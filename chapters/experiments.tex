\chapter{Experiments} \label{ch:exp} 
In this section, we present the results of evaluating our proposed algorithms on
three real-world datasets and compare them to the state-of-the-art in sequential
level set estimation.
In more detail, the algorithms and their setup are as follows.
\begin{description}[labelindent=0pt,leftmargin=7pt,itemindent=-2pt,itemsep=0pt]
\item[\acl/\iacl:] Since the bound of \theoremref{thm:acl} is fairly conservative,
  in our experiments we used a constant value of $\beta_t^{1/2} = 3$, which is
  somewhat smaller than the values suggested by the theorem.
\item[\bacl/\ibacl:] We used somewhat larger value of $\eta_t^{1/2} = 4$ (compared
  to $\beta_t^{1/2}=3$ of the sequential case) and a batch size of $B = 30$.
\item[\str:] The state-of-the-art straddle heuristic,
  as proposed by \citet{bryan05}, with the selection rule
  ${\*x_t = \argmax_{\*x \in D} \left(1.96\sigma_{t-1}(\*x) - |\mu_{t-1}(\*x) - h|\right)}$.
\item[\istr:] For the implicit threshold setting, we have defined a
  variant of the straddle heuristic that uses at each step the
  implicitly defined threshold level with respect to the maximum of the
  inferred mean, i.e.
  ${h_t = \omega \max_{\*x\in D} \mu_{t-1}(\*x)}$.
\item[\rstr/\bstr:] We have defined two batch versions of the straddle heuristic:
  \rstr selects the $B = 30$ points with
  the largest straddle score, while \bstr follows a similar approach to \bacl
  by using the following selection rule
  \begin{align*}
  {\*x_t = \argmax_{\*x \in D}\left(1.96\sigma_{t-1}(\*x) - |\mu_{\fb[t]}(\*x) - h|\right)}.
  \end{align*}
\item[\var:]  The maximum variance rule
  $\*x_t = \argmax_{\*x \in D} \sigma_{t-1}(\*x)$.
\end{description}

We implemented all algorithms in MATLAB and we used an approximate TSP
solver based on genetic
programming\footnote{\url{http://www.mathworks.com/matlabcentral/fileexchange/21198}}
for the path planning evaluation.

\paragraph{Dataset 1: Network latency}
Our first dataset consists of round-trip time (RTT) measurements obtained
by ``pinging'' $1768$ servers spread around the world (see \figref{fig:map}).
The sample space consists of the longitude and latitude coordinates of each
server, as determined by a commercial geolocation
database\footnote{\url{http://www.maxmind.com}}.
Example applications for geographic RTT level set estimation, include
monitoring global networks and improving quality of service for applications
such as internet telephony or online games. Furthermore,
selecting samples in batches results in significant time savings, since
sending out and receiving multiple ICMP packets can be virtually performed in
parallel.

We used $200$ randomly selected measurements to fit suitable hyperparameters for
an anisotropic Mat\'{e}rn-5~\mbox{\cite{rasmussen06}} kernel by maximum
likelihood  and the remaining $1568$ for evaluation.
The threshold level we chose for the experiments was $h = 200$ ms.

%\setlength\figureheight{1.7in}\setlength\figurewidth{3.4in}
%% This file was created by matlab2tikz v0.2.3.
% Copyright (c) 2008--2012, Nico Schlömer <nico.schloemer@gmail.com>
% All rights reserved.
% 
% 
% 

% defining custom colors
\definecolor{mycolor1}{rgb}{1,0.98989898989899,0}
\definecolor{mycolor2}{rgb}{1,0.434343434343434,0}
\definecolor{mycolor3}{rgb}{1,0.464646464646465,0}
\definecolor{mycolor4}{rgb}{1,0.424242424242424,0}
\definecolor{mycolor5}{rgb}{1,0.282828282828283,0}
\definecolor{mycolor6}{rgb}{1,0.383838383838384,0}
\definecolor{mycolor7}{rgb}{1,0.404040404040404,0}
\definecolor{mycolor8}{rgb}{1,0.323232323232323,0}
\definecolor{mycolor9}{rgb}{1,0.373737373737374,0}
\definecolor{mycolor10}{rgb}{1,0.313131313131313,0}
\definecolor{mycolor11}{rgb}{1,0.292929292929293,0}
\definecolor{mycolor12}{rgb}{1,0.343434343434343,0}
\definecolor{mycolor13}{rgb}{1,0.202020202020202,0}
\definecolor{mycolor14}{rgb}{1,0.929292929292929,0}
\definecolor{mycolor15}{rgb}{1,0.94949494949495,0}
\definecolor{mycolor16}{rgb}{1,0.545454545454545,0}
\definecolor{mycolor17}{rgb}{1,0.595959595959596,0}
\definecolor{mycolor18}{rgb}{1,0.898989898989899,0}
\definecolor{mycolor19}{rgb}{1,0.95959595959596,0}
\definecolor{mycolor20}{rgb}{1,0.0505050505050505,0}
\definecolor{mycolor21}{rgb}{1,0.484848484848485,0}
\definecolor{mycolor22}{rgb}{1,0.494949494949495,0}
\definecolor{mycolor23}{rgb}{1,0.505050505050505,0}
\definecolor{mycolor24}{rgb}{1,0.838383838383838,0}
\definecolor{mycolor25}{rgb}{1,0.919191919191919,0}
\definecolor{mycolor26}{rgb}{1,0.909090909090909,0}
\definecolor{mycolor27}{rgb}{1,0.777777777777778,0}
\definecolor{mycolor28}{rgb}{1,0.737373737373737,0}
\definecolor{mycolor29}{rgb}{1,0.636363636363636,0}
\definecolor{mycolor30}{rgb}{1,0.747474747474748,0}
\definecolor{mycolor31}{rgb}{1,0.717171717171717,0}
\definecolor{mycolor32}{rgb}{1,0.525252525252525,0}
\definecolor{mycolor33}{rgb}{1,0.0303030303030303,0}
\definecolor{mycolor34}{rgb}{1,0.191919191919192,0}
\definecolor{mycolor35}{rgb}{1,0.262626262626263,0}
\definecolor{mycolor36}{rgb}{1,0.212121212121212,0}
\definecolor{mycolor37}{rgb}{1,0.303030303030303,0}
\definecolor{mycolor38}{rgb}{1,0.333333333333333,0}
\definecolor{mycolor39}{rgb}{1,0.181818181818182,0}
\definecolor{mycolor40}{rgb}{1,0.131313131313131,0}
\definecolor{mycolor41}{rgb}{1,0.353535353535354,0}
\definecolor{mycolor42}{rgb}{1,0.606060606060606,0}
\definecolor{mycolor43}{rgb}{1,0.939393939393939,0}
\definecolor{mycolor44}{rgb}{1,0.767676767676768,0}
\definecolor{mycolor45}{rgb}{1,0.878787878787879,0}
\definecolor{mycolor46}{rgb}{1,0.888888888888889,0}
\definecolor{mycolor47}{rgb}{1,0.414141414141414,0}
\definecolor{mycolor48}{rgb}{1,0.858585858585859,0}
\definecolor{mycolor49}{rgb}{1,0.96969696969697,0}
\definecolor{mycolor50}{rgb}{1,0.848484848484849,0}
\definecolor{mycolor51}{rgb}{1,0.868686868686869,0}
\definecolor{mycolor52}{rgb}{1,0.696969696969697,0}
\definecolor{mycolor53}{rgb}{1,0.151515151515152,0}
\definecolor{mycolor54}{rgb}{1,0.535353535353535,0}
\definecolor{mycolor55}{rgb}{1,0.161616161616162,0}
\definecolor{mycolor56}{rgb}{1,0.252525252525253,0}
\definecolor{mycolor57}{rgb}{1,0.242424242424242,0}
\definecolor{mycolor58}{rgb}{1,0.515151515151515,0}
\definecolor{mycolor59}{rgb}{1,0.808080808080808,0}
\definecolor{mycolor60}{rgb}{1,0.97979797979798,0}
\definecolor{mycolor61}{rgb}{1,0.393939393939394,0}
\definecolor{mycolor62}{rgb}{1,0.363636363636364,0}
\definecolor{mycolor63}{rgb}{1,0.626262626262626,0}
\definecolor{mycolor64}{rgb}{1,0.676767676767677,0}
\definecolor{mycolor65}{rgb}{1,0.787878787878788,0}
\definecolor{mycolor66}{rgb}{1,0.818181818181818,0}
\definecolor{mycolor67}{rgb}{1,0.232323232323232,0}
\definecolor{mycolor68}{rgb}{1,0.565656565656566,0}
\definecolor{mycolor69}{rgb}{1,0.616161616161616,0}
\definecolor{mycolor70}{rgb}{1,0.666666666666667,0}
\definecolor{mycolor71}{rgb}{1,0.0606060606060606,0}
\definecolor{mycolor72}{rgb}{1,0.272727272727273,0}
\definecolor{mycolor73}{rgb}{1,0.757575757575758,0}
\definecolor{mycolor74}{rgb}{1,0.686868686868687,0}
\definecolor{mycolor75}{rgb}{1,0.727272727272727,0}
\definecolor{mycolor76}{rgb}{1,0.707070707070707,0}
\definecolor{mycolor77}{rgb}{1,0.646464646464647,0}
\definecolor{mycolor78}{rgb}{1,0.656565656565657,0}
\definecolor{mycolor79}{rgb}{1,0.555555555555556,0}
\definecolor{mycolor80}{rgb}{1,0.585858585858586,0}
\definecolor{mycolor81}{rgb}{1,0.454545454545455,0}
\definecolor{mycolor82}{rgb}{1,0.575757575757576,0}
\definecolor{mycolor83}{rgb}{1,0.171717171717172,0}
\definecolor{mycolor84}{rgb}{1,0.444444444444444,0}
\definecolor{mycolor85}{rgb}{1,0.828282828282828,0}
\definecolor{mycolor86}{rgb}{1,0.797979797979798,0}
\definecolor{mycolor87}{rgb}{1,0.141414141414141,0}
\definecolor{mycolor88}{rgb}{1,0.474747474747475,0}
\definecolor{mycolor89}{rgb}{1,0.222222222222222,0}

\begin{tikzpicture}

\begin{axis}[%
tick label style={font=\tiny},
label style={font=\tiny},
xlabel shift={-10pt},
ylabel shift={-17pt},
legend style={font=\tiny},
view={0}{90},
width=\figurewidth,
height=\figureheight,
scale only axis,
%every outer x axis line/.append style={white},
%every x tick label/.append style={font=\color{white}},
xmin=-180, xmax=180,
xtick={-150, -100, -50, 50, 100, 150},
xlabel=Longitude (deg),
%xtick={\empty},
%every outer y axis line/.append style={white},
%every y tick label/.append style={font=\color{white}},
ymin=-90, ymax=90,
ytick={-50, 50},
ylabel=Latitude (deg),
%ytick={\empty},
name=plot1,
unbounded coords=jump,
colormap={mymap}{[1pt] rgb(0pt)=(1,1,0); rgb(63pt)=(1,0,0)},
colorbar right,
colorbar style={
  tickwidth=0.03cm,
  at={(1.04, 0)},
  anchor=south east,
  yticklabel pos=right,
  yticklabel shift={-2pt}
},
tickwidth=0.1cm,
colorbar/width=0.15cm,
point meta min=0,
point meta max=500]
\addplot [
color=black,
solid,
forget plot
]
coordinates{
 (-180,-83.83)(-178,-84.33)(-174,-84.5)(-170,-84.67)(-166,-84.92)(-163,-85.42)(-158,-85.42)(-152,-85.58)(-146,-85.33)(-147,-84.83)(-151,-84.5)(-153.5,-84)(-153,-83.5)(-154,-83)(-154,-82.5)(-154,-82)(-154.5,-81.5)(-153,-81.17)(-150,-81)(-146.5,-80.92)(-145.5,-80.67)(-148,-80.33)(-150,-80)(-152.5,-79.67)(-155,-79.25)(-157,-78.83)(-157.255422337107,-78.7478175670774)(-157.507183341852,-78.6654203423938)(-157.75535304677,-78.5828129769986)(-158,-78.5)(-157.665802438229,-78.4805682858952)(-157.332726125013,-78.4607564072689)(-157.000786850852,-78.4405663191189)(-156.67,-78.42)(-154.5,-78.5)(-154.5,-78.17)(-154.5,-78.17)(-156.67,-78.08)(-158,-77.83)(-158.33,-77.5)(-158.67,-77.17)(-157,-77)(-154,-77.17)(-153,-77.58)(-150.5,-77.83)(-148,-77.67)(-146,-77.25)(-146,-76.75)(-146.33,-76.33)(-146.67,-75.92)(-144,-75.83)(-142,-75.58)(-140.67,-75.75)(-140,-75.42)(-138.33,-75.25)(-136.67,-75)(-135.5,-74.75)(-133,-74.75)(-132,-75)(-130.83,-75.42)(-129.67,-75.75)(-128,-76)(-127,-76)(-126.83,-75.67)(-126.33,-75.17)(-125.5,-74.67)(-124.5,-74.33)(-124.5,-74.33)(-123.33,-73.83)(-123.67,-73.25)(-122.33,-73.25)(-121.33,-73.75)(-120,-74)(-120,-73.58)(-119.584192021421,-73.6037325911182)(-119.167231996011,-73.6266457352614)(-118.749155810557,-73.6487359854033)(-118.33,-73.67)(-117.91502005204,-73.6712165775355)(-117.5,-73.6716221235359)(-117.084979947959,-73.6712165775355)(-116.67,-73.67)(-116.585609443933,-73.7100506890652)(-116.500814364581,-73.7500677519198)(-116.415612107796,-73.7900509393775)(-116.33,-73.83)(-116.659817277184,-73.9157742527243)(-116.993066927812,-74.0010378330404)(-117.329783158363,-74.0857825281661)(-117.67,-74.17)(-117.67,-74.295)(-117.67,-74.4199999999999)(-117.67,-74.5449999999999)(-117.67,-74.67)(-117.343081218212,-74.7957317872771)(-117.010850320323,-74.9209838162284)(-116.673194849942,-75.0457440264729)(-116.33,-75.17)(-115.667717906718,-75.1928534787836)(-115.003517406365,-75.2138099091036)(-114.337557532468,-75.2328612112005)(-113.67,-75.25)(-112.33,-75)(-112.5,-74.5)(-112.543494166835,-74.3750124497283)(-112.586315059339,-74.250016467828)(-112.628478545448,-74.1250122535671)(-112.67,-74)(-112.499992333616,-73.9176973942665)(-112.331671740615,-73.835261848855)(-112.165015187667,-73.7526953884558)(-112,-73.67)(-111.5,-74)(-111,-74.5)(-110.33,-75)(-108.67,-75.17)(-108.5,-74.67)(-109.5,-74.5)(-109.5,-74)(-108.33,-74)(-108,-74.33)(-108.5,-74.67)(-107.5,-75.08)(-106,-75.08)(-104.5,-74.92)(-104.5,-74.92)(-104.5,-74.58)(-102.67,-74.58)(-101.33,-74.75)(-100,-75)(-99.5,-74.75)(-101.33,-74.33)(-100.67,-74)(-101.5,-73.75)(-101.5,-73.5)(-100,-73.58)(-99,-73.42)(-100,-73.08)(-101.33,-73.25)(-102.5,-72.75)(-101.33,-72.67)(-101.5,-73)(-99.67,-72.92)(-98.5,-73.33)(-98,-73.08)(-97.5,-73.42)(-97.5,-73.75)(-96.33,-73.75)(-96,-73.42)(-96,-73.08)(-97.5,-72.92)(-98.5,-72.5)(-99.67,-72.17)(-100.005859467039,-72.1083517079824)(-100.339477288113,-72.0461316628678)(-100.67085633472,-71.9833457935171)(-101,-71.92)(-100.705518400152,-71.8781650224846)(-100.412354835263,-71.8358846254382)(-100.120513927715,-71.7931619148565)(-99.83,-71.75)(-98.33,-72)(-98.33,-72)(-97,-71.83)(-95.67,-72.08)(-95.67,-72.42)(-94,-72.58)(-92.25,-72.75)(-90.5,-72.92)(-88.75,-72.92)(-87,-72.92)(-85.5,-73.08)(-84.33,-72.92)(-83,-72.83)(-81.25,-72.92)(-79.5,-73)(-78.5,-73.25)(-78.67,-72.83)(-78,-72.42)(-76.5,-72.58)(-75,-72.83)(-74,-73)(-74.5,-73.5)(-73,-73.42)(-71.5,-73.33)(-70,-73.17)(-68.5,-73.08)(-67.5,-72.83)(-67,-72.33)(-67,-71.83)(-67.5,-71.5)(-67.5,-70.92)(-68,-70.42)(-68,-70.42)(-68.5,-70)(-68.5,-69.42)(-67.33,-69.42)(-66.67,-69.08)(-67.33,-68.83)(-66.83,-68.5)(-66.83,-68.08)(-67.5,-67.75)(-67.5,-67.42)(-68.5,-67.75)(-69.25,-67.75)(-69.33,-67.33)(-68.67,-66.92)(-67.83,-66.67)(-67.67,-67.08)(-67,-67)(-66.33,-67.33)(-66.33,-66.83)(-65.5,-66.58)(-65.83,-66.25)(-65,-65.92)(-64,-65.5)(-64,-65.08)(-63,-65.17)(-62.67,-64.75)(-61.67,-64.67)(-61.5,-64.25)(-61,-64)(-59.67,-63.83)(-58.83,-63.58)(-58.83,-63.58)(-58,-63.33)(-57.17,-63.25)(-56.67,-63.58)(-57.33,-63.58)(-57.5386102377023,-63.6429574267071)(-57.7481455686199,-63.7056113601135)(-57.9586081319424,-63.7679596159831)(-58.17,-63.83)(-57.9400029854285,-63.8305481842232)(-57.7100000000004,-63.8307309148603)(-57.4799970145724,-63.8305481842232)(-57.25,-63.83)(-57.33,-64.42)(-58.17,-64.33)(-58.67,-64)(-59,-64.5)(-59.67,-64.33)(-60.5,-64.67)(-61.33,-65)(-62,-65.42)(-62.17,-65.75)(-62.33,-66.08)(-62.67,-66.5)(-63.5,-66.25)(-64,-66.5)(-63.67,-66.92)(-64.67,-66.92)(-65.33,-67.33)(-65.5,-67.83)(-65.3338510403035,-67.8927547997984)(-65.1668048745368,-67.9553406989404)(-64.9988562723319,-68.0177562517368)(-64.83,-68.08)(-64.953979428701,-68.1426406141349)(-65.0786360082278,-68.2051880244368)(-65.2039745756343,-68.2676414245381)(-65.33,-68.33)(-65.2067401949143,-68.4351385889792)(-65.0823309536829,-68.540185691339)(-64.9567563042736,-68.6451399544047)(-64.83,-68.75)(-63.67,-68.75)(-62.67,-69.08)(-62.33,-69.67)(-62,-70.17)(-62,-70.17)(-61.5,-70.58)(-61.67,-70.92)(-61,-71.17)(-60.83,-71.58)(-61.67,-72)(-60.83,-72.33)(-61,-72.67)(-60.7115780127866,-72.7531285087468)(-60.4204543154898,-72.8358420477363)(-60.1266034487312,-72.918134584303)(-59.83,-73)(-59.9137922876238,-73.0825521725768)(-59.9983820463683,-73.1650699046062)(-60.0837807286573,-73.2475526867978)(-60.17,-73.33)(-61.5,-73.42)(-61,-73.75)(-60.67,-74.33)(-61.67,-74.67)(-61.67,-75.08)(-63,-75.17)(-63,-75.67)(-63.67,-76.08)(-65.5,-76.5)(-67.33,-77)(-65,-77.17)(-62.5,-77.08)(-60,-77.08)(-57.67,-77.25)(-56,-77.58)(-53.67,-77.92)(-51.67,-78.25)(-49.33,-78.5)(-47.67,-78.75)(-44,-78.83)(-41,-78.83)(-41,-78.83)(-38.5,-78.75)(-36.33,-78.33)(-35,-77.83)(-33.33,-77.5)(-31.67,-77.17)(-30,-76.83)(-28.33,-76.5)(-27,-76.08)(-26,-75.67)(-24,-75.33)(-23.67,-74.75)(-22.33,-74.42)(-20.67,-74.08)(-19.33,-73.75)(-17.67,-73.42)(-16.33,-73.08)(-14.67,-72.92)(-13.5,-72.83)(-12.33,-72.58)(-11.33,-72.25)(-11.17,-71.92)(-12,-71.67)(-12.17,-71.33)(-11.5,-71.25)(-11,-71.58)(-10.5,-71.25)(-9.75,-71)(-9,-71.17)(-8.67,-71.42)(-8.5,-71.83)(-8.5,-71.83)(-8.20564497011808,-71.7906676054245)(-7.91252284564118,-71.7508881930406)(-7.62063937098969,-71.7106646824286)(-7.33,-71.67)(-7.41583618209822,-71.6075569886788)(-7.50111135584679,-71.5450757273537)(-7.58583086738204,-71.4825566035972)(-7.67,-71.42)(-7.58388064667478,-71.3350578316363)(-7.49851407634196,-71.2500767589533)(-7.4138904265624,-71.1650573091027)(-7.33,-71.08)(-7.41580827518496,-71.0175585331279)(-7.50107438383996,-70.9550777878001)(-7.58580331465019,-70.8925581497706)(-7.67,-70.83)(-7.45937070849571,-70.8103590976097)(-7.24915845804159,-70.7904782992163)(-7.03936699293089,-70.7703583497576)(-6.83,-70.75)(-5.83,-70.75)(-5.83,-71)(-6,-71.33)(-4.5,-71.33)(-3,-71.25)(-3,-70.67)(-3.33,-70.42)(-2.5,-70.25)(-1.67,-70.42)(-1,-70.67)(-1.5,-71)(-0.67,-71.25)(0,-71.5)(1.33,-71.25)(2.67,-70.92)(3.5,-70.58)(4.67,-70.33)(6.33,-70.33)(7.33,-70.08)(8.67,-70.08)(10,-70.17)(11.33,-70.08)(12.67,-70.08)(14.33,-70.17)(15.67,-70.17)(15.67,-70.17)(17,-70.17)(18.33,-70.17)(20,-70.33)(21.5,-70.42)(23,-70.58)(24.67,-70.42)(25.33,-70.17)(27,-70.17)(28.33,-69.92)(30,-69.83)(31.33,-69.58)(32,-69.5)(33.5,-69.5)(33.33,-69)(33.5,-68.67)(34.33,-68.58)(35,-68.92)(36,-69.33)(37,-69.5)(38.33,-69.58)(38.67,-70.08)(39.33,-69.67)(39.83,-69.5)(39.83,-69.17)(40,-68.83)(41.33,-68.5)(42,-68.17)(43.33,-67.83)(44.5,-67.75)(46,-67.75)(46,-67.75)(47,-67.58)(48.17,-67.67)(48.5,-67.42)(49.33,-67.42)(49.83,-67.83)(50.5,-67.58)(50.17,-67)(50.5,-66.5)(51.5,-66.25)(52.17,-65.92)(53.67,-65.83)(55.17,-65.92)(56,-66.25)(57.17,-66.42)(57.17,-66.67)(56.33,-66.75)(57,-66.92)(58,-67)(59,-67.17)(59.33,-67.5)(60.67,-67.33)(61.67,-67.58)(62.67,-67.67)(63.67,-67.5)(65,-67.58)(66.33,-67.75)(67.67,-67.83)(68.67,-67.83)(69.5,-67.67)(69.5,-68.17)(69.5,-68.17)(70,-68.58)(70.5,-69)(69.5,-69.17)(69.83,-69.58)(69.83,-70.08)(70.67,-70.5)(71.67,-70.42)(71,-70.08)(72.33,-69.75)(73.5,-69.58)(74.5,-69.67)(75.5,-69.83)(76.17,-69.5)(77.17,-69.17)(78,-69.08)(78.33,-68.5)(79,-68.25)(80.5,-68)(81.67,-67.83)(82.5,-67.33)(83.67,-67.17)(85,-67)(86.33,-66.83)(87.67,-66.67)(88.5,-66.75)(89.67,-66.75)(91,-66.67)(92,-66.5)(93,-66.58)(94,-66.58)(94,-66.58)(95,-66.5)(96,-66.67)(97,-66.5)(97.67,-66.67)(98.5,-66.5)(99.33,-66.58)(100,-66.42)(100.67,-66.42)(101.5,-66.08)(102.67,-65.83)(104,-65.92)(105,-66.08)(106,-66.25)(107,-66.5)(108,-66.58)(109,-66.83)(109.5,-66.58)(110.33,-66.67)(110.67,-66.42)(110.67,-66.08)(111.67,-65.92)(112.67,-65.83)(113.67,-65.75)(114.5,-66)(115.5,-66.33)(116.5,-66.5)(117.67,-66.75)(119,-66.83)(120.33,-66.83)(121.5,-66.67)(121.5,-66.67)(122.5,-66.42)(123.33,-66.75)(124.33,-66.75)(124.83,-66.5)(125.83,-66.58)(126.5,-66.25)(127,-66.25)(127.67,-66.5)(128.33,-66.92)(129,-67)(129.5,-67.17)(130,-66.92)(130,-66.33)(130.83,-66.08)(132,-66.17)(133.33,-66.08)(134.33,-66.17)(135.33,-66.08)(136.33,-66.33)(137.33,-66.33)(138.33,-66.5)(139.33,-66.58)(140.5,-66.67)(142,-66.75)(142.5,-67)(143.33,-66.83)(144.33,-67)(144.33,-67.5)(145,-67.58)(146,-67.5)(146,-67.5)(146.83,-67.83)(147,-68.25)(148.33,-68.42)(149,-68.25)(150.33,-68.5)(151,-68.33)(151.5,-68.67)(152.67,-68.58)(154.33,-68.58)(155,-69.08)(156,-69.33)(156.5,-69)(157.5,-69.08)(158.67,-69.25)(160,-69.5)(161,-69.67)(161,-70.08)(161.17,-70.5)(161.67,-70.83)(162.17,-70.58)(162.33,-70.17)(163.33,-70.08)(163.17,-70.58)(164.67,-70.58)(166,-70.5)(167.33,-70.83)(168.67,-71.17)(170,-71.5)(170.17,-71.17)(171,-71.75)(171,-71.75)(170.17,-72)(170.5,-72.5)(169.67,-72.83)(168.5,-73.17)(167,-73.25)(167,-73.67)(166.67,-74.08)(165.33,-74.17)(165.33,-74.5)(164.33,-74.83)(163,-75.17)(162.67,-75.67)(162.67,-76.08)(163,-76.5)(163.33,-76.92)(163.67,-77.33)(163.67,-77.75)(164.67,-78)(164.993808606302,-78.1055736155981)(165.323322295135,-78.2107716399456)(165.658674020009,-78.315583926681)(166,-78.42)(165.502596593669,-78.4412805859242)(165.003419554567,-78.4617104395498)(164.50253237335,-78.4812850335626)(164,-78.5)(161.5,-78.75)(160,-79.17)(160,-79.67)(160,-80)(160.33,-80.5)(160.33,-81)(161,-81.5)(162.33,-81.92)(164,-82.42)(166,-82.83)(166,-82.83)(168.5,-83.25)(171.5,-83.5)(176,-83.5)(180,-83.83) 
};
\addplot [
color=black,
solid,
forget plot
]
coordinates{
 (166.33,-77.58)(166.67,-77.08)(168.17,-77.33)(169.67,-77.42)(168.17,-77.67)(166.33,-77.58) 
};
\addplot [
color=black,
solid,
forget plot
]
coordinates{
 (-164,-78.75)(-161.67,-78.75)(-160.67,-79.08)(-160,-79.5)(-160,-79.92)(-161.67,-80.25)(-163.67,-79.83)(-163.67,-79.33)(-163,-79.08)(-164,-78.75) 
};
\addplot [
color=black,
solid,
forget plot
]
coordinates{
 (-150.17,-76.58)(-148.17,-76.17)(-147.923344089767,-76.232874883209)(-147.674473191872,-76.2955021268067)(-147.423365714669,-76.3578783192194)(-147.17,-76.42)(-147.496504185226,-76.5031490777776)(-147.826970351106,-76.5858707681257)(-148.161451276784,-76.6681571132266)(-148.5,-76.75)(-150.17,-76.58) 
};
\addplot [
color=black,
solid,
forget plot
]
coordinates{
 (-76.17,-71.08)(-74.67,-71.08)(-73,-71)(-71.83,-70.92)(-71,-70.83)(-70.67,-70.58)(-71,-70.08)(-71.5,-69.58)(-71.67,-69.08)(-71.17,-68.92)(-70.33,-68.83)(-69.67,-69.42)(-69.33,-70)(-68.83,-70.5)(-68.5,-71)(-68.33,-71.5)(-68.5,-71.92)(-68.83,-72.33)(-70,-72.58)(-71.5,-72.67)(-73,-72.58)(-73.33,-72.25)(-74.33,-72.17)(-75,-71.83)(-76,-71.67)(-76.5,-71.33)(-76.17,-71.08) 
};
\addplot [
color=black,
solid,
forget plot
]
coordinates{
 (-76.17,-70.58)(-75.67,-70.25)(-75.75,-70)(-74.5,-69.75)(-73.67,-69.92)(-73.75,-70.25)(-74,-70.5)(-75,-70.58)(-76.17,-70.58) 
};
\addplot [
color=black,
solid,
forget plot
]
coordinates{
 (-63.6633303741216,-64.7989959192451)(-64.0273318151905,-64.7633657732706)(-64.1488446695359,-64.6919374914007)(-64.0043646946572,-64.5632966594665)(-63.6482008624201,-64.4418395054606)(-63.2838262445918,-64.3347368915286)(-63.0403533092285,-64.5629349530888)(-63.4542032133668,-64.755996322801)(-63.6633303741216,-64.7989959192451) 
};
\addplot [
color=black,
solid,
forget plot
]
coordinates{
 (-62.5913662569744,-64.5413511845796)(-62.4589453073987,-64.2560344163493)(-62.478112668493,-64.0994011387612)(-62.2524988298515,-64.085095533338)(-62.1151478940334,-64.248782106647)(-62.5913662569744,-64.5413511845796) 
};
\addplot [
color=black,
solid,
forget plot
]
coordinates{
 (-56.5,-63.25)(-55.83,-63)(-55.6247526915323,-63.1054517217184)(-55.4180125484956,-63.2106046692346)(-55.2097661496206,-63.3154552935421)(-55,-63.42)(-55.2494724106988,-63.440653571045)(-55.4993010866733,-63.4608720913568)(-55.7494792360173,-63.4806545613378)(-56,-63.5)(-56.5,-63.25) 
};
\addplot [
color=black,
solid,
forget plot
]
coordinates{
 (-58.7408216907335,-62.2320263927402)(-58.752164115065,-62.1267610168571)(-58.5804285832797,-62.0074841192257)(-57.9444859664825,-61.9579892530209)(-57.8247837451924,-61.9859270938015)(-57.8207931715965,-62.0770934372158)(-58.047645458526,-62.1615061608537)(-58.7408216907335,-62.2320263927402) 
};
\addplot [
color=black,
solid,
forget plot
]
coordinates{
 (-60.2334317286292,-62.7883469405551)(-60.3025360899468,-62.6897195759722)(-60.7806802031877,-62.6687835109152)(-60.7730998755804,-62.5068976608135)(-60.3265698625862,-62.4574985682265)(-60.0342840543814,-62.4784754622075)(-59.9605193987969,-62.5699043252387)(-59.9509493286603,-62.6755087527817)(-60.2334317286292,-62.7883469405551) 
};
\addplot [
color=black,
solid,
forget plot
]
coordinates{
 (-46.33,-60.63)(-46,-60.33)(-45.7521383180011,-60.4231988159565)(-45.5028553208436,-60.5159347023893)(-45.2521446372527,-60.6082032458992)(-45,-60.7)(-45.3330339620452,-60.6837368990792)(-45.665722882662,-60.6666482197008)(-45.9980503316596,-60.6487354138707)(-46.33,-60.63) 
};
\addplot [
color=black,
solid,
forget plot
]
coordinates{
 (-80.08,0)(-80,0.67)(-79.33,1)(-78.83,1.25)(-79,1.67)(-78.58,1.75)(-78.67,2.17)(-78.33,2.67)(-77.75,2.67)(-77.5,3.25)(-77.17,3.67)(-77.5,4)(-77.33,4.5)(-77.42,5)(-77.42,5.5)(-77.5,6.17)(-77.42,6.58)(-77.75,7)(-78.17,7.42)(-78.42,8)(-78.17,8.33)(-78.5,8.33)(-78.75,8.75)(-79.17,9)(-79.5,9)(-79.75,8.58)(-80.08,8.33)(-80.08,8.33)(-80.5,8.17)(-80.42,7.83)(-80.3149387336758,7.74753829793484)(-80.2099186092594,7.66505087257074)(-80.1049391801817,7.5825380110076)(-80,7.5)(-80.1250530931188,7.43755222799571)(-80.2500705956681,7.37506943244509)(-80.3750528002957,7.31255192071402)(-80.5,7.25)(-80.92,7.17)(-81,7.75)(-81.5,7.75)(-81.75,8.08)(-82.25,8.25)(-82.75,8.33)(-83,8.17)(-83.25,8.42)(-83.75,8.5)(-83.67,9)(-84,9.33)(-84.58,9.67)(-84.83,10)(-85.17,9.67)(-85.33,9.83)(-85.67,9.92)(-85.83,10.33)(-85.75,10.75)(-85.75,11.17)(-86.17,11.5)(-86.58,11.83)(-87,12.25)(-87.33,12.58)(-87.4148942345818,12.6850407080801)(-87.4998585524282,12.7900544464804)(-87.5848935937196,12.895040961821)(-87.67,13)(-87.5850000094671,13.0000414590983)(-87.5000000000004,13.0000552788089)(-87.4149999905337,13.0000414590983)(-87.33,13)(-87.4148906288233,13.1050419535466)(-87.4998537414893,13.2100561080881)(-87.5848899830567,13.3150422087725)(-87.67,13.42)(-87.67,13.42)(-87.83,13.17)(-88.42,13.17)(-88.83,13.17)(-89.33,13.42)(-89.75,13.42)(-90,13.67)(-90.5,13.92)(-91,13.92)(-91.5,14.08)(-92,14.33)(-92.33,14.67)(-92.75,15)(-93.17,15.42)(-93.42,15.67)(-93.83,15.92)(-94.25,16.08)(-94.67,16.17)(-95.17,16.17)(-95.67,15.92)(-96.25,15.67)(-96.83,15.83)(-97.42,16)(-98,16.08)(-98.5,16.33)(-99,16.58)(-99.5,16.67)(-100,16.92)(-100.5,17.08)(-101,17.25)(-101.58,17.67)(-101.58,17.67)(-102,18)(-102.25,18)(-102.83,18.08)(-103.5,18.33)(-103.83,18.75)(-104.33,19.08)(-104.92,19.33)(-105.33,19.83)(-105.67,20.33)(-105.33,20.58)(-105.5,20.83)(-105.17,21.17)(-105.17,21.5)(-105.5,21.75)(-105.67,22.42)(-106,22.75)(-106.42,23.17)(-106.75,23.5)(-107.17,24)(-107.58,24.33)(-108,24.67)(-108.33,25.25)(-109,25.5)(-109.42,25.75)(-109.42,26)(-109.25,26.33)(-109.5,26.67)(-109.75,26.67)(-110,27)(-110.58,27.42)(-110.58,27.42)(-110.58,27.92)(-111.17,28)(-111.67,28.5)(-112.17,29)(-112.42,29.5)(-112.75,29.92)(-112.83,30.25)(-113.08,30.75)(-113.08,31.17)(-113.5,31.33)(-113.83,31.58)(-114.17,31.5)(-114.75,31.75)(-114.83,31.42)(-114.75,31)(-114.58,30.58)(-114.67,30.17)(-114.42,29.83)(-114,29.58)(-113.67,29.17)(-113.25,28.75)(-112.92,28.33)(-112.75,27.83)(-112.33,27.5)(-112,27)(-111.5,26.5)(-111.33,26)(-111,25.5)(-111,25.17)(-110.67,24.83)(-110.67,24.83)(-110.83,24.67)(-110.58,24.25)(-110.17,24.25)(-109.75,23.75)(-109.42,23.33)(-110,22.83)(-110.33,23.5)(-110.83,23.83)(-111.25,24.25)(-111.67,24.58)(-112.33,24.83)(-112.08,25.17)(-112.08,25.67)(-112.33,26.17)(-112.83,26.42)(-113.17,26.75)(-113.58,26.75)(-114,27)(-114.33,27.17)(-114.5,27.5)(-115,27.83)(-114.42,27.75)(-114,28)(-114,28.42)(-114.33,28.75)(-114.67,29.08)(-115.17,29.5)(-115.67,29.67)(-115.75,30.25)(-116,30.33)(-116,30.33)(-116,30.75)(-116.33,31)(-116.33,31.25)(-116.58,31.58)(-116.75,32)(-117.08,32.5)(-117.25,33)(-117.5,33.42)(-118,33.75)(-118.33,33.75)(-118.42,34.08)(-118.75,34.08)(-119.25,34.25)(-119.83,34.42)(-120.58,34.58)(-120.58,35.08)(-121.08,35.58)(-121.42,36)(-121.92,36.33)(-121.83,36.58)(-121.83,37)(-122.17,37)(-122.42,37.33)(-122.5,37.75)(-122.08,37.5)(-122.33,37.92)(-122.17,38.08)(-122.5,38.17)(-122.5,37.92)(-123,38)(-123,38)(-122.92,38.33)(-123.33,38.58)(-123.67,38.92)(-123.75,39.33)(-123.75,39.67)(-124,40)(-124.33,40.25)(-124.33,40.5)(-124.08,41)(-124,41.5)(-124.17,42)(-124.42,42.33)(-124.5,42.83)(-124.25,43.42)(-124.08,44)(-124,44.5)(-124,45)(-123.83,45.42)(-123.92,45.83)(-123.92,46.17)(-123.17,46.17)(-123.92,46.33)(-124,46.67)(-124.08,47)(-124.33,47.5)(-124.58,47.92)(-124.75,48.17)(-124.67,48.42)(-123.83,48.17)(-123.17,48.17)(-123.17,48.17)(-122.58,47.92)(-123,47.5)(-122.58,47.67)(-122.5,47.5)(-122.25,48)(-122.331993750491,48.1050883041792)(-122.414323374038,48.2101180446756)(-122.496991305508,48.3150887639105)(-122.58,48.42)(-122.540094608219,48.4600207671279)(-122.500126255258,48.5000277170568)(-122.460094774781,48.5400208084937)(-122.42,48.58)(-122.501983334192,48.6850880821232)(-122.584309430342,48.7901177514985)(-122.666980805992,48.8950885462286)(-122.75,49)(-123.17,49.25)(-123.17,49.75)(-123.75,49.5)(-124.33,49.75)(-124.92,50.08)(-125.5,50.42)(-126.25,50.67)(-127,50.83)(-127.75,51.17)(-127.75,51.5)(-128.17,51.83)(-128.42,52.33)(-129.17,52.58)(-129.75,53.17)(-130.33,53.42)(-130.75,53.92)(-130.17,54.17)(-130.42,54.42)(-130.08,54.92)(-130.75,54.75)(-131.33,55)(-131.75,55.33)(-131.75,55.33)(-132.08,55.75)(-132,56.08)(-132.5,56.08)(-133,56.25)(-132.58,55.83)(-132.08,55.25)(-132.08,54.67)(-132.5,55.08)(-133.08,55.33)(-133.42,55.58)(-133.67,55.92)(-133.67,56.33)(-134.17,56.25)(-134.33,56.75)(-133.83,57.08)(-133,56.92)(-133.42,57.25)(-133.75,57.58)(-134,57.25)(-134.5,57)(-134.5,57.42)(-134.75,57.92)(-134.67,58.25)(-135,58.67)(-135.17,58.25)(-135.67,58.25)(-135,57.92)(-135,57.42)(-134.75,56.83)(-134.75,56.17)(-134.75,56.17)(-135.17,56.75)(-135.5,57.25)(-136.17,57.67)(-136.5,58.17)(-137.25,58.42)(-137.92,58.75)(-138.67,59.17)(-139.5,59.42)(-139.67,59.92)(-140.25,59.67)(-141.08,59.75)(-142,60)(-143,60)(-144,60)(-145,60.25)(-146,60.58)(-146.67,60.83)(-147.5,60.83)(-148.33,61)(-148.17,60.42)(-148.67,59.92)(-149.5,59.92)(-150.17,59.58)(-150.92,59.25)(-151.92,59.17)(-151.42,59.5)(-151.92,59.67)(-151.42,60.17)(-151.33,60.75)(-150.42,60.92)(-150.42,60.92)(-150.295975194561,61.0026726290314)(-150.171303389393,61.0852308333188)(-150.045979897661,61.1676736234305)(-149.92,61.25)(-150.107498456546,61.2503881235412)(-150.295000000001,61.2505174992267)(-150.482501543455,61.2503881235412)(-150.67,61.25)(-151.67,60.92)(-152.25,60.58)(-152.5,60.08)(-153.08,59.67)(-153.83,59.42)(-154.08,59.08)(-153.33,58.92)(-153.83,58.5)(-154.33,58.17)(-155.25,57.83)(-156.17,57.42)(-156.67,57)(-157.67,56.67)(-158.42,56.42)(-158.75,56)(-159.58,55.83)(-160.5,55.58)(-161.42,55.42)(-162,55.17)(-162.83,55)(-163.67,54.67)(-164.75,54.33)(-164.92,54.58)(-163.92,55)(-162.92,55.25)(-162.33,55.67)(-161.42,55.92)(-160.5,56)(-160.5,56)(-159.92,56.5)(-159,56.83)(-158.33,57.25)(-157.67,57.58)(-157.5,58.17)(-157.42,58.75)(-158,58.67)(-158.67,58.75)(-158.92,58.42)(-159.58,58.92)(-161,58.92)(-161.58,58.75)(-161.92,59.17)(-161.67,59.5)(-162.17,60.08)(-163,59.67)(-163.83,59.67)(-164.33,60.08)(-164.92,60.58)(-165.25,61.08)(-166,61.5)(-165.5,62.17)(-164.83,62.67)(-164.67,63)(-163.92,63.25)(-163,63)(-162.25,63.42)(-161,63.5)(-160.75,64)(-161.25,64.42)(-161.25,64.42)(-161,64.83)(-161.92,64.67)(-162.67,64.42)(-163.67,64.58)(-164.92,64.5)(-166.33,64.67)(-166.67,65)(-166.33,65.33)(-167.25,65.42)(-167.394333937499,65.4602073667155)(-167.539111483918,65.5002769422319)(-167.684333292374,65.5402080471534)(-167.83,65.58)(-167.583047612547,65.6856092864369)(-167.334075929404,65.7908159197552)(-167.083066293475,65.8956146110974)(-166.83,66)(-165.75,66.25)(-164.67,66.5)(-163.83,66.5)(-163.75,66.08)(-162.17,66.08)(-161.75,66.42)(-162.5,66.83)(-163.5,67.17)(-164,67.67)(-165.25,67.98)(-166.17,68.33)(-166,68.83)(-164.83,68.92)(-163.5,69.08)(-163,69.45)(-162.67,69.92)(-161.58,70.33)(-160.42,70.42)(-159,70.83)(-159,70.83)(-157.58,70.87)(-156.33,71.33)(-154.83,71.13)(-154.33,70.87)(-152.5,70.87)(-151.5,70.48)(-150.33,70.48)(-149,70.47)(-147.5,70.22)(-146.17,70.22)(-145.08,70)(-144,70.08)(-142.83,70.08)(-141.83,69.83)(-141,69.62)(-139.5,69.58)(-138.33,69.32)(-137.33,69)(-136,68.75)(-135.92,69.17)(-135.17,69.5)(-134.17,69.58)(-133.33,69.38)(-132.58,69.67)(-131.17,69.98)(-129.83,70.25)(-129.42,69.83)(-128.5,70)(-128,70.58)(-126.67,70.08)(-126.67,70.08)(-126.17,69.58)(-125.33,69.42)(-124.42,70)(-124.33,69.42)(-123.5,69.37)(-123,69.83)(-121.33,69.83)(-120.25,69.47)(-118.83,69.22)(-117.5,69)(-116,68.97)(-114.67,68.72)(-114.08,68.3)(-115.08,68.17)(-115.33,67.83)(-114.17,67.83)(-113,67.67)(-111.58,67.83)(-110.25,68)(-109.25,67.75)(-108.17,67.58)(-107.83,68)(-109,68.25)(-108.25,68.58)(-106.92,68.75)(-105.67,68.9)(-104.58,68.35)(-103,68.03)(-101.75,67.75)(-100,67.58)(-100,67.58)(-98.75,67.67)(-98.67,68.25)(-98,68.58)(-99.58,69)(-98.58,69.33)(-97.92,69.83)(-96.42,69.33)(-95.58,68.75)(-96.92,68.38)(-96.83,67.82)(-95.83,67.5)(-95.5,67.98)(-95.17,68)(-94.25,68.5)(-94.67,69)(-94,69.42)(-95.67,69.83)(-96.75,70.42)(-96.25,70.97)(-95.83,71.5)(-95,71.75)(-95,72.25)(-95,72.67)(-95.25,73.17)(-95.58,73.58)(-95.17,74)(-93.17,74.17)(-91.17,74.02)(-90.17,73.83)(-90.67,73.58)(-90.67,73.58)(-91.33,73.17)(-91.92,72.75)(-93.08,72.75)(-94,72.58)(-93.5,72.25)(-94.17,71.92)(-93.5,71.42)(-92.67,71)(-91.92,70.67)(-91.33,70.17)(-92.42,69.75)(-91.33,69.5)(-90.67,69.08)(-89.92,68.58)(-88.92,69.25)(-88,68.75)(-88,68.33)(-88.17,67.83)(-87.25,67.25)(-86.5,67.42)(-86.5,67.83)(-85.42,68.17)(-85.42,68.75)(-85.5,69.33)(-85.5,69.83)(-84.25,69.83)(-82.92,69.67)(-82.75,69.33)(-81.33,69.2)(-81.33,68.58)(-81.33,68.58)(-82.42,68.33)(-82.42,67.83)(-81.17,67.58)(-81.42,67)(-82.67,66.58)(-83.75,66.17)(-85.08,66.33)(-85.5,66.58)(-86.75,66.5)(-86,66.17)(-87,65.5)(-87,65)(-87.75,64.5)(-88.5,64)(-90,64)(-90.67,63.42)(-90.58,62.92)(-91.58,62.75)(-92.5,62.58)(-92.5,62.08)(-93.17,61.92)(-93.67,61.5)(-94.17,61)(-94.42,60.58)(-94.67,60.17)(-94.75,59.58)(-94.75,59.08)(-94.42,58.67)(-93.75,58.75)(-93.08,58.67)(-93.08,58.67)(-92.67,58.08)(-92.67,57.67)(-92.33,57.25)(-92.33,56.92)(-91.75,57.08)(-91,57.25)(-90,56.92)(-89,56.75)(-88.08,56.42)(-87.9338056093021,56.315255209495)(-87.7884116304059,56.2103392088403)(-87.6438118252163,56.1052536080741)(-87.5,56)(-87.3319494414551,55.9578413499627)(-87.1642655368941,55.9154545589934)(-86.996948869602,55.872840488445)(-86.83,55.83)(-86,55.67)(-85,55.25)(-84,55.25)(-83,55.25)(-82.17,55)(-82.42,54.25)(-82.17,53.83)(-82.17,53.42)(-82.33,52.92)(-81.67,52.5)(-81.25,52.08)(-80.75,51.75)(-80.42,51.33)(-79.75,51)(-79.83,51.33)(-79,51.67)(-78.42,52.17)(-78.83,52.67)(-78.83,53.17)(-78.83,53.17)(-78.92,53.75)(-79.17,54.17)(-79.5,54.58)(-78.75,54.83)(-78,55.17)(-77.25,55.58)(-76.75,56)(-76.75,56.67)(-77,57.25)(-77.17,57.67)(-77.5,58.25)(-78.08,58.42)(-78.67,58.75)(-78.58,59.08)(-77.92,59.17)(-77.42,59.58)(-77.42,60)(-77.58,60.5)(-78.08,60.75)(-77.75,61.17)(-77.6881124384374,61.2725428483386)(-77.6258194511825,61.3750573369968)(-77.5631167529266,61.4775431582423)(-77.5,61.58)(-77.6441197826654,61.6427292242723)(-77.7888247246953,61.7053063119304)(-77.9341173053397,61.7677302452488)(-78.08,61.83)(-78.08,62.25)(-77.42,62.5)(-76.42,62.42)(-75.5,62.25)(-74.83,62.25)(-73.67,62.42)(-72.92,62.17)(-72.25,61.83)(-72.25,61.83)(-71.5,61.58)(-71.75,61.25)(-71,61)(-70.25,61)(-69.5,60.92)(-69.67,60.5)(-69.42,60.08)(-69.5,59.58)(-69.5,59.25)(-69,58.83)(-68.42,58.75)(-67.75,58.17)(-66.75,58.42)(-66.42,58.75)(-65.67,59)(-65.25,59.42)(-65.58,59.75)(-65.17,60)(-64.67,60.33)(-64.33,60)(-63.83,59.5)(-63.25,59)(-62.83,58.5)(-62.33,58)(-61.83,57.67)(-61.83,57.33)(-61.33,57.08)(-61.75,56.75)(-61.75,56.25)(-61.33,55.92)(-61.33,55.92)(-60.5,55.67)(-60.33,55.25)(-59.25,55.17)(-58.92,54.83)(-58,54.75)(-57.874423219653,54.6876934406893)(-57.7492318102636,54.6252574583237)(-57.6244244963679,54.5626927476577)(-57.5,54.5)(-57.7087501516841,54.4180351975872)(-57.916665356893,54.3357119210838)(-58.1237478718261,54.2530326871059)(-58.33,54.17)(-58.079595617536,54.1482774680051)(-57.829456994954,54.1260359667913)(-57.5795898865283,54.1032764770967)(-57.33,54.08)(-57.25,53.67)(-56.5,53.67)(-55.83,53.17)(-56,52.75)(-55.67,52.08)(-56.17,51.83)(-56.83,51.5)(-57.83,51.42)(-58.67,51.17)(-59.08,50.83)(-59.5,50.5)(-60,50.17)(-60.92,50.17)(-61.67,50.08)(-62.17,50.25)(-62.83,50.25)(-63.42,50.25)(-64,50.33)(-64.75,50.25)(-65.42,50.25)(-66,50.25)(-66.75,50.08)(-66.75,50.08)(-67.17,49.75)(-67.33,49.33)(-68,49.33)(-68.67,48.92)(-69.25,48.58)(-69.67,48.17)(-70,47.83)(-70.42,47.5)(-70.75,47.17)(-71.17,46.83)(-70.5,47)(-70.08,47.33)(-69.67,47.67)(-69.25,48.08)(-68.67,48.42)(-68,48.67)(-67.42,48.83)(-66.75,49.08)(-66,49.17)(-65.25,49.25)(-64.58,49.08)(-64.25,48.83)(-64.25,48.5)(-64.83,48.25)(-65.33,48.08)(-66.33,48.08)(-65.67,47.58)(-64.83,47.75)(-65,47.33)(-64.92,46.75)(-64.92,46.75)(-64.58,46.25)(-64,46.08)(-63.25,45.75)(-62.58,45.67)(-62,45.83)(-61.5,45.75)(-61.5,46.17)(-61,46.58)(-60.67,47)(-60.42,46.83)(-60.5,46.33)(-60.3317456714901,46.2478676707003)(-60.1639952032256,46.1654892673137)(-59.996747136637,46.0828662318686)(-59.83,46)(-59.9781567837507,45.9177852218078)(-60.1258746786771,45.8353795551792)(-60.2731552330809,45.7527841123611)(-60.42,45.67)(-61,45.5)(-61.17,45.17)(-62,45)(-63,44.75)(-63.67,44.5)(-64.17,44.5)(-64.5,44.17)(-65,43.83)(-65.5,43.5)(-66,43.75)(-66.17,44.17)(-65.92,44.58)(-65.33,44.92)(-64.92,45.08)(-64.25,45.42)(-64.92,45.33)(-64.83,45.67)(-64.83,45.67)(-65.58,45.33)(-66.25,45.17)(-67,45.17)(-67,44.83)(-67.5,44.58)(-68,44.42)(-68.58,44.25)(-68.83,44.42)(-69.17,44)(-69.75,43.83)(-70.25,43.67)(-70.25,43.5)(-70.58,43.17)(-70.83,42.83)(-70.75,42.58)(-71,42.33)(-70.67,42.17)(-70.5,41.75)(-70,41.79)(-70.04,42.02)(-69.93,41.67)(-70.67,41.5)(-70.67,41.67)(-71.08,41.5)(-71.33,41.75)(-71.5,41.42)(-72,41.25)(-72.5,41.25)(-73,41.25)(-73.5,41.08)(-73.5,41.08)(-74,40.83)(-74.33,40.5)(-74,40.33)(-74.17,39.83)(-74.5,39.33)(-75,39)(-75.25,39.33)(-75.58,39.5)(-75.42,39)(-75.314359267803,38.8551412854205)(-75.2091481363962,38.7101878005004)(-75.1043629286282,38.5651404175513)(-75,38.42)(-75.0626616852532,38.3575498042996)(-75.1252153336054,38.2950663179822)(-75.1876613154244,38.2325496728087)(-75.25,38.17)(-75.58,37.75)(-75.75,37.25)(-76,37.17)(-75.83,37.67)(-75.67,38)(-76,38)(-75.92,38.25)(-76.33,38.42)(-76.17,38.83)(-76.17,39.17)(-76,39.5)(-76.42,39.33)(-76.58,39)(-76.5,38.5)(-76.42,38.17)(-76.92,38.25)(-76.33,37.92)(-76.42,37.5)(-76.42,37.17)(-76.42,37.17)(-76,36.83)(-75.83,36.5)(-75.66,36.11)(-75.53,35.85)(-75.91,36.21)(-76,36.17)(-76.33,36.08)(-76.75,35.92)(-76.25,35.92)(-75.83,35.92)(-75.83,35.58)(-76,35.33)(-76.33,35.33)(-76.83,35)(-76.33,35)(-76.58,34.75)(-77.17,34.67)(-77.58,34.33)(-78,34)(-78.42,33.92)(-78.92,33.67)(-79.17,33.33)(-79.5,33)(-80,32.67)(-80.5,32.42)(-81,32)(-81.25,31.5)(-81.5,31)(-81.42,30.5)(-81.33,30)(-81.33,30)(-81.08,29.5)(-80.92,29)(-80.58,28.5)(-80.58,28.17)(-80.33,27.58)(-80.08,27)(-80.08,26.42)(-80.17,25.83)(-80.42,25.33)(-80.67,25.17)(-81.08,25.08)(-81.25,25.5)(-81.67,25.83)(-81.92,26.33)(-82.17,26.75)(-82.5,27.08)(-82.519954234216,27.1650042553486)(-82.5399388743637,27.2500056835862)(-82.5599540771468,27.3350042700481)(-82.58,27.42)(-82.5401369312649,27.5450171836611)(-82.5001830318643,27.6700229695792)(-82.4601376176982,27.7950172708601)(-82.42,27.92)(-82.5025513928908,27.8975736775494)(-82.5850685425664,27.8750981920465)(-82.6675514208719,27.8525736105052)(-82.75,27.83)(-82.75,28.17)(-82.67,28.58)(-82.83,29.08)(-83.08,29.17)(-83.5,29.75)(-83.92,30.08)(-84.42,30)(-84.83,29.75)(-85.33,29.67)(-85.5,30.08)(-86,30.25)(-86,30.25)(-86.33,30.5)(-86.83,30.42)(-87.33,30.33)(-87.83,30.33)(-88,30.67)(-88.25,30.42)(-88.75,30.5)(-89.17,30.5)(-89.67,30.25)(-90.17,30.5)(-90.42,30.08)(-90.08,30)(-89.67,30.08)(-89.25,30.08)(-89.33,29.83)(-89.58,29.67)(-89.58,29.5)(-89.08,29.17)(-89.25,29)(-89.5,29.25)(-89.83,29.33)(-90.17,29.17)(-90.75,29.08)(-91.25,29.17)(-91.33,29.5)(-91.83,29.83)(-92.25,29.58)(-92.75,29.67)(-93.25,29.83)(-93.83,29.75)(-93.83,29.75)(-94.25,29.67)(-94.75,29.42)(-94.75,29.83)(-94.8325987940275,29.7900768276896)(-94.9151316754455,29.7501023543806)(-94.9975987190326,29.7100767039162)(-95.08,29.67)(-95.0173055158737,29.5650438851623)(-94.9547412123031,29.4600583897768)(-94.8923063015243,29.3550436997799)(-94.83,29.25)(-94.9151551336428,29.187580523613)(-95.0002066434796,29.1251072295714)(-95.0851548316019,29.0625803208878)(-95.17,29)(-95.5,28.75)(-96,28.67)(-96.5,28.33)(-97,28)(-97.33,27.67)(-97.42,27.17)(-97.25,26.67)(-97.25,26.17)(-97.17,25.75)(-97.42,25.33)(-97.58,24.83)(-97.75,24.33)(-97.75,23.83)(-97.75,23.33)(-97.75,22.83)(-97.92,22.42)(-97.75,21.92)(-97.25,21.58)(-97.42,21.25)(-97.25,20.83)(-97,20.5)(-96.67,20.17)(-96.42,19.67)(-96.17,19.25)(-96.17,19.25)(-95.83,18.83)(-95.33,18.67)(-94.83,18.5)(-94.5,18.17)(-94.17,18.17)(-93.58,18.33)(-93,18.42)(-92.42,18.67)(-91.83,18.58)(-91.83,18.42)(-91.25,18.58)(-91.33,18.83)(-91,19.08)(-90.67,19.25)(-90.67,19.67)(-90.5,20)(-90.42,20.42)(-90.33,21)(-90,21.17)(-89.5,21.25)(-89,21.33)(-88.5,21.5)(-88,21.58)(-87.58,21.5)(-87.08,21.5)(-86.83,21.33)(-86.83,20.92)(-87.25,20.5)(-87.42,20)(-87.4400471300025,19.8750033398573)(-87.460062664131,19.7500044406535)(-87.4800468665453,19.6250033211481)(-87.5,19.5)(-87.5,19.5)(-87.5,19.5)(-87.5,19.5)(-87.5,19.5)(-87.5200458424419,19.375003268755)(-87.5400609500227,19.2500043459626)(-87.5600455829308,19.1250032502132)(-87.58,19)(-87.67,18.42)(-87.92,17.92)(-88.17,18.33)(-88.25,17.5)(-88.25,17)(-88.5,16.33)(-89,16)(-88.58,15.83)(-88.17,15.58)(-87.58,15.83)(-87,15.67)(-86.5,15.75)(-86,15.92)(-85.5,15.83)(-85,15.92)(-84.67,15.83)(-84.25,15.75)(-83.83,15.33)(-83.42,15.17)(-83.25,15)(-83.25,14.25)(-83.5,13.83)(-83.5,13.25)(-83.5,12.75)(-83.67,12.17)(-83.67,11.67)(-83.83,11.25)(-83.67,10.83)(-83.42,10.42)(-83.42,10.42)(-83.17,10)(-82.75,9.67)(-82.33,9.42)(-82.17,9)(-81.75,9)(-81.42,8.75)(-80.83,8.92)(-80.33,9.08)(-79.83,9.25)(-79.58,9.58)(-79.08,9.58)(-78.67,9.42)(-78.25,9.33)(-77.83,9)(-77.5,8.67)(-77.17,8.33)(-76.75,7.83)(-76.75,8.58)(-76.25,9)(-76,9.42)(-75.58,9.5)(-75.58,10)(-75.5,10.5)(-75.25,10.83)(-74.83,11)(-74.42,10.92)(-74.17,11.25)(-73.5,11.17)(-73,11.42)(-72.67,11.67)(-72.67,11.67)(-72.25,11.83)(-72,12.17)(-71.67,12.42)(-71.33,12.33)(-71.17,12.08)(-71.42,11.67)(-71.92,11.5)(-71.67,11)(-71.5,10.58)(-71.75,10.25)(-72,9.75)(-71.75,9.42)(-71.67,9)(-71.08,9.17)(-71.08,9.75)(-71.42,10.33)(-71.42,10.92)(-71,11.08)(-70.58,11.25)(-70.17,11.5)(-70.33,11.92)(-69.92,12.17)(-69.75,11.5)(-69.33,11.5)(-68.83,11.42)(-68.42,11.17)(-68.33,10.75)(-68.17,10.42)(-67.58,10.5)(-67,10.5)(-67,10.5)(-66.58,10.58)(-66.17,10.58)(-66,10.33)(-65.58,10.17)(-65,10.08)(-64.5,10.25)(-64,10.67)(-63.42,10.67)(-62.75,10.67)(-62.6050000318369,10.6701001527273)(-62.4600000000003,10.6701335370459)(-62.3149999681637,10.6701001527273)(-62.17,10.67)(-62.3775863242996,10.6277038398687)(-62.585115053832,10.5852713900106)(-62.7925862557096,10.5427032448294)(-63,10.5)(-62.67,10.17)(-62.33,9.75)(-62,9.83)(-61.58,9.67)(-61,9.5)(-60.83,9)(-61,8.42)(-60.25,8.58)(-59.67,8.25)(-59.17,7.92)(-58.67,7.58)(-58.5,7.17)(-58.58,6.75)(-58.17,6.83)(-57.67,6.5)(-57.33,6.25)(-57.08,5.92)(-56.5,5.83)(-56,5.83)(-56,5.83)(-55.5,5.92)(-55,5.92)(-54.5,5.83)(-54,5.67)(-53.67,5.58)(-53.25,5.42)(-52.75,5.17)(-52.33,4.83)(-51.83,4.5)(-51.5,4.25)(-51,3.75)(-51,3.17)(-50.67,2.5)(-50.5,1.83)(-49.92,1.75)(-50,1)(-50.42,0.5)(-50.75,0.17)(-51.17,-0.17)(-51.33,-0.67)(-51.42,-1.25)(-50.83,-1)(-50.58,-1.67)(-50.42,-1)(-50.42,-0.42)(-50,-0.17)(-49.5,-0.25)(-49,-0.17)(-48.42,-0.33)(-48.58,-1)(-48.58,-1)(-48.75,-1.42)(-48.5,-1.58)(-48.17,-0.83)(-47.83,-0.67)(-47.25,-0.67)(-46.67,-1)(-46.17,-1.17)(-45.5,-1.33)(-45.33,-1.67)(-44.83,-1.5)(-44.67,-1.92)(-44.42,-2.33)(-44.58,-2.83)(-44,-2.67)(-43.5,-2.42)(-43,-2.5)(-42.5,-2.75)(-42,-2.75)(-41.5,-3)(-41,-3)(-40.5,-2.83)(-40,-2.83)(-39.5,-3.17)(-39,-3.5)(-38.5,-3.83)(-38.17,-4.17)(-37.83,-4.5)(-37.33,-4.75)(-37,-5.08)(-36.5,-5.17)(-36.5,-5.17)(-36,-5.08)(-35.42,-5.25)(-35.17,-5.83)(-35,-6.42)(-34.83,-7)(-34.75,-7.5)(-34.83,-8.17)(-35.08,-8.67)(-35.33,-9.17)(-35.58,-9.58)(-36,-10)(-36.25,-10.42)(-36.75,-10.75)(-37,-11)(-37.33,-11.5)(-37.58,-12)(-38,-12.58)(-38.42,-13)(-38.92,-13.33)(-39,-14)(-39.08,-14.5)(-39,-15)(-39,-15.5)(-38.92,-16)(-39.08,-16.58)(-39.17,-17.17)(-39.17,-17.67)(-39.5,-18)(-39.58,-18.5)(-39.58,-19)(-39.58,-19)(-39.67,-19.5)(-40.17,-20)(-40.33,-20.5)(-40.83,-21)(-41,-21.5)(-41,-22)(-41.5,-22.25)(-42,-22.5)(-42,-22.92)(-42.5,-22.92)(-43,-23)(-43.5,-23)(-44,-22.92)(-44.67,-23)(-44.5,-23.33)(-45,-23.33)(-45.42,-23.83)(-45.75,-23.67)(-46.33,-24)(-46.83,-24.17)(-47.25,-24.58)(-47.75,-25)(-48.17,-25.42)(-48.58,-25.83)(-48.58,-26.33)(-48.58,-27)(-48.5,-27.5)(-48.58,-28)(-48.75,-28.5)(-49,-28.67)(-49,-28.67)(-49.42,-29)(-49.83,-29.5)(-50.17,-30)(-50.33,-30.5)(-50.67,-31)(-51,-31.5)(-51.5,-31.83)(-52,-32.17)(-52.42,-32.5)(-52.58,-33)(-53,-33.5)(-53.5,-33.83)(-53.58,-34.17)(-54,-34.5)(-54.5,-34.75)(-55,-34.83)(-55.67,-34.75)(-56.25,-34.83)(-56.83,-34.67)(-57.17,-34.42)(-57.83,-34.42)(-58.33,-34)(-58.5,-34.33)(-58.33,-34.67)(-57.5,-35)(-57.17,-35.33)(-57.42,-35.83)(-57.17,-36.25)(-56.75,-36.33)(-56.67,-36.83)(-56.67,-36.83)(-57,-37.33)(-57.42,-37.75)(-57.58,-38.17)(-58.17,-38.42)(-58.67,-38.58)(-59.33,-38.75)(-60,-38.83)(-60.5,-38.92)(-61.25,-39)(-62,-39)(-62.25,-38.75)(-62.33,-39.17)(-62,-39.5)(-62.17,-39.83)(-62.42,-40.33)(-62.25,-40.58)(-62.33,-40.92)(-63,-41.17)(-63.83,-41.17)(-64.33,-41)(-64.83,-40.75)(-64.9147595081485,-40.812593468583)(-64.9996790132277,-40.8751247948558)(-65.084759011538,-40.9375937239525)(-65.17,-41)(-65.1277432858537,-41.1250233770185)(-65.0853252574962,-41.2500312548644)(-65.0427446035575,-41.3750235055943)(-65,-41.5)(-65,-42.08)(-64.5,-42.33)(-63.75,-42.08)(-63.58,-42.58)(-63.67,-42.83)(-64.17,-42.83)(-64.5,-42.5)(-64.5,-42.5)(-65,-42.83)(-64.33,-43)(-65,-43.33)(-65.25,-43.67)(-65.17,-44)(-65.17,-44.5)(-65.58,-44.67)(-65.58,-45)(-66.17,-45)(-66.83,-45.25)(-67.17,-45.5)(-67.5,-46)(-67.42,-46.5)(-67,-46.83)(-66.5,-47.17)(-65.75,-47.17)(-65.67,-47.5)(-65.83,-48)(-66.33,-48.33)(-67,-48.67)(-67.5,-49)(-67.67,-49.5)(-67.83,-50)(-68.5,-50.17)(-69,-50.5)(-69.17,-51)(-69,-51.42)(-68.83,-51.83)(-68.33,-52.33)(-69,-52.25)(-69,-52.25)(-69.5,-52.25)(-69.58,-52.5)(-70,-52.5)(-70.75,-52.75)(-70.92,-53.25)(-70.92,-53.75)(-71.25,-53.83)(-72,-53.67)(-72.25,-53.42)(-71.33,-53.17)(-71.08,-52.83)(-71.5,-52.75)(-72,-53.17)(-72.5,-53.42)(-73.17,-53.17)(-73.5,-52.92)(-73.5,-52.67)(-74,-52.58)(-74.17,-52.17)(-74.5,-52.42)(-75,-52.25)(-75,-51.75)(-74.33,-51.83)(-74.17,-51.5)(-74.33,-51)(-74.5,-51.33)(-74.7066474406907,-51.3930484462007)(-74.9138635808113,-51.4557324765105)(-75.1216479425486,-51.5180502688262)(-75.33,-51.58)(-75.246944442522,-51.4775871316498)(-75.1642611280328,-51.375115860134)(-75.0819472433073,-51.2725866597267)(-75,-51.17)(-74.83,-50.67)(-74.5,-50.75)(-74.5,-50.75)(-74.67,-50.42)(-74.5,-50)(-75.33,-50.75)(-75.33,-50.17)(-75.247177316949,-50.1075878562742)(-75.164570312725,-50.0451169535423)(-75.0821781515034,-49.9825875744099)(-75,-49.92)(-75.1251744780687,-49.8977016928952)(-75.2502329677141,-49.8752687691246)(-75.3751749724113,-49.8527014605738)(-75.5,-49.83)(-75.42,-49.42)(-75.58,-49.17)(-75.42,-48.58)(-75.42,-48.17)(-75.33,-47.83)(-74.92,-47.75)(-74.5,-48.33)(-74.5,-47.83)(-74.42,-47.42)(-74,-46.83)(-74.5,-46.83)(-75,-46.67)(-75.5,-46.92)(-75.5,-46.58)(-74.75,-46.17)(-75,-45.92)(-74.5,-45.83)(-74.67,-45.58)(-74.5,-45.25)(-74.33,-45)(-74.5,-44.58)(-74.25,-44.17)(-73.83,-43.83)(-73.5,-44.17)(-73.5,-44.17)(-73.67,-44.42)(-73.58,-44.67)(-73.83,-45)(-73.67,-45.17)(-73.58,-45.5)(-73.33,-45)(-73.42,-44.67)(-73,-44.42)(-73.17,-44.17)(-72.83,-43.75)(-73,-43.42)(-72.75,-43)(-72.75,-42.33)(-72.6873810265864,-42.2900509425956)(-72.6248414650984,-42.2500678630034)(-72.562381171068,-42.2100508519644)(-72.5,-42.17)(-72.5626256885556,-42.1275509162072)(-72.6251674712363,-42.0850678242299)(-72.6876255183144,-42.0425508202015)(-72.75,-42)(-72.6872579029007,-41.9175509135654)(-72.6246777486531,-41.8350677608861)(-72.56225871897,-41.7525507280541)(-72.5,-41.67)(-72.625246803806,-41.6277032699699)(-72.7503291135976,-41.5852707729349)(-72.8752468655119,-41.5427028894544)(-73,-41.5)(-73.17,-41.83)(-73.58,-41.75)(-73.75,-41.42)(-73.83,-41)(-73.67,-40.58)(-73.67,-40)(-73.25,-39.5)(-73.33,-39)(-73.42,-38.58)(-73.42,-38.17)(-73.67,-37.75)(-73.5,-37.17)(-73.08,-37.17)(-73.08,-37.17)(-73.08,-36.67)(-72.83,-36.5)(-72.67,-36)(-72.5,-35.5)(-72.17,-35)(-71.92,-34.5)(-71.83,-34)(-71.58,-33.67)(-71.67,-33.17)(-71.33,-32.5)(-71.5,-32.17)(-71.5,-31.83)(-71.58,-31.33)(-71.67,-30.75)(-71.58,-30.25)(-71.25,-30)(-71.33,-29.33)(-71.5,-29)(-71.25,-28.5)(-71.17,-28)(-70.92,-27.5)(-70.92,-27)(-70.58,-26.33)(-70.75,-25.75)(-70.5,-25.42)(-70.5,-25)(-70.58,-24.5)(-70.5,-24)(-70.5,-23.5)(-70.5,-23)(-70.5,-23)(-70.25,-22.5)(-70.17,-22)(-70.08,-21.5)(-70.17,-21)(-70.17,-20.5)(-70.08,-20)(-70.25,-19.33)(-70.33,-18.75)(-70.33,-18.25)(-70.83,-17.83)(-71.17,-17.67)(-71.5,-17.25)(-72,-17)(-72.5,-16.67)(-72.83,-16.58)(-73.5,-16.25)(-74,-15.83)(-74.5,-15.75)(-75.17,-15.33)(-75.5,-15)(-75.83,-14.67)(-76.25,-14.17)(-76.17,-13.67)(-76.5,-13)(-76.67,-12.5)(-77,-12.17)(-77.17,-11.83)(-77.5,-11.33)(-77.67,-10.83)(-78.17,-10.17)(-78.17,-10.17)(-78.33,-9.67)(-78.5,-9.17)(-78.67,-8.67)(-79,-8.17)(-79.5,-7.83)(-79.67,-7.25)(-80,-6.83)(-80.33,-6.5)(-80.83,-6.25)(-81.17,-6)(-80.83,-5.67)(-81.17,-5.17)(-81.33,-4.67)(-81.33,-4.25)(-80.83,-3.83)(-80.33,-3.5)(-80,-3.33)(-79.83,-2.83)(-79.67,-2.5)(-80.08,-2.5)(-80.17,-3)(-80.58,-2.5)(-80.92,-2.33)(-80.75,-2.17)(-80.75,-1.67)(-80.92,-1)(-80.5,-0.83)(-80.5,-0.33)(-80.08,0) 
};
\addplot [
color=black,
solid,
forget plot
]
coordinates{
 (-125.08,66.08)(-123.58,66.13)(-122,66.25)(-121,66)(-122,65.75)(-122.75,65.58)(-123.25,65)(-121.75,65)(-121.58,65.33)(-120.67,65.67)(-120.5,65.33)(-121,64.83)(-119.33,65.37)(-120,65.82)(-118.42,65.67)(-118.08,66.08)(-117.67,66.42)(-119.08,66.33)(-119.434033315195,66.3537102142424)(-119.788725080211,66.3766151739887)(-120.144054368942,66.3987125281792)(-120.5,66.42)(-120.313912132169,66.4828351664328)(-120.126885632702,66.5454480837195)(-119.938916309916,66.6078369623189)(-119.75,66.67)(-118.92,66.92)(-120,67.08)(-121.33,66.75)(-122.67,66.58)(-123.83,66.37)(-125.08,66.08) 
};
\addplot [
color=black,
solid,
forget plot
]
coordinates{
 (-117,61.17)(-116,60.83)(-115.25,60.83)(-114.42,61)(-113.92,60.92)(-113.75,61.25)(-112.92,61.42)(-112.17,61.58)(-111.75,62.08)(-110.92,62.33)(-110.648117768157,62.3532979933177)(-110.375817924255,62.3760648675512)(-110.103109095266,62.3982993013346)(-109.83,62.42)(-109.91446570352,62.482577207951)(-109.999286169917,62.5451031795658)(-110.08446354781,62.6075775621295)(-110.17,62.67)(-109.877506016488,62.6709135608646)(-109.585000000001,62.6712180879742)(-109.292493983513,62.6709135608646)(-109,62.67)(-110,62.83)(-110.75,62.83)(-111.58,62.67)(-111.92,62.42)(-112.42,62.08)(-113.08,62)(-113.83,62.17)(-114.17,62.42)(-115.33,62.42)(-115.08,62.17)(-114.933783300335,62.0852288563474)(-114.788381715384,62.0003042106129)(-114.643789263544,61.9152274631708)(-114.5,61.83)(-114.687864740793,61.8103836669833)(-114.87548805629,61.7905111923247)(-115.062867337339,61.7703831201436)(-115.25,61.75)(-115.33,61.42)(-115.83,61.17)(-117,61.17) 
};
\addplot [
color=black,
solid,
forget plot
]
coordinates{
 (-111.25,58.67)(-110.33,58.58)(-110,58.92)(-109.25,59.08)(-108.25,59.08)(-108.043274454248,59.1229954095684)(-107.836031373306,59.1656614661962)(-107.62827259193,59.2079967889383)(-107.42,59.25)(-107.564455878864,59.292741135793)(-107.709274404741,59.3353219641819)(-107.854455731907,59.3777418107737)(-108,59.42)(-108.58,59.42)(-109.17,59.67)(-109.67,59.67)(-110.17,59.25)(-110.67,59)(-111.25,58.67) 
};
\addplot [
color=black,
solid,
forget plot
]
coordinates{
 (-103.08,56.33)(-102.17,56.58)(-102.17,57)(-102.17,57.42)(-101.58,57.67)(-102.08,58.17)(-102.5,57.75)(-102.583062559044,57.6675808404474)(-102.665748312672,57.5851075100487)(-102.748059915405,57.5025804256103)(-102.83,57.42)(-102.746793760951,57.3150813841502)(-102.664061234883,57.2101081608723)(-102.581798079174,57.1050808587671)(-102.5,57)(-102.562914115913,56.9175469483072)(-102.625550834669,56.8350624402057)(-102.687912140456,56.752546712563)(-102.75,56.67)(-103.08,56.33) 
};
\addplot [
color=black,
solid,
forget plot
]
coordinates{
 (-99.25,53.33)(-98.92,53)(-98.67,52.5)(-98.17,52.25)(-98.17,51.92)(-97.42,51.92)(-97.33,51.42)(-96.83,51.58)(-96.92,51.17)(-97,50.75)(-97,50.33)(-96.42,50.5)(-96.17,51.17)(-96.58,51.58)(-97,52.17)(-97.25,52.67)(-97.58,53.17)(-97.83,53.67)(-98.42,53.75)(-99,53.75)(-99.25,53.33) 
};
\addplot [
color=black,
solid,
forget plot
]
coordinates{
 (-92.08,46.67)(-91.58,46.67)(-90.92,46.92)(-90.5,46.5)(-89.75,46.75)(-89,47)(-88.25,47.42)(-87.75,47.42)(-88.42,47.08)(-88.42,46.75)(-87.75,46.83)(-87.33,46.42)(-86.67,46.42)(-85.83,46.67)(-85,46.75)(-85,46.5)(-84.8750003130499,46.5002042509004)(-84.7500000000005,46.5002723347555)(-84.6249996869511,46.5002042509004)(-84.5,46.5)(-84.5621361165904,46.6050509125237)(-84.624513651179,46.7100680545239)(-84.6871343563797,46.815051169856)(-84.75,46.92)(-84.67,47.33)(-85,47.58)(-85,47.92)(-85.75,47.92)(-86.17,48.25)(-86.5,48.75)(-87.33,48.75)(-88.08,49)(-88.42,48.67)(-89.08,48.42)(-89.42,48.08)(-89.83,47.83)(-89.83,47.83)(-90.5,47.67)(-91,47.33)(-91.58,47)(-92.08,46.67) 
};
\addplot [
color=black,
solid,
forget plot
]
coordinates{
 (-88,44.58)(-87.8133266122204,44.6654592705956)(-87.6261029988252,44.7506135680812)(-87.4383278802626,44.8354610832317)(-87.25,44.92)(-87.312840944691,44.8150512540212)(-87.3754535371956,44.710068172588)(-87.4378393643276,44.6050510054061)(-87.5,44.5)(-87.5200845708254,44.4175052444892)(-87.5401125486898,44.3350069793813)(-87.5600842527632,44.2525052246176)(-87.58,44.17)(-87.75,43.58)(-87.92,43.17)(-87.83,42.75)(-87.83,42.17)(-87.7470139069391,42.0450888238191)(-87.6643535676617,41.9201181039613)(-87.5820164390243,41.7950883333069)(-87.5,41.67)(-87.3950001558688,41.6701433433428)(-87.2900000000001,41.6701911245579)(-87.1849998441314,41.6701433433429)(-87.08,41.67)(-86.67,41.83)(-86.42,42.17)(-86.25,42.75)(-86.33,43.25)(-86.5,43.67)(-86.5,44.08)(-86.25,44.5)(-86.08,44.92)(-85.33,45.17)(-85,45.67)(-84.8750003044375,45.6702044753)(-84.7500000000002,45.6702726339522)(-84.6249996955629,45.6702044753)(-84.5,45.67)(-84.3947614639943,45.6276444310506)(-84.2896820294391,45.585192382409)(-84.1847615807952,45.5426441426711)(-84.08,45.5)(-83.58,45.33)(-83.33,45)(-83.33,44.42)(-83.83,43.92)(-83.83,43.58)(-83.42,43.92)(-82.92,44.08)(-82.67,43.58)(-82.67,43.58)(-82.42,43.08)(-81.75,43.33)(-81.75,43.92)(-81.5,44.42)(-81.33,44.83)(-81.5,45.17)(-80.92,44.75)(-80.08,44.5)(-79.9978516461096,44.5825889171427)(-79.915469644271,44.6651187825557)(-79.8328528217291,44.747589257237)(-79.75,44.83)(-79.8740334076097,44.977703822537)(-79.9987071713761,45.1252726986667)(-80.1240273340882,45.27270522968)(-80.25,45.42)(-80.75,45.92)(-81.67,46.08)(-82.33,46.17)(-83,46.17)(-83.67,46.25)(-84.17,46.25)(-84,45.92)(-84.83,45.92)(-85.25,46)(-85.75,45.92)(-86.75,45.75)(-87,45.75)(-87.33,45.42)(-87.67,45)(-88,44.58) 
};
\addplot [
color=black,
solid,
forget plot
]
coordinates{
 (-83.08,45.83)(-82.33,45.83)(-81.75,45.92)(-81.92,45.5)(-82.42,45.67)(-83.08,45.83) 
};
\addplot [
color=black,
solid,
forget plot
]
coordinates{
 (-83.5,41.75)(-82.83,41.5)(-82.25,41.5)(-81.67,41.58)(-81.08,41.83)(-80.5,42)(-79.83,42.25)(-79.25,42.5)(-78.83,42.83)(-79.5,42.83)(-80.17,42.83)(-80.5,42.58)(-80.92,42.67)(-81.5,42.58)(-81.83,42.25)(-82.5,42)(-83.08,42)(-83.5,41.75) 
};
\addplot [
color=black,
solid,
forget plot
]
coordinates{
 (-79.75,43.25)(-79.25,43.17)(-78.58,43.33)(-78,43.33)(-77.42,43.25)(-76.83,43.33)(-76.17,43.5)(-76.25,44)(-76.08,44.33)(-76.67,44.17)(-77,43.92)(-77.67,44)(-78.17,43.92)(-78.83,43.83)(-79.42,43.67)(-79.75,43.25) 
};
\addplot [
color=black,
solid,
forget plot
]
coordinates{
 (-91.42,0.08)(-91.08,-0.33)(-90.83,-0.75)(-91.17,-1.08)(-91.5,-0.92)(-91.17,-0.67)(-91.5,-0.42)(-91.42,0.08) 
};
\addplot [
color=black,
solid,
forget plot
]
coordinates{
 (-74,-41.83)(-73.42,-42)(-73.33,-42.33)(-73.67,-42.67)(-73.5,-43)(-73.75,-43.42)(-74.33,-43.33)(-74.17,-42.67)(-74.17,-42.33)(-74,-41.83) 
};
\addplot [
color=black,
solid,
forget plot
]
coordinates{
 (-68.75,-52.58)(-68.25,-53)(-68.17,-53.33)(-68,-53.67)(-67.42,-54)(-66.75,-54.25)(-66.33,-54.5)(-65.83,-54.67)(-65.17,-54.67)(-65.25,-54.92)(-65.67,-55)(-66.5,-55)(-67.33,-55.25)(-68.17,-55.25)(-68,-55.67)(-68.67,-55.42)(-69.33,-55.5)(-69.83,-55.33)(-70,-55)(-70.5,-55.17)(-71.17,-55)(-72,-54.67)(-72,-54.67)(-72,-54.42)(-72.5,-54.42)(-72.67,-54.08)(-73.33,-54.08)(-73.33,-53.83)(-73.83,-53.58)(-73.5,-53.33)(-74,-53.25)(-74.5,-53)(-74.67,-52.75)(-74,-53)(-73.33,-53.25)(-72.5,-53.58)(-72.08,-53.83)(-71.33,-54)(-71,-54.17)(-70.5,-53.67)(-70.33,-54.17)(-70.2476198767059,-54.1900844809197)(-70.1651597765114,-54.210112705259)(-70.0826197877533,-54.2300845769545)(-70,-54.25)(-70.0201808059418,-54.125004997185)(-70.0402402255263,-54.0000066393877)(-70.0601795367065,-53.8750049620177)(-70.08,-53.75)(-69.8916668189764,-53.6879402615448)(-69.7038892929089,-53.6255859900808)(-69.51666712907,-53.5629387244)(-69.33,-53.5)(-69.42,-53.33)(-70.17,-53.5)(-70.42,-53.33)(-70.25,-52.83)(-69.75,-52.83)(-69.5,-52.58)(-69.17,-52.67)(-68.75,-52.58) 
};
\addplot [
color=black,
solid,
forget plot
]
coordinates{
 (-61.17,-51.83)(-60.5,-52)(-60.17,-51.75)(-60.5,-51.42)(-59.67,-51.33)(-59,-51.5)(-58.33,-51.33)(-57.75,-51.5)(-57.75,-51.83)(-58.5,-52)(-58.83,-52.25)(-59.5,-52.33)(-59.67,-52.08)(-60,-52)(-60.5,-52.25)(-61.17,-51.83) 
};
\addplot [
color=black,
solid,
forget plot
]
coordinates{
 (-38,-54)(-37,-54.08)(-36.25,-54.33)(-35.83,-54.58)(-36,-54.92)(-36.58,-54.5)(-37.25,-54.25)(-38,-54) 
};
\addplot [
color=black,
solid,
forget plot
]
coordinates{
 (-61.83,10)(-61.5,10.17)(-61.5,10.58)(-61.67,10.67)(-61,10.75)(-61,10.25)(-61.25,10)(-61.83,10) 
};
\addplot [
color=black,
solid,
forget plot
]
coordinates{
 (-78.33,18.17)(-78.25,18.33)(-77.83,18.5)(-77.33,18.42)(-76.83,18.33)(-76.42,18.17)(-76.25,17.83)(-76.58,17.83)(-76.92,17.92)(-77.25,17.75)(-77.67,17.83)(-78,18.17)(-78.33,18.17) 
};
\addplot [
color=black,
solid,
forget plot
]
coordinates{
 (-83.07,21.77)(-82.87,21.98)(-82.65,21.82)(-82.6124433361097,21.7475126558478)(-82.5749245633057,21.6750168480891)(-82.5374435087201,21.6025126163181)(-82.5,21.53)(-82.5825424959051,21.5050607492187)(-82.6650566611557,21.4800809551418)(-82.7475424957677,21.4550606834863)(-82.83,21.43)(-83.05,21.47)(-82.95,21.6)(-83.07,21.77) 
};
\addplot [
color=black,
solid,
forget plot
]
coordinates{
 (-84.83,21.83)(-84.33,22)(-84.33,22.33)(-84,22.58)(-83.5,22.83)(-83,22.92)(-82.5,23)(-82,23.17)(-81.5,23.17)(-81,23)(-80.5,23)(-80.08,22.83)(-79.67,22.67)(-79.33,22.33)(-78.83,22.33)(-78.33,22.17)(-77.83,21.92)(-77.42,21.67)(-77,21.5)(-76.5,21.17)(-76,21.08)(-75.58,21)(-75.67,20.58)(-75.33,20.67)(-74.83,20.58)(-74.5,20.25)(-74.25,20.25)(-74.17,20.08)(-74.17,20.08)(-74.67,20)(-75,19.83)(-75.5,19.83)(-76,19.92)(-76.5,19.92)(-77,19.83)(-77.16749983526,19.8302343963114)(-77.3350000000001,19.830312528689)(-77.5025001647401,19.8302343963114)(-77.67,19.83)(-77.6075741663965,19.8925327601089)(-77.5450990153092,19.955043741238)(-77.4825743566357,20.0175328518024)(-77.42,20.08)(-77.08,20.33)(-77.25,20.67)(-78,20.67)(-78.5,21)(-78.58,21.42)(-78.75,21.58)(-79.25,21.5)(-79.75,21.67)(-80.17,21.75)(-80.5,22)(-81.25,22)(-81.75,22.08)(-82.08,22.33)(-81.58,22.5)(-82.25,22.58)(-82.75,22.67)(-83.08,22.42)(-83.42,22.17)(-84,22.17)(-84.0425765580477,22.0850164542984)(-84.0851018958192,22.0000218990967)(-84.1275762859245,21.9150163944045)(-84.17,21.83)(-84.3349998107833,21.8302460355041)(-84.5000000000005,21.8303280476282)(-84.6650001892177,21.8302460355041)(-84.83,21.83) 
};
\addplot [
color=black,
solid,
forget plot
]
coordinates{
 (-74.5,18.33)(-74.17,18.58)(-73.67,18.5)(-73.25,18.33)(-72.75,18.33)(-72.645114540159,18.3925865453814)(-72.5401528954061,18.4551155617534)(-72.4351148028543,18.5175867973508)(-72.33,18.58)(-72.4348040309855,18.685087801943)(-72.5397381236334,18.7901173528574)(-72.6448031539635,18.8950882277462)(-72.75,19)(-72.75,19.33)(-73,19.67)(-73,19.67)(-73.5,19.67)(-73.17,19.92)(-72.58,19.92)(-72,19.67)(-71.67,19.83)(-71,19.83)(-70.5,19.75)(-69.92,19.67)(-69.75,19.33)(-69.17,19.33)(-69.58,19.08)(-68.75,18.92)(-68.33,18.5)(-68.67,18.08)(-69,18.42)(-69.5,18.42)(-70.08,18.25)(-70.58,18.17)(-70.58,18.42)(-71.08,18.25)(-71.08,18)(-71.42,17.58)(-71.83,18.08)(-72.33,18.17)(-72.75,18.08)(-73.17,18.17)(-73.67,18.17)(-73.83,18)(-74.5,18.33) 
};
\addplot [
color=black,
solid,
forget plot
]
coordinates{
 (-67.17,18)(-67.17,18.5)(-66.58,18.5)(-66,18.5)(-65.67,18.25)(-65.92,18)(-66.33,18)(-66.75,17.92)(-67.17,18) 
};
\addplot [
color=black,
solid,
forget plot
]
coordinates{
 (-61.7519395710623,16.3688751081682)(-61.5520583856025,16.281084177942)(-61.5439566710806,16.4582246640639)(-61.4947246161506,16.5029640016532)(-61.4275619278705,16.4717323721938)(-61.3917853345298,16.3672086280307)(-61.1898210922401,16.2654349661684)(-61.4639967931012,16.177479316112)(-61.5943465512396,16.2367871833565)(-61.5661265947591,16.0384248946353)(-61.6919277799882,15.9542607522394)(-61.7601946929947,16.0688983098815)(-61.790493552906,16.3294871211233)(-61.7519395710623,16.3688751081682) 
};
\addplot [
color=black,
solid,
forget plot
]
coordinates{
 (-60.951078849822,14.5483862204799)(-61.1413264083548,14.8553121599439)(-61.1213363624138,14.9656321793812)(-60.9787423293317,14.9616208661169)(-60.8309965128035,14.7803051565814)(-60.7498614884859,14.5404253607997)(-60.951078849822,14.5483862204799) 
};
\addplot [
color=black,
solid,
forget plot
]
coordinates{
 (-78.25,25.17)(-78,25.17)(-77.75,24.67)(-77.75,24.33)(-77.6873786680155,24.247538261021)(-77.6248384847328,24.1650509272301)(-77.5623790587759,24.0825381299514)(-77.5,24)(-77.5425812417333,23.9175175094475)(-77.5851081409574,23.8350233057445)(-77.6275809699532,23.7525174492296)(-77.67,23.67)(-77.92,24.17)(-78.42,24.58)(-78.17,24.83)(-78.25,25.17) 
};
\addplot [
color=black,
solid,
forget plot
]
coordinates{
 (-73.67,20.92)(-73.5,21.17)(-73.17,21.08)(-73,21.33)(-73.17,20.92)(-73.67,20.92) 
};
\addplot [
color=black,
solid,
forget plot
]
coordinates{
 (-78.7880424535408,26.5650673955282)(-78.6800213103482,26.5914597207128)(-78.5757534204153,26.8000302870433)(-78.5222198674216,26.7467656883192)(-77.9000185082285,26.7474968925176)(-77.8692903143642,26.6033503094876)(-78.237938698261,26.6251602271793)(-78.7256702378887,26.4758950566538)(-78.7880424535408,26.5650673955282) 
};
\addplot [
color=black,
solid,
forget plot
]
coordinates{
 (-77.8707055365136,26.8874163795946)(-77.7723089030764,26.9323903835192)(-77.5532153852355,26.9096505348708)(-77.0443960488348,26.5530524821212)(-76.9677555111525,26.2887737772182)(-77.1278679435662,26.2609985668928)(-77.183795447204,25.8781082433424)(-77.2401031486272,25.8489932419356)(-77.3858997903128,25.9953657953991)(-77.2254268680678,26.119013653305)(-77.2625847993564,26.1833601730454)(-77.2208830225664,26.2706092404661)(-77.2372364598824,26.4374001221387)(-77.1452537522541,26.5537429678914)(-77.3198527587381,26.6159943541836)(-77.5773297912924,26.8522180179261)(-77.8707055365136,26.8874163795946) 
};
\addplot [
color=black,
solid,
forget plot
]
coordinates{
 (-76.7079244688259,25.564178230382)(-76.647621911446,25.5636609698507)(-76.6140248916536,25.4894184666273)(-76.2797766472629,25.2975649742391)(-76.1185422712376,25.148275251695)(-76.0972806133105,25.0494942284349)(-76.1645194220738,24.7520290310533)(-76.1316858515019,24.6403999304653)(-76.2097660214783,24.7462422045058)(-76.2899720455301,24.7971193190467)(-76.1943350699228,24.8399257713717)(-76.2046522982015,24.9190596890181)(-76.1354882632531,25.018140488891)(-76.1838375466383,25.1324038465551)(-76.2966435100082,25.2566221994092)(-76.7492810249881,25.4238879112604)(-76.7079244688259,25.564178230382) 
};
\addplot [
color=black,
solid,
forget plot
]
coordinates{
 (-75.7233610813439,24.6848056210295)(-75.6203435333992,24.6313530155198)(-75.2979296790122,24.1452191598339)(-75.3665197301533,24.1705498746618)(-75.5053198268863,24.1267989486499)(-75.4386094820872,24.2187345388334)(-75.5914101045463,24.4899899298181)(-75.6879880623594,24.5163010466313)(-75.7233610813439,24.6848056210295) 
};
\addplot [
color=black,
solid,
forget plot
]
coordinates{
 (-75.2971406109266,23.6925225741425)(-75.0780927394295,23.3716783491433)(-75.0256389485464,23.1621000298154)(-74.9373513544841,23.1307797423322)(-74.8087974668878,22.9059048553992)(-74.847202553531,22.8569565584847)(-74.9638845258824,23.0658002824714)(-75.2076689207778,23.143841548141)(-75.1493506063077,23.3546914704486)(-75.331412486647,23.6193741346449)(-75.2971406109266,23.6925225741425) 
};
\addplot [
color=black,
solid,
forget plot
]
coordinates{
 (-73.8579865322794,22.7389770200097)(-73.8081827149825,22.5610375833775)(-73.9569389271629,22.3158981592398)(-74.1653825195328,22.2507595612519)(-74.1498568455114,22.3160489656477)(-74.0363824425152,22.3806001801931)(-74.0162887022204,22.4703817896562)(-73.8946867435247,22.5600367098073)(-74.2731769248798,22.7342147386424)(-74.3477000269466,22.8546191643548)(-74.0269997071741,22.6966258309788)(-73.8579865322794,22.7389770200097) 
};
\addplot [
color=black,
solid,
forget plot
]
coordinates{
 (-74,40.58)(-73.92,40.83)(-73.33,40.92)(-72.83,41)(-72.33,41.17)(-72,41)(-72.67,40.75)(-73.33,40.67)(-74,40.58) 
};
\addplot [
color=black,
solid,
forget plot
]
coordinates{
 (-64.42,49.83)(-63.75,49.83)(-63.08,49.75)(-62.5,49.58)(-62,49.42)(-61.67,49.08)(-62.25,49.08)(-62.92,49.17)(-63.5,49.33)(-63.75,49.58)(-64.42,49.83) 
};
\addplot [
color=black,
solid,
forget plot
]
coordinates{
 (-64.25,46.67)(-64,46.92)(-63.92,46.58)(-63.33,46.42)(-62.75,46.42)(-62,46.5)(-62.5,46.25)(-62.5,45.92)(-63.33,46.17)(-63.92,46.42)(-64.25,46.67) 
};
\addplot [
color=black,
solid,
forget plot
]
coordinates{
 (-83.67,62.17)(-83.33,62.58)(-82.58,62.75)(-81.92,62.75)(-81.92,62.67)(-82.83,62.25)(-83.67,62.17) 
};
\addplot [
color=black,
solid,
forget plot
]
coordinates{
 (-80.08,62.33)(-79.25,62.33)(-79.33,61.92)(-79.75,61.5)(-80.17,61.75)(-80.08,62.33) 
};
\addplot [
color=black,
solid,
forget plot
]
coordinates{
 (-80,56.25)(-79.17,56.58)(-78.92,56.33)(-79.17,55.75)(-79.17,56.17)(-79.5,55.92)(-79.644999362922,55.9202554634611)(-79.7899999999999,55.9203406183794)(-79.9350006370777,55.9202554634611)(-80.08,55.92)(-79.9163159839294,56.0228288218997)(-79.7517587737869,56.1254397738347)(-79.586322183767,56.2278308438598)(-79.42,56.33)(-80,56.25) 
};
\addplot [
color=black,
solid,
forget plot
]
coordinates{
 (-82,53)(-81.42,53.17)(-81,53.08)(-80.75,52.67)(-81.42,52.83)(-82,53) 
};
\addplot [
color=black,
solid,
forget plot
]
coordinates{
 (-59.33,47.92)(-58.92,48.17)(-58.42,48.5)(-58.75,48.58)(-58.42,49)(-57.92,49.5)(-57.67,50.08)(-57.25,50.58)(-57,51)(-56.67,51.33)(-56,51.58)(-55.5,51.58)(-55.75,51.08)(-56.17,50.58)(-56.5,50.25)(-56.83,49.75)(-56.08,50.08)(-55.5,49.92)(-55.92,49.67)(-55.33,49.33)(-54.67,49.42)(-54,49.42)(-54,49.42)(-53.5,49.25)(-54,48.83)(-53.75,48.5)(-53,48.58)(-53.75,48.08)(-53.92,47.75)(-53.67,47.58)(-53.33,48)(-53,48.08)(-53.25,47.58)(-52.75,47.67)(-52.92,47.17)(-53.08,46.67)(-53.67,46.67)(-53.67,47.08)(-54.17,46.83)(-53.92,47.33)(-54.42,47.42)(-54.83,47.33)(-55.33,46.83)(-55.92,46.92)(-55.42,47.17)(-55.42,47.5)(-56.08,47.58)(-56.83,47.58)(-57.58,47.58)(-58.42,47.67)(-59.17,47.5)(-59.33,47.92) 
};
\addplot [
color=black,
solid,
forget plot
]
coordinates{
 (-87.17,63.67)(-86.17,64.08)(-86.42,64.67)(-86.25,65.25)(-86,65.67)(-85,66.08)(-84.67,65.58)(-83.67,65.17)(-82.5,64.83)(-81.75,64.5)(-81.25,64.08)(-80.25,63.83)(-81.17,63.42)(-82,63.75)(-82.83,64)(-83.58,64)(-84.17,63.67)(-84.83,63.25)(-85.75,63.17)(-85.83,63.67)(-87.17,63.67) 
};
\addplot [
color=black,
solid,
forget plot
]
coordinates{
 (-77.25,67.67)(-76.5,68.25)(-75.33,68.25)(-74.83,67.92)(-75,67.5)(-75.67,67.25)(-76.83,67.17)(-77.25,67.67) 
};
\addplot [
color=black,
solid,
forget plot
]
coordinates{
 (-90,71.75)(-90,72.33)(-89.58,72.75)(-89.08,73.17)(-88,73.58)(-87,73.83)(-85.42,73.83)(-85.0645320245947,73.8233835993553)(-84.7093595466696,73.8161775969027)(-84.3545073301758,73.8083827803029)(-84,73.8)(-84.379526186654,73.7334976944216)(-84.7560298480398,73.6663247720544)(-85.1295183729203,73.5984894774536)(-85.5,73.53)(-86.08,73.25)(-86.5,72.83)(-86.42,72.42)(-86.33,72)(-85.33,71.67)(-85.67,72.08)(-85.58,72.5)(-85.5,73)(-84.33,73.42)(-82.92,73.75)(-81.67,73.83)(-81.17,73.58)(-81.5,73.13)(-80.83,72.67)(-81.17,72.17)(-81.17,72.17)(-80.17,72.47)(-79,72.47)(-77.83,72.8)(-76.83,72.7)(-75.75,72.5)(-74.33,72.58)(-74,72.25)(-74,71.75)(-72.42,71.67)(-71.25,71.42)(-69.75,70.83)(-68.33,70.5)(-68.1186956623776,70.3753607284782)(-67.9099564422788,70.2504779245599)(-67.7037387893354,70.1253561849371)(-67.5,70)(-67.6065760232819,69.8950940530024)(-67.7120908492722,69.7901247516203)(-67.816560325681,69.6850930793401)(-67.92,69.58)(-67.6233148507322,69.4782420411892)(-67.3294385866509,69.3759844692327)(-67.0383430223729,69.2732346903519)(-66.75,69.17)(-67.0844303198915,69.1284670998657)(-67.4175816018666,69.086286851639)(-67.7494419795812,69.0434631707025)(-68.08,69)(-68.08,68.5)(-66.75,68.08)(-65.17,68.08)(-64.25,67.67)(-63.17,67.33)(-62.17,67)(-61.42,66.67)(-62.08,66.08)(-62.33,65.67)(-63.5,65.5)(-63.83,65)(-65,65.33)(-65.5,65.92)(-67.17,66.42)(-67.17,66.42)(-67.42,65.75)(-66.83,65.25)(-65.75,64.83)(-65.08,64.5)(-65,64)(-64.25,63.58)(-64.5,62.92)(-65.75,62.92)(-66.83,63.25)(-67.75,63.58)(-68.92,63.83)(-68.5,63.42)(-67.75,63.08)(-66.92,62.67)(-66.17,62.33)(-66.17,61.92)(-67.17,62)(-68.08,62.17)(-69,62.42)(-69.33,62.67)(-70.33,62.83)(-71.08,63)(-71.92,63.33)(-71.92,63.75)(-72.75,64)(-73.67,64.17)(-74.67,64.42)(-75.75,64.5)(-76.5,64.25)(-78,64.25)(-78,64.25)(-78.58,64.67)(-78.25,65.08)(-77.67,65.17)(-77.5,65.5)(-76,65.33)(-74.67,65.33)(-73.67,65.67)(-74.58,66.17)(-73.75,66.5)(-73,66.92)(-72.8342066980222,67.0027624632733)(-72.6672822028683,67.0853512147491)(-72.4992166244331,67.1677643646793)(-72.33,67.25)(-72.413879100483,67.3550666514849)(-72.4984986998715,67.4600892836331)(-72.5838689048446,67.5650672768093)(-72.67,67.67)(-73,68.17)(-74.25,68.5)(-75,69)(-76.17,68.67)(-75.83,69.33)(-77.5,69.83)(-78.08,70.25)(-79,69.75)(-79.83,69.58)(-81.33,70)(-82.33,69.92)(-84.33,70.08)(-85.75,70.08)(-87.33,70.42)(-88.25,70.83)(-89,71.33)(-90,71.75) 
};
\addplot [
color=black,
solid,
forget plot
]
coordinates{
 (-80.5,73.75)(-79.25,73.83)(-77.83,73.78)(-76.75,73.58)(-76.42,73.17)(-76.58,72.87)(-78.17,72.98)(-79.17,72.83)(-80.5,72.98)(-80.83,73.42)(-80.5,73.75) 
};
\addplot [
color=black,
solid,
forget plot
]
coordinates{
 (-96.58,75)(-95.67,75.55)(-94.5,75.55)(-93.75,75.2)(-93.75,74.67)(-95.33,74.82)(-96.58,75) 
};
\addplot [
color=black,
solid,
forget plot
]
coordinates{
 (-96.17,77.03)(-94.33,76.92)(-93,76.65)(-91.25,76.65)(-90.75,76.37)(-90.4771019880083,76.3204416184273)(-90.206142461367,76.2705866826931)(-89.9371117509498,76.2204384125846)(-89.67,76.17)(-89.7962576248564,76.1225958709229)(-89.9216718637893,76.0751273918633)(-90.04625018854,76.0275952188777)(-90.17,75.98)(-89.8098298919039,75.9107863949742)(-89.4531233624307,75.8410433508055)(-89.0998552797384,75.7707786550758)(-88.75,75.7)(-86.75,75.58)(-84.83,75.82)(-83.08,75.72)(-81.58,75.72)(-79.83,75.47)(-79.83,75.08)(-80.42,74.67)(-82.17,74.5)(-83.67,74.75)(-84.75,74.42)(-86.5,74.42)(-88.17,74.47)(-90,74.53)(-92,74.75)(-92.17,75.25)(-91.83,75.67)(-92.17,76.08)(-93.25,76.33)(-95.08,76.25)(-96.25,76.67)(-96.17,77.03) 
};
\addplot [
color=black,
solid,
forget plot
]
coordinates{
 (-95,80)(-95,80.45)(-93.75,80.8)(-93.58,81.25)(-92,81.25)(-91.33,80.8)(-90.25,80.5)(-88,80.42)(-87.7885064070389,80.3576884826279)(-87.5797059337473,80.2952496923913)(-87.3735522663571,80.2326860702897)(-87.17,80.17)(-87.2989220640512,80.0650709335665)(-87.4251755085079,79.9600935889943)(-87.5488420117774,79.8550694651185)(-87.67,79.75)(-87.2287357878862,79.7208848781443)(-86.789962542604,79.6911764206034)(-86.3537083932134,79.6608797425322)(-85.92,79.63)(-85.92,79.25)(-87.25,79)(-88.08,78.53)(-89.25,78.17)(-92,78.17)(-93.33,78.53)(-93.67,78.98)(-93.17,79.33)(-93.83,79.72)(-95,80) 
};
\addplot [
color=black,
solid,
forget plot
]
coordinates{
 (-90.5,81.5)(-89,81.92)(-86.67,81.92)(-86.67,81.92)(-86.67,82.13)(-85,82.33)(-82.58,82.58)(-80.75,82.92)(-76.83,83.05)(-74.5,82.98)(-72.5,83.08)(-69.75,83.08)(-67.17,82.92)(-64.42,82.8)(-62.92,82.55)(-61.17,82.42)(-61.17,82.2)(-62.5,81.95)(-64.25,81.67)(-64.33,81.42)(-66.75,80.95)(-68.83,80.58)(-70,80.2)(-71.5,79.72)(-74,79.5)(-75,79)(-75,78.5)(-75.83,78)(-78,77.92)(-77.67,77.58)(-79.25,77.25)(-77.75,76.83)(-79,76.42)(-81,76.17)(-81,76.17)(-81.58,76.5)(-82.67,76.33)(-84,76.42)(-85.17,76.25)(-87.75,76.33)(-89.5,76.42)(-89.5,76.83)(-88.17,77.13)(-87.8795232250299,77.1604831097556)(-87.5876943334351,77.1906456593506)(-87.2945181954161,77.2204853784799)(-87,77.25)(-87.2475280787676,77.2928488292626)(-87.4966990426658,77.3354666689745)(-87.7475205207071,77.3778511788585)(-88,77.42)(-88,77.83)(-87.3983760871943,77.8544640799208)(-86.7944241425026,77.8776237054737)(-86.1882594311378,77.8994713853611)(-85.58,77.92)(-86.0722757631521,77.9838192556119)(-86.5696952409989,78.0467682387755)(-87.0722678191949,78.1088331260219)(-87.1987169628735,78.1242095788564)(-87.3254886690791,78.1395296879541)(-87.4525829985504,78.1547932346599)(-87.58,78.17)(-87.4857506539161,78.2064946534432)(-87.39092475336,78.2429580988463)(-87.2955174165633,78.2793900448908)(-87.1995237133875,78.3157901968797)(-86.8095849144825,78.4610668495102)(-86.4098569400991,78.6058103116471)(-86,78.75)(-85.647824449343,78.7806263960583)(-85.2937586376395,78.8108374421294)(-84.9378133366375,78.8406297644133)(-84.58,78.87)(-84.7237359774167,78.9651051853301)(-84.8699389507124,79.060141465363)(-85.0186719251763,79.1551070283757)(-85.17,79.25)(-85.17,79.7)(-86.67,79.87)(-85.83,80.42)(-83.67,80.25)(-82.5,80.47)(-79.58,80.63)(-77.67,80.92)(-80.67,80.8)(-83.5,80.8)(-85.5,80.67)(-88.25,80.67)(-90,81)(-90.5,81.5) 
};
\addplot [
color=black,
solid,
forget plot
]
coordinates{
 (-99.5,79.9)(-98.83,80.17)(-97.67,79.75)(-99.5,79.9) 
};
\addplot [
color=black,
solid,
forget plot
]
coordinates{
 (-106.33,79.2)(-105.67,79.38)(-104.25,79.38)(-103,79.2)(-101.42,79.2)(-100,78.88)(-100,78.48)(-99.58,78.13)(-98.58,78.08)(-98.17,78.48)(-98,78.9)(-95.83,78.55)(-95.42,78.3)(-95.83,78)(-97.17,77.97)(-98.83,77.92)(-100,77.75)(-100.83,78.17)(-102.17,78.17)(-104.17,78.3)(-105,78.53)(-104.17,78.83)(-105.92,79)(-106.33,79.2) 
};
\addplot [
color=black,
solid,
forget plot
]
coordinates{
 (-97,77.67)(-95,77.83)(-92.5,77.8)(-92.33,77.57)(-93.17,77.35)(-95.17,77.47)(-97,77.67) 
};
\addplot [
color=black,
solid,
forget plot
]
coordinates{
 (-105.67,77.67)(-104.75,77.75)(-104.58,77.43)(-104,77.08)(-105.33,77.18)(-105.67,77.67) 
};
\addplot [
color=black,
solid,
forget plot
]
coordinates{
 (-105,76.58)(-104,76.63)(-103,76.37)(-102,76)(-101.92,76.42)(-101.25,76.7)(-99.5,76.67)(-98,76.38)(-98,76)(-97.83,75.58)(-98.33,75.07)(-100.58,75.07)(-101.5,75.63)(-103,75.42)(-104,75.83)(-104.25,76.17)(-105,76.58) 
};
\addplot [
color=black,
solid,
forget plot
]
coordinates{
 (-104.75,75.12)(-104.25,75.47)(-103.58,75.13)(-104.75,75.12) 
};
\addplot [
color=black,
solid,
forget plot
]
coordinates{
 (-102.92,72.92)(-102,73.07)(-100.5,73.03)(-101.42,73.5)(-101.25,73.83)(-100.17,74)(-98.83,73.92)(-97.75,74.13)(-97.5786784247043,74.0051974049157)(-97.4099423126632,73.8802611648959)(-97.2437346493732,73.755194365202)(-97.08,73.63)(-97.3350805288007,73.5154510160312)(-97.5867329862074,73.4005971885788)(-97.8350192537093,73.2854448058225)(-98.08,73.17)(-97.7246230102136,73.1234085108766)(-97.3711584137149,73.0762079564)(-97.019614770736,73.028403421827)(-96.67,72.98)(-96.25,72.42)(-96.42,71.83)(-97.83,71.67)(-99,71.33)(-99.92,71.83)(-100.92,72.25)(-102,72.25)(-102.33,72.63)(-102.92,72.92) 
};
\addplot [
color=black,
solid,
forget plot
]
coordinates{
 (-113.75,78.25)(-113,78.5)(-110.75,78.58)(-108.92,78.38)(-109.58,78.2)(-111,78.17)(-112,78.3)(-113.75,78.25) 
};
\addplot [
color=black,
solid,
forget plot
]
coordinates{
 (-115.5,77.92)(-114.25,78.08)(-114.08,77.75)(-114.83,77.7)(-115.5,77.92) 
};
\addplot [
color=black,
solid,
forget plot
]
coordinates{
 (-113.58,77.75)(-112.25,77.87)(-111.08,78.02)(-110.810172337194,78.0228872859565)(-110.540222183695,78.0255164918867)(-110.270160931175,78.0278874477296)(-110,78.03)(-110.043669155578,77.9175098512521)(-110.086544521012,77.8050130138104)(-110.128647821911,77.6925096711225)(-110.17,77.58)(-111.75,77.33)(-113.25,77.42)(-113.58,77.75) 
};
\addplot [
color=black,
solid,
forget plot
]
coordinates{
 (-119.5,75.58)(-118.92,75.9)(-117.75,76.13)(-118.08,75.75)(-118.5,75.48)(-119.5,75.58) 
};
\addplot [
color=black,
solid,
forget plot
]
coordinates{
 (-124.17,76.25)(-122.75,76.5)(-121.67,76.88)(-120.5,77.15)(-119.33,77.33)(-117.67,77.3)(-116.75,77.58)(-115.5,77.33)(-116.58,77.08)(-116.17,76.72)(-117.5,76.3)(-118.92,76.5)(-119.5,76.17)(-120.58,75.83)(-121.92,75.92)(-123.08,75.87)(-124.17,76.25) 
};
\addplot [
color=black,
solid,
forget plot
]
coordinates{
 (-117.75,75.25)(-117.17,75.67)(-117.17,75.67)(-116.42,76.13)(-115.75,76.47)(-114.33,76.42)(-112.75,76.17)(-112,75.92)(-111.42,75.57)(-110.918257495623,75.5540848932154)(-110.41763175213,75.5371106716657)(-109.918190386677,75.5190810611455)(-109.42,75.5)(-109.582225559651,75.582669974726)(-109.746281141229,75.6652279382139)(-109.912196011737,75.7476719432432)(-110.08,75.83)(-110.018379495532,75.8975239431438)(-109.956178209967,75.9650320777886)(-109.893387858289,76.0325241746385)(-109.83,76.1)(-110.058166314517,76.1378204680574)(-110.287553086216,76.1754284518073)(-110.518163344689,76.2128222118929)(-110.75,76.25)(-110.565833418844,76.332708839956)(-110.379469059478,76.4152801455265)(-110.19087034853,76.4977113927523)(-110,76.58)(-109.17,76.8)(-108.58,76.33)(-107,76.02)(-105.67,75.87)(-106,75.47)(-106.17,75.03)(-107.42,74.95)(-109.5,75)(-110.67,74.8)(-111.67,74.53)(-112.75,74.42)(-114,74.5)(-114.5,74.72)(-113.25,74.92)(-114.25,75.25)(-115.25,74.97)(-116.33,75.13)(-117.75,75.25) 
};
\addplot [
color=black,
solid,
forget plot
]
coordinates{
 (-119,71.6)(-118.83,72.03)(-118,72.3)(-118,72.58)(-117,72.83)(-115.75,73.02)(-114.83,73.3)(-113.67,73.3)(-113.25,72.72)(-111.92,73.08)(-110.83,72.85)(-109.58,73)(-108,72.67)(-107.42,73.17)(-106.58,73.67)(-105,73.7)(-104.17,73.42)(-104.67,73.05)(-105.5,72.58)(-105.17,72.17)(-104.58,71.63)(-104.58,71.63)(-104.75,71.05)(-103.75,70.67)(-102.33,70.4)(-100.67,70.22)(-100.83,69.75)(-102,69.67)(-101.5,69.3)(-102.58,69)(-103.83,68.87)(-105.08,69.15)(-106,69.5)(-107.08,69.17)(-108.42,69)(-110,68.75)(-111.67,68.65)(-113.08,68.58)(-113.75,69.25)(-115.17,69.25)(-116.42,69.5)(-116.75,70.12)(-115.5,70.17)(-114,70.17)(-112.67,70.42)(-114,70.7)(-115.42,70.67)(-117.08,70.75)(-118.08,71.03)(-117.75,71.37)(-119,71.6) 
};
\addplot [
color=black,
solid,
forget plot
]
coordinates{
 (-125.25,72.08)(-124.67,72.5)(-124,72.98)(-123.75,73.5)(-124.42,74.08)(-122.67,74.25)(-121.17,74.5)(-119.42,74.25)(-117.58,74.25)(-116.58,73.83)(-115.08,73.5)(-116.5,73.2)(-117.83,72.92)(-119.25,72.58)(-119.67,71.92)(-120.58,71.42)(-121.75,71.33)(-123.08,71.08)(-123.83,71.67)(-125.25,72.08) 
};
\addplot [
color=black,
solid,
forget plot
]
coordinates{
 (165.685937217858,55.3371386993368)(166.22918106749,55.3719686002233)(166.174899470977,55.2209096821551)(166.648854854896,54.8974859660883)(166.614361444468,54.7189660245458)(166.474906382073,54.7479519431988)(166.442504868719,54.8704087895514)(166.298243931367,54.8618051578507)(165.993324722901,55.0990342675988)(166.022596390489,55.1461222746398)(165.685937217858,55.3371386993368) 
};
\addplot [
color=black,
solid,
forget plot
]
coordinates{
 (172.5,52.92)(172.83,53)(173.33,52.83)(172.92,52.75)(172.5,52.92) 
};
\addplot [
color=black,
solid,
forget plot
]
coordinates{
 (-177.919064514333,51.71328910552)(-178.231266931886,51.7604208146286)(-178.300671889919,51.886799649729)(-178.177933421784,51.9131625290537)(-177.906814559514,51.8660775978535)(-177.804293948233,51.7468555716488)(-177.919064514333,51.71328910552) 
};
\addplot [
color=black,
solid,
forget plot
]
coordinates{
 (-176.92,51.67)(-176.75,51.92)(-176.33,51.75)(-176.92,51.67) 
};
\addplot [
color=black,
solid,
forget plot
]
coordinates{
 (-174.185733667661,52.0723019000094)(-174.379578432379,52.0906582267429)(-174.547347431368,52.2027589942188)(-174.301597749641,52.2717716876289)(-174.347587326582,52.4184671182327)(-173.983963524551,52.1743046092601)(-174.185733667661,52.0723019000094) 
};
\addplot [
color=black,
solid,
forget plot
]
coordinates{
 (-169.17,52.75)(-168.5,53.5)(-168,53.58)(-168,53.33)(-168.33,53.17)(-169.17,52.75) 
};
\addplot [
color=black,
solid,
forget plot
]
coordinates{
 (-167.67,53.33)(-167.17,53.67)(-167.17,54)(-166.42,54)(-166.42,53.7)(-167,53.5)(-167.67,53.33) 
};
\addplot [
color=black,
solid,
forget plot
]
coordinates{
 (-171.75,63.67)(-170.5,63.67)(-169.83,63.42)(-168.83,63.33)(-169.83,63.08)(-170.75,63.42)(-171.75,63.33)(-171.75,63.67) 
};
\addplot [
color=black,
solid,
forget plot
]
coordinates{
 (-167.42,60.17)(-166.75,60.25)(-166.08,60.33)(-165.5,60.25)(-165.5,60)(-166.17,59.83)(-166.83,59.92)(-167.42,60.17) 
};
\addplot [
color=black,
solid,
forget plot
]
coordinates{
 (-154.83,57.25)(-153.92,57.75)(-153.25,58.08)(-152.5,58.42)(-152.08,58.17)(-152.311206369129,58.108123124419)(-152.541608886114,58.045829193927)(-152.771206937709,57.9831206671871)(-153,57.92)(-152.874765731413,57.89768435069)(-152.749687293146,57.8752456272285)(-152.624765209886,57.852684089979)(-152.5,57.83)(-152.25,57.5)(-153.08,57.33)(-153.42,57)(-154.08,56.75)(-154.83,57.25) 
};
\addplot [
color=black,
solid,
forget plot
]
coordinates{
 (-133.08,54.17)(-132.33,54.08)(-131.67,54.08)(-132,53.58)(-131.67,53.08)(-131.83,52.75)(-131.33,52.5)(-131,52)(-131.83,52.42)(-132.25,52.83)(-132.5,53.33)(-133,53.58)(-133.08,54.17) 
};
\addplot [
color=black,
solid,
forget plot
]
coordinates{
 (-128.33,50.67)(-127.92,50.83)(-127.17,50.58)(-126.25,50.42)(-125.5,50.25)(-125,49.75)(-124.42,49.33)(-123.67,48.92)(-123.17,48.5)(-123.67,48.33)(-124.25,48.5)(-125,48.75)(-125,49)(-125.5,49)(-125.92,49.25)(-126.5,49.42)(-126.42,49.67)(-127.08,49.92)(-127.75,50.17)(-128.33,50.67) 
};
\addplot [
color=black,
solid,
forget plot
]
coordinates{
 (-73,78.17)(-72.33,78.58)(-70.33,78.75)(-68.67,79.02)(-65.67,79.2)(-64.5,79.75)(-65,80.08)(-65,80.08)(-65.6674120801586,80.0819793182335)(-66.3349999999979,80.0826391787706)(-67.0025879198373,80.0819793182335)(-67.67,80.08)(-67.4478428523179,80.1927271347789)(-67.2205738047149,80.3053063621336)(-66.9880197403059,80.417732468499)(-66.75,80.53)(-65,80.92)(-63.75,81.18)(-61.83,81.13)(-61.25,81.42)(-62,81.75)(-59.33,82.08)(-54.5,82.33)(-51.25,82)(-50.5,82.47)(-47.33,82.42)(-46.5,82.83)(-44,83.2)(-39.5,83.42)(-36,83.65)(-31.5,83.58)(-27.33,83.55)(-24.83,83.25)(-24,82.98)(-21.67,82.85)(-21.2404437936883,82.7805797074353)(-20.8190497994449,82.7107655305856)(-20.4056299552587,82.6405686792751)(-20,82.57)(-20.5559499743137,82.5085122870749)(-21.1028496865035,82.4463385782089)(-21.6408240161031,82.38349567323)(-22.17,82.32)(-25.58,82.22)(-26.2454905074185,82.2290599335615)(-26.912436464532,82.2370819906047)(-27.5806647014548,82.2440629505673)(-28.25,82.25)(-28.6517798737887,82.2055550781201)(-29.0490127849107,82.1607358643152)(-29.4417391145285,82.1155487417739)(-29.83,82.07)(-28.7054256377637,82.074526609391)(-27.5799999999992,82.0760360538405)(-26.4545743622346,82.074526609391)(-25.33,82.07)(-25.33,81.9725000000001)(-25.33,81.8750000000001)(-25.33,81.7775)(-25.33,81.68)(-23.67,81.75)(-23.67,82.03)(-23.67,82.03)(-21.67,82.08)(-21.33,81.8)(-22,81.63)(-23.67,81.37)(-24.83,80.92)(-22.5,81.08)(-20.75,81.32)(-19.5,81.58)(-18.17,81.48)(-16.42,81.8)(-13.25,81.68)(-12.8424299282934,81.5906032378728)(-12.4433923924321,81.5007958092121)(-12.0526570329995,81.4105905958431)(-11.67,81.32)(-12.3668741889116,81.1793089443701)(-13.0419433522427,81.0373736567896)(-13.6960484153323,80.8942523232038)(-14.33,80.75)(-15.83,80.42)(-17.17,79.97)(-18.08,79.53)(-18.5,79.13)(-19.5,78.83)(-19.17,78.42)(-19.67,77.92)(-18.83,77.58)(-18.17,77.08)(-18.42,76.75)(-20.33,76.92)(-21.42,76.67)(-21.5,76.3)(-20,76.2)(-19.42,75.7)(-19.42,75.2)(-20.08,74.67)(-20.08,74.67)(-18.83,74.67)(-19.25,74.25)(-21.5,74)(-20.33,73.83)(-20.33,73.47)(-21.58,73.42)(-22.17,73.25)(-22.3792964787519,73.2003136323798)(-22.5873915356806,73.1504169360371)(-22.7942908119598,73.1003117752499)(-23,73.05)(-22.7486114813021,73.0179587107631)(-22.498146355605,72.9856104479864)(-22.2486080918705,72.9529569604496)(-22,72.92)(-21.75,72.42)(-22,72.08)(-23.33,72.3)(-24.58,72.5)(-23.67,72.17)(-22.58,71.83)(-21.75,71.42)(-21.75,71)(-21.83,70.5)(-23.08,70.42)(-24,70.58)(-24.25,71)(-24.5,71.33)(-25.5,71.17)(-25.25,70.67)(-26.33,70.42)(-26.33,70.17)(-25.08,70.42)(-23.75,70.17)(-23.4364886472888,70.1483183217019)(-23.1236420709084,70.1260898661471)(-22.8114745123873,70.1033164683947)(-22.5,70.08)(-22.669485716123,69.997737864024)(-22.8376384280895,69.9153158480765)(-23.0044719616931,69.8327359151599)(-23.17,69.75)(-23.17,69.75)(-24.25,69.42)(-25.25,69.08)(-26.25,68.83)(-27.5,68.58)(-28.75,68.42)(-30,68.17)(-31.33,68.17)(-32.17,67.92)(-33,67.5)(-33.42,67.08)(-33.92,66.67)(-34.75,66.33)(-35.67,66)(-37,65.67)(-38.5,65.67)(-39.67,65.5)(-40,65.08)(-41,65.17)(-40.42,64.5)(-40.5,64)(-40.75,63.5)(-41.5,63)(-42.42,62.67)(-42,62)(-42.5,61.5)(-42.75,61)(-42.83,60.5)(-43.25,60)(-43.92,59.83)(-45.17,60.17)(-45.17,60.17)(-45.75,60.58)(-46.67,60.75)(-48,60.83)(-49,61.42)(-49.5,61.92)(-50.08,62.5)(-50.33,62.83)(-51.25,63.58)(-51.33,64)(-52,64.33)(-52,64.92)(-52.42,65.5)(-53.5,66)(-53.5,66.5)(-53.75,67)(-53.67,67.58)(-53.33,68.17)(-52.58,68.58)(-51,68.58)(-50.75,69.33)(-51.25,69.92)(-52.17,69.47)(-53.67,69.3)(-54.83,69.67)(-54.58,70.25)(-54.25,70.83)(-52.83,70.8)(-51.25,70.5)(-51.58,71)(-52.75,71.17)(-52.75,71.17)(-53.33,71.75)(-54,71.42)(-55.25,71.42)(-55.75,71.68)(-55.33,72.17)(-55.83,72.58)(-55.5,73.17)(-55.67,73.58)(-56.5,74)(-56.5,74.5)(-57.5,74.97)(-58.33,75.32)(-58.33,75.67)(-60,75.82)(-61.67,76.17)(-63.42,76.17)(-65.33,76.13)(-66.58,75.92)(-68.5,76.13)(-69.5,76.38)(-68.75,76.6)(-70.25,76.8)(-71.17,77.02)(-70,77.25)(-68.42,77.35)(-70,77.58)(-70.33,77.92)(-71.67,77.87)(-73,78.17) 
};
\addplot [
color=black,
solid,
forget plot
]
coordinates{
 (-24.33,65.5)(-23.75,65.83)(-23.67,66.17)(-23,66.17)(-23,66.42)(-22.25,66.33)(-21.5,66)(-21.42,65.42)(-21.42,65.42)(-20.33,65.58)(-20.25,66.08)(-19.67,65.83)(-19.25,66.08)(-18.25,66.17)(-17.58,66)(-16.83,66.17)(-16.25,66.5)(-15.67,66.25)(-14.75,66.33)(-14.67,65.75)(-13.67,65.5)(-13.67,64.92)(-14.5,64.42)(-15.83,64.17)(-16.75,63.83)(-17.67,63.75)(-18.75,63.42)(-20,63.58)(-21,63.83)(-21.4374794513765,63.8319834625633)(-21.8750000000006,63.8326446503042)(-22.3125205486247,63.8319834625633)(-22.75,63.83)(-22.5646103536837,63.9353607556321)(-22.377823086861,64.0404829622972)(-22.1896243065338,64.1453636980085)(-22,64.25)(-22.1430193006782,64.3752128180541)(-22.2873470703295,64.5002851541685)(-22.4330012312677,64.6252149229047)(-22.58,64.75)(-24,64.75)(-22.83,65.08)(-22.6443055925602,65.1653487174702)(-22.4574135603253,65.2504665600156)(-22.2693147564928,65.335351129173)(-22.08,65.42)(-22.246730285172,65.4602767148512)(-22.4139736078,65.5003695581766)(-22.5817301340636,65.5402776228459)(-22.75,65.58)(-23.58,65.42)(-24.33,65.5) 
};
\addplot [
color=black,
solid,
forget plot
]
coordinates{
 (-9.05404021662837,70.894810609535)(-8.50616453697261,71.0174216196454)(-8.30805216593745,71.1505287761176)(-7.9248969863564,71.1521484616041)(-8.00499903734917,70.9906688958469)(-8.25562382606616,70.922329138016)(-8.60835917376846,70.9070844189087)(-9.04743081058271,70.8064645107309)(-9.05404021662837,70.894810609535) 
};
\addplot [
color=black,
solid,
forget plot
]
coordinates{
 (-18.83,75.4)(-17.92,75.05)(-18.83,75)(-18.83,75.4) 
};
\addplot [
color=black,
solid,
forget plot
]
coordinates{
 (9.33,0)(9.33,0.5)(9.58,0.92)(9.25,1.08)(9.5,1.5)(9.67,1.92)(9.75,2.5)(9.92,3.08)(9.58,3.5)(9.33,3.92)(8.92,4)(8.83,4.5)(8.42,4.5)(8,4.42)(7.5,4.42)(7,4.33)(6.5,4.25)(6,4.25)(5.42,4.58)(5.25,5.25)(5,5.92)(4.5,6.33)(4,6.5)(3.42,6.5)(2.83,6.42)(2.33,6.42)(1.83,6.33)(1.83,6.33)(1.25,6.17)(1,5.83)(0.33,5.83)(-0.25,5.5)(-0.67,5.25)(-1.08,5.17)(-1.58,5)(-2,4.75)(-2.58,5)(-3.08,5.08)(-3.83,5.25)(-4.33,5.25)(-4.75,5.17)(-5.33,5.08)(-5.92,5)(-6.33,4.67)(-6.92,4.58)(-7.42,4.25)(-7.83,4.33)(-8.33,4.5)(-8.92,4.92)(-9.42,5.25)(-9.83,5.58)(-10.33,6.08)(-11,6.42)(-11.42,6.83)(-11.92,7.08)(-12.42,7.33)(-12.92,7.83)(-13.08,8.25)(-13.08,8.25)(-13.17,8.67)(-13.25,9.17)(-13.58,9.58)(-14,10)(-14.42,10.25)(-14.67,10.67)(-15,11.08)(-15.42,11.25)(-15.5,11.92)(-15.92,11.83)(-16.33,12.08)(-16.75,12.33)(-16.75,12.92)(-16.67,13.42)(-16.75,14)(-17,14.5)(-17.42,14.75)(-17,15)(-16.75,15.42)(-16.42,15.83)(-16.42,16.5)(-16.25,17)(-16.08,17.5)(-16,18)(-16,18.5)(-16.17,19)(-16.5,19.5)(-16.17,19.83)(-16.17,20.33)(-16.58,20.75)(-16.58,20.75)(-17,21.08)(-16.92,21.58)(-16.83,22.08)(-16.5,22.33)(-16.25,22.75)(-16,23.42)(-15.75,23.83)(-15.42,24.25)(-15,24.67)(-14.75,25.25)(-14.58,25.67)(-14.42,26.17)(-14.17,26.42)(-13.67,26.67)(-13.42,27)(-13.25,27.5)(-12.92,27.92)(-12.17,28)(-11.5,28.25)(-11.17,28.67)(-10.58,28.92)(-10.17,29.33)(-9.83,29.75)(-9.58,30.33)(-9.83,30.67)(-9.83,31.5)(-9.5,31.75)(-9.25,32.17)(-9.25,32.5)(-8.83,32.83)(-8.83,32.83)(-8.5,33.25)(-8,33.5)(-7.42,33.67)(-6.83,34)(-6.58,34.42)(-6.33,34.83)(-6.17,35.25)(-5.92,35.75)(-5.42,35.83)(-5.17,35.5)(-4.5,35.17)(-4.08,35.25)(-3.42,35.25)(-3,35.33)(-2.42,35.08)(-1.92,35.17)(-1.25,35.42)(-1,35.75)(-0.5,35.83)(0,35.92)(0.5,36.25)(1,36.5)(1.75,36.58)(2.42,36.67)(2.92,36.83)(3.42,36.83)(3.83,36.92)(4.42,36.92)(5,36.83)(5.33,36.67)(5.33,36.67)(5.83,36.75)(6.25,37)(7,36.83)(7.42,37)(8.17,36.83)(8.67,36.92)(9.08,37.17)(9.58,37.25)(10.08,37.17)(10.33,36.67)(11,36.92)(10.75,36.5)(10.42,36.33)(10.58,35.75)(10.92,35.58)(11,35.17)(10.75,34.75)(10.33,34.42)(10,34.17)(10.08,33.75)(10.5,33.58)(10.92,33.67)(11.08,33.25)(11.67,33)(12.17,32.83)(12.83,32.83)(13.58,32.75)(14.17,32.67)(14.75,32.42)(15.33,32.25)(15.33,32.25)(15.33,31.75)(15.58,31.5)(16.08,31.25)(16.67,31.17)(17.5,31)(18.17,30.67)(18.67,30.33)(19.17,30.25)(19.58,30.42)(19.92,30.67)(20.08,31.08)(20,31.5)(19.9576130490445,31.582521093001)(19.9151509747484,31.6650281708827)(19.8726134135052,31.7475211634118)(19.83,31.83)(19.8923264619019,31.9150458787878)(19.9547682458655,32.0000612768826)(20.0173259063027,32.0850460367353)(20.08,32.17)(20.67,32.58)(21.25,32.75)(21.83,32.83)(22.5,32.75)(23.08,32.58)(23.17,32.17)(23.75,32.08)(24.33,31.92)(24.92,31.92)(25.17,31.5)(25.75,31.58)(26.33,31.5)(26.92,31.42)(27.33,31.25)(27.83,31.08)(28.42,31.08)(28.42,31.08)(29,30.83)(29.5,30.92)(30,31.17)(30.58,31.42)(31.17,31.5)(31.83,31.42)(31.75,31.08)(32.15,30.92)(32.2532292582192,30.6901207470446)(32.3559679549481,30.4601602505747)(32.4582226955484,30.2301196326425)(32.56,30)(32.5174199262357,29.9575442440698)(32.454893301703,29.9150589415848)(32.392420026326,29.8725441683427)(32.33,29.83)(32.3926911221724,29.7275440345626)(32.455254327708,29.6250585915253)(32.517690370373,29.5225438529911)(32.58,29.42)(32.67,29)(32.92,28.58)(33.25,28.17)(33.58,27.83)(33.67,27.42)(33.92,27.08)(34,26.67)(34.33,26.17)(34.5,25.75)(34.75,25.25)(35,24.83)(35.25,24.33)(35.58,24)(35.5,23.58)(35.58,23.25)(35.83,22.75)(36.17,22.5)(36.58,22.25)(36.92,22.08)(36.92,21.5)(37.25,21.08)(37.25,20.75)(37.17,20.25)(37.25,19.67)(37.33,19.25)(37.58,18.67)(38.08,18.42)(38.08,18.42)(38.58,18)(38.83,17.58)(39.08,17.08)(39.17,16.5)(39.25,16)(39.5,15.58)(39.83,15.5)(40.08,15)(40.67,14.75)(41.17,14.58)(41.58,14.08)(42.08,13.67)(42.42,13.25)(42.83,12.83)(43.25,12.42)(43.42,11.92)(42.92,11.67)(43.25,11.42)(43.58,11.08)(43.92,10.67)(44.25,10.42)(44.75,10.33)(45.25,10.5)(45.75,10.75)(46.33,10.67)(46.83,10.75)(47.42,11.08)(48,11.08)(48.5,11.25)(49,11.17)(49,11.17)(49.58,11.33)(50.17,11.5)(50.58,11.92)(51.17,11.75)(51.08,11)(51,10.42)(50.75,10)(50.75,9.42)(50.42,8.92)(50.17,8.5)(49.92,8.17)(49.75,7.67)(49.5,7.25)(49.25,6.75)(49.08,6.33)(48.92,5.75)(48.5,5.42)(48.17,5.08)(47.92,4.58)(47.67,4.17)(47.25,3.75)(46.92,3.42)(46.5,3)(46.17,2.58)(45.75,2.25)(45.33,1.92)(44.83,1.75)(44.42,1.42)(44,1)(43.58,0.75)(43.58,0.75)(43.17,0.42)(42.92,0)(42.42,-0.42)(42.08,-0.92)(41.75,-1.25)(41.5,-1.75)(41.08,-2.08)(40.67,-2.5)(40.25,-2.42)(40.25,-3)(39.92,-3.58)(39.58,-4.17)(39.33,-4.58)(39.08,-5.08)(38.92,-5.58)(38.75,-6.08)(39.17,-6.58)(39.58,-7.08)(39.33,-7.58)(39.33,-8.17)(39.42,-8.83)(39.67,-9.42)(39.83,-9.92)(40.17,-10.17)(40.5,-10.5)(40.5,-11)(40.42,-12.5)(40.5,-12)(40.5,-12.42)(40.42,-13)(40.42,-13)(40.58,-13.58)(40.58,-14.08)(40.75,-14.67)(40.67,-15.17)(40.42,-15.58)(40.08,-16)(39.75,-16.33)(39.25,-16.75)(38.92,-17.08)(38.25,-17.17)(37.83,-17.33)(37.33,-17.5)(37,-17.92)(36.67,-18.33)(36.25,-18.83)(35.83,-19)(35.5,-19.42)(35.17,-19.67)(34.75,-20)(34.67,-20.5)(35,-20.83)(35.17,-21.42)(35.33,-22.08)(35.5,-22.5)(35.5,-23.08)(35.42,-23.5)(35.5,-24)(35.25,-24.5)(34.92,-24.75)(34.42,-24.92)(34.42,-24.92)(33.75,-25.17)(33.42,-25.33)(32.92,-25.5)(32.58,-26.08)(32.92,-26.25)(32.92,-26.58)(32.92,-27)(32.75,-27.5)(32.58,-28.08)(32.42,-28.58)(32,-28.83)(31.5,-29.17)(31.25,-29.58)(30.92,-30)(30.67,-30.5)(30.42,-30.92)(30,-31.33)(29.58,-31.67)(29.25,-32)(28.83,-32.42)(28.5,-32.75)(28.08,-33)(27.5,-33.33)(27.08,-33.67)(26.58,-33.75)(26,-33.75)(25.75,-34)(25.08,-34)(24.75,-34.25)(24.25,-34.17)(24.25,-34.17)(23.67,-34)(23.25,-34.17)(22.67,-34.08)(22.08,-34.17)(21.75,-34.42)(21.08,-34.42)(20.5,-34.5)(20.17,-34.75)(19.58,-34.75)(19.17,-34.42)(18.75,-34.08)(18.42,-34.25)(18.42,-33.83)(18.25,-33.42)(17.83,-32.83)(18.08,-32.83)(18.33,-32.58)(18.25,-32)(17.92,-31.42)(17.58,-31)(17.33,-30.58)(17.17,-30)(16.92,-29.58)(16.75,-29)(16.42,-28.58)(15.92,-28.25)(15.67,-27.83)(15.33,-27.5)(15.25,-27)(15.08,-26.42)(15.08,-26.42)(14.92,-25.92)(14.83,-25.5)(14.83,-25)(14.58,-24.42)(14.5,-24)(14.5,-23.5)(14.42,-22.92)(14.5,-22.5)(14.33,-22.08)(13.92,-21.67)(13.75,-21.25)(13.42,-20.83)(13.25,-20.25)(12.92,-19.75)(12.67,-19.25)(12.42,-18.83)(12.08,-18.5)(11.83,-17.92)(11.75,-17.42)(11.75,-17)(11.83,-16.5)(11.83,-16)(12.08,-15.5)(12.17,-15)(12.33,-14.5)(12.5,-13.92)(12.67,-13.33)(13,-12.92)(13.42,-12.67)(13.67,-12.25)(13.67,-12.25)(13.83,-11.83)(13.92,-11.25)(13.83,-10.67)(13.58,-10.33)(13.33,-9.92)(13.17,-9.42)(13,-9.17)(13.42,-8.67)(13.25,-8.17)(13,-7.75)(12.92,-7.25)(12.67,-6.75)(12.42,-6.42)(12.33,-6)(12.17,-5.75)(12.17,-5.33)(11.92,-4.92)(11.58,-4.5)(11.33,-4.08)(11,-3.75)(10.67,-3.42)(10.33,-3.08)(9.83,-2.75)(9.5,-2.25)(9.17,-1.83)(9,-1.33)(8.75,-0.83)(9.25,-0.5)(9.33,0) 
};
\addplot [
color=black,
solid,
forget plot
]
coordinates{
 (32.58,30)(32.17,30.92)(32.33,31.25)(32.67,31)(33.25,31.08)(34,31.17)(34.33,31.5)(34.67,32)(34.83,32.5)(35,33)(35.25,33.5)(35.5,34.08)(35.83,34.5)(35.75,35)(35.75,35.67)(35.92,36)(35.75,36.33)(36.08,36.67)(36,36.92)(35.5,36.58)(35.17,36.58)(34.5,36.83)(34.08,36.5)(33.58,36.17)(33.58,36.17)(33.17,36.17)(32.75,36)(32.25,36.17)(32,36.5)(31.25,36.75)(30.67,36.83)(30.5,36.42)(29.75,36.17)(29.25,36.33)(28.58,36.75)(28,36.75)(28.17,37)(27.67,37)(27.5450002154839,37.0001966083438)(27.4199999999998,37.0002621446377)(27.2949997845156,37.0001966083438)(27.17,37)(27.2522302417128,37.0825856342946)(27.3346397847517,37.1651143760625)(27.4172294346703,37.2475859301897)(27.5,37.33)(27.3952630209891,37.3926391581951)(27.2903509808162,37.4551857875113)(27.1852634499949,37.517639523352)(27.08,37.58)(27.17,37.92)(26.75,38.08)(26.25,38.25)(26.33,38.58)(26.67,38.33)(26.92,38.42)(26.75,38.67)(26.67,39.25)(26.83,39.5)(26,39.42)(26.17,40)(26.17,40.25)(26.58,40.58)(26.08,40.5)(25.75,40.83)(25.17,40.92)(24.75,40.83)(24.25,40.92)(24,40.67)(23.5,40.67)(23.75,40.42)(23.58,40.17)(23.08,40.17)(22.58,40.58)(22.5,40)(22.83,39.58)(23.17,39.25)(23.33,38.92)(23.67,38.67)(24.08,38.5)(24.17,38.17)(24.17,38.17)(24.58,38)(24,38.17)(24,37.58)(23.5,37.92)(22.92,38)(22.58,38.33)(21.83,38.3)(21.08,38.33)(20.67,38.75)(20.33,39.25)(20.08,39.58)(19.75,40)(19.33,40.25)(19.33,40.75)(19.42,41.33)(19.5,41.75)(19,42)(18.5,42.33)(17.92,42.58)(17.33,43)(16.58,43.42)(16,43.5)(15.75,43.75)(15.17,44)(15.17,44.33)(14.92,44.58)(14.83,45)(14.17,45.33)(14.08,45)(13.75,44.83)(13.75,44.83)(13.5,45.08)(13.42,45.42)(13.58,45.67)(13.08,45.75)(12.58,45.5)(12.17,45.5)(12.08,45.17)(12.33,44.92)(12.17,44.58)(12.25,44.25)(12.67,43.92)(13.08,43.67)(13.58,43.42)(13.83,43)(13.92,42.58)(14.17,42.33)(14.67,42.08)(15.25,41.83)(16.08,41.83)(15.83,41.5)(16.33,41.33)(16.92,41.08)(17.42,40.83)(17.92,40.58)(18.42,40.17)(18.25,39.83)(17.92,39.92)(17.83,40.25)(17.33,40.33)(16.92,40.5)(16.92,40.5)(16.67,40.17)(16.5,39.67)(17,39.42)(17,39)(16.5,38.83)(16.5,38.42)(16.17,38.17)(16,37.92)(15.67,37.92)(15.67,38.17)(15.83,38.58)(16.08,38.83)(16,39.25)(15.75,39.67)(15.67,40)(15,40.17)(14.75,40.58)(14.42,40.67)(14,40.92)(13.58,41.25)(12.92,41.25)(12.42,41.5)(12,41.92)(11.5,42.33)(11,42.33)(10.92,42.67)(10.42,43.08)(10.17,43.5)(10.08,44)(9.5,44.08)(9.5,44.08)(9.17,44.25)(8.67,44.33)(8.25,44.17)(8,43.83)(7.42,43.67)(6.83,43.42)(6.42,43.08)(5.83,43)(5.25,43.25)(4.58,43.33)(4,43.5)(3.58,43.25)(3,43)(3.08,42.5)(3.17,41.92)(2.67,41.67)(2.08,41.33)(1.5,41.17)(1,41)(0.5,40.42)(0,39.92)(-0.33,39.5)(-0.25,39.08)(-0.124563537673831,38.9977002856715)(0.000581241619058977,38.9152665786469)(0.125435400010871,38.8326995831081)(0.25,38.75)(0.104499737124542,38.6677688156289)(-0.0406663331775021,38.5853577939253)(-0.185499235797881,38.5027678761999)(-0.33,38.42)(-0.67,38.08)(-0.83,37.67)(-1.33,37.67)(-1.83,37.17)(-2.17,36.75)(-2.17,36.75)(-2.75,36.75)(-3.33,36.75)(-3.92,36.75)(-4.5,36.67)(-5,36.5)(-5.5,36.08)(-6,36.25)(-6.33,36.75)(-6.5,37)(-7,37.25)(-7.42,37.25)(-7.83,37.08)(-8.33,37.17)(-8.92,37.08)(-8.75,37.5)(-8.75,38)(-8.67,38.5)(-9.08,38.5)(-9.33,38.83)(-9.25,39.33)(-8.92,39.67)(-8.83,40.25)(-8.67,40.75)(-8.58,41.17)(-8.75,41.67)(-8.83,42)(-8.75,42.42)(-9.17,43)(-8.92,43.25)(-8.33,43.33)(-8.33,43.33)(-8,43.75)(-7.25,43.5)(-6.58,43.5)(-5.83,43.58)(-5.08,43.5)(-4.42,43.33)(-3.58,43.5)(-2.75,43.42)(-2.17,43.33)(-1.67,43.42)(-1.42,43.67)(-1.25,44.25)(-1.17,44.67)(-1.08,45.25)(-1,45.83)(-1.25,46.25)(-1.75,46.5)(-2,46.75)(-1.92,47)(-2.33,47.33)(-2.5,47.5)(-3.17,47.58)(-3.75,47.83)(-4.25,47.83)(-4.67,48)(-4.75,48.5)(-4.17,48.58)(-3.67,48.75)(-3,48.83)(-2.67,48.5)(-2.67,48.5)(-2,48.58)(-1.42,48.58)(-1.58,49.17)(-1.92,49.67)(-1.33,49.67)(-1.08,49.25)(-0.58,49.33)(0,49.25)(0.17,49.67)(0.58,49.83)(1.08,49.92)(1.5,50.08)(1.58,50.5)(1.75,50.83)(2.33,51)(3,51.25)(3.58,51.42)(4,51.75)(4.33,52.08)(4.58,52.42)(4.75,52.83)(5.08,52.92)(5.08,52.42)(5.5,52.25)(5.92,52.5)(5.92,52.75)(5.5,52.83)(5.5,53.25)(6.08,53.42)(7,53.42)(7,53.42)(7.33,53.67)(8,53.67)(8.5,53.5)(8.58,53.83)(9,54.08)(8.67,54.33)(9,54.5)(8.67,54.83)(8.67,55.33)(8.17,55.58)(8.08,56)(8.08,56.5)(8.25,56.83)(8.67,57.08)(9.42,57.17)(9.83,57.5)(10.5,57.58)(10.5,57.25)(10.25,56.83)(10.25,56.5)(10.83,56.5)(10.83,56.17)(10.17,56)(9.75,55.5)(9.42,55)(10,54.75)(10,54.5)(10.92,54.25)(10.75,53.92)(11.33,53.92)(11.33,53.92)(11.83,54.17)(12.25,54.33)(12.92,54.42)(13.33,54.67)(13.67,54.33)(13.25,54.17)(13.83,54)(14.33,53.92)(15.17,54.17)(16,54.25)(16.67,54.58)(17.5,54.75)(18.33,54.83)(18.67,54.33)(19.25,54.33)(19.75,54.5)(20,54.92)(20.75,55.17)(21.08,55.67)(21,56.17)(21,56.75)(21.33,57.08)(21.67,57.58)(22.42,57.67)(22.92,57.42)(23.25,57.08)(23.83,57)(24.33,57.25)(24.33,57.75)(24.33,58.25)(24.33,58.25)(23.75,58.33)(23.5,58.75)(23.5,59.17)(24.17,59.42)(25,59.5)(25.92,59.58)(26.83,59.42)(27.92,59.42)(28.33,59.67)(29.17,60)(29.420958103315,59.9582107326128)(29.6712804175488,59.9159462981316)(29.9209624964206,59.8732087118081)(30.17,59.83)(30.0659982174343,59.9351244367202)(29.961335306474,60.0401665016784)(29.856004761399,60.1451253187064)(29.75,60.25)(29.08,60.17)(28.33,60.58)(27.17,60.58)(26.25,60.42)(25.33,60.25)(24.42,60.08)(23.5,60)(22.58,60.17)(22.08,60.42)(21.33,60.67)(21.33,61)(21.67,61.5)(21.25,62)(21.17,62.5)(21.58,63)(22.42,63.42)(23.25,63.92)(24.08,64.33)(24.08,64.33)(24.67,64.83)(25.33,65)(25.33,65.5)(24.5,65.83)(23.83,65.75)(23.17,65.75)(22.42,65.83)(22,65.5)(21.42,65.33)(21.5,65.08)(21,64.75)(21.67,64.5)(21,64.17)(20.67,63.83)(19.83,63.58)(19.17,63.33)(18.42,63)(17.75,62.5)(17.42,62.25)(17.33,61.75)(17.17,61.25)(17.25,60.83)(17.92,60.5)(18.58,60.17)(18.83,59.83)(18.58,59.58)(18,59.33)(18.33,59.17)(17.58,58.92)(16.83,58.58)(16.83,58.58)(16.67,58.25)(16.67,57.75)(16.5,57.33)(16.33,56.75)(15.92,56.17)(15.33,56.17)(14.67,56.17)(14.17,55.75)(14.25,55.42)(13.67,55.42)(13,55.42)(13,55.75)(12.67,56.17)(12.92,56.5)(12.33,56.92)(12,57.42)(11.67,58)(11.33,58.5)(11.08,59.08)(10.58,59.33)(10.25,59)(9.75,59)(9,58.58)(8.5,58.25)(7.67,58)(6.83,58.08)(6,58.33)(5.58,58.58)(5.67,59)(6.17,59.33)(6.17,59.33)(5.42,59.25)(5.25,59.58)(5.58,59.92)(5.08,60.33)(5,60.75)(4.95791932696489,60.8550199948626)(4.91556111406005,60.9600267568599)(4.87292235478889,61.0650201409497)(4.83,61.17)(4.93417246023938,61.2526211308045)(5.03889377778098,61.3351619756325)(5.14416819734473,61.4176218344926)(5.25,61.5)(5.16815909643137,61.5825742199289)(5.08588119372288,61.6650992498179)(5.00316270307605,61.7475746559997)(4.92,61.83)(5.10593376008368,61.9153803447475)(5.292907378572,62.0005086758287)(5.48092730740793,62.0853826743128)(5.67,62.17)(6.33,62.5)(7,62.92)(8,63)(8.5,63.33)(8.92,63.75)(9.5,63.5)(10.08,63.92)(10.67,64.33)(11.33,64.92)(12.17,65.25)(12.58,65.75)(13,66.17)(13.33,66.58)(13.83,67)(14.67,67.42)(15,67.83)(16.17,68.33)(15,68.25)(14,68.17)(12.92,67.83)(13.67,68.25)(13.67,68.25)(14.0006313300639,68.2934910809463)(14.3325163877054,68.3363240488366)(14.6656433512038,68.3784949847604)(15,68.42)(14.8562046933395,68.4826868294695)(14.7116098794915,68.5452498333576)(14.5662101323322,68.6076879232048)(14.42,68.67)(14.6455835979782,68.7329555571461)(14.8724413844397,68.7956092052521)(15.1005785005263,68.8579582553937)(15.33,68.92)(16.17,69.25)(15.92,68.75)(16.58,69)(16.67,68.67)(16.8949685047713,68.7529544181381)(17.1216156187534,68.8356082590186)(17.3499549192014,68.9179579809876)(17.58,69)(17.435396825581,69.0201836935526)(17.2905285287903,69.0402451592022)(17.1453959636822,69.0601840450474)(17,69.08)(17,69.42)(18.08,69.58)(18.92,70.08)(20.17,70.08)(21.5,70.25)(22.33,70.67)(23.33,70.92)(24.25,70.67)(24.83,71)(25.67,71.17)(25.58,70.58)(26.58,70.92)(26.83,70.5)(27.75,71.08)(29.08,70.92)(30.17,70.67)(31.17,70.42)(30,70.08)(29.6885254876126,70.1033164683947)(29.3763579290915,70.1260898661471)(29.0635113527111,70.1483183217019)(28.75,70.17)(29.0663293997428,70.0858251357219)(29.3800951730363,70.0010955060012)(29.6913132883916,69.9158181417272)(30,69.83)(31.25,69.75)(31.25,69.75)(32.08,69.92)(33.08,69.75)(33.08,69.42)(34.5,69.33)(35.83,69.17)(37,68.92)(38,68.58)(39,68.25)(40,68)(40.92,67.67)(41.25,67.25)(41.25,66.75)(40.58,66.42)(39.5,66.17)(38.33,66.08)(37.25,66.25)(36.25,66.33)(35.33,66.42)(34.33,66.67)(33.17,66.83)(32.9424057968269,66.9154962486231)(32.713215457671,67.0006641187067)(32.4824173858646,67.0854999394631)(32.25,67.17)(32.3972384995999,67.0451995698219)(32.542969519732,66.9202646486918)(32.687215885292,66.795197414311)(32.83,66.67)(33.67,66.42)(34.67,66)(34.67,65.5)(34.67,65.08)(34.67,64.58)(35.75,64.33)(36.75,63.92)(38,63.92)(38,63.92)(38,64.42)(37.25,64.42)(36.5,64.83)(36.92,65.17)(37.75,65.08)(38.67,64.83)(39.5,64.67)(40.67,64.58)(40.17,65.08)(40.0464876043803,65.1851546713177)(39.9219912804442,65.2902071079409)(39.7964993577892,65.3951559958902)(39.67,65.5)(39.8755225068712,65.5829217685991)(40.0823573671648,65.6655642723538)(40.2905135443167,65.7479246472207)(40.5,65.83)(41.5,66.08)(42.17,66.42)(43.17,66.33)(43.92,66.17)(44.5,66.5)(44.5,67)(43.83,67.17)(44.17,67.67)(44.17,68.25)(43.25,68.67)(44.5,68.5)(45.67,68.5)(46.25,68.25)(46.67,67.83)(45.5,67.75)(45.3530633337134,67.6451952347854)(45.2074288100464,67.5402590937166)(45.0630798192209,67.4351934137356)(44.92,67.33)(45.14863522793,67.29048406424)(45.3765150510125,67.2506442837438)(45.6036373233695,67.2104823609391)(45.83,67.17)(46.33,66.83)(47.33,66.83)(47.33,66.83)(48.17,67.08)(48.17,67.67)(49.08,67.83)(50.17,68.08)(51,68.33)(52.08,68.5)(53,68.75)(53.92,69)(54,68.5)(54.83,68.17)(55.17,68.5)(56.17,68.58)(57.25,68.5)(58.17,68.83)(59.08,69)(59.08,68.42)(59.83,68.33)(59.83,68.67)(60.5,68.75)(61.08,69)(60.5,69.33)(60.17,69.67)(59.17,69.92)(58.58,70.17)(59.08,70.5)(60.17,70.08)(61.75,69.75)(63,69.67)(64.08,69.33)(65.08,69.25)(65.08,69.25)(66,69)(67.08,68.83)(67.75,68.5)(68.58,68.25)(69.08,68.75)(68.33,69)(67.83,69.5)(66.92,69.67)(66.92,70)(67.33,70.42)(66.83,70.83)(67,71.25)(68.25,71.58)(68.75,72)(68.92,72.5)(69.5,72.92)(70.08,73.33)(70.3504634756357,73.3730304338205)(70.622285181329,73.4157090624811)(70.8954643567124,73.4580331612014)(71.17,73.5)(71.2968122295569,73.4176129303691)(71.4224050713842,73.3351498280832)(71.5467954497506,73.2526118170136)(71.67,73.17)(71.584084402054,73.1075529550552)(71.4987834761844,73.0450703463188)(71.4140907912021,72.982552565803)(71.33,72.92)(71.7089630445262,72.858543969592)(72.0852845350576,72.7963869128146)(72.4589635429764,72.7335364061226)(72.83,72.67)(72.83,72.5849999999995)(72.83,72.4999999999997)(72.83,72.4149999999998)(72.83,72.33)(72.7257847363807,72.2275808035767)(72.6227257070869,72.1251071198396)(72.5208037222296,72.0225798813493)(72.42,71.92)(72.2700332408033,71.8151706777575)(72.1217286994251,71.7102262629782)(71.9750596768738,71.6051687273786)(71.83,71.5)(72.0434257315025,71.3953527172671)(72.2545453928567,71.290467649061)(72.4633924873939,71.1853487762415)(72.67,71.08)(72.67,70.9549999999999)(72.67,70.8299999999999)(72.67,70.705)(72.67,70.58)(72.67,70.4549999999999)(72.67,70.33)(72.67,70.205)(72.67,70.08)(72.6470326044044,69.9350043269973)(72.624381279427,69.7900057278771)(72.6020392670413,69.6450042652608)(72.58,69.5)(72.58,69.3749999999999)(72.58,69.2499999999999)(72.58,69.125)(72.58,69)(72.833555262159,68.895554571169)(73.0847224100886,68.7907357642993)(73.333528403724,68.6855490975701)(73.58,68.58)(73.58,68.58)(73.17,68.17)(73.17,67.75)(72.5,67.33)(71.83,67)(70.5,66.67)(71.58,66.25)(72.5,66.5)(73.5,66.83)(74,67.25)(74.83,67.67)(74.5,68.25)(74.58,68.75)(74.8080544486473,68.812964193789)(75.0374025173529,68.8756207617418)(75.2680493505809,68.9379669535588)(75.5,69)(75.1060951922062,69.0213631592953)(74.7114398642789,69.0418193028636)(74.3160645122796,69.0613657678273)(73.92,69.08)(73.67,69.67)(73.83,70.17)(74.33,70.58)(73.83,71)(73.17,71.33)(73.75,71.83)(74.67,72)(75.17,72.33)(75.08,72.67)(74.17,72.92)(74.5,73.08)(75.5,72.75)(75.75,72.25)(75.33,71.83)(75.33,71.33)(76.92,71.17)(76.92,71.17)(78.5,71)(78,71.25)(76.58,71.5)(76.08,71.92)(76.92,72.08)(77.75,71.83)(78.33,72)(78.0019538143979,72.0833399053044)(77.6709525103982,72.1661250676607)(77.3369748605117,72.2483477190051)(77,72.33)(77.1855643080788,72.392763594913)(77.3724122923571,72.4553527138128)(77.5605541202185,72.5177654811937)(77.75,72.58)(78.5,72.33)(79.5,72.33)(80.58,72.08)(81.58,71.67)(81.8905111066848,71.7107586857063)(82.202353454559,71.7510138073812)(82.515519166444,71.7907620221065)(82.83,71.83)(82.6677834992719,71.9352078253468)(82.5037314184487,72.0402787225106)(82.337813776374,72.1452102719074)(82.17,72.25)(80.83,72.42)(80.83,72.92)(80.33,73.5)(82,73.58)(83.75,73.67)(85.25,73.83)(87,73.92)(86.5,74.33)(86.58,74.67)(87.5,75)(88.67,75.33)(90.58,75.58)(92.17,75.75)(93.25,76.08)(94.75,76.08)(94.75,76.08)(95.3049852904074,76.1444093240076)(95.8650025320627,76.2075573637068)(96.4300195904023,76.2694267114063)(97,76.33)(96.9160443923319,76.2475416156848)(96.8330705491423,76.1650551566243)(96.7510613342105,76.0825411221531)(96.67,76)(97.351250139759,76.0453793324944)(98.0367649912682,76.0888509750881)(98.726398942127,76.1303969841282)(99.42,76.17)(99.3168547019012,76.2525658509434)(99.2124874317886,76.3350883315742)(99.1068765697524,76.4175666511415)(99,76.5)(101,76.5)(101,77)(102.5,77.42)(104.08,77.75)(104.589840932426,77.668883117166)(105.09308308458,77.5868319314625)(105.589783113029,77.5038648507254)(106.08,77.42)(105.613468631059,77.3361871980386)(105.153000991631,77.2515723987795)(104.698533166343,77.166171468631)(104.25,77.08)(104.68747576496,77.0810920166681)(105.124999999999,77.0814560427421)(105.562524235039,77.0810920166681)(106,77.08)(106.376693809735,77.0608058528436)(106.752275331407,77.0410728229314)(107.126719054197,77.0208033687306)(107.5,77)(107.222429273359,76.8754296924055)(106.949993240074,76.7505675197438)(106.682559571936,76.6254216584417)(106.42,76.5)(106.772487384239,76.5007384073147)(107.124999999998,76.5009845520647)(107.477512615757,76.5007384073147)(107.83,76.5)(108.17,76.75)(109.58,76.75)(111.17,76.67)(112.42,76.5)(113.5,76.17)(113.83,75.75)(113.67,75.25)(112.5,74.92)(111,74.5)(109.67,74.17)(108.33,73.75)(107.17,73.58)(106.5,73.17)(107.58,73.17)(108.83,73.33)(110.17,73.5)(110.17,73.5)(110.004359662467,73.562697908322)(109.837487276381,73.6252648861022)(109.669371276287,73.6876994258185)(109.5,73.75)(109.665610015687,73.8126952039305)(109.832471995465,73.8752612818421)(110.000597951257,73.9376967239291)(110.17,74)(111.17,73.83)(112,73.67)(113,73.67)(113.67,73.5)(114.83,73.67)(115.92,73.75)(117.25,73.58)(118.67,73.58)(118.5,73.17)(120,73)(122,72.92)(123.33,73)(122.83,73.33)(123,73.67)(124.83,73.67)(126.92,73.42)(128.58,73.25)(129.83,72.75)(129.5,72.25)(128.75,71.83)(129.42,71.33)(130.33,70.92)(131.25,70.75)(131.25,70.75)(132,71.08)(132.25,71.5)(132.58,71.92)(133.5,71.5)(134.42,71.33)(135,71.58)(135.83,71.67)(136.67,71.5)(138.58,71.58)(138.9787311287,71.5612437462325)(139.376662727596,71.5416565483117)(139.773762853595,71.5212410512517)(140.17,71.5)(140.047067858494,71.6051211774616)(139.92277228172,71.7101624985892)(139.797090682748,71.8151225781905)(139.67,71.92)(139.33,72.25)(139.75,72.5)(141.17,72.5)(140.67,72.83)(142.42,72.75)(143.83,72.67)(145.25,72.58)(146.83,72.33)(148.5,72.33)(149.83,72.17)(150,71.67)(150.67,71.42)(151.58,71.42)(152.42,71.17)(152.42,70.83)(153.5,70.92)(154.58,70.92)(156,71.08)(157.5,71)(157.5,71)(158.83,70.92)(159.67,70.67)(159.753784767042,70.565056462805)(159.836704010686,70.4600748787195)(159.918771317959,70.3550558584667)(160,70.25)(159.873098252544,70.1451319323723)(159.747476985979,70.0401749833618)(159.623117138198,69.9351305496535)(159.5,69.83)(159.795089686805,69.7682308950718)(160.088451262517,69.7059715269011)(160.380087132024,69.6432264003832)(160.67,69.58)(162.25,69.58)(164,69.67)(165.67,69.58)(167,69.42)(167.67,69.67)(168.25,70)(169.42,69.83)(169.17,69.58)(168.25,69.58)(168.33,69.25)(169.33,69.08)(169.58,68.75)(170.42,68.83)(171,69.08)(170.58,69.5)(170.58,70.08)(172,70)(173.33,69.92)(174.67,69.83)(176.17,69.83)(177.42,69.58)(178.67,69.42)(179.75,69.17)(180,69.08)(180,69.08)(180.92,68.75) 
};
\addplot [
color=black,
solid,
forget plot
]
coordinates{
 (180.33,65.17)(180,65.05)(180,65.05)(179.58,64.92)(178.67,64.67)(177.5,64.75)(177.5,64.33)(178.33,64.33)(178.58,64)(178.83,63.5)(179.25,63)(179.67,62.67)(179,62.25)(178.17,62.5)(177,62.5)(176.25,62.25)(175.33,62.08)(174.58,61.75)(173.83,61.67)(173.08,61.33)(172.33,61)(171.5,60.67)(170.67,60.42)(170.67,60.42)(170.33,59.92)(169.83,60.25)(169.17,60.5)(168.25,60.58)(167.5,60.42)(166.83,60.17)(166.17,59.75)(166.17,60.33)(165.33,60.17)(164.75,59.83)(164.17,60)(163.33,59.83)(163.17,59.42)(162.92,59)(162.33,58.58)(161.92,58.17)(162.08,57.75)(162.58,57.92)(163.17,57.67)(162.67,57.25)(162.67,56.83)(163.08,56.67)(163.25,56.17)(162.83,56)(162.42,56.25)(161.92,56.08)(161.58,55.67)(161.67,55.17)(162.08,54.75)(161.67,54.5)(161.67,54.5)(161.17,54.58)(160.42,54.42)(159.92,54.08)(159.83,53.58)(159.92,53.17)(159.17,53.17)(158.58,52.83)(158.42,52.25)(157.92,51.67)(157.33,51.25)(156.67,50.92)(156.5,51.25)(156.5,51.67)(156.42,52.17)(156.08,52.75)(156,53.33)(155.83,53.92)(155.67,54.33)(155.5,54.83)(155.42,55.25)(155.5,55.83)(155.75,56.33)(156,56.83)(156.67,57)(157,57.42)(156.83,57.75)(157.5,57.75)(158.25,58)(159,58.42)(159.67,58.83)(159.67,58.83)(160.08,59.25)(160.92,59.67)(161.58,60.08)(161.92,60.42)(162.67,60.67)(163.5,60.83)(163.75,61.17)(163.92,61.75)(164.08,62.25)(164.75,62.5)(164.42,62.67)(163.25,62.5)(162.92,62.08)(162.92,61.67)(162.25,61.58)(161.67,61.25)(161,60.92)(160.17,60.58)(159.92,60.92)(159.83,61.25)(159.953257654372,61.3951704892396)(160.077665570262,61.5402284845195)(160.203240623881,61.685172246128)(160.33,61.83)(160.12169214681,61.7904706487975)(159.913921572827,61.7506266396635)(159.706690227222,61.7104693099681)(159.5,61.67)(158.83,61.83)(157.92,61.75)(156.92,61.58)(156.33,61.17)(155.75,60.75)(155,60.42)(154.5,60)(154.17,59.5)(154.17,59.5)(154.83,59.42)(154.915469211237,59.3575832128528)(155.000624515934,59.2951107227724)(155.085467564864,59.2325828719225)(155.17,59.17)(154.876928839879,59.1484872585002)(154.584231514158,59.1263153841096)(154.291918455511,59.1034858081194)(154,59.08)(153.33,59.17)(152.92,59)(152,58.83)(151.33,58.83)(151.08,59.08)(151.67,59.33)(151.33,59.58)(150.83,59.5)(150.17,59.58)(149.5,59.75)(149,59.67)(148.92,59.42)(148.83,59.25)(148.17,59.33)(147.42,59.25)(146.5,59.42)(145.92,59.17)(145.58,59.33)(144.67,59.33)(143.67,59.33)(142.67,59.25)(142,59)(141.67,58.67)(141,58.42)(140.58,58.08)(140.25,57.75)(139.67,57.5)(139.67,57.5)(139,57.17)(138.5,56.83)(137.92,56.33)(137.25,55.92)(136.58,55.58)(135.92,55.17)(135.17,54.92)(135.25,54.67)(135.92,54.5)(136.67,54.58)(136.67,54.17)(136.67,53.75)(137.08,54.17)(137.67,54.25)(137.67,53.83)(137.584499756351,53.7475904895266)(137.499334386991,53.6651203739753)(137.414501820612,53.582590072302)(137.33,53.5)(137.559304583772,53.5431608127869)(137.789074869564,53.5858821331621)(138.01930773577,53.6281623847012)(138.25,53.67)(138.83,54.17)(139.5,54.17)(140.17,54)(140.25,53.75)(140.75,53.5)(141.33,53.25)(141.08,52.83)(141.08,52.42)(141.33,52.17)(141.08,51.83)(140.75,51.42)(140.5,51)(140.42,50.5)(140.42,50.5)(140.5,50)(140.5,49.42)(140.33,49)(140.17,48.5)(139.5,48)(139,47.5)(138.5,47.08)(138.25,46.5)(137.92,46.08)(137.5,45.67)(136.92,45.25)(136.33,44.75)(135.67,44.33)(135.5,43.92)(135,43.42)(134.33,43.17)(133.75,42.83)(133.08,42.67)(132.33,42.92)(132.33,43.25)(131.75,43.25)(131.5,43)(131.17,42.58)(130.83,42.58)(130.67,42.25)(130.08,42.08)(129.58,41.58)(129.67,41.33)(129.67,40.83)(129.25,40.67)(129.25,40.67)(128.75,40.33)(128.25,40.08)(127.83,39.92)(127.704877163298,39.8977013647871)(127.579835970742,39.8752683564343)(127.454876793475,39.8527011696582)(127.33,39.83)(127.372615633764,39.7675232722421)(127.415153983974,39.7050309879708)(127.457615342493,39.6425232097852)(127.468218686256,39.6268938516306)(127.478817240748,39.6112635299637)(127.489411010489,39.595632245764)(127.5,39.58)(127.489327708671,39.5593822711506)(127.478661760542,39.5387635668066)(127.4680021475,39.5181438882614)(127.457348861441,39.4975232368069)(127.414798825398,39.4150309276643)(127.372349375917,39.3325231548149)(127.33,39.25)(127.455332941058,39.1877005124886)(127.580443645881,39.1252669935001)(127.705332526918,39.0626999781093)(127.83,39)(128.25,38.67)(128.5,38.25)(128.83,37.92)(129.17,37.5)(129.33,37)(129.42,36.5)(129.42,36)(129.33,35.58)(129.08,35.17)(128.5,34.92)(127.75,34.83)(127.17,34.67)(126.58,34.33)(126.33,34.83)(126.33,35.33)(126.75,35.75)(126.5,36.33)(126.17,36.92)(126.75,37)(126.58,37.42)(126.08,37.75)(125.33,37.67)(124.75,38.08)(124.75,38.08)(125,38.5)(125.17,38.92)(125.42,39.42)(125.08,39.58)(124.67,39.67)(124.08,39.83)(123.5,39.75)(123,39.58)(122.42,39.42)(122.08,39.08)(121.58,38.83)(121.17,38.75)(121.67,39.17)(121.33,39.5)(121.5,39.75)(121.92,40)(122.25,40.42)(121.92,40.83)(121.25,40.92)(120.83,40.58)(120.42,40.17)(119.92,40)(119.42,39.75)(119.25,39.42)(119,39.17)(118.42,39.08)(118,39.25)(117.67,38.92)(117.58,38.58)(117.92,38.25)(117.92,38.25)(118.33,38)(118.83,37.75)(118.92,37.33)(119.33,37.08)(119.75,37.08)(120,37.33)(120.33,37.58)(120.83,37.75)(121.17,37.5)(121.58,37.42)(122.08,37.42)(122.205112322874,37.3976974224082)(122.330149985834,37.3752631059973)(122.455112655301,37.3526972363918)(122.58,37.33)(122.517190069358,37.2050493193052)(122.454587801513,37.080065587447)(122.392191629692,36.9550490624278)(122.33,36.83)(122,37)(121.5,36.75)(120.83,36.58)(120.67,36.17)(120.17,35.92)(119.67,35.58)(119.42,35.25)(119.17,34.83)(119.75,34.58)(120.33,34.33)(120.5,33.83)(120.67,33.42)(120.83,33)(120.92,32.58)(121.25,32.42)(121.67,32.08)(121.83,31.67)(121.83,31.67)(121.25,31.67)(121.83,31.25)(121.92,30.92)(121.5,30.83)(121.17,30.67)(120.92,30.42)(120.42,30.33)(120.75,30.17)(121.33,30.25)(121.67,30.08)(122.08,30)(122,29.75)(121.92,29.17)(121.67,29.17)(121.5,28.67)(121.58,28.42)(121.42,28.17)(121.08,28.17)(120.83,27.83)(120.58,27.42)(120.33,27)(120,26.67)(119.67,26.75)(119.75,26.33)(119.67,26)(119.42,25.58)(119.17,25.25)(118.83,25)(118.67,24.58)(118.17,24.58)(118.17,24.58)(118.17,24.25)(117.83,24)(117.42,23.67)(117,23.42)(116.5,23.08)(116,22.83)(115.5,22.75)(115,22.67)(114.5,22.67)(114.17,22.25)(113.67,22.67)(113.5,22.17)(113.08,22.5)(113,22)(112.5,21.83)(112,21.75)(111.5,21.5)(110.92,21.42)(110.42,21.25)(110.25,20.92)(110.5,20.5)(110.25,20.33)(109.92,20.33)(109.75,20.75)(109.67,21)(109.75,21.42)(109.42,21.42)(109,21.67)(108.5,21.67)(108,21.5)(108,21.5)(107.58,21.25)(107.08,21)(106.67,20.67)(106.42,20.25)(105.92,19.92)(105.75,19.5)(105.58,19)(105.92,18.42)(106.33,18.08)(106.42,17.75)(106.83,17.33)(107.17,16.92)(107.67,16.5)(108.08,16.17)(108.42,15.75)(108.83,15.33)(109,14.75)(109.17,14.25)(109.25,13.75)(109.25,13.25)(109.25,12.83)(109.17,12.17)(109.17,11.75)(109,11.42)(108.5,11.17)(108.08,10.92)(107.67,10.67)(107.25,10.42)(106.67,10.42)(106.67,9.83)(106.67,9.83)(106.33,9.42)(105.83,9.33)(105.42,9)(105,8.67)(104.83,9.08)(104.83,9.58)(104.872452022433,9.70500791829167)(104.914935737014,9.83001060621156)(104.957451582786,9.95500799107766)(105,10.08)(104.875049817224,10.122570952539)(104.750066498891,10.1650947466465)(104.625049930983,10.2075711674255)(104.5,10.25)(104.25,10.58)(103.67,10.58)(103.25,10.92)(103,11.33)(102.83,11.75)(102.58,12.08)(102.25,12.25)(101.83,12.67)(101.42,12.67)(100.92,12.75)(100.92,13.33)(100.42,13.5)(100,13.33)(99.92,12.75)(99.92,12.08)(99.75,11.67)(99.58,11.17)(99.33,10.83)(99.17,10.25)(99.17,9.67)(99.33,9.17)(99.83,9.25)(99.83,9.25)(99.92,8.58)(100.25,8.25)(100.42,7.75)(100.58,7.17)(100.92,6.92)(101.5,6.83)(101.83,6.42)(102.25,6.08)(102.67,5.75)(103.08,5.42)(103.42,4.92)(103.42,4.42)(103.42,3.92)(103.42,3.42)(103.42,2.92)(103.75,2.58)(104.08,2)(104.25,1.42)(103.5,1.42)(103.17,1.67)(102.75,1.92)(102.25,2.17)(101.92,2.5)(101.42,2.83)(101.33,3.33)(100.92,3.75)(100.58,4.25)(100.67,4.75)(100.42,5.08)(100.33,5.67)(100.33,5.67)(100.33,6.08)(100.08,6.67)(99.75,7)(99.42,7.33)(99.08,7.83)(98.67,8.25)(98.42,8.08)(98.25,8.5)(98.25,9)(98.5,9.58)(98.5,10.17)(98.5,10.67)(98.75,11)(98.75,11.42)(98.5,11.92)(98.75,12.5)(98.58,13.08)(98.25,13.67)(98.08,14.25)(97.92,14.75)(97.83,15.33)(97.75,15.83)(97.58,16.33)(97.25,16.75)(96.92,16.92)(96.75,16.58)(96.25,16.33)(95.83,16.08)(95.5,15.75)(95.08,15.75)(95.08,15.75)(94.75,15.83)(94.25,16.08)(94.42,16.58)(94.58,17.17)(94.5,17.75)(94.33,18.33)(94.17,18.83)(93.83,18.83)(93.58,19.25)(93.75,19.5)(93.5,19.92)(93.17,19.92)(92.92,20.25)(92.5,20.67)(92.17,21.08)(92,21.5)(91.92,21.92)(91.75,22.33)(91.42,22.75)(90.92,22.42)(90.58,22.17)(90.25,21.83)(89.75,21.75)(89.25,21.67)(88.75,21.58)(88.33,21.5)(88.17,22.08)(88,21.75)(87.58,21.58)(87.08,21.42)(87.08,21.42)(86.92,21.08)(87,20.75)(86.75,20.33)(86.42,19.92)(85.92,19.75)(85.5,19.67)(85.08,19.33)(84.83,19)(84.5,18.58)(84.08,18.25)(83.67,18.08)(83.33,17.83)(83,17.42)(82.58,17.17)(82.33,17)(82.25,16.58)(82,16.33)(81.5,16.25)(81.17,16.17)(81,15.75)(80.42,15.83)(80.17,15.42)(80.08,15)(80.17,14.42)(80.25,13.83)(80.33,13.25)(80.25,12.67)(80,12.17)(79.75,11.75)(79.75,11.25)(79.75,11.25)(79.83,10.83)(79.83,10.25)(79.33,10.25)(79,9.75)(79,9.25)(78.58,9.17)(78.17,8.92)(78,8.33)(77.58,8.08)(77.17,8.25)(76.83,8.58)(76.5,9)(76.33,9.58)(76.17,10.17)(76,10.67)(75.83,11.17)(75.58,11.58)(75.25,12)(75,12.42)(74.83,13)(74.67,13.5)(74.5,14)(74.42,14.5)(74.08,15)(73.75,15.5)(73.5,16)(73.33,16.42)(73.25,17)(73.17,17.5)(73,18.08)(73,18.08)(72.83,18.67)(73,19)(72.83,19.25)(72.67,19.75)(72.75,20.25)(72.83,20.83)(72.58,21.25)(72.5,21.67)(72.58,22.17)(72.33,22.25)(72.17,21.83)(72.08,21.25)(71.67,20.92)(71.17,20.75)(70.58,20.75)(70.08,21)(69.67,21.42)(69.33,21.75)(69.08,22.25)(69.67,22.25)(70.17,22.42)(70.33,22.92)(69.75,22.75)(69.17,22.75)(68.67,23.08)(68.33,23.5)(67.92,23.75)(67.42,23.92)(67.33,24.33)(67.17,24.75)(67.17,24.75)(66.83,25)(66.58,25.33)(66.08,25.42)(65.58,25.33)(65.17,25.25)(64.67,25.17)(64.25,25.25)(63.83,25.42)(63.5,25.25)(62.92,25.25)(62.42,25.17)(61.83,25.08)(61.17,25.17)(60.75,25.25)(60.17,25.33)(59.67,25.33)(59.17,25.42)(58.58,25.58)(58,25.67)(57.33,25.75)(57.17,26.25)(57.08,26.67)(56.75,27)(56.25,27)(55.75,26.83)(55.33,26.67)(54.75,26.42)(54.33,26.67)(53.83,26.58)(53.5,26.92)(53.5,26.92)(53,27)(52.58,27.42)(52.17,27.67)(51.5,27.83)(51.25,28.25)(51.08,28.75)(50.67,29.25)(50.33,29.83)(50,30.17)(49.58,30)(49.25,30.25)(48.92,30.5)(48.58,30)(48.33,29.58)(47.75,29.42)(48.17,29.25)(48.33,28.75)(48.58,28.25)(48.83,27.83)(49.25,27.42)(49.67,27)(50.08,26.67)(50.25,26.25)(50.17,25.83)(50.5,25.42)(50.67,25.17)(51,25.58)(51.08,26)(51.33,26.08)(51.58,25.75)(51.58,25.75)(51.67,25.25)(51.58,24.75)(51.25,24.25)(51.67,24.25)(51.83,24)(52.42,23.92)(52.75,24.17)(53.25,24.17)(53.83,24.08)(54.25,24.17)(54.5,24.58)(55,24.92)(55.33,25.25)(55.67,25.58)(56.08,25.83)(56.25,26.17)(56.5,26.08)(56.42,25.83)(56.42,25.33)(56.42,24.75)(56.83,24.33)(57.25,24)(57.75,23.75)(58.25,23.67)(58.83,23.5)(59.08,23.08)(59.42,22.67)(59.83,22.42)(59.58,21.92)(59.33,21.42)(59.33,21.42)(59,21.17)(58.75,20.83)(58.5,20.42)(58.08,20.42)(57.83,20)(57.75,19.5)(57.92,19.08)(57.5,18.92)(57.08,18.83)(56.75,18.58)(56.58,18.08)(56.33,17.92)(55.92,17.83)(55.42,17.67)(55.25,17.17)(54.83,16.92)(54.42,17)(53.92,16.92)(53.42,16.67)(52.92,16.5)(52.42,16.25)(52.25,15.92)(52.25,15.58)(51.75,15.42)(51.25,15.25)(50.75,15.08)(50.25,14.92)(49.67,14.75)(49.17,14.5)(48.83,14.08)(48.83,14.08)(48.33,13.92)(47.83,13.92)(47.42,13.58)(47.08,13.33)(46.58,13.25)(46.08,13.33)(45.67,13.33)(45.42,13)(44.92,12.83)(44.42,12.75)(43.92,12.67)(43.5,12.75)(43.17,13.25)(43.25,13.75)(43.08,14)(42.92,14.58)(42.75,15.17)(42.67,15.67)(42.75,16.08)(42.67,16.5)(42.42,17)(42.08,17.5)(41.75,17.75)(41.5,18.17)(41.25,18.67)(41,19.17)(40.75,19.67)(40.33,20.08)(39.67,20.25)(39.5,20.58)(39.5,20.58)(39.17,20.92)(39.08,21.5)(39,21.92)(39.08,22.25)(39,22.75)(38.75,23.25)(38.42,23.75)(38,24.08)(37.5,24.33)(37.25,24.83)(37.08,25.33)(36.75,25.83)(36.42,26.17)(36.08,26.67)(35.75,27.17)(35.5,27.58)(35.17,28)(34.67,28)(34.83,28.5)(35,29)(35,29.5)(34.67,28.92)(34.42,28.33)(34.42,27.92)(34.17,27.75)(33.75,28.08)(33.42,28.42)(33.25,28.75)(33,29.08)(32.75,29.58)(32.75,29.58)(32.58,30) 
};
\addplot [
color=black,
solid,
forget plot
]
coordinates{
 (112.83,74.08)(111.58,74.33)(112.08,74.5)(113.33,74.42)(112.83,74.08) 
};
\addplot [
color=black,
solid,
forget plot
]
coordinates{
 (26.58,40.33)(27.17,40.33)(27.75,40.33)(27.75,40.33)(28.42,40.33)(29,40.33)(28.92,40.58)(29.42,40.67)(28.83,41)(28.17,41)(27.42,40.92)(27.08,40.67)(26.58,40.33) 
};
\addplot [
color=black,
solid,
forget plot
]
coordinates{
 (13.08,14.08)(13.33,13.75)(13.5,13.33)(13.83,13.08)(13.92,12.75)(14.17,12.5)(14.58,12.75)(15.08,12.92)(15.25,13.42)(14.75,13.42)(14.17,13.5)(14.17,13.83)(13.92,14.17)(13.42,14.25)(13.08,14.08) 
};
\addplot [
color=black,
solid,
forget plot
]
coordinates{
 (31.67,-2.08)(31.83,-2.75)(32.25,-2.33)(32.75,-2.5)(33.42,-2.5)(33.83,-2.17)(33.25,-2.08)(33.58,-1.75)(33.92,-1.33)(34.08,-0.75)(34.25,-0.42)(34,0.08)(33.5,0.25)(33,0.25)(32.42,0.17)(31.83,-0.17)(31.67,-0.75)(31.83,-1.17)(31.75,-1.58)(31.67,-2.08) 
};
\addplot [
color=black,
solid,
forget plot
]
coordinates{
 (29.25,-6)(29.5,-6.5)(29.83,-7)(30.33,-7.42)(30.58,-8.08)(30.5,-8.58)(31.08,-8.75)(31,-8.25)(30.75,-7.75)(30.5,-7.17)(30.58,-6.83)(30.17,-6.5)(30.0649494233928,-6.41753179592029)(29.9599328598126,-6.33504220530285)(29.8549498663,-6.25253151210506)(29.75,-6.17)(29.8125360887731,-6.06501062509922)(29.8750478303248,-5.96001408167863)(29.9375356568483,-5.85501049746473)(30,-5.75)(29.9574688781091,-5.60500451966582)(29.9149588769091,-5.46000597204522)(29.8724694369991,-5.31500443844044)(29.83,-5.17)(29.67,-4.58)(29.42,-4)(29.33,-3.33)(29.08,-4)(29.25,-4.5)(29.08,-5)(29.33,-5.58)(29.25,-6) 
};
\addplot [
color=black,
solid,
forget plot
]
coordinates{
 (34.08,-12.17)(34.17,-12.58)(34.33,-13)(34.33,-13.42)(34.67,-13.75)(34.75,-14.17)(35.08,-14.08)(35.33,-14.5)(35.25,-13.83)(34.92,-13.42)(34.83,-12.83)(34.83,-12.25)(34.872568658504,-12.1050096299713)(34.9150911338309,-11.9600127821487)(34.9575680428381,-11.8150095433381)(35,-11.67)(34.9173712079684,-11.5225346185063)(34.8348290866225,-11.3750459380663)(34.7523724209548,-11.2275342889031)(34.67,-11.08)(34.58,-10.58)(34.5,-9.92)(34.08,-9.58)(34,-9.83)(34.25,-10.83)(34.25,-11)(34.42,-11.75)(34.08,-12.17) 
};
\addplot [
color=black,
solid,
forget plot
]
coordinates{
 (27.5,42.5)(27.92,42.08)(28.08,41.67)(28.58,41.33)(29.17,41.17)(29.83,41.08)(30.5,41.17)(31.08,41.08)(31.58,41.33)(32.08,41.58)(32.58,41.83)(33.25,42)(33.83,41.92)(34.5,41.92)(35,42)(35.33,41.67)(36,41.67)(36.33,41.25)(36.75,41.25)(37.25,41)(37.67,41.08)(38.17,40.92)(38.75,41)(39.33,41.08)(39.92,41)(40.5,41.08)(41.08,41.42)(41.58,41.67)(41.67,42)(41.42,42.33)(41.42,42.33)(41.25,42.75)(41,42.92)(40.33,43.17)(39.83,43.42)(39.5,43.67)(39,44.08)(38.5,44.33)(38,44.42)(37.67,44.58)(37.25,44.75)(37,45.08)(36.67,45.08)(36.08,45)(35.58,45.08)(35.08,44.83)(34.58,44.75)(34.25,44.5)(33.75,44.33)(33.42,44.58)(33.58,45)(33.25,45.17)(32.67,45.33)(33.08,45.75)(33.75,45.92)(33.17,46.17)(32.58,46)(31.83,46.25)(32.08,46.58)(31.42,46.58)(30.83,46.58)(30.83,46.58)(30.58,46.17)(30.17,45.83)(29.58,45.67)(29.67,45.17)(29.5,44.83)(29,44.67)(28.67,44.33)(28.67,43.92)(28.58,43.42)(28,43.33)(27.92,42.83)(27.5,42.5) 
};
\addplot [
color=black,
solid,
forget plot
]
coordinates{
 (34.42,45.92)(34.75,45.75)(35,45.42)(35.5,45.25)(36.08,45.42)(36.67,45.42)(37.33,45.33)(37.75,45.83)(38.25,46.25)(37.92,46.42)(37.67,46.67)(38.17,46.67)(38.75,47)(39.33,47.08)(39,47.25)(38.42,47.17)(37.58,47.08)(36.83,46.75)(36.08,46.67)(35.42,46.42)(35,46.08)(34.42,45.92) 
};
\addplot [
color=black,
solid,
forget plot
]
coordinates{
 (46.67,44.75)(46.67,44.42)(47.08,44.25)(47.42,43.75)(47.42,43.42)(47.42,43)(47.83,42.5)(48.25,42.08)(48.67,41.83)(49,41.42)(49.17,41)(49.67,40.58)(49.815259208616,40.5402720653442)(49.9603458344429,40.5003624390396)(50.1052595413597,40.4602715930502)(50.25,40.42)(50.1248749689283,40.397701970215)(49.9998330400221,40.3752691624447)(49.874874591782,40.352701773244)(49.75,40.33)(49.42,40)(49.25,39.5)(49.17,39.17)(48.75,39.17)(48.83,38.58)(48.92,38.08)(49,37.58)(49.5,37.42)(50.17,37.33)(50.33,37.08)(51,36.75)(51.58,36.58)(52.42,36.75)(53.17,36.92)(53.92,36.83)(54,37.17)(54,37.17)(53.75,37.83)(53.75,38.5)(54,39.08)(53.58,39.17)(53.42,39.58)(53.5,40)(52.75,40)(52.67,40.33)(52.92,40.83)(53.5,40.67)(54.17,40.67)(54.75,41)(54.33,41.33)(53.92,41.67)(53.67,42.08)(53,42)(52.67,41.67)(52.75,41.25)(52.42,41.75)(52.397610967671,41.8550065741483)(52.3751482992219,41.9600087859341)(52.3526114822422,42.0650066048179)(52.33,42.17)(52.4145846722513,42.2725939625859)(52.4994450764392,42.3751255699114)(52.5845829395797,42.4775943931191)(52.67,42.58)(52.6276281940953,42.6425235348704)(52.5851711506072,42.7050314237896)(52.542628532189,42.7675236008948)(52.5,42.83)(51.83,42.83)(51.33,43.17)(51.17,43.67)(50.75,44.17)(50.17,44.33)(50.25,44.58)(50.92,44.58)(51.42,44.5)(51.42,44.5)(51.17,45)(51.314370766348,45.0827746816596)(51.4591600264912,45.1653669478635)(51.6043692743999,45.2477757414615)(51.75,45.33)(51.9372752953093,45.3529599144958)(52.1247015905973,45.3756135436531)(52.3122770936865,45.3979603995707)(52.5,45.42)(53.08,45.17)(53.67,45.25)(54.25,45.25)(53.92,45.5)(53.83,45.92)(53.67,46.42)(53.17,46.83)(52.58,47.17)(52.08,47.17)(51.42,47.17)(50.92,47.25)(50.17,46.83)(49.5,46.75)(48.92,46.58)(49,46.33)(48.5,46.25)(48.25,45.92)(47.58,46.08)(47.33,45.67)(47,45.17)(46.67,44.75) 
};
\addplot [
color=black,
solid,
forget plot
]
coordinates{
 (58.17,45)(58.17,44.5)(58.33,44.25)(58.33,43.67)(59,43.67)(59.67,43.67)(59.83,43.5)(60.5,43.67)(61.08,44.17)(61.08,44.67)(61.5,44.75)(61.83,45)(61.5,45.33)(61,45.75)(61,46)(61.33,46.42)(61.67,46.75)(61,46.5)(60.75,46.67)(60.08,46.5)(60.08,46.17)(59.75,46.33)(59.5,45.92)(59,45.92)(58.67,45.83)(58.58,45.42)(58.17,45) 
};
\addplot [
color=black,
solid,
forget plot
]
coordinates{
 (73.42,45.58)(73.83,45.25)(74.08,44.92)(74.25,45.67)(74.17,46)(74.75,46.17)(75.17,46.42)(76,46.58)(76.83,46.42)(77.33,46.48)(78,46.33)(78.75,46.42)(79.25,46.75)(78.75,46.75)(78.33,46.58)(77.83,46.67)(77.33,46.58)(76.83,46.67)(76.08,46.75)(75.25,46.67)(74.75,46.75)(74.17,46.42)(73.67,46.17)(73.42,45.58) 
};
\addplot [
color=black,
solid,
forget plot
]
coordinates{
 (103.58,51.58)(104.5,51.33)(105.25,51.5)(105.92,51.67)(106.25,52.17)(107,52.5)(107.83,52.67)(108.33,53)(109,53.5)(109.42,54)(109.5,54.5)(109.67,55.08)(109.92,55.58)(109.5,55.75)(109.25,55.5)(109.17,55.17)(108.83,54.75)(108.5,54.33)(108.17,53.92)(107.58,53.5)(107,53.08)(106.58,52.75)(106.08,52.5)(105.67,52.17)(105.25,51.83)(104.5,51.75)(103.58,51.58) 
};
\addplot [
color=black,
solid,
forget plot
]
coordinates{
 (30,61.08)(30.58,60.75)(30.92,60.33)(31.08,59.92)(31.5,59.92)(31.75,60.17)(32.5,60.17)(32.83,60.5)(32.83,60.83)(32.5,61.17)(31.83,61.33)(31.42,61.58)(30.75,61.58)(30.17,61.33)(30,61.08) 
};
\addplot [
color=black,
solid,
forget plot
]
coordinates{
 (34.5,61.75)(34.83,61.5)(35.5,61.33)(35.5,60.92)(36,60.92)(36.42,61.08)(36.42,61.42)(36,61.67)(35.75,62)(35.83,62.42)(35.25,62.67)(34.5,62.83)(34.83,62.58)(35.33,62.5)(35.33,62.17)(34.67,62.17)(34.5,61.75) 
};
\addplot [
color=black,
solid,
forget plot
]
coordinates{
 (43.25,-21.92)(43.58,-21.33)(43.92,-20.92)(44.08,-20.5)(44.42,-20)(44.5,-19.58)(44.25,-19)(44.25,-18.58)(44,-18)(44,-18)(43.92,-17.42)(44.25,-16.92)(44.5,-16.17)(44.92,-16.17)(45.42,-15.92)(45.92,-15.75)(46.42,-15.5)(46.83,-15.25)(47.25,-14.75)(47.67,-14.75)(47.58,-14.33)(47.92,-14.08)(47.83,-13.58)(48.33,-13.5)(48.75,-13.33)(48.83,-12.92)(48.83,-12.42)(49.25,-12)(49.58,-12.5)(49.83,-13)(49.92,-13.58)(50.17,-14.17)(50.33,-14.92)(50.42,-15.42)(50.08,-15.92)(49.67,-15.5)(49.75,-16.33)(49.67,-16.75)(49.42,-17)(49.42,-17.42)(49.42,-17.42)(49.33,-18)(49.25,-18.5)(49,-19.08)(48.83,-19.58)(48.67,-20.08)(48.5,-20.5)(48.42,-20.92)(48.25,-21.33)(48.08,-21.75)(47.92,-22.17)(47.83,-22.58)(47.75,-23)(47.67,-23.42)(47.5,-23.75)(47.33,-24.25)(47.17,-24.67)(46.83,-25.17)(46.42,-25.08)(45.92,-25.25)(45.5,-25.5)(45,-25.58)(44.58,-25.33)(44.17,-25)(43.75,-24.5)(43.67,-23.92)(43.58,-23.25)(43.33,-22.83)(43.25,-22.33)(43.25,-21.92) 
};
\addplot [
color=black,
solid,
forget plot
]
coordinates{
 (55.17,-21)(55.58,-20.92)(55.83,-21.17)(55.75,-21.33)(55.33,-21.33)(55.17,-21) 
};
\addplot [
color=black,
solid,
forget plot
]
coordinates{
 (57.33,-20.5)(57.5,-20)(57.75,-20.17)(57.75,-20.5)(57.33,-20.5) 
};
\addplot [
color=black,
solid,
forget plot
]
coordinates{
 (53.33,12.42)(53.67,12.67)(54.17,12.58)(54.58,12.5)(54.08,12.25)(53.67,12.25)(53.33,12.42) 
};
\addplot [
color=black,
solid,
forget plot
]
coordinates{
 (68.83,-48.83)(69,-48.67)(69.33,-49)(69.67,-49.25)(70,-49.17)(70.58,-49.08)(70.42,-49.42)(70,-49.42)(69.83,-49.67)(69.5,-49.5)(68.92,-49.67)(69.17,-49.25)(68.83,-48.83) 
};
\addplot [
color=black,
solid,
forget plot
]
coordinates{
 (92.6807250291134,11.5628310745619)(92.5312247144373,11.8211899644143)(92.7533132170992,12.7376824229216)(92.914330048594,12.7509134105868)(92.9652278118068,12.6121693538125)(92.6807250291134,11.5628310745619) 
};
\addplot [
color=black,
solid,
forget plot
]
coordinates{
 (92.8677246456791,12.9970189100361)(92.8272477119554,13.1610567732303)(92.8831930971716,13.4075404254359)(92.979454853423,13.4899831184673)(93.06455463069,13.4713620370773)(93.0769155006349,13.1872410914329)(92.8677246456791,12.9970189100361) 
};
\addplot [
color=black,
solid,
forget plot
]
coordinates{
 (96.4070710432786,2.36634815631389)(96.0803149633811,2.51322183889086)(95.809086015527,2.7189282267963)(95.8480427695392,2.85555812834429)(95.9739957390652,2.88238737331295)(96.2558388152619,2.61840095485437)(96.4070710432786,2.36634815631389) 
};
\addplot [
color=black,
solid,
forget plot
]
coordinates{
 (97.25,1.33)(97.5,1.5)(98,1)(98,0.58)(97.58,1)(97.25,1.33) 
};
\addplot [
color=black,
solid,
forget plot
]
coordinates{
 (98.75,-1)(99,-0.92)(99.42,-1.67)(99.08,-1.75)(98.83,-1.5)(98.75,-1) 
};
\addplot [
color=black,
solid,
forget plot
]
coordinates{
 (79.75,8.08)(79.83,8.5)(79.92,9)(80.17,9.42)(80,9.83)(80.58,9.5)(80.92,9)(81.25,8.5)(81.5,8)(81.83,7.58)(81.83,7)(81.67,6.5)(81.25,6.17)(80.67,6)(80.25,6)(80,6.42)(79.83,7.08)(79.75,7.5)(79.75,8.08) 
};
\addplot [
color=black,
solid,
forget plot
]
coordinates{
 (108.67,19.33)(109.08,19.58)(109.58,19.92)(110,20)(110.5,20)(111,20)(111,19.67)(110.67,19.33)(110.5,18.83)(110.17,18.5)(109.67,18.25)(109.25,18.25)(108.75,18.5)(108.67,18.92)(108.67,19.33) 
};
\addplot [
color=black,
solid,
forget plot
]
coordinates{
 (120.17,23.67)(120.5,24.17)(120.75,24.58)(121.08,25)(121.58,25.25)(121.92,25.08)(121.92,24.58)(121.67,24.08)(121.5,23.5)(121.33,23)(121,22.58)(120.83,22)(120.67,22.42)(120.33,22.58)(120.17,23.17)(120.17,23.67) 
};
\addplot [
color=black,
solid,
forget plot
]
coordinates{
 (129.5,33.25)(130,33.5)(130.42,33.83)(130.92,33.92)(131.08,33.67)(131.58,33.67)(131.67,33.25)(131.92,32.83)(131.67,32.42)(131.42,31.92)(131.33,31.42)(130.75,31.08)(130.17,31.25)(130.17,31.67)(130.17,32.08)(130.5,32.33)(130.5,32.75)(130.25,33.08)(130.17,32.75)(129.75,32.75)(129.5,33.25) 
};
\addplot [
color=black,
solid,
forget plot
]
coordinates{
 (132.25,33.42)(132.67,33.67)(132.83,34)(133.33,33.92)(133.58,34.25)(134.08,34.33)(134.58,34.17)(134.67,33.83)(134.33,33.58)(134.08,33.25)(133.75,33.5)(133.25,33.33)(133.08,33.08)(132.92,32.75)(132.42,32.92)(132.25,33.42) 
};
\addplot [
color=black,
solid,
forget plot
]
coordinates{
 (130.92,34.33)(131.42,34.5)(131.83,34.75)(132.33,35.08)(132.75,35.5)(133.25,35.5)(134,35.58)(134.67,35.67)(135.25,35.75)(135.67,35.5)(136,35.67)(136,36.08)(136.33,36.42)(136.67,36.83)(136.75,37.25)(137.17,37.5)(137,37.08)(137,36.75)(137.58,37)(138.17,37.17)(138.17,37.17)(138.58,37.42)(138.83,37.83)(139.33,38.08)(139.5,38.58)(139.83,39)(140,39.42)(140,39.92)(140,40.33)(140,40.67)(140.33,41.17)(140.67,40.92)(141.08,40.92)(141.08,41.17)(140.75,41.17)(140.83,41.5)(141.42,41.33)(141.33,41)(141.42,40.58)(141.75,40.17)(141.92,39.67)(141.83,39.08)(141.5,38.75)(141.5,38.33)(141.08,38.33)(140.92,38.08)(141,37.58)(140.92,37)(140.67,36.75)(140.58,36.17)(140.75,35.75)(140.75,35.75)(140.42,35.5)(140.33,35.17)(139.75,34.92)(139.83,35.5)(139.58,35.25)(139.08,35.25)(138.83,34.58)(138.67,35.08)(138.08,34.58)(137.25,34.58)(136.75,35)(136.5,34.67)(136.75,34.33)(136.17,33.92)(135.75,33.42)(135.42,33.5)(135.17,33.92)(135.17,34.33)(135.5,34.75)(134.92,34.75)(134.5,34.83)(133.92,34.58)(133.5,34.5)(132.92,34.33)(132.33,34.25)(132,33.92)(131.5,34)(130.92,33.92)(130.92,34.33) 
};
\addplot [
color=black,
solid,
forget plot
]
coordinates{
 (139.83,42.58)(140.42,42.92)(140.33,43.25)(140.75,43.17)(141.25,43.17)(141.33,43.42)(141.25,43.75)(141.58,44)(141.67,44.33)(141.75,44.75)(141.5,45.25)(141.92,45.5)(142.33,45.17)(142.67,44.75)(142.67,44.75)(143,44.5)(143.58,44.17)(144.17,44)(144.67,43.92)(145.33,44.25)(145,43.75)(145.25,43.58)(145.25,43.33)(145.83,43.42)(145.33,43.17)(144.75,42.92)(144.17,43)(143.67,42.67)(143.33,42.33)(143.25,42)(142.5,42.25)(141.83,42.58)(141.33,42.5)(140.92,42.33)(140.67,42.5)(140.33,42.5)(140.25,42.25)(140.67,42.08)(141.08,41.83)(140.5,41.67)(140,41.42)(140,42)(139.75,42.25)(139.83,42.58) 
};
\addplot [
color=black,
solid,
forget plot
]
coordinates{
 (127.667993973762,26.079590772794)(127.656370237941,26.3182613929584)(127.817396903917,26.563335547369)(128.303264680748,26.8714925225757)(128.340149504718,26.8272743638904)(128.347796362212,26.738194004126)(127.96154581203,26.4189924871751)(127.792724706209,26.1224784272633)(127.667993973762,26.079590772794) 
};
\addplot [
color=black,
solid,
forget plot
]
coordinates{
 (129.295547449471,28.0781422840092)(129.24272942846,28.1157882544363)(129.231555240267,28.266503271161)(129.293335899178,28.3733150541133)(129.524954544999,28.4989357818367)(129.726385560287,28.4427377926625)(129.295547449471,28.0781422840092) 
};
\addplot [
color=black,
solid,
forget plot
]
coordinates{
 (145.442132687183,43.956452059924)(146.070049041262,44.5230760466893)(146.394907446495,44.4503364352471)(145.442132687183,43.956452059924) 
};
\addplot [
color=black,
solid,
forget plot
]
coordinates{
 (146.92,44.42)(147.42,45)(147.92,45.33)(148.33,45.25)(148.83,45.5)(148.17,45.08)(147.5,44.75)(146.92,44.42) 
};
\addplot [
color=black,
solid,
forget plot
]
coordinates{
 (149.5,45.67)(150,46.08)(150.5,46.25)(149.83,45.75)(149.5,45.67) 
};
\addplot [
color=black,
solid,
forget plot
]
coordinates{
 (155.17,50.27)(155.67,50.42)(156.08,50.75)(156.08,50.5)(155.83,50.2)(155.17,50)(155.17,50.27) 
};
\addplot [
color=black,
solid,
forget plot
]
coordinates{
 (137.25,54.83)(137.5,55.17)(138.08,55)(137.58,54.58)(137.25,54.83) 
};
\addplot [
color=black,
solid,
forget plot
]
coordinates{
 (163.33,58.5)(163.67,59)(164.25,59.17)(164.42,58.92)(163.33,58.5) 
};
\addplot [
color=black,
solid,
forget plot
]
coordinates{
 (141.75,53.33)(142.25,53.58)(142.312221952382,53.6425487044387)(142.374628684371,53.7050650530651)(142.437221073011,53.7675488754291)(142.5,53.83)(142.43797228482,53.9350484628332)(142.3756315184,54.0400648090015)(142.312975000078,54.1450487514848)(142.25,54.25)(142.354846890951,54.2701367171866)(142.459796048035,54.2901823934683)(142.564847181798,54.310136872925)(142.67,54.33)(142.92,54)(143,53.5)(143.17,53)(143.17,52.5)(143.08,52)(143.33,51.5)(143.5,51)(143.67,50.5)(143.92,50)(144.17,49.5)(144.33,49.08)(143.75,49.33)(143.25,49.33)(143.25,49.33)(142.92,49.08)(142.83,48.75)(142.58,48.33)(142.5,47.83)(142.67,47.5)(143,47.25)(143.08,46.92)(143.42,46.75)(143.42,46.25)(143,46.58)(142.58,46.67)(142.25,46.42)(142,45.92)(141.75,46.58)(142,47)(142,47.58)(142.17,48)(142,48.5)(141.83,48.75)(142,49.08)(142.08,49.5)(142.08,50)(142,50.5)(142.08,51)(142,51.42)(141.67,51.75)(141.67,52.25)(141.83,52.75)(141.75,53.33) 
};
\addplot [
color=black,
solid,
forget plot
]
coordinates{
 (180,71)(178.83,70.83)(178.67,71.12)(180,71.53)(181.75,71.53) 
};
\addplot [
color=black,
solid,
forget plot
]
coordinates{
 (180.83,70.9)(180,71) 
};
\addplot [
color=black,
solid,
forget plot
]
coordinates{
 (140.58,73.5)(141,73.83)(142.25,73.92)(143.5,73.5)(143.5,73.17)(141.58,73.25)(140.58,73.5) 
};
\addplot [
color=black,
solid,
forget plot
]
coordinates{
 (140.17,74.25)(141,74.25)(141,74)(140.33,73.92)(140.17,74.25) 
};
\addplot [
color=black,
solid,
forget plot
]
coordinates{
 (146.5,75.58)(147.33,75.42)(149.25,75.33)(151,75.17)(150.5,74.75)(149.25,74.75)(147.25,75.08)(146.5,75.58) 
};
\addplot [
color=black,
solid,
forget plot
]
coordinates{
 (137.17,75.25)(137.5,75.92)(139,76.17)(141,75.75)(142,76.17)(143.25,75.83)(144,75.83)(145,75.67)(145,75.42)(144,75.08)(142.83,74.83)(142,75)(140.5,74.83)(139.25,74.67)(138.08,74.75)(137.17,75.25) 
};
\addplot [
color=black,
solid,
forget plot
]
coordinates{
 (99.33,77.92)(99.83,78.33)(100.58,78.83)(101.25,79.25)(102.33,79.42)(103.92,79.17)(105,78.83)(105,78.33)(103.17,78.17)(101,78.17)(99.33,77.92) 
};
\addplot [
color=black,
solid,
forget plot
]
coordinates{
 (91.17,80.08)(92.67,80.5)(93.17,80.92)(95.58,81.25)(96.42,81)(96.6955331083988,80.9378008414093)(96.9673348412742,80.8753983763956)(97.2354694295564,80.8127967479608)(97.5,80.75)(97.3725062919965,80.6875661738073)(97.2466964414902,80.6250876435922)(97.1225381731421,80.5625652972709)(97,80.5)(97.14870863738,80.4175918568378)(97.2949038608617,80.3351214304415)(97.4386476437407,80.2525903018119)(97.58,80.17)(99.25,80)(99.4855830234185,79.9177424830354)(99.7173884722545,79.835320687667)(99.9455002402843,79.7527385782599)(100.17,79.67)(99.9975022036828,79.5651333681195)(99.8284007827857,79.4601760528307)(99.6625980705934,79.3551307367883)(99.5,79.25)(99.83,78.83)(98.33,78.75)(96.75,78.92)(94.67,79.08)(93.83,79.5)(91.83,79.67)(91.17,80.08) 
};
\addplot [
color=black,
solid,
forget plot
]
coordinates{
 (51.67,71.58)(51.67,71.7050000000001)(51.67,71.8300000000001)(51.67,71.955)(51.67,72.08)(51.9372513464786,72.1430533847988)(52.2063294417687,72.2057404440702)(52.47724286286,72.2680572892352)(52.75,72.33)(52.688378019499,72.4150291840309)(52.6261762942193,72.5000391012074)(52.5633864615091,72.5850294691485)(52.5,72.67)(52.5456416833647,72.7013297118346)(52.591443891652,72.7326490652087)(52.6374074470827,72.7639580036132)(52.6835331770781,72.7952564701456)(52.729821914297,72.8265444075061)(52.7762744966828,72.8578217580013)(52.8228917674847,72.8890884635299)(52.8696745753075,72.9203444655871)(53.0584783383816,73.0452602546236)(53.25,73.17)(53.517061283295,73.2330234478199)(53.7860743195746,73.2957005471722)(54.0570502371517,73.3580273821441)(54.33,73.42)(54.206824243497,73.5026103203792)(54.0824438344976,73.5851478337515)(53.9568415843733,73.6676114356024)(53.83,73.75)(54.92,74.17)(55.75,74.75)(55.75,75.08)(57.5,75.33)(57.5,75.33)(58.17,75.58)(59.25,75.92)(61.17,76.25)(63.17,76.25)(65.17,76.42)(66,76.75)(67.75,77)(69.08,76.83)(68.83,76.33)(67.25,76.08)(65.5,75.83)(63.83,75.67)(62.25,75.33)(60.75,75)(59.83,74.58)(58.67,74.17)(57.83,73.75)(57,73.33)(56.17,72.92)(55.58,72.5)(55.5,72)(55.75,71.5)(56.33,71.08)(56.6691905277101,70.9984117229016)(57.0055769432049,70.9162103789705)(57.3391747779739,70.8334038651029)(57.67,70.75)(57.4813100298068,70.7077881729807)(57.293414470616,70.6653833859811)(57.1063116882251,70.6227869073451)(56.92,70.58)(55.33,70.67)(53.83,70.83)(53.5832212433306,70.9155014003511)(53.3343086055768,71.0006715310699)(53.0832416801531,71.0855059119766)(52.83,71.17)(52.5445995091537,71.2731626159639)(52.2561576693501,71.3758883606905)(51.9646370543405,71.4781699575122)(51.67,71.58) 
};
\addplot [
color=black,
solid,
forget plot
]
coordinates{
 (48.75,68.75)(48.5,69.25)(49.17,69.58)(50.33,69.25)(50,68.92)(48.75,68.75) 
};
\addplot [
color=black,
solid,
forget plot
]
coordinates{
 (63.2489368203742,80.6666754169739)(62.9479843141808,80.6745569077346)(63.1449756880921,80.8487189631018)(63.3672238491184,80.9379945870164)(64.3083405649349,81.0590142670808)(64.7424302831953,81.2686854929056)(65.141998670896,81.1606228564974)(65.3645532056556,81.0426940893492)(65.23470514731,80.8940394356959)(64.8409013953816,80.7740079487712)(63.2489368203742,80.6666754169739) 
};
\addplot [
color=black,
solid,
forget plot
]
coordinates{
 (59.6481436751265,80.3914706656765)(59.456365590854,80.7020921969396)(59.579587420369,80.8246532008455)(59.789793960684,80.8689267222993)(60.4063904131603,80.800055202714)(61.367126445532,80.8086511883287)(61.9936147579593,80.8966716757807)(62.217430409412,80.6700442669082)(61.8583929379152,80.5571380427097)(61.2053397208823,80.4004180732853)(60.7563504534356,80.4882825276409)(59.6481436751265,80.3914706656765) 
};
\addplot [
color=black,
solid,
forget plot
]
coordinates{
 (56.6393603475741,79.8720763762114)(56.5422918291816,79.9258121962952)(56.0693506506449,79.9820451209876)(56.1000771471188,80.1152435379924)(56.2996418004118,80.2778949281676)(57.0281262321584,80.2387936307562)(57.043950303732,80.1812689407859)(56.9259888813792,79.9343604620209)(56.6393603475741,79.8720763762114) 
};
\addplot [
color=black,
solid,
forget plot
]
coordinates{
 (57.50830976225,80.4523325647651)(58.450147985325,80.427747800857)(59.0024798431139,80.3428611752054)(58.6139275050242,80.1989998792141)(57.7287372123551,80.0374961336987)(57.6403322247771,80.1042686647626)(57.50830976225,80.4523325647651) 
};
\addplot [
color=black,
solid,
forget plot
]
coordinates{
 (54.25,80.8)(54.83,81.03)(56.5,81.03)(56.5,81.42)(58.5,81.75)(58.83,81.33)(57.67,81)(58.5,80.8)(56.33,80.63)(54.25,80.8) 
};
\addplot [
color=black,
solid,
forget plot
]
coordinates{
 (53.17,80.25)(54,80.5)(54.2542719621673,80.4452708359674)(54.5056695864294,80.3903590493634)(54.7542325371929,80.3352677532055)(55,80.28)(54.5414803049968,80.2734138874779)(54.0836036130491,80.2662175967534)(53.6264252443471,80.2584124801774)(53.17,80.25) 
};
\addplot [
color=black,
solid,
forget plot
]
coordinates{
 (44.5,80.55)(47,80.9)(48,80.75)(48,80.55)(50.33,80.78)(51.17,80.67)(50,80.4)(50,80.12)(48.42,80.05)(47,80.17)(47,80.42)(45.67,80.42)(44.5,80.55) 
};
\addplot [
color=black,
solid,
forget plot
]
coordinates{
 (11,79.75)(13.83,79.83)(15.5,79.67)(16.33,80.03)(18,79.92)(18,80.25)(20,80.5)(21,80.17)(23,80.42)(24.83,80.25)(24.83,80.25)(27,80.17)(27,79.87)(25.67,79.42)(23.83,79.17)(20.92,79.33)(18.75,79.7)(19,79.17)(21.5,78.83)(22.17,78.42)(23.17,78.08)(24.83,77.75)(22.67,77.25)(20.83,77.42)(21.67,77.92)(20.67,78.18)(20.25,78.63)(19,78.42)(18.92,78.03)(18.08,77.5)(17.33,77)(17,76.5)(15.17,77.02)(14,77.5)(13.67,77.95)(14.0166161408189,78.0181533534696)(14.3671272374509,78.0858760822362)(14.7215747705073,78.153160799991)(15.08,78.22)(14.6025317847339,78.2211929291399)(14.1250000000016,78.2215905990401)(13.6474682152693,78.22119292914)(13.17,78.22)(11.67,78.7)(11,79.33)(11,79.75) 
};
\addplot [
color=black,
solid,
forget plot
]
coordinates{
 (-7.42,62.08)(-7.25,62.3)(-6.33,62.37)(-6.92,62)(-7.42,62.08) 
};
\addplot [
color=black,
solid,
forget plot
]
coordinates{
 (-1.38343958643956,59.9485027188086)(-1.44895988871044,60.0779174778201)(-1.40667454715998,60.2220852429074)(-1.48211854718674,60.3864762186728)(-1.28568358986622,60.517910131491)(-1.22755782940787,60.2983366822249)(-1.08100916185137,60.1686759484348)(-1.38343958643956,59.9485027188086) 
};
\addplot [
color=black,
solid,
forget plot
]
coordinates{
 (-3.03740021907134,58.8595972966356)(-3.28080657166615,58.9399113750031)(-3.40120235343843,59.0396805602362)(-3.33469190451689,59.1908789323131)(-3.15674380221521,59.2120648219004)(-2.92567125930044,58.9822021959451)(-2.96661614952165,58.8734798359035)(-3.03740021907134,58.8595972966356) 
};
\addplot [
color=black,
solid,
forget plot
]
coordinates{
 (-7.25,57.58)(-7,57.83)(-7,58.17)(-6.17,58.5)(-6.42,58)(-7.25,57.58) 
};
\addplot [
color=black,
solid,
forget plot
]
coordinates{
 (9.75,55.47)(10.28,55.58)(10.75,55.37)(10.7,55.05)(10.08,55.05)(9.75,55.47) 
};
\addplot [
color=black,
solid,
forget plot
]
coordinates{
 (11.08,55.7)(11.75,55.9)(12.42,56.1)(12.4603102508089,56.0000195036025)(12.5004124564284,55.9000259275303)(12.5403084389631,55.8000193880268)(12.58,55.7)(12.4770490174094,55.6426284617988)(12.3743994098077,55.5851709923572)(12.2720500970333,55.5276280272981)(12.17,55.47)(12.42,55.28)(11.83,55.13)(11.25,55.2)(11.08,55.7) 
};
\addplot [
color=black,
solid,
forget plot
]
coordinates{
 (11,54.83)(11.58,54.92)(12.08,54.83)(11.83,54.67)(11.33,54.58)(11,54.83) 
};
\addplot [
color=black,
solid,
forget plot
]
coordinates{
 (18.08,57.5)(18.42,57.78)(19.17,57.93)(18.75,57.7)(18.75,57.28)(18.25,57.08)(18.08,57.5) 
};
\addplot [
color=black,
solid,
forget plot
]
coordinates{
 (21.83,58.33)(22.33,58.62)(23.17,58.5)(22.67,58.3)(22.17,58.1)(21.83,58.33) 
};
\addplot [
color=black,
solid,
forget plot
]
coordinates{
 (22.33,58.88)(22.5,59.08)(23,58.83)(22.42,58.72)(22.33,58.88) 
};
\addplot [
color=black,
solid,
forget plot
]
coordinates{
 (-5.58,50.08)(-5,50.42)(-4.5,50.75)(-4.08,51.17)(-3.5,51.17)(-3,51.17)(-2.67,51.58)(-3.33,51.33)(-3.83,51.58)(-4.33,51.67)(-5,51.58)(-5.04232414009193,51.6425229606329)(-5.08476514799345,51.705030665266)(-5.12732358109048,51.7675230373853)(-5.17,51.83)(-5.02311639520253,51.8927759721379)(-4.87582243418472,51.9553685809607)(-4.72811725334462,52.0177769001319)(-4.58,52.08)(-4.08,52.25)(-4.08,52.75)(-4.67,52.75)(-4.33,53.08)(-4.5,53.33)(-3.58,53.25)(-3,53.33)(-3,53.75)(-2.67,54.17)(-3.17,54.17)(-3.58,54.5)(-3.25,54.92)(-4,54.75)(-4.75,54.58)(-4.83198161424289,54.6650837767226)(-4.91430727780454,54.750111975874)(-4.9969792980365,54.8350841880072)(-5.08,54.92)(-4.97861561400311,55.0651283348393)(-4.87649348347627,55.2101718368697)(-4.77362464488855,55.3551294248244)(-4.67,55.5)(-4.7718531256622,55.5826277925148)(-4.87413569662005,55.6651708051968)(-4.9768504149839,55.7476284166117)(-5.08,55.83)(-5.08,55.83)(-5.58,55.42)(-5.58,55.75)(-6.17,55.75)(-5.5,56.08)(-5.42,56.5)(-6,56.33)(-6,56.67)(-5.58,57.17)(-6.25,57.17)(-6.67,57.42)(-6.33,57.58)(-5.83,57.42)(-5.75,57.75)(-5.33,58.08)(-5,58.5)(-4.08,58.5)(-3.17,58.58)(-3,58.33)(-3.83,58)(-4.17,57.58)(-3.42,57.67)(-2.75,57.67)(-1.92,57.67)(-1.83,57.33)(-2.08,57)(-2.42,56.58)(-2.83,56.25)(-3.33,56)(-2.67,56)(-2,55.83)(-2,55.83)(-1.67,55.5)(-1.5,55)(-1.17,54.58)(-0.5,54.42)(-0.25,54)(-0.08,53.67)(0.17,53.42)(0.33,53.08)(0,52.83)(0.25,52.67)(0.58,52.92)(1,52.92)(1.58,52.75)(1.83,52.42)(1.58,52)(1,51.75)(0.58,51.33)(1.42,51.25)(0.92,50.83)(0.08,50.67)(-0.67,50.67)(-1.42,50.75)(-2.08,50.5)(-2.83,50.67)(-3.5,50.58)(-3.67,50.17)(-4.5,50.33)(-5.08,50)(-5.58,50.08) 
};
\addplot [
color=black,
solid,
forget plot
]
coordinates{
 (-10.33,52.17)(-9.75,52.17)(-9.5,52.67)(-9,53.17)(-9.5,53.17)(-10,53.5)(-9.5,53.75)(-10,53.92)(-9.75,54.25)(-9.08,54.25)(-8.5,54.25)(-8.17,54.5)(-8.67,54.67)(-8.25,55.08)(-7.5,55.25)(-6.75,55.17)(-6.08,55.08)(-5.75,54.75)(-5.58,54.25)(-6.25,53.92)(-6.08,53.5)(-6,53)(-6.17,52.58)(-6.33,52.17)(-7.25,52.08)(-8,51.75)(-8.75,51.58)(-9.58,51.5)(-10.08,51.75)(-10.33,52.17) 
};
\addplot [
color=black,
solid,
forget plot
]
coordinates{
 (8.67,42.5)(9.17,42.67)(9.42,43)(9.42,42.5)(9.5,42.17)(9.33,41.83)(9.17,41.33)(8.67,41.75)(8.58,42.08)(8.67,42.5) 
};
\addplot [
color=black,
solid,
forget plot
]
coordinates{
 (8.17,40.92)(8.42,40.83)(8.75,40.92)(9.25,41.25)(9.67,40.83)(9.75,40.42)(9.75,40)(9.67,39.58)(9.5,39.17)(9.08,39.17)(8.75,38.83)(8.33,39.17)(8.42,39.67)(8.42,40.17)(8.17,40.58)(8.17,40.92) 
};
\addplot [
color=black,
solid,
forget plot
]
coordinates{
 (12.33,37.92)(12.75,38.17)(13.25,38.17)(13.67,38)(14.25,38)(15,38.08)(15.67,38.25)(15.33,37.83)(15.17,37.33)(15.33,37)(15.17,36.67)(14.5,36.75)(14.17,37.08)(13.67,37.17)(13,37.5)(12.67,37.5)(12.33,37.92) 
};
\addplot [
color=black,
solid,
forget plot
]
coordinates{
 (21.08,37.75)(21.33,38.08)(21.83,38.25)(22.33,38.08)(22.83,37.83)(23.17,37.33)(22.67,37.42)(22.83,37)(23.08,36.42)(22.67,36.75)(22.33,36.42)(22.08,36.75)(21.67,36.75)(21.5,37)(21.58,37.42)(21.08,37.75) 
};
\addplot [
color=black,
solid,
forget plot
]
coordinates{
 (23.5,35.17)(23.58,35.58)(24.17,35.5)(24.25,35.33)(24.75,35.42)(25.17,35.3)(25.75,35.33)(25.75,35.08)(26.25,35.25)(26.17,34.97)(25.5,34.93)(24.83,34.9)(24.5,35.08)(24,35.17)(23.5,35.17) 
};
\addplot [
color=black,
solid,
forget plot
]
coordinates{
 (32.25,35)(32.75,35.08)(33,35.25)(33.42,35.33)(33.92,35.42)(34.5,35.58)(33.92,35.17)(34.08,34.92)(33.67,34.92)(33.5,34.7)(33,34.58)(32.42,34.67)(32.25,35) 
};
\addplot [
color=black,
solid,
forget plot
]
coordinates{
 (2.25,39.58)(2.92,39.92)(3.42,39.67)(3.08,39.27)(2.73,39.5)(2.25,39.58) 
};
\addplot [
color=black,
solid,
forget plot
]
coordinates{
 (27.67,36.17)(28.25,36.45)(28.08,36.08)(27.75,35.88)(27.67,36.17) 
};
\addplot [
color=black,
solid,
forget plot
]
coordinates{
 (26.4216215347702,39.0073952822487)(26.3299677129587,39.0479510917006)(26.212250985767,39.20596678632)(26.0299533870444,39.1036273472459)(25.9003374018617,39.1469304274766)(25.8589423223557,39.2177280900838)(25.8896235694489,39.2743582217914)(26.3033175689478,39.3774659437758)(26.3831308795929,39.3552741227911)(26.5096723858048,39.1366036028302)(26.4216215347702,39.0073952822487) 
};
\addplot [
color=black,
solid,
forget plot
]
coordinates{
 (26.0002418228706,38.2570834577285)(25.8130822631503,38.6004044822847)(26.0178199243303,38.5677866331395)(26.1785613169641,38.4164823190358)(26.0884004834629,38.2841471734031)(26.0002418228706,38.2570834577285) 
};
\addplot [
color=black,
solid,
forget plot
]
coordinates{
 (-28.1597661594428,38.5122903980545)(-28.0293893694369,38.4166856182679)(-28.1929632679821,38.3839155864956)(-28.4735454097547,38.39997977563)(-28.5366321878669,38.4655060432826)(-28.4901511525856,38.577908585877)(-28.1597661594428,38.5122903980545) 
};
\addplot [
color=black,
solid,
forget plot
]
coordinates{
 (-25.2548488430309,37.6972938234956)(-25.7897496335887,37.7382064684788)(-25.8258174925971,37.7829779636578)(-25.8139606469972,37.9246567674286)(-25.4882663179953,37.8627799910007)(-25.2349431796117,37.895945706446)(-25.1679281544546,37.7880436033691)(-25.2548488430309,37.6972938234956) 
};
\addplot [
color=black,
solid,
forget plot
]
coordinates{
 (-16.83,28.33)(-16.5,28.38)(-16.17,28.58)(-16.33,28.17)(-16.58,28)(-16.83,28.33) 
};
\addplot [
color=black,
solid,
forget plot
]
coordinates{
 (-15.8,28)(-15.67,28.17)(-15.37,28.08)(-15.33,27.83)(-15.67,27.75)(-15.8,28) 
};
\addplot [
color=black,
solid,
forget plot
]
coordinates{
 (-13.7387047684927,28.8331574988704)(-13.7870708634293,28.9411547621885)(-13.7669375010774,29.0434560063252)(-13.541560382708,29.1478747199861)(-13.4282554645504,29.1371633158594)(-13.412159507578,29.0418169209511)(-13.7387047684927,28.8331574988704) 
};
\addplot [
color=black,
solid,
forget plot
]
coordinates{
 (-14.2095413230737,28.1886056482134)(-14.1822604675603,28.3599452791151)(-13.9612548658688,28.641902186767)(-13.8980852355612,28.4556789848521)(-13.9338371551099,28.3387778294499)(-14.2095413230737,28.1886056482134) 
};
\addplot [
color=black,
solid,
forget plot
]
coordinates{
 (8.5,3.33)(8.67,3.75)(8.92,3.58)(8.75,3.25)(8.5,3.33) 
};
\addplot [
color=black,
solid,
forget plot
]
coordinates{
 (115.17,-34.17)(115.17,-33.58)(115.58,-33.5)(115.83,-33.25)(115.75,-32.92)(115.83,-32.33)(115.83,-31.75)(115.58,-31.42)(115.42,-31)(115.17,-30.5)(115,-30)(115,-29.5)(114.92,-29.08)(114.58,-28.58)(114.17,-28.08)(114.17,-27.67)(113.92,-27)(113.5,-26.67)(113.33,-26.17)(113.83,-26.58)(113.5,-25.67)(113.75,-26.08)(113.75,-26.08)(114.17,-26.42)(114.17,-25.83)(113.92,-25.42)(113.67,-25)(113.5,-24.42)(113.42,-23.92)(113.83,-23.5)(113.83,-23.08)(113.67,-22.67)(113.83,-22.25)(114.08,-21.83)(114.33,-22.33)(114.58,-21.92)(115.08,-21.67)(115.5,-21.5)(115.83,-21.08)(116.25,-20.83)(116.75,-20.67)(117.25,-20.67)(117.75,-20.67)(118.25,-20.33)(118.75,-20.33)(119.17,-20)(119.75,-20)(120.33,-19.83)(120.92,-19.75)(121.33,-19.42)(121.67,-19.08)(121.92,-18.58)(122.33,-18.17)(122.33,-18.17)(122.25,-17.83)(122.17,-17.42)(122.58,-17)(123,-16.5)(123.25,-17.08)(123.5,-17.5)(123.92,-17)(123.58,-16.67)(123.75,-16.17)(124.33,-16.42)(124.58,-16)(124.67,-15.5)(125.17,-15.08)(125.25,-14.67)(125.92,-14.67)(126.08,-14.33)(126.58,-14.25)(126.83,-13.92)(127.42,-14)(127.83,-14.42)(128.17,-14.75)(128.17,-15.25)(128.67,-14.92)(129.08,-14.92)(129.67,-15.25)(129.83,-14.83)(129.33,-14.5)(129.75,-14.08)(129.92,-13.67)(130.42,-13.58)(130.42,-13.58)(130.25,-13.17)(130.5,-12.75)(131.08,-12.33)(131.75,-12.33)(132.42,-12.33)(132.482587881908,-12.2050210088767)(132.545116728733,-12.0800279038991)(132.607587211711,-11.9550208471094)(132.67,-11.83)(132.584905851333,-11.7275374550532)(132.499874867265,-11.6250497773624)(132.414906449288,-11.5225372111452)(132.33,-11.42)(132.497428660869,-11.4601433154948)(132.664904834399,-11.5001913334305)(132.83242859112,-11.5401436845294)(133,-11.58)(133.5,-11.92)(134.08,-12)(134.67,-12.17)(135.25,-12.33)(135.83,-12.17)(136.08,-12.58)(136.5,-12.08)(137,-12.42)(136.67,-12.83)(136.58,-13.33)(136,-13.33)(136,-13.83)(136,-14.25)(135.75,-14.58)(135.58,-15)(136,-15.33)(136.42,-15.5)(136.58,-16)(137.33,-16)(137.75,-16.33)(138.08,-16.58)(138.67,-16.83)(138.67,-16.83)(139.17,-17)(139.42,-17.42)(140,-17.83)(140.5,-17.75)(141,-17.5)(141,-17)(141.25,-16.58)(141.5,-16.08)(141.58,-15.58)(141.67,-15.08)(141.58,-14.58)(141.58,-14.08)(141.67,-13.58)(141.75,-13)(141.75,-12.67)(141.75,-12.17)(142.08,-11.83)(142.08,-11.25)(142.42,-10.83)(142.83,-11.08)(142.92,-11.67)(143.08,-12.17)(143.33,-12.67)(143.58,-13.25)(143.58,-13.75)(143.75,-14.33)(144.08,-14.5)(144.58,-14.33)(144.92,-14.67)(145.33,-15)(145.33,-15)(145.33,-15.5)(145.5,-16)(145.5,-16.58)(145.83,-17)(146.08,-17.33)(146.17,-17.83)(146.08,-18.33)(146.33,-18.67)(146.42,-19)(147,-19.33)(147.5,-19.42)(147.83,-19.83)(148.33,-20.17)(148.83,-20.42)(148.83,-20.83)(149.17,-21.08)(149.33,-21.42)(149.5,-21.92)(149.75,-22.42)(150,-22.08)(150.42,-22.42)(150.83,-22.67)(150.83,-23.17)(151,-23.58)(151.33,-23.92)(151.83,-24.08)(152.17,-24.58)(152.58,-25.08)(152.83,-25.5)(153.08,-26.08)(153.08,-26.08)(153.08,-26.5)(153.17,-27.08)(153.33,-27.58)(153.5,-28.17)(153.67,-28.67)(153.5,-29)(153.42,-29.5)(153.33,-30)(153.08,-30.58)(153.08,-31)(152.92,-31.5)(152.75,-31.92)(152.5,-32.33)(152.25,-32.75)(151.83,-33)(151.5,-33.42)(151.33,-33.83)(151.08,-34.33)(150.83,-34.92)(150.58,-35.25)(150.25,-35.75)(150.17,-36.25)(150,-36.75)(150,-37.17)(149.92,-37.58)(149.42,-37.83)(148.75,-37.83)(148.17,-37.92)(147.83,-38.08)(147.25,-38.42)(147.25,-38.42)(146.83,-38.67)(146.42,-39.17)(146,-38.92)(145.58,-38.67)(145.5,-38.25)(145,-38.42)(144.83,-37.83)(144.42,-38.25)(144,-38.5)(143.5,-38.83)(143.08,-38.67)(142.67,-38.42)(142.25,-38.33)(141.75,-38.25)(141.42,-38.42)(141.17,-38.08)(140.67,-38)(140.25,-37.67)(139.92,-37.33)(139.75,-37)(139.92,-36.75)(139.67,-36.33)(139.33,-35.92)(138.83,-35.58)(138.17,-35.67)(138.42,-35.42)(138.58,-35)(138.42,-34.67)(138.08,-34.25)(137.92,-34.75)(137.92,-34.75)(137.75,-35.17)(137.33,-35.17)(136.92,-35.25)(137,-35)(137.42,-34.83)(137.5,-34.5)(137.58,-34)(137.92,-33.58)(137.83,-32.83)(137.5,-33.17)(137.25,-33.67)(136.75,-33.92)(136.33,-34.25)(135.92,-34.58)(135.67,-35)(135.42,-34.58)(135.25,-34.17)(135,-33.75)(134.83,-33.25)(134.17,-33)(134.33,-32.58)(133.92,-32.33)(133.42,-32.17)(132.83,-32)(132.25,-32)(131.83,-31.75)(131.33,-31.58)(130.83,-31.58)(130.33,-31.58)(129.67,-31.58)(129.67,-31.58)(129.08,-31.67)(128.58,-31.83)(128,-32.08)(127.5,-32.17)(126.92,-32.25)(126.25,-32.17)(125.83,-32.33)(125.25,-32.67)(124.75,-32.83)(124.08,-33)(124,-33.5)(123.58,-33.83)(123.08,-33.83)(122.58,-33.83)(121.92,-33.83)(121.33,-33.83)(120.75,-33.83)(120.08,-33.92)(119.75,-34)(119.42,-34.42)(119,-34.42)(118.58,-34.75)(118.33,-35)(117.75,-35)(117.17,-35)(116.67,-35)(116.08,-34.83)(115.67,-34.33)(115.17,-34.17) 
};
\addplot [
color=black,
solid,
forget plot
]
coordinates{
 (130.08,-11.83)(130.33,-11.33)(130.75,-11.5)(131.33,-11.33)(131.5,-11.58)(131,-12)(130.67,-11.83)(130.08,-11.83) 
};
\addplot [
color=black,
solid,
forget plot
]
coordinates{
 (136.58,-35.92)(136.83,-35.75)(137.42,-35.67)(138.08,-35.92)(137.58,-36.17)(136.83,-36.17)(136.58,-35.92) 
};
\addplot [
color=black,
solid,
forget plot
]
coordinates{
 (144.83,-40.67)(145.5,-40.83)(146.17,-41.17)(146.75,-41.17)(147.33,-41)(148,-40.83)(148.33,-41)(148.33,-41.5)(148.33,-42)(148.17,-42.17)(148,-42.58)(147.92,-43.08)(147.5,-42.83)(147.17,-43.17)(146.83,-43.58)(146.25,-43.5)(145.75,-43.25)(145.5,-42.92)(145.33,-42.5)(145.25,-42.17)(145,-41.75)(144.83,-41.17)(144.83,-40.67) 
};
\addplot [
color=black,
solid,
forget plot
]
coordinates{
 (175.189005765312,-36.9214313780351)(175.520126812359,-37.0138259675808)(175.533582906009,-36.9496236032213)(175.444655302468,-36.7884883281087)(175.480570269194,-36.706081429688)(175.758633229727,-36.7878838467347)(175.896899326676,-37.02490195245)(176.010263119138,-37.4262759765111)(177.028533390313,-37.8988032180634)(177.258394962553,-37.9644603486482)(177.425212056562,-37.935249852552)(177.737062876409,-37.7053778066973)(178.13822463126,-37.5651908272572)(178.467019452492,-37.6605675516206)(178.497151745077,-37.8642775226856)(178.314079918127,-38.5342748262274)(177.83993457634,-38.6129681644187)(177.827487791252,-38.9277043585977)(177.77207817333,-39.0020396345478)(177.244069923124,-39.0448125301667)(177.055653310959,-39.1462282899703)(176.934493612025,-39.2824687336338)(176.945466935341,-39.4561537125148)(177.062199303558,-39.6930788408725)(177.062588539133,-39.7913091952717)(176.703383937069,-40.1345351475686)(175.922020708825,-41.1731476974194)(175.581198253924,-41.4572471810745)(175.2737982,-41.6149175033622)(174.891908850954,-41.4290097657017)(174.764908248106,-41.3134833766605)(175.121887534503,-40.5968077518065)(175.297338198038,-40.3534366161144)(175.280404122763,-40.1813872496992)(174.80733974072,-39.8067082890437)(173.95356393105,-39.463077451295)(173.773153553097,-39.261368958778)(173.806198717872,-39.196833448992)(174.56561918852,-38.7129399625059)(174.649455525552,-38.449120117956)(174.609860198227,-38.1027760135149)(174.643901260388,-37.9430247718638)(174.811166376129,-37.7147544951605)(174.822275692258,-37.5232014721948)(174.611755278214,-37.1316936376544)(174.652394606107,-36.8796091872976)(174.600921532298,-36.7860579476267)(174.236103134326,-36.4499589263557)(174.287715187337,-36.4031112106002)(174.110270622357,-36.1374391943823)(173.838672344628,-35.9730163945766)(174.106785628447,-36.3185421108039)(174.128098805651,-36.3814171621978)(174.052673284457,-36.3777906104963)(173.23620611583,-35.3438456032304)(173.236754936069,-35.2917409583199)(173.117433772103,-35.1716989677222)(173.120762462063,-35.0382533525486)(172.980902412508,-34.7601104856269)(172.63595963751,-34.4376213707464)(172.709751594712,-34.4180645954477)(172.802409438781,-34.4633999932764)(173.024274852514,-34.4069334273684)(173.061915743281,-34.5352663973375)(172.994412982901,-34.6178608783104)(173.171146106754,-34.8603809796037)(173.352121058372,-34.9940870970795)(173.359543319682,-34.927324558089)(173.477332535099,-34.9405530224395)(174.026354682989,-35.1597334094204)(174.079259302489,-35.3418503720428)(174.163151934302,-35.3551443286381)(174.229211162641,-35.2723535279908)(174.313042363657,-35.285645520301)(174.302922354492,-35.35961356821)(174.405260995742,-35.5221075807479)(174.622422166657,-35.7385031205974)(174.545836283744,-35.7506031820566)(174.513698784614,-35.8246259400191)(174.55338610918,-35.8986890861489)(174.374120848829,-35.8648831066903)(174.789766761627,-36.2330559446755)(174.80662925115,-36.3753319067354)(174.699150994225,-36.3667340419923)(174.917332370418,-36.5628945925701)(174.6060155159,-36.553854625301)(174.595981610608,-36.5920373045209)(174.848277668433,-36.6714432777169)(174.795029032346,-36.7241100481202)(175.199667461032,-36.8036580046453)(175.123499799617,-36.8563974332845)(175.189005765312,-36.9214313780351) 
};
\addplot [
color=black,
solid,
forget plot
]
coordinates{
 (166.5,-45.92)(166.83,-45.42)(167.08,-45.08)(167.58,-44.75)(167.92,-44.42)(168.33,-44.08)(168.83,-44)(169.5,-43.67)(170,-43.33)(170.5,-43)(171.08,-42.67)(171.33,-42.25)(171.5,-41.83)(172,-41.67)(172.17,-41.33)(172.17,-40.92)(172.58,-40.67)(172.58,-40.67)(173.08,-40.92)(173.17,-41.33)(173.5,-41.17)(173.83,-40.92)(173.92,-41.25)(174.33,-41.33)(174.33,-41.83)(174,-42.17)(173.58,-42.58)(173.25,-43)(172.83,-43.17)(172.75,-43.5)(173.08,-43.75)(172.42,-43.75)(172,-44.08)(171.42,-44.33)(171.25,-44.67)(171.08,-45.08)(170.92,-45.5)(170.58,-45.92)(170.17,-46.25)(169.67,-46.5)(169,-46.67)(168.33,-46.5)(167.83,-46.42)(167.58,-46.17)(167.25,-46.25)(166.75,-46.17)(166.5,-45.92) 
};
\addplot [
color=black,
solid,
forget plot
]
coordinates{
 (167.58,-47.33)(168,-46.67)(168.25,-47)(167.58,-47.33) 
};
\addplot [
color=black,
solid,
forget plot
]
coordinates{
 (95.33,5.67)(95.92,5.67)(96.33,5.33)(96.92,5.33)(97.58,5.25)(98,4.92)(98.25,4.5)(98.58,4.08)(98.58,4.08)(99,3.75)(99.5,3.5)(100,3.17)(100.33,2.67)(100.75,2.17)(101.25,2.25)(101.75,1.75)(102.25,1.67)(102.67,1.17)(103.08,1.08)(103,0.58)(103.42,0.58)(103.83,0.17)(103.75,-0.25)(103.5,-0.67)(103.83,-1)(104.42,-1)(104.58,-1.67)(104.92,-1.92)(105.08,-2.33)(105.58,-2.33)(105.92,-2.67)(106.17,-3)(105.92,-3.5)(105.92,-4.17)(105.92,-4.67)(105.83,-5.25)(105.75,-5.75)(105.33,-5.5)(104.92,-5.5)(104.92,-5.5)(104.75,-5.75)(104.42,-5.58)(104,-5.08)(103.5,-4.75)(102.83,-4.25)(102.42,-3.83)(102.25,-3.5)(101.75,-3.17)(101.33,-2.58)(100.92,-2.08)(100.83,-1.67)(100.5,-1.08)(100.42,-0.67)(100,-0.25)(99.83,0.08)(99.25,0.33)(99.08,1)(98.83,1.5)(98.83,1.83)(98.33,2.17)(97.83,2.42)(97.67,2.92)(97.25,3.33)(96.92,3.75)(96.58,3.83)(96.08,4.33)(95.67,4.75)(95.42,5.17)(95.33,5.67) 
};
\addplot [
color=black,
solid,
forget plot
]
coordinates{
 (105.5,-6.67)(105.83,-6.33)(106.08,-5.83)(106.67,-6)(107.08,-5.83)(107.42,-5.92)(107.92,-6.17)(108.42,-6.25)(108.67,-6.67)(109.08,-6.75)(109.58,-6.75)(110.08,-6.83)(110.5,-6.83)(110.83,-6.33)(110.83,-6.33)(111.33,-6.58)(111.92,-6.75)(112.42,-6.83)(112.75,-7.08)(112.83,-7.5)(113.42,-7.67)(113.92,-7.58)(114.42,-7.67)(114.42,-8.17)(114.42,-8.58)(114,-8.5)(113.58,-8.33)(113.08,-8.17)(112.5,-8.33)(112,-8.17)(111.5,-8.17)(111,-8.08)(110.5,-8)(110,-7.75)(109.58,-7.67)(109,-7.67)(108.58,-7.75)(108,-7.67)(107.58,-7.42)(107,-7.33)(106.5,-7.25)(106.58,-7)(106,-6.75)(105.5,-6.67) 
};
\addplot [
color=black,
solid,
forget plot
]
coordinates{
 (115.25,-8.74)(114.7,-8.39)(114.66,-8.25)(115.25,-8.17)(115.66,-8.41)(115.25,-8.74) 
};
\addplot [
color=black,
solid,
forget plot
]
coordinates{
 (118.028059442558,-8.41104686737567)(117.800233775704,-8.31513720330521)(117.771133889163,-8.23850275663592)(117.78201565897,-8.14112393707432)(117.891597216745,-8.04693716012936)(118.028678971886,-8.02632336439236)(118.209863132734,-8.2045125956636)(118.341150003211,-8.13412136544481)(118.490577916196,-8.17755131869922)(118.538161387257,-8.24549682209156)(118.538011755498,-8.45229962510046)(118.683989142027,-8.24047674160013)(118.822552454983,-8.2229397160943)(118.924438381285,-8.32303393577706)(118.938166827376,-8.51660162952348)(119.01687533373,-8.49766161526527)(119.045063562893,-8.54484714508597)(119.038311667375,-8.72669262138055)(118.897912175047,-8.63484539966358)(118.762062486925,-8.61235565715134)(118.701251114399,-8.64093883774496)(118.734552731523,-8.71092798000638)(118.364386654044,-8.79437514604275)(118.301320534076,-8.75478928660519)(118.340398476142,-8.63075970987202)(118.302890396088,-8.60522193470337)(118.073807185621,-8.85261847265665)(117.944273460881,-8.83527108123706)(117.401645482742,-9.04628387252848)(117.245169969743,-9.02781109505081)(117.111221280244,-9.08827341335281)(116.939654348515,-9.06163672559732)(116.804445183712,-8.97738879499421)(116.773523336899,-8.85018716540458)(116.83315276587,-8.7288715260848)(116.795232314044,-8.61455696645733)(117.077801681411,-8.42958305171861)(117.618649472466,-8.49826401865994)(117.660887096159,-8.62511406456012)(117.917841225883,-8.69280312510885)(117.997664320666,-8.57196026340501)(118.124347818222,-8.56269505419067)(118.144375898145,-8.47663467713424)(118.028059442558,-8.41104686737567) 
};
\addplot [
color=black,
solid,
forget plot
]
coordinates{
 (123.121894797409,-8.217147185206)(122.8140789876,-8.37717688086487)(122.828746858869,-8.48047251303811)(122.153250689542,-8.78891983332545)(122.010507328041,-8.77217583443668)(121.81964078335,-8.91879755691578)(121.426605479542,-8.8424742045866)(121.280082027804,-8.96829670675575)(121.140253574465,-8.98540185961284)(120.610408669538,-8.77019179860951)(119.975495478487,-8.75374619466182)(119.877209938693,-8.4983110354637)(119.886321644487,-8.38806259951003)(120.051539281222,-8.29202678960401)(120.365265686815,-8.21220852858177)(120.668172517252,-8.26389127802245)(121.245945768375,-8.51062940649318)(121.429475067864,-8.64800035662666)(121.572101915315,-8.65237824075205)(121.709233800311,-8.518838368821)(122.085632890189,-8.4761458554594)(122.260422672658,-8.55708755566288)(122.45721336682,-8.50313307502545)(122.4264745333,-8.40506946331763)(122.843243159426,-8.1727453755201)(122.896024060187,-8.09667607028572)(122.856567018172,-8.04962935658149)(122.756753349161,-8.07446822895798)(122.822579993711,-7.97545059520764)(123.026743465421,-7.99363350175412)(123.119484876657,-8.08061928585358)(123.121894797409,-8.217147185206) 
};
\addplot [
color=black,
solid,
forget plot
]
coordinates{
 (123.556630976114,-8.48030162933665)(123.375269496328,-8.3706895723723)(123.52562619993,-8.25228615033961)(123.864887322583,-8.24990993107508)(123.886639121697,-8.35134764177375)(123.556630976114,-8.48030162933665) 
};
\addplot [
color=black,
solid,
forget plot
]
coordinates{
 (116.41458176826,-8.88506847700632)(115.958998793949,-8.8290843858457)(116.125779049575,-8.68125088548129)(116.13262296429,-8.48053453359116)(116.236062836227,-8.32834935339669)(116.441914388614,-8.28269026263688)(116.655553453564,-8.37906860233304)(116.41458176826,-8.88506847700632) 
};
\addplot [
color=black,
solid,
forget plot
]
coordinates{
 (124.33,-8.42)(124.5,-8.08)(124.667446588662,-8.1226030733851)(124.834928692729,-8.16513768314841)(125.002446450707,-8.20760345126261)(125.17,-8.25)(124.96006840228,-8.29266530041313)(124.750091288369,-8.33522079734325)(124.540068529687,-8.37766589534082)(124.33,-8.42) 
};
\addplot [
color=black,
solid,
forget plot
]
coordinates{
 (125.83,-7.92)(126,-7.58)(126.67,-7.5)(126.42,-7.92)(125.83,-7.92) 
};
\addplot [
color=black,
solid,
forget plot
]
coordinates{
 (119,-9.42)(119.5,-9.33)(120,-9.33)(120.5,-9.58)(120.92,-10)(120.42,-10.25)(120,-9.92)(119.67,-9.75)(119.17,-9.75)(119,-9.42) 
};
\addplot [
color=black,
solid,
forget plot
]
coordinates{
 (123.5,-10.33)(123.67,-9.67)(123.92,-9.33)(124.33,-9.17)(124.83,-9)(125.08,-8.67)(125.67,-8.5)(126.25,-8.42)(126.75,-8.33)(127.25,-8.33)(126.92,-8.67)(126.42,-8.92)(125.92,-9.08)(125.42,-9.25)(124.92,-9.58)(124.5,-10.08)(124,-10.25)(123.5,-10.33) 
};
\addplot [
color=black,
solid,
forget plot
]
coordinates{
 (131.17,-7.92)(131.25,-7.42)(131.58,-7.08)(131.58,-7.58)(131.17,-7.92) 
};
\addplot [
color=black,
solid,
forget plot
]
coordinates{
 (134.08,-6.92)(134.08,-6.33)(134.25,-5.83)(134.5,-5.42)(134.67,-5.92)(134.5,-6.5)(134.08,-6.92) 
};
\addplot [
color=black,
solid,
forget plot
]
coordinates{
 (105.25,-1.92)(105.5,-1.58)(106,-1.5)(106.25,-2)(106.42,-2.42)(106.75,-2.5)(106.67,-3)(106,-2.62)(106,-2.25)(105.75,-2)(105.25,-1.92) 
};
\addplot [
color=black,
solid,
forget plot
]
coordinates{
 (107.67,-3.08)(107.75,-2.5)(108.33,-2.67)(108.08,-3.17)(107.67,-3.08) 
};
\addplot [
color=black,
solid,
forget plot
]
coordinates{
 (113.058819945317,-7.17947357427633)(112.930397733542,-7.13003635652306)(112.839982106007,-6.98003584545403)(113.003180387849,-6.90380885719669)(113.142026881074,-6.92181325660225)(113.795387206126,-6.85526626130036)(113.95088824877,-6.89199111624995)(113.639357619324,-7.16353292357206)(113.058819945317,-7.17947357427633) 
};
\addplot [
color=black,
solid,
forget plot
]
coordinates{
 (108.92,1)(109.08,1.58)(109.5,2)(110,1.75)(110.58,1.67)(111.08,1.58)(111.33,2.25)(111.5,2.83)(112,2.92)(112.58,3)(113.08,3.33)(113.33,3.75)(113.75,4.08)(113.75,4.08)(114.25,4.67)(114.75,4.92)(115.42,4.92)(115.42,5.33)(116,5.83)(116.17,6.25)(116.67,6.58)(117,7)(117.25,6.75)(117.75,6.42)(117.75,5.92)(118.25,5.75)(118.75,5.42)(119.25,5.17)(118.75,5)(118.25,4.83)(118.58,4.42)(117.92,4.25)(117.83,3.67)(117.5,3.25)(117.75,2.75)(118.08,2.33)(117.83,2)(118.25,1.58)(118.67,1.25)(119,0.92)(118.42,0.83)(117.92,0.83)(117.67,0.5)(117.5,0)(117.5,0)(117.42,-0.58)(117.17,-1)(116.75,-1.33)(116.33,-1.75)(116.67,-2.25)(116.33,-2.92)(116.17,-3.42)(115.67,-3.67)(115.17,-3.92)(114.75,-4.08)(114.67,-3.67)(114.33,-3.33)(113.83,-3.42)(113.25,-3.08)(112.83,-3.33)(112.33,-3.33)(111.83,-3.5)(111.75,-2.75)(111.42,-2.92)(110.92,-3)(110.33,-2.92)(110.25,-2.33)(110.08,-1.75)(110.08,-1.25)(109.75,-0.75)(109.25,-0.42)(109.25,0.08)(109,0.42)(108.92,1) 
};
\addplot [
color=black,
solid,
forget plot
]
coordinates{
 (118.83,-2.83)(119.17,-2.42)(119.33,-1.92)(119.33,-1.33)(119.5,-0.92)(119.83,-0.58)(119.83,-0.08)(119.92,0.42)(120,0.75)(120.67,0.75)(120.92,1.33)(121.42,1.17)(122,1.08)(122.5,1)(122.92,0.83)(123.33,0.92)(123.92,0.92)(124.42,1.17)(124.75,1.5)(125,1.75)(125,1.75)(125.25,1.5)(125,1.08)(124.58,0.58)(124,0.42)(123.5,0.33)(123,0.5)(122.5,0.5)(122,0.5)(121.42,0.5)(121,0.42)(120.67,0.5)(120.33,0.42)(120.08,-0.08)(120.08,-0.67)(120.58,-0.92)(120.67,-1.33)(121.17,-1.33)(121.58,-0.92)(122.08,-0.92)(122.58,-0.75)(123.17,-0.58)(123.5,-0.92)(122.92,-0.92)(122.5,-1.33)(122.08,-1.67)(121.5,-1.92)(121.92,-2.42)(122.17,-2.75)(122.42,-3.17)(122.08,-3.5)(122.08,-3.5)(122.5,-3.83)(122.83,-4.33)(123.17,-4.75)(123.17,-5.33)(122.67,-5.67)(122.33,-5.33)(122.33,-4.58)(122,-4.83)(121.58,-4.67)(121.67,-4.08)(121.17,-3.83)(120.92,-3.42)(121.08,-3)(121.08,-2.67)(120.67,-2.67)(120.25,-3)(120.42,-3.5)(120.42,-4)(120.42,-4.5)(120.33,-5)(120.42,-5.5)(119.92,-5.58)(119.5,-5.58)(119.42,-5.17)(119.58,-4.58)(119.67,-4)(119.5,-3.5)(119,-3.5)(118.83,-2.83) 
};
\addplot [
color=black,
solid,
forget plot
]
coordinates{
 (127.33,1.08)(127.5,1.5)(127.58,1.92)(128,2.17)(127.83,1.75)(128.08,1.33)(128.67,1.58)(128.67,1.08)(128.25,0.83)(128.5,0.5)(128,0.5)(127.92,0)(128,-0.42)(128.42,-0.92)(127.92,-0.58)(127.58,-0.17)(127.67,0.25)(127.58,0.67)(127.51750492286,0.772501500511419)(127.455006830665,0.875002082026049)(127.392505323155,0.977501622534029)(127.33,1.08) 
};
\addplot [
color=black,
solid,
forget plot
]
coordinates{
 (126,-3.25)(126.5,-3.08)(127,-3.17)(127.17,-3.58)(126.83,-3.83)(126.33,-3.67)(126,-3.25) 
};
\addplot [
color=black,
solid,
forget plot
]
coordinates{
 (127.92,-3.17)(128.25,-2.83)(128.92,-2.83)(129.5,-2.75)(129.92,-3)(130.42,-3)(130.75,-3.42)(130.75,-3.83)(130.25,-3.58)(129.42,-3.42)(128.92,-3.33)(128.5,-3.42)(127.92,-3.17) 
};
\addplot [
color=black,
solid,
forget plot
]
coordinates{
 (124.615533401065,-1.97171215397888)(124.508466684028,-1.93596686941992)(124.54876487104,-1.73058441962512)(124.701538962367,-1.6838264624442)(125.190895688033,-1.78353929485945)(125.371678786335,-1.89914675908637)(124.615533401065,-1.97171215397888) 
};
\addplot [
color=black,
solid,
forget plot
]
coordinates{
 (127.33,-1.67)(127.58,-1.33)(128.17,-1.67)(127.960000002396,-1.67003363289756)(127.75,-1.67004484391371)(127.539999997604,-1.67003363289756)(127.33,-1.67) 
};
\addplot [
color=black,
solid,
forget plot
]
coordinates{
 (130.33,-0.17)(130.75,0)(131.25,-0.25)(130.75,-0.42)(130.33,-0.17) 
};
\addplot [
color=black,
solid,
forget plot
]
coordinates{
 (135.33,-0.67)(135.92,-0.67)(136.25,-1.08)(135.83,-1.17)(135.33,-0.67) 
};
\addplot [
color=black,
solid,
forget plot
]
coordinates{
 (130.83,-1.33)(131.25,-0.83)(131.25,-0.83)(131.83,-0.67)(132.17,-0.33)(132.83,-0.33)(133.25,-0.67)(133.92,-0.67)(134.17,-1.33)(134,-1.83)(134.17,-2.33)(134.58,-2.5)(134.75,-3)(135.17,-3.33)(135.58,-3.33)(135.83,-2.92)(136.08,-2.58)(136.5,-2.17)(137,-2.08)(137.25,-1.67)(137.92,-1.42)(138.42,-1.67)(139,-2)(139.5,-2.17)(140,-2.33)(140.42,-2.33)(141,-2.58)(141.5,-2.75)(142,-2.92)(142.42,-3.08)(143,-3.33)(143.5,-3.42)(144,-3.67)(144,-3.67)(144.5,-3.92)(145,-4.25)(145.42,-4.5)(145.75,-4.92)(145.75,-5.5)(146.25,-5.58)(146.83,-5.92)(147.42,-5.83)(147.83,-6.17)(147.83,-6.67)(147.42,-6.75)(147,-6.83)(147.17,-7.25)(147.67,-7.75)(148.17,-8)(148.25,-8.58)(148.58,-9.08)(149.25,-9)(149.08,-9.42)(149.58,-9.58)(149.684999990233,-9.58004736610185)(149.79,-9.58006315482112)(149.895000009767,-9.58004736610185)(150,-9.58)(149.937547649611,-9.6650170150247)(149.875063729551,-9.75002275796275)(149.812547944822,-9.83501712196827)(149.75,-9.92)(149.854961268822,-9.96004928833762)(149.959948297964,-10.0000658127392)(150.06496117819,-10.0400494307755)(150.17,-10.08)(150.75,-10.17)(150.58,-10.58)(150,-10.67)(149.75,-10.33)(149.25,-10.25)(148.67,-10.17)(148.08,-10.08)(148.08,-10.08)(147.58,-10)(147.42,-9.67)(147,-9.33)(146.5,-8.92)(146.25,-8.5)(146,-8)(145.58,-7.92)(145,-7.67)(144.67,-7.5)(144.17,-7.5)(143.83,-7.92)(143.58,-8.17)(143.17,-8.33)(143.33,-8.75)(143.17,-9)(142.58,-9.25)(142,-9.08)(141.5,-9.08)(141,-9)(140.58,-8.67)(140.25,-8.33)(140,-8)(139.42,-8.08)(138.83,-8.08)(138.25,-8.33)(137.67,-8.33)(138,-7.75)(138.25,-7.33)(138.75,-7.25)(138.58,-6.75)(138.58,-6.75)(138.42,-6.17)(138.17,-5.58)(137.83,-5.25)(137.33,-5)(136.75,-4.83)(136.17,-4.58)(135.58,-4.42)(135,-4.33)(134.58,-4.08)(134.08,-3.83)(133.67,-3.5)(133.25,-4)(132.83,-4)(132.75,-3.58)(132.67,-3.25)(132.25,-2.92)(131.92,-2.83)(132.17,-2.67)(132.67,-2.75)(133.08,-2.5)(133.67,-2.5)(133.83,-2)(133.25,-2.17)(132.75,-2.25)(132.25,-2.25)(131.83,-2)(131.83,-1.58)(131.42,-1.5)(130.83,-1.33) 
};
\addplot [
color=black,
solid,
forget plot
]
coordinates{
 (119.83,16.25)(120.33,16.08)(120.33,16.58)(120.5,17.17)(120.33,17.58)(120.5,18)(120.67,18.5)(121.08,18.58)(121.5,18.33)(121.92,18.25)(122.33,18.5)(122.17,17.92)(122.17,17.42)(122.252614230244,17.3350505669733)(122.335152014794,17.2500672843816)(122.417613792146,17.1650503597482)(122.5,17.08)(122.457415602305,16.9550131450595)(122.414887811259,16.8300174731454)(122.372416114012,16.7050130647518)(122.33,16.58)(122,16.17)(121.58,15.75)(121.58,15.75)(121.42,15.25)(121.67,14.58)(121.83,14.08)(122.25,13.92)(122.58,14.17)(122.92,14.17)(123.33,14)(123.92,13.75)(123.58,13.67)(123.75,13.33)(124.17,12.92)(123.83,12.83)(123.42,13)(123.08,13.58)(122.58,13.83)(122.67,13.5)(122.67,13.08)(122.17,13.67)(121.75,13.92)(121.25,13.58)(120.67,13.75)(120.67,14.17)(121,14.5)(120.67,14.75)(120.58,14.42)(120.08,14.75)(120,15.25)(119.92,15.67)(119.83,16.25) 
};
\addplot [
color=black,
solid,
forget plot
]
coordinates{
 (117.25,8.42)(117.5,8.92)(117.92,9.25)(118.42,9.67)(118.75,10)(119.17,10.5)(119.5,11.25)(119.67,10.5)(119.17,10.08)(118.83,9.92)(118.58,9.42)(118.17,9.08)(117.92,8.75)(117.25,8.42) 
};
\addplot [
color=black,
solid,
forget plot
]
coordinates{
 (120.42,13.42)(120.83,13.42)(121.25,13.33)(121.58,13)(121.58,12.5)(121.17,12.17)(120.92,12.5)(120.75,13)(120.42,13.42) 
};
\addplot [
color=black,
solid,
forget plot
]
coordinates{
 (124.33,12.5)(124.75,12.42)(125.25,12.5)(125.5,12.08)(125.5,11.67)(125.67,11.08)(125,11.08)(125,10.67)(125.25,10.25)(124.75,10.17)(124.75,10.75)(124.42,10.83)(124.42,11.42)(124.83,11.33)(124.872452451556,11.4350092375575)(124.914936389349,11.5400123583124)(124.957452132259,11.6450092999543)(125,11.75)(124.875085873722,11.8125822133927)(124.750114689432,11.8751098329257)(124.625086160275,11.9375825360643)(124.5,12)(124.33,12.5) 
};
\addplot [
color=black,
solid,
forget plot
]
coordinates{
 (121.92,10.5)(122,10.92)(122.08,11.42)(121.92,11.83)(122.5,11.75)(122.58,11.5)(123.17,11.5)(123.08,10.92)(123.5,10.83)(123.42,10.42)(123.83,10.67)(124,11.08)(124,10.58)(124,10.17)(124.5,10.08)(124.5,9.67)(124,9.58)(123.75,9.92)(123.5,9.58)(123.17,9.75)(123.33,9.25)(123,9)(122.67,9.33)(122.607557522328,9.43501664882601)(122.545076997148,9.5400222862546)(122.482557973615,9.64501678063041)(122.42,9.75)(122.522441731899,9.81254636686287)(122.624922144635,9.87506196364376)(122.727441485088,9.93754657865356)(122.83,10)(122.83,10.42)(122.5,10.58)(121.92,10.5) 
};
\addplot [
color=black,
solid,
forget plot
]
coordinates{
 (122,7.17)(122.08,7.75)(122.33,8)(122.83,8.17)(123.08,8.5)(123.5,8.67)(123.83,8.5)(123.92,8.08)(124.25,8.42)(124.67,8.5)(124.83,9)(125.17,8.83)(125.5,9.08)(125.42,9.75)(125.42,9.75)(126,9.33)(126.33,8.83)(126.33,8.25)(126.58,7.67)(126.58,7.25)(126.33,6.83)(126.17,6.33)(126,6.92)(125.75,7.33)(125.5,7)(125.33,6.75)(125.58,6.42)(125.75,6.08)(125.42,5.58)(125.08,5.92)(124.75,5.92)(124.33,6.17)(124,6.42)(124,7)(124.25,7.33)(124,7.58)(123.58,7.75)(123.25,7.42)(122.92,7.33)(122.83,7.75)(122.5,7.67)(122.25,7.25)(122.08,6.83)(122,7.17) 
};
\addplot [
color=black,
solid,
forget plot
]
coordinates{
 (148.25,-5.5)(148.83,-5.58)(149.42,-5.58)(150,-5.5)(150.42,-5.5)(150.83,-5.5)(151.25,-4.92)(151.67,-4.92)(151.5,-4.17)(152.25,-4.17)(152.42,-4.67)(152.08,-5)(152,-5.58)(151.58,-5.58)(151.17,-6)(150.67,-6.25)(150.17,-6.25)(149.67,-6.25)(149.17,-6.17)(148.83,-5.83)(148.33,-5.75)(148.25,-5.5) 
};
\addplot [
color=black,
solid,
forget plot
]
coordinates{
 (147.19809472474,-2.21576668400953)(146.7185549723,-2.18724679446117)(146.595908484641,-2.1378862505337)(146.756982740847,-1.9867182048445)(147.092420697003,-1.96600247905389)(147.366351090669,-2.07707403790962)(147.19809472474,-2.21576668400953) 
};
\addplot [
color=black,
solid,
forget plot
]
coordinates{
 (152.468043392879,-3.73634530808361)(153.061019328995,-4.16655953262886)(153.092343683017,-4.28377638714791)(153.048863813647,-4.60660515756206)(152.954666247085,-4.74546190406671)(152.881033736061,-4.78953628748618)(152.723341270769,-4.55772838101401)(152.720908552785,-4.25506601392839)(152.468043392879,-3.73634530808361) 
};
\addplot [
color=black,
solid,
forget plot
]
coordinates{
 (152.022494078021,-3.45722171984702)(151.938009576222,-3.45087566150437)(151.303894544589,-3.04935695015838)(151.071722959216,-2.8361616428279)(151.146382736326,-2.76105478411147)(151.621798602667,-3.10603463486951)(152.066838698159,-3.35671289773501)(152.084122496299,-3.43183068784051)(152.022494078021,-3.45722171984702) 
};
\addplot [
color=black,
solid,
forget plot
]
coordinates{
 (154.67,-5.42)(155.08,-5.5)(155.58,-6.17)(156,-6.58)(155.5,-6.83)(155.17,-6.5)(154.83,-6)(154.67,-5.42) 
};
\addplot [
color=black,
solid,
forget plot
]
coordinates{
 (157.408439343801,-7.30804153601413)(157.304967472915,-7.34627172412957)(157.033128660566,-7.26191908064289)(156.487534339115,-6.79430735124937)(156.478749495251,-6.67685911501031)(156.886237649908,-6.82916558514358)(157.139389253712,-7.11623192478533)(157.408439343801,-7.30804153601413) 
};
\addplot [
color=black,
solid,
forget plot
]
coordinates{
 (157.849108564222,-8.53102643736693)(157.698151375293,-8.51260740557505)(157.551577944195,-8.33215703657426)(157.462360681115,-8.29246350817381)(157.300027804215,-8.29963058243987)(157.362143839028,-8.02900797011925)(157.437187938194,-7.96993317813849)(157.569982053477,-7.99715261863547)(157.88452843002,-8.4167301992257)(157.849108564222,-8.53102643736693) 
};
\addplot [
color=black,
solid,
forget plot
]
coordinates{
 (159.783088995371,-8.47312000521855)(158.725924476823,-7.8870868090001)(158.553723532268,-7.67923904171745)(158.605784570939,-7.5588923447561)(159.167725792182,-7.96475955962262)(159.74144109164,-8.26925313382878)(159.839164571601,-8.35813737917717)(159.783088995371,-8.47312000521855) 
};
\addplot [
color=black,
solid,
forget plot
]
coordinates{
 (161.285080612403,-9.51514385708004)(160.889902754384,-9.16309677321365)(160.833856752539,-8.92188442739894)(160.667803074198,-8.57258054981554)(160.677589483136,-8.47112565544728)(160.802617888886,-8.45287667460012)(160.903693859702,-8.53599257033957)(161.285080612403,-9.51514385708004) 
};
\addplot [
color=black,
solid,
forget plot
]
coordinates{
 (159.5,-9.25)(160.25,-9.33)(160.83,-9.67)(160.5,-9.92)(160.08,-9.83)(159.67,-9.83)(159.42,-9.5)(159.5,-9.25) 
};
\addplot [
color=black,
solid,
forget plot
]
coordinates{
 (161.25,-10.17)(162,-10.33)(162.33,-10.75)(161.75,-10.67)(161.25,-10.17) 
};
\addplot [
color=black,
solid,
forget plot
]
coordinates{
 (166.776223660329,-15.6617954311137)(166.558845436436,-14.7995872882631)(166.57800113136,-14.7171371588901)(166.713079245306,-14.825510159286)(166.844604230259,-15.1995390237467)(166.986271541137,-15.1803320949799)(167.063028082739,-14.9902972231577)(167.237741980548,-15.4711729054563)(167.200458548863,-15.5345346103726)(166.912676157075,-15.5856162315677)(166.776223660329,-15.6617954311137) 
};
\addplot [
color=black,
solid,
forget plot
]
coordinates{
 (167.33,-15.92)(167.92,-16.42)(167.5,-16.5)(167.33,-15.92) 
};
\addplot [
color=black,
solid,
forget plot
]
coordinates{
 (164.25,-20.25)(164.75,-20.5)(165.33,-20.75)(165.67,-21.17)(166,-21.5)(166.5,-21.67)(166.83,-22)(167.17,-22.33)(166.67,-22.33)(166.17,-22)(165.75,-21.75)(165.42,-21.5)(165,-21.25)(164.58,-20.83)(164.25,-20.25) 
};
\addplot [
color=black,
solid,
forget plot
]
coordinates{
 (-176.83,-43.77)(-176.17,-43.8)(-176.58,-44.17)(-176.5,-43.9)(-176.83,-43.77) 
};
\addplot [
color=black,
solid,
forget plot
]
coordinates{
 (177.33,-18)(177.42,-17.58)(177.75,-17.33)(178.25,-17.33)(178.67,-17.58)(178.67,-18)(178.17,-18.17)(177.75,-18.17)(177.33,-18) 
};
\addplot [
color=black,
solid,
forget plot
]
coordinates{
 (178.58,-16.58)(179.17,-16.42)(179.83,-16.17)(179.67,-16.5)(179.92,-16.67)(179.33,-16.67)(178.75,-16.92)(178.58,-16.58) 
};
\addplot [
color=black,
solid,
forget plot
]
coordinates{
 (-149.58,-17.47)(-149.33,-17.53)(-149.17,-17.83)(-149.5,-17.73)(-149.58,-17.47) 
};
\addplot [
color=black,
solid,
forget plot
]
coordinates{
 (-159.82,22.03)(-159.58,22.22)(-159.3,22.22)(-159.38,21.9)(-159.62,21.9)(-159.82,22.03) 
};
\addplot [
color=black,
solid,
forget plot
]
coordinates{
 (-158.28,21.58)(-157.98,21.72)(-157.67,21.3)(-158.12,21.3)(-158.28,21.58) 
};
\addplot [
color=black,
solid,
forget plot
]
coordinates{
 (-157.212685821526,21.2134048517516)(-156.716359348044,21.1607621184613)(-156.857545274522,21.0424385030412)(-157.300456378137,21.1017003345928)(-157.212685821526,21.2134048517516) 
};
\addplot [
color=black,
solid,
forget plot
]
coordinates{
 (-156.68,20.95)(-156.47,20.9)(-156.28,20.97)(-155.98,20.73)(-156.2,20.62)(-156.43,20.57)(-156.47,20.78)(-156.68,20.95) 
};
\addplot [
color=black,
solid,
forget plot
]
coordinates{
 (-155.88,20.27)(-155.63,20.15)(-155.37,20.05)(-155.13,19.92)(-155.02,19.68)(-154.8,19.52)(-155,19.33)(-155.3,19.27)(-155.52,19.13)(-155.65,18.93)(-155.92,19.08)(-155.88,19.35)(-156.05,19.77)(-155.87,19.95)(-155.88,20.27) 
};
\addplot [
color=blue,
solid,
mark=*,mark size=1.3pt,
mark options={solid,fill=mycolor1,draw=black,line width=0.15pt},
forget plot
]
coordinates{
 (7.5733,47.558399) 
};
\addplot [
color=blue,
solid,
mark=*,mark size=1.3pt,
mark options={solid,fill=mycolor2,draw=black,line width=0.15pt},
forget plot
]
coordinates{
 (-58.672501,-34.587502) 
};
\addplot [
color=blue,
solid,
mark=*,mark size=1.3pt,
mark options={solid,fill=mycolor3,draw=black,line width=0.15pt},
forget plot
]
coordinates{
 (-57.954498,-34.921501) 
};
\addplot [
color=blue,
solid,
mark=*,mark size=1.3pt,
mark options={solid,fill=mycolor2,draw=black,line width=0.15pt},
forget plot
]
coordinates{
 (-58.672501,-34.587502) 
};
\addplot [
color=blue,
solid,
mark=*,mark size=1.3pt,
mark options={solid,fill=mycolor4,draw=black,line width=0.15pt},
forget plot
]
coordinates{
 (-64.181099,-31.4135) 
};
\addplot [
color=blue,
solid,
mark=*,mark size=1.3pt,
mark options={solid,fill=mycolor5,draw=black,line width=0.15pt},
forget plot
]
coordinates{
 (145.116699,-37.883301) 
};
\addplot [
color=blue,
solid,
mark=*,mark size=1.3pt,
mark options={solid,fill=mycolor6,draw=black,line width=0.15pt},
forget plot
]
coordinates{
 (133,-27) 
};
\addplot [
color=blue,
solid,
mark=*,mark size=1.3pt,
mark options={solid,fill=mycolor7,draw=black,line width=0.15pt},
forget plot
]
coordinates{
 (138.598602,-34.928699) 
};
\addplot [
color=blue,
solid,
mark=*,mark size=1.3pt,
mark options={solid,fill=mycolor8,draw=black,line width=0.15pt},
forget plot
]
coordinates{
 (133,-27) 
};
\addplot [
color=blue,
solid,
mark=*,mark size=1.3pt,
mark options={solid,fill=mycolor9,draw=black,line width=0.15pt},
forget plot
]
coordinates{
 (133,-27) 
};
\addplot [
color=blue,
solid,
mark=*,mark size=1.3pt,
mark options={solid,fill=mycolor10,draw=black,line width=0.15pt},
forget plot
]
coordinates{
 (133,-27) 
};
\addplot [
color=blue,
solid,
mark=*,mark size=1.3pt,
mark options={solid,fill=mycolor11,draw=black,line width=0.15pt},
forget plot
]
coordinates{
 (149.133301,-35.266701) 
};
\addplot [
color=blue,
solid,
mark=*,mark size=1.3pt,
mark options={solid,fill=mycolor12,draw=black,line width=0.15pt},
forget plot
]
coordinates{
 (151.205505,-33.8615) 
};
\addplot [
color=blue,
solid,
mark=*,mark size=1.3pt,
mark options={solid,fill=mycolor13,draw=black,line width=0.15pt},
forget plot
]
coordinates{
 (116.674202,-31.6493) 
};
\addplot [
color=blue,
solid,
mark=*,mark size=1.3pt,
mark options={solid,fill=mycolor14,draw=black,line width=0.15pt},
forget plot
]
coordinates{
 (15.45,47.0667) 
};
\addplot [
color=blue,
solid,
mark=*,mark size=1.3pt,
mark options={solid,fill=mycolor14,draw=black,line width=0.15pt},
forget plot
]
coordinates{
 (14.4167,48.349998) 
};
\addplot [
color=blue,
solid,
mark=*,mark size=1.3pt,
mark options={solid,fill=mycolor15,draw=black,line width=0.15pt},
forget plot
]
coordinates{
 (16.366699,48.200001) 
};
\addplot [
color=blue,
solid,
mark=*,mark size=1.3pt,
mark options={solid,fill=mycolor16,draw=black,line width=0.15pt},
forget plot
]
coordinates{
 (85.166702,26.6833) 
};
\addplot [
color=blue,
solid,
mark=*,mark size=1.3pt,
mark options={solid,fill=mycolor17,draw=black,line width=0.15pt},
forget plot
]
coordinates{
 (90.4086,23.723101) 
};
\addplot [
color=blue,
solid,
mark=*,mark size=1.3pt,
mark options={solid,fill=mycolor18,draw=black,line width=0.15pt},
forget plot
]
coordinates{
 (27.5667,53.900002) 
};
\addplot [
color=blue,
solid,
mark=*,mark size=1.3pt,
mark options={solid,fill=mycolor18,draw=black,line width=0.15pt},
forget plot
]
coordinates{
 (27.5667,53.900002) 
};
\addplot [
color=blue,
solid,
mark=*,mark size=1.3pt,
mark options={solid,fill=mycolor19,draw=black,line width=0.15pt},
forget plot
]
coordinates{
 (4,50.833302) 
};
\addplot [
color=blue,
solid,
mark=*,mark size=1.3pt,
mark options={solid,fill=mycolor19,draw=black,line width=0.15pt},
forget plot
]
coordinates{
 (4.4167,51.216702) 
};
\addplot [
color=blue,
solid,
mark=*,mark size=1.3pt,
mark options={solid,fill=mycolor19,draw=black,line width=0.15pt},
forget plot
]
coordinates{
 (4,50.833302) 
};
\addplot [
color=blue,
solid,
mark=*,mark size=1.3pt,
mark options={solid,fill=mycolor18,draw=black,line width=0.15pt},
forget plot
]
coordinates{
 (18.383301,43.849998) 
};
\addplot [
color=blue,
solid,
mark=*,mark size=1.3pt,
mark options={solid,fill=mycolor20,draw=black,line width=0.15pt},
forget plot
]
coordinates{
 (-55,-10) 
};
\addplot [
color=blue,
solid,
mark=*,mark size=1.3pt,
mark options={solid,fill=mycolor21,draw=black,line width=0.15pt},
forget plot
]
coordinates{
 (-49.25,-25.4167) 
};
\addplot [
color=blue,
solid,
mark=*,mark size=1.3pt,
mark options={solid,fill=mycolor3,draw=black,line width=0.15pt},
forget plot
]
coordinates{
 (-55,-10) 
};
\addplot [
color=blue,
solid,
mark=*,mark size=1.3pt,
mark options={solid,fill=mycolor2,draw=black,line width=0.15pt},
forget plot
]
coordinates{
 (-38.516701,-12.9833) 
};
\addplot [
color=blue,
solid,
mark=*,mark size=1.3pt,
mark options={solid,fill=mycolor22,draw=black,line width=0.15pt},
forget plot
]
coordinates{
 (-46.665798,-23.473301) 
};
\addplot [
color=blue,
solid,
mark=*,mark size=1.3pt,
mark options={solid,fill=mycolor21,draw=black,line width=0.15pt},
forget plot
]
coordinates{
 (-46.665798,-23.473301) 
};
\addplot [
color=blue,
solid,
mark=*,mark size=1.3pt,
mark options={solid,fill=mycolor23,draw=black,line width=0.15pt},
forget plot
]
coordinates{
 (-51.916698,-23.4167) 
};
\addplot [
color=blue,
solid,
mark=*,mark size=1.3pt,
mark options={solid,fill=mycolor21,draw=black,line width=0.15pt},
forget plot
]
coordinates{
 (-47.083302,-22.9) 
};
\addplot [
color=blue,
solid,
mark=*,mark size=1.3pt,
mark options={solid,fill=mycolor24,draw=black,line width=0.15pt},
forget plot
]
coordinates{
 (25,43) 
};
\addplot [
color=blue,
solid,
mark=*,mark size=1.3pt,
mark options={solid,fill=mycolor25,draw=black,line width=0.15pt},
forget plot
]
coordinates{
 (23.3167,42.6833) 
};
\addplot [
color=blue,
solid,
mark=*,mark size=1.3pt,
mark options={solid,fill=mycolor26,draw=black,line width=0.15pt},
forget plot
]
coordinates{
 (23.3167,42.6833) 
};
\addplot [
color=blue,
solid,
mark=*,mark size=1.3pt,
mark options={solid,fill=mycolor27,draw=black,line width=0.15pt},
forget plot
]
coordinates{
 (-77.949997,44.299999) 
};
\addplot [
color=blue,
solid,
mark=*,mark size=1.3pt,
mark options={solid,fill=mycolor28,draw=black,line width=0.15pt},
forget plot
]
coordinates{
 (-80.533302,43.466702) 
};
\addplot [
color=blue,
solid,
mark=*,mark size=1.3pt,
mark options={solid,fill=mycolor29,draw=black,line width=0.15pt},
forget plot
]
coordinates{
 (-74.020302,40.688801) 
};
\addplot [
color=blue,
solid,
mark=*,mark size=1.3pt,
mark options={solid,fill=mycolor27,draw=black,line width=0.15pt},
forget plot
]
coordinates{
 (-73.583298,45.5) 
};
\addplot [
color=blue,
solid,
mark=*,mark size=1.3pt,
mark options={solid,fill=mycolor30,draw=black,line width=0.15pt},
forget plot
]
coordinates{
 (-63.599998,44.650002) 
};
\addplot [
color=blue,
solid,
mark=*,mark size=1.3pt,
mark options={solid,fill=mycolor27,draw=black,line width=0.15pt},
forget plot
]
coordinates{
 (-73.733299,45.599998) 
};
\addplot [
color=blue,
solid,
mark=*,mark size=1.3pt,
mark options={solid,fill=mycolor31,draw=black,line width=0.15pt},
forget plot
]
coordinates{
 (-79.833298,43.25) 
};
\addplot [
color=blue,
solid,
mark=*,mark size=1.3pt,
mark options={solid,fill=mycolor32,draw=black,line width=0.15pt},
forget plot
]
coordinates{
 (-70.666702,-33.450001) 
};
\addplot [
color=blue,
solid,
mark=*,mark size=1.3pt,
mark options={solid,fill=mycolor33,draw=black,line width=0.15pt},
forget plot
]
coordinates{
 (103.792198,36.0564) 
};
\addplot [
color=blue,
solid,
mark=*,mark size=1.3pt,
mark options={solid,fill=mycolor34,draw=black,line width=0.15pt},
forget plot
]
coordinates{
 (123.4328,41.792198) 
};
\addplot [
color=blue,
solid,
mark=*,mark size=1.3pt,
mark options={solid,fill=mycolor33,draw=black,line width=0.15pt},
forget plot
]
coordinates{
 (116.9972,36.668301) 
};
\addplot [
color=blue,
solid,
mark=*,mark size=1.3pt,
mark options={solid,fill=mycolor35,draw=black,line width=0.15pt},
forget plot
]
coordinates{
 (116.388298,39.928902) 
};
\addplot [
color=blue,
solid,
mark=*,mark size=1.3pt,
mark options={solid,fill=mycolor36,draw=black,line width=0.15pt},
forget plot
]
coordinates{
 (117.2808,31.863899) 
};
\addplot [
color=blue,
solid,
mark=*,mark size=1.3pt,
mark options={solid,fill=mycolor37,draw=black,line width=0.15pt},
forget plot
]
coordinates{
 (116.388298,39.928902) 
};
\addplot [
color=blue,
solid,
mark=*,mark size=1.3pt,
mark options={solid,fill=mycolor38,draw=black,line width=0.15pt},
forget plot
]
coordinates{
 (120.1614,30.2936) 
};
\addplot [
color=blue,
solid,
mark=*,mark size=1.3pt,
mark options={solid,fill=mycolor39,draw=black,line width=0.15pt},
forget plot
]
coordinates{
 (123.4328,41.792198) 
};
\addplot [
color=blue,
solid,
mark=*,mark size=1.3pt,
mark options={solid,fill=mycolor40,draw=black,line width=0.15pt},
forget plot
]
coordinates{
 (104.066704,30.6667) 
};
\addplot [
color=blue,
solid,
mark=*,mark size=1.3pt,
mark options={solid,fill=mycolor40,draw=black,line width=0.15pt},
forget plot
]
coordinates{
 (104.066704,30.6667) 
};
\addplot [
color=blue,
solid,
mark=*,mark size=1.3pt,
mark options={solid,fill=mycolor41,draw=black,line width=0.15pt},
forget plot
]
coordinates{
 (116.388298,39.928902) 
};
\addplot [
color=blue,
solid,
mark=*,mark size=1.3pt,
mark options={solid,fill=mycolor42,draw=black,line width=0.15pt},
forget plot
]
coordinates{
 (-75.563599,6.2518) 
};
\addplot [
color=blue,
solid,
mark=*,mark size=1.3pt,
mark options={solid,fill=mycolor19,draw=black,line width=0.15pt},
forget plot
]
coordinates{
 (-2,54) 
};
\addplot [
color=blue,
solid,
mark=*,mark size=1.3pt,
mark options={solid,fill=mycolor25,draw=black,line width=0.15pt},
forget plot
]
coordinates{
 (15.5,49.75) 
};
\addplot [
color=blue,
solid,
mark=*,mark size=1.3pt,
mark options={solid,fill=mycolor43,draw=black,line width=0.15pt},
forget plot
]
coordinates{
 (14.4667,50.083302) 
};
\addplot [
color=blue,
solid,
mark=*,mark size=1.3pt,
mark options={solid,fill=mycolor43,draw=black,line width=0.15pt},
forget plot
]
coordinates{
 (14.4667,50.083302) 
};
\addplot [
color=blue,
solid,
mark=*,mark size=1.3pt,
mark options={solid,fill=mycolor15,draw=black,line width=0.15pt},
forget plot
]
coordinates{
 (14.4667,50.083302) 
};
\addplot [
color=blue,
solid,
mark=*,mark size=1.3pt,
mark options={solid,fill=mycolor25,draw=black,line width=0.15pt},
forget plot
]
coordinates{
 (12.5221,55.678501) 
};
\addplot [
color=blue,
solid,
mark=*,mark size=1.3pt,
mark options={solid,fill=mycolor44,draw=black,line width=0.15pt},
forget plot
]
coordinates{
 (10,56) 
};
\addplot [
color=blue,
solid,
mark=*,mark size=1.3pt,
mark options={solid,fill=mycolor26,draw=black,line width=0.15pt},
forget plot
]
coordinates{
 (12.5833,55.666698) 
};
\addplot [
color=blue,
solid,
mark=*,mark size=1.3pt,
mark options={solid,fill=mycolor43,draw=black,line width=0.15pt},
forget plot
]
coordinates{
 (10,56) 
};
\addplot [
color=blue,
solid,
mark=*,mark size=1.3pt,
mark options={solid,fill=mycolor14,draw=black,line width=0.15pt},
forget plot
]
coordinates{
 (8.4807,55.621601) 
};
\addplot [
color=blue,
solid,
mark=*,mark size=1.3pt,
mark options={solid,fill=mycolor18,draw=black,line width=0.15pt},
forget plot
]
coordinates{
 (24.7281,59.433899) 
};
\addplot [
color=blue,
solid,
mark=*,mark size=1.3pt,
mark options={solid,fill=mycolor45,draw=black,line width=0.15pt},
forget plot
]
coordinates{
 (24.6667,60.216702) 
};
\addplot [
color=blue,
solid,
mark=*,mark size=1.3pt,
mark options={solid,fill=mycolor46,draw=black,line width=0.15pt},
forget plot
]
coordinates{
 (24.6667,60.216702) 
};
\addplot [
color=blue,
solid,
mark=*,mark size=1.3pt,
mark options={solid,fill=mycolor46,draw=black,line width=0.15pt},
forget plot
]
coordinates{
 (23.75,61.5) 
};
\addplot [
color=blue,
solid,
mark=*,mark size=1.3pt,
mark options={solid,fill=mycolor15,draw=black,line width=0.15pt},
forget plot
]
coordinates{
 (2,46) 
};
\addplot [
color=blue,
solid,
mark=*,mark size=1.3pt,
mark options={solid,fill=mycolor15,draw=black,line width=0.15pt},
forget plot
]
coordinates{
 (2.3333,48.866699) 
};
\addplot [
color=blue,
solid,
mark=*,mark size=1.3pt,
mark options={solid,fill=mycolor15,draw=black,line width=0.15pt},
forget plot
]
coordinates{
 (2,46) 
};
\addplot [
color=blue,
solid,
mark=*,mark size=1.3pt,
mark options={solid,fill=mycolor43,draw=black,line width=0.15pt},
forget plot
]
coordinates{
 (2,46) 
};
\addplot [
color=blue,
solid,
mark=*,mark size=1.3pt,
mark options={solid,fill=mycolor14,draw=black,line width=0.15pt},
forget plot
]
coordinates{
 (2.3333,48.866699) 
};
\addplot [
color=blue,
solid,
mark=*,mark size=1.3pt,
mark options={solid,fill=mycolor43,draw=black,line width=0.15pt},
forget plot
]
coordinates{
 (2.2473,48.835701) 
};
\addplot [
color=blue,
solid,
mark=*,mark size=1.3pt,
mark options={solid,fill=mycolor14,draw=black,line width=0.15pt},
forget plot
]
coordinates{
 (2.3333,48.866699) 
};
\addplot [
color=blue,
solid,
mark=*,mark size=1.3pt,
mark options={solid,fill=mycolor43,draw=black,line width=0.15pt},
forget plot
]
coordinates{
 (2.3333,48.866699) 
};
\addplot [
color=blue,
solid,
mark=*,mark size=1.3pt,
mark options={solid,fill=mycolor25,draw=black,line width=0.15pt},
forget plot
]
coordinates{
 (2,46) 
};
\addplot [
color=blue,
solid,
mark=*,mark size=1.3pt,
mark options={solid,fill=mycolor25,draw=black,line width=0.15pt},
forget plot
]
coordinates{
 (2.3,49.900002) 
};
\addplot [
color=blue,
solid,
mark=*,mark size=1.3pt,
mark options={solid,fill=mycolor47,draw=black,line width=0.15pt},
forget plot
]
coordinates{
 (-149.566696,-17.5333) 
};
\addplot [
color=blue,
solid,
mark=*,mark size=1.3pt,
mark options={solid,fill=mycolor48,draw=black,line width=0.15pt},
forget plot
]
coordinates{
 (44.790798,41.724998) 
};
\addplot [
color=blue,
solid,
mark=*,mark size=1.3pt,
mark options={solid,fill=mycolor19,draw=black,line width=0.15pt},
forget plot
]
coordinates{
 (6.1053,50.770802) 
};
\addplot [
color=blue,
solid,
mark=*,mark size=1.3pt,
mark options={solid,fill=mycolor14,draw=black,line width=0.15pt},
forget plot
]
coordinates{
 (13.75,51.049999) 
};
\addplot [
color=blue,
solid,
mark=*,mark size=1.3pt,
mark options={solid,fill=mycolor43,draw=black,line width=0.15pt},
forget plot
]
coordinates{
 (9.3167,48.6833) 
};
\addplot [
color=blue,
solid,
mark=*,mark size=1.3pt,
mark options={solid,fill=mycolor15,draw=black,line width=0.15pt},
forget plot
]
coordinates{
 (7.75,49.450001) 
};
\addplot [
color=blue,
solid,
mark=*,mark size=1.3pt,
mark options={solid,fill=mycolor49,draw=black,line width=0.15pt},
forget plot
]
coordinates{
 (9,51) 
};
\addplot [
color=blue,
solid,
mark=*,mark size=1.3pt,
mark options={solid,fill=mycolor49,draw=black,line width=0.15pt},
forget plot
]
coordinates{
 (8.4647,49.4883) 
};
\addplot [
color=blue,
solid,
mark=*,mark size=1.3pt,
mark options={solid,fill=mycolor43,draw=black,line width=0.15pt},
forget plot
]
coordinates{
 (6.95,50.9333) 
};
\addplot [
color=blue,
solid,
mark=*,mark size=1.3pt,
mark options={solid,fill=mycolor15,draw=black,line width=0.15pt},
forget plot
]
coordinates{
 (9,51) 
};
\addplot [
color=blue,
solid,
mark=*,mark size=1.3pt,
mark options={solid,fill=mycolor15,draw=black,line width=0.15pt},
forget plot
]
coordinates{
 (9.2,52.3167) 
};
\addplot [
color=blue,
solid,
mark=*,mark size=1.3pt,
mark options={solid,fill=mycolor26,draw=black,line width=0.15pt},
forget plot
]
coordinates{
 (13.4,52.516701) 
};
\addplot [
color=blue,
solid,
mark=*,mark size=1.3pt,
mark options={solid,fill=mycolor25,draw=black,line width=0.15pt},
forget plot
]
coordinates{
 (11.0039,49.589699) 
};
\addplot [
color=blue,
solid,
mark=*,mark size=1.3pt,
mark options={solid,fill=mycolor43,draw=black,line width=0.15pt},
forget plot
]
coordinates{
 (10.5333,52.266701) 
};
\addplot [
color=blue,
solid,
mark=*,mark size=1.3pt,
mark options={solid,fill=mycolor19,draw=black,line width=0.15pt},
forget plot
]
coordinates{
 (7.1,50.733299) 
};
\addplot [
color=blue,
solid,
mark=*,mark size=1.3pt,
mark options={solid,fill=mycolor25,draw=black,line width=0.15pt},
forget plot
]
coordinates{
 (10.9,50.6833) 
};
\addplot [
color=blue,
solid,
mark=*,mark size=1.3pt,
mark options={solid,fill=mycolor26,draw=black,line width=0.15pt},
forget plot
]
coordinates{
 (11.5783,49.948101) 
};
\addplot [
color=blue,
solid,
mark=*,mark size=1.3pt,
mark options={solid,fill=mycolor25,draw=black,line width=0.15pt},
forget plot
]
coordinates{
 (11.0039,49.589699) 
};
\addplot [
color=blue,
solid,
mark=*,mark size=1.3pt,
mark options={solid,fill=mycolor15,draw=black,line width=0.15pt},
forget plot
]
coordinates{
 (8.2711,50) 
};
\addplot [
color=blue,
solid,
mark=*,mark size=1.3pt,
mark options={solid,fill=mycolor15,draw=black,line width=0.15pt},
forget plot
]
coordinates{
 (7.6333,51.966702) 
};
\addplot [
color=blue,
solid,
mark=*,mark size=1.3pt,
mark options={solid,fill=mycolor43,draw=black,line width=0.15pt},
forget plot
]
coordinates{
 (9.9333,51.533298) 
};
\addplot [
color=blue,
solid,
mark=*,mark size=1.3pt,
mark options={solid,fill=mycolor19,draw=black,line width=0.15pt},
forget plot
]
coordinates{
 (6.1053,50.770802) 
};
\addplot [
color=blue,
solid,
mark=*,mark size=1.3pt,
mark options={solid,fill=mycolor43,draw=black,line width=0.15pt},
forget plot
]
coordinates{
 (6.1053,50.770802) 
};
\addplot [
color=blue,
solid,
mark=*,mark size=1.3pt,
mark options={solid,fill=mycolor14,draw=black,line width=0.15pt},
forget plot
]
coordinates{
 (12.3333,51.299999) 
};
\addplot [
color=blue,
solid,
mark=*,mark size=1.3pt,
mark options={solid,fill=mycolor14,draw=black,line width=0.15pt},
forget plot
]
coordinates{
 (9,51) 
};
\addplot [
color=blue,
solid,
mark=*,mark size=1.3pt,
mark options={solid,fill=mycolor19,draw=black,line width=0.15pt},
forget plot
]
coordinates{
 (6.1833,51.650002) 
};
\addplot [
color=blue,
solid,
mark=*,mark size=1.3pt,
mark options={solid,fill=mycolor15,draw=black,line width=0.15pt},
forget plot
]
coordinates{
 (9.5,51.3167) 
};
\addplot [
color=blue,
solid,
mark=*,mark size=1.3pt,
mark options={solid,fill=mycolor14,draw=black,line width=0.15pt},
forget plot
]
coordinates{
 (13.75,51.049999) 
};
\addplot [
color=blue,
solid,
mark=*,mark size=1.3pt,
mark options={solid,fill=mycolor50,draw=black,line width=0.15pt},
forget plot
]
coordinates{
 (25.1306,35.325001) 
};
\addplot [
color=blue,
solid,
mark=*,mark size=1.3pt,
mark options={solid,fill=mycolor51,draw=black,line width=0.15pt},
forget plot
]
coordinates{
 (23.733299,37.983299) 
};
\addplot [
color=blue,
solid,
mark=*,mark size=1.3pt,
mark options={solid,fill=mycolor24,draw=black,line width=0.15pt},
forget plot
]
coordinates{
 (22,39) 
};
\addplot [
color=blue,
solid,
mark=*,mark size=1.3pt,
mark options={solid,fill=mycolor48,draw=black,line width=0.15pt},
forget plot
]
coordinates{
 (25.1306,35.325001) 
};
\addplot [
color=blue,
solid,
mark=*,mark size=1.3pt,
mark options={solid,fill=mycolor32,draw=black,line width=0.15pt},
forget plot
]
coordinates{
 (114.216698,22.3167) 
};
\addplot [
color=blue,
solid,
mark=*,mark size=1.3pt,
mark options={solid,fill=mycolor25,draw=black,line width=0.15pt},
forget plot
]
coordinates{
 (19.0833,47.5) 
};
\addplot [
color=blue,
solid,
mark=*,mark size=1.3pt,
mark options={solid,fill=mycolor18,draw=black,line width=0.15pt},
forget plot
]
coordinates{
 (20,47) 
};
\addplot [
color=blue,
solid,
mark=*,mark size=1.3pt,
mark options={solid,fill=mycolor45,draw=black,line width=0.15pt},
forget plot
]
coordinates{
 (-18,65) 
};
\addplot [
color=blue,
solid,
mark=*,mark size=1.3pt,
mark options={solid,fill=mycolor52,draw=black,line width=0.15pt},
forget plot
]
coordinates{
 (72.966698,19.200001) 
};
\addplot [
color=blue,
solid,
mark=*,mark size=1.3pt,
mark options={solid,fill=mycolor42,draw=black,line width=0.15pt},
forget plot
]
coordinates{
 (76.516701,31.6833) 
};
\addplot [
color=blue,
solid,
mark=*,mark size=1.3pt,
mark options={solid,fill=mycolor32,draw=black,line width=0.15pt},
forget plot
]
coordinates{
 (106.829399,-6.1744) 
};
\addplot [
color=blue,
solid,
mark=*,mark size=1.3pt,
mark options={solid,fill=mycolor53,draw=black,line width=0.15pt},
forget plot
]
coordinates{
 (120,-5) 
};
\addplot [
color=blue,
solid,
mark=*,mark size=1.3pt,
mark options={solid,fill=mycolor54,draw=black,line width=0.15pt},
forget plot
]
coordinates{
 (106.829399,-6.1744) 
};
\addplot [
color=blue,
solid,
mark=*,mark size=1.3pt,
mark options={solid,fill=mycolor55,draw=black,line width=0.15pt},
forget plot
]
coordinates{
 (112.750801,-7.2492) 
};
\addplot [
color=blue,
solid,
mark=*,mark size=1.3pt,
mark options={solid,fill=mycolor56,draw=black,line width=0.15pt},
forget plot
]
coordinates{
 (112.630402,-7.9797) 
};
\addplot [
color=blue,
solid,
mark=*,mark size=1.3pt,
mark options={solid,fill=mycolor57,draw=black,line width=0.15pt},
forget plot
]
coordinates{
 (120,-5) 
};
\addplot [
color=blue,
solid,
mark=*,mark size=1.3pt,
mark options={solid,fill=mycolor12,draw=black,line width=0.15pt},
forget plot
]
coordinates{
 (110.831703,-7.5561) 
};
\addplot [
color=blue,
solid,
mark=*,mark size=1.3pt,
mark options={solid,fill=mycolor58,draw=black,line width=0.15pt},
forget plot
]
coordinates{
 (107.618599,-6.9039) 
};
\addplot [
color=blue,
solid,
mark=*,mark size=1.3pt,
mark options={solid,fill=mycolor25,draw=black,line width=0.15pt},
forget plot
]
coordinates{
 (-8,53) 
};
\addplot [
color=blue,
solid,
mark=*,mark size=1.3pt,
mark options={solid,fill=mycolor14,draw=black,line width=0.15pt},
forget plot
]
coordinates{
 (-6.2489,53.333099) 
};
\addplot [
color=blue,
solid,
mark=*,mark size=1.3pt,
mark options={solid,fill=mycolor59,draw=black,line width=0.15pt},
forget plot
]
coordinates{
 (34.766701,32.0667) 
};
\addplot [
color=blue,
solid,
mark=*,mark size=1.3pt,
mark options={solid,fill=mycolor60,draw=black,line width=0.15pt},
forget plot
]
coordinates{
 (7.6667,45.049999) 
};
\addplot [
color=blue,
solid,
mark=*,mark size=1.3pt,
mark options={solid,fill=mycolor43,draw=black,line width=0.15pt},
forget plot
]
coordinates{
 (12.4833,41.900002) 
};
\addplot [
color=blue,
solid,
mark=*,mark size=1.3pt,
mark options={solid,fill=mycolor19,draw=black,line width=0.15pt},
forget plot
]
coordinates{
 (12.6,42.783298) 
};
\addplot [
color=blue,
solid,
mark=*,mark size=1.3pt,
mark options={solid,fill=mycolor31,draw=black,line width=0.15pt},
forget plot
]
coordinates{
 (14.0667,41.533298) 
};
\addplot [
color=blue,
solid,
mark=*,mark size=1.3pt,
mark options={solid,fill=mycolor15,draw=black,line width=0.15pt},
forget plot
]
coordinates{
 (9.15,44.366699) 
};
\addplot [
color=blue,
solid,
mark=*,mark size=1.3pt,
mark options={solid,fill=mycolor7,draw=black,line width=0.15pt},
forget plot
]
coordinates{
 (136.482498,36.438301) 
};
\addplot [
color=blue,
solid,
mark=*,mark size=1.3pt,
mark options={solid,fill=mycolor61,draw=black,line width=0.15pt},
forget plot
]
coordinates{
 (139.623306,35.789398) 
};
\addplot [
color=blue,
solid,
mark=*,mark size=1.3pt,
mark options={solid,fill=mycolor6,draw=black,line width=0.15pt},
forget plot
]
coordinates{
 (140.116699,36.083302) 
};
\addplot [
color=blue,
solid,
mark=*,mark size=1.3pt,
mark options={solid,fill=mycolor62,draw=black,line width=0.15pt},
forget plot
]
coordinates{
 (137.211395,36.695301) 
};
\addplot [
color=blue,
solid,
mark=*,mark size=1.3pt,
mark options={solid,fill=mycolor7,draw=black,line width=0.15pt},
forget plot
]
coordinates{
 (132.766098,33.841702) 
};
\addplot [
color=blue,
solid,
mark=*,mark size=1.3pt,
mark options={solid,fill=mycolor28,draw=black,line width=0.15pt},
forget plot
]
coordinates{
 (76.949997,43.25) 
};
\addplot [
color=blue,
solid,
mark=*,mark size=1.3pt,
mark options={solid,fill=mycolor17,draw=black,line width=0.15pt},
forget plot
]
coordinates{
 (36.8167,-1.2833) 
};
\addplot [
color=blue,
solid,
mark=*,mark size=1.3pt,
mark options={solid,fill=mycolor41,draw=black,line width=0.15pt},
forget plot
]
coordinates{
 (126.978302,37.598499) 
};
\addplot [
color=blue,
solid,
mark=*,mark size=1.3pt,
mark options={solid,fill=mycolor11,draw=black,line width=0.15pt},
forget plot
]
coordinates{
 (127.419701,36.3214) 
};
\addplot [
color=blue,
solid,
mark=*,mark size=1.3pt,
mark options={solid,fill=mycolor61,draw=black,line width=0.15pt},
forget plot
]
coordinates{
 (127.5,37) 
};
\addplot [
color=blue,
solid,
mark=*,mark size=1.3pt,
mark options={solid,fill=mycolor30,draw=black,line width=0.15pt},
forget plot
]
coordinates{
 (47.978298,29.369699) 
};
\addplot [
color=blue,
solid,
mark=*,mark size=1.3pt,
mark options={solid,fill=mycolor26,draw=black,line width=0.15pt},
forget plot
]
coordinates{
 (24.1,56.950001) 
};
\addplot [
color=blue,
solid,
mark=*,mark size=1.3pt,
mark options={solid,fill=mycolor45,draw=black,line width=0.15pt},
forget plot
]
coordinates{
 (24.1,56.950001) 
};
\addplot [
color=blue,
solid,
mark=*,mark size=1.3pt,
mark options={solid,fill=mycolor18,draw=black,line width=0.15pt},
forget plot
]
coordinates{
 (24.2022,56.598598) 
};
\addplot [
color=blue,
solid,
mark=*,mark size=1.3pt,
mark options={solid,fill=mycolor45,draw=black,line width=0.15pt},
forget plot
]
coordinates{
 (23.9,54.900002) 
};
\addplot [
color=blue,
solid,
mark=*,mark size=1.3pt,
mark options={solid,fill=mycolor18,draw=black,line width=0.15pt},
forget plot
]
coordinates{
 (25.3167,54.6833) 
};
\addplot [
color=blue,
solid,
mark=*,mark size=1.3pt,
mark options={solid,fill=mycolor45,draw=black,line width=0.15pt},
forget plot
]
coordinates{
 (23.9,54.900002) 
};
\addplot [
color=blue,
solid,
mark=*,mark size=1.3pt,
mark options={solid,fill=mycolor15,draw=black,line width=0.15pt},
forget plot
]
coordinates{
 (6.1667,49.75) 
};
\addplot [
color=blue,
solid,
mark=*,mark size=1.3pt,
mark options={solid,fill=mycolor50,draw=black,line width=0.15pt},
forget plot
]
coordinates{
 (21.4333,42) 
};
\addplot [
color=blue,
solid,
mark=*,mark size=1.3pt,
mark options={solid,fill=mycolor8,draw=black,line width=0.15pt},
forget plot
]
coordinates{
 (101.716698,3.0333) 
};
\addplot [
color=blue,
solid,
mark=*,mark size=1.3pt,
mark options={solid,fill=mycolor48,draw=black,line width=0.15pt},
forget plot
]
coordinates{
 (14.4611,35.897202) 
};
\addplot [
color=blue,
solid,
mark=*,mark size=1.3pt,
mark options={solid,fill=mycolor11,draw=black,line width=0.15pt},
forget plot
]
coordinates{
 (105,46) 
};
\addplot [
color=blue,
solid,
mark=*,mark size=1.3pt,
mark options={solid,fill=mycolor2,draw=black,line width=0.15pt},
forget plot
]
coordinates{
 (17.083599,-22.57) 
};
\addplot [
color=blue,
solid,
mark=*,mark size=1.3pt,
mark options={solid,fill=mycolor29,draw=black,line width=0.15pt},
forget plot
]
coordinates{
 (84,28) 
};
\addplot [
color=blue,
solid,
mark=*,mark size=1.3pt,
mark options={solid,fill=mycolor14,draw=black,line width=0.15pt},
forget plot
]
coordinates{
 (5.6777,52.035301) 
};
\addplot [
color=blue,
solid,
mark=*,mark size=1.3pt,
mark options={solid,fill=mycolor19,draw=black,line width=0.15pt},
forget plot
]
coordinates{
 (6.8912,52.219501) 
};
\addplot [
color=blue,
solid,
mark=*,mark size=1.3pt,
mark options={solid,fill=mycolor19,draw=black,line width=0.15pt},
forget plot
]
coordinates{
 (5.75,52.5) 
};
\addplot [
color=blue,
solid,
mark=*,mark size=1.3pt,
mark options={solid,fill=mycolor19,draw=black,line width=0.15pt},
forget plot
]
coordinates{
 (4.9167,52.349998) 
};
\addplot [
color=blue,
solid,
mark=*,mark size=1.3pt,
mark options={solid,fill=mycolor26,draw=black,line width=0.15pt},
forget plot
]
coordinates{
 (4.9167,52.349998) 
};
\addplot [
color=blue,
solid,
mark=*,mark size=1.3pt,
mark options={solid,fill=mycolor19,draw=black,line width=0.15pt},
forget plot
]
coordinates{
 (5.75,52.5) 
};
\addplot [
color=blue,
solid,
mark=*,mark size=1.3pt,
mark options={solid,fill=mycolor19,draw=black,line width=0.15pt},
forget plot
]
coordinates{
 (5.75,52.5) 
};
\addplot [
color=blue,
solid,
mark=*,mark size=1.3pt,
mark options={solid,fill=mycolor19,draw=black,line width=0.15pt},
forget plot
]
coordinates{
 (4.3782,52.0186) 
};
\addplot [
color=blue,
solid,
mark=*,mark size=1.3pt,
mark options={solid,fill=mycolor63,draw=black,line width=0.15pt},
forget plot
]
coordinates{
 (-122.210999,37.491402) 
};
\addplot [
color=blue,
solid,
mark=*,mark size=1.3pt,
mark options={solid,fill=mycolor15,draw=black,line width=0.15pt},
forget plot
]
coordinates{
 (5.75,52.5) 
};
\addplot [
color=blue,
solid,
mark=*,mark size=1.3pt,
mark options={solid,fill=mycolor37,draw=black,line width=0.15pt},
forget plot
]
coordinates{
 (166.449997,-22.266701) 
};
\addplot [
color=blue,
solid,
mark=*,mark size=1.3pt,
mark options={solid,fill=mycolor63,draw=black,line width=0.15pt},
forget plot
]
coordinates{
 (174.783295,-41.299999) 
};
\addplot [
color=blue,
solid,
mark=*,mark size=1.3pt,
mark options={solid,fill=mycolor7,draw=black,line width=0.15pt},
forget plot
]
coordinates{
 (174.766693,-36.866699) 
};
\addplot [
color=blue,
solid,
mark=*,mark size=1.3pt,
mark options={solid,fill=mycolor12,draw=black,line width=0.15pt},
forget plot
]
coordinates{
 (172.633301,-43.533298) 
};
\addplot [
color=blue,
solid,
mark=*,mark size=1.3pt,
mark options={solid,fill=mycolor17,draw=black,line width=0.15pt},
forget plot
]
coordinates{
 (174.783295,-41.299999) 
};
\addplot [
color=blue,
solid,
mark=*,mark size=1.3pt,
mark options={solid,fill=mycolor18,draw=black,line width=0.15pt},
forget plot
]
coordinates{
 (10,62) 
};
\addplot [
color=blue,
solid,
mark=*,mark size=1.3pt,
mark options={solid,fill=mycolor18,draw=black,line width=0.15pt},
forget plot
]
coordinates{
 (11.0333,59.983299) 
};
\addplot [
color=blue,
solid,
mark=*,mark size=1.3pt,
mark options={solid,fill=mycolor46,draw=black,line width=0.15pt},
forget plot
]
coordinates{
 (20.25,63.833302) 
};
\addplot [
color=blue,
solid,
mark=*,mark size=1.3pt,
mark options={solid,fill=mycolor46,draw=black,line width=0.15pt},
forget plot
]
coordinates{
 (5.3247,60.391102) 
};
\addplot [
color=blue,
solid,
mark=*,mark size=1.3pt,
mark options={solid,fill=mycolor16,draw=black,line width=0.15pt},
forget plot
]
coordinates{
 (58.5933,23.6133) 
};
\addplot [
color=blue,
solid,
mark=*,mark size=1.3pt,
mark options={solid,fill=mycolor64,draw=black,line width=0.15pt},
forget plot
]
coordinates{
 (67.082199,24.9056) 
};
\addplot [
color=blue,
solid,
mark=*,mark size=1.3pt,
mark options={solid,fill=mycolor63,draw=black,line width=0.15pt},
forget plot
]
coordinates{
 (67.082199,24.9056) 
};
\addplot [
color=blue,
solid,
mark=*,mark size=1.3pt,
mark options={solid,fill=mycolor37,draw=black,line width=0.15pt},
forget plot
]
coordinates{
 (121.050903,14.6488) 
};
\addplot [
color=blue,
solid,
mark=*,mark size=1.3pt,
mark options={solid,fill=mycolor46,draw=black,line width=0.15pt},
forget plot
]
coordinates{
 (21,52.25) 
};
\addplot [
color=blue,
solid,
mark=*,mark size=1.3pt,
mark options={solid,fill=mycolor45,draw=black,line width=0.15pt},
forget plot
]
coordinates{
 (16.1619,51.210098) 
};
\addplot [
color=blue,
solid,
mark=*,mark size=1.3pt,
mark options={solid,fill=mycolor45,draw=black,line width=0.15pt},
forget plot
]
coordinates{
 (18.549999,54.5) 
};
\addplot [
color=blue,
solid,
mark=*,mark size=1.3pt,
mark options={solid,fill=mycolor18,draw=black,line width=0.15pt},
forget plot
]
coordinates{
 (23.15,53.133301) 
};
\addplot [
color=blue,
solid,
mark=*,mark size=1.3pt,
mark options={solid,fill=mycolor45,draw=black,line width=0.15pt},
forget plot
]
coordinates{
 (-8.4201,41.550301) 
};
\addplot [
color=blue,
solid,
mark=*,mark size=1.3pt,
mark options={solid,fill=mycolor46,draw=black,line width=0.15pt},
forget plot
]
coordinates{
 (-8.611,41.149601) 
};
\addplot [
color=blue,
solid,
mark=*,mark size=1.3pt,
mark options={solid,fill=mycolor46,draw=black,line width=0.15pt},
forget plot
]
coordinates{
 (-8.4196,40.205601) 
};
\addplot [
color=blue,
solid,
mark=*,mark size=1.3pt,
mark options={solid,fill=mycolor46,draw=black,line width=0.15pt},
forget plot
]
coordinates{
 (-9.1333,38.716702) 
};
\addplot [
color=blue,
solid,
mark=*,mark size=1.3pt,
mark options={solid,fill=mycolor51,draw=black,line width=0.15pt},
forget plot
]
coordinates{
 (-8,39.5) 
};
\addplot [
color=blue,
solid,
mark=*,mark size=1.3pt,
mark options={solid,fill=mycolor45,draw=black,line width=0.15pt},
forget plot
]
coordinates{
 (-8.4196,40.205601) 
};
\addplot [
color=blue,
solid,
mark=*,mark size=1.3pt,
mark options={solid,fill=mycolor45,draw=black,line width=0.15pt},
forget plot
]
coordinates{
 (-8.805,39.7477) 
};
\addplot [
color=blue,
solid,
mark=*,mark size=1.3pt,
mark options={solid,fill=mycolor51,draw=black,line width=0.15pt},
forget plot
]
coordinates{
 (25,46) 
};
\addplot [
color=blue,
solid,
mark=*,mark size=1.3pt,
mark options={solid,fill=mycolor26,draw=black,line width=0.15pt},
forget plot
]
coordinates{
 (21.3167,46.1833) 
};
\addplot [
color=blue,
solid,
mark=*,mark size=1.3pt,
mark options={solid,fill=mycolor51,draw=black,line width=0.15pt},
forget plot
]
coordinates{
 (37.615601,55.752201) 
};
\addplot [
color=blue,
solid,
mark=*,mark size=1.3pt,
mark options={solid,fill=mycolor45,draw=black,line width=0.15pt},
forget plot
]
coordinates{
 (37.615601,55.752201) 
};
\addplot [
color=blue,
solid,
mark=*,mark size=1.3pt,
mark options={solid,fill=mycolor50,draw=black,line width=0.15pt},
forget plot
]
coordinates{
 (37.615601,55.752201) 
};
\addplot [
color=blue,
solid,
mark=*,mark size=1.3pt,
mark options={solid,fill=mycolor51,draw=black,line width=0.15pt},
forget plot
]
coordinates{
 (37.615601,55.752201) 
};
\addplot [
color=blue,
solid,
mark=*,mark size=1.3pt,
mark options={solid,fill=mycolor46,draw=black,line width=0.15pt},
forget plot
]
coordinates{
 (37.615601,55.752201) 
};
\addplot [
color=blue,
solid,
mark=*,mark size=1.3pt,
mark options={solid,fill=mycolor44,draw=black,line width=0.15pt},
forget plot
]
coordinates{
 (82.934601,55.0415) 
};
\addplot [
color=blue,
solid,
mark=*,mark size=1.3pt,
mark options={solid,fill=mycolor65,draw=black,line width=0.15pt},
forget plot
]
coordinates{
 (56.2841,58.0093) 
};
\addplot [
color=blue,
solid,
mark=*,mark size=1.3pt,
mark options={solid,fill=mycolor18,draw=black,line width=0.15pt},
forget plot
]
coordinates{
 (30.2642,59.894402) 
};
\addplot [
color=blue,
solid,
mark=*,mark size=1.3pt,
mark options={solid,fill=mycolor66,draw=black,line width=0.15pt},
forget plot
]
coordinates{
 (44.501801,48.719398) 
};
\addplot [
color=blue,
solid,
mark=*,mark size=1.3pt,
mark options={solid,fill=mycolor28,draw=black,line width=0.15pt},
forget plot
]
coordinates{
 (46.715199,24.6537) 
};
\addplot [
color=blue,
solid,
mark=*,mark size=1.3pt,
mark options={solid,fill=mycolor14,draw=black,line width=0.15pt},
forget plot
]
coordinates{
 (20.4681,44.8186) 
};
\addplot [
color=blue,
solid,
mark=*,mark size=1.3pt,
mark options={solid,fill=mycolor56,draw=black,line width=0.15pt},
forget plot
]
coordinates{
 (103.800003,1.3667) 
};
\addplot [
color=blue,
solid,
mark=*,mark size=1.3pt,
mark options={solid,fill=mycolor67,draw=black,line width=0.15pt},
forget plot
]
coordinates{
 (103.800003,1.3667) 
};
\addplot [
color=blue,
solid,
mark=*,mark size=1.3pt,
mark options={solid,fill=mycolor65,draw=black,line width=0.15pt},
forget plot
]
coordinates{
 (-122.113403,47.605099) 
};
\addplot [
color=blue,
solid,
mark=*,mark size=1.3pt,
mark options={solid,fill=mycolor25,draw=black,line width=0.15pt},
forget plot
]
coordinates{
 (19.5,48.666698) 
};
\addplot [
color=blue,
solid,
mark=*,mark size=1.3pt,
mark options={solid,fill=mycolor43,draw=black,line width=0.15pt},
forget plot
]
coordinates{
 (18.4,48.383301) 
};
\addplot [
color=blue,
solid,
mark=*,mark size=1.3pt,
mark options={solid,fill=mycolor43,draw=black,line width=0.15pt},
forget plot
]
coordinates{
 (17.116699,48.150002) 
};
\addplot [
color=blue,
solid,
mark=*,mark size=1.3pt,
mark options={solid,fill=mycolor43,draw=black,line width=0.15pt},
forget plot
]
coordinates{
 (14.5144,46.055302) 
};
\addplot [
color=blue,
solid,
mark=*,mark size=1.3pt,
mark options={solid,fill=mycolor43,draw=black,line width=0.15pt},
forget plot
]
coordinates{
 (14.5667,46.133301) 
};
\addplot [
color=blue,
solid,
mark=*,mark size=1.3pt,
mark options={solid,fill=mycolor29,draw=black,line width=0.15pt},
forget plot
]
coordinates{
 (24,-29) 
};
\addplot [
color=blue,
solid,
mark=*,mark size=1.3pt,
mark options={solid,fill=mycolor68,draw=black,line width=0.15pt},
forget plot
]
coordinates{
 (18.423201,-33.9258) 
};
\addplot [
color=blue,
solid,
mark=*,mark size=1.3pt,
mark options={solid,fill=mycolor69,draw=black,line width=0.15pt},
forget plot
]
coordinates{
 (28.049801,-26.2052) 
};
\addplot [
color=blue,
solid,
mark=*,mark size=1.3pt,
mark options={solid,fill=mycolor58,draw=black,line width=0.15pt},
forget plot
]
coordinates{
 (18.8668,-33.934601) 
};
\addplot [
color=blue,
solid,
mark=*,mark size=1.3pt,
mark options={solid,fill=mycolor46,draw=black,line width=0.15pt},
forget plot
]
coordinates{
 (-8.4068,43.3666) 
};
\addplot [
color=blue,
solid,
mark=*,mark size=1.3pt,
mark options={solid,fill=mycolor18,draw=black,line width=0.15pt},
forget plot
]
coordinates{
 (-5.9761,37.382401) 
};
\addplot [
color=blue,
solid,
mark=*,mark size=1.3pt,
mark options={solid,fill=mycolor26,draw=black,line width=0.15pt},
forget plot
]
coordinates{
 (-3.8986,43.4454) 
};
\addplot [
color=blue,
solid,
mark=*,mark size=1.3pt,
mark options={solid,fill=mycolor45,draw=black,line width=0.15pt},
forget plot
]
coordinates{
 (-1.13,37.987) 
};
\addplot [
color=blue,
solid,
mark=*,mark size=1.3pt,
mark options={solid,fill=mycolor30,draw=black,line width=0.15pt},
forget plot
]
coordinates{
 (2.8211,41.984402) 
};
\addplot [
color=blue,
solid,
mark=*,mark size=1.3pt,
mark options={solid,fill=mycolor70,draw=black,line width=0.15pt},
forget plot
]
coordinates{
 (80.584099,7.2622) 
};
\addplot [
color=blue,
solid,
mark=*,mark size=1.3pt,
mark options={solid,fill=mycolor18,draw=black,line width=0.15pt},
forget plot
]
coordinates{
 (15,62) 
};
\addplot [
color=blue,
solid,
mark=*,mark size=1.3pt,
mark options={solid,fill=mycolor26,draw=black,line width=0.15pt},
forget plot
]
coordinates{
 (8,47) 
};
\addplot [
color=blue,
solid,
mark=*,mark size=1.3pt,
mark options={solid,fill=mycolor25,draw=black,line width=0.15pt},
forget plot
]
coordinates{
 (15,62) 
};
\addplot [
color=blue,
solid,
mark=*,mark size=1.3pt,
mark options={solid,fill=mycolor45,draw=black,line width=0.15pt},
forget plot
]
coordinates{
 (13.1833,55.700001) 
};
\addplot [
color=blue,
solid,
mark=*,mark size=1.3pt,
mark options={solid,fill=mycolor18,draw=black,line width=0.15pt},
forget plot
]
coordinates{
 (14.1833,57.783298) 
};
\addplot [
color=blue,
solid,
mark=*,mark size=1.3pt,
mark options={solid,fill=mycolor18,draw=black,line width=0.15pt},
forget plot
]
coordinates{
 (15.6167,58.416698) 
};
\addplot [
color=blue,
solid,
mark=*,mark size=1.3pt,
mark options={solid,fill=mycolor14,draw=black,line width=0.15pt},
forget plot
]
coordinates{
 (15,62) 
};
\addplot [
color=blue,
solid,
mark=*,mark size=1.3pt,
mark options={solid,fill=mycolor14,draw=black,line width=0.15pt},
forget plot
]
coordinates{
 (13,55.599998) 
};
\addplot [
color=blue,
solid,
mark=*,mark size=1.3pt,
mark options={solid,fill=mycolor18,draw=black,line width=0.15pt},
forget plot
]
coordinates{
 (18.049999,59.333302) 
};
\addplot [
color=blue,
solid,
mark=*,mark size=1.3pt,
mark options={solid,fill=mycolor1,draw=black,line width=0.15pt},
forget plot
]
coordinates{
 (8.55,47.366699) 
};
\addplot [
color=blue,
solid,
mark=*,mark size=1.3pt,
mark options={solid,fill=mycolor1,draw=black,line width=0.15pt},
forget plot
]
coordinates{
 (8.55,47.366699) 
};
\addplot [
color=blue,
solid,
mark=*,mark size=1.3pt,
mark options={solid,fill=mycolor38,draw=black,line width=0.15pt},
forget plot
]
coordinates{
 (120.971397,24.804701) 
};
\addplot [
color=blue,
solid,
mark=*,mark size=1.3pt,
mark options={solid,fill=mycolor12,draw=black,line width=0.15pt},
forget plot
]
coordinates{
 (121.525002,25.0392) 
};
\addplot [
color=blue,
solid,
mark=*,mark size=1.3pt,
mark options={solid,fill=mycolor71,draw=black,line width=0.15pt},
forget plot
]
coordinates{
 (121.525002,25.0392) 
};
\addplot [
color=blue,
solid,
mark=*,mark size=1.3pt,
mark options={solid,fill=mycolor9,draw=black,line width=0.15pt},
forget plot
]
coordinates{
 (121.525002,25.0392) 
};
\addplot [
color=blue,
solid,
mark=*,mark size=1.3pt,
mark options={solid,fill=mycolor10,draw=black,line width=0.15pt},
forget plot
]
coordinates{
 (121.525002,25.0392) 
};
\addplot [
color=blue,
solid,
mark=*,mark size=1.3pt,
mark options={solid,fill=mycolor9,draw=black,line width=0.15pt},
forget plot
]
coordinates{
 (120.971397,24.804701) 
};
\addplot [
color=blue,
solid,
mark=*,mark size=1.3pt,
mark options={solid,fill=mycolor41,draw=black,line width=0.15pt},
forget plot
]
coordinates{
 (120.349998,22.633301) 
};
\addplot [
color=blue,
solid,
mark=*,mark size=1.3pt,
mark options={solid,fill=mycolor72,draw=black,line width=0.15pt},
forget plot
]
coordinates{
 (121.525002,25.0392) 
};
\addplot [
color=blue,
solid,
mark=*,mark size=1.3pt,
mark options={solid,fill=mycolor41,draw=black,line width=0.15pt},
forget plot
]
coordinates{
 (100.501404,13.754) 
};
\addplot [
color=blue,
solid,
mark=*,mark size=1.3pt,
mark options={solid,fill=mycolor8,draw=black,line width=0.15pt},
forget plot
]
coordinates{
 (100.501404,13.754) 
};
\addplot [
color=blue,
solid,
mark=*,mark size=1.3pt,
mark options={solid,fill=mycolor54,draw=black,line width=0.15pt},
forget plot
]
coordinates{
 (100,15) 
};
\addplot [
color=blue,
solid,
mark=*,mark size=1.3pt,
mark options={solid,fill=mycolor62,draw=black,line width=0.15pt},
forget plot
]
coordinates{
 (100,15) 
};
\addplot [
color=blue,
solid,
mark=*,mark size=1.3pt,
mark options={solid,fill=mycolor46,draw=black,line width=0.15pt},
forget plot
]
coordinates{
 (10.1711,36.806099) 
};
\addplot [
color=blue,
solid,
mark=*,mark size=1.3pt,
mark options={solid,fill=mycolor45,draw=black,line width=0.15pt},
forget plot
]
coordinates{
 (32.840302,39.911701) 
};
\addplot [
color=blue,
solid,
mark=*,mark size=1.3pt,
mark options={solid,fill=mycolor45,draw=black,line width=0.15pt},
forget plot
]
coordinates{
 (32,49) 
};
\addplot [
color=blue,
solid,
mark=*,mark size=1.3pt,
mark options={solid,fill=mycolor18,draw=black,line width=0.15pt},
forget plot
]
coordinates{
 (32,49) 
};
\addplot [
color=blue,
solid,
mark=*,mark size=1.3pt,
mark options={solid,fill=mycolor26,draw=black,line width=0.15pt},
forget plot
]
coordinates{
 (30.516701,50.4333) 
};
\addplot [
color=blue,
solid,
mark=*,mark size=1.3pt,
mark options={solid,fill=mycolor43,draw=black,line width=0.15pt},
forget plot
]
coordinates{
 (1.0833,51.266701) 
};
\addplot [
color=blue,
solid,
mark=*,mark size=1.3pt,
mark options={solid,fill=mycolor15,draw=black,line width=0.15pt},
forget plot
]
coordinates{
 (-2,54) 
};
\addplot [
color=blue,
solid,
mark=*,mark size=1.3pt,
mark options={solid,fill=mycolor43,draw=black,line width=0.15pt},
forget plot
]
coordinates{
 (-2,54) 
};
\addplot [
color=blue,
solid,
mark=*,mark size=1.3pt,
mark options={solid,fill=mycolor44,draw=black,line width=0.15pt},
forget plot
]
coordinates{
 (-2,54) 
};
\addplot [
color=blue,
solid,
mark=*,mark size=1.3pt,
mark options={solid,fill=mycolor14,draw=black,line width=0.15pt},
forget plot
]
coordinates{
 (-1.25,51.75) 
};
\addplot [
color=blue,
solid,
mark=*,mark size=1.3pt,
mark options={solid,fill=mycolor65,draw=black,line width=0.15pt},
forget plot
]
coordinates{
 (-2,54) 
};
\addplot [
color=blue,
solid,
mark=*,mark size=1.3pt,
mark options={solid,fill=mycolor14,draw=black,line width=0.15pt},
forget plot
]
coordinates{
 (-2,54) 
};
\addplot [
color=blue,
solid,
mark=*,mark size=1.3pt,
mark options={solid,fill=mycolor15,draw=black,line width=0.15pt},
forget plot
]
coordinates{
 (-2,54) 
};
\addplot [
color=blue,
solid,
mark=*,mark size=1.3pt,
mark options={solid,fill=mycolor25,draw=black,line width=0.15pt},
forget plot
]
coordinates{
 (-2.2167,53.5) 
};
\addplot [
color=blue,
solid,
mark=*,mark size=1.3pt,
mark options={solid,fill=mycolor15,draw=black,line width=0.15pt},
forget plot
]
coordinates{
 (-2,54) 
};
\addplot [
color=blue,
solid,
mark=*,mark size=1.3pt,
mark options={solid,fill=mycolor14,draw=black,line width=0.15pt},
forget plot
]
coordinates{
 (-2,54) 
};
\addplot [
color=blue,
solid,
mark=*,mark size=1.3pt,
mark options={solid,fill=mycolor19,draw=black,line width=0.15pt},
forget plot
]
coordinates{
 (-2,54) 
};
\addplot [
color=blue,
solid,
mark=*,mark size=1.3pt,
mark options={solid,fill=mycolor26,draw=black,line width=0.15pt},
forget plot
]
coordinates{
 (-0.0931,51.514198) 
};
\addplot [
color=blue,
solid,
mark=*,mark size=1.3pt,
mark options={solid,fill=mycolor43,draw=black,line width=0.15pt},
forget plot
]
coordinates{
 (-0.4667,51.75) 
};
\addplot [
color=blue,
solid,
mark=*,mark size=1.3pt,
mark options={solid,fill=mycolor14,draw=black,line width=0.15pt},
forget plot
]
coordinates{
 (-2.6,53.383301) 
};
\addplot [
color=blue,
solid,
mark=*,mark size=1.3pt,
mark options={solid,fill=mycolor73,draw=black,line width=0.15pt},
forget plot
]
coordinates{
 (-88.083298,41.786999) 
};
\addplot [
color=blue,
solid,
mark=*,mark size=1.3pt,
mark options={solid,fill=mycolor73,draw=black,line width=0.15pt},
forget plot
]
coordinates{
 (-75.514801,39.3498) 
};
\addplot [
color=blue,
solid,
mark=*,mark size=1.3pt,
mark options={solid,fill=mycolor73,draw=black,line width=0.15pt},
forget plot
]
coordinates{
 (-73.965302,40.800598) 
};
\addplot [
color=blue,
solid,
mark=*,mark size=1.3pt,
mark options={solid,fill=mycolor69,draw=black,line width=0.15pt},
forget plot
]
coordinates{
 (-122.737701,45.5508) 
};
\addplot [
color=blue,
solid,
mark=*,mark size=1.3pt,
mark options={solid,fill=mycolor63,draw=black,line width=0.15pt},
forget plot
]
coordinates{
 (-123.262001,44.564602) 
};
\addplot [
color=blue,
solid,
mark=*,mark size=1.3pt,
mark options={solid,fill=mycolor74,draw=black,line width=0.15pt},
forget plot
]
coordinates{
 (-104.999496,39.752499) 
};
\addplot [
color=blue,
solid,
mark=*,mark size=1.3pt,
mark options={solid,fill=mycolor31,draw=black,line width=0.15pt},
forget plot
]
coordinates{
 (-84.488197,42.728298) 
};
\addplot [
color=blue,
solid,
mark=*,mark size=1.3pt,
mark options={solid,fill=mycolor64,draw=black,line width=0.15pt},
forget plot
]
coordinates{
 (-82.412804,28.063101) 
};
\addplot [
color=blue,
solid,
mark=*,mark size=1.3pt,
mark options={solid,fill=mycolor74,draw=black,line width=0.15pt},
forget plot
]
coordinates{
 (-96.171097,41.291698) 
};
\addplot [
color=blue,
solid,
mark=*,mark size=1.3pt,
mark options={solid,fill=mycolor27,draw=black,line width=0.15pt},
forget plot
]
coordinates{
 (-77.076302,38.914398) 
};
\addplot [
color=blue,
solid,
mark=*,mark size=1.3pt,
mark options={solid,fill=mycolor30,draw=black,line width=0.15pt},
forget plot
]
coordinates{
 (-75.1791,40.4259) 
};
\addplot [
color=blue,
solid,
mark=*,mark size=1.3pt,
mark options={solid,fill=mycolor75,draw=black,line width=0.15pt},
forget plot
]
coordinates{
 (-70.975601,43.304501) 
};
\addplot [
color=blue,
solid,
mark=*,mark size=1.3pt,
mark options={solid,fill=mycolor28,draw=black,line width=0.15pt},
forget plot
]
coordinates{
 (-87.650299,41.8745) 
};
\addplot [
color=blue,
solid,
mark=*,mark size=1.3pt,
mark options={solid,fill=mycolor65,draw=black,line width=0.15pt},
forget plot
]
coordinates{
 (-97,38) 
};
\addplot [
color=blue,
solid,
mark=*,mark size=1.3pt,
mark options={solid,fill=mycolor28,draw=black,line width=0.15pt},
forget plot
]
coordinates{
 (-97,38) 
};
\addplot [
color=blue,
solid,
mark=*,mark size=1.3pt,
mark options={solid,fill=mycolor44,draw=black,line width=0.15pt},
forget plot
]
coordinates{
 (-76.877701,38.833599) 
};
\addplot [
color=blue,
solid,
mark=*,mark size=1.3pt,
mark options={solid,fill=mycolor42,draw=black,line width=0.15pt},
forget plot
]
coordinates{
 (-123.054703,44.0364) 
};
\addplot [
color=blue,
solid,
mark=*,mark size=1.3pt,
mark options={solid,fill=mycolor63,draw=black,line width=0.15pt},
forget plot
]
coordinates{
 (-118.449303,34.3078) 
};
\addplot [
color=blue,
solid,
mark=*,mark size=1.3pt,
mark options={solid,fill=mycolor29,draw=black,line width=0.15pt},
forget plot
]
coordinates{
 (-122.693199,45.507301) 
};
\addplot [
color=blue,
solid,
mark=*,mark size=1.3pt,
mark options={solid,fill=mycolor75,draw=black,line width=0.15pt},
forget plot
]
coordinates{
 (-71.087799,42.3424) 
};
\addplot [
color=blue,
solid,
mark=*,mark size=1.3pt,
mark options={solid,fill=mycolor70,draw=black,line width=0.15pt},
forget plot
]
coordinates{
 (-118.4925,34.015999) 
};
\addplot [
color=blue,
solid,
mark=*,mark size=1.3pt,
mark options={solid,fill=mycolor76,draw=black,line width=0.15pt},
forget plot
]
coordinates{
 (-84.636002,42.7257) 
};
\addplot [
color=blue,
solid,
mark=*,mark size=1.3pt,
mark options={solid,fill=mycolor28,draw=black,line width=0.15pt},
forget plot
]
coordinates{
 (-71.102798,42.364601) 
};
\addplot [
color=blue,
solid,
mark=*,mark size=1.3pt,
mark options={solid,fill=mycolor44,draw=black,line width=0.15pt},
forget plot
]
coordinates{
 (-77.615601,43.1548) 
};
\addplot [
color=blue,
solid,
mark=*,mark size=1.3pt,
mark options={solid,fill=mycolor70,draw=black,line width=0.15pt},
forget plot
]
coordinates{
 (-116.203499,43.613499) 
};
\addplot [
color=blue,
solid,
mark=*,mark size=1.3pt,
mark options={solid,fill=mycolor63,draw=black,line width=0.15pt},
forget plot
]
coordinates{
 (-111.891098,40.760799) 
};
\addplot [
color=blue,
solid,
mark=*,mark size=1.3pt,
mark options={solid,fill=mycolor31,draw=black,line width=0.15pt},
forget plot
]
coordinates{
 (-89.527397,43.0737) 
};
\addplot [
color=blue,
solid,
mark=*,mark size=1.3pt,
mark options={solid,fill=mycolor27,draw=black,line width=0.15pt},
forget plot
]
coordinates{
 (-74.024597,40.790401) 
};
\addplot [
color=blue,
solid,
mark=*,mark size=1.3pt,
mark options={solid,fill=mycolor30,draw=black,line width=0.15pt},
forget plot
]
coordinates{
 (-81.844299,32.372898) 
};
\addplot [
color=blue,
solid,
mark=*,mark size=1.3pt,
mark options={solid,fill=mycolor73,draw=black,line width=0.15pt},
forget plot
]
coordinates{
 (-79.929398,40.453098) 
};
\addplot [
color=blue,
solid,
mark=*,mark size=1.3pt,
mark options={solid,fill=mycolor35,draw=black,line width=0.15pt},
forget plot
]
coordinates{
 (-118.408798,46.044701) 
};
\addplot [
color=blue,
solid,
mark=*,mark size=1.3pt,
mark options={solid,fill=mycolor75,draw=black,line width=0.15pt},
forget plot
]
coordinates{
 (-122.270798,37.804401) 
};
\addplot [
color=blue,
solid,
mark=*,mark size=1.3pt,
mark options={solid,fill=mycolor28,draw=black,line width=0.15pt},
forget plot
]
coordinates{
 (-93.2323,44.973301) 
};
\addplot [
color=blue,
solid,
mark=*,mark size=1.3pt,
mark options={solid,fill=mycolor44,draw=black,line width=0.15pt},
forget plot
]
coordinates{
 (-79.9561,40.443901) 
};
\addplot [
color=blue,
solid,
mark=*,mark size=1.3pt,
mark options={solid,fill=mycolor75,draw=black,line width=0.15pt},
forget plot
]
coordinates{
 (-78.901901,35.995399) 
};
\addplot [
color=blue,
solid,
mark=*,mark size=1.3pt,
mark options={solid,fill=mycolor69,draw=black,line width=0.15pt},
forget plot
]
coordinates{
 (-122.182602,37.376202) 
};
\addplot [
color=blue,
solid,
mark=*,mark size=1.3pt,
mark options={solid,fill=mycolor73,draw=black,line width=0.15pt},
forget plot
]
coordinates{
 (-88.212303,40.109501) 
};
\addplot [
color=blue,
solid,
mark=*,mark size=1.3pt,
mark options={solid,fill=mycolor52,draw=black,line width=0.15pt},
forget plot
]
coordinates{
 (-88.647102,47.1544) 
};
\addplot [
color=blue,
solid,
mark=*,mark size=1.3pt,
mark options={solid,fill=mycolor54,draw=black,line width=0.15pt},
forget plot
]
coordinates{
 (-157.816696,21.3267) 
};
\addplot [
color=blue,
solid,
mark=*,mark size=1.3pt,
mark options={solid,fill=mycolor30,draw=black,line width=0.15pt},
forget plot
]
coordinates{
 (-74.925797,44.6609) 
};
\addplot [
color=blue,
solid,
mark=*,mark size=1.3pt,
mark options={solid,fill=mycolor69,draw=black,line width=0.15pt},
forget plot
]
coordinates{
 (-117.714302,34.122299) 
};
\addplot [
color=blue,
solid,
mark=*,mark size=1.3pt,
mark options={solid,fill=mycolor69,draw=black,line width=0.15pt},
forget plot
]
coordinates{
 (-121.639801,38.482899) 
};
\addplot [
color=blue,
solid,
mark=*,mark size=1.3pt,
mark options={solid,fill=mycolor77,draw=black,line width=0.15pt},
forget plot
]
coordinates{
 (-121.896202,37.515499) 
};
\addplot [
color=blue,
solid,
mark=*,mark size=1.3pt,
mark options={solid,fill=mycolor69,draw=black,line width=0.15pt},
forget plot
]
coordinates{
 (-120.503899,37.411499) 
};
\addplot [
color=blue,
solid,
mark=*,mark size=1.3pt,
mark options={solid,fill=mycolor78,draw=black,line width=0.15pt},
forget plot
]
coordinates{
 (-111.917099,33.435699) 
};
\addplot [
color=blue,
solid,
mark=*,mark size=1.3pt,
mark options={solid,fill=mycolor73,draw=black,line width=0.15pt},
forget plot
]
coordinates{
 (-71.059799,42.358398) 
};
\addplot [
color=blue,
solid,
mark=*,mark size=1.3pt,
mark options={solid,fill=mycolor73,draw=black,line width=0.15pt},
forget plot
]
coordinates{
 (-71.099297,42.3451) 
};
\addplot [
color=blue,
solid,
mark=*,mark size=1.3pt,
mark options={solid,fill=mycolor73,draw=black,line width=0.15pt},
forget plot
]
coordinates{
 (-73.843399,40.844898) 
};
\addplot [
color=blue,
solid,
mark=*,mark size=1.3pt,
mark options={solid,fill=mycolor28,draw=black,line width=0.15pt},
forget plot
]
coordinates{
 (-86.526398,39.165298) 
};
\addplot [
color=blue,
solid,
mark=*,mark size=1.3pt,
mark options={solid,fill=mycolor74,draw=black,line width=0.15pt},
forget plot
]
coordinates{
 (-97.7369,30.296101) 
};
\addplot [
color=blue,
solid,
mark=*,mark size=1.3pt,
mark options={solid,fill=mycolor30,draw=black,line width=0.15pt},
forget plot
]
coordinates{
 (-77.460297,37.553799) 
};
\addplot [
color=blue,
solid,
mark=*,mark size=1.3pt,
mark options={solid,fill=mycolor31,draw=black,line width=0.15pt},
forget plot
]
coordinates{
 (-85.616699,42.268299) 
};
\addplot [
color=blue,
solid,
mark=*,mark size=1.3pt,
mark options={solid,fill=mycolor31,draw=black,line width=0.15pt},
forget plot
]
coordinates{
 (64,41) 
};
\addplot [
color=blue,
solid,
mark=*,mark size=1.3pt,
mark options={solid,fill=mycolor31,draw=black,line width=0.15pt},
forget plot
]
coordinates{
 (64,41) 
};
\addplot [
color=blue,
solid,
mark=*,mark size=1.3pt,
mark options={solid,fill=mycolor56,draw=black,line width=0.15pt},
forget plot
]
coordinates{
 (106.643799,10.8142) 
};
\addplot [
color=blue,
solid,
mark=*,mark size=1.3pt,
mark options={solid,fill=mycolor24,draw=black,line width=0.15pt},
forget plot
]
coordinates{
 (-25.6667,37.733299) 
};
\addplot [
color=blue,
solid,
mark=*,mark size=1.3pt,
mark options={solid,fill=mycolor59,draw=black,line width=0.15pt},
forget plot
]
coordinates{
 (-16.9,32.633301) 
};
\addplot [
color=blue,
solid,
mark=*,mark size=1.3pt,
mark options={solid,fill=mycolor50,draw=black,line width=0.15pt},
forget plot
]
coordinates{
 (-21.950001,64.150002) 
};
\addplot [
color=blue,
solid,
mark=*,mark size=1.3pt,
mark options={solid,fill=mycolor42,draw=black,line width=0.15pt},
forget plot
]
coordinates{
 (-111.890602,33.6119) 
};
\addplot [
color=blue,
solid,
mark=*,mark size=1.3pt,
mark options={solid,fill=mycolor70,draw=black,line width=0.15pt},
forget plot
]
coordinates{
 (39.283298,-6.8) 
};
\addplot [
color=blue,
solid,
mark=*,mark size=1.3pt,
mark options={solid,fill=mycolor23,draw=black,line width=0.15pt},
forget plot
]
coordinates{
 (47.516701,-18.9167) 
};
\addplot [
color=blue,
solid,
mark=*,mark size=1.3pt,
mark options={solid,fill=mycolor79,draw=black,line width=0.15pt},
forget plot
]
coordinates{
 (39.283298,-6.8) 
};
\addplot [
color=blue,
solid,
mark=*,mark size=1.3pt,
mark options={solid,fill=mycolor80,draw=black,line width=0.15pt},
forget plot
]
coordinates{
 (-77.050003,-12.05) 
};
\addplot [
color=blue,
solid,
mark=*,mark size=1.3pt,
mark options={solid,fill=mycolor29,draw=black,line width=0.15pt},
forget plot
]
coordinates{
 (30,15) 
};
\addplot [
color=blue,
solid,
mark=*,mark size=1.3pt,
mark options={solid,fill=mycolor30,draw=black,line width=0.15pt},
forget plot
]
coordinates{
 (30,15) 
};
\addplot [
color=blue,
solid,
mark=*,mark size=1.3pt,
mark options={solid,fill=mycolor52,draw=black,line width=0.15pt},
forget plot
]
coordinates{
 (-95.366997,29.7523) 
};
\addplot [
color=blue,
solid,
mark=*,mark size=1.3pt,
mark options={solid,fill=mycolor44,draw=black,line width=0.15pt},
forget plot
]
coordinates{
 (32.299999,31.266701) 
};
\addplot [
color=blue,
solid,
mark=*,mark size=1.3pt,
mark options={solid,fill=mycolor73,draw=black,line width=0.15pt},
forget plot
]
coordinates{
 (29.919201,31.198099) 
};
\addplot [
color=blue,
solid,
mark=*,mark size=1.3pt,
mark options={solid,fill=mycolor44,draw=black,line width=0.15pt},
forget plot
]
coordinates{
 (31.25,30.049999) 
};
\addplot [
color=blue,
solid,
mark=*,mark size=1.3pt,
mark options={solid,fill=mycolor31,draw=black,line width=0.15pt},
forget plot
]
coordinates{
 (-81.370598,28.5445) 
};
\addplot [
color=blue,
solid,
mark=*,mark size=1.3pt,
mark options={solid,fill=mycolor30,draw=black,line width=0.15pt},
forget plot
]
coordinates{
 (30.84,29.3078) 
};
\addplot [
color=blue,
solid,
mark=*,mark size=1.3pt,
mark options={solid,fill=mycolor70,draw=black,line width=0.15pt},
forget plot
]
coordinates{
 (31.25,30.049999) 
};
\addplot [
color=blue,
solid,
mark=*,mark size=1.3pt,
mark options={solid,fill=mycolor44,draw=black,line width=0.15pt},
forget plot
]
coordinates{
 (30,27) 
};
\addplot [
color=blue,
solid,
mark=*,mark size=1.3pt,
mark options={solid,fill=mycolor31,draw=black,line width=0.15pt},
forget plot
]
coordinates{
 (31.376699,31.0431) 
};
\addplot [
color=blue,
solid,
mark=*,mark size=1.3pt,
mark options={solid,fill=mycolor44,draw=black,line width=0.15pt},
forget plot
]
coordinates{
 (31.25,30.049999) 
};
\addplot [
color=blue,
solid,
mark=*,mark size=1.3pt,
mark options={solid,fill=mycolor28,draw=black,line width=0.15pt},
forget plot
]
coordinates{
 (30,27) 
};
\addplot [
color=blue,
solid,
mark=*,mark size=1.3pt,
mark options={solid,fill=mycolor31,draw=black,line width=0.15pt},
forget plot
]
coordinates{
 (-2,8) 
};
\addplot [
color=blue,
solid,
mark=*,mark size=1.3pt,
mark options={solid,fill=mycolor81,draw=black,line width=0.15pt},
forget plot
]
coordinates{
 (57.498901,-20.1619) 
};
\addplot [
color=blue,
solid,
mark=*,mark size=1.3pt,
mark options={solid,fill=mycolor29,draw=black,line width=0.15pt},
forget plot
]
coordinates{
 (-118.468201,34.011902) 
};
\addplot [
color=blue,
solid,
mark=*,mark size=1.3pt,
mark options={solid,fill=mycolor76,draw=black,line width=0.15pt},
forget plot
]
coordinates{
 (-96.835297,32.929901) 
};
\addplot [
color=blue,
solid,
mark=*,mark size=1.3pt,
mark options={solid,fill=mycolor28,draw=black,line width=0.15pt},
forget plot
]
coordinates{
 (-93.605797,41.729698) 
};
\addplot [
color=blue,
solid,
mark=*,mark size=1.3pt,
mark options={solid,fill=mycolor44,draw=black,line width=0.15pt},
forget plot
]
coordinates{
 (-113.994003,46.872101) 
};
\addplot [
color=blue,
solid,
mark=*,mark size=1.3pt,
mark options={solid,fill=mycolor52,draw=black,line width=0.15pt},
forget plot
]
coordinates{
 (-84.636002,42.7257) 
};
\addplot [
color=blue,
solid,
mark=*,mark size=1.3pt,
mark options={solid,fill=mycolor19,draw=black,line width=0.15pt},
forget plot
]
coordinates{
 (9,51) 
};
\addplot [
color=blue,
solid,
mark=*,mark size=1.3pt,
mark options={solid,fill=mycolor23,draw=black,line width=0.15pt},
forget plot
]
coordinates{
 (-66.5,18.25) 
};
\addplot [
color=blue,
solid,
mark=*,mark size=1.3pt,
mark options={solid,fill=mycolor69,draw=black,line width=0.15pt},
forget plot
]
coordinates{
 (-96.835297,32.929901) 
};
\addplot [
color=blue,
solid,
mark=*,mark size=1.3pt,
mark options={solid,fill=mycolor80,draw=black,line width=0.15pt},
forget plot
]
coordinates{
 (-77.050003,-12.05) 
};
\addplot [
color=blue,
solid,
mark=*,mark size=1.3pt,
mark options={solid,fill=mycolor54,draw=black,line width=0.15pt},
forget plot
]
coordinates{
 (-77.050003,-12.05) 
};
\addplot [
color=blue,
solid,
mark=*,mark size=1.3pt,
mark options={solid,fill=mycolor80,draw=black,line width=0.15pt},
forget plot
]
coordinates{
 (-77.050003,-12.05) 
};
\addplot [
color=blue,
solid,
mark=*,mark size=1.3pt,
mark options={solid,fill=mycolor80,draw=black,line width=0.15pt},
forget plot
]
coordinates{
 (-77.050003,-12.05) 
};
\addplot [
color=blue,
solid,
mark=*,mark size=1.3pt,
mark options={solid,fill=mycolor74,draw=black,line width=0.15pt},
forget plot
]
coordinates{
 (-94.573502,39.147202) 
};
\addplot [
color=blue,
solid,
mark=*,mark size=1.3pt,
mark options={solid,fill=mycolor64,draw=black,line width=0.15pt},
forget plot
]
coordinates{
 (-98.398697,29.488899) 
};
\addplot [
color=blue,
solid,
mark=*,mark size=1.3pt,
mark options={solid,fill=mycolor76,draw=black,line width=0.15pt},
forget plot
]
coordinates{
 (-95.366997,29.7523) 
};
\addplot [
color=blue,
solid,
mark=*,mark size=1.3pt,
mark options={solid,fill=mycolor82,draw=black,line width=0.15pt},
forget plot
]
coordinates{
 (-77.050003,-12.05) 
};
\addplot [
color=blue,
solid,
mark=*,mark size=1.3pt,
mark options={solid,fill=mycolor3,draw=black,line width=0.15pt},
forget plot
]
coordinates{
 (-77.050003,-12.05) 
};
\addplot [
color=blue,
solid,
mark=*,mark size=1.3pt,
mark options={solid,fill=mycolor76,draw=black,line width=0.15pt},
forget plot
]
coordinates{
 (-95.913803,29.701799) 
};
\addplot [
color=blue,
solid,
mark=*,mark size=1.3pt,
mark options={solid,fill=mycolor16,draw=black,line width=0.15pt},
forget plot
]
coordinates{
 (-70.248299,-18.0056) 
};
\addplot [
color=blue,
solid,
mark=*,mark size=1.3pt,
mark options={solid,fill=mycolor52,draw=black,line width=0.15pt},
forget plot
]
coordinates{
 (-96.835297,32.929901) 
};
\addplot [
color=blue,
solid,
mark=*,mark size=1.3pt,
mark options={solid,fill=mycolor17,draw=black,line width=0.15pt},
forget plot
]
coordinates{
 (-79.550003,9.0167) 
};
\addplot [
color=blue,
solid,
mark=*,mark size=1.3pt,
mark options={solid,fill=mycolor42,draw=black,line width=0.15pt},
forget plot
]
coordinates{
 (-80,9) 
};
\addplot [
color=blue,
solid,
mark=*,mark size=1.3pt,
mark options={solid,fill=mycolor52,draw=black,line width=0.15pt},
forget plot
]
coordinates{
 (-82.737602,27.720699) 
};
\addplot [
color=blue,
solid,
mark=*,mark size=1.3pt,
mark options={solid,fill=mycolor52,draw=black,line width=0.15pt},
forget plot
]
coordinates{
 (-96.835297,32.929901) 
};
\addplot [
color=blue,
solid,
mark=*,mark size=1.3pt,
mark options={solid,fill=mycolor7,draw=black,line width=0.15pt},
forget plot
]
coordinates{
 (-99.138603,19.4342) 
};
\addplot [
color=blue,
solid,
mark=*,mark size=1.3pt,
mark options={solid,fill=mycolor54,draw=black,line width=0.15pt},
forget plot
]
coordinates{
 (-99.138603,19.4342) 
};
\addplot [
color=blue,
solid,
mark=*,mark size=1.3pt,
mark options={solid,fill=mycolor76,draw=black,line width=0.15pt},
forget plot
]
coordinates{
 (-96.835297,32.929901) 
};
\addplot [
color=blue,
solid,
mark=*,mark size=1.3pt,
mark options={solid,fill=mycolor80,draw=black,line width=0.15pt},
forget plot
]
coordinates{
 (-110.966698,29.0667) 
};
\addplot [
color=blue,
solid,
mark=*,mark size=1.3pt,
mark options={solid,fill=mycolor3,draw=black,line width=0.15pt},
forget plot
]
coordinates{
 (-102.266701,19.983299) 
};
\addplot [
color=blue,
solid,
mark=*,mark size=1.3pt,
mark options={solid,fill=mycolor80,draw=black,line width=0.15pt},
forget plot
]
coordinates{
 (-102,23) 
};
\addplot [
color=blue,
solid,
mark=*,mark size=1.3pt,
mark options={solid,fill=mycolor32,draw=black,line width=0.15pt},
forget plot
]
coordinates{
 (-99.138603,19.4342) 
};
\addplot [
color=blue,
solid,
mark=*,mark size=1.3pt,
mark options={solid,fill=mycolor17,draw=black,line width=0.15pt},
forget plot
]
coordinates{
 (-117.016701,32.533298) 
};
\addplot [
color=blue,
solid,
mark=*,mark size=1.3pt,
mark options={solid,fill=mycolor23,draw=black,line width=0.15pt},
forget plot
]
coordinates{
 (-99.138603,19.4342) 
};
\addplot [
color=blue,
solid,
mark=*,mark size=1.3pt,
mark options={solid,fill=mycolor42,draw=black,line width=0.15pt},
forget plot
]
coordinates{
 (-122.949997,49.25) 
};
\addplot [
color=blue,
solid,
mark=*,mark size=1.3pt,
mark options={solid,fill=mycolor68,draw=black,line width=0.15pt},
forget plot
]
coordinates{
 (-99.138603,19.4342) 
};
\addplot [
color=blue,
solid,
mark=*,mark size=1.3pt,
mark options={solid,fill=mycolor80,draw=black,line width=0.15pt},
forget plot
]
coordinates{
 (-116.616699,31.866699) 
};
\addplot [
color=blue,
solid,
mark=*,mark size=1.3pt,
mark options={solid,fill=mycolor28,draw=black,line width=0.15pt},
forget plot
]
coordinates{
 (-79.666702,43.4333) 
};
\addplot [
color=blue,
solid,
mark=*,mark size=1.3pt,
mark options={solid,fill=mycolor42,draw=black,line width=0.15pt},
forget plot
]
coordinates{
 (-103.333298,20.6667) 
};
\addplot [
color=blue,
solid,
mark=*,mark size=1.3pt,
mark options={solid,fill=mycolor19,draw=black,line width=0.15pt},
forget plot
]
coordinates{
 (9,51) 
};
\addplot [
color=blue,
solid,
mark=*,mark size=1.3pt,
mark options={solid,fill=mycolor42,draw=black,line width=0.15pt},
forget plot
]
coordinates{
 (-99.138603,19.4342) 
};
\addplot [
color=blue,
solid,
mark=*,mark size=1.3pt,
mark options={solid,fill=mycolor76,draw=black,line width=0.15pt},
forget plot
]
coordinates{
 (-122.007401,37.4249) 
};
\addplot [
color=blue,
solid,
mark=*,mark size=1.3pt,
mark options={solid,fill=mycolor70,draw=black,line width=0.15pt},
forget plot
]
coordinates{
 (-87.216698,14.1) 
};
\addplot [
color=blue,
solid,
mark=*,mark size=1.3pt,
mark options={solid,fill=mycolor31,draw=black,line width=0.15pt},
forget plot
]
coordinates{
 (-75.408302,40.054798) 
};
\addplot [
color=blue,
solid,
mark=*,mark size=1.3pt,
mark options={solid,fill=mycolor29,draw=black,line width=0.15pt},
forget plot
]
coordinates{
 (-87.216698,14.0833) 
};
\addplot [
color=blue,
solid,
mark=*,mark size=1.3pt,
mark options={solid,fill=mycolor52,draw=black,line width=0.15pt},
forget plot
]
coordinates{
 (105,46) 
};
\addplot [
color=blue,
solid,
mark=*,mark size=1.3pt,
mark options={solid,fill=mycolor5,draw=black,line width=0.15pt},
forget plot
]
coordinates{
 (106.916702,47.916698) 
};
\addplot [
color=blue,
solid,
mark=*,mark size=1.3pt,
mark options={solid,fill=mycolor72,draw=black,line width=0.15pt},
forget plot
]
coordinates{
 (106.916702,47.916698) 
};
\addplot [
color=blue,
solid,
mark=*,mark size=1.3pt,
mark options={solid,fill=mycolor11,draw=black,line width=0.15pt},
forget plot
]
coordinates{
 (106.916702,47.916698) 
};
\addplot [
color=blue,
solid,
mark=*,mark size=1.3pt,
mark options={solid,fill=mycolor75,draw=black,line width=0.15pt},
forget plot
]
coordinates{
 (-71.102798,42.364601) 
};
\addplot [
color=blue,
solid,
mark=*,mark size=1.3pt,
mark options={solid,fill=mycolor43,draw=black,line width=0.15pt},
forget plot
]
coordinates{
 (-122.007401,37.4249) 
};
\addplot [
color=blue,
solid,
mark=*,mark size=1.3pt,
mark options={solid,fill=mycolor42,draw=black,line width=0.15pt},
forget plot
]
coordinates{
 (-118.468201,34.011902) 
};
\addplot [
color=blue,
solid,
mark=*,mark size=1.3pt,
mark options={solid,fill=mycolor42,draw=black,line width=0.15pt},
forget plot
]
coordinates{
 (2.4333,6.35) 
};
\addplot [
color=blue,
solid,
mark=*,mark size=1.3pt,
mark options={solid,fill=mycolor28,draw=black,line width=0.15pt},
forget plot
]
coordinates{
 (-71.259003,42.403) 
};
\addplot [
color=blue,
solid,
mark=*,mark size=1.3pt,
mark options={solid,fill=mycolor25,draw=black,line width=0.15pt},
forget plot
]
coordinates{
 (1.4332,43.599499) 
};
\addplot [
color=blue,
solid,
mark=*,mark size=1.3pt,
mark options={solid,fill=mycolor76,draw=black,line width=0.15pt},
forget plot
]
coordinates{
 (-86.144997,39.822701) 
};
\addplot [
color=blue,
solid,
mark=*,mark size=1.3pt,
mark options={solid,fill=mycolor76,draw=black,line width=0.15pt},
forget plot
]
coordinates{
 (-122.461899,37.705799) 
};
\addplot [
color=blue,
solid,
mark=*,mark size=1.3pt,
mark options={solid,fill=mycolor52,draw=black,line width=0.15pt},
forget plot
]
coordinates{
 (12,6) 
};
\addplot [
color=blue,
solid,
mark=*,mark size=1.3pt,
mark options={solid,fill=mycolor76,draw=black,line width=0.15pt},
forget plot
]
coordinates{
 (11.5167,3.8667) 
};
\addplot [
color=blue,
solid,
mark=*,mark size=1.3pt,
mark options={solid,fill=mycolor27,draw=black,line width=0.15pt},
forget plot
]
coordinates{
 (2.3333,48.866699) 
};
\addplot [
color=blue,
solid,
mark=*,mark size=1.3pt,
mark options={solid,fill=mycolor52,draw=black,line width=0.15pt},
forget plot
]
coordinates{
 (-95.9627,36.097099) 
};
\addplot [
color=blue,
solid,
mark=*,mark size=1.3pt,
mark options={solid,fill=mycolor19,draw=black,line width=0.15pt},
forget plot
]
coordinates{
 (9,51) 
};
\addplot [
color=blue,
solid,
mark=*,mark size=1.3pt,
mark options={solid,fill=mycolor25,draw=black,line width=0.15pt},
forget plot
]
coordinates{
 (2,46) 
};
\addplot [
color=blue,
solid,
mark=*,mark size=1.3pt,
mark options={solid,fill=mycolor29,draw=black,line width=0.15pt},
forget plot
]
coordinates{
 (-118.3965,34.021099) 
};
\addplot [
color=blue,
solid,
mark=*,mark size=1.3pt,
mark options={solid,fill=mycolor77,draw=black,line width=0.15pt},
forget plot
]
coordinates{
 (-122.007401,37.4249) 
};
\addplot [
color=blue,
solid,
mark=*,mark size=1.3pt,
mark options={solid,fill=mycolor25,draw=black,line width=0.15pt},
forget plot
]
coordinates{
 (10.75,59.916698) 
};
\addplot [
color=blue,
solid,
mark=*,mark size=1.3pt,
mark options={solid,fill=mycolor29,draw=black,line width=0.15pt},
forget plot
]
coordinates{
 (-104.873802,39.623699) 
};
\addplot [
color=blue,
solid,
mark=*,mark size=1.3pt,
mark options={solid,fill=mycolor76,draw=black,line width=0.15pt},
forget plot
]
coordinates{
 (-95.366997,29.7523) 
};
\addplot [
color=blue,
solid,
mark=*,mark size=1.3pt,
mark options={solid,fill=mycolor76,draw=black,line width=0.15pt},
forget plot
]
coordinates{
 (-95.366997,29.7523) 
};
\addplot [
color=blue,
solid,
mark=*,mark size=1.3pt,
mark options={solid,fill=mycolor76,draw=black,line width=0.15pt},
forget plot
]
coordinates{
 (-95.366997,29.7523) 
};
\addplot [
color=blue,
solid,
mark=*,mark size=1.3pt,
mark options={solid,fill=mycolor30,draw=black,line width=0.15pt},
forget plot
]
coordinates{
 (-2,8) 
};
\addplot [
color=blue,
solid,
mark=*,mark size=1.3pt,
mark options={solid,fill=mycolor63,draw=black,line width=0.15pt},
forget plot
]
coordinates{
 (-112.082397,33.509102) 
};
\addplot [
color=blue,
solid,
mark=*,mark size=1.3pt,
mark options={solid,fill=mycolor65,draw=black,line width=0.15pt},
forget plot
]
coordinates{
 (-122.113403,47.605099) 
};
\addplot [
color=blue,
solid,
mark=*,mark size=1.3pt,
mark options={solid,fill=mycolor31,draw=black,line width=0.15pt},
forget plot
]
coordinates{
 (-83.138298,39.964901) 
};
\addplot [
color=blue,
solid,
mark=*,mark size=1.3pt,
mark options={solid,fill=mycolor75,draw=black,line width=0.15pt},
forget plot
]
coordinates{
 (-2,8) 
};
\addplot [
color=blue,
solid,
mark=*,mark size=1.3pt,
mark options={solid,fill=mycolor77,draw=black,line width=0.15pt},
forget plot
]
coordinates{
 (-117.791199,33.850201) 
};
\addplot [
color=blue,
solid,
mark=*,mark size=1.3pt,
mark options={solid,fill=mycolor76,draw=black,line width=0.15pt},
forget plot
]
coordinates{
 (-95.366997,29.7523) 
};
\addplot [
color=blue,
solid,
mark=*,mark size=1.3pt,
mark options={solid,fill=mycolor42,draw=black,line width=0.15pt},
forget plot
]
coordinates{
 (-112.088898,33.7248) 
};
\addplot [
color=blue,
solid,
mark=*,mark size=1.3pt,
mark options={solid,fill=mycolor17,draw=black,line width=0.15pt},
forget plot
]
coordinates{
 (-87.650002,41.849998) 
};
\addplot [
color=blue,
solid,
mark=*,mark size=1.3pt,
mark options={solid,fill=mycolor30,draw=black,line width=0.15pt},
forget plot
]
coordinates{
 (-95.913803,29.701799) 
};
\addplot [
color=blue,
solid,
mark=*,mark size=1.3pt,
mark options={solid,fill=mycolor68,draw=black,line width=0.15pt},
forget plot
]
coordinates{
 (36.8167,-1.2833) 
};
\addplot [
color=blue,
solid,
mark=*,mark size=1.3pt,
mark options={solid,fill=mycolor44,draw=black,line width=0.15pt},
forget plot
]
coordinates{
 (-77.162102,38.935799) 
};
\addplot [
color=blue,
solid,
mark=*,mark size=1.3pt,
mark options={solid,fill=mycolor17,draw=black,line width=0.15pt},
forget plot
]
coordinates{
 (36.8167,-1.2833) 
};
\addplot [
color=blue,
solid,
mark=*,mark size=1.3pt,
mark options={solid,fill=mycolor30,draw=black,line width=0.15pt},
forget plot
]
coordinates{
 (-87.650002,41.849998) 
};
\addplot [
color=blue,
solid,
mark=*,mark size=1.3pt,
mark options={solid,fill=mycolor69,draw=black,line width=0.15pt},
forget plot
]
coordinates{
 (-84.430901,33.7257) 
};
\addplot [
color=blue,
solid,
mark=*,mark size=1.3pt,
mark options={solid,fill=mycolor69,draw=black,line width=0.15pt},
forget plot
]
coordinates{
 (-122.007401,37.4249) 
};
\addplot [
color=blue,
solid,
mark=*,mark size=1.3pt,
mark options={solid,fill=mycolor4,draw=black,line width=0.15pt},
forget plot
]
coordinates{
 (-106.666702,52.133301) 
};
\addplot [
color=blue,
solid,
mark=*,mark size=1.3pt,
mark options={solid,fill=mycolor23,draw=black,line width=0.15pt},
forget plot
]
coordinates{
 (47,-20) 
};
\addplot [
color=blue,
solid,
mark=*,mark size=1.3pt,
mark options={solid,fill=mycolor18,draw=black,line width=0.15pt},
forget plot
]
coordinates{
 (15,62) 
};
\addplot [
color=blue,
solid,
mark=*,mark size=1.3pt,
mark options={solid,fill=mycolor68,draw=black,line width=0.15pt},
forget plot
]
coordinates{
 (34,-13.5) 
};
\addplot [
color=blue,
solid,
mark=*,mark size=1.3pt,
mark options={solid,fill=mycolor79,draw=black,line width=0.15pt},
forget plot
]
coordinates{
 (34,-13.5) 
};
\addplot [
color=blue,
solid,
mark=*,mark size=1.3pt,
mark options={solid,fill=mycolor27,draw=black,line width=0.15pt},
forget plot
]
coordinates{
 (-98.398697,29.488899) 
};
\addplot [
color=blue,
solid,
mark=*,mark size=1.3pt,
mark options={solid,fill=mycolor79,draw=black,line width=0.15pt},
forget plot
]
coordinates{
 (32.589199,-25.9653) 
};
\addplot [
color=blue,
solid,
mark=*,mark size=1.3pt,
mark options={solid,fill=mycolor58,draw=black,line width=0.15pt},
forget plot
]
coordinates{
 (32.589199,-25.9653) 
};
\addplot [
color=blue,
solid,
mark=*,mark size=1.3pt,
mark options={solid,fill=mycolor81,draw=black,line width=0.15pt},
forget plot
]
coordinates{
 (17.083599,-22.57) 
};
\addplot [
color=blue,
solid,
mark=*,mark size=1.3pt,
mark options={solid,fill=mycolor14,draw=black,line width=0.15pt},
forget plot
]
coordinates{
 (9,51) 
};
\addplot [
color=blue,
solid,
mark=*,mark size=1.3pt,
mark options={solid,fill=mycolor30,draw=black,line width=0.15pt},
forget plot
]
coordinates{
 (-84.769096,42.751202) 
};
\addplot [
color=blue,
solid,
mark=*,mark size=1.3pt,
mark options={solid,fill=mycolor28,draw=black,line width=0.15pt},
forget plot
]
coordinates{
 (-87.644096,41.8825) 
};
\addplot [
color=blue,
solid,
mark=*,mark size=1.3pt,
mark options={solid,fill=mycolor76,draw=black,line width=0.15pt},
forget plot
]
coordinates{
 (12.4833,9.2) 
};
\addplot [
color=blue,
solid,
mark=*,mark size=1.3pt,
mark options={solid,fill=mycolor15,draw=black,line width=0.15pt},
forget plot
]
coordinates{
 (-0.0931,51.514198) 
};
\addplot [
color=blue,
solid,
mark=*,mark size=1.3pt,
mark options={solid,fill=mycolor42,draw=black,line width=0.15pt},
forget plot
]
coordinates{
 (-117.861198,33.926899) 
};
\addplot [
color=blue,
solid,
mark=*,mark size=1.3pt,
mark options={solid,fill=mycolor74,draw=black,line width=0.15pt},
forget plot
]
coordinates{
 (-80.5,19.5) 
};
\addplot [
color=blue,
solid,
mark=*,mark size=1.3pt,
mark options={solid,fill=mycolor76,draw=black,line width=0.15pt},
forget plot
]
coordinates{
 (-95.366997,29.7523) 
};
\addplot [
color=blue,
solid,
mark=*,mark size=1.3pt,
mark options={solid,fill=mycolor75,draw=black,line width=0.15pt},
forget plot
]
coordinates{
 (-83.230698,42.465) 
};
\addplot [
color=blue,
solid,
mark=*,mark size=1.3pt,
mark options={solid,fill=mycolor69,draw=black,line width=0.15pt},
forget plot
]
coordinates{
 (-97,38) 
};
\addplot [
color=blue,
solid,
mark=*,mark size=1.3pt,
mark options={solid,fill=mycolor52,draw=black,line width=0.15pt},
forget plot
]
coordinates{
 (-95.366997,29.7523) 
};
\addplot [
color=blue,
solid,
mark=*,mark size=1.3pt,
mark options={solid,fill=mycolor52,draw=black,line width=0.15pt},
forget plot
]
coordinates{
 (-95.4739,29.830099) 
};
\addplot [
color=blue,
solid,
mark=*,mark size=1.3pt,
mark options={solid,fill=mycolor15,draw=black,line width=0.15pt},
forget plot
]
coordinates{
 (-2,54) 
};
\addplot [
color=blue,
solid,
mark=*,mark size=1.3pt,
mark options={solid,fill=mycolor76,draw=black,line width=0.15pt},
forget plot
]
coordinates{
 (-95.4739,29.830099) 
};
\addplot [
color=blue,
solid,
mark=*,mark size=1.3pt,
mark options={solid,fill=mycolor31,draw=black,line width=0.15pt},
forget plot
]
coordinates{
 (-95.4739,29.830099) 
};
\addplot [
color=blue,
solid,
mark=*,mark size=1.3pt,
mark options={solid,fill=mycolor76,draw=black,line width=0.15pt},
forget plot
]
coordinates{
 (-77.0261,38.9907) 
};
\addplot [
color=blue,
solid,
mark=*,mark size=1.3pt,
mark options={solid,fill=mycolor70,draw=black,line width=0.15pt},
forget plot
]
coordinates{
 (-97.737297,30.176001) 
};
\addplot [
color=blue,
solid,
mark=*,mark size=1.3pt,
mark options={solid,fill=mycolor27,draw=black,line width=0.15pt},
forget plot
]
coordinates{
 (-84.213501,33.9412) 
};
\addplot [
color=blue,
solid,
mark=*,mark size=1.3pt,
mark options={solid,fill=mycolor31,draw=black,line width=0.15pt},
forget plot
]
coordinates{
 (-84.636002,42.7257) 
};
\addplot [
color=blue,
solid,
mark=*,mark size=1.3pt,
mark options={solid,fill=mycolor73,draw=black,line width=0.15pt},
forget plot
]
coordinates{
 (-91.544701,41.664001) 
};
\addplot [
color=blue,
solid,
mark=*,mark size=1.3pt,
mark options={solid,fill=mycolor6,draw=black,line width=0.15pt},
forget plot
]
coordinates{
 (135.466705,34.583302) 
};
\addplot [
color=blue,
solid,
mark=*,mark size=1.3pt,
mark options={solid,fill=mycolor29,draw=black,line width=0.15pt},
forget plot
]
coordinates{
 (-112.071198,33.449902) 
};
\addplot [
color=blue,
solid,
mark=*,mark size=1.3pt,
mark options={solid,fill=mycolor54,draw=black,line width=0.15pt},
forget plot
]
coordinates{
 (30.0606,-1.9536) 
};
\addplot [
color=blue,
solid,
mark=*,mark size=1.3pt,
mark options={solid,fill=mycolor42,draw=black,line width=0.15pt},
forget plot
]
coordinates{
 (30,-2) 
};
\addplot [
color=blue,
solid,
mark=*,mark size=1.3pt,
mark options={solid,fill=mycolor73,draw=black,line width=0.15pt},
forget plot
]
coordinates{
 (-2,54) 
};
\addplot [
color=blue,
solid,
mark=*,mark size=1.3pt,
mark options={solid,fill=mycolor83,draw=black,line width=0.15pt},
forget plot
]
coordinates{
 (-17.438101,14.6708) 
};
\addplot [
color=blue,
solid,
mark=*,mark size=1.3pt,
mark options={solid,fill=mycolor74,draw=black,line width=0.15pt},
forget plot
]
coordinates{
 (-16.4953,16.0189) 
};
\addplot [
color=blue,
solid,
mark=*,mark size=1.3pt,
mark options={solid,fill=mycolor78,draw=black,line width=0.15pt},
forget plot
]
coordinates{
 (-111.843597,40.498199) 
};
\addplot [
color=blue,
solid,
mark=*,mark size=1.3pt,
mark options={solid,fill=mycolor7,draw=black,line width=0.15pt},
forget plot
]
coordinates{
 (102.280502,2.3125) 
};
\addplot [
color=blue,
solid,
mark=*,mark size=1.3pt,
mark options={solid,fill=mycolor28,draw=black,line width=0.15pt},
forget plot
]
coordinates{
 (-80.171097,26.1882) 
};
\addplot [
color=blue,
solid,
mark=*,mark size=1.3pt,
mark options={solid,fill=mycolor64,draw=black,line width=0.15pt},
forget plot
]
coordinates{
 (-118.264198,34.053001) 
};
\addplot [
color=blue,
solid,
mark=*,mark size=1.3pt,
mark options={solid,fill=mycolor29,draw=black,line width=0.15pt},
forget plot
]
coordinates{
 (24,-29) 
};
\addplot [
color=blue,
solid,
mark=*,mark size=1.3pt,
mark options={solid,fill=mycolor75,draw=black,line width=0.15pt},
forget plot
]
coordinates{
 (-84.488197,42.728298) 
};
\addplot [
color=blue,
solid,
mark=*,mark size=1.3pt,
mark options={solid,fill=mycolor79,draw=black,line width=0.15pt},
forget plot
]
coordinates{
 (24,-29) 
};
\addplot [
color=blue,
solid,
mark=*,mark size=1.3pt,
mark options={solid,fill=mycolor44,draw=black,line width=0.15pt},
forget plot
]
coordinates{
 (-122.007401,37.4249) 
};
\addplot [
color=blue,
solid,
mark=*,mark size=1.3pt,
mark options={solid,fill=mycolor79,draw=black,line width=0.15pt},
forget plot
]
coordinates{
 (27.1,-26.7167) 
};
\addplot [
color=blue,
solid,
mark=*,mark size=1.3pt,
mark options={solid,fill=mycolor23,draw=black,line width=0.15pt},
forget plot
]
coordinates{
 (18.6285,-33.9002) 
};
\addplot [
color=blue,
solid,
mark=*,mark size=1.3pt,
mark options={solid,fill=mycolor16,draw=black,line width=0.15pt},
forget plot
]
coordinates{
 (28.049801,-26.2052) 
};
\addplot [
color=blue,
solid,
mark=*,mark size=1.3pt,
mark options={solid,fill=mycolor68,draw=black,line width=0.15pt},
forget plot
]
coordinates{
 (24,-29) 
};
\addplot [
color=blue,
solid,
mark=*,mark size=1.3pt,
mark options={solid,fill=mycolor43,draw=black,line width=0.15pt},
forget plot
]
coordinates{
 (-2,54) 
};
\addplot [
color=blue,
solid,
mark=*,mark size=1.3pt,
mark options={solid,fill=mycolor28,draw=black,line width=0.15pt},
forget plot
]
coordinates{
 (-71.204697,42.5051) 
};
\addplot [
color=blue,
solid,
mark=*,mark size=1.3pt,
mark options={solid,fill=mycolor43,draw=black,line width=0.15pt},
forget plot
]
coordinates{
 (-2,54) 
};
\addplot [
color=blue,
solid,
mark=*,mark size=1.3pt,
mark options={solid,fill=mycolor16,draw=black,line width=0.15pt},
forget plot
]
coordinates{
 (31.366699,-26.483299) 
};
\addplot [
color=blue,
solid,
mark=*,mark size=1.3pt,
mark options={solid,fill=mycolor77,draw=black,line width=0.15pt},
forget plot
]
coordinates{
 (35,-6) 
};
\addplot [
color=blue,
solid,
mark=*,mark size=1.3pt,
mark options={solid,fill=mycolor74,draw=black,line width=0.15pt},
forget plot
]
coordinates{
 (-95.366997,29.7523) 
};
\addplot [
color=blue,
solid,
mark=*,mark size=1.3pt,
mark options={solid,fill=mycolor19,draw=black,line width=0.15pt},
forget plot
]
coordinates{
 (13.4,52.516701) 
};
\addplot [
color=blue,
solid,
mark=*,mark size=1.3pt,
mark options={solid,fill=mycolor29,draw=black,line width=0.15pt},
forget plot
]
coordinates{
 (-111.613297,40.218102) 
};
\addplot [
color=blue,
solid,
mark=*,mark size=1.3pt,
mark options={solid,fill=mycolor44,draw=black,line width=0.15pt},
forget plot
]
coordinates{
 (-104.8582,39.569) 
};
\addplot [
color=blue,
solid,
mark=*,mark size=1.3pt,
mark options={solid,fill=mycolor75,draw=black,line width=0.15pt},
forget plot
]
coordinates{
 (-71.204697,42.5051) 
};
\addplot [
color=blue,
solid,
mark=*,mark size=1.3pt,
mark options={solid,fill=mycolor69,draw=black,line width=0.15pt},
forget plot
]
coordinates{
 (-118.468201,34.011902) 
};
\addplot [
color=blue,
solid,
mark=*,mark size=1.3pt,
mark options={solid,fill=mycolor69,draw=black,line width=0.15pt},
forget plot
]
coordinates{
 (-97,38) 
};
\addplot [
color=blue,
solid,
mark=*,mark size=1.3pt,
mark options={solid,fill=mycolor15,draw=black,line width=0.15pt},
forget plot
]
coordinates{
 (103.855797,1.2931) 
};
\addplot [
color=blue,
solid,
mark=*,mark size=1.3pt,
mark options={solid,fill=mycolor16,draw=black,line width=0.15pt},
forget plot
]
coordinates{
 (-90.1922,38.631199) 
};
\addplot [
color=blue,
solid,
mark=*,mark size=1.3pt,
mark options={solid,fill=mycolor75,draw=black,line width=0.15pt},
forget plot
]
coordinates{
 (-75.408302,40.054798) 
};
\addplot [
color=blue,
solid,
mark=*,mark size=1.3pt,
mark options={solid,fill=mycolor23,draw=black,line width=0.15pt},
forget plot
]
coordinates{
 (30,-15) 
};
\addplot [
color=blue,
solid,
mark=*,mark size=1.3pt,
mark options={solid,fill=mycolor32,draw=black,line width=0.15pt},
forget plot
]
coordinates{
 (31.044701,-17.817801) 
};
\addplot [
color=blue,
solid,
mark=*,mark size=1.3pt,
mark options={solid,fill=mycolor63,draw=black,line width=0.15pt},
forget plot
]
coordinates{
 (-96.700302,43.549999) 
};
\addplot [
color=blue,
solid,
mark=*,mark size=1.3pt,
mark options={solid,fill=mycolor59,draw=black,line width=0.15pt},
forget plot
]
coordinates{
 (31.25,30.049999) 
};
\addplot [
color=blue,
solid,
mark=*,mark size=1.3pt,
mark options={solid,fill=mycolor31,draw=black,line width=0.15pt},
forget plot
]
coordinates{
 (-75.408302,40.054798) 
};
\addplot [
color=blue,
solid,
mark=*,mark size=1.3pt,
mark options={solid,fill=mycolor79,draw=black,line width=0.15pt},
forget plot
]
coordinates{
 (-73.583298,45.5) 
};
\addplot [
color=blue,
solid,
mark=*,mark size=1.3pt,
mark options={solid,fill=mycolor75,draw=black,line width=0.15pt},
forget plot
]
coordinates{
 (-71.102798,42.364601) 
};
\addplot [
color=blue,
solid,
mark=*,mark size=1.3pt,
mark options={solid,fill=mycolor79,draw=black,line width=0.15pt},
forget plot
]
coordinates{
 (28.049801,-26.2052) 
};
\addplot [
color=blue,
solid,
mark=*,mark size=1.3pt,
mark options={solid,fill=mycolor54,draw=black,line width=0.15pt},
forget plot
]
coordinates{
 (27.964701,-30.795401) 
};
\addplot [
color=blue,
solid,
mark=*,mark size=1.3pt,
mark options={solid,fill=mycolor19,draw=black,line width=0.15pt},
forget plot
]
coordinates{
 (9,51) 
};
\addplot [
color=blue,
solid,
mark=*,mark size=1.3pt,
mark options={solid,fill=mycolor15,draw=black,line width=0.15pt},
forget plot
]
coordinates{
 (8,47) 
};
\addplot [
color=blue,
solid,
mark=*,mark size=1.3pt,
mark options={solid,fill=mycolor43,draw=black,line width=0.15pt},
forget plot
]
coordinates{
 (-2,54) 
};
\addplot [
color=blue,
solid,
mark=*,mark size=1.3pt,
mark options={solid,fill=mycolor66,draw=black,line width=0.15pt},
forget plot
]
coordinates{
 (-1.1667,52.966702) 
};
\addplot [
color=blue,
solid,
mark=*,mark size=1.3pt,
mark options={solid,fill=mycolor1,draw=black,line width=0.15pt},
forget plot
]
coordinates{
 (-0.1833,51.8167) 
};
\addplot [
color=blue,
solid,
mark=*,mark size=1.3pt,
mark options={solid,fill=mycolor25,draw=black,line width=0.15pt},
forget plot
]
coordinates{
 (-2,54) 
};
\addplot [
color=blue,
solid,
mark=*,mark size=1.3pt,
mark options={solid,fill=mycolor14,draw=black,line width=0.15pt},
forget plot
]
coordinates{
 (-1.9167,52.1833) 
};
\addplot [
color=blue,
solid,
mark=*,mark size=1.3pt,
mark options={solid,fill=mycolor46,draw=black,line width=0.15pt},
forget plot
]
coordinates{
 (24.1,56.950001) 
};
\addplot [
color=blue,
solid,
mark=*,mark size=1.3pt,
mark options={solid,fill=mycolor14,draw=black,line width=0.15pt},
forget plot
]
coordinates{
 (4,50.833302) 
};
\addplot [
color=blue,
solid,
mark=*,mark size=1.3pt,
mark options={solid,fill=mycolor52,draw=black,line width=0.15pt},
forget plot
]
coordinates{
 (-95.366997,29.7523) 
};
\addplot [
color=blue,
solid,
mark=*,mark size=1.3pt,
mark options={solid,fill=mycolor26,draw=black,line width=0.15pt},
forget plot
]
coordinates{
 (7.7874,48.600399) 
};
\addplot [
color=blue,
solid,
mark=*,mark size=1.3pt,
mark options={solid,fill=mycolor25,draw=black,line width=0.15pt},
forget plot
]
coordinates{
 (10.75,59.916698) 
};
\addplot [
color=blue,
solid,
mark=*,mark size=1.3pt,
mark options={solid,fill=mycolor64,draw=black,line width=0.15pt},
forget plot
]
coordinates{
 (-72,4) 
};
\addplot [
color=blue,
solid,
mark=*,mark size=1.3pt,
mark options={solid,fill=mycolor77,draw=black,line width=0.15pt},
forget plot
]
coordinates{
 (-90.526901,14.6211) 
};
\addplot [
color=blue,
solid,
mark=*,mark size=1.3pt,
mark options={solid,fill=mycolor76,draw=black,line width=0.15pt},
forget plot
]
coordinates{
 (-95.366997,29.7523) 
};
\addplot [
color=blue,
solid,
mark=*,mark size=1.3pt,
mark options={solid,fill=mycolor27,draw=black,line width=0.15pt},
forget plot
]
coordinates{
 (-79.416702,43.666698) 
};
\addplot [
color=blue,
solid,
mark=*,mark size=1.3pt,
mark options={solid,fill=mycolor15,draw=black,line width=0.15pt},
forget plot
]
coordinates{
 (8,47) 
};
\addplot [
color=blue,
solid,
mark=*,mark size=1.3pt,
mark options={solid,fill=mycolor73,draw=black,line width=0.15pt},
forget plot
]
coordinates{
 (-77.0364,38.8951) 
};
\addplot [
color=blue,
solid,
mark=*,mark size=1.3pt,
mark options={solid,fill=mycolor60,draw=black,line width=0.15pt},
forget plot
]
coordinates{
 (8,47) 
};
\addplot [
color=blue,
solid,
mark=*,mark size=1.3pt,
mark options={solid,fill=mycolor1,draw=black,line width=0.15pt},
forget plot
]
coordinates{
 (8,47) 
};
\addplot [
color=blue,
solid,
mark=*,mark size=1.3pt,
mark options={solid,fill=mycolor28,draw=black,line width=0.15pt},
forget plot
]
coordinates{
 (-77.0364,38.8951) 
};
\addplot [
color=blue,
solid,
mark=*,mark size=1.3pt,
mark options={solid,fill=mycolor49,draw=black,line width=0.15pt},
forget plot
]
coordinates{
 (8,47) 
};
\addplot [
color=blue,
solid,
mark=*,mark size=1.3pt,
mark options={solid,fill=mycolor28,draw=black,line width=0.15pt},
forget plot
]
coordinates{
 (-90.533401,38.650002) 
};
\addplot [
color=blue,
solid,
mark=*,mark size=1.3pt,
mark options={solid,fill=mycolor69,draw=black,line width=0.15pt},
forget plot
]
coordinates{
 (-104.873802,39.623699) 
};
\addplot [
color=blue,
solid,
mark=*,mark size=1.3pt,
mark options={solid,fill=mycolor47,draw=black,line width=0.15pt},
forget plot
]
coordinates{
 (-48.633301,-26.983299) 
};
\addplot [
color=blue,
solid,
mark=*,mark size=1.3pt,
mark options={solid,fill=mycolor81,draw=black,line width=0.15pt},
forget plot
]
coordinates{
 (-58.672501,-34.587502) 
};
\addplot [
color=blue,
solid,
mark=*,mark size=1.3pt,
mark options={solid,fill=mycolor81,draw=black,line width=0.15pt},
forget plot
]
coordinates{
 (-58.672501,-34.587502) 
};
\addplot [
color=blue,
solid,
mark=*,mark size=1.3pt,
mark options={solid,fill=mycolor15,draw=black,line width=0.15pt},
forget plot
]
coordinates{
 (5.75,52.5) 
};
\addplot [
color=blue,
solid,
mark=*,mark size=1.3pt,
mark options={solid,fill=mycolor12,draw=black,line width=0.15pt},
forget plot
]
coordinates{
 (133,-27) 
};
\addplot [
color=blue,
solid,
mark=*,mark size=1.3pt,
mark options={solid,fill=mycolor41,draw=black,line width=0.15pt},
forget plot
]
coordinates{
 (149.134399,-35.276001) 
};
\addplot [
color=blue,
solid,
mark=*,mark size=1.3pt,
mark options={solid,fill=mycolor25,draw=black,line width=0.15pt},
forget plot
]
coordinates{
 (16.366699,48.200001) 
};
\addplot [
color=blue,
solid,
mark=*,mark size=1.3pt,
mark options={solid,fill=mycolor65,draw=black,line width=0.15pt},
forget plot
]
coordinates{
 (49.882198,40.395302) 
};
\addplot [
color=blue,
solid,
mark=*,mark size=1.3pt,
mark options={solid,fill=mycolor42,draw=black,line width=0.15pt},
forget plot
]
coordinates{
 (90,24) 
};
\addplot [
color=blue,
solid,
mark=*,mark size=1.3pt,
mark options={solid,fill=mycolor42,draw=black,line width=0.15pt},
forget plot
]
coordinates{
 (90,24) 
};
\addplot [
color=blue,
solid,
mark=*,mark size=1.3pt,
mark options={solid,fill=mycolor26,draw=black,line width=0.15pt},
forget plot
]
coordinates{
 (27.5667,53.900002) 
};
\addplot [
color=blue,
solid,
mark=*,mark size=1.3pt,
mark options={solid,fill=mycolor26,draw=black,line width=0.15pt},
forget plot
]
coordinates{
 (27.5667,53.900002) 
};
\addplot [
color=blue,
solid,
mark=*,mark size=1.3pt,
mark options={solid,fill=mycolor19,draw=black,line width=0.15pt},
forget plot
]
coordinates{
 (4.3333,50.833302) 
};
\addplot [
color=blue,
solid,
mark=*,mark size=1.3pt,
mark options={solid,fill=mycolor14,draw=black,line width=0.15pt},
forget plot
]
coordinates{
 (6.2667,50.416698) 
};
\addplot [
color=blue,
solid,
mark=*,mark size=1.3pt,
mark options={solid,fill=mycolor30,draw=black,line width=0.15pt},
forget plot
]
coordinates{
 (2.4333,6.35) 
};
\addplot [
color=blue,
solid,
mark=*,mark size=1.3pt,
mark options={solid,fill=mycolor44,draw=black,line width=0.15pt},
forget plot
]
coordinates{
 (-74.081299,40.8783) 
};
\addplot [
color=blue,
solid,
mark=*,mark size=1.3pt,
mark options={solid,fill=mycolor21,draw=black,line width=0.15pt},
forget plot
]
coordinates{
 (-68.150002,-16.5) 
};
\addplot [
color=blue,
solid,
mark=*,mark size=1.3pt,
mark options={solid,fill=mycolor18,draw=black,line width=0.15pt},
forget plot
]
coordinates{
 (18.383301,43.849998) 
};
\addplot [
color=blue,
solid,
mark=*,mark size=1.3pt,
mark options={solid,fill=mycolor14,draw=black,line width=0.15pt},
forget plot
]
coordinates{
 (18.383301,43.849998) 
};
\addplot [
color=blue,
solid,
mark=*,mark size=1.3pt,
mark options={solid,fill=mycolor26,draw=black,line width=0.15pt},
forget plot
]
coordinates{
 (18.383301,43.849998) 
};
\addplot [
color=blue,
solid,
mark=*,mark size=1.3pt,
mark options={solid,fill=mycolor26,draw=black,line width=0.15pt},
forget plot
]
coordinates{
 (17.1856,44.775799) 
};
\addplot [
color=blue,
solid,
mark=*,mark size=1.3pt,
mark options={solid,fill=mycolor82,draw=black,line width=0.15pt},
forget plot
]
coordinates{
 (25.9119,-24.6464) 
};
\addplot [
color=blue,
solid,
mark=*,mark size=1.3pt,
mark options={solid,fill=mycolor81,draw=black,line width=0.15pt},
forget plot
]
coordinates{
 (-47.916698,-15.7833) 
};
\addplot [
color=blue,
solid,
mark=*,mark size=1.3pt,
mark options={solid,fill=mycolor12,draw=black,line width=0.15pt},
forget plot
]
coordinates{
 (114.933296,4.8833) 
};
\addplot [
color=blue,
solid,
mark=*,mark size=1.3pt,
mark options={solid,fill=mycolor75,draw=black,line width=0.15pt},
forget plot
]
coordinates{
 (-87.650299,41.8745) 
};
\addplot [
color=blue,
solid,
mark=*,mark size=1.3pt,
mark options={solid,fill=mycolor52,draw=black,line width=0.15pt},
forget plot
]
coordinates{
 (-82.511703,27.8867) 
};
\addplot [
color=blue,
solid,
mark=*,mark size=1.3pt,
mark options={solid,fill=mycolor84,draw=black,line width=0.15pt},
forget plot
]
coordinates{
 (103.800003,1.3667) 
};
\addplot [
color=blue,
solid,
mark=*,mark size=1.3pt,
mark options={solid,fill=mycolor13,draw=black,line width=0.15pt},
forget plot
]
coordinates{
 (104.916702,11.55) 
};
\addplot [
color=blue,
solid,
mark=*,mark size=1.3pt,
mark options={solid,fill=mycolor31,draw=black,line width=0.15pt},
forget plot
]
coordinates{
 (-95.366997,29.7523) 
};
\addplot [
color=blue,
solid,
mark=*,mark size=1.3pt,
mark options={solid,fill=mycolor8,draw=black,line width=0.15pt},
forget plot
]
coordinates{
 (105,13) 
};
\addplot [
color=blue,
solid,
mark=*,mark size=1.3pt,
mark options={solid,fill=mycolor64,draw=black,line width=0.15pt},
forget plot
]
coordinates{
 (11.5167,3.8667) 
};
\addplot [
color=blue,
solid,
mark=*,mark size=1.3pt,
mark options={solid,fill=mycolor64,draw=black,line width=0.15pt},
forget plot
]
coordinates{
 (-81.383301,19.299999) 
};
\addplot [
color=blue,
solid,
mark=*,mark size=1.3pt,
mark options={solid,fill=mycolor27,draw=black,line width=0.15pt},
forget plot
]
coordinates{
 (-77.487503,39.043701) 
};
\addplot [
color=blue,
solid,
mark=*,mark size=1.3pt,
mark options={solid,fill=mycolor45,draw=black,line width=0.15pt},
forget plot
]
coordinates{
 (2,46) 
};
\addplot [
color=blue,
solid,
mark=*,mark size=1.3pt,
mark options={solid,fill=mycolor58,draw=black,line width=0.15pt},
forget plot
]
coordinates{
 (-70.666702,-33.450001) 
};
\addplot [
color=blue,
solid,
mark=*,mark size=1.3pt,
mark options={solid,fill=mycolor22,draw=black,line width=0.15pt},
forget plot
]
coordinates{
 (-70.666702,-33.450001) 
};
\addplot [
color=blue,
solid,
mark=*,mark size=1.3pt,
mark options={solid,fill=mycolor15,draw=black,line width=0.15pt},
forget plot
]
coordinates{
 (113.25,23.116699) 
};
\addplot [
color=blue,
solid,
mark=*,mark size=1.3pt,
mark options={solid,fill=mycolor5,draw=black,line width=0.15pt},
forget plot
]
coordinates{
 (113.550003,22.200001) 
};
\addplot [
color=blue,
solid,
mark=*,mark size=1.3pt,
mark options={solid,fill=mycolor52,draw=black,line width=0.15pt},
forget plot
]
coordinates{
 (-92.289597,34.746498) 
};
\addplot [
color=blue,
solid,
mark=*,mark size=1.3pt,
mark options={solid,fill=mycolor63,draw=black,line width=0.15pt},
forget plot
]
coordinates{
 (-74.062798,4.6492) 
};
\addplot [
color=blue,
solid,
mark=*,mark size=1.3pt,
mark options={solid,fill=mycolor14,draw=black,line width=0.15pt},
forget plot
]
coordinates{
 (6.8698,44.048302) 
};
\addplot [
color=blue,
solid,
mark=*,mark size=1.3pt,
mark options={solid,fill=mycolor14,draw=black,line width=0.15pt},
forget plot
]
coordinates{
 (4.0333,49.25) 
};
\addplot [
color=blue,
solid,
mark=*,mark size=1.3pt,
mark options={solid,fill=mycolor77,draw=black,line width=0.15pt},
forget plot
]
coordinates{
 (-159.774994,-21.2078) 
};
\addplot [
color=blue,
solid,
mark=*,mark size=1.3pt,
mark options={solid,fill=mycolor15,draw=black,line width=0.15pt},
forget plot
]
coordinates{
 (8,47) 
};
\addplot [
color=blue,
solid,
mark=*,mark size=1.3pt,
mark options={solid,fill=mycolor15,draw=black,line width=0.15pt},
forget plot
]
coordinates{
 (14.4667,50.083302) 
};
\addplot [
color=blue,
solid,
mark=*,mark size=1.3pt,
mark options={solid,fill=mycolor43,draw=black,line width=0.15pt},
forget plot
]
coordinates{
 (14.4667,50.083302) 
};
\addplot [
color=blue,
solid,
mark=*,mark size=1.3pt,
mark options={solid,fill=mycolor14,draw=black,line width=0.15pt},
forget plot
]
coordinates{
 (15.5,49.75) 
};
\addplot [
color=blue,
solid,
mark=*,mark size=1.3pt,
mark options={solid,fill=mycolor14,draw=black,line width=0.15pt},
forget plot
]
coordinates{
 (15.5,49.75) 
};
\addplot [
color=blue,
solid,
mark=*,mark size=1.3pt,
mark options={solid,fill=mycolor43,draw=black,line width=0.15pt},
forget plot
]
coordinates{
 (14.4667,50.083302) 
};
\addplot [
color=blue,
solid,
mark=*,mark size=1.3pt,
mark options={solid,fill=mycolor27,draw=black,line width=0.15pt},
forget plot
]
coordinates{
 (-75.704399,39.615501) 
};
\addplot [
color=blue,
solid,
mark=*,mark size=1.3pt,
mark options={solid,fill=mycolor78,draw=black,line width=0.15pt},
forget plot
]
coordinates{
 (-69.900002,18.4667) 
};
\addplot [
color=blue,
solid,
mark=*,mark size=1.3pt,
mark options={solid,fill=mycolor82,draw=black,line width=0.15pt},
forget plot
]
coordinates{
 (-78.5,-0.2167) 
};
\addplot [
color=blue,
solid,
mark=*,mark size=1.3pt,
mark options={solid,fill=mycolor29,draw=black,line width=0.15pt},
forget plot
]
coordinates{
 (-89.203102,13.7086) 
};
\addplot [
color=blue,
solid,
mark=*,mark size=1.3pt,
mark options={solid,fill=mycolor28,draw=black,line width=0.15pt},
forget plot
]
coordinates{
 (-87.650002,41.849998) 
};
\addplot [
color=blue,
solid,
mark=*,mark size=1.3pt,
mark options={solid,fill=mycolor46,draw=black,line width=0.15pt},
forget plot
]
coordinates{
 (24.7281,59.433899) 
};
\addplot [
color=blue,
solid,
mark=*,mark size=1.3pt,
mark options={solid,fill=mycolor18,draw=black,line width=0.15pt},
forget plot
]
coordinates{
 (26,59) 
};
\addplot [
color=blue,
solid,
mark=*,mark size=1.3pt,
mark options={solid,fill=mycolor27,draw=black,line width=0.15pt},
forget plot
]
coordinates{
 (-77.0364,38.8951) 
};
\addplot [
color=blue,
solid,
mark=*,mark size=1.3pt,
mark options={solid,fill=mycolor26,draw=black,line width=0.15pt},
forget plot
]
coordinates{
 (7.7874,48.600399) 
};
\addplot [
color=blue,
solid,
mark=*,mark size=1.3pt,
mark options={solid,fill=mycolor15,draw=black,line width=0.15pt},
forget plot
]
coordinates{
 (-2,54) 
};
\addplot [
color=blue,
solid,
mark=*,mark size=1.3pt,
mark options={solid,fill=mycolor63,draw=black,line width=0.15pt},
forget plot
]
coordinates{
 (-104.873802,39.623699) 
};
\addplot [
color=blue,
solid,
mark=*,mark size=1.3pt,
mark options={solid,fill=mycolor45,draw=black,line width=0.15pt},
forget plot
]
coordinates{
 (26,64) 
};
\addplot [
color=blue,
solid,
mark=*,mark size=1.3pt,
mark options={solid,fill=mycolor44,draw=black,line width=0.15pt},
forget plot
]
coordinates{
 (2.3333,48.866699) 
};
\addplot [
color=blue,
solid,
mark=*,mark size=1.3pt,
mark options={solid,fill=mycolor43,draw=black,line width=0.15pt},
forget plot
]
coordinates{
 (2.2667,48.883301) 
};
\addplot [
color=blue,
solid,
mark=*,mark size=1.3pt,
mark options={solid,fill=mycolor43,draw=black,line width=0.15pt},
forget plot
]
coordinates{
 (2,46) 
};
\addplot [
color=blue,
solid,
mark=*,mark size=1.3pt,
mark options={solid,fill=mycolor18,draw=black,line width=0.15pt},
forget plot
]
coordinates{
 (2,46) 
};
\addplot [
color=blue,
solid,
mark=*,mark size=1.3pt,
mark options={solid,fill=mycolor44,draw=black,line width=0.15pt},
forget plot
]
coordinates{
 (-16.5667,13.4667) 
};
\addplot [
color=blue,
solid,
mark=*,mark size=1.3pt,
mark options={solid,fill=mycolor27,draw=black,line width=0.15pt},
forget plot
]
coordinates{
 (-122.3284,47.6026) 
};
\addplot [
color=blue,
solid,
mark=*,mark size=1.3pt,
mark options={solid,fill=mycolor19,draw=black,line width=0.15pt},
forget plot
]
coordinates{
 (13.4,52.516701) 
};
\addplot [
color=blue,
solid,
mark=*,mark size=1.3pt,
mark options={solid,fill=mycolor28,draw=black,line width=0.15pt},
forget plot
]
coordinates{
 (-87.644096,41.8825) 
};
\addplot [
color=blue,
solid,
mark=*,mark size=1.3pt,
mark options={solid,fill=mycolor76,draw=black,line width=0.15pt},
forget plot
]
coordinates{
 (-71.846298,42.195499) 
};
\addplot [
color=blue,
solid,
mark=*,mark size=1.3pt,
mark options={solid,fill=mycolor76,draw=black,line width=0.15pt},
forget plot
]
coordinates{
 (-81.379204,28.5383) 
};
\addplot [
color=blue,
solid,
mark=*,mark size=1.3pt,
mark options={solid,fill=mycolor44,draw=black,line width=0.15pt},
forget plot
]
coordinates{
 (-75.662399,41.409) 
};
\addplot [
color=blue,
solid,
mark=*,mark size=1.3pt,
mark options={solid,fill=mycolor26,draw=black,line width=0.15pt},
forget plot
]
coordinates{
 (19.0833,47.5) 
};
\addplot [
color=blue,
solid,
mark=*,mark size=1.3pt,
mark options={solid,fill=mycolor50,draw=black,line width=0.15pt},
forget plot
]
coordinates{
 (-21.950001,64.150002) 
};
\addplot [
color=blue,
solid,
mark=*,mark size=1.3pt,
mark options={solid,fill=mycolor24,draw=black,line width=0.15pt},
forget plot
]
coordinates{
 (-21.950001,64.150002) 
};
\addplot [
color=blue,
solid,
mark=*,mark size=1.3pt,
mark options={solid,fill=mycolor45,draw=black,line width=0.15pt},
forget plot
]
coordinates{
 (-21.950001,64.150002) 
};
\addplot [
color=blue,
solid,
mark=*,mark size=1.3pt,
mark options={solid,fill=mycolor68,draw=black,line width=0.15pt},
forget plot
]
coordinates{
 (77.216698,28.6667) 
};
\addplot [
color=blue,
solid,
mark=*,mark size=1.3pt,
mark options={solid,fill=mycolor68,draw=black,line width=0.15pt},
forget plot
]
coordinates{
 (77.216698,28.6667) 
};
\addplot [
color=blue,
solid,
mark=*,mark size=1.3pt,
mark options={solid,fill=mycolor82,draw=black,line width=0.15pt},
forget plot
]
coordinates{
 (77.216698,28.6667) 
};
\addplot [
color=blue,
solid,
mark=*,mark size=1.3pt,
mark options={solid,fill=mycolor17,draw=black,line width=0.15pt},
forget plot
]
coordinates{
 (77.216698,28.6667) 
};
\addplot [
color=blue,
solid,
mark=*,mark size=1.3pt,
mark options={solid,fill=mycolor9,draw=black,line width=0.15pt},
forget plot
]
coordinates{
 (120,-5) 
};
\addplot [
color=blue,
solid,
mark=*,mark size=1.3pt,
mark options={solid,fill=mycolor58,draw=black,line width=0.15pt},
forget plot
]
coordinates{
 (106.829399,-6.1744) 
};
\addplot [
color=blue,
solid,
mark=*,mark size=1.3pt,
mark options={solid,fill=mycolor14,draw=black,line width=0.15pt},
forget plot
]
coordinates{
 (-2,54) 
};
\addplot [
color=blue,
solid,
mark=*,mark size=1.3pt,
mark options={solid,fill=mycolor51,draw=black,line width=0.15pt},
forget plot
]
coordinates{
 (37.615601,55.752201) 
};
\addplot [
color=blue,
solid,
mark=*,mark size=1.3pt,
mark options={solid,fill=mycolor25,draw=black,line width=0.15pt},
forget plot
]
coordinates{
 (-8,53) 
};
\addplot [
color=blue,
solid,
mark=*,mark size=1.3pt,
mark options={solid,fill=mycolor14,draw=black,line width=0.15pt},
forget plot
]
coordinates{
 (-8,53) 
};
\addplot [
color=blue,
solid,
mark=*,mark size=1.3pt,
mark options={solid,fill=mycolor19,draw=black,line width=0.15pt},
forget plot
]
coordinates{
 (5.75,52.5) 
};
\addplot [
color=blue,
solid,
mark=*,mark size=1.3pt,
mark options={solid,fill=mycolor44,draw=black,line width=0.15pt},
forget plot
]
coordinates{
 (-4.0127,5.3097) 
};
\addplot [
color=blue,
solid,
mark=*,mark size=1.3pt,
mark options={solid,fill=mycolor31,draw=black,line width=0.15pt},
forget plot
]
coordinates{
 (-95.366997,29.7523) 
};
\addplot [
color=blue,
solid,
mark=*,mark size=1.3pt,
mark options={solid,fill=mycolor64,draw=black,line width=0.15pt},
forget plot
]
coordinates{
 (-98.398697,29.488899) 
};
\addplot [
color=blue,
solid,
mark=*,mark size=1.3pt,
mark options={solid,fill=mycolor64,draw=black,line width=0.15pt},
forget plot
]
coordinates{
 (-84.2304,33.968102) 
};
\addplot [
color=blue,
solid,
mark=*,mark size=1.3pt,
mark options={solid,fill=mycolor29,draw=black,line width=0.15pt},
forget plot
]
coordinates{
 (-117.791199,33.850201) 
};
\addplot [
color=blue,
solid,
mark=*,mark size=1.3pt,
mark options={solid,fill=mycolor61,draw=black,line width=0.15pt},
forget plot
]
coordinates{
 (138,36) 
};
\addplot [
color=blue,
solid,
mark=*,mark size=1.3pt,
mark options={solid,fill=mycolor81,draw=black,line width=0.15pt},
forget plot
]
coordinates{
 (103.800003,1.3667) 
};
\addplot [
color=blue,
solid,
mark=*,mark size=1.3pt,
mark options={solid,fill=mycolor15,draw=black,line width=0.15pt},
forget plot
]
coordinates{
 (2,46) 
};
\addplot [
color=blue,
solid,
mark=*,mark size=1.3pt,
mark options={solid,fill=mycolor65,draw=black,line width=0.15pt},
forget plot
]
coordinates{
 (-74.130699,40.8326) 
};
\addplot [
color=blue,
solid,
mark=*,mark size=1.3pt,
mark options={solid,fill=mycolor75,draw=black,line width=0.15pt},
forget plot
]
coordinates{
 (47.978298,29.369699) 
};
\addplot [
color=blue,
solid,
mark=*,mark size=1.3pt,
mark options={solid,fill=mycolor44,draw=black,line width=0.15pt},
forget plot
]
coordinates{
 (-96.835297,32.929901) 
};
\addplot [
color=blue,
solid,
mark=*,mark size=1.3pt,
mark options={solid,fill=mycolor44,draw=black,line width=0.15pt},
forget plot
]
coordinates{
 (74.600304,42.8731) 
};
\addplot [
color=blue,
solid,
mark=*,mark size=1.3pt,
mark options={solid,fill=mycolor27,draw=black,line width=0.15pt},
forget plot
]
coordinates{
 (74.600304,42.8731) 
};
\addplot [
color=blue,
solid,
mark=*,mark size=1.3pt,
mark options={solid,fill=mycolor45,draw=black,line width=0.15pt},
forget plot
]
coordinates{
 (25,57) 
};
\addplot [
color=blue,
solid,
mark=*,mark size=1.3pt,
mark options={solid,fill=mycolor43,draw=black,line width=0.15pt},
forget plot
]
coordinates{
 (-0.0931,51.514198) 
};
\addplot [
color=blue,
solid,
mark=*,mark size=1.3pt,
mark options={solid,fill=mycolor14,draw=black,line width=0.15pt},
forget plot
]
coordinates{
 (11.5833,48.150002) 
};
\addplot [
color=blue,
solid,
mark=*,mark size=1.3pt,
mark options={solid,fill=mycolor19,draw=black,line width=0.15pt},
forget plot
]
coordinates{
 (8,47) 
};
\addplot [
color=blue,
solid,
mark=*,mark size=1.3pt,
mark options={solid,fill=mycolor46,draw=black,line width=0.15pt},
forget plot
]
coordinates{
 (22,41.833302) 
};
\addplot [
color=blue,
solid,
mark=*,mark size=1.3pt,
mark options={solid,fill=mycolor54,draw=black,line width=0.15pt},
forget plot
]
coordinates{
 (34,-13.5) 
};
\addplot [
color=blue,
solid,
mark=*,mark size=1.3pt,
mark options={solid,fill=mycolor31,draw=black,line width=0.15pt},
forget plot
]
coordinates{
 (-81.5401,30.143801) 
};
\addplot [
color=blue,
solid,
mark=*,mark size=1.3pt,
mark options={solid,fill=mycolor31,draw=black,line width=0.15pt},
forget plot
]
coordinates{
 (-87.637604,41.882401) 
};
\addplot [
color=blue,
solid,
mark=*,mark size=1.3pt,
mark options={solid,fill=mycolor27,draw=black,line width=0.15pt},
forget plot
]
coordinates{
 (-104.873802,39.623699) 
};
\addplot [
color=blue,
solid,
mark=*,mark size=1.3pt,
mark options={solid,fill=mycolor32,draw=black,line width=0.15pt},
forget plot
]
coordinates{
 (-157.898193,21.409401) 
};
\addplot [
color=blue,
solid,
mark=*,mark size=1.3pt,
mark options={solid,fill=mycolor79,draw=black,line width=0.15pt},
forget plot
]
coordinates{
 (-99.138603,19.4342) 
};
\addplot [
color=blue,
solid,
mark=*,mark size=1.3pt,
mark options={solid,fill=mycolor41,draw=black,line width=0.15pt},
forget plot
]
coordinates{
 (138,36) 
};
\addplot [
color=blue,
solid,
mark=*,mark size=1.3pt,
mark options={solid,fill=mycolor46,draw=black,line width=0.15pt},
forget plot
]
coordinates{
 (28.8575,47.0056) 
};
\addplot [
color=blue,
solid,
mark=*,mark size=1.3pt,
mark options={solid,fill=mycolor5,draw=black,line width=0.15pt},
forget plot
]
coordinates{
 (106.916702,47.916698) 
};
\addplot [
color=blue,
solid,
mark=*,mark size=1.3pt,
mark options={solid,fill=mycolor72,draw=black,line width=0.15pt},
forget plot
]
coordinates{
 (106.916702,47.916698) 
};
\addplot [
color=blue,
solid,
mark=*,mark size=1.3pt,
mark options={solid,fill=mycolor26,draw=black,line width=0.15pt},
forget plot
]
coordinates{
 (19.263599,42.441101) 
};
\addplot [
color=blue,
solid,
mark=*,mark size=1.3pt,
mark options={solid,fill=mycolor18,draw=black,line width=0.15pt},
forget plot
]
coordinates{
 (19.263599,42.441101) 
};
\addplot [
color=blue,
solid,
mark=*,mark size=1.3pt,
mark options={solid,fill=mycolor43,draw=black,line width=0.15pt},
forget plot
]
coordinates{
 (9,51) 
};
\addplot [
color=blue,
solid,
mark=*,mark size=1.3pt,
mark options={solid,fill=mycolor84,draw=black,line width=0.15pt},
forget plot
]
coordinates{
 (-7.6192,33.5928) 
};
\addplot [
color=blue,
solid,
mark=*,mark size=1.3pt,
mark options={solid,fill=mycolor68,draw=black,line width=0.15pt},
forget plot
]
coordinates{
 (17.083599,-22.57) 
};
\addplot [
color=blue,
solid,
mark=*,mark size=1.3pt,
mark options={solid,fill=mycolor43,draw=black,line width=0.15pt},
forget plot
]
coordinates{
 (13.4,52.516701) 
};
\addplot [
color=blue,
solid,
mark=*,mark size=1.3pt,
mark options={solid,fill=mycolor12,draw=black,line width=0.15pt},
forget plot
]
coordinates{
 (85.316704,27.7167) 
};
\addplot [
color=blue,
solid,
mark=*,mark size=1.3pt,
mark options={solid,fill=mycolor3,draw=black,line width=0.15pt},
forget plot
]
coordinates{
 (84,28) 
};
\addplot [
color=blue,
solid,
mark=*,mark size=1.3pt,
mark options={solid,fill=mycolor81,draw=black,line width=0.15pt},
forget plot
]
coordinates{
 (85.316704,27.7167) 
};
\addplot [
color=blue,
solid,
mark=*,mark size=1.3pt,
mark options={solid,fill=mycolor8,draw=black,line width=0.15pt},
forget plot
]
coordinates{
 (174,-41) 
};
\addplot [
color=blue,
solid,
mark=*,mark size=1.3pt,
mark options={solid,fill=mycolor41,draw=black,line width=0.15pt},
forget plot
]
coordinates{
 (174.783295,-41.299999) 
};
\addplot [
color=blue,
solid,
mark=*,mark size=1.3pt,
mark options={solid,fill=mycolor69,draw=black,line width=0.15pt},
forget plot
]
coordinates{
 (-86.268303,12.1508) 
};
\addplot [
color=blue,
solid,
mark=*,mark size=1.3pt,
mark options={solid,fill=mycolor63,draw=black,line width=0.15pt},
forget plot
]
coordinates{
 (-121.891502,37.333801) 
};
\addplot [
color=blue,
solid,
mark=*,mark size=1.3pt,
mark options={solid,fill=mycolor27,draw=black,line width=0.15pt},
forget plot
]
coordinates{
 (-77.382698,38.9841) 
};
\addplot [
color=blue,
solid,
mark=*,mark size=1.3pt,
mark options={solid,fill=mycolor63,draw=black,line width=0.15pt},
forget plot
]
coordinates{
 (-117.861198,33.926899) 
};
\addplot [
color=blue,
solid,
mark=*,mark size=1.3pt,
mark options={solid,fill=mycolor31,draw=black,line width=0.15pt},
forget plot
]
coordinates{
 (-81.932098,34.9496) 
};
\addplot [
color=blue,
solid,
mark=*,mark size=1.3pt,
mark options={solid,fill=mycolor27,draw=black,line width=0.15pt},
forget plot
]
coordinates{
 (-96.835297,32.929901) 
};
\addplot [
color=blue,
solid,
mark=*,mark size=1.3pt,
mark options={solid,fill=mycolor52,draw=black,line width=0.15pt},
forget plot
]
coordinates{
 (-95.4739,29.830099) 
};
\addplot [
color=blue,
solid,
mark=*,mark size=1.3pt,
mark options={solid,fill=mycolor25,draw=black,line width=0.15pt},
forget plot
]
coordinates{
 (10,62) 
};
\addplot [
color=blue,
solid,
mark=*,mark size=1.3pt,
mark options={solid,fill=mycolor41,draw=black,line width=0.15pt},
forget plot
]
coordinates{
 (57,21) 
};
\addplot [
color=blue,
solid,
mark=*,mark size=1.3pt,
mark options={solid,fill=mycolor73,draw=black,line width=0.15pt},
forget plot
]
coordinates{
 (-111.814697,41.692902) 
};
\addplot [
color=blue,
solid,
mark=*,mark size=1.3pt,
mark options={solid,fill=mycolor64,draw=black,line width=0.15pt},
forget plot
]
coordinates{
 (-118.381104,34.094299) 
};
\addplot [
color=blue,
solid,
mark=*,mark size=1.3pt,
mark options={solid,fill=mycolor29,draw=black,line width=0.15pt},
forget plot
]
coordinates{
 (67.082199,24.9056) 
};
\addplot [
color=blue,
solid,
mark=*,mark size=1.3pt,
mark options={solid,fill=mycolor42,draw=black,line width=0.15pt},
forget plot
]
coordinates{
 (-111.890602,33.6119) 
};
\addplot [
color=blue,
solid,
mark=*,mark size=1.3pt,
mark options={solid,fill=mycolor61,draw=black,line width=0.15pt},
forget plot
]
coordinates{
 (-57.666698,-25.266701) 
};
\addplot [
color=blue,
solid,
mark=*,mark size=1.3pt,
mark options={solid,fill=mycolor62,draw=black,line width=0.15pt},
forget plot
]
coordinates{
 (-57.666698,-25.266701) 
};
\addplot [
color=blue,
solid,
mark=*,mark size=1.3pt,
mark options={solid,fill=mycolor21,draw=black,line width=0.15pt},
forget plot
]
coordinates{
 (-77.050003,-12.05) 
};
\addplot [
color=blue,
solid,
mark=*,mark size=1.3pt,
mark options={solid,fill=mycolor11,draw=black,line width=0.15pt},
forget plot
]
coordinates{
 (121.050903,14.6488) 
};
\addplot [
color=blue,
solid,
mark=*,mark size=1.3pt,
mark options={solid,fill=mycolor38,draw=black,line width=0.15pt},
forget plot
]
coordinates{
 (121.050903,14.6488) 
};
\addplot [
color=blue,
solid,
mark=*,mark size=1.3pt,
mark options={solid,fill=mycolor45,draw=black,line width=0.15pt},
forget plot
]
coordinates{
 (20,52) 
};
\addplot [
color=blue,
solid,
mark=*,mark size=1.3pt,
mark options={solid,fill=mycolor25,draw=black,line width=0.15pt},
forget plot
]
coordinates{
 (21,52.25) 
};
\addplot [
color=blue,
solid,
mark=*,mark size=1.3pt,
mark options={solid,fill=mycolor45,draw=black,line width=0.15pt},
forget plot
]
coordinates{
 (-8,39.5) 
};
\addplot [
color=blue,
solid,
mark=*,mark size=1.3pt,
mark options={solid,fill=mycolor51,draw=black,line width=0.15pt},
forget plot
]
coordinates{
 (-9.1333,38.716702) 
};
\addplot [
color=blue,
solid,
mark=*,mark size=1.3pt,
mark options={solid,fill=mycolor14,draw=black,line width=0.15pt},
forget plot
]
coordinates{
 (-71.084297,42.362598) 
};
\addplot [
color=blue,
solid,
mark=*,mark size=1.3pt,
mark options={solid,fill=mycolor50,draw=black,line width=0.15pt},
forget plot
]
coordinates{
 (-7.8377,40.636299) 
};
\addplot [
color=blue,
solid,
mark=*,mark size=1.3pt,
mark options={solid,fill=mycolor26,draw=black,line width=0.15pt},
forget plot
]
coordinates{
 (25,46) 
};
\addplot [
color=blue,
solid,
mark=*,mark size=1.3pt,
mark options={solid,fill=mycolor18,draw=black,line width=0.15pt},
forget plot
]
coordinates{
 (26.1,44.4333) 
};
\addplot [
color=blue,
solid,
mark=*,mark size=1.3pt,
mark options={solid,fill=mycolor48,draw=black,line width=0.15pt},
forget plot
]
coordinates{
 (37.615601,55.752201) 
};
\addplot [
color=blue,
solid,
mark=*,mark size=1.3pt,
mark options={solid,fill=mycolor50,draw=black,line width=0.15pt},
forget plot
]
coordinates{
 (49.122101,55.7887) 
};
\addplot [
color=blue,
solid,
mark=*,mark size=1.3pt,
mark options={solid,fill=mycolor77,draw=black,line width=0.15pt},
forget plot
]
coordinates{
 (-122.208702,47.681499) 
};
\addplot [
color=blue,
solid,
mark=*,mark size=1.3pt,
mark options={solid,fill=mycolor68,draw=black,line width=0.15pt},
forget plot
]
coordinates{
 (30.0606,-1.9536) 
};
\addplot [
color=blue,
solid,
mark=*,mark size=1.3pt,
mark options={solid,fill=mycolor82,draw=black,line width=0.15pt},
forget plot
]
coordinates{
 (30.0606,-1.9536) 
};
\addplot [
color=blue,
solid,
mark=*,mark size=1.3pt,
mark options={solid,fill=mycolor30,draw=black,line width=0.15pt},
forget plot
]
coordinates{
 (-87.644096,41.8825) 
};
\addplot [
color=blue,
solid,
mark=*,mark size=1.3pt,
mark options={solid,fill=mycolor78,draw=black,line width=0.15pt},
forget plot
]
coordinates{
 (-61,14) 
};
\addplot [
color=blue,
solid,
mark=*,mark size=1.3pt,
mark options={solid,fill=mycolor2,draw=black,line width=0.15pt},
forget plot
]
coordinates{
 (-172.333298,-13.5833) 
};
\addplot [
color=blue,
solid,
mark=*,mark size=1.3pt,
mark options={solid,fill=mycolor4,draw=black,line width=0.15pt},
forget plot
]
coordinates{
 (-172.333298,-13.5833) 
};
\addplot [
color=blue,
solid,
mark=*,mark size=1.3pt,
mark options={solid,fill=mycolor75,draw=black,line width=0.15pt},
forget plot
]
coordinates{
 (-87.650002,41.849998) 
};
\addplot [
color=blue,
solid,
mark=*,mark size=1.3pt,
mark options={solid,fill=mycolor26,draw=black,line width=0.15pt},
forget plot
]
coordinates{
 (17.1667,60.666698) 
};
\addplot [
color=blue,
solid,
mark=*,mark size=1.3pt,
mark options={solid,fill=mycolor31,draw=black,line width=0.15pt},
forget plot
]
coordinates{
 (-96.835297,32.929901) 
};
\addplot [
color=blue,
solid,
mark=*,mark size=1.3pt,
mark options={solid,fill=mycolor74,draw=black,line width=0.15pt},
forget plot
]
coordinates{
 (-87.650002,41.849998) 
};
\addplot [
color=blue,
solid,
mark=*,mark size=1.3pt,
mark options={solid,fill=mycolor73,draw=black,line width=0.15pt},
forget plot
]
coordinates{
 (-77.041702,38.8979) 
};
\addplot [
color=blue,
solid,
mark=*,mark size=1.3pt,
mark options={solid,fill=mycolor14,draw=black,line width=0.15pt},
forget plot
]
coordinates{
 (-2,54) 
};
\addplot [
color=blue,
solid,
mark=*,mark size=1.3pt,
mark options={solid,fill=mycolor65,draw=black,line width=0.15pt},
forget plot
]
coordinates{
 (-122.113403,47.605099) 
};
\addplot [
color=blue,
solid,
mark=*,mark size=1.3pt,
mark options={solid,fill=mycolor25,draw=black,line width=0.15pt},
forget plot
]
coordinates{
 (18.0333,48.900002) 
};
\addplot [
color=blue,
solid,
mark=*,mark size=1.3pt,
mark options={solid,fill=mycolor14,draw=black,line width=0.15pt},
forget plot
]
coordinates{
 (19.5,48.666698) 
};
\addplot [
color=blue,
solid,
mark=*,mark size=1.3pt,
mark options={solid,fill=mycolor14,draw=black,line width=0.15pt},
forget plot
]
coordinates{
 (17.116699,48.150002) 
};
\addplot [
color=blue,
solid,
mark=*,mark size=1.3pt,
mark options={solid,fill=mycolor15,draw=black,line width=0.15pt},
forget plot
]
coordinates{
 (9,51) 
};
\addplot [
color=blue,
solid,
mark=*,mark size=1.3pt,
mark options={solid,fill=mycolor61,draw=black,line width=0.15pt},
forget plot
]
coordinates{
 (138,36) 
};
\addplot [
color=blue,
solid,
mark=*,mark size=1.3pt,
mark options={solid,fill=mycolor31,draw=black,line width=0.15pt},
forget plot
]
coordinates{
 (-122.007401,37.4249) 
};
\addplot [
color=blue,
solid,
mark=*,mark size=1.3pt,
mark options={solid,fill=mycolor15,draw=black,line width=0.15pt},
forget plot
]
coordinates{
 (5.75,52.5) 
};
\addplot [
color=blue,
solid,
mark=*,mark size=1.3pt,
mark options={solid,fill=mycolor52,draw=black,line width=0.15pt},
forget plot
]
coordinates{
 (-75.408302,40.054798) 
};
\addplot [
color=blue,
solid,
mark=*,mark size=1.3pt,
mark options={solid,fill=mycolor75,draw=black,line width=0.15pt},
forget plot
]
coordinates{
 (-83.138298,39.964901) 
};
\addplot [
color=blue,
solid,
mark=*,mark size=1.3pt,
mark options={solid,fill=mycolor14,draw=black,line width=0.15pt},
forget plot
]
coordinates{
 (-4,40) 
};
\addplot [
color=blue,
solid,
mark=*,mark size=1.3pt,
mark options={solid,fill=mycolor51,draw=black,line width=0.15pt},
forget plot
]
coordinates{
 (-8.4068,43.3666) 
};
\addplot [
color=blue,
solid,
mark=*,mark size=1.3pt,
mark options={solid,fill=mycolor74,draw=black,line width=0.15pt},
forget plot
]
coordinates{
 (-87.650002,41.849998) 
};
\addplot [
color=blue,
solid,
mark=*,mark size=1.3pt,
mark options={solid,fill=mycolor34,draw=black,line width=0.15pt},
forget plot
]
coordinates{
 (31.133301,-26.3167) 
};
\addplot [
color=blue,
solid,
mark=*,mark size=1.3pt,
mark options={solid,fill=mycolor43,draw=black,line width=0.15pt},
forget plot
]
coordinates{
 (11.0683,49.4478) 
};
\addplot [
color=blue,
solid,
mark=*,mark size=1.3pt,
mark options={solid,fill=mycolor26,draw=black,line width=0.15pt},
forget plot
]
coordinates{
 (15.4167,60.483299) 
};
\addplot [
color=blue,
solid,
mark=*,mark size=1.3pt,
mark options={solid,fill=mycolor51,draw=black,line width=0.15pt},
forget plot
]
coordinates{
 (15,62) 
};
\addplot [
color=blue,
solid,
mark=*,mark size=1.3pt,
mark options={solid,fill=mycolor85,draw=black,line width=0.15pt},
forget plot
]
coordinates{
 (32,49) 
};
\addplot [
color=blue,
solid,
mark=*,mark size=1.3pt,
mark options={solid,fill=mycolor14,draw=black,line width=0.15pt},
forget plot
]
coordinates{
 (-2,54) 
};
\addplot [
color=blue,
solid,
mark=*,mark size=1.3pt,
mark options={solid,fill=mycolor34,draw=black,line width=0.15pt},
forget plot
]
coordinates{
 (100.501404,13.754) 
};
\addplot [
color=blue,
solid,
mark=*,mark size=1.3pt,
mark options={solid,fill=mycolor15,draw=black,line width=0.15pt},
forget plot
]
coordinates{
 (2,46) 
};
\addplot [
color=blue,
solid,
mark=*,mark size=1.3pt,
mark options={solid,fill=mycolor75,draw=black,line width=0.15pt},
forget plot
]
coordinates{
 (-93.605797,41.729698) 
};
\addplot [
color=blue,
solid,
mark=*,mark size=1.3pt,
mark options={solid,fill=mycolor51,draw=black,line width=0.15pt},
forget plot
]
coordinates{
 (10.1711,36.806099) 
};
\addplot [
color=blue,
solid,
mark=*,mark size=1.3pt,
mark options={solid,fill=mycolor59,draw=black,line width=0.15pt},
forget plot
]
coordinates{
 (28.964701,41.0186) 
};
\addplot [
color=blue,
solid,
mark=*,mark size=1.3pt,
mark options={solid,fill=mycolor27,draw=black,line width=0.15pt},
forget plot
]
coordinates{
 (28.964701,41.0186) 
};
\addplot [
color=blue,
solid,
mark=*,mark size=1.3pt,
mark options={solid,fill=mycolor69,draw=black,line width=0.15pt},
forget plot
]
coordinates{
 (-118.3965,34.021099) 
};
\addplot [
color=blue,
solid,
mark=*,mark size=1.3pt,
mark options={solid,fill=mycolor52,draw=black,line width=0.15pt},
forget plot
]
coordinates{
 (55.304699,25.2582) 
};
\addplot [
color=blue,
solid,
mark=*,mark size=1.3pt,
mark options={solid,fill=mycolor19,draw=black,line width=0.15pt},
forget plot
]
coordinates{
 (-71.084297,42.362598) 
};
\addplot [
color=blue,
solid,
mark=*,mark size=1.3pt,
mark options={solid,fill=mycolor25,draw=black,line width=0.15pt},
forget plot
]
coordinates{
 (-8,53) 
};
\addplot [
color=blue,
solid,
mark=*,mark size=1.3pt,
mark options={solid,fill=mycolor19,draw=black,line width=0.15pt},
forget plot
]
coordinates{
 (-0.0931,51.514198) 
};
\addplot [
color=blue,
solid,
mark=*,mark size=1.3pt,
mark options={solid,fill=mycolor43,draw=black,line width=0.15pt},
forget plot
]
coordinates{
 (9,51) 
};
\addplot [
color=blue,
solid,
mark=*,mark size=1.3pt,
mark options={solid,fill=mycolor14,draw=black,line width=0.15pt},
forget plot
]
coordinates{
 (-2,54) 
};
\addplot [
color=blue,
solid,
mark=*,mark size=1.3pt,
mark options={solid,fill=mycolor15,draw=black,line width=0.15pt},
forget plot
]
coordinates{
 (9,51) 
};
\addplot [
color=blue,
solid,
mark=*,mark size=1.3pt,
mark options={solid,fill=mycolor14,draw=black,line width=0.15pt},
forget plot
]
coordinates{
 (-4.57,54.23) 
};
\addplot [
color=blue,
solid,
mark=*,mark size=1.3pt,
mark options={solid,fill=mycolor26,draw=black,line width=0.15pt},
forget plot
]
coordinates{
 (-2,54) 
};
\addplot [
color=blue,
solid,
mark=*,mark size=1.3pt,
mark options={solid,fill=mycolor15,draw=black,line width=0.15pt},
forget plot
]
coordinates{
 (-2,54) 
};
\addplot [
color=blue,
solid,
mark=*,mark size=1.3pt,
mark options={solid,fill=mycolor27,draw=black,line width=0.15pt},
forget plot
]
coordinates{
 (-104.873802,39.623699) 
};
\addplot [
color=blue,
solid,
mark=*,mark size=1.3pt,
mark options={solid,fill=mycolor61,draw=black,line width=0.15pt},
forget plot
]
coordinates{
 (-64.783897,32.294201) 
};
\addplot [
color=blue,
solid,
mark=*,mark size=1.3pt,
mark options={solid,fill=mycolor15,draw=black,line width=0.15pt},
forget plot
]
coordinates{
 (-2,54) 
};
\addplot [
color=blue,
solid,
mark=*,mark size=1.3pt,
mark options={solid,fill=mycolor64,draw=black,line width=0.15pt},
forget plot
]
coordinates{
 (-62.200001,16.75) 
};
\addplot [
color=blue,
solid,
mark=*,mark size=1.3pt,
mark options={solid,fill=mycolor43,draw=black,line width=0.15pt},
forget plot
]
coordinates{
 (-0.0931,51.514198) 
};
\addplot [
color=blue,
solid,
mark=*,mark size=1.3pt,
mark options={solid,fill=mycolor14,draw=black,line width=0.15pt},
forget plot
]
coordinates{
 (-2,54) 
};
\addplot [
color=blue,
solid,
mark=*,mark size=1.3pt,
mark options={solid,fill=mycolor7,draw=black,line width=0.15pt},
forget plot
]
coordinates{
 (-56.170799,-34.858101) 
};
\addplot [
color=blue,
solid,
mark=*,mark size=1.3pt,
mark options={solid,fill=mycolor2,draw=black,line width=0.15pt},
forget plot
]
coordinates{
 (-56.170799,-34.858101) 
};
\addplot [
color=blue,
solid,
mark=*,mark size=1.3pt,
mark options={solid,fill=mycolor63,draw=black,line width=0.15pt},
forget plot
]
coordinates{
 (-66.916702,10.5) 
};
\addplot [
color=blue,
solid,
mark=*,mark size=1.3pt,
mark options={solid,fill=mycolor76,draw=black,line width=0.15pt},
forget plot
]
coordinates{
 (-75.408302,40.054798) 
};
\addplot [
color=blue,
solid,
mark=*,mark size=1.3pt,
mark options={solid,fill=mycolor10,draw=black,line width=0.15pt},
forget plot
]
coordinates{
 (105.849998,21.0333) 
};
\addplot [
color=blue,
solid,
mark=*,mark size=1.3pt,
mark options={solid,fill=mycolor63,draw=black,line width=0.15pt},
forget plot
]
coordinates{
 (48,15) 
};
\addplot [
color=blue,
solid,
mark=*,mark size=1.3pt,
mark options={solid,fill=mycolor78,draw=black,line width=0.15pt},
forget plot
]
coordinates{
 (48,15) 
};
\addplot [
color=blue,
solid,
mark=*,mark size=1.3pt,
mark options={solid,fill=mycolor81,draw=black,line width=0.15pt},
forget plot
]
coordinates{
 (28.2833,-15.4167) 
};
\addplot [
color=blue,
solid,
mark=*,mark size=1.3pt,
mark options={solid,fill=mycolor69,draw=black,line width=0.15pt},
forget plot
]
coordinates{
 (-76.800003,18) 
};
\addplot [
color=blue,
solid,
mark=*,mark size=1.3pt,
mark options={solid,fill=mycolor18,draw=black,line width=0.15pt},
forget plot
]
coordinates{
 (1.5167,42.5) 
};
\addplot [
color=blue,
solid,
mark=*,mark size=1.3pt,
mark options={solid,fill=mycolor18,draw=black,line width=0.15pt},
forget plot
]
coordinates{
 (1.5167,42.583302) 
};
\addplot [
color=blue,
solid,
mark=*,mark size=1.3pt,
mark options={solid,fill=mycolor18,draw=black,line width=0.15pt},
forget plot
]
coordinates{
 (1.4833,42.5667) 
};
\addplot [
color=blue,
solid,
mark=*,mark size=1.3pt,
mark options={solid,fill=mycolor45,draw=black,line width=0.15pt},
forget plot
]
coordinates{
 (1.5167,42.5) 
};
\addplot [
color=blue,
solid,
mark=*,mark size=1.3pt,
mark options={solid,fill=mycolor48,draw=black,line width=0.15pt},
forget plot
]
coordinates{
 (1.5,42.466702) 
};
\addplot [
color=blue,
solid,
mark=*,mark size=1.3pt,
mark options={solid,fill=mycolor51,draw=black,line width=0.15pt},
forget plot
]
coordinates{
 (1.5167,42.5) 
};
\addplot [
color=blue,
solid,
mark=*,mark size=1.3pt,
mark options={solid,fill=mycolor18,draw=black,line width=0.15pt},
forget plot
]
coordinates{
 (1.5167,42.5) 
};
\addplot [
color=blue,
solid,
mark=*,mark size=1.3pt,
mark options={solid,fill=mycolor51,draw=black,line width=0.15pt},
forget plot
]
coordinates{
 (1.5167,42.5) 
};
\addplot [
color=blue,
solid,
mark=*,mark size=1.3pt,
mark options={solid,fill=mycolor18,draw=black,line width=0.15pt},
forget plot
]
coordinates{
 (1.55,42.516701) 
};
\addplot [
color=blue,
solid,
mark=*,mark size=1.3pt,
mark options={solid,fill=mycolor18,draw=black,line width=0.15pt},
forget plot
]
coordinates{
 (1.5167,42.5) 
};
\addplot [
color=blue,
solid,
mark=*,mark size=1.3pt,
mark options={solid,fill=mycolor76,draw=black,line width=0.15pt},
forget plot
]
coordinates{
 (55.304699,25.2582) 
};
\addplot [
color=blue,
solid,
mark=*,mark size=1.3pt,
mark options={solid,fill=mycolor70,draw=black,line width=0.15pt},
forget plot
]
coordinates{
 (55.304699,25.2582) 
};
\addplot [
color=blue,
solid,
mark=*,mark size=1.3pt,
mark options={solid,fill=mycolor32,draw=black,line width=0.15pt},
forget plot
]
coordinates{
 (55.304699,25.2582) 
};
\addplot [
color=blue,
solid,
mark=*,mark size=1.3pt,
mark options={solid,fill=mycolor64,draw=black,line width=0.15pt},
forget plot
]
coordinates{
 (55.304699,25.2582) 
};
\addplot [
color=blue,
solid,
mark=*,mark size=1.3pt,
mark options={solid,fill=mycolor42,draw=black,line width=0.15pt},
forget plot
]
coordinates{
 (69.183296,34.516701) 
};
\addplot [
color=blue,
solid,
mark=*,mark size=1.3pt,
mark options={solid,fill=mycolor17,draw=black,line width=0.15pt},
forget plot
]
coordinates{
 (69.183296,34.516701) 
};
\addplot [
color=blue,
solid,
mark=*,mark size=1.3pt,
mark options={solid,fill=mycolor52,draw=black,line width=0.15pt},
forget plot
]
coordinates{
 (-61.849998,17.116699) 
};
\addplot [
color=blue,
solid,
mark=*,mark size=1.3pt,
mark options={solid,fill=mycolor79,draw=black,line width=0.15pt},
forget plot
]
coordinates{
 (-61.849998,17.116699) 
};
\addplot [
color=blue,
solid,
mark=*,mark size=1.3pt,
mark options={solid,fill=mycolor23,draw=black,line width=0.15pt},
forget plot
]
coordinates{
 (-61.849998,17.116699) 
};
\addplot [
color=blue,
solid,
mark=*,mark size=1.3pt,
mark options={solid,fill=mycolor42,draw=black,line width=0.15pt},
forget plot
]
coordinates{
 (-62.616699,17.133301) 
};
\addplot [
color=blue,
solid,
mark=*,mark size=1.3pt,
mark options={solid,fill=mycolor69,draw=black,line width=0.15pt},
forget plot
]
coordinates{
 (-0.0931,51.514198) 
};
\addplot [
color=blue,
solid,
mark=*,mark size=1.3pt,
mark options={solid,fill=mycolor54,draw=black,line width=0.15pt},
forget plot
]
coordinates{
 (-61.849998,17.116699) 
};
\addplot [
color=blue,
solid,
mark=*,mark size=1.3pt,
mark options={solid,fill=mycolor77,draw=black,line width=0.15pt},
forget plot
]
coordinates{
 (-61.849998,17.116699) 
};
\addplot [
color=blue,
solid,
mark=*,mark size=1.3pt,
mark options={solid,fill=mycolor77,draw=black,line width=0.15pt},
forget plot
]
coordinates{
 (-61.849998,17.116699) 
};
\addplot [
color=blue,
solid,
mark=*,mark size=1.3pt,
mark options={solid,fill=mycolor16,draw=black,line width=0.15pt},
forget plot
]
coordinates{
 (-61.849998,17.116699) 
};
\addplot [
color=blue,
solid,
mark=*,mark size=1.3pt,
mark options={solid,fill=mycolor63,draw=black,line width=0.15pt},
forget plot
]
coordinates{
 (-63.049999,18.2167) 
};
\addplot [
color=blue,
solid,
mark=*,mark size=1.3pt,
mark options={solid,fill=mycolor29,draw=black,line width=0.15pt},
forget plot
]
coordinates{
 (-63.049999,18.2167) 
};
\addplot [
color=blue,
solid,
mark=*,mark size=1.3pt,
mark options={solid,fill=mycolor77,draw=black,line width=0.15pt},
forget plot
]
coordinates{
 (-63.049999,18.2167) 
};
\addplot [
color=blue,
solid,
mark=*,mark size=1.3pt,
mark options={solid,fill=mycolor44,draw=black,line width=0.15pt},
forget plot
]
coordinates{
 (-63.049999,18.2167) 
};
\addplot [
color=blue,
solid,
mark=*,mark size=1.3pt,
mark options={solid,fill=mycolor78,draw=black,line width=0.15pt},
forget plot
]
coordinates{
 (-63.049999,18.2167) 
};
\addplot [
color=blue,
solid,
mark=*,mark size=1.3pt,
mark options={solid,fill=mycolor69,draw=black,line width=0.15pt},
forget plot
]
coordinates{
 (-63.049999,18.2167) 
};
\addplot [
color=blue,
solid,
mark=*,mark size=1.3pt,
mark options={solid,fill=mycolor44,draw=black,line width=0.15pt},
forget plot
]
coordinates{
 (-63.049999,18.2167) 
};
\addplot [
color=blue,
solid,
mark=*,mark size=1.3pt,
mark options={solid,fill=mycolor29,draw=black,line width=0.15pt},
forget plot
]
coordinates{
 (-63.049999,18.2167) 
};
\addplot [
color=blue,
solid,
mark=*,mark size=1.3pt,
mark options={solid,fill=mycolor85,draw=black,line width=0.15pt},
forget plot
]
coordinates{
 (20,41) 
};
\addplot [
color=blue,
solid,
mark=*,mark size=1.3pt,
mark options={solid,fill=mycolor26,draw=black,line width=0.15pt},
forget plot
]
coordinates{
 (20,41) 
};
\addplot [
color=blue,
solid,
mark=*,mark size=1.3pt,
mark options={solid,fill=mycolor48,draw=black,line width=0.15pt},
forget plot
]
coordinates{
 (19.818899,41.327499) 
};
\addplot [
color=blue,
solid,
mark=*,mark size=1.3pt,
mark options={solid,fill=mycolor45,draw=black,line width=0.15pt},
forget plot
]
coordinates{
 (20,41) 
};
\addplot [
color=blue,
solid,
mark=*,mark size=1.3pt,
mark options={solid,fill=mycolor26,draw=black,line width=0.15pt},
forget plot
]
coordinates{
 (19.818899,41.327499) 
};
\addplot [
color=blue,
solid,
mark=*,mark size=1.3pt,
mark options={solid,fill=mycolor86,draw=black,line width=0.15pt},
forget plot
]
coordinates{
 (20,41) 
};
\addplot [
color=blue,
solid,
mark=*,mark size=1.3pt,
mark options={solid,fill=mycolor45,draw=black,line width=0.15pt},
forget plot
]
coordinates{
 (19.818899,41.327499) 
};
\addplot [
color=blue,
solid,
mark=*,mark size=1.3pt,
mark options={solid,fill=mycolor50,draw=black,line width=0.15pt},
forget plot
]
coordinates{
 (44.513599,40.181099) 
};
\addplot [
color=blue,
solid,
mark=*,mark size=1.3pt,
mark options={solid,fill=mycolor24,draw=black,line width=0.15pt},
forget plot
]
coordinates{
 (44.513599,40.181099) 
};
\addplot [
color=blue,
solid,
mark=*,mark size=1.3pt,
mark options={solid,fill=mycolor27,draw=black,line width=0.15pt},
forget plot
]
coordinates{
 (44.513599,40.181099) 
};
\addplot [
color=blue,
solid,
mark=*,mark size=1.3pt,
mark options={solid,fill=mycolor50,draw=black,line width=0.15pt},
forget plot
]
coordinates{
 (44.513599,40.181099) 
};
\addplot [
color=blue,
solid,
mark=*,mark size=1.3pt,
mark options={solid,fill=mycolor85,draw=black,line width=0.15pt},
forget plot
]
coordinates{
 (45,40) 
};
\addplot [
color=blue,
solid,
mark=*,mark size=1.3pt,
mark options={solid,fill=mycolor72,draw=black,line width=0.15pt},
forget plot
]
coordinates{
 (18.5,-12.5) 
};
\addplot [
color=blue,
solid,
mark=*,mark size=1.3pt,
mark options={solid,fill=mycolor64,draw=black,line width=0.15pt},
forget plot
]
coordinates{
 (13.2332,-8.8368) 
};
\addplot [
color=blue,
solid,
mark=*,mark size=1.3pt,
mark options={solid,fill=mycolor64,draw=black,line width=0.15pt},
forget plot
]
coordinates{
 (13.2332,-8.8368) 
};
\addplot [
color=blue,
solid,
mark=*,mark size=1.3pt,
mark options={solid,fill=mycolor3,draw=black,line width=0.15pt},
forget plot
]
coordinates{
 (12.2,-5.55) 
};
\addplot [
color=blue,
solid,
mark=*,mark size=1.3pt,
mark options={solid,fill=mycolor84,draw=black,line width=0.15pt},
forget plot
]
coordinates{
 (-60.639301,-32.9468) 
};
\addplot [
color=blue,
solid,
mark=*,mark size=1.3pt,
mark options={solid,fill=mycolor4,draw=black,line width=0.15pt},
forget plot
]
coordinates{
 (-58.7878,-34.4501) 
};
\addplot [
color=blue,
solid,
mark=*,mark size=1.3pt,
mark options={solid,fill=mycolor4,draw=black,line width=0.15pt},
forget plot
]
coordinates{
 (-58.672501,-34.587502) 
};
\addplot [
color=blue,
solid,
mark=*,mark size=1.3pt,
mark options={solid,fill=mycolor4,draw=black,line width=0.15pt},
forget plot
]
coordinates{
 (-60.322498,-36.8927) 
};
\addplot [
color=blue,
solid,
mark=*,mark size=1.3pt,
mark options={solid,fill=mycolor3,draw=black,line width=0.15pt},
forget plot
]
coordinates{
 (-58.672501,-34.587502) 
};
\addplot [
color=blue,
solid,
mark=*,mark size=1.3pt,
mark options={solid,fill=mycolor6,draw=black,line width=0.15pt},
forget plot
]
coordinates{
 (-57.557499,-38.0023) 
};
\addplot [
color=blue,
solid,
mark=*,mark size=1.3pt,
mark options={solid,fill=mycolor7,draw=black,line width=0.15pt},
forget plot
]
coordinates{
 (-58.252602,-34.724201) 
};
\addplot [
color=blue,
solid,
mark=*,mark size=1.3pt,
mark options={solid,fill=mycolor2,draw=black,line width=0.15pt},
forget plot
]
coordinates{
 (-58.487,-34.5075) 
};
\addplot [
color=blue,
solid,
mark=*,mark size=1.3pt,
mark options={solid,fill=mycolor7,draw=black,line width=0.15pt},
forget plot
]
coordinates{
 (-60.4739,-34.641701) 
};
\addplot [
color=blue,
solid,
mark=*,mark size=1.3pt,
mark options={solid,fill=mycolor44,draw=black,line width=0.15pt},
forget plot
]
coordinates{
 (13.0333,47.799999) 
};
\addplot [
color=blue,
solid,
mark=*,mark size=1.3pt,
mark options={solid,fill=mycolor18,draw=black,line width=0.15pt},
forget plot
]
coordinates{
 (16.366699,48.200001) 
};
\addplot [
color=blue,
solid,
mark=*,mark size=1.3pt,
mark options={solid,fill=mycolor87,draw=black,line width=0.15pt},
forget plot
]
coordinates{
 (150.983307,-33.833302) 
};
\addplot [
color=blue,
solid,
mark=*,mark size=1.3pt,
mark options={solid,fill=mycolor29,draw=black,line width=0.15pt},
forget plot
]
coordinates{
 (-70.033302,12.5167) 
};
\addplot [
color=blue,
solid,
mark=*,mark size=1.3pt,
mark options={solid,fill=mycolor42,draw=black,line width=0.15pt},
forget plot
]
coordinates{
 (-70.033302,12.5167) 
};
\addplot [
color=blue,
solid,
mark=*,mark size=1.3pt,
mark options={solid,fill=mycolor69,draw=black,line width=0.15pt},
forget plot
]
coordinates{
 (-70.033302,12.5167) 
};
\addplot [
color=blue,
solid,
mark=*,mark size=1.3pt,
mark options={solid,fill=mycolor63,draw=black,line width=0.15pt},
forget plot
]
coordinates{
 (-70.033302,12.5167) 
};
\addplot [
color=blue,
solid,
mark=*,mark size=1.3pt,
mark options={solid,fill=mycolor75,draw=black,line width=0.15pt},
forget plot
]
coordinates{
 (49.882198,40.395302) 
};
\addplot [
color=blue,
solid,
mark=*,mark size=1.3pt,
mark options={solid,fill=mycolor44,draw=black,line width=0.15pt},
forget plot
]
coordinates{
 (49.882198,40.395302) 
};
\addplot [
color=blue,
solid,
mark=*,mark size=1.3pt,
mark options={solid,fill=mycolor18,draw=black,line width=0.15pt},
forget plot
]
coordinates{
 (18.383301,43.849998) 
};
\addplot [
color=blue,
solid,
mark=*,mark size=1.3pt,
mark options={solid,fill=mycolor85,draw=black,line width=0.15pt},
forget plot
]
coordinates{
 (18.383301,43.849998) 
};
\addplot [
color=blue,
solid,
mark=*,mark size=1.3pt,
mark options={solid,fill=mycolor85,draw=black,line width=0.15pt},
forget plot
]
coordinates{
 (18.2953,44.8139) 
};
\addplot [
color=blue,
solid,
mark=*,mark size=1.3pt,
mark options={solid,fill=mycolor45,draw=black,line width=0.15pt},
forget plot
]
coordinates{
 (18.669399,44.542801) 
};
\addplot [
color=blue,
solid,
mark=*,mark size=1.3pt,
mark options={solid,fill=mycolor51,draw=black,line width=0.15pt},
forget plot
]
coordinates{
 (17.8564,44.2047) 
};
\addplot [
color=blue,
solid,
mark=*,mark size=1.3pt,
mark options={solid,fill=mycolor26,draw=black,line width=0.15pt},
forget plot
]
coordinates{
 (17.834999,43.042801) 
};
\addplot [
color=blue,
solid,
mark=*,mark size=1.3pt,
mark options={solid,fill=mycolor24,draw=black,line width=0.15pt},
forget plot
]
coordinates{
 (18.2953,44.8139) 
};
\addplot [
color=blue,
solid,
mark=*,mark size=1.3pt,
mark options={solid,fill=mycolor18,draw=black,line width=0.15pt},
forget plot
]
coordinates{
 (18,44) 
};
\addplot [
color=blue,
solid,
mark=*,mark size=1.3pt,
mark options={solid,fill=mycolor45,draw=black,line width=0.15pt},
forget plot
]
coordinates{
 (17.1856,44.775799) 
};
\addplot [
color=blue,
solid,
mark=*,mark size=1.3pt,
mark options={solid,fill=mycolor51,draw=black,line width=0.15pt},
forget plot
]
coordinates{
 (18,44) 
};
\addplot [
color=blue,
solid,
mark=*,mark size=1.3pt,
mark options={solid,fill=mycolor46,draw=black,line width=0.15pt},
forget plot
]
coordinates{
 (18.669399,44.542801) 
};
\addplot [
color=blue,
solid,
mark=*,mark size=1.3pt,
mark options={solid,fill=mycolor18,draw=black,line width=0.15pt},
forget plot
]
coordinates{
 (17.670601,44.2267) 
};
\addplot [
color=blue,
solid,
mark=*,mark size=1.3pt,
mark options={solid,fill=mycolor18,draw=black,line width=0.15pt},
forget plot
]
coordinates{
 (18.669399,44.542801) 
};
\addplot [
color=blue,
solid,
mark=*,mark size=1.3pt,
mark options={solid,fill=mycolor82,draw=black,line width=0.15pt},
forget plot
]
coordinates{
 (-59.616699,13.1) 
};
\addplot [
color=blue,
solid,
mark=*,mark size=1.3pt,
mark options={solid,fill=mycolor61,draw=black,line width=0.15pt},
forget plot
]
coordinates{
 (-111.883698,40.7561) 
};
\addplot [
color=blue,
solid,
mark=*,mark size=1.3pt,
mark options={solid,fill=mycolor77,draw=black,line width=0.15pt},
forget plot
]
coordinates{
 (-59.616699,13.1) 
};
\addplot [
color=blue,
solid,
mark=*,mark size=1.3pt,
mark options={solid,fill=mycolor69,draw=black,line width=0.15pt},
forget plot
]
coordinates{
 (-59.616699,13.1) 
};
\addplot [
color=blue,
solid,
mark=*,mark size=1.3pt,
mark options={solid,fill=mycolor80,draw=black,line width=0.15pt},
forget plot
]
coordinates{
 (-59.616699,13.1) 
};
\addplot [
color=blue,
solid,
mark=*,mark size=1.3pt,
mark options={solid,fill=mycolor79,draw=black,line width=0.15pt},
forget plot
]
coordinates{
 (-59.616699,13.1) 
};
\addplot [
color=blue,
solid,
mark=*,mark size=1.3pt,
mark options={solid,fill=mycolor80,draw=black,line width=0.15pt},
forget plot
]
coordinates{
 (-59.616699,13.1) 
};
\addplot [
color=blue,
solid,
mark=*,mark size=1.3pt,
mark options={solid,fill=mycolor42,draw=black,line width=0.15pt},
forget plot
]
coordinates{
 (-59.616699,13.1) 
};
\addplot [
color=blue,
solid,
mark=*,mark size=1.3pt,
mark options={solid,fill=mycolor79,draw=black,line width=0.15pt},
forget plot
]
coordinates{
 (-59.616699,13.1) 
};
\addplot [
color=blue,
solid,
mark=*,mark size=1.3pt,
mark options={solid,fill=mycolor29,draw=black,line width=0.15pt},
forget plot
]
coordinates{
 (-59.616699,13.1) 
};
\addplot [
color=blue,
solid,
mark=*,mark size=1.3pt,
mark options={solid,fill=mycolor17,draw=black,line width=0.15pt},
forget plot
]
coordinates{
 (-59.616699,13.1) 
};
\addplot [
color=blue,
solid,
mark=*,mark size=1.3pt,
mark options={solid,fill=mycolor80,draw=black,line width=0.15pt},
forget plot
]
coordinates{
 (-59.616699,13.1) 
};
\addplot [
color=blue,
solid,
mark=*,mark size=1.3pt,
mark options={solid,fill=mycolor79,draw=black,line width=0.15pt},
forget plot
]
coordinates{
 (-59.616699,13.1) 
};
\addplot [
color=blue,
solid,
mark=*,mark size=1.3pt,
mark options={solid,fill=mycolor17,draw=black,line width=0.15pt},
forget plot
]
coordinates{
 (90.4086,23.723101) 
};
\addplot [
color=blue,
solid,
mark=*,mark size=1.3pt,
mark options={solid,fill=mycolor43,draw=black,line width=0.15pt},
forget plot
]
coordinates{
 (3.7167,51.049999) 
};
\addplot [
color=blue,
solid,
mark=*,mark size=1.3pt,
mark options={solid,fill=mycolor26,draw=black,line width=0.15pt},
forget plot
]
coordinates{
 (4.95,51.25) 
};
\addplot [
color=blue,
solid,
mark=*,mark size=1.3pt,
mark options={solid,fill=mycolor43,draw=black,line width=0.15pt},
forget plot
]
coordinates{
 (4.2333,50.733299) 
};
\addplot [
color=blue,
solid,
mark=*,mark size=1.3pt,
mark options={solid,fill=mycolor27,draw=black,line width=0.15pt},
forget plot
]
coordinates{
 (25,43) 
};
\addplot [
color=blue,
solid,
mark=*,mark size=1.3pt,
mark options={solid,fill=mycolor46,draw=black,line width=0.15pt},
forget plot
]
coordinates{
 (23.3167,42.6833) 
};
\addplot [
color=blue,
solid,
mark=*,mark size=1.3pt,
mark options={solid,fill=mycolor44,draw=black,line width=0.15pt},
forget plot
]
coordinates{
 (24.75,42.150002) 
};
\addplot [
color=blue,
solid,
mark=*,mark size=1.3pt,
mark options={solid,fill=mycolor18,draw=black,line width=0.15pt},
forget plot
]
coordinates{
 (23.3167,42.6833) 
};
\addplot [
color=blue,
solid,
mark=*,mark size=1.3pt,
mark options={solid,fill=mycolor59,draw=black,line width=0.15pt},
forget plot
]
coordinates{
 (24.75,42.150002) 
};
\addplot [
color=blue,
solid,
mark=*,mark size=1.3pt,
mark options={solid,fill=mycolor27,draw=black,line width=0.15pt},
forget plot
]
coordinates{
 (24.75,42.150002) 
};
\addplot [
color=blue,
solid,
mark=*,mark size=1.3pt,
mark options={solid,fill=mycolor86,draw=black,line width=0.15pt},
forget plot
]
coordinates{
 (25.1136,43.025799) 
};
\addplot [
color=blue,
solid,
mark=*,mark size=1.3pt,
mark options={solid,fill=mycolor45,draw=black,line width=0.15pt},
forget plot
]
coordinates{
 (23.3167,42.6833) 
};
\addplot [
color=blue,
solid,
mark=*,mark size=1.3pt,
mark options={solid,fill=mycolor48,draw=black,line width=0.15pt},
forget plot
]
coordinates{
 (23.0333,42.599998) 
};
\addplot [
color=blue,
solid,
mark=*,mark size=1.3pt,
mark options={solid,fill=mycolor25,draw=black,line width=0.15pt},
forget plot
]
coordinates{
 (23.3167,42.6833) 
};
\addplot [
color=blue,
solid,
mark=*,mark size=1.3pt,
mark options={solid,fill=mycolor18,draw=black,line width=0.15pt},
forget plot
]
coordinates{
 (23.766701,42.950001) 
};
\addplot [
color=blue,
solid,
mark=*,mark size=1.3pt,
mark options={solid,fill=mycolor27,draw=black,line width=0.15pt},
forget plot
]
coordinates{
 (26.329201,42.685799) 
};
\addplot [
color=blue,
solid,
mark=*,mark size=1.3pt,
mark options={solid,fill=mycolor46,draw=black,line width=0.15pt},
forget plot
]
coordinates{
 (26.516701,43.533298) 
};
\addplot [
color=blue,
solid,
mark=*,mark size=1.3pt,
mark options={solid,fill=mycolor46,draw=black,line width=0.15pt},
forget plot
]
coordinates{
 (24.75,42.150002) 
};
\addplot [
color=blue,
solid,
mark=*,mark size=1.3pt,
mark options={solid,fill=mycolor18,draw=black,line width=0.15pt},
forget plot
]
coordinates{
 (23.3167,42.6833) 
};
\addplot [
color=blue,
solid,
mark=*,mark size=1.3pt,
mark options={solid,fill=mycolor45,draw=black,line width=0.15pt},
forget plot
]
coordinates{
 (25,43) 
};
\addplot [
color=blue,
solid,
mark=*,mark size=1.3pt,
mark options={solid,fill=mycolor45,draw=black,line width=0.15pt},
forget plot
]
coordinates{
 (24.1833,42.700001) 
};
\addplot [
color=blue,
solid,
mark=*,mark size=1.3pt,
mark options={solid,fill=mycolor46,draw=black,line width=0.15pt},
forget plot
]
coordinates{
 (25.883301,42.9333) 
};
\addplot [
color=blue,
solid,
mark=*,mark size=1.3pt,
mark options={solid,fill=mycolor59,draw=black,line width=0.15pt},
forget plot
]
coordinates{
 (23.3167,42.6833) 
};
\addplot [
color=blue,
solid,
mark=*,mark size=1.3pt,
mark options={solid,fill=mycolor52,draw=black,line width=0.15pt},
forget plot
]
coordinates{
 (50.583099,26.236099) 
};
\addplot [
color=blue,
solid,
mark=*,mark size=1.3pt,
mark options={solid,fill=mycolor68,draw=black,line width=0.15pt},
forget plot
]
coordinates{
 (50.581902,26.150299) 
};
\addplot [
color=blue,
solid,
mark=*,mark size=1.3pt,
mark options={solid,fill=mycolor32,draw=black,line width=0.15pt},
forget plot
]
coordinates{
 (50.583099,26.236099) 
};
\addplot [
color=blue,
solid,
mark=*,mark size=1.3pt,
mark options={solid,fill=mycolor78,draw=black,line width=0.15pt},
forget plot
]
coordinates{
 (50.583099,26.236099) 
};
\addplot [
color=blue,
solid,
mark=*,mark size=1.3pt,
mark options={solid,fill=mycolor68,draw=black,line width=0.15pt},
forget plot
]
coordinates{
 (50.583099,26.236099) 
};
\addplot [
color=blue,
solid,
mark=*,mark size=1.3pt,
mark options={solid,fill=mycolor17,draw=black,line width=0.15pt},
forget plot
]
coordinates{
 (30,-3.5) 
};
\addplot [
color=blue,
solid,
mark=*,mark size=1.3pt,
mark options={solid,fill=mycolor22,draw=black,line width=0.15pt},
forget plot
]
coordinates{
 (30,-3.5) 
};
\addplot [
color=blue,
solid,
mark=*,mark size=1.3pt,
mark options={solid,fill=mycolor79,draw=black,line width=0.15pt},
forget plot
]
coordinates{
 (30,-3.5) 
};
\addplot [
color=blue,
solid,
mark=*,mark size=1.3pt,
mark options={solid,fill=mycolor69,draw=black,line width=0.15pt},
forget plot
]
coordinates{
 (29.360001,-3.3761) 
};
\addplot [
color=blue,
solid,
mark=*,mark size=1.3pt,
mark options={solid,fill=mycolor79,draw=black,line width=0.15pt},
forget plot
]
coordinates{
 (30,-3.5) 
};
\addplot [
color=blue,
solid,
mark=*,mark size=1.3pt,
mark options={solid,fill=mycolor74,draw=black,line width=0.15pt},
forget plot
]
coordinates{
 (2.25,9.5) 
};
\addplot [
color=blue,
solid,
mark=*,mark size=1.3pt,
mark options={solid,fill=mycolor61,draw=black,line width=0.15pt},
forget plot
]
coordinates{
 (-64.783897,32.294201) 
};
\addplot [
color=blue,
solid,
mark=*,mark size=1.3pt,
mark options={solid,fill=mycolor79,draw=black,line width=0.15pt},
forget plot
]
coordinates{
 (-64.783897,32.294201) 
};
\addplot [
color=blue,
solid,
mark=*,mark size=1.3pt,
mark options={solid,fill=mycolor75,draw=black,line width=0.15pt},
forget plot
]
coordinates{
 (-64.783897,32.294201) 
};
\addplot [
color=blue,
solid,
mark=*,mark size=1.3pt,
mark options={solid,fill=mycolor79,draw=black,line width=0.15pt},
forget plot
]
coordinates{
 (-64.783897,32.294201) 
};
\addplot [
color=blue,
solid,
mark=*,mark size=1.3pt,
mark options={solid,fill=mycolor75,draw=black,line width=0.15pt},
forget plot
]
coordinates{
 (-64.783897,32.294201) 
};
\addplot [
color=blue,
solid,
mark=*,mark size=1.3pt,
mark options={solid,fill=mycolor87,draw=black,line width=0.15pt},
forget plot
]
coordinates{
 (-64.783897,32.294201) 
};
\addplot [
color=blue,
solid,
mark=*,mark size=1.3pt,
mark options={solid,fill=mycolor28,draw=black,line width=0.15pt},
forget plot
]
coordinates{
 (-64.783897,32.294201) 
};
\addplot [
color=blue,
solid,
mark=*,mark size=1.3pt,
mark options={solid,fill=mycolor57,draw=black,line width=0.15pt},
forget plot
]
coordinates{
 (114.666702,4.5) 
};
\addplot [
color=blue,
solid,
mark=*,mark size=1.3pt,
mark options={solid,fill=mycolor32,draw=black,line width=0.15pt},
forget plot
]
coordinates{
 (-66.25,-12.2833) 
};
\addplot [
color=blue,
solid,
mark=*,mark size=1.3pt,
mark options={solid,fill=mycolor58,draw=black,line width=0.15pt},
forget plot
]
coordinates{
 (-43.233299,-22.9) 
};
\addplot [
color=blue,
solid,
mark=*,mark size=1.3pt,
mark options={solid,fill=mycolor42,draw=black,line width=0.15pt},
forget plot
]
coordinates{
 (-78.699997,26.5333) 
};
\addplot [
color=blue,
solid,
mark=*,mark size=1.3pt,
mark options={solid,fill=mycolor52,draw=black,line width=0.15pt},
forget plot
]
coordinates{
 (-78.699997,26.5333) 
};
\addplot [
color=blue,
solid,
mark=*,mark size=1.3pt,
mark options={solid,fill=mycolor76,draw=black,line width=0.15pt},
forget plot
]
coordinates{
 (-77.349998,25.0833) 
};
\addplot [
color=blue,
solid,
mark=*,mark size=1.3pt,
mark options={solid,fill=mycolor76,draw=black,line width=0.15pt},
forget plot
]
coordinates{
 (-77.349998,25.0833) 
};
\addplot [
color=blue,
solid,
mark=*,mark size=1.3pt,
mark options={solid,fill=mycolor52,draw=black,line width=0.15pt},
forget plot
]
coordinates{
 (-77.349998,25.0833) 
};
\addplot [
color=blue,
solid,
mark=*,mark size=1.3pt,
mark options={solid,fill=mycolor74,draw=black,line width=0.15pt},
forget plot
]
coordinates{
 (-77.349998,25.0833) 
};
\addplot [
color=blue,
solid,
mark=*,mark size=1.3pt,
mark options={solid,fill=mycolor22,draw=black,line width=0.15pt},
forget plot
]
coordinates{
 (77,20) 
};
\addplot [
color=blue,
solid,
mark=*,mark size=1.3pt,
mark options={solid,fill=mycolor48,draw=black,line width=0.15pt},
forget plot
]
coordinates{
 (30.194401,55.192501) 
};
\addplot [
color=blue,
solid,
mark=*,mark size=1.3pt,
mark options={solid,fill=mycolor48,draw=black,line width=0.15pt},
forget plot
]
coordinates{
 (30.336399,53.913898) 
};
\addplot [
color=blue,
solid,
mark=*,mark size=1.3pt,
mark options={solid,fill=mycolor51,draw=black,line width=0.15pt},
forget plot
]
coordinates{
 (27.5667,53.900002) 
};
\addplot [
color=blue,
solid,
mark=*,mark size=1.3pt,
mark options={solid,fill=mycolor78,draw=black,line width=0.15pt},
forget plot
]
coordinates{
 (-88.183296,17.483299) 
};
\addplot [
color=blue,
solid,
mark=*,mark size=1.3pt,
mark options={solid,fill=mycolor64,draw=black,line width=0.15pt},
forget plot
]
coordinates{
 (-87.949997,17.9167) 
};
\addplot [
color=blue,
solid,
mark=*,mark size=1.3pt,
mark options={solid,fill=mycolor63,draw=black,line width=0.15pt},
forget plot
]
coordinates{
 (-88.183296,17.483299) 
};
\addplot [
color=blue,
solid,
mark=*,mark size=1.3pt,
mark options={solid,fill=mycolor73,draw=black,line width=0.15pt},
forget plot
]
coordinates{
 (-88.75,17.25) 
};
\addplot [
color=blue,
solid,
mark=*,mark size=1.3pt,
mark options={solid,fill=mycolor52,draw=black,line width=0.15pt},
forget plot
]
coordinates{
 (-75.699997,45.416698) 
};
\addplot [
color=blue,
solid,
mark=*,mark size=1.3pt,
mark options={solid,fill=mycolor75,draw=black,line width=0.15pt},
forget plot
]
coordinates{
 (-71.25,46.799999) 
};
\addplot [
color=blue,
solid,
mark=*,mark size=1.3pt,
mark options={solid,fill=mycolor42,draw=black,line width=0.15pt},
forget plot
]
coordinates{
 (-113.5,53.549999) 
};
\addplot [
color=blue,
solid,
mark=*,mark size=1.3pt,
mark options={solid,fill=mycolor28,draw=black,line width=0.15pt},
forget plot
]
coordinates{
 (-72.433296,46.333302) 
};
\addplot [
color=blue,
solid,
mark=*,mark size=1.3pt,
mark options={solid,fill=mycolor52,draw=black,line width=0.15pt},
forget plot
]
coordinates{
 (-72.216698,48.516701) 
};
\addplot [
color=blue,
solid,
mark=*,mark size=1.3pt,
mark options={solid,fill=mycolor58,draw=black,line width=0.15pt},
forget plot
]
coordinates{
 (-114.083298,51.083302) 
};
\addplot [
color=blue,
solid,
mark=*,mark size=1.3pt,
mark options={solid,fill=mycolor70,draw=black,line width=0.15pt},
forget plot
]
coordinates{
 (-79.416702,43.666698) 
};
\addplot [
color=blue,
solid,
mark=*,mark size=1.3pt,
mark options={solid,fill=mycolor65,draw=black,line width=0.15pt},
forget plot
]
coordinates{
 (-73.666702,45.549999) 
};
\addplot [
color=blue,
solid,
mark=*,mark size=1.3pt,
mark options={solid,fill=mycolor68,draw=black,line width=0.15pt},
forget plot
]
coordinates{
 (-122.821297,49.136398) 
};
\addplot [
color=blue,
solid,
mark=*,mark size=1.3pt,
mark options={solid,fill=mycolor74,draw=black,line width=0.15pt},
forget plot
]
coordinates{
 (21,7) 
};
\addplot [
color=blue,
solid,
mark=*,mark size=1.3pt,
mark options={solid,fill=mycolor19,draw=black,line width=0.15pt},
forget plot
]
coordinates{
 (8,47) 
};
\addplot [
color=blue,
solid,
mark=*,mark size=1.3pt,
mark options={solid,fill=mycolor26,draw=black,line width=0.15pt},
forget plot
]
coordinates{
 (7.6325,46.778099) 
};
\addplot [
color=blue,
solid,
mark=*,mark size=1.3pt,
mark options={solid,fill=mycolor43,draw=black,line width=0.15pt},
forget plot
]
coordinates{
 (7.5733,47.558399) 
};
\addplot [
color=blue,
solid,
mark=*,mark size=1.3pt,
mark options={solid,fill=mycolor60,draw=black,line width=0.15pt},
forget plot
]
coordinates{
 (7.4667,46.916698) 
};
\addplot [
color=blue,
solid,
mark=*,mark size=1.3pt,
mark options={solid,fill=mycolor14,draw=black,line width=0.15pt},
forget plot
]
coordinates{
 (9.3426,47.092499) 
};
\addplot [
color=blue,
solid,
mark=*,mark size=1.3pt,
mark options={solid,fill=mycolor14,draw=black,line width=0.15pt},
forget plot
]
coordinates{
 (8,47) 
};
\addplot [
color=blue,
solid,
mark=*,mark size=1.3pt,
mark options={solid,fill=mycolor74,draw=black,line width=0.15pt},
forget plot
]
coordinates{
 (-4.0127,5.3097) 
};
\addplot [
color=blue,
solid,
mark=*,mark size=1.3pt,
mark options={solid,fill=mycolor81,draw=black,line width=0.15pt},
forget plot
]
coordinates{
 (-70.304199,-18.475) 
};
\addplot [
color=blue,
solid,
mark=*,mark size=1.3pt,
mark options={solid,fill=mycolor16,draw=black,line width=0.15pt},
forget plot
]
coordinates{
 (-74.062798,4.6492) 
};
\addplot [
color=blue,
solid,
mark=*,mark size=1.3pt,
mark options={solid,fill=mycolor82,draw=black,line width=0.15pt},
forget plot
]
coordinates{
 (-74.062798,4.6492) 
};
\addplot [
color=blue,
solid,
mark=*,mark size=1.3pt,
mark options={solid,fill=mycolor63,draw=black,line width=0.15pt},
forget plot
]
coordinates{
 (-72,4) 
};
\addplot [
color=blue,
solid,
mark=*,mark size=1.3pt,
mark options={solid,fill=mycolor80,draw=black,line width=0.15pt},
forget plot
]
coordinates{
 (-74.062798,4.6492) 
};
\addplot [
color=blue,
solid,
mark=*,mark size=1.3pt,
mark options={solid,fill=mycolor63,draw=black,line width=0.15pt},
forget plot
]
coordinates{
 (-84.083298,9.9333) 
};
\addplot [
color=blue,
solid,
mark=*,mark size=1.3pt,
mark options={solid,fill=mycolor80,draw=black,line width=0.15pt},
forget plot
]
coordinates{
 (-84.083298,9.9333) 
};
\addplot [
color=blue,
solid,
mark=*,mark size=1.3pt,
mark options={solid,fill=mycolor74,draw=black,line width=0.15pt},
forget plot
]
coordinates{
 (-84,10) 
};
\addplot [
color=blue,
solid,
mark=*,mark size=1.3pt,
mark options={solid,fill=mycolor63,draw=black,line width=0.15pt},
forget plot
]
coordinates{
 (-84.083298,9.9333) 
};
\addplot [
color=blue,
solid,
mark=*,mark size=1.3pt,
mark options={solid,fill=mycolor44,draw=black,line width=0.15pt},
forget plot
]
coordinates{
 (-23.516701,14.9167) 
};
\addplot [
color=blue,
solid,
mark=*,mark size=1.3pt,
mark options={solid,fill=mycolor27,draw=black,line width=0.15pt},
forget plot
]
coordinates{
 (-23.516701,14.9167) 
};
\addplot [
color=blue,
solid,
mark=*,mark size=1.3pt,
mark options={solid,fill=mycolor44,draw=black,line width=0.15pt},
forget plot
]
coordinates{
 (33,35) 
};
\addplot [
color=blue,
solid,
mark=*,mark size=1.3pt,
mark options={solid,fill=mycolor52,draw=black,line width=0.15pt},
forget plot
]
coordinates{
 (33.366699,35.166698) 
};
\addplot [
color=blue,
solid,
mark=*,mark size=1.3pt,
mark options={solid,fill=mycolor27,draw=black,line width=0.15pt},
forget plot
]
coordinates{
 (33,35) 
};
\addplot [
color=blue,
solid,
mark=*,mark size=1.3pt,
mark options={solid,fill=mycolor27,draw=black,line width=0.15pt},
forget plot
]
coordinates{
 (33.983299,35.037498) 
};
\addplot [
color=blue,
solid,
mark=*,mark size=1.3pt,
mark options={solid,fill=mycolor27,draw=black,line width=0.15pt},
forget plot
]
coordinates{
 (33,35) 
};
\addplot [
color=blue,
solid,
mark=*,mark size=1.3pt,
mark options={solid,fill=mycolor65,draw=black,line width=0.15pt},
forget plot
]
coordinates{
 (22,39) 
};
\addplot [
color=blue,
solid,
mark=*,mark size=1.3pt,
mark options={solid,fill=mycolor44,draw=black,line width=0.15pt},
forget plot
]
coordinates{
 (23.733299,37.983299) 
};
\addplot [
color=blue,
solid,
mark=*,mark size=1.3pt,
mark options={solid,fill=mycolor27,draw=black,line width=0.15pt},
forget plot
]
coordinates{
 (33.366699,35.166698) 
};
\addplot [
color=blue,
solid,
mark=*,mark size=1.3pt,
mark options={solid,fill=mycolor30,draw=black,line width=0.15pt},
forget plot
]
coordinates{
 (33.366699,35.166698) 
};
\addplot [
color=blue,
solid,
mark=*,mark size=1.3pt,
mark options={solid,fill=mycolor44,draw=black,line width=0.15pt},
forget plot
]
coordinates{
 (33.6292,34.916698) 
};
\addplot [
color=blue,
solid,
mark=*,mark size=1.3pt,
mark options={solid,fill=mycolor28,draw=black,line width=0.15pt},
forget plot
]
coordinates{
 (33.033298,34.674999) 
};
\addplot [
color=blue,
solid,
mark=*,mark size=1.3pt,
mark options={solid,fill=mycolor44,draw=black,line width=0.15pt},
forget plot
]
coordinates{
 (33.366699,35.166698) 
};
\addplot [
color=blue,
solid,
mark=*,mark size=1.3pt,
mark options={solid,fill=mycolor73,draw=black,line width=0.15pt},
forget plot
]
coordinates{
 (33,35) 
};
\addplot [
color=blue,
solid,
mark=*,mark size=1.3pt,
mark options={solid,fill=mycolor59,draw=black,line width=0.15pt},
forget plot
]
coordinates{
 (33,35) 
};
\addplot [
color=blue,
solid,
mark=*,mark size=1.3pt,
mark options={solid,fill=mycolor44,draw=black,line width=0.15pt},
forget plot
]
coordinates{
 (33.366699,35.166698) 
};
\addplot [
color=blue,
solid,
mark=*,mark size=1.3pt,
mark options={solid,fill=mycolor66,draw=black,line width=0.15pt},
forget plot
]
coordinates{
 (37.767502,44.7244) 
};
\addplot [
color=blue,
solid,
mark=*,mark size=1.3pt,
mark options={solid,fill=mycolor76,draw=black,line width=0.15pt},
forget plot
]
coordinates{
 (32.416698,34.766701) 
};
\addplot [
color=blue,
solid,
mark=*,mark size=1.3pt,
mark options={solid,fill=mycolor27,draw=black,line width=0.15pt},
forget plot
]
coordinates{
 (33.366699,35.166698) 
};
\addplot [
color=blue,
solid,
mark=*,mark size=1.3pt,
mark options={solid,fill=mycolor65,draw=black,line width=0.15pt},
forget plot
]
coordinates{
 (33.366699,35.166698) 
};
\addplot [
color=blue,
solid,
mark=*,mark size=1.3pt,
mark options={solid,fill=mycolor29,draw=black,line width=0.15pt},
forget plot
]
coordinates{
 (33,35) 
};
\addplot [
color=blue,
solid,
mark=*,mark size=1.3pt,
mark options={solid,fill=mycolor30,draw=black,line width=0.15pt},
forget plot
]
coordinates{
 (33.366699,35.166698) 
};
\addplot [
color=blue,
solid,
mark=*,mark size=1.3pt,
mark options={solid,fill=mycolor66,draw=black,line width=0.15pt},
forget plot
]
coordinates{
 (33.033298,34.674999) 
};
\addplot [
color=blue,
solid,
mark=*,mark size=1.3pt,
mark options={solid,fill=mycolor49,draw=black,line width=0.15pt},
forget plot
]
coordinates{
 (33,35) 
};
\addplot [
color=blue,
solid,
mark=*,mark size=1.3pt,
mark options={solid,fill=mycolor25,draw=black,line width=0.15pt},
forget plot
]
coordinates{
 (13.6391,50.509602) 
};
\addplot [
color=blue,
solid,
mark=*,mark size=1.3pt,
mark options={solid,fill=mycolor46,draw=black,line width=0.15pt},
forget plot
]
coordinates{
 (16.7258,49.165401) 
};
\addplot [
color=blue,
solid,
mark=*,mark size=1.3pt,
mark options={solid,fill=mycolor14,draw=black,line width=0.15pt},
forget plot
]
coordinates{
 (16.814199,48.856998) 
};
\addplot [
color=blue,
solid,
mark=*,mark size=1.3pt,
mark options={solid,fill=mycolor15,draw=black,line width=0.15pt},
forget plot
]
coordinates{
 (16.633301,49.200001) 
};
\addplot [
color=blue,
solid,
mark=*,mark size=1.3pt,
mark options={solid,fill=mycolor18,draw=black,line width=0.15pt},
forget plot
]
coordinates{
 (16.5648,49.281399) 
};
\addplot [
color=blue,
solid,
mark=*,mark size=1.3pt,
mark options={solid,fill=mycolor25,draw=black,line width=0.15pt},
forget plot
]
coordinates{
 (18.255301,49.811901) 
};
\addplot [
color=blue,
solid,
mark=*,mark size=1.3pt,
mark options={solid,fill=mycolor26,draw=black,line width=0.15pt},
forget plot
]
coordinates{
 (15.5646,49.614101) 
};
\addplot [
color=blue,
solid,
mark=*,mark size=1.3pt,
mark options={solid,fill=mycolor15,draw=black,line width=0.15pt},
forget plot
]
coordinates{
 (15.5,49.75) 
};
\addplot [
color=blue,
solid,
mark=*,mark size=1.3pt,
mark options={solid,fill=mycolor85,draw=black,line width=0.15pt},
forget plot
]
coordinates{
 (18.3333,49.133301) 
};
\addplot [
color=blue,
solid,
mark=*,mark size=1.3pt,
mark options={solid,fill=mycolor46,draw=black,line width=0.15pt},
forget plot
]
coordinates{
 (10.1333,54.333302) 
};
\addplot [
color=blue,
solid,
mark=*,mark size=1.3pt,
mark options={solid,fill=mycolor14,draw=black,line width=0.15pt},
forget plot
]
coordinates{
 (7.05,51.516701) 
};
\addplot [
color=blue,
solid,
mark=*,mark size=1.3pt,
mark options={solid,fill=mycolor14,draw=black,line width=0.15pt},
forget plot
]
coordinates{
 (6.1833,51.650002) 
};
\addplot [
color=blue,
solid,
mark=*,mark size=1.3pt,
mark options={solid,fill=mycolor19,draw=black,line width=0.15pt},
forget plot
]
coordinates{
 (6.95,50.9333) 
};
\addplot [
color=blue,
solid,
mark=*,mark size=1.3pt,
mark options={solid,fill=mycolor15,draw=black,line width=0.15pt},
forget plot
]
coordinates{
 (9.1833,48.766701) 
};
\addplot [
color=blue,
solid,
mark=*,mark size=1.3pt,
mark options={solid,fill=mycolor43,draw=black,line width=0.15pt},
forget plot
]
coordinates{
 (6.1833,51.650002) 
};
\addplot [
color=blue,
solid,
mark=*,mark size=1.3pt,
mark options={solid,fill=mycolor45,draw=black,line width=0.15pt},
forget plot
]
coordinates{
 (8.7667,51.716702) 
};
\addplot [
color=blue,
solid,
mark=*,mark size=1.3pt,
mark options={solid,fill=mycolor48,draw=black,line width=0.15pt},
forget plot
]
coordinates{
 (11.5833,48.150002) 
};
\addplot [
color=blue,
solid,
mark=*,mark size=1.3pt,
mark options={solid,fill=mycolor26,draw=black,line width=0.15pt},
forget plot
]
coordinates{
 (13.4,52.516701) 
};
\addplot [
color=blue,
solid,
mark=*,mark size=1.3pt,
mark options={solid,fill=mycolor14,draw=black,line width=0.15pt},
forget plot
]
coordinates{
 (8.2167,50.75) 
};
\addplot [
color=blue,
solid,
mark=*,mark size=1.3pt,
mark options={solid,fill=mycolor18,draw=black,line width=0.15pt},
forget plot
]
coordinates{
 (9,51) 
};
\addplot [
color=blue,
solid,
mark=*,mark size=1.3pt,
mark options={solid,fill=mycolor14,draw=black,line width=0.15pt},
forget plot
]
coordinates{
 (7.2167,51.549999) 
};
\addplot [
color=blue,
solid,
mark=*,mark size=1.3pt,
mark options={solid,fill=mycolor25,draw=black,line width=0.15pt},
forget plot
]
coordinates{
 (9.5,51.3167) 
};
\addplot [
color=blue,
solid,
mark=*,mark size=1.3pt,
mark options={solid,fill=mycolor49,draw=black,line width=0.15pt},
forget plot
]
coordinates{
 (9,51) 
};
\addplot [
color=blue,
solid,
mark=*,mark size=1.3pt,
mark options={solid,fill=mycolor25,draw=black,line width=0.15pt},
forget plot
]
coordinates{
 (7.4,51.700001) 
};
\addplot [
color=blue,
solid,
mark=*,mark size=1.3pt,
mark options={solid,fill=mycolor15,draw=black,line width=0.15pt},
forget plot
]
coordinates{
 (7.6333,51.966702) 
};
\addplot [
color=blue,
solid,
mark=*,mark size=1.3pt,
mark options={solid,fill=mycolor18,draw=black,line width=0.15pt},
forget plot
]
coordinates{
 (9.7617,55.575802) 
};
\addplot [
color=blue,
solid,
mark=*,mark size=1.3pt,
mark options={solid,fill=mycolor46,draw=black,line width=0.15pt},
forget plot
]
coordinates{
 (10.2255,56.1534) 
};
\addplot [
color=blue,
solid,
mark=*,mark size=1.3pt,
mark options={solid,fill=mycolor45,draw=black,line width=0.15pt},
forget plot
]
coordinates{
 (10.3949,55.393501) 
};
\addplot [
color=blue,
solid,
mark=*,mark size=1.3pt,
mark options={solid,fill=mycolor43,draw=black,line width=0.15pt},
forget plot
]
coordinates{
 (10,56) 
};
\addplot [
color=blue,
solid,
mark=*,mark size=1.3pt,
mark options={solid,fill=mycolor46,draw=black,line width=0.15pt},
forget plot
]
coordinates{
 (10.3949,55.393501) 
};
\addplot [
color=blue,
solid,
mark=*,mark size=1.3pt,
mark options={solid,fill=mycolor45,draw=black,line width=0.15pt},
forget plot
]
coordinates{
 (12.3167,55.9333) 
};
\addplot [
color=blue,
solid,
mark=*,mark size=1.3pt,
mark options={solid,fill=mycolor46,draw=black,line width=0.15pt},
forget plot
]
coordinates{
 (12.3281,55.618401) 
};
\addplot [
color=blue,
solid,
mark=*,mark size=1.3pt,
mark options={solid,fill=mycolor26,draw=black,line width=0.15pt},
forget plot
]
coordinates{
 (10.2929,57.1604) 
};
\addplot [
color=blue,
solid,
mark=*,mark size=1.3pt,
mark options={solid,fill=mycolor51,draw=black,line width=0.15pt},
forget plot
]
coordinates{
 (12.5918,55.7537) 
};
\addplot [
color=blue,
solid,
mark=*,mark size=1.3pt,
mark options={solid,fill=mycolor48,draw=black,line width=0.15pt},
forget plot
]
coordinates{
 (12.2258,55.572601) 
};
\addplot [
color=blue,
solid,
mark=*,mark size=1.3pt,
mark options={solid,fill=mycolor45,draw=black,line width=0.15pt},
forget plot
]
coordinates{
 (10.0413,55.373199) 
};
\addplot [
color=blue,
solid,
mark=*,mark size=1.3pt,
mark options={solid,fill=mycolor45,draw=black,line width=0.15pt},
forget plot
]
coordinates{
 (12.4871,55.683601) 
};
\addplot [
color=blue,
solid,
mark=*,mark size=1.3pt,
mark options={solid,fill=mycolor80,draw=black,line width=0.15pt},
forget plot
]
coordinates{
 (-61.400002,15.3) 
};
\addplot [
color=blue,
solid,
mark=*,mark size=1.3pt,
mark options={solid,fill=mycolor32,draw=black,line width=0.15pt},
forget plot
]
coordinates{
 (-61.400002,15.3) 
};
\addplot [
color=blue,
solid,
mark=*,mark size=1.3pt,
mark options={solid,fill=mycolor27,draw=black,line width=0.15pt},
forget plot
]
coordinates{
 (-61.400002,15.3) 
};
\addplot [
color=blue,
solid,
mark=*,mark size=1.3pt,
mark options={solid,fill=mycolor80,draw=black,line width=0.15pt},
forget plot
]
coordinates{
 (-61.400002,15.3) 
};
\addplot [
color=blue,
solid,
mark=*,mark size=1.3pt,
mark options={solid,fill=mycolor65,draw=black,line width=0.15pt},
forget plot
]
coordinates{
 (-74.005997,40.714298) 
};
\addplot [
color=blue,
solid,
mark=*,mark size=1.3pt,
mark options={solid,fill=mycolor17,draw=black,line width=0.15pt},
forget plot
]
coordinates{
 (-61.400002,15.3) 
};
\addplot [
color=blue,
solid,
mark=*,mark size=1.3pt,
mark options={solid,fill=mycolor68,draw=black,line width=0.15pt},
forget plot
]
coordinates{
 (-61.400002,15.3) 
};
\addplot [
color=blue,
solid,
mark=*,mark size=1.3pt,
mark options={solid,fill=mycolor54,draw=black,line width=0.15pt},
forget plot
]
coordinates{
 (-61.400002,15.3) 
};
\addplot [
color=blue,
solid,
mark=*,mark size=1.3pt,
mark options={solid,fill=mycolor27,draw=black,line width=0.15pt},
forget plot
]
coordinates{
 (-61.400002,15.3) 
};
\addplot [
color=blue,
solid,
mark=*,mark size=1.3pt,
mark options={solid,fill=mycolor79,draw=black,line width=0.15pt},
forget plot
]
coordinates{
 (-61.400002,15.3) 
};
\addplot [
color=blue,
solid,
mark=*,mark size=1.3pt,
mark options={solid,fill=mycolor65,draw=black,line width=0.15pt},
forget plot
]
coordinates{
 (-61.400002,15.3) 
};
\addplot [
color=blue,
solid,
mark=*,mark size=1.3pt,
mark options={solid,fill=mycolor27,draw=black,line width=0.15pt},
forget plot
]
coordinates{
 (-61.400002,15.3) 
};
\addplot [
color=blue,
solid,
mark=*,mark size=1.3pt,
mark options={solid,fill=mycolor80,draw=black,line width=0.15pt},
forget plot
]
coordinates{
 (-61.400002,15.3) 
};
\addplot [
color=blue,
solid,
mark=*,mark size=1.3pt,
mark options={solid,fill=mycolor42,draw=black,line width=0.15pt},
forget plot
]
coordinates{
 (-61.400002,15.3) 
};
\addplot [
color=blue,
solid,
mark=*,mark size=1.3pt,
mark options={solid,fill=mycolor44,draw=black,line width=0.15pt},
forget plot
]
coordinates{
 (-61.400002,15.3) 
};
\addplot [
color=blue,
solid,
mark=*,mark size=1.3pt,
mark options={solid,fill=mycolor65,draw=black,line width=0.15pt},
forget plot
]
coordinates{
 (-61.400002,15.3) 
};
\addplot [
color=blue,
solid,
mark=*,mark size=1.3pt,
mark options={solid,fill=mycolor27,draw=black,line width=0.15pt},
forget plot
]
coordinates{
 (-61.400002,15.3) 
};
\addplot [
color=blue,
solid,
mark=*,mark size=1.3pt,
mark options={solid,fill=mycolor42,draw=black,line width=0.15pt},
forget plot
]
coordinates{
 (-61.400002,15.3) 
};
\addplot [
color=blue,
solid,
mark=*,mark size=1.3pt,
mark options={solid,fill=mycolor79,draw=black,line width=0.15pt},
forget plot
]
coordinates{
 (-61.400002,15.3) 
};
\addplot [
color=blue,
solid,
mark=*,mark size=1.3pt,
mark options={solid,fill=mycolor78,draw=black,line width=0.15pt},
forget plot
]
coordinates{
 (2,46) 
};
\addplot [
color=blue,
solid,
mark=*,mark size=1.3pt,
mark options={solid,fill=mycolor27,draw=black,line width=0.15pt},
forget plot
]
coordinates{
 (3.0506,36.7631) 
};
\addplot [
color=blue,
solid,
mark=*,mark size=1.3pt,
mark options={solid,fill=mycolor16,draw=black,line width=0.15pt},
forget plot
]
coordinates{
 (-78.5,-0.2167) 
};
\addplot [
color=blue,
solid,
mark=*,mark size=1.3pt,
mark options={solid,fill=mycolor54,draw=black,line width=0.15pt},
forget plot
]
coordinates{
 (-78.5,-0.2167) 
};
\addplot [
color=blue,
solid,
mark=*,mark size=1.3pt,
mark options={solid,fill=mycolor79,draw=black,line width=0.15pt},
forget plot
]
coordinates{
 (-78.983299,-2.8833) 
};
\addplot [
color=blue,
solid,
mark=*,mark size=1.3pt,
mark options={solid,fill=mycolor21,draw=black,line width=0.15pt},
forget plot
]
coordinates{
 (-78.5,-0.2167) 
};
\addplot [
color=blue,
solid,
mark=*,mark size=1.3pt,
mark options={solid,fill=mycolor42,draw=black,line width=0.15pt},
forget plot
]
coordinates{
 (-79.900002,-2.1667) 
};
\addplot [
color=blue,
solid,
mark=*,mark size=1.3pt,
mark options={solid,fill=mycolor48,draw=black,line width=0.15pt},
forget plot
]
coordinates{
 (-83.1763,42.3223) 
};
\addplot [
color=blue,
solid,
mark=*,mark size=1.3pt,
mark options={solid,fill=mycolor46,draw=black,line width=0.15pt},
forget plot
]
coordinates{
 (26,59) 
};
\addplot [
color=blue,
solid,
mark=*,mark size=1.3pt,
mark options={solid,fill=mycolor50,draw=black,line width=0.15pt},
forget plot
]
coordinates{
 (24.7281,59.433899) 
};
\addplot [
color=blue,
solid,
mark=*,mark size=1.3pt,
mark options={solid,fill=mycolor45,draw=black,line width=0.15pt},
forget plot
]
coordinates{
 (24.7281,59.433899) 
};
\addplot [
color=blue,
solid,
mark=*,mark size=1.3pt,
mark options={solid,fill=mycolor24,draw=black,line width=0.15pt},
forget plot
]
coordinates{
 (-110.360703,31.5273) 
};
\addplot [
color=blue,
solid,
mark=*,mark size=1.3pt,
mark options={solid,fill=mycolor24,draw=black,line width=0.15pt},
forget plot
]
coordinates{
 (114.273399,30.580099) 
};
\addplot [
color=blue,
solid,
mark=*,mark size=1.3pt,
mark options={solid,fill=mycolor46,draw=black,line width=0.15pt},
forget plot
]
coordinates{
 (4.9889,47.270901) 
};
\addplot [
color=blue,
solid,
mark=*,mark size=1.3pt,
mark options={solid,fill=mycolor18,draw=black,line width=0.15pt},
forget plot
]
coordinates{
 (26.355801,59.346401) 
};
\addplot [
color=blue,
solid,
mark=*,mark size=1.3pt,
mark options={solid,fill=mycolor51,draw=black,line width=0.15pt},
forget plot
]
coordinates{
 (-77.341103,38.9687) 
};
\addplot [
color=blue,
solid,
mark=*,mark size=1.3pt,
mark options={solid,fill=mycolor18,draw=black,line width=0.15pt},
forget plot
]
coordinates{
 (25.59,58.363899) 
};
\addplot [
color=blue,
solid,
mark=*,mark size=1.3pt,
mark options={solid,fill=mycolor76,draw=black,line width=0.15pt},
forget plot
]
coordinates{
 (31.25,30.049999) 
};
\addplot [
color=blue,
solid,
mark=*,mark size=1.3pt,
mark options={solid,fill=mycolor5,draw=black,line width=0.15pt},
forget plot
]
coordinates{
 (30,27) 
};
\addplot [
color=blue,
solid,
mark=*,mark size=1.3pt,
mark options={solid,fill=mycolor24,draw=black,line width=0.15pt},
forget plot
]
coordinates{
 (135.7556,35.0214) 
};
\addplot [
color=blue,
solid,
mark=*,mark size=1.3pt,
mark options={solid,fill=mycolor65,draw=black,line width=0.15pt},
forget plot
]
coordinates{
 (-4.7727,37.891602) 
};
\addplot [
color=blue,
solid,
mark=*,mark size=1.3pt,
mark options={solid,fill=mycolor73,draw=black,line width=0.15pt},
forget plot
]
coordinates{
 (-16.3727,28.354799) 
};
\addplot [
color=blue,
solid,
mark=*,mark size=1.3pt,
mark options={solid,fill=mycolor48,draw=black,line width=0.15pt},
forget plot
]
coordinates{
 (-3.8044,43.464699) 
};
\addplot [
color=blue,
solid,
mark=*,mark size=1.3pt,
mark options={solid,fill=mycolor48,draw=black,line width=0.15pt},
forget plot
]
coordinates{
 (2.1094,41.543301) 
};
\addplot [
color=blue,
solid,
mark=*,mark size=1.3pt,
mark options={solid,fill=mycolor85,draw=black,line width=0.15pt},
forget plot
]
coordinates{
 (2.1741,41.398399) 
};
\addplot [
color=blue,
solid,
mark=*,mark size=1.3pt,
mark options={solid,fill=mycolor86,draw=black,line width=0.15pt},
forget plot
]
coordinates{
 (-3.1618,40.628601) 
};
\addplot [
color=blue,
solid,
mark=*,mark size=1.3pt,
mark options={solid,fill=mycolor45,draw=black,line width=0.15pt},
forget plot
]
coordinates{
 (-3.6922,40.4086) 
};
\addplot [
color=blue,
solid,
mark=*,mark size=1.3pt,
mark options={solid,fill=mycolor59,draw=black,line width=0.15pt},
forget plot
]
coordinates{
 (1.8941,41.543701) 
};
\addplot [
color=blue,
solid,
mark=*,mark size=1.3pt,
mark options={solid,fill=mycolor59,draw=black,line width=0.15pt},
forget plot
]
coordinates{
 (-15.4134,28.099701) 
};
\addplot [
color=blue,
solid,
mark=*,mark size=1.3pt,
mark options={solid,fill=mycolor66,draw=black,line width=0.15pt},
forget plot
]
coordinates{
 (-1.1333,52.633301) 
};
\addplot [
color=blue,
solid,
mark=*,mark size=1.3pt,
mark options={solid,fill=mycolor18,draw=black,line width=0.15pt},
forget plot
]
coordinates{
 (-4,40) 
};
\addplot [
color=blue,
solid,
mark=*,mark size=1.3pt,
mark options={solid,fill=mycolor25,draw=black,line width=0.15pt},
forget plot
]
coordinates{
 (-4,40) 
};
\addplot [
color=blue,
solid,
mark=*,mark size=1.3pt,
mark options={solid,fill=mycolor52,draw=black,line width=0.15pt},
forget plot
]
coordinates{
 (38,8) 
};
\addplot [
color=blue,
solid,
mark=*,mark size=1.3pt,
mark options={solid,fill=mycolor48,draw=black,line width=0.15pt},
forget plot
]
coordinates{
 (24.85,60.299999) 
};
\addplot [
color=blue,
solid,
mark=*,mark size=1.3pt,
mark options={solid,fill=mycolor46,draw=black,line width=0.15pt},
forget plot
]
coordinates{
 (23.75,61.5) 
};
\addplot [
color=blue,
solid,
mark=*,mark size=1.3pt,
mark options={solid,fill=mycolor50,draw=black,line width=0.15pt},
forget plot
]
coordinates{
 (25.7167,62.283298) 
};
\addplot [
color=blue,
solid,
mark=*,mark size=1.3pt,
mark options={solid,fill=mycolor48,draw=black,line width=0.15pt},
forget plot
]
coordinates{
 (24.6667,60.216702) 
};
\addplot [
color=blue,
solid,
mark=*,mark size=1.3pt,
mark options={solid,fill=mycolor11,draw=black,line width=0.15pt},
forget plot
]
coordinates{
 (158.25,6.9167) 
};
\addplot [
color=blue,
solid,
mark=*,mark size=1.3pt,
mark options={solid,fill=mycolor37,draw=black,line width=0.15pt},
forget plot
]
coordinates{
 (158.25,6.9167) 
};
\addplot [
color=blue,
solid,
mark=*,mark size=1.3pt,
mark options={solid,fill=mycolor11,draw=black,line width=0.15pt},
forget plot
]
coordinates{
 (158.25,6.9167) 
};
\addplot [
color=blue,
solid,
mark=*,mark size=1.3pt,
mark options={solid,fill=mycolor37,draw=black,line width=0.15pt},
forget plot
]
coordinates{
 (158.25,6.9167) 
};
\addplot [
color=blue,
solid,
mark=*,mark size=1.3pt,
mark options={solid,fill=mycolor10,draw=black,line width=0.15pt},
forget plot
]
coordinates{
 (158.25,6.9167) 
};
\addplot [
color=blue,
solid,
mark=*,mark size=1.3pt,
mark options={solid,fill=mycolor57,draw=black,line width=0.15pt},
forget plot
]
coordinates{
 (158.25,6.9167) 
};
\addplot [
color=blue,
solid,
mark=*,mark size=1.3pt,
mark options={solid,fill=mycolor37,draw=black,line width=0.15pt},
forget plot
]
coordinates{
 (158.25,6.9167) 
};
\addplot [
color=blue,
solid,
mark=*,mark size=1.3pt,
mark options={solid,fill=mycolor48,draw=black,line width=0.15pt},
forget plot
]
coordinates{
 (-7.3,62.0667) 
};
\addplot [
color=blue,
solid,
mark=*,mark size=1.3pt,
mark options={solid,fill=mycolor85,draw=black,line width=0.15pt},
forget plot
]
coordinates{
 (-6.7667,62.016701) 
};
\addplot [
color=blue,
solid,
mark=*,mark size=1.3pt,
mark options={solid,fill=mycolor45,draw=black,line width=0.15pt},
forget plot
]
coordinates{
 (-6.7667,62.016701) 
};
\addplot [
color=blue,
solid,
mark=*,mark size=1.3pt,
mark options={solid,fill=mycolor24,draw=black,line width=0.15pt},
forget plot
]
coordinates{
 (-6.7667,62.016701) 
};
\addplot [
color=blue,
solid,
mark=*,mark size=1.3pt,
mark options={solid,fill=mycolor66,draw=black,line width=0.15pt},
forget plot
]
coordinates{
 (-6.7667,62.016701) 
};
\addplot [
color=blue,
solid,
mark=*,mark size=1.3pt,
mark options={solid,fill=mycolor45,draw=black,line width=0.15pt},
forget plot
]
coordinates{
 (-6.6833,62.133301) 
};
\addplot [
color=blue,
solid,
mark=*,mark size=1.3pt,
mark options={solid,fill=mycolor50,draw=black,line width=0.15pt},
forget plot
]
coordinates{
 (-6.7667,62.016701) 
};
\addplot [
color=blue,
solid,
mark=*,mark size=1.3pt,
mark options={solid,fill=mycolor48,draw=black,line width=0.15pt},
forget plot
]
coordinates{
 (-6.7667,62.016701) 
};
\addplot [
color=blue,
solid,
mark=*,mark size=1.3pt,
mark options={solid,fill=mycolor66,draw=black,line width=0.15pt},
forget plot
]
coordinates{
 (-51.950001,-22.983299) 
};
\addplot [
color=blue,
solid,
mark=*,mark size=1.3pt,
mark options={solid,fill=mycolor48,draw=black,line width=0.15pt},
forget plot
]
coordinates{
 (2,46) 
};
\addplot [
color=blue,
solid,
mark=*,mark size=1.3pt,
mark options={solid,fill=mycolor85,draw=black,line width=0.15pt},
forget plot
]
coordinates{
 (138,36) 
};
\addplot [
color=blue,
solid,
mark=*,mark size=1.3pt,
mark options={solid,fill=mycolor48,draw=black,line width=0.15pt},
forget plot
]
coordinates{
 (-86.526398,39.165298) 
};
\addplot [
color=blue,
solid,
mark=*,mark size=1.3pt,
mark options={solid,fill=mycolor28,draw=black,line width=0.15pt},
forget plot
]
coordinates{
 (2.4432,48.864201) 
};
\addplot [
color=blue,
solid,
mark=*,mark size=1.3pt,
mark options={solid,fill=mycolor26,draw=black,line width=0.15pt},
forget plot
]
coordinates{
 (4.267,45.388199) 
};
\addplot [
color=blue,
solid,
mark=*,mark size=1.3pt,
mark options={solid,fill=mycolor85,draw=black,line width=0.15pt},
forget plot
]
coordinates{
 (-82.998802,39.961201) 
};
\addplot [
color=blue,
solid,
mark=*,mark size=1.3pt,
mark options={solid,fill=mycolor50,draw=black,line width=0.15pt},
forget plot
]
coordinates{
 (-75.900002,45.333302) 
};
\addplot [
color=blue,
solid,
mark=*,mark size=1.3pt,
mark options={solid,fill=mycolor73,draw=black,line width=0.15pt},
forget plot
]
coordinates{
 (2.3366,48.796299) 
};
\addplot [
color=blue,
solid,
mark=*,mark size=1.3pt,
mark options={solid,fill=mycolor50,draw=black,line width=0.15pt},
forget plot
]
coordinates{
 (-2,54) 
};
\addplot [
color=blue,
solid,
mark=*,mark size=1.3pt,
mark options={solid,fill=mycolor45,draw=black,line width=0.15pt},
forget plot
]
coordinates{
 (2.3333,48.866699) 
};
\addplot [
color=blue,
solid,
mark=*,mark size=1.3pt,
mark options={solid,fill=mycolor46,draw=black,line width=0.15pt},
forget plot
]
coordinates{
 (2,46) 
};
\addplot [
color=blue,
solid,
mark=*,mark size=1.3pt,
mark options={solid,fill=mycolor81,draw=black,line width=0.15pt},
forget plot
]
coordinates{
 (55.466702,-20.866699) 
};
\addplot [
color=blue,
solid,
mark=*,mark size=1.3pt,
mark options={solid,fill=mycolor68,draw=black,line width=0.15pt},
forget plot
]
coordinates{
 (2.3833,48.950001) 
};
\addplot [
color=blue,
solid,
mark=*,mark size=1.3pt,
mark options={solid,fill=mycolor24,draw=black,line width=0.15pt},
forget plot
]
coordinates{
 (-97,38) 
};
\addplot [
color=blue,
solid,
mark=*,mark size=1.3pt,
mark options={solid,fill=mycolor15,draw=black,line width=0.15pt},
forget plot
]
coordinates{
 (-0.0931,51.514198) 
};
\addplot [
color=blue,
solid,
mark=*,mark size=1.3pt,
mark options={solid,fill=mycolor43,draw=black,line width=0.15pt},
forget plot
]
coordinates{
 (-2,54) 
};
\addplot [
color=blue,
solid,
mark=*,mark size=1.3pt,
mark options={solid,fill=mycolor42,draw=black,line width=0.15pt},
forget plot
]
coordinates{
 (-61.75,12.05) 
};
\addplot [
color=blue,
solid,
mark=*,mark size=1.3pt,
mark options={solid,fill=mycolor64,draw=black,line width=0.15pt},
forget plot
]
coordinates{
 (-61.666698,12.1167) 
};
\addplot [
color=blue,
solid,
mark=*,mark size=1.3pt,
mark options={solid,fill=mycolor63,draw=black,line width=0.15pt},
forget plot
]
coordinates{
 (-61.75,12.05) 
};
\addplot [
color=blue,
solid,
mark=*,mark size=1.3pt,
mark options={solid,fill=mycolor48,draw=black,line width=0.15pt},
forget plot
]
coordinates{
 (44.790798,41.724998) 
};
\addplot [
color=blue,
solid,
mark=*,mark size=1.3pt,
mark options={solid,fill=mycolor65,draw=black,line width=0.15pt},
forget plot
]
coordinates{
 (44.790798,41.724998) 
};
\addplot [
color=blue,
solid,
mark=*,mark size=1.3pt,
mark options={solid,fill=mycolor85,draw=black,line width=0.15pt},
forget plot
]
coordinates{
 (44.790798,41.724998) 
};
\addplot [
color=blue,
solid,
mark=*,mark size=1.3pt,
mark options={solid,fill=mycolor86,draw=black,line width=0.15pt},
forget plot
]
coordinates{
 (44.790798,41.724998) 
};
\addplot [
color=blue,
solid,
mark=*,mark size=1.3pt,
mark options={solid,fill=mycolor24,draw=black,line width=0.15pt},
forget plot
]
coordinates{
 (44.790798,41.724998) 
};
\addplot [
color=blue,
solid,
mark=*,mark size=1.3pt,
mark options={solid,fill=mycolor51,draw=black,line width=0.15pt},
forget plot
]
coordinates{
 (44.790798,41.724998) 
};
\addplot [
color=blue,
solid,
mark=*,mark size=1.3pt,
mark options={solid,fill=mycolor65,draw=black,line width=0.15pt},
forget plot
]
coordinates{
 (42.7164,42.262199) 
};
\addplot [
color=blue,
solid,
mark=*,mark size=1.3pt,
mark options={solid,fill=mycolor21,draw=black,line width=0.15pt},
forget plot
]
coordinates{
 (-52.333302,4.9333) 
};
\addplot [
color=blue,
solid,
mark=*,mark size=1.3pt,
mark options={solid,fill=mycolor84,draw=black,line width=0.15pt},
forget plot
]
coordinates{
 (-0.2167,5.55) 
};
\addplot [
color=blue,
solid,
mark=*,mark size=1.3pt,
mark options={solid,fill=mycolor3,draw=black,line width=0.15pt},
forget plot
]
coordinates{
 (-0.2167,5.55) 
};
\addplot [
color=blue,
solid,
mark=*,mark size=1.3pt,
mark options={solid,fill=mycolor8,draw=black,line width=0.15pt},
forget plot
]
coordinates{
 (-0.2167,5.55) 
};
\addplot [
color=blue,
solid,
mark=*,mark size=1.3pt,
mark options={solid,fill=mycolor21,draw=black,line width=0.15pt},
forget plot
]
coordinates{
 (-0.2167,5.55) 
};
\addplot [
color=blue,
solid,
mark=*,mark size=1.3pt,
mark options={solid,fill=mycolor81,draw=black,line width=0.15pt},
forget plot
]
coordinates{
 (-0.2167,5.55) 
};
\addplot [
color=blue,
solid,
mark=*,mark size=1.3pt,
mark options={solid,fill=mycolor58,draw=black,line width=0.15pt},
forget plot
]
coordinates{
 (-2,8) 
};
\addplot [
color=blue,
solid,
mark=*,mark size=1.3pt,
mark options={solid,fill=mycolor10,draw=black,line width=0.15pt},
forget plot
]
coordinates{
 (-0.2167,5.55) 
};
\addplot [
color=blue,
solid,
mark=*,mark size=1.3pt,
mark options={solid,fill=mycolor7,draw=black,line width=0.15pt},
forget plot
]
coordinates{
 (-0.2167,5.55) 
};
\addplot [
color=blue,
solid,
mark=*,mark size=1.3pt,
mark options={solid,fill=mycolor52,draw=black,line width=0.15pt},
forget plot
]
coordinates{
 (-0.2167,5.55) 
};
\addplot [
color=blue,
solid,
mark=*,mark size=1.3pt,
mark options={solid,fill=mycolor84,draw=black,line width=0.15pt},
forget plot
]
coordinates{
 (-0.2167,5.55) 
};
\addplot [
color=blue,
solid,
mark=*,mark size=1.3pt,
mark options={solid,fill=mycolor68,draw=black,line width=0.15pt},
forget plot
]
coordinates{
 (-2,8) 
};
\addplot [
color=blue,
solid,
mark=*,mark size=1.3pt,
mark options={solid,fill=mycolor45,draw=black,line width=0.15pt},
forget plot
]
coordinates{
 (-5.35,36.133301) 
};
\addplot [
color=blue,
solid,
mark=*,mark size=1.3pt,
mark options={solid,fill=mycolor85,draw=black,line width=0.15pt},
forget plot
]
coordinates{
 (138,36) 
};
\addplot [
color=blue,
solid,
mark=*,mark size=1.3pt,
mark options={solid,fill=mycolor45,draw=black,line width=0.15pt},
forget plot
]
coordinates{
 (-5.35,36.133301) 
};
\addplot [
color=blue,
solid,
mark=*,mark size=1.3pt,
mark options={solid,fill=mycolor85,draw=black,line width=0.15pt},
forget plot
]
coordinates{
 (134.616699,34.1833) 
};
\addplot [
color=blue,
solid,
mark=*,mark size=1.3pt,
mark options={solid,fill=mycolor45,draw=black,line width=0.15pt},
forget plot
]
coordinates{
 (-5.3667,36.1833) 
};
\addplot [
color=blue,
solid,
mark=*,mark size=1.3pt,
mark options={solid,fill=mycolor48,draw=black,line width=0.15pt},
forget plot
]
coordinates{
 (-102,23) 
};
\addplot [
color=blue,
solid,
mark=*,mark size=1.3pt,
mark options={solid,fill=mycolor51,draw=black,line width=0.15pt},
forget plot
]
coordinates{
 (-5.35,36.133301) 
};
\addplot [
color=blue,
solid,
mark=*,mark size=1.3pt,
mark options={solid,fill=mycolor85,draw=black,line width=0.15pt},
forget plot
]
coordinates{
 (-51.75,64.183296) 
};
\addplot [
color=blue,
solid,
mark=*,mark size=1.3pt,
mark options={solid,fill=mycolor86,draw=black,line width=0.15pt},
forget plot
]
coordinates{
 (-51.75,64.183296) 
};
\addplot [
color=blue,
solid,
mark=*,mark size=1.3pt,
mark options={solid,fill=mycolor63,draw=black,line width=0.15pt},
forget plot
]
coordinates{
 (-16.577499,13.4531) 
};
\addplot [
color=blue,
solid,
mark=*,mark size=1.3pt,
mark options={solid,fill=mycolor63,draw=black,line width=0.15pt},
forget plot
]
coordinates{
 (-61.5,16.266701) 
};
\addplot [
color=blue,
solid,
mark=*,mark size=1.3pt,
mark options={solid,fill=mycolor65,draw=black,line width=0.15pt},
forget plot
]
coordinates{
 (23.733299,38.033298) 
};
\addplot [
color=blue,
solid,
mark=*,mark size=1.3pt,
mark options={solid,fill=mycolor59,draw=black,line width=0.15pt},
forget plot
]
coordinates{
 (23.733299,37.983299) 
};
\addplot [
color=blue,
solid,
mark=*,mark size=1.3pt,
mark options={solid,fill=mycolor44,draw=black,line width=0.15pt},
forget plot
]
coordinates{
 (-122.234802,47.380901) 
};
\addplot [
color=blue,
solid,
mark=*,mark size=1.3pt,
mark options={solid,fill=mycolor50,draw=black,line width=0.15pt},
forget plot
]
coordinates{
 (23.733299,37.983299) 
};
\addplot [
color=blue,
solid,
mark=*,mark size=1.3pt,
mark options={solid,fill=mycolor86,draw=black,line width=0.15pt},
forget plot
]
coordinates{
 (126.978302,37.598499) 
};
\addplot [
color=blue,
solid,
mark=*,mark size=1.3pt,
mark options={solid,fill=mycolor85,draw=black,line width=0.15pt},
forget plot
]
coordinates{
 (23.733299,37.983299) 
};
\addplot [
color=blue,
solid,
mark=*,mark size=1.3pt,
mark options={solid,fill=mycolor69,draw=black,line width=0.15pt},
forget plot
]
coordinates{
 (139.751404,35.685001) 
};
\addplot [
color=blue,
solid,
mark=*,mark size=1.3pt,
mark options={solid,fill=mycolor77,draw=black,line width=0.15pt},
forget plot
]
coordinates{
 (8.55,47.366699) 
};
\addplot [
color=blue,
solid,
mark=*,mark size=1.3pt,
mark options={solid,fill=mycolor69,draw=black,line width=0.15pt},
forget plot
]
coordinates{
 (-71.640602,10.6317) 
};
\addplot [
color=blue,
solid,
mark=*,mark size=1.3pt,
mark options={solid,fill=mycolor78,draw=black,line width=0.15pt},
forget plot
]
coordinates{
 (126.978302,37.598499) 
};
\addplot [
color=blue,
solid,
mark=*,mark size=1.3pt,
mark options={solid,fill=mycolor63,draw=black,line width=0.15pt},
forget plot
]
coordinates{
 (9.2333,48.599998) 
};
\addplot [
color=blue,
solid,
mark=*,mark size=1.3pt,
mark options={solid,fill=mycolor35,draw=black,line width=0.15pt},
forget plot
]
coordinates{
 (144.786301,13.4443) 
};
\addplot [
color=blue,
solid,
mark=*,mark size=1.3pt,
mark options={solid,fill=mycolor61,draw=black,line width=0.15pt},
forget plot
]
coordinates{
 (144.786301,13.4443) 
};
\addplot [
color=blue,
solid,
mark=*,mark size=1.3pt,
mark options={solid,fill=mycolor12,draw=black,line width=0.15pt},
forget plot
]
coordinates{
 (144.786301,13.4443) 
};
\addplot [
color=blue,
solid,
mark=*,mark size=1.3pt,
mark options={solid,fill=mycolor41,draw=black,line width=0.15pt},
forget plot
]
coordinates{
 (144.786301,13.4443) 
};
\addplot [
color=blue,
solid,
mark=*,mark size=1.3pt,
mark options={solid,fill=mycolor36,draw=black,line width=0.15pt},
forget plot
]
coordinates{
 (120.929001,14.459) 
};
\addplot [
color=blue,
solid,
mark=*,mark size=1.3pt,
mark options={solid,fill=mycolor80,draw=black,line width=0.15pt},
forget plot
]
coordinates{
 (-78.898598,35.993999) 
};
\addplot [
color=blue,
solid,
mark=*,mark size=1.3pt,
mark options={solid,fill=mycolor21,draw=black,line width=0.15pt},
forget plot
]
coordinates{
 (-58.166698,6.8) 
};
\addplot [
color=blue,
solid,
mark=*,mark size=1.3pt,
mark options={solid,fill=mycolor80,draw=black,line width=0.15pt},
forget plot
]
coordinates{
 (-58.299999,6) 
};
\addplot [
color=blue,
solid,
mark=*,mark size=1.3pt,
mark options={solid,fill=mycolor78,draw=black,line width=0.15pt},
forget plot
]
coordinates{
 (-87.216698,14.1) 
};
\addplot [
color=blue,
solid,
mark=*,mark size=1.3pt,
mark options={solid,fill=mycolor77,draw=black,line width=0.15pt},
forget plot
]
coordinates{
 (-87.216698,14.1) 
};
\addplot [
color=blue,
solid,
mark=*,mark size=1.3pt,
mark options={solid,fill=mycolor29,draw=black,line width=0.15pt},
forget plot
]
coordinates{
 (-87.216698,14.1) 
};
\addplot [
color=blue,
solid,
mark=*,mark size=1.3pt,
mark options={solid,fill=mycolor64,draw=black,line width=0.15pt},
forget plot
]
coordinates{
 (-87.216698,14.1) 
};
\addplot [
color=blue,
solid,
mark=*,mark size=1.3pt,
mark options={solid,fill=mycolor78,draw=black,line width=0.15pt},
forget plot
]
coordinates{
 (-87.216698,14.1) 
};
\addplot [
color=blue,
solid,
mark=*,mark size=1.3pt,
mark options={solid,fill=mycolor77,draw=black,line width=0.15pt},
forget plot
]
coordinates{
 (-87.216698,14.1) 
};
\addplot [
color=blue,
solid,
mark=*,mark size=1.3pt,
mark options={solid,fill=mycolor29,draw=black,line width=0.15pt},
forget plot
]
coordinates{
 (-88.033302,15.5) 
};
\addplot [
color=blue,
solid,
mark=*,mark size=1.3pt,
mark options={solid,fill=mycolor77,draw=black,line width=0.15pt},
forget plot
]
coordinates{
 (-90.25,15.5) 
};
\addplot [
color=blue,
solid,
mark=*,mark size=1.3pt,
mark options={solid,fill=mycolor78,draw=black,line width=0.15pt},
forget plot
]
coordinates{
 (-88.033302,15.5) 
};
\addplot [
color=blue,
solid,
mark=*,mark size=1.3pt,
mark options={solid,fill=mycolor64,draw=black,line width=0.15pt},
forget plot
]
coordinates{
 (-87.216698,14.1) 
};
\addplot [
color=blue,
solid,
mark=*,mark size=1.3pt,
mark options={solid,fill=mycolor69,draw=black,line width=0.15pt},
forget plot
]
coordinates{
 (-87.216698,14.1) 
};
\addplot [
color=blue,
solid,
mark=*,mark size=1.3pt,
mark options={solid,fill=mycolor42,draw=black,line width=0.15pt},
forget plot
]
coordinates{
 (-88.033302,15.5) 
};
\addplot [
color=blue,
solid,
mark=*,mark size=1.3pt,
mark options={solid,fill=mycolor29,draw=black,line width=0.15pt},
forget plot
]
coordinates{
 (-87.216698,14.1) 
};
\addplot [
color=blue,
solid,
mark=*,mark size=1.3pt,
mark options={solid,fill=mycolor64,draw=black,line width=0.15pt},
forget plot
]
coordinates{
 (-88.033302,15.5) 
};
\addplot [
color=blue,
solid,
mark=*,mark size=1.3pt,
mark options={solid,fill=mycolor29,draw=black,line width=0.15pt},
forget plot
]
coordinates{
 (-87.216698,14.1) 
};
\addplot [
color=blue,
solid,
mark=*,mark size=1.3pt,
mark options={solid,fill=mycolor43,draw=black,line width=0.15pt},
forget plot
]
coordinates{
 (15.5,45.166698) 
};
\addplot [
color=blue,
solid,
mark=*,mark size=1.3pt,
mark options={solid,fill=mycolor46,draw=black,line width=0.15pt},
forget plot
]
coordinates{
 (15.5,45.166698) 
};
\addplot [
color=blue,
solid,
mark=*,mark size=1.3pt,
mark options={solid,fill=mycolor14,draw=black,line width=0.15pt},
forget plot
]
coordinates{
 (16,45.799999) 
};
\addplot [
color=blue,
solid,
mark=*,mark size=1.3pt,
mark options={solid,fill=mycolor88,draw=black,line width=0.15pt},
forget plot
]
coordinates{
 (-72.334999,18.5392) 
};
\addplot [
color=blue,
solid,
mark=*,mark size=1.3pt,
mark options={solid,fill=mycolor88,draw=black,line width=0.15pt},
forget plot
]
coordinates{
 (-72.334999,18.5392) 
};
\addplot [
color=blue,
solid,
mark=*,mark size=1.3pt,
mark options={solid,fill=mycolor77,draw=black,line width=0.15pt},
forget plot
]
coordinates{
 (-72.334999,18.5392) 
};
\addplot [
color=blue,
solid,
mark=*,mark size=1.3pt,
mark options={solid,fill=mycolor25,draw=black,line width=0.15pt},
forget plot
]
coordinates{
 (18.6481,47.741501) 
};
\addplot [
color=blue,
solid,
mark=*,mark size=1.3pt,
mark options={solid,fill=mycolor26,draw=black,line width=0.15pt},
forget plot
]
coordinates{
 (19.0833,47.5) 
};
\addplot [
color=blue,
solid,
mark=*,mark size=1.3pt,
mark options={solid,fill=mycolor18,draw=black,line width=0.15pt},
forget plot
]
coordinates{
 (17.269899,47.867901) 
};
\addplot [
color=blue,
solid,
mark=*,mark size=1.3pt,
mark options={solid,fill=mycolor77,draw=black,line width=0.15pt},
forget plot
]
coordinates{
 (20,47) 
};
\addplot [
color=blue,
solid,
mark=*,mark size=1.3pt,
mark options={solid,fill=mycolor26,draw=black,line width=0.15pt},
forget plot
]
coordinates{
 (19.0833,47.5) 
};
\addplot [
color=blue,
solid,
mark=*,mark size=1.3pt,
mark options={solid,fill=mycolor46,draw=black,line width=0.15pt},
forget plot
]
coordinates{
 (20,47) 
};
\addplot [
color=blue,
solid,
mark=*,mark size=1.3pt,
mark options={solid,fill=mycolor14,draw=black,line width=0.15pt},
forget plot
]
coordinates{
 (20.200001,47.1833) 
};
\addplot [
color=blue,
solid,
mark=*,mark size=1.3pt,
mark options={solid,fill=mycolor46,draw=black,line width=0.15pt},
forget plot
]
coordinates{
 (17.6376,47.687302) 
};
\addplot [
color=blue,
solid,
mark=*,mark size=1.3pt,
mark options={solid,fill=mycolor75,draw=black,line width=0.15pt},
forget plot
]
coordinates{
 (20.7833,48.166698) 
};
\addplot [
color=blue,
solid,
mark=*,mark size=1.3pt,
mark options={solid,fill=mycolor44,draw=black,line width=0.15pt},
forget plot
]
coordinates{
 (18.9111,47.617802) 
};
\addplot [
color=blue,
solid,
mark=*,mark size=1.3pt,
mark options={solid,fill=mycolor26,draw=black,line width=0.15pt},
forget plot
]
coordinates{
 (16.6,47.6833) 
};
\addplot [
color=blue,
solid,
mark=*,mark size=1.3pt,
mark options={solid,fill=mycolor18,draw=black,line width=0.15pt},
forget plot
]
coordinates{
 (19.0833,47.5) 
};
\addplot [
color=blue,
solid,
mark=*,mark size=1.3pt,
mark options={solid,fill=mycolor26,draw=black,line width=0.15pt},
forget plot
]
coordinates{
 (19.698299,47.9202) 
};
\addplot [
color=blue,
solid,
mark=*,mark size=1.3pt,
mark options={solid,fill=mycolor61,draw=black,line width=0.15pt},
forget plot
]
coordinates{
 (120,-5) 
};
\addplot [
color=blue,
solid,
mark=*,mark size=1.3pt,
mark options={solid,fill=mycolor57,draw=black,line width=0.15pt},
forget plot
]
coordinates{
 (120,-5) 
};
\addplot [
color=blue,
solid,
mark=*,mark size=1.3pt,
mark options={solid,fill=mycolor26,draw=black,line width=0.15pt},
forget plot
]
coordinates{
 (-6.2489,53.333099) 
};
\addplot [
color=blue,
solid,
mark=*,mark size=1.3pt,
mark options={solid,fill=mycolor59,draw=black,line width=0.15pt},
forget plot
]
coordinates{
 (-8.92,53.268299) 
};
\addplot [
color=blue,
solid,
mark=*,mark size=1.3pt,
mark options={solid,fill=mycolor18,draw=black,line width=0.15pt},
forget plot
]
coordinates{
 (-8.4833,54.266701) 
};
\addplot [
color=blue,
solid,
mark=*,mark size=1.3pt,
mark options={solid,fill=mycolor25,draw=black,line width=0.15pt},
forget plot
]
coordinates{
 (-8,53) 
};
\addplot [
color=blue,
solid,
mark=*,mark size=1.3pt,
mark options={solid,fill=mycolor86,draw=black,line width=0.15pt},
forget plot
]
coordinates{
 (34.766701,32.0667) 
};
\addplot [
color=blue,
solid,
mark=*,mark size=1.3pt,
mark options={solid,fill=mycolor59,draw=black,line width=0.15pt},
forget plot
]
coordinates{
 (34.830299,32.0014) 
};
\addplot [
color=blue,
solid,
mark=*,mark size=1.3pt,
mark options={solid,fill=mycolor27,draw=black,line width=0.15pt},
forget plot
]
coordinates{
 (34.766701,32.0667) 
};
\addplot [
color=blue,
solid,
mark=*,mark size=1.3pt,
mark options={solid,fill=mycolor86,draw=black,line width=0.15pt},
forget plot
]
coordinates{
 (34.75,31.5) 
};
\addplot [
color=blue,
solid,
mark=*,mark size=1.3pt,
mark options={solid,fill=mycolor86,draw=black,line width=0.15pt},
forget plot
]
coordinates{
 (34.75,31.5) 
};
\addplot [
color=blue,
solid,
mark=*,mark size=1.3pt,
mark options={solid,fill=mycolor78,draw=black,line width=0.15pt},
forget plot
]
coordinates{
 (34.766701,32.0667) 
};
\addplot [
color=blue,
solid,
mark=*,mark size=1.3pt,
mark options={solid,fill=mycolor86,draw=black,line width=0.15pt},
forget plot
]
coordinates{
 (34.766701,32.0667) 
};
\addplot [
color=blue,
solid,
mark=*,mark size=1.3pt,
mark options={solid,fill=mycolor70,draw=black,line width=0.15pt},
forget plot
]
coordinates{
 (34.772202,32.011398) 
};
\addplot [
color=blue,
solid,
mark=*,mark size=1.3pt,
mark options={solid,fill=mycolor27,draw=black,line width=0.15pt},
forget plot
]
coordinates{
 (34.8367,32.165798) 
};
\addplot [
color=blue,
solid,
mark=*,mark size=1.3pt,
mark options={solid,fill=mycolor65,draw=black,line width=0.15pt},
forget plot
]
coordinates{
 (34.75,31.5) 
};
\addplot [
color=blue,
solid,
mark=*,mark size=1.3pt,
mark options={solid,fill=mycolor74,draw=black,line width=0.15pt},
forget plot
]
coordinates{
 (34.766701,32.0667) 
};
\addplot [
color=blue,
solid,
mark=*,mark size=1.3pt,
mark options={solid,fill=mycolor74,draw=black,line width=0.15pt},
forget plot
]
coordinates{
 (34.8367,32.165798) 
};
\addplot [
color=blue,
solid,
mark=*,mark size=1.3pt,
mark options={solid,fill=mycolor79,draw=black,line width=0.15pt},
forget plot
]
coordinates{
 (44.393902,33.3386) 
};
\addplot [
color=blue,
solid,
mark=*,mark size=1.3pt,
mark options={solid,fill=mycolor44,draw=black,line width=0.15pt},
forget plot
]
coordinates{
 (55.304699,25.2582) 
};
\addplot [
color=blue,
solid,
mark=*,mark size=1.3pt,
mark options={solid,fill=mycolor54,draw=black,line width=0.15pt},
forget plot
]
coordinates{
 (44.393902,33.3386) 
};
\addplot [
color=blue,
solid,
mark=*,mark size=1.3pt,
mark options={solid,fill=mycolor84,draw=black,line width=0.15pt},
forget plot
]
coordinates{
 (53,32) 
};
\addplot [
color=blue,
solid,
mark=*,mark size=1.3pt,
mark options={solid,fill=mycolor68,draw=black,line width=0.15pt},
forget plot
]
coordinates{
 (50.004902,36.279701) 
};
\addplot [
color=blue,
solid,
mark=*,mark size=1.3pt,
mark options={solid,fill=mycolor68,draw=black,line width=0.15pt},
forget plot
]
coordinates{
 (46.291901,38.080002) 
};
\addplot [
color=blue,
solid,
mark=*,mark size=1.3pt,
mark options={solid,fill=mycolor48,draw=black,line width=0.15pt},
forget plot
]
coordinates{
 (-21.950001,64.150002) 
};
\addplot [
color=blue,
solid,
mark=*,mark size=1.3pt,
mark options={solid,fill=mycolor24,draw=black,line width=0.15pt},
forget plot
]
coordinates{
 (-21.950001,64.150002) 
};
\addplot [
color=blue,
solid,
mark=*,mark size=1.3pt,
mark options={solid,fill=mycolor48,draw=black,line width=0.15pt},
forget plot
]
coordinates{
 (-18,65) 
};
\addplot [
color=blue,
solid,
mark=*,mark size=1.3pt,
mark options={solid,fill=mycolor65,draw=black,line width=0.15pt},
forget plot
]
coordinates{
 (11.1333,46.0667) 
};
\addplot [
color=blue,
solid,
mark=*,mark size=1.3pt,
mark options={solid,fill=mycolor24,draw=black,line width=0.15pt},
forget plot
]
coordinates{
 (12.5808,44.063301) 
};
\addplot [
color=blue,
solid,
mark=*,mark size=1.3pt,
mark options={solid,fill=mycolor48,draw=black,line width=0.15pt},
forget plot
]
coordinates{
 (17.2297,40.476101) 
};
\addplot [
color=blue,
solid,
mark=*,mark size=1.3pt,
mark options={solid,fill=mycolor70,draw=black,line width=0.15pt},
forget plot
]
coordinates{
 (-76.800003,18) 
};
\addplot [
color=blue,
solid,
mark=*,mark size=1.3pt,
mark options={solid,fill=mycolor29,draw=black,line width=0.15pt},
forget plot
]
coordinates{
 (-76.800003,18) 
};
\addplot [
color=blue,
solid,
mark=*,mark size=1.3pt,
mark options={solid,fill=mycolor63,draw=black,line width=0.15pt},
forget plot
]
coordinates{
 (-76.800003,18) 
};
\addplot [
color=blue,
solid,
mark=*,mark size=1.3pt,
mark options={solid,fill=mycolor29,draw=black,line width=0.15pt},
forget plot
]
coordinates{
 (-76.866699,17.9667) 
};
\addplot [
color=blue,
solid,
mark=*,mark size=1.3pt,
mark options={solid,fill=mycolor23,draw=black,line width=0.15pt},
forget plot
]
coordinates{
 (-76.800003,18) 
};
\addplot [
color=blue,
solid,
mark=*,mark size=1.3pt,
mark options={solid,fill=mycolor70,draw=black,line width=0.15pt},
forget plot
]
coordinates{
 (-76.866699,17.9667) 
};
\addplot [
color=blue,
solid,
mark=*,mark size=1.3pt,
mark options={solid,fill=mycolor77,draw=black,line width=0.15pt},
forget plot
]
coordinates{
 (-76.800003,18) 
};
\addplot [
color=blue,
solid,
mark=*,mark size=1.3pt,
mark options={solid,fill=mycolor64,draw=black,line width=0.15pt},
forget plot
]
coordinates{
 (-76.800003,18) 
};
\addplot [
color=blue,
solid,
mark=*,mark size=1.3pt,
mark options={solid,fill=mycolor29,draw=black,line width=0.15pt},
forget plot
]
coordinates{
 (-76.800003,18) 
};
\addplot [
color=blue,
solid,
mark=*,mark size=1.3pt,
mark options={solid,fill=mycolor78,draw=black,line width=0.15pt},
forget plot
]
coordinates{
 (-76.800003,18) 
};
\addplot [
color=blue,
solid,
mark=*,mark size=1.3pt,
mark options={solid,fill=mycolor12,draw=black,line width=0.15pt},
forget plot
]
coordinates{
 (-76.800003,18) 
};
\addplot [
color=blue,
solid,
mark=*,mark size=1.3pt,
mark options={solid,fill=mycolor70,draw=black,line width=0.15pt},
forget plot
]
coordinates{
 (-76.800003,18) 
};
\addplot [
color=blue,
solid,
mark=*,mark size=1.3pt,
mark options={solid,fill=mycolor64,draw=black,line width=0.15pt},
forget plot
]
coordinates{
 (-76.800003,18) 
};
\addplot [
color=blue,
solid,
mark=*,mark size=1.3pt,
mark options={solid,fill=mycolor64,draw=black,line width=0.15pt},
forget plot
]
coordinates{
 (-76.800003,18) 
};
\addplot [
color=blue,
solid,
mark=*,mark size=1.3pt,
mark options={solid,fill=mycolor5,draw=black,line width=0.15pt},
forget plot
]
coordinates{
 (35.9333,31.950001) 
};
\addplot [
color=blue,
solid,
mark=*,mark size=1.3pt,
mark options={solid,fill=mycolor44,draw=black,line width=0.15pt},
forget plot
]
coordinates{
 (35.9333,31.950001) 
};
\addplot [
color=blue,
solid,
mark=*,mark size=1.3pt,
mark options={solid,fill=mycolor73,draw=black,line width=0.15pt},
forget plot
]
coordinates{
 (35.9333,31.950001) 
};
\addplot [
color=blue,
solid,
mark=*,mark size=1.3pt,
mark options={solid,fill=mycolor30,draw=black,line width=0.15pt},
forget plot
]
coordinates{
 (35.9333,31.950001) 
};
\addplot [
color=blue,
solid,
mark=*,mark size=1.3pt,
mark options={solid,fill=mycolor31,draw=black,line width=0.15pt},
forget plot
]
coordinates{
 (35.9333,31.950001) 
};
\addplot [
color=blue,
solid,
mark=*,mark size=1.3pt,
mark options={solid,fill=mycolor63,draw=black,line width=0.15pt},
forget plot
]
coordinates{
 (35.9333,31.950001) 
};
\addplot [
color=blue,
solid,
mark=*,mark size=1.3pt,
mark options={solid,fill=mycolor6,draw=black,line width=0.15pt},
forget plot
]
coordinates{
 (135.316696,34.25) 
};
\addplot [
color=blue,
solid,
mark=*,mark size=1.3pt,
mark options={solid,fill=mycolor6,draw=black,line width=0.15pt},
forget plot
]
coordinates{
 (139.642502,35.4478) 
};
\addplot [
color=blue,
solid,
mark=*,mark size=1.3pt,
mark options={solid,fill=mycolor76,draw=black,line width=0.15pt},
forget plot
]
coordinates{
 (74.600304,42.8731) 
};
\addplot [
color=blue,
solid,
mark=*,mark size=1.3pt,
mark options={solid,fill=mycolor64,draw=black,line width=0.15pt},
forget plot
]
coordinates{
 (74.600304,42.8731) 
};
\addplot [
color=blue,
solid,
mark=*,mark size=1.3pt,
mark options={solid,fill=mycolor63,draw=black,line width=0.15pt},
forget plot
]
coordinates{
 (75,41) 
};
\addplot [
color=blue,
solid,
mark=*,mark size=1.3pt,
mark options={solid,fill=mycolor31,draw=black,line width=0.15pt},
forget plot
]
coordinates{
 (74.600304,42.8731) 
};
\addplot [
color=blue,
solid,
mark=*,mark size=1.3pt,
mark options={solid,fill=mycolor70,draw=black,line width=0.15pt},
forget plot
]
coordinates{
 (73.8442,42.826099) 
};
\addplot [
color=blue,
solid,
mark=*,mark size=1.3pt,
mark options={solid,fill=mycolor80,draw=black,line width=0.15pt},
forget plot
]
coordinates{
 (74.600304,42.8731) 
};
\addplot [
color=blue,
solid,
mark=*,mark size=1.3pt,
mark options={solid,fill=mycolor31,draw=black,line width=0.15pt},
forget plot
]
coordinates{
 (74.600304,42.8731) 
};
\addplot [
color=blue,
solid,
mark=*,mark size=1.3pt,
mark options={solid,fill=mycolor82,draw=black,line width=0.15pt},
forget plot
]
coordinates{
 (74.600304,42.8731) 
};
\addplot [
color=blue,
solid,
mark=*,mark size=1.3pt,
mark options={solid,fill=mycolor76,draw=black,line width=0.15pt},
forget plot
]
coordinates{
 (74.600304,42.8731) 
};
\addplot [
color=blue,
solid,
mark=*,mark size=1.3pt,
mark options={solid,fill=mycolor55,draw=black,line width=0.15pt},
forget plot
]
coordinates{
 (105,13) 
};
\addplot [
color=blue,
solid,
mark=*,mark size=1.3pt,
mark options={solid,fill=mycolor34,draw=black,line width=0.15pt},
forget plot
]
coordinates{
 (104.916702,11.55) 
};
\addplot [
color=blue,
solid,
mark=*,mark size=1.3pt,
mark options={solid,fill=mycolor67,draw=black,line width=0.15pt},
forget plot
]
coordinates{
 (104.866699,11.4) 
};
\addplot [
color=blue,
solid,
mark=*,mark size=1.3pt,
mark options={solid,fill=mycolor13,draw=black,line width=0.15pt},
forget plot
]
coordinates{
 (104.916702,11.55) 
};
\addplot [
color=blue,
solid,
mark=*,mark size=1.3pt,
mark options={solid,fill=mycolor67,draw=black,line width=0.15pt},
forget plot
]
coordinates{
 (104.916702,11.55) 
};
\addplot [
color=blue,
solid,
mark=*,mark size=1.3pt,
mark options={solid,fill=mycolor67,draw=black,line width=0.15pt},
forget plot
]
coordinates{
 (105,13) 
};
\addplot [
color=blue,
solid,
mark=*,mark size=1.3pt,
mark options={solid,fill=mycolor80,draw=black,line width=0.15pt},
forget plot
]
coordinates{
 (44.25,-12.1667) 
};
\addplot [
color=blue,
solid,
mark=*,mark size=1.3pt,
mark options={solid,fill=mycolor82,draw=black,line width=0.15pt},
forget plot
]
coordinates{
 (44.25,-12.1667) 
};
\addplot [
color=blue,
solid,
mark=*,mark size=1.3pt,
mark options={solid,fill=mycolor2,draw=black,line width=0.15pt},
forget plot
]
coordinates{
 (43.240299,-11.7042) 
};
\addplot [
color=blue,
solid,
mark=*,mark size=1.3pt,
mark options={solid,fill=mycolor70,draw=black,line width=0.15pt},
forget plot
]
coordinates{
 (43.240299,-11.7042) 
};
\addplot [
color=blue,
solid,
mark=*,mark size=1.3pt,
mark options={solid,fill=mycolor29,draw=black,line width=0.15pt},
forget plot
]
coordinates{
 (44.25,-12.1667) 
};
\addplot [
color=blue,
solid,
mark=*,mark size=1.3pt,
mark options={solid,fill=mycolor20,draw=black,line width=0.15pt},
forget plot
]
coordinates{
 (44.25,-12.1667) 
};
\addplot [
color=blue,
solid,
mark=*,mark size=1.3pt,
mark options={solid,fill=mycolor19,draw=black,line width=0.15pt},
forget plot
]
coordinates{
 (-62.75,17.3333) 
};
\addplot [
color=blue,
solid,
mark=*,mark size=1.3pt,
mark options={solid,fill=mycolor42,draw=black,line width=0.15pt},
forget plot
]
coordinates{
 (-62.716702,17.299999) 
};
\addplot [
color=blue,
solid,
mark=*,mark size=1.3pt,
mark options={solid,fill=mycolor29,draw=black,line width=0.15pt},
forget plot
]
coordinates{
 (-62.716702,17.299999) 
};
\addplot [
color=blue,
solid,
mark=*,mark size=1.3pt,
mark options={solid,fill=mycolor70,draw=black,line width=0.15pt},
forget plot
]
coordinates{
 (-62.716702,17.299999) 
};
\addplot [
color=blue,
solid,
mark=*,mark size=1.3pt,
mark options={solid,fill=mycolor63,draw=black,line width=0.15pt},
forget plot
]
coordinates{
 (-62.716702,17.299999) 
};
\addplot [
color=blue,
solid,
mark=*,mark size=1.3pt,
mark options={solid,fill=mycolor42,draw=black,line width=0.15pt},
forget plot
]
coordinates{
 (-62.616699,17.133301) 
};
\addplot [
color=blue,
solid,
mark=*,mark size=1.3pt,
mark options={solid,fill=mycolor79,draw=black,line width=0.15pt},
forget plot
]
coordinates{
 (-62.716702,17.299999) 
};
\addplot [
color=blue,
solid,
mark=*,mark size=1.3pt,
mark options={solid,fill=mycolor63,draw=black,line width=0.15pt},
forget plot
]
coordinates{
 (-62.716702,17.299999) 
};
\addplot [
color=blue,
solid,
mark=*,mark size=1.3pt,
mark options={solid,fill=mycolor77,draw=black,line width=0.15pt},
forget plot
]
coordinates{
 (-62.716702,17.299999) 
};
\addplot [
color=blue,
solid,
mark=*,mark size=1.3pt,
mark options={solid,fill=mycolor77,draw=black,line width=0.15pt},
forget plot
]
coordinates{
 (-62.716702,17.299999) 
};
\addplot [
color=blue,
solid,
mark=*,mark size=1.3pt,
mark options={solid,fill=mycolor8,draw=black,line width=0.15pt},
forget plot
]
coordinates{
 (127.5,37) 
};
\addplot [
color=blue,
solid,
mark=*,mark size=1.3pt,
mark options={solid,fill=mycolor11,draw=black,line width=0.15pt},
forget plot
]
coordinates{
 (126.978302,37.598499) 
};
\addplot [
color=blue,
solid,
mark=*,mark size=1.3pt,
mark options={solid,fill=mycolor11,draw=black,line width=0.15pt},
forget plot
]
coordinates{
 (126.978302,37.598499) 
};
\addplot [
color=blue,
solid,
mark=*,mark size=1.3pt,
mark options={solid,fill=mycolor10,draw=black,line width=0.15pt},
forget plot
]
coordinates{
 (126.978302,37.598499) 
};
\addplot [
color=blue,
solid,
mark=*,mark size=1.3pt,
mark options={solid,fill=mycolor11,draw=black,line width=0.15pt},
forget plot
]
coordinates{
 (129.154694,35.16) 
};
\addplot [
color=blue,
solid,
mark=*,mark size=1.3pt,
mark options={solid,fill=mycolor37,draw=black,line width=0.15pt},
forget plot
]
coordinates{
 (127.5,37) 
};
\addplot [
color=blue,
solid,
mark=*,mark size=1.3pt,
mark options={solid,fill=mycolor41,draw=black,line width=0.15pt},
forget plot
]
coordinates{
 (126.978302,37.598499) 
};
\addplot [
color=blue,
solid,
mark=*,mark size=1.3pt,
mark options={solid,fill=mycolor12,draw=black,line width=0.15pt},
forget plot
]
coordinates{
 (126.978302,37.598499) 
};
\addplot [
color=blue,
solid,
mark=*,mark size=1.3pt,
mark options={solid,fill=mycolor52,draw=black,line width=0.15pt},
forget plot
]
coordinates{
 (47.978298,29.369699) 
};
\addplot [
color=blue,
solid,
mark=*,mark size=1.3pt,
mark options={solid,fill=mycolor70,draw=black,line width=0.15pt},
forget plot
]
coordinates{
 (-81.383301,19.299999) 
};
\addplot [
color=blue,
solid,
mark=*,mark size=1.3pt,
mark options={solid,fill=mycolor42,draw=black,line width=0.15pt},
forget plot
]
coordinates{
 (-81.383301,19.299999) 
};
\addplot [
color=blue,
solid,
mark=*,mark size=1.3pt,
mark options={solid,fill=mycolor42,draw=black,line width=0.15pt},
forget plot
]
coordinates{
 (-81.383301,19.299999) 
};
\addplot [
color=blue,
solid,
mark=*,mark size=1.3pt,
mark options={solid,fill=mycolor77,draw=black,line width=0.15pt},
forget plot
]
coordinates{
 (-81.383301,19.299999) 
};
\addplot [
color=blue,
solid,
mark=*,mark size=1.3pt,
mark options={solid,fill=mycolor78,draw=black,line width=0.15pt},
forget plot
]
coordinates{
 (-81.383301,19.299999) 
};
\addplot [
color=blue,
solid,
mark=*,mark size=1.3pt,
mark options={solid,fill=mycolor73,draw=black,line width=0.15pt},
forget plot
]
coordinates{
 (-81.383301,19.299999) 
};
\addplot [
color=blue,
solid,
mark=*,mark size=1.3pt,
mark options={solid,fill=mycolor77,draw=black,line width=0.15pt},
forget plot
]
coordinates{
 (-81.383301,19.299999) 
};
\addplot [
color=blue,
solid,
mark=*,mark size=1.3pt,
mark options={solid,fill=mycolor78,draw=black,line width=0.15pt},
forget plot
]
coordinates{
 (-81.383301,19.299999) 
};
\addplot [
color=blue,
solid,
mark=*,mark size=1.3pt,
mark options={solid,fill=mycolor77,draw=black,line width=0.15pt},
forget plot
]
coordinates{
 (-81.383301,19.299999) 
};
\addplot [
color=blue,
solid,
mark=*,mark size=1.3pt,
mark options={solid,fill=mycolor64,draw=black,line width=0.15pt},
forget plot
]
coordinates{
 (-81.383301,19.299999) 
};
\addplot [
color=blue,
solid,
mark=*,mark size=1.3pt,
mark options={solid,fill=mycolor78,draw=black,line width=0.15pt},
forget plot
]
coordinates{
 (-81.383301,19.299999) 
};
\addplot [
color=blue,
solid,
mark=*,mark size=1.3pt,
mark options={solid,fill=mycolor64,draw=black,line width=0.15pt},
forget plot
]
coordinates{
 (-81.383301,19.299999) 
};
\addplot [
color=blue,
solid,
mark=*,mark size=1.3pt,
mark options={solid,fill=mycolor31,draw=black,line width=0.15pt},
forget plot
]
coordinates{
 (73.099403,49.798901) 
};
\addplot [
color=blue,
solid,
mark=*,mark size=1.3pt,
mark options={solid,fill=mycolor74,draw=black,line width=0.15pt},
forget plot
]
coordinates{
 (73.099403,49.798901) 
};
\addplot [
color=blue,
solid,
mark=*,mark size=1.3pt,
mark options={solid,fill=mycolor31,draw=black,line width=0.15pt},
forget plot
]
coordinates{
 (76.949997,43.25) 
};
\addplot [
color=blue,
solid,
mark=*,mark size=1.3pt,
mark options={solid,fill=mycolor13,draw=black,line width=0.15pt},
forget plot
]
coordinates{
 (105,18) 
};
\addplot [
color=blue,
solid,
mark=*,mark size=1.3pt,
mark options={solid,fill=mycolor13,draw=black,line width=0.15pt},
forget plot
]
coordinates{
 (105,18) 
};
\addplot [
color=blue,
solid,
mark=*,mark size=1.3pt,
mark options={solid,fill=mycolor48,draw=black,line width=0.15pt},
forget plot
]
coordinates{
 (35.509701,33.871899) 
};
\addplot [
color=blue,
solid,
mark=*,mark size=1.3pt,
mark options={solid,fill=mycolor17,draw=black,line width=0.15pt},
forget plot
]
coordinates{
 (-61,14) 
};
\addplot [
color=blue,
solid,
mark=*,mark size=1.3pt,
mark options={solid,fill=mycolor69,draw=black,line width=0.15pt},
forget plot
]
coordinates{
 (-61,14) 
};
\addplot [
color=blue,
solid,
mark=*,mark size=1.3pt,
mark options={solid,fill=mycolor78,draw=black,line width=0.15pt},
forget plot
]
coordinates{
 (-61,14) 
};
\addplot [
color=blue,
solid,
mark=*,mark size=1.3pt,
mark options={solid,fill=mycolor39,draw=black,line width=0.15pt},
forget plot
]
coordinates{
 (-61,14) 
};
\addplot [
color=blue,
solid,
mark=*,mark size=1.3pt,
mark options={solid,fill=mycolor29,draw=black,line width=0.15pt},
forget plot
]
coordinates{
 (-61,14) 
};
\addplot [
color=blue,
solid,
mark=*,mark size=1.3pt,
mark options={solid,fill=mycolor69,draw=black,line width=0.15pt},
forget plot
]
coordinates{
 (-61,14) 
};
\addplot [
color=blue,
solid,
mark=*,mark size=1.3pt,
mark options={solid,fill=mycolor74,draw=black,line width=0.15pt},
forget plot
]
coordinates{
 (-61,14) 
};
\addplot [
color=blue,
solid,
mark=*,mark size=1.3pt,
mark options={solid,fill=mycolor29,draw=black,line width=0.15pt},
forget plot
]
coordinates{
 (-61,14) 
};
\addplot [
color=blue,
solid,
mark=*,mark size=1.3pt,
mark options={solid,fill=mycolor77,draw=black,line width=0.15pt},
forget plot
]
coordinates{
 (-61,14) 
};
\addplot [
color=blue,
solid,
mark=*,mark size=1.3pt,
mark options={solid,fill=mycolor29,draw=black,line width=0.15pt},
forget plot
]
coordinates{
 (-61,14) 
};
\addplot [
color=blue,
solid,
mark=*,mark size=1.3pt,
mark options={solid,fill=mycolor82,draw=black,line width=0.15pt},
forget plot
]
coordinates{
 (-61,14) 
};
\addplot [
color=blue,
solid,
mark=*,mark size=1.3pt,
mark options={solid,fill=mycolor54,draw=black,line width=0.15pt},
forget plot
]
coordinates{
 (-61,14) 
};
\addplot [
color=blue,
solid,
mark=*,mark size=1.3pt,
mark options={solid,fill=mycolor70,draw=black,line width=0.15pt},
forget plot
]
coordinates{
 (-61,14) 
};
\addplot [
color=blue,
solid,
mark=*,mark size=1.3pt,
mark options={solid,fill=mycolor77,draw=black,line width=0.15pt},
forget plot
]
coordinates{
 (-61,14) 
};
\addplot [
color=blue,
solid,
mark=*,mark size=1.3pt,
mark options={solid,fill=mycolor29,draw=black,line width=0.15pt},
forget plot
]
coordinates{
 (-61,14) 
};
\addplot [
color=blue,
solid,
mark=*,mark size=1.3pt,
mark options={solid,fill=mycolor78,draw=black,line width=0.15pt},
forget plot
]
coordinates{
 (-61,14) 
};
\addplot [
color=blue,
solid,
mark=*,mark size=1.3pt,
mark options={solid,fill=mycolor63,draw=black,line width=0.15pt},
forget plot
]
coordinates{
 (-61,14) 
};
\addplot [
color=blue,
solid,
mark=*,mark size=1.3pt,
mark options={solid,fill=mycolor29,draw=black,line width=0.15pt},
forget plot
]
coordinates{
 (-61,14) 
};
\addplot [
color=blue,
solid,
mark=*,mark size=1.3pt,
mark options={solid,fill=mycolor78,draw=black,line width=0.15pt},
forget plot
]
coordinates{
 (-61,14) 
};
\addplot [
color=blue,
solid,
mark=*,mark size=1.3pt,
mark options={solid,fill=mycolor70,draw=black,line width=0.15pt},
forget plot
]
coordinates{
 (-61,14) 
};
\addplot [
color=blue,
solid,
mark=*,mark size=1.3pt,
mark options={solid,fill=mycolor64,draw=black,line width=0.15pt},
forget plot
]
coordinates{
 (-61,14) 
};
\addplot [
color=blue,
solid,
mark=*,mark size=1.3pt,
mark options={solid,fill=mycolor19,draw=black,line width=0.15pt},
forget plot
]
coordinates{
 (9.5,47.0667) 
};
\addplot [
color=blue,
solid,
mark=*,mark size=1.3pt,
mark options={solid,fill=mycolor49,draw=black,line width=0.15pt},
forget plot
]
coordinates{
 (9.5167,47.216702) 
};
\addplot [
color=blue,
solid,
mark=*,mark size=1.3pt,
mark options={solid,fill=mycolor49,draw=black,line width=0.15pt},
forget plot
]
coordinates{
 (9.5333,47.099998) 
};
\addplot [
color=blue,
solid,
mark=*,mark size=1.3pt,
mark options={solid,fill=mycolor60,draw=black,line width=0.15pt},
forget plot
]
coordinates{
 (9.5333,47.166698) 
};
\addplot [
color=blue,
solid,
mark=*,mark size=1.3pt,
mark options={solid,fill=mycolor77,draw=black,line width=0.15pt},
forget plot
]
coordinates{
 (126.978302,37.598499) 
};
\addplot [
color=blue,
solid,
mark=*,mark size=1.3pt,
mark options={solid,fill=mycolor59,draw=black,line width=0.15pt},
forget plot
]
coordinates{
 (-9.5,6.5) 
};
\addplot [
color=blue,
solid,
mark=*,mark size=1.3pt,
mark options={solid,fill=mycolor52,draw=black,line width=0.15pt},
forget plot
]
coordinates{
 (25.049999,54.650002) 
};
\addplot [
color=blue,
solid,
mark=*,mark size=1.3pt,
mark options={solid,fill=mycolor51,draw=black,line width=0.15pt},
forget plot
]
coordinates{
 (24.35,55.733299) 
};
\addplot [
color=blue,
solid,
mark=*,mark size=1.3pt,
mark options={solid,fill=mycolor46,draw=black,line width=0.15pt},
forget plot
]
coordinates{
 (24,56) 
};
\addplot [
color=blue,
solid,
mark=*,mark size=1.3pt,
mark options={solid,fill=mycolor18,draw=black,line width=0.15pt},
forget plot
]
coordinates{
 (25.3167,54.6833) 
};
\addplot [
color=blue,
solid,
mark=*,mark size=1.3pt,
mark options={solid,fill=mycolor18,draw=black,line width=0.15pt},
forget plot
]
coordinates{
 (6.1667,49.75) 
};
\addplot [
color=blue,
solid,
mark=*,mark size=1.3pt,
mark options={solid,fill=mycolor63,draw=black,line width=0.15pt},
forget plot
]
coordinates{
 (6.1667,49.75) 
};
\addplot [
color=blue,
solid,
mark=*,mark size=1.3pt,
mark options={solid,fill=mycolor51,draw=black,line width=0.15pt},
forget plot
]
coordinates{
 (23.712799,56.650002) 
};
\addplot [
color=blue,
solid,
mark=*,mark size=1.3pt,
mark options={solid,fill=mycolor65,draw=black,line width=0.15pt},
forget plot
]
coordinates{
 (24.1,56.950001) 
};
\addplot [
color=blue,
solid,
mark=*,mark size=1.3pt,
mark options={solid,fill=mycolor50,draw=black,line width=0.15pt},
forget plot
]
coordinates{
 (24.1,56.950001) 
};
\addplot [
color=blue,
solid,
mark=*,mark size=1.3pt,
mark options={solid,fill=mycolor46,draw=black,line width=0.15pt},
forget plot
]
coordinates{
 (24.1,56.950001) 
};
\addplot [
color=blue,
solid,
mark=*,mark size=1.3pt,
mark options={solid,fill=mycolor24,draw=black,line width=0.15pt},
forget plot
]
coordinates{
 (26.950001,56.549999) 
};
\addplot [
color=blue,
solid,
mark=*,mark size=1.3pt,
mark options={solid,fill=mycolor45,draw=black,line width=0.15pt},
forget plot
]
coordinates{
 (25.4,57.549999) 
};
\addplot [
color=blue,
solid,
mark=*,mark size=1.3pt,
mark options={solid,fill=mycolor64,draw=black,line width=0.15pt},
forget plot
]
coordinates{
 (24.1,56.950001) 
};
\addplot [
color=blue,
solid,
mark=*,mark size=1.3pt,
mark options={solid,fill=mycolor45,draw=black,line width=0.15pt},
forget plot
]
coordinates{
 (24.1,56.950001) 
};
\addplot [
color=blue,
solid,
mark=*,mark size=1.3pt,
mark options={solid,fill=mycolor46,draw=black,line width=0.15pt},
forget plot
]
coordinates{
 (24.1,56.950001) 
};
\addplot [
color=blue,
solid,
mark=*,mark size=1.3pt,
mark options={solid,fill=mycolor18,draw=black,line width=0.15pt},
forget plot
]
coordinates{
 (24.1,56.950001) 
};
\addplot [
color=blue,
solid,
mark=*,mark size=1.3pt,
mark options={solid,fill=mycolor46,draw=black,line width=0.15pt},
forget plot
]
coordinates{
 (24.1,56.950001) 
};
\addplot [
color=blue,
solid,
mark=*,mark size=1.3pt,
mark options={solid,fill=mycolor51,draw=black,line width=0.15pt},
forget plot
]
coordinates{
 (25,57) 
};
\addplot [
color=blue,
solid,
mark=*,mark size=1.3pt,
mark options={solid,fill=mycolor45,draw=black,line width=0.15pt},
forget plot
]
coordinates{
 (24.1,56.950001) 
};
\addplot [
color=blue,
solid,
mark=*,mark size=1.3pt,
mark options={solid,fill=mycolor85,draw=black,line width=0.15pt},
forget plot
]
coordinates{
 (25.25,57.299999) 
};
\addplot [
color=blue,
solid,
mark=*,mark size=1.3pt,
mark options={solid,fill=mycolor46,draw=black,line width=0.15pt},
forget plot
]
coordinates{
 (24.1,56.950001) 
};
\addplot [
color=blue,
solid,
mark=*,mark size=1.3pt,
mark options={solid,fill=mycolor46,draw=black,line width=0.15pt},
forget plot
]
coordinates{
 (24.1,56.950001) 
};
\addplot [
color=blue,
solid,
mark=*,mark size=1.3pt,
mark options={solid,fill=mycolor48,draw=black,line width=0.15pt},
forget plot
]
coordinates{
 (25,57) 
};
\addplot [
color=blue,
solid,
mark=*,mark size=1.3pt,
mark options={solid,fill=mycolor45,draw=black,line width=0.15pt},
forget plot
]
coordinates{
 (25.4,57.549999) 
};
\addplot [
color=blue,
solid,
mark=*,mark size=1.3pt,
mark options={solid,fill=mycolor27,draw=black,line width=0.15pt},
forget plot
]
coordinates{
 (17,25) 
};
\addplot [
color=blue,
solid,
mark=*,mark size=1.3pt,
mark options={solid,fill=mycolor73,draw=black,line width=0.15pt},
forget plot
]
coordinates{
 (17,25) 
};
\addplot [
color=blue,
solid,
mark=*,mark size=1.3pt,
mark options={solid,fill=mycolor66,draw=black,line width=0.15pt},
forget plot
]
coordinates{
 (-7.6192,33.5928) 
};
\addplot [
color=blue,
solid,
mark=*,mark size=1.3pt,
mark options={solid,fill=mycolor50,draw=black,line width=0.15pt},
forget plot
]
coordinates{
 (120.1614,30.2936) 
};
\addplot [
color=blue,
solid,
mark=*,mark size=1.3pt,
mark options={solid,fill=mycolor66,draw=black,line width=0.15pt},
forget plot
]
coordinates{
 (28.8575,47.0056) 
};
\addplot [
color=blue,
solid,
mark=*,mark size=1.3pt,
mark options={solid,fill=mycolor45,draw=black,line width=0.15pt},
forget plot
]
coordinates{
 (28.8575,47.0056) 
};
\addplot [
color=blue,
solid,
mark=*,mark size=1.3pt,
mark options={solid,fill=mycolor59,draw=black,line width=0.15pt},
forget plot
]
coordinates{
 (28.8575,47.0056) 
};
\addplot [
color=blue,
solid,
mark=*,mark size=1.3pt,
mark options={solid,fill=mycolor45,draw=black,line width=0.15pt},
forget plot
]
coordinates{
 (28.8575,47.0056) 
};
\addplot [
color=blue,
solid,
mark=*,mark size=1.3pt,
mark options={solid,fill=mycolor18,draw=black,line width=0.15pt},
forget plot
]
coordinates{
 (29,47) 
};
\addplot [
color=blue,
solid,
mark=*,mark size=1.3pt,
mark options={solid,fill=mycolor45,draw=black,line width=0.15pt},
forget plot
]
coordinates{
 (27.9289,47.7617) 
};
\addplot [
color=blue,
solid,
mark=*,mark size=1.3pt,
mark options={solid,fill=mycolor85,draw=black,line width=0.15pt},
forget plot
]
coordinates{
 (29,47) 
};
\addplot [
color=blue,
solid,
mark=*,mark size=1.3pt,
mark options={solid,fill=mycolor86,draw=black,line width=0.15pt},
forget plot
]
coordinates{
 (28.8575,47.0056) 
};
\addplot [
color=blue,
solid,
mark=*,mark size=1.3pt,
mark options={solid,fill=mycolor66,draw=black,line width=0.15pt},
forget plot
]
coordinates{
 (28.8575,47.0056) 
};
\addplot [
color=blue,
solid,
mark=*,mark size=1.3pt,
mark options={solid,fill=mycolor45,draw=black,line width=0.15pt},
forget plot
]
coordinates{
 (28.725,46.9408) 
};
\addplot [
color=blue,
solid,
mark=*,mark size=1.3pt,
mark options={solid,fill=mycolor52,draw=black,line width=0.15pt},
forget plot
]
coordinates{
 (28.8575,47.0056) 
};
\addplot [
color=blue,
solid,
mark=*,mark size=1.3pt,
mark options={solid,fill=mycolor86,draw=black,line width=0.15pt},
forget plot
]
coordinates{
 (20,41) 
};
\addplot [
color=blue,
solid,
mark=*,mark size=1.3pt,
mark options={solid,fill=mycolor51,draw=black,line width=0.15pt},
forget plot
]
coordinates{
 (19.263599,42.441101) 
};
\addplot [
color=blue,
solid,
mark=*,mark size=1.3pt,
mark options={solid,fill=mycolor48,draw=black,line width=0.15pt},
forget plot
]
coordinates{
 (19.263599,42.441101) 
};
\addplot [
color=blue,
solid,
mark=*,mark size=1.3pt,
mark options={solid,fill=mycolor85,draw=black,line width=0.15pt},
forget plot
]
coordinates{
 (20,41) 
};
\addplot [
color=blue,
solid,
mark=*,mark size=1.3pt,
mark options={solid,fill=mycolor65,draw=black,line width=0.15pt},
forget plot
]
coordinates{
 (21.0986,42.408901) 
};
\addplot [
color=blue,
solid,
mark=*,mark size=1.3pt,
mark options={solid,fill=mycolor48,draw=black,line width=0.15pt},
forget plot
]
coordinates{
 (20,41) 
};
\addplot [
color=blue,
solid,
mark=*,mark size=1.3pt,
mark options={solid,fill=mycolor66,draw=black,line width=0.15pt},
forget plot
]
coordinates{
 (20,41) 
};
\addplot [
color=blue,
solid,
mark=*,mark size=1.3pt,
mark options={solid,fill=mycolor45,draw=black,line width=0.15pt},
forget plot
]
coordinates{
 (19.263599,42.441101) 
};
\addplot [
color=blue,
solid,
mark=*,mark size=1.3pt,
mark options={solid,fill=mycolor32,draw=black,line width=0.15pt},
forget plot
]
coordinates{
 (19.263599,42.441101) 
};
\addplot [
color=blue,
solid,
mark=*,mark size=1.3pt,
mark options={solid,fill=mycolor48,draw=black,line width=0.15pt},
forget plot
]
coordinates{
 (19.263599,42.441101) 
};
\addplot [
color=blue,
solid,
mark=*,mark size=1.3pt,
mark options={solid,fill=mycolor44,draw=black,line width=0.15pt},
forget plot
]
coordinates{
 (21,44) 
};
\addplot [
color=blue,
solid,
mark=*,mark size=1.3pt,
mark options={solid,fill=mycolor66,draw=black,line width=0.15pt},
forget plot
]
coordinates{
 (19.263599,42.441101) 
};
\addplot [
color=blue,
solid,
mark=*,mark size=1.3pt,
mark options={solid,fill=mycolor48,draw=black,line width=0.15pt},
forget plot
]
coordinates{
 (19.263599,42.441101) 
};
\addplot [
color=blue,
solid,
mark=*,mark size=1.3pt,
mark options={solid,fill=mycolor48,draw=black,line width=0.15pt},
forget plot
]
coordinates{
 (19.263599,42.441101) 
};
\addplot [
color=blue,
solid,
mark=*,mark size=1.3pt,
mark options={solid,fill=mycolor66,draw=black,line width=0.15pt},
forget plot
]
coordinates{
 (18.84,42.2864) 
};
\addplot [
color=blue,
solid,
mark=*,mark size=1.3pt,
mark options={solid,fill=mycolor46,draw=black,line width=0.15pt},
forget plot
]
coordinates{
 (19.263599,42.441101) 
};
\addplot [
color=blue,
solid,
mark=*,mark size=1.3pt,
mark options={solid,fill=mycolor76,draw=black,line width=0.15pt},
forget plot
]
coordinates{
 (21.1667,42.666698) 
};
\addplot [
color=blue,
solid,
mark=*,mark size=1.3pt,
mark options={solid,fill=mycolor2,draw=black,line width=0.15pt},
forget plot
]
coordinates{
 (47,-20) 
};
\addplot [
color=blue,
solid,
mark=*,mark size=1.3pt,
mark options={solid,fill=mycolor58,draw=black,line width=0.15pt},
forget plot
]
coordinates{
 (47,-20) 
};
\addplot [
color=blue,
solid,
mark=*,mark size=1.3pt,
mark options={solid,fill=mycolor58,draw=black,line width=0.15pt},
forget plot
]
coordinates{
 (47.516701,-18.9167) 
};
\addplot [
color=blue,
solid,
mark=*,mark size=1.3pt,
mark options={solid,fill=mycolor10,draw=black,line width=0.15pt},
forget plot
]
coordinates{
 (46.3167,-15.7167) 
};
\addplot [
color=blue,
solid,
mark=*,mark size=1.3pt,
mark options={solid,fill=mycolor3,draw=black,line width=0.15pt},
forget plot
]
coordinates{
 (47,-20) 
};
\addplot [
color=blue,
solid,
mark=*,mark size=1.3pt,
mark options={solid,fill=mycolor59,draw=black,line width=0.15pt},
forget plot
]
coordinates{
 (22,41.833302) 
};
\addplot [
color=blue,
solid,
mark=*,mark size=1.3pt,
mark options={solid,fill=mycolor45,draw=black,line width=0.15pt},
forget plot
]
coordinates{
 (20.9083,41.797199) 
};
\addplot [
color=blue,
solid,
mark=*,mark size=1.3pt,
mark options={solid,fill=mycolor45,draw=black,line width=0.15pt},
forget plot
]
coordinates{
 (21.4333,42) 
};
\addplot [
color=blue,
solid,
mark=*,mark size=1.3pt,
mark options={solid,fill=mycolor48,draw=black,line width=0.15pt},
forget plot
]
coordinates{
 (21.714399,42.132198) 
};
\addplot [
color=blue,
solid,
mark=*,mark size=1.3pt,
mark options={solid,fill=mycolor24,draw=black,line width=0.15pt},
forget plot
]
coordinates{
 (22.1786,42.003101) 
};
\addplot [
color=blue,
solid,
mark=*,mark size=1.3pt,
mark options={solid,fill=mycolor48,draw=black,line width=0.15pt},
forget plot
]
coordinates{
 (21.4333,42) 
};
\addplot [
color=blue,
solid,
mark=*,mark size=1.3pt,
mark options={solid,fill=mycolor24,draw=black,line width=0.15pt},
forget plot
]
coordinates{
 (22.195801,41.7458) 
};
\addplot [
color=blue,
solid,
mark=*,mark size=1.3pt,
mark options={solid,fill=mycolor45,draw=black,line width=0.15pt},
forget plot
]
coordinates{
 (21.4333,42) 
};
\addplot [
color=blue,
solid,
mark=*,mark size=1.3pt,
mark options={solid,fill=mycolor85,draw=black,line width=0.15pt},
forget plot
]
coordinates{
 (22.509199,41.882801) 
};
\addplot [
color=blue,
solid,
mark=*,mark size=1.3pt,
mark options={solid,fill=mycolor51,draw=black,line width=0.15pt},
forget plot
]
coordinates{
 (21.4333,42) 
};
\addplot [
color=blue,
solid,
mark=*,mark size=1.3pt,
mark options={solid,fill=mycolor45,draw=black,line width=0.15pt},
forget plot
]
coordinates{
 (21.714399,42.132198) 
};
\addplot [
color=blue,
solid,
mark=*,mark size=1.3pt,
mark options={solid,fill=mycolor48,draw=black,line width=0.15pt},
forget plot
]
coordinates{
 (22,41.833302) 
};
\addplot [
color=blue,
solid,
mark=*,mark size=1.3pt,
mark options={solid,fill=mycolor48,draw=black,line width=0.15pt},
forget plot
]
coordinates{
 (20.801901,41.117199) 
};
\addplot [
color=blue,
solid,
mark=*,mark size=1.3pt,
mark options={solid,fill=mycolor45,draw=black,line width=0.15pt},
forget plot
]
coordinates{
 (21.714399,42.132198) 
};
\addplot [
color=blue,
solid,
mark=*,mark size=1.3pt,
mark options={solid,fill=mycolor25,draw=black,line width=0.15pt},
forget plot
]
coordinates{
 (21.4333,42) 
};
\addplot [
color=blue,
solid,
mark=*,mark size=1.3pt,
mark options={solid,fill=mycolor45,draw=black,line width=0.15pt},
forget plot
]
coordinates{
 (22.464701,41.638302) 
};
\addplot [
color=blue,
solid,
mark=*,mark size=1.3pt,
mark options={solid,fill=mycolor45,draw=black,line width=0.15pt},
forget plot
]
coordinates{
 (22.464701,41.638302) 
};
\addplot [
color=blue,
solid,
mark=*,mark size=1.3pt,
mark options={solid,fill=mycolor45,draw=black,line width=0.15pt},
forget plot
]
coordinates{
 (21.4333,42) 
};
\addplot [
color=blue,
solid,
mark=*,mark size=1.3pt,
mark options={solid,fill=mycolor51,draw=black,line width=0.15pt},
forget plot
]
coordinates{
 (21.4333,42) 
};
\addplot [
color=blue,
solid,
mark=*,mark size=1.3pt,
mark options={solid,fill=mycolor45,draw=black,line width=0.15pt},
forget plot
]
coordinates{
 (21.554399,41.346401) 
};
\addplot [
color=blue,
solid,
mark=*,mark size=1.3pt,
mark options={solid,fill=mycolor18,draw=black,line width=0.15pt},
forget plot
]
coordinates{
 (21.4333,42) 
};
\addplot [
color=blue,
solid,
mark=*,mark size=1.3pt,
mark options={solid,fill=mycolor73,draw=black,line width=0.15pt},
forget plot
]
coordinates{
 (21.4333,42) 
};
\addplot [
color=blue,
solid,
mark=*,mark size=1.3pt,
mark options={solid,fill=mycolor51,draw=black,line width=0.15pt},
forget plot
]
coordinates{
 (21.714399,42.132198) 
};
\addplot [
color=blue,
solid,
mark=*,mark size=1.3pt,
mark options={solid,fill=mycolor50,draw=black,line width=0.15pt},
forget plot
]
coordinates{
 (22,41.833302) 
};
\addplot [
color=blue,
solid,
mark=*,mark size=1.3pt,
mark options={solid,fill=mycolor47,draw=black,line width=0.15pt},
forget plot
]
coordinates{
 (-8,12.65) 
};
\addplot [
color=blue,
solid,
mark=*,mark size=1.3pt,
mark options={solid,fill=mycolor69,draw=black,line width=0.15pt},
forget plot
]
coordinates{
 (-4,17) 
};
\addplot [
color=blue,
solid,
mark=*,mark size=1.3pt,
mark options={solid,fill=mycolor22,draw=black,line width=0.15pt},
forget plot
]
coordinates{
 (-8,12.65) 
};
\addplot [
color=blue,
solid,
mark=*,mark size=1.3pt,
mark options={solid,fill=mycolor69,draw=black,line width=0.15pt},
forget plot
]
coordinates{
 (-8,12.65) 
};
\addplot [
color=blue,
solid,
mark=*,mark size=1.3pt,
mark options={solid,fill=mycolor37,draw=black,line width=0.15pt},
forget plot
]
coordinates{
 (113.550003,22.200001) 
};
\addplot [
color=blue,
solid,
mark=*,mark size=1.3pt,
mark options={solid,fill=mycolor5,draw=black,line width=0.15pt},
forget plot
]
coordinates{
 (113.550003,22.200001) 
};
\addplot [
color=blue,
solid,
mark=*,mark size=1.3pt,
mark options={solid,fill=mycolor11,draw=black,line width=0.15pt},
forget plot
]
coordinates{
 (113.550003,22.200001) 
};
\addplot [
color=blue,
solid,
mark=*,mark size=1.3pt,
mark options={solid,fill=mycolor5,draw=black,line width=0.15pt},
forget plot
]
coordinates{
 (113.550003,22.200001) 
};
\addplot [
color=blue,
solid,
mark=*,mark size=1.3pt,
mark options={solid,fill=mycolor11,draw=black,line width=0.15pt},
forget plot
]
coordinates{
 (113.550003,22.200001) 
};
\addplot [
color=blue,
solid,
mark=*,mark size=1.3pt,
mark options={solid,fill=mycolor11,draw=black,line width=0.15pt},
forget plot
]
coordinates{
 (113.550003,22.200001) 
};
\addplot [
color=blue,
solid,
mark=*,mark size=1.3pt,
mark options={solid,fill=mycolor55,draw=black,line width=0.15pt},
forget plot
]
coordinates{
 (145.756699,15.1819) 
};
\addplot [
color=blue,
solid,
mark=*,mark size=1.3pt,
mark options={solid,fill=mycolor89,draw=black,line width=0.15pt},
forget plot
]
coordinates{
 (145.75,15.2) 
};
\addplot [
color=blue,
solid,
mark=*,mark size=1.3pt,
mark options={solid,fill=mycolor32,draw=black,line width=0.15pt},
forget plot
]
coordinates{
 (-61.083302,14.6) 
};
\addplot [
color=blue,
solid,
mark=*,mark size=1.3pt,
mark options={solid,fill=mycolor32,draw=black,line width=0.15pt},
forget plot
]
coordinates{
 (-61.083302,14.6) 
};
\addplot [
color=blue,
solid,
mark=*,mark size=1.3pt,
mark options={solid,fill=mycolor21,draw=black,line width=0.15pt},
forget plot
]
coordinates{
 (-61.083302,14.6) 
};
\addplot [
color=blue,
solid,
mark=*,mark size=1.3pt,
mark options={solid,fill=mycolor58,draw=black,line width=0.15pt},
forget plot
]
coordinates{
 (-61.083302,14.6) 
};
\addplot [
color=blue,
solid,
mark=*,mark size=1.3pt,
mark options={solid,fill=mycolor52,draw=black,line width=0.15pt},
forget plot
]
coordinates{
 (-61.083302,14.6) 
};
\addplot [
color=blue,
solid,
mark=*,mark size=1.3pt,
mark options={solid,fill=mycolor54,draw=black,line width=0.15pt},
forget plot
]
coordinates{
 (-61,14.6667) 
};
\addplot [
color=blue,
solid,
mark=*,mark size=1.3pt,
mark options={solid,fill=mycolor54,draw=black,line width=0.15pt},
forget plot
]
coordinates{
 (-61.083302,14.6) 
};
\addplot [
color=blue,
solid,
mark=*,mark size=1.3pt,
mark options={solid,fill=mycolor21,draw=black,line width=0.15pt},
forget plot
]
coordinates{
 (-61,14.6667) 
};
\addplot [
color=blue,
solid,
mark=*,mark size=1.3pt,
mark options={solid,fill=mycolor32,draw=black,line width=0.15pt},
forget plot
]
coordinates{
 (-61,14.6667) 
};
\addplot [
color=blue,
solid,
mark=*,mark size=1.3pt,
mark options={solid,fill=mycolor32,draw=black,line width=0.15pt},
forget plot
]
coordinates{
 (-61.083302,14.6) 
};
\addplot [
color=blue,
solid,
mark=*,mark size=1.3pt,
mark options={solid,fill=mycolor64,draw=black,line width=0.15pt},
forget plot
]
coordinates{
 (-61.083302,14.6) 
};
\addplot [
color=blue,
solid,
mark=*,mark size=1.3pt,
mark options={solid,fill=mycolor64,draw=black,line width=0.15pt},
forget plot
]
coordinates{
 (-61,14.6) 
};
\addplot [
color=blue,
solid,
mark=*,mark size=1.3pt,
mark options={solid,fill=mycolor52,draw=black,line width=0.15pt},
forget plot
]
coordinates{
 (-15.9753,18.086399) 
};
\addplot [
color=blue,
solid,
mark=*,mark size=1.3pt,
mark options={solid,fill=mycolor5,draw=black,line width=0.15pt},
forget plot
]
coordinates{
 (-15.9753,18.086399) 
};
\addplot [
color=blue,
solid,
mark=*,mark size=1.3pt,
mark options={solid,fill=mycolor81,draw=black,line width=0.15pt},
forget plot
]
coordinates{
 (-15.9753,18.086399) 
};
\addplot [
color=blue,
solid,
mark=*,mark size=1.3pt,
mark options={solid,fill=mycolor64,draw=black,line width=0.15pt},
forget plot
]
coordinates{
 (-62.200001,16.75) 
};
\addplot [
color=blue,
solid,
mark=*,mark size=1.3pt,
mark options={solid,fill=mycolor70,draw=black,line width=0.15pt},
forget plot
]
coordinates{
 (-62.200001,16.75) 
};
\addplot [
color=blue,
solid,
mark=*,mark size=1.3pt,
mark options={solid,fill=mycolor78,draw=black,line width=0.15pt},
forget plot
]
coordinates{
 (-62.200001,16.75) 
};
\addplot [
color=blue,
solid,
mark=*,mark size=1.3pt,
mark options={solid,fill=mycolor29,draw=black,line width=0.15pt},
forget plot
]
coordinates{
 (-62.216702,16.75) 
};
\addplot [
color=blue,
solid,
mark=*,mark size=1.3pt,
mark options={solid,fill=mycolor78,draw=black,line width=0.15pt},
forget plot
]
coordinates{
 (-62.216702,16.733299) 
};
\addplot [
color=blue,
solid,
mark=*,mark size=1.3pt,
mark options={solid,fill=mycolor77,draw=black,line width=0.15pt},
forget plot
]
coordinates{
 (-62.216702,16.75) 
};
\addplot [
color=blue,
solid,
mark=*,mark size=1.3pt,
mark options={solid,fill=mycolor77,draw=black,line width=0.15pt},
forget plot
]
coordinates{
 (-62.216702,16.75) 
};
\addplot [
color=blue,
solid,
mark=*,mark size=1.3pt,
mark options={solid,fill=mycolor70,draw=black,line width=0.15pt},
forget plot
]
coordinates{
 (-62.216702,16.75) 
};
\addplot [
color=blue,
solid,
mark=*,mark size=1.3pt,
mark options={solid,fill=mycolor45,draw=black,line width=0.15pt},
forget plot
]
coordinates{
 (14.4025,35.886902) 
};
\addplot [
color=blue,
solid,
mark=*,mark size=1.3pt,
mark options={solid,fill=mycolor45,draw=black,line width=0.15pt},
forget plot
]
coordinates{
 (14.4025,35.886902) 
};
\addplot [
color=blue,
solid,
mark=*,mark size=1.3pt,
mark options={solid,fill=mycolor45,draw=black,line width=0.15pt},
forget plot
]
coordinates{
 (14.4847,35.833099) 
};
\addplot [
color=blue,
solid,
mark=*,mark size=1.3pt,
mark options={solid,fill=mycolor45,draw=black,line width=0.15pt},
forget plot
]
coordinates{
 (14.4025,35.886902) 
};
\addplot [
color=blue,
solid,
mark=*,mark size=1.3pt,
mark options={solid,fill=mycolor74,draw=black,line width=0.15pt},
forget plot
]
coordinates{
 (-90.816803,35.074501) 
};
\addplot [
color=blue,
solid,
mark=*,mark size=1.3pt,
mark options={solid,fill=mycolor46,draw=black,line width=0.15pt},
forget plot
]
coordinates{
 (14.4025,35.886902) 
};
\addplot [
color=blue,
solid,
mark=*,mark size=1.3pt,
mark options={solid,fill=mycolor18,draw=black,line width=0.15pt},
forget plot
]
coordinates{
 (14.4025,35.886902) 
};
\addplot [
color=blue,
solid,
mark=*,mark size=1.3pt,
mark options={solid,fill=mycolor46,draw=black,line width=0.15pt},
forget plot
]
coordinates{
 (14.5019,35.912498) 
};
\addplot [
color=blue,
solid,
mark=*,mark size=1.3pt,
mark options={solid,fill=mycolor51,draw=black,line width=0.15pt},
forget plot
]
coordinates{
 (14.4761,35.905602) 
};
\addplot [
color=blue,
solid,
mark=*,mark size=1.3pt,
mark options={solid,fill=mycolor46,draw=black,line width=0.15pt},
forget plot
]
coordinates{
 (14.4025,35.886902) 
};
\addplot [
color=blue,
solid,
mark=*,mark size=1.3pt,
mark options={solid,fill=mycolor46,draw=black,line width=0.15pt},
forget plot
]
coordinates{
 (14.4025,35.886902) 
};
\addplot [
color=blue,
solid,
mark=*,mark size=1.3pt,
mark options={solid,fill=mycolor14,draw=black,line width=0.15pt},
forget plot
]
coordinates{
 (-2,54) 
};
\addplot [
color=blue,
solid,
mark=*,mark size=1.3pt,
mark options={solid,fill=mycolor45,draw=black,line width=0.15pt},
forget plot
]
coordinates{
 (14.4025,35.886902) 
};
\addplot [
color=blue,
solid,
mark=*,mark size=1.3pt,
mark options={solid,fill=mycolor45,draw=black,line width=0.15pt},
forget plot
]
coordinates{
 (14.4025,35.886902) 
};
\addplot [
color=blue,
solid,
mark=*,mark size=1.3pt,
mark options={solid,fill=mycolor46,draw=black,line width=0.15pt},
forget plot
]
coordinates{
 (14.48,35.922501) 
};
\addplot [
color=blue,
solid,
mark=*,mark size=1.3pt,
mark options={solid,fill=mycolor46,draw=black,line width=0.15pt},
forget plot
]
coordinates{
 (14.4989,35.8731) 
};
\addplot [
color=blue,
solid,
mark=*,mark size=1.3pt,
mark options={solid,fill=mycolor46,draw=black,line width=0.15pt},
forget plot
]
coordinates{
 (14.4025,35.886902) 
};
\addplot [
color=blue,
solid,
mark=*,mark size=1.3pt,
mark options={solid,fill=mycolor45,draw=black,line width=0.15pt},
forget plot
]
coordinates{
 (14.5833,35.833302) 
};
\addplot [
color=blue,
solid,
mark=*,mark size=1.3pt,
mark options={solid,fill=mycolor46,draw=black,line width=0.15pt},
forget plot
]
coordinates{
 (14.4025,35.886902) 
};
\addplot [
color=blue,
solid,
mark=*,mark size=1.3pt,
mark options={solid,fill=mycolor46,draw=black,line width=0.15pt},
forget plot
]
coordinates{
 (14.4436,35.913601) 
};
\addplot [
color=blue,
solid,
mark=*,mark size=1.3pt,
mark options={solid,fill=mycolor46,draw=black,line width=0.15pt},
forget plot
]
coordinates{
 (14.2333,36.029999) 
};
\addplot [
color=blue,
solid,
mark=*,mark size=1.3pt,
mark options={solid,fill=mycolor45,draw=black,line width=0.15pt},
forget plot
]
coordinates{
 (14.4436,35.913601) 
};
\addplot [
color=blue,
solid,
mark=*,mark size=1.3pt,
mark options={solid,fill=mycolor85,draw=black,line width=0.15pt},
forget plot
]
coordinates{
 (14.4425,35.889702) 
};
\addplot [
color=blue,
solid,
mark=*,mark size=1.3pt,
mark options={solid,fill=mycolor5,draw=black,line width=0.15pt},
forget plot
]
coordinates{
 (57.498901,-20.1619) 
};
\addplot [
color=blue,
solid,
mark=*,mark size=1.3pt,
mark options={solid,fill=mycolor80,draw=black,line width=0.15pt},
forget plot
]
coordinates{
 (73.5,4.1667) 
};
\addplot [
color=blue,
solid,
mark=*,mark size=1.3pt,
mark options={solid,fill=mycolor81,draw=black,line width=0.15pt},
forget plot
]
coordinates{
 (73.5,4.1667) 
};
\addplot [
color=blue,
solid,
mark=*,mark size=1.3pt,
mark options={solid,fill=mycolor80,draw=black,line width=0.15pt},
forget plot
]
coordinates{
 (73.5,4.1667) 
};
\addplot [
color=blue,
solid,
mark=*,mark size=1.3pt,
mark options={solid,fill=mycolor84,draw=black,line width=0.15pt},
forget plot
]
coordinates{
 (73.5,4.1667) 
};
\addplot [
color=blue,
solid,
mark=*,mark size=1.3pt,
mark options={solid,fill=mycolor42,draw=black,line width=0.15pt},
forget plot
]
coordinates{
 (-100.25,25.6833) 
};
\addplot [
color=blue,
solid,
mark=*,mark size=1.3pt,
mark options={solid,fill=mycolor32,draw=black,line width=0.15pt},
forget plot
]
coordinates{
 (-99.75,25.1833) 
};
\addplot [
color=blue,
solid,
mark=*,mark size=1.3pt,
mark options={solid,fill=mycolor69,draw=black,line width=0.15pt},
forget plot
]
coordinates{
 (-100.300003,25.75) 
};
\addplot [
color=blue,
solid,
mark=*,mark size=1.3pt,
mark options={solid,fill=mycolor58,draw=black,line width=0.15pt},
forget plot
]
coordinates{
 (-99.138603,19.4342) 
};
\addplot [
color=blue,
solid,
mark=*,mark size=1.3pt,
mark options={solid,fill=mycolor21,draw=black,line width=0.15pt},
forget plot
]
coordinates{
 (-101.666702,21.116699) 
};
\addplot [
color=blue,
solid,
mark=*,mark size=1.3pt,
mark options={solid,fill=mycolor58,draw=black,line width=0.15pt},
forget plot
]
coordinates{
 (-99.221703,19.526899) 
};
\addplot [
color=blue,
solid,
mark=*,mark size=1.3pt,
mark options={solid,fill=mycolor58,draw=black,line width=0.15pt},
forget plot
]
coordinates{
 (-110.966698,29.0667) 
};
\addplot [
color=blue,
solid,
mark=*,mark size=1.3pt,
mark options={solid,fill=mycolor80,draw=black,line width=0.15pt},
forget plot
]
coordinates{
 (-103.316704,20.65) 
};
\addplot [
color=blue,
solid,
mark=*,mark size=1.3pt,
mark options={solid,fill=mycolor61,draw=black,line width=0.15pt},
forget plot
]
coordinates{
 (101.533302,3.0833) 
};
\addplot [
color=blue,
solid,
mark=*,mark size=1.3pt,
mark options={solid,fill=mycolor9,draw=black,line width=0.15pt},
forget plot
]
coordinates{
 (100.3601,6.121) 
};
\addplot [
color=blue,
solid,
mark=*,mark size=1.3pt,
mark options={solid,fill=mycolor20,draw=black,line width=0.15pt},
forget plot
]
coordinates{
 (101.699997,3.1667) 
};
\addplot [
color=blue,
solid,
mark=*,mark size=1.3pt,
mark options={solid,fill=mycolor7,draw=black,line width=0.15pt},
forget plot
]
coordinates{
 (101.650002,3.0833) 
};
\addplot [
color=blue,
solid,
mark=*,mark size=1.3pt,
mark options={solid,fill=mycolor61,draw=black,line width=0.15pt},
forget plot
]
coordinates{
 (101.699997,3.1667) 
};
\addplot [
color=blue,
solid,
mark=*,mark size=1.3pt,
mark options={solid,fill=mycolor12,draw=black,line width=0.15pt},
forget plot
]
coordinates{
 (101.699997,3.1667) 
};
\addplot [
color=blue,
solid,
mark=*,mark size=1.3pt,
mark options={solid,fill=mycolor61,draw=black,line width=0.15pt},
forget plot
]
coordinates{
 (101.650002,3.0833) 
};
\addplot [
color=blue,
solid,
mark=*,mark size=1.3pt,
mark options={solid,fill=mycolor5,draw=black,line width=0.15pt},
forget plot
]
coordinates{
 (102.279404,6.0912) 
};
\addplot [
color=blue,
solid,
mark=*,mark size=1.3pt,
mark options={solid,fill=mycolor7,draw=black,line width=0.15pt},
forget plot
]
coordinates{
 (101.790901,2.9927) 
};
\addplot [
color=blue,
solid,
mark=*,mark size=1.3pt,
mark options={solid,fill=mycolor62,draw=black,line width=0.15pt},
forget plot
]
coordinates{
 (101.533302,3.0833) 
};
\addplot [
color=blue,
solid,
mark=*,mark size=1.3pt,
mark options={solid,fill=mycolor7,draw=black,line width=0.15pt},
forget plot
]
coordinates{
 (101.716698,3.0333) 
};
\addplot [
color=blue,
solid,
mark=*,mark size=1.3pt,
mark options={solid,fill=mycolor9,draw=black,line width=0.15pt},
forget plot
]
coordinates{
 (100.306503,5.3451) 
};
\addplot [
color=blue,
solid,
mark=*,mark size=1.3pt,
mark options={solid,fill=mycolor88,draw=black,line width=0.15pt},
forget plot
]
coordinates{
 (17.083599,-22.57) 
};
\addplot [
color=blue,
solid,
mark=*,mark size=1.3pt,
mark options={solid,fill=mycolor21,draw=black,line width=0.15pt},
forget plot
]
coordinates{
 (17.7167,-19.233299) 
};
\addplot [
color=blue,
solid,
mark=*,mark size=1.3pt,
mark options={solid,fill=mycolor21,draw=black,line width=0.15pt},
forget plot
]
coordinates{
 (17.083599,-22.57) 
};
\addplot [
color=blue,
solid,
mark=*,mark size=1.3pt,
mark options={solid,fill=mycolor35,draw=black,line width=0.15pt},
forget plot
]
coordinates{
 (166.449997,-22.266701) 
};
\addplot [
color=blue,
solid,
mark=*,mark size=1.3pt,
mark options={solid,fill=mycolor5,draw=black,line width=0.15pt},
forget plot
]
coordinates{
 (166.449997,-22.266701) 
};
\addplot [
color=blue,
solid,
mark=*,mark size=1.3pt,
mark options={solid,fill=mycolor28,draw=black,line width=0.15pt},
forget plot
]
coordinates{
 (8,10) 
};
\addplot [
color=blue,
solid,
mark=*,mark size=1.3pt,
mark options={solid,fill=mycolor42,draw=black,line width=0.15pt},
forget plot
]
coordinates{
 (-86.199997,11.9167) 
};
\addplot [
color=blue,
solid,
mark=*,mark size=1.3pt,
mark options={solid,fill=mycolor42,draw=black,line width=0.15pt},
forget plot
]
coordinates{
 (-86.268303,12.1508) 
};
\addplot [
color=blue,
solid,
mark=*,mark size=1.3pt,
mark options={solid,fill=mycolor63,draw=black,line width=0.15pt},
forget plot
]
coordinates{
 (-86.268303,12.1508) 
};
\addplot [
color=blue,
solid,
mark=*,mark size=1.3pt,
mark options={solid,fill=mycolor42,draw=black,line width=0.15pt},
forget plot
]
coordinates{
 (-86.268303,12.1508) 
};
\addplot [
color=blue,
solid,
mark=*,mark size=1.3pt,
mark options={solid,fill=mycolor69,draw=black,line width=0.15pt},
forget plot
]
coordinates{
 (-86.268303,12.1508) 
};
\addplot [
color=blue,
solid,
mark=*,mark size=1.3pt,
mark options={solid,fill=mycolor46,draw=black,line width=0.15pt},
forget plot
]
coordinates{
 (5.1191,52.0938) 
};
\addplot [
color=blue,
solid,
mark=*,mark size=1.3pt,
mark options={solid,fill=mycolor18,draw=black,line width=0.15pt},
forget plot
]
coordinates{
 (6.8912,52.219501) 
};
\addplot [
color=blue,
solid,
mark=*,mark size=1.3pt,
mark options={solid,fill=mycolor18,draw=black,line width=0.15pt},
forget plot
]
coordinates{
 (4.9167,52.349998) 
};
\addplot [
color=blue,
solid,
mark=*,mark size=1.3pt,
mark options={solid,fill=mycolor43,draw=black,line width=0.15pt},
forget plot
]
coordinates{
 (5.4667,51.450001) 
};
\addplot [
color=blue,
solid,
mark=*,mark size=1.3pt,
mark options={solid,fill=mycolor46,draw=black,line width=0.15pt},
forget plot
]
coordinates{
 (10.2333,59.133301) 
};
\addplot [
color=blue,
solid,
mark=*,mark size=1.3pt,
mark options={solid,fill=mycolor18,draw=black,line width=0.15pt},
forget plot
]
coordinates{
 (8.5886,58.340599) 
};
\addplot [
color=blue,
solid,
mark=*,mark size=1.3pt,
mark options={solid,fill=mycolor46,draw=black,line width=0.15pt},
forget plot
]
coordinates{
 (5.3247,60.391102) 
};
\addplot [
color=blue,
solid,
mark=*,mark size=1.3pt,
mark options={solid,fill=mycolor24,draw=black,line width=0.15pt},
forget plot
]
coordinates{
 (12,60.200001) 
};
\addplot [
color=blue,
solid,
mark=*,mark size=1.3pt,
mark options={solid,fill=mycolor48,draw=black,line width=0.15pt},
forget plot
]
coordinates{
 (-110.360703,31.5273) 
};
\addplot [
color=blue,
solid,
mark=*,mark size=1.3pt,
mark options={solid,fill=mycolor46,draw=black,line width=0.15pt},
forget plot
]
coordinates{
 (10.6311,59.6633) 
};
\addplot [
color=blue,
solid,
mark=*,mark size=1.3pt,
mark options={solid,fill=mycolor46,draw=black,line width=0.15pt},
forget plot
]
coordinates{
 (5.3247,60.391102) 
};
\addplot [
color=blue,
solid,
mark=*,mark size=1.3pt,
mark options={solid,fill=mycolor26,draw=black,line width=0.15pt},
forget plot
]
coordinates{
 (10.75,59.916698) 
};
\addplot [
color=blue,
solid,
mark=*,mark size=1.3pt,
mark options={solid,fill=mycolor23,draw=black,line width=0.15pt},
forget plot
]
coordinates{
 (85.316704,27.7167) 
};
\addplot [
color=blue,
solid,
mark=*,mark size=1.3pt,
mark options={solid,fill=mycolor22,draw=black,line width=0.15pt},
forget plot
]
coordinates{
 (58.5933,23.6133) 
};
\addplot [
color=blue,
solid,
mark=*,mark size=1.3pt,
mark options={solid,fill=mycolor21,draw=black,line width=0.15pt},
forget plot
]
coordinates{
 (58.5933,23.6133) 
};
\addplot [
color=blue,
solid,
mark=*,mark size=1.3pt,
mark options={solid,fill=mycolor77,draw=black,line width=0.15pt},
forget plot
]
coordinates{
 (-80.233299,7.75) 
};
\addplot [
color=blue,
solid,
mark=*,mark size=1.3pt,
mark options={solid,fill=mycolor16,draw=black,line width=0.15pt},
forget plot
]
coordinates{
 (-77.050003,-12.05) 
};
\addplot [
color=blue,
solid,
mark=*,mark size=1.3pt,
mark options={solid,fill=mycolor16,draw=black,line width=0.15pt},
forget plot
]
coordinates{
 (-77.050003,-12.05) 
};
\addplot [
color=blue,
solid,
mark=*,mark size=1.3pt,
mark options={solid,fill=mycolor16,draw=black,line width=0.15pt},
forget plot
]
coordinates{
 (-77.050003,-12.05) 
};
\addplot [
color=blue,
solid,
mark=*,mark size=1.3pt,
mark options={solid,fill=mycolor32,draw=black,line width=0.15pt},
forget plot
]
coordinates{
 (-77.050003,-12.05) 
};
\addplot [
color=blue,
solid,
mark=*,mark size=1.3pt,
mark options={solid,fill=mycolor79,draw=black,line width=0.15pt},
forget plot
]
coordinates{
 (-77.050003,-12.05) 
};
\addplot [
color=blue,
solid,
mark=*,mark size=1.3pt,
mark options={solid,fill=mycolor58,draw=black,line width=0.15pt},
forget plot
]
coordinates{
 (-75.233299,-12.0667) 
};
\addplot [
color=blue,
solid,
mark=*,mark size=1.3pt,
mark options={solid,fill=mycolor17,draw=black,line width=0.15pt},
forget plot
]
coordinates{
 (-77.150002,-12.0667) 
};
\addplot [
color=blue,
solid,
mark=*,mark size=1.3pt,
mark options={solid,fill=mycolor82,draw=black,line width=0.15pt},
forget plot
]
coordinates{
 (-77.050003,-12.05) 
};
\addplot [
color=blue,
solid,
mark=*,mark size=1.3pt,
mark options={solid,fill=mycolor32,draw=black,line width=0.15pt},
forget plot
]
coordinates{
 (-77.050003,-12.05) 
};
\addplot [
color=blue,
solid,
mark=*,mark size=1.3pt,
mark options={solid,fill=mycolor82,draw=black,line width=0.15pt},
forget plot
]
coordinates{
 (-77.050003,-12.05) 
};
\addplot [
color=blue,
solid,
mark=*,mark size=1.3pt,
mark options={solid,fill=mycolor58,draw=black,line width=0.15pt},
forget plot
]
coordinates{
 (-77.050003,-12.05) 
};
\addplot [
color=blue,
solid,
mark=*,mark size=1.3pt,
mark options={solid,fill=mycolor68,draw=black,line width=0.15pt},
forget plot
]
coordinates{
 (-77.050003,-12.05) 
};
\addplot [
color=blue,
solid,
mark=*,mark size=1.3pt,
mark options={solid,fill=mycolor54,draw=black,line width=0.15pt},
forget plot
]
coordinates{
 (-77.050003,-12.05) 
};
\addplot [
color=blue,
solid,
mark=*,mark size=1.3pt,
mark options={solid,fill=mycolor68,draw=black,line width=0.15pt},
forget plot
]
coordinates{
 (-77.050003,-12.05) 
};
\addplot [
color=blue,
solid,
mark=*,mark size=1.3pt,
mark options={solid,fill=mycolor56,draw=black,line width=0.15pt},
forget plot
]
coordinates{
 (147.199997,-9.4667) 
};
\addplot [
color=blue,
solid,
mark=*,mark size=1.3pt,
mark options={solid,fill=mycolor67,draw=black,line width=0.15pt},
forget plot
]
coordinates{
 (147,-6) 
};
\addplot [
color=blue,
solid,
mark=*,mark size=1.3pt,
mark options={solid,fill=mycolor56,draw=black,line width=0.15pt},
forget plot
]
coordinates{
 (147.192505,-9.4647) 
};
\addplot [
color=blue,
solid,
mark=*,mark size=1.3pt,
mark options={solid,fill=mycolor56,draw=black,line width=0.15pt},
forget plot
]
coordinates{
 (147,-6) 
};
\addplot [
color=blue,
solid,
mark=*,mark size=1.3pt,
mark options={solid,fill=mycolor67,draw=black,line width=0.15pt},
forget plot
]
coordinates{
 (147,-6) 
};
\addplot [
color=blue,
solid,
mark=*,mark size=1.3pt,
mark options={solid,fill=mycolor67,draw=black,line width=0.15pt},
forget plot
]
coordinates{
 (121.360199,14.5402) 
};
\addplot [
color=blue,
solid,
mark=*,mark size=1.3pt,
mark options={solid,fill=mycolor41,draw=black,line width=0.15pt},
forget plot
]
coordinates{
 (73.055099,33.689999) 
};
\addplot [
color=blue,
solid,
mark=*,mark size=1.3pt,
mark options={solid,fill=mycolor69,draw=black,line width=0.15pt},
forget plot
]
coordinates{
 (67.082199,24.9056) 
};
\addplot [
color=blue,
solid,
mark=*,mark size=1.3pt,
mark options={solid,fill=mycolor63,draw=black,line width=0.15pt},
forget plot
]
coordinates{
 (67.082199,24.9056) 
};
\addplot [
color=blue,
solid,
mark=*,mark size=1.3pt,
mark options={solid,fill=mycolor46,draw=black,line width=0.15pt},
forget plot
]
coordinates{
 (20,52) 
};
\addplot [
color=blue,
solid,
mark=*,mark size=1.3pt,
mark options={solid,fill=mycolor46,draw=black,line width=0.15pt},
forget plot
]
coordinates{
 (16.075701,51.662998) 
};
\addplot [
color=blue,
solid,
mark=*,mark size=1.3pt,
mark options={solid,fill=mycolor26,draw=black,line width=0.15pt},
forget plot
]
coordinates{
 (18.9328,50.348) 
};
\addplot [
color=blue,
solid,
mark=*,mark size=1.3pt,
mark options={solid,fill=mycolor46,draw=black,line width=0.15pt},
forget plot
]
coordinates{
 (16.746099,53.154999) 
};
\addplot [
color=blue,
solid,
mark=*,mark size=1.3pt,
mark options={solid,fill=mycolor46,draw=black,line width=0.15pt},
forget plot
]
coordinates{
 (20,52) 
};
\addplot [
color=blue,
solid,
mark=*,mark size=1.3pt,
mark options={solid,fill=mycolor51,draw=black,line width=0.15pt},
forget plot
]
coordinates{
 (22.065399,53.173599) 
};
\addplot [
color=blue,
solid,
mark=*,mark size=1.3pt,
mark options={solid,fill=mycolor45,draw=black,line width=0.15pt},
forget plot
]
coordinates{
 (17.2407,54.058498) 
};
\addplot [
color=blue,
solid,
mark=*,mark size=1.3pt,
mark options={solid,fill=mycolor46,draw=black,line width=0.15pt},
forget plot
]
coordinates{
 (18.6766,50.2976) 
};
\addplot [
color=blue,
solid,
mark=*,mark size=1.3pt,
mark options={solid,fill=mycolor82,draw=black,line width=0.15pt},
forget plot
]
coordinates{
 (-66.057999,18.227301) 
};
\addplot [
color=blue,
solid,
mark=*,mark size=1.3pt,
mark options={solid,fill=mycolor42,draw=black,line width=0.15pt},
forget plot
]
coordinates{
 (-66.0616,18.420799) 
};
\addplot [
color=blue,
solid,
mark=*,mark size=1.3pt,
mark options={solid,fill=mycolor78,draw=black,line width=0.15pt},
forget plot
]
coordinates{
 (-67.127502,18.184299) 
};
\addplot [
color=blue,
solid,
mark=*,mark size=1.3pt,
mark options={solid,fill=mycolor42,draw=black,line width=0.15pt},
forget plot
]
coordinates{
 (-66.636597,18.023899) 
};
\addplot [
color=blue,
solid,
mark=*,mark size=1.3pt,
mark options={solid,fill=mycolor64,draw=black,line width=0.15pt},
forget plot
]
coordinates{
 (-66.0616,18.420799) 
};
\addplot [
color=blue,
solid,
mark=*,mark size=1.3pt,
mark options={solid,fill=mycolor75,draw=black,line width=0.15pt},
forget plot
]
coordinates{
 (35.25,32) 
};
\addplot [
color=blue,
solid,
mark=*,mark size=1.3pt,
mark options={solid,fill=mycolor85,draw=black,line width=0.15pt},
forget plot
]
coordinates{
 (34.466702,31.5) 
};
\addplot [
color=blue,
solid,
mark=*,mark size=1.3pt,
mark options={solid,fill=mycolor65,draw=black,line width=0.15pt},
forget plot
]
coordinates{
 (35.25,32) 
};
\addplot [
color=blue,
solid,
mark=*,mark size=1.3pt,
mark options={solid,fill=mycolor32,draw=black,line width=0.15pt},
forget plot
]
coordinates{
 (35.200001,31.9) 
};
\addplot [
color=blue,
solid,
mark=*,mark size=1.3pt,
mark options={solid,fill=mycolor73,draw=black,line width=0.15pt},
forget plot
]
coordinates{
 (35.25,32) 
};
\addplot [
color=blue,
solid,
mark=*,mark size=1.3pt,
mark options={solid,fill=mycolor30,draw=black,line width=0.15pt},
forget plot
]
coordinates{
 (34.466702,31.5) 
};
\addplot [
color=blue,
solid,
mark=*,mark size=1.3pt,
mark options={solid,fill=mycolor75,draw=black,line width=0.15pt},
forget plot
]
coordinates{
 (35.25,32) 
};
\addplot [
color=blue,
solid,
mark=*,mark size=1.3pt,
mark options={solid,fill=mycolor24,draw=black,line width=0.15pt},
forget plot
]
coordinates{
 (-9.1569,38.679001) 
};
\addplot [
color=blue,
solid,
mark=*,mark size=1.3pt,
mark options={solid,fill=mycolor85,draw=black,line width=0.15pt},
forget plot
]
coordinates{
 (-7.9097,40.660999) 
};
\addplot [
color=blue,
solid,
mark=*,mark size=1.3pt,
mark options={solid,fill=mycolor24,draw=black,line width=0.15pt},
forget plot
]
coordinates{
 (-8.2962,41.444401) 
};
\addplot [
color=blue,
solid,
mark=*,mark size=1.3pt,
mark options={solid,fill=mycolor24,draw=black,line width=0.15pt},
forget plot
]
coordinates{
 (-7.8632,38.015099) 
};
\addplot [
color=blue,
solid,
mark=*,mark size=1.3pt,
mark options={solid,fill=mycolor48,draw=black,line width=0.15pt},
forget plot
]
coordinates{
 (-8.4196,40.205601) 
};
\addplot [
color=blue,
solid,
mark=*,mark size=1.3pt,
mark options={solid,fill=mycolor50,draw=black,line width=0.15pt},
forget plot
]
coordinates{
 (-8.611,41.149601) 
};
\addplot [
color=blue,
solid,
mark=*,mark size=1.3pt,
mark options={solid,fill=mycolor24,draw=black,line width=0.15pt},
forget plot
]
coordinates{
 (-8.6135,41.2146) 
};
\addplot [
color=blue,
solid,
mark=*,mark size=1.3pt,
mark options={solid,fill=mycolor50,draw=black,line width=0.15pt},
forget plot
]
coordinates{
 (-9.1874,38.7952) 
};
\addplot [
color=blue,
solid,
mark=*,mark size=1.3pt,
mark options={solid,fill=mycolor63,draw=black,line width=0.15pt},
forget plot
]
coordinates{
 (51.533298,25.286699) 
};
\addplot [
color=blue,
solid,
mark=*,mark size=1.3pt,
mark options={solid,fill=mycolor70,draw=black,line width=0.15pt},
forget plot
]
coordinates{
 (51.533298,25.286699) 
};
\addplot [
color=blue,
solid,
mark=*,mark size=1.3pt,
mark options={solid,fill=mycolor17,draw=black,line width=0.15pt},
forget plot
]
coordinates{
 (51.533298,25.286699) 
};
\addplot [
color=blue,
solid,
mark=*,mark size=1.3pt,
mark options={solid,fill=mycolor29,draw=black,line width=0.15pt},
forget plot
]
coordinates{
 (51.533298,25.286699) 
};
\addplot [
color=blue,
solid,
mark=*,mark size=1.3pt,
mark options={solid,fill=mycolor23,draw=black,line width=0.15pt},
forget plot
]
coordinates{
 (55.599998,-21.1) 
};
\addplot [
color=blue,
solid,
mark=*,mark size=1.3pt,
mark options={solid,fill=mycolor67,draw=black,line width=0.15pt},
forget plot
]
coordinates{
 (9,51) 
};
\addplot [
color=blue,
solid,
mark=*,mark size=1.3pt,
mark options={solid,fill=mycolor58,draw=black,line width=0.15pt},
forget plot
]
coordinates{
 (138,36) 
};
\addplot [
color=blue,
solid,
mark=*,mark size=1.3pt,
mark options={solid,fill=mycolor53,draw=black,line width=0.15pt},
forget plot
]
coordinates{
 (-110.360703,31.5273) 
};
\addplot [
color=blue,
solid,
mark=*,mark size=1.3pt,
mark options={solid,fill=mycolor23,draw=black,line width=0.15pt},
forget plot
]
coordinates{
 (55.599998,-21.1) 
};
\addplot [
color=blue,
solid,
mark=*,mark size=1.3pt,
mark options={solid,fill=mycolor21,draw=black,line width=0.15pt},
forget plot
]
coordinates{
 (55.466702,-20.866699) 
};
\addplot [
color=blue,
solid,
mark=*,mark size=1.3pt,
mark options={solid,fill=mycolor22,draw=black,line width=0.15pt},
forget plot
]
coordinates{
 (55.483299,-20.883301) 
};
\addplot [
color=blue,
solid,
mark=*,mark size=1.3pt,
mark options={solid,fill=mycolor81,draw=black,line width=0.15pt},
forget plot
]
coordinates{
 (135.616699,34.849998) 
};
\addplot [
color=blue,
solid,
mark=*,mark size=1.3pt,
mark options={solid,fill=mycolor54,draw=black,line width=0.15pt},
forget plot
]
coordinates{
 (55.466702,-20.866699) 
};
\addplot [
color=blue,
solid,
mark=*,mark size=1.3pt,
mark options={solid,fill=mycolor84,draw=black,line width=0.15pt},
forget plot
]
coordinates{
 (55.599998,-21.1) 
};
\addplot [
color=blue,
solid,
mark=*,mark size=1.3pt,
mark options={solid,fill=mycolor62,draw=black,line width=0.15pt},
forget plot
]
coordinates{
 (55.233299,-21.0333) 
};
\addplot [
color=blue,
solid,
mark=*,mark size=1.3pt,
mark options={solid,fill=mycolor88,draw=black,line width=0.15pt},
forget plot
]
coordinates{
 (55.599998,-21.1) 
};
\addplot [
color=blue,
solid,
mark=*,mark size=1.3pt,
mark options={solid,fill=mycolor21,draw=black,line width=0.15pt},
forget plot
]
coordinates{
 (-2,54) 
};
\addplot [
color=blue,
solid,
mark=*,mark size=1.3pt,
mark options={solid,fill=mycolor3,draw=black,line width=0.15pt},
forget plot
]
coordinates{
 (10,53.549999) 
};
\addplot [
color=blue,
solid,
mark=*,mark size=1.3pt,
mark options={solid,fill=mycolor54,draw=black,line width=0.15pt},
forget plot
]
coordinates{
 (55.599998,-21.1) 
};
\addplot [
color=blue,
solid,
mark=*,mark size=1.3pt,
mark options={solid,fill=mycolor81,draw=black,line width=0.15pt},
forget plot
]
coordinates{
 (55.483299,-21.3167) 
};
\addplot [
color=blue,
solid,
mark=*,mark size=1.3pt,
mark options={solid,fill=mycolor81,draw=black,line width=0.15pt},
forget plot
]
coordinates{
 (55.466702,-20.866699) 
};
\addplot [
color=blue,
solid,
mark=*,mark size=1.3pt,
mark options={solid,fill=mycolor32,draw=black,line width=0.15pt},
forget plot
]
coordinates{
 (55.599998,-21.1) 
};
\addplot [
color=blue,
solid,
mark=*,mark size=1.3pt,
mark options={solid,fill=mycolor23,draw=black,line width=0.15pt},
forget plot
]
coordinates{
 (55.599998,-21.1) 
};
\addplot [
color=blue,
solid,
mark=*,mark size=1.3pt,
mark options={solid,fill=mycolor47,draw=black,line width=0.15pt},
forget plot
]
coordinates{
 (55.466702,-20.866699) 
};
\addplot [
color=blue,
solid,
mark=*,mark size=1.3pt,
mark options={solid,fill=mycolor22,draw=black,line width=0.15pt},
forget plot
]
coordinates{
 (55.483299,-21.3167) 
};
\addplot [
color=blue,
solid,
mark=*,mark size=1.3pt,
mark options={solid,fill=mycolor76,draw=black,line width=0.15pt},
forget plot
]
coordinates{
 (4.7283,45.987202) 
};
\addplot [
color=blue,
solid,
mark=*,mark size=1.3pt,
mark options={solid,fill=mycolor88,draw=black,line width=0.15pt},
forget plot
]
coordinates{
 (55.650002,-20.950001) 
};
\addplot [
color=blue,
solid,
mark=*,mark size=1.3pt,
mark options={solid,fill=mycolor23,draw=black,line width=0.15pt},
forget plot
]
coordinates{
 (-82.998802,39.961201) 
};
\addplot [
color=blue,
solid,
mark=*,mark size=1.3pt,
mark options={solid,fill=mycolor23,draw=black,line width=0.15pt},
forget plot
]
coordinates{
 (55.483299,-20.9) 
};
\addplot [
color=blue,
solid,
mark=*,mark size=1.3pt,
mark options={solid,fill=mycolor7,draw=black,line width=0.15pt},
forget plot
]
coordinates{
 (55.516701,-21.266701) 
};
\addplot [
color=blue,
solid,
mark=*,mark size=1.3pt,
mark options={solid,fill=mycolor3,draw=black,line width=0.15pt},
forget plot
]
coordinates{
 (55.599998,-21.1) 
};
\addplot [
color=blue,
solid,
mark=*,mark size=1.3pt,
mark options={solid,fill=mycolor6,draw=black,line width=0.15pt},
forget plot
]
coordinates{
 (55.483299,-21.3167) 
};
\addplot [
color=blue,
solid,
mark=*,mark size=1.3pt,
mark options={solid,fill=mycolor28,draw=black,line width=0.15pt},
forget plot
]
coordinates{
 (25,46) 
};
\addplot [
color=blue,
solid,
mark=*,mark size=1.3pt,
mark options={solid,fill=mycolor30,draw=black,line width=0.15pt},
forget plot
]
coordinates{
 (25,46) 
};
\addplot [
color=blue,
solid,
mark=*,mark size=1.3pt,
mark options={solid,fill=mycolor26,draw=black,line width=0.15pt},
forget plot
]
coordinates{
 (25,46) 
};
\addplot [
color=blue,
solid,
mark=*,mark size=1.3pt,
mark options={solid,fill=mycolor18,draw=black,line width=0.15pt},
forget plot
]
coordinates{
 (26.1,44.4333) 
};
\addplot [
color=blue,
solid,
mark=*,mark size=1.3pt,
mark options={solid,fill=mycolor25,draw=black,line width=0.15pt},
forget plot
]
coordinates{
 (24.2833,44.900002) 
};
\addplot [
color=blue,
solid,
mark=*,mark size=1.3pt,
mark options={solid,fill=mycolor51,draw=black,line width=0.15pt},
forget plot
]
coordinates{
 (26.1,44.4333) 
};
\addplot [
color=blue,
solid,
mark=*,mark size=1.3pt,
mark options={solid,fill=mycolor26,draw=black,line width=0.15pt},
forget plot
]
coordinates{
 (20.4681,44.8186) 
};
\addplot [
color=blue,
solid,
mark=*,mark size=1.3pt,
mark options={solid,fill=mycolor18,draw=black,line width=0.15pt},
forget plot
]
coordinates{
 (22.083099,44.040001) 
};
\addplot [
color=blue,
solid,
mark=*,mark size=1.3pt,
mark options={solid,fill=mycolor18,draw=black,line width=0.15pt},
forget plot
]
coordinates{
 (21.903299,43.324699) 
};
\addplot [
color=blue,
solid,
mark=*,mark size=1.3pt,
mark options={solid,fill=mycolor25,draw=black,line width=0.15pt},
forget plot
]
coordinates{
 (21.1667,42.666698) 
};
\addplot [
color=blue,
solid,
mark=*,mark size=1.3pt,
mark options={solid,fill=mycolor51,draw=black,line width=0.15pt},
forget plot
]
coordinates{
 (21,44) 
};
\addplot [
color=blue,
solid,
mark=*,mark size=1.3pt,
mark options={solid,fill=mycolor59,draw=black,line width=0.15pt},
forget plot
]
coordinates{
 (20.4681,44.8186) 
};
\addplot [
color=blue,
solid,
mark=*,mark size=1.3pt,
mark options={solid,fill=mycolor65,draw=black,line width=0.15pt},
forget plot
]
coordinates{
 (20.9167,44.016701) 
};
\addplot [
color=blue,
solid,
mark=*,mark size=1.3pt,
mark options={solid,fill=mycolor48,draw=black,line width=0.15pt},
forget plot
]
coordinates{
 (20.4681,44.8186) 
};
\addplot [
color=blue,
solid,
mark=*,mark size=1.3pt,
mark options={solid,fill=mycolor18,draw=black,line width=0.15pt},
forget plot
]
coordinates{
 (20.9167,44.016701) 
};
\addplot [
color=blue,
solid,
mark=*,mark size=1.3pt,
mark options={solid,fill=mycolor64,draw=black,line width=0.15pt},
forget plot
]
coordinates{
 (20.4681,44.8186) 
};
\addplot [
color=blue,
solid,
mark=*,mark size=1.3pt,
mark options={solid,fill=mycolor48,draw=black,line width=0.15pt},
forget plot
]
coordinates{
 (21.903299,43.324699) 
};
\addplot [
color=blue,
solid,
mark=*,mark size=1.3pt,
mark options={solid,fill=mycolor51,draw=black,line width=0.15pt},
forget plot
]
coordinates{
 (20.0361,45.6189) 
};
\addplot [
color=blue,
solid,
mark=*,mark size=1.3pt,
mark options={solid,fill=mycolor26,draw=black,line width=0.15pt},
forget plot
]
coordinates{
 (20.4681,44.8186) 
};
\addplot [
color=blue,
solid,
mark=*,mark size=1.3pt,
mark options={solid,fill=mycolor18,draw=black,line width=0.15pt},
forget plot
]
coordinates{
 (20.4681,44.8186) 
};
\addplot [
color=blue,
solid,
mark=*,mark size=1.3pt,
mark options={solid,fill=mycolor25,draw=black,line width=0.15pt},
forget plot
]
coordinates{
 (20.4681,44.8186) 
};
\addplot [
color=blue,
solid,
mark=*,mark size=1.3pt,
mark options={solid,fill=mycolor25,draw=black,line width=0.15pt},
forget plot
]
coordinates{
 (20.4681,44.8186) 
};
\addplot [
color=blue,
solid,
mark=*,mark size=1.3pt,
mark options={solid,fill=mycolor46,draw=black,line width=0.15pt},
forget plot
]
coordinates{
 (20.9167,44.016701) 
};
\addplot [
color=blue,
solid,
mark=*,mark size=1.3pt,
mark options={solid,fill=mycolor26,draw=black,line width=0.15pt},
forget plot
]
coordinates{
 (20.4681,44.8186) 
};
\addplot [
color=blue,
solid,
mark=*,mark size=1.3pt,
mark options={solid,fill=mycolor50,draw=black,line width=0.15pt},
forget plot
]
coordinates{
 (37.615601,55.752201) 
};
\addplot [
color=blue,
solid,
mark=*,mark size=1.3pt,
mark options={solid,fill=mycolor66,draw=black,line width=0.15pt},
forget plot
]
coordinates{
 (56.2841,58.0093) 
};
\addplot [
color=blue,
solid,
mark=*,mark size=1.3pt,
mark options={solid,fill=mycolor45,draw=black,line width=0.15pt},
forget plot
]
coordinates{
 (37.900002,59.133301) 
};
\addplot [
color=blue,
solid,
mark=*,mark size=1.3pt,
mark options={solid,fill=mycolor5,draw=black,line width=0.15pt},
forget plot
]
coordinates{
 (58.326698,51.202999) 
};
\addplot [
color=blue,
solid,
mark=*,mark size=1.3pt,
mark options={solid,fill=mycolor59,draw=black,line width=0.15pt},
forget plot
]
coordinates{
 (56.045601,54.785198) 
};
\addplot [
color=blue,
solid,
mark=*,mark size=1.3pt,
mark options={solid,fill=mycolor54,draw=black,line width=0.15pt},
forget plot
]
coordinates{
 (39.826099,21.426701) 
};
\addplot [
color=blue,
solid,
mark=*,mark size=1.3pt,
mark options={solid,fill=mycolor23,draw=black,line width=0.15pt},
forget plot
]
coordinates{
 (46.715199,24.6537) 
};
\addplot [
color=blue,
solid,
mark=*,mark size=1.3pt,
mark options={solid,fill=mycolor28,draw=black,line width=0.15pt},
forget plot
]
coordinates{
 (30,15) 
};
\addplot [
color=blue,
solid,
mark=*,mark size=1.3pt,
mark options={solid,fill=mycolor76,draw=black,line width=0.15pt},
forget plot
]
coordinates{
 (32.534199,15.5881) 
};
\addplot [
color=blue,
solid,
mark=*,mark size=1.3pt,
mark options={solid,fill=mycolor46,draw=black,line width=0.15pt},
forget plot
]
coordinates{
 (-2,54) 
};
\addplot [
color=blue,
solid,
mark=*,mark size=1.3pt,
mark options={solid,fill=mycolor46,draw=black,line width=0.15pt},
forget plot
]
coordinates{
 (15,62) 
};
\addplot [
color=blue,
solid,
mark=*,mark size=1.3pt,
mark options={solid,fill=mycolor50,draw=black,line width=0.15pt},
forget plot
]
coordinates{
 (21.700001,65.833298) 
};
\addplot [
color=blue,
solid,
mark=*,mark size=1.3pt,
mark options={solid,fill=mycolor48,draw=black,line width=0.15pt},
forget plot
]
coordinates{
 (11.9667,57.716702) 
};
\addplot [
color=blue,
solid,
mark=*,mark size=1.3pt,
mark options={solid,fill=mycolor26,draw=black,line width=0.15pt},
forget plot
]
coordinates{
 (17.9464,58.9044) 
};
\addplot [
color=blue,
solid,
mark=*,mark size=1.3pt,
mark options={solid,fill=mycolor25,draw=black,line width=0.15pt},
forget plot
]
coordinates{
 (2.4786,48.838799) 
};
\addplot [
color=blue,
solid,
mark=*,mark size=1.3pt,
mark options={solid,fill=mycolor43,draw=black,line width=0.15pt},
forget plot
]
coordinates{
 (14.5144,46.055302) 
};
\addplot [
color=blue,
solid,
mark=*,mark size=1.3pt,
mark options={solid,fill=mycolor43,draw=black,line width=0.15pt},
forget plot
]
coordinates{
 (15.6467,46.554699) 
};
\addplot [
color=blue,
solid,
mark=*,mark size=1.3pt,
mark options={solid,fill=mycolor43,draw=black,line width=0.15pt},
forget plot
]
coordinates{
 (14.5144,46.055302) 
};
\addplot [
color=blue,
solid,
mark=*,mark size=1.3pt,
mark options={solid,fill=mycolor46,draw=black,line width=0.15pt},
forget plot
]
coordinates{
 (13.7361,46.186401) 
};
\addplot [
color=blue,
solid,
mark=*,mark size=1.3pt,
mark options={solid,fill=mycolor45,draw=black,line width=0.15pt},
forget plot
]
coordinates{
 (14.6094,46.2197) 
};
\addplot [
color=blue,
solid,
mark=*,mark size=1.3pt,
mark options={solid,fill=mycolor25,draw=black,line width=0.15pt},
forget plot
]
coordinates{
 (15.6467,46.554699) 
};
\addplot [
color=blue,
solid,
mark=*,mark size=1.3pt,
mark options={solid,fill=mycolor18,draw=black,line width=0.15pt},
forget plot
]
coordinates{
 (14.4333,46.133301) 
};
\addplot [
color=blue,
solid,
mark=*,mark size=1.3pt,
mark options={solid,fill=mycolor18,draw=black,line width=0.15pt},
forget plot
]
coordinates{
 (19.5,48.666698) 
};
\addplot [
color=blue,
solid,
mark=*,mark size=1.3pt,
mark options={solid,fill=mycolor18,draw=black,line width=0.15pt},
forget plot
]
coordinates{
 (17.116699,48.150002) 
};
\addplot [
color=blue,
solid,
mark=*,mark size=1.3pt,
mark options={solid,fill=mycolor14,draw=black,line width=0.15pt},
forget plot
]
coordinates{
 (21.25,48.716702) 
};
\addplot [
color=blue,
solid,
mark=*,mark size=1.3pt,
mark options={solid,fill=mycolor46,draw=black,line width=0.15pt},
forget plot
]
coordinates{
 (19.5,48.666698) 
};
\addplot [
color=blue,
solid,
mark=*,mark size=1.3pt,
mark options={solid,fill=mycolor43,draw=black,line width=0.15pt},
forget plot
]
coordinates{
 (17.116699,48.150002) 
};
\addplot [
color=blue,
solid,
mark=*,mark size=1.3pt,
mark options={solid,fill=mycolor26,draw=black,line width=0.15pt},
forget plot
]
coordinates{
 (19.5,48.666698) 
};
\addplot [
color=blue,
solid,
mark=*,mark size=1.3pt,
mark options={solid,fill=mycolor18,draw=black,line width=0.15pt},
forget plot
]
coordinates{
 (19.5,48.666698) 
};
\addplot [
color=blue,
solid,
mark=*,mark size=1.3pt,
mark options={solid,fill=mycolor46,draw=black,line width=0.15pt},
forget plot
]
coordinates{
 (19.5,48.666698) 
};
\addplot [
color=blue,
solid,
mark=*,mark size=1.3pt,
mark options={solid,fill=mycolor43,draw=black,line width=0.15pt},
forget plot
]
coordinates{
 (17.6,48.366699) 
};
\addplot [
color=blue,
solid,
mark=*,mark size=1.3pt,
mark options={solid,fill=mycolor45,draw=black,line width=0.15pt},
forget plot
]
coordinates{
 (17.116699,48.150002) 
};
\addplot [
color=blue,
solid,
mark=*,mark size=1.3pt,
mark options={solid,fill=mycolor26,draw=black,line width=0.15pt},
forget plot
]
coordinates{
 (21.25,48.716702) 
};
\addplot [
color=blue,
solid,
mark=*,mark size=1.3pt,
mark options={solid,fill=mycolor14,draw=black,line width=0.15pt},
forget plot
]
coordinates{
 (21.25,48.716702) 
};
\addplot [
color=blue,
solid,
mark=*,mark size=1.3pt,
mark options={solid,fill=mycolor50,draw=black,line width=0.15pt},
forget plot
]
coordinates{
 (12.4667,43.9333) 
};
\addplot [
color=blue,
solid,
mark=*,mark size=1.3pt,
mark options={solid,fill=mycolor50,draw=black,line width=0.15pt},
forget plot
]
coordinates{
 (12.4667,43.9333) 
};
\addplot [
color=blue,
solid,
mark=*,mark size=1.3pt,
mark options={solid,fill=mycolor58,draw=black,line width=0.15pt},
forget plot
]
coordinates{
 (12.2833,43.883301) 
};
\addplot [
color=blue,
solid,
mark=*,mark size=1.3pt,
mark options={solid,fill=mycolor25,draw=black,line width=0.15pt},
forget plot
]
coordinates{
 (12.5,43.983299) 
};
\addplot [
color=blue,
solid,
mark=*,mark size=1.3pt,
mark options={solid,fill=mycolor85,draw=black,line width=0.15pt},
forget plot
]
coordinates{
 (-17.438101,14.6708) 
};
\addplot [
color=blue,
solid,
mark=*,mark size=1.3pt,
mark options={solid,fill=mycolor70,draw=black,line width=0.15pt},
forget plot
]
coordinates{
 (-14,14) 
};
\addplot [
color=blue,
solid,
mark=*,mark size=1.3pt,
mark options={solid,fill=mycolor71,draw=black,line width=0.15pt},
forget plot
]
coordinates{
 (-14,14) 
};
\addplot [
color=blue,
solid,
mark=*,mark size=1.3pt,
mark options={solid,fill=mycolor85,draw=black,line width=0.15pt},
forget plot
]
coordinates{
 (-14,14) 
};
\addplot [
color=blue,
solid,
mark=*,mark size=1.3pt,
mark options={solid,fill=mycolor69,draw=black,line width=0.15pt},
forget plot
]
coordinates{
 (-89.203102,13.7086) 
};
\addplot [
color=blue,
solid,
mark=*,mark size=1.3pt,
mark options={solid,fill=mycolor69,draw=black,line width=0.15pt},
forget plot
]
coordinates{
 (-89.724197,13.7189) 
};
\addplot [
color=blue,
solid,
mark=*,mark size=1.3pt,
mark options={solid,fill=mycolor69,draw=black,line width=0.15pt},
forget plot
]
coordinates{
 (-89.724197,13.7189) 
};
\addplot [
color=blue,
solid,
mark=*,mark size=1.3pt,
mark options={solid,fill=mycolor63,draw=black,line width=0.15pt},
forget plot
]
coordinates{
 (-89.203102,13.7086) 
};
\addplot [
color=blue,
solid,
mark=*,mark size=1.3pt,
mark options={solid,fill=mycolor63,draw=black,line width=0.15pt},
forget plot
]
coordinates{
 (-89.566704,14.15) 
};
\addplot [
color=blue,
solid,
mark=*,mark size=1.3pt,
mark options={solid,fill=mycolor77,draw=black,line width=0.15pt},
forget plot
]
coordinates{
 (-89.203102,13.7086) 
};
\addplot [
color=blue,
solid,
mark=*,mark size=1.3pt,
mark options={solid,fill=mycolor29,draw=black,line width=0.15pt},
forget plot
]
coordinates{
 (-89.566704,14.15) 
};
\addplot [
color=blue,
solid,
mark=*,mark size=1.3pt,
mark options={solid,fill=mycolor63,draw=black,line width=0.15pt},
forget plot
]
coordinates{
 (-89.203102,13.7086) 
};
\addplot [
color=blue,
solid,
mark=*,mark size=1.3pt,
mark options={solid,fill=mycolor77,draw=black,line width=0.15pt},
forget plot
]
coordinates{
 (-89.279701,13.6769) 
};
\addplot [
color=blue,
solid,
mark=*,mark size=1.3pt,
mark options={solid,fill=mycolor82,draw=black,line width=0.15pt},
forget plot
]
coordinates{
 (31.5,-26.5) 
};
\addplot [
color=blue,
solid,
mark=*,mark size=1.3pt,
mark options={solid,fill=mycolor80,draw=black,line width=0.15pt},
forget plot
]
coordinates{
 (31.5,-26.5) 
};
\addplot [
color=blue,
solid,
mark=*,mark size=1.3pt,
mark options={solid,fill=mycolor68,draw=black,line width=0.15pt},
forget plot
]
coordinates{
 (31.5,-26.5) 
};
\addplot [
color=blue,
solid,
mark=*,mark size=1.3pt,
mark options={solid,fill=mycolor71,draw=black,line width=0.15pt},
forget plot
]
coordinates{
 (31.133301,-26.3167) 
};
\addplot [
color=blue,
solid,
mark=*,mark size=1.3pt,
mark options={solid,fill=mycolor77,draw=black,line width=0.15pt},
forget plot
]
coordinates{
 (-71.133301,21.4667) 
};
\addplot [
color=blue,
solid,
mark=*,mark size=1.3pt,
mark options={solid,fill=mycolor29,draw=black,line width=0.15pt},
forget plot
]
coordinates{
 (-71.133301,21.4667) 
};
\addplot [
color=blue,
solid,
mark=*,mark size=1.3pt,
mark options={solid,fill=mycolor54,draw=black,line width=0.15pt},
forget plot
]
coordinates{
 (1.1667,8) 
};
\addplot [
color=blue,
solid,
mark=*,mark size=1.3pt,
mark options={solid,fill=mycolor87,draw=black,line width=0.15pt},
forget plot
]
coordinates{
 (99.832497,19.9086) 
};
\addplot [
color=blue,
solid,
mark=*,mark size=1.3pt,
mark options={solid,fill=mycolor80,draw=black,line width=0.15pt},
forget plot
]
coordinates{
 (71,39) 
};
\addplot [
color=blue,
solid,
mark=*,mark size=1.3pt,
mark options={solid,fill=mycolor76,draw=black,line width=0.15pt},
forget plot
]
coordinates{
 (58.383301,37.950001) 
};
\addplot [
color=blue,
solid,
mark=*,mark size=1.3pt,
mark options={solid,fill=mycolor70,draw=black,line width=0.15pt},
forget plot
]
coordinates{
 (58.383301,37.950001) 
};
\addplot [
color=blue,
solid,
mark=*,mark size=1.3pt,
mark options={solid,fill=mycolor44,draw=black,line width=0.15pt},
forget plot
]
coordinates{
 (10.1711,36.806099) 
};
\addplot [
color=blue,
solid,
mark=*,mark size=1.3pt,
mark options={solid,fill=mycolor65,draw=black,line width=0.15pt},
forget plot
]
coordinates{
 (10.1711,36.806099) 
};
\addplot [
color=blue,
solid,
mark=*,mark size=1.3pt,
mark options={solid,fill=mycolor86,draw=black,line width=0.15pt},
forget plot
]
coordinates{
 (9,34) 
};
\addplot [
color=blue,
solid,
mark=*,mark size=1.3pt,
mark options={solid,fill=mycolor50,draw=black,line width=0.15pt},
forget plot
]
coordinates{
 (28.964701,41.0186) 
};
\addplot [
color=blue,
solid,
mark=*,mark size=1.3pt,
mark options={solid,fill=mycolor66,draw=black,line width=0.15pt},
forget plot
]
coordinates{
 (28.964701,41.0186) 
};
\addplot [
color=blue,
solid,
mark=*,mark size=1.3pt,
mark options={solid,fill=mycolor30,draw=black,line width=0.15pt},
forget plot
]
coordinates{
 (29.916901,40.766899) 
};
\addplot [
color=blue,
solid,
mark=*,mark size=1.3pt,
mark options={solid,fill=mycolor69,draw=black,line width=0.15pt},
forget plot
]
coordinates{
 (-61.516701,10.65) 
};
\addplot [
color=blue,
solid,
mark=*,mark size=1.3pt,
mark options={solid,fill=mycolor79,draw=black,line width=0.15pt},
forget plot
]
coordinates{
 (-61.450001,10.65) 
};
\addplot [
color=blue,
solid,
mark=*,mark size=1.3pt,
mark options={solid,fill=mycolor79,draw=black,line width=0.15pt},
forget plot
]
coordinates{
 (-61.466702,10.2833) 
};
\addplot [
color=blue,
solid,
mark=*,mark size=1.3pt,
mark options={solid,fill=mycolor42,draw=black,line width=0.15pt},
forget plot
]
coordinates{
 (-61.283298,10.6333) 
};
\addplot [
color=blue,
solid,
mark=*,mark size=1.3pt,
mark options={solid,fill=mycolor47,draw=black,line width=0.15pt},
forget plot
]
coordinates{
 (-61.466702,10.2833) 
};
\addplot [
color=blue,
solid,
mark=*,mark size=1.3pt,
mark options={solid,fill=mycolor17,draw=black,line width=0.15pt},
forget plot
]
coordinates{
 (-61.466702,10.2833) 
};
\addplot [
color=blue,
solid,
mark=*,mark size=1.3pt,
mark options={solid,fill=mycolor17,draw=black,line width=0.15pt},
forget plot
]
coordinates{
 (-61.466702,10.2833) 
};
\addplot [
color=blue,
solid,
mark=*,mark size=1.3pt,
mark options={solid,fill=mycolor22,draw=black,line width=0.15pt},
forget plot
]
coordinates{
 (-61.516701,10.65) 
};
\addplot [
color=blue,
solid,
mark=*,mark size=1.3pt,
mark options={solid,fill=mycolor80,draw=black,line width=0.15pt},
forget plot
]
coordinates{
 (-61.516701,10.65) 
};
\addplot [
color=blue,
solid,
mark=*,mark size=1.3pt,
mark options={solid,fill=mycolor29,draw=black,line width=0.15pt},
forget plot
]
coordinates{
 (-61.516701,10.65) 
};
\addplot [
color=blue,
solid,
mark=*,mark size=1.3pt,
mark options={solid,fill=mycolor42,draw=black,line width=0.15pt},
forget plot
]
coordinates{
 (-61.416698,10.65) 
};
\addplot [
color=blue,
solid,
mark=*,mark size=1.3pt,
mark options={solid,fill=mycolor80,draw=black,line width=0.15pt},
forget plot
]
coordinates{
 (-61.516701,10.65) 
};
\addplot [
color=blue,
solid,
mark=*,mark size=1.3pt,
mark options={solid,fill=mycolor69,draw=black,line width=0.15pt},
forget plot
]
coordinates{
 (-60.733299,11.1833) 
};
\addplot [
color=blue,
solid,
mark=*,mark size=1.3pt,
mark options={solid,fill=mycolor62,draw=black,line width=0.15pt},
forget plot
]
coordinates{
 (121.525002,25.0392) 
};
\addplot [
color=blue,
solid,
mark=*,mark size=1.3pt,
mark options={solid,fill=mycolor37,draw=black,line width=0.15pt},
forget plot
]
coordinates{
 (121.525002,25.0392) 
};
\addplot [
color=blue,
solid,
mark=*,mark size=1.3pt,
mark options={solid,fill=mycolor41,draw=black,line width=0.15pt},
forget plot
]
coordinates{
 (120.681396,24.143299) 
};
\addplot [
color=blue,
solid,
mark=*,mark size=1.3pt,
mark options={solid,fill=mycolor72,draw=black,line width=0.15pt},
forget plot
]
coordinates{
 (121.525002,25.0392) 
};
\addplot [
color=blue,
solid,
mark=*,mark size=1.3pt,
mark options={solid,fill=mycolor78,draw=black,line width=0.15pt},
forget plot
]
coordinates{
 (35,-6) 
};
\addplot [
color=blue,
solid,
mark=*,mark size=1.3pt,
mark options={solid,fill=mycolor18,draw=black,line width=0.15pt},
forget plot
]
coordinates{
 (31.9974,46.9659) 
};
\addplot [
color=blue,
solid,
mark=*,mark size=1.3pt,
mark options={solid,fill=mycolor48,draw=black,line width=0.15pt},
forget plot
]
coordinates{
 (30.7386,46.463902) 
};
\addplot [
color=blue,
solid,
mark=*,mark size=1.3pt,
mark options={solid,fill=mycolor41,draw=black,line width=0.15pt},
forget plot
]
coordinates{
 (30.7386,46.463902) 
};
\addplot [
color=blue,
solid,
mark=*,mark size=1.3pt,
mark options={solid,fill=mycolor42,draw=black,line width=0.15pt},
forget plot
]
coordinates{
 (32,49) 
};
\addplot [
color=blue,
solid,
mark=*,mark size=1.3pt,
mark options={solid,fill=mycolor50,draw=black,line width=0.15pt},
forget plot
]
coordinates{
 (30.516701,50.4333) 
};
\addplot [
color=blue,
solid,
mark=*,mark size=1.3pt,
mark options={solid,fill=mycolor74,draw=black,line width=0.15pt},
forget plot
]
coordinates{
 (-96.891502,32.9478) 
};
\addplot [
color=blue,
solid,
mark=*,mark size=1.3pt,
mark options={solid,fill=mycolor74,draw=black,line width=0.15pt},
forget plot
]
coordinates{
 (-96.891502,32.9478) 
};
\addplot [
color=blue,
solid,
mark=*,mark size=1.3pt,
mark options={solid,fill=mycolor52,draw=black,line width=0.15pt},
forget plot
]
coordinates{
 (-96.891502,32.9478) 
};
\addplot [
color=blue,
solid,
mark=*,mark size=1.3pt,
mark options={solid,fill=mycolor42,draw=black,line width=0.15pt},
forget plot
]
coordinates{
 (-105.115303,41.982601) 
};
\addplot [
color=blue,
solid,
mark=*,mark size=1.3pt,
mark options={solid,fill=mycolor70,draw=black,line width=0.15pt},
forget plot
]
coordinates{
 (-92.003197,34.228401) 
};
\addplot [
color=blue,
solid,
mark=*,mark size=1.3pt,
mark options={solid,fill=mycolor31,draw=black,line width=0.15pt},
forget plot
]
coordinates{
 (-71.471901,42.076599) 
};
\addplot [
color=blue,
solid,
mark=*,mark size=1.3pt,
mark options={solid,fill=mycolor16,draw=black,line width=0.15pt},
forget plot
]
coordinates{
 (64,41) 
};
\addplot [
color=blue,
solid,
mark=*,mark size=1.3pt,
mark options={solid,fill=mycolor75,draw=black,line width=0.15pt},
forget plot
]
coordinates{
 (69.25,41.3167) 
};
\addplot [
color=blue,
solid,
mark=*,mark size=1.3pt,
mark options={solid,fill=mycolor31,draw=black,line width=0.15pt},
forget plot
]
coordinates{
 (64,41) 
};
\addplot [
color=blue,
solid,
mark=*,mark size=1.3pt,
mark options={solid,fill=mycolor75,draw=black,line width=0.15pt},
forget plot
]
coordinates{
 (64,41) 
};
\addplot [
color=blue,
solid,
mark=*,mark size=1.3pt,
mark options={solid,fill=mycolor31,draw=black,line width=0.15pt},
forget plot
]
coordinates{
 (64,41) 
};
\addplot [
color=blue,
solid,
mark=*,mark size=1.3pt,
mark options={solid,fill=mycolor76,draw=black,line width=0.15pt},
forget plot
]
coordinates{
 (64,41) 
};
\addplot [
color=blue,
solid,
mark=*,mark size=1.3pt,
mark options={solid,fill=mycolor65,draw=black,line width=0.15pt},
forget plot
]
coordinates{
 (-76.266701,-10.6833) 
};
\addplot [
color=blue,
solid,
mark=*,mark size=1.3pt,
mark options={solid,fill=mycolor86,draw=black,line width=0.15pt},
forget plot
]
coordinates{
 (-74.005997,40.714298) 
};
\addplot [
color=blue,
solid,
mark=*,mark size=1.3pt,
mark options={solid,fill=mycolor86,draw=black,line width=0.15pt},
forget plot
]
coordinates{
 (12.5,43.950001) 
};
\addplot [
color=blue,
solid,
mark=*,mark size=1.3pt,
mark options={solid,fill=mycolor80,draw=black,line width=0.15pt},
forget plot
]
coordinates{
 (-61.216702,13.1333) 
};
\addplot [
color=blue,
solid,
mark=*,mark size=1.3pt,
mark options={solid,fill=mycolor78,draw=black,line width=0.15pt},
forget plot
]
coordinates{
 (-61.216702,13.1333) 
};
\addplot [
color=blue,
solid,
mark=*,mark size=1.3pt,
mark options={solid,fill=mycolor17,draw=black,line width=0.15pt},
forget plot
]
coordinates{
 (-61.216702,13.1333) 
};
\addplot [
color=blue,
solid,
mark=*,mark size=1.3pt,
mark options={solid,fill=mycolor80,draw=black,line width=0.15pt},
forget plot
]
coordinates{
 (-61.216702,13.1333) 
};
\addplot [
color=blue,
solid,
mark=*,mark size=1.3pt,
mark options={solid,fill=mycolor78,draw=black,line width=0.15pt},
forget plot
]
coordinates{
 (-61.216702,13.1333) 
};
\addplot [
color=blue,
solid,
mark=*,mark size=1.3pt,
mark options={solid,fill=mycolor42,draw=black,line width=0.15pt},
forget plot
]
coordinates{
 (-61.216702,13.1333) 
};
\addplot [
color=blue,
solid,
mark=*,mark size=1.3pt,
mark options={solid,fill=mycolor78,draw=black,line width=0.15pt},
forget plot
]
coordinates{
 (-61.216702,13.1333) 
};
\addplot [
color=blue,
solid,
mark=*,mark size=1.3pt,
mark options={solid,fill=mycolor63,draw=black,line width=0.15pt},
forget plot
]
coordinates{
 (-61.216702,13.1333) 
};
\addplot [
color=blue,
solid,
mark=*,mark size=1.3pt,
mark options={solid,fill=mycolor16,draw=black,line width=0.15pt},
forget plot
]
coordinates{
 (-61.216702,13.1333) 
};
\addplot [
color=blue,
solid,
mark=*,mark size=1.3pt,
mark options={solid,fill=mycolor69,draw=black,line width=0.15pt},
forget plot
]
coordinates{
 (-61.216702,13.1333) 
};
\addplot [
color=blue,
solid,
mark=*,mark size=1.3pt,
mark options={solid,fill=mycolor29,draw=black,line width=0.15pt},
forget plot
]
coordinates{
 (-61.216702,13.1333) 
};
\addplot [
color=blue,
solid,
mark=*,mark size=1.3pt,
mark options={solid,fill=mycolor63,draw=black,line width=0.15pt},
forget plot
]
coordinates{
 (-61.216702,13.1333) 
};
\addplot [
color=blue,
solid,
mark=*,mark size=1.3pt,
mark options={solid,fill=mycolor78,draw=black,line width=0.15pt},
forget plot
]
coordinates{
 (-61.216702,13.1333) 
};
\addplot [
color=blue,
solid,
mark=*,mark size=1.3pt,
mark options={solid,fill=mycolor23,draw=black,line width=0.15pt},
forget plot
]
coordinates{
 (-61.216702,13.1333) 
};
\addplot [
color=blue,
solid,
mark=*,mark size=1.3pt,
mark options={solid,fill=mycolor63,draw=black,line width=0.15pt},
forget plot
]
coordinates{
 (-61.216702,13.1333) 
};
\addplot [
color=blue,
solid,
mark=*,mark size=1.3pt,
mark options={solid,fill=mycolor42,draw=black,line width=0.15pt},
forget plot
]
coordinates{
 (-61.216702,13.1333) 
};
\addplot [
color=blue,
solid,
mark=*,mark size=1.3pt,
mark options={solid,fill=mycolor63,draw=black,line width=0.15pt},
forget plot
]
coordinates{
 (-61.216702,13.1333) 
};
\addplot [
color=blue,
solid,
mark=*,mark size=1.3pt,
mark options={solid,fill=mycolor68,draw=black,line width=0.15pt},
forget plot
]
coordinates{
 (-64.166702,10.4667) 
};
\addplot [
color=blue,
solid,
mark=*,mark size=1.3pt,
mark options={solid,fill=mycolor23,draw=black,line width=0.15pt},
forget plot
]
coordinates{
 (-71.640602,10.6317) 
};
\addplot [
color=blue,
solid,
mark=*,mark size=1.3pt,
mark options={solid,fill=mycolor74,draw=black,line width=0.15pt},
forget plot
]
coordinates{
 (-64.616699,18.4167) 
};
\addplot [
color=blue,
solid,
mark=*,mark size=1.3pt,
mark options={solid,fill=mycolor79,draw=black,line width=0.15pt},
forget plot
]
coordinates{
 (-64.616699,18.4167) 
};
\addplot [
color=blue,
solid,
mark=*,mark size=1.3pt,
mark options={solid,fill=mycolor54,draw=black,line width=0.15pt},
forget plot
]
coordinates{
 (-64.616699,18.4167) 
};
\addplot [
color=blue,
solid,
mark=*,mark size=1.3pt,
mark options={solid,fill=mycolor79,draw=black,line width=0.15pt},
forget plot
]
coordinates{
 (-64.616699,18.4167) 
};
\addplot [
color=blue,
solid,
mark=*,mark size=1.3pt,
mark options={solid,fill=mycolor58,draw=black,line width=0.15pt},
forget plot
]
coordinates{
 (-64.616699,18.4167) 
};
\addplot [
color=blue,
solid,
mark=*,mark size=1.3pt,
mark options={solid,fill=mycolor86,draw=black,line width=0.15pt},
forget plot
]
coordinates{
 (-64.616699,18.4167) 
};
\addplot [
color=blue,
solid,
mark=*,mark size=1.3pt,
mark options={solid,fill=mycolor64,draw=black,line width=0.15pt},
forget plot
]
coordinates{
 (-64.616699,18.4167) 
};
\addplot [
color=blue,
solid,
mark=*,mark size=1.3pt,
mark options={solid,fill=mycolor63,draw=black,line width=0.15pt},
forget plot
]
coordinates{
 (-64.616699,18.4167) 
};
\addplot [
color=blue,
solid,
mark=*,mark size=1.3pt,
mark options={solid,fill=mycolor39,draw=black,line width=0.15pt},
forget plot
]
coordinates{
 (106.643799,10.8142) 
};
\addplot [
color=blue,
solid,
mark=*,mark size=1.3pt,
mark options={solid,fill=mycolor35,draw=black,line width=0.15pt},
forget plot
]
coordinates{
 (106.643799,10.8142) 
};
\addplot [
color=blue,
solid,
mark=*,mark size=1.3pt,
mark options={solid,fill=mycolor67,draw=black,line width=0.15pt},
forget plot
]
coordinates{
 (106.643799,10.8142) 
};
\addplot [
color=blue,
solid,
mark=*,mark size=1.3pt,
mark options={solid,fill=mycolor57,draw=black,line width=0.15pt},
forget plot
]
coordinates{
 (105.783302,10.0333) 
};
\addplot [
color=blue,
solid,
mark=*,mark size=1.3pt,
mark options={solid,fill=mycolor38,draw=black,line width=0.15pt},
forget plot
]
coordinates{
 (-172.333298,-13.5833) 
};
\addplot [
color=blue,
solid,
mark=*,mark size=1.3pt,
mark options={solid,fill=mycolor82,draw=black,line width=0.15pt},
forget plot
]
coordinates{
 (48,15) 
};
\addplot [
color=blue,
solid,
mark=*,mark size=1.3pt,
mark options={solid,fill=mycolor16,draw=black,line width=0.15pt},
forget plot
]
coordinates{
 (48,15) 
};
\addplot [
color=blue,
solid,
mark=*,mark size=1.3pt,
mark options={solid,fill=mycolor23,draw=black,line width=0.15pt},
forget plot
]
coordinates{
 (48,15) 
};
\addplot [
color=blue,
solid,
mark=*,mark size=1.3pt,
mark options={solid,fill=mycolor16,draw=black,line width=0.15pt},
forget plot
]
coordinates{
 (48,15) 
};
\addplot [
color=blue,
solid,
mark=*,mark size=1.3pt,
mark options={solid,fill=mycolor29,draw=black,line width=0.15pt},
forget plot
]
coordinates{
 (24,-29) 
};
\addplot [
color=blue,
solid,
mark=*,mark size=1.3pt,
mark options={solid,fill=mycolor82,draw=black,line width=0.15pt},
forget plot
]
coordinates{
 (28.2833,-15.4167) 
};
\addplot [
color=blue,
solid,
mark=*,mark size=1.3pt,
mark options={solid,fill=mycolor68,draw=black,line width=0.15pt},
forget plot
]
coordinates{
 (30,-15) 
};
\addplot [
color=blue,
solid,
mark=*,mark size=1.3pt,
mark options={solid,fill=mycolor14,draw=black,line width=0.15pt},
forget plot
]
coordinates{
 (8.7842,47.202202) 
};
\addplot [
color=blue,
solid,
mark=*,mark size=1.3pt,
mark options={solid,fill=mycolor43,draw=black,line width=0.15pt},
forget plot
]
coordinates{
 (6.1481,46.195599) 
};
\addplot [
color=blue,
solid,
mark=*,mark size=1.3pt,
mark options={solid,fill=mycolor43,draw=black,line width=0.15pt},
forget plot
]
coordinates{
 (7.4667,46.916698) 
};
\addplot [
color=blue,
solid,
mark=*,mark size=1.3pt,
mark options={solid,fill=mycolor14,draw=black,line width=0.15pt},
forget plot
]
coordinates{
 (8.6437,47.270401) 
};
\addplot [
color=blue,
solid,
mark=*,mark size=1.3pt,
mark options={solid,fill=mycolor14,draw=black,line width=0.15pt},
forget plot
]
coordinates{
 (9.3776,47.4254) 
};
\addplot [
color=blue,
solid,
mark=*,mark size=1.3pt,
mark options={solid,fill=mycolor49,draw=black,line width=0.15pt},
forget plot
]
coordinates{
 (8,47) 
};
\addplot [
color=blue,
solid,
mark=*,mark size=1.3pt,
mark options={solid,fill=mycolor43,draw=black,line width=0.15pt},
forget plot
]
coordinates{
 (6.6667,46.533298) 
};
\addplot [
color=blue,
solid,
mark=*,mark size=1.3pt,
mark options={solid,fill=mycolor43,draw=black,line width=0.15pt},
forget plot
]
coordinates{
 (7.4667,46.916698) 
};
\addplot [
color=blue,
solid,
mark=*,mark size=1.3pt,
mark options={solid,fill=mycolor43,draw=black,line width=0.15pt},
forget plot
]
coordinates{
 (8,47) 
};
\addplot [
color=blue,
solid,
mark=*,mark size=1.3pt,
mark options={solid,fill=mycolor75,draw=black,line width=0.15pt},
forget plot
]
coordinates{
 (4.7963,43.968399) 
};
\addplot [
color=blue,
solid,
mark=*,mark size=1.3pt,
mark options={solid,fill=mycolor85,draw=black,line width=0.15pt},
forget plot
]
coordinates{
 (139.751404,35.685001) 
};
\addplot [
color=blue,
solid,
mark=*,mark size=1.3pt,
mark options={solid,fill=mycolor24,draw=black,line width=0.15pt},
forget plot
]
coordinates{
 (-0.6338,44.8325) 
};
\addplot [
color=blue,
solid,
mark=*,mark size=1.3pt,
mark options={solid,fill=mycolor29,draw=black,line width=0.15pt},
forget plot
]
coordinates{
 (2.3056,48.758301) 
};
\addplot [
color=blue,
solid,
mark=*,mark size=1.3pt,
mark options={solid,fill=mycolor43,draw=black,line width=0.15pt},
forget plot
]
coordinates{
 (-8,53) 
};
\addplot [
color=blue,
solid,
mark=*,mark size=1.3pt,
mark options={solid,fill=mycolor66,draw=black,line width=0.15pt},
forget plot
]
coordinates{
 (120.595398,31.3041) 
};
\addplot [
color=blue,
solid,
mark=*,mark size=1.3pt,
mark options={solid,fill=mycolor46,draw=black,line width=0.15pt},
forget plot
]
coordinates{
 (2,46) 
};
\addplot [
color=blue,
solid,
mark=*,mark size=1.3pt,
mark options={solid,fill=mycolor24,draw=black,line width=0.15pt},
forget plot
]
coordinates{
 (2,46) 
};
\addplot [
color=blue,
solid,
mark=*,mark size=1.3pt,
mark options={solid,fill=mycolor85,draw=black,line width=0.15pt},
forget plot
]
coordinates{
 (-73.263702,41.141201) 
};
\addplot [
color=blue,
solid,
mark=*,mark size=1.3pt,
mark options={solid,fill=mycolor50,draw=black,line width=0.15pt},
forget plot
]
coordinates{
 (23.733299,37.983299) 
};
\addplot [
color=blue,
solid,
mark=*,mark size=1.3pt,
mark options={solid,fill=mycolor17,draw=black,line width=0.15pt},
forget plot
]
coordinates{
 (22,39) 
};
\addplot [
color=blue,
solid,
mark=*,mark size=1.3pt,
mark options={solid,fill=mycolor66,draw=black,line width=0.15pt},
forget plot
]
coordinates{
 (23.733299,37.983299) 
};
\addplot [
color=blue,
solid,
mark=*,mark size=1.3pt,
mark options={solid,fill=mycolor27,draw=black,line width=0.15pt},
forget plot
]
coordinates{
 (116.388298,39.928902) 
};
\addplot [
color=blue,
solid,
mark=*,mark size=1.3pt,
mark options={solid,fill=mycolor65,draw=black,line width=0.15pt},
forget plot
]
coordinates{
 (23.733299,37.983299) 
};
\addplot [
color=blue,
solid,
mark=*,mark size=1.3pt,
mark options={solid,fill=mycolor86,draw=black,line width=0.15pt},
forget plot
]
coordinates{
 (23.733299,37.983299) 
};
\addplot [
color=blue,
solid,
mark=*,mark size=1.3pt,
mark options={solid,fill=mycolor27,draw=black,line width=0.15pt},
forget plot
]
coordinates{
 (23.700001,37.950001) 
};
\addplot [
color=blue,
solid,
mark=*,mark size=1.3pt,
mark options={solid,fill=mycolor59,draw=black,line width=0.15pt},
forget plot
]
coordinates{
 (23.733299,37.983299) 
};
\addplot [
color=blue,
solid,
mark=*,mark size=1.3pt,
mark options={solid,fill=mycolor27,draw=black,line width=0.15pt},
forget plot
]
coordinates{
 (13.3333,47.333302) 
};
\addplot [
color=blue,
solid,
mark=*,mark size=1.3pt,
mark options={solid,fill=mycolor86,draw=black,line width=0.15pt},
forget plot
]
coordinates{
 (112.453598,34.683601) 
};
\end{axis}
\end{tikzpicture}%

\begin{figure}[tb]
  \centering
  \adjincludegraphics[width=4in]{figures/map}
  \caption{Round-trip times in milliseconds obtained by pinging from Zurich
           1768 servers around the world.}
  \label{fig:map}
\end{figure}

\paragraph{Datasets 2 \& 3: Environmental monitoring}
Our second and third datasets come from the domain of environmental monitoring
of inland
waters and consist of $2024$ \emph{in situ} measurements of chlorophyll
and \emph{Planktothrix rubescens}\footnote{Planktothrix rubescens is a genus of
blue-green algae that can produce toxins.}
concentration respectively, which were collected by an autonomous surface
vessel within a vertical transect plane of Lake Zurich \cite{hitz12}.
Monitoring chlorophyll and algae concentration is useful in analyzing
limnological phenomena such as algal bloom. Since the concentration levels can
vary throughout the year, in addition to having a fixed threshold
concentration, it can also be useful to be able to detect
relative ``hotspots'' of chlorophyll or algae, i.e. regions of high
concentration with respect to the current maximum. Furthermore, selecting
batches of points can be used to plan sampling paths and reduce the required
traveling distances.

In our evaluation, we used $10,000$ points sampled in a $100 \times 100$ grid
from the GP posteriors that were derived using the $2024$ original measurements
(see \figsref{fig:limno_chl} and \ref{fig:limno_bgape}).
Again, anisotropic Mat\'{e}rn-5 kernels were used and suitable
hyperparameters were fitted by maximizing the
likelihood of two different chlorophyll and algae concentration datasets from
the same lake.
As illustrated in \figsref{fig:limno_chl} and \ref{fig:limno_bgape}, we used
explicit threshold levels of $h = 1.3$ RFU (relative fluorescence units) for the
chlorophyll dataset and
$h = 7$ RFU for the algae concentration dataset. For the implicit threshold
experiments, we chose the values of $\omega$ so that the resulting implicit
levels are identical to the explicit ones, which enables us to compare the two
settings on equal ground.
%The effect of batch sampling on the required traveling distance was evaluated
%using an approximate Euclidean TSP solver to create paths connecting each
%batch of samples selected by \bacl, as described in \sectref{sect:pp}.

\paragraph{Evaluation methodology}
We assess the classification accuracy for all algorithms using the $F_1$-score,
i.e. the harmonic mean of precision and recall, by considering points in the
super- and sublevel sets as positives and negatives respectively.

Note that the execution of the algorithms is not deterministic, because
ties in the next point selection rules are resolved uniformly at random.
Additionally, in the case of the environmental monitoring datasets, sampling
from the GP posterior inherently involves randomness because of the noise
included in the GP model. For these reasons, in our evaluation we repeated
multiple executions of each algorithm. In particular, \str and its
extensions as well as \var were executed $50$ times on each dataset and the
$F_1$-score at each iteration of each execution was computed by classifying
points according to the posterior mean of the GP trained on the already
selected points, i.e.
\begin{align*}
H_t &= \{\*x \in D \mid \mu_t(\*x) \geq h\}\\
L_t &= \{\*x \in D \mid \mu_t(\*x) < h\}.
\end{align*}
On the other hand, \acl and its extensions were evaluated by repeatedly
executing each algorithm for a range of values of the accuracy parameter
$\epsilon$ and  recording the total number of samples and the $F_1$-score after
each execution's termination (in total $2000$ executions per algorithm
per dataset).
The parameter $\epsilon$ was chosen to increase
exponentially within a suitable range depending on the experiment (roughly
between $1\%$ and $20\%$ of the respective threshold level).

%\setlength\figureheight{1.3in}\setlength\figurewidth{2.1in}
%% This file was created by matlab2tikz v0.2.3.
% Copyright (c) 2008--2012, Nico Schlömer <nico.schloemer@gmail.com>
% All rights reserved.
% 
% 
% 
\begin{tikzpicture}

\begin{axis}[%
tick label style={font=\tiny},
label style={font=\tiny},
label shift={-4pt},
xlabel shift={-6pt},
legend style={font=\tiny},
view={0}{90},
width=\figurewidth,
height=\figureheight,
scale only axis,
xmin=0, xmax=200,
xlabel={Samples},
ymin=0.48, ymax=0.85,
ylabel={$F_1$-score},
axis lines*=left,
legend cell align=left,
legend style={at={(1.03,0)},anchor=south east,fill=none,draw=none,align=left,row sep=-0.2em},
clip=false]

\addplot [
color=red,
densely dotted,
line width=1.0pt,
]
coordinates{
 (1,0.546400352044879)(2,0.56166869343547)(3,0.560876391398024)(4,0.579675310166485)(5,0.598979796761476)(6,0.63668436247015)(7,0.657055543732906)(8,0.664697717531494)(9,0.668581423455002)(10,0.675491451952887)(11,0.69459143800243)(12,0.702230421003333)(13,0.70480967603004)(14,0.708690785600837)(15,0.709135806646603)(16,0.709939160878352)(17,0.711812572843289)(18,0.711518608295787)(19,0.709518011389869)(20,0.711239139206519)(21,0.710496724962167)(22,0.71123534139012)(23,0.711723211344969)(24,0.710519197014501)(25,0.713011254715555)(26,0.713171659382865)(27,0.71338657346086)(28,0.716348613291698)(29,0.717096043175512)(30,0.713536643297698)(31,0.714733170311306)(32,0.727857508299209)(33,0.730780959635134)(34,0.730074201488478)(35,0.733674172941041)(36,0.736030065386755)(37,0.736723853888598)(38,0.734497035860791)(39,0.735000078229227)(40,0.734721188804642)(41,0.73279906817303)(42,0.733947315032938)(43,0.734058017856816)(44,0.742824658490753)(45,0.741495584772496)(46,0.742315981008647)(47,0.740704468921484)(48,0.74031591517697)(49,0.740197136013231)(50,0.741286002739978)(51,0.74539364890449)(52,0.745880766898206)(53,0.745789697161033)(54,0.749317852639895)(55,0.753509263458708)(56,0.755407232589305)(57,0.758155107462861)(58,0.755967495678694)(59,0.755666497396904)(60,0.758144364252442)(61,0.761012280488404)(62,0.76035590433262)(63,0.76504255111696)(64,0.766776885522283)(65,0.770304091296343)(66,0.772227950496226)(67,0.772930692583039)(68,0.773544441891429)(69,0.773733148341332)(70,0.772971734766962)(71,0.772696986971335)(72,0.774144426823316)(73,0.775917488363573)(74,0.775263356426386)(75,0.776757712667269)(76,0.778483701672349)(77,0.780064076341726)(78,0.779895259912255)(79,0.780734379756713)(80,0.781095245042932)(81,0.78130859816154)(82,0.781868233549773)(83,0.783615687630672)(84,0.784081820385593)(85,0.785363983058159)(86,0.78528362539496)(87,0.786164155639427)(88,0.784934164199082)(89,0.785889854876922)(90,0.785921175799231)(91,0.787322359375225)(92,0.78872249423461)(93,0.786757080400144)(94,0.787418239648933)(95,0.787581104807533)(96,0.788089005159753)(97,0.788640775855612)(98,0.789884366037676)(99,0.789708564045549)(100,0.7899960002653)(101,0.79022853411162)(102,0.789352431226563)(103,0.78946154285715)(104,0.790020339756281)(105,0.790835563472153)(106,0.790598296177559)(107,0.790164954625906)(108,0.789358209911386)(109,0.789241239962202)(110,0.789705763758909)(111,0.78987058447003)(112,0.788508172075383)(113,0.789090630619167)(114,0.78953153118042)(115,0.791460661365453)(116,0.790941859465784)(117,0.792924621599125)(118,0.791894450735339)(119,0.7911417320409)(120,0.791462588056238)(121,0.791422421972462)(122,0.790958025647203)(123,0.792884271492723)(124,0.79290753478404)(125,0.792320818963452)(126,0.792357647936961)(127,0.792276705439153)(128,0.791922288039078)(129,0.791914757120911)(130,0.79147100254784)(131,0.792053961927661)(132,0.792053961927661)(133,0.792120709849987)(134,0.792538088578453)(135,0.792270747748579)(136,0.792704095321573)(137,0.792790352408512)(138,0.792744868429222)(139,0.79294407571578)(140,0.793033351084664)(141,0.792033568411695)(142,0.793077492959995)(143,0.792956044276595)(144,0.793147777102478)(145,0.79321219766146)(146,0.793910605091245)(147,0.794650824286166)(148,0.794743068042185)(149,0.795631150982308)(150,0.795204165904794)(151,0.795343677881981)(152,0.796087929076797)(153,0.796087929076797)(154,0.796647416085026)(155,0.7964512540673)(156,0.796128728992552)(157,0.796563641912045)(158,0.796492011182469)(159,0.796572342407195)(160,0.796449307951216)(161,0.796341472940362)(162,0.796421600256054)(163,0.795516549549351)(164,0.795361418071629)(165,0.796691893743234)(166,0.798285231673976)(167,0.798365682366336)(168,0.798253710507448)(169,0.798088505676031)(170,0.798169076140295)(171,0.797633389864208)(172,0.798066092834123)(173,0.797956561645955)(174,0.798423008140885)(175,0.79831217639827)(176,0.798390224098062)(177,0.798498688181881)(178,0.798411220354249)(179,0.798242692763272)(180,0.797262450873547)(181,0.7971071551397)(182,0.798304742492213)(183,0.797465327409474)(184,0.797540136461524)(185,0.798469190515578)(186,0.798506507131068)(187,0.797540136461524)(188,0.797400367044749)(189,0.797071147890219)(190,0.798212905137287)(191,0.79779186861094)(192,0.798867691587115)(193,0.798794129922471)(194,0.798880639907564)(195,0.798551420753033)(196,0.798977473472727)(197,0.799191795586357)(198,0.799841657347304)(199,0.799365439956872)(200,0.799841876639432)
 %(201,0.800171095793963)(202,0.800139849950287)(203,0.799771895601775)(204,0.800065853067271)(205,0.799759408328664)(206,0.798438944517992)(207,0.798438944517992)(208,0.798438944517992)(209,0.798206793684613)(210,0.79728174035623)(211,0.79728174035623)(212,0.797668083419471)(213,0.798641088217355)(214,0.798641088217355)(215,0.798272006220718)(216,0.798708309481009)(217,0.798611384069556)(218,0.798908105771199)(219,0.799072064421035)(220,0.798511879736315)(221,0.799352333043826)(222,0.799942431794075)(223,0.799451935375807)(224,0.799201110109272)(225,0.799775520171277)(226,0.799289790290139)(227,0.7992004370879)(228,0.799412014469923)(229,0.799296894724831)(230,0.799212557016652)(231,0.79915923018881)(232,0.799451186192663)(233,0.799428392486258)(234,0.799052835457501)(235,0.799339349152359)(236,0.799380671466409)(237,0.799654677721753)(238,0.798982083561364)(239,0.799873688052868)(240,0.799795340139164)(241,0.799206895443151)(242,0.799669870164587)(243,0.799980522201021)(244,0.799980522201021)(245,0.800208659450849)(246,0.799562463400354)(247,0.799210655000011)(248,0.799409542912859)(249,0.799375773865005)(250,0.799375773865005)(251,0.799375773865005)(252,0.799569414456515)(253,0.799406011169843)(254,0.798643023457637)(255,0.798561432496172)(256,0.798561432496172)(257,0.798561432496172)(258,0.798134796801243)(259,0.797102611069087)(260,0.797735817076634)(261,0.797735817076634)(262,0.797983606369878)(263,0.798065590086063)(264,0.799616188996853)(265,0.79968063185583)(266,0.799732823762498)(267,0.799872268392712)(268,0.799856681395137)(269,0.799635773866637)(270,0.799488858198839)(271,0.799539329120651)(272,0.799539329120651)(273,0.799680848247277)(274,0.798256198981965)(275,0.798364478765716)(276,0.799179462173657)(277,0.798573490966701)(278,0.798697617554005)(279,0.798805268527268)(280,0.799526659859649)(281,0.798789262466448)(282,0.798789262466448)(283,0.798385803282783)(284,0.798616363546383)(285,0.79958588128637)(286,0.798402735452024)(287,0.799385335247316)(288,0.799018612803566)(289,0.799445355170236)(290,0.799963448115996)(291,0.800151194376885)(292,0.800002503571668)(293,0.799590949090049)(294,0.800364969249051)(295,0.799602009983437)(296,0.800577430199934)(297,0.800800457308474)(298,0.800546244920842)(299,0.800593230348497)(300,0.801027926600444)(301,0.80105324183662)(302,0.802390377415543)(303,0.800878325351999)(304,0.801433962173152)(305,0.801939534791411)(306,0.800592029115992)(307,0.801570193803298)(308,0.800592029115992)(309,0.800840737745386)(310,0.80088266610031)(311,0.802212074143749)(312,0.802499318209326)(313,0.80182286748948)(314,0.802006787942737)(315,0.80196828889238)(316,0.802676504524364)(317,0.801305539342211)(318,0.800927901709469)(319,0.801564305330631)(320,0.802208977960473)(321,0.802125204892402)(322,0.802625626244785)(323,0.802541253008165)(324,0.80233901927053)(325,0.802046214602074)(326,0.801507610622813)(327,0.802625706060401)(328,0.802463929489528)(329,0.803082197497406)(330,0.802739537614976)(331,0.804213982497451)(332,0.80295574815075)(333,0.805300123695637)(334,0.804911646378288)(335,0.80436748044311)(336,0.804024977498421)(337,0.805363778826475)(338,0.806246082209095)(339,0.805375599291769)(340,0.805544903523551)(341,0.806085673132462)(342,0.805719446466524)(343,0.805598215532081)(344,0.805598215532081)(345,0.80655053725259)(346,0.806330058857465)(347,0.805825491097452)(348,0.804980030404723)(349,0.805855309699984)(350,0.804567598374107)(351,0.804964590107823)(352,0.805087018680472)(353,0.805506405227771)(354,0.806167363801184)(355,0.805891280976843)(356,0.805891280976843)(357,0.805471894429544)(358,0.805145871245274)(359,0.80532741781513)(360,0.805212594269344)(361,0.804787583842024)(362,0.804998820895067)(363,0.804212227755879)(364,0.804051509429549)(365,0.805672376026798)(366,0.805672376026798)(367,0.8052529894795)(368,0.805097135520631)(369,0.805013541136258)(370,0.805169410706421)(371,0.805130197025962)(372,0.805130197025962)(373,0.805130197025962)(374,0.805012722311223)(375,0.805102008025508)(376,0.804693043650732)(377,0.805065052781866)(378,0.804168054709231)(379,0.804217002206874)(380,0.803254138491538)(381,0.803932271960593)(382,0.804100959502529)(383,0.803673177387957)(384,0.803643292501661)(385,0.803754121712081)(386,0.803754121712081)(387,0.803754121712081)(388,0.803754121712081)(389,0.804828909140181)(390,0.804822244293806)(391,0.804822244293806)(392,0.80475678051774)(393,0.8050539967691)(394,0.804225054473765)(395,0.803927838222405)(396,0.803927838222405)(397,0.803966471212204)(398,0.802168299263143)(399,0.803037902514523)(400,0.802089601520862) 

};
\addlegendentry{\str};

\addplot [
color=orange,
densely dotted,
line width=1.0pt,
]
coordinates{
 %(1,0.470216301241586)(2,0.470539009624162)
 (3,0.529885670577277)(4,0.529843588740064)(5,0.524114225843782)(6,0.5519040941856)(7,0.570614778741484)(8,0.597870621270671)(9,0.603900239804462)(10,0.605505889789711)(11,0.628012153331517)(12,0.635657902273215)(13,0.649414359157191)(14,0.645843347863916)(15,0.658689747947712)(16,0.664140357849611)(17,0.659407755089051)(18,0.66032728802541)(19,0.664065901611842)(20,0.661379778379683)(21,0.665654923156283)(22,0.665683641465307)(23,0.671935812122478)(24,0.675127384443139)(25,0.679248749498767)(26,0.682859587929769)(27,0.680034333799083)(28,0.679449506928499)(29,0.681464064140961)(30,0.683765967628062)(31,0.687507543237132)(32,0.688239264355242)(33,0.693167398743821)(34,0.693542460067808)(35,0.69437146586143)(36,0.694979151183891)(37,0.69440132837617)(38,0.688760986171016)(39,0.689787698386667)(40,0.687706569824396)(41,0.686786914423253)(42,0.687414142540112)(43,0.687625898890657)(44,0.685115567197897)(45,0.691250489029271)(46,0.696100058014429)(47,0.686539251745561)(48,0.689238260241043)(49,0.691980411096248)(50,0.695848791660666)(51,0.696044236412898)(52,0.705459178230364)(53,0.704214335294907)(54,0.708699392230451)(55,0.710071908938195)(56,0.708779080563867)(57,0.706750842471612)(58,0.708754144419424)(59,0.707655495309179)(60,0.709065761042602)(61,0.711381383036992)(62,0.711609955539584)(63,0.710853781121195)(64,0.712610143082108)(65,0.716975388572815)(66,0.716217562293735)(67,0.720426823461158)(68,0.724836311391687)(69,0.724436748215109)(70,0.718569332224268)(71,0.719057764032513)(72,0.720182828215946)(73,0.722481800118811)(74,0.724881353022124)(75,0.726812977394103)(76,0.727880686939858)(77,0.727635691692697)(78,0.727037366403538)(79,0.727379976975789)(80,0.726951536288008)(81,0.726877559383331)(82,0.7301967298349)(83,0.732714907001867)(84,0.732795517963862)(85,0.733831628722151)(86,0.736601476671502)(87,0.737954305432674)(88,0.744589620398604)(89,0.742465727200403)(90,0.742328539575824)(91,0.743091726054681)(92,0.743729501160541)(93,0.744507896923761)(94,0.745218527778483)(95,0.74677810907214)(96,0.746100857462296)(97,0.748712600387868)(98,0.74905888560449)(99,0.748522677261229)(100,0.748537038928545)(101,0.748537038928545)(102,0.748925808149851)(103,0.75318662974284)(104,0.75318662974284)(105,0.753448158881041)(106,0.753448158881041)(107,0.753448158881041)(108,0.753665196567531)(109,0.751990390020027)(110,0.750614630304116)(111,0.749732532255277)(112,0.749732532255277)(113,0.750352328972793)(114,0.750017407483551)(115,0.749510498864578)(116,0.752099716919845)(117,0.752594702522069)(118,0.75267708521145)(119,0.752338878573369)(120,0.75250152660266)(121,0.752392147378459)(122,0.752518845117274)(123,0.752544365638472)(124,0.752965235882305)(125,0.752746246463625)(126,0.753857071905308)(127,0.756276061809886)(128,0.757766464522244)(129,0.757057031807639)(130,0.758442385837553)(131,0.7583639239851)(132,0.760007490262129)(133,0.760390137114096)(134,0.760861164625323)(135,0.76132040367338)(136,0.76098341084395)(137,0.762058684455356)(138,0.7629191123568)(139,0.762919918803364)(140,0.765299518013368)(141,0.764613638580786)(142,0.764415856636005)(143,0.763832538548918)(144,0.764364522614635)(145,0.764451902218873)(146,0.763396273817034)(147,0.763464560322398)(148,0.763880205652354)(149,0.765766227936117)(150,0.767283980481775)(151,0.767416137163271)(152,0.767870942308001)(153,0.766173402688127)(154,0.76726774679801)(155,0.767695210281549)(156,0.767807719924637)(157,0.767253921248296)(158,0.767253921248296)(159,0.76800643901219)(160,0.766885882929815)(161,0.766885882929815)(162,0.767265257281685)(163,0.767265257281685)(164,0.769035549682838)(165,0.76935307002408)(166,0.769727430210032)(167,0.768863130319417)(168,0.768202368072665)(169,0.767431234107964)(170,0.767167227817826)(171,0.767430578738714)(172,0.766715082948637)(173,0.766351357915775)(174,0.766469492986171)(175,0.7689482925571)(176,0.769125182021664)(177,0.768661868877175)(178,0.76912184756066)(179,0.769445375935179)(180,0.769650161868317)(181,0.768892022283844)(182,0.768892022283844)(183,0.769713971764806)(184,0.769069769652524)(185,0.770778667770684)(186,0.770944682745329)(187,0.77186962242669)(188,0.77186962242669)(189,0.772122261820134)(190,0.772195645094862)(191,0.772231380091438)(192,0.772115010045484)(193,0.771776894967998)(194,0.772630518001133)(195,0.773017530817131)(196,0.77332077406715)(197,0.773064121400656)(198,0.772900101723903)(199,0.772557747885649)(200,0.772768234199896)
 %(201,0.773873851433923)(202,0.774067978386628)(203,0.77475347298454)(204,0.775368783133615)(205,0.775447207633604)(206,0.77592146380891)(207,0.775951185586505)(208,0.777454539845965)(209,0.778241588197198)(210,0.779255727148367)(211,0.779513763044891)(212,0.779758650741205)(213,0.780550417861294)(214,0.779925591230452)(215,0.779495473362572)(216,0.779277997546227)(217,0.779942886160672)(218,0.779805240147773)(219,0.779805240147773)(220,0.779766629103217)(221,0.779651433794808)(222,0.779811013218976)(223,0.780068356919014)(224,0.779700418003666)(225,0.780080635840455)(226,0.779244435784164)(227,0.780120947631595)(228,0.780114080282301)(229,0.780082345771865)(230,0.781975083841414)(231,0.781961880380686)(232,0.782379036235233)(233,0.782053157810205)(234,0.782120067086282)(235,0.781355099048743)(236,0.781929997913674)(237,0.78224182400581)(238,0.782205879805358)(239,0.782437895464852)(240,0.782437895464852)(241,0.782801498200489)(242,0.782573697602261)(243,0.783040563436511)(244,0.782914604634416)(245,0.782737724334084)(246,0.783046986528504)(247,0.782947054459542)(248,0.783091723575825)(249,0.782274943638159)(250,0.781712601099892)(251,0.781596792309114)(252,0.781305967914229)(253,0.781234919718752)(254,0.781129784411894)(255,0.781461174411217)(256,0.781606383824545)(257,0.781614433965423)(258,0.781492824961792)(259,0.781437692877715)(260,0.781365479349481)(261,0.781294940467306)(262,0.781320999027398)(263,0.781169157435113)(264,0.780776342427092)(265,0.782111153157115)(266,0.782149251892512)(267,0.782193026789485)(268,0.782104439610324)(269,0.782104439610324)(270,0.782217110181049)(271,0.781783387669991)(272,0.781783387669991)(273,0.782113546341914)(274,0.781849497588806)(275,0.782384068271032)(276,0.782133352673154)(277,0.781459059734556)(278,0.781347478721261)(279,0.781347478721261)(280,0.781570429589626)(281,0.781885617364965)(282,0.782149072233622)(283,0.782714234959149)(284,0.783175369267201)(285,0.782724119656247)(286,0.783293366025207)(287,0.783323502499206)(288,0.783250242558466)(289,0.78453605058045)(290,0.784462070569286)(291,0.784543603095589)(292,0.784587253574301)(293,0.784587253574301)(294,0.784715427873521)(295,0.784715427873521)(296,0.784748692531432)(297,0.784904285286174)(298,0.784938289075093)(299,0.7849038618964)(300,0.785094835153176)(301,0.785094835153176)(302,0.785473982467933)(303,0.785473982467933)(304,0.785029707831604)(305,0.785253551412423)(306,0.785011768634357)(307,0.784938320104627)(308,0.784905172894414)(309,0.784905172894414)(310,0.784905172894414)(311,0.784162155829584)(312,0.783717025660044)(313,0.783493182079225)(314,0.783493182079225)(315,0.783445303690333)(316,0.783173215030269)(317,0.783202912007597)(318,0.782090790984597)(319,0.782555859624025)(320,0.782482045884805)(321,0.782515176369129)(322,0.781995006578724)(323,0.782542265622323)(324,0.78246878373642)(325,0.782788647349647)(326,0.782747803615006)(327,0.78277823933792)(328,0.782871187499167)(329,0.783250242641352)(330,0.783637458586075)(331,0.783944721155907)(332,0.784617770714583)(333,0.785303328822156)(334,0.786599909529275)(335,0.786672617810338)(336,0.787002441893015)(337,0.787038696163815)(338,0.788449780755196)(339,0.788412303913255)(340,0.788841950100513)(341,0.789242926657905)(342,0.789566618340548)(343,0.789646967732683)(344,0.789646967732683)(345,0.789646967732683)(346,0.789498577575414)(347,0.789675218467861)(348,0.789745248045408)(349,0.789636562519011)(350,0.789636562519011)(351,0.789614950726949)(352,0.789763015818088)(353,0.789614950726949)(354,0.790251251017696)(355,0.790032385913078)(356,0.788945319530755)(357,0.788945319530755)(358,0.788945319530755)(359,0.788945319530755)(360,0.788634768616483)(361,0.78843667857387)(362,0.788400532282463)(363,0.788440951112443)(364,0.789201101756949)(365,0.789201101756949)(366,0.789163216958857)(367,0.789277272087271)(368,0.789312707405039)(369,0.789090432133826)(370,0.788719126129283)(371,0.788373084206139)(372,0.788849297202953)(373,0.788783150344722)(374,0.788078332027527)(375,0.786702161239923)(376,0.784499984371322)(377,0.784274477452486)(378,0.784126980502334)(379,0.784126980502334)(380,0.783872353096913)(381,0.783643736660626)(382,0.783674425038241)(383,0.784315236570686)(384,0.783941636633794)(385,0.783902523565876)(386,0.784318150089652)(387,0.785506029977422)(388,0.785578980476512)(389,0.785856135513443)(390,0.785439654575634)(391,0.785440361896203)(392,0.786002226024854)(393,0.786143971346927)(394,0.786255764595981)(395,0.786255764595981)(396,0.786150097359782)(397,0.785726402103585)(398,0.785639157534425)(399,0.785639157534425)(400,0.785639157534425) 
};
\addlegendentry{\var};

\addplot [
color=blue,
solid,
line width=1.3pt,
]
coordinates{
 (16,0.667733354091681)(16,0.667733354091681)(18,0.683897116669292)(18,0.683897116669292)(18,0.683897116669292)(19,0.689452322658859)(19,0.689452322658859)(19,0.689452322658859)(20,0.693702557466728)(20,0.693702557466728)(20,0.693702557466728)(20,0.693702557466728)(20,0.693702557466728)(21,0.696646748427684)(21,0.696646748427684)(21,0.696646748427684)(21,0.696646748427684)(22,0.697836300278999)(22,0.697836300278999)(22,0.697836300278999)(22,0.697836300278999)(22,0.697836300278999)(22,0.697836300278999)(22,0.697836300278999)(22,0.697836300278999)(22,0.697836300278999)(22,0.697836300278999)(23,0.696600635310183)(23,0.696600635310183)(23,0.696600635310183)(23,0.696600635310183)(23,0.696600635310183)(24,0.696375658072051)(24,0.696375658072051)(24,0.696375658072051)(24,0.696375658072051)(24,0.696375658072051)(24,0.696375658072051)(24,0.696375658072051)(24,0.696375658072051)(24,0.696375658072051)(25,0.70072688145614)(25,0.70072688145614)(25,0.70072688145614)(25,0.70072688145614)(25,0.70072688145614)(25,0.70072688145614)(25,0.70072688145614)(25,0.70072688145614)(26,0.706266702094706)(26,0.706266702094706)(26,0.706266702094706)(26,0.706266702094706)(27,0.711789463374266)(27,0.711789463374266)(27,0.711789463374266)(27,0.711789463374266)(27,0.711789463374266)(27,0.711789463374266)(27,0.711789463374266)(28,0.712089954701152)(28,0.712089954701152)(28,0.712089954701152)(28,0.712089954701152)(28,0.712089954701152)(28,0.712089954701152)(28,0.712089954701152)(28,0.712089954701152)(29,0.707067635272622)(29,0.707067635272622)(29,0.707067635272622)(30,0.702611638310362)(30,0.702611638310362)(30,0.702611638310362)(30,0.702611638310362)(30,0.702611638310362)(30,0.702611638310362)(30,0.702611638310362)(31,0.702511885042345)(31,0.702511885042345)(31,0.702511885042345)(31,0.702511885042345)(31,0.702511885042345)(31,0.702511885042345)(31,0.702511885042345)(31,0.702511885042345)(31,0.702511885042345)(31,0.702511885042345)(32,0.704909402849649)(32,0.704909402849649)(32,0.704909402849649)(32,0.704909402849649)(32,0.704909402849649)(32,0.704909402849649)(32,0.704909402849649)(33,0.710683722915031)(33,0.710683722915031)(33,0.710683722915031)(33,0.710683722915031)(33,0.710683722915031)(33,0.710683722915031)(33,0.710683722915031)(33,0.710683722915031)(34,0.719859622238175)(34,0.719859622238175)(34,0.719859622238175)(34,0.719859622238175)(34,0.719859622238175)(35,0.724480825718809)(35,0.724480825718809)(35,0.724480825718809)(35,0.724480825718809)(36,0.725202472667685)(36,0.725202472667685)(36,0.725202472667685)(36,0.725202472667685)(36,0.725202472667685)(36,0.725202472667685)(36,0.725202472667685)(36,0.725202472667685)(37,0.722316269423803)(37,0.722316269423803)(37,0.722316269423803)(37,0.722316269423803)(37,0.722316269423803)(37,0.722316269423803)(37,0.722316269423803)(37,0.722316269423803)(38,0.716203259755281)(38,0.716203259755281)(38,0.716203259755281)(38,0.716203259755281)(38,0.716203259755281)(38,0.716203259755281)(39,0.709972657685087)(39,0.709972657685087)(39,0.709972657685087)(39,0.709972657685087)(39,0.709972657685087)(39,0.709972657685087)(39,0.709972657685087)(40,0.712012642162277)(40,0.712012642162277)(40,0.712012642162277)(40,0.712012642162277)(40,0.712012642162277)(40,0.712012642162277)(41,0.713818793255306)(41,0.713818793255306)(41,0.713818793255306)(41,0.713818793255306)(41,0.713818793255306)(41,0.713818793255306)(42,0.717562502574405)(42,0.717562502574405)(42,0.717562502574405)(42,0.717562502574405)(42,0.717562502574405)(42,0.717562502574405)(43,0.71783798441135)(43,0.71783798441135)(43,0.71783798441135)(43,0.71783798441135)(43,0.71783798441135)(43,0.71783798441135)(43,0.71783798441135)(43,0.71783798441135)(43,0.71783798441135)(44,0.718432718937133)(44,0.718432718937133)(44,0.718432718937133)(44,0.718432718937133)(44,0.718432718937133)(45,0.718514367349046)(45,0.718514367349046)(45,0.718514367349046)(45,0.718514367349046)(45,0.718514367349046)(45,0.718514367349046)(45,0.718514367349046)(45,0.718514367349046)(45,0.718514367349046)(46,0.719592387403055)(47,0.720615295513403)(47,0.720615295513403)(47,0.720615295513403)(47,0.720615295513403)(47,0.720615295513403)(47,0.720615295513403)(48,0.722412766820268)(48,0.722412766820268)(48,0.722412766820268)(48,0.722412766820268)(48,0.722412766820268)(48,0.722412766820268)(48,0.722412766820268)(49,0.725527446987011)(49,0.725527446987011)(49,0.725527446987011)(49,0.725527446987011)(49,0.725527446987011)(50,0.72905569992371)(50,0.72905569992371)(51,0.731329274769408)(51,0.731329274769408)(51,0.731329274769408)(51,0.731329274769408)(52,0.732035382872626)(52,0.732035382872626)(53,0.733312034897521)(53,0.733312034897521)(53,0.733312034897521)(54,0.735686821177466)(54,0.735686821177466)(54,0.735686821177466)(54,0.735686821177466)(54,0.735686821177466)(54,0.735686821177466)(54,0.735686821177466)(54,0.735686821177466)(54,0.735686821177466)(55,0.736247249821717)(55,0.736247249821717)(55,0.736247249821717)(55,0.736247249821717)(55,0.736247249821717)(55,0.736247249821717)(56,0.737681759190814)(56,0.737681759190814)(56,0.737681759190814)(56,0.737681759190814)(56,0.737681759190814)(56,0.737681759190814)(56,0.737681759190814)(56,0.737681759190814)(57,0.738472944298363)(57,0.738472944298363)(57,0.738472944298363)(57,0.738472944298363)(57,0.738472944298363)(57,0.738472944298363)(57,0.738472944298363)(58,0.739915668913179)(58,0.739915668913179)(58,0.739915668913179)(58,0.739915668913179)(59,0.740099238453627)(59,0.740099238453627)(59,0.740099238453627)(60,0.739517002677836)(60,0.739517002677836)(60,0.739517002677836)(60,0.739517002677836)(60,0.739517002677836)(60,0.739517002677836)(60,0.739517002677836)(61,0.737967088593692)(61,0.737967088593692)(61,0.737967088593692)(61,0.737967088593692)(62,0.738858469857056)(62,0.738858469857056)(62,0.738858469857056)(63,0.742604432816141)(63,0.742604432816141)(63,0.742604432816141)(63,0.742604432816141)(63,0.742604432816141)(63,0.742604432816141)(64,0.748197418480488)(64,0.748197418480488)(64,0.748197418480488)(64,0.748197418480488)(64,0.748197418480488)(64,0.748197418480488)(65,0.752179350915116)(65,0.752179350915116)(66,0.755712198666309)(66,0.755712198666309)(66,0.755712198666309)(66,0.755712198666309)(67,0.757262880906437)(67,0.757262880906437)(67,0.757262880906437)(67,0.757262880906437)(67,0.757262880906437)(68,0.75893512931078)(68,0.75893512931078)(68,0.75893512931078)(68,0.75893512931078)(68,0.75893512931078)(68,0.75893512931078)(69,0.759896954419146)(69,0.759896954419146)(69,0.759896954419146)(69,0.759896954419146)(70,0.761004567521)(70,0.761004567521)(70,0.761004567521)(70,0.761004567521)(70,0.761004567521)(70,0.761004567521)(70,0.761004567521)(70,0.761004567521)(71,0.761314392374807)(71,0.761314392374807)(71,0.761314392374807)(71,0.761314392374807)(71,0.761314392374807)(71,0.761314392374807)(72,0.761123710513821)(72,0.761123710513821)(73,0.760343778515144)(73,0.760343778515144)(74,0.75949943339395)(74,0.75949943339395)(74,0.75949943339395)(75,0.760190556153384)(75,0.760190556153384)(75,0.760190556153384)(76,0.76151995468781)(77,0.763660617592232)(77,0.763660617592232)(77,0.763660617592232)(77,0.763660617592232)(77,0.763660617592232)(78,0.765940144057935)(79,0.767735186154637)(80,0.769047783988815)(80,0.769047783988815)(80,0.769047783988815)(80,0.769047783988815)(80,0.769047783988815)(80,0.769047783988815)(80,0.769047783988815)(80,0.769047783988815)(80,0.769047783988815)(81,0.769690425912887)(81,0.769690425912887)(82,0.770436529642019)(82,0.770436529642019)(83,0.770987274227462)(83,0.770987274227462)(83,0.770987274227462)(83,0.770987274227462)(83,0.770987274227462)(84,0.771906679644753)(85,0.772956655427339)(85,0.772956655427339)(85,0.772956655427339)(85,0.772956655427339)(86,0.773771250012881)(87,0.774033080865089)(87,0.774033080865089)(88,0.774294763867503)(89,0.774497577195663)(89,0.774497577195663)(89,0.774497577195663)(89,0.774497577195663)(90,0.774465583751633)(91,0.774664767877785)(91,0.774664767877785)(91,0.774664767877785)(91,0.774664767877785)(91,0.774664767877785)(92,0.775118173248884)(92,0.775118173248884)(92,0.775118173248884)(92,0.775118173248884)(92,0.775118173248884)(93,0.775556081387871)(93,0.775556081387871)(93,0.775556081387871)(94,0.776443274878899)(95,0.776654925694697)(95,0.776654925694697)(95,0.776654925694697)(96,0.776964868147717)(96,0.776964868147717)(97,0.776520210915009)(97,0.776520210915009)(98,0.776199048542742)(98,0.776199048542742)(98,0.776199048542742)(99,0.77566090118787)(99,0.77566090118787)(100,0.775256796526401)(100,0.775256796526401)(100,0.775256796526401)(100,0.775256796526401)(101,0.775576920393688)(101,0.775576920393688)(101,0.775576920393688)(103,0.77622064532544)(104,0.777221945509281)(104,0.777221945509281)(104,0.777221945509281)(105,0.777977755260091)(105,0.777977755260091)(105,0.777977755260091)(105,0.777977755260091)(105,0.777977755260091)(105,0.777977755260091)(105,0.777977755260091)(106,0.778964673067522)(106,0.778964673067522)(106,0.778964673067522)(107,0.77990629233716)(107,0.77990629233716)(107,0.77990629233716)(107,0.77990629233716)(108,0.78083876548677)(108,0.78083876548677)(108,0.78083876548677)(109,0.782006040669816)(110,0.783156342321489)(110,0.783156342321489)(110,0.783156342321489)(111,0.784512131873486)(111,0.784512131873486)(111,0.784512131873486)(113,0.787266711530008)(114,0.788702480975759)(114,0.788702480975759)(114,0.788702480975759)(114,0.788702480975759)(115,0.790065369093699)(116,0.790496184504916)(116,0.790496184504916)(116,0.790496184504916)(118,0.790436226585343)(118,0.790436226585343)(118,0.790436226585343)(118,0.790436226585343)(119,0.790143715672104)(119,0.790143715672104)(120,0.789760764415783)(121,0.789093970503513)(121,0.789093970503513)(121,0.789093970503513)(122,0.788610970578975)(123,0.788196248905252)(123,0.788196248905252)(124,0.787943039758245)(124,0.787943039758245)(124,0.787943039758245)(124,0.787943039758245)(124,0.787943039758245)(124,0.787943039758245)(125,0.787710814390734)(126,0.787513881464323)(126,0.787513881464323)(126,0.787513881464323)(126,0.787513881464323)(126,0.787513881464323)(127,0.787367912528375)(127,0.787367912528375)(127,0.787367912528375)(128,0.787168579917541)(129,0.786881945785787)(129,0.786881945785787)(130,0.786689273439725)(130,0.786689273439725)(131,0.786478567524394)(131,0.786478567524394)(131,0.786478567524394)(131,0.786478567524394)(131,0.786478567524394)(131,0.786478567524394)(132,0.786360901962584)(133,0.786284978624195)(134,0.786622695233259)(135,0.786947580818259)(136,0.787359727850614)(136,0.787359727850614)(136,0.787359727850614)(136,0.787359727850614)(136,0.787359727850614)(138,0.788370508870756)(138,0.788370508870756)(139,0.788816899556286)(139,0.788816899556286)(139,0.788816899556286)(139,0.788816899556286)(139,0.788816899556286)(140,0.789163985007055)(140,0.789163985007055)(140,0.789163985007055)(141,0.78952271169384)(141,0.78952271169384)(142,0.78992944207987)(143,0.790077963836642)(143,0.790077963836642)(144,0.790390693316187)(144,0.790390693316187)(145,0.790452430507669)(146,0.790230624496094)(148,0.789872163279454)(148,0.789872163279454)(149,0.78964932686879)(150,0.789467842673859)(151,0.78927853541045)(151,0.78927853541045)(152,0.789084717005457)(152,0.789084717005457)(152,0.789084717005457)(154,0.78874135003309)(154,0.78874135003309)(155,0.788638130103016)(155,0.788638130103016)(156,0.788616614960394)(157,0.788538970645238)(157,0.788538970645238)(158,0.788294682517534)(159,0.788202764801006)(161,0.788386687300345)(161,0.788386687300345)(162,0.788578695000899)(163,0.788610875890519)(163,0.788610875890519)(163,0.788610875890519)(165,0.789070543344757)(165,0.789070543344757)(165,0.789070543344757)(165,0.789070543344757)(165,0.789070543344757)(166,0.789351370870402)(166,0.789351370870402)(166,0.789351370870402)(167,0.789669363784956)(167,0.789669363784956)(168,0.790008740241868)(170,0.790683682951524)(170,0.790683682951524)(170,0.790683682951524)(171,0.791062946491539)(171,0.791062946491539)(172,0.791389477313675)(172,0.791389477313675)(174,0.792018098716952)(174,0.792018098716952)(174,0.792018098716952)(174,0.792018098716952)(174,0.792018098716952)(175,0.792224217678216)(176,0.792366825797504)(177,0.792495635185097)(177,0.792495635185097)(177,0.792495635185097)(180,0.792773661100279)(180,0.792773661100279)(181,0.792817197691661)(182,0.792813851087019)(182,0.792813851087019)(182,0.792813851087019)(183,0.79286340055075)(183,0.79286340055075)(184,0.792869011566158)(186,0.793009383574827)(187,0.793192307492358)(188,0.793332808087568)(188,0.793332808087568)(189,0.793593486338552)(190,0.793979769553074)(190,0.793979769553074)(191,0.794324386819211)(191,0.794324386819211)(191,0.794324386819211)(192,0.794680447955706)(192,0.794680447955706)(192,0.794680447955706)(193,0.79507188935576)(193,0.79507188935576)(194,0.795423167740793)(194,0.795423167740793)(195,0.795723405476347)(195,0.795723405476347)(195,0.795723405476347)(196,0.796028938210089)(196,0.796028938210089)(197,0.79633037842111)(197,0.79633037842111)(197,0.79633037842111)(197,0.79633037842111)(198,0.796636051529974)(198,0.796636051529974)(198,0.796636051529974)(199,0.796928014950214)(199,0.796928014950214)
 %(201,0.797560959751857)(202,0.797819006990831)(203,0.798122358537082)(204,0.79841628518789)(204,0.79841628518789)(204,0.79841628518789)(205,0.798700381379615)(207,0.799231281129846)(207,0.799231281129846)(209,0.799683192283813)(209,0.799683192283813)(209,0.799683192283813)(212,0.800119825436441)(212,0.800119825436441)(212,0.800119825436441)(212,0.800119825436441)(214,0.800263623252051)(214,0.800263623252051)(215,0.80029241445587)(215,0.80029241445587)(216,0.800320326578744)(216,0.800320326578744)(217,0.800281401180719)(217,0.800281401180719)(220,0.800128717782799)(221,0.800018939358603)(221,0.800018939358603)(221,0.800018939358603)(221,0.800018939358603)(222,0.799890384611263)(223,0.799754703464872)(224,0.799596294845774)(224,0.799596294845774)(226,0.799418579176836)(227,0.799415622761108)(228,0.799421360704851)(228,0.799421360704851)(230,0.799523054686333)(231,0.799581728686016)(231,0.799581728686016)(232,0.799650322759703)(232,0.799650322759703)(234,0.799732653871919)(235,0.799809428975196)(236,0.799889663425405)(237,0.800012877267597)(238,0.800064907009371)(239,0.800089966717174)(239,0.800089966717174)(239,0.800089966717174)(240,0.800150897470986)(240,0.800150897470986)(242,0.800237655502814)(242,0.800237655502814)(245,0.800408396445436)(245,0.800408396445436)(245,0.800408396445436)(246,0.800464744218651)(247,0.80047527580861)(247,0.80047527580861)(248,0.800541946903549)(248,0.800541946903549)(249,0.800618773749222)(249,0.800618773749222)(250,0.800702708155069)(251,0.800787298858049)(251,0.800787298858049)(252,0.800925941984395)(254,0.801008178073216)(255,0.80107497653017)(256,0.801143885632162)(259,0.801217713193946)(259,0.801217713193946)(261,0.801316397816392)(262,0.801385809615921)(263,0.80146506531577)(263,0.80146506531577)(264,0.80147865979944)(264,0.80147865979944)(264,0.80147865979944)(265,0.801582741444967)(265,0.801582741444967)(265,0.801582741444967)(266,0.801656349167779)(266,0.801656349167779)(266,0.801656349167779)(267,0.801834120691746)(267,0.801834120691746)(270,0.802308939059823)(270,0.802308939059823)(271,0.802492387636842)(273,0.80291225330217)(273,0.80291225330217)(274,0.803146021843844)(274,0.803146021843844)(275,0.803445409550777)(275,0.803445409550777)(276,0.803609539852562)(276,0.803609539852562)(277,0.803770345761516)(277,0.803770345761516)(280,0.804195193387316)(282,0.804289769628312)(282,0.804289769628312)(282,0.804289769628312)(283,0.80430426703494)(284,0.80429436564894)(284,0.80429436564894)(284,0.80429436564894)(286,0.804237817556614)(286,0.804237817556614)(287,0.804187676990588)(287,0.804187676990588)(288,0.804090463198347)(288,0.804090463198347)(289,0.803982668649655)(289,0.803982668649655)(289,0.803982668649655)(290,0.803845657738362)(290,0.803845657738362)(290,0.803845657738362)(292,0.803537386165853)(293,0.803374276222152)(293,0.803374276222152)(293,0.803374276222152)(294,0.803210327580703)(294,0.803210327580703)(295,0.803058336670561)(295,0.803058336670561)(299,0.802506076954176)(299,0.802506076954176)(300,0.802347414378396)(300,0.802347414378396)(302,0.80195879069658)(303,0.80175865955167)(303,0.80175865955167)(304,0.801557258754404)(305,0.801371412220286)(306,0.801239134333061)(308,0.800940354514626)(309,0.800851381281337)(309,0.800851381281337)(310,0.800791241904408)(311,0.800754833387424)(313,0.800733076463598)(314,0.800684328645487)(315,0.800658576693933)(315,0.800658576693933)(316,0.800711048618316)(317,0.800782237309499)(317,0.800782237309499)(318,0.800883540328276)(318,0.800883540328276)(318,0.800883540328276)(319,0.80097305720306)(320,0.801071216980531)(320,0.801071216980531)(321,0.801173083025523)(323,0.801374596667666)(324,0.801467610329881)(326,0.801615708335145)(329,0.801671777097037)(330,0.80167776325728)(330,0.80167776325728)(331,0.801612259598829)(332,0.801576489108497)(332,0.801576489108497)(336,0.801351992916951)(336,0.801351992916951)(337,0.801332663089539)(337,0.801332663089539)(339,0.801148400578989)(341,0.801033735930878)(342,0.800976484601784)(342,0.800976484601784)(342,0.800976484601784)(343,0.80094723103224)(343,0.80094723103224)(345,0.800847092226502)(345,0.800847092226502)(345,0.800847092226502)(346,0.800829585858338)(346,0.800829585858338)(347,0.800889639200582)(348,0.800875755022397)(348,0.800875755022397)(348,0.800875755022397)(349,0.800908647929591)(349,0.800908647929591)(350,0.80088953625081)(351,0.800876402172229)(352,0.800875324134474)(353,0.800905367257639)(354,0.800940740767013)(354,0.800940740767013)(356,0.801099651967593)(356,0.801099651967593)(357,0.801219804268806)(357,0.801219804268806)(358,0.801364210025852)(358,0.801364210025852)(358,0.801364210025852)(358,0.801364210025852)(358,0.801364210025852)(358,0.801364210025852)(363,0.802264437368391)(363,0.802264437368391)(363,0.802264437368391)(364,0.802454422414837)(365,0.802649540516046)(365,0.802649540516046)(366,0.802834604992129)(368,0.803169925013661)(368,0.803169925013661)(369,0.803323445466239)(371,0.803610708642787)(371,0.803610708642787)(371,0.803610708642787)(372,0.803746840935968)(372,0.803746840935968)(372,0.803746840935968)(373,0.803879260540608)(374,0.804008964458969)(374,0.804008964458969)(374,0.804008964458969)(374,0.804008964458969)(376,0.804262907734918)(377,0.804387969802118)(377,0.804387969802118)(380,0.80475800838941)(382,0.805000137707543)(383,0.8051194977485)(385,0.805354045206626)(386,0.805468880563667)(386,0.805468880563667)(386,0.805468880563667)(388,0.8056928733768)(389,0.805801814828107)(389,0.805801814828107)(390,0.805908607759269)(390,0.805908607759269)(391,0.806013108314948)(393,0.806214805842097)(396,0.806498270907724)(397,0.806587534252591)(397,0.806587534252591) 
};
\addlegendentry{\acl};

\end{axis}
\end{tikzpicture}%

%% This file was created by matlab2tikz v0.2.3.
% Copyright (c) 2008--2012, Nico Schlömer <nico.schloemer@gmail.com>
% All rights reserved.
% 
% 
% 
\begin{tikzpicture}

\begin{axis}[%
tick label style={font=\tiny},
label style={font=\tiny},
label shift={-4pt},
xlabel shift={-6pt},
legend style={font=\tiny},
view={0}{90},
width=\figurewidth,
height=\figureheight,
scale only axis,
xmin=0, xmax=250,
xlabel={Samples},
ymin=0.5, ymax=1,
ylabel={$F_1$-score},
axis lines*=left,
legend cell align=left,
legend style={at={(1.03,0)},anchor=south east,fill=none,draw=none,align=left,row sep=-0.2em},
clip=false]

\addplot [
color=red,
densely dotted,
line width=1.0pt,
]
coordinates{
 (10,0.52533310220537)(11,0.557672196428727)(12,0.575743758960597)(13,0.590646289243376)(14,0.605782970458242)(15,0.610980719543227)(16,0.624921328074933)(17,0.654759495695639)(18,0.668653587208841)(19,0.682044536684701)(20,0.696293746592038)(21,0.705923688065395)(22,0.720668803482832)(23,0.73027939559258)(24,0.740084104094547)(25,0.750240000950907)(26,0.762407681699535)(27,0.776711713706289)(28,0.794784592754037)(29,0.804611236600242)(30,0.812354597786025)(31,0.819690320327739)(32,0.825631497703322)(33,0.833298860587863)(34,0.837138897546174)(35,0.840222230492171)(36,0.846952477148692)(37,0.849051638772518)(38,0.854088176386332)(39,0.856872998128136)(40,0.861232576095433)(41,0.863103040986308)(42,0.865595016979865)(43,0.867579392024461)(44,0.869195668242486)(45,0.872130971324911)(46,0.875708420343264)(47,0.87799305722999)(48,0.879870916960692)(49,0.882138220233942)(50,0.883677114318825)(51,0.886026716425645)(52,0.887580403126912)(53,0.889612901875371)(54,0.89204330766938)(55,0.893416135510093)(56,0.895040260048166)(57,0.896340328635769)(58,0.898034920927625)(59,0.89939390744677)(60,0.901194284082291)(61,0.901747453262801)(62,0.902640561508487)(63,0.903358052690958)(64,0.90487396123068)(65,0.905728241873442)(66,0.90717754806677)(67,0.908006865839977)(68,0.908543697880697)(69,0.909837027876651)(70,0.910585344445978)(71,0.911366469772231)(72,0.911858062931837)(73,0.912583315647871)(74,0.914013783949776)(75,0.914927069786691)(76,0.915526873755034)(77,0.9166539293755)(78,0.917013906742577)(79,0.917515186242694)(80,0.918000244428204)(81,0.918049750151561)(82,0.918720967012093)(83,0.919412221023926)(84,0.920261152500483)(85,0.920968110798626)(86,0.921983384813368)(87,0.922716786279053)(88,0.923187824647256)(89,0.923843242526898)(90,0.92410515525381)(91,0.924461621888038)(92,0.925100283044281)(93,0.925580901547949)(94,0.926295864766597)(95,0.926469219782392)(96,0.926818254143752)(97,0.927260341333934)(98,0.927597248750238)(99,0.928594154383763)(100,0.929160522708893)(101,0.929394264342322)(102,0.92981265843383)(103,0.930058852825857)(104,0.930714611111779)(105,0.931104729881569)(106,0.931622936747441)(107,0.932187280438798)(108,0.932372879802343)(109,0.932560537175996)(110,0.932829836541668)(111,0.933290390193197)(112,0.933213244006758)(113,0.933346973000961)(114,0.933459631747173)(115,0.933558154971042)(116,0.933796005537234)(117,0.934141210218833)(118,0.934697744373268)(119,0.935238621314867)(120,0.935662639350973)(121,0.936005605168867)(122,0.936474267821892)(123,0.936886015222962)(124,0.937223389556857)(125,0.937513101996976)(126,0.937977993420033)(127,0.938170950788463)(128,0.938137474812014)(129,0.938356873435003)(130,0.938520270322147)(131,0.939047392557674)(132,0.939475248473771)(133,0.939881049714253)(134,0.939946337219991)(135,0.940035080163984)(136,0.940318520779432)(137,0.940433137797788)(138,0.940872848835832)(139,0.941359476576867)(140,0.941540187870658)(141,0.941845836672395)(142,0.942053908656337)(143,0.942197534586605)(144,0.942148951263702)(145,0.9425760461771)(146,0.942774915517072)(147,0.942925166430249)(148,0.943047783520236)(149,0.943110092747764)(150,0.943272779726815)(151,0.943568265017602)(152,0.943374110307765)(153,0.943649180270698)(154,0.943839052318403)(155,0.94407532633884)(156,0.944103890275209)(157,0.944250495262415)(158,0.944561689776027)(159,0.944874130678816)(160,0.945352110712912)(161,0.945706319018414)(162,0.945748422090303)(163,0.946049088422217)(164,0.946335802921303)(165,0.946487738890851)(166,0.94647536614588)(167,0.946840417539018)(168,0.947055591805982)(169,0.947417481744414)(170,0.947591992622851)(171,0.94769020986477)(172,0.947869524837937)(173,0.947971834772995)(174,0.948007069762396)(175,0.94823465035008)(176,0.948226688476758)(177,0.948252916989002)(178,0.948384877421998)(179,0.948661804097849)(180,0.948650001312223)(181,0.948845546847204)(182,0.948955705713231)(183,0.949052434854085)(184,0.949273772496655)(185,0.94938615172545)(186,0.949475387976992)(187,0.949651725714835)(188,0.949712325644604)(189,0.949825809086236)(190,0.950308336093611)(191,0.950776471565836)(192,0.951207632419819)(193,0.951408050206678)(194,0.951677902371648)(195,0.951789332954319)(196,0.951882475127331)(197,0.951853722351573)(198,0.951803122226002)(199,0.952011719683136)(200,0.952103028443106)(201,0.952074109436277)(202,0.952437684080902)(203,0.952531085698127)(204,0.952711750413993)(205,0.952863072911667)(206,0.95313650580238)(207,0.953488499349935)(208,0.953360085966174)(209,0.953407737244314)(210,0.95353863304214)(211,0.95357138093994)(212,0.9537584754864)(213,0.953894418952067)(214,0.95413498583262)(215,0.95420724481816)(216,0.95430919511106)(217,0.954513512770046)(218,0.954686635397325)(219,0.954929978421261)(220,0.955001998893767)(221,0.955121276405886)(222,0.955127148554544)(223,0.955215107141375)(224,0.955366526804986)(225,0.955441058824362)(226,0.955514359606562)(227,0.955679487338711)(228,0.955757417904964)(229,0.955788424629029)(230,0.955970692685351)(231,0.956031632022781)(232,0.956185798058961)(233,0.956251810801452)(234,0.956362151684898)(235,0.956457411912621)(236,0.956568693341014)(237,0.956671322428597)(238,0.95667302473472)(239,0.95679591860453)(240,0.957030060645719)(241,0.95717369620451)(242,0.957416071684338)(243,0.957512321478542)(244,0.957832382198916)(245,0.957927780017532)(246,0.958133947469559)(247,0.958112983585751)(248,0.958287716431921)(249,0.958394193412201)(250,0.958474124445125) 
};
\addlegendentry{\str};

\addplot [
color=orange,
densely dotted,
line width=1.0pt,
]
coordinates{
 (13,0.507584034777479)(14,0.517247700423277)(15,0.572238215574885)(16,0.582422327241963)(17,0.604926003337122)(18,0.636375694634852)(19,0.655758426106328)(20,0.669256404545698)(21,0.674364714530316)(22,0.678639588479459)(23,0.682535165934502)(24,0.688525310381419)(25,0.692975361392489)(26,0.697012231822675)(27,0.697311089697725)(28,0.700858439472114)(29,0.70740837514674)(30,0.709958246739738)(31,0.717352494167362)(32,0.720483243291611)(33,0.728099616818936)(34,0.734598083294621)(35,0.738131696700902)(36,0.74211160007762)(37,0.746441269096912)(38,0.753498116345641)(39,0.761692285967329)(40,0.768782767643523)(41,0.773837487553921)(42,0.782627081949437)(43,0.786275662260016)(44,0.792622226027352)(45,0.795958646181168)(46,0.798634760647099)(47,0.802576261359332)(48,0.805820027847354)(49,0.808242385990205)(50,0.809936629538819)(51,0.81208034104347)(52,0.814991074387453)(53,0.816593622102033)(54,0.817096830424648)(55,0.821116315086098)(56,0.821701923630705)(57,0.823027506237456)(58,0.826081497574548)(59,0.825911441373957)(60,0.826569481404181)(61,0.829932116871272)(62,0.831697431520055)(63,0.834735086384987)(64,0.835884847329633)(65,0.838629439225837)(66,0.840761751582692)(67,0.845477668026043)(68,0.846239911674701)(69,0.849202028926581)(70,0.850033715987167)(71,0.851063004691418)(72,0.852651888408583)(73,0.853564033341907)(74,0.854646353565926)(75,0.856162425956943)(76,0.85849007693264)(77,0.86112827904605)(78,0.862826140975189)(79,0.863797293003795)(80,0.863530075288295)(81,0.864630346333055)(82,0.86552089730048)(83,0.866307444599484)(84,0.867342067570914)(85,0.86769565718161)(86,0.869749697770077)(87,0.870036397280863)(88,0.871274444919727)(89,0.872660978088627)(90,0.872498010995331)(91,0.872712164148922)(92,0.874041432306972)(93,0.874812787372527)(94,0.875216836032846)(95,0.875889152595632)(96,0.876895412490436)(97,0.877934834942897)(98,0.878702237769499)(99,0.879905378227846)(100,0.880949770047229)(101,0.881292039578384)(102,0.882272304450795)(103,0.882588755967421)(104,0.882264251716728)(105,0.882720392251831)(106,0.884645694422261)(107,0.885583734439967)(108,0.886548146947634)(109,0.88730276050149)(110,0.888315827508603)(111,0.888622174503191)(112,0.888737688289676)(113,0.889007390593952)(114,0.889009042299928)(115,0.889111712466203)(116,0.889773720164739)(117,0.889888563672103)(118,0.89067002251236)(119,0.891749021493381)(120,0.892553463850082)(121,0.893439237945534)(122,0.893200317118098)(123,0.893333166960484)(124,0.89368008030734)(125,0.894026792443352)(126,0.895034179798507)(127,0.896200300237591)(128,0.89645797889405)(129,0.896910327350752)(130,0.89777923371852)(131,0.897939767369577)(132,0.898460848770436)(133,0.898899555326756)(134,0.899016934607784)(135,0.89921646375836)(136,0.899579777697076)(137,0.900359820675551)(138,0.900512310165175)(139,0.900711961721959)(140,0.90186466724932)(141,0.901945904952295)(142,0.902272771401321)(143,0.902978658480172)(144,0.90304709839765)(145,0.90306748897384)(146,0.903327764434363)(147,0.903503528337146)(148,0.90358149380707)(149,0.903822210516242)(150,0.904170980119813)(151,0.904454118360591)(152,0.904563670298197)(153,0.904756928474757)(154,0.905326351370848)(155,0.90540632737451)(156,0.905126741146053)(157,0.904744890913584)(158,0.904744374174924)(159,0.904753907589627)(160,0.904943214185696)(161,0.905109096508829)(162,0.905606973888005)(163,0.905812046089449)(164,0.90601840112168)(165,0.906100245767127)(166,0.906520849515284)(167,0.907094426391329)(168,0.907313005234039)(169,0.907220441176807)(170,0.907557508437351)(171,0.90780580636184)(172,0.907925815092495)(173,0.908079097892732)(174,0.907800298598454)(175,0.908015880159913)(176,0.908212615727068)(177,0.908366969462898)(178,0.908292274270809)(179,0.90835306786236)(180,0.909091322321647)(181,0.909118647840401)(182,0.90923547160423)(183,0.909350941020779)(184,0.909561087825921)(185,0.909557504388699)(186,0.909347004918345)(187,0.910095236837435)(188,0.910093654681919)(189,0.910076929876632)(190,0.910384254964817)(191,0.91055401034678)(192,0.910690171448364)(193,0.910934554349643)(194,0.911081456737378)(195,0.911162530041387)(196,0.911235440160654)(197,0.911625910980671)(198,0.911761633909168)(199,0.911831925088714)(200,0.911848660523484)(201,0.912388206166487)(202,0.912919736950128)(203,0.91295317789517)(204,0.913103563491324)(205,0.913623597144312)(206,0.913668037069722)(207,0.913921919064996)(208,0.913920914225615)(209,0.914091452005187)(210,0.914347155700563)(211,0.914332836369352)(212,0.914795677615513)(213,0.915071978025465)(214,0.915158992669344)(215,0.915193136010929)(216,0.915428337022903)(217,0.915902467665848)(218,0.915964887355616)(219,0.915864918073847)(220,0.916004725233405)(221,0.916045577563489)(222,0.91623313302245)(223,0.91661933470131)(224,0.916757007299151)(225,0.91682303515078)(226,0.91704710680855)(227,0.917136473592758)(228,0.917304358648358)(229,0.917352846527502)(230,0.917322361995041)(231,0.917352299780932)(232,0.917559900175025)(233,0.917405227647042)(234,0.917227202892635)(235,0.917292634429919)(236,0.917266428655332)(237,0.917356511858611)(238,0.917359401480053)(239,0.917633666474858)(240,0.917936263810671)(241,0.91803913549424)(242,0.918199634518663)(243,0.91815039038759)(244,0.918300332622768)(245,0.91869804633947)(246,0.918838622845131)(247,0.91901105515893)(248,0.919061583798719)(249,0.919430374221418)(250,0.919705153681425) 
};
\addlegendentry{\var};

\addplot [
color=blue,
solid,
line width=1.3pt,
]
coordinates{
 (11,0.557805350508368)(12,0.584920290621236)(13,0.60743962410603)(14,0.629736776257156)(15,0.647853538545617)(16,0.654587634378721)(17,0.664901024864382)(18,0.675632467077312)(19,0.691612852270182)(20,0.711102929742341)(21,0.722761755061544)(22,0.737122844225225)(23,0.749131234530275)(24,0.755665889606608)(25,0.764025931414285)(26,0.771639327573556)(27,0.778483009534706)(28,0.786099602657274)(29,0.793204045369064)(30,0.798645689752734)(31,0.805479008712953)(32,0.810842657790998)(33,0.814859881964581)(34,0.819956815695211)(35,0.82497065966054)(36,0.827990186800739)(37,0.832845076520112)(38,0.836171892704204)(39,0.839044654134234)(40,0.842707577447977)(41,0.845033789614353)(42,0.84726538167709)(43,0.850138633875384)(44,0.851972249222479)(45,0.85525479599921)(46,0.857767097894571)(47,0.860565309133078)(48,0.862033529761079)(49,0.864861964190868)(50,0.866526220819859)(51,0.868516020619239)(52,0.869942315260919)(53,0.8721428789246)(54,0.874227491451549)(55,0.876214824492852)(56,0.877744387481661)(57,0.88049551060649)(58,0.88193978277466)(59,0.883572011999064)(60,0.885397172142527)(61,0.886997694373075)(62,0.888898470064641)(63,0.890322501546843)(64,0.891294365918149)(65,0.893180400254651)(66,0.895373443361924)(67,0.89620074137114)(68,0.896506493445616)(69,0.89676199152541)(70,0.896911277268515)(71,0.898237252394432)(72,0.900027326029483)(73,0.900280884257965)(74,0.901305891661412)(75,0.903537643203334)(76,0.905119878148378)(77,0.906521246143092)(78,0.906605546095009)(79,0.906875817611285)(80,0.908276385083401)(81,0.90987661240627)(82,0.909767593014742)(83,0.910418087320106)(84,0.910963234082237)(85,0.912067147785186)(86,0.913201334433638)(87,0.913571256735177)(88,0.913717235405004)(89,0.914969199330645)(90,0.91574460313923)(91,0.916428398525439)(92,0.916265792170056)(93,0.916714809578789)(94,0.917151430840143)(95,0.918203968009942)(96,0.919100552009751)(97,0.919722462602353)(98,0.920117741237143)(99,0.921451842288492)(100,0.92246275512306)(101,0.923073144284829)(102,0.923144679282998)(103,0.92326658690924)(104,0.922830375181854)(105,0.923746179373901)(106,0.923977889534646)(107,0.923924275121816)(108,0.924500094399489)(109,0.925239942654375)(110,0.926226853834868)(111,0.927160092745498)(112,0.927140439980406)(113,0.927609885920603)(114,0.928376921725666)(115,0.928302485831713)(116,0.928207869112932)(117,0.927875526999115)(118,0.928817989647843)(119,0.929309123121027)(120,0.929828172397114)(121,0.930133592937048)(122,0.930976472232016)(123,0.932171464753438)(124,0.932856588858147)(125,0.933058923995853)(126,0.933837384173235)(127,0.934074064006003)(128,0.93482162971248)(129,0.934737929850902)(130,0.934020663576096)(131,0.934278692607279)(132,0.934156687761956)(133,0.93419122966007)(134,0.93443845717058)(135,0.934174914861057)(136,0.934551526013876)(137,0.935317738115796)(138,0.934901790165197)(139,0.935526528269265)(140,0.936496510677214)(141,0.936911539116373)(142,0.937044538395074)(143,0.937302179803076)(144,0.937301019817956)(145,0.938121073396593)(146,0.939021189720733)(147,0.939159348138848)(148,0.939040703602209)(149,0.939400636482935)(150,0.94012849792747)(151,0.940851123272936)(152,0.940558433126631)(153,0.939265249309497)(154,0.939463146966182)(155,0.940331401675218)(156,0.94088728783807)(157,0.940244299614817)(158,0.940887573966253)(159,0.941512704975099)(160,0.942169944172755)(161,0.941998674388153)(162,0.941659789331516)(163,0.941639942386097)(164,0.942431895789111)(165,0.941606990366595)(166,0.94204849323498)(167,0.942287517782532)(168,0.9426555339986)(169,0.943500175340806)(170,0.943279813261148)(171,0.943102513117787)(172,0.943887390117868)(173,0.944280980534773)(174,0.944383860821696)(175,0.945010567080586)(176,0.944849178817932)(177,0.94541586773477)(178,0.945640441872523)(179,0.945333075626632)(180,0.945567289217471)(181,0.946119996327517)(182,0.945452588817017)(183,0.946000042898311)(184,0.945798974905748)(185,0.946049689659921)(186,0.946527805974106)(187,0.946961544842994)(188,0.946997781780621)(189,0.947340583688351)(190,0.946869754865649)(191,0.947304108945921)(192,0.947120298592354)(193,0.947103237267777)(194,0.946840150691466)(195,0.946833623493308)(196,0.947553364789403)(197,0.947565234030534)(198,0.947110001053178)(199,0.947539519616885)(200,0.948563518260031)(201,0.948654378345864)(202,0.948664572204736)(203,0.948585297830896)(204,0.94965642220641)(205,0.950015040214423)(206,0.950602325648697)(207,0.94973018829716)(208,0.949503746571536)(209,0.949863983498012)(210,0.949594842144012)(211,0.948956796613854)(212,0.949180481515474)(213,0.949189540674723)(214,0.949560921202021)(215,0.949956536231531)(216,0.950492121106577)(217,0.9507443070823)(218,0.950928126658356)(219,0.950656084240814)(220,0.95064310557963)(221,0.950808950530687)(222,0.95122714114615)(223,0.950872959580171)(224,0.950730872721056)(225,0.950934063366391)(226,0.951564259740184)(227,0.951486190570636)(228,0.951841584642564)(229,0.950608457891829)(230,0.950870585665999)(231,0.952162416605227)(232,0.952004713818938)(233,0.95324318836938)(234,0.953620808031985)(235,0.954488442525752)(236,0.955678409838087)(237,0.952798014608032)(238,0.951963795577935)(239,0.954477409936002)(240,0.955538114191448)(241,0.955684515949372)(242,0.955845932819166)(243,0.95496116225368)(244,0.959048524538359)(245,0.958568211123904)(246,0.956206149506114)(247,0.9554938718859)(248,0.954528922694781)(249,0.954411492426072)(250,0.953461628566717) 
};
\addlegendentry{\acl};

\end{axis}
\end{tikzpicture}%

%% This file was created by matlab2tikz v0.2.3.
% Copyright (c) 2008--2012, Nico Schlömer <nico.schloemer@gmail.com>
% All rights reserved.
% 
% 
% 
\begin{tikzpicture}

\begin{axis}[%
tick label style={font=\tiny},
label style={font=\tiny},
label shift={-4pt},
xlabel shift={-6pt},
legend style={font=\tiny},
view={0}{90},
width=\figurewidth,
height=\figureheight,
scale only axis,
xmin=0, xmax=400,
xlabel={Samples},
ymin=0.48, ymax=1,
ylabel={$F_1$-score},
axis lines*=left,
legend cell align=left,
legend style={at={(1.03,0)},anchor=south east,fill=none,draw=none,align=left,row sep=-0.2em},
clip=false]

\addplot [
color=red,
densely dotted,
line width=1.0pt,
]
coordinates{
 %(1,0.00139759036144578)(2,0.00465389921183217)(3,0.0364138133023919)(4,0.079258283481824)(5,0.113187007367813)(6,0.217955112043304)(7,0.293236735011939)(8,0.328487107231426)(9,0.379219328967168)(10,0.407941591798234)(11,0.427586771234057)(12,0.444779940991805)(13,0.467359459052809)(14,0.476258556711259)(15,0.496216293978091)
 (16,0.514877831417121)(17,0.52854357991793)(18,0.535258000241356)(19,0.549213231394572)(20,0.553651369822796)(21,0.553933435279276)(22,0.55572822692818)(23,0.563627103435266)(24,0.565275953425249)(25,0.568774168597364)(26,0.573990313215478)(27,0.575197521016709)(28,0.575784015419952)(29,0.577353399475848)(30,0.579268673378126)(31,0.580494081533373)(32,0.582512785882977)(33,0.583061296886681)(34,0.58370802912286)(35,0.584850747214315)(36,0.586947567712069)(37,0.58920953150354)(38,0.590068001311278)(39,0.59105342652139)(40,0.591712258607971)(41,0.592933612386182)(42,0.60454060172143)(43,0.607502915139737)(44,0.608090750785295)(45,0.609264071985588)(46,0.611196190632394)(47,0.611836175622033)(48,0.612487163041322)(49,0.613824067589886)(50,0.614640775233563)(51,0.615196028204479)(52,0.616309434271518)(53,0.616989981140434)(54,0.617022436909172)(55,0.618275410142257)(56,0.619098684190892)(57,0.619706807783053)(58,0.621407477401435)(59,0.622901801935725)(60,0.623687128253362)(61,0.624442583682318)(62,0.625986011066538)(63,0.627442634352511)(64,0.628447692197199)(65,0.628699954105504)(66,0.629161439957669)(67,0.628834965049008)(68,0.629176061058916)(69,0.629509201416352)(70,0.629962978077947)(71,0.631101008973919)(72,0.631791243137759)(73,0.632756179774225)(74,0.632793949173012)(75,0.633152486621175)(76,0.63334459593346)(77,0.633131870309633)(78,0.633102249384337)(79,0.633715626904399)(80,0.633740588216734)(81,0.634261454346518)(82,0.634809215567952)(83,0.635021925311526)(84,0.6354039575982)(85,0.638160692861799)(86,0.638504009190309)(87,0.639000960527351)(88,0.643229726894854)(89,0.64524088434333)(90,0.645356726742684)(91,0.644989673405909)(92,0.652862130908198)(93,0.654118494413225)(94,0.663081793067044)(95,0.662886274818271)(96,0.663330427481672)(97,0.664248237800867)(98,0.664500466127785)(99,0.666822097990091)(100,0.666221821693735)(101,0.666332687131882)(102,0.667270985499063)(103,0.667161926591252)(104,0.667419473635039)(105,0.667384203732804)(106,0.668275633114068)(107,0.668838279046097)(108,0.668932931098312)(109,0.669050511678908)(110,0.669240306034598)(111,0.66923709976007)(112,0.669557288647218)(113,0.669985151026762)(114,0.67044577112058)(115,0.670537350503865)(116,0.670724637148796)(117,0.670730112416019)(118,0.670884165187504)(119,0.671081576302063)(120,0.67195192258241)(121,0.672095653182923)(122,0.672382579091359)(123,0.672864608796835)(124,0.673360438439585)(125,0.67354769438679)(126,0.673528990635654)(127,0.674196355645044)(128,0.674815174552903)(129,0.674721706786954)(130,0.675723471280189)(131,0.675202883266239)(132,0.675424106208164)(133,0.676498190865111)(134,0.676340887066405)(135,0.676350631101913)(136,0.676727967476128)(137,0.676891595964666)(138,0.676848040846931)(139,0.676899167459829)(140,0.676982699794639)(141,0.676865618346513)(142,0.678096687062245)(143,0.678241469359211)(144,0.679151035178983)(145,0.679215416315326)(146,0.679624517756823)(147,0.690161207772219)(148,0.693139849001488)(149,0.701736505657032)(150,0.703088154052863)(151,0.702828639896247)(152,0.703286615502668)(153,0.704906787719845)(154,0.705095241633679)(155,0.706912999143827)(156,0.707125721230829)(157,0.70766419833855)(158,0.70861419318796)(159,0.709272104631397)(160,0.709631186938525)(161,0.709992762962769)(162,0.710192705457696)(163,0.709766923199156)(164,0.70976936089695)(165,0.710016198487136)(166,0.710241989438912)(167,0.710456216880442)(168,0.710277674438049)(169,0.710314688433967)(170,0.710687637059355)(171,0.710819415401459)(172,0.710709842205335)(173,0.710478356112178)(174,0.710822956519857)(175,0.710674620895784)(176,0.710850348650157)(177,0.711461203501253)(178,0.711319562393315)(179,0.711913244283724)(180,0.711906460574403)(181,0.711960666821191)(182,0.712038532874426)(183,0.712068004520364)(184,0.712108584698679)(185,0.712177237106964)(186,0.712228203398754)(187,0.712492202125944)(188,0.712550018306977)(189,0.712531387229472)(190,0.712504299572054)(191,0.712602454425849)(192,0.712930397728489)(193,0.713024736675323)(194,0.71305733736303)(195,0.71336098801621)(196,0.713982303092315)(197,0.717189504587765)(198,0.718099715178158)(199,0.718780324119999)(200,0.719268366180957)(201,0.719263096201933)(202,0.720168068659664)(203,0.721036751617841)(204,0.721277722691142)(205,0.721504031356868)(206,0.722031127783398)(207,0.722762102098156)(208,0.723014009459114)(209,0.723880124033509)(210,0.723570304310589)(211,0.724110297163403)(212,0.724117779515804)(213,0.724095247243937)(214,0.72460840264125)(215,0.724375783477677)(216,0.724468691583774)(217,0.724733078551096)(218,0.725066972424546)(219,0.725337947028454)(220,0.725377899007707)(221,0.725951225261727)(222,0.726161803368533)(223,0.726097086425473)(224,0.726243217857816)(225,0.726486755968957)(226,0.726972522218222)(227,0.726772598439893)(228,0.727047805995507)(229,0.727075662682324)(230,0.727573875471391)(231,0.727706496278611)(232,0.728146583997998)(233,0.728610358182861)(234,0.728647852620854)(235,0.728745433928207)(236,0.728771777570294)(237,0.728907508427867)(238,0.728694962831681)(239,0.728750610724236)(240,0.728589699115689)(241,0.728695125651372)(242,0.728716686879977)(243,0.728702097993093)(244,0.72872327061183)(245,0.729140071902817)(246,0.729147401609121)(247,0.72929901441531)(248,0.729131426857658)(249,0.729633877701887)(250,0.729541504118162)(251,0.729708678020464)(252,0.730103524480756)(253,0.730166939793213)(254,0.730268364889533)(255,0.730302544522968)(256,0.730720040380181)(257,0.730809428880569)(258,0.730697067154925)(259,0.730780244610008)(260,0.731139436098116)(261,0.73129982134558)(262,0.731457976752607)(263,0.731600076941229)(264,0.731553691464401)(265,0.731693940960734)(266,0.732091537468934)(267,0.731941176897988)(268,0.732222157840117)(269,0.732551286623175)(270,0.732798223345482)(271,0.732875551655641)(272,0.733086386241835)(273,0.733233968929551)(274,0.733270429328403)(275,0.733490957351414)(276,0.733412283439812)(277,0.733621989730248)(278,0.733617400090941)(279,0.733709730022942)(280,0.734407003096129)(281,0.73439302237359)(282,0.73458840987138)(283,0.735372959936309)(284,0.735390647272275)(285,0.735503964766132)(286,0.735872262401135)(287,0.73615538419535)(288,0.736308650652397)(289,0.736499502961472)(290,0.736785522207581)(291,0.736844435165791)(292,0.736570328773845)(293,0.736801866223249)(294,0.737038604243604)(295,0.737130007547535)(296,0.737386158574637)(297,0.737345906901436)(298,0.737225066370591)(299,0.737288393131448)(300,0.737283540430072)(301,0.737382236629989)(302,0.737483593145379)(303,0.73762468510962)(304,0.737765681538077)(305,0.737676984552825)(306,0.73763048369546)(307,0.737609452308673)(308,0.737545494941235)(309,0.737498659197234)(310,0.737804402505834)(311,0.737925924095759)(312,0.73808497669102)(313,0.738232970496811)(314,0.7384423251127)(315,0.738419862390899)(316,0.738524183029256)(317,0.738352716421059)(318,0.738487974238649)(319,0.738611851007436)(320,0.738747137465759)(321,0.738639144665042)(322,0.738584318203569)(323,0.738985793340583)(324,0.739013202383749)(325,0.7393900747726)(326,0.73918549703446)(327,0.739477473127542)(328,0.739522045717354)(329,0.739524847835501)(330,0.739766063727333)(331,0.73997346351208)(332,0.740222479673738)(333,0.740309090933788)(334,0.740326730911898)(335,0.740340469144833)(336,0.740374668940585)(337,0.740681308658634)(338,0.740726022169745)(339,0.740617124594404)(340,0.740626912350138)(341,0.740968974662367)(342,0.740746938426901)(343,0.740891855681557)(344,0.740961732344678)(345,0.741011782108424)(346,0.741151787195524)(347,0.741222376636415)(348,0.741530028519909)(349,0.741636953414424)(350,0.74200301651631)(351,0.742063121621251)(352,0.742199599671191)(353,0.742206125142217)(354,0.742426992319195)(355,0.742531520639912)(356,0.742353143089436)(357,0.742470666363032)(358,0.742425210855321)(359,0.742678913481229)(360,0.742802267262362)(361,0.74276665979098)(362,0.742883906772324)(363,0.742996545008769)(364,0.743002161295302)(365,0.742904973684877)(366,0.742838148916951)(367,0.742806959270424)(368,0.742890635491975)(369,0.742972363359733)(370,0.74306784685871)(371,0.743168596344616)(372,0.743058573180948)(373,0.743091519438503)(374,0.743040901181387)(375,0.74307036941489)(376,0.74306816650591)(377,0.743062302771393)(378,0.742770060155063)(379,0.742805029307801)(380,0.742807639841606)(381,0.742748014727813)(382,0.743002652365808)(383,0.743057752705661)(384,0.743088923370519)(385,0.743045998151112)(386,0.743085087523649)(387,0.743053498445627)(388,0.743136975045398)(389,0.743353499223567)(390,0.743372348826104)(391,0.743425938733279)(392,0.74343401991877)(393,0.743360419860995)(394,0.743497256659677)(395,0.743589847759661)(396,0.743724006749351)(397,0.743606534803721)(398,0.743635049049537)(399,0.743837321912305)(400,0.743874116942055) 
};
\addlegendentry{\str};

\addplot [
color=orange,
densely dotted,
line width=1.0pt,
]
coordinates{
 %(1,0.0168355194759798)(2,0.0168355194759798)(3,0.0168355194759798)(4,0.0168374610889054)(5,0.0167699930780963)(6,0.048208589049083)(7,0.0477307785738257)(8,0.0476460981374464)(9,0.0474410305321807)(10,0.0485550543005947)(11,0.0507299655269844)(12,0.0532599869110817)(13,0.0669352482510405)(14,0.0856605578885116)(15,0.123749024176474)(16,0.132778182367498)(17,0.173523899962934)(18,0.228390698777483)(19,0.267670379610717)(20,0.328391421207214)(21,0.35553555784785)(22,0.388809416890549)(23,0.390987890598007)(24,0.398668883538064)(25,0.398607087878894)(26,0.413139948110276)(27,0.418235643086597)(28,0.421605492316469)(29,0.423641706313654)(30,0.430336010974062)(31,0.463627316393779)(32,0.464837285908879)
 (33,0.501703648534391)(34,0.515453153799586)(35,0.539465051247664)(36,0.555947493149717)(37,0.57339876711497)(38,0.627532176488351)(39,0.651856985149348)(40,0.662254513534309)(41,0.66133725322749)(42,0.675204859726853)(43,0.683783026853696)(44,0.685788959960219)(45,0.70282708160731)(46,0.706140469390923)(47,0.707702871846997)(48,0.714364336393848)(49,0.7151327654894)(50,0.718845886658305)(51,0.720697686870771)(52,0.723220881976813)(53,0.722307575815388)(54,0.731157566241356)(55,0.73185815760204)(56,0.737072589368828)(57,0.741196582140359)(58,0.745478161918016)(59,0.74668888008618)(60,0.748434036728539)(61,0.74866006828087)(62,0.749375203100507)(63,0.748366076922985)(64,0.756285447780586)(65,0.756545733769362)(66,0.758999038373228)(67,0.760875728901419)(68,0.760511098667807)(69,0.763315523563978)(70,0.762907675019111)(71,0.766780101808984)(72,0.767574594140362)(73,0.768873861635267)(74,0.77132158968897)(75,0.773969779107824)(76,0.776988811797957)(77,0.777269579257164)(78,0.77873812748378)(79,0.779590427010662)(80,0.779770542255994)(81,0.779814316292224)(82,0.779773865949109)(83,0.780821478558642)(84,0.781316430697107)(85,0.780119438999362)(86,0.782783511154606)(87,0.782935221683918)(88,0.784722878402314)(89,0.784528601892085)(90,0.785755530764448)(91,0.786579202746182)(92,0.787541394366187)(93,0.787037131978506)(94,0.787796126417614)(95,0.788657082673002)(96,0.788866586595803)(97,0.79245415547999)(98,0.79299645313061)(99,0.793239263423599)(100,0.793410417040492)(101,0.793398485158353)(102,0.79504230683255)(103,0.79538670163636)(104,0.795433929052428)(105,0.796636136401633)(106,0.797395298683088)(107,0.797478768764061)(108,0.79873184209954)(109,0.798188612412934)(110,0.798158006197179)(111,0.798737463747929)(112,0.799981252856107)(113,0.79988901329743)(114,0.800639170469608)(115,0.801271163621793)(116,0.802610363789278)(117,0.803153598680265)(118,0.805186114721705)(119,0.805857404220269)(120,0.805855969142446)(121,0.806059302750794)(122,0.807082294191735)(123,0.807535786430194)(124,0.807935321562945)(125,0.808146105261737)(126,0.809740449986898)(127,0.81113816625602)(128,0.812492651952222)(129,0.812520183313194)(130,0.813667241844308)(131,0.814520012562363)(132,0.81547005237087)(133,0.816599866145591)(134,0.817372719028093)(135,0.818284061575476)(136,0.818782014650651)(137,0.820492189331931)(138,0.821052916225409)(139,0.821660943450204)(140,0.822346179095648)(141,0.823463028124937)(142,0.823975253142869)(143,0.823888218841833)(144,0.823604992068404)(145,0.824073479069842)(146,0.823936301680092)(147,0.824320049556736)(148,0.824838356442337)(149,0.82507894986564)(150,0.825613977232026)(151,0.826323183825629)(152,0.826962847313908)(153,0.828598820170037)(154,0.82985830017499)(155,0.830091612363561)(156,0.830469168697509)(157,0.830480264249874)(158,0.830985835212206)(159,0.831601569407353)(160,0.831969784089505)(161,0.833068392622128)(162,0.832680461999549)(163,0.832665677203082)(164,0.832503845548131)(165,0.832504585295565)(166,0.832349902077364)(167,0.832451006253437)(168,0.833279823733234)(169,0.833362393545589)(170,0.833740108485466)(171,0.833553881327365)(172,0.834186084568046)(173,0.834493014200965)(174,0.834425251195595)(175,0.834158024728102)(176,0.834716512131478)(177,0.836077174064374)(178,0.836360618691598)(179,0.837543084209139)(180,0.838204625710763)(181,0.838210529312491)(182,0.838833858819765)(183,0.839193669209591)(184,0.839575527917163)(185,0.839979112792977)(186,0.839694962095681)(187,0.839634934091879)(188,0.839907908839505)(189,0.839925215110266)(190,0.839570148194517)(191,0.839775667052517)(192,0.839771032628145)(193,0.840681861570861)(194,0.840707042102804)(195,0.840789249207689)(196,0.842234796288293)(197,0.841964361273978)(198,0.842725131134611)(199,0.843522617295772)(200,0.843943100392778)(201,0.844022305405221)(202,0.844581235753074)(203,0.844343308636547)(204,0.845484770318911)(205,0.845885106866485)(206,0.845833825786078)(207,0.846020622495698)(208,0.846504034140925)(209,0.846422434529814)(210,0.846357244773762)(211,0.847324221779135)(212,0.848073796411742)(213,0.847929383752023)(214,0.848110535671909)(215,0.848096479489859)(216,0.849233244979107)(217,0.849938734058634)(218,0.850329026327976)(219,0.850896478217075)(220,0.851873469501244)(221,0.852716046868096)(222,0.852740261467841)(223,0.852625704924477)(224,0.852708835540035)(225,0.852931651320368)(226,0.853418221977078)(227,0.853504187290808)(228,0.853618563902009)(229,0.853832051812884)(230,0.853746659263614)(231,0.853745845298752)(232,0.853849741689243)(233,0.853895308109015)(234,0.854093567819473)(235,0.854157679336484)(236,0.854933714308131)(237,0.854520929241889)(238,0.854680414970597)(239,0.856186910172796)(240,0.856461855092529)(241,0.856484273237266)(242,0.856581839465739)(243,0.856423170046712)(244,0.856148413712893)(245,0.85649616902178)(246,0.856743762563026)(247,0.857780261905147)(248,0.857918172575755)(249,0.85760562973369)(250,0.857593875818756)(251,0.858281078706888)(252,0.858809733905514)(253,0.858757182008126)(254,0.859223736175564)(255,0.859720795726467)(256,0.860129612462307)(257,0.860066164277161)(258,0.8602412857742)(259,0.860650812063371)(260,0.860919496049057)(261,0.861295476059137)(262,0.861291150914811)(263,0.861211916856695)(264,0.861233584896272)(265,0.861401795085125)(266,0.861692472791896)(267,0.862428046006644)(268,0.862398187435361)(269,0.862452123071314)(270,0.863261396924492)(271,0.863331417005527)(272,0.863587034960837)(273,0.863475169304848)(274,0.863686904227101)(275,0.864615257121548)(276,0.8649558583527)(277,0.865063161884179)(278,0.865298523771288)(279,0.865380557862139)(280,0.865818113016894)(281,0.865871485443187)(282,0.866198881028621)(283,0.865974049433974)(284,0.866383400350716)(285,0.866963561479466)(286,0.867030460957878)(287,0.867462326764827)(288,0.867565296934443)(289,0.867779957070339)(290,0.868098175012602)(291,0.867873072942303)(292,0.867871784402578)(293,0.867658087915904)(294,0.867657801654585)(295,0.868627435317739)(296,0.868807588071323)(297,0.869113551102529)(298,0.869051069554459)(299,0.869151219951336)(300,0.869517058151138)(301,0.869809701337033)(302,0.870036146648058)(303,0.870223460310499)(304,0.870495709497698)(305,0.870916483540594)(306,0.871311438940446)(307,0.871096734376483)(308,0.871082497470043)(309,0.871394714041076)(310,0.871662718036576)(311,0.872020896966391)(312,0.872094417370846)(313,0.872265891374552)(314,0.872380685120301)(315,0.873012924052392)(316,0.873159201400678)(317,0.873155344608866)(318,0.873146245838206)(319,0.873504941352877)(320,0.873570951318809)(321,0.873699738445901)(322,0.874054054609346)(323,0.874332868915781)(324,0.874523001002456)(325,0.874880592675815)(326,0.875031016390726)(327,0.875156231565368)(328,0.875390336627467)(329,0.875057547909804)(330,0.874953682735061)(331,0.874843933292625)(332,0.875346992212524)(333,0.875262431664718)(334,0.875245973009823)(335,0.875111736792106)(336,0.875256555961774)(337,0.875507707723524)(338,0.875667862243695)(339,0.875881150457632)(340,0.875915799800559)(341,0.875880545598203)(342,0.87551929626444)(343,0.875903339532271)(344,0.876755387162011)(345,0.876983903548128)(346,0.877287282257713)(347,0.877278833482363)(348,0.877578993172259)(349,0.87757437882039)(350,0.878041331443303)(351,0.878008560125416)(352,0.878947126559563)(353,0.878761743389138)(354,0.878727279059182)(355,0.878791126245787)(356,0.878789270274246)(357,0.878879369457419)(358,0.878929332879307)(359,0.878869288487378)(360,0.87889432516828)(361,0.879159686486771)(362,0.879363120582601)(363,0.879844176991724)(364,0.879873960354928)(365,0.880035159406256)(366,0.880175209002008)(367,0.880306951854499)(368,0.880402699118724)(369,0.880431631472416)(370,0.880226654793281)(371,0.880316828539134)(372,0.880423225535735)(373,0.880254638796996)(374,0.880252375022937)(375,0.880482707617788)(376,0.880957396108648)(377,0.880980920378297)(378,0.881475374942296)(379,0.881509432889557)(380,0.881538203795952)(381,0.88178829347817)(382,0.882028813151936)(383,0.882509888101808)(384,0.882617992606114)(385,0.882582759302722)(386,0.882864494641784)(387,0.882857438270651)(388,0.882929266247059)(389,0.883337517327456)(390,0.88332394745124)(391,0.883351528922512)(392,0.883450625693178)(393,0.883786359642393)(394,0.884406657062666)(395,0.884541154944127)(396,0.884563591161788)(397,0.884919352416905)(398,0.88491506952062)(399,0.884526098247838)(400,0.884542989925364) 
};
\addlegendentry{\var};

\addplot [
color=blue,
solid,
line width=1.3pt,
]
coordinates{
 %(4,-0.0343851593873003)(5,0.0465617551911736)(6,0.124179461435001)(8,0.269318713729605)(8,0.269318713729605)(8,0.269318713729605)(8,0.269318713729605)(8,0.269318713729605)(8,0.269318713729605)(8,0.269318713729605)(8,0.269318713729605)(9,0.336676358760064)(9,0.336676358760064)(10,0.40006301618624)(10,0.40006301618624)(10,0.40006301618624)(10,0.40006301618624)(10,0.40006301618624)(11,0.458985475390362)(11,0.458985475390362)(11,0.458985475390362)(11,0.458985475390362)(11,0.458985475390362)(11,0.458985475390362)(11,0.458985475390362)(11,0.458985475390362)
 (12,0.513242753000793)(12,0.513242753000793)(12,0.513242753000793)(12,0.513242753000793)(12,0.513242753000793)(12,0.513242753000793)(12,0.513242753000793)(12,0.513242753000793)(13,0.563071802960045)(13,0.563071802960045)(13,0.563071802960045)(13,0.563071802960045)(13,0.563071802960045)(13,0.563071802960045)(13,0.563071802960045)(13,0.563071802960045)(13,0.563071802960045)(14,0.609412089731185)(14,0.609412089731185)(14,0.609412089731185)(14,0.609412089731185)(14,0.609412089731185)(14,0.609412089731185)(14,0.609412089731185)(14,0.609412089731185)(14,0.609412089731185)(15,0.654045205749223)(15,0.654045205749223)(15,0.654045205749223)(15,0.654045205749223)(15,0.654045205749223)(15,0.654045205749223)(15,0.654045205749223)(15,0.654045205749223)(15,0.654045205749223)(15,0.654045205749223)(15,0.654045205749223)(15,0.654045205749223)(16,0.693800447635594)(16,0.693800447635594)(16,0.693800447635594)(16,0.693800447635594)(16,0.693800447635594)(16,0.693800447635594)(16,0.693800447635594)(16,0.693800447635594)(16,0.693800447635594)(16,0.693800447635594)(16,0.693800447635594)(16,0.693800447635594)(16,0.693800447635594)(16,0.693800447635594)(16,0.693800447635594)(16,0.693800447635594)(16,0.693800447635594)(16,0.693800447635594)(16,0.693800447635594)(16,0.693800447635594)(16,0.693800447635594)(17,0.719964994412613)(17,0.719964994412613)(17,0.719964994412613)(17,0.719964994412613)(17,0.719964994412613)(17,0.719964994412613)(17,0.719964994412613)(17,0.719964994412613)(17,0.719964994412613)(17,0.719964994412613)(17,0.719964994412613)(17,0.719964994412613)(17,0.719964994412613)(17,0.719964994412613)(17,0.719964994412613)(17,0.719964994412613)(17,0.719964994412613)(17,0.719964994412613)(17,0.719964994412613)(17,0.719964994412613)(17,0.719964994412613)(17,0.719964994412613)(17,0.719964994412613)(18,0.738956378279693)(18,0.738956378279693)(18,0.738956378279693)(18,0.738956378279693)(18,0.738956378279693)(18,0.738956378279693)(18,0.738956378279693)(18,0.738956378279693)(18,0.738956378279693)(18,0.738956378279693)(18,0.738956378279693)(18,0.738956378279693)(18,0.738956378279693)(18,0.738956378279693)(18,0.738956378279693)(18,0.738956378279693)(18,0.738956378279693)(18,0.738956378279693)(18,0.738956378279693)(18,0.738956378279693)(18,0.738956378279693)(18,0.738956378279693)(18,0.738956378279693)(18,0.738956378279693)(18,0.738956378279693)(18,0.738956378279693)(18,0.738956378279693)(19,0.756578089042498)(19,0.756578089042498)(19,0.756578089042498)(19,0.756578089042498)(19,0.756578089042498)(19,0.756578089042498)(19,0.756578089042498)(19,0.756578089042498)(19,0.756578089042498)(19,0.756578089042498)(19,0.756578089042498)(19,0.756578089042498)(19,0.756578089042498)(19,0.756578089042498)(19,0.756578089042498)(19,0.756578089042498)(19,0.756578089042498)(19,0.756578089042498)(19,0.756578089042498)(19,0.756578089042498)(19,0.756578089042498)(19,0.756578089042498)(19,0.756578089042498)(19,0.756578089042498)(19,0.756578089042498)(19,0.756578089042498)(19,0.756578089042498)(19,0.756578089042498)(19,0.756578089042498)(20,0.766154555098519)(20,0.766154555098519)(20,0.766154555098519)(20,0.766154555098519)(20,0.766154555098519)(20,0.766154555098519)(20,0.766154555098519)(20,0.766154555098519)(20,0.766154555098519)(20,0.766154555098519)(20,0.766154555098519)(20,0.766154555098519)(20,0.766154555098519)(20,0.766154555098519)(20,0.766154555098519)(20,0.766154555098519)(20,0.766154555098519)(20,0.766154555098519)(20,0.766154555098519)(20,0.766154555098519)(20,0.766154555098519)(20,0.766154555098519)(20,0.766154555098519)(20,0.766154555098519)(20,0.766154555098519)(20,0.766154555098519)(20,0.766154555098519)(20,0.766154555098519)(20,0.766154555098519)(20,0.766154555098519)(20,0.766154555098519)(20,0.766154555098519)(20,0.766154555098519)(20,0.766154555098519)(20,0.766154555098519)(20,0.766154555098519)(20,0.766154555098519)(20,0.766154555098519)(20,0.766154555098519)(21,0.772168694536424)(21,0.772168694536424)(21,0.772168694536424)(21,0.772168694536424)(21,0.772168694536424)(21,0.772168694536424)(21,0.772168694536424)(21,0.772168694536424)(21,0.772168694536424)(21,0.772168694536424)(21,0.772168694536424)(21,0.772168694536424)(21,0.772168694536424)(21,0.772168694536424)(21,0.772168694536424)(21,0.772168694536424)(21,0.772168694536424)(21,0.772168694536424)(21,0.772168694536424)(21,0.772168694536424)(21,0.772168694536424)(21,0.772168694536424)(21,0.772168694536424)(21,0.772168694536424)(21,0.772168694536424)(21,0.772168694536424)(21,0.772168694536424)(21,0.772168694536424)(21,0.772168694536424)(21,0.772168694536424)(21,0.772168694536424)(21,0.772168694536424)(21,0.772168694536424)(21,0.772168694536424)(21,0.772168694536424)(21,0.772168694536424)(21,0.772168694536424)(21,0.772168694536424)(21,0.772168694536424)(21,0.772168694536424)(21,0.772168694536424)(21,0.772168694536424)(21,0.772168694536424)(21,0.772168694536424)(21,0.772168694536424)(22,0.775246122095812)(22,0.775246122095812)(22,0.775246122095812)(22,0.775246122095812)(22,0.775246122095812)(22,0.775246122095812)(22,0.775246122095812)(22,0.775246122095812)(22,0.775246122095812)(22,0.775246122095812)(22,0.775246122095812)(22,0.775246122095812)(22,0.775246122095812)(22,0.775246122095812)(22,0.775246122095812)(22,0.775246122095812)(22,0.775246122095812)(22,0.775246122095812)(22,0.775246122095812)(22,0.775246122095812)(22,0.775246122095812)(22,0.775246122095812)(22,0.775246122095812)(22,0.775246122095812)(22,0.775246122095812)(22,0.775246122095812)(22,0.775246122095812)(22,0.775246122095812)(22,0.775246122095812)(22,0.775246122095812)(22,0.775246122095812)(22,0.775246122095812)(22,0.775246122095812)(22,0.775246122095812)(22,0.775246122095812)(22,0.775246122095812)(22,0.775246122095812)(22,0.775246122095812)(22,0.775246122095812)(22,0.775246122095812)(22,0.775246122095812)(22,0.775246122095812)(23,0.779159487806811)(23,0.779159487806811)(23,0.779159487806811)(23,0.779159487806811)(23,0.779159487806811)(23,0.779159487806811)(23,0.779159487806811)(23,0.779159487806811)(23,0.779159487806811)(23,0.779159487806811)(23,0.779159487806811)(23,0.779159487806811)(23,0.779159487806811)(23,0.779159487806811)(23,0.779159487806811)(23,0.779159487806811)(23,0.779159487806811)(23,0.779159487806811)(23,0.779159487806811)(23,0.779159487806811)(23,0.779159487806811)(23,0.779159487806811)(23,0.779159487806811)(23,0.779159487806811)(23,0.779159487806811)(23,0.779159487806811)(23,0.779159487806811)(23,0.779159487806811)(23,0.779159487806811)(23,0.779159487806811)(23,0.779159487806811)(23,0.779159487806811)(23,0.779159487806811)(23,0.779159487806811)(23,0.779159487806811)(23,0.779159487806811)(23,0.779159487806811)(23,0.779159487806811)(23,0.779159487806811)(23,0.779159487806811)(23,0.779159487806811)(23,0.779159487806811)(23,0.779159487806811)(23,0.779159487806811)(23,0.779159487806811)(23,0.779159487806811)(23,0.779159487806811)(23,0.779159487806811)(23,0.779159487806811)(24,0.784632419978595)(24,0.784632419978595)(24,0.784632419978595)(24,0.784632419978595)(24,0.784632419978595)(24,0.784632419978595)(24,0.784632419978595)(24,0.784632419978595)(24,0.784632419978595)(24,0.784632419978595)(24,0.784632419978595)(24,0.784632419978595)(24,0.784632419978595)(24,0.784632419978595)(24,0.784632419978595)(24,0.784632419978595)(24,0.784632419978595)(24,0.784632419978595)(24,0.784632419978595)(24,0.784632419978595)(24,0.784632419978595)(24,0.784632419978595)(24,0.784632419978595)(24,0.784632419978595)(24,0.784632419978595)(24,0.784632419978595)(24,0.784632419978595)(24,0.784632419978595)(24,0.784632419978595)(24,0.784632419978595)(24,0.784632419978595)(24,0.784632419978595)(24,0.784632419978595)(24,0.784632419978595)(24,0.784632419978595)(24,0.784632419978595)(24,0.784632419978595)(24,0.784632419978595)(24,0.784632419978595)(24,0.784632419978595)(24,0.784632419978595)(24,0.784632419978595)(24,0.784632419978595)(24,0.784632419978595)(24,0.784632419978595)(24,0.784632419978595)(24,0.784632419978595)(24,0.784632419978595)(24,0.784632419978595)(24,0.784632419978595)(24,0.784632419978595)(25,0.790027700330571)(25,0.790027700330571)(25,0.790027700330571)(25,0.790027700330571)(25,0.790027700330571)(25,0.790027700330571)(25,0.790027700330571)(25,0.790027700330571)(25,0.790027700330571)(25,0.790027700330571)(25,0.790027700330571)(25,0.790027700330571)(25,0.790027700330571)(25,0.790027700330571)(25,0.790027700330571)(25,0.790027700330571)(25,0.790027700330571)(25,0.790027700330571)(25,0.790027700330571)(25,0.790027700330571)(25,0.790027700330571)(25,0.790027700330571)(25,0.790027700330571)(25,0.790027700330571)(25,0.790027700330571)(25,0.790027700330571)(25,0.790027700330571)(25,0.790027700330571)(25,0.790027700330571)(25,0.790027700330571)(25,0.790027700330571)(25,0.790027700330571)(25,0.790027700330571)(25,0.790027700330571)(25,0.790027700330571)(25,0.790027700330571)(25,0.790027700330571)(25,0.790027700330571)(25,0.790027700330571)(25,0.790027700330571)(26,0.79447497049185)(26,0.79447497049185)(26,0.79447497049185)(26,0.79447497049185)(26,0.79447497049185)(26,0.79447497049185)(26,0.79447497049185)(26,0.79447497049185)(26,0.79447497049185)(26,0.79447497049185)(26,0.79447497049185)(26,0.79447497049185)(26,0.79447497049185)(26,0.79447497049185)(26,0.79447497049185)(26,0.79447497049185)(26,0.79447497049185)(26,0.79447497049185)(26,0.79447497049185)(26,0.79447497049185)(26,0.79447497049185)(26,0.79447497049185)(26,0.79447497049185)(26,0.79447497049185)(26,0.79447497049185)(26,0.79447497049185)(26,0.79447497049185)(26,0.79447497049185)(26,0.79447497049185)(26,0.79447497049185)(26,0.79447497049185)(26,0.79447497049185)(26,0.79447497049185)(26,0.79447497049185)(26,0.79447497049185)(26,0.79447497049185)(26,0.79447497049185)(26,0.79447497049185)(26,0.79447497049185)(26,0.79447497049185)(26,0.79447497049185)(27,0.796074899938778)(27,0.796074899938778)(27,0.796074899938778)(27,0.796074899938778)(27,0.796074899938778)(27,0.796074899938778)(27,0.796074899938778)(27,0.796074899938778)(27,0.796074899938778)(27,0.796074899938778)(27,0.796074899938778)(27,0.796074899938778)(27,0.796074899938778)(27,0.796074899938778)(27,0.796074899938778)(27,0.796074899938778)(27,0.796074899938778)(27,0.796074899938778)(27,0.796074899938778)(27,0.796074899938778)(27,0.796074899938778)(27,0.796074899938778)(27,0.796074899938778)(27,0.796074899938778)(27,0.796074899938778)(27,0.796074899938778)(27,0.796074899938778)(27,0.796074899938778)(27,0.796074899938778)(27,0.796074899938778)(27,0.796074899938778)(27,0.796074899938778)(27,0.796074899938778)(27,0.796074899938778)(27,0.796074899938778)(27,0.796074899938778)(27,0.796074899938778)(27,0.796074899938778)(27,0.796074899938778)(28,0.795999379478335)(28,0.795999379478335)(28,0.795999379478335)(28,0.795999379478335)(28,0.795999379478335)(28,0.795999379478335)(28,0.795999379478335)(28,0.795999379478335)(28,0.795999379478335)(28,0.795999379478335)(28,0.795999379478335)(28,0.795999379478335)(28,0.795999379478335)(28,0.795999379478335)(28,0.795999379478335)(28,0.795999379478335)(28,0.795999379478335)(28,0.795999379478335)(28,0.795999379478335)(28,0.795999379478335)(28,0.795999379478335)(28,0.795999379478335)(28,0.795999379478335)(28,0.795999379478335)(28,0.795999379478335)(28,0.795999379478335)(28,0.795999379478335)(28,0.795999379478335)(28,0.795999379478335)(28,0.795999379478335)(28,0.795999379478335)(28,0.795999379478335)(28,0.795999379478335)(28,0.795999379478335)(28,0.795999379478335)(28,0.795999379478335)(28,0.795999379478335)(28,0.795999379478335)(28,0.795999379478335)(29,0.798409856541205)(29,0.798409856541205)(29,0.798409856541205)(29,0.798409856541205)(29,0.798409856541205)(29,0.798409856541205)(29,0.798409856541205)(29,0.798409856541205)(29,0.798409856541205)(29,0.798409856541205)(29,0.798409856541205)(29,0.798409856541205)(29,0.798409856541205)(29,0.798409856541205)(29,0.798409856541205)(29,0.798409856541205)(29,0.798409856541205)(29,0.798409856541205)(29,0.798409856541205)(29,0.798409856541205)(29,0.798409856541205)(29,0.798409856541205)(29,0.798409856541205)(29,0.798409856541205)(29,0.798409856541205)(29,0.798409856541205)(29,0.798409856541205)(29,0.798409856541205)(29,0.798409856541205)(29,0.798409856541205)(29,0.798409856541205)(29,0.798409856541205)(29,0.798409856541205)(29,0.798409856541205)(29,0.798409856541205)(29,0.798409856541205)(29,0.798409856541205)(30,0.801907554024585)(30,0.801907554024585)(30,0.801907554024585)(30,0.801907554024585)(30,0.801907554024585)(30,0.801907554024585)(30,0.801907554024585)(30,0.801907554024585)(30,0.801907554024585)(30,0.801907554024585)(30,0.801907554024585)(30,0.801907554024585)(30,0.801907554024585)(30,0.801907554024585)(30,0.801907554024585)(30,0.801907554024585)(30,0.801907554024585)(30,0.801907554024585)(30,0.801907554024585)(30,0.801907554024585)(30,0.801907554024585)(30,0.801907554024585)(30,0.801907554024585)(30,0.801907554024585)(30,0.801907554024585)(30,0.801907554024585)(30,0.801907554024585)(30,0.801907554024585)(30,0.801907554024585)(31,0.806226472417736)(31,0.806226472417736)(31,0.806226472417736)(31,0.806226472417736)(31,0.806226472417736)(31,0.806226472417736)(31,0.806226472417736)(31,0.806226472417736)(31,0.806226472417736)(31,0.806226472417736)(31,0.806226472417736)(31,0.806226472417736)(31,0.806226472417736)(31,0.806226472417736)(31,0.806226472417736)(31,0.806226472417736)(31,0.806226472417736)(31,0.806226472417736)(31,0.806226472417736)(31,0.806226472417736)(31,0.806226472417736)(31,0.806226472417736)(31,0.806226472417736)(31,0.806226472417736)(31,0.806226472417736)(31,0.806226472417736)(31,0.806226472417736)(31,0.806226472417736)(31,0.806226472417736)(31,0.806226472417736)(31,0.806226472417736)(31,0.806226472417736)(31,0.806226472417736)(31,0.806226472417736)(31,0.806226472417736)(32,0.810075821596545)(32,0.810075821596545)(32,0.810075821596545)(32,0.810075821596545)(32,0.810075821596545)(32,0.810075821596545)(32,0.810075821596545)(32,0.810075821596545)(32,0.810075821596545)(32,0.810075821596545)(32,0.810075821596545)(32,0.810075821596545)(32,0.810075821596545)(32,0.810075821596545)(32,0.810075821596545)(32,0.810075821596545)(32,0.810075821596545)(32,0.810075821596545)(32,0.810075821596545)(32,0.810075821596545)(32,0.810075821596545)(32,0.810075821596545)(32,0.810075821596545)(32,0.810075821596545)(32,0.810075821596545)(32,0.810075821596545)(32,0.810075821596545)(32,0.810075821596545)(32,0.810075821596545)(32,0.810075821596545)(32,0.810075821596545)(32,0.810075821596545)(32,0.810075821596545)(32,0.810075821596545)(32,0.810075821596545)(32,0.810075821596545)(32,0.810075821596545)(33,0.81300247138301)(33,0.81300247138301)(33,0.81300247138301)(33,0.81300247138301)(33,0.81300247138301)(33,0.81300247138301)(33,0.81300247138301)(33,0.81300247138301)(33,0.81300247138301)(33,0.81300247138301)(33,0.81300247138301)(33,0.81300247138301)(33,0.81300247138301)(33,0.81300247138301)(33,0.81300247138301)(33,0.81300247138301)(33,0.81300247138301)(33,0.81300247138301)(33,0.81300247138301)(33,0.81300247138301)(33,0.81300247138301)(33,0.81300247138301)(33,0.81300247138301)(33,0.81300247138301)(33,0.81300247138301)(33,0.81300247138301)(33,0.81300247138301)(33,0.81300247138301)(33,0.81300247138301)(33,0.81300247138301)(33,0.81300247138301)(33,0.81300247138301)(33,0.81300247138301)(33,0.81300247138301)(33,0.81300247138301)(34,0.815634494830405)(34,0.815634494830405)(34,0.815634494830405)(34,0.815634494830405)(34,0.815634494830405)(34,0.815634494830405)(34,0.815634494830405)(34,0.815634494830405)(34,0.815634494830405)(34,0.815634494830405)(34,0.815634494830405)(34,0.815634494830405)(34,0.815634494830405)(34,0.815634494830405)(34,0.815634494830405)(34,0.815634494830405)(34,0.815634494830405)(34,0.815634494830405)(34,0.815634494830405)(34,0.815634494830405)(34,0.815634494830405)(34,0.815634494830405)(34,0.815634494830405)(34,0.815634494830405)(34,0.815634494830405)(34,0.815634494830405)(34,0.815634494830405)(34,0.815634494830405)(35,0.817866616393611)(35,0.817866616393611)(35,0.817866616393611)(35,0.817866616393611)(35,0.817866616393611)(35,0.817866616393611)(35,0.817866616393611)(35,0.817866616393611)(35,0.817866616393611)(35,0.817866616393611)(35,0.817866616393611)(35,0.817866616393611)(35,0.817866616393611)(35,0.817866616393611)(35,0.817866616393611)(35,0.817866616393611)(35,0.817866616393611)(35,0.817866616393611)(35,0.817866616393611)(35,0.817866616393611)(35,0.817866616393611)(35,0.817866616393611)(36,0.820093974807631)(36,0.820093974807631)(36,0.820093974807631)(36,0.820093974807631)(36,0.820093974807631)(36,0.820093974807631)(36,0.820093974807631)(36,0.820093974807631)(36,0.820093974807631)(36,0.820093974807631)(36,0.820093974807631)(36,0.820093974807631)(36,0.820093974807631)(36,0.820093974807631)(36,0.820093974807631)(36,0.820093974807631)(36,0.820093974807631)(36,0.820093974807631)(36,0.820093974807631)(36,0.820093974807631)(36,0.820093974807631)(36,0.820093974807631)(36,0.820093974807631)(37,0.823267231828805)(37,0.823267231828805)(37,0.823267231828805)(37,0.823267231828805)(37,0.823267231828805)(37,0.823267231828805)(37,0.823267231828805)(37,0.823267231828805)(37,0.823267231828805)(37,0.823267231828805)(37,0.823267231828805)(37,0.823267231828805)(37,0.823267231828805)(37,0.823267231828805)(37,0.823267231828805)(37,0.823267231828805)(37,0.823267231828805)(37,0.823267231828805)(37,0.823267231828805)(37,0.823267231828805)(37,0.823267231828805)(37,0.823267231828805)(37,0.823267231828805)(37,0.823267231828805)(37,0.823267231828805)(37,0.823267231828805)(38,0.826497293362125)(38,0.826497293362125)(38,0.826497293362125)(38,0.826497293362125)(38,0.826497293362125)(38,0.826497293362125)(38,0.826497293362125)(38,0.826497293362125)(38,0.826497293362125)(38,0.826497293362125)(38,0.826497293362125)(38,0.826497293362125)(38,0.826497293362125)(38,0.826497293362125)(38,0.826497293362125)(38,0.826497293362125)(38,0.826497293362125)(38,0.826497293362125)(38,0.826497293362125)(38,0.826497293362125)(38,0.826497293362125)(38,0.826497293362125)(38,0.826497293362125)(38,0.826497293362125)(38,0.826497293362125)(39,0.82935832921807)(39,0.82935832921807)(39,0.82935832921807)(39,0.82935832921807)(39,0.82935832921807)(39,0.82935832921807)(39,0.82935832921807)(39,0.82935832921807)(39,0.82935832921807)(39,0.82935832921807)(39,0.82935832921807)(39,0.82935832921807)(39,0.82935832921807)(39,0.82935832921807)(39,0.82935832921807)(39,0.82935832921807)(39,0.82935832921807)(39,0.82935832921807)(39,0.82935832921807)(39,0.82935832921807)(39,0.82935832921807)(39,0.82935832921807)(39,0.82935832921807)(39,0.82935832921807)(39,0.82935832921807)(39,0.82935832921807)(39,0.82935832921807)(40,0.83140553472798)(40,0.83140553472798)(40,0.83140553472798)(40,0.83140553472798)(40,0.83140553472798)(40,0.83140553472798)(40,0.83140553472798)(40,0.83140553472798)(40,0.83140553472798)(40,0.83140553472798)(40,0.83140553472798)(40,0.83140553472798)(40,0.83140553472798)(40,0.83140553472798)(40,0.83140553472798)(40,0.83140553472798)(40,0.83140553472798)(40,0.83140553472798)(40,0.83140553472798)(40,0.83140553472798)(41,0.832979516062166)(41,0.832979516062166)(41,0.832979516062166)(41,0.832979516062166)(41,0.832979516062166)(41,0.832979516062166)(41,0.832979516062166)(41,0.832979516062166)(41,0.832979516062166)(41,0.832979516062166)(41,0.832979516062166)(41,0.832979516062166)(41,0.832979516062166)(41,0.832979516062166)(41,0.832979516062166)(41,0.832979516062166)(41,0.832979516062166)(41,0.832979516062166)(41,0.832979516062166)(41,0.832979516062166)(41,0.832979516062166)(41,0.832979516062166)(42,0.835016811547283)(42,0.835016811547283)(42,0.835016811547283)(42,0.835016811547283)(42,0.835016811547283)(42,0.835016811547283)(42,0.835016811547283)(42,0.835016811547283)(42,0.835016811547283)(42,0.835016811547283)(42,0.835016811547283)(42,0.835016811547283)(42,0.835016811547283)(42,0.835016811547283)(42,0.835016811547283)(42,0.835016811547283)(43,0.836782158554706)(43,0.836782158554706)(43,0.836782158554706)(43,0.836782158554706)(43,0.836782158554706)(43,0.836782158554706)(43,0.836782158554706)(43,0.836782158554706)(43,0.836782158554706)(43,0.836782158554706)(43,0.836782158554706)(43,0.836782158554706)(43,0.836782158554706)(43,0.836782158554706)(43,0.836782158554706)(44,0.838048687172316)(44,0.838048687172316)(44,0.838048687172316)(44,0.838048687172316)(44,0.838048687172316)(44,0.838048687172316)(44,0.838048687172316)(44,0.838048687172316)(44,0.838048687172316)(44,0.838048687172316)(44,0.838048687172316)(44,0.838048687172316)(44,0.838048687172316)(44,0.838048687172316)(44,0.838048687172316)(44,0.838048687172316)(44,0.838048687172316)(44,0.838048687172316)(44,0.838048687172316)(44,0.838048687172316)(44,0.838048687172316)(44,0.838048687172316)(44,0.838048687172316)(44,0.838048687172316)(44,0.838048687172316)(44,0.838048687172316)(45,0.83911360475653)(45,0.83911360475653)(45,0.83911360475653)(45,0.83911360475653)(45,0.83911360475653)(45,0.83911360475653)(45,0.83911360475653)(45,0.83911360475653)(45,0.83911360475653)(45,0.83911360475653)(45,0.83911360475653)(45,0.83911360475653)(45,0.83911360475653)(45,0.83911360475653)(45,0.83911360475653)(45,0.83911360475653)(45,0.83911360475653)(45,0.83911360475653)(45,0.83911360475653)(45,0.83911360475653)(45,0.83911360475653)(45,0.83911360475653)(45,0.83911360475653)(46,0.840165489288512)(46,0.840165489288512)(46,0.840165489288512)(46,0.840165489288512)(46,0.840165489288512)(46,0.840165489288512)(46,0.840165489288512)(46,0.840165489288512)(46,0.840165489288512)(46,0.840165489288512)(46,0.840165489288512)(46,0.840165489288512)(46,0.840165489288512)(46,0.840165489288512)(46,0.840165489288512)(46,0.840165489288512)(46,0.840165489288512)(46,0.840165489288512)(46,0.840165489288512)(46,0.840165489288512)(47,0.841519534640496)(47,0.841519534640496)(47,0.841519534640496)(47,0.841519534640496)(47,0.841519534640496)(47,0.841519534640496)(47,0.841519534640496)(47,0.841519534640496)(47,0.841519534640496)(47,0.841519534640496)(47,0.841519534640496)(47,0.841519534640496)(47,0.841519534640496)(47,0.841519534640496)(47,0.841519534640496)(47,0.841519534640496)(47,0.841519534640496)(47,0.841519534640496)(47,0.841519534640496)(47,0.841519534640496)(47,0.841519534640496)(47,0.841519534640496)(48,0.842937547168656)(48,0.842937547168656)(48,0.842937547168656)(48,0.842937547168656)(48,0.842937547168656)(48,0.842937547168656)(48,0.842937547168656)(48,0.842937547168656)(48,0.842937547168656)(49,0.844463742087862)(49,0.844463742087862)(49,0.844463742087862)(49,0.844463742087862)(49,0.844463742087862)(49,0.844463742087862)(49,0.844463742087862)(49,0.844463742087862)(49,0.844463742087862)(49,0.844463742087862)(49,0.844463742087862)(49,0.844463742087862)(49,0.844463742087862)(49,0.844463742087862)(49,0.844463742087862)(49,0.844463742087862)(49,0.844463742087862)(49,0.844463742087862)(49,0.844463742087862)(49,0.844463742087862)(50,0.845997992207294)(50,0.845997992207294)(50,0.845997992207294)(50,0.845997992207294)(50,0.845997992207294)(50,0.845997992207294)(50,0.845997992207294)(50,0.845997992207294)(50,0.845997992207294)(50,0.845997992207294)(50,0.845997992207294)(50,0.845997992207294)(50,0.845997992207294)(50,0.845997992207294)(50,0.845997992207294)(50,0.845997992207294)(50,0.845997992207294)(50,0.845997992207294)(50,0.845997992207294)(51,0.847649750433022)(51,0.847649750433022)(51,0.847649750433022)(51,0.847649750433022)(51,0.847649750433022)(51,0.847649750433022)(51,0.847649750433022)(51,0.847649750433022)(51,0.847649750433022)(51,0.847649750433022)(51,0.847649750433022)(51,0.847649750433022)(51,0.847649750433022)(51,0.847649750433022)(52,0.850004427720027)(52,0.850004427720027)(52,0.850004427720027)(52,0.850004427720027)(52,0.850004427720027)(52,0.850004427720027)(52,0.850004427720027)(52,0.850004427720027)(52,0.850004427720027)(52,0.850004427720027)(52,0.850004427720027)(52,0.850004427720027)(52,0.850004427720027)(52,0.850004427720027)(52,0.850004427720027)(52,0.850004427720027)(53,0.852261417456051)(53,0.852261417456051)(53,0.852261417456051)(53,0.852261417456051)(53,0.852261417456051)(53,0.852261417456051)(53,0.852261417456051)(54,0.854751441376187)(54,0.854751441376187)(54,0.854751441376187)(54,0.854751441376187)(54,0.854751441376187)(54,0.854751441376187)(54,0.854751441376187)(54,0.854751441376187)(54,0.854751441376187)(54,0.854751441376187)(54,0.854751441376187)(54,0.854751441376187)(54,0.854751441376187)(54,0.854751441376187)(54,0.854751441376187)(55,0.857406982603522)(55,0.857406982603522)(55,0.857406982603522)(55,0.857406982603522)(55,0.857406982603522)(55,0.857406982603522)(55,0.857406982603522)(55,0.857406982603522)(55,0.857406982603522)(55,0.857406982603522)(55,0.857406982603522)(55,0.857406982603522)(55,0.857406982603522)(55,0.857406982603522)(55,0.857406982603522)(55,0.857406982603522)(55,0.857406982603522)(55,0.857406982603522)(56,0.859328278945926)(56,0.859328278945926)(56,0.859328278945926)(56,0.859328278945926)(56,0.859328278945926)(56,0.859328278945926)(56,0.859328278945926)(56,0.859328278945926)(56,0.859328278945926)(56,0.859328278945926)(56,0.859328278945926)(56,0.859328278945926)(56,0.859328278945926)(57,0.861063335311418)(57,0.861063335311418)(57,0.861063335311418)(57,0.861063335311418)(57,0.861063335311418)(57,0.861063335311418)(57,0.861063335311418)(57,0.861063335311418)(57,0.861063335311418)(57,0.861063335311418)(57,0.861063335311418)(57,0.861063335311418)(57,0.861063335311418)(57,0.861063335311418)(58,0.862397066154262)(58,0.862397066154262)(58,0.862397066154262)(58,0.862397066154262)(58,0.862397066154262)(58,0.862397066154262)(58,0.862397066154262)(58,0.862397066154262)(58,0.862397066154262)(58,0.862397066154262)(58,0.862397066154262)(58,0.862397066154262)(58,0.862397066154262)(58,0.862397066154262)(58,0.862397066154262)(59,0.863641311211628)(59,0.863641311211628)(59,0.863641311211628)(59,0.863641311211628)(59,0.863641311211628)(59,0.863641311211628)(59,0.863641311211628)(59,0.863641311211628)(59,0.863641311211628)(59,0.863641311211628)(60,0.864923564406168)(60,0.864923564406168)(60,0.864923564406168)(60,0.864923564406168)(60,0.864923564406168)(60,0.864923564406168)(60,0.864923564406168)(60,0.864923564406168)(60,0.864923564406168)(60,0.864923564406168)(60,0.864923564406168)(60,0.864923564406168)(61,0.866069771229191)(61,0.866069771229191)(61,0.866069771229191)(61,0.866069771229191)(61,0.866069771229191)(61,0.866069771229191)(61,0.866069771229191)(61,0.866069771229191)(61,0.866069771229191)(61,0.866069771229191)(61,0.866069771229191)(61,0.866069771229191)(61,0.866069771229191)(61,0.866069771229191)(61,0.866069771229191)(61,0.866069771229191)(61,0.866069771229191)(61,0.866069771229191)(62,0.86714028889457)(62,0.86714028889457)(62,0.86714028889457)(62,0.86714028889457)(62,0.86714028889457)(62,0.86714028889457)(62,0.86714028889457)(62,0.86714028889457)(62,0.86714028889457)(62,0.86714028889457)(62,0.86714028889457)(63,0.868379210237588)(63,0.868379210237588)(63,0.868379210237588)(63,0.868379210237588)(63,0.868379210237588)(63,0.868379210237588)(64,0.86945236496554)(64,0.86945236496554)(64,0.86945236496554)(64,0.86945236496554)(64,0.86945236496554)(64,0.86945236496554)(64,0.86945236496554)(64,0.86945236496554)(64,0.86945236496554)(64,0.86945236496554)(64,0.86945236496554)(64,0.86945236496554)(64,0.86945236496554)(64,0.86945236496554)(65,0.870630399980386)(65,0.870630399980386)(65,0.870630399980386)(65,0.870630399980386)(65,0.870630399980386)(65,0.870630399980386)(65,0.870630399980386)(65,0.870630399980386)(65,0.870630399980386)(65,0.870630399980386)(65,0.870630399980386)(65,0.870630399980386)(66,0.871661181225682)(66,0.871661181225682)(66,0.871661181225682)(66,0.871661181225682)(66,0.871661181225682)(66,0.871661181225682)(66,0.871661181225682)(66,0.871661181225682)(66,0.871661181225682)(66,0.871661181225682)(67,0.872777656661999)(67,0.872777656661999)(67,0.872777656661999)(67,0.872777656661999)(67,0.872777656661999)(67,0.872777656661999)(67,0.872777656661999)(67,0.872777656661999)(67,0.872777656661999)(67,0.872777656661999)(67,0.872777656661999)(67,0.872777656661999)(67,0.872777656661999)(67,0.872777656661999)(67,0.872777656661999)(68,0.874094440732809)(68,0.874094440732809)(68,0.874094440732809)(68,0.874094440732809)(68,0.874094440732809)(68,0.874094440732809)(68,0.874094440732809)(68,0.874094440732809)(69,0.875274710403873)(69,0.875274710403873)(69,0.875274710403873)(69,0.875274710403873)(69,0.875274710403873)(69,0.875274710403873)(69,0.875274710403873)(69,0.875274710403873)(69,0.875274710403873)(69,0.875274710403873)(70,0.876213199750608)(70,0.876213199750608)(70,0.876213199750608)(70,0.876213199750608)(70,0.876213199750608)(70,0.876213199750608)(70,0.876213199750608)(70,0.876213199750608)(70,0.876213199750608)(70,0.876213199750608)(70,0.876213199750608)(70,0.876213199750608)(70,0.876213199750608)(70,0.876213199750608)(70,0.876213199750608)(71,0.877107762438385)(71,0.877107762438385)(71,0.877107762438385)(71,0.877107762438385)(71,0.877107762438385)(71,0.877107762438385)(71,0.877107762438385)(71,0.877107762438385)(71,0.877107762438385)(71,0.877107762438385)(71,0.877107762438385)(71,0.877107762438385)(71,0.877107762438385)(71,0.877107762438385)(71,0.877107762438385)(71,0.877107762438385)(71,0.877107762438385)(71,0.877107762438385)(72,0.877849086914781)(72,0.877849086914781)(72,0.877849086914781)(72,0.877849086914781)(72,0.877849086914781)(72,0.877849086914781)(72,0.877849086914781)(72,0.877849086914781)(72,0.877849086914781)(72,0.877849086914781)(72,0.877849086914781)(73,0.878339661501157)(73,0.878339661501157)(73,0.878339661501157)(73,0.878339661501157)(74,0.87895365306575)(74,0.87895365306575)(74,0.87895365306575)(74,0.87895365306575)(74,0.87895365306575)(74,0.87895365306575)(74,0.87895365306575)(75,0.879497079213324)(75,0.879497079213324)(75,0.879497079213324)(75,0.879497079213324)(75,0.879497079213324)(75,0.879497079213324)(75,0.879497079213324)(75,0.879497079213324)(75,0.879497079213324)(76,0.879983979577299)(76,0.879983979577299)(76,0.879983979577299)(76,0.879983979577299)(76,0.879983979577299)(76,0.879983979577299)(76,0.879983979577299)(76,0.879983979577299)(77,0.880428644612614)(77,0.880428644612614)(77,0.880428644612614)(77,0.880428644612614)(77,0.880428644612614)(77,0.880428644612614)(77,0.880428644612614)(77,0.880428644612614)(77,0.880428644612614)(77,0.880428644612614)(77,0.880428644612614)(78,0.880884441235036)(78,0.880884441235036)(78,0.880884441235036)(78,0.880884441235036)(78,0.880884441235036)(78,0.880884441235036)(78,0.880884441235036)(78,0.880884441235036)(78,0.880884441235036)(78,0.880884441235036)(79,0.881525322533769)(79,0.881525322533769)(79,0.881525322533769)(79,0.881525322533769)(79,0.881525322533769)(79,0.881525322533769)(79,0.881525322533769)(79,0.881525322533769)(80,0.882069292779447)(80,0.882069292779447)(80,0.882069292779447)(80,0.882069292779447)(80,0.882069292779447)(80,0.882069292779447)(80,0.882069292779447)(80,0.882069292779447)(80,0.882069292779447)(80,0.882069292779447)(80,0.882069292779447)(81,0.882661011436284)(81,0.882661011436284)(81,0.882661011436284)(81,0.882661011436284)(81,0.882661011436284)(81,0.882661011436284)(82,0.883283510096945)(82,0.883283510096945)(82,0.883283510096945)(82,0.883283510096945)(82,0.883283510096945)(82,0.883283510096945)(83,0.883868112059667)(83,0.883868112059667)(83,0.883868112059667)(83,0.883868112059667)(83,0.883868112059667)(83,0.883868112059667)(84,0.884270575794794)(84,0.884270575794794)(84,0.884270575794794)(84,0.884270575794794)(84,0.884270575794794)(84,0.884270575794794)(84,0.884270575794794)(84,0.884270575794794)(84,0.884270575794794)(84,0.884270575794794)(85,0.884624618568871)(85,0.884624618568871)(85,0.884624618568871)(85,0.884624618568871)(85,0.884624618568871)(85,0.884624618568871)(85,0.884624618568871)(85,0.884624618568871)(85,0.884624618568871)(86,0.884937108886295)(86,0.884937108886295)(86,0.884937108886295)(86,0.884937108886295)(86,0.884937108886295)(86,0.884937108886295)(86,0.884937108886295)(86,0.884937108886295)(86,0.884937108886295)(87,0.885080252494528)(87,0.885080252494528)(87,0.885080252494528)(87,0.885080252494528)(87,0.885080252494528)(87,0.885080252494528)(87,0.885080252494528)(87,0.885080252494528)(87,0.885080252494528)(87,0.885080252494528)(87,0.885080252494528)(88,0.885296675031833)(88,0.885296675031833)(88,0.885296675031833)(88,0.885296675031833)(88,0.885296675031833)(88,0.885296675031833)(88,0.885296675031833)(89,0.885504136120846)(89,0.885504136120846)(89,0.885504136120846)(89,0.885504136120846)(89,0.885504136120846)(89,0.885504136120846)(89,0.885504136120846)(90,0.885617405637599)(90,0.885617405637599)(90,0.885617405637599)(90,0.885617405637599)(90,0.885617405637599)(90,0.885617405637599)(90,0.885617405637599)(91,0.885896246301623)(91,0.885896246301623)(91,0.885896246301623)(91,0.885896246301623)(91,0.885896246301623)(91,0.885896246301623)(91,0.885896246301623)(91,0.885896246301623)(91,0.885896246301623)(91,0.885896246301623)(92,0.88621120038927)(92,0.88621120038927)(92,0.88621120038927)(92,0.88621120038927)(92,0.88621120038927)(92,0.88621120038927)(92,0.88621120038927)(92,0.88621120038927)(92,0.88621120038927)(92,0.88621120038927)(92,0.88621120038927)(93,0.8866810990757)(93,0.8866810990757)(93,0.8866810990757)(93,0.8866810990757)(93,0.8866810990757)(93,0.8866810990757)(93,0.8866810990757)(93,0.8866810990757)(93,0.8866810990757)(93,0.8866810990757)(93,0.8866810990757)(94,0.887144966388669)(94,0.887144966388669)(94,0.887144966388669)(94,0.887144966388669)(94,0.887144966388669)(94,0.887144966388669)(94,0.887144966388669)(94,0.887144966388669)(94,0.887144966388669)(95,0.887618801414262)(95,0.887618801414262)(95,0.887618801414262)(95,0.887618801414262)(95,0.887618801414262)(95,0.887618801414262)(95,0.887618801414262)(95,0.887618801414262)(95,0.887618801414262)(95,0.887618801414262)(96,0.888085972647855)(96,0.888085972647855)(96,0.888085972647855)(96,0.888085972647855)(96,0.888085972647855)(96,0.888085972647855)(97,0.888555526147879)(97,0.888555526147879)(97,0.888555526147879)(97,0.888555526147879)(97,0.888555526147879)(97,0.888555526147879)(97,0.888555526147879)(97,0.888555526147879)(97,0.888555526147879)(97,0.888555526147879)(98,0.889203931927193)(98,0.889203931927193)(98,0.889203931927193)(98,0.889203931927193)(98,0.889203931927193)(98,0.889203931927193)(98,0.889203931927193)(99,0.889779126406695)(99,0.889779126406695)(99,0.889779126406695)(99,0.889779126406695)(99,0.889779126406695)(100,0.890410517944253)(100,0.890410517944253)(100,0.890410517944253)(100,0.890410517944253)(100,0.890410517944253)(100,0.890410517944253)(100,0.890410517944253)(100,0.890410517944253)(100,0.890410517944253)(100,0.890410517944253)(100,0.890410517944253)(100,0.890410517944253)(100,0.890410517944253)(100,0.890410517944253)(101,0.891098887988349)(101,0.891098887988349)(101,0.891098887988349)(101,0.891098887988349)(102,0.891861261990652)(102,0.891861261990652)(102,0.891861261990652)(102,0.891861261990652)(102,0.891861261990652)(103,0.89262418989616)(103,0.89262418989616)(103,0.89262418989616)(103,0.89262418989616)(103,0.89262418989616)(103,0.89262418989616)(103,0.89262418989616)(104,0.893483787450481)(104,0.893483787450481)(104,0.893483787450481)(104,0.893483787450481)(104,0.893483787450481)(104,0.893483787450481)(104,0.893483787450481)(104,0.893483787450481)(104,0.893483787450481)(105,0.894412274850794)(105,0.894412274850794)(105,0.894412274850794)(105,0.894412274850794)(105,0.894412274850794)(105,0.894412274850794)(105,0.894412274850794)(106,0.895370698323643)(106,0.895370698323643)(106,0.895370698323643)(107,0.896344780150986)(107,0.896344780150986)(107,0.896344780150986)(107,0.896344780150986)(107,0.896344780150986)(107,0.896344780150986)(107,0.896344780150986)(108,0.897336442355504)(108,0.897336442355504)(108,0.897336442355504)(108,0.897336442355504)(108,0.897336442355504)(109,0.89832614256793)(109,0.89832614256793)(109,0.89832614256793)(109,0.89832614256793)(109,0.89832614256793)(110,0.899215723414439)(110,0.899215723414439)(111,0.900091671229955)(111,0.900091671229955)(111,0.900091671229955)(111,0.900091671229955)(111,0.900091671229955)(111,0.900091671229955)(111,0.900091671229955)(111,0.900091671229955)(112,0.900895387171251)(112,0.900895387171251)(112,0.900895387171251)(112,0.900895387171251)(112,0.900895387171251)(112,0.900895387171251)(113,0.901621042139023)(113,0.901621042139023)(113,0.901621042139023)(113,0.901621042139023)(113,0.901621042139023)(113,0.901621042139023)(114,0.902265862459695)(114,0.902265862459695)(114,0.902265862459695)(114,0.902265862459695)(114,0.902265862459695)(114,0.902265862459695)(114,0.902265862459695)(114,0.902265862459695)(115,0.90282856860764)(115,0.90282856860764)(115,0.90282856860764)(115,0.90282856860764)(115,0.90282856860764)(115,0.90282856860764)(115,0.90282856860764)(115,0.90282856860764)(115,0.90282856860764)(116,0.903302394529468)(116,0.903302394529468)(116,0.903302394529468)(116,0.903302394529468)(116,0.903302394529468)(117,0.90366445589929)(117,0.90366445589929)(117,0.90366445589929)(117,0.90366445589929)(117,0.90366445589929)(117,0.90366445589929)(117,0.90366445589929)(118,0.904052072986014)(118,0.904052072986014)(118,0.904052072986014)(118,0.904052072986014)(118,0.904052072986014)(119,0.90442940929921)(119,0.90442940929921)(119,0.90442940929921)(119,0.90442940929921)(119,0.90442940929921)(119,0.90442940929921)(120,0.904820989336109)(120,0.904820989336109)(120,0.904820989336109)(120,0.904820989336109)(121,0.905290476626895)(121,0.905290476626895)(121,0.905290476626895)(121,0.905290476626895)(121,0.905290476626895)(122,0.905762975609968)(122,0.905762975609968)(122,0.905762975609968)(122,0.905762975609968)(122,0.905762975609968)(122,0.905762975609968)(123,0.906288210386602)(123,0.906288210386602)(123,0.906288210386602)(123,0.906288210386602)(123,0.906288210386602)(123,0.906288210386602)(124,0.906857921242519)(124,0.906857921242519)(124,0.906857921242519)(124,0.906857921242519)(124,0.906857921242519)(124,0.906857921242519)(124,0.906857921242519)(125,0.907458971897663)(125,0.907458971897663)(125,0.907458971897663)(125,0.907458971897663)(125,0.907458971897663)(125,0.907458971897663)(125,0.907458971897663)(125,0.907458971897663)(125,0.907458971897663)(126,0.908073697422993)(126,0.908073697422993)(126,0.908073697422993)(126,0.908073697422993)(126,0.908073697422993)(127,0.908682117119428)(127,0.908682117119428)(127,0.908682117119428)(127,0.908682117119428)(127,0.908682117119428)(127,0.908682117119428)(127,0.908682117119428)(127,0.908682117119428)(128,0.909266995551112)(128,0.909266995551112)(128,0.909266995551112)(128,0.909266995551112)(128,0.909266995551112)(128,0.909266995551112)(128,0.909266995551112)(129,0.909804698477688)(129,0.909804698477688)(129,0.909804698477688)(129,0.909804698477688)(129,0.909804698477688)(129,0.909804698477688)(130,0.910271424406664)(130,0.910271424406664)(130,0.910271424406664)(130,0.910271424406664)(130,0.910271424406664)(130,0.910271424406664)(130,0.910271424406664)(131,0.910666087030618)(131,0.910666087030618)(131,0.910666087030618)(132,0.910994912722979)(132,0.910994912722979)(132,0.910994912722979)(132,0.910994912722979)(132,0.910994912722979)(132,0.910994912722979)(132,0.910994912722979)(133,0.911265224144584)(133,0.911265224144584)(133,0.911265224144584)(133,0.911265224144584)(133,0.911265224144584)(133,0.911265224144584)(134,0.911479562692085)(134,0.911479562692085)(134,0.911479562692085)(134,0.911479562692085)(134,0.911479562692085)(135,0.911647070464452)(135,0.911647070464452)(135,0.911647070464452)(135,0.911647070464452)(135,0.911647070464452)(135,0.911647070464452)(135,0.911647070464452)(135,0.911647070464452)(136,0.911782068522749)(136,0.911782068522749)(136,0.911782068522749)(136,0.911782068522749)(136,0.911782068522749)(137,0.911902289743262)(137,0.911902289743262)(137,0.911902289743262)(137,0.911902289743262)(138,0.912025538806745)(138,0.912025538806745)(138,0.912025538806745)(138,0.912025538806745)(138,0.912025538806745)(138,0.912025538806745)(138,0.912025538806745)(138,0.912025538806745)(138,0.912025538806745)(139,0.912157017168855)(139,0.912157017168855)(139,0.912157017168855)(139,0.912157017168855)(139,0.912157017168855)(139,0.912157017168855)(140,0.912288966028396)(140,0.912288966028396)(140,0.912288966028396)(140,0.912288966028396)(141,0.912411892748327)(141,0.912411892748327)(141,0.912411892748327)(141,0.912411892748327)(141,0.912411892748327)(141,0.912411892748327)(141,0.912411892748327)(141,0.912411892748327)(142,0.912531494087763)(142,0.912531494087763)(142,0.912531494087763)(142,0.912531494087763)(142,0.912531494087763)(142,0.912531494087763)(143,0.912659214864361)(143,0.912659214864361)(143,0.912659214864361)(143,0.912659214864361)(143,0.912659214864361)(143,0.912659214864361)(143,0.912659214864361)(143,0.912659214864361)(143,0.912659214864361)(144,0.912797002438885)(144,0.912797002438885)(144,0.912797002438885)(144,0.912797002438885)(145,0.912970339199763)(145,0.912970339199763)(145,0.912970339199763)(145,0.912970339199763)(145,0.912970339199763)(145,0.912970339199763)(145,0.912970339199763)(145,0.912970339199763)(146,0.913049215410552)(146,0.913049215410552)(146,0.913049215410552)(146,0.913049215410552)(147,0.913085221356563)(147,0.913085221356563)(147,0.913085221356563)(147,0.913085221356563)(147,0.913085221356563)(148,0.913100331670334)(148,0.913100331670334)(148,0.913100331670334)(148,0.913100331670334)(148,0.913100331670334)(148,0.913100331670334)(148,0.913100331670334)(148,0.913100331670334)(149,0.91312152334848)(149,0.91312152334848)(149,0.91312152334848)(149,0.91312152334848)(149,0.91312152334848)(149,0.91312152334848)(150,0.913170983171252)(150,0.913170983171252)(150,0.913170983171252)(150,0.913170983171252)(150,0.913170983171252)(150,0.913170983171252)(150,0.913170983171252)(150,0.913170983171252)(150,0.913170983171252)(151,0.913453143866604)(151,0.913453143866604)(151,0.913453143866604)(152,0.913646758974046)(152,0.913646758974046)(152,0.913646758974046)(152,0.913646758974046)(152,0.913646758974046)(152,0.913646758974046)(152,0.913646758974046)(152,0.913646758974046)(152,0.913646758974046)(152,0.913646758974046)(153,0.913917254304116)(153,0.913917254304116)(153,0.913917254304116)(153,0.913917254304116)(153,0.913917254304116)(154,0.914265043301814)(154,0.914265043301814)(154,0.914265043301814)(154,0.914265043301814)(154,0.914265043301814)(154,0.914265043301814)(154,0.914265043301814)(154,0.914265043301814)(155,0.914675968349094)(155,0.914675968349094)(155,0.914675968349094)(155,0.914675968349094)(155,0.914675968349094)(155,0.914675968349094)(155,0.914675968349094)(156,0.915138446186076)(156,0.915138446186076)(156,0.915138446186076)(156,0.915138446186076)(156,0.915138446186076)(156,0.915138446186076)(157,0.915640033422324)(157,0.915640033422324)(157,0.915640033422324)(157,0.915640033422324)(157,0.915640033422324)(157,0.915640033422324)(158,0.916175342762926)(158,0.916175342762926)(158,0.916175342762926)(158,0.916175342762926)(158,0.916175342762926)(158,0.916175342762926)(158,0.916175342762926)(159,0.916696768984411)(159,0.916696768984411)(159,0.916696768984411)(159,0.916696768984411)(159,0.916696768984411)(159,0.916696768984411)(159,0.916696768984411)(159,0.916696768984411)(160,0.917326693682256)(160,0.917326693682256)(160,0.917326693682256)(160,0.917326693682256)(160,0.917326693682256)(161,0.917867490864265)(161,0.917867490864265)(161,0.917867490864265)(161,0.917867490864265)(161,0.917867490864265)(161,0.917867490864265)(161,0.917867490864265)(161,0.917867490864265)(161,0.917867490864265)(162,0.918341190066385)(162,0.918341190066385)(162,0.918341190066385)(162,0.918341190066385)(162,0.918341190066385)(162,0.918341190066385)(162,0.918341190066385)(162,0.918341190066385)(163,0.918717931029981)(163,0.918717931029981)(163,0.918717931029981)(164,0.918974578453551)(164,0.918974578453551)(164,0.918974578453551)(164,0.918974578453551)(164,0.918974578453551)(164,0.918974578453551)(165,0.918892313269438)(165,0.918892313269438)(165,0.918892313269438)(165,0.918892313269438)(165,0.918892313269438)(166,0.919004286565191)(166,0.919004286565191)(166,0.919004286565191)(166,0.919004286565191)(166,0.919004286565191)(166,0.919004286565191)(166,0.919004286565191)(167,0.919208466633211)(167,0.919208466633211)(167,0.919208466633211)(167,0.919208466633211)(167,0.919208466633211)(167,0.919208466633211)(167,0.919208466633211)(167,0.919208466633211)(168,0.919209525369934)(168,0.919209525369934)(168,0.919209525369934)(168,0.919209525369934)(168,0.919209525369934)(169,0.919227589419664)(169,0.919227589419664)(169,0.919227589419664)(169,0.919227589419664)(169,0.919227589419664)(169,0.919227589419664)(169,0.919227589419664)(169,0.919227589419664)(169,0.919227589419664)(170,0.919228938879919)(170,0.919228938879919)(170,0.919228938879919)(170,0.919228938879919)(170,0.919228938879919)(170,0.919228938879919)(170,0.919228938879919)(170,0.919228938879919)(170,0.919228938879919)(171,0.919302490922463)(171,0.919302490922463)(171,0.919302490922463)(171,0.919302490922463)(171,0.919302490922463)(171,0.919302490922463)(171,0.919302490922463)(172,0.919447206153786)(172,0.919447206153786)(172,0.919447206153786)(172,0.919447206153786)(172,0.919447206153786)(172,0.919447206153786)(172,0.919447206153786)(172,0.919447206153786)(172,0.919447206153786)(173,0.919678970076311)(173,0.919678970076311)(173,0.919678970076311)(173,0.919678970076311)(173,0.919678970076311)(174,0.920003216764254)(174,0.920003216764254)(174,0.920003216764254)(175,0.920147093836222)(175,0.920147093836222)(175,0.920147093836222)(175,0.920147093836222)(175,0.920147093836222)(176,0.920589996613363)(176,0.920589996613363)(176,0.920589996613363)(176,0.920589996613363)(176,0.920589996613363)(176,0.920589996613363)(176,0.920589996613363)(177,0.92106950351799)(177,0.92106950351799)(177,0.92106950351799)(178,0.921693998148347)(178,0.921693998148347)(178,0.921693998148347)(178,0.921693998148347)(178,0.921693998148347)(178,0.921693998148347)(178,0.921693998148347)(179,0.922058867817332)(179,0.922058867817332)(179,0.922058867817332)(180,0.922369993301596)(180,0.922369993301596)(180,0.922369993301596)(180,0.922369993301596)(180,0.922369993301596)(180,0.922369993301596)(181,0.9226598000701)(181,0.9226598000701)(181,0.9226598000701)(181,0.9226598000701)(181,0.9226598000701)(181,0.9226598000701)(181,0.9226598000701)(182,0.922879363294031)(182,0.922879363294031)(182,0.922879363294031)(182,0.922879363294031)(183,0.923010624099443)(183,0.923010624099443)(183,0.923010624099443)(183,0.923010624099443)(183,0.923010624099443)(183,0.923010624099443)(183,0.923010624099443)(183,0.923010624099443)(183,0.923010624099443)(183,0.923010624099443)(183,0.923010624099443)(184,0.923081888901675)(184,0.923081888901675)(184,0.923081888901675)(184,0.923081888901675)(184,0.923081888901675)(184,0.923081888901675)(184,0.923081888901675)(184,0.923081888901675)(185,0.923142748599112)(185,0.923142748599112)(185,0.923142748599112)(185,0.923142748599112)(185,0.923142748599112)(185,0.923142748599112)(186,0.923233115335294)(186,0.923233115335294)(186,0.923233115335294)(186,0.923233115335294)(186,0.923233115335294)(186,0.923233115335294)(186,0.923233115335294)(187,0.923366134857737)(187,0.923366134857737)(187,0.923366134857737)(187,0.923366134857737)(187,0.923366134857737)(188,0.923539277155194)(188,0.923539277155194)(188,0.923539277155194)(188,0.923539277155194)(188,0.923539277155194)(188,0.923539277155194)(188,0.923539277155194)(188,0.923539277155194)(189,0.92387069681485)(189,0.92387069681485)(189,0.92387069681485)(189,0.92387069681485)(189,0.92387069681485)(189,0.92387069681485)(189,0.92387069681485)(189,0.92387069681485)(189,0.92387069681485)(190,0.924032343083996)(190,0.924032343083996)(190,0.924032343083996)(190,0.924032343083996)(191,0.924164300915139)(191,0.924164300915139)(191,0.924164300915139)(191,0.924164300915139)(192,0.924230284390243)(192,0.924230284390243)(192,0.924230284390243)(192,0.924230284390243)(192,0.924230284390243)(192,0.924230284390243)(193,0.924206051309703)(193,0.924206051309703)(193,0.924206051309703)(193,0.924206051309703)(193,0.924206051309703)(193,0.924206051309703)(193,0.924206051309703)(193,0.924206051309703)(193,0.924206051309703)(193,0.924206051309703)(194,0.92408993528439)(194,0.92408993528439)(194,0.92408993528439)(195,0.923921577475897)(195,0.923921577475897)(195,0.923921577475897)(195,0.923921577475897)(195,0.923921577475897)(196,0.923686836089653)(196,0.923686836089653)(196,0.923686836089653)(196,0.923686836089653)(196,0.923686836089653)(196,0.923686836089653)(196,0.923686836089653)(196,0.923686836089653)(196,0.923686836089653)(196,0.923686836089653)(197,0.92340494913761)(197,0.92340494913761)(197,0.92340494913761)(197,0.92340494913761)(197,0.92340494913761)(197,0.92340494913761)(197,0.92340494913761)(198,0.923096399506605)(198,0.923096399506605)(198,0.923096399506605)(198,0.923096399506605)(198,0.923096399506605)(198,0.923096399506605)(198,0.923096399506605)(198,0.923096399506605)(198,0.923096399506605)(198,0.923096399506605)(198,0.923096399506605)(198,0.923096399506605)(199,0.922788035664489)(199,0.922788035664489)(199,0.922788035664489)(199,0.922788035664489)(199,0.922788035664489)(199,0.922788035664489)(199,0.922788035664489)(199,0.922788035664489)(199,0.922788035664489)(200,0.922501962114206)(200,0.922501962114206)(200,0.922501962114206)(200,0.922501962114206)(200,0.922501962114206)(200,0.922501962114206)(200,0.922501962114206)(201,0.922252395594164)(201,0.922252395594164)(201,0.922252395594164)(201,0.922252395594164)(201,0.922252395594164)(201,0.922252395594164)(201,0.922252395594164)(201,0.922252395594164)(201,0.922252395594164)(201,0.922252395594164)(202,0.921987535368166)(202,0.921987535368166)(202,0.921987535368166)(202,0.921987535368166)(202,0.921987535368166)(203,0.921866979232156)(203,0.921866979232156)(203,0.921866979232156)(203,0.921866979232156)(203,0.921866979232156)(203,0.921866979232156)(204,0.921862204062474)(204,0.921862204062474)(204,0.921862204062474)(204,0.921862204062474)(204,0.921862204062474)(205,0.922018885850143)(205,0.922018885850143)(205,0.922018885850143)(205,0.922018885850143)(205,0.922018885850143)(205,0.922018885850143)(205,0.922018885850143)(206,0.922404613114075)(206,0.922404613114075)(206,0.922404613114075)(206,0.922404613114075)(206,0.922404613114075)(206,0.922404613114075)(206,0.922404613114075)(206,0.922404613114075)(206,0.922404613114075)(206,0.922404613114075)(207,0.92287352706863)(207,0.92287352706863)(207,0.92287352706863)(207,0.92287352706863)(207,0.92287352706863)(207,0.92287352706863)(207,0.92287352706863)(207,0.92287352706863)(208,0.923395728059796)(208,0.923395728059796)(208,0.923395728059796)(208,0.923395728059796)(208,0.923395728059796)(208,0.923395728059796)(209,0.923946230294363)(209,0.923946230294363)(209,0.923946230294363)(209,0.923946230294363)(209,0.923946230294363)(210,0.924521795051165)(210,0.924521795051165)(210,0.924521795051165)(210,0.924521795051165)(210,0.924521795051165)(210,0.924521795051165)(210,0.924521795051165)(210,0.924521795051165)(210,0.924521795051165)(211,0.92507066762613)(211,0.92507066762613)(211,0.92507066762613)(211,0.92507066762613)(211,0.92507066762613)(211,0.92507066762613)(211,0.92507066762613)(211,0.92507066762613)(211,0.92507066762613)(212,0.925593336605208)(212,0.925593336605208)(212,0.925593336605208)(212,0.925593336605208)(212,0.925593336605208)(212,0.925593336605208)(212,0.925593336605208)(212,0.925593336605208)(212,0.925593336605208)(212,0.925593336605208)(212,0.925593336605208)(212,0.925593336605208)(213,0.92609027066118)(213,0.92609027066118)(213,0.92609027066118)(213,0.92609027066118)(213,0.92609027066118)(213,0.92609027066118)(213,0.92609027066118)(213,0.92609027066118)(214,0.926557249575485)(214,0.926557249575485)(214,0.926557249575485)(214,0.926557249575485)(214,0.926557249575485)(215,0.926909574017588)(215,0.926909574017588)(215,0.926909574017588)(215,0.926909574017588)(215,0.926909574017588)(215,0.926909574017588)(215,0.926909574017588)(216,0.9273384048181)(216,0.9273384048181)(216,0.9273384048181)(216,0.9273384048181)(216,0.9273384048181)(216,0.9273384048181)(216,0.9273384048181)(216,0.9273384048181)(216,0.9273384048181)(216,0.9273384048181)(216,0.9273384048181)(216,0.9273384048181)(217,0.927795265164317)(217,0.927795265164317)(217,0.927795265164317)(217,0.927795265164317)(217,0.927795265164317)(217,0.927795265164317)(218,0.928300989943222)(218,0.928300989943222)(218,0.928300989943222)(218,0.928300989943222)(218,0.928300989943222)(218,0.928300989943222)(219,0.928858880520682)(219,0.928858880520682)(219,0.928858880520682)(220,0.929448392740186)(220,0.929448392740186)(220,0.929448392740186)(220,0.929448392740186)(220,0.929448392740186)(220,0.929448392740186)(220,0.929448392740186)(221,0.930009018963276)(221,0.930009018963276)(221,0.930009018963276)(222,0.930486980244315)(222,0.930486980244315)(222,0.930486980244315)(223,0.930590901576675)(223,0.930590901576675)(223,0.930590901576675)(223,0.930590901576675)(223,0.930590901576675)(223,0.930590901576675)(223,0.930590901576675)(223,0.930590901576675)(223,0.930590901576675)(223,0.930590901576675)(224,0.93114736483215)(224,0.93114736483215)(224,0.93114736483215)(224,0.93114736483215)(225,0.931330264124905)(225,0.931330264124905)(225,0.931330264124905)(225,0.931330264124905)(225,0.931330264124905)(225,0.931330264124905)(225,0.931330264124905)(225,0.931330264124905)(225,0.931330264124905)(225,0.931330264124905)(226,0.931425315510514)(226,0.931425315510514)(226,0.931425315510514)(226,0.931425315510514)(226,0.931425315510514)(226,0.931425315510514)(226,0.931425315510514)(226,0.931425315510514)(226,0.931425315510514)(227,0.931485377316195)(227,0.931485377316195)(227,0.931485377316195)(227,0.931485377316195)(227,0.931485377316195)(227,0.931485377316195)(227,0.931485377316195)(227,0.931485377316195)(227,0.931485377316195)(227,0.931485377316195)(227,0.931485377316195)(228,0.93135741700128)(228,0.93135741700128)(228,0.93135741700128)(228,0.93135741700128)(228,0.93135741700128)(228,0.93135741700128)(228,0.93135741700128)(228,0.93135741700128)(229,0.931229067391955)(229,0.931229067391955)(229,0.931229067391955)(229,0.931229067391955)(229,0.931229067391955)(229,0.931229067391955)(229,0.931229067391955)(229,0.931229067391955)(229,0.931229067391955)(229,0.931229067391955)(230,0.931123285836686)(230,0.931123285836686)(230,0.931123285836686)(230,0.931123285836686)(230,0.931123285836686)(230,0.931123285836686)(230,0.931123285836686)(230,0.931123285836686)(231,0.931040292263274)(231,0.931040292263274)(231,0.931040292263274)(231,0.931040292263274)(231,0.931040292263274)(231,0.931040292263274)(232,0.930990176843429)(232,0.930990176843429)(232,0.930990176843429)(232,0.930990176843429)(233,0.930969949307365)(233,0.930969949307365)(233,0.930969949307365)(233,0.930969949307365)(233,0.930969949307365)(233,0.930969949307365)(233,0.930969949307365)(234,0.93095587475416)(234,0.93095587475416)(234,0.93095587475416)(234,0.93095587475416)(234,0.93095587475416)(235,0.930932180950164)(235,0.930932180950164)(235,0.930932180950164)(235,0.930932180950164)(235,0.930932180950164)(235,0.930932180950164)(235,0.930932180950164)(236,0.930864021882447)(236,0.930864021882447)(236,0.930864021882447)(236,0.930864021882447)(236,0.930864021882447)(236,0.930864021882447)(237,0.930785214149691)(237,0.930785214149691)(237,0.930785214149691)(237,0.930785214149691)(237,0.930785214149691)(237,0.930785214149691)(237,0.930785214149691)(237,0.930785214149691)(237,0.930785214149691)(238,0.930545805270038)(238,0.930545805270038)(238,0.930545805270038)(238,0.930545805270038)(238,0.930545805270038)(238,0.930545805270038)(238,0.930545805270038)(238,0.930545805270038)(239,0.930439951570301)(239,0.930439951570301)(239,0.930439951570301)(239,0.930439951570301)(239,0.930439951570301)(240,0.930370388221364)(240,0.930370388221364)(240,0.930370388221364)(240,0.930370388221364)(240,0.930370388221364)(240,0.930370388221364)(240,0.930370388221364)(240,0.930370388221364)(240,0.930370388221364)(241,0.930344219139873)(241,0.930344219139873)(241,0.930344219139873)(241,0.930344219139873)(241,0.930344219139873)(241,0.930344219139873)(241,0.930344219139873)(241,0.930344219139873)(241,0.930344219139873)(241,0.930344219139873)(241,0.930344219139873)(241,0.930344219139873)(242,0.930359789430444)(242,0.930359789430444)(242,0.930359789430444)(242,0.930359789430444)(242,0.930359789430444)(242,0.930359789430444)(242,0.930359789430444)(242,0.930359789430444)(243,0.930414236149959)(243,0.930414236149959)(243,0.930414236149959)(243,0.930414236149959)(243,0.930414236149959)(243,0.930414236149959)(243,0.930414236149959)(243,0.930414236149959)(243,0.930414236149959)(243,0.930414236149959)(244,0.930556630074303)(244,0.930556630074303)(244,0.930556630074303)(244,0.930556630074303)(244,0.930556630074303)(244,0.930556630074303)(244,0.930556630074303)(244,0.930556630074303)(245,0.930647427813468)(245,0.930647427813468)(245,0.930647427813468)(245,0.930647427813468)(246,0.930785229644968)(246,0.930785229644968)(247,0.930972597381396)(247,0.930972597381396)(247,0.930972597381396)(247,0.930972597381396)(247,0.930972597381396)(247,0.930972597381396)(247,0.930972597381396)(247,0.930972597381396)(247,0.930972597381396)(247,0.930972597381396)(247,0.930972597381396)(247,0.930972597381396)(247,0.930972597381396)(247,0.930972597381396)(247,0.930972597381396)(248,0.931226570515829)(248,0.931226570515829)(248,0.931226570515829)(248,0.931226570515829)(248,0.931226570515829)(248,0.931226570515829)(248,0.931226570515829)(248,0.931226570515829)(248,0.931226570515829)(248,0.931226570515829)(248,0.931226570515829)(249,0.931564331868745)(249,0.931564331868745)(249,0.931564331868745)(249,0.931564331868745)(249,0.931564331868745)(249,0.931564331868745)(249,0.931564331868745)(249,0.931564331868745)(249,0.931564331868745)(249,0.931564331868745)(249,0.931564331868745)(250,0.931971380568781)(250,0.931971380568781)(250,0.931971380568781)(250,0.931971380568781)(250,0.931971380568781)(250,0.931971380568781)(250,0.931971380568781)(250,0.931971380568781)(251,0.932419799014061)(251,0.932419799014061)(251,0.932419799014061)(251,0.932419799014061)(251,0.932419799014061)(251,0.932419799014061)(251,0.932419799014061)(252,0.932887537332004)(252,0.932887537332004)(252,0.932887537332004)(252,0.932887537332004)(252,0.932887537332004)(252,0.932887537332004)(252,0.932887537332004)(252,0.932887537332004)(252,0.932887537332004)(253,0.933346342165302)(253,0.933346342165302)(253,0.933346342165302)(253,0.933346342165302)(254,0.933785813449489)(254,0.933785813449489)(254,0.933785813449489)(254,0.933785813449489)(254,0.933785813449489)(254,0.933785813449489)(254,0.933785813449489)(254,0.933785813449489)(254,0.933785813449489)(255,0.934212384311136)(255,0.934212384311136)(255,0.934212384311136)(255,0.934212384311136)(255,0.934212384311136)(255,0.934212384311136)(256,0.934631342762802)(256,0.934631342762802)(256,0.934631342762802)(256,0.934631342762802)(256,0.934631342762802)(256,0.934631342762802)(256,0.934631342762802)(256,0.934631342762802)(256,0.934631342762802)(257,0.935052139100331)(257,0.935052139100331)(257,0.935052139100331)(257,0.935052139100331)(257,0.935052139100331)(257,0.935052139100331)(257,0.935052139100331)(257,0.935052139100331)(257,0.935052139100331)(257,0.935052139100331)(257,0.935052139100331)(257,0.935052139100331)(257,0.935052139100331)(257,0.935052139100331)(258,0.935508809929941)(258,0.935508809929941)(258,0.935508809929941)(259,0.936016701477708)(259,0.936016701477708)(259,0.936016701477708)(259,0.936016701477708)(259,0.936016701477708)(259,0.936016701477708)(259,0.936016701477708)(260,0.93633319614183)(260,0.93633319614183)(260,0.93633319614183)(260,0.93633319614183)(261,0.937102954477552)(261,0.937102954477552)(261,0.937102954477552)(261,0.937102954477552)(262,0.937606462772912)(262,0.937606462772912)(262,0.937606462772912)(262,0.937606462772912)(262,0.937606462772912)(262,0.937606462772912)(262,0.937606462772912)(263,0.937912104505906)(263,0.937912104505906)(263,0.937912104505906)(263,0.937912104505906)(263,0.937912104505906)(263,0.937912104505906)(263,0.937912104505906)(263,0.937912104505906)(263,0.937912104505906)(263,0.937912104505906)(263,0.937912104505906)(263,0.937912104505906)(264,0.938342346329133)(264,0.938342346329133)(264,0.938342346329133)(265,0.938622825548547)(265,0.938622825548547)(265,0.938622825548547)(265,0.938622825548547)(265,0.938622825548547)(265,0.938622825548547)(265,0.938622825548547)(266,0.93891634941725)(266,0.93891634941725)(266,0.93891634941725)(266,0.93891634941725)(266,0.93891634941725)(266,0.93891634941725)(266,0.93891634941725)(266,0.93891634941725)(266,0.93891634941725)(267,0.939141462624407)(267,0.939141462624407)(267,0.939141462624407)(267,0.939141462624407)(267,0.939141462624407)(268,0.939370326124655)(268,0.939370326124655)(268,0.939370326124655)(268,0.939370326124655)(268,0.939370326124655)(268,0.939370326124655)(269,0.939452165348286)(269,0.939452165348286)(269,0.939452165348286)(269,0.939452165348286)(269,0.939452165348286)(269,0.939452165348286)(269,0.939452165348286)(269,0.939452165348286)(269,0.939452165348286)(269,0.939452165348286)(269,0.939452165348286)(270,0.939579828203442)(270,0.939579828203442)(270,0.939579828203442)(270,0.939579828203442)(270,0.939579828203442)(270,0.939579828203442)(270,0.939579828203442)(270,0.939579828203442)(271,0.939735728780331)(271,0.939735728780331)(271,0.939735728780331)(271,0.939735728780331)(271,0.939735728780331)(271,0.939735728780331)(271,0.939735728780331)(271,0.939735728780331)(271,0.939735728780331)(271,0.939735728780331)(271,0.939735728780331)(271,0.939735728780331)(272,0.940097573071384)(272,0.940097573071384)(272,0.940097573071384)(272,0.940097573071384)(272,0.940097573071384)(272,0.940097573071384)(272,0.940097573071384)(272,0.940097573071384)(273,0.940360968649988)(273,0.940360968649988)(273,0.940360968649988)(273,0.940360968649988)(273,0.940360968649988)(273,0.940360968649988)(273,0.940360968649988)(274,0.940679299029566)(274,0.940679299029566)(274,0.940679299029566)(274,0.940679299029566)(274,0.940679299029566)(274,0.940679299029566)(274,0.940679299029566)(275,0.941044841474603)(275,0.941044841474603)(275,0.941044841474603)(275,0.941044841474603)(275,0.941044841474603)(275,0.941044841474603)(275,0.941044841474603)(275,0.941044841474603)(276,0.941449229898842)(276,0.941449229898842)(276,0.941449229898842)(276,0.941449229898842)(276,0.941449229898842)(276,0.941449229898842)(276,0.941449229898842)(276,0.941449229898842)(276,0.941449229898842)(277,0.941873046097793)(277,0.941873046097793)(277,0.941873046097793)(277,0.941873046097793)(277,0.941873046097793)(277,0.941873046097793)(277,0.941873046097793)(278,0.942285337021387)(278,0.942285337021387)(279,0.942664781164883)(279,0.942664781164883)(279,0.942664781164883)(279,0.942664781164883)(279,0.942664781164883)(279,0.942664781164883)(280,0.943000623995104)(280,0.943000623995104)(280,0.943000623995104)(280,0.943000623995104)(280,0.943000623995104)(280,0.943000623995104)(281,0.943291953907347)(281,0.943291953907347)(281,0.943291953907347)(281,0.943291953907347)(281,0.943291953907347)(281,0.943291953907347)(282,0.943552364856462)(282,0.943552364856462)(282,0.943552364856462)(282,0.943552364856462)(282,0.943552364856462)(283,0.943784962514867)(283,0.943784962514867)(283,0.943784962514867)(283,0.943784962514867)(283,0.943784962514867)(283,0.943784962514867)(283,0.943784962514867)(283,0.943784962514867)(283,0.943784962514867)(283,0.943784962514867)(284,0.943984390365138)(284,0.943984390365138)(284,0.943984390365138)(284,0.943984390365138)(284,0.943984390365138)(284,0.943984390365138)(285,0.944144980860743)(285,0.944144980860743)(285,0.944144980860743)(285,0.944144980860743)(285,0.944144980860743)(286,0.944286228332822)(286,0.944286228332822)(286,0.944286228332822)(287,0.944394054141991)(287,0.944394054141991)(287,0.944394054141991)(287,0.944394054141991)(287,0.944394054141991)(287,0.944394054141991)(287,0.944394054141991)(288,0.94452553944118)(288,0.94452553944118)(288,0.94452553944118)(288,0.94452553944118)(289,0.944692690010736)(289,0.944692690010736)(289,0.944692690010736)(290,0.944917019878289)(290,0.944917019878289)(290,0.944917019878289)(290,0.944917019878289)(291,0.945207576196414)(291,0.945207576196414)(291,0.945207576196414)(291,0.945207576196414)(291,0.945207576196414)(291,0.945207576196414)(291,0.945207576196414)(291,0.945207576196414)(292,0.94557134831913)(292,0.94557134831913)(292,0.94557134831913)(293,0.94589533748509)(293,0.94589533748509)(293,0.94589533748509)(293,0.94589533748509)(293,0.94589533748509)(293,0.94589533748509)(293,0.94589533748509)(293,0.94589533748509)(293,0.94589533748509)(293,0.94589533748509)(293,0.94589533748509)(294,0.946519065796333)(294,0.946519065796333)(294,0.946519065796333)(295,0.94706717789734)(295,0.94706717789734)(295,0.94706717789734)(295,0.94706717789734)(295,0.94706717789734)(296,0.947644294913962)(296,0.947644294913962)(296,0.947644294913962)(296,0.947644294913962)(296,0.947644294913962)(296,0.947644294913962)(296,0.947644294913962)(296,0.947644294913962)(297,0.948322053791683)(297,0.948322053791683)(297,0.948322053791683)(298,0.948937412765218)(298,0.948937412765218)(298,0.948937412765218)(298,0.948937412765218)(298,0.948937412765218)(298,0.948937412765218)(298,0.948937412765218)(298,0.948937412765218)(299,0.949572832647687)(299,0.949572832647687)(299,0.949572832647687)(299,0.949572832647687)(299,0.949572832647687)(299,0.949572832647687)(299,0.949572832647687)(300,0.950228249802406)(300,0.950228249802406)(300,0.950228249802406)(300,0.950228249802406)(300,0.950228249802406)(300,0.950228249802406)(300,0.950228249802406)(300,0.950228249802406)(300,0.950228249802406)(300,0.950228249802406)(300,0.950228249802406)(301,0.95090035956721)(301,0.95090035956721)(301,0.95090035956721)(301,0.95090035956721)(301,0.95090035956721)(301,0.95090035956721)(301,0.95090035956721)(301,0.95090035956721)(302,0.951583610289534)(302,0.951583610289534)(302,0.951583610289534)(302,0.951583610289534)(302,0.951583610289534)(302,0.951583610289534)(302,0.951583610289534)(302,0.951583610289534)(302,0.951583610289534)(302,0.951583610289534)(303,0.952278762226028)(303,0.952278762226028)(303,0.952278762226028)(303,0.952278762226028)(303,0.952278762226028)(303,0.952278762226028)(303,0.952278762226028)(303,0.952278762226028)(303,0.952278762226028)(304,0.952978886687073)(304,0.952978886687073)(304,0.952978886687073)(304,0.952978886687073)(304,0.952978886687073)(305,0.953669302637451)(305,0.953669302637451)(305,0.953669302637451)(305,0.953669302637451)(305,0.953669302637451)(305,0.953669302637451)(305,0.953669302637451)(305,0.953669302637451)(305,0.953669302637451)(305,0.953669302637451)(305,0.953669302637451)(306,0.954336361550128)(306,0.954336361550128)(306,0.954336361550128)(306,0.954336361550128)(306,0.954336361550128)(306,0.954336361550128)(306,0.954336361550128)(307,0.954961117376979)(307,0.954961117376979)(307,0.954961117376979)(307,0.954961117376979)(307,0.954961117376979)(307,0.954961117376979)(307,0.954961117376979)(307,0.954961117376979)(307,0.954961117376979)(307,0.954961117376979)(308,0.955524529771646)(308,0.955524529771646)(308,0.955524529771646)(308,0.955524529771646)(308,0.955524529771646)(308,0.955524529771646)(308,0.955524529771646)(309,0.956009312602022)(309,0.956009312602022)(309,0.956009312602022)(309,0.956009312602022)(309,0.956009312602022)(310,0.956406909840839)(310,0.956406909840839)(310,0.956406909840839)(310,0.956406909840839)(310,0.956406909840839)(310,0.956406909840839)(310,0.956406909840839)(310,0.956406909840839)(310,0.956406909840839)(311,0.956712479644112)(311,0.956712479644112)(311,0.956712479644112)(312,0.956933905690378)(312,0.956933905690378)(312,0.956933905690378)(312,0.956933905690378)(312,0.956933905690378)(312,0.956933905690378)(312,0.956933905690378)(312,0.956933905690378)(313,0.957087205412894)(313,0.957087205412894)(313,0.957087205412894)(313,0.957087205412894)(313,0.957087205412894)(314,0.957153415696524)(314,0.957153415696524)(314,0.957153415696524)(314,0.957153415696524)(314,0.957153415696524)(315,0.957257115405326)(315,0.957257115405326)(315,0.957257115405326)(315,0.957257115405326)(315,0.957257115405326)(315,0.957257115405326)(315,0.957257115405326)(315,0.957257115405326)(315,0.957257115405326)(315,0.957257115405326)(315,0.957257115405326)(316,0.957299632416401)(316,0.957299632416401)(316,0.957299632416401)(316,0.957299632416401)(316,0.957299632416401)(316,0.957299632416401)(317,0.957332779610485)(317,0.957332779610485)(317,0.957332779610485)(317,0.957332779610485)(317,0.957332779610485)(317,0.957332779610485)(317,0.957332779610485)(317,0.957332779610485)(317,0.957332779610485)(317,0.957332779610485)(318,0.957369601128973)(318,0.957369601128973)(318,0.957369601128973)(318,0.957369601128973)(319,0.957419501599356)(319,0.957419501599356)(319,0.957419501599356)(319,0.957419501599356)(319,0.957419501599356)(319,0.957419501599356)(319,0.957419501599356)(320,0.957554578418521)(320,0.957554578418521)(320,0.957554578418521)(321,0.95757200454874)(321,0.95757200454874)(321,0.95757200454874)(321,0.95757200454874)(321,0.95757200454874)(321,0.95757200454874)(321,0.95757200454874)(322,0.957744224569694)(322,0.957744224569694)(322,0.957744224569694)(322,0.957744224569694)(322,0.957744224569694)(322,0.957744224569694)(322,0.957744224569694)(323,0.957865192539705)(323,0.957865192539705)(323,0.957865192539705)(323,0.957865192539705)(324,0.957997338079995)(324,0.957997338079995)(324,0.957997338079995)(324,0.957997338079995)(324,0.957997338079995)(325,0.958144410462858)(325,0.958144410462858)(325,0.958144410462858)(325,0.958144410462858)(326,0.958309372346608)(326,0.958309372346608)(326,0.958309372346608)(326,0.958309372346608)(326,0.958309372346608)(326,0.958309372346608)(326,0.958309372346608)(326,0.958309372346608)(327,0.958565900893205)(327,0.958565900893205)(327,0.958565900893205)(327,0.958565900893205)(327,0.958565900893205)(327,0.958565900893205)(327,0.958565900893205)(327,0.958565900893205)(328,0.958757008128417)(328,0.958757008128417)(328,0.958757008128417)(328,0.958757008128417)(328,0.958757008128417)(328,0.958757008128417)(328,0.958757008128417)(328,0.958757008128417)(328,0.958757008128417)(328,0.958757008128417)(329,0.958965442289426)(329,0.958965442289426)(329,0.958965442289426)(329,0.958965442289426)(330,0.959192098622117)(330,0.959192098622117)(330,0.959192098622117)(330,0.959192098622117)(330,0.959192098622117)(330,0.959192098622117)(330,0.959192098622117)(330,0.959192098622117)(330,0.959192098622117)(330,0.959192098622117)(330,0.959192098622117)(331,0.959461168890595)(331,0.959461168890595)(331,0.959461168890595)(331,0.959461168890595)(331,0.959461168890595)(331,0.959461168890595)(332,0.959712129129329)(332,0.959712129129329)(332,0.959712129129329)(332,0.959712129129329)(333,0.959975800674616)(333,0.959975800674616)(333,0.959975800674616)(333,0.959975800674616)(333,0.959975800674616)(333,0.959975800674616)(334,0.960239914606844)(334,0.960239914606844)(334,0.960239914606844)(334,0.960239914606844)(335,0.960511767772069)(335,0.960511767772069)(335,0.960511767772069)(335,0.960511767772069)(335,0.960511767772069)(335,0.960511767772069)(335,0.960511767772069)(335,0.960511767772069)(336,0.960787048933944)(336,0.960787048933944)(336,0.960787048933944)(336,0.960787048933944)(337,0.961016494061257)(337,0.961016494061257)(337,0.961016494061257)(337,0.961016494061257)(337,0.961016494061257)(338,0.961277467829354)(338,0.961277467829354)(338,0.961277467829354)(338,0.961277467829354)(339,0.961460023501389)(339,0.961460023501389)(339,0.961460023501389)(339,0.961460023501389)(339,0.961460023501389)(340,0.961682983258959)(340,0.961682983258959)(340,0.961682983258959)(341,0.9618102486193)(341,0.9618102486193)(341,0.9618102486193)(341,0.9618102486193)(341,0.9618102486193)(341,0.9618102486193)(341,0.9618102486193)(341,0.9618102486193)(342,0.961996971644293)(342,0.961996971644293)(342,0.961996971644293)(343,0.962094787604586)(343,0.962094787604586)(343,0.962094787604586)(343,0.962094787604586)(344,0.962252116891891)(344,0.962252116891891)(344,0.962252116891891)(344,0.962252116891891)(344,0.962252116891891)(344,0.962252116891891)(344,0.962252116891891)(344,0.962252116891891)(345,0.962333738978964)(345,0.962333738978964)(345,0.962333738978964)(346,0.962415743467908)(346,0.962415743467908)(346,0.962415743467908)(346,0.962415743467908)(347,0.962533489062679)(347,0.962533489062679)(347,0.962533489062679)(347,0.962533489062679)(347,0.962533489062679)(347,0.962533489062679)(347,0.962533489062679)(347,0.962533489062679)(347,0.962533489062679)(347,0.962533489062679)(348,0.962628978097089)(349,0.962678182208823)(349,0.962678182208823)(349,0.962678182208823)(349,0.962678182208823)(349,0.962678182208823)(350,0.962713468270638)(351,0.962756014630323)(351,0.962756014630323)(351,0.962756014630323)(351,0.962756014630323)(351,0.962756014630323)(352,0.962771307621729)(352,0.962771307621729)(352,0.962771307621729)(352,0.962771307621729)(352,0.962771307621729)(353,0.962781379915339)(353,0.962781379915339)(353,0.962781379915339)(353,0.962781379915339)(353,0.962781379915339)(353,0.962781379915339)(354,0.962791240004302)(354,0.962791240004302)(354,0.962791240004302)(354,0.962791240004302)(354,0.962791240004302)(355,0.962795724513849)(355,0.962795724513849)(355,0.962795724513849)(355,0.962795724513849)(355,0.962795724513849)(356,0.962803454805453)(356,0.962803454805453)(356,0.962803454805453)(356,0.962803454805453)(357,0.962814412623668)(357,0.962814412623668)(357,0.962814412623668)(357,0.962814412623668)(357,0.962814412623668)(357,0.962814412623668)(357,0.962814412623668)(357,0.962814412623668)(358,0.96283229362331)(358,0.96283229362331)(358,0.96283229362331)(359,0.962859207368946)(359,0.962859207368946)(359,0.962859207368946)(359,0.962859207368946)(360,0.96289508040129)(360,0.96289508040129)(360,0.96289508040129)(360,0.96289508040129)(360,0.96289508040129)(360,0.96289508040129)(360,0.96289508040129)(361,0.962938760176549)(361,0.962938760176549)(361,0.962938760176549)(361,0.962938760176549)(362,0.962988834705437)(362,0.962988834705437)(363,0.963043960579464)(363,0.963043960579464)(363,0.963043960579464)(363,0.963043960579464)(363,0.963043960579464)(364,0.96310301669742)(364,0.96310301669742)(364,0.96310301669742)(364,0.96310301669742)(365,0.963165115333559)(365,0.963165115333559)(365,0.963165115333559)(365,0.963165115333559)(366,0.963229596987339)(366,0.963229596987339)(366,0.963229596987339)(366,0.963229596987339)(366,0.963229596987339)(367,0.963295992720769)(367,0.963295992720769)(368,0.96336398418692)(368,0.96336398418692)(368,0.96336398418692)(368,0.96336398418692)(368,0.96336398418692)(369,0.963433325115759)(369,0.963433325115759)(370,0.963503798971485)(370,0.963503798971485)(370,0.963503798971485)(370,0.963503798971485)(372,0.963647623799517)(372,0.963647623799517)(372,0.963647623799517)(372,0.963647623799517)(373,0.963720855700045)(374,0.963794965505029)(374,0.963794965505029)(374,0.963794965505029)(374,0.963794965505029)(375,0.963869999387688)(375,0.963869999387688)(376,0.963946037131751)(376,0.963946037131751)(377,0.964023179324354)(377,0.964023179324354)(378,0.964101532858321)(379,0.964181216008743)(379,0.964181216008743)(379,0.964181216008743)(380,0.964262373287412)(380,0.964262373287412)(381,0.964345173191453)(381,0.964345173191453)(382,0.964429775192126)(382,0.964429775192126)(382,0.964429775192126)(384,0.96460503106239)(384,0.96460503106239)(384,0.96460503106239)(384,0.96460503106239)(386,0.964789293406043)(386,0.964789293406043)(386,0.964789293406043)(388,0.96498346986617)(388,0.96498346986617)(388,0.96498346986617)(389,0.965084580603826)(390,0.965188508045022)(394,0.965633587009639)(394,0.965633587009639)(395,0.965752382541782)(395,0.965752382541782)(399,0.966256640961473) 
};
\addlegendentry{\acl};

\end{axis}
\end{tikzpicture}%

%% This file was created by matlab2tikz v0.2.3.
% Copyright (c) 2008--2012, Nico Schlömer <nico.schloemer@gmail.com>
% All rights reserved.
% 
% 
% 
\begin{tikzpicture}

\begin{axis}[%
tick label style={font=\tiny},
label style={font=\tiny},
label shift={-4pt},
xlabel shift={-6pt},
legend style={font=\tiny},
view={0}{90},
width=\figurewidth,
height=\figureheight,
scale only axis,
xmin=0, xmax=400,
xlabel={Samples},
ymin=0, ymax=1,
ylabel={$F_1$-score},
axis lines*=left,
legend cell align=left,
legend style={at={(1.03,0)},anchor=south east,fill=none,draw=none,align=left,row sep=-0.2em},
clip=false]

\addplot [
color=red,
mark size=0.1pt,
only marks,
mark=*,
mark options={solid,fill=red},
forget plot
]
coordinates{
 (1,0)(2,0.0490797546012269)(3,0.0594059405940594)(4,0.0279542566709021)(5,0.030379746835443)(6,0.0203562340966921)(7,0.0203562340966921)(8,0)(9,0)(10,0)(11,0)(12,0.0137174211248285)(13,0.0366492146596858)(14,0.107448107448107)(15,0.118483412322275)(16,0.123809523809524)(17,0.113116726835138)(18,0.116915422885572)(19,0.219883040935672)(20,0.226327944572748)(21,0.226327944572748)(22,0.221187427240978)(23,0.204705882352941)(24,0.241019698725377)(25,0.241019698725377)(26,0.235981308411215)(27,0.235981308411215)(28,0.239336492890995)(29,0.230952380952381)(30,0.22673031026253)(31,0.22673031026253)(32,0.219806763285024)(33,0.227817745803357)(34,0.232335329341317)(35,0.247030878859857)(36,0.240762812872467)(37,0.249110320284697)(38,0.253254437869822)(39,0.253855278766311)(40,0.251482799525504)(41,0.247324613555291)(42,0.247619047619048)(43,0.247619047619048)(44,0.247324613555291)(45,0.248506571087216)(46,0.250602409638554)(47,0.254501800720288)(48,0.250602409638554)(49,0.250602409638554)(50,0.250602409638554)(51,0.251207729468599)(52,0.255421686746988)(53,0.25511432009627)(54,0.255421686746988)(55,0.255421686746988)(56,0.253317249698432)(57,0.255421686746988)(58,0.253317249698432)(59,0.25181598062954)(60,0.25181598062954)(61,0.252427184466019)(62,0.251219512195122)(63,0.24235006119951)(64,0.250913520097442)(65,0.261501210653753)(66,0.26360338573156)(67,0.255164034021871)(68,0.253041362530414)(69,0.244798041615667)(70,0.244798041615667)(71,0.251219512195122)(72,0.251219512195122)(73,0.251219512195122)(74,0.251219512195122)(75,0.250913520097442)(76,0.250913520097442)(77,0.253041362530414)(78,0.253041362530414)(79,0.25273390036452)(80,0.254854368932039)(81,0.253041362530414)(82,0.253041362530414)(83,0.248175182481752)(84,0.246642246642247)(85,0.246943765281174)(86,0.246642246642247)(87,0.248175182481752)(88,0.248175182481752)(89,0.248477466504263)(90,0.252427184466019)(91,0.25273390036452)(92,0.25273390036452)(93,0.253349573690621)(94,0.253349573690621)(95,0.253041362530414)(96,0.253041362530414)(97,0.253041362530414)(98,0.253041362530414)(99,0.253041362530414)(100,0.250913520097442)(101,0.251219512195122)(102,0.251219512195122)(103,0.251219512195122)(104,0.251219512195122)(105,0.251219512195122)(106,0.251219512195122)(107,0.251219512195122)(108,0.251219512195122)(109,0.253349573690621)(110,0.253041362530414)(111,0.257281553398058)(112,0.255164034021871)(113,0.253041362530414)(114,0.253349573690621)(115,0.253349573690621)(116,0.253349573690621)(117,0.253041362530414)(118,0.250913520097442)(119,0.250913520097442)(120,0.253041362530414)(121,0.250913520097442)(122,0.255164034021871)(123,0.255164034021871)(124,0.255164034021871)(125,0.257281553398058)(126,0.257281553398058)(127,0.257281553398058)(128,0.259393939393939)(129,0.259393939393939)(130,0.276110444177671)(131,0.259393939393939)(132,0.280239520958084)(133,0.324009324009324)(134,0.310223266745006)(135,0.310223266745006)(136,0.327906976744186)(137,0.339839265212399)(138,0.326388888888889)(139,0.326011560693642)(140,0.325581395348837)(141,0.324009324009324)(142,0.324009324009324)(143,0.324510932105869)(144,0.324510932105869)(145,0.325259515570934)(146,0.331034482758621)(147,0.338672768878718)(148,0.336769759450172)(149,0.342465753424657)(150,0.336769759450172)(151,0.336)(152,0.338672768878718)(153,0.338672768878718)(154,0.342465753424657)(155,0.340961098398169)(156,0.337155963302752)(157,0.335246842709529)(158,0.333333333333333)(159,0.333716915995397)(160,0.334486735870819)(161,0.333333333333333)(162,0.334101382488479)(163,0.329849012775842)(164,0.329849012775842)(165,0.329849012775842)(166,0.331402085747393)(167,0.337931034482759)(168,0.337931034482759)(169,0.336018411967779)(170,0.334486735870819)(171,0.338709677419355)(172,0.33679354094579)(173,0.33679354094579)(174,0.338709677419355)(175,0.338709677419355)(176,0.338709677419355)(177,0.338319907940161)(178,0.338319907940161)(179,0.340229885057471)(180,0.338319907940161)(181,0.342922899884925)(182,0.342922899884925)(183,0.344036697247706)(184,0.345933562428408)(185,0.345933562428408)(186,0.345933562428408)(187,0.341743119266055)(188,0.342528735632184)(189,0.342528735632184)(190,0.342922899884925)(191,0.342922899884925)(192,0.342922899884925)(193,0.344827586206896)(194,0.344827586206896)(195,0.344827586206896)(196,0.344827586206896)(197,0.344827586206896)(198,0.344827586206896)(199,0.344827586206896)(200,0.344827586206896)(201,0.344827586206896)(202,0.344827586206896)(203,0.346727898966705)(204,0.34902411021814)(205,0.350917431192661)(206,0.350917431192661)(207,0.350917431192661)(208,0.350917431192661)(209,0.350917431192661)(210,0.350917431192661)(211,0.352402745995423)(212,0.352402745995423)(213,0.352402745995423)(214,0.354691075514874)(215,0.356571428571429)(216,0.356571428571429)(217,0.356571428571429)(218,0.356571428571429)(219,0.356571428571429)(220,0.352806414662085)(221,0.352806414662085)(222,0.354691075514874)(223,0.354691075514874)(224,0.354691075514874)(225,0.354285714285714)(226,0.354285714285714)(227,0.34902411021814)(228,0.348623853211009)(229,0.348623853211009)(230,0.348623853211009)(231,0.352)(232,0.3557582668187)(233,0.3557582668187)(234,0.357630979498861)(235,0.3557582668187)(236,0.3557582668187)(237,0.358447488584475)(238,0.358447488584475)(239,0.356979405034325)(240,0.357388316151203)(241,0.357388316151203)(242,0.357388316151203)(243,0.359267734553776)(244,0.357388316151203)(245,0.355504587155963)(246,0.355504587155963)(247,0.356979405034325)(248,0.355097365406644)(249,0.355097365406644)(250,0.355504587155963)(251,0.355504587155963)(252,0.355504587155963)(253,0.355504587155963)(254,0.355504587155963)(255,0.355504587155963)(256,0.356979405034325)(257,0.358857142857143)(258,0.358857142857143)(259,0.356979405034325)(260,0.358857142857143)(261,0.358857142857143)(262,0.358857142857143)(263,0.360730593607306)(264,0.360730593607306)(265,0.360730593607306)(266,0.360730593607306)(267,0.362599771949829)(268,0.362599771949829)(269,0.362599771949829)(270,0.364464692482916)(271,0.362599771949829)(272,0.362599771949829)(273,0.362599771949829)(274,0.362599771949829)(275,0.362599771949829)(276,0.360730593607306)(277,0.360730593607306)(278,0.355097365406644)(279,0.355097365406644)(280,0.355097365406644)(281,0.355097365406644)(282,0.355097365406644)(283,0.356979405034325)(284,0.355097365406644)(285,0.355097365406644)(286,0.355097365406644)(287,0.358857142857143)(288,0.358857142857143)(289,0.358857142857143)(290,0.358857142857143)(291,0.355504587155963)(292,0.355504587155963)(293,0.355504587155963)(294,0.355504587155963)(295,0.35361653272101)(296,0.35361653272101)(297,0.35361653272101)(298,0.35361653272101)(299,0.353211009174312)(300,0.353211009174312)(301,0.353211009174312)(302,0.357388316151203)(303,0.359267734553776)(304,0.359267734553776)(305,0.357388316151203)(306,0.357388316151203)(307,0.355504587155963)(308,0.355097365406644)(309,0.355097365406644)(310,0.355097365406644)(311,0.355097365406644)(312,0.356571428571429)(313,0.356571428571429)(314,0.356571428571429)(315,0.356571428571429)(316,0.358447488584475)(317,0.358447488584475)(318,0.358447488584475)(319,0.358447488584475)(320,0.358447488584475)(321,0.360319270239453)(322,0.358447488584475)(323,0.358447488584475)(324,0.358447488584475)(325,0.358447488584475)(326,0.358447488584475)(327,0.360319270239453)(328,0.358447488584475)(329,0.358447488584475)(330,0.358447488584475)(331,0.360319270239453)(332,0.360319270239453)(333,0.360319270239453)(334,0.360319270239453)(335,0.362186788154897)(336,0.362186788154897)(337,0.362186788154897)(338,0.362186788154897)(339,0.362186788154897)(340,0.362186788154897)(341,0.362186788154897)(342,0.362186788154897)(343,0.362186788154897)(344,0.362186788154897)(345,0.362599771949829)(346,0.362599771949829)(347,0.362599771949829)(348,0.362599771949829)(349,0.362599771949829)(350,0.366325369738339)(351,0.366325369738339)(352,0.366325369738339)(353,0.368181818181818)(354,0.368181818181818)(355,0.368181818181818)(356,0.366325369738339)(357,0.366325369738339)(358,0.366325369738339)(359,0.366325369738339)(360,0.366325369738339)(361,0.366325369738339)(362,0.366325369738339)(363,0.366325369738339)(364,0.366325369738339)(365,0.362599771949829)(366,0.362599771949829)(367,0.360730593607306)(368,0.360730593607306)(369,0.360730593607306)(370,0.360730593607306)(371,0.362599771949829)(372,0.362599771949829)(373,0.363013698630137)(374,0.362599771949829)(375,0.362599771949829)(376,0.362599771949829)(377,0.362599771949829)(378,0.362599771949829)(379,0.363013698630137)(380,0.363013698630137)(381,0.363013698630137)(382,0.364880273660205)(383,0.364880273660205)(384,0.364880273660205)(385,0.364880273660205)(386,0.363013698630137)(387,0.363013698630137)(388,0.364880273660205)(389,0.364880273660205)(390,0.364880273660205)(391,0.364880273660205)(392,0.364880273660205)(393,0.364880273660205)(394,0.364880273660205)(395,0.364880273660205)(396,0.368600682593857)(397,0.368600682593857)(398,0.366742596810934)(399,0.366742596810934)(400,0.366742596810934) 
};
\addplot [
color=red,
mark size=0.1pt,
only marks,
mark=*,
mark options={solid,fill=red},
forget plot
]
coordinates{
 (1,0)(2,0)(3,0)(4,0.183982683982684)(5,0.215122470713525)(6,0.202393906420022)(7,0.227114716106605)(8,0.223946784922395)(9,0.756373937677054)(10,0.780939774983455)(11,0.822085889570552)(12,0.825531914893617)(13,0.824946846208363)(14,0.836008374040474)(15,0.844036697247706)(16,0.832857142857143)(17,0.830079537237889)(18,0.837209302325581)(19,0.84149855907781)(20,0.845315904139433)(21,0.847953216374269)(22,0.846661775495231)(23,0.844802342606149)(24,0.843653250773994)(25,0.845440494590417)(26,0.844547563805104)(27,0.860569715142429)(28,0.860569715142429)(29,0.854802680565897)(30,0.865143699336772)(31,0.864667154352597)(32,0.866909090909091)(33,0.87168458781362)(34,0.87168458781362)(35,0.870875179340029)(36,0.871757925072046)(37,0.87724335965542)(38,0.874105865522174)(39,0.874105865522174)(40,0.867435158501441)(41,0.875455207574654)(42,0.870967741935484)(43,0.870139398385913)(44,0.870139398385913)(45,0.874269005847953)(46,0.874908558888076)(47,0.874722016308377)(48,0.879411764705882)(49,0.879765395894428)(50,0.882395909422936)(51,0.885149963423555)(52,0.889855072463768)(53,0.887272727272727)(54,0.887272727272727)(55,0.891161431701972)(56,0.891161431701972)(57,0.897568165070007)(58,0.896907216494845)(59,0.898230088495575)(60,0.897810218978102)(61,0.899926953981008)(62,0.899122807017544)(63,0.899926953981008)(64,0.900729927007299)(65,0.899780541331382)(66,0.897360703812317)(67,0.896854425749817)(68,0.89554419284149)(69,0.898761835396941)(70,0.897305171158048)(71,0.897454545454545)(72,0.897305171158048)(73,0.899270072992701)(74,0.899270072992701)(75,0.900218499635834)(76,0.900072939460248)(77,0.892149669845928)(78,0.894273127753304)(79,0.893460690668626)(80,0.892961876832845)(81,0.900218499635834)(82,0.901960784313725)(83,0.900363636363636)(84,0.907514450867052)(85,0.906993511175198)(86,0.906993511175198)(87,0.906859205776173)(88,0.904247660187185)(89,0.905308464849354)(90,0.905308464849354)(91,0.905308464849354)(92,0.905308464849354)(93,0.906876790830945)(94,0.907010014306152)(95,0.908309455587392)(96,0.908309455587392)(97,0.908698777857656)(98,0.908698777857656)(99,0.908829863603733)(100,0.908829863603733)(101,0.910662824207493)(102,0.911319394376352)(103,0.910791366906475)(104,0.912482065997131)(105,0.911827956989247)(106,0.912607449856734)(107,0.913137114142139)(108,0.913916786226686)(109,0.914695340501792)(110,0.91404011461318)(111,0.913916786226686)(112,0.91588785046729)(113,0.91588785046729)(114,0.91588785046729)(115,0.916666666666667)(116,0.918571428571428)(117,0.918803418803419)(118,0.918803418803419)(119,0.918803418803419)(120,0.921108742004264)(121,0.920454545454545)(122,0.920679886685552)(123,0.920679886685552)(124,0.918079096045198)(125,0.918727915194346)(126,0.918079096045198)(127,0.916784203102962)(128,0.91743119266055)(129,0.91743119266055)(130,0.918079096045198)(131,0.916784203102962)(132,0.917314487632509)(133,0.917314487632509)(134,0.916784203102962)(135,0.91861288039632)(136,0.919377652050919)(137,0.920679886685552)(138,0.919148936170213)(139,0.919148936170213)(140,0.918265813788202)(141,0.919572953736655)(142,0.921540656205421)(143,0.921540656205421)(144,0.921540656205421)(145,0.921540656205421)(146,0.922857142857143)(147,0.926164874551971)(148,0.927494615936827)(149,0.925952552120776)(150,0.927494615936827)(151,0.927494615936827)(152,0.930565497494631)(153,0.930565497494631)(154,0.927390366642703)(155,0.926724137931034)(156,0.927849927849928)(157,0.927075812274368)(158,0.927075812274368)(159,0.927745664739884)(160,0.925308194343727)(161,0.925308194343727)(162,0.925308194343727)(163,0.925308194343727)(164,0.925979680696662)(165,0.92675852066715)(166,0.927431059506531)(167,0.92463768115942)(168,0.926652142338417)(169,0.928104575163399)(170,0.927325581395349)(171,0.926652142338417)(172,0.921954777534646)(173,0.921954777534646)(174,0.925979680696662)(175,0.925979680696662)(176,0.925979680696662)(177,0.925979680696662)(178,0.925979680696662)(179,0.926652142338417)(180,0.930232558139535)(181,0.930131004366812)(182,0.930232558139535)(183,0.930232558139535)(184,0.92955700798838)(185,0.92955700798838)(186,0.930434782608696)(187,0.931209268645909)(188,0.930535455861071)(189,0.929761042722665)(190,0.932178932178932)(191,0.93294881038212)(192,0.932178932178932)(193,0.932178932178932)(194,0.932178932178932)(195,0.932178932178932)(196,0.932178932178932)(197,0.93294881038212)(198,0.93294881038212)(199,0.935064935064935)(200,0.934296028880866)(201,0.935064935064935)(202,0.934296028880866)(203,0.933621933621934)(204,0.933621933621934)(205,0.932851985559567)(206,0.931308749096168)(207,0.932080924855491)(208,0.932080924855491)(209,0.932080924855491)(210,0.932754880694143)(211,0.932754880694143)(212,0.933526011560693)(213,0.933526011560693)(214,0.934296028880866)(215,0.933621933621934)(216,0.933621933621934)(217,0.933621933621934)(218,0.931308749096168)(219,0.932080924855491)(220,0.931407942238267)(221,0.931407942238267)(222,0.931407942238267)(223,0.932080924855491)(224,0.932851985559567)(225,0.932080924855491)(226,0.932080924855491)(227,0.931982633863965)(228,0.932754880694143)(229,0.933621933621934)(230,0.933621933621934)(231,0.934296028880866)(232,0.934971098265896)(233,0.937093275488069)(234,0.936324167872648)(235,0.935553946415641)(236,0.935553946415641)(237,0.936324167872648)(238,0.936324167872648)(239,0.936324167872648)(240,0.936324167872648)(241,0.937093275488069)(242,0.936324167872648)(243,0.938040345821326)(244,0.938040345821326)(245,0.938628158844765)(246,0.938628158844765)(247,0.937950937950938)(248,0.937950937950938)(249,0.938716654650324)(250,0.938040345821326)(251,0.938716654650324)(252,0.938716654650324)(253,0.937950937950938)(254,0.934105720492397)(255,0.935647143890094)(256,0.936416184971098)(257,0.935740072202166)(258,0.935740072202166)(259,0.937093275488069)(260,0.935740072202166)(261,0.936324167872648)(262,0.937093275488069)(263,0.937093275488069)(264,0.9378612716763)(265,0.9378612716763)(266,0.935832732516222)(267,0.935832732516222)(268,0.935832732516222)(269,0.935832732516222)(270,0.935832732516222)(271,0.936599423631124)(272,0.936599423631124)(273,0.936599423631124)(274,0.93745506829619)(275,0.93745506829619)(276,0.938218390804598)(277,0.937544867193108)(278,0.936872309899569)(279,0.937544867193108)(280,0.936872309899569)(281,0.937544867193108)(282,0.937544867193108)(283,0.937544867193108)(284,0.936872309899569)(285,0.938892882818116)(286,0.938892882818116)(287,0.938892882818116)(288,0.938892882818116)(289,0.938892882818116)(290,0.939568345323741)(291,0.938804895608351)(292,0.936507936507936)(293,0.936507936507936)(294,0.936507936507936)(295,0.935740072202166)(296,0.935832732516222)(297,0.935832732516222)(298,0.936507936507936)(299,0.936507936507936)(300,0.936507936507936)(301,0.936507936507936)(302,0.935925125989921)(303,0.935925125989921)(304,0.935925125989921)(305,0.935925125989921)(306,0.936690647482014)(307,0.934485241180705)(308,0.935158501440922)(309,0.935158501440922)(310,0.936690647482014)(311,0.935925125989921)(312,0.936017253774263)(313,0.936017253774263)(314,0.936599423631124)(315,0.936599423631124)(316,0.937274693583273)(317,0.936599423631124)(318,0.936599423631124)(319,0.937365010799136)(320,0.937365010799136)(321,0.936599423631124)(322,0.938129496402878)(323,0.937365010799136)(324,0.936690647482014)(325,0.936690647482014)(326,0.936690647482014)(327,0.936690647482014)(328,0.936690647482014)(329,0.936690647482014)(330,0.935925125989921)(331,0.93745506829619)(332,0.93745506829619)(333,0.938129496402878)(334,0.938218390804598)(335,0.938218390804598)(336,0.939655172413793)(337,0.939568345323741)(338,0.940330697340043)(339,0.939655172413793)(340,0.939655172413793)(341,0.939655172413793)(342,0.939655172413793)(343,0.938892882818116)(344,0.939655172413793)(345,0.939655172413793)(346,0.939655172413793)(347,0.939655172413793)(348,0.940416367552046)(349,0.939655172413793)(350,0.939481268011527)(351,0.939481268011527)(352,0.938804895608351)(353,0.938804895608351)(354,0.938804895608351)(355,0.939568345323741)(356,0.938892882818116)(357,0.938892882818116)(358,0.938892882818116)(359,0.939568345323741)(360,0.939568345323741)(361,0.939568345323741)(362,0.938129496402878)(363,0.938129496402878)(364,0.938129496402878)(365,0.938129496402878)(366,0.938129496402878)(367,0.936599423631124)(368,0.938129496402878)(369,0.936690647482014)(370,0.938218390804598)(371,0.938218390804598)(372,0.938218390804598)(373,0.938218390804598)(374,0.938218390804598)(375,0.938218390804598)(376,0.938218390804598)(377,0.938218390804598)(378,0.938218390804598)(379,0.938218390804598)(380,0.938218390804598)(381,0.93745506829619)(382,0.93745506829619)(383,0.938129496402878)(384,0.938129496402878)(385,0.938129496402878)(386,0.938129496402878)(387,0.938129496402878)(388,0.938218390804598)(389,0.938218390804598)(390,0.938892882818116)(391,0.938892882818116)(392,0.938892882818116)(393,0.938218390804598)(394,0.939568345323741)(395,0.940244780417567)(396,0.940244780417567)(397,0.938892882818116)(398,0.938892882818116)(399,0.938129496402878)(400,0.938892882818116) 
};
\addplot [
color=red,
mark size=0.1pt,
only marks,
mark=*,
mark options={solid,fill=red},
forget plot
]
coordinates{
 (1,0)(2,0)(3,0)(4,0)(5,0)(6,0.593424218123496)(7,0.623853211009174)(8,0.635205992509363)(9,0.670347003154574)(10,0.738461538461538)(11,0.801509433962264)(12,0.804201050262566)(13,0.817204301075269)(14,0.812644564379337)(15,0.824261275272162)(16,0.824902723735408)(17,0.821036106750392)(18,0.822974036191975)(19,0.820143884892086)(20,0.822134387351778)(21,0.820754716981132)(22,0.822429906542056)(23,0.822981366459627)(24,0.823620823620824)(25,0.820592823712948)(26,0.821233411397346)(27,0.821233411397346)(28,0.819390148553557)(29,0.815396700706991)(30,0.818325434439178)(31,0.81897233201581)(32,0.81897233201581)(33,0.819258089976322)(34,0.819258089976322)(35,0.823343848580441)(36,0.822695035460993)(37,0.822695035460993)(38,0.822695035460993)(39,0.822695035460993)(40,0.821570182394925)(41,0.818759936406995)(42,0.819411296738266)(43,0.818759936406995)(44,0.819698173153296)(45,0.814046288906624)(46,0.813343923749007)(47,0.813990461049284)(48,0.818685669041964)(49,0.822415153906866)(50,0.824367088607595)(51,0.823156225218081)(52,0.823715415019763)(53,0.824547600314713)(54,0.824271079590228)(55,0.825471698113207)(56,0.824921135646688)(57,0.825949367088607)(58,0.825296442687747)(59,0.825296442687747)(60,0.826224328593997)(61,0.82818685669042)(62,0.829113924050633)(63,0.829113924050633)(64,0.82818685669042)(65,0.82818685669042)(66,0.82818685669042)(67,0.82818685669042)(68,0.82818685669042)(69,0.82818685669042)(70,0.83147853736089)(71,0.832406671961874)(72,0.829617834394904)(73,0.829617834394904)(74,0.828685258964143)(75,0.829346092503987)(76,0.828685258964143)(77,0.829617834394904)(78,0.829617834394904)(79,0.829617834394904)(80,0.827751196172249)(81,0.829073482428115)(82,0.828411811652035)(83,0.828411811652035)(84,0.828411811652035)(85,0.827751196172249)(86,0.829617834394904)(87,0.83054892601432)(88,0.83054892601432)(89,0.83054892601432)(90,0.831210191082802)(91,0.829888712241653)(92,0.829888712241653)(93,0.828025477707006)(94,0.829346092503987)(95,0.830278884462151)(96,0.829346092503987)(97,0.830670926517572)(98,0.832933653077538)(99,0.83000798084597)(100,0.83000798084597)(101,0.83000798084597)(102,0.830670926517572)(103,0.835322195704057)(104,0.832535885167464)(105,0.833200319233839)(106,0.833466135458167)(107,0.834130781499202)(108,0.835059760956175)(109,0.833731105807478)(110,0.833333333333333)(111,0.833731105807478)(112,0.833731105807478)(113,0.833068362480127)(114,0.833068362480127)(115,0.833068362480127)(116,0.833068362480127)(117,0.833731105807478)(118,0.833731105807478)(119,0.834658187599364)(120,0.833731105807478)(121,0.833731105807478)(122,0.833731105807478)(123,0.835059760956175)(124,0.839486356340289)(125,0.841346153846154)(126,0.839071257005604)(127,0.839071257005604)(128,0.839071257005604)(129,0.837209302325581)(130,0.838141025641025)(131,0.839071257005604)(132,0.838141025641025)(133,0.839328537170264)(134,0.839328537170264)(135,0.837469975980785)(136,0.837469975980785)(137,0.837469975980785)(138,0.839328537170264)(139,0.839328537170264)(140,0.839328537170264)(141,0.839328537170264)(142,0.839328537170264)(143,0.839328537170264)(144,0.839328537170264)(145,0.840255591054313)(146,0.840255591054313)(147,0.839584996009577)(148,0.839584996009577)(149,0.839584996009577)(150,0.839584996009577)(151,0.839584996009577)(152,0.839584996009577)(153,0.839584996009577)(154,0.838915470494418)(155,0.839584996009577)(156,0.840255591054313)(157,0.840255591054313)(158,0.841181165203511)(159,0.842105263157895)(160,0.844868735083532)(161,0.844868735083532)(162,0.844868735083532)(163,0.844868735083532)(164,0.844868735083532)(165,0.845541401273885)(166,0.845541401273885)(167,0.845541401273885)(168,0.845541401273885)(169,0.845786963434022)(170,0.846946867565424)(171,0.846275752773376)(172,0.846275752773376)(173,0.846275752773376)(174,0.846275752773376)(175,0.846275752773376)(176,0.846275752773376)(177,0.846275752773376)(178,0.845360824742268)(179,0.844444444444444)(180,0.844444444444444)(181,0.844444444444444)(182,0.844444444444444)(183,0.844444444444444)(184,0.844444444444444)(185,0.844444444444444)(186,0.844444444444444)(187,0.844197138314785)(188,0.844197138314785)(189,0.843949044585987)(190,0.844197138314785)(191,0.844444444444444)(192,0.844444444444444)(193,0.844690966719493)(194,0.844690966719493)(195,0.844690966719493)(196,0.844936708860759)(197,0.845605700712589)(198,0.843774781919112)(199,0.843774781919112)(200,0.843774781919112)(201,0.843774781919112)(202,0.842188739095955)(203,0.842188739095955)(204,0.842188739095955)(205,0.842188739095955)(206,0.843354430379747)(207,0.843354430379747)(208,0.843354430379747)(209,0.842438638163104)(210,0.843354430379747)(211,0.843354430379747)(212,0.843354430379747)(213,0.843354430379747)(214,0.843354430379747)(215,0.844022169437846)(216,0.84310618066561)(217,0.84310618066561)(218,0.84310618066561)(219,0.843774781919112)(220,0.846518987341772)(221,0.846518987341772)(222,0.846518987341772)(223,0.846518987341772)(224,0.846518987341772)(225,0.845849802371541)(226,0.845849802371541)(227,0.845181674565561)(228,0.845181674565561)(229,0.845181674565561)(230,0.847189231987332)(231,0.846518987341772)(232,0.846518987341772)(233,0.846518987341772)(234,0.847189231987332)(235,0.849921011058452)(236,0.849921011058452)(237,0.849250197316495)(238,0.849250197316495)(239,0.849250197316495)(240,0.849250197316495)(241,0.849250197316495)(242,0.849250197316495)(243,0.848580441640378)(244,0.848580441640378)(245,0.848580441640378)(246,0.848580441640378)(247,0.848580441640378)(248,0.848580441640378)(249,0.848580441640378)(250,0.848580441640378)(251,0.848580441640378)(252,0.848580441640378)(253,0.848580441640378)(254,0.848580441640378)(255,0.848580441640378)(256,0.848580441640378)(257,0.848580441640378)(258,0.848580441640378)(259,0.848580441640378)(260,0.848580441640378)(261,0.848580441640378)(262,0.848580441640378)(263,0.848580441640378)(264,0.848580441640378)(265,0.848580441640378)(266,0.849250197316495)(267,0.849250197316495)(268,0.850157728706624)(269,0.850157728706624)(270,0.847911741528763)(271,0.847911741528763)(272,0.847911741528763)(273,0.847003154574132)(274,0.846335697399527)(275,0.846335697399527)(276,0.846335697399527)(277,0.846335697399527)(278,0.845669291338583)(279,0.845669291338583)(280,0.845669291338583)(281,0.845669291338583)(282,0.846577498033045)(283,0.846577498033045)(284,0.847244094488189)(285,0.849056603773585)(286,0.848151062155783)(287,0.848151062155783)(288,0.848151062155783)(289,0.848151062155783)(290,0.849056603773585)(291,0.849056603773585)(292,0.849056603773585)(293,0.849056603773585)(294,0.849056603773585)(295,0.849056603773585)(296,0.848151062155783)(297,0.848151062155783)(298,0.848151062155783)(299,0.848151062155783)(300,0.848151062155783)(301,0.848151062155783)(302,0.848818897637795)(303,0.848818897637795)(304,0.848818897637795)(305,0.848818897637795)(306,0.848818897637795)(307,0.848818897637795)(308,0.848151062155783)(309,0.848818897637795)(310,0.846093133385951)(311,0.847003154574132)(312,0.847911741528763)(313,0.847911741528763)(314,0.847244094488189)(315,0.845425867507886)(316,0.845425867507886)(317,0.845425867507886)(318,0.845425867507886)(319,0.845425867507886)(320,0.845425867507886)(321,0.846335697399527)(322,0.846335697399527)(323,0.846335697399527)(324,0.846335697399527)(325,0.846335697399527)(326,0.846335697399527)(327,0.846335697399527)(328,0.847244094488189)(329,0.846335697399527)(330,0.848151062155783)(331,0.848818897637795)(332,0.848818897637795)(333,0.848151062155783)(334,0.848151062155783)(335,0.848151062155783)(336,0.848151062155783)(337,0.848151062155783)(338,0.848818897637795)(339,0.848818897637795)(340,0.849487785657998)(341,0.849487785657998)(342,0.849487785657998)(343,0.849487785657998)(344,0.851500789889415)(345,0.851500789889415)(346,0.851500789889415)(347,0.852173913043478)(348,0.852173913043478)(349,0.852173913043478)(350,0.852173913043478)(351,0.852173913043478)(352,0.852173913043478)(353,0.852173913043478)(354,0.852173913043478)(355,0.852173913043478)(356,0.852173913043478)(357,0.852173913043478)(358,0.852848101265823)(359,0.852173913043478)(360,0.852173913043478)(361,0.852848101265823)(362,0.852848101265823)(363,0.852848101265823)(364,0.852848101265823)(365,0.852848101265823)(366,0.852848101265823)(367,0.852848101265823)(368,0.852848101265823)(369,0.852848101265823)(370,0.852848101265823)(371,0.852848101265823)(372,0.852848101265823)(373,0.852848101265823)(374,0.852848101265823)(375,0.852848101265823)(376,0.852848101265823)(377,0.852848101265823)(378,0.852848101265823)(379,0.85126582278481)(380,0.851030110935024)(381,0.851030110935024)(382,0.850118953211737)(383,0.850118953211737)(384,0.850118953211737)(385,0.850356294536817)(386,0.851030110935024)(387,0.851030110935024)(388,0.851030110935024)(389,0.851030110935024)(390,0.851030110935024)(391,0.851030110935024)(392,0.851030110935024)(393,0.851030110935024)(394,0.851030110935024)(395,0.851030110935024)(396,0.851030110935024)(397,0.853754940711462)(398,0.853754940711462)(399,0.853754940711462)(400,0.853754940711462) 
};
\addplot [
color=red,
mark size=0.1pt,
only marks,
mark=*,
mark options={solid,fill=red},
forget plot
]
coordinates{
 (1,0.0698795180722891)(2,0.0462287104622871)(3,0.0391676866585067)(4,0.0768245838668374)(5,0.0839694656488549)(6,0.0787401574803149)(7,0.0513513513513513)(8,0.0512129380053908)(9,0.0328767123287671)(10,0)(11,0)(12,0)(13,0)(14,0)(15,0.0220082530949106)(16,0)(17,0.00555555555555555)(18,0.00555555555555555)(19,0.00278551532033426)(20,0.00277777777777778)(21,0.00278164116828929)(22,0)(23,0.0110344827586207)(24,0.00277777777777778)(25,0)(26,0.0138312586445366)(27,0.0138312586445366)(28,0.0488466757123473)(29,0.0435374149659864)(30,0.064516129032258)(31,0.0723860589812332)(32,0.095364238410596)(33,0.0748663101604278)(34,0.0697050938337801)(35,0.0825565912117177)(36,0.0825565912117177)(37,0.0775401069518716)(38,0.0775401069518716)(39,0.095364238410596)(40,0.0928381962864721)(41,0.0928381962864721)(42,0.0928381962864721)(43,0.0903054448871182)(44,0.0901856763925729)(45,0.0927152317880794)(46,0.0927152317880794)(47,0.0927152317880794)(48,0.0977542932628798)(49,0.0976253298153034)(50,0.0972404730617608)(51,0.0992167101827676)(52,0.0996068152031455)(53,0.0973684210526316)(54,0.0973684210526316)(55,0.102362204724409)(56,0.102496714848883)(57,0.0973684210526316)(58,0.0973684210526316)(59,0.0998685939553219)(60,0.1)(61,0.0973684210526316)(62,0.0973684210526316)(63,0.0976253298153034)(64,0.0976253298153034)(65,0.0998685939553219)(66,0.0998685939553219)(67,0.102362204724409)(68,0.102362204724409)(69,0.102362204724409)(70,0.104849279161206)(71,0.107049608355091)(72,0.104849279161206)(73,0.10498687664042)(74,0.107329842931937)(75,0.104849279161206)(76,0.105124835742444)(77,0.10761154855643)(78,0.10761154855643)(79,0.109803921568627)(80,0.109803921568627)(81,0.109803921568627)(82,0.11963589076723)(83,0.1171875)(84,0.11963589076723)(85,0.11963589076723)(86,0.119947848761408)(87,0.119947848761408)(88,0.119947848761408)(89,0.122395833333333)(90,0.124837451235371)(91,0.127272727272727)(92,0.127272727272727)(93,0.12970168612192)(94,0.129198966408269)(95,0.128865979381443)(96,0.128865979381443)(97,0.128865979381443)(98,0.128865979381443)(99,0.128865979381443)(100,0.128865979381443)(101,0.129032258064516)(102,0.129032258064516)(103,0.129533678756477)(104,0.129366106080207)(105,0.129198966408269)(106,0.12970168612192)(107,0.132124352331606)(108,0.132124352331606)(109,0.132124352331606)(110,0.132124352331606)(111,0.12970168612192)(112,0.12970168612192)(113,0.12970168612192)(114,0.12970168612192)(115,0.12970168612192)(116,0.12987012987013)(117,0.12987012987013)(118,0.12987012987013)(119,0.12970168612192)(120,0.12970168612192)(121,0.12970168612192)(122,0.12970168612192)(123,0.12970168612192)(124,0.12970168612192)(125,0.131953428201811)(126,0.131953428201811)(127,0.131953428201811)(128,0.131953428201811)(129,0.131782945736434)(130,0.131953428201811)(131,0.131953428201811)(132,0.131953428201811)(133,0.131953428201811)(134,0.132124352331606)(135,0.131953428201811)(136,0.131953428201811)(137,0.131782945736434)(138,0.131782945736434)(139,0.132295719844358)(140,0.131953428201811)(141,0.131782945736434)(142,0.131953428201811)(143,0.132124352331606)(144,0.132124352331606)(145,0.131782945736434)(146,0.131782945736434)(147,0.131782945736434)(148,0.131782945736434)(149,0.132124352331606)(150,0.132124352331606)(151,0.131782945736434)(152,0.131953428201811)(153,0.134366925064599)(154,0.134366925064599)(155,0.134193548387097)(156,0.134193548387097)(157,0.134366925064599)(158,0.134366925064599)(159,0.134366925064599)(160,0.134366925064599)(161,0.134366925064599)(162,0.134366925064599)(163,0.134366925064599)(164,0.134366925064599)(165,0.134366925064599)(166,0.134366925064599)(167,0.134366925064599)(168,0.134193548387097)(169,0.131953428201811)(170,0.134540750323415)(171,0.134540750323415)(172,0.134540750323415)(173,0.134366925064599)(174,0.134366925064599)(175,0.134366925064599)(176,0.134366925064599)(177,0.134366925064599)(178,0.134366925064599)(179,0.134366925064599)(180,0.134366925064599)(181,0.134366925064599)(182,0.134366925064599)(183,0.131953428201811)(184,0.131782945736434)(185,0.131782945736434)(186,0.131782945736434)(187,0.134193548387097)(188,0.134193548387097)(189,0.134020618556701)(190,0.134020618556701)(191,0.134020618556701)(192,0.134020618556701)(193,0.134020618556701)(194,0.134366925064599)(195,0.134366925064599)(196,0.134193548387097)(197,0.297921478060046)(198,0.320090805902384)(199,0.312428734321551)(200,0.321759259259259)(201,0.321016166281755)(202,0.344671201814059)(203,0.354260089686099)(204,0.354260089686099)(205,0.354657687991021)(206,0.365256124721603)(207,0.372246696035242)(208,0.369469026548673)(209,0.373068432671082)(210,0.372246696035242)(211,0.373216245883644)(212,0.373626373626374)(213,0.373626373626374)(214,0.373626373626374)(215,0.371584699453552)(216,0.37117903930131)(217,0.36980306345733)(218,0.368421052631579)(219,0.368017524644031)(220,0.36761487964989)(221,0.366228070175439)(222,0.368653421633554)(223,0.368246968026461)(224,0.368246968026461)(225,0.369878183831672)(226,0.369878183831672)(227,0.360400444938821)(228,0.358803986710963)(229,0.35920177383592)(230,0.358803986710963)(231,0.364640883977901)(232,0.36283185840708)(233,0.35920177383592)(234,0.361018826135105)(235,0.360619469026549)(236,0.359823399558499)(237,0.360220994475138)(238,0.360220994475138)(239,0.361419068736142)(240,0.361018826135105)(241,0.361820199778024)(242,0.361419068736142)(243,0.361820199778024)(244,0.361419068736142)(245,0.363028953229399)(246,0.361607142857143)(247,0.358342665173572)(248,0.363028953229399)(249,0.363028953229399)(250,0.362222222222222)(251,0.363433667781494)(252,0.362011173184357)(253,0.361607142857143)(254,0.36241610738255)(255,0.362011173184357)(256,0.36)(257,0.360400444938821)(258,0.358574610244989)(259,0.357380688124306)(260,0.361018826135105)(261,0.361018826135105)(262,0.361018826135105)(263,0.360619469026549)(264,0.360619469026549)(265,0.360619469026549)(266,0.361018826135105)(267,0.361419068736142)(268,0.361419068736142)(269,0.36)(270,0.359600443951165)(271,0.359600443951165)(272,0.35920177383592)(273,0.361419068736142)(274,0.359600443951165)(275,0.359600443951165)(276,0.359600443951165)(277,0.361018826135105)(278,0.361018826135105)(279,0.361018826135105)(280,0.361419068736142)(281,0.361419068736142)(282,0.359600443951165)(283,0.360801781737194)(284,0.356744704570791)(285,0.360400444938821)(286,0.360400444938821)(287,0.360400444938821)(288,0.360400444938821)(289,0.361204013377926)(290,0.361204013377926)(291,0.359375)(292,0.359776536312849)(293,0.360801781737194)(294,0.360801781737194)(295,0.362625139043382)(296,0.360801781737194)(297,0.361204013377926)(298,0.36241610738255)(299,0.36241610738255)(300,0.36241610738255)(301,0.364245810055866)(302,0.368303571428571)(303,0.368303571428571)(304,0.368303571428571)(305,0.368303571428571)(306,0.368303571428571)(307,0.366480446927374)(308,0.364653243847875)(309,0.366480446927374)(310,0.368303571428571)(311,0.368715083798883)(312,0.371937639198218)(313,0.371937639198218)(314,0.371937639198218)(315,0.372352285395764)(316,0.372352285395764)(317,0.372352285395764)(318,0.372352285395764)(319,0.372352285395764)(320,0.372352285395764)(321,0.372352285395764)(322,0.372352285395764)(323,0.372352285395764)(324,0.371523915461624)(325,0.371523915461624)(326,0.371937639198218)(327,0.371937639198218)(328,0.371937639198218)(329,0.371523915461624)(330,0.371523915461624)(331,0.371523915461624)(332,0.371523915461624)(333,0.371523915461624)(334,0.371523915461624)(335,0.371523915461624)(336,0.371523915461624)(337,0.371523915461624)(338,0.371523915461624)(339,0.371523915461624)(340,0.371523915461624)(341,0.371523915461624)(342,0.371523915461624)(343,0.371523915461624)(344,0.371523915461624)(345,0.371523915461624)(346,0.371523915461624)(347,0.371523915461624)(348,0.371523915461624)(349,0.371523915461624)(350,0.371523915461624)(351,0.371523915461624)(352,0.371523915461624)(353,0.371523915461624)(354,0.371523915461624)(355,0.371523915461624)(356,0.371523915461624)(357,0.371937639198218)(358,0.371937639198218)(359,0.371937639198218)(360,0.371937639198218)(361,0.371937639198218)(362,0.371937639198218)(363,0.371937639198218)(364,0.371937639198218)(365,0.371937639198218)(366,0.371937639198218)(367,0.371937639198218)(368,0.371937639198218)(369,0.374164810690423)(370,0.374164810690423)(371,0.372352285395764)(372,0.372352285395764)(373,0.374164810690423)(374,0.374164810690423)(375,0.374164810690423)(376,0.374164810690423)(377,0.374164810690423)(378,0.374164810690423)(379,0.374164810690423)(380,0.374164810690423)(381,0.374164810690423)(382,0.374164810690423)(383,0.373748609566185)(384,0.373748609566185)(385,0.373748609566185)(386,0.373748609566185)(387,0.373748609566185)(388,0.371937639198218)(389,0.371937639198218)(390,0.372352285395764)(391,0.372352285395764)(392,0.372352285395764)(393,0.374164810690423)(394,0.374581939799331)(395,0.374581939799331)(396,0.374581939799331)(397,0.374581939799331)(398,0.374581939799331)(399,0.374581939799331)(400,0.374581939799331) 
};
\addplot [
color=red,
mark size=0.1pt,
only marks,
mark=*,
mark options={solid,fill=red},
forget plot
]
coordinates{
 (1,0)(2,0)(3,0)(4,0)(5,0)(6,0)(7,0.0697050938337801)(8,0.0839895013123359)(9,0)(10,0.240722166499498)(11,0.236686390532544)(12,0.215447154471545)(13,0.232356134636265)(14,0.236813778256189)(15,0.229284903518729)(16,0.227990970654628)(17,0.204946996466431)(18,0.440111420612813)(19,0.705551651440618)(20,0.777943368107302)(21,0.772865853658536)(22,0.783520599250936)(23,0.825688073394495)(24,0.826025459688826)(25,0.825870646766169)(26,0.838574423480084)(27,0.83785664578984)(28,0.838664812239221)(29,0.833218943033631)(30,0.835512732278045)(31,0.837307152875175)(32,0.838255977496484)(33,0.843905915894512)(34,0.847482014388489)(35,0.84809215262779)(36,0.849964106245513)(37,0.854209445585216)(38,0.850653819683414)(39,0.860382707299787)(40,0.857142857142857)(41,0.869942196531792)(42,0.876632801161103)(43,0.876632801161103)(44,0.880694143167028)(45,0.878679109834889)(46,0.879942487419123)(47,0.881038211968277)(48,0.871035940803383)(49,0.871035940803383)(50,0.870488322717622)(51,0.871287128712871)(52,0.872521246458923)(53,0.872521246458923)(54,0.871722182849043)(55,0.874115983026874)(56,0.876595744680851)(57,0.872265349329569)(58,0.871903750884643)(59,0.878118317890235)(60,0.877667140825035)(61,0.877667140825035)(62,0.88872936109117)(63,0.888412017167382)(64,0.887616320687187)(65,0.886981402002861)(66,0.889048991354467)(67,0.889526542324247)(68,0.889208633093525)(69,0.884393063583815)(70,0.886002886002886)(71,0.88647866955893)(72,0.88647866955893)(73,0.886314265025344)(74,0.888243831640058)(75,0.890341321713871)(76,0.888399412628488)(77,0.887582659808964)(78,0.888235294117647)(79,0.890029325513196)(80,0.89068231841526)(81,0.89068231841526)(82,0.890842490842491)(83,0.891654465592972)(84,0.892307692307692)(85,0.890350877192982)(86,0.893460690668626)(87,0.891161431701972)(88,0.891970802919708)(89,0.892622352081811)(90,0.892778993435449)(91,0.895238095238095)(92,0.895238095238095)(93,0.895741556534508)(94,0.895741556534508)(95,0.895741556534508)(96,0.894930198383541)(97,0.8983174835406)(98,0.900218499635834)(99,0.901387874360847)(100,0.898826979472141)(101,0.900662251655629)(102,0.901325478645066)(103,0.905325443786982)(104,0.909361069836553)(105,0.908685968819599)(106,0.908002991772625)(107,0.91449814126394)(108,0.918595967139656)(109,0.918595967139656)(110,0.918595967139656)(111,0.919402985074627)(112,0.919402985074627)(113,0.918717375093214)(114,0.918717375093214)(115,0.917102315160568)(116,0.917102315160568)(117,0.91304347826087)(118,0.915482423335826)(119,0.915482423335826)(120,0.915482423335826)(121,0.919523099850969)(122,0.919523099850969)(123,0.92020879940343)(124,0.918838421444527)(125,0.92020879940343)(126,0.92020879940343)(127,0.921130952380952)(128,0.922619047619048)(129,0.922734026745914)(130,0.922734026745914)(131,0.922734026745914)(132,0.922048997772828)(133,0.920681986656783)(134,0.920681986656783)(135,0.922734026745914)(136,0.923076923076923)(137,0.923076923076923)(138,0.923076923076923)(139,0.922279792746114)(140,0.921713441654357)(141,0.920916481892091)(142,0.920916481892091)(143,0.920916481892091)(144,0.928728875826598)(145,0.92557111274871)(146,0.924778761061947)(147,0.92557111274871)(148,0.92557111274871)(149,0.927152317880795)(150,0.926470588235294)(151,0.926362297496318)(152,0.924431401320616)(153,0.924542124542124)(154,0.923753665689149)(155,0.923753665689149)(156,0.922964049889949)(157,0.922287390029325)(158,0.922287390029325)(159,0.923076923076923)(160,0.923076923076923)(161,0.923076923076923)(162,0.922287390029325)(163,0.920704845814978)(164,0.922287390029325)(165,0.922287390029325)(166,0.922964049889949)(167,0.924542124542124)(168,0.926900584795322)(169,0.927684441197955)(170,0.927684441197955)(171,0.927684441197955)(172,0.927684441197955)(173,0.927684441197955)(174,0.927684441197955)(175,0.927684441197955)(176,0.928362573099415)(177,0.928362573099415)(178,0.931768158473954)(179,0.931768158473954)(180,0.931768158473954)(181,0.930882352941176)(182,0.932452276064611)(183,0.933137398971345)(184,0.93108504398827)(185,0.93108504398827)(186,0.93108504398827)(187,0.931868131868132)(188,0.931868131868132)(189,0.931868131868132)(190,0.93108504398827)(191,0.931185944363104)(192,0.931967812728603)(193,0.931967812728603)(194,0.935083880379285)(195,0.934402332361516)(196,0.935272727272727)(197,0.935953420669578)(198,0.935953420669578)(199,0.935953420669578)(200,0.935272727272727)(201,0.935272727272727)(202,0.935272727272727)(203,0.933721777130371)(204,0.933721777130371)(205,0.933721777130371)(206,0.940665701881331)(207,0.940244780417567)(208,0.939898624185373)(209,0.946762589928057)(210,0.939898624185373)(211,0.943804034582133)(212,0.943804034582133)(213,0.942196531791907)(214,0.94759511844939)(215,0.94759511844939)(216,0.948350071736011)(217,0.953604568165596)(218,0.958748221906117)(219,0.959660297239915)(220,0.957386363636364)(221,0.957386363636364)(222,0.964689265536723)(223,0.963328631875881)(224,0.967651195499297)(225,0.96914446002805)(226,0.967696629213483)(227,0.968332160450387)(228,0.965419901199718)(229,0.965371024734982)(230,0.964059196617336)(231,0.962649753347428)(232,0.960618846694796)(233,0.959269662921348)(234,0.955244755244755)(235,0.959943780744905)(236,0.95859649122807)(237,0.959326788218794)(238,0.96)(239,0.957865168539326)(240,0.959154929577465)(241,0.959154929577465)(242,0.959154929577465)(243,0.96056338028169)(244,0.959097320169252)(245,0.958362738179252)(246,0.958362738179252)(247,0.958362738179252)(248,0.958362738179252)(249,0.959717314487632)(250,0.959774170783345)(251,0.959039548022599)(252,0.958244869072894)(253,0.958923512747875)(254,0.958923512747875)(255,0.96039603960396)(256,0.96039603960396)(257,0.96039603960396)(258,0.961020552799433)(259,0.961020552799433)(260,0.961020552799433)(261,0.961020552799433)(262,0.961020552799433)(263,0.961020552799433)(264,0.961020552799433)(265,0.961020552799433)(266,0.961020552799433)(267,0.961020552799433)(268,0.961020552799433)(269,0.961020552799433)(270,0.958981612446959)(271,0.958981612446959)(272,0.958981612446959)(273,0.958981612446959)(274,0.958981612446959)(275,0.958981612446959)(276,0.959717314487632)(277,0.96045197740113)(278,0.959774170783345)(279,0.959774170783345)(280,0.961240310077519)(281,0.962597035991531)(282,0.962597035991531)(283,0.961971830985915)(284,0.961971830985915)(285,0.961294862772695)(286,0.96056338028169)(287,0.96056338028169)(288,0.961240310077519)(289,0.961185603387438)(290,0.961918194640338)(291,0.961918194640338)(292,0.961864406779661)(293,0.962597035991531)(294,0.965371024734982)(295,0.965371024734982)(296,0.962491153573956)(297,0.962491153573956)(298,0.962491153573956)(299,0.961810466760962)(300,0.961810466760962)(301,0.961810466760962)(302,0.961810466760962)(303,0.961810466760962)(304,0.961810466760962)(305,0.961810466760962)(306,0.961810466760962)(307,0.961810466760962)(308,0.962597035991531)(309,0.962597035991531)(310,0.96056338028169)(311,0.961918194640338)(312,0.961918194640338)(313,0.962649753347428)(314,0.961240310077519)(315,0.961240310077519)(316,0.961240310077519)(317,0.961240310077519)(318,0.961240310077519)(319,0.962597035991531)(320,0.962597035991531)(321,0.96056338028169)(322,0.959830866807611)(323,0.959830866807611)(324,0.959830866807611)(325,0.96056338028169)(326,0.961294862772695)(327,0.96056338028169)(328,0.961294862772695)(329,0.961294862772695)(330,0.961294862772695)(331,0.961971830985915)(332,0.961971830985915)(333,0.961294862772695)(334,0.962702322308233)(335,0.962754743499649)(336,0.962025316455696)(337,0.962754743499649)(338,0.962754743499649)(339,0.964788732394366)(340,0.964109781843772)(341,0.964059196617336)(342,0.964059196617336)(343,0.964059196617336)(344,0.964059196617336)(345,0.963328631875881)(346,0.964059196617336)(347,0.964739069111424)(348,0.966197183098591)(349,0.966197183098591)(350,0.966197183098591)(351,0.966924700914849)(352,0.966244725738396)(353,0.966244725738396)(354,0.966244725738396)(355,0.966924700914849)(356,0.966924700914849)(357,0.966924700914849)(358,0.966878083157153)(359,0.966878083157153)(360,0.967559943582511)(361,0.966197183098591)(362,0.966197183098591)(363,0.96551724137931)(364,0.966197183098591)(365,0.966197183098591)(366,0.966197183098591)(367,0.964838255977496)(368,0.964838255977496)(369,0.964838255977496)(370,0.967559943582511)(371,0.967559943582511)(372,0.967559943582511)(373,0.967559943582511)(374,0.968242766407904)(375,0.968242766407904)(376,0.968926553672316)(377,0.968926553672316)(378,0.968926553672316)(379,0.968926553672316)(380,0.968926553672316)(381,0.968926553672316)(382,0.968926553672316)(383,0.969611307420495)(384,0.969611307420495)(385,0.969654199011997)(386,0.968970380818054)(387,0.968970380818054)(388,0.968970380818054)(389,0.968970380818054)(390,0.969654199011997)(391,0.968926553672316)(392,0.968926553672316)(393,0.968926553672316)(394,0.968926553672316)(395,0.968926553672316)(396,0.968926553672316)(397,0.969654199011997)(398,0.970380818053596)(399,0.969654199011997)(400,0.969654199011997) 
};
\addplot [
color=red,
mark size=0.1pt,
only marks,
mark=*,
mark options={solid,fill=red},
forget plot
]
coordinates{
 (1,0)(2,0)(3,0)(4,0)(5,0)(6,0)(7,0)(8,0)(9,0.155019059720457)(10,0.131736526946108)(11,0.141826923076923)(12,0.0930232558139535)(13,0.217036172695449)(14,0.21784472769409)(15,0.211150652431791)(16,0.697635135135135)(17,0.782978723404255)(18,0.741839762611276)(19,0.777864380358535)(20,0.776729559748428)(21,0.775888717156105)(22,0.809806835066865)(23,0.818045112781955)(24,0.839378238341969)(25,0.842572062084257)(26,0.870748299319728)(27,0.859623733719247)(28,0.864)(29,0.859649122807017)(30,0.855882352941176)(31,0.856309263311451)(32,0.862537764350453)(33,0.875)(34,0.87673231218089)(35,0.876371616678859)(36,0.87762490948588)(37,0.876093294460641)(38,0.876093294460641)(39,0.880410858400587)(40,0.880177514792899)(41,0.885416666666667)(42,0.882440476190476)(43,0.882309400444115)(44,0.882440476190476)(45,0.882265275707899)(46,0.894620486366986)(47,0.894930198383541)(48,0.899925871015567)(49,0.899925871015567)(50,0.904552129221733)(51,0.907095830285296)(52,0.906705539358601)(53,0.908959537572254)(54,0.906976744186046)(55,0.906976744186046)(56,0.905797101449275)(57,0.90658942795076)(58,0.904624277456647)(59,0.905958363244795)(60,0.903971119133574)(61,0.903597122302158)(62,0.901367890568754)(63,0.901734104046243)(64,0.907501820830299)(65,0.906976744186046)(66,0.910688140556369)(67,0.909625275532696)(68,0.907624633431085)(69,0.908155767817781)(70,0.908155767817781)(71,0.904831625183016)(72,0.905632772494513)(73,0.908424908424908)(74,0.908424908424908)(75,0.908558888076079)(76,0.905882352941176)(77,0.905882352941176)(78,0.904129793510324)(79,0.904937361827561)(80,0.905882352941176)(81,0.90521675238795)(82,0.903888481291269)(83,0.903888481291269)(84,0.905494505494505)(85,0.907095830285296)(86,0.907894736842105)(87,0.915697674418605)(88,0.918722786647315)(89,0.921071687183201)(90,0.921071687183201)(91,0.921071687183201)(92,0.923076923076923)(93,0.922407541696882)(94,0.924308588064046)(95,0.923525127458121)(96,0.928208846990573)(97,0.929761042722665)(98,0.929761042722665)(99,0.929761042722665)(100,0.930434782608696)(101,0.929761042722665)(102,0.929659173313996)(103,0.92955700798838)(104,0.929659173313996)(105,0.928985507246377)(106,0.927431059506531)(107,0.926652142338417)(108,0.926652142338417)(109,0.928208846990573)(110,0.928985507246377)(111,0.928985507246377)(112,0.928985507246377)(113,0.93033381712627)(114,0.931009440813362)(115,0.930434782608696)(116,0.930434782608696)(117,0.930434782608696)(118,0.930434782608696)(119,0.930434782608696)(120,0.9288824383164)(121,0.9288824383164)(122,0.933526011560693)(123,0.932754880694143)(124,0.932754880694143)(125,0.930835734870317)(126,0.932851985559567)(127,0.933526011560693)(128,0.93342981186686)(129,0.934782608695652)(130,0.937184115523466)(131,0.936324167872648)(132,0.936324167872648)(133,0.936416184971098)(134,0.936324167872648)(135,0.936324167872648)(136,0.936324167872648)(137,0.936324167872648)(138,0.937002172338885)(139,0.934687953555878)(140,0.936231884057971)(141,0.933914306463326)(142,0.933914306463326)(143,0.933914306463326)(144,0.933914306463326)(145,0.931586608442504)(146,0.933236574746009)(147,0.933914306463326)(148,0.932265112891478)(149,0.930909090909091)(150,0.932265112891478)(151,0.932265112891478)(152,0.932265112891478)(153,0.932265112891478)(154,0.932166301969365)(155,0.931486880466472)(156,0.929041697147037)(157,0.929824561403509)(158,0.929824561403509)(159,0.929824561403509)(160,0.929824561403509)(161,0.930606281957633)(162,0.930606281957633)(163,0.929824561403509)(164,0.928257686676427)(165,0.928257686676427)(166,0.927472527472527)(167,0.927472527472527)(168,0.927472527472527)(169,0.928257686676427)(170,0.928257686676427)(171,0.929041697147037)(172,0.928362573099415)(173,0.929145361577794)(174,0.929145361577794)(175,0.928467153284671)(176,0.928467153284671)(177,0.929248723559446)(178,0.928571428571428)(179,0.928571428571428)(180,0.929248723559446)(181,0.929248723559446)(182,0.929248723559446)(183,0.929248723559446)(184,0.929248723559446)(185,0.929248723559446)(186,0.931586608442504)(187,0.932363636363636)(188,0.932363636363636)(189,0.931586608442504)(190,0.931586608442504)(191,0.931586608442504)(192,0.931586608442504)(193,0.931586608442504)(194,0.931586608442504)(195,0.932265112891478)(196,0.931486880466472)(197,0.928571428571428)(198,0.928571428571428)(199,0.928571428571428)(200,0.929351784413692)(201,0.928571428571428)(202,0.928571428571428)(203,0.928571428571428)(204,0.928571428571428)(205,0.928571428571428)(206,0.93002915451895)(207,0.93002915451895)(208,0.932067202337473)(209,0.932067202337473)(210,0.932067202337473)(211,0.932067202337473)(212,0.932067202337473)(213,0.932067202337473)(214,0.932846715328467)(215,0.932067202337473)(216,0.931286549707602)(217,0.931286549707602)(218,0.931286549707602)(219,0.931286549707602)(220,0.93050475493782)(221,0.931185944363104)(222,0.931185944363104)(223,0.931967812728603)(224,0.932748538011696)(225,0.931967812728603)(226,0.931967812728603)(227,0.931967812728603)(228,0.931967812728603)(229,0.931967812728603)(230,0.931967812728603)(231,0.931967812728603)(232,0.932846715328467)(233,0.932846715328467)(234,0.932846715328467)(235,0.931967812728603)(236,0.931185944363104)(237,0.931967812728603)(238,0.929618768328446)(239,0.93040293040293)(240,0.932748538011696)(241,0.932748538011696)(242,0.932748538011696)(243,0.931967812728603)(244,0.933625091174325)(245,0.933625091174325)(246,0.933625091174325)(247,0.932748538011696)(248,0.93411420204978)(249,0.93411420204978)(250,0.93411420204978)(251,0.934893928310168)(252,0.937226277372263)(253,0.937226277372263)(254,0.937226277372263)(255,0.935672514619883)(256,0.934893928310168)(257,0.934893928310168)(258,0.934893928310168)(259,0.934893928310168)(260,0.935672514619883)(261,0.935672514619883)(262,0.935672514619883)(263,0.935672514619883)(264,0.935672514619883)(265,0.93644996347699)(266,0.934989043097151)(267,0.935766423357664)(268,0.935766423357664)(269,0.935766423357664)(270,0.935766423357664)(271,0.935766423357664)(272,0.935766423357664)(273,0.935766423357664)(274,0.935766423357664)(275,0.935766423357664)(276,0.93731778425656)(277,0.93731778425656)(278,0.93809176984705)(279,0.93731778425656)(280,0.93731778425656)(281,0.936542669584245)(282,0.936635105608157)(283,0.936635105608157)(284,0.935860058309038)(285,0.935178441369264)(286,0.935178441369264)(287,0.935178441369264)(288,0.935178441369264)(289,0.935953420669578)(290,0.935953420669578)(291,0.936727272727273)(292,0.936727272727273)(293,0.937409024745269)(294,0.937409024745269)(295,0.935178441369264)(296,0.935366739288308)(297,0.936635105608157)(298,0.936635105608157)(299,0.937409024745269)(300,0.937409024745269)(301,0.937409024745269)(302,0.937409024745269)(303,0.937409024745269)(304,0.937409024745269)(305,0.937409024745269)(306,0.937409024745269)(307,0.937409024745269)(308,0.93809176984705)(309,0.93809176984705)(310,0.938864628820961)(311,0.93809176984705)(312,0.93809176984705)(313,0.938864628820961)(314,0.938864628820961)(315,0.9375)(316,0.9375)(317,0.9375)(318,0.9375)(319,0.9375)(320,0.9375)(321,0.9375)(322,0.938271604938271)(323,0.939811457577955)(324,0.938271604938271)(325,0.940406976744186)(326,0.939636363636364)(327,0.941176470588235)(328,0.941176470588235)(329,0.940406976744186)(330,0.940406976744186)(331,0.940406976744186)(332,0.940406976744186)(333,0.940406976744186)(334,0.941176470588235)(335,0.941176470588235)(336,0.940406976744186)(337,0.940406976744186)(338,0.938864628820961)(339,0.938864628820961)(340,0.938864628820961)(341,0.938864628820961)(342,0.938864628820961)(343,0.938181818181818)(344,0.938181818181818)(345,0.9375)(346,0.938864628820961)(347,0.938271604938271)(348,0.938953488372093)(349,0.937409024745269)(350,0.937409024745269)(351,0.938953488372093)(352,0.938953488372093)(353,0.938181818181818)(354,0.938181818181818)(355,0.938864628820961)(356,0.938864628820961)(357,0.940406976744186)(358,0.941176470588235)(359,0.941176470588235)(360,0.941176470588235)(361,0.941176470588235)(362,0.941176470588235)(363,0.941176470588235)(364,0.940406976744186)(365,0.939636363636364)(366,0.939636363636364)(367,0.940406976744186)(368,0.940406976744186)(369,0.939636363636364)(370,0.939636363636364)(371,0.939636363636364)(372,0.939636363636364)(373,0.939636363636364)(374,0.939636363636364)(375,0.939636363636364)(376,0.939636363636364)(377,0.939636363636364)(378,0.939636363636364)(379,0.939636363636364)(380,0.938953488372093)(381,0.938181818181818)(382,0.938181818181818)(383,0.938181818181818)(384,0.938181818181818)(385,0.938181818181818)(386,0.938181818181818)(387,0.938181818181818)(388,0.939724037763253)(389,0.940406976744186)(390,0.939724037763253)(391,0.939724037763253)(392,0.939724037763253)(393,0.939724037763253)(394,0.939724037763253)(395,0.939724037763253)(396,0.939724037763253)(397,0.937590711175617)(398,0.937590711175617)(399,0.937590711175617)(400,0.939898624185373) 
};
\addplot [
color=red,
mark size=0.1pt,
only marks,
mark=*,
mark options={solid,fill=red},
forget plot
]
coordinates{
 (1,0)(2,0)(3,0)(4,0)(5,0)(6,0)(7,0)(8,0.0590604026845637)(9,0.0927152317880794)(10,0.198067632850242)(11,0.191387559808612)(12,0.167694204685573)(13,0.647814910025707)(14,0.660869565217391)(15,0.700414937759336)(16,0.708505367464905)(17,0.691126279863481)(18,0.680034873583261)(19,0.837461300309597)(20,0.84526558891455)(21,0.85041761579347)(22,0.846212700841622)(23,0.853231939163498)(24,0.859925093632959)(25,0.857787810383747)(26,0.850870552611658)(27,0.846737481031866)(28,0.847380410022779)(29,0.841135840368381)(30,0.842428900845503)(31,0.853073463268366)(32,0.858440575321726)(33,0.860166288737717)(34,0.861089792785879)(35,0.859101294744859)(36,0.861937452326468)(37,0.883755588673621)(38,0.883789785344189)(39,0.891654465592972)(40,0.891654465592972)(41,0.897058823529412)(42,0.905908096280087)(43,0.904169714703731)(44,0.903367496339678)(45,0.903367496339678)(46,0.903367496339678)(47,0.909225092250922)(48,0.909492273730684)(49,0.907885040530582)(50,0.912152269399707)(51,0.911355311355311)(52,0.910425844346549)(53,0.914536157779401)(54,0.90280777537797)(55,0.913454545454545)(56,0.915451895043732)(57,0.915451895043732)(58,0.91332847778587)(59,0.913202042304887)(60,0.914785142024763)(61,0.919472913616398)(62,0.918681318681319)(63,0.919354838709677)(64,0.919354838709677)(65,0.921149594694178)(66,0.920353982300885)(67,0.921033210332103)(68,0.918639053254438)(69,0.920353982300885)(70,0.920353982300885)(71,0.920353982300885)(72,0.920118343195266)(73,0.922509225092251)(74,0.923416789396171)(75,0.922173274596182)(76,0.922850844966936)(77,0.922623434045689)(78,0.923872875092387)(79,0.924667651403249)(80,0.92079940784604)(81,0.920118343195266)(82,0.919319022945966)(83,0.920118343195266)(84,0.918518518518519)(85,0.919319022945966)(86,0.91899852724595)(87,0.919793966151582)(88,0.922173274596182)(89,0.924208977189109)(90,0.925)(91,0.925)(92,0.923416789396171)(93,0.924208977189109)(94,0.923416789396171)(95,0.925)(96,0.925898752751284)(97,0.92511013215859)(98,0.924320352681852)(99,0.92511013215859)(100,0.923529411764706)(101,0.923529411764706)(102,0.923529411764706)(103,0.923529411764706)(104,0.924431401320616)(105,0.924431401320616)(106,0.925329428989751)(107,0.925329428989751)(108,0.925329428989751)(109,0.922627737226277)(110,0.925329428989751)(111,0.923753665689149)(112,0.923641703377386)(113,0.922964049889949)(114,0.923753665689149)(115,0.922401171303075)(116,0.920379839298758)(117,0.918367346938775)(118,0.916363636363636)(119,0.917698470502549)(120,0.919272727272727)(121,0.918486171761281)(122,0.918486171761281)(123,0.920495988329686)(124,0.921282798833819)(125,0.922627737226277)(126,0.922627737226277)(127,0.922627737226277)(128,0.921840759678597)(129,0.923865300146413)(130,0.924652523774689)(131,0.928467153284671)(132,0.928467153284671)(133,0.928467153284671)(134,0.929824561403509)(135,0.932944606413994)(136,0.931386861313869)(137,0.931386861313869)(138,0.931386861313869)(139,0.932748538011696)(140,0.932748538011696)(141,0.932748538011696)(142,0.933528122717312)(143,0.933528122717312)(144,0.932748538011696)(145,0.931967812728603)(146,0.931967812728603)(147,0.933528122717312)(148,0.931967812728603)(149,0.933528122717312)(150,0.933528122717312)(151,0.933528122717312)(152,0.933528122717312)(153,0.933528122717312)(154,0.934306569343066)(155,0.93411420204978)(156,0.933430870519385)(157,0.934210526315789)(158,0.930300807043287)(159,0.930300807043287)(160,0.930300807043287)(161,0.930300807043287)(162,0.928937728937729)(163,0.930300807043287)(164,0.930300807043287)(165,0.929618768328446)(166,0.93050475493782)(167,0.931286549707602)(168,0.93040293040293)(169,0.929721815519766)(170,0.93040293040293)(171,0.929721815519766)(172,0.929618768328446)(173,0.93040293040293)(174,0.93040293040293)(175,0.929721815519766)(176,0.930300807043287)(177,0.931868131868132)(178,0.933430870519385)(179,0.933430870519385)(180,0.933430870519385)(181,0.933430870519385)(182,0.933430870519385)(183,0.932067202337473)(184,0.932748538011696)(185,0.931185944363104)(186,0.931185944363104)(187,0.931967812728603)(188,0.931967812728603)(189,0.932748538011696)(190,0.932748538011696)(191,0.932748538011696)(192,0.933430870519385)(193,0.931286549707602)(194,0.931286549707602)(195,0.931286549707602)(196,0.931286549707602)(197,0.93050475493782)(198,0.929618768328446)(199,0.929515418502203)(200,0.929411764705882)(201,0.931667891256429)(202,0.928518791451732)(203,0.928518791451732)(204,0.928518791451732)(205,0.929098966026588)(206,0.928307464892831)(207,0.928307464892831)(208,0.928307464892831)(209,0.926829268292683)(210,0.926829268292683)(211,0.926829268292683)(212,0.9240972733972)(213,0.924778761061947)(214,0.920704845814978)(215,0.92138133725202)(216,0.92138133725202)(217,0.92138133725202)(218,0.920704845814978)(219,0.92138133725202)(220,0.922850844966936)(221,0.922850844966936)(222,0.922058823529412)(223,0.922737306843267)(224,0.922737306843267)(225,0.922737306843267)(226,0.922737306843267)(227,0.920353982300885)(228,0.920353982300885)(229,0.920236336779911)(230,0.919438285291944)(231,0.921828908554572)(232,0.921828908554572)(233,0.922623434045689)(234,0.922623434045689)(235,0.923416789396171)(236,0.922623434045689)(237,0.923416789396171)(238,0.922509225092251)(239,0.922509225092251)(240,0.921713441654357)(241,0.921713441654357)(242,0.921713441654357)(243,0.921033210332103)(244,0.921713441654357)(245,0.922509225092251)(246,0.928046989720998)(247,0.928833455612619)(248,0.928046989720998)(249,0.928833455612619)(250,0.928833455612619)(251,0.928728875826598)(252,0.928728875826598)(253,0.928728875826598)(254,0.928728875826598)(255,0.929515418502203)(256,0.929515418502203)(257,0.930300807043287)(258,0.93108504398827)(259,0.93108504398827)(260,0.927578639356254)(261,0.93108504398827)(262,0.931868131868132)(263,0.93108504398827)(264,0.93108504398827)(265,0.930300807043287)(266,0.930300807043287)(267,0.93040293040293)(268,0.93040293040293)(269,0.931185944363104)(270,0.931967812728603)(271,0.931967812728603)(272,0.932650073206442)(273,0.932650073206442)(274,0.932650073206442)(275,0.932650073206442)(276,0.932650073206442)(277,0.932650073206442)(278,0.931967812728603)(279,0.932650073206442)(280,0.934306569343066)(281,0.934306569343066)(282,0.934306569343066)(283,0.935083880379285)(284,0.935083880379285)(285,0.934306569343066)(286,0.933625091174325)(287,0.933625091174325)(288,0.933625091174325)(289,0.933625091174325)(290,0.934306569343066)(291,0.934306569343066)(292,0.934306569343066)(293,0.934306569343066)(294,0.934306569343066)(295,0.934306569343066)(296,0.934306569343066)(297,0.934306569343066)(298,0.938181818181818)(299,0.938181818181818)(300,0.938181818181818)(301,0.937409024745269)(302,0.937409024745269)(303,0.937409024745269)(304,0.937409024745269)(305,0.937409024745269)(306,0.938181818181818)(307,0.936635105608157)(308,0.936635105608157)(309,0.936542669584245)(310,0.93731778425656)(311,0.93731778425656)(312,0.936542669584245)(313,0.936542669584245)(314,0.935766423357664)(315,0.935766423357664)(316,0.935766423357664)(317,0.935766423357664)(318,0.934989043097151)(319,0.935766423357664)(320,0.935766423357664)(321,0.935766423357664)(322,0.934989043097151)(323,0.934989043097151)(324,0.934989043097151)(325,0.934989043097151)(326,0.934989043097151)(327,0.934989043097151)(328,0.934989043097151)(329,0.934989043097151)(330,0.934989043097151)(331,0.933625091174325)(332,0.934989043097151)(333,0.934989043097151)(334,0.934989043097151)(335,0.934989043097151)(336,0.934989043097151)(337,0.934989043097151)(338,0.934989043097151)(339,0.934989043097151)(340,0.934989043097151)(341,0.934989043097151)(342,0.934989043097151)(343,0.934989043097151)(344,0.93731778425656)(345,0.941261783901378)(346,0.94356005788712)(347,0.947368421052631)(348,0.95435092724679)(349,0.958126330731015)(350,0.96045197740113)(351,0.958126330731015)(352,0.966197183098591)(353,0.966197183098591)(354,0.966971187631764)(355,0.966924700914849)(356,0.968421052631579)(357,0.965662228451296)(358,0.966339410939691)(359,0.96629213483146)(360,0.965468639887244)(361,0.963276836158192)(362,0.962491153573956)(363,0.962491153573956)(364,0.965419901199718)(365,0.964059196617336)(366,0.964008468595625)(367,0.967605633802817)(368,0.967605633802817)(369,0.968970380818054)(370,0.96969696969697)(371,0.96969696969697)(372,0.969014084507042)(373,0.969739619985925)(374,0.969014084507042)(375,0.968242766407904)(376,0.970422535211268)(377,0.971147079521464)(378,0.968970380818054)(379,0.968970380818054)(380,0.969654199011997)(381,0.969654199011997)(382,0.969611307420495)(383,0.968926553672316)(384,0.968242766407904)(385,0.966737438075018)(386,0.968926553672316)(387,0.969654199011997)(388,0.971106412966878)(389,0.969654199011997)(390,0.969654199011997)(391,0.971830985915493)(392,0.971830985915493)(393,0.971106412966878)(394,0.971106412966878)(395,0.971106412966878)(396,0.970380818053596)(397,0.969654199011997)(398,0.970380818053596)(399,0.971106412966878)(400,0.968197879858657) 
};
\addplot [
color=red,
mark size=0.1pt,
only marks,
mark=*,
mark options={solid,fill=red},
forget plot
]
coordinates{
 (1,0)(2,0)(3,0)(4,0)(5,0)(6,0)(7,0)(8,0)(9,0.693153000845308)(10,0.719367588932806)(11,0.770124481327801)(12,0.770096463022508)(13,0.757505773672055)(14,0.767315175097276)(15,0.774703557312253)(16,0.785824345146379)(17,0.791893998441153)(18,0.792511700468019)(19,0.803996925441968)(20,0.800618238021638)(21,0.806354009077156)(22,0.806964420893263)(23,0.806964420893263)(24,0.803315749811605)(25,0.785871964679912)(26,0.805744520030234)(27,0.800901577761082)(28,0.811263318112633)(29,0.814024390243902)(30,0.820906994619523)(31,0.82)(32,0.82)(33,0.806015037593985)(34,0.802996254681648)(35,0.802700675168792)(36,0.849826989619377)(37,0.8582995951417)(38,0.857916102841678)(39,0.855363321799308)(40,0.860810810810811)(41,0.863201094391245)(42,0.862210095497954)(43,0.860259032038173)(44,0.863013698630137)(45,0.868184955141477)(46,0.86878453038674)(47,0.863387978142076)(48,0.867950481430536)(49,0.877607788595271)(50,0.877607788595271)(51,0.877607788595271)(52,0.873961218836565)(53,0.866163349347975)(54,0.867453157529493)(55,0.865051903114187)(56,0.864453665283541)(57,0.8646408839779)(58,0.869625520110957)(59,0.866988283942109)(60,0.864010989010989)(61,0.869146005509642)(62,0.8707015130674)(63,0.87130075705437)(64,0.87250172294969)(65,0.869505494505494)(66,0.870588235294117)(67,0.871654083733699)(68,0.878116343490305)(69,0.88268156424581)(70,0.885774351786966)(71,0.885774351786966)(72,0.892655367231638)(73,0.898571428571428)(74,0.894251242015614)(75,0.896159317211949)(76,0.895035460992908)(77,0.893917963224894)(78,0.894550601556971)(79,0.896113074204947)(80,0.893108298171589)(81,0.897130860741777)(82,0.897130860741777)(83,0.897759103641456)(84,0.897759103641456)(85,0.895877009084556)(86,0.895877009084556)(87,0.897759103641456)(88,0.898591549295775)(89,0.900494001411433)(90,0.897381457891012)(91,0.898509581263307)(92,0.901709401709402)(93,0.902352102637206)(94,0.902352102637206)(95,0.909352517985611)(96,0.911319394376352)(97,0.915229885057471)(98,0.910533910533911)(99,0.914450035945363)(100,0.9164265129683)(101,0.914985590778098)(102,0.914203316510454)(103,0.913544668587896)(104,0.913544668587896)(105,0.909222948438635)(106,0.910545454545454)(107,0.91383055756698)(108,0.915645277577505)(109,0.915523465703971)(110,0.916184971098266)(111,0.916063675832127)(112,0.916847433116413)(113,0.92141312184571)(114,0.92129963898917)(115,0.922077922077922)(116,0.922077922077922)(117,0.918646508279338)(118,0.922743682310469)(119,0.921071687183201)(120,0.92040520984081)(121,0.92040520984081)(122,0.916363636363636)(123,0.918840579710145)(124,0.92118582791034)(125,0.92118582791034)(126,0.92118582791034)(127,0.919623461259956)(128,0.918056562726613)(129,0.916241806263656)(130,0.916241806263656)(131,0.917818181818182)(132,0.917818181818182)(133,0.915820029027576)(134,0.916909620991254)(135,0.916241806263656)(136,0.916241806263656)(137,0.917030567685589)(138,0.917271407837445)(139,0.915278783490224)(140,0.912917271407837)(141,0.912917271407837)(142,0.912)(143,0.911078717201166)(144,0.911743253099927)(145,0.912408759124087)(146,0.913075237399562)(147,0.912948061448427)(148,0.912152269399707)(149,0.912152269399707)(150,0.912820512820513)(151,0.913489736070381)(152,0.913489736070381)(153,0.914159941305943)(154,0.915503306392358)(155,0.921033210332103)(156,0.916850625459897)(157,0.91844232182219)(158,0.916850625459897)(159,0.916176470588235)(160,0.917094644167278)(161,0.918800292611558)(162,0.917216117216117)(163,0.918681318681319)(164,0.916422287390029)(165,0.917216117216117)(166,0.917888563049853)(167,0.917216117216117)(168,0.917216117216117)(169,0.916299559471366)(170,0.918322295805739)(171,0.918322295805739)(172,0.917647058823529)(173,0.917525773195876)(174,0.918322295805739)(175,0.91899852724595)(176,0.918879056047198)(177,0.918879056047198)(178,0.919793966151582)(179,0.920588235294117)(180,0.920704845814978)(181,0.921496698459281)(182,0.920704845814978)(183,0.920704845814978)(184,0.920704845814978)(185,0.920704845814978)(186,0.920704845814978)(187,0.920704845814978)(188,0.92675852066715)(189,0.927849927849928)(190,0.920883820384889)(191,0.924623115577889)(192,0.930635838150289)(193,0.93294881038212)(194,0.929862617498192)(195,0.935437589670014)(196,0.926970354302241)(197,0.927536231884058)(198,0.926864590876176)(199,0.927536231884058)(200,0.925631768953068)(201,0.925631768953068)(202,0.926300578034682)(203,0.925416364952933)(204,0.925416364952933)(205,0.926193921852388)(206,0.926193921852388)(207,0.926193921852388)(208,0.925416364952933)(209,0.926193921852388)(210,0.926864590876176)(211,0.928208846990573)(212,0.928208846990573)(213,0.928208846990573)(214,0.927536231884058)(215,0.92955700798838)(216,0.930434782608696)(217,0.930434782608696)(218,0.92955700798838)(219,0.932657494569153)(220,0.931884057971014)(221,0.931884057971014)(222,0.931884057971014)(223,0.931109499637418)(224,0.928104575163399)(225,0.9288824383164)(226,0.929659173313996)(227,0.931109499637418)(228,0.931109499637418)(229,0.931884057971014)(230,0.931982633863965)(231,0.929963898916967)(232,0.931506849315068)(233,0.930935251798561)(234,0.929597701149425)(235,0.929597701149425)(236,0.93371757925072)(237,0.93371757925072)(238,0.93371757925072)(239,0.931605471562275)(240,0.931605471562275)(241,0.929496402877698)(242,0.928725701943844)(243,0.93016558675306)(244,0.93016558675306)(245,0.93016558675306)(246,0.93016558675306)(247,0.929394812680115)(248,0.929190751445086)(249,0.932568149210904)(250,0.93103448275862)(251,0.933333333333333)(252,0.934097421203438)(253,0.934097421203438)(254,0.934953538241601)(255,0.934953538241601)(256,0.939415538132573)(257,0.939415538132573)(258,0.94017094017094)(259,0.940925266903915)(260,0.942430703624733)(261,0.941678520625889)(262,0.941009239516702)(263,0.941761363636364)(264,0.941009239516702)(265,0.941761363636364)(266,0.947666195190948)(267,0.949152542372881)(268,0.954992967651195)(269,0.94982332155477)(270,0.94982332155477)(271,0.950564971751412)(272,0.951305575158786)(273,0.950564971751412)(274,0.952112676056338)(275,0.954321855235418)(276,0.956582633053221)(277,0.956582633053221)(278,0.953910614525139)(279,0.954703832752613)(280,0.955369595536959)(281,0.956036287508723)(282,0.956036287508723)(283,0.95676429567643)(284,0.954102920723227)(285,0.952050034746352)(286,0.957431960921144)(287,0.957431960921144)(288,0.957372466806429)(289,0.957372466806429)(290,0.958041958041958)(291,0.958041958041958)(292,0.958770090845562)(293,0.958041958041958)(294,0.958041958041958)(295,0.958041958041958)(296,0.958712386284115)(297,0.958654519971969)(298,0.956460674157303)(299,0.957983193277311)(300,0.958712386284115)(301,0.959326788218794)(302,0.959326788218794)(303,0.96)(304,0.96078431372549)(305,0.96078431372549)(306,0.962237762237762)(307,0.962237762237762)(308,0.962237762237762)(309,0.960893854748603)(310,0.960893854748603)(311,0.959553695955369)(312,0.958217270194986)(313,0.957550452331246)(314,0.95615866388309)(315,0.957491289198606)(316,0.957491289198606)(317,0.958217270194986)(318,0.958217270194986)(319,0.958885017421603)(320,0.958827634333566)(321,0.958100558659218)(322,0.958100558659218)(323,0.958100558659218)(324,0.958100558659218)(325,0.958100558659218)(326,0.958827634333566)(327,0.958827634333566)(328,0.958827634333566)(329,0.958827634333566)(330,0.958827634333566)(331,0.958827634333566)(332,0.960278745644599)(333,0.959553695955369)(334,0.959553695955369)(335,0.960223307745987)(336,0.958770090845562)(337,0.960111966410077)(338,0.959440559440559)(339,0.959440559440559)(340,0.959440559440559)(341,0.958770090845562)(342,0.958770090845562)(343,0.958770090845562)(344,0.960278745644599)(345,0.96100278551532)(346,0.96100278551532)(347,0.96105702364395)(348,0.959722222222222)(349,0.96105702364395)(350,0.96105702364395)(351,0.961725817675713)(352,0.961672473867596)(353,0.961672473867596)(354,0.962395543175487)(355,0.962395543175487)(356,0.961672473867596)(357,0.961725817675713)(358,0.961672473867596)(359,0.963687150837989)(360,0.963636363636364)(361,0.964310706787963)(362,0.964310706787963)(363,0.964985994397759)(364,0.964985994397759)(365,0.964260686755431)(366,0.964260686755431)(367,0.963585434173669)(368,0.964260686755431)(369,0.964260686755431)(370,0.964260686755431)(371,0.964260686755431)(372,0.964260686755431)(373,0.964260686755431)(374,0.963534361851332)(375,0.964985994397759)(376,0.964985994397759)(377,0.965710286913926)(378,0.966386554621849)(379,0.966386554621849)(380,0.967109867039888)(381,0.966433566433566)(382,0.96575821104123)(383,0.966433566433566)(384,0.966433566433566)(385,0.966433566433566)(386,0.966433566433566)(387,0.967832167832168)(388,0.966433566433566)(389,0.966433566433566)(390,0.966433566433566)(391,0.966433566433566)(392,0.966433566433566)(393,0.964360587002096)(394,0.964360587002096)(395,0.966480446927374)(396,0.96575821104123)(397,0.966480446927374)(398,0.966480446927374)(399,0.966480446927374)(400,0.967201674808095) 
};
\addplot [
color=red,
mark size=0.1pt,
only marks,
mark=*,
mark options={solid,fill=red},
forget plot
]
coordinates{
 (1,0)(2,0)(3,0)(4,0)(5,0)(6,0)(7,0)(8,0)(9,0)(10,0)(11,0)(12,0)(13,0)(14,0.0802139037433155)(15,0.113842173350582)(16,0.119897959183673)(17,0.179802955665025)(18,0.198787878787879)(19,0.233766233766234)(20,0.261168384879725)(21,0.263279445727483)(22,0.258741258741259)(23,0.263888888888889)(24,0.263888888888889)(25,0.256111757857974)(26,0.256410256410256)(27,0.256410256410256)(28,0.252955082742317)(29,0.254156769596199)(30,0.255017709563164)(31,0.254156769596199)(32,0.257988165680473)(33,0.261792452830189)(34,0.267605633802817)(35,0.261792452830189)(36,0.261792452830189)(37,0.261484098939929)(38,0.258907363420428)(39,0.26241134751773)(40,0.260355029585799)(41,0.266509433962264)(42,0.265882352941176)(43,0.248506571087216)(44,0.244897959183673)(45,0.244311377245509)(46,0.248210023866348)(47,0.246411483253588)(48,0.26303317535545)(49,0.264462809917355)(50,0.252080856123662)(51,0.255675029868578)(52,0.255675029868578)(53,0.253588516746411)(54,0.249101796407186)(55,0.257449344457688)(56,0.259215219976219)(57,0.254761904761905)(58,0.251497005988024)(59,0.251497005988024)(60,0.255675029868578)(61,0.255675029868578)(62,0.255675029868578)(63,0.255675029868578)(64,0.261904761904762)(65,0.255675029868578)(66,0.253588516746411)(67,0.249400479616307)(68,0.249400479616307)(69,0.248803827751196)(70,0.247596153846154)(71,0.247596153846154)(72,0.247596153846154)(73,0.249699879951981)(74,0.249699879951981)(75,0.253892215568862)(76,0.253892215568862)(77,0.253892215568862)(78,0.254196642685851)(79,0.254196642685851)(80,0.254196642685851)(81,0.252100840336134)(82,0.252100840336134)(83,0.256287425149701)(84,0.256287425149701)(85,0.254196642685851)(86,0.255980861244019)(87,0.255980861244019)(88,0.258064516129032)(89,0.255980861244019)(90,0.255980861244019)(91,0.253892215568862)(92,0.253892215568862)(93,0.258064516129032)(94,0.262216924910608)(95,0.255980861244019)(96,0.258064516129032)(97,0.260143198090692)(98,0.256287425149701)(99,0.256287425149701)(100,0.256287425149701)(101,0.258373205741627)(102,0.260454002389486)(103,0.258373205741627)(104,0.258682634730539)(105,0.257521058965102)(106,0.257521058965102)(107,0.257521058965102)(108,0.253317249698432)(109,0.253317249698432)(110,0.253317249698432)(111,0.251207729468599)(112,0.253317249698432)(113,0.253317249698432)(114,0.246973365617433)(115,0.246973365617433)(116,0.246973365617433)(117,0.255421686746988)(118,0.255421686746988)(119,0.255421686746988)(120,0.255421686746988)(121,0.253317249698432)(122,0.253317249698432)(123,0.255421686746988)(124,0.253317249698432)(125,0.253317249698432)(126,0.253317249698432)(127,0.253317249698432)(128,0.253317249698432)(129,0.253317249698432)(130,0.253317249698432)(131,0.253317249698432)(132,0.253317249698432)(133,0.253317249698432)(134,0.253317249698432)(135,0.255421686746988)(136,0.249394673123487)(137,0.249394673123487)(138,0.249394673123487)(139,0.251511487303507)(140,0.249394673123487)(141,0.249394673123487)(142,0.249394673123487)(143,0.249394673123487)(144,0.249394673123487)(145,0.249394673123487)(146,0.253317249698432)(147,0.259615384615385)(148,0.259615384615385)(149,0.259615384615385)(150,0.261704681872749)(151,0.261704681872749)(152,0.261704681872749)(153,0.261704681872749)(154,0.261704681872749)(155,0.261704681872749)(156,0.261704681872749)(157,0.261704681872749)(158,0.26378896882494)(159,0.261704681872749)(160,0.265868263473054)(161,0.265868263473054)(162,0.265868263473054)(163,0.265868263473054)(164,0.265868263473054)(165,0.265868263473054)(166,0.265868263473054)(167,0.265868263473054)(168,0.265868263473054)(169,0.266506602641056)(170,0.268585131894484)(171,0.268585131894484)(172,0.268585131894484)(173,0.266506602641056)(174,0.266506602641056)(175,0.266506602641056)(176,0.266506602641056)(177,0.266506602641056)(178,0.268585131894484)(179,0.268585131894484)(180,0.268585131894484)(181,0.268585131894484)(182,0.268585131894484)(183,0.268585131894484)(184,0.268585131894484)(185,0.268585131894484)(186,0.268585131894484)(187,0.268585131894484)(188,0.266506602641056)(189,0.266506602641056)(190,0.266506602641056)(191,0.266506602641056)(192,0.266506602641056)(193,0.266506602641056)(194,0.276849642004773)(195,0.27479091995221)(196,0.313450292397661)(197,0.28944246737841)(198,0.305522914218566)(199,0.326388888888889)(200,0.337543053960964)(201,0.331797235023041)(202,0.333333333333333)(203,0.335246842709529)(204,0.335246842709529)(205,0.332179930795848)(206,0.336018411967779)(207,0.336018411967779)(208,0.334101382488479)(209,0.337931034482759)(210,0.337931034482759)(211,0.339060710194731)(212,0.34135166093929)(213,0.337155963302752)(214,0.337543053960964)(215,0.333716915995397)(216,0.333716915995397)(217,0.333716915995397)(218,0.34135166093929)(219,0.345142857142857)(220,0.34703196347032)(221,0.348916761687571)(222,0.34703196347032)(223,0.346636259977195)(224,0.34703196347032)(225,0.348916761687571)(226,0.348916761687571)(227,0.350797266514806)(228,0.350797266514806)(229,0.350797266514806)(230,0.351473922902494)(231,0.351872871736663)(232,0.351872871736663)(233,0.352673492605233)(234,0.352673492605233)(235,0.352272727272727)(236,0.352272727272727)(237,0.35)(238,0.351197263397947)(239,0.351197263397947)(240,0.350797266514806)(241,0.349315068493151)(242,0.349315068493151)(243,0.349315068493151)(244,0.345933562428408)(245,0.34443168771527)(246,0.34443168771527)(247,0.34443168771527)(248,0.345537757437071)(249,0.345142857142857)(250,0.345142857142857)(251,0.34443168771527)(252,0.348224513172967)(253,0.353881278538813)(254,0.353881278538813)(255,0.353881278538813)(256,0.353881278538813)(257,0.353881278538813)(258,0.353477765108324)(259,0.353477765108324)(260,0.353477765108324)(261,0.359090909090909)(262,0.359090909090909)(263,0.359090909090909)(264,0.359090909090909)(265,0.360953461975028)(266,0.36281179138322)(267,0.360953461975028)(268,0.360953461975028)(269,0.360953461975028)(270,0.360953461975028)(271,0.360953461975028)(272,0.360953461975028)(273,0.36281179138322)(274,0.36281179138322)(275,0.36281179138322)(276,0.36281179138322)(277,0.36281179138322)(278,0.36281179138322)(279,0.361581920903955)(280,0.358916478555305)(281,0.360766629086809)(282,0.36281179138322)(283,0.36281179138322)(284,0.36281179138322)(285,0.36281179138322)(286,0.36281179138322)(287,0.364665911664779)(288,0.364665911664779)(289,0.364665911664779)(290,0.364665911664779)(291,0.36281179138322)(292,0.36281179138322)(293,0.360953461975028)(294,0.36281179138322)(295,0.36281179138322)(296,0.36281179138322)(297,0.36281179138322)(298,0.36281179138322)(299,0.36281179138322)(300,0.36281179138322)(301,0.36281179138322)(302,0.36281179138322)(303,0.36281179138322)(304,0.36281179138322)(305,0.366515837104072)(306,0.364665911664779)(307,0.364665911664779)(308,0.366515837104072)(309,0.366515837104072)(310,0.366515837104072)(311,0.366515837104072)(312,0.366515837104072)(313,0.366515837104072)(314,0.366515837104072)(315,0.366515837104072)(316,0.366515837104072)(317,0.366515837104072)(318,0.366515837104072)(319,0.366515837104072)(320,0.366515837104072)(321,0.366515837104072)(322,0.366515837104072)(323,0.366515837104072)(324,0.366515837104072)(325,0.364665911664779)(326,0.364665911664779)(327,0.364665911664779)(328,0.364665911664779)(329,0.364665911664779)(330,0.364665911664779)(331,0.364665911664779)(332,0.366515837104072)(333,0.366515837104072)(334,0.366515837104072)(335,0.366515837104072)(336,0.366515837104072)(337,0.364665911664779)(338,0.364665911664779)(339,0.36281179138322)(340,0.36281179138322)(341,0.366515837104072)(342,0.366515837104072)(343,0.364665911664779)(344,0.364665911664779)(345,0.36281179138322)(346,0.366515837104072)(347,0.366515837104072)(348,0.368361581920904)(349,0.368361581920904)(350,0.368361581920904)(351,0.368361581920904)(352,0.368361581920904)(353,0.368361581920904)(354,0.37020316027088)(355,0.37020316027088)(356,0.37020316027088)(357,0.368361581920904)(358,0.368361581920904)(359,0.368361581920904)(360,0.367945823927765)(361,0.367945823927765)(362,0.367945823927765)(363,0.367945823927765)(364,0.367945823927765)(365,0.367945823927765)(366,0.367945823927765)(367,0.367945823927765)(368,0.368361581920904)(369,0.368361581920904)(370,0.366515837104072)(371,0.366515837104072)(372,0.366515837104072)(373,0.366515837104072)(374,0.366515837104072)(375,0.366515837104072)(376,0.366515837104072)(377,0.366515837104072)(378,0.366515837104072)(379,0.366515837104072)(380,0.366101694915254)(381,0.366101694915254)(382,0.366101694915254)(383,0.366101694915254)(384,0.36568848758465)(385,0.36568848758465)(386,0.36568848758465)(387,0.36568848758465)(388,0.36568848758465)(389,0.36568848758465)(390,0.36568848758465)(391,0.36568848758465)(392,0.36568848758465)(393,0.36568848758465)(394,0.36568848758465)(395,0.36568848758465)(396,0.36568848758465)(397,0.36568848758465)(398,0.36568848758465)(399,0.36568848758465)(400,0.36568848758465) 
};
\addplot [
color=red,
mark size=0.1pt,
only marks,
mark=*,
mark options={solid,fill=red},
forget plot
]
coordinates{
 (1,0)(2,0)(3,0)(4,0)(5,0)(6,0)(7,0.173053152039555)(8,0.163892445582586)(9,0.147286821705426)(10,0.154440154440154)(11,0.15681233933162)(12,0.188679245283019)(13,0.184810126582278)(14,0.180203045685279)(15,0.206467661691542)(16,0.213135068153655)(17,0.213135068153655)(18,0.228501228501228)(19,0.232843137254902)(20,0.249394673123487)(21,0.249093107617896)(22,0.251511487303507)(23,0.250602409638554)(24,0.258992805755396)(25,0.257211538461538)(26,0.256902761104442)(27,0.257521058965102)(28,0.257211538461538)(29,0.258454106280193)(30,0.266187050359712)(31,0.265232974910394)(32,0.265232974910394)(33,0.26555023923445)(34,0.265232974910394)(35,0.262019230769231)(36,0.263473053892216)(37,0.261704681872749)(38,0.261704681872749)(39,0.26378896882494)(40,0.261704681872749)(41,0.261704681872749)(42,0.261704681872749)(43,0.261704681872749)(44,0.262019230769231)(45,0.262650602409639)(46,0.264105642256903)(47,0.259927797833935)(48,0.264105642256903)(49,0.258454106280193)(50,0.262650602409639)(51,0.262650602409639)(52,0.262019230769231)(53,0.261704681872749)(54,0.261704681872749)(55,0.261704681872749)(56,0.262334536702768)(57,0.261704681872749)(58,0.261704681872749)(59,0.26555023923445)(60,0.26555023923445)(61,0.26555023923445)(62,0.265232974910394)(63,0.265232974910394)(64,0.265232974910394)(65,0.265232974910394)(66,0.26491646778043)(67,0.26491646778043)(68,0.26491646778043)(69,0.265232974910394)(70,0.265232974910394)(71,0.265868263473054)(72,0.265868263473054)(73,0.265868263473054)(74,0.265868263473054)(75,0.270011947431302)(76,0.270011947431302)(77,0.270011947431302)(78,0.272401433691756)(79,0.278571428571429)(80,0.280618311533888)(81,0.280618311533888)(82,0.280618311533888)(83,0.274463007159904)(84,0.274463007159904)(85,0.415015641293013)(86,0.420274551214361)(87,0.445396145610278)(88,0.648506151142355)(89,0.719798657718121)(90,0.71662360034453)(91,0.713164777680906)(92,0.845553822152886)(93,0.882352941176471)(94,0.908823529411765)(95,0.906020558002937)(96,0.904970760233918)(97,0.90546528803545)(98,0.904970760233918)(99,0.912255257432922)(100,0.911743253099927)(101,0.906453952139231)(102,0.906453952139231)(103,0.904865649963689)(104,0.90566037735849)(105,0.910948905109489)(106,0.911871813546977)(107,0.913235294117647)(108,0.913235294117647)(109,0.911225238444607)(110,0.912408759124087)(111,0.912408759124087)(112,0.92118582791034)(113,0.921526277897768)(114,0.925287356321839)(115,0.925394548063127)(116,0.926270579813887)(117,0.922857142857143)(118,0.924177396280401)(119,0.921863799283154)(120,0.924393723252496)(121,0.925053533190578)(122,0.924068767908309)(123,0.924731182795699)(124,0.926829268292683)(125,0.926164874551971)(126,0.922419928825623)(127,0.923734853884533)(128,0.923186344238976)(129,0.92264017033357)(130,0.926585887384177)(131,0.92790863668808)(132,0.926829268292683)(133,0.923407301360057)(134,0.922302158273381)(135,0.923631123919308)(136,0.923631123919308)(137,0.923631123919308)(138,0.925179856115108)(139,0.922413793103448)(140,0.923850574712644)(141,0.923850574712644)(142,0.923187365398421)(143,0.924731182795699)(144,0.925394548063127)(145,0.926058865757358)(146,0.925394548063127)(147,0.925394548063127)(148,0.926058865757358)(149,0.926164874551971)(150,0.926058865757358)(151,0.925179856115108)(152,0.919770773638968)(153,0.923076923076923)(154,0.923076923076923)(155,0.923076923076923)(156,0.923741007194245)(157,0.922966162706983)(158,0.922966162706983)(159,0.922966162706983)(160,0.923631123919308)(161,0.92451473759885)(162,0.932094353109364)(163,0.932094353109364)(164,0.932094353109364)(165,0.931330472103004)(166,0.930664760543245)(167,0.931428571428571)(168,0.930763740185582)(169,0.931428571428571)(170,0.930763740185582)(171,0.931428571428571)(172,0.932761087267525)(173,0.933428775948461)(174,0.933428775948461)(175,0.933428775948461)(176,0.933428775948461)(177,0.932568149210904)(178,0.934579439252336)(179,0.939306358381503)(180,0.939306358381503)(181,0.938539407086045)(182,0.938539407086045)(183,0.938539407086045)(184,0.940665701881331)(185,0.939898624185373)(186,0.939898624185373)(187,0.939130434782609)(188,0.9378612716763)(189,0.9378612716763)(190,0.937093275488069)(191,0.937093275488069)(192,0.937274693583273)(193,0.937274693583273)(194,0.936599423631124)(195,0.936599423631124)(196,0.937365010799136)(197,0.937274693583273)(198,0.937274693583273)(199,0.935832732516222)(200,0.935158501440922)(201,0.934390771449171)(202,0.935064935064935)(203,0.934296028880866)(204,0.935647143890094)(205,0.935647143890094)(206,0.937771345875543)(207,0.937771345875543)(208,0.937771345875543)(209,0.937771345875543)(210,0.937771345875543)(211,0.938450398262129)(212,0.938450398262129)(213,0.938450398262129)(214,0.93768115942029)(215,0.938450398262129)(216,0.93768115942029)(217,0.936910804931109)(218,0.938539407086045)(219,0.937002172338885)(220,0.937002172338885)(221,0.937002172338885)(222,0.938539407086045)(223,0.937184115523466)(224,0.937184115523466)(225,0.937184115523466)(226,0.936507936507936)(227,0.937184115523466)(228,0.937950937950938)(229,0.938716654650324)(230,0.937365010799136)(231,0.936599423631124)(232,0.938129496402878)(233,0.938129496402878)(234,0.937365010799136)(235,0.937365010799136)(236,0.939655172413793)(237,0.939655172413793)(238,0.939655172413793)(239,0.938892882818116)(240,0.939568345323741)(241,0.939655172413793)(242,0.939655172413793)(243,0.939655172413793)(244,0.939655172413793)(245,0.940330697340043)(246,0.939655172413793)(247,0.941007194244604)(248,0.939568345323741)(249,0.939568345323741)(250,0.939568345323741)(251,0.939568345323741)(252,0.940922190201729)(253,0.940922190201729)(254,0.942446043165467)(255,0.941007194244604)(256,0.941007194244604)(257,0.94176851186197)(258,0.943965517241379)(259,0.941260744985673)(260,0.943287867910983)(261,0.943287867910983)(262,0.943884892086331)(263,0.944564434845212)(264,0.943042537851478)(265,0.943042537851478)(266,0.943042537851478)(267,0.943042537851478)(268,0.943042537851478)(269,0.943722943722944)(270,0.943722943722944)(271,0.943722943722944)(272,0.943722943722944)(273,0.944404332129964)(274,0.94364161849711)(275,0.94364161849711)(276,0.94364161849711)(277,0.94364161849711)(278,0.94364161849711)(279,0.94364161849711)(280,0.94364161849711)(281,0.94364161849711)(282,0.942877801879971)(283,0.942877801879971)(284,0.942877801879971)(285,0.942877801879971)(286,0.942877801879971)(287,0.94364161849711)(288,0.942112879884226)(289,0.942795076031861)(290,0.942795076031861)(291,0.943478260869565)(292,0.943478260869565)(293,0.943478260869565)(294,0.9427121102248)(295,0.946531791907514)(296,0.946531791907514)(297,0.945770065075922)(298,0.946454413892909)(299,0.944323933477946)(300,0.944323933477946)(301,0.945086705202312)(302,0.946608946608946)(303,0.945848375451263)(304,0.946608946608946)(305,0.945086705202312)(306,0.945086705202312)(307,0.945848375451263)(308,0.945848375451263)(309,0.945086705202312)(310,0.945086705202312)(311,0.945086705202312)(312,0.944323933477946)(313,0.945086705202312)(314,0.945086705202312)(315,0.942960288808664)(316,0.942960288808664)(317,0.942196531791907)(318,0.942196531791907)(319,0.942960288808664)(320,0.941600576784427)(321,0.941600576784427)(322,0.941600576784427)(323,0.941600576784427)(324,0.941600576784427)(325,0.941600576784427)(326,0.941600576784427)(327,0.945770065075922)(328,0.945770065075922)(329,0.946531791907514)(330,0.947976878612717)(331,0.947216196673897)(332,0.947216196673897)(333,0.947216196673897)(334,0.947216196673897)(335,0.947216196673897)(336,0.946454413892909)(337,0.946454413892909)(338,0.946454413892909)(339,0.946454413892909)(340,0.946454413892909)(341,0.946454413892909)(342,0.946454413892909)(343,0.947901591895803)(344,0.947901591895803)(345,0.947139753801593)(346,0.947139753801593)(347,0.947139753801593)(348,0.947139753801593)(349,0.947139753801593)(350,0.947139753801593)(351,0.947139753801593)(352,0.947139753801593)(353,0.947826086956522)(354,0.947139753801593)(355,0.946376811594203)(356,0.946376811594203)(357,0.946376811594203)(358,0.946376811594203)(359,0.946376811594203)(360,0.946376811594203)(361,0.946376811594203)(362,0.947139753801593)(363,0.947139753801593)(364,0.947139753801593)(365,0.947139753801593)(366,0.946454413892909)(367,0.947139753801593)(368,0.947139753801593)(369,0.947139753801593)(370,0.947139753801593)(371,0.947139753801593)(372,0.945086705202312)(373,0.945086705202312)(374,0.945086705202312)(375,0.945086705202312)(376,0.944404332129964)(377,0.945086705202312)(378,0.943722943722944)(379,0.943722943722944)(380,0.943722943722944)(381,0.942960288808664)(382,0.942960288808664)(383,0.94364161849711)(384,0.94364161849711)(385,0.94364161849711)(386,0.944323933477946)(387,0.944323933477946)(388,0.945086705202312)(389,0.945848375451263)(390,0.945848375451263)(391,0.945848375451263)(392,0.945165945165945)(393,0.945165945165945)(394,0.945165945165945)(395,0.945165945165945)(396,0.945165945165945)(397,0.944484498918529)(398,0.944484498918529)(399,0.944484498918529)(400,0.944484498918529) 
};
\addplot [
color=red,
mark size=0.1pt,
only marks,
mark=*,
mark options={solid,fill=red},
forget plot
]
coordinates{
 (1,0)(2,0)(3,0)(4,0)(5,0)(6,0)(7,0.65139116202946)(8,0.684528954191875)(9,0.701858736059479)(10,0.685982772122161)(11,0.742424242424242)(12,0.758051197357556)(13,0.756756756756757)(14,0.763877381938691)(15,0.757998359310911)(16,0.757998359310911)(17,0.767726161369193)(18,0.767726161369193)(19,0.764279967819791)(20,0.766773162939297)(21,0.770452740270056)(22,0.772545889864326)(23,0.770087509944312)(24,0.771678599840891)(25,0.774603174603175)(26,0.791243158717748)(27,0.797202797202797)(28,0.798136645962733)(29,0.798136645962733)(30,0.796267496111975)(31,0.798129384255651)(32,0.798751950078003)(33,0.793725490196078)(34,0.792156862745098)(35,0.792156862745098)(36,0.797814207650273)(37,0.800936768149883)(38,0.797498045347928)(39,0.797498045347928)(40,0.799375487900078)(41,0.8)(42,0.8)(43,0.804669260700389)(44,0.806526806526806)(45,0.806526806526806)(46,0.806526806526806)(47,0.805900621118012)(48,0.807126258714175)(49,0.807126258714175)(50,0.807453416149068)(51,0.807453416149068)(52,0.807782101167315)(53,0.8109375)(54,0.817322834645669)(55,0.819905213270142)(56,0.822047244094488)(57,0.824823252160251)(58,0.830696202531646)(59,0.830963665086888)(60,0.833597464342314)(61,0.832672482157018)(62,0.835443037974684)(63,0.835844567803331)(64,0.837430610626487)(65,0.837430610626487)(66,0.834394904458599)(67,0.833731105807478)(68,0.830940988835726)(69,0.830940988835726)(70,0.830940988835726)(71,0.831604150039904)(72,0.834935897435897)(73,0.834935897435897)(74,0.833333333333333)(75,0.835868694955965)(76,0.834935897435897)(77,0.834935897435897)(78,0.834935897435897)(79,0.835868694955965)(80,0.835868694955965)(81,0.839228295819936)(82,0.839228295819936)(83,0.838033843674456)(84,0.837096774193548)(85,0.838969404186795)(86,0.838969404186795)(87,0.838969404186795)(88,0.838969404186795)(89,0.839903459372486)(90,0.839903459372486)(91,0.839903459372486)(92,0.839903459372486)(93,0.842696629213483)(94,0.844551282051282)(95,0.844551282051282)(96,0.84664536741214)(97,0.84664536741214)(98,0.845723421262989)(99,0.843875100080064)(100,0.845723421262989)(101,0.84664536741214)(102,0.84664536741214)(103,0.842948717948718)(104,0.842948717948718)(105,0.842948717948718)(106,0.842948717948718)(107,0.842948717948718)(108,0.842948717948718)(109,0.842948717948718)(110,0.842273819055244)(111,0.842273819055244)(112,0.842948717948718)(113,0.845047923322684)(114,0.8448)(115,0.8448)(116,0.8448)(117,0.8448)(118,0.8448)(119,0.8448)(120,0.8448)(121,0.8448)(122,0.8448)(123,0.8448)(124,0.8448)(125,0.8448)(126,0.8448)(127,0.8448)(128,0.8448)(129,0.8448)(130,0.8448)(131,0.845476381104884)(132,0.843624699278268)(133,0.843624699278268)(134,0.843624699278268)(135,0.843624699278268)(136,0.842020850040096)(137,0.842273819055244)(138,0.842273819055244)(139,0.841346153846154)(140,0.841346153846154)(141,0.844124700239808)(142,0.845047923322684)(143,0.845969672785315)(144,0.84664536741214)(145,0.84664536741214)(146,0.84756584197925)(147,0.848484848484848)(148,0.848484848484848)(149,0.84756584197925)(150,0.848484848484848)(151,0.848484848484848)(152,0.848484848484848)(153,0.848484848484848)(154,0.848484848484848)(155,0.848484848484848)(156,0.848484848484848)(157,0.848484848484848)(158,0.848484848484848)(159,0.848484848484848)(160,0.848484848484848)(161,0.850079744816587)(162,0.849402390438247)(163,0.849642004773269)(164,0.849642004773269)(165,0.849642004773269)(166,0.849642004773269)(167,0.849642004773269)(168,0.849642004773269)(169,0.849642004773269)(170,0.849642004773269)(171,0.848726114649681)(172,0.848726114649681)(173,0.848726114649681)(174,0.848726114649681)(175,0.848726114649681)(176,0.848726114649681)(177,0.848726114649681)(178,0.849642004773269)(179,0.847808764940239)(180,0.847808764940239)(181,0.848726114649681)(182,0.847808764940239)(183,0.847808764940239)(184,0.847808764940239)(185,0.847808764940239)(186,0.848726114649681)(187,0.849642004773269)(188,0.848726114649681)(189,0.848726114649681)(190,0.849642004773269)(191,0.849642004773269)(192,0.849642004773269)(193,0.850556438791733)(194,0.851469420174742)(195,0.851469420174742)(196,0.851469420174742)(197,0.849642004773269)(198,0.849642004773269)(199,0.849642004773269)(200,0.848726114649681)(201,0.850556438791733)(202,0.850556438791733)(203,0.850556438791733)(204,0.850556438791733)(205,0.850556438791733)(206,0.850556438791733)(207,0.850793650793651)(208,0.850793650793651)(209,0.850793650793651)(210,0.850793650793651)(211,0.850793650793651)(212,0.850793650793651)(213,0.850793650793651)(214,0.848966613672496)(215,0.848966613672496)(216,0.848966613672496)(217,0.848966613672496)(218,0.847133757961783)(219,0.84805091487669)(220,0.84805091487669)(221,0.84805091487669)(222,0.84805091487669)(223,0.849880857823669)(224,0.848966613672496)(225,0.849880857823669)(226,0.849880857823669)(227,0.850793650793651)(228,0.84805091487669)(229,0.849880857823669)(230,0.849206349206349)(231,0.850118953211737)(232,0.850118953211737)(233,0.848772763262074)(234,0.849445324881141)(235,0.849445324881141)(236,0.848772763262074)(237,0.848772763262074)(238,0.848772763262074)(239,0.847860538827258)(240,0.847860538827258)(241,0.847189231987332)(242,0.848101265822785)(243,0.848772763262074)(244,0.847430830039526)(245,0.847430830039526)(246,0.847430830039526)(247,0.847430830039526)(248,0.847430830039526)(249,0.847430830039526)(250,0.847430830039526)(251,0.848101265822785)(252,0.848101265822785)(253,0.848772763262074)(254,0.848772763262074)(255,0.848772763262074)(256,0.847860538827258)(257,0.847860538827258)(258,0.847860538827258)(259,0.847189231987332)(260,0.846518987341772)(261,0.846946867565424)(262,0.846946867565424)(263,0.846946867565424)(264,0.847619047619047)(265,0.846946867565424)(266,0.846946867565424)(267,0.846946867565424)(268,0.846946867565424)(269,0.847619047619047)(270,0.847619047619047)(271,0.847619047619047)(272,0.848292295472597)(273,0.851030110935024)(274,0.85193982581156)(275,0.852848101265823)(276,0.852848101265823)(277,0.852848101265823)(278,0.852848101265823)(279,0.852848101265823)(280,0.853523357086302)(281,0.853523357086302)(282,0.854199683042789)(283,0.854877081681205)(284,0.854199683042789)(285,0.854199683042789)(286,0.854199683042789)(287,0.854199683042789)(288,0.854199683042789)(289,0.854199683042789)(290,0.854199683042789)(291,0.854199683042789)(292,0.854199683042789)(293,0.854199683042789)(294,0.854199683042789)(295,0.854199683042789)(296,0.853291038858049)(297,0.852380952380952)(298,0.852380952380952)(299,0.851704996034893)(300,0.851704996034893)(301,0.853291038858049)(302,0.853291038858049)(303,0.853291038858049)(304,0.852614896988906)(305,0.852614896988906)(306,0.852614896988906)(307,0.852614896988906)(308,0.852614896988906)(309,0.852380952380952)(310,0.852380952380952)(311,0.852380952380952)(312,0.852380952380952)(313,0.855106888361045)(314,0.855106888361045)(315,0.855106888361045)(316,0.855106888361045)(317,0.855106888361045)(318,0.855106888361045)(319,0.855106888361045)(320,0.855106888361045)(321,0.855106888361045)(322,0.855106888361045)(323,0.855106888361045)(324,0.855106888361045)(325,0.855106888361045)(326,0.855106888361045)(327,0.855106888361045)(328,0.855106888361045)(329,0.854199683042789)(330,0.855555555555555)(331,0.855555555555555)(332,0.856235107227959)(333,0.856235107227959)(334,0.856235107227959)(335,0.856235107227959)(336,0.856235107227959)(337,0.857142857142857)(338,0.857142857142857)(339,0.856235107227959)(340,0.856235107227959)(341,0.856235107227959)(342,0.856235107227959)(343,0.856235107227959)(344,0.856235107227959)(345,0.856235107227959)(346,0.856235107227959)(347,0.856235107227959)(348,0.856235107227959)(349,0.855325914149443)(350,0.855325914149443)(351,0.855325914149443)(352,0.855325914149443)(353,0.856235107227959)(354,0.856235107227959)(355,0.855325914149443)(356,0.855325914149443)(357,0.855325914149443)(358,0.855325914149443)(359,0.855325914149443)(360,0.855325914149443)(361,0.855325914149443)(362,0.855325914149443)(363,0.855325914149443)(364,0.856235107227959)(365,0.856235107227959)(366,0.857142857142857)(367,0.857142857142857)(368,0.857142857142857)(369,0.857142857142857)(370,0.856463124504361)(371,0.856463124504361)(372,0.857369255150555)(373,0.857369255150555)(374,0.857369255150555)(375,0.857369255150555)(376,0.857369255150555)(377,0.857369255150555)(378,0.857369255150555)(379,0.857369255150555)(380,0.857369255150555)(381,0.857369255150555)(382,0.857369255150555)(383,0.857369255150555)(384,0.857369255150555)(385,0.857369255150555)(386,0.857369255150555)(387,0.857369255150555)(388,0.857369255150555)(389,0.857369255150555)(390,0.857369255150555)(391,0.857369255150555)(392,0.857369255150555)(393,0.857369255150555)(394,0.857369255150555)(395,0.857369255150555)(396,0.857369255150555)(397,0.857369255150555)(398,0.857369255150555)(399,0.857369255150555)(400,0.857369255150555) 
};
\addplot [
color=red,
mark size=0.1pt,
only marks,
mark=*,
mark options={solid,fill=red},
forget plot
]
coordinates{
 (1,0)(2,0)(3,0)(4,0)(5,0)(6,0)(7,0)(8,0.0828877005347593)(9,0.12303664921466)(10,0.056910569105691)(11,0.151133501259446)(12,0.124191461836999)(13,0.16060225846926)(14,0.156010230179028)(15,0.183168316831683)(16,0.178391959798995)(17,0.223039215686274)(18,0.214285714285714)(19,0.218137254901961)(20,0.215757575757576)(21,0.241338112305854)(22,0.24256837098692)(23,0.244655581947743)(24,0.243786982248521)(25,0.241050119331742)(26,0.241626794258373)(27,0.240762812872467)(28,0.235860409145608)(29,0.23642943305187)(30,0.235722964763062)(31,0.233009708737864)(32,0.23728813559322)(33,0.240963855421687)(34,0.239419588875453)(35,0.239419588875453)(36,0.242788461538462)(37,0.253285543608124)(38,0.259215219976219)(39,0.269185360094451)(40,0.261282660332542)(41,0.259523809523809)(42,0.254196642685851)(43,0.254196642685851)(44,0.256348246674728)(45,0.256348246674728)(46,0.268585131894484)(47,0.268585131894484)(48,0.272727272727273)(49,0.271634615384615)(50,0.271634615384615)(51,0.271634615384615)(52,0.271634615384615)(53,0.272401433691756)(54,0.270334928229665)(55,0.270334928229665)(56,0.272401433691756)(57,0.273709483793517)(58,0.273709483793517)(59,0.273709483793517)(60,0.273709483793517)(61,0.268115942028985)(62,0.272289156626506)(63,0.270205066344994)(64,0.270205066344994)(65,0.270205066344994)(66,0.270205066344994)(67,0.272289156626506)(68,0.272289156626506)(69,0.271961492178099)(70,0.269879518072289)(71,0.270205066344994)(72,0.278511404561825)(73,0.274368231046931)(74,0.276442307692308)(75,0.276442307692308)(76,0.276442307692308)(77,0.275779376498801)(78,0.27511961722488)(79,0.275449101796407)(80,0.271308523409364)(81,0.271308523409364)(82,0.272727272727273)(83,0.274463007159904)(84,0.274463007159904)(85,0.273809523809524)(86,0.270011947431302)(87,0.269689737470167)(88,0.270011947431302)(89,0.269689737470167)(90,0.269689737470167)(91,0.269689737470167)(92,0.269689737470167)(93,0.269689737470167)(94,0.269689737470167)(95,0.270011947431302)(96,0.270011947431302)(97,0.269689737470167)(98,0.271428571428571)(99,0.271428571428571)(100,0.271428571428571)(101,0.271105826397146)(102,0.271105826397146)(103,0.269047619047619)(104,0.268727705112961)(105,0.268727705112961)(106,0.268727705112961)(107,0.266666666666667)(108,0.264600715137068)(109,0.26491646778043)(110,0.269047619047619)(111,0.269047619047619)(112,0.26491646778043)(113,0.263157894736842)(114,0.263157894736842)(115,0.261077844311377)(116,0.263157894736842)(117,0.26491646778043)(118,0.266984505363528)(119,0.26491646778043)(120,0.26491646778043)(121,0.267303102625298)(122,0.267303102625298)(123,0.263157894736842)(124,0.265232974910394)(125,0.269368295589988)(126,0.269368295589988)(127,0.269368295589988)(128,0.269368295589988)(129,0.263157894736842)(130,0.263157894736842)(131,0.263157894736842)(132,0.265232974910394)(133,0.267303102625298)(134,0.267303102625298)(135,0.267303102625298)(136,0.266984505363528)(137,0.266984505363528)(138,0.267303102625298)(139,0.267303102625298)(140,0.269047619047619)(141,0.269047619047619)(142,0.269047619047619)(143,0.269047619047619)(144,0.269047619047619)(145,0.269047619047619)(146,0.269368295589988)(147,0.269368295589988)(148,0.269368295589988)(149,0.269689737470167)(150,0.269689737470167)(151,0.269689737470167)(152,0.269689737470167)(153,0.269689737470167)(154,0.273809523809524)(155,0.288075560802834)(156,0.286052009456265)(157,0.286052009456265)(158,0.290094339622641)(159,0.290094339622641)(160,0.290094339622641)(161,0.290094339622641)(162,0.289752650176678)(163,0.289752650176678)(164,0.285714285714286)(165,0.285714285714286)(166,0.287735849056604)(167,0.285714285714286)(168,0.285714285714286)(169,0.285714285714286)(170,0.290094339622641)(171,0.289411764705882)(172,0.292397660818713)(173,0.289752650176678)(174,0.301754385964912)(175,0.305717619603267)(176,0.311627906976744)(177,0.323325635103926)(178,0.309662398137369)(179,0.322209436133487)(180,0.32258064516129)(181,0.321100917431193)(182,0.324571428571428)(183,0.324571428571428)(184,0.329142857142857)(185,0.324942791762014)(186,0.326436781609195)(187,0.326812428078251)(188,0.330275229357798)(189,0.330275229357798)(190,0.328358208955224)(191,0.332187857961054)(192,0.332187857961054)(193,0.332187857961054)(194,0.330654420206659)(195,0.330654420206659)(196,0.326812428078251)(197,0.330654420206659)(198,0.330654420206659)(199,0.340182648401826)(200,0.342075256556442)(201,0.343963553530752)(202,0.343963553530752)(203,0.34584755403868)(204,0.355203619909502)(205,0.351473922902494)(206,0.358916478555305)(207,0.360766629086809)(208,0.357062146892655)(209,0.357062146892655)(210,0.348122866894198)(211,0.349602724177071)(212,0.353340883352208)(213,0.354802259887006)(214,0.355203619909502)(215,0.355203619909502)(216,0.355203619909502)(217,0.355203619909502)(218,0.351872871736663)(219,0.351872871736663)(220,0.352272727272727)(221,0.356009070294785)(222,0.351872871736663)(223,0.351872871736663)(224,0.348122866894198)(225,0.350398179749716)(226,0.348519362186788)(227,0.350398179749716)(228,0.348916761687571)(229,0.34703196347032)(230,0.34703196347032)(231,0.348122866894198)(232,0.34733257661748)(233,0.34733257661748)(234,0.348122866894198)(235,0.348519362186788)(236,0.348519362186788)(237,0.348519362186788)(238,0.347727272727273)(239,0.348122866894198)(240,0.348122866894198)(241,0.347727272727273)(242,0.347727272727273)(243,0.349602724177071)(244,0.35)(245,0.35)(246,0.346636259977195)(247,0.346636259977195)(248,0.346636259977195)(249,0.34624145785877)(250,0.34624145785877)(251,0.34624145785877)(252,0.350398179749716)(253,0.350398179749716)(254,0.350398179749716)(255,0.350398179749716)(256,0.350398179749716)(257,0.348519362186788)(258,0.348122866894198)(259,0.351872871736663)(260,0.351473922902494)(261,0.351473922902494)(262,0.351473922902494)(263,0.351473922902494)(264,0.351473922902494)(265,0.351872871736663)(266,0.352272727272727)(267,0.352272727272727)(268,0.351075877689694)(269,0.351075877689694)(270,0.351075877689694)(271,0.351075877689694)(272,0.350678733031674)(273,0.350678733031674)(274,0.350678733031674)(275,0.351075877689694)(276,0.351075877689694)(277,0.351473922902494)(278,0.350678733031674)(279,0.350678733031674)(280,0.350678733031674)(281,0.350678733031674)(282,0.350678733031674)(283,0.346938775510204)(284,0.346938775510204)(285,0.346153846153846)(286,0.346545866364666)(287,0.346545866364666)(288,0.346545866364666)(289,0.346545866364666)(290,0.346545866364666)(291,0.346545866364666)(292,0.344671201814059)(293,0.344671201814059)(294,0.34881087202718)(295,0.34881087202718)(296,0.34881087202718)(297,0.34881087202718)(298,0.346938775510204)(299,0.34881087202718)(300,0.34881087202718)(301,0.34881087202718)(302,0.346938775510204)(303,0.344671201814059)(304,0.346938775510204)(305,0.346938775510204)(306,0.34733257661748)(307,0.34733257661748)(308,0.34733257661748)(309,0.347727272727273)(310,0.34584755403868)(311,0.345454545454545)(312,0.345454545454545)(313,0.345454545454545)(314,0.345454545454545)(315,0.345454545454545)(316,0.343572241183163)(317,0.343572241183163)(318,0.343963553530752)(319,0.343572241183163)(320,0.343572241183163)(321,0.343572241183163)(322,0.343572241183163)(323,0.343572241183163)(324,0.344355758266819)(325,0.343963553530752)(326,0.34584755403868)(327,0.34584755403868)(328,0.34584755403868)(329,0.34584755403868)(330,0.34624145785877)(331,0.348122866894198)(332,0.348122866894198)(333,0.34624145785877)(334,0.34584755403868)(335,0.34584755403868)(336,0.34584755403868)(337,0.345454545454545)(338,0.345454545454545)(339,0.346938775510204)(340,0.348022598870056)(341,0.348022598870056)(342,0.348022598870056)(343,0.348022598870056)(344,0.348022598870056)(345,0.348022598870056)(346,0.348022598870056)(347,0.348022598870056)(348,0.348022598870056)(349,0.349887133182844)(350,0.349887133182844)(351,0.349887133182844)(352,0.349887133182844)(353,0.350282485875706)(354,0.350282485875706)(355,0.350678733031674)(356,0.350678733031674)(357,0.350678733031674)(358,0.350678733031674)(359,0.350678733031674)(360,0.350678733031674)(361,0.354401805869074)(362,0.354002254791432)(363,0.353603603603603)(364,0.353205849268841)(365,0.353205849268841)(366,0.353205849268841)(367,0.351747463359639)(368,0.349887133182844)(369,0.352144469525959)(370,0.351747463359639)(371,0.352144469525959)(372,0.352144469525959)(373,0.352144469525959)(374,0.352144469525959)(375,0.352144469525959)(376,0.352144469525959)(377,0.352144469525959)(378,0.351747463359639)(379,0.352144469525959)(380,0.352941176470588)(381,0.352941176470588)(382,0.352542372881356)(383,0.352542372881356)(384,0.352542372881356)(385,0.352144469525959)(386,0.352542372881356)(387,0.352144469525959)(388,0.352144469525959)(389,0.352144469525959)(390,0.352144469525959)(391,0.352144469525959)(392,0.352144469525959)(393,0.352542372881356)(394,0.352941176470588)(395,0.352542372881356)(396,0.352542372881356)(397,0.352542372881356)(398,0.352542372881356)(399,0.352542372881356)(400,0.352542372881356) 
};
\addplot [
color=red,
mark size=0.1pt,
only marks,
mark=*,
mark options={solid,fill=red},
forget plot
]
coordinates{
 (1,0)(2,0)(3,0.0677083333333333)(4,0.68114926419061)(5,0.671988388969521)(6,0.743630573248408)(7,0.745283018867924)(8,0.763285024154589)(9,0.800650935720097)(10,0.793755135579293)(11,0.808475957620212)(12,0.805825242718447)(13,0.81629392971246)(14,0.82753164556962)(15,0.818759936406995)(16,0.820553359683794)(17,0.819047619047619)(18,0.818109610802224)(19,0.818109610802224)(20,0.8224)(21,0.82051282051282)(22,0.82051282051282)(23,0.821917808219178)(24,0.82475884244373)(25,0.824476650563607)(26,0.828087167070218)(27,0.82914979757085)(28,0.829268292682927)(29,0.833602584814216)(30,0.83454398708636)(31,0.824104234527687)(32,0.825345247766044)(33,0.825732899022801)(34,0.826298701298701)(35,0.827642276422764)(36,0.823817292006525)(37,0.824489795918367)(38,0.824489795918367)(39,0.823145884270579)(40,0.824489795918367)(41,0.826688364524003)(42,0.825448613376835)(43,0.82859463850528)(44,0.834276475343573)(45,0.835218093699515)(46,0.835218093699515)(47,0.836158192090396)(48,0.833333333333333)(49,0.832528180354267)(50,0.834276475343573)(51,0.835218093699515)(52,0.835218093699515)(53,0.836158192090396)(54,0.840836012861736)(55,0.842190016103059)(56,0.841512469831054)(57,0.841512469831054)(58,0.840579710144927)(59,0.845228548516439)(60,0.847077662129704)(61,0.847077662129704)(62,0.845295055821372)(63,0.845295055821372)(64,0.845969672785315)(65,0.845295055821372)(66,0.843949044585987)(67,0.845295055821372)(68,0.844868735083532)(69,0.846703733121525)(70,0.847619047619047)(71,0.844936708860759)(72,0.846031746031746)(73,0.846946867565424)(74,0.846946867565424)(75,0.846946867565424)(76,0.847808764940239)(77,0.847808764940239)(78,0.844124700239808)(79,0.843949044585987)(80,0.844197138314785)(81,0.844197138314785)(82,0.845295055821372)(83,0.8448)(84,0.845476381104884)(85,0.8464)(86,0.845723421262989)(87,0.8464)(88,0.847808764940239)(89,0.84805091487669)(90,0.847808764940239)(91,0.84756584197925)(92,0.84664536741214)(93,0.84664536741214)(94,0.84664536741214)(95,0.842948717948718)(96,0.84664536741214)(97,0.84664536741214)(98,0.848242811501597)(99,0.8464)(100,0.844551282051282)(101,0.845476381104884)(102,0.845476381104884)(103,0.845476381104884)(104,0.843624699278268)(105,0.844551282051282)(106,0.8464)(107,0.8464)(108,0.8464)(109,0.8464)(110,0.8464)(111,0.8464)(112,0.8464)(113,0.844551282051282)(114,0.841767068273092)(115,0.843624699278268)(116,0.843624699278268)(117,0.843624699278268)(118,0.842696629213483)(119,0.843624699278268)(120,0.842948717948718)(121,0.842948717948718)(122,0.842948717948718)(123,0.844551282051282)(124,0.845228548516439)(125,0.844124700239808)(126,0.844124700239808)(127,0.845047923322684)(128,0.845047923322684)(129,0.843700159489633)(130,0.843700159489633)(131,0.843700159489633)(132,0.843700159489633)(133,0.843700159489633)(134,0.843700159489633)(135,0.844621513944223)(136,0.844621513944223)(137,0.844621513944223)(138,0.844621513944223)(139,0.844621513944223)(140,0.844621513944223)(141,0.844621513944223)(142,0.843700159489633)(143,0.843700159489633)(144,0.843027888446215)(145,0.843027888446215)(146,0.843027888446215)(147,0.843027888446215)(148,0.843027888446215)(149,0.843027888446215)(150,0.843700159489633)(151,0.842356687898089)(152,0.842356687898089)(153,0.843027888446215)(154,0.843027888446215)(155,0.843027888446215)(156,0.843027888446215)(157,0.843027888446215)(158,0.843027888446215)(159,0.843027888446215)(160,0.843027888446215)(161,0.843027888446215)(162,0.843027888446215)(163,0.843027888446215)(164,0.843027888446215)(165,0.843027888446215)(166,0.843027888446215)(167,0.843027888446215)(168,0.842105263157895)(169,0.843949044585987)(170,0.843949044585987)(171,0.843949044585987)(172,0.843949044585987)(173,0.843027888446215)(174,0.842105263157895)(175,0.842105263157895)(176,0.842105263157895)(177,0.842105263157895)(178,0.842105263157895)(179,0.842105263157895)(180,0.842105263157895)(181,0.842105263157895)(182,0.842105263157895)(183,0.842105263157895)(184,0.842105263157895)(185,0.841181165203511)(186,0.841181165203511)(187,0.842105263157895)(188,0.842105263157895)(189,0.842105263157895)(190,0.842105263157895)(191,0.842105263157895)(192,0.843027888446215)(193,0.843027888446215)(194,0.843949044585987)(195,0.843277645186953)(196,0.843277645186953)(197,0.843277645186953)(198,0.843277645186953)(199,0.843277645186953)(200,0.843277645186953)(201,0.844197138314785)(202,0.844197138314785)(203,0.843949044585987)(204,0.843949044585987)(205,0.843277645186953)(206,0.843277645186953)(207,0.843949044585987)(208,0.844868735083532)(209,0.844868735083532)(210,0.844868735083532)(211,0.844197138314785)(212,0.844868735083532)(213,0.844868735083532)(214,0.844868735083532)(215,0.844868735083532)(216,0.844868735083532)(217,0.844868735083532)(218,0.844868735083532)(219,0.845786963434022)(220,0.845786963434022)(221,0.846459824980111)(222,0.84805091487669)(223,0.84805091487669)(224,0.84805091487669)(225,0.847376788553259)(226,0.847376788553259)(227,0.847376788553259)(228,0.847133757961783)(229,0.84805091487669)(230,0.84805091487669)(231,0.84805091487669)(232,0.84805091487669)(233,0.84805091487669)(234,0.847376788553259)(235,0.847376788553259)(236,0.847376788553259)(237,0.846703733121525)(238,0.846703733121525)(239,0.846703733121525)(240,0.846703733121525)(241,0.846703733121525)(242,0.846703733121525)(243,0.846703733121525)(244,0.846703733121525)(245,0.846703733121525)(246,0.846703733121525)(247,0.846703733121525)(248,0.846031746031746)(249,0.846031746031746)(250,0.846703733121525)(251,0.846703733121525)(252,0.846703733121525)(253,0.846703733121525)(254,0.846031746031746)(255,0.845360824742268)(256,0.845360824742268)(257,0.845360824742268)(258,0.844444444444444)(259,0.845786963434022)(260,0.845786963434022)(261,0.844868735083532)(262,0.844868735083532)(263,0.845541401273885)(264,0.845541401273885)(265,0.846459824980111)(266,0.848292295472597)(267,0.846459824980111)(268,0.846459824980111)(269,0.846459824980111)(270,0.846459824980111)(271,0.846459824980111)(272,0.846459824980111)(273,0.846459824980111)(274,0.846459824980111)(275,0.846459824980111)(276,0.846459824980111)(277,0.846459824980111)(278,0.846459824980111)(279,0.846459824980111)(280,0.846459824980111)(281,0.846459824980111)(282,0.846459824980111)(283,0.847376788553259)(284,0.845541401273885)(285,0.845541401273885)(286,0.847133757961783)(287,0.847133757961783)(288,0.847133757961783)(289,0.847133757961783)(290,0.847133757961783)(291,0.84805091487669)(292,0.848726114649681)(293,0.848726114649681)(294,0.848726114649681)(295,0.84805091487669)(296,0.84805091487669)(297,0.84805091487669)(298,0.848966613672496)(299,0.849642004773269)(300,0.849642004773269)(301,0.849642004773269)(302,0.849642004773269)(303,0.849642004773269)(304,0.850556438791733)(305,0.848726114649681)(306,0.849402390438247)(307,0.849402390438247)(308,0.849402390438247)(309,0.846273291925466)(310,0.855335968379446)(311,0.855590062111801)(312,0.849056603773585)(313,0.85031847133758)(314,0.862096138691883)(315,0.854877081681205)(316,0.860299921073402)(317,0.860299921073402)(318,0.862992125984252)(319,0.867450980392157)(320,0.865625)(321,0.864949258391881)(322,0.863886703383163)(323,0.873151750972762)(324,0.876648564778898)(325,0.89025326170376)(326,0.878787878787879)(327,0.884322678843227)(328,0.881588999236058)(329,0.879756468797565)(330,0.881588999236058)(331,0.881407804131599)(332,0.876712328767123)(333,0.880061115355233)(334,0.880061115355233)(335,0.880916030534351)(336,0.880916030534351)(337,0.881588999236058)(338,0.882758620689655)(339,0.882758620689655)(340,0.882758620689655)(341,0.884113584036838)(342,0.883435582822086)(343,0.885321100917431)(344,0.886172650878533)(345,0.885670731707317)(346,0.888212927756654)(347,0.887366818873668)(348,0.885670731707317)(349,0.885670731707317)(350,0.885670731707317)(351,0.886346300533943)(352,0.885670731707317)(353,0.885670731707317)(354,0.885670731707317)(355,0.885670731707317)(356,0.885670731707317)(357,0.885670731707317)(358,0.885670731707317)(359,0.887700534759358)(360,0.887700534759358)(361,0.887700534759358)(362,0.887700534759358)(363,0.887700534759358)(364,0.888379204892966)(365,0.889228418640183)(366,0.888379204892966)(367,0.888379204892966)(368,0.888379204892966)(369,0.890421455938697)(370,0.890421455938697)(371,0.892119357306809)(372,0.892284186401833)(373,0.891437308868501)(374,0.891437308868501)(375,0.890756302521008)(376,0.890756302521008)(377,0.888888888888889)(378,0.886503067484662)(379,0.884792626728111)(380,0.883935434281322)(381,0.885648503453568)(382,0.88735632183908)(383,0.886503067484662)(384,0.891104294478527)(385,0.891104294478527)(386,0.888547271329746)(387,0.888547271329746)(388,0.888547271329746)(389,0.889570552147239)(390,0.890421455938697)(391,0.890421455938697)(392,0.890589135424636)(393,0.889058913542463)(394,0.889908256880734)(395,0.889739663093415)(396,0.889739663093415)(397,0.889739663093415)(398,0.889739663093415)(399,0.889739663093415)(400,0.889739663093415) 
};
\addplot [
color=red,
mark size=0.1pt,
only marks,
mark=*,
mark options={solid,fill=red},
forget plot
]
coordinates{
 (1,0)(2,0)(3,0)(4,0)(5,0)(6,0.706624605678233)(7,0.702314445331205)(8,0.671641791044776)(9,0.7)(10,0.703801945181255)(11,0.743221690590112)(12,0.771404821280133)(13,0.779605263157894)(14,0.805488297013721)(15,0.822115384615385)(16,0.821742605915268)(17,0.827697262479871)(18,0.821052631578947)(19,0.824193548387097)(20,0.824193548387097)(21,0.825141015310234)(22,0.82475884244373)(23,0.823529411764706)(24,0.823151125401929)(25,0.821256038647343)(26,0.824096385542169)(27,0.827862289831865)(28,0.831210191082802)(29,0.832278481012658)(30,0.834258524980174)(31,0.837688044338875)(32,0.838351822503962)(33,0.837688044338875)(34,0.837688044338875)(35,0.836879432624113)(36,0.835043409629045)(37,0.833201581027668)(38,0.834123222748815)(39,0.835043409629045)(40,0.835443037974684)(41,0.835844567803331)(42,0.835182250396196)(43,0.837025316455696)(44,0.836248012718601)(45,0.838095238095238)(46,0.837430610626487)(47,0.837430610626487)(48,0.837430610626487)(49,0.836767036450079)(50,0.836767036450079)(51,0.836507936507936)(52,0.836507936507936)(53,0.837172359015091)(54,0.838504375497215)(55,0.839171974522293)(56,0.840095465393795)(57,0.841938046068308)(58,0.841017488076311)(59,0.837025316455696)(60,0.835703001579779)(61,0.837282780410742)(62,0.837282780410742)(63,0.838200473559589)(64,0.837539432176656)(65,0.838455476753349)(66,0.837539432176656)(67,0.834782608695652)(68,0.835703001579779)(69,0.840349483717236)(70,0.842188739095955)(71,0.842188739095955)(72,0.839427662957075)(73,0.840764331210191)(74,0.841181165203511)(75,0.84185303514377)(76,0.841181165203511)(77,0.842777334397446)(78,0.841434262948207)(79,0.840510366826156)(80,0.840927258193445)(81,0.84185303514377)(82,0.84185303514377)(83,0.84185303514377)(84,0.84185303514377)(85,0.84185303514377)(86,0.84185303514377)(87,0.84185303514377)(88,0.842525979216627)(89,0.842525979216627)(90,0.842525979216627)(91,0.842525979216627)(92,0.842525979216627)(93,0.842525979216627)(94,0.843875100080064)(95,0.843875100080064)(96,0.845047923322684)(97,0.843875100080064)(98,0.844124700239808)(99,0.844124700239808)(100,0.845969672785315)(101,0.8432)(102,0.8432)(103,0.8432)(104,0.84185303514377)(105,0.842105263157895)(106,0.842356687898089)(107,0.843027888446215)(108,0.842356687898089)(109,0.842356687898089)(110,0.842356687898089)(111,0.842356687898089)(112,0.842356687898089)(113,0.843700159489633)(114,0.843700159489633)(115,0.843700159489633)(116,0.842356687898089)(117,0.842356687898089)(118,0.841017488076311)(119,0.841017488076311)(120,0.841017488076311)(121,0.841017488076311)(122,0.841938046068308)(123,0.841938046068308)(124,0.842857142857143)(125,0.842857142857143)(126,0.841938046068308)(127,0.842607313195548)(128,0.843277645186953)(129,0.842607313195548)(130,0.844197138314785)(131,0.842607313195548)(132,0.84352660841938)(133,0.84352660841938)(134,0.84352660841938)(135,0.84352660841938)(136,0.844444444444444)(137,0.844444444444444)(138,0.845360824742268)(139,0.845360824742268)(140,0.846275752773376)(141,0.846275752773376)(142,0.846275752773376)(143,0.846275752773376)(144,0.846275752773376)(145,0.846275752773376)(146,0.846275752773376)(147,0.846275752773376)(148,0.846275752773376)(149,0.846275752773376)(150,0.846275752773376)(151,0.846275752773376)(152,0.846275752773376)(153,0.846275752773376)(154,0.846275752773376)(155,0.846275752773376)(156,0.846275752773376)(157,0.846275752773376)(158,0.846275752773376)(159,0.846275752773376)(160,0.845605700712589)(161,0.846275752773376)(162,0.846275752773376)(163,0.846275752773376)(164,0.846275752773376)(165,0.846275752773376)(166,0.846275752773376)(167,0.846275752773376)(168,0.846275752773376)(169,0.847619047619047)(170,0.847619047619047)(171,0.847619047619047)(172,0.84853291038858)(173,0.84853291038858)(174,0.849206349206349)(175,0.84853291038858)(176,0.849206349206349)(177,0.84853291038858)(178,0.84853291038858)(179,0.84853291038858)(180,0.84853291038858)(181,0.84853291038858)(182,0.847619047619047)(183,0.84853291038858)(184,0.846518987341772)(185,0.847189231987332)(186,0.846518987341772)(187,0.846518987341772)(188,0.846518987341772)(189,0.846518987341772)(190,0.846518987341772)(191,0.846761453396524)(192,0.846761453396524)(193,0.846761453396524)(194,0.846761453396524)(195,0.847189231987332)(196,0.846518987341772)(197,0.846518987341772)(198,0.846518987341772)(199,0.846518987341772)(200,0.846518987341772)(201,0.846518987341772)(202,0.846518987341772)(203,0.846518987341772)(204,0.846518987341772)(205,0.845181674565561)(206,0.845181674565561)(207,0.845181674565561)(208,0.844514601420679)(209,0.844514601420679)(210,0.845181674565561)(211,0.845181674565561)(212,0.845849802371541)(213,0.845849802371541)(214,0.845849802371541)(215,0.845181674565561)(216,0.845181674565561)(217,0.845181674565561)(218,0.845181674565561)(219,0.845181674565561)(220,0.845849802371541)(221,0.845849802371541)(222,0.845849802371541)(223,0.846518987341772)(224,0.846518987341772)(225,0.848101265822785)(226,0.848101265822785)(227,0.848101265822785)(228,0.848101265822785)(229,0.848101265822785)(230,0.848101265822785)(231,0.848101265822785)(232,0.848101265822785)(233,0.848101265822785)(234,0.848101265822785)(235,0.847430830039526)(236,0.847430830039526)(237,0.847430830039526)(238,0.847430830039526)(239,0.847430830039526)(240,0.847430830039526)(241,0.846761453396524)(242,0.846761453396524)(243,0.846761453396524)(244,0.846761453396524)(245,0.846761453396524)(246,0.846761453396524)(247,0.846761453396524)(248,0.846761453396524)(249,0.846761453396524)(250,0.848818897637795)(251,0.848818897637795)(252,0.848580441640378)(253,0.848580441640378)(254,0.848580441640378)(255,0.848580441640378)(256,0.848580441640378)(257,0.848580441640378)(258,0.848580441640378)(259,0.848580441640378)(260,0.848580441640378)(261,0.848580441640378)(262,0.848580441640378)(263,0.847911741528763)(264,0.847911741528763)(265,0.847911741528763)(266,0.847911741528763)(267,0.847911741528763)(268,0.847911741528763)(269,0.847911741528763)(270,0.847911741528763)(271,0.847911741528763)(272,0.848580441640378)(273,0.848580441640378)(274,0.848580441640378)(275,0.848580441640378)(276,0.847911741528763)(277,0.847911741528763)(278,0.847911741528763)(279,0.847911741528763)(280,0.847911741528763)(281,0.847911741528763)(282,0.847911741528763)(283,0.847911741528763)(284,0.847671665351223)(285,0.847671665351223)(286,0.848341232227488)(287,0.848341232227488)(288,0.847671665351223)(289,0.848341232227488)(290,0.848341232227488)(291,0.848341232227488)(292,0.848341232227488)(293,0.848341232227488)(294,0.848341232227488)(295,0.848341232227488)(296,0.848341232227488)(297,0.848341232227488)(298,0.848341232227488)(299,0.848341232227488)(300,0.848341232227488)(301,0.848341232227488)(302,0.848341232227488)(303,0.848341232227488)(304,0.848341232227488)(305,0.848341232227488)(306,0.848341232227488)(307,0.848341232227488)(308,0.849250197316495)(309,0.849250197316495)(310,0.849250197316495)(311,0.849250197316495)(312,0.849250197316495)(313,0.849250197316495)(314,0.849250197316495)(315,0.849250197316495)(316,0.849250197316495)(317,0.849250197316495)(318,0.849250197316495)(319,0.848580441640378)(320,0.848580441640378)(321,0.848580441640378)(322,0.848580441640378)(323,0.848580441640378)(324,0.848580441640378)(325,0.848580441640378)(326,0.848580441640378)(327,0.848580441640378)(328,0.849250197316495)(329,0.849921011058452)(330,0.850828729281768)(331,0.850828729281768)(332,0.850828729281768)(333,0.850828729281768)(334,0.850828729281768)(335,0.850157728706624)(336,0.850157728706624)(337,0.850157728706624)(338,0.849487785657998)(339,0.849487785657998)(340,0.849487785657998)(341,0.849487785657998)(342,0.848818897637795)(343,0.848818897637795)(344,0.848818897637795)(345,0.848818897637795)(346,0.848818897637795)(347,0.848818897637795)(348,0.849487785657998)(349,0.849487785657998)(350,0.849487785657998)(351,0.849487785657998)(352,0.849487785657998)(353,0.849487785657998)(354,0.849487785657998)(355,0.848818897637795)(356,0.848818897637795)(357,0.848818897637795)(358,0.848818897637795)(359,0.849487785657998)(360,0.849487785657998)(361,0.848818897637795)(362,0.848818897637795)(363,0.848818897637795)(364,0.848818897637795)(365,0.848818897637795)(366,0.848818897637795)(367,0.848818897637795)(368,0.848151062155783)(369,0.848151062155783)(370,0.848151062155783)(371,0.848151062155783)(372,0.84748427672956)(373,0.848151062155783)(374,0.84748427672956)(375,0.848151062155783)(376,0.848151062155783)(377,0.848151062155783)(378,0.848151062155783)(379,0.848151062155783)(380,0.848151062155783)(381,0.848151062155783)(382,0.848818897637795)(383,0.848818897637795)(384,0.848818897637795)(385,0.848818897637795)(386,0.848818897637795)(387,0.848818897637795)(388,0.848818897637795)(389,0.848818897637795)(390,0.848151062155783)(391,0.848151062155783)(392,0.848151062155783)(393,0.848151062155783)(394,0.848151062155783)(395,0.848818897637795)(396,0.848818897637795)(397,0.848818897637795)(398,0.849487785657998)(399,0.849487785657998)(400,0.849487785657998) 
};
\addplot [
color=red,
mark size=0.1pt,
only marks,
mark=*,
mark options={solid,fill=red},
forget plot
]
coordinates{
 (1,0)(2,0)(3,0)(4,0)(5,0)(6,0)(7,0.337246531483458)(8,0.465462274176408)(9,0.729086722947045)(10,0.777134587554269)(11,0.764525993883792)(12,0.7808)(13,0.800327332242226)(14,0.802610114192496)(15,0.805194805194805)(16,0.801302931596091)(17,0.802250803858521)(18,0.816229116945107)(19,0.819047619047619)(20,0.819334389857369)(21,0.816229116945107)(22,0.812943962115233)(23,0.812351543942993)(24,0.813990461049284)(25,0.818545163868905)(26,0.818254603682946)(27,0.818910256410256)(28,0.814516129032258)(29,0.814516129032258)(30,0.813859790491539)(31,0.816855753646677)(32,0.815896188158962)(33,0.819063004846527)(34,0.823813354786806)(35,0.823813354786806)(36,0.82475884244373)(37,0.82475884244373)(38,0.832)(39,0.832268370607029)(40,0.833200319233839)(41,0.833466135458167)(42,0.833466135458167)(43,0.833466135458167)(44,0.833068362480127)(45,0.832140015910899)(46,0.832802547770701)(47,0.832140015910899)(48,0.83015873015873)(49,0.82922954725973)(50,0.82922954725973)(51,0.831086439333862)(52,0.835703001579779)(53,0.833333333333333)(54,0.83794466403162)(55,0.838862559241706)(56,0.844868735083532)(57,0.844868735083532)(58,0.840255591054313)(59,0.839840637450199)(60,0.838915470494418)(61,0.837060702875399)(62,0.835725677830941)(63,0.832)(64,0.833865814696486)(65,0.8352)(66,0.834532374100719)(67,0.834532374100719)(68,0.839840637450199)(69,0.837988826815642)(70,0.837988826815642)(71,0.837988826815642)(72,0.837988826815642)(73,0.836131095123901)(74,0.8352)(75,0.835725677830941)(76,0.834394904458599)(77,0.832268370607029)(78,0.832933653077538)(79,0.832933653077538)(80,0.833865814696486)(81,0.838760921366164)(82,0.838504375497215)(83,0.841017488076311)(84,0.840095465393795)(85,0.840095465393795)(86,0.840095465393795)(87,0.839171974522293)(88,0.840095465393795)(89,0.841017488076311)(90,0.839271575613618)(91,0.839271575613618)(92,0.838607594936709)(93,0.839271575613618)(94,0.839271575613618)(95,0.839271575613618)(96,0.839271575613618)(97,0.839271575613618)(98,0.839271575613618)(99,0.841106719367589)(100,0.840189873417721)(101,0.840189873417721)(102,0.840189873417721)(103,0.839936608557845)(104,0.841938046068308)(105,0.84352660841938)(106,0.844868735083532)(107,0.844868735083532)(108,0.845295055821372)(109,0.845969672785315)(110,0.845295055821372)(111,0.845295055821372)(112,0.845295055821372)(113,0.846215139442231)(114,0.846215139442231)(115,0.846215139442231)(116,0.846215139442231)(117,0.846215139442231)(118,0.844373503591381)(119,0.844373503591381)(120,0.84688995215311)(121,0.84688995215311)(122,0.847808764940239)(123,0.84688995215311)(124,0.847808764940239)(125,0.847808764940239)(126,0.847808764940239)(127,0.847808764940239)(128,0.847808764940239)(129,0.845786963434022)(130,0.845786963434022)(131,0.846459824980111)(132,0.842105263157895)(133,0.844868735083532)(134,0.844868735083532)(135,0.844868735083532)(136,0.845786963434022)(137,0.847619047619047)(138,0.847430830039526)(139,0.848772763262074)(140,0.844197138314785)(141,0.846031746031746)(142,0.846031746031746)(143,0.845360824742268)(144,0.845360824742268)(145,0.845360824742268)(146,0.845360824742268)(147,0.845360824742268)(148,0.845360824742268)(149,0.844444444444444)(150,0.845115170770453)(151,0.845115170770453)(152,0.845115170770453)(153,0.846031746031746)(154,0.846031746031746)(155,0.846703733121525)(156,0.846031746031746)(157,0.846031746031746)(158,0.846946867565424)(159,0.846946867565424)(160,0.846946867565424)(161,0.848292295472597)(162,0.848292295472597)(163,0.848292295472597)(164,0.847133757961783)(165,0.847808764940239)(166,0.847133757961783)(167,0.847133757961783)(168,0.847133757961783)(169,0.846215139442231)(170,0.846215139442231)(171,0.845786963434022)(172,0.845786963434022)(173,0.846459824980111)(174,0.847376788553259)(175,0.847376788553259)(176,0.847376788553259)(177,0.846703733121525)(178,0.846703733121525)(179,0.84853291038858)(180,0.847619047619047)(181,0.847619047619047)(182,0.847619047619047)(183,0.846703733121525)(184,0.845786963434022)(185,0.845786963434022)(186,0.845786963434022)(187,0.845786963434022)(188,0.846703733121525)(189,0.847376788553259)(190,0.847376788553259)(191,0.845786963434022)(192,0.845115170770453)(193,0.845115170770453)(194,0.843277645186953)(195,0.84352660841938)(196,0.844868735083532)(197,0.844868735083532)(198,0.844868735083532)(199,0.844621513944223)(200,0.845295055821372)(201,0.844621513944223)(202,0.844621513944223)(203,0.844621513944223)(204,0.843949044585987)(205,0.844868735083532)(206,0.845786963434022)(207,0.847860538827258)(208,0.846031746031746)(209,0.848966613672496)(210,0.84805091487669)(211,0.848726114649681)(212,0.848726114649681)(213,0.848726114649681)(214,0.847808764940239)(215,0.847808764940239)(216,0.847808764940239)(217,0.848484848484848)(218,0.848484848484848)(219,0.848484848484848)(220,0.848484848484848)(221,0.848484848484848)(222,0.848484848484848)(223,0.847133757961783)(224,0.84805091487669)(225,0.848966613672496)(226,0.848292295472597)(227,0.848292295472597)(228,0.848292295472597)(229,0.849206349206349)(230,0.849206349206349)(231,0.849206349206349)(232,0.848292295472597)(233,0.850118953211737)(234,0.85193982581156)(235,0.85193982581156)(236,0.853754940711462)(237,0.853754940711462)(238,0.853754940711462)(239,0.853754940711462)(240,0.852848101265823)(241,0.852614896988906)(242,0.852614896988906)(243,0.852614896988906)(244,0.853523357086302)(245,0.853523357086302)(246,0.853523357086302)(247,0.853523357086302)(248,0.853523357086302)(249,0.853523357086302)(250,0.852848101265823)(251,0.852848101265823)(252,0.852848101265823)(253,0.853523357086302)(254,0.853523357086302)(255,0.853523357086302)(256,0.854199683042789)(257,0.854199683042789)(258,0.854199683042789)(259,0.853291038858049)(260,0.853291038858049)(261,0.852380952380952)(262,0.853291038858049)(263,0.853291038858049)(264,0.853057982525814)(265,0.852380952380952)(266,0.852380952380952)(267,0.852380952380952)(268,0.852614896988906)(269,0.853968253968254)(270,0.853057982525814)(271,0.853057982525814)(272,0.853057982525814)(273,0.853057982525814)(274,0.853057982525814)(275,0.853057982525814)(276,0.853057982525814)(277,0.853736089030207)(278,0.854646544876886)(279,0.853736089030207)(280,0.854646544876886)(281,0.854646544876886)(282,0.853736089030207)(283,0.854646544876886)(284,0.853736089030207)(285,0.853736089030207)(286,0.853736089030207)(287,0.852824184566428)(288,0.85214626391097)(289,0.853057982525814)(290,0.853057982525814)(291,0.85214626391097)(292,0.85214626391097)(293,0.85214626391097)(294,0.85214626391097)(295,0.85214626391097)(296,0.85214626391097)(297,0.85214626391097)(298,0.85214626391097)(299,0.853057982525814)(300,0.853057982525814)(301,0.853057982525814)(302,0.853057982525814)(303,0.853057982525814)(304,0.853968253968254)(305,0.853968253968254)(306,0.853968253968254)(307,0.854646544876886)(308,0.854646544876886)(309,0.854646544876886)(310,0.853736089030207)(311,0.854646544876886)(312,0.853968253968254)(313,0.853057982525814)(314,0.853057982525814)(315,0.853057982525814)(316,0.853057982525814)(317,0.853968253968254)(318,0.853968253968254)(319,0.853968253968254)(320,0.853968253968254)(321,0.854877081681205)(322,0.854877081681205)(323,0.854877081681205)(324,0.854877081681205)(325,0.854877081681205)(326,0.854877081681205)(327,0.854877081681205)(328,0.854877081681205)(329,0.854877081681205)(330,0.854877081681205)(331,0.854877081681205)(332,0.854877081681205)(333,0.854877081681205)(334,0.854877081681205)(335,0.854877081681205)(336,0.854877081681205)(337,0.854877081681205)(338,0.854877081681205)(339,0.854877081681205)(340,0.854877081681205)(341,0.854877081681205)(342,0.854877081681205)(343,0.854877081681205)(344,0.854877081681205)(345,0.854877081681205)(346,0.854877081681205)(347,0.854199683042789)(348,0.854199683042789)(349,0.854199683042789)(350,0.854199683042789)(351,0.854877081681205)(352,0.854877081681205)(353,0.854877081681205)(354,0.854877081681205)(355,0.854877081681205)(356,0.854877081681205)(357,0.854877081681205)(358,0.854877081681205)(359,0.854877081681205)(360,0.854877081681205)(361,0.854877081681205)(362,0.854877081681205)(363,0.854877081681205)(364,0.854877081681205)(365,0.853968253968254)(366,0.854646544876886)(367,0.853968253968254)(368,0.853968253968254)(369,0.853968253968254)(370,0.853968253968254)(371,0.853968253968254)(372,0.853968253968254)(373,0.853968253968254)(374,0.853968253968254)(375,0.853968253968254)(376,0.853968253968254)(377,0.853968253968254)(378,0.853968253968254)(379,0.853968253968254)(380,0.853736089030207)(381,0.853736089030207)(382,0.853736089030207)(383,0.853736089030207)(384,0.853736089030207)(385,0.853736089030207)(386,0.854415274463007)(387,0.853057982525814)(388,0.853057982525814)(389,0.853057982525814)(390,0.853057982525814)(391,0.853057982525814)(392,0.853057982525814)(393,0.853057982525814)(394,0.853057982525814)(395,0.853057982525814)(396,0.853057982525814)(397,0.853057982525814)(398,0.853057982525814)(399,0.853057982525814)(400,0.853057982525814) 
};
\addplot [
color=red,
mark size=0.1pt,
only marks,
mark=*,
mark options={solid,fill=red},
forget plot
]
coordinates{
 (1,0)(2,0)(3,0)(4,0)(5,0)(6,0.23695652173913)(7,0.232161874334398)(8,0.217438105489774)(9,0.226851851851852)(10,0.241183162684869)(11,0.225149700598802)(12,0.22700119474313)(13,0.221686746987952)(14,0.231404958677686)(15,0.227218934911243)(16,0.241299303944315)(17,0.242068155111633)(18,0.242924528301887)(19,0.242352941176471)(20,0.242068155111633)(21,0.241784037558685)(22,0.23696682464455)(23,0.476618705035971)(24,0.460973370064279)(25,0.637759710930442)(26,0.762258543833581)(27,0.765990639625585)(28,0.760159362549801)(29,0.833080424886191)(30,0.832955404383976)(31,0.866176470588235)(32,0.867060561299852)(33,0.877873563218391)(34,0.880986937590711)(35,0.885672937771346)(36,0.88536409516943)(37,0.880057803468208)(38,0.879248011569053)(39,0.87797833935018)(40,0.883620689655172)(41,0.878397711015737)(42,0.879596250901226)(43,0.881159420289855)(44,0.881159420289855)(45,0.882096069868996)(46,0.881281864530226)(47,0.884531590413943)(48,0.888567293777135)(49,0.884559884559884)(50,0.886002886002886)(51,0.886642599277978)(52,0.88728323699422)(53,0.891461649782923)(54,0.891618497109826)(55,0.893217893217893)(56,0.894168466522678)(57,0.892573900504686)(58,0.891618497109826)(59,0.893555394641564)(60,0.892908827785817)(61,0.892908827785817)(62,0.889689978370584)(63,0.890974729241877)(64,0.892573900504686)(65,0.894964028776978)(66,0.895114942528736)(67,0.894812680115274)(68,0.895608351331893)(69,0.896551724137931)(70,0.896551724137931)(71,0.901734104046243)(72,0.901083032490975)(73,0.904347826086956)(74,0.904347826086956)(75,0.906993511175198)(76,0.905172413793103)(77,0.905172413793103)(78,0.904385334291876)(79,0.905172413793103)(80,0.904659498207885)(81,0.905444126074499)(82,0.905444126074499)(83,0.907394113424264)(84,0.911976911976912)(85,0.911976911976912)(86,0.91383055756698)(87,0.914492753623188)(88,0.915155910079768)(89,0.915278783490224)(90,0.914616497829233)(91,0.914492753623188)(92,0.915278783490224)(93,0.916063675832127)(94,0.916063675832127)(95,0.911447084233261)(96,0.911447084233261)(97,0.913419913419913)(98,0.913294797687861)(99,0.914079422382671)(100,0.914079422382671)(101,0.916727009413468)(102,0.916727009413468)(103,0.918293564714389)(104,0.916305916305916)(105,0.918958031837916)(106,0.919623461259956)(107,0.919623461259956)(108,0.919739696312364)(109,0.92040520984081)(110,0.920520231213873)(111,0.920520231213873)(112,0.919739696312364)(113,0.919739696312364)(114,0.919739696312364)(115,0.921625544267053)(116,0.921965317919075)(117,0.922077922077922)(118,0.922855082912761)(119,0.922190201729106)(120,0.923520923520923)(121,0.923520923520923)(122,0.924187725631769)(123,0.924963924963925)(124,0.925631768953068)(125,0.923299565846599)(126,0.922519913106445)(127,0.923299565846599)(128,0.923299565846599)(129,0.923299565846599)(130,0.923299565846599)(131,0.923299565846599)(132,0.924078091106291)(133,0.921739130434783)(134,0.921739130434783)(135,0.920957215373459)(136,0.920957215373459)(137,0.91985559566787)(138,0.91985559566787)(139,0.919739696312364)(140,0.919739696312364)(141,0.92129963898917)(142,0.91985559566787)(143,0.91985559566787)(144,0.91985559566787)(145,0.91985559566787)(146,0.919191919191919)(147,0.919191919191919)(148,0.919971160778659)(149,0.919971160778659)(150,0.920749279538905)(151,0.918646508279338)(152,0.921090387374462)(153,0.920315865039483)(154,0.921090387374462)(155,0.920863309352518)(156,0.919424460431655)(157,0.918763479511143)(158,0.918763479511143)(159,0.919424460431655)(160,0.918646508279338)(161,0.918646508279338)(162,0.920086393088553)(163,0.920749279538905)(164,0.919191919191919)(165,0.919191919191919)(166,0.919308357348703)(167,0.918646508279338)(168,0.918646508279338)(169,0.918646508279338)(170,0.918646508279338)(171,0.918646508279338)(172,0.918646508279338)(173,0.917867435158501)(174,0.918646508279338)(175,0.922077922077922)(176,0.922743682310469)(177,0.922631959508315)(178,0.922631959508315)(179,0.922631959508315)(180,0.922631959508315)(181,0.921965317919075)(182,0.922743682310469)(183,0.922077922077922)(184,0.922077922077922)(185,0.922077922077922)(186,0.925179856115108)(187,0.925179856115108)(188,0.925845932325414)(189,0.925845932325414)(190,0.92507204610951)(191,0.924406047516199)(192,0.925179856115108)(193,0.925179856115108)(194,0.925179856115108)(195,0.924406047516199)(196,0.92507204610951)(197,0.92507204610951)(198,0.92507204610951)(199,0.92507204610951)(200,0.924406047516199)(201,0.925845932325414)(202,0.925179856115108)(203,0.928725701943844)(204,0.928057553956834)(205,0.928057553956834)(206,0.928057553956834)(207,0.928057553956834)(208,0.927390366642703)(209,0.927390366642703)(210,0.927390366642703)(211,0.926618705035971)(212,0.926618705035971)(213,0.927285817134629)(214,0.927285817134629)(215,0.927285817134629)(216,0.927953890489913)(217,0.929496402877698)(218,0.929496402877698)(219,0.92882818116463)(220,0.926829268292683)(221,0.927494615936827)(222,0.929496402877698)(223,0.930935251798561)(224,0.93016558675306)(225,0.93016558675306)(226,0.93016558675306)(227,0.93016558675306)(228,0.93016558675306)(229,0.93016558675306)(230,0.93016558675306)(231,0.93016558675306)(232,0.93016558675306)(233,0.93016558675306)(234,0.93016558675306)(235,0.93016558675306)(236,0.93016558675306)(237,0.93016558675306)(238,0.93016558675306)(239,0.930935251798561)(240,0.930935251798561)(241,0.929597701149425)(242,0.930935251798561)(243,0.929597701149425)(244,0.932094353109364)(245,0.932094353109364)(246,0.936079545454545)(247,0.946778711484594)(248,0.941508104298802)(249,0.943661971830986)(250,0.940845070422535)(251,0.940267041461701)(252,0.941340782122905)(253,0.93986013986014)(254,0.941258741258741)(255,0.941258741258741)(256,0.939775910364146)(257,0.939775910364146)(258,0.939775910364146)(259,0.939775910364146)(260,0.941258741258741)(261,0.942079553384508)(262,0.941998602375961)(263,0.941258741258741)(264,0.941258741258741)(265,0.941258741258741)(266,0.941258741258741)(267,0.941258741258741)(268,0.941258741258741)(269,0.941998602375961)(270,0.942737430167598)(271,0.943396226415094)(272,0.94413407821229)(273,0.943475226796929)(274,0.944367176634214)(275,0.945176960444136)(276,0.945176960444136)(277,0.945908460471567)(278,0.945908460471567)(279,0.951724137931034)(280,0.950344827586207)(281,0.951000690131125)(282,0.951724137931034)(283,0.952446588559614)(284,0.952446588559614)(285,0.952446588559614)(286,0.95807560137457)(287,0.95807560137457)(288,0.95807560137457)(289,0.95807560137457)(290,0.957417582417582)(291,0.953488372093023)(292,0.954140999315537)(293,0.956104252400549)(294,0.956104252400549)(295,0.958017894012388)(296,0.96)(297,0.96)(298,0.960662525879917)(299,0.960716747070985)(300,0.959944751381215)(301,0.960553633217993)(302,0.959944751381215)(303,0.95850622406639)(304,0.95916955017301)(305,0.95850622406639)(306,0.95983379501385)(307,0.95983379501385)(308,0.959112959112959)(309,0.95983379501385)(310,0.95983379501385)(311,0.96049896049896)(312,0.959778085991678)(313,0.959778085991678)(314,0.959056210964608)(315,0.959056210964608)(316,0.959056210964608)(317,0.958333333333333)(318,0.958333333333333)(319,0.958333333333333)(320,0.959666203059805)(321,0.959666203059805)(322,0.96038915913829)(323,0.96038915913829)(324,0.96038915913829)(325,0.96105702364395)(326,0.96105702364395)(327,0.96105702364395)(328,0.96105702364395)(329,0.96105702364395)(330,0.961725817675713)(331,0.961672473867596)(332,0.96100278551532)(333,0.960278745644599)(334,0.960278745644599)(335,0.960278745644599)(336,0.960278745644599)(337,0.959553695955369)(338,0.959553695955369)(339,0.959553695955369)(340,0.959553695955369)(341,0.959553695955369)(342,0.960223307745987)(343,0.960893854748603)(344,0.960893854748603)(345,0.964509394572025)(346,0.964509394572025)(347,0.964509394572025)(348,0.964509394572025)(349,0.965229485396384)(350,0.965229485396384)(351,0.965229485396384)(352,0.965229485396384)(353,0.964607911172797)(354,0.964607911172797)(355,0.964607911172797)(356,0.96455872133426)(357,0.966620305980528)(358,0.96455872133426)(359,0.96455872133426)(360,0.966620305980528)(361,0.966573816155989)(362,0.966573816155989)(363,0.966573816155989)(364,0.965853658536585)(365,0.965901183020181)(366,0.964459930313589)(367,0.964459930313589)(368,0.963788300835654)(369,0.964459930313589)(370,0.96513249651325)(371,0.96513249651325)(372,0.966480446927374)(373,0.967832167832168)(374,0.967832167832168)(375,0.967155835080363)(376,0.967155835080363)(377,0.965806001395673)(378,0.965806001395673)(379,0.96652719665272)(380,0.966620305980528)(381,0.966573816155989)(382,0.967338429464906)(383,0.968011126564673)(384,0.966666666666666)(385,0.964656964656964)(386,0.964656964656964)(387,0.965995836224844)(388,0.965995836224844)(389,0.967428967428967)(390,0.96814404432133)(391,0.96814404432133)(392,0.968814968814969)(393,0.969486823855756)(394,0.970914127423823)(395,0.970914127423823)(396,0.970914127423823)(397,0.970914127423823)(398,0.970242214532872)(399,0.971665514858327)(400,0.971665514858327) 
};
\addplot [
color=red,
mark size=0.1pt,
only marks,
mark=*,
mark options={solid,fill=red},
forget plot
]
coordinates{
 (1,0)(2,0)(3,0)(4,0)(5,0.0369393139841688)(6,0)(7,0.661157024793388)(8,0.740791268758527)(9,0.742778541953232)(10,0.792592592592592)(11,0.775687409551375)(12,0.787967718268525)(13,0.783060921248143)(14,0.79333838001514)(15,0.799694189602446)(16,0.80061115355233)(17,0.809930178432894)(18,0.809930178432894)(19,0.813717848791894)(20,0.811276429130775)(21,0.807453416149068)(22,0.811232449297972)(23,0.810600155884645)(24,0.812596006144393)(25,0.811349693251534)(26,0.814186584425597)(27,0.813559322033898)(28,0.820872274143302)(29,0.819953234606391)(30,0.818958818958819)(31,0.818110850897736)(32,0.818466353677621)(33,0.818611987381703)(34,0.820553359683794)(35,0.822415153906866)(36,0.826394344069128)(37,0.826224328593997)(38,0.8274231678487)(39,0.824175824175824)(40,0.823622047244094)(41,0.822695035460993)(42,0.817460317460317)(43,0.817460317460317)(44,0.818398096748612)(45,0.818398096748612)(46,0.817170111287758)(47,0.817170111287758)(48,0.818109610802224)(49,0.818759936406995)(50,0.817820206841686)(51,0.817820206841686)(52,0.819047619047619)(53,0.816455696202532)(54,0.818325434439178)(55,0.817391304347826)(56,0.817391304347826)(57,0.819258089976322)(58,0.819542947202522)(59,0.822327044025157)(60,0.822327044025157)(61,0.822327044025157)(62,0.822047244094488)(63,0.823899371069182)(64,0.823899371069182)(65,0.822047244094488)(66,0.823715415019763)(67,0.823715415019763)(68,0.822222222222222)(69,0.822222222222222)(70,0.821286735504368)(71,0.822875297855441)(72,0.822875297855441)(73,0.82484076433121)(74,0.828137490007994)(75,0.828137490007994)(76,0.828411811652035)(77,0.826538768984812)(78,0.827476038338658)(79,0.8268156424581)(80,0.8268156424581)(81,0.82484076433121)(82,0.823904382470119)(83,0.822966507177033)(84,0.823904382470119)(85,0.822966507177033)(86,0.822966507177033)(87,0.822683706070288)(88,0.822683706070288)(89,0.823623304070231)(90,0.823623304070231)(91,0.824281150159744)(92,0.824281150159744)(93,0.823904382470119)(94,0.823623304070231)(95,0.823623304070231)(96,0.823623304070231)(97,0.826433121019108)(98,0.825775656324582)(99,0.828571428571429)(100,0.82922954725973)(101,0.828298887122416)(102,0.82922954725973)(103,0.831086439333862)(104,0.833201581027668)(105,0.833597464342314)(106,0.833597464342314)(107,0.833597464342314)(108,0.833860759493671)(109,0.834782608695652)(110,0.833860759493671)(111,0.833860759493671)(112,0.833860759493671)(113,0.834782608695652)(114,0.836363636363636)(115,0.836621941594317)(116,0.836621941594317)(117,0.836621941594317)(118,0.835043409629045)(119,0.835043409629045)(120,0.834384858044164)(121,0.834384858044164)(122,0.834384858044164)(123,0.833727344365642)(124,0.834645669291339)(125,0.833727344365642)(126,0.833727344365642)(127,0.833464877663773)(128,0.833201581027668)(129,0.83596214511041)(130,0.83596214511041)(131,0.836621941594317)(132,0.836621941594317)(133,0.840031520882585)(134,0.840031520882585)(135,0.840283241542093)(136,0.84160756501182)(137,0.84160756501182)(138,0.841856805664831)(139,0.837282780410742)(140,0.837282780410742)(141,0.838862559241706)(142,0.837539432176656)(143,0.838200473559589)(144,0.837539432176656)(145,0.836220472440945)(146,0.836879432624113)(147,0.837539432176656)(148,0.837539432176656)(149,0.838862559241706)(150,0.838862559241706)(151,0.838862559241706)(152,0.840944881889764)(153,0.840694006309148)(154,0.840031520882585)(155,0.840283241542093)(156,0.840031520882585)(157,0.839370078740157)(158,0.840283241542093)(159,0.840944881889764)(160,0.842271293375394)(161,0.842936069455406)(162,0.842936069455406)(163,0.84384858044164)(164,0.84384858044164)(165,0.844268774703557)(166,0.842936069455406)(167,0.842271293375394)(168,0.84160756501182)(169,0.842271293375394)(170,0.842271293375394)(171,0.842271293375394)(172,0.842271293375394)(173,0.842271293375394)(174,0.842936069455406)(175,0.842271293375394)(176,0.842271293375394)(177,0.84160756501182)(178,0.842271293375394)(179,0.842022116903633)(180,0.842022116903633)(181,0.842936069455406)(182,0.842936069455406)(183,0.842936069455406)(184,0.84160756501182)(185,0.840944881889764)(186,0.840283241542093)(187,0.839622641509434)(188,0.840283241542093)(189,0.842105263157895)(190,0.843921568627451)(191,0.843921568627451)(192,0.844827586206896)(193,0.845732184808144)(194,0.845732184808144)(195,0.845732184808144)(196,0.846635367762128)(197,0.845973416731822)(198,0.845973416731822)(199,0.845973416731822)(200,0.846635367762128)(201,0.847298355520752)(202,0.847298355520752)(203,0.846394984326019)(204,0.845490196078431)(205,0.846153846153846)(206,0.846153846153846)(207,0.846153846153846)(208,0.846153846153846)(209,0.845490196078431)(210,0.845490196078431)(211,0.845490196078431)(212,0.844827586206896)(213,0.844827586206896)(214,0.846818538884525)(215,0.846818538884525)(216,0.846818538884525)(217,0.846818538884525)(218,0.847058823529412)(219,0.847058823529412)(220,0.847058823529412)(221,0.847058823529412)(222,0.847058823529412)(223,0.846394984326019)(224,0.846394984326019)(225,0.846394984326019)(226,0.845070422535211)(227,0.844409695074277)(228,0.843505477308294)(229,0.844409695074277)(230,0.845070422535211)(231,0.845070422535211)(232,0.845973416731822)(233,0.846875)(234,0.846875)(235,0.846875)(236,0.846875)(237,0.846875)(238,0.846875)(239,0.846875)(240,0.847775175644028)(241,0.847775175644028)(242,0.847775175644028)(243,0.847775175644028)(244,0.847775175644028)(245,0.849100860046912)(246,0.849100860046912)(247,0.849100860046912)(248,0.849765258215962)(249,0.850430696945967)(250,0.850430696945967)(251,0.850430696945967)(252,0.850430696945967)(253,0.850430696945967)(254,0.850430696945967)(255,0.849765258215962)(256,0.849765258215962)(257,0.849100860046912)(258,0.8484375)(259,0.8484375)(260,0.8484375)(261,0.849100860046912)(262,0.8484375)(263,0.847537138389367)(264,0.846875)(265,0.846875)(266,0.847113884555382)(267,0.847113884555382)(268,0.847775175644028)(269,0.847775175644028)(270,0.847775175644028)(271,0.847775175644028)(272,0.847775175644028)(273,0.846875)(274,0.846875)(275,0.847775175644028)(276,0.846453624318005)(277,0.846453624318005)(278,0.847775175644028)(279,0.847775175644028)(280,0.847775175644028)(281,0.847775175644028)(282,0.847775175644028)(283,0.847775175644028)(284,0.847775175644028)(285,0.8484375)(286,0.847775175644028)(287,0.847775175644028)(288,0.847775175644028)(289,0.8484375)(290,0.8484375)(291,0.847775175644028)(292,0.847775175644028)(293,0.847775175644028)(294,0.847775175644028)(295,0.847775175644028)(296,0.848673946957878)(297,0.848673946957878)(298,0.848673946957878)(299,0.848673946957878)(300,0.848673946957878)(301,0.848673946957878)(302,0.848673946957878)(303,0.848673946957878)(304,0.848012470771629)(305,0.848673946957878)(306,0.849336455893833)(307,0.849336455893833)(308,0.849336455893833)(309,0.849336455893833)(310,0.848673946957878)(311,0.848673946957878)(312,0.849336455893833)(313,0.849336455893833)(314,0.849336455893833)(315,0.849336455893833)(316,0.85)(317,0.85)(318,0.85)(319,0.85)(320,0.850664581704456)(321,0.849765258215962)(322,0.849765258215962)(323,0.849765258215962)(324,0.849765258215962)(325,0.849765258215962)(326,0.849765258215962)(327,0.85)(328,0.850664581704456)(329,0.850664581704456)(330,0.850664581704456)(331,0.850664581704456)(332,0.850664581704456)(333,0.850664581704456)(334,0.850664581704456)(335,0.850664581704456)(336,0.850664581704456)(337,0.850664581704456)(338,0.850664581704456)(339,0.850664581704456)(340,0.850664581704456)(341,0.850664581704456)(342,0.850664581704456)(343,0.850664581704456)(344,0.850664581704456)(345,0.851330203442879)(346,0.851330203442879)(347,0.850430696945967)(348,0.849765258215962)(349,0.849100860046912)(350,0.849100860046912)(351,0.849336455893833)(352,0.85)(353,0.850897736143638)(354,0.850234009360374)(355,0.850897736143638)(356,0.850897736143638)(357,0.8515625)(358,0.8515625)(359,0.8515625)(360,0.8515625)(361,0.8515625)(362,0.850897736143638)(363,0.850897736143638)(364,0.850897736143638)(365,0.850897736143638)(366,0.850234009360374)(367,0.850234009360374)(368,0.850234009360374)(369,0.849571317225253)(370,0.849571317225253)(371,0.849336455893833)(372,0.849336455893833)(373,0.849336455893833)(374,0.849336455893833)(375,0.849336455893833)(376,0.85)(377,0.85)(378,0.85)(379,0.85)(380,0.8515625)(381,0.8515625)(382,0.8515625)(383,0.8515625)(384,0.850897736143638)(385,0.850897736143638)(386,0.850897736143638)(387,0.850897736143638)(388,0.850897736143638)(389,0.850897736143638)(390,0.850897736143638)(391,0.850897736143638)(392,0.8515625)(393,0.850897736143638)(394,0.850897736143638)(395,0.8515625)(396,0.8515625)(397,0.850897736143638)(398,0.850897736143638)(399,0.850897736143638)(400,0.850897736143638) 
};
\addplot [
color=red,
mark size=0.1pt,
only marks,
mark=*,
mark options={solid,fill=red},
forget plot
]
coordinates{
 (1,0)(2,0)(3,0)(4,0)(5,0)(6,0.742944317315027)(7,0.753726046841732)(8,0.742980561555076)(9,0.764011799410029)(10,0.771641791044776)(11,0.788432267884323)(12,0.789230769230769)(13,0.791056283731688)(14,0.785208497246263)(15,0.809230769230769)(16,0.805471124620061)(17,0.807926829268293)(18,0.806130268199234)(19,0.804281345565749)(20,0.806748466257669)(21,0.796661608497724)(22,0.80091533180778)(23,0.804297774366846)(24,0.807044410413476)(25,0.806426931905126)(26,0.808314087759815)(27,0.808314087759815)(28,0.813455657492355)(29,0.812213740458015)(30,0.8125)(31,0.814362108479755)(32,0.814362108479755)(33,0.813740458015267)(34,0.811263318112633)(35,0.811881188118812)(36,0.810030395136778)(37,0.808510638297872)(38,0.809125475285171)(39,0.809741248097412)(40,0.815325670498084)(41,0.812213740458015)(42,0.811306340718105)(43,0.812547819433818)(44,0.812547819433818)(45,0.81283422459893)(46,0.813455657492355)(47,0.813740458015267)(48,0.810975609756098)(49,0.813793103448276)(50,0.813169984686064)(51,0.818812644564379)(52,0.818181818181818)(53,0.819722650231125)(54,0.82)(55,0.821263482280431)(56,0.82)(57,0.819369715603382)(58,0.82)(59,0.818740399385561)(60,0.819369715603382)(61,0.819369715603382)(62,0.819369715603382)(63,0.82571649883811)(64,0.82571649883811)(65,0.825077399380805)(66,0.82571649883811)(67,0.825986078886311)(68,0.826522744795682)(69,0.827799227799228)(70,0.828173374613003)(71,0.829721362229102)(72,0.828703703703704)(73,0.828064764841943)(74,0.829721362229102)(75,0.830626450116009)(76,0.83579766536965)(77,0.83579766536965)(78,0.837100545596259)(79,0.837100545596259)(80,0.836193447737909)(81,0.838006230529595)(82,0.837354085603113)(83,0.834755624515128)(84,0.836052836052836)(85,0.836052836052836)(86,0.839968774395004)(87,0.838659392049883)(88,0.837606837606838)(89,0.839563862928349)(90,0.841530054644809)(91,0.841530054644809)(92,0.842845973416732)(93,0.843505477308294)(94,0.844166014095536)(95,0.8421875)(96,0.841940532081377)(97,0.841940532081377)(98,0.841940532081377)(99,0.842599843382929)(100,0.841444270015699)(101,0.840534171249018)(102,0.840534171249018)(103,0.839622641509434)(104,0.84160756501182)(105,0.84160756501182)(106,0.84160756501182)(107,0.841357537490134)(108,0.842271293375394)(109,0.842767295597484)(110,0.843676355066771)(111,0.846394984326019)(112,0.843676355066771)(113,0.842767295597484)(114,0.842767295597484)(115,0.842767295597484)(116,0.842767295597484)(117,0.844759653270291)(118,0.843430369787569)(119,0.844759653270291)(120,0.845425867507886)(121,0.844759653270291)(122,0.844759653270291)(123,0.844759653270291)(124,0.844759653270291)(125,0.844759653270291)(126,0.846761453396524)(127,0.846761453396524)(128,0.846761453396524)(129,0.846761453396524)(130,0.846761453396524)(131,0.846761453396524)(132,0.846761453396524)(133,0.846761453396524)(134,0.847671665351223)(135,0.846335697399527)(136,0.847244094488189)(137,0.846335697399527)(138,0.846335697399527)(139,0.847003154574132)(140,0.847244094488189)(141,0.845911949685534)(142,0.845911949685534)(143,0.846577498033045)(144,0.846577498033045)(145,0.846577498033045)(146,0.847003154574132)(147,0.846335697399527)(148,0.847003154574132)(149,0.847671665351223)(150,0.847671665351223)(151,0.847671665351223)(152,0.847430830039526)(153,0.847671665351223)(154,0.847244094488189)(155,0.848580441640378)(156,0.848580441640378)(157,0.847911741528763)(158,0.847911741528763)(159,0.847244094488189)(160,0.847244094488189)(161,0.847244094488189)(162,0.848580441640378)(163,0.849250197316495)(164,0.847911741528763)(165,0.847911741528763)(166,0.848580441640378)(167,0.848580441640378)(168,0.848580441640378)(169,0.848580441640378)(170,0.847671665351223)(171,0.847671665351223)(172,0.848341232227488)(173,0.848341232227488)(174,0.847671665351223)(175,0.847671665351223)(176,0.84901185770751)(177,0.84901185770751)(178,0.84901185770751)(179,0.849683544303797)(180,0.850592885375494)(181,0.850592885375494)(182,0.850592885375494)(183,0.850592885375494)(184,0.850592885375494)(185,0.850592885375494)(186,0.850592885375494)(187,0.850592885375494)(188,0.85126582278481)(189,0.851500789889415)(190,0.850592885375494)(191,0.850356294536817)(192,0.85126582278481)(193,0.852173913043478)(194,0.85126582278481)(195,0.852173913043478)(196,0.85126582278481)(197,0.85126582278481)(198,0.85126582278481)(199,0.852173913043478)(200,0.85126582278481)(201,0.85126582278481)(202,0.85126582278481)(203,0.85126582278481)(204,0.85126582278481)(205,0.85126582278481)(206,0.851704996034893)(207,0.850592885375494)(208,0.850592885375494)(209,0.850592885375494)(210,0.850592885375494)(211,0.850592885375494)(212,0.85126582278481)(213,0.85126582278481)(214,0.85126582278481)(215,0.850356294536817)(216,0.850356294536817)(217,0.850356294536817)(218,0.850356294536817)(219,0.850356294536817)(220,0.850356294536817)(221,0.850356294536817)(222,0.850356294536817)(223,0.850356294536817)(224,0.851030110935024)(225,0.851030110935024)(226,0.851030110935024)(227,0.851030110935024)(228,0.851030110935024)(229,0.850793650793651)(230,0.850793650793651)(231,0.850793650793651)(232,0.850793650793651)(233,0.850793650793651)(234,0.850793650793651)(235,0.850793650793651)(236,0.850793650793651)(237,0.84853291038858)(238,0.849445324881141)(239,0.849445324881141)(240,0.849445324881141)(241,0.849445324881141)(242,0.849445324881141)(243,0.849445324881141)(244,0.850356294536817)(245,0.850356294536817)(246,0.849683544303797)(247,0.849683544303797)(248,0.849683544303797)(249,0.849683544303797)(250,0.850356294536817)(251,0.851030110935024)(252,0.849683544303797)(253,0.851704996034893)(254,0.852614896988906)(255,0.852614896988906)(256,0.852614896988906)(257,0.852614896988906)(258,0.852614896988906)(259,0.852614896988906)(260,0.852614896988906)(261,0.852614896988906)(262,0.851704996034893)(263,0.851704996034893)(264,0.851704996034893)(265,0.852614896988906)(266,0.852380952380952)(267,0.852380952380952)(268,0.852380952380952)(269,0.852380952380952)(270,0.853291038858049)(271,0.853291038858049)(272,0.853291038858049)(273,0.853291038858049)(274,0.853291038858049)(275,0.852380952380952)(276,0.852380952380952)(277,0.852380952380952)(278,0.852380952380952)(279,0.852380952380952)(280,0.850556438791733)(281,0.850556438791733)(282,0.850556438791733)(283,0.851469420174742)(284,0.851469420174742)(285,0.851469420174742)(286,0.851469420174742)(287,0.851469420174742)(288,0.851469420174742)(289,0.851469420174742)(290,0.851469420174742)(291,0.851469420174742)(292,0.852380952380952)(293,0.854199683042789)(294,0.854199683042789)(295,0.854430379746835)(296,0.855335968379446)(297,0.856240126382306)(298,0.856240126382306)(299,0.856240126382306)(300,0.855335968379446)(301,0.854660347551343)(302,0.854660347551343)(303,0.854660347551343)(304,0.853985793212312)(305,0.854660347551343)(306,0.854660347551343)(307,0.853754940711462)(308,0.853754940711462)(309,0.853754940711462)(310,0.853754940711462)(311,0.853754940711462)(312,0.853754940711462)(313,0.853523357086302)(314,0.853523357086302)(315,0.853523357086302)(316,0.853523357086302)(317,0.853523357086302)(318,0.853523357086302)(319,0.852614896988906)(320,0.852614896988906)(321,0.853523357086302)(322,0.853523357086302)(323,0.853523357086302)(324,0.852614896988906)(325,0.853523357086302)(326,0.853523357086302)(327,0.851500789889415)(328,0.852173913043478)(329,0.852173913043478)(330,0.851500789889415)(331,0.851500789889415)(332,0.851500789889415)(333,0.852173913043478)(334,0.852173913043478)(335,0.852173913043478)(336,0.852173913043478)(337,0.852173913043478)(338,0.852173913043478)(339,0.852173913043478)(340,0.852173913043478)(341,0.852173913043478)(342,0.852173913043478)(343,0.852173913043478)(344,0.852173913043478)(345,0.851500789889415)(346,0.851500789889415)(347,0.851500789889415)(348,0.852173913043478)(349,0.852173913043478)(350,0.852173913043478)(351,0.852848101265823)(352,0.852848101265823)(353,0.852173913043478)(354,0.852173913043478)(355,0.852173913043478)(356,0.852173913043478)(357,0.852173913043478)(358,0.852173913043478)(359,0.852173913043478)(360,0.85240726124704)(361,0.85173501577287)(362,0.85173501577287)(363,0.85240726124704)(364,0.85240726124704)(365,0.85240726124704)(366,0.85240726124704)(367,0.853312302839117)(368,0.85240726124704)(369,0.85240726124704)(370,0.853312302839117)(371,0.853312302839117)(372,0.853312302839117)(373,0.853312302839117)(374,0.85263987391647)(375,0.85263987391647)(376,0.85263987391647)(377,0.853312302839117)(378,0.85240726124704)(379,0.85240726124704)(380,0.85240726124704)(381,0.853312302839117)(382,0.853312302839117)(383,0.85173501577287)(384,0.85173501577287)(385,0.85173501577287)(386,0.85173501577287)(387,0.85263987391647)(388,0.85263987391647)(389,0.85263987391647)(390,0.85263987391647)(391,0.85263987391647)(392,0.85173501577287)(393,0.85173501577287)(394,0.850828729281768)(395,0.85173501577287)(396,0.85173501577287)(397,0.851063829787234)(398,0.851063829787234)(399,0.851063829787234)(400,0.851063829787234) 
};
\addplot [
color=red,
mark size=0.1pt,
only marks,
mark=*,
mark options={solid,fill=red},
forget plot
]
coordinates{
 (1,0)(2,0)(3,0)(4,0)(5,0.167082294264339)(6,0.321678321678322)(7,0.748355263157895)(8,0.749016522423289)(9,0.749399519615692)(10,0.769991755976917)(11,0.787096774193548)(12,0.805237315875614)(13,0.811382113821138)(14,0.813311688311688)(15,0.813114754098361)(16,0.821138211382114)(17,0.821138211382114)(18,0.820846905537459)(19,0.813114754098361)(20,0.807160292921074)(21,0.809716599190283)(22,0.809716599190283)(23,0.813859790491539)(24,0.811945117029863)(25,0.812297734627832)(26,0.814574898785425)(27,0.813008130081301)(28,0.814574898785425)(29,0.815831987075929)(30,0.818846466287571)(31,0.822294022617124)(32,0.818181818181818)(33,0.820388349514563)(34,0.821630347054076)(35,0.821630347054076)(36,0.819433198380567)(37,0.820097244732577)(38,0.821052631578947)(39,0.821052631578947)(40,0.824858757062147)(41,0.825806451612903)(42,0.825806451612903)(43,0.827809215844786)(44,0.827809215844786)(45,0.824858757062147)(46,0.835341365461847)(47,0.837469975980785)(48,0.838141025641025)(49,0.838141025641025)(50,0.836131095123901)(51,0.837060702875399)(52,0.8352)(53,0.835868694955965)(54,0.836131095123901)(55,0.837729816147082)(56,0.835987261146497)(57,0.837837837837838)(58,0.836913285600636)(59,0.838247011952191)(60,0.837729816147082)(61,0.837729816147082)(62,0.838658146964856)(63,0.840510366826156)(64,0.840510366826156)(65,0.840927258193445)(66,0.838658146964856)(67,0.8416)(68,0.83974358974359)(69,0.839071257005604)(70,0.839071257005604)(71,0.839071257005604)(72,0.84185303514377)(73,0.841434262948207)(74,0.839171974522293)(75,0.839171974522293)(76,0.839171974522293)(77,0.838247011952191)(78,0.839171974522293)(79,0.840095465393795)(80,0.839171974522293)(81,0.840095465393795)(82,0.839427662957075)(83,0.837579617834395)(84,0.838504375497215)(85,0.840349483717236)(86,0.833860759493671)(87,0.8318863456985)(88,0.833201581027668)(89,0.833201581027668)(90,0.83452098178939)(91,0.833201581027668)(92,0.833860759493671)(93,0.833860759493671)(94,0.833860759493671)(95,0.835703001579779)(96,0.837430610626487)(97,0.839271575613618)(98,0.839271575613618)(99,0.841938046068308)(100,0.840095465393795)(101,0.841017488076311)(102,0.840349483717236)(103,0.842188739095955)(104,0.841269841269841)(105,0.840602696272799)(106,0.840602696272799)(107,0.838760921366164)(108,0.838760921366164)(109,0.83968253968254)(110,0.839427662957075)(111,0.839427662957075)(112,0.839427662957075)(113,0.838504375497215)(114,0.840349483717236)(115,0.840349483717236)(116,0.841269841269841)(117,0.841269841269841)(118,0.83968253968254)(119,0.83968253968254)(120,0.840602696272799)(121,0.841521394611727)(122,0.842438638163104)(123,0.842438638163104)(124,0.843354430379747)(125,0.843354430379747)(126,0.843354430379747)(127,0.842022116903633)(128,0.843354430379747)(129,0.842188739095955)(130,0.841938046068308)(131,0.842857142857143)(132,0.839525691699605)(133,0.838862559241706)(134,0.839779005524862)(135,0.839116719242902)(136,0.839116719242902)(137,0.839779005524862)(138,0.840694006309148)(139,0.840694006309148)(140,0.838862559241706)(141,0.838455476753349)(142,0.836879432624113)(143,0.836879432624113)(144,0.838455476753349)(145,0.838455476753349)(146,0.838455476753349)(147,0.839370078740157)(148,0.839116719242902)(149,0.838200473559589)(150,0.838200473559589)(151,0.836621941594317)(152,0.836621941594317)(153,0.839370078740157)(154,0.839779005524862)(155,0.84044233807267)(156,0.84044233807267)(157,0.841357537490134)(158,0.841357537490134)(159,0.841357537490134)(160,0.841357537490134)(161,0.841357537490134)(162,0.841357537490134)(163,0.839525691699605)(164,0.839525691699605)(165,0.839525691699605)(166,0.839525691699605)(167,0.84044233807267)(168,0.84044233807267)(169,0.840189873417721)(170,0.840189873417721)(171,0.837430610626487)(172,0.838351822503962)(173,0.838351822503962)(174,0.839525691699605)(175,0.839525691699605)(176,0.839525691699605)(177,0.839525691699605)(178,0.840189873417721)(179,0.841357537490134)(180,0.841106719367589)(181,0.841106719367589)(182,0.841106719367589)(183,0.840189873417721)(184,0.841357537490134)(185,0.841357537490134)(186,0.840694006309148)(187,0.840694006309148)(188,0.840694006309148)(189,0.838455476753349)(190,0.839116719242902)(191,0.838455476753349)(192,0.838455476753349)(193,0.838455476753349)(194,0.838455476753349)(195,0.837795275590551)(196,0.837795275590551)(197,0.837795275590551)(198,0.838455476753349)(199,0.838455476753349)(200,0.837795275590551)(201,0.836477987421384)(202,0.837136113296617)(203,0.838050314465409)(204,0.838050314465409)(205,0.838963079340141)(206,0.838963079340141)(207,0.838963079340141)(208,0.842271293375394)(209,0.842271293375394)(210,0.84251968503937)(211,0.84251968503937)(212,0.843183609141056)(213,0.843183609141056)(214,0.84251968503937)(215,0.841856805664831)(216,0.84251968503937)(217,0.84251968503937)(218,0.84251968503937)(219,0.84160756501182)(220,0.84160756501182)(221,0.842271293375394)(222,0.842936069455406)(223,0.844759653270291)(224,0.844759653270291)(225,0.845669291338583)(226,0.844759653270291)(227,0.843183609141056)(228,0.845849802371541)(229,0.845849802371541)(230,0.841856805664831)(231,0.842767295597484)(232,0.842767295597484)(233,0.842767295597484)(234,0.842767295597484)(235,0.843676355066771)(236,0.843676355066771)(237,0.843676355066771)(238,0.843676355066771)(239,0.845490196078431)(240,0.845490196078431)(241,0.845490196078431)(242,0.845490196078431)(243,0.844827586206896)(244,0.844827586206896)(245,0.844827586206896)(246,0.844827586206896)(247,0.844827586206896)(248,0.844827586206896)(249,0.845490196078431)(250,0.84458398744113)(251,0.84458398744113)(252,0.84458398744113)(253,0.84458398744113)(254,0.845490196078431)(255,0.846394984326019)(256,0.845490196078431)(257,0.845490196078431)(258,0.845490196078431)(259,0.846153846153846)(260,0.846153846153846)(261,0.846153846153846)(262,0.846153846153846)(263,0.846153846153846)(264,0.846153846153846)(265,0.846153846153846)(266,0.850157728706624)(267,0.850157728706624)(268,0.850157728706624)(269,0.850157728706624)(270,0.850157728706624)(271,0.850157728706624)(272,0.850157728706624)(273,0.851968503937008)(274,0.851968503937008)(275,0.851968503937008)(276,0.851968503937008)(277,0.851968503937008)(278,0.853312302839117)(279,0.853312302839117)(280,0.854215918045705)(281,0.853543307086614)(282,0.853543307086614)(283,0.853543307086614)(284,0.854215918045705)(285,0.854215918045705)(286,0.853312302839117)(287,0.854215918045705)(288,0.853312302839117)(289,0.853312302839117)(290,0.853312302839117)(291,0.853312302839117)(292,0.85240726124704)(293,0.85240726124704)(294,0.85240726124704)(295,0.85240726124704)(296,0.85240726124704)(297,0.85240726124704)(298,0.85240726124704)(299,0.85240726124704)(300,0.85240726124704)(301,0.85240726124704)(302,0.85240726124704)(303,0.85240726124704)(304,0.85240726124704)(305,0.85240726124704)(306,0.85240726124704)(307,0.850592885375494)(308,0.850592885375494)(309,0.850592885375494)(310,0.850592885375494)(311,0.850592885375494)(312,0.850592885375494)(313,0.849683544303797)(314,0.849683544303797)(315,0.848772763262074)(316,0.848772763262074)(317,0.848772763262074)(318,0.848772763262074)(319,0.848772763262074)(320,0.848772763262074)(321,0.848772763262074)(322,0.848772763262074)(323,0.848772763262074)(324,0.848772763262074)(325,0.848772763262074)(326,0.848772763262074)(327,0.853563038371182)(328,0.852895148669796)(329,0.853568800588668)(330,0.858407079646018)(331,0.860119047619048)(332,0.861152141802068)(333,0.863703703703703)(334,0.862657757980698)(335,0.857354028085735)(336,0.858407079646018)(337,0.870342771982116)(338,0.873303167420814)(339,0.86910197869102)(340,0.86910197869102)(341,0.872030651340996)(342,0.87148288973384)(343,0.875945537065053)(344,0.8763197586727)(345,0.873684210526316)(346,0.874341610233258)(347,0.877643504531722)(348,0.878306878306878)(349,0.8763197586727)(350,0.879699248120301)(351,0.878378378378378)(352,0.879219804951238)(353,0.879219804951238)(354,0.878195488721804)(355,0.879699248120301)(356,0.880540946656649)(357,0.880540946656649)(358,0.879699248120301)(359,0.880540946656649)(360,0.880540946656649)(361,0.881381381381381)(362,0.881381381381381)(363,0.880059970014992)(364,0.882043576258452)(365,0.883370955605718)(366,0.883370955605718)(367,0.882530120481928)(368,0.882530120481928)(369,0.882530120481928)(370,0.882530120481928)(371,0.882530120481928)(372,0.882706766917293)(373,0.883546205860255)(374,0.882706766917293)(375,0.882043576258452)(376,0.882706766917293)(377,0.883370955605718)(378,0.882706766917293)(379,0.883370955605718)(380,0.883370955605718)(381,0.882706766917293)(382,0.883370955605718)(383,0.883370955605718)(384,0.882706766917293)(385,0.884702336096458)(386,0.886037735849057)(387,0.885865457294029)(388,0.887547169811321)(389,0.88821752265861)(390,0.88821752265861)(391,0.88821752265861)(392,0.88821752265861)(393,0.887547169811321)(394,0.887547169811321)(395,0.886877828054299)(396,0.886877828054299)(397,0.887547169811321)(398,0.888888888888889)(399,0.888888888888889)(400,0.888888888888889) 
};
\addplot [
color=red,
mark size=0.1pt,
only marks,
mark=*,
mark options={solid,fill=red},
forget plot
]
coordinates{
 (1,0)(2,0)(3,0.23953488372093)(4,0.241784037558685)(5,0.172808132147395)(6,0.205256570713392)(7,0.648506151142355)(8,0.733853797019162)(9,0.770149253731343)(10,0.794348508634223)(11,0.801271860095389)(12,0.801909307875895)(13,0.797488226059655)(14,0.792094861660079)(15,0.797458300238284)(16,0.797136038186158)(17,0.795836669335468)(18,0.792880258899676)(19,0.792239288601455)(20,0.794188861985472)(21,0.794212218649518)(22,0.789431545236189)(23,0.794585987261146)(24,0.796116504854369)(25,0.790996784565916)(26,0.793269230769231)(27,0.79646017699115)(28,0.807017543859649)(29,0.815518606492478)(30,0.814580031695721)(31,0.812698412698413)(32,0.812400635930048)(33,0.8152260111023)(34,0.821484992101106)(35,0.823436262866192)(36,0.823809523809524)(37,0.823809523809524)(38,0.823809523809524)(39,0.824463860206513)(40,0.823529411764706)(41,0.82818685669042)(42,0.824463860206513)(43,0.823156225218081)(44,0.825019794140934)(45,0.825949367088607)(46,0.823809523809524)(47,0.823156225218081)(48,0.822134387351778)(49,0.820836621941594)(50,0.820836621941594)(51,0.820189274447949)(52,0.820754716981132)(53,0.823343848580441)(54,0.824644549763033)(55,0.823993685872139)(56,0.823622047244094)(57,0.834115805946792)(58,0.834115805946792)(59,0.834115805946792)(60,0.833855799373041)(61,0.836193447737909)(62,0.837753510140405)(63,0.835147744945568)(64,0.835147744945568)(65,0.834241245136187)(66,0.834241245136187)(67,0.834241245136187)(68,0.834241245136187)(69,0.835541699142634)(70,0.835541699142634)(71,0.838154808444097)(72,0.838154808444097)(73,0.838154808444097)(74,0.838407494145199)(75,0.838154808444097)(76,0.838154808444097)(77,0.836846213895394)(78,0.835541699142634)(79,0.836193447737909)(80,0.836193447737909)(81,0.8375)(82,0.836846213895394)(83,0.8375)(84,0.8375)(85,0.8375)(86,0.838407494145199)(87,0.837100545596259)(88,0.837753510140405)(89,0.838407494145199)(90,0.838407494145199)(91,0.838407494145199)(92,0.839718530101642)(93,0.8390625)(94,0.8390625)(95,0.8390625)(96,0.83579766536965)(97,0.83579766536965)(98,0.83579766536965)(99,0.837100545596259)(100,0.837753510140405)(101,0.84037558685446)(102,0.839718530101642)(103,0.838810641627543)(104,0.8390625)(105,0.8390625)(106,0.8390625)(107,0.8390625)(108,0.8390625)(109,0.838407494145199)(110,0.837753510140405)(111,0.837100545596259)(112,0.837100545596259)(113,0.839813374805599)(114,0.841614906832298)(115,0.839813374805599)(116,0.839813374805599)(117,0.840466926070039)(118,0.839813374805599)(119,0.839813374805599)(120,0.839813374805599)(121,0.841777084957132)(122,0.841777084957132)(123,0.841777084957132)(124,0.842433697347894)(125,0.842433697347894)(126,0.843091334894613)(127,0.843091334894613)(128,0.843091334894613)(129,0.843091334894613)(130,0.843091334894613)(131,0.843091334894613)(132,0.843091334894613)(133,0.841121495327103)(134,0.841121495327103)(135,0.843091334894613)(136,0.84375)(137,0.844409695074277)(138,0.844409695074277)(139,0.844409695074277)(140,0.845070422535211)(141,0.844409695074277)(142,0.844409695074277)(143,0.8453125)(144,0.843091334894613)(145,0.844652615144418)(146,0.844652615144418)(147,0.8453125)(148,0.8453125)(149,0.8453125)(150,0.8453125)(151,0.8453125)(152,0.8453125)(153,0.845973416731822)(154,0.845973416731822)(155,0.8453125)(156,0.8453125)(157,0.845973416731822)(158,0.846635367762128)(159,0.845973416731822)(160,0.8453125)(161,0.8453125)(162,0.8453125)(163,0.846635367762128)(164,0.846635367762128)(165,0.846635367762128)(166,0.844827586206896)(167,0.844827586206896)(168,0.844827586206896)(169,0.844827586206896)(170,0.844827586206896)(171,0.845490196078431)(172,0.845490196078431)(173,0.845490196078431)(174,0.845490196078431)(175,0.845490196078431)(176,0.845490196078431)(177,0.846153846153846)(178,0.846153846153846)(179,0.847962382445141)(180,0.847298355520752)(181,0.848200312989045)(182,0.847537138389367)(183,0.847537138389367)(184,0.846635367762128)(185,0.847298355520752)(186,0.845973416731822)(187,0.845973416731822)(188,0.845973416731822)(189,0.845973416731822)(190,0.847962382445141)(191,0.847298355520752)(192,0.847962382445141)(193,0.847962382445141)(194,0.847962382445141)(195,0.847962382445141)(196,0.847962382445141)(197,0.847962382445141)(198,0.847962382445141)(199,0.847298355520752)(200,0.847298355520752)(201,0.846635367762128)(202,0.846635367762128)(203,0.846635367762128)(204,0.846635367762128)(205,0.846635367762128)(206,0.846635367762128)(207,0.846635367762128)(208,0.847298355520752)(209,0.847298355520752)(210,0.847298355520752)(211,0.847962382445141)(212,0.847962382445141)(213,0.847962382445141)(214,0.847962382445141)(215,0.847962382445141)(216,0.847962382445141)(217,0.847962382445141)(218,0.847962382445141)(219,0.847962382445141)(220,0.847962382445141)(221,0.847298355520752)(222,0.847298355520752)(223,0.846394984326019)(224,0.846394984326019)(225,0.846394984326019)(226,0.846394984326019)(227,0.845070422535211)(228,0.845070422535211)(229,0.845070422535211)(230,0.845732184808144)(231,0.846394984326019)(232,0.846394984326019)(233,0.845732184808144)(234,0.846394984326019)(235,0.84458398744113)(236,0.845247446975648)(237,0.845247446975648)(238,0.845247446975648)(239,0.846153846153846)(240,0.846153846153846)(241,0.846153846153846)(242,0.847723704866562)(243,0.846153846153846)(244,0.846153846153846)(245,0.846153846153846)(246,0.846153846153846)(247,0.847058823529412)(248,0.846153846153846)(249,0.846153846153846)(250,0.846153846153846)(251,0.848864526233359)(252,0.848864526233359)(253,0.847058823529412)(254,0.847962382445141)(255,0.847962382445141)(256,0.847962382445141)(257,0.847723704866562)(258,0.847723704866562)(259,0.848389630793401)(260,0.848389630793401)(261,0.849529780564263)(262,0.849529780564263)(263,0.849529780564263)(264,0.849529780564263)(265,0.849529780564263)(266,0.849529780564263)(267,0.850196078431372)(268,0.849529780564263)(269,0.849529780564263)(270,0.849529780564263)(271,0.849529780564263)(272,0.849529780564263)(273,0.849529780564263)(274,0.849529780564263)(275,0.849529780564263)(276,0.849529780564263)(277,0.849529780564263)(278,0.850430696945967)(279,0.850430696945967)(280,0.850430696945967)(281,0.851097178683385)(282,0.851764705882353)(283,0.851764705882353)(284,0.851097178683385)(285,0.850430696945967)(286,0.851097178683385)(287,0.850430696945967)(288,0.851097178683385)(289,0.850430696945967)(290,0.849765258215962)(291,0.849765258215962)(292,0.849100860046912)(293,0.849100860046912)(294,0.849100860046912)(295,0.849100860046912)(296,0.849100860046912)(297,0.849100860046912)(298,0.849100860046912)(299,0.849100860046912)(300,0.849100860046912)(301,0.849100860046912)(302,0.849100860046912)(303,0.849100860046912)(304,0.849100860046912)(305,0.849100860046912)(306,0.849765258215962)(307,0.850430696945967)(308,0.850430696945967)(309,0.850430696945967)(310,0.850430696945967)(311,0.851330203442879)(312,0.850664581704456)(313,0.850664581704456)(314,0.850664581704456)(315,0.850664581704456)(316,0.850664581704456)(317,0.850664581704456)(318,0.850664581704456)(319,0.850664581704456)(320,0.850664581704456)(321,0.850664581704456)(322,0.85)(323,0.85)(324,0.85)(325,0.85)(326,0.849336455893833)(327,0.849336455893833)(328,0.849336455893833)(329,0.85)(330,0.85)(331,0.85)(332,0.85)(333,0.849336455893833)(334,0.85)(335,0.85)(336,0.85)(337,0.85)(338,0.85)(339,0.85)(340,0.85)(341,0.85)(342,0.85)(343,0.85)(344,0.85)(345,0.85)(346,0.85)(347,0.849336455893833)(348,0.849336455893833)(349,0.848673946957878)(350,0.848673946957878)(351,0.8484375)(352,0.8484375)(353,0.8484375)(354,0.8484375)(355,0.8484375)(356,0.8484375)(357,0.8484375)(358,0.8484375)(359,0.849100860046912)(360,0.849100860046912)(361,0.849100860046912)(362,0.849100860046912)(363,0.849100860046912)(364,0.849100860046912)(365,0.849100860046912)(366,0.849100860046912)(367,0.849100860046912)(368,0.850430696945967)(369,0.849765258215962)(370,0.849100860046912)(371,0.849765258215962)(372,0.849100860046912)(373,0.849100860046912)(374,0.849100860046912)(375,0.849100860046912)(376,0.849100860046912)(377,0.849100860046912)(378,0.849100860046912)(379,0.849100860046912)(380,0.849100860046912)(381,0.849100860046912)(382,0.849100860046912)(383,0.851764705882353)(384,0.851764705882353)(385,0.851764705882353)(386,0.851764705882353)(387,0.851764705882353)(388,0.851764705882353)(389,0.851764705882353)(390,0.851764705882353)(391,0.851764705882353)(392,0.851764705882353)(393,0.851764705882353)(394,0.851764705882353)(395,0.851764705882353)(396,0.851764705882353)(397,0.851764705882353)(398,0.851764705882353)(399,0.851764705882353)(400,0.851764705882353) 
};
\addplot [
color=red,
mark size=0.1pt,
only marks,
mark=*,
mark options={solid,fill=red},
forget plot
]
coordinates{
 (1,0)(2,0)(3,0)(4,0)(5,0.0275103163686382)(6,0.163892445582586)(7,0.199029126213592)(8,0.160714285714286)(9,0.148802017654477)(10,0.245322245322245)(11,0.499184339314845)(12,0.614357262103506)(13,0.748122866894198)(14,0.746331236897274)(15,0.766787003610108)(16,0.767341040462428)(17,0.761837455830389)(18,0.771014492753623)(19,0.770788141720896)(20,0.785714285714286)(21,0.783882783882784)(22,0.780096308186196)(23,0.785425101214575)(24,0.7840260798696)(25,0.78530612244898)(26,0.823899371069182)(27,0.823255813953488)(28,0.82307092751364)(29,0.813532651455547)(30,0.81259842519685)(31,0.82004555808656)(32,0.820668693009118)(33,0.82004555808656)(34,0.825018615040953)(35,0.829962546816479)(36,0.837878787878788)(37,0.837878787878788)(38,0.841463414634146)(39,0.841945288753799)(40,0.849733028222731)(41,0.849128127369219)(42,0.89033189033189)(43,0.892219842969308)(44,0.9004329004329)(45,0.892597968069666)(46,0.894660894660895)(47,0.896601590744758)(48,0.899705014749262)(49,0.905355832721937)(50,0.905355832721937)(51,0.907759882869692)(52,0.906959706959707)(53,0.906959706959707)(54,0.914244186046511)(55,0.914244186046511)(56,0.912663755458515)(57,0.915451895043732)(58,0.917337234820775)(59,0.919941775836972)(60,0.906227630637079)(61,0.914909090909091)(62,0.916909620991254)(63,0.916909620991254)(64,0.91358024691358)(65,0.916241806263656)(66,0.918958031837916)(67,0.918056562726613)(68,0.917271407837445)(69,0.918056562726613)(70,0.918604651162791)(71,0.919272727272727)(72,0.920611798980335)(73,0.920058139534884)(74,0.917271407837445)(75,0.917937545388526)(76,0.917937545388526)(77,0.917937545388526)(78,0.915278783490224)(79,0.917937545388526)(80,0.916606236403191)(81,0.916606236403191)(82,0.917578409919767)(83,0.918367346938775)(84,0.91599707815924)(85,0.916666666666667)(86,0.916666666666667)(87,0.91599707815924)(88,0.916909620991254)(89,0.915328467153285)(90,0.916788321167883)(91,0.916119620714807)(92,0.916909620991254)(93,0.91599707815924)(94,0.91599707815924)(95,0.912)(96,0.910675381263616)(97,0.91332847778587)(98,0.914119359534206)(99,0.913454545454545)(100,0.913454545454545)(101,0.914119359534206)(102,0.915451895043732)(103,0.915328467153285)(104,0.914536157779401)(105,0.916422287390029)(106,0.91562729273661)(107,0.91562729273661)(108,0.914033798677443)(109,0.916666666666667)(110,0.918486171761281)(111,0.920957215373459)(112,0.920957215373459)(113,0.920957215373459)(114,0.922631959508315)(115,0.924078091106291)(116,0.92485549132948)(117,0.922519913106445)(118,0.923299565846599)(119,0.923520923520923)(120,0.922855082912761)(121,0.926193921852388)(122,0.92463768115942)(123,0.92463768115942)(124,0.92463768115942)(125,0.925308194343727)(126,0.925308194343727)(127,0.925308194343727)(128,0.925308194343727)(129,0.925524222704266)(130,0.925979680696662)(131,0.925979680696662)(132,0.925524222704266)(133,0.924297043979812)(134,0.924297043979812)(135,0.923520923520923)(136,0.923741007194245)(137,0.922190201729106)(138,0.922190201729106)(139,0.922190201729106)(140,0.922077922077922)(141,0.922077922077922)(142,0.924078091106291)(143,0.924078091106291)(144,0.923410404624277)(145,0.92507204610951)(146,0.925179856115108)(147,0.923741007194245)(148,0.925179856115108)(149,0.924297043979812)(150,0.924297043979812)(151,0.924297043979812)(152,0.926618705035971)(153,0.926618705035971)(154,0.928057553956834)(155,0.927390366642703)(156,0.927390366642703)(157,0.927390366642703)(158,0.926724137931034)(159,0.926512968299712)(160,0.926512968299712)(161,0.930366116295764)(162,0.930366116295764)(163,0.931899641577061)(164,0.93103448275862)(165,0.931133428981348)(166,0.930465949820788)(167,0.931133428981348)(168,0.931133428981348)(169,0.931133428981348)(170,0.930465949820788)(171,0.930465949820788)(172,0.930465949820788)(173,0.929698708751793)(174,0.929032258064516)(175,0.927038626609442)(176,0.927702219040802)(177,0.929698708751793)(178,0.929698708751793)(179,0.929698708751793)(180,0.929698708751793)(181,0.929698708751793)(182,0.929698708751793)(183,0.929032258064516)(184,0.929032258064516)(185,0.93016558675306)(186,0.93016558675306)(187,0.93016558675306)(188,0.93016558675306)(189,0.93016558675306)(190,0.93016558675306)(191,0.93016558675306)(192,0.930835734870317)(193,0.93016558675306)(194,0.93016558675306)(195,0.938571428571429)(196,0.947592067988668)(197,0.953048353188507)(198,0.959944751381215)(199,0.960662525879917)(200,0.959450171821306)(201,0.960110041265475)(202,0.953551912568306)(203,0.960055096418733)(204,0.962758620689655)(205,0.957417582417582)(206,0.957417582417582)(207,0.956760466712423)(208,0.960716747070985)(209,0.963372494816862)(210,0.963372494816862)(211,0.961218836565097)(212,0.961832061068702)(213,0.963168867268937)(214,0.965325936199723)(215,0.965325936199723)(216,0.964656964656964)(217,0.964656964656964)(218,0.964656964656964)(219,0.963938973647712)(220,0.963321799307958)(221,0.963321799307958)(222,0.962042788129744)(223,0.960716747070985)(224,0.962042788129744)(225,0.962707182320442)(226,0.962707182320442)(227,0.962707182320442)(228,0.962042788129744)(229,0.961379310344827)(230,0.962042788129744)(231,0.960716747070985)(232,0.960716747070985)(233,0.96)(234,0.960662525879917)(235,0.961379310344827)(236,0.960716747070985)(237,0.962042788129744)(238,0.959394356503785)(239,0.961379310344827)(240,0.959394356503785)(241,0.959394356503785)(242,0.959394356503785)(243,0.959394356503785)(244,0.958734525447043)(245,0.958734525447043)(246,0.958017894012388)(247,0.957359009628611)(248,0.957359009628611)(249,0.958017894012388)(250,0.958017894012388)(251,0.957417582417582)(252,0.95807560137457)(253,0.960716747070985)(254,0.960716747070985)(255,0.960716747070985)(256,0.960716747070985)(257,0.962147281486579)(258,0.962147281486579)(259,0.962147281486579)(260,0.962147281486579)(261,0.960164835164835)(262,0.961485557083906)(263,0.961432506887052)(264,0.961432506887052)(265,0.961432506887052)(266,0.961432506887052)(267,0.961432506887052)(268,0.962095106822881)(269,0.960662525879917)(270,0.961379310344827)(271,0.961379310344827)(272,0.960716747070985)(273,0.959394356503785)(274,0.960770818995182)(275,0.960110041265475)(276,0.960110041265475)(277,0.960110041265475)(278,0.960110041265475)(279,0.959394356503785)(280,0.959394356503785)(281,0.960770818995182)(282,0.960770818995182)(283,0.961432506887052)(284,0.961432506887052)(285,0.961379310344827)(286,0.962809917355372)(287,0.962809917355372)(288,0.962809917355372)(289,0.962147281486579)(290,0.962147281486579)(291,0.962809917355372)(292,0.962861072902338)(293,0.962861072902338)(294,0.962861072902338)(295,0.962861072902338)(296,0.962861072902338)(297,0.961432506887052)(298,0.960770818995182)(299,0.960716747070985)(300,0.960716747070985)(301,0.960716747070985)(302,0.960716747070985)(303,0.961432506887052)(304,0.961432506887052)(305,0.962095106822881)(306,0.962095106822881)(307,0.962095106822881)(308,0.961379310344827)(309,0.961379310344827)(310,0.960716747070985)(311,0.96)(312,0.96)(313,0.96)(314,0.96)(315,0.95933838731909)(316,0.95933838731909)(317,0.95933838731909)(318,0.95928226363009)(319,0.96)(320,0.96)(321,0.95933838731909)(322,0.958677685950413)(323,0.958620689655172)(324,0.957902001380262)(325,0.957902001380262)(326,0.957241379310345)(327,0.9565816678153)(328,0.9565816678153)(329,0.956521739130435)(330,0.9565816678153)(331,0.957300275482094)(332,0.957300275482094)(333,0.957300275482094)(334,0.957300275482094)(335,0.957960027567195)(336,0.957960027567195)(337,0.957960027567195)(338,0.960055096418733)(339,0.959394356503785)(340,0.959394356503785)(341,0.960055096418733)(342,0.960055096418733)(343,0.959394356503785)(344,0.959394356503785)(345,0.958734525447043)(346,0.958734525447043)(347,0.958734525447043)(348,0.960055096418733)(349,0.960055096418733)(350,0.961379310344827)(351,0.961379310344827)(352,0.961379310344827)(353,0.962707182320442)(354,0.964038727524205)(355,0.964038727524205)(356,0.964038727524205)(357,0.965469613259669)(358,0.965469613259669)(359,0.965469613259669)(360,0.965469613259669)(361,0.965469613259669)(362,0.965469613259669)(363,0.965469613259669)(364,0.96551724137931)(365,0.966230186078566)(366,0.96423658872077)(367,0.96423658872077)(368,0.964900206469374)(369,0.964900206469374)(370,0.964900206469374)(371,0.964900206469374)(372,0.965564738292011)(373,0.965564738292011)(374,0.964900206469374)(375,0.96694214876033)(376,0.96694214876033)(377,0.967608545830462)(378,0.966230186078566)(379,0.967608545830462)(380,0.967608545830462)(381,0.966276668960771)(382,0.966896551724138)(383,0.96694214876033)(384,0.96694214876033)(385,0.967608545830462)(386,0.968232044198895)(387,0.968232044198895)(388,0.966896551724138)(389,0.965469613259669)(390,0.966136834830684)(391,0.966183574879227)(392,0.966850828729282)(393,0.966183574879227)(394,0.966850828729282)(395,0.966850828729282)(396,0.964754664823773)(397,0.964088397790055)(398,0.964088397790055)(399,0.964038727524205)(400,0.961884961884962) 
};
\addplot [
color=red,
mark size=0.1pt,
only marks,
mark=*,
mark options={solid,fill=red},
forget plot
]
coordinates{
 (1,0)(2,0)(3,0)(4,0)(5,0)(6,0.610088070456365)(7,0.634691195795007)(8,0.694986072423398)(9,0.733333333333333)(10,0.747678018575851)(11,0.73186119873817)(12,0.7383100902379)(13,0.746887966804979)(14,0.73666384419983)(15,0.807045636509207)(16,0.807692307692308)(17,0.804761904761905)(18,0.813186813186813)(19,0.805379746835443)(20,0.811320754716981)(21,0.80814408770556)(22,0.811959087332809)(23,0.813586097946287)(24,0.812893081761006)(25,0.814814814814815)(26,0.814814814814815)(27,0.812893081761006)(28,0.812254516889238)(29,0.81042654028436)(30,0.815396700706991)(31,0.818466353677621)(32,0.82818685669042)(33,0.82818685669042)(34,0.825572217837411)(35,0.824367088607595)(36,0.828298887122416)(37,0.828411811652035)(38,0.829346092503987)(39,0.83147853736089)(40,0.832151300236407)(41,0.832151300236407)(42,0.827586206896551)(43,0.827586206896551)(44,0.827856025039124)(45,0.827208756841282)(46,0.82850430696946)(47,0.830336200156372)(48,0.83125)(49,0.829953198127925)(50,0.829953198127925)(51,0.828015564202335)(52,0.826728826728827)(53,0.824806201550388)(54,0.823255813953488)(55,0.828015564202335)(56,0.828015564202335)(57,0.828282828282828)(58,0.828282828282828)(59,0.828015564202335)(60,0.827371695178849)(61,0.824629773967264)(62,0.827856025039124)(63,0.827856025039124)(64,0.828125)(65,0.824902723735408)(66,0.827102803738318)(67,0.828393135725429)(68,0.82903981264637)(69,0.829420970266041)(70,0.829420970266041)(71,0.830070477682067)(72,0.830985915492958)(73,0.830601092896175)(74,0.83125)(75,0.830985915492958)(76,0.831636648394675)(77,0.831636648394675)(78,0.828885400313972)(79,0.832550860719875)(80,0.833855799373041)(81,0.833855799373041)(82,0.831107619795758)(83,0.831107619795758)(84,0.831761006289308)(85,0.832678711704635)(86,0.832678711704635)(87,0.832025117739403)(88,0.832678711704635)(89,0.833988985051141)(90,0.835303388494878)(91,0.83794466403162)(92,0.838607594936709)(93,0.838607594936709)(94,0.840602696272799)(95,0.841269841269841)(96,0.841938046068308)(97,0.841938046068308)(98,0.841269841269841)(99,0.840349483717236)(100,0.840349483717236)(101,0.83968253968254)(102,0.840602696272799)(103,0.840602696272799)(104,0.840602696272799)(105,0.840349483717236)(106,0.840349483717236)(107,0.839016653449643)(108,0.839016653449643)(109,0.839016653449643)(110,0.839016653449643)(111,0.839016653449643)(112,0.840855106888361)(113,0.841772151898734)(114,0.840855106888361)(115,0.840855106888361)(116,0.838862559241706)(117,0.838862559241706)(118,0.839370078740157)(119,0.838709677419355)(120,0.837795275590551)(121,0.837795275590551)(122,0.837795275590551)(123,0.836879432624113)(124,0.837136113296617)(125,0.837795275590551)(126,0.837795275590551)(127,0.837795275590551)(128,0.838709677419355)(129,0.839622641509434)(130,0.839622641509434)(131,0.839622641509434)(132,0.839622641509434)(133,0.836734693877551)(134,0.836734693877551)(135,0.836734693877551)(136,0.835820895522388)(137,0.838557993730407)(138,0.839467501957713)(139,0.839467501957713)(140,0.839467501957713)(141,0.839467501957713)(142,0.84037558685446)(143,0.84037558685446)(144,0.84012539184953)(145,0.84012539184953)(146,0.84078431372549)(147,0.84078431372549)(148,0.839874411302983)(149,0.842352941176471)(150,0.844166014095536)(151,0.840534171249018)(152,0.840534171249018)(153,0.840534171249018)(154,0.840534171249018)(155,0.838963079340141)(156,0.84078431372549)(157,0.839467501957713)(158,0.839874411302983)(159,0.839874411302983)(160,0.841194968553459)(161,0.841856805664831)(162,0.841856805664831)(163,0.84458398744113)(164,0.84458398744113)(165,0.84458398744113)(166,0.843676355066771)(167,0.843676355066771)(168,0.843676355066771)(169,0.84458398744113)(170,0.84458398744113)(171,0.845911949685534)(172,0.845911949685534)(173,0.845911949685534)(174,0.845247446975648)(175,0.845247446975648)(176,0.847058823529412)(177,0.847962382445141)(178,0.847962382445141)(179,0.847962382445141)(180,0.847298355520752)(181,0.847962382445141)(182,0.847962382445141)(183,0.847962382445141)(184,0.847962382445141)(185,0.846153846153846)(186,0.846153846153846)(187,0.846153846153846)(188,0.846153846153846)(189,0.845490196078431)(190,0.845490196078431)(191,0.845490196078431)(192,0.846153846153846)(193,0.845247446975648)(194,0.847962382445141)(195,0.847962382445141)(196,0.847962382445141)(197,0.847962382445141)(198,0.849529780564263)(199,0.849529780564263)(200,0.849529780564263)(201,0.849529780564263)(202,0.849529780564263)(203,0.849529780564263)(204,0.849529780564263)(205,0.849529780564263)(206,0.849529780564263)(207,0.850196078431372)(208,0.850196078431372)(209,0.850196078431372)(210,0.850196078431372)(211,0.850196078431372)(212,0.850196078431372)(213,0.850196078431372)(214,0.850196078431372)(215,0.850196078431372)(216,0.850196078431372)(217,0.849529780564263)(218,0.849529780564263)(219,0.849529780564263)(220,0.849529780564263)(221,0.849529780564263)(222,0.848627450980392)(223,0.848627450980392)(224,0.848627450980392)(225,0.849293563579278)(226,0.849293563579278)(227,0.849293563579278)(228,0.849293563579278)(229,0.849960722702278)(230,0.849293563579278)(231,0.848389630793401)(232,0.849293563579278)(233,0.850196078431372)(234,0.848389630793401)(235,0.848389630793401)(236,0.847058823529412)(237,0.848389630793401)(238,0.847723704866562)(239,0.847723704866562)(240,0.847723704866562)(241,0.847723704866562)(242,0.846394984326019)(243,0.846394984326019)(244,0.846394984326019)(245,0.85)(246,0.85)(247,0.85)(248,0.850664581704456)(249,0.850664581704456)(250,0.851330203442879)(251,0.851330203442879)(252,0.851330203442879)(253,0.85)(254,0.85)(255,0.850664581704456)(256,0.851330203442879)(257,0.851996867658575)(258,0.851996867658575)(259,0.851996867658575)(260,0.851996867658575)(261,0.851996867658575)(262,0.851097178683385)(263,0.850196078431372)(264,0.850863422291994)(265,0.849529780564263)(266,0.849529780564263)(267,0.849529780564263)(268,0.848200312989045)(269,0.848200312989045)(270,0.848200312989045)(271,0.849529780564263)(272,0.848200312989045)(273,0.848200312989045)(274,0.848200312989045)(275,0.848200312989045)(276,0.847537138389367)(277,0.847537138389367)(278,0.8484375)(279,0.847537138389367)(280,0.846635367762128)(281,0.846635367762128)(282,0.847537138389367)(283,0.847537138389367)(284,0.848200312989045)(285,0.848864526233359)(286,0.848864526233359)(287,0.848864526233359)(288,0.848864526233359)(289,0.848864526233359)(290,0.848864526233359)(291,0.848864526233359)(292,0.846394984326019)(293,0.848389630793401)(294,0.847723704866562)(295,0.847723704866562)(296,0.847723704866562)(297,0.849293563579278)(298,0.849293563579278)(299,0.849293563579278)(300,0.849293563579278)(301,0.849293563579278)(302,0.848389630793401)(303,0.848389630793401)(304,0.848389630793401)(305,0.847723704866562)(306,0.848389630793401)(307,0.848389630793401)(308,0.847723704866562)(309,0.847723704866562)(310,0.848389630793401)(311,0.849056603773585)(312,0.850393700787401)(313,0.849724626278521)(314,0.849724626278521)(315,0.848818897637795)(316,0.848818897637795)(317,0.848818897637795)(318,0.848818897637795)(319,0.848818897637795)(320,0.849487785657998)(321,0.849487785657998)(322,0.847911741528763)(323,0.848151062155783)(324,0.84748427672956)(325,0.849056603773585)(326,0.849056603773585)(327,0.848389630793401)(328,0.847723704866562)(329,0.848389630793401)(330,0.848389630793401)(331,0.848389630793401)(332,0.849056603773585)(333,0.849724626278521)(334,0.848151062155783)(335,0.848151062155783)(336,0.848151062155783)(337,0.849724626278521)(338,0.849724626278521)(339,0.848818897637795)(340,0.848818897637795)(341,0.848818897637795)(342,0.848151062155783)(343,0.848151062155783)(344,0.84748427672956)(345,0.84748427672956)(346,0.846818538884525)(347,0.846818538884525)(348,0.848151062155783)(349,0.84748427672956)(350,0.848580441640378)(351,0.848580441640378)(352,0.848580441640378)(353,0.848580441640378)(354,0.849921011058452)(355,0.849921011058452)(356,0.849921011058452)(357,0.850828729281768)(358,0.850828729281768)(359,0.849921011058452)(360,0.850828729281768)(361,0.850828729281768)(362,0.849921011058452)(363,0.850828729281768)(364,0.850828729281768)(365,0.850828729281768)(366,0.850828729281768)(367,0.85173501577287)(368,0.85173501577287)(369,0.85173501577287)(370,0.85173501577287)(371,0.85263987391647)(372,0.85263987391647)(373,0.85263987391647)(374,0.85263987391647)(375,0.85263987391647)(376,0.85263987391647)(377,0.85263987391647)(378,0.85173501577287)(379,0.85173501577287)(380,0.85173501577287)(381,0.85263987391647)(382,0.85263987391647)(383,0.85263987391647)(384,0.85263987391647)(385,0.85263987391647)(386,0.85263987391647)(387,0.85263987391647)(388,0.85263987391647)(389,0.851968503937008)(390,0.851968503937008)(391,0.851968503937008)(392,0.851968503937008)(393,0.851968503937008)(394,0.851968503937008)(395,0.851968503937008)(396,0.851968503937008)(397,0.851968503937008)(398,0.851968503937008)(399,0.851968503937008)(400,0.851968503937008) 
};
\addplot [
color=red,
mark size=0.1pt,
only marks,
mark=*,
mark options={solid,fill=red},
forget plot
]
coordinates{
 (1,0)(2,0)(3,0)(4,0)(5,0)(6,0.716535433070866)(7,0.714754098360656)(8,0.715111478117258)(9,0.801526717557252)(10,0.778898370830101)(11,0.783653846153846)(12,0.782264449722882)(13,0.771875)(14,0.771764705882353)(15,0.765888978278359)(16,0.792156862745098)(17,0.796875)(18,0.804724409448819)(19,0.802250803858521)(20,0.810853950518755)(21,0.809943865276664)(22,0.811849479583667)(23,0.811849479583667)(24,0.815936254980079)(25,0.816653322658126)(26,0.81255028157683)(27,0.813204508856683)(28,0.81715210355987)(29,0.817518248175182)(30,0.809172809172809)(31,0.809172809172809)(32,0.808510638297872)(33,0.802631578947368)(34,0.796036333608588)(35,0.79504132231405)(36,0.792703150912106)(37,0.79933110367893)(38,0.811930405965203)(39,0.810945273631841)(40,0.818780889621087)(41,0.818780889621087)(42,0.818780889621087)(43,0.818780889621087)(44,0.818780889621087)(45,0.823914823914824)(46,0.823914823914824)(47,0.825838103025347)(48,0.830618892508143)(49,0.830618892508143)(50,0.828711256117455)(51,0.827473426001635)(52,0.829387755102041)(53,0.82843137254902)(54,0.828711256117455)(55,0.828711256117455)(56,0.831570382424735)(57,0.831570382424735)(58,0.833468724614135)(59,0.835360908353609)(60,0.835360908353609)(61,0.833063209076175)(62,0.827309236947791)(63,0.827309236947791)(64,0.827974276527331)(65,0.827031375703942)(66,0.826752618855761)(67,0.827031375703942)(68,0.827309236947791)(69,0.827974276527331)(70,0.831315577078289)(71,0.834008097165992)(72,0.834008097165992)(73,0.834008097165992)(74,0.831715210355987)(75,0.831315577078289)(76,0.82864038616251)(77,0.828250401284109)(78,0.828250401284109)(79,0.8304)(80,0.831334932054356)(81,0.832268370607029)(82,0.832268370607029)(83,0.832933653077538)(84,0.832)(85,0.833865814696486)(86,0.834532374100719)(87,0.834532374100719)(88,0.832535885167464)(89,0.836248012718601)(90,0.836104513064133)(91,0.835182250396196)(92,0.836104513064133)(93,0.835443037974684)(94,0.835443037974684)(95,0.834782608695652)(96,0.832012678288431)(97,0.832012678288431)(98,0.831620553359684)(99,0.832937450514648)(100,0.832937450514648)(101,0.833597464342314)(102,0.835182250396196)(103,0.834258524980174)(104,0.834258524980174)(105,0.835182250396196)(106,0.835844567803331)(107,0.836507936507936)(108,0.836507936507936)(109,0.836248012718601)(110,0.835322195704057)(111,0.834658187599364)(112,0.833995234312947)(113,0.831746031746032)(114,0.834658187599364)(115,0.833995234312947)(116,0.833995234312947)(117,0.833995234312947)(118,0.834258524980174)(119,0.835182250396196)(120,0.835182250396196)(121,0.836104513064133)(122,0.83794466403162)(123,0.83794466403162)(124,0.83794466403162)(125,0.836363636363636)(126,0.838200473559589)(127,0.839116719242902)(128,0.838200473559589)(129,0.840031520882585)(130,0.840031520882585)(131,0.839116719242902)(132,0.837282780410742)(133,0.837282780410742)(134,0.837282780410742)(135,0.836363636363636)(136,0.836363636363636)(137,0.836363636363636)(138,0.839779005524862)(139,0.839779005524862)(140,0.841357537490134)(141,0.841357537490134)(142,0.842271293375394)(143,0.842271293375394)(144,0.84251968503937)(145,0.84458398744113)(146,0.843014128728414)(147,0.842105263157895)(148,0.843921568627451)(149,0.844827586206896)(150,0.843921568627451)(151,0.843921568627451)(152,0.843014128728414)(153,0.842352941176471)(154,0.842352941176471)(155,0.840534171249018)(156,0.840534171249018)(157,0.840534171249018)(158,0.839622641509434)(159,0.839622641509434)(160,0.842352941176471)(161,0.842352941176471)(162,0.841444270015699)(163,0.841194968553459)(164,0.841856805664831)(165,0.843676355066771)(166,0.843676355066771)(167,0.843676355066771)(168,0.843676355066771)(169,0.843676355066771)(170,0.843676355066771)(171,0.843014128728414)(172,0.843014128728414)(173,0.843014128728414)(174,0.843014128728414)(175,0.843014128728414)(176,0.843676355066771)(177,0.845003933910307)(178,0.843676355066771)(179,0.845247446975648)(180,0.845247446975648)(181,0.845247446975648)(182,0.843676355066771)(183,0.844339622641509)(184,0.844339622641509)(185,0.843430369787569)(186,0.842271293375394)(187,0.842271293375394)(188,0.841357537490134)(189,0.841357537490134)(190,0.842022116903633)(191,0.839525691699605)(192,0.841357537490134)(193,0.842271293375394)(194,0.842271293375394)(195,0.842271293375394)(196,0.842271293375394)(197,0.84384858044164)(198,0.842271293375394)(199,0.843601895734597)(200,0.842687747035573)(201,0.843601895734597)(202,0.843601895734597)(203,0.843601895734597)(204,0.843601895734597)(205,0.84384858044164)(206,0.84384858044164)(207,0.84384858044164)(208,0.844514601420679)(209,0.844514601420679)(210,0.846335697399527)(211,0.846335697399527)(212,0.846335697399527)(213,0.845669291338583)(214,0.845669291338583)(215,0.845669291338583)(216,0.845669291338583)(217,0.845669291338583)(218,0.845669291338583)(219,0.844759653270291)(220,0.844759653270291)(221,0.845425867507886)(222,0.845425867507886)(223,0.844759653270291)(224,0.844759653270291)(225,0.845669291338583)(226,0.845669291338583)(227,0.845669291338583)(228,0.845669291338583)(229,0.845669291338583)(230,0.845669291338583)(231,0.845669291338583)(232,0.845669291338583)(233,0.845669291338583)(234,0.845669291338583)(235,0.845003933910307)(236,0.845003933910307)(237,0.845669291338583)(238,0.845669291338583)(239,0.844759653270291)(240,0.845669291338583)(241,0.845669291338583)(242,0.845669291338583)(243,0.846577498033045)(244,0.846577498033045)(245,0.847911741528763)(246,0.847911741528763)(247,0.847911741528763)(248,0.846577498033045)(249,0.846577498033045)(250,0.846577498033045)(251,0.845911949685534)(252,0.846577498033045)(253,0.847911741528763)(254,0.847911741528763)(255,0.847911741528763)(256,0.847911741528763)(257,0.849724626278521)(258,0.847244094488189)(259,0.847244094488189)(260,0.847244094488189)(261,0.847244094488189)(262,0.845849802371541)(263,0.845849802371541)(264,0.846518987341772)(265,0.846518987341772)(266,0.846518987341772)(267,0.845849802371541)(268,0.845849802371541)(269,0.844514601420679)(270,0.843601895734597)(271,0.843601895734597)(272,0.844514601420679)(273,0.844514601420679)(274,0.844514601420679)(275,0.845605700712589)(276,0.844268774703557)(277,0.844268774703557)(278,0.844268774703557)(279,0.842936069455406)(280,0.843601895734597)(281,0.844268774703557)(282,0.844268774703557)(283,0.844268774703557)(284,0.843601895734597)(285,0.843601895734597)(286,0.843601895734597)(287,0.843601895734597)(288,0.843601895734597)(289,0.843601895734597)(290,0.844268774703557)(291,0.844268774703557)(292,0.844268774703557)(293,0.844268774703557)(294,0.845181674565561)(295,0.845181674565561)(296,0.845181674565561)(297,0.845181674565561)(298,0.845849802371541)(299,0.846518987341772)(300,0.845849802371541)(301,0.845849802371541)(302,0.845849802371541)(303,0.845849802371541)(304,0.846518987341772)(305,0.846518987341772)(306,0.846518987341772)(307,0.845849802371541)(308,0.845849802371541)(309,0.845181674565561)(310,0.845181674565561)(311,0.845181674565561)(312,0.845181674565561)(313,0.847189231987332)(314,0.847860538827258)(315,0.849683544303797)(316,0.849683544303797)(317,0.849683544303797)(318,0.849683544303797)(319,0.849683544303797)(320,0.849683544303797)(321,0.849683544303797)(322,0.849683544303797)(323,0.849683544303797)(324,0.849683544303797)(325,0.849683544303797)(326,0.849683544303797)(327,0.850592885375494)(328,0.850592885375494)(329,0.850592885375494)(330,0.850592885375494)(331,0.850592885375494)(332,0.849921011058452)(333,0.848580441640378)(334,0.848580441640378)(335,0.848580441640378)(336,0.848580441640378)(337,0.848580441640378)(338,0.848580441640378)(339,0.848580441640378)(340,0.847671665351223)(341,0.84901185770751)(342,0.848341232227488)(343,0.848341232227488)(344,0.848341232227488)(345,0.848341232227488)(346,0.848341232227488)(347,0.848341232227488)(348,0.849250197316495)(349,0.849250197316495)(350,0.849921011058452)(351,0.849921011058452)(352,0.849921011058452)(353,0.849921011058452)(354,0.849921011058452)(355,0.849921011058452)(356,0.849921011058452)(357,0.849921011058452)(358,0.849921011058452)(359,0.849921011058452)(360,0.849250197316495)(361,0.849250197316495)(362,0.849250197316495)(363,0.849250197316495)(364,0.849250197316495)(365,0.849250197316495)(366,0.849250197316495)(367,0.849250197316495)(368,0.849250197316495)(369,0.849250197316495)(370,0.849250197316495)(371,0.849250197316495)(372,0.849250197316495)(373,0.849250197316495)(374,0.849250197316495)(375,0.849250197316495)(376,0.849250197316495)(377,0.849250197316495)(378,0.850157728706624)(379,0.849250197316495)(380,0.850157728706624)(381,0.849250197316495)(382,0.850157728706624)(383,0.850157728706624)(384,0.850157728706624)(385,0.850157728706624)(386,0.850157728706624)(387,0.850157728706624)(388,0.850157728706624)(389,0.850828729281768)(390,0.850828729281768)(391,0.850828729281768)(392,0.850828729281768)(393,0.850157728706624)(394,0.850157728706624)(395,0.849487785657998)(396,0.849487785657998)(397,0.849487785657998)(398,0.849487785657998)(399,0.849487785657998)(400,0.849487785657998) 
};
\addplot [
color=red,
mark size=0.1pt,
only marks,
mark=*,
mark options={solid,fill=red},
forget plot
]
coordinates{
 (1,0)(2,0)(3,0)(4,0)(5,0)(6,0)(7,0)(8,0)(9,0)(10,0.211442786069652)(11,0.205256570713392)(12,0.235571260306243)(13,0.240384615384615)(14,0.25206124852768)(15,0.252955082742317)(16,0.245530393325387)(17,0.250887573964497)(18,0.251184834123223)(19,0.251184834123223)(20,0.258823529411765)(21,0.261988304093567)(22,0.261988304093567)(23,0.262910798122066)(24,0.263529411764706)(25,0.251482799525504)(26,0.243146603098927)(27,0.243146603098927)(28,0.244897959183673)(29,0.255369928400955)(30,0.256837098692033)(31,0.256227758007117)(32,0.256227758007117)(33,0.257142857142857)(34,0.250596658711217)(35,0.248803827751196)(36,0.247298919567827)(37,0.243667068757539)(38,0.243961352657005)(39,0.243961352657005)(40,0.243667068757539)(41,0.243667068757539)(42,0.247596153846154)(43,0.24578313253012)(44,0.247596153846154)(45,0.251798561151079)(46,0.250602409638554)(47,0.250602409638554)(48,0.250300842358604)(49,0.250602409638554)(50,0.248492159227985)(51,0.24455205811138)(52,0.248792270531401)(53,0.250300842358604)(54,0.248192771084337)(55,0.249093107617896)(56,0.248792270531401)(57,0.248792270531401)(58,0.248792270531401)(59,0.248492159227985)(60,0.248492159227985)(61,0.257521058965102)(62,0.259615384615385)(63,0.259303721488595)(64,0.259303721488595)(65,0.259303721488595)(66,0.261390887290168)(67,0.263473053892216)(68,0.267622461170848)(69,0.267622461170848)(70,0.267622461170848)(71,0.266666666666667)(72,0.269689737470167)(73,0.269689737470167)(74,0.269689737470167)(75,0.269368295589988)(76,0.269689737470167)(77,0.271752085816448)(78,0.271752085816448)(79,0.271752085816448)(80,0.271752085816448)(81,0.271752085816448)(82,0.272076372315036)(83,0.272076372315036)(84,0.270011947431302)(85,0.270011947431302)(86,0.272076372315036)(87,0.270011947431302)(88,0.275862068965517)(89,0.275862068965517)(90,0.270011947431302)(91,0.270011947431302)(92,0.270011947431302)(93,0.267942583732057)(94,0.267942583732057)(95,0.270011947431302)(96,0.270011947431302)(97,0.267942583732057)(98,0.265868263473054)(99,0.265868263473054)(100,0.265868263473054)(101,0.269689737470167)(102,0.267942583732057)(103,0.267942583732057)(104,0.270011947431302)(105,0.270011947431302)(106,0.267942583732057)(107,0.267942583732057)(108,0.267942583732057)(109,0.269689737470167)(110,0.269689737470167)(111,0.269689737470167)(112,0.267942583732057)(113,0.267942583732057)(114,0.270011947431302)(115,0.270011947431302)(116,0.270011947431302)(117,0.270011947431302)(118,0.270011947431302)(119,0.270011947431302)(120,0.272401433691756)(121,0.272401433691756)(122,0.272727272727273)(123,0.272727272727273)(124,0.268585131894484)(125,0.268263473053892)(126,0.268263473053892)(127,0.268263473053892)(128,0.268263473053892)(129,0.266506602641056)(130,0.264423076923077)(131,0.264423076923077)(132,0.266506602641056)(133,0.268263473053892)(134,0.268263473053892)(135,0.268263473053892)(136,0.268263473053892)(137,0.268263473053892)(138,0.270334928229665)(139,0.270334928229665)(140,0.270334928229665)(141,0.268263473053892)(142,0.272401433691756)(143,0.270334928229665)(144,0.270334928229665)(145,0.269230769230769)(146,0.269230769230769)(147,0.269230769230769)(148,0.271308523409364)(149,0.273381294964029)(150,0.273381294964029)(151,0.273381294964029)(152,0.273381294964029)(153,0.273053892215569)(154,0.272727272727273)(155,0.272727272727273)(156,0.272727272727273)(157,0.272727272727273)(158,0.272727272727273)(159,0.272727272727273)(160,0.272727272727273)(161,0.272727272727273)(162,0.272727272727273)(163,0.272727272727273)(164,0.272727272727273)(165,0.272727272727273)(166,0.272727272727273)(167,0.272727272727273)(168,0.272401433691756)(169,0.272401433691756)(170,0.271752085816448)(171,0.271752085816448)(172,0.271752085816448)(173,0.271752085816448)(174,0.271752085816448)(175,0.271752085816448)(176,0.271752085816448)(177,0.271752085816448)(178,0.271752085816448)(179,0.271752085816448)(180,0.273483947681332)(181,0.271428571428571)(182,0.271428571428571)(183,0.272076372315036)(184,0.274135876042908)(185,0.274135876042908)(186,0.274135876042908)(187,0.274135876042908)(188,0.273809523809524)(189,0.276190476190476)(190,0.276190476190476)(191,0.276190476190476)(192,0.278240190249703)(193,0.278240190249703)(194,0.278240190249703)(195,0.278240190249703)(196,0.278240190249703)(197,0.278240190249703)(198,0.278240190249703)(199,0.278240190249703)(200,0.278240190249703)(201,0.278240190249703)(202,0.278240190249703)(203,0.27790973871734)(204,0.279952550415184)(205,0.280618311533888)(206,0.280618311533888)(207,0.280952380952381)(208,0.281287246722288)(209,0.281287246722288)(210,0.280618311533888)(211,0.280285035629454)(212,0.280285035629454)(213,0.280618311533888)(214,0.280952380952381)(215,0.280952380952381)(216,0.280952380952381)(217,0.280952380952381)(218,0.280952380952381)(219,0.280952380952381)(220,0.280952380952381)(221,0.280952380952381)(222,0.280952380952381)(223,0.280952380952381)(224,0.280952380952381)(225,0.280952380952381)(226,0.280952380952381)(227,0.280952380952381)(228,0.280952380952381)(229,0.280952380952381)(230,0.280952380952381)(231,0.280952380952381)(232,0.280952380952381)(233,0.280952380952381)(234,0.280952380952381)(235,0.280952380952381)(236,0.280952380952381)(237,0.280952380952381)(238,0.280952380952381)(239,0.280952380952381)(240,0.280952380952381)(241,0.280952380952381)(242,0.280952380952381)(243,0.280952380952381)(244,0.280952380952381)(245,0.280952380952381)(246,0.280952380952381)(247,0.280952380952381)(248,0.280952380952381)(249,0.281287246722288)(250,0.281287246722288)(251,0.281287246722288)(252,0.281287246722288)(253,0.281287246722288)(254,0.281287246722288)(255,0.281287246722288)(256,0.281287246722288)(257,0.281287246722288)(258,0.281287246722288)(259,0.281287246722288)(260,0.281287246722288)(261,0.281287246722288)(262,0.281287246722288)(263,0.281287246722288)(264,0.281287246722288)(265,0.281287246722288)(266,0.29112426035503)(267,0.291469194312796)(268,0.297169811320755)(269,0.317016317016317)(270,0.320930232558139)(271,0.320930232558139)(272,0.320930232558139)(273,0.324449594438007)(274,0.324449594438007)(275,0.332179930795848)(276,0.330254041570439)(277,0.332179930795848)(278,0.332179930795848)(279,0.334101382488479)(280,0.334101382488479)(281,0.334101382488479)(282,0.336018411967779)(283,0.344036697247706)(284,0.351598173515982)(285,0.351598173515982)(286,0.353075170842825)(287,0.351197263397947)(288,0.356818181818182)(289,0.362400906002265)(290,0.363841807909604)(291,0.371203599550056)(292,0.364253393665158)(293,0.36978579481398)(294,0.367945823927765)(295,0.36978579481398)(296,0.36978579481398)(297,0.373453318335208)(298,0.373033707865168)(299,0.371203599550056)(300,0.36978579481398)(301,0.369369369369369)(302,0.369369369369369)(303,0.369369369369369)(304,0.36978579481398)(305,0.36978579481398)(306,0.37020316027088)(307,0.37020316027088)(308,0.368361581920904)(309,0.367945823927765)(310,0.36978579481398)(311,0.367945823927765)(312,0.371621621621621)(313,0.371621621621621)(314,0.371621621621621)(315,0.371621621621621)(316,0.371621621621621)(317,0.371621621621621)(318,0.371621621621621)(319,0.371621621621621)(320,0.371621621621621)(321,0.371621621621621)(322,0.36978579481398)(323,0.371621621621621)(324,0.373453318335208)(325,0.373453318335208)(326,0.373453318335208)(327,0.371621621621621)(328,0.371621621621621)(329,0.372040586245772)(330,0.372040586245772)(331,0.372040586245772)(332,0.372040586245772)(333,0.372040586245772)(334,0.372040586245772)(335,0.372040586245772)(336,0.372040586245772)(337,0.373873873873874)(338,0.373453318335208)(339,0.373453318335208)(340,0.373453318335208)(341,0.375280898876404)(342,0.373453318335208)(343,0.373873873873874)(344,0.373873873873874)(345,0.373873873873874)(346,0.373873873873874)(347,0.373873873873874)(348,0.373873873873874)(349,0.373873873873874)(350,0.373873873873874)(351,0.373873873873874)(352,0.373873873873874)(353,0.37020316027088)(354,0.372040586245772)(355,0.372040586245772)(356,0.372040586245772)(357,0.373873873873874)(358,0.373873873873874)(359,0.373873873873874)(360,0.373873873873874)(361,0.373873873873874)(362,0.37570303712036)(363,0.373873873873874)(364,0.373873873873874)(365,0.373873873873874)(366,0.373873873873874)(367,0.373873873873874)(368,0.373873873873874)(369,0.373873873873874)(370,0.373453318335208)(371,0.373453318335208)(372,0.373873873873874)(373,0.373453318335208)(374,0.373453318335208)(375,0.373453318335208)(376,0.373453318335208)(377,0.373453318335208)(378,0.373453318335208)(379,0.373453318335208)(380,0.373453318335208)(381,0.373453318335208)(382,0.373453318335208)(383,0.373453318335208)(384,0.373453318335208)(385,0.373453318335208)(386,0.373873873873874)(387,0.373873873873874)(388,0.373873873873874)(389,0.373873873873874)(390,0.373873873873874)(391,0.373873873873874)(392,0.373873873873874)(393,0.37570303712036)(394,0.37570303712036)(395,0.37570303712036)(396,0.37570303712036)(397,0.37570303712036)(398,0.37570303712036)(399,0.37570303712036)(400,0.37570303712036) 
};
\addplot [
color=red,
mark size=0.1pt,
only marks,
mark=*,
mark options={solid,fill=red},
forget plot
]
coordinates{
 (1,0)(2,0)(3,0)(4,0)(5,0)(6,0)(7,0.397299903567984)(8,0.549707602339181)(9,0.573346116970278)(10,0.762836185819071)(11,0.763754045307443)(12,0.744838976052849)(13,0.748735244519393)(14,0.756260434056761)(15,0.754371357202331)(16,0.748752079866888)(17,0.760726072607261)(18,0.76126878130217)(19,0.76759410801964)(20,0.768979591836735)(21,0.786806114239742)(22,0.7904)(23,0.801587301587302)(24,0.789983844911147)(25,0.7984)(26,0.798387096774194)(27,0.801282051282051)(28,0.793021411578112)(29,0.7984)(30,0.791967871485944)(31,0.792574656981436)(32,0.790996784565916)(33,0.791633145615447)(34,0.794498381877022)(35,0.794830371567044)(36,0.804487179487179)(37,0.805132317562149)(38,0.806709265175719)(39,0.809943865276664)(40,0.8096)(41,0.810593900481541)(42,0.811244979919679)(43,0.820063694267516)(44,0.820063694267516)(45,0.819047619047619)(46,0.822503961965135)(47,0.822310756972111)(48,0.824088748019017)(49,0.819620253164557)(50,0.821484992101106)(51,0.815686274509804)(52,0.816967792615868)(53,0.818253343823761)(54,0.818253343823761)(55,0.818253343823761)(56,0.817610062893082)(57,0.817966903073286)(58,0.819542947202522)(59,0.818253343823761)(60,0.819542947202522)(61,0.819542947202522)(62,0.820754716981132)(63,0.825371965544244)(64,0.827315541601256)(65,0.826666666666666)(66,0.823899371069182)(67,0.826224328593997)(68,0.82753164556962)(69,0.826877470355731)(70,0.8256735340729)(71,0.826603325415677)(72,0.82791435368755)(73,0.833865814696486)(74,0.835868694955965)(75,0.835868694955965)(76,0.838658146964856)(77,0.838658146964856)(78,0.838658146964856)(79,0.838658146964856)(80,0.838658146964856)(81,0.838658146964856)(82,0.839584996009577)(83,0.840255591054313)(84,0.840672538030424)(85,0.840927258193445)(86,0.840927258193445)(87,0.842948717948718)(88,0.842948717948718)(89,0.842948717948718)(90,0.842948717948718)(91,0.8416)(92,0.840672538030424)(93,0.842525979216627)(94,0.844373503591381)(95,0.844621513944223)(96,0.846215139442231)(97,0.846215139442231)(98,0.845295055821372)(99,0.843450479233227)(100,0.843027888446215)(101,0.842356687898089)(102,0.843027888446215)(103,0.843027888446215)(104,0.843277645186953)(105,0.842607313195548)(106,0.842356687898089)(107,0.843277645186953)(108,0.843277645186953)(109,0.843027888446215)(110,0.846703733121525)(111,0.84853291038858)(112,0.847860538827258)(113,0.846703733121525)(114,0.847376788553259)(115,0.84805091487669)(116,0.84805091487669)(117,0.848966613672496)(118,0.848966613672496)(119,0.84805091487669)(120,0.848966613672496)(121,0.847133757961783)(122,0.847133757961783)(123,0.84688995215311)(124,0.847808764940239)(125,0.847133757961783)(126,0.847133757961783)(127,0.847808764940239)(128,0.849642004773269)(129,0.849642004773269)(130,0.849642004773269)(131,0.849642004773269)(132,0.848292295472597)(133,0.848966613672496)(134,0.85031847133758)(135,0.85031847133758)(136,0.849402390438247)(137,0.85031847133758)(138,0.85031847133758)(139,0.85031847133758)(140,0.85031847133758)(141,0.85031847133758)(142,0.85031847133758)(143,0.849402390438247)(144,0.849402390438247)(145,0.848484848484848)(146,0.848484848484848)(147,0.850079744816587)(148,0.850079744816587)(149,0.850079744816587)(150,0.8464)(151,0.8464)(152,0.847322142286171)(153,0.845723421262989)(154,0.845723421262989)(155,0.845723421262989)(156,0.845047923322684)(157,0.844124700239808)(158,0.844124700239808)(159,0.8432)(160,0.844124700239808)(161,0.844124700239808)(162,0.845476381104884)(163,0.8448)(164,0.8448)(165,0.8448)(166,0.845047923322684)(167,0.845047923322684)(168,0.845047923322684)(169,0.842273819055244)(170,0.8432)(171,0.845047923322684)(172,0.845047923322684)(173,0.845047923322684)(174,0.845295055821372)(175,0.844373503591381)(176,0.845295055821372)(177,0.845295055821372)(178,0.845295055821372)(179,0.844373503591381)(180,0.844373503591381)(181,0.844373503591381)(182,0.844373503591381)(183,0.845969672785315)(184,0.844373503591381)(185,0.844373503591381)(186,0.846215139442231)(187,0.844373503591381)(188,0.843450479233227)(189,0.843450479233227)(190,0.843450479233227)(191,0.845047923322684)(192,0.845047923322684)(193,0.8448)(194,0.8448)(195,0.8448)(196,0.8448)(197,0.845723421262989)(198,0.845969672785315)(199,0.845047923322684)(200,0.845047923322684)(201,0.844373503591381)(202,0.844373503591381)(203,0.844373503591381)(204,0.843700159489633)(205,0.843700159489633)(206,0.843700159489633)(207,0.844621513944223)(208,0.844621513944223)(209,0.844621513944223)(210,0.844868735083532)(211,0.844868735083532)(212,0.845786963434022)(213,0.845786963434022)(214,0.845786963434022)(215,0.847133757961783)(216,0.847133757961783)(217,0.847133757961783)(218,0.847133757961783)(219,0.847133757961783)(220,0.847133757961783)(221,0.84805091487669)(222,0.848726114649681)(223,0.848484848484848)(224,0.847808764940239)(225,0.847133757961783)(226,0.847808764940239)(227,0.85031847133758)(228,0.85214626391097)(229,0.85214626391097)(230,0.85214626391097)(231,0.851233094669849)(232,0.851233094669849)(233,0.851233094669849)(234,0.851233094669849)(235,0.851233094669849)(236,0.854415274463007)(237,0.854415274463007)(238,0.853736089030207)(239,0.853057982525814)(240,0.853057982525814)(241,0.853736089030207)(242,0.853736089030207)(243,0.853736089030207)(244,0.853736089030207)(245,0.854415274463007)(246,0.854415274463007)(247,0.854415274463007)(248,0.854415274463007)(249,0.854415274463007)(250,0.853503184713376)(251,0.853503184713376)(252,0.853503184713376)(253,0.853503184713376)(254,0.854415274463007)(255,0.853736089030207)(256,0.853736089030207)(257,0.853736089030207)(258,0.853736089030207)(259,0.853736089030207)(260,0.853736089030207)(261,0.853736089030207)(262,0.854646544876886)(263,0.854646544876886)(264,0.853968253968254)(265,0.853968253968254)(266,0.853968253968254)(267,0.853968253968254)(268,0.853968253968254)(269,0.853291038858049)(270,0.853291038858049)(271,0.853291038858049)(272,0.853291038858049)(273,0.853291038858049)(274,0.853291038858049)(275,0.854199683042789)(276,0.854199683042789)(277,0.855555555555555)(278,0.855555555555555)(279,0.855555555555555)(280,0.855555555555555)(281,0.855555555555555)(282,0.855555555555555)(283,0.855555555555555)(284,0.856463124504361)(285,0.856463124504361)(286,0.856463124504361)(287,0.857142857142857)(288,0.856235107227959)(289,0.857142857142857)(290,0.857142857142857)(291,0.856463124504361)(292,0.856463124504361)(293,0.856463124504361)(294,0.856463124504361)(295,0.856463124504361)(296,0.856463124504361)(297,0.856463124504361)(298,0.855555555555555)(299,0.856463124504361)(300,0.856463124504361)(301,0.856463124504361)(302,0.856463124504361)(303,0.856463124504361)(304,0.856463124504361)(305,0.856463124504361)(306,0.856463124504361)(307,0.856463124504361)(308,0.856463124504361)(309,0.855555555555555)(310,0.855325914149443)(311,0.855325914149443)(312,0.855325914149443)(313,0.855325914149443)(314,0.855325914149443)(315,0.856235107227959)(316,0.856235107227959)(317,0.856235107227959)(318,0.856235107227959)(319,0.857823669579031)(320,0.857823669579031)(321,0.857823669579031)(322,0.857823669579031)(323,0.857823669579031)(324,0.857823669579031)(325,0.857823669579031)(326,0.857823669579031)(327,0.857823669579031)(328,0.857823669579031)(329,0.857823669579031)(330,0.857823669579031)(331,0.858730158730159)(332,0.858049167327518)(333,0.858049167327518)(334,0.858049167327518)(335,0.858049167327518)(336,0.858954041204437)(337,0.85827395091053)(338,0.858954041204437)(339,0.858954041204437)(340,0.86053882725832)(341,0.859635210150674)(342,0.858049167327518)(343,0.858049167327518)(344,0.858049167327518)(345,0.859857482185273)(346,0.859857482185273)(347,0.859857482185273)(348,0.859857482185273)(349,0.859857482185273)(350,0.859857482185273)(351,0.859857482185273)(352,0.859857482185273)(353,0.859857482185273)(354,0.859857482185273)(355,0.859857482185273)(356,0.859857482185273)(357,0.859857482185273)(358,0.859857482185273)(359,0.859857482185273)(360,0.859857482185273)(361,0.859857482185273)(362,0.859857482185273)(363,0.859857482185273)(364,0.859857482185273)(365,0.859857482185273)(366,0.859857482185273)(367,0.859857482185273)(368,0.859857482185273)(369,0.859857482185273)(370,0.859857482185273)(371,0.859857482185273)(372,0.859857482185273)(373,0.859857482185273)(374,0.859857482185273)(375,0.859857482185273)(376,0.859857482185273)(377,0.859857482185273)(378,0.859857482185273)(379,0.858954041204437)(380,0.859857482185273)(381,0.859857482185273)(382,0.859857482185273)(383,0.86053882725832)(384,0.86053882725832)(385,0.86053882725832)(386,0.86053882725832)(387,0.86053882725832)(388,0.86053882725832)(389,0.86053882725832)(390,0.86053882725832)(391,0.86053882725832)(392,0.86053882725832)(393,0.86053882725832)(394,0.86053882725832)(395,0.86053882725832)(396,0.86053882725832)(397,0.86053882725832)(398,0.86053882725832)(399,0.86053882725832)(400,0.86053882725832) 
};
\addplot [
color=red,
mark size=0.1pt,
only marks,
mark=*,
mark options={solid,fill=red},
forget plot
]
coordinates{
 (1,0)(2,0)(3,0)(4,0)(5,0)(6,0)(7,0)(8,0)(9,0)(10,0)(11,0.056910569105691)(12,0.259459459459459)(13,0.258620689655172)(14,0.259459459459459)(15,0.264026402640264)(16,0.267558528428094)(17,0.259681093394077)(18,0.26027397260274)(19,0.262068965517241)(20,0.263219741480611)(21,0.256532066508313)(22,0.256532066508313)(23,0.259433962264151)(24,0.259215219976219)(25,0.261390887290168)(26,0.272511848341232)(27,0.264285714285714)(28,0.260143198090692)(29,0.261593341260404)(30,0.257449344457688)(31,0.257988165680473)(32,0.260143198090692)(33,0.260143198090692)(34,0.259833134684148)(35,0.259833134684148)(36,0.259833134684148)(37,0.260143198090692)(38,0.253317249698432)(39,0.253317249698432)(40,0.251207729468599)(41,0.253012048192771)(42,0.254501800720288)(43,0.254501800720288)(44,0.252707581227437)(45,0.246674727932285)(46,0.245487364620939)(47,0.244897959183673)(48,0.245487364620939)(49,0.24578313253012)(50,0.241545893719807)(51,0.247894103489771)(52,0.247894103489771)(53,0.247894103489771)(54,0.247596153846154)(55,0.246674727932285)(56,0.246674727932285)(57,0.255421686746988)(58,0.255421686746988)(59,0.265868263473054)(60,0.25511432009627)(61,0.257211538461538)(62,0.263473053892216)(63,0.267622461170848)(64,0.267942583732057)(65,0.26378896882494)(66,0.26378896882494)(67,0.26378896882494)(68,0.259615384615385)(69,0.261704681872749)(70,0.26378896882494)(71,0.26491646778043)(72,0.26491646778043)(73,0.26491646778043)(74,0.26491646778043)(75,0.26555023923445)(76,0.26555023923445)(77,0.26378896882494)(78,0.265868263473054)(79,0.26378896882494)(80,0.265868263473054)(81,0.265868263473054)(82,0.264105642256903)(83,0.259927797833935)(84,0.266187050359712)(85,0.268263473053892)(86,0.270334928229665)(87,0.270334928229665)(88,0.270334928229665)(89,0.268263473053892)(90,0.27065868263473)(91,0.268263473053892)(92,0.268263473053892)(93,0.270334928229665)(94,0.270334928229665)(95,0.270334928229665)(96,0.270334928229665)(97,0.270334928229665)(98,0.268585131894484)(99,0.266506602641056)(100,0.268263473053892)(101,0.268263473053892)(102,0.268585131894484)(103,0.268585131894484)(104,0.268585131894484)(105,0.268585131894484)(106,0.268585131894484)(107,0.268585131894484)(108,0.268585131894484)(109,0.27065868263473)(110,0.270011947431302)(111,0.270334928229665)(112,0.268263473053892)(113,0.270334928229665)(114,0.270334928229665)(115,0.268263473053892)(116,0.270011947431302)(117,0.270011947431302)(118,0.270011947431302)(119,0.27065868263473)(120,0.270334928229665)(121,0.270334928229665)(122,0.268585131894484)(123,0.270334928229665)(124,0.270334928229665)(125,0.270011947431302)(126,0.270334928229665)(127,0.270334928229665)(128,0.270334928229665)(129,0.270334928229665)(130,0.270334928229665)(131,0.270011947431302)(132,0.269689737470167)(133,0.269689737470167)(134,0.270011947431302)(135,0.271752085816448)(136,0.270011947431302)(137,0.270334928229665)(138,0.270334928229665)(139,0.272076372315036)(140,0.272076372315036)(141,0.270334928229665)(142,0.27065868263473)(143,0.270334928229665)(144,0.270334928229665)(145,0.270334928229665)(146,0.270334928229665)(147,0.272401433691756)(148,0.272401433691756)(149,0.272401433691756)(150,0.274135876042908)(151,0.274135876042908)(152,0.274135876042908)(153,0.27790973871734)(154,0.27790973871734)(155,0.277580071174377)(156,0.279620853080569)(157,0.279289940828402)(158,0.279289940828402)(159,0.279289940828402)(160,0.279289940828402)(161,0.278959810874704)(162,0.278959810874704)(163,0.278959810874704)(164,0.276923076923077)(165,0.276923076923077)(166,0.276923076923077)(167,0.276269185360094)(168,0.276269185360094)(169,0.276269185360094)(170,0.274556213017751)(171,0.274881516587678)(172,0.274881516587678)(173,0.276595744680851)(174,0.274881516587678)(175,0.274881516587678)(176,0.274881516587678)(177,0.276923076923077)(178,0.277251184834123)(179,0.277251184834123)(180,0.277251184834123)(181,0.277251184834123)(182,0.277251184834123)(183,0.277251184834123)(184,0.277251184834123)(185,0.277251184834123)(186,0.277251184834123)(187,0.277251184834123)(188,0.277251184834123)(189,0.277251184834123)(190,0.277251184834123)(191,0.277251184834123)(192,0.277251184834123)(193,0.277251184834123)(194,0.277251184834123)(195,0.277251184834123)(196,0.277251184834123)(197,0.277251184834123)(198,0.277251184834123)(199,0.277251184834123)(200,0.277251184834123)(201,0.277251184834123)(202,0.277251184834123)(203,0.276923076923077)(204,0.277580071174377)(205,0.277580071174377)(206,0.277580071174377)(207,0.277580071174377)(208,0.277580071174377)(209,0.295081967213115)(210,0.281656804733728)(211,0.292397660818713)(212,0.285714285714286)(213,0.294736842105263)(214,0.308227114716107)(215,0.308227114716107)(216,0.316151202749141)(217,0.326530612244898)(218,0.328828828828829)(219,0.328828828828829)(220,0.326160815402038)(221,0.332584269662921)(222,0.338565022421525)(223,0.33970753655793)(224,0.351893095768374)(225,0.353726362625139)(226,0.360619469026549)(227,0.35920177383592)(228,0.359600443951165)(229,0.355704697986577)(230,0.355704697986577)(231,0.357541899441341)(232,0.359375)(233,0.363028953229399)(234,0.369878183831672)(235,0.369878183831672)(236,0.364849833147942)(237,0.364039955604883)(238,0.365663322185061)(239,0.366071428571428)(240,0.366071428571428)(241,0.365061590145577)(242,0.366480446927374)(243,0.364653243847875)(244,0.364653243847875)(245,0.370535714285714)(246,0.370122630992196)(247,0.367483296213808)(248,0.367892976588629)(249,0.368715083798883)(250,0.367301231802911)(251,0.366890380313199)(252,0.366071428571428)(253,0.366071428571428)(254,0.365663322185061)(255,0.364444444444444)(256,0.364444444444444)(257,0.364444444444444)(258,0.361607142857143)(259,0.361607142857143)(260,0.361607142857143)(261,0.360801781737194)(262,0.365853658536585)(263,0.365853658536585)(264,0.365853658536585)(265,0.365853658536585)(266,0.36766334440753)(267,0.36766334440753)(268,0.36766334440753)(269,0.36766334440753)(270,0.36766334440753)(271,0.36766334440753)(272,0.368070953436807)(273,0.367256637168142)(274,0.366666666666667)(275,0.368070953436807)(276,0.368479467258601)(277,0.36766334440753)(278,0.36766334440753)(279,0.36766334440753)(280,0.36766334440753)(281,0.367256637168142)(282,0.367256637168142)(283,0.367256637168142)(284,0.368070953436807)(285,0.366259711431742)(286,0.366259711431742)(287,0.366259711431742)(288,0.366259711431742)(289,0.366666666666667)(290,0.368479467258601)(291,0.368479467258601)(292,0.368479467258601)(293,0.366666666666667)(294,0.366666666666667)(295,0.366666666666667)(296,0.366666666666667)(297,0.366666666666667)(298,0.366666666666667)(299,0.367074527252503)(300,0.367074527252503)(301,0.367074527252503)(302,0.367074527252503)(303,0.367074527252503)(304,0.368888888888889)(305,0.369299221357063)(306,0.369299221357063)(307,0.368888888888889)(308,0.367074527252503)(309,0.367074527252503)(310,0.367483296213808)(311,0.371111111111111)(312,0.371111111111111)(313,0.370288248337029)(314,0.370288248337029)(315,0.370288248337029)(316,0.369710467706013)(317,0.367892976588629)(318,0.367892976588629)(319,0.367483296213808)(320,0.369299221357063)(321,0.368888888888889)(322,0.367483296213808)(323,0.367483296213808)(324,0.370122630992196)(325,0.370122630992196)(326,0.368715083798883)(327,0.368715083798883)(328,0.368715083798883)(329,0.366890380313199)(330,0.366890380313199)(331,0.366890380313199)(332,0.368715083798883)(333,0.368715083798883)(334,0.368715083798883)(335,0.368715083798883)(336,0.368715083798883)(337,0.368303571428571)(338,0.366480446927374)(339,0.366480446927374)(340,0.366480446927374)(341,0.366480446927374)(342,0.366480446927374)(343,0.366890380313199)(344,0.365061590145577)(345,0.366480446927374)(346,0.366480446927374)(347,0.366480446927374)(348,0.366890380313199)(349,0.366890380313199)(350,0.366890380313199)(351,0.366890380313199)(352,0.366890380313199)(353,0.366890380313199)(354,0.371364653243848)(355,0.371364653243848)(356,0.369540873460246)(357,0.369540873460246)(358,0.369540873460246)(359,0.371364653243848)(360,0.371364653243848)(361,0.371364653243848)(362,0.371364653243848)(363,0.371364653243848)(364,0.371364653243848)(365,0.370949720670391)(366,0.371364653243848)(367,0.370949720670391)(368,0.370949720670391)(369,0.370535714285714)(370,0.372352285395764)(371,0.372352285395764)(372,0.373184357541899)(373,0.372767857142857)(374,0.370535714285714)(375,0.370535714285714)(376,0.370122630992196)(377,0.370949720670391)(378,0.370535714285714)(379,0.372352285395764)(380,0.370949720670391)(381,0.370535714285714)(382,0.370535714285714)(383,0.370535714285714)(384,0.370535714285714)(385,0.370535714285714)(386,0.370949720670391)(387,0.370949720670391)(388,0.370949720670391)(389,0.370949720670391)(390,0.370949720670391)(391,0.368715083798883)(392,0.368715083798883)(393,0.368715083798883)(394,0.368715083798883)(395,0.369127516778523)(396,0.368715083798883)(397,0.368715083798883)(398,0.368715083798883)(399,0.368715083798883)(400,0.368715083798883) 
};
\addplot [
color=red,
mark size=0.1pt,
only marks,
mark=*,
mark options={solid,fill=red},
forget plot
]
coordinates{
 (1,0)(2,0)(3,0)(4,0)(5,0)(6,0)(7,0)(8,0)(9,0.0302197802197802)(10,0.164179104477612)(11,0.146987951807229)(12,0.238095238095238)(13,0.242491657397108)(14,0.242761692650334)(15,0.242424242424242)(16,0.244013683010262)(17,0.244224422442244)(18,0.245847176079734)(19,0.245847176079734)(20,0.248322147651007)(21,0.247491638795987)(22,0.236632536973834)(23,0.233409610983982)(24,0.238317757009346)(25,0.237367802585194)(26,0.236533957845433)(27,0.235981308411215)(28,0.235981308411215)(29,0.235981308411215)(30,0.23625730994152)(31,0.237089201877934)(32,0.235431235431235)(33,0.237089201877934)(34,0.238488783943329)(35,0.236406619385343)(36,0.236686390532544)(37,0.23696682464455)(38,0.238095238095238)(39,0.237812128418549)(40,0.237410071942446)(41,0.243727598566308)(42,0.243146603098927)(43,0.241050119331742)(44,0.240762812872467)(45,0.242280285035629)(46,0.242280285035629)(47,0.242280285035629)(48,0.242857142857143)(49,0.243146603098927)(50,0.243436754176611)(51,0.245530393325387)(52,0.255924170616114)(53,0.262101534828807)(54,0.26241134751773)(55,0.26241134751773)(56,0.262721893491124)(57,0.262721893491124)(58,0.262721893491124)(59,0.261792452830189)(60,0.261176470588235)(61,0.262101534828807)(62,0.266195524146054)(63,0.271981242672919)(64,0.275700934579439)(65,0.27027027027027)(66,0.276346604215457)(67,0.27432590855803)(68,0.279394644935972)(69,0.277066356228172)(70,0.279394644935972)(71,0.278688524590164)(72,0.279015240328253)(73,0.278362573099415)(74,0.279015240328253)(75,0.280329799764429)(76,0.280329799764429)(77,0.279670975323149)(78,0.279670975323149)(79,0.279670975323149)(80,0.279670975323149)(81,0.28)(82,0.280329799764429)(83,0.28099173553719)(84,0.276923076923077)(85,0.276923076923077)(86,0.276923076923077)(87,0.276923076923077)(88,0.274881516587678)(89,0.274881516587678)(90,0.27520759193357)(91,0.275862068965517)(92,0.274556213017751)(93,0.274881516587678)(94,0.274881516587678)(95,0.274881516587678)(96,0.274881516587678)(97,0.274881516587678)(98,0.276923076923077)(99,0.276923076923077)(100,0.276923076923077)(101,0.276923076923077)(102,0.278959810874704)(103,0.278959810874704)(104,0.278959810874704)(105,0.278959810874704)(106,0.278959810874704)(107,0.278959810874704)(108,0.278959810874704)(109,0.279289940828402)(110,0.278630460448642)(111,0.276595744680851)(112,0.275943396226415)(113,0.275943396226415)(114,0.276595744680851)(115,0.276595744680851)(116,0.278630460448642)(117,0.278959810874704)(118,0.276923076923077)(119,0.276923076923077)(120,0.277251184834123)(121,0.276923076923077)(122,0.276923076923077)(123,0.276923076923077)(124,0.276923076923077)(125,0.276923076923077)(126,0.277251184834123)(127,0.277580071174377)(128,0.277580071174377)(129,0.277251184834123)(130,0.277251184834123)(131,0.277251184834123)(132,0.281323877068558)(133,0.279289940828402)(134,0.277251184834123)(135,0.277251184834123)(136,0.277251184834123)(137,0.277251184834123)(138,0.276923076923077)(139,0.276923076923077)(140,0.276923076923077)(141,0.277251184834123)(142,0.276923076923077)(143,0.276923076923077)(144,0.276923076923077)(145,0.276923076923077)(146,0.277251184834123)(147,0.277251184834123)(148,0.277251184834123)(149,0.672514619883041)(150,0.722521137586472)(151,0.725718725718726)(152,0.728019720624486)(153,0.774593338497289)(154,0.775289575289575)(155,0.849415204678362)(156,0.856716417910448)(157,0.887955182072829)(158,0.899357601713062)(159,0.899147727272727)(160,0.898220640569395)(161,0.895627644569817)(162,0.905172413793103)(163,0.910128388017118)(164,0.910394265232975)(165,0.910778015703069)(166,0.913385826771653)(167,0.913385826771653)(168,0.912330719885958)(169,0.911827956989247)(170,0.910374029640085)(171,0.910500352360817)(172,0.91114245416079)(173,0.909476661951909)(174,0.908833922261484)(175,0.906927921623513)(176,0.911392405063291)(177,0.910626319493315)(178,0.91114245416079)(179,0.911785462244178)(180,0.912429378531073)(181,0.912429378531073)(182,0.914205344585091)(183,0.914325842696629)(184,0.914968376669009)(185,0.918194640338505)(186,0.922206506364922)(187,0.927453769559033)(188,0.927453769559033)(189,0.927453769559033)(190,0.926794598436389)(191,0.926794598436389)(192,0.930099857346647)(193,0.929537366548043)(194,0.925373134328358)(195,0.925584691708008)(196,0.926241134751773)(197,0.924929178470255)(198,0.923620933521924)(199,0.923512747875354)(200,0.920141342756184)(201,0.920141342756184)(202,0.920141342756184)(203,0.920141342756184)(204,0.922096317280453)(205,0.925479063165365)(206,0.925479063165365)(207,0.924167257264351)(208,0.923295454545454)(209,0.922530206112296)(210,0.922530206112296)(211,0.92264017033357)(212,0.92264017033357)(213,0.921875)(214,0.923295454545454)(215,0.922749822820695)(216,0.922096317280453)(217,0.918842625264644)(218,0.922096317280453)(219,0.922096317280453)(220,0.922749822820695)(221,0.926794598436389)(222,0.926794598436389)(223,0.926031294452347)(224,0.926031294452347)(225,0.926690391459075)(226,0.928216062544421)(227,0.928216062544421)(228,0.928216062544421)(229,0.928876244665718)(230,0.928876244665718)(231,0.931526390870185)(232,0.934191702432046)(233,0.933428775948461)(234,0.933428775948461)(235,0.936200716845878)(236,0.934767025089606)(237,0.935344827586207)(238,0.935437589670014)(239,0.935437589670014)(240,0.928011404133999)(241,0.926031294452347)(242,0.926031294452347)(243,0.926690391459075)(244,0.926031294452347)(245,0.926136363636364)(246,0.926031294452347)(247,0.927556818181818)(248,0.925373134328358)(249,0.925373134328358)(250,0.925373134328358)(251,0.926031294452347)(252,0.926031294452347)(253,0.926031294452347)(254,0.929336188436831)(255,0.927350427350427)(256,0.928011404133999)(257,0.927246790299572)(258,0.927246790299572)(259,0.927246790299572)(260,0.927246790299572)(261,0.927246790299572)(262,0.928113879003559)(263,0.928876244665718)(264,0.928876244665718)(265,0.928216062544421)(266,0.928216062544421)(267,0.928216062544421)(268,0.929637526652452)(269,0.929637526652452)(270,0.929637526652452)(271,0.929637526652452)(272,0.928876244665718)(273,0.928977272727273)(274,0.928317955997161)(275,0.928317955997161)(276,0.926794598436389)(277,0.927659574468085)(278,0.928317955997161)(279,0.926898509581263)(280,0.928419560595322)(281,0.928419560595322)(282,0.929078014184397)(283,0.929078014184397)(284,0.929078014184397)(285,0.929078014184397)(286,0.929078014184397)(287,0.929936305732484)(288,0.930693069306931)(289,0.929378531073446)(290,0.929378531073446)(291,0.930035335689046)(292,0.930035335689046)(293,0.929278642149929)(294,0.929278642149929)(295,0.929278642149929)(296,0.929278642149929)(297,0.929278642149929)(298,0.929936305732484)(299,0.929936305732484)(300,0.929936305732484)(301,0.929936305732484)(302,0.929936305732484)(303,0.927659574468085)(304,0.926136363636364)(305,0.924167257264351)(306,0.923620933521924)(307,0.925899788285109)(308,0.925141242937853)(309,0.925795053003533)(310,0.925035360678925)(311,0.925795053003533)(312,0.92524682651622)(313,0.925899788285109)(314,0.925899788285109)(315,0.925899788285109)(316,0.925899788285109)(317,0.925899788285109)(318,0.925899788285109)(319,0.925899788285109)(320,0.926657263751763)(321,0.925141242937853)(322,0.925141242937853)(323,0.924488355681016)(324,0.926449787835926)(325,0.926449787835926)(326,0.927208480565371)(327,0.927105449398443)(328,0.927105449398443)(329,0.927105449398443)(330,0.927762039660057)(331,0.927762039660057)(332,0.929278642149929)(333,0.929278642149929)(334,0.930594900849858)(335,0.929936305732484)(336,0.929936305732484)(337,0.930594900849858)(338,0.930594900849858)(339,0.932107496463932)(340,0.930594900849858)(341,0.930594900849858)(342,0.929936305732484)(343,0.929936305732484)(344,0.929936305732484)(345,0.929178470254957)(346,0.930594900849858)(347,0.931254429482636)(348,0.932011331444759)(349,0.930397727272727)(350,0.930397727272727)(351,0.931818181818182)(352,0.931818181818182)(353,0.932480454868515)(354,0.931818181818182)(355,0.931156848828957)(356,0.929836995038979)(357,0.929836995038979)(358,0.929836995038979)(359,0.929836995038979)(360,0.929836995038979)(361,0.929836995038979)(362,0.929836995038979)(363,0.931156848828957)(364,0.931156848828957)(365,0.931156848828957)(366,0.931156848828957)(367,0.931156848828957)(368,0.931156848828957)(369,0.931818181818182)(370,0.931914893617021)(371,0.932576295244855)(372,0.931058990760483)(373,0.931058990760483)(374,0.931058990760483)(375,0.931058990760483)(376,0.931818181818182)(377,0.931818181818182)(378,0.931156848828957)(379,0.931156848828957)(380,0.931156848828957)(381,0.931721194879089)(382,0.931721194879089)(383,0.93238434163701)(384,0.93238434163701)(385,0.931721194879089)(386,0.930960854092526)(387,0.930960854092526)(388,0.930960854092526)(389,0.931623931623932)(390,0.930960854092526)(391,0.931623931623932)(392,0.931623931623932)(393,0.931623931623932)(394,0.931623931623932)(395,0.931623931623932)(396,0.931623931623932)(397,0.931623931623932)(398,0.93238434163701)(399,0.940014114326041)(400,0.941672522839072) 
};
\addplot [
color=red,
mark size=0.1pt,
only marks,
mark=*,
mark options={solid,fill=red},
forget plot
]
coordinates{
 (1,0)(2,0)(3,0)(4,0)(5,0)(6,0.10443864229765)(7,0.102994011976048)(8,0.10281517747858)(9,0.135552913198573)(10,0.0877419354838709)(11,0.0776196636481242)(12,0.0775193798449612)(13,0.0987341772151898)(14,0.098984771573604)(15,0.212910532276331)(16,0.218961625282167)(17,0.277899343544858)(18,0.279781420765027)(19,0.258563535911602)(20,0.275409836065574)(21,0.267108167770419)(22,0.269018743109151)(23,0.274211099020675)(24,0.282584884994524)(25,0.281009879253567)(26,0.273637374860957)(27,0.273942093541203)(28,0.25754060324826)(29,0.258741258741259)(30,0.250292397660819)(31,0.242068155111633)(32,0.24)(33,0.245901639344262)(34,0.246768507638073)(35,0.246768507638073)(36,0.245283018867924)(37,0.242924528301887)(38,0.242924528301887)(39,0.24582338902148)(40,0.252080856123662)(41,0.252080856123662)(42,0.256227758007117)(43,0.306451612903226)(44,0.309633027522936)(45,0.320541760722348)(46,0.317351598173516)(47,0.317351598173516)(48,0.30379746835443)(49,0.317351598173516)(50,0.320454545454545)(51,0.320454545454545)(52,0.324263038548753)(53,0.324263038548753)(54,0.321917808219178)(55,0.322727272727273)(56,0.322727272727273)(57,0.311212814645309)(58,0.320819112627986)(59,0.322360953461975)(60,0.332958380202475)(61,0.334080717488789)(62,0.333333333333333)(63,0.336700336700337)(64,0.337078651685393)(65,0.338218714768884)(66,0.33970753655793)(67,0.336700336700337)(68,0.336700336700337)(69,0.342281879194631)(70,0.342281879194631)(71,0.340425531914894)(72,0.332958380202475)(73,0.331454340473506)(74,0.332579185520362)(75,0.332579185520362)(76,0.332579185520362)(77,0.328018223234624)(78,0.328018223234624)(79,0.329519450800915)(80,0.335233751425313)(81,0.336)(82,0.33674630261661)(83,0.336)(84,0.339794754846066)(85,0.337868480725624)(86,0.338251986379115)(87,0.33257403189066)(88,0.33257403189066)(89,0.340136054421769)(90,0.347237880496054)(91,0.346846846846847)(92,0.346846846846847)(93,0.347629796839729)(94,0.344983089064261)(95,0.346846846846847)(96,0.350561797752809)(97,0.360400444938821)(98,0.360400444938821)(99,0.360619469026549)(100,0.358241758241758)(101,0.358635863586359)(102,0.358635863586359)(103,0.361018826135105)(104,0.359600443951165)(105,0.359600443951165)(106,0.363636363636364)(107,0.362625139043382)(108,0.366259711431742)(109,0.363836824696803)(110,0.363436123348018)(111,0.363036303630363)(112,0.362637362637363)(113,0.362637362637363)(114,0.364640883977901)(115,0.364640883977901)(116,0.36644591611479)(117,0.360220994475138)(118,0.361018826135105)(119,0.361018826135105)(120,0.359823399558499)(121,0.359030837004405)(122,0.358635863586359)(123,0.359030837004405)(124,0.358635863586359)(125,0.359823399558499)(126,0.352549889135255)(127,0.352549889135255)(128,0.352549889135255)(129,0.350721420643729)(130,0.352159468438538)(131,0.352159468438538)(132,0.352549889135255)(133,0.352549889135255)(134,0.351111111111111)(135,0.351111111111111)(136,0.349665924276169)(137,0.350055741360089)(138,0.349276974416018)(139,0.351111111111111)(140,0.350721420643729)(141,0.350721420643729)(142,0.351111111111111)(143,0.351111111111111)(144,0.350721420643729)(145,0.350721420643729)(146,0.350721420643729)(147,0.351111111111111)(148,0.350721420643729)(149,0.351501668520578)(150,0.351501668520578)(151,0.351501668520578)(152,0.352285395763657)(153,0.352678571428571)(154,0.354120267260579)(155,0.356347438752784)(156,0.356347438752784)(157,0.356347438752784)(158,0.356347438752784)(159,0.356347438752784)(160,0.356347438752784)(161,0.356744704570791)(162,0.354910714285714)(163,0.354910714285714)(164,0.360400444938821)(165,0.358574610244989)(166,0.358574610244989)(167,0.356744704570791)(168,0.356744704570791)(169,0.358974358974359)(170,0.358974358974359)(171,0.358574610244989)(172,0.358574610244989)(173,0.354910714285714)(174,0.354910714285714)(175,0.352678571428571)(176,0.348993288590604)(177,0.349384098544233)(178,0.35016835016835)(179,0.35016835016835)(180,0.351230425055928)(181,0.352017937219731)(182,0.352017937219731)(183,0.350561797752809)(184,0.350956130483689)(185,0.350561797752809)(186,0.35016835016835)(187,0.352017937219731)(188,0.35016835016835)(189,0.35016835016835)(190,0.350956130483689)(191,0.350956130483689)(192,0.354260089686099)(193,0.356103023516237)(194,0.350561797752809)(195,0.350956130483689)(196,0.350561797752809)(197,0.351351351351351)(198,0.351351351351351)(199,0.353205849268841)(200,0.353205849268841)(201,0.355855855855856)(202,0.355855855855856)(203,0.355855855855856)(204,0.355855855855856)(205,0.354401805869074)(206,0.352542372881356)(207,0.352542372881356)(208,0.352144469525959)(209,0.352144469525959)(210,0.352144469525959)(211,0.352144469525959)(212,0.352144469525959)(213,0.352144469525959)(214,0.351747463359639)(215,0.351747463359639)(216,0.351747463359639)(217,0.351747463359639)(218,0.351747463359639)(219,0.350956130483689)(220,0.351351351351351)(221,0.353205849268841)(222,0.353205849268841)(223,0.353205849268841)(224,0.356902356902357)(225,0.358744394618834)(226,0.358744394618834)(227,0.358744394618834)(228,0.358744394618834)(229,0.359550561797753)(230,0.359955005624297)(231,0.358511837655017)(232,0.358511837655017)(233,0.358511837655017)(234,0.358511837655017)(235,0.358511837655017)(236,0.358511837655017)(237,0.358511837655017)(238,0.358511837655017)(239,0.358511837655017)(240,0.358511837655017)(241,0.358511837655017)(242,0.358108108108108)(243,0.358511837655017)(244,0.358511837655017)(245,0.358511837655017)(246,0.35665914221219)(247,0.35665914221219)(248,0.35665914221219)(249,0.356257046223224)(250,0.35665914221219)(251,0.35665914221219)(252,0.358511837655017)(253,0.358108108108108)(254,0.358108108108108)(255,0.359955005624297)(256,0.359955005624297)(257,0.359955005624297)(258,0.359955005624297)(259,0.359955005624297)(260,0.359955005624297)(261,0.359955005624297)(262,0.359955005624297)(263,0.359955005624297)(264,0.359955005624297)(265,0.359955005624297)(266,0.359955005624297)(267,0.359955005624297)(268,0.359955005624297)(269,0.359955005624297)(270,0.359955005624297)(271,0.359955005624297)(272,0.361797752808989)(273,0.361797752808989)(274,0.361797752808989)(275,0.361797752808989)(276,0.359955005624297)(277,0.359955005624297)(278,0.359955005624297)(279,0.359955005624297)(280,0.359955005624297)(281,0.359955005624297)(282,0.359955005624297)(283,0.36036036036036)(284,0.36036036036036)(285,0.36036036036036)(286,0.36036036036036)(287,0.36036036036036)(288,0.360766629086809)(289,0.36036036036036)(290,0.36036036036036)(291,0.36036036036036)(292,0.359955005624297)(293,0.359955005624297)(294,0.359955005624297)(295,0.359955005624297)(296,0.369127516778523)(297,0.369127516778523)(298,0.363636363636364)(299,0.363636363636364)(300,0.367301231802911)(301,0.367301231802911)(302,0.367301231802911)(303,0.369127516778523)(304,0.369540873460246)(305,0.369540873460246)(306,0.369540873460246)(307,0.369540873460246)(308,0.367713004484305)(309,0.367713004484305)(310,0.367713004484305)(311,0.369540873460246)(312,0.369540873460246)(313,0.369540873460246)(314,0.369540873460246)(315,0.369540873460246)(316,0.369540873460246)(317,0.369540873460246)(318,0.367301231802911)(319,0.367301231802911)(320,0.369540873460246)(321,0.369540873460246)(322,0.37037037037037)(323,0.37037037037037)(324,0.37037037037037)(325,0.369955156950673)(326,0.37037037037037)(327,0.369955156950673)(328,0.37037037037037)(329,0.37037037037037)(330,0.37037037037037)(331,0.37037037037037)(332,0.37037037037037)(333,0.37037037037037)(334,0.37037037037037)(335,0.369955156950673)(336,0.37037037037037)(337,0.369540873460246)(338,0.369955156950673)(339,0.369127516778523)(340,0.369127516778523)(341,0.370949720670391)(342,0.370949720670391)(343,0.370949720670391)(344,0.370949720670391)(345,0.370949720670391)(346,0.370949720670391)(347,0.370949720670391)(348,0.370949720670391)(349,0.371364653243848)(350,0.371364653243848)(351,0.371364653243848)(352,0.371364653243848)(353,0.371364653243848)(354,0.371364653243848)(355,0.370949720670391)(356,0.370949720670391)(357,0.371364653243848)(358,0.371780515117581)(359,0.371364653243848)(360,0.370949720670391)(361,0.370949720670391)(362,0.370949720670391)(363,0.370949720670391)(364,0.371364653243848)(365,0.371364653243848)(366,0.371364653243848)(367,0.370949720670391)(368,0.371364653243848)(369,0.371364653243848)(370,0.371364653243848)(371,0.370949720670391)(372,0.370949720670391)(373,0.370949720670391)(374,0.370949720670391)(375,0.371364653243848)(376,0.371364653243848)(377,0.371364653243848)(378,0.371364653243848)(379,0.370949720670391)(380,0.370949720670391)(381,0.370949720670391)(382,0.370949720670391)(383,0.372352285395764)(384,0.372352285395764)(385,0.371937639198218)(386,0.372767857142857)(387,0.372767857142857)(388,0.372352285395764)(389,0.372352285395764)(390,0.372352285395764)(391,0.373184357541899)(392,0.373184357541899)(393,0.373184357541899)(394,0.373184357541899)(395,0.372767857142857)(396,0.373184357541899)(397,0.372352285395764)(398,0.372352285395764)(399,0.372352285395764)(400,0.372352285395764) 
};
\addplot [
color=red,
mark size=0.1pt,
only marks,
mark=*,
mark options={solid,fill=red},
forget plot
]
coordinates{
 (1,0)(2,0)(3,0)(4,0)(5,0)(6,0)(7,0.0329218106995885)(8,0.255609756097561)(9,0.24853228962818)(10,0.26051282051282)(11,0.244111349036403)(12,0.235162374020157)(13,0.236220472440945)(14,0.226285714285714)(15,0.230941704035874)(16,0.27103825136612)(17,0.265193370165746)(18,0.251670378619154)(19,0.234498308906426)(20,0.224277456647399)(21,0.188725490196078)(22,0.201701093560146)(23,0.201970443349754)(24,0.199753390875462)(25,0.198773006134969)(26,0.200980392156863)(27,0.253554502369668)(28,0.209500609013398)(29,0.213685474189676)(30,0.277904328018223)(31,0.27384960718294)(32,0.27765237020316)(33,0.276571428571429)(34,0.273878020713464)(35,0.273878020713464)(36,0.275229357798165)(37,0.276744186046512)(38,0.26463700234192)(39,0.26463700234192)(40,0.267877412031782)(41,0.268486916951081)(42,0.2692750287687)(43,0.276887871853547)(44,0.289592760180995)(45,0.29654403567447)(46,0.298672566371681)(47,0.299003322259136)(48,0.297117516629712)(49,0.298013245033112)(50,0.298013245033112)(51,0.305494505494505)(52,0.306843267108168)(53,0.30989010989011)(54,0.314410480349345)(55,0.314410480349345)(56,0.323561346362649)(57,0.321739130434783)(58,0.316248636859324)(59,0.316248636859324)(60,0.316702819956616)(61,0.316939890710382)(62,0.315098468271335)(63,0.313253012048193)(64,0.314631463146315)(65,0.316022099447514)(66,0.318232044198895)(67,0.314507198228128)(68,0.314855875831486)(69,0.314855875831486)(70,0.314855875831486)(71,0.314507198228128)(72,0.316258351893096)(73,0.319290465631929)(74,0.319290465631929)(75,0.31672203765227)(76,0.314507198228128)(77,0.314507198228128)(78,0.313812154696133)(79,0.311804008908686)(80,0.311804008908686)(81,0.315325248070562)(82,0.316371681415929)(83,0.321348314606741)(84,0.321348314606741)(85,0.324263038548753)(86,0.323895809739524)(87,0.32579185520362)(88,0.32579185520362)(89,0.32579185520362)(90,0.32579185520362)(91,0.326530612244898)(92,0.326530612244898)(93,0.326530612244898)(94,0.325)(95,0.325)(96,0.323462414578588)(97,0.323462414578588)(98,0.323462414578588)(99,0.324200913242009)(100,0.324200913242009)(101,0.326111744583808)(102,0.328018223234624)(103,0.32648401826484)(104,0.32648401826484)(105,0.328767123287671)(106,0.326857142857143)(107,0.326111744583808)(108,0.326111744583808)(109,0.32648401826484)(110,0.327231121281464)(111,0.327231121281464)(112,0.324942791762014)(113,0.324571428571428)(114,0.324942791762014)(115,0.326857142857143)(116,0.323765786452354)(117,0.323024054982818)(118,0.324510932105869)(119,0.327981651376147)(120,0.326061997703789)(121,0.326061997703789)(122,0.326061997703789)(123,0.329896907216495)(124,0.327605956471936)(125,0.329519450800915)(126,0.333333333333333)(127,0.333333333333333)(128,0.334851936218679)(129,0.332953249714937)(130,0.333714285714286)(131,0.335233751425313)(132,0.335233751425313)(133,0.335616438356164)(134,0.333714285714286)(135,0.333714285714286)(136,0.334478808705613)(137,0.334862385321101)(138,0.335246842709529)(139,0.337155963302752)(140,0.337899543378995)(141,0.337899543378995)(142,0.334862385321101)(143,0.334862385321101)(144,0.337155963302752)(145,0.340961098398169)(146,0.337155963302752)(147,0.342857142857143)(148,0.342857142857143)(149,0.344748858447488)(150,0.342857142857143)(151,0.342857142857143)(152,0.342075256556442)(153,0.342075256556442)(154,0.342075256556442)(155,0.343963553530752)(156,0.34584755403868)(157,0.34584755403868)(158,0.34584755403868)(159,0.34584755403868)(160,0.34584755403868)(161,0.34584755403868)(162,0.34584755403868)(163,0.342075256556442)(164,0.342075256556442)(165,0.343963553530752)(166,0.343963553530752)(167,0.344355758266819)(168,0.344748858447488)(169,0.345142857142857)(170,0.345142857142857)(171,0.344748858447488)(172,0.34135166093929)(173,0.34135166093929)(174,0.340961098398169)(175,0.340182648401826)(176,0.340961098398169)(177,0.340961098398169)(178,0.340571428571428)(179,0.342857142857143)(180,0.344748858447488)(181,0.344748858447488)(182,0.344748858447488)(183,0.345537757437071)(184,0.345142857142857)(185,0.345142857142857)(186,0.34135166093929)(187,0.341743119266055)(188,0.340961098398169)(189,0.340961098398169)(190,0.340961098398169)(191,0.343642611683849)(192,0.34324942791762)(193,0.343963553530752)(194,0.342075256556442)(195,0.338285714285714)(196,0.338285714285714)(197,0.338285714285714)(198,0.338285714285714)(199,0.337899543378995)(200,0.337899543378995)(201,0.337899543378995)(202,0.337899543378995)(203,0.339794754846066)(204,0.336)(205,0.336)(206,0.337899543378995)(207,0.337899543378995)(208,0.337899543378995)(209,0.339794754846066)(210,0.337899543378995)(211,0.337899543378995)(212,0.337899543378995)(213,0.336)(214,0.339449541284404)(215,0.336384439359268)(216,0.336384439359268)(217,0.338285714285714)(218,0.336384439359268)(219,0.342075256556442)(220,0.34584755403868)(221,0.34584755403868)(222,0.34584755403868)(223,0.34584755403868)(224,0.34584755403868)(225,0.343963553530752)(226,0.343963553530752)(227,0.345062429057889)(228,0.345062429057889)(229,0.343572241183163)(230,0.343572241183163)(231,0.345454545454545)(232,0.343572241183163)(233,0.343572241183163)(234,0.343572241183163)(235,0.34733257661748)(236,0.345454545454545)(237,0.343572241183163)(238,0.343572241183163)(239,0.343572241183163)(240,0.341685649202733)(241,0.341685649202733)(242,0.342075256556442)(243,0.342075256556442)(244,0.342075256556442)(245,0.343963553530752)(246,0.343963553530752)(247,0.343963553530752)(248,0.343963553530752)(249,0.344355758266819)(250,0.343963553530752)(251,0.343963553530752)(252,0.343963553530752)(253,0.344355758266819)(254,0.342857142857143)(255,0.34584755403868)(256,0.343963553530752)(257,0.343963553530752)(258,0.343572241183163)(259,0.343572241183163)(260,0.343963553530752)(261,0.343963553530752)(262,0.34584755403868)(263,0.344748858447488)(264,0.344748858447488)(265,0.344748858447488)(266,0.344748858447488)(267,0.344748858447488)(268,0.348519362186788)(269,0.350398179749716)(270,0.348519362186788)(271,0.348916761687571)(272,0.352673492605233)(273,0.354545454545454)(274,0.352673492605233)(275,0.350797266514806)(276,0.34703196347032)(277,0.347428571428571)(278,0.347428571428571)(279,0.347428571428571)(280,0.344036697247706)(281,0.347826086956522)(282,0.348916761687571)(283,0.348916761687571)(284,0.348916761687571)(285,0.345142857142857)(286,0.348916761687571)(287,0.350797266514806)(288,0.350797266514806)(289,0.35)(290,0.351197263397947)(291,0.352272727272727)(292,0.351872871736663)(293,0.352272727272727)(294,0.353741496598639)(295,0.353741496598639)(296,0.353741496598639)(297,0.353741496598639)(298,0.353340883352208)(299,0.353340883352208)(300,0.353741496598639)(301,0.353741496598639)(302,0.353340883352208)(303,0.354802259887006)(304,0.354802259887006)(305,0.354802259887006)(306,0.354802259887006)(307,0.354802259887006)(308,0.354802259887006)(309,0.354802259887006)(310,0.352941176470588)(311,0.352941176470588)(312,0.352542372881356)(313,0.352542372881356)(314,0.352542372881356)(315,0.352542372881356)(316,0.352542372881356)(317,0.354401805869074)(318,0.354401805869074)(319,0.354401805869074)(320,0.354401805869074)(321,0.354401805869074)(322,0.354401805869074)(323,0.356257046223224)(324,0.356257046223224)(325,0.356257046223224)(326,0.356257046223224)(327,0.356257046223224)(328,0.356257046223224)(329,0.356257046223224)(330,0.356257046223224)(331,0.356257046223224)(332,0.358108108108108)(333,0.356257046223224)(334,0.356257046223224)(335,0.356257046223224)(336,0.356257046223224)(337,0.358108108108108)(338,0.356257046223224)(339,0.358108108108108)(340,0.356257046223224)(341,0.356257046223224)(342,0.356257046223224)(343,0.356257046223224)(344,0.356257046223224)(345,0.356257046223224)(346,0.356257046223224)(347,0.356257046223224)(348,0.358108108108108)(349,0.359955005624297)(350,0.359955005624297)(351,0.359955005624297)(352,0.359955005624297)(353,0.365470852017937)(354,0.363636363636364)(355,0.361797752808989)(356,0.359955005624297)(357,0.363228699551569)(358,0.363228699551569)(359,0.363228699551569)(360,0.365061590145577)(361,0.363228699551569)(362,0.365470852017937)(363,0.365470852017937)(364,0.363636363636364)(365,0.363228699551569)(366,0.363636363636364)(367,0.361797752808989)(368,0.359955005624297)(369,0.359955005624297)(370,0.359955005624297)(371,0.359955005624297)(372,0.359955005624297)(373,0.359955005624297)(374,0.359955005624297)(375,0.359955005624297)(376,0.359955005624297)(377,0.359955005624297)(378,0.359550561797753)(379,0.359550561797753)(380,0.359550561797753)(381,0.359550561797753)(382,0.363228699551569)(383,0.363228699551569)(384,0.363228699551569)(385,0.363228699551569)(386,0.363228699551569)(387,0.363228699551569)(388,0.363228699551569)(389,0.363228699551569)(390,0.363228699551569)(391,0.365061590145577)(392,0.363228699551569)(393,0.363228699551569)(394,0.365061590145577)(395,0.363228699551569)(396,0.363228699551569)(397,0.363228699551569)(398,0.363228699551569)(399,0.363228699551569)(400,0.359550561797753) 
};
\addplot [
color=red,
mark size=0.1pt,
only marks,
mark=*,
mark options={solid,fill=red},
forget plot
]
coordinates{
 (1,0)(2,0)(3,0)(4,0)(5,0)(6,0)(7,0)(8,0.12200956937799)(9,0.112469437652812)(10,0.104738154613466)(11,0.0110957004160888)(12,0.0723860589812332)(13,0.0775401069518716)(14,0.0775401069518716)(15,0.08)(16,0.0875331564986737)(17,0.0949868073878628)(18,0.0952380952380952)(19,0.105263157894737)(20,0.0876494023904382)(21,0.0922266139657444)(22,0.107049608355091)(23,0.114137483787289)(24,0.115979381443299)(25,0.116129032258064)(26,0.116429495472186)(27,0.116580310880829)(28,0.125319693094629)(29,0.127877237851662)(30,0.129770992366412)(31,0.129770992366412)(32,0.129770992366412)(33,0.130434782608696)(34,0.130937098844673)(35,0.132992327365729)(36,0.13265306122449)(37,0.132484076433121)(38,0.129936305732484)(39,0.129770992366412)(40,0.129277566539924)(41,0.129441624365482)(42,0.129936305732484)(43,0.127877237851662)(44,0.132315521628499)(45,0.130102040816327)(46,0.130434782608696)(47,0.131274131274131)(48,0.133333333333333)(49,0.133676092544987)(50,0.133676092544987)(51,0.130601792573624)(52,0.128205128205128)(53,0.128369704749679)(54,0.128369704749679)(55,0.128700128700129)(56,0.12853470437018)(57,0.12853470437018)(58,0.194444444444444)(59,0.210274790919952)(60,0.255760368663594)(61,0.2548794489093)(62,0.301339285714286)(63,0.35027027027027)(64,0.345276872964169)(65,0.360725720384205)(66,0.358050847457627)(67,0.35244161358811)(68,0.355460385438972)(69,0.355841371918542)(70,0.355841371918542)(71,0.356076759061834)(72,0.356076759061834)(73,0.359913793103448)(74,0.356371490280777)(75,0.356989247311828)(76,0.356989247311828)(77,0.358369098712446)(78,0.359139784946237)(79,0.363834422657952)(80,0.366630076838639)(81,0.367032967032967)(82,0.370208105147864)(83,0.368995633187773)(84,0.36761487964989)(85,0.364628820960699)(86,0.365027322404372)(87,0.36144578313253)(88,0.360043907793633)(89,0.360220994475138)(90,0.359823399558499)(91,0.361631753031973)(92,0.363836824696803)(93,0.367074527252503)(94,0.365256124721603)(95,0.366071428571428)(96,0.365256124721603)(97,0.365256124721603)(98,0.365663322185061)(99,0.367483296213808)(100,0.366666666666667)(101,0.367074527252503)(102,0.369299221357063)(103,0.367074527252503)(104,0.367074527252503)(105,0.365663322185061)(106,0.367483296213808)(107,0.365663322185061)(108,0.365663322185061)(109,0.365663322185061)(110,0.365256124721603)(111,0.365663322185061)(112,0.365663322185061)(113,0.371111111111111)(114,0.370288248337029)(115,0.365256124721603)(116,0.361607142857143)(117,0.362222222222222)(118,0.363028953229399)(119,0.364653243847875)(120,0.363839285714286)(121,0.358342665173572)(122,0.358342665173572)(123,0.359550561797753)(124,0.359550561797753)(125,0.360582306830907)(126,0.361391694725028)(127,0.361391694725028)(128,0.361797752808989)(129,0.363636363636364)(130,0.363636363636364)(131,0.357705286839145)(132,0.357705286839145)(133,0.357705286839145)(134,0.359550561797753)(135,0.361391694725028)(136,0.357705286839145)(137,0.361391694725028)(138,0.360986547085202)(139,0.361391694725028)(140,0.365663322185061)(141,0.365663322185061)(142,0.365663322185061)(143,0.365470852017937)(144,0.365061590145577)(145,0.364653243847875)(146,0.366480446927374)(147,0.365061590145577)(148,0.363228699551569)(149,0.362821948488242)(150,0.364044943820225)(151,0.369369369369369)(152,0.368953880764904)(153,0.368953880764904)(154,0.363841807909604)(155,0.363841807909604)(156,0.362400906002265)(157,0.360953461975028)(158,0.36281179138322)(159,0.362400906002265)(160,0.362400906002265)(161,0.364665911664779)(162,0.366101694915254)(163,0.366101694915254)(164,0.366101694915254)(165,0.366101694915254)(166,0.366101694915254)(167,0.365276211950395)(168,0.365276211950395)(169,0.365276211950395)(170,0.36568848758465)(171,0.36568848758465)(172,0.364864864864865)(173,0.364454443194601)(174,0.364454443194601)(175,0.364454443194601)(176,0.365276211950395)(177,0.370786516853932)(178,0.370786516853932)(179,0.373033707865168)(180,0.371203599550056)(181,0.371203599550056)(182,0.373033707865168)(183,0.373033707865168)(184,0.372615039281706)(185,0.371780515117581)(186,0.370122630992196)(187,0.369710467706013)(188,0.370122630992196)(189,0.369710467706013)(190,0.369710467706013)(191,0.369710467706013)(192,0.368303571428571)(193,0.367892976588629)(194,0.367892976588629)(195,0.367892976588629)(196,0.367892976588629)(197,0.369710467706013)(198,0.369710467706013)(199,0.367892976588629)(200,0.367892976588629)(201,0.370122630992196)(202,0.368303571428571)(203,0.369710467706013)(204,0.369710467706013)(205,0.373333333333333)(206,0.373333333333333)(207,0.371681415929203)(208,0.371681415929203)(209,0.371270718232044)(210,0.370044052863436)(211,0.370044052863436)(212,0.371681415929203)(213,0.373333333333333)(214,0.373333333333333)(215,0.373333333333333)(216,0.373748609566185)(217,0.371937639198218)(218,0.371937639198218)(219,0.373748609566185)(220,0.373748609566185)(221,0.373748609566185)(222,0.373748609566185)(223,0.373333333333333)(224,0.373333333333333)(225,0.373333333333333)(226,0.373333333333333)(227,0.373333333333333)(228,0.373748609566185)(229,0.373748609566185)(230,0.375555555555556)(231,0.375555555555556)(232,0.375555555555556)(233,0.373333333333333)(234,0.373333333333333)(235,0.373333333333333)(236,0.373333333333333)(237,0.373333333333333)(238,0.373333333333333)(239,0.374722838137472)(240,0.374722838137472)(241,0.373333333333333)(242,0.373333333333333)(243,0.373333333333333)(244,0.373333333333333)(245,0.373333333333333)(246,0.373333333333333)(247,0.373333333333333)(248,0.373333333333333)(249,0.373333333333333)(250,0.373333333333333)(251,0.373333333333333)(252,0.373748609566185)(253,0.373748609566185)(254,0.373748609566185)(255,0.375555555555556)(256,0.375555555555556)(257,0.375555555555556)(258,0.374581939799331)(259,0.375555555555556)(260,0.375555555555556)(261,0.375555555555556)(262,0.375973303670745)(263,0.377777777777778)(264,0.377777777777778)(265,0.377777777777778)(266,0.377777777777778)(267,0.377777777777778)(268,0.377777777777778)(269,0.377777777777778)(270,0.377777777777778)(271,0.377777777777778)(272,0.377777777777778)(273,0.377777777777778)(274,0.377777777777778)(275,0.377777777777778)(276,0.377777777777778)(277,0.377777777777778)(278,0.377358490566038)(279,0.377358490566038)(280,0.377358490566038)(281,0.377358490566038)(282,0.376940133037694)(283,0.376940133037694)(284,0.375973303670745)(285,0.375973303670745)(286,0.375973303670745)(287,0.375973303670745)(288,0.375973303670745)(289,0.375973303670745)(290,0.375973303670745)(291,0.375973303670745)(292,0.375973303670745)(293,0.375973303670745)(294,0.377777777777778)(295,0.377777777777778)(296,0.378197997775306)(297,0.374581939799331)(298,0.374581939799331)(299,0.374581939799331)(300,0.374164810690423)(301,0.374164810690423)(302,0.374581939799331)(303,0.375418994413408)(304,0.375838926174497)(305,0.375838926174497)(306,0.375838926174497)(307,0.375838926174497)(308,0.375838926174497)(309,0.376259798432251)(310,0.374859708193041)(311,0.374859708193041)(312,0.374859708193041)(313,0.374859708193041)(314,0.374859708193041)(315,0.378499440089586)(316,0.378499440089586)(317,0.376681614349776)(318,0.376681614349776)(319,0.376681614349776)(320,0.376681614349776)(321,0.376681614349776)(322,0.378499440089586)(323,0.378499440089586)(324,0.378076062639821)(325,0.378076062639821)(326,0.378499440089586)(327,0.378499440089586)(328,0.378499440089586)(329,0.378499440089586)(330,0.378499440089586)(331,0.378499440089586)(332,0.378499440089586)(333,0.378499440089586)(334,0.378499440089586)(335,0.378499440089586)(336,0.378499440089586)(337,0.378499440089586)(338,0.378499440089586)(339,0.378499440089586)(340,0.378499440089586)(341,0.378499440089586)(342,0.378499440089586)(343,0.378499440089586)(344,0.376681614349776)(345,0.378499440089586)(346,0.374859708193041)(347,0.374859708193041)(348,0.374859708193041)(349,0.375280898876404)(350,0.377104377104377)(351,0.377104377104377)(352,0.377104377104377)(353,0.377104377104377)(354,0.377104377104377)(355,0.378923766816143)(356,0.378923766816143)(357,0.378923766816143)(358,0.378923766816143)(359,0.378923766816143)(360,0.378923766816143)(361,0.378923766816143)(362,0.378923766816143)(363,0.378923766816143)(364,0.378923766816143)(365,0.378923766816143)(366,0.378923766816143)(367,0.378923766816143)(368,0.378923766816143)(369,0.378923766816143)(370,0.378923766816143)(371,0.378923766816143)(372,0.378923766816143)(373,0.378923766816143)(374,0.378923766816143)(375,0.378923766816143)(376,0.378923766816143)(377,0.378923766816143)(378,0.378923766816143)(379,0.378923766816143)(380,0.378923766816143)(381,0.379349046015713)(382,0.379349046015713)(383,0.37752808988764)(384,0.37752808988764)(385,0.37570303712036)(386,0.37570303712036)(387,0.37570303712036)(388,0.37570303712036)(389,0.37752808988764)(390,0.37752808988764)(391,0.37752808988764)(392,0.37570303712036)(393,0.37570303712036)(394,0.37570303712036)(395,0.37570303712036)(396,0.37570303712036)(397,0.373873873873874)(398,0.373873873873874)(399,0.373873873873874)(400,0.37570303712036) 
};
\addplot [
color=red,
mark size=0.1pt,
only marks,
mark=*,
mark options={solid,fill=red},
forget plot
]
coordinates{
 (1,0)(2,0)(3,0)(4,0)(5,0.228903976721629)(6,0.18144750254842)(7,0.192640692640693)(8,0.184782608695652)(9,0.18876404494382)(10,0.188129899216125)(11,0.190156599552573)(12,0.222464558342421)(13,0.221734357848518)(14,0.222469410456062)(15,0.227477477477477)(16,0.237668161434978)(17,0.247216035634744)(18,0.246941045606229)(19,0.246049661399548)(20,0.246885617214043)(21,0.246885617214043)(22,0.257847533632287)(23,0.254791431792559)(24,0.257110352673493)(25,0.257110352673493)(26,0.242840778923253)(27,0.244013683010262)(28,0.244292237442922)(29,0.245412844036697)(30,0.25)(31,0.246294184720638)(32,0.242562929061785)(33,0.242774566473988)(34,0.24390243902439)(35,0.243619489559165)(36,0.245614035087719)(37,0.257839721254355)(38,0.256772673733804)(39,0.258600237247924)(40,0.257378984651712)(41,0.261176470588235)(42,0.26056338028169)(43,0.264018691588785)(44,0.262295081967213)(45,0.262295081967213)(46,0.262910798122066)(47,0.258215962441315)(48,0.260257913247362)(49,0.262295081967213)(50,0.262295081967213)(51,0.264018691588785)(52,0.263710618436406)(53,0.264327485380117)(54,0.266195524146054)(55,0.265569917743831)(56,0.265569917743831)(57,0.266509433962264)(58,0.26792009400705)(59,0.268551236749117)(60,0.26792009400705)(61,0.267605633802817)(62,0.265569917743831)(63,0.265258215962441)(64,0.265258215962441)(65,0.265258215962441)(66,0.265258215962441)(67,0.271345029239766)(68,0.272727272727273)(69,0.268691588785047)(70,0.268691588785047)(71,0.268691588785047)(72,0.269636576787808)(73,0.272941176470588)(74,0.273262661955241)(75,0.27122641509434)(76,0.27122641509434)(77,0.27122641509434)(78,0.267772511848341)(79,0.266509433962264)(80,0.266824085005903)(81,0.266824085005903)(82,0.266824085005903)(83,0.267455621301775)(84,0.267772511848341)(85,0.266666666666667)(86,0.266033254156769)(87,0.266984505363528)(88,0.266984505363528)(89,0.266984505363528)(90,0.266984505363528)(91,0.266984505363528)(92,0.266984505363528)(93,0.265868263473054)(94,0.26378896882494)(95,0.261704681872749)(96,0.261704681872749)(97,0.261704681872749)(98,0.28232502965599)(99,0.339794754846066)(100,0.337899543378995)(101,0.337899543378995)(102,0.341242937853107)(103,0.340857787810384)(104,0.344206974128234)(105,0.340473506200676)(106,0.340807174887892)(107,0.342342342342342)(108,0.343963553530752)(109,0.343181818181818)(110,0.343115124153499)(111,0.343115124153499)(112,0.344280860702152)(113,0.343963553530752)(114,0.346938775510204)(115,0.343572241183163)(116,0.343572241183163)(117,0.343963553530752)(118,0.338672768878718)(119,0.331415420023015)(120,0.331415420023015)(121,0.340961098398169)(122,0.340961098398169)(123,0.342465753424657)(124,0.342465753424657)(125,0.355203619909502)(126,0.357062146892655)(127,0.358916478555305)(128,0.360766629086809)(129,0.359955005624297)(130,0.359955005624297)(131,0.358108108108108)(132,0.356257046223224)(133,0.354802259887006)(134,0.355203619909502)(135,0.347727272727273)(136,0.347727272727273)(137,0.347727272727273)(138,0.348122866894198)(139,0.348122866894198)(140,0.347727272727273)(141,0.34584755403868)(142,0.34624145785877)(143,0.34624145785877)(144,0.346636259977195)(145,0.34624145785877)(146,0.34624145785877)(147,0.35)(148,0.35)(149,0.35)(150,0.351872871736663)(151,0.348122866894198)(152,0.35)(153,0.351872871736663)(154,0.351872871736663)(155,0.351872871736663)(156,0.351872871736663)(157,0.349315068493151)(158,0.347428571428571)(159,0.347428571428571)(160,0.347428571428571)(161,0.347428571428571)(162,0.347428571428571)(163,0.347428571428571)(164,0.347428571428571)(165,0.353075170842825)(166,0.353075170842825)(167,0.356818181818182)(168,0.351197263397947)(169,0.353075170842825)(170,0.353075170842825)(171,0.353075170842825)(172,0.354948805460751)(173,0.351197263397947)(174,0.351197263397947)(175,0.349714285714286)(176,0.349315068493151)(177,0.351197263397947)(178,0.351598173515982)(179,0.353075170842825)(180,0.350797266514806)(181,0.350797266514806)(182,0.350797266514806)(183,0.350797266514806)(184,0.352673492605233)(185,0.352673492605233)(186,0.352673492605233)(187,0.352673492605233)(188,0.352673492605233)(189,0.352673492605233)(190,0.352673492605233)(191,0.352673492605233)(192,0.352673492605233)(193,0.352673492605233)(194,0.352673492605233)(195,0.352673492605233)(196,0.353075170842825)(197,0.353075170842825)(198,0.353075170842825)(199,0.354948805460751)(200,0.354948805460751)(201,0.354948805460751)(202,0.354948805460751)(203,0.354948805460751)(204,0.354948805460751)(205,0.354948805460751)(206,0.353075170842825)(207,0.351598173515982)(208,0.354545454545454)(209,0.354545454545454)(210,0.356413166855846)(211,0.356413166855846)(212,0.354545454545454)(213,0.354545454545454)(214,0.356413166855846)(215,0.356413166855846)(216,0.354545454545454)(217,0.354545454545454)(218,0.354545454545454)(219,0.352673492605233)(220,0.352673492605233)(221,0.352673492605233)(222,0.350797266514806)(223,0.350797266514806)(224,0.352272727272727)(225,0.352272727272727)(226,0.352272727272727)(227,0.352272727272727)(228,0.354545454545454)(229,0.354545454545454)(230,0.354545454545454)(231,0.354143019296254)(232,0.354143019296254)(233,0.354143019296254)(234,0.354143019296254)(235,0.354143019296254)(236,0.354143019296254)(237,0.354143019296254)(238,0.354143019296254)(239,0.356009070294785)(240,0.357466063348416)(241,0.357870894677237)(242,0.357466063348416)(243,0.357466063348416)(244,0.357466063348416)(245,0.357466063348416)(246,0.357466063348416)(247,0.357466063348416)(248,0.357062146892655)(249,0.357062146892655)(250,0.357870894677237)(251,0.357870894677237)(252,0.357870894677237)(253,0.357870894677237)(254,0.357870894677237)(255,0.357870894677237)(256,0.357870894677237)(257,0.357870894677237)(258,0.357870894677237)(259,0.357870894677237)(260,0.357870894677237)(261,0.357870894677237)(262,0.357870894677237)(263,0.357870894677237)(264,0.35827664399093)(265,0.35827664399093)(266,0.35827664399093)(267,0.35827664399093)(268,0.35827664399093)(269,0.35827664399093)(270,0.35827664399093)(271,0.357870894677237)(272,0.357870894677237)(273,0.357870894677237)(274,0.357870894677237)(275,0.357870894677237)(276,0.357870894677237)(277,0.357870894677237)(278,0.357870894677237)(279,0.357870894677237)(280,0.357870894677237)(281,0.357870894677237)(282,0.357466063348416)(283,0.357062146892655)(284,0.357466063348416)(285,0.357466063348416)(286,0.357466063348416)(287,0.357466063348416)(288,0.357466063348416)(289,0.357466063348416)(290,0.357466063348416)(291,0.357466063348416)(292,0.357466063348416)(293,0.357466063348416)(294,0.357466063348416)(295,0.357870894677237)(296,0.357870894677237)(297,0.357870894677237)(298,0.35827664399093)(299,0.35827664399093)(300,0.35827664399093)(301,0.35827664399093)(302,0.35827664399093)(303,0.35827664399093)(304,0.35827664399093)(305,0.358683314415437)(306,0.358683314415437)(307,0.358683314415437)(308,0.358683314415437)(309,0.358683314415437)(310,0.358683314415437)(311,0.358683314415437)(312,0.358683314415437)(313,0.358683314415437)(314,0.357870894677237)(315,0.35827664399093)(316,0.358683314415437)(317,0.35827664399093)(318,0.357870894677237)(319,0.357870894677237)(320,0.357870894677237)(321,0.357870894677237)(322,0.357870894677237)(323,0.357870894677237)(324,0.357870894677237)(325,0.357466063348416)(326,0.357466063348416)(327,0.357466063348416)(328,0.357870894677237)(329,0.357870894677237)(330,0.357870894677237)(331,0.357870894677237)(332,0.35972850678733)(333,0.35972850678733)(334,0.359322033898305)(335,0.359322033898305)(336,0.359322033898305)(337,0.359322033898305)(338,0.359322033898305)(339,0.359322033898305)(340,0.359322033898305)(341,0.359322033898305)(342,0.359322033898305)(343,0.359322033898305)(344,0.359322033898305)(345,0.359322033898305)(346,0.359322033898305)(347,0.359322033898305)(348,0.359322033898305)(349,0.359322033898305)(350,0.359322033898305)(351,0.359322033898305)(352,0.359322033898305)(353,0.359322033898305)(354,0.359322033898305)(355,0.359322033898305)(356,0.359322033898305)(357,0.359322033898305)(358,0.358916478555305)(359,0.359322033898305)(360,0.359322033898305)(361,0.359322033898305)(362,0.359322033898305)(363,0.359322033898305)(364,0.359322033898305)(365,0.359322033898305)(366,0.359322033898305)(367,0.359322033898305)(368,0.359322033898305)(369,0.359322033898305)(370,0.359322033898305)(371,0.359322033898305)(372,0.359322033898305)(373,0.359322033898305)(374,0.36117381489842)(375,0.36117381489842)(376,0.36117381489842)(377,0.36117381489842)(378,0.36117381489842)(379,0.363021420518602)(380,0.36117381489842)(381,0.36117381489842)(382,0.364864864864865)(383,0.364864864864865)(384,0.364864864864865)(385,0.366704161979753)(386,0.363021420518602)(387,0.363021420518602)(388,0.363021420518602)(389,0.364864864864865)(390,0.36117381489842)(391,0.36117381489842)(392,0.363021420518602)(393,0.363021420518602)(394,0.363431151241535)(395,0.365276211950395)(396,0.365276211950395)(397,0.363431151241535)(398,0.363431151241535)(399,0.363431151241535)(400,0.363431151241535) 
};
\addplot [
color=red,
mark size=0.1pt,
only marks,
mark=*,
mark options={solid,fill=red},
forget plot
]
coordinates{
 (1,0)(2,0)(3,0)(4,0)(5,0)(6,0)(7,0.0830090791180285)(8,0.111675126903553)(9,0.110414052697616)(10,0.109452736318408)(11,0.118811881188119)(12,0.120906801007557)(13,0.128712871287129)(14,0.129032258064516)(15,0.129353233830846)(16,0.130653266331658)(17,0.130653266331658)(18,0.131479140328698)(19,0.126742712294043)(20,0.126582278481013)(21,0.126582278481013)(22,0.122448979591837)(23,0.122292993630573)(24,0.122292993630573)(25,0.122292993630573)(26,0.125480153649168)(27,0.121761658031088)(28,0.121290322580645)(29,0.125641025641026)(30,0.106493506493506)(31,0.115979381443299)(32,0.118100128369705)(33,0.118709677419355)(34,0.118251928020566)(35,0.127551020408163)(36,0.127226463104326)(37,0.127877237851662)(38,0.128040973111396)(39,0.125802310654685)(40,0.128205128205128)(41,0.128700128700129)(42,0.129198966408269)(43,0.129198966408269)(44,0.129198966408269)(45,0.129198966408269)(46,0.12970168612192)(47,0.12970168612192)(48,0.127272727272727)(49,0.127272727272727)(50,0.127272727272727)(51,0.127107652399481)(52,0.126778783958603)(53,0.127272727272727)(54,0.127438231469441)(55,0.122715404699739)(56,0.1251629726206)(57,0.1251629726206)(58,0.127438231469441)(59,0.12970168612192)(60,0.12970168612192)(61,0.12970168612192)(62,0.12970168612192)(63,0.12970168612192)(64,0.129366106080207)(65,0.129366106080207)(66,0.129366106080207)(67,0.129366106080207)(68,0.129366106080207)(69,0.129366106080207)(70,0.129366106080207)(71,0.129366106080207)(72,0.126943005181347)(73,0.126943005181347)(74,0.126778783958603)(75,0.127272727272727)(76,0.127107652399481)(77,0.129533678756477)(78,0.129533678756477)(79,0.129366106080207)(80,0.129366106080207)(81,0.129198966408269)(82,0.129198966408269)(83,0.129366106080207)(84,0.129366106080207)(85,0.129366106080207)(86,0.129198966408269)(87,0.129198966408269)(88,0.129198966408269)(89,0.129198966408269)(90,0.129366106080207)(91,0.129366106080207)(92,0.12970168612192)(93,0.12970168612192)(94,0.130039011703511)(95,0.127604166666667)(96,0.132467532467532)(97,0.127604166666667)(98,0.130039011703511)(99,0.130039011703511)(100,0.130039011703511)(101,0.130039011703511)(102,0.130039011703511)(103,0.130039011703511)(104,0.130039011703511)(105,0.127604166666667)(106,0.130039011703511)(107,0.130039011703511)(108,0.132467532467532)(109,0.132467532467532)(110,0.132467532467532)(111,0.132467532467532)(112,0.132467532467532)(113,0.134889753566796)(114,0.134889753566796)(115,0.134889753566796)(116,0.134889753566796)(117,0.134889753566796)(118,0.134889753566796)(119,0.134540750323415)(120,0.134540750323415)(121,0.134540750323415)(122,0.134540750323415)(123,0.134540750323415)(124,0.134540750323415)(125,0.134540750323415)(126,0.134540750323415)(127,0.134540750323415)(128,0.134540750323415)(129,0.134540750323415)(130,0.134540750323415)(131,0.134540750323415)(132,0.132124352331606)(133,0.132124352331606)(134,0.132124352331606)(135,0.132295719844358)(136,0.132295719844358)(137,0.132295719844358)(138,0.132295719844358)(139,0.132295719844358)(140,0.132295719844358)(141,0.132295719844358)(142,0.132124352331606)(143,0.132124352331606)(144,0.132124352331606)(145,0.132124352331606)(146,0.132124352331606)(147,0.624027657735523)(148,0.778932778932779)(149,0.782060266292922)(150,0.8)(151,0.791500664010624)(152,0.811179277436946)(153,0.823855755894591)(154,0.822714681440443)(155,0.819310344827586)(156,0.832402234636871)(157,0.831241283124128)(158,0.846975088967971)(159,0.872620790629575)(160,0.872620790629575)(161,0.874631268436578)(162,0.876957494407159)(163,0.863602110022607)(164,0.867524602573808)(165,0.864253393665158)(166,0.868322046651618)(167,0.867378048780488)(168,0.867378048780488)(169,0.863636363636364)(170,0.867579908675799)(171,0.873580620741862)(172,0.873860182370821)(173,0.873196659073652)(174,0.871601208459214)(175,0.871137905048983)(176,0.870943396226415)(177,0.872836719337848)(178,0.871137905048983)(179,0.875375375375375)(180,0.873493975903614)(181,0.868263473053892)(182,0.869109947643979)(183,0.873684210526316)(184,0.872372372372372)(185,0.873218304576144)(186,0.873408239700374)(187,0.874251497005988)(188,0.872944693572496)(189,0.872292755787901)(190,0.872292755787901)(191,0.870411985018726)(192,0.873684210526316)(193,0.874530428249437)(194,0.877902621722846)(195,0.87874251497006)(196,0.87874251497006)(197,0.87874251497006)(198,0.882043576258452)(199,0.883370955605718)(200,0.882352941176471)(201,0.882842025699168)(202,0.883018867924528)(203,0.885022692889561)(204,0.885022692889561)(205,0.884353741496599)(206,0.884353741496599)(207,0.884036144578313)(208,0.884702336096458)(209,0.882882882882883)(210,0.885542168674699)(211,0.886877828054299)(212,0.886877828054299)(213,0.885196374622356)(214,0.884528301886792)(215,0.886535552193646)(216,0.887206661619985)(217,0.889057750759878)(218,0.888888888888889)(219,0.889733840304182)(220,0.889733840304182)(221,0.889733840304182)(222,0.890577507598784)(223,0.889733840304182)(224,0.889733840304182)(225,0.889733840304182)(226,0.889733840304182)(227,0.889733840304182)(228,0.888719512195122)(229,0.887871853546911)(230,0.888719512195122)(231,0.890410958904109)(232,0.889565879664889)(233,0.889565879664889)(234,0.889565879664889)(235,0.888888888888889)(236,0.888888888888889)(237,0.887871853546911)(238,0.887871853546911)(239,0.886519421172887)(240,0.888212927756654)(241,0.888888888888889)(242,0.888719512195122)(243,0.889397406559878)(244,0.887700534759358)(245,0.887700534759358)(246,0.887022900763359)(247,0.887022900763359)(248,0.886519421172887)(249,0.888549618320611)(250,0.888549618320611)(251,0.889058913542463)(252,0.889739663093415)(253,0.889739663093415)(254,0.890589135424636)(255,0.889908256880734)(256,0.889908256880734)(257,0.890922959572845)(258,0.890922959572845)(259,0.889058913542463)(260,0.889908256880734)(261,0.892284186401833)(262,0.892284186401833)(263,0.892284186401833)(264,0.893292682926829)(265,0.893292682926829)(266,0.892448512585812)(267,0.891603053435114)(268,0.892448512585812)(269,0.892448512585812)(270,0.891768292682927)(271,0.891768292682927)(272,0.891768292682927)(273,0.891768292682927)(274,0.891603053435114)(275,0.891768292682927)(276,0.892448512585812)(277,0.892448512585812)(278,0.893292682926829)(279,0.892448512585812)(280,0.892448512585812)(281,0.892448512585812)(282,0.891603053435114)(283,0.892448512585812)(284,0.892448512585812)(285,0.892448512585812)(286,0.892448512585812)(287,0.892448512585812)(288,0.892448512585812)(289,0.892448512585812)(290,0.892448512585812)(291,0.892448512585812)(292,0.892448512585812)(293,0.892448512585812)(294,0.892448512585812)(295,0.891768292682927)(296,0.891768292682927)(297,0.89193302891933)(298,0.89193302891933)(299,0.89193302891933)(300,0.89193302891933)(301,0.89193302891933)(302,0.89193302891933)(303,0.891254752851711)(304,0.891254752851711)(305,0.891254752851711)(306,0.89193302891933)(307,0.891089108910891)(308,0.891089108910891)(309,0.891089108910891)(310,0.891089108910891)(311,0.891768292682927)(312,0.892448512585812)(313,0.892448512585812)(314,0.892448512585812)(315,0.892448512585812)(316,0.892448512585812)(317,0.891603053435114)(318,0.892284186401833)(319,0.892284186401833)(320,0.892284186401833)(321,0.891271056661562)(322,0.892119357306809)(323,0.892119357306809)(324,0.891271056661562)(325,0.891271056661562)(326,0.891437308868501)(327,0.891437308868501)(328,0.890589135424636)(329,0.891437308868501)(330,0.890589135424636)(331,0.890589135424636)(332,0.890589135424636)(333,0.892284186401833)(334,0.892284186401833)(335,0.892284186401833)(336,0.892284186401833)(337,0.892284186401833)(338,0.893129770992366)(339,0.893129770992366)(340,0.892284186401833)(341,0.893129770992366)(342,0.893129770992366)(343,0.893974065598779)(344,0.893974065598779)(345,0.893974065598779)(346,0.893974065598779)(347,0.893974065598779)(348,0.893974065598779)(349,0.893974065598779)(350,0.893974065598779)(351,0.893974065598779)(352,0.893974065598779)(353,0.893974065598779)(354,0.893974065598779)(355,0.893974065598779)(356,0.893974065598779)(357,0.893974065598779)(358,0.891437308868501)(359,0.892284186401833)(360,0.892284186401833)(361,0.892284186401833)(362,0.892284186401833)(363,0.892284186401833)(364,0.892284186401833)(365,0.892284186401833)(366,0.892284186401833)(367,0.892284186401833)(368,0.892284186401833)(369,0.892284186401833)(370,0.892284186401833)(371,0.892284186401833)(372,0.892284186401833)(373,0.892284186401833)(374,0.892284186401833)(375,0.892284186401833)(376,0.892284186401833)(377,0.892284186401833)(378,0.892284186401833)(379,0.891437308868501)(380,0.890589135424636)(381,0.890589135424636)(382,0.890589135424636)(383,0.890589135424636)(384,0.890589135424636)(385,0.890589135424636)(386,0.891271056661562)(387,0.891271056661562)(388,0.891271056661562)(389,0.891271056661562)(390,0.891271056661562)(391,0.891271056661562)(392,0.891271056661562)(393,0.891271056661562)(394,0.891271056661562)(395,0.891271056661562)(396,0.891271056661562)(397,0.891271056661562)(398,0.891271056661562)(399,0.891271056661562)(400,0.891271056661562) 
};
\addplot [
color=red,
mark size=0.1pt,
only marks,
mark=*,
mark options={solid,fill=red},
forget plot
]
coordinates{
 (1,0)(2,0)(3,0)(4,0)(5,0)(6,0.115485564304462)(7,0.0751677852348993)(8,0.18227215980025)(9,0.179974651457541)(10,0.184576485461441)(11,0.184343434343434)(12,0.184343434343434)(13,0.15681233933162)(14,0.189393939393939)(15,0.215346534653465)(16,0.208955223880597)(17,0.211180124223602)(18,0.197994987468672)(19,0.209215442092154)(20,0.22029702970297)(21,0.244200244200244)(22,0.25273390036452)(23,0.256969696969697)(24,0.26360338573156)(25,0.2647412755716)(26,0.254237288135593)(27,0.250608272506083)(28,0.249084249084249)(29,0.246943765281174)(30,0.244200244200244)(31,0.245742092457421)(32,0.245742092457421)(33,0.247873633049818)(34,0.256348246674728)(35,0.25)(36,0.247873633049818)(37,0.248175182481752)(38,0.25)(39,0.247873633049818)(40,0.25)(41,0.254237288135593)(42,0.254237288135593)(43,0.258454106280193)(44,0.258454106280193)(45,0.258454106280193)(46,0.258454106280193)(47,0.260240963855422)(48,0.24969696969697)(49,0.245742092457421)(50,0.254237288135593)(51,0.25)(52,0.25)(53,0.252121212121212)(54,0.252121212121212)(55,0.25181598062954)(56,0.255729794933655)(57,0.255729794933655)(58,0.255421686746988)(59,0.255729794933655)(60,0.256038647342995)(61,0.256038647342995)(62,0.256038647342995)(63,0.259927797833935)(64,0.260240963855422)(65,0.259927797833935)(66,0.260240963855422)(67,0.258142340168878)(68,0.258142340168878)(69,0.258766626360338)(70,0.265060240963855)(71,0.265060240963855)(72,0.265060240963855)(73,0.265379975874548)(74,0.265379975874548)(75,0.271634615384615)(76,0.265379975874548)(77,0.265379975874548)(78,0.265379975874548)(79,0.2647412755716)(80,0.265379975874548)(81,0.267469879518072)(82,0.267469879518072)(83,0.269554753309266)(84,0.269554753309266)(85,0.269554753309266)(86,0.269554753309266)(87,0.267469879518072)(88,0.265379975874548)(89,0.265379975874548)(90,0.265379975874548)(91,0.2590799031477)(92,0.2590799031477)(93,0.2590799031477)(94,0.2590799031477)(95,0.2590799031477)(96,0.254854368932039)(97,0.2590799031477)(98,0.2590799031477)(99,0.261185006045949)(100,0.260869565217391)(101,0.260869565217391)(102,0.263285024154589)(103,0.261501210653753)(104,0.261501210653753)(105,0.265700483091787)(106,0.265379975874548)(107,0.267469879518072)(108,0.261185006045949)(109,0.263285024154589)(110,0.263922518159806)(111,0.263922518159806)(112,0.265700483091787)(113,0.265700483091787)(114,0.267792521109771)(115,0.267792521109771)(116,0.269879518072289)(117,0.269879518072289)(118,0.271961492178099)(119,0.271961492178099)(120,0.269879518072289)(121,0.269879518072289)(122,0.269879518072289)(123,0.269879518072289)(124,0.271961492178099)(125,0.276110444177671)(126,0.276110444177671)(127,0.276110444177671)(128,0.276110444177671)(129,0.276442307692308)(130,0.276442307692308)(131,0.276110444177671)(132,0.276110444177671)(133,0.276110444177671)(134,0.274368231046931)(135,0.274368231046931)(136,0.274368231046931)(137,0.274368231046931)(138,0.274368231046931)(139,0.274368231046931)(140,0.274368231046931)(141,0.276442307692308)(142,0.274038461538461)(143,0.274038461538461)(144,0.274038461538461)(145,0.273709483793517)(146,0.275779376498801)(147,0.276442307692308)(148,0.274368231046931)(149,0.274368231046931)(150,0.274368231046931)(151,0.274368231046931)(152,0.274368231046931)(153,0.276110444177671)(154,0.275779376498801)(155,0.276110444177671)(156,0.276110444177671)(157,0.273381294964029)(158,0.273381294964029)(159,0.27511961722488)(160,0.275449101796407)(161,0.275449101796407)(162,0.275779376498801)(163,0.275779376498801)(164,0.275779376498801)(165,0.275779376498801)(166,0.275449101796407)(167,0.275779376498801)(168,0.277511961722488)(169,0.277511961722488)(170,0.277511961722488)(171,0.277511961722488)(172,0.277511961722488)(173,0.277511961722488)(174,0.275779376498801)(175,0.277844311377245)(176,0.277844311377245)(177,0.277844311377245)(178,0.277844311377245)(179,0.277844311377245)(180,0.275779376498801)(181,0.275779376498801)(182,0.277844311377245)(183,0.275779376498801)(184,0.275779376498801)(185,0.277511961722488)(186,0.277844311377245)(187,0.277844311377245)(188,0.277844311377245)(189,0.277844311377245)(190,0.277844311377245)(191,0.277844311377245)(192,0.277511961722488)(193,0.277511961722488)(194,0.277844311377245)(195,0.274038461538461)(196,0.274038461538461)(197,0.276110444177671)(198,0.275779376498801)(199,0.276110444177671)(200,0.276110444177671)(201,0.276110444177671)(202,0.280239520958084)(203,0.280239520958084)(204,0.280239520958084)(205,0.280239520958084)(206,0.278177458033573)(207,0.278177458033573)(208,0.277844311377245)(209,0.277844311377245)(210,0.277844311377245)(211,0.278177458033573)(212,0.278177458033573)(213,0.278177458033573)(214,0.278177458033573)(215,0.278177458033573)(216,0.278177458033573)(217,0.278177458033573)(218,0.278177458033573)(219,0.278177458033573)(220,0.278177458033573)(221,0.278177458033573)(222,0.278177458033573)(223,0.278177458033573)(224,0.278177458033573)(225,0.280239520958084)(226,0.292508917954816)(227,0.292508917954816)(228,0.302600472813239)(229,0.302600472813239)(230,0.32046783625731)(231,0.314185228604924)(232,0.325960419091967)(233,0.345537757437071)(234,0.345454545454545)(235,0.346727898966705)(236,0.345224395857307)(237,0.34331797235023)(238,0.341407151095732)(239,0.339491916859122)(240,0.339491916859122)(241,0.352806414662085)(242,0.350917431192661)(243,0.345224395857307)(244,0.348224513172967)(245,0.348224513172967)(246,0.346330275229358)(247,0.346330275229358)(248,0.342528735632184)(249,0.352402745995423)(250,0.357630979498861)(251,0.354691075514874)(252,0.3557582668187)(253,0.354285714285714)(254,0.354285714285714)(255,0.353881278538813)(256,0.353881278538813)(257,0.353881278538813)(258,0.3557582668187)(259,0.357224118316268)(260,0.355353075170843)(261,0.355353075170843)(262,0.355353075170843)(263,0.357224118316268)(264,0.357224118316268)(265,0.356818181818182)(266,0.357224118316268)(267,0.355353075170843)(268,0.355353075170843)(269,0.355353075170843)(270,0.355353075170843)(271,0.355353075170843)(272,0.357224118316268)(273,0.355353075170843)(274,0.355353075170843)(275,0.355353075170843)(276,0.355353075170843)(277,0.357224118316268)(278,0.357224118316268)(279,0.360953461975028)(280,0.357224118316268)(281,0.357224118316268)(282,0.356818181818182)(283,0.357224118316268)(284,0.357224118316268)(285,0.356818181818182)(286,0.356818181818182)(287,0.356818181818182)(288,0.357224118316268)(289,0.357224118316268)(290,0.357630979498861)(291,0.357224118316268)(292,0.357224118316268)(293,0.357224118316268)(294,0.359090909090909)(295,0.359090909090909)(296,0.359090909090909)(297,0.355353075170843)(298,0.3557582668187)(299,0.355353075170843)(300,0.356818181818182)(301,0.355353075170843)(302,0.3557582668187)(303,0.3557582668187)(304,0.3557582668187)(305,0.356164383561644)(306,0.353881278538813)(307,0.356571428571429)(308,0.356571428571429)(309,0.3557582668187)(310,0.356571428571429)(311,0.356164383561644)(312,0.358038768529076)(313,0.358447488584475)(314,0.358447488584475)(315,0.358447488584475)(316,0.358447488584475)(317,0.358447488584475)(318,0.358447488584475)(319,0.356571428571429)(320,0.356571428571429)(321,0.360319270239453)(322,0.360319270239453)(323,0.360319270239453)(324,0.360319270239453)(325,0.360319270239453)(326,0.360319270239453)(327,0.360319270239453)(328,0.364050056882821)(329,0.365909090909091)(330,0.367763904653802)(331,0.365909090909091)(332,0.369614512471655)(333,0.369614512471655)(334,0.367763904653802)(335,0.369614512471655)(336,0.369614512471655)(337,0.369614512471655)(338,0.369614512471655)(339,0.371460928652322)(340,0.371460928652322)(341,0.369614512471655)(342,0.367763904653802)(343,0.367763904653802)(344,0.371460928652322)(345,0.369614512471655)(346,0.371460928652322)(347,0.371460928652322)(348,0.371460928652322)(349,0.371460928652322)(350,0.373303167420814)(351,0.373303167420814)(352,0.371460928652322)(353,0.369614512471655)(354,0.369614512471655)(355,0.369614512471655)(356,0.369614512471655)(357,0.369614512471655)(358,0.369614512471655)(359,0.369614512471655)(360,0.369614512471655)(361,0.369614512471655)(362,0.371460928652322)(363,0.371460928652322)(364,0.367763904653802)(365,0.369614512471655)(366,0.369614512471655)(367,0.369614512471655)(368,0.369614512471655)(369,0.369614512471655)(370,0.369614512471655)(371,0.369614512471655)(372,0.369614512471655)(373,0.369614512471655)(374,0.369614512471655)(375,0.369614512471655)(376,0.369614512471655)(377,0.369614512471655)(378,0.369614512471655)(379,0.369614512471655)(380,0.369614512471655)(381,0.369614512471655)(382,0.369614512471655)(383,0.369614512471655)(384,0.371460928652322)(385,0.371460928652322)(386,0.371460928652322)(387,0.371460928652322)(388,0.371460928652322)(389,0.371460928652322)(390,0.371460928652322)(391,0.371460928652322)(392,0.371460928652322)(393,0.371460928652322)(394,0.371460928652322)(395,0.371460928652322)(396,0.371460928652322)(397,0.371460928652322)(398,0.369614512471655)(399,0.369614512471655)(400,0.369614512471655) 
};
\addplot [
color=red,
mark size=0.1pt,
only marks,
mark=*,
mark options={solid,fill=red},
forget plot
]
coordinates{
 (1,0)(2,0)(3,0)(4,0)(5,0)(6,0.023841059602649)(7,0.0341655716162943)(8,0.195121951219512)(9,0.525023607176582)(10,0.710606060606061)(11,0.760111576011157)(12,0.77542062911485)(13,0.772594752186589)(14,0.794272795779955)(15,0.793602437166793)(16,0.799382716049383)(17,0.795665634674923)(18,0.80030487804878)(19,0.8)(20,0.81441717791411)(21,0.818812644564379)(22,0.816923076923077)(23,0.82051282051282)(24,0.82051282051282)(25,0.816705336426914)(26,0.82051282051282)(27,0.825617283950617)(28,0.82716049382716)(29,0.821455938697318)(30,0.82217090069284)(31,0.824074074074074)(32,0.825986078886311)(33,0.829869130100077)(34,0.830246913580247)(35,0.829606784888204)(36,0.831408775981524)(37,0.831408775981524)(38,0.830769230769231)(39,0.834621329211746)(40,0.833976833976834)(41,0.833333333333333)(42,0.833976833976834)(43,0.833976833976834)(44,0.836307214895268)(45,0.83695652173913)(46,0.83695652173913)(47,0.835658914728682)(48,0.837606837606838)(49,0.838258164852255)(50,0.838910505836576)(51,0.839563862928349)(52,0.837858805275407)(53,0.838258164852255)(54,0.839313572542902)(55,0.839313572542902)(56,0.839313572542902)(57,0.841282251759187)(58,0.840625)(59,0.840625)(60,0.840625)(61,0.840625)(62,0.841940532081377)(63,0.842845973416732)(64,0.8421875)(65,0.842845973416732)(66,0.841940532081377)(67,0.837354085603113)(68,0.841282251759187)(69,0.841940532081377)(70,0.841940532081377)(71,0.841692789968652)(72,0.842599843382929)(73,0.840625)(74,0.840625)(75,0.841940532081377)(76,0.841940532081377)(77,0.841940532081377)(78,0.841940532081377)(79,0.84458398744113)(80,0.84458398744113)(81,0.84458398744113)(82,0.84458398744113)(83,0.84458398744113)(84,0.843676355066771)(85,0.844339622641509)(86,0.845247446975648)(87,0.84458398744113)(88,0.845247446975648)(89,0.846577498033045)(90,0.845911949685534)(91,0.846335697399527)(92,0.846577498033045)(93,0.846577498033045)(94,0.848341232227488)(95,0.850356294536817)(96,0.849683544303797)(97,0.849683544303797)(98,0.849250197316495)(99,0.848341232227488)(100,0.848341232227488)(101,0.848341232227488)(102,0.847003154574132)(103,0.848341232227488)(104,0.847671665351223)(105,0.84901185770751)(106,0.849683544303797)(107,0.849683544303797)(108,0.849683544303797)(109,0.84901185770751)(110,0.84901185770751)(111,0.847671665351223)(112,0.847671665351223)(113,0.847671665351223)(114,0.847671665351223)(115,0.847671665351223)(116,0.847671665351223)(117,0.847671665351223)(118,0.847671665351223)(119,0.848341232227488)(120,0.847671665351223)(121,0.847671665351223)(122,0.847671665351223)(123,0.847671665351223)(124,0.847671665351223)(125,0.848341232227488)(126,0.848341232227488)(127,0.84901185770751)(128,0.84901185770751)(129,0.84901185770751)(130,0.84901185770751)(131,0.84901185770751)(132,0.848341232227488)(133,0.848341232227488)(134,0.848341232227488)(135,0.84901185770751)(136,0.84901185770751)(137,0.84901185770751)(138,0.847003154574132)(139,0.848341232227488)(140,0.84901185770751)(141,0.84901185770751)(142,0.848341232227488)(143,0.848341232227488)(144,0.84901185770751)(145,0.848341232227488)(146,0.848341232227488)(147,0.848341232227488)(148,0.847003154574132)(149,0.848580441640378)(150,0.848580441640378)(151,0.848580441640378)(152,0.848580441640378)(153,0.848580441640378)(154,0.848580441640378)(155,0.846761453396524)(156,0.846761453396524)(157,0.848580441640378)(158,0.847244094488189)(159,0.847244094488189)(160,0.847244094488189)(161,0.847244094488189)(162,0.847244094488189)(163,0.847244094488189)(164,0.846577498033045)(165,0.847244094488189)(166,0.847244094488189)(167,0.847244094488189)(168,0.845425867507886)(169,0.845425867507886)(170,0.845425867507886)(171,0.847671665351223)(172,0.848341232227488)(173,0.847671665351223)(174,0.848580441640378)(175,0.848818897637795)(176,0.848151062155783)(177,0.848151062155783)(178,0.848151062155783)(179,0.848151062155783)(180,0.848818897637795)(181,0.84748427672956)(182,0.84748427672956)(183,0.84748427672956)(184,0.84748427672956)(185,0.84748427672956)(186,0.84748427672956)(187,0.84748427672956)(188,0.848151062155783)(189,0.848151062155783)(190,0.848818897637795)(191,0.848818897637795)(192,0.848818897637795)(193,0.848818897637795)(194,0.848818897637795)(195,0.848818897637795)(196,0.847911741528763)(197,0.849250197316495)(198,0.849250197316495)(199,0.849250197316495)(200,0.848580441640378)(201,0.848580441640378)(202,0.848580441640378)(203,0.848580441640378)(204,0.848580441640378)(205,0.848580441640378)(206,0.849250197316495)(207,0.849250197316495)(208,0.848580441640378)(209,0.84901185770751)(210,0.849921011058452)(211,0.849921011058452)(212,0.850592885375494)(213,0.849683544303797)(214,0.849683544303797)(215,0.849683544303797)(216,0.84901185770751)(217,0.84901185770751)(218,0.848341232227488)(219,0.849250197316495)(220,0.849250197316495)(221,0.85193982581156)(222,0.85193982581156)(223,0.85193982581156)(224,0.85126582278481)(225,0.85126582278481)(226,0.85126582278481)(227,0.850592885375494)(228,0.850592885375494)(229,0.85126582278481)(230,0.850592885375494)(231,0.850592885375494)(232,0.852173913043478)(233,0.852173913043478)(234,0.851500789889415)(235,0.851500789889415)(236,0.853080568720379)(237,0.853985793212312)(238,0.854889589905363)(239,0.854889589905363)(240,0.854215918045705)(241,0.854889589905363)(242,0.854215918045705)(243,0.854215918045705)(244,0.854215918045705)(245,0.854889589905363)(246,0.854889589905363)(247,0.854889589905363)(248,0.854889589905363)(249,0.854889589905363)(250,0.854889589905363)(251,0.854889589905363)(252,0.854215918045705)(253,0.854215918045705)(254,0.854215918045705)(255,0.854215918045705)(256,0.855564325177585)(257,0.856240126382306)(258,0.856240126382306)(259,0.855564325177585)(260,0.856240126382306)(261,0.856240126382306)(262,0.855335968379446)(263,0.856240126382306)(264,0.856240126382306)(265,0.855335968379446)(266,0.855335968379446)(267,0.855335968379446)(268,0.856012658227848)(269,0.854660347551343)(270,0.855564325177585)(271,0.856240126382306)(272,0.856240126382306)(273,0.856916996047431)(274,0.856916996047431)(275,0.857594936708861)(276,0.857594936708861)(277,0.857594936708861)(278,0.857594936708861)(279,0.857594936708861)(280,0.857594936708861)(281,0.857594936708861)(282,0.857594936708861)(283,0.857594936708861)(284,0.857594936708861)(285,0.856916996047431)(286,0.85827395091053)(287,0.85827395091053)(288,0.85827395091053)(289,0.857594936708861)(290,0.857594936708861)(291,0.856240126382306)(292,0.856240126382306)(293,0.856240126382306)(294,0.856240126382306)(295,0.856240126382306)(296,0.856240126382306)(297,0.856240126382306)(298,0.855791962174941)(299,0.855791962174941)(300,0.85511811023622)(301,0.85511811023622)(302,0.85511811023622)(303,0.855791962174941)(304,0.855791962174941)(305,0.855791962174941)(306,0.854445318646735)(307,0.854445318646735)(308,0.85511811023622)(309,0.855791962174941)(310,0.85511811023622)(311,0.85511811023622)(312,0.855791962174941)(313,0.855791962174941)(314,0.855791962174941)(315,0.855791962174941)(316,0.855791962174941)(317,0.855791962174941)(318,0.855791962174941)(319,0.855791962174941)(320,0.855791962174941)(321,0.855791962174941)(322,0.855791962174941)(323,0.855791962174941)(324,0.855791962174941)(325,0.855791962174941)(326,0.855791962174941)(327,0.855791962174941)(328,0.855791962174941)(329,0.855791962174941)(330,0.857142857142857)(331,0.857142857142857)(332,0.857142857142857)(333,0.857142857142857)(334,0.857594936708861)(335,0.857594936708861)(336,0.857594936708861)(337,0.85827395091053)(338,0.85827395091053)(339,0.85827395091053)(340,0.85827395091053)(341,0.856916996047431)(342,0.856916996047431)(343,0.856916996047431)(344,0.856916996047431)(345,0.856916996047431)(346,0.856916996047431)(347,0.856240126382306)(348,0.856240126382306)(349,0.856916996047431)(350,0.856916996047431)(351,0.856916996047431)(352,0.856916996047431)(353,0.854889589905363)(354,0.855791962174941)(355,0.855791962174941)(356,0.855791962174941)(357,0.855791962174941)(358,0.855791962174941)(359,0.855791962174941)(360,0.855791962174941)(361,0.855791962174941)(362,0.855791962174941)(363,0.856466876971609)(364,0.856466876971609)(365,0.855791962174941)(366,0.856466876971609)(367,0.856466876971609)(368,0.856466876971609)(369,0.856466876971609)(370,0.857142857142857)(371,0.857142857142857)(372,0.857142857142857)(373,0.856466876971609)(374,0.856466876971609)(375,0.856466876971609)(376,0.856466876971609)(377,0.856466876971609)(378,0.856466876971609)(379,0.857142857142857)(380,0.857142857142857)(381,0.857142857142857)(382,0.857142857142857)(383,0.857142857142857)(384,0.857142857142857)(385,0.857142857142857)(386,0.857142857142857)(387,0.857142857142857)(388,0.857142857142857)(389,0.857142857142857)(390,0.857142857142857)(391,0.857142857142857)(392,0.857142857142857)(393,0.857142857142857)(394,0.857142857142857)(395,0.857142857142857)(396,0.857142857142857)(397,0.857142857142857)(398,0.857142857142857)(399,0.857142857142857)(400,0.85781990521327) 
};
\addplot [
color=red,
mark size=0.1pt,
only marks,
mark=*,
mark options={solid,fill=red},
forget plot
]
coordinates{
 (1,0)(2,0)(3,0)(4,0)(5,0)(6,0)(7,0)(8,0)(9,0)(10,0.106217616580311)(11,0.0731070496083551)(12,0.0636604774535809)(13,0.0193370165745856)(14,0.0461329715061058)(15,0.0695187165775401)(16,0.0644295302013423)(17,0.0968586387434555)(18,0.0993464052287581)(19,0.113110539845758)(20,0.112388250319285)(21,0.112965340179718)(22,0.125)(23,0.127388535031847)(24,0.126903553299492)(25,0.127388535031847)(26,0.126262626262626)(27,0.127551020408163)(28,0.126742712294043)(29,0.12484076433121)(30,0.125159642401022)(31,0.123393316195373)(32,0.125802310654685)(33,0.125964010282776)(34,0.125964010282776)(35,0.12291933418694)(36,0.12291933418694)(37,0.122292993630573)(38,0.122605363984674)(39,0.122605363984674)(40,0.122605363984674)(41,0.125480153649168)(42,0.120667522464698)(43,0.12291933418694)(44,0.12291933418694)(45,0.123076923076923)(46,0.123076923076923)(47,0.123076923076923)(48,0.125480153649168)(49,0.127877237851662)(50,0.127877237851662)(51,0.127877237851662)(52,0.130268199233716)(53,0.127877237851662)(54,0.127877237851662)(55,0.127713920817369)(56,0.127551020408163)(57,0.127551020408163)(58,0.127551020408163)(59,0.127226463104326)(60,0.127388535031847)(61,0.127388535031847)(62,0.127388535031847)(63,0.127713920817369)(64,0.127551020408163)(65,0.127551020408163)(66,0.128040973111396)(67,0.128040973111396)(68,0.127713920817369)(69,0.127713920817369)(70,0.127388535031847)(71,0.127388535031847)(72,0.125)(73,0.127388535031847)(74,0.127388535031847)(75,0.127713920817369)(76,0.128040973111396)(77,0.127877237851662)(78,0.127713920817369)(79,0.127713920817369)(80,0.127713920817369)(81,0.127551020408163)(82,0.127551020408163)(83,0.127713920817369)(84,0.127713920817369)(85,0.127713920817369)(86,0.127713920817369)(87,0.127877237851662)(88,0.127877237851662)(89,0.128040973111396)(90,0.127877237851662)(91,0.127877237851662)(92,0.361251261352169)(93,0.36976506639428)(94,0.769544527532291)(95,0.768702814001373)(96,0.765613519470977)(97,0.784790874524715)(98,0.778035576179428)(99,0.815033161385409)(100,0.803792851932896)(101,0.799148332150461)(102,0.802826855123675)(103,0.807965860597439)(104,0.807965860597439)(105,0.819102749638206)(106,0.842261904761905)(107,0.84375)(108,0.845637583892617)(109,0.842342342342342)(110,0.842342342342342)(111,0.84377358490566)(112,0.845112781954887)(113,0.846328538985617)(114,0.843465045592705)(115,0.842824601366743)(116,0.843537414965986)(117,0.852852852852853)(118,0.853512705530643)(119,0.856929955290611)(120,0.857777777777778)(121,0.857571214392803)(122,0.85821455363841)(123,0.858426966292135)(124,0.858426966292135)(125,0.857357357357357)(126,0.85821455363841)(127,0.85821455363841)(128,0.859910581222056)(129,0.859479553903346)(130,0.864179104477612)(131,0.865470852017937)(132,0.863534675615212)(133,0.864824495892457)(134,0.867756315007429)(135,0.866071428571428)(136,0.866071428571428)(137,0.866517524235645)(138,0.865627319970304)(139,0.866118175018698)(140,0.865871833084948)(141,0.865871833084948)(142,0.865067466266866)(143,0.867017280240421)(144,0.866165413533834)(145,0.867867867867868)(146,0.868717179294824)(147,0.868913857677903)(148,0.868263473053892)(149,0.869565217391304)(150,0.869565217391304)(151,0.867017280240421)(152,0.863602110022607)(153,0.864661654135338)(154,0.865312264860797)(155,0.868717179294824)(156,0.867867867867868)(157,0.868519909842224)(158,0.87125748502994)(159,0.871064467766117)(160,0.870217554388597)(161,0.870217554388597)(162,0.870411985018726)(163,0.871064467766117)(164,0.872754491017964)(165,0.872101720269259)(166,0.872101720269259)(167,0.873408239700374)(168,0.872754491017964)(169,0.875651526433358)(170,0.876304023845007)(171,0.876773711725168)(172,0.87593423019432)(173,0.874626865671642)(174,0.875280059746079)(175,0.874439461883408)(176,0.871064467766117)(177,0.87171792948237)(178,0.869172932330827)(179,0.872452830188679)(180,0.87642153146323)(181,0.87642153146323)(182,0.875757575757576)(183,0.875757575757576)(184,0.87642153146323)(185,0.87642153146323)(186,0.875757575757576)(187,0.877458396369138)(188,0.878306878306878)(189,0.87642153146323)(190,0.87642153146323)(191,0.87642153146323)(192,0.87556904400607)(193,0.874715261958998)(194,0.874715261958998)(195,0.874715261958998)(196,0.873860182370821)(197,0.874524714828897)(198,0.873196659073652)(199,0.874051593323217)(200,0.871678056188307)(201,0.871678056188307)(202,0.872534142640364)(203,0.890380313199105)(204,0.892193308550186)(205,0.901033973412112)(206,0.899408284023668)(207,0.92507204610951)(208,0.933428775948461)(209,0.944639103013315)(210,0.94413407821229)(211,0.948524365133836)(212,0.948524365133836)(213,0.945280437756498)(214,0.943989071038251)(215,0.941015089163237)(216,0.94287680660702)(217,0.941740918437286)(218,0.941740918437286)(219,0.938608458390177)(220,0.938608458390177)(221,0.946280991735537)(222,0.948594928032899)(223,0.945554789800138)(224,0.941908713692946)(225,0.942560553633218)(226,0.942560553633218)(227,0.945176960444136)(228,0.944521497919556)(229,0.945983379501385)(230,0.95324494068388)(231,0.953910614525139)(232,0.95324494068388)(233,0.953974895397489)(234,0.955493741307371)(235,0.955493741307371)(236,0.956219596942321)(237,0.95615866388309)(238,0.953504510756419)(239,0.953504510756419)(240,0.952843273231623)(241,0.953568953568953)(242,0.950932964754665)(243,0.951590594744122)(244,0.953568953568953)(245,0.954892435808466)(246,0.955555555555555)(247,0.955555555555555)(248,0.955555555555555)(249,0.955555555555555)(250,0.954230235783634)(251,0.954892435808466)(252,0.95676429567643)(253,0.958770090845562)(254,0.960111966410077)(255,0.960111966410077)(256,0.957431960921144)(257,0.958770090845562)(258,0.958100558659218)(259,0.95676429567643)(260,0.95676429567643)(261,0.95676429567643)(262,0.95676429567643)(263,0.957491289198606)(264,0.957491289198606)(265,0.958942240779401)(266,0.958217270194986)(267,0.958217270194986)(268,0.957550452331246)(269,0.957550452331246)(270,0.958217270194986)(271,0.958885017421603)(272,0.958885017421603)(273,0.958885017421603)(274,0.958942240779401)(275,0.958942240779401)(276,0.959610027855153)(277,0.95827538247566)(278,0.95827538247566)(279,0.95827538247566)(280,0.960278745644599)(281,0.958942240779401)(282,0.958942240779401)(283,0.959666203059805)(284,0.961672473867596)(285,0.96301465457083)(286,0.96301465457083)(287,0.96301465457083)(288,0.961672473867596)(289,0.96100278551532)(290,0.963687150837989)(291,0.963687150837989)(292,0.96373779637378)(293,0.96373779637378)(294,0.962395543175487)(295,0.962395543175487)(296,0.96513249651325)(297,0.966480446927374)(298,0.967155835080363)(299,0.967155835080363)(300,0.966433566433566)(301,0.96792189679219)(302,0.96792189679219)(303,0.967877094972067)(304,0.967201674808095)(305,0.967201674808095)(306,0.967201674808095)(307,0.96575821104123)(308,0.966480446927374)(309,0.966433566433566)(310,0.966433566433566)(311,0.966433566433566)(312,0.966433566433566)(313,0.96652719665272)(314,0.965181058495822)(315,0.965853658536585)(316,0.965853658536585)(317,0.963838664812239)(318,0.963838664812239)(319,0.962447844228094)(320,0.961725817675713)(321,0.96105702364395)(322,0.96105702364395)(323,0.9625)(324,0.962447844228094)(325,0.962447844228094)(326,0.962447844228094)(327,0.962447844228094)(328,0.964459930313589)(329,0.964459930313589)(330,0.965181058495822)(331,0.96373779637378)(332,0.965181058495822)(333,0.965901183020181)(334,0.965901183020181)(335,0.965901183020181)(336,0.965229485396384)(337,0.965948575399583)(338,0.965229485396384)(339,0.965901183020181)(340,0.965901183020181)(341,0.966620305980528)(342,0.967338429464906)(343,0.966666666666666)(344,0.966666666666666)(345,0.966666666666666)(346,0.965325936199723)(347,0.965995836224844)(348,0.965995836224844)(349,0.965995836224844)(350,0.965995836224844)(351,0.965995836224844)(352,0.967338429464906)(353,0.967338429464906)(354,0.970034843205575)(355,0.97071129707113)(356,0.970034843205575)(357,0.969359331476323)(358,0.968011126564673)(359,0.968011126564673)(360,0.968011126564673)(361,0.965995836224844)(362,0.965995836224844)(363,0.965995836224844)(364,0.965995836224844)(365,0.965995836224844)(366,0.966666666666666)(367,0.966666666666666)(368,0.966666666666666)(369,0.966666666666666)(370,0.967383761276891)(371,0.967383761276891)(372,0.966666666666666)(373,0.966666666666666)(374,0.965948575399583)(375,0.965948575399583)(376,0.965948575399583)(377,0.964509394572025)(378,0.962395543175487)(379,0.962447844228094)(380,0.962447844228094)(381,0.962447844228094)(382,0.961779013203613)(383,0.962447844228094)(384,0.962447844228094)(385,0.962447844228094)(386,0.962447844228094)(387,0.962447844228094)(388,0.961779013203613)(389,0.962447844228094)(390,0.963117606123869)(391,0.963117606123869)(392,0.961725817675713)(393,0.963117606123869)(394,0.963117606123869)(395,0.963117606123869)(396,0.963788300835654)(397,0.963788300835654)(398,0.96513249651325)(399,0.96513249651325)(400,0.966480446927374) 
};
\addplot [
color=red,
mark size=0.1pt,
only marks,
mark=*,
mark options={solid,fill=red},
forget plot
]
coordinates{
 (1,0)(2,0)(3,0)(4,0)(5,0)(6,0)(7,0)(8,0)(9,0)(10,0)(11,0.272164948453608)(12,0.524720893141946)(13,0.563404255319149)(14,0.594813614262561)(15,0.615763546798029)(16,0.758064516129032)(17,0.765467625899281)(18,0.761280931586608)(19,0.764444444444444)(20,0.759969902182092)(21,0.765375854214123)(22,0.761398176291793)(23,0.761614623000762)(24,0.781990521327014)(25,0.779714738510301)(26,0.805275407292475)(27,0.80867544539117)(28,0.834937083641747)(29,0.837797619047619)(30,0.840030326004549)(31,0.839820359281437)(32,0.831620553359684)(33,0.832678711704635)(34,0.830963665086888)(35,0.837245696400626)(36,0.837245696400626)(37,0.862147753236862)(38,0.871406959152799)(39,0.871794871794872)(40,0.873771730914588)(41,0.873771730914588)(42,0.872919818456883)(43,0.876233864844343)(44,0.877086494688923)(45,0.873962264150943)(46,0.872919818456883)(47,0.876628352490421)(48,0.875477463712758)(49,0.898959881129272)(50,0.896449704142012)(51,0.896449704142012)(52,0.895125553914328)(53,0.896755162241888)(54,0.897869213813372)(55,0.897869213813372)(56,0.898379970544919)(57,0.89807976366322)(58,0.895478131949592)(59,0.89814126394052)(60,0.89814126394052)(61,0.897168405365127)(62,0.893553223388306)(63,0.896035901271503)(64,0.899328859060403)(65,0.9)(66,0.899178491411501)(67,0.897989575577066)(68,0.897473997028232)(69,0.896706586826347)(70,0.896706586826347)(71,0.905716406829992)(72,0.907475943745374)(73,0.907475943745374)(74,0.914705882352941)(75,0.914705882352941)(76,0.913907284768212)(77,0.919590643274854)(78,0.918681318681319)(79,0.911504424778761)(80,0.90989010989011)(81,0.911485003657644)(82,0.912152269399707)(83,0.913075237399562)(84,0.913075237399562)(85,0.914660831509847)(86,0.91399416909621)(87,0.913202042304887)(88,0.910818713450292)(89,0.910818713450292)(90,0.908424908424908)(91,0.908424908424908)(92,0.910021945866862)(93,0.910021945866862)(94,0.909356725146199)(95,0.909356725146199)(96,0.911871813546977)(97,0.912536443148688)(98,0.912536443148688)(99,0.912536443148688)(100,0.911871813546977)(101,0.912663755458515)(102,0.913202042304887)(103,0.912408759124087)(104,0.911614317019722)(105,0.910948905109489)(106,0.909752547307132)(107,0.912127814088598)(108,0.914119359534206)(109,0.915328467153285)(110,0.916909620991254)(111,0.915328467153285)(112,0.914536157779401)(113,0.913616398243045)(114,0.915204678362573)(115,0.916788321167883)(116,0.917578409919767)(117,0.917578409919767)(118,0.915574963609898)(119,0.918367346938775)(120,0.918367346938775)(121,0.919155134741442)(122,0.918248175182482)(123,0.920611798980335)(124,0.920611798980335)(125,0.920611798980335)(126,0.920611798980335)(127,0.920611798980335)(128,0.920611798980335)(129,0.919941775836972)(130,0.920611798980335)(131,0.919941775836972)(132,0.919941775836972)(133,0.919941775836972)(134,0.922740524781341)(135,0.922740524781341)(136,0.922740524781341)(137,0.922068463219228)(138,0.923525127458121)(139,0.924872355944566)(140,0.924872355944566)(141,0.924872355944566)(142,0.924872355944566)(143,0.923525127458121)(144,0.923525127458121)(145,0.922852983988355)(146,0.925199709513435)(147,0.923188405797101)(148,0.922519913106445)(149,0.923747276688453)(150,0.923076923076923)(151,0.92296511627907)(152,0.920611798980335)(153,0.91970802919708)(154,0.91970802919708)(155,0.918918918918919)(156,0.918918918918919)(157,0.918918918918919)(158,0.91970802919708)(159,0.91970802919708)(160,0.921052631578947)(161,0.925789860396767)(162,0.924431401320616)(163,0.924431401320616)(164,0.924431401320616)(165,0.922401171303075)(166,0.922401171303075)(167,0.921726408193123)(168,0.922514619883041)(169,0.922514619883041)(170,0.922514619883041)(171,0.922514619883041)(172,0.922514619883041)(173,0.922514619883041)(174,0.922514619883041)(175,0.922514619883041)(176,0.923301680058437)(177,0.924087591240876)(178,0.921726408193123)(179,0.920937042459736)(180,0.921840759678597)(181,0.921840759678597)(182,0.921052631578947)(183,0.921052631578947)(184,0.921726408193123)(185,0.922514619883041)(186,0.922514619883041)(187,0.923189465983906)(188,0.923976608187134)(189,0.923976608187134)(190,0.924652523774689)(191,0.924652523774689)(192,0.924652523774689)(193,0.925329428989751)(194,0.922964049889949)(195,0.924542124542124)(196,0.925329428989751)(197,0.925329428989751)(198,0.925329428989751)(199,0.925329428989751)(200,0.926900584795322)(201,0.925329428989751)(202,0.924542124542124)(203,0.923753665689149)(204,0.923753665689149)(205,0.923753665689149)(206,0.923753665689149)(207,0.923865300146413)(208,0.923976608187134)(209,0.923301680058437)(210,0.923976608187134)(211,0.923976608187134)(212,0.923976608187134)(213,0.923189465983906)(214,0.923976608187134)(215,0.923865300146413)(216,0.923865300146413)(217,0.923865300146413)(218,0.923865300146413)(219,0.923865300146413)(220,0.923865300146413)(221,0.923865300146413)(222,0.923189465983906)(223,0.923189465983906)(224,0.923189465983906)(225,0.923865300146413)(226,0.924652523774689)(227,0.925438596491228)(228,0.925438596491228)(229,0.925438596491228)(230,0.925438596491228)(231,0.926223520818115)(232,0.925438596491228)(233,0.925438596491228)(234,0.925438596491228)(235,0.925438596491228)(236,0.925438596491228)(237,0.926223520818115)(238,0.926223520818115)(239,0.926223520818115)(240,0.926223520818115)(241,0.925547445255474)(242,0.925547445255474)(243,0.925547445255474)(244,0.925547445255474)(245,0.926223520818115)(246,0.926223520818115)(247,0.926223520818115)(248,0.926223520818115)(249,0.926223520818115)(250,0.926223520818115)(251,0.926900584795322)(252,0.927578639356254)(253,0.927578639356254)(254,0.927578639356254)(255,0.926900584795322)(256,0.926900584795322)(257,0.926900584795322)(258,0.927684441197955)(259,0.927684441197955)(260,0.926900584795322)(261,0.926900584795322)(262,0.926900584795322)(263,0.926900584795322)(264,0.926900584795322)(265,0.926900584795322)(266,0.926115581565472)(267,0.925329428989751)(268,0.924652523774689)(269,0.924652523774689)(270,0.924652523774689)(271,0.924652523774689)(272,0.924652523774689)(273,0.924652523774689)(274,0.924652523774689)(275,0.924652523774689)(276,0.924652523774689)(277,0.923865300146413)(278,0.923076923076923)(279,0.922287390029325)(280,0.922287390029325)(281,0.922964049889949)(282,0.922964049889949)(283,0.922964049889949)(284,0.922964049889949)(285,0.922964049889949)(286,0.922287390029325)(287,0.922287390029325)(288,0.922287390029325)(289,0.922287390029325)(290,0.923076923076923)(291,0.923076923076923)(292,0.923865300146413)(293,0.923865300146413)(294,0.923865300146413)(295,0.923865300146413)(296,0.923865300146413)(297,0.923865300146413)(298,0.923865300146413)(299,0.923865300146413)(300,0.923865300146413)(301,0.923865300146413)(302,0.923865300146413)(303,0.924652523774689)(304,0.924652523774689)(305,0.923076923076923)(306,0.923076923076923)(307,0.923076923076923)(308,0.921726408193123)(309,0.921726408193123)(310,0.921726408193123)(311,0.921726408193123)(312,0.921726408193123)(313,0.923301680058437)(314,0.923976608187134)(315,0.923976608187134)(316,0.923976608187134)(317,0.923976608187134)(318,0.923976608187134)(319,0.923976608187134)(320,0.924652523774689)(321,0.924652523774689)(322,0.926115581565472)(323,0.926115581565472)(324,0.925438596491228)(325,0.926223520818115)(326,0.925438596491228)(327,0.924652523774689)(328,0.924652523774689)(329,0.924652523774689)(330,0.924872355944566)(331,0.924762600438276)(332,0.924762600438276)(333,0.924762600438276)(334,0.924087591240876)(335,0.924087591240876)(336,0.924762600438276)(337,0.924762600438276)(338,0.924087591240876)(339,0.924087591240876)(340,0.923976608187134)(341,0.925547445255474)(342,0.924762600438276)(343,0.924762600438276)(344,0.925438596491228)(345,0.925438596491228)(346,0.926223520818115)(347,0.927007299270073)(348,0.925655976676385)(349,0.925655976676385)(350,0.925655976676385)(351,0.925655976676385)(352,0.924981791697014)(353,0.924981791697014)(354,0.923413566739606)(355,0.925764192139738)(356,0.924198250728863)(357,0.924198250728863)(358,0.924198250728863)(359,0.924198250728863)(360,0.927219796215429)(361,0.928)(362,0.927219796215429)(363,0.927219796215429)(364,0.927219796215429)(365,0.928)(366,0.928)(367,0.927219796215429)(368,0.928)(369,0.928)(370,0.928)(371,0.928675400291121)(372,0.928675400291121)(373,0.9278951201748)(374,0.9278951201748)(375,0.926331145149526)(376,0.927113702623907)(377,0.9278951201748)(378,0.927113702623907)(379,0.9278951201748)(380,0.9278951201748)(381,0.9278951201748)(382,0.9278951201748)(383,0.9278951201748)(384,0.928675400291121)(385,0.9278951201748)(386,0.9278951201748)(387,0.9278951201748)(388,0.9278951201748)(389,0.9278951201748)(390,0.928675400291121)(391,0.928675400291121)(392,0.930232558139535)(393,0.931586608442504)(394,0.931586608442504)(395,0.932363636363636)(396,0.933042212518195)(397,0.933042212518195)(398,0.933818181818182)(399,0.933818181818182)(400,0.933042212518195) 
};
\addplot [
color=red,
mark size=0.1pt,
only marks,
mark=*,
mark options={solid,fill=red},
forget plot
]
coordinates{
 (1,0)(2,0)(3,0)(4,0)(5,0)(6,0)(7,0.064118372379778)(8,0.244197780020182)(9,0.235655737704918)(10,0.199143468950749)(11,0.201069518716577)(12,0.203579418344519)(13,0.252699784017279)(14,0.245385450597177)(15,0.243478260869565)(16,0.231947483588621)(17,0.291581108829569)(18,0.284221525600836)(19,0.279749478079332)(20,0.271293375394322)(21,0.277248677248677)(22,0.285115303983228)(23,0.303719008264463)(24,0.299896587383661)(25,0.304213771839671)(26,0.302658486707566)(27,0.299794661190965)(28,0.300613496932515)(29,0.301030927835051)(30,0.30625)(31,0.298850574712644)(32,0.304166666666667)(33,0.301255230125523)(34,0.301570680628272)(35,0.301570680628272)(36,0.301570680628272)(37,0.30625)(38,0.310598111227702)(39,0.304761904761905)(40,0.305408271474019)(41,0.309799789251844)(42,0.313025210084034)(43,0.310126582278481)(44,0.311300639658849)(45,0.30752453653217)(46,0.305676855895196)(47,0.312968917470525)(48,0.324152542372881)(49,0.325925925925926)(50,0.334736842105263)(51,0.3276955602537)(52,0.325581395348837)(53,0.327349524815206)(54,0.328767123287671)(55,0.328767123287671)(56,0.327004219409283)(57,0.336909871244635)(58,0.33763440860215)(59,0.342541436464088)(60,0.340285400658617)(61,0.344978165938865)(62,0.344978165938865)(63,0.341730558598028)(64,0.347729789590255)(65,0.358635863586359)(66,0.359030837004405)(67,0.359649122807017)(68,0.361842105263158)(69,0.361842105263158)(70,0.36144578313253)(71,0.36043956043956)(72,0.36043956043956)(73,0.359426681367144)(74,0.355016538037486)(75,0.353846153846154)(76,0.353070175438596)(77,0.356673960612691)(78,0.356673960612691)(79,0.357848518111965)(80,0.354235423542354)(81,0.354625550660793)(82,0.358695652173913)(83,0.358306188925081)(84,0.356134636264929)(85,0.356134636264929)(86,0.356284153005464)(87,0.35589519650655)(88,0.350802139037433)(89,0.352813852813853)(90,0.352813852813853)(91,0.344528710725894)(92,0.346695557963163)(93,0.344902386117137)(94,0.344902386117137)(95,0.343105320304017)(96,0.344902386117137)(97,0.344902386117137)(98,0.350109409190372)(99,0.345054945054945)(100,0.343680709534368)(101,0.343680709534368)(102,0.344370860927152)(103,0.342920353982301)(104,0.344370860927152)(105,0.344370860927152)(106,0.343300110741971)(107,0.342222222222222)(108,0.342602892102336)(109,0.341137123745819)(110,0.341517857142857)(111,0.344904815229563)(112,0.34675615212528)(113,0.347144456886898)(114,0.348603351955307)(115,0.349384098544233)(116,0.346368715083799)(117,0.34675615212528)(118,0.34675615212528)(119,0.34675615212528)(120,0.347144456886898)(121,0.347144456886898)(122,0.34675615212528)(123,0.348314606741573)(124,0.349775784753363)(125,0.349775784753363)(126,0.349384098544233)(127,0.349775784753363)(128,0.351623740201568)(129,0.351230425055928)(130,0.361419068736142)(131,0.356347438752784)(132,0.356347438752784)(133,0.356347438752784)(134,0.355307262569832)(135,0.353863381858902)(136,0.355056179775281)(137,0.354657687991021)(138,0.357142857142857)(139,0.360801781737194)(140,0.360801781737194)(141,0.356744704570791)(142,0.358574610244989)(143,0.356744704570791)(144,0.362222222222222)(145,0.356744704570791)(146,0.353467561521253)(147,0.355307262569832)(148,0.354910714285714)(149,0.353072625698324)(150,0.355307262569832)(151,0.355307262569832)(152,0.353467561521253)(153,0.355704697986577)(154,0.357142857142857)(155,0.357142857142857)(156,0.357142857142857)(157,0.360801781737194)(158,0.360801781737194)(159,0.360801781737194)(160,0.360801781737194)(161,0.361204013377926)(162,0.360178970917226)(163,0.360178970917226)(164,0.360178970917226)(165,0.360178970917226)(166,0.361607142857143)(167,0.360178970917226)(168,0.360178970917226)(169,0.360178970917226)(170,0.359776536312849)(171,0.362011173184357)(172,0.362011173184357)(173,0.361607142857143)(174,0.36036036036036)(175,0.36036036036036)(176,0.359955005624297)(177,0.359550561797753)(178,0.359147025813692)(179,0.359147025813692)(180,0.359147025813692)(181,0.359550561797753)(182,0.359147025813692)(183,0.359147025813692)(184,0.359147025813692)(185,0.359147025813692)(186,0.359147025813692)(187,0.359147025813692)(188,0.359955005624297)(189,0.36036036036036)(190,0.36036036036036)(191,0.359955005624297)(192,0.359955005624297)(193,0.359955005624297)(194,0.359550561797753)(195,0.359955005624297)(196,0.36036036036036)(197,0.362204724409449)(198,0.362204724409449)(199,0.361391694725028)(200,0.361797752808989)(201,0.361391694725028)(202,0.361391694725028)(203,0.363228699551569)(204,0.36241610738255)(205,0.363228699551569)(206,0.36241610738255)(207,0.360582306830907)(208,0.364245810055866)(209,0.367892976588629)(210,0.369710467706013)(211,0.366071428571428)(212,0.366071428571428)(213,0.366071428571428)(214,0.365663322185061)(215,0.366480446927374)(216,0.366480446927374)(217,0.366480446927374)(218,0.366480446927374)(219,0.370699223085461)(220,0.370699223085461)(221,0.366480446927374)(222,0.366480446927374)(223,0.367301231802911)(224,0.367713004484305)(225,0.367713004484305)(226,0.367713004484305)(227,0.367713004484305)(228,0.367713004484305)(229,0.367713004484305)(230,0.368125701459035)(231,0.368125701459035)(232,0.371780515117581)(233,0.375418994413408)(234,0.371780515117581)(235,0.371780515117581)(236,0.371780515117581)(237,0.371780515117581)(238,0.369540873460246)(239,0.369540873460246)(240,0.369540873460246)(241,0.371780515117581)(242,0.371780515117581)(243,0.37219730941704)(244,0.368539325842697)(245,0.368539325842697)(246,0.368125701459035)(247,0.368125701459035)(248,0.368125701459035)(249,0.368125701459035)(250,0.368125701459035)(251,0.367713004484305)(252,0.367713004484305)(253,0.368125701459035)(254,0.368125701459035)(255,0.368125701459035)(256,0.368539325842697)(257,0.368125701459035)(258,0.368125701459035)(259,0.368125701459035)(260,0.368125701459035)(261,0.368125701459035)(262,0.368125701459035)(263,0.367713004484305)(264,0.368125701459035)(265,0.367301231802911)(266,0.367301231802911)(267,0.367301231802911)(268,0.367301231802911)(269,0.367301231802911)(270,0.367301231802911)(271,0.367301231802911)(272,0.367301231802911)(273,0.366480446927374)(274,0.366480446927374)(275,0.366071428571428)(276,0.366890380313199)(277,0.370535714285714)(278,0.370535714285714)(279,0.370535714285714)(280,0.370535714285714)(281,0.370535714285714)(282,0.370535714285714)(283,0.370535714285714)(284,0.370122630992196)(285,0.370122630992196)(286,0.370122630992196)(287,0.368303571428571)(288,0.366480446927374)(289,0.366480446927374)(290,0.366071428571428)(291,0.366071428571428)(292,0.365663322185061)(293,0.366071428571428)(294,0.366071428571428)(295,0.366071428571428)(296,0.366071428571428)(297,0.366071428571428)(298,0.366890380313199)(299,0.366480446927374)(300,0.366480446927374)(301,0.366480446927374)(302,0.366480446927374)(303,0.366480446927374)(304,0.366480446927374)(305,0.366480446927374)(306,0.366480446927374)(307,0.364653243847875)(308,0.364653243847875)(309,0.364653243847875)(310,0.366480446927374)(311,0.366890380313199)(312,0.366480446927374)(313,0.366480446927374)(314,0.366480446927374)(315,0.366480446927374)(316,0.366480446927374)(317,0.366480446927374)(318,0.366890380313199)(319,0.367301231802911)(320,0.368125701459035)(321,0.368125701459035)(322,0.368539325842697)(323,0.368539325842697)(324,0.368539325842697)(325,0.371780515117581)(326,0.371780515117581)(327,0.371780515117581)(328,0.371780515117581)(329,0.371780515117581)(330,0.371780515117581)(331,0.369955156950673)(332,0.371364653243848)(333,0.371364653243848)(334,0.371364653243848)(335,0.371364653243848)(336,0.371364653243848)(337,0.371364653243848)(338,0.370949720670391)(339,0.370949720670391)(340,0.370949720670391)(341,0.370949720670391)(342,0.371364653243848)(343,0.370949720670391)(344,0.370949720670391)(345,0.370949720670391)(346,0.369127516778523)(347,0.367713004484305)(348,0.368125701459035)(349,0.370949720670391)(350,0.370949720670391)(351,0.370949720670391)(352,0.371364653243848)(353,0.370949720670391)(354,0.370949720670391)(355,0.370949720670391)(356,0.370949720670391)(357,0.370949720670391)(358,0.370949720670391)(359,0.370949720670391)(360,0.370949720670391)(361,0.371364653243848)(362,0.370949720670391)(363,0.370949720670391)(364,0.370949720670391)(365,0.372767857142857)(366,0.372767857142857)(367,0.372767857142857)(368,0.376811594202898)(369,0.376811594202898)(370,0.376811594202898)(371,0.375)(372,0.376811594202898)(373,0.375)(374,0.376811594202898)(375,0.376811594202898)(376,0.376811594202898)(377,0.380846325167038)(378,0.380422691879866)(379,0.380422691879866)(380,0.380846325167038)(381,0.380422691879866)(382,0.380422691879866)(383,0.380422691879866)(384,0.378619153674833)(385,0.380422691879866)(386,0.380422691879866)(387,0.380422691879866)(388,0.380422691879866)(389,0.378619153674833)(390,0.378619153674833)(391,0.378619153674833)(392,0.378619153674833)(393,0.378619153674833)(394,0.380846325167038)(395,0.380846325167038)(396,0.380846325167038)(397,0.380846325167038)(398,0.380846325167038)(399,0.380846325167038)(400,0.381270903010033) 
};
\addplot [
color=red,
mark size=0.1pt,
only marks,
mark=*,
mark options={solid,fill=red},
forget plot
]
coordinates{
 (1,0)(2,0)(3,0)(4,0)(5,0.0547263681592039)(6,0.504215851602023)(7,0.685350318471337)(8,0.708724832214765)(9,0.733579335793358)(10,0.742897727272727)(11,0.772291820191599)(12,0.777614138438881)(13,0.779911373707533)(14,0.770045385779123)(15,0.76402767102229)(16,0.794141578519121)(17,0.800322061191626)(18,0.797725426482534)(19,0.802287581699346)(20,0.835562549173879)(21,0.819875776397516)(22,0.813186813186813)(23,0.809786898184688)(24,0.811410459587955)(25,0.8203125)(26,0.81974921630094)(27,0.825545171339564)(28,0.82307092751364)(29,0.81974921630094)(30,0.830278884462151)(31,0.834935897435897)(32,0.856697819314642)(33,0.85580670303975)(34,0.85580670303975)(35,0.860031104199067)(36,0.86748844375963)(37,0.865947611710324)(38,0.868702290076336)(39,0.877272727272727)(40,0.876045627376426)(41,0.87910447761194)(42,0.873730043541364)(43,0.879882092851879)(44,0.885100074128984)(45,0.885925925925926)(46,0.8735976065819)(47,0.876957494407159)(48,0.876387860843819)(49,0.877868245743893)(50,0.878518518518518)(51,0.878518518518518)(52,0.880354505169867)(53,0.879348630643967)(54,0.879348630643967)(55,0.881831610044313)(56,0.881831610044313)(57,0.878765613519471)(58,0.881231671554252)(59,0.885317750182615)(60,0.884502923976608)(61,0.896500372300819)(62,0.899850523168909)(63,0.90187265917603)(64,0.905153099327856)(65,0.904334828101644)(66,0.905153099327856)(67,0.905153099327856)(68,0.904619970193741)(69,0.905970149253731)(70,0.905153099327856)(71,0.905970149253731)(72,0.908413998510797)(73,0.912592592592593)(74,0.914201183431952)(75,0.913269088213491)(76,0.918918918918919)(77,0.917293233082707)(78,0.916604057099925)(79,0.919850187265918)(80,0.91904047976012)(81,0.919729932483121)(82,0.92042042042042)(83,0.922038980509745)(84,0.921230307576894)(85,0.92065868263473)(86,0.921465968586387)(87,0.919850187265918)(88,0.92089552238806)(89,0.919523099850969)(90,0.918154761904762)(91,0.913986537023186)(92,0.913986537023186)(93,0.914414414414414)(94,0.915482423335826)(95,0.913986537023186)(96,0.914798206278027)(97,0.915608663181479)(98,0.915860014892033)(99,0.915178571428571)(100,0.914370811615785)(101,0.91475166790215)(102,0.915304606240713)(103,0.916234247590808)(104,0.916358253145818)(105,0.919557195571956)(106,0.919117647058823)(107,0.919472913616398)(108,0.919590643274854)(109,0.918486171761281)(110,0.918486171761281)(111,0.917698470502549)(112,0.917698470502549)(113,0.919037199124726)(114,0.919037199124726)(115,0.91970802919708)(116,0.91970802919708)(117,0.921282798833819)(118,0.922068463219228)(119,0.923636363636364)(120,0.923857868020305)(121,0.924528301886792)(122,0.924528301886792)(123,0.925416364952933)(124,0.923299565846599)(125,0.925416364952933)(126,0.923857868020305)(127,0.926193921852388)(128,0.928985507246377)(129,0.926300578034682)(130,0.926300578034682)(131,0.925524222704266)(132,0.928104575163399)(133,0.927325581395349)(134,0.931308749096168)(135,0.931308749096168)(136,0.931308749096168)(137,0.931308749096168)(138,0.931308749096168)(139,0.931308749096168)(140,0.931982633863965)(141,0.932657494569153)(142,0.932657494569153)(143,0.934010152284264)(144,0.932657494569153)(145,0.93033381712627)(146,0.928985507246377)(147,0.93033381712627)(148,0.93033381712627)(149,0.931686046511628)(150,0.932461873638344)(151,0.932461873638344)(152,0.932363636363636)(153,0.93002915451895)(154,0.931586608442504)(155,0.931586608442504)(156,0.931586608442504)(157,0.931586608442504)(158,0.931586608442504)(159,0.931586608442504)(160,0.931586608442504)(161,0.931586608442504)(162,0.931586608442504)(163,0.931586608442504)(164,0.932166301969365)(165,0.933528122717312)(166,0.933528122717312)(167,0.934306569343066)(168,0.934989043097151)(169,0.934989043097151)(170,0.934989043097151)(171,0.933625091174325)(172,0.933625091174325)(173,0.933430870519385)(174,0.934306569343066)(175,0.933721777130371)(176,0.934402332361516)(177,0.934306569343066)(178,0.935860058309038)(179,0.936635105608157)(180,0.936635105608157)(181,0.934497816593886)(182,0.93644996347699)(183,0.93644996347699)(184,0.93644996347699)(185,0.938001458789205)(186,0.938775510204082)(187,0.938775510204082)(188,0.938775510204082)(189,0.938775510204082)(190,0.938775510204082)(191,0.93809176984705)(192,0.936542669584245)(193,0.936542669584245)(194,0.936542669584245)(195,0.936542669584245)(196,0.93731778425656)(197,0.93731778425656)(198,0.93731778425656)(199,0.938001458789205)(200,0.944162436548223)(201,0.938001458789205)(202,0.96011396011396)(203,0.96096522356281)(204,0.956335003579098)(205,0.957142857142857)(206,0.959372772630078)(207,0.961428571428571)(208,0.963701067615658)(209,0.964539007092199)(210,0.968242766407904)(211,0.968197879858657)(212,0.965957446808511)(213,0.965272856130404)(214,0.966737438075018)(215,0.966737438075018)(216,0.967468175388967)(217,0.968107725017718)(218,0.968107725017718)(219,0.968242766407904)(220,0.971024734982332)(221,0.971106412966878)(222,0.967514124293785)(223,0.968242766407904)(224,0.968197879858657)(225,0.966690290574061)(226,0.96742209631728)(227,0.968152866242038)(228,0.968838526912181)(229,0.970338983050847)(230,0.968882602545969)(231,0.970422535211268)(232,0.97179125528914)(233,0.97029702970297)(234,0.971024734982332)(235,0.971024734982332)(236,0.97029702970297)(237,0.97179125528914)(238,0.97179125528914)(239,0.97179125528914)(240,0.97106563161609)(241,0.971106412966878)(242,0.970422535211268)(243,0.971711456859972)(244,0.971711456859972)(245,0.971711456859972)(246,0.970983722576079)(247,0.971711456859972)(248,0.972515856236786)(249,0.972515856236786)(250,0.973201692524683)(251,0.972515856236786)(252,0.969611307420495)(253,0.968926553672316)(254,0.969654199011997)(255,0.968926553672316)(256,0.97029702970297)(257,0.969568294409059)(258,0.970254957507082)(259,0.970983722576079)(260,0.971711456859972)(261,0.971711456859972)(262,0.971711456859972)(263,0.972438162544169)(264,0.971711456859972)(265,0.972438162544169)(266,0.972438162544169)(267,0.972438162544169)(268,0.970380818053596)(269,0.971106412966878)(270,0.971106412966878)(271,0.970422535211268)(272,0.970422535211268)(273,0.970422535211268)(274,0.97179125528914)(275,0.97179125528914)(276,0.97179125528914)(277,0.97179125528914)(278,0.972477064220183)(279,0.972477064220183)(280,0.972477064220183)(281,0.968838526912181)(282,0.96742209631728)(283,0.968197879858657)(284,0.966878083157153)(285,0.967514124293785)(286,0.966831333803811)(287,0.966831333803811)(288,0.966878083157153)(289,0.966878083157153)(290,0.968926553672316)(291,0.969654199011997)(292,0.968970380818054)(293,0.971870604781997)(294,0.973314606741573)(295,0.972593113141251)(296,0.972593113141251)(297,0.973314606741573)(298,0.973314606741573)(299,0.973352033660589)(300,0.973352033660589)(301,0.973352033660589)(302,0.972669936930624)(303,0.974035087719298)(304,0.973314606741573)(305,0.973998594518622)(306,0.973277074542897)(307,0.973277074542897)(308,0.973277074542897)(309,0.973277074542897)(310,0.972593113141251)(311,0.972593113141251)(312,0.972593113141251)(313,0.972593113141251)(314,0.973277074542897)(315,0.973277074542897)(316,0.973277074542897)(317,0.973277074542897)(318,0.97396199859254)(319,0.97191011235955)(320,0.97191011235955)(321,0.972593113141251)(322,0.972593113141251)(323,0.973277074542897)(324,0.973239436619718)(325,0.973925299506695)(326,0.972477064220183)(327,0.973201692524683)(328,0.973201692524683)(329,0.973888496824276)(330,0.973888496824276)(331,0.974612129760225)(332,0.974612129760225)(333,0.974612129760225)(334,0.974612129760225)(335,0.974612129760225)(336,0.974612129760225)(337,0.973888496824276)(338,0.972477064220183)(339,0.972477064220183)(340,0.974612129760225)(341,0.974612129760225)(342,0.973163841807909)(343,0.973163841807909)(344,0.972438162544169)(345,0.972438162544169)(346,0.972438162544169)(347,0.972438162544169)(348,0.970254957507082)(349,0.971711456859972)(350,0.971711456859972)(351,0.971711456859972)(352,0.971711456859972)(353,0.970338983050847)(354,0.971024734982332)(355,0.971024734982332)(356,0.971024734982332)(357,0.96969696969697)(358,0.971711456859972)(359,0.972438162544169)(360,0.972438162544169)(361,0.9723991507431)(362,0.971671388101983)(363,0.971671388101983)(364,0.971671388101983)(365,0.970983722576079)(366,0.971671388101983)(367,0.971671388101983)(368,0.972360028348689)(369,0.973011363636364)(370,0.973011363636364)(371,0.97228144989339)(372,0.971590909090909)(373,0.971590909090909)(374,0.971590909090909)(375,0.972320794889993)(376,0.971631205673759)(377,0.971631205673759)(378,0.971631205673759)(379,0.971671388101983)(380,0.971671388101983)(381,0.9723991507431)(382,0.971711456859972)(383,0.968970380818054)(384,0.968970380818054)(385,0.968970380818054)(386,0.969654199011997)(387,0.968970380818054)(388,0.968970380818054)(389,0.968970380818054)(390,0.968242766407904)(391,0.967559943582511)(392,0.968242766407904)(393,0.968242766407904)(394,0.968242766407904)(395,0.968242766407904)(396,0.96969696969697)(397,0.96969696969697)(398,0.96969696969697)(399,0.96969696969697)(400,0.970422535211268) 
};
\addplot [
color=red,
mark size=0.1pt,
only marks,
mark=*,
mark options={solid,fill=red},
forget plot
]
coordinates{
 (1,0)(2,0)(3,0.604087812263437)(4,0.665546218487395)(5,0.717069368667186)(6,0.765613519470977)(7,0.745481927710843)(8,0.763921568627451)(9,0.754633061527057)(10,0.776506483600305)(11,0.786604361370716)(12,0.790007806401249)(13,0.778830963665087)(14,0.786217697729052)(15,0.786551993745113)(16,0.784375)(17,0.788069073783359)(18,0.791208791208791)(19,0.793774319066148)(20,0.795348837209302)(21,0.795348837209302)(22,0.802773497688752)(23,0.810264385692068)(24,0.809968847352025)(25,0.812742812742813)(26,0.807453416149068)(27,0.809338521400778)(28,0.809338521400778)(29,0.807126258714175)(30,0.801556420233463)(31,0.800933125972006)(32,0.8003108003108)(33,0.798751950078003)(34,0.800623052959501)(35,0.8015625)(36,0.798751950078003)(37,0.806201550387597)(38,0.808080808080808)(39,0.809302325581395)(40,0.81055900621118)(41,0.81055900621118)(42,0.812451361867704)(43,0.819798917246713)(44,0.818604651162791)(45,0.82051282051282)(46,0.821705426356589)(47,0.821705426356589)(48,0.821068938807126)(49,0.820872274143302)(50,0.822706065318818)(51,0.824629773967264)(52,0.82591725214676)(53,0.824261275272162)(54,0.824261275272162)(55,0.826356589147287)(56,0.826356589147287)(57,0.82571649883811)(58,0.824806201550388)(59,0.826728826728827)(60,0.830601092896175)(61,0.829953198127925)(62,0.829953198127925)(63,0.829953198127925)(64,0.8296875)(65,0.830336200156372)(66,0.830601092896175)(67,0.830865159781761)(68,0.831128404669261)(69,0.831128404669261)(70,0.83203732503888)(71,0.831128404669261)(72,0.82983682983683)(73,0.82983682983683)(74,0.82983682983683)(75,0.826356589147287)(76,0.826356589147287)(77,0.827639751552795)(78,0.827639751552795)(79,0.827639751552795)(80,0.828282828282828)(81,0.828015564202335)(82,0.82699767261443)(83,0.827639751552795)(84,0.826728826728827)(85,0.826728826728827)(86,0.825816485225505)(87,0.825816485225505)(88,0.82699767261443)(89,0.827639751552795)(90,0.830865159781761)(91,0.83476898981989)(92,0.836078431372549)(93,0.838709677419355)(94,0.841357537490134)(95,0.840189873417721)(96,0.840189873417721)(97,0.840189873417721)(98,0.840189873417721)(99,0.840189873417721)(100,0.839525691699605)(101,0.839525691699605)(102,0.839525691699605)(103,0.839525691699605)(104,0.841106719367589)(105,0.841106719367589)(106,0.841106719367589)(107,0.839525691699605)(108,0.840855106888361)(109,0.840855106888361)(110,0.842857142857143)(111,0.842188739095955)(112,0.84352660841938)(113,0.84352660841938)(114,0.844197138314785)(115,0.842777334397446)(116,0.843700159489633)(117,0.843700159489633)(118,0.845541401273885)(119,0.844621513944223)(120,0.845541401273885)(121,0.845541401273885)(122,0.845541401273885)(123,0.845541401273885)(124,0.844444444444444)(125,0.843774781919112)(126,0.842857142857143)(127,0.842857142857143)(128,0.841938046068308)(129,0.843774781919112)(130,0.846275752773376)(131,0.845360824742268)(132,0.843774781919112)(133,0.844022169437846)(134,0.844690966719493)(135,0.844690966719493)(136,0.844690966719493)(137,0.844690966719493)(138,0.844690966719493)(139,0.844690966719493)(140,0.844690966719493)(141,0.844022169437846)(142,0.845360824742268)(143,0.845360824742268)(144,0.845360824742268)(145,0.844444444444444)(146,0.844444444444444)(147,0.844444444444444)(148,0.844444444444444)(149,0.844444444444444)(150,0.844444444444444)(151,0.844444444444444)(152,0.844444444444444)(153,0.844444444444444)(154,0.844444444444444)(155,0.844444444444444)(156,0.844444444444444)(157,0.845115170770453)(158,0.845786963434022)(159,0.845786963434022)(160,0.845786963434022)(161,0.845786963434022)(162,0.845786963434022)(163,0.844868735083532)(164,0.844868735083532)(165,0.844868735083532)(166,0.844868735083532)(167,0.847619047619047)(168,0.846703733121525)(169,0.84853291038858)(170,0.849445324881141)(171,0.849445324881141)(172,0.849445324881141)(173,0.848772763262074)(174,0.848772763262074)(175,0.848772763262074)(176,0.848101265822785)(177,0.848772763262074)(178,0.848772763262074)(179,0.848772763262074)(180,0.848772763262074)(181,0.848772763262074)(182,0.848772763262074)(183,0.848772763262074)(184,0.848772763262074)(185,0.848772763262074)(186,0.848772763262074)(187,0.848772763262074)(188,0.848772763262074)(189,0.848772763262074)(190,0.847860538827258)(191,0.847860538827258)(192,0.847860538827258)(193,0.847860538827258)(194,0.847860538827258)(195,0.847860538827258)(196,0.847860538827258)(197,0.847860538827258)(198,0.847860538827258)(199,0.848772763262074)(200,0.848772763262074)(201,0.848772763262074)(202,0.848772763262074)(203,0.849683544303797)(204,0.849683544303797)(205,0.849683544303797)(206,0.850592885375494)(207,0.850592885375494)(208,0.849683544303797)(209,0.849683544303797)(210,0.849683544303797)(211,0.849683544303797)(212,0.849683544303797)(213,0.849683544303797)(214,0.849683544303797)(215,0.849683544303797)(216,0.849683544303797)(217,0.849683544303797)(218,0.850592885375494)(219,0.850592885375494)(220,0.849683544303797)(221,0.849683544303797)(222,0.849683544303797)(223,0.84901185770751)(224,0.847430830039526)(225,0.84901185770751)(226,0.84901185770751)(227,0.848341232227488)(228,0.848341232227488)(229,0.847671665351223)(230,0.846761453396524)(231,0.846761453396524)(232,0.846761453396524)(233,0.846761453396524)(234,0.846761453396524)(235,0.846761453396524)(236,0.846761453396524)(237,0.846761453396524)(238,0.846761453396524)(239,0.846761453396524)(240,0.846761453396524)(241,0.846761453396524)(242,0.846761453396524)(243,0.846761453396524)(244,0.846761453396524)(245,0.846761453396524)(246,0.846761453396524)(247,0.846761453396524)(248,0.847430830039526)(249,0.848101265822785)(250,0.848101265822785)(251,0.848101265822785)(252,0.848101265822785)(253,0.848101265822785)(254,0.848101265822785)(255,0.848101265822785)(256,0.848772763262074)(257,0.848772763262074)(258,0.84853291038858)(259,0.84853291038858)(260,0.84853291038858)(261,0.84853291038858)(262,0.84853291038858)(263,0.849206349206349)(264,0.849206349206349)(265,0.849880857823669)(266,0.849880857823669)(267,0.849880857823669)(268,0.849880857823669)(269,0.849880857823669)(270,0.849880857823669)(271,0.849880857823669)(272,0.849206349206349)(273,0.849880857823669)(274,0.849880857823669)(275,0.849880857823669)(276,0.849880857823669)(277,0.849880857823669)(278,0.849880857823669)(279,0.849880857823669)(280,0.849880857823669)(281,0.849880857823669)(282,0.849642004773269)(283,0.849642004773269)(284,0.849642004773269)(285,0.849642004773269)(286,0.849642004773269)(287,0.849642004773269)(288,0.849642004773269)(289,0.849642004773269)(290,0.849642004773269)(291,0.849642004773269)(292,0.849642004773269)(293,0.849642004773269)(294,0.848966613672496)(295,0.848966613672496)(296,0.848966613672496)(297,0.849880857823669)(298,0.849880857823669)(299,0.849880857823669)(300,0.849880857823669)(301,0.849880857823669)(302,0.849880857823669)(303,0.849206349206349)(304,0.849206349206349)(305,0.849206349206349)(306,0.849206349206349)(307,0.849206349206349)(308,0.849206349206349)(309,0.849206349206349)(310,0.850793650793651)(311,0.850793650793651)(312,0.850793650793651)(313,0.850793650793651)(314,0.850793650793651)(315,0.850793650793651)(316,0.849880857823669)(317,0.849880857823669)(318,0.849880857823669)(319,0.849880857823669)(320,0.849880857823669)(321,0.849880857823669)(322,0.849880857823669)(323,0.849880857823669)(324,0.849880857823669)(325,0.849880857823669)(326,0.849880857823669)(327,0.849880857823669)(328,0.849880857823669)(329,0.849880857823669)(330,0.849880857823669)(331,0.849880857823669)(332,0.849880857823669)(333,0.849880857823669)(334,0.849880857823669)(335,0.850793650793651)(336,0.850118953211737)(337,0.850118953211737)(338,0.850118953211737)(339,0.850118953211737)(340,0.850118953211737)(341,0.850118953211737)(342,0.850118953211737)(343,0.850118953211737)(344,0.850118953211737)(345,0.850118953211737)(346,0.850118953211737)(347,0.849206349206349)(348,0.849206349206349)(349,0.849206349206349)(350,0.849206349206349)(351,0.849206349206349)(352,0.849206349206349)(353,0.849206349206349)(354,0.849206349206349)(355,0.849206349206349)(356,0.849206349206349)(357,0.849880857823669)(358,0.849880857823669)(359,0.849880857823669)(360,0.849206349206349)(361,0.849206349206349)(362,0.84853291038858)(363,0.849206349206349)(364,0.849206349206349)(365,0.849206349206349)(366,0.849206349206349)(367,0.849445324881141)(368,0.849445324881141)(369,0.849445324881141)(370,0.849445324881141)(371,0.849445324881141)(372,0.849445324881141)(373,0.849445324881141)(374,0.849445324881141)(375,0.849445324881141)(376,0.849445324881141)(377,0.849445324881141)(378,0.84853291038858)(379,0.849206349206349)(380,0.849206349206349)(381,0.849206349206349)(382,0.849445324881141)(383,0.849445324881141)(384,0.849445324881141)(385,0.849445324881141)(386,0.849445324881141)(387,0.850118953211737)(388,0.850118953211737)(389,0.850118953211737)(390,0.850118953211737)(391,0.850118953211737)(392,0.851030110935024)(393,0.850118953211737)(394,0.850118953211737)(395,0.850118953211737)(396,0.851704996034893)(397,0.851704996034893)(398,0.851704996034893)(399,0.851704996034893)(400,0.851704996034893) 
};
\addplot [
color=red,
mark size=0.1pt,
only marks,
mark=*,
mark options={solid,fill=red},
forget plot
]
coordinates{
 (1,0)(2,0)(3,0)(4,0.135040745052386)(5,0.169965075669383)(6,0.228443449048152)(7,0.226332970620239)(8,0.227518959913326)(9,0.211111111111111)(10,0.203351955307262)(11,0.2)(12,0.223929747530187)(13,0.229007633587786)(14,0.232608695652174)(15,0.756906077348066)(16,0.75705437026841)(17,0.760137457044673)(18,0.784514243973703)(19,0.8005698005698)(20,0.8)(21,0.806201550387597)(22,0.817379497623897)(23,0.824913494809688)(24,0.824343015214384)(25,0.8283378746594)(26,0.841397849462366)(27,0.839273705447209)(28,0.831978319783198)(29,0.835942818243703)(30,0.835942818243703)(31,0.837559972583962)(32,0.836438923395445)(33,0.838664812239221)(34,0.839112343966713)(35,0.843205574912892)(36,0.844382414515003)(37,0.857547838412473)(38,0.85348506401138)(39,0.853884533143264)(40,0.859021183345507)(41,0.860482103725347)(42,0.860888565185725)(43,0.860058309037901)(44,0.857351865398683)(45,0.857979502196193)(46,0.856304985337243)(47,0.866318147871546)(48,0.84996191926885)(49,0.857791225416036)(50,0.857142857142857)(51,0.865648854961832)(52,0.864741641337386)(53,0.865399239543726)(54,0.860198624904507)(55,0.86280487804878)(56,0.86969696969697)(57,0.866920152091255)(58,0.874622356495468)(59,0.880608365019011)(60,0.881818181818182)(61,0.881818181818182)(62,0.885865457294029)(63,0.893393393393393)(64,0.890895410082769)(65,0.888721804511278)(66,0.892883895131086)(67,0.896500372300819)(68,0.898292501855976)(69,0.897473997028232)(70,0.895522388059701)(71,0.903703703703704)(72,0.906942392909897)(73,0.910303928836175)(74,0.910303928836175)(75,0.90922619047619)(76,0.914877868245744)(77,0.914074074074074)(78,0.912462908011869)(79,0.912462908011869)(80,0.910037174721189)(81,0.910037174721189)(82,0.911655530809206)(83,0.910979228486647)(84,0.918322295805739)(85,0.916119620714807)(86,0.918561995597946)(87,0.918561995597946)(88,0.91812865497076)(89,0.919941775836972)(90,0.91776798825257)(91,0.919354838709677)(92,0.91800878477306)(93,0.918800292611558)(94,0.921282798833819)(95,0.921954777534646)(96,0.922514619883041)(97,0.924087591240876)(98,0.924087591240876)(99,0.924652523774689)(100,0.919911829537105)(101,0.919911829537105)(102,0.919911829537105)(103,0.919911829537105)(104,0.919793966151582)(105,0.919793966151582)(106,0.916422287390029)(107,0.915750915750916)(108,0.912306558585114)(109,0.912306558585114)(110,0.912306558585114)(111,0.911504424778761)(112,0.913525498891352)(113,0.918322295805739)(114,0.923076923076923)(115,0.922287390029325)(116,0.922401171303075)(117,0.925438596491228)(118,0.928571428571428)(119,0.926223520818115)(120,0.926900584795322)(121,0.932363636363636)(122,0.930808448652586)(123,0.930808448652586)(124,0.933139534883721)(125,0.932461873638344)(126,0.932461873638344)(127,0.932461873638344)(128,0.930909090909091)(129,0.931686046511628)(130,0.931686046511628)(131,0.931009440813362)(132,0.931009440813362)(133,0.933042212518195)(134,0.933042212518195)(135,0.933042212518195)(136,0.931686046511628)(137,0.927789934354486)(138,0.927789934354486)(139,0.928675400291121)(140,0.9278951201748)(141,0.9278951201748)(142,0.93002915451895)(143,0.930131004366812)(144,0.930909090909091)(145,0.929659173313996)(146,0.930232558139535)(147,0.928208846990573)(148,0.9288824383164)(149,0.9288824383164)(150,0.930909090909091)(151,0.932265112891478)(152,0.932944606413994)(153,0.932944606413994)(154,0.933721777130371)(155,0.932944606413994)(156,0.932944606413994)(157,0.932846715328467)(158,0.947444204463643)(159,0.949712643678161)(160,0.955714285714286)(161,0.955714285714286)(162,0.957203994293866)(163,0.959430604982206)(164,0.957203994293866)(165,0.956459671663098)(166,0.954967834167262)(167,0.956459671663098)(168,0.96090973702914)(169,0.956459671663098)(170,0.958689458689459)(171,0.959430604982206)(172,0.957947255880257)(173,0.958689458689459)(174,0.962384669978708)(175,0.959430604982206)(176,0.957947255880257)(177,0.960170697012802)(178,0.959430604982206)(179,0.959430604982206)(180,0.959430604982206)(181,0.96011396011396)(182,0.9607982893799)(183,0.959314775160599)(184,0.958630527817404)(185,0.96)(186,0.959314775160599)(187,0.959314775160599)(188,0.957325746799431)(189,0.956583629893238)(190,0.956521739130435)(191,0.955840455840456)(192,0.953604568165596)(193,0.95435092724679)(194,0.955096222380613)(195,0.957264957264957)(196,0.957264957264957)(197,0.958689458689459)(198,0.958689458689459)(199,0.958689458689459)(200,0.958689458689459)(201,0.960227272727273)(202,0.96011396011396)(203,0.958126330731015)(204,0.958066808813077)(205,0.96011396011396)(206,0.959372772630078)(207,0.959372772630078)(208,0.958689458689459)(209,0.957081545064378)(210,0.956335003579098)(211,0.956335003579098)(212,0.956335003579098)(213,0.955650929899857)(214,0.957081545064378)(215,0.952380952380952)(216,0.95373665480427)(217,0.952380952380952)(218,0.955096222380613)(219,0.955096222380613)(220,0.955714285714286)(221,0.957081545064378)(222,0.956335003579098)(223,0.957081545064378)(224,0.958571428571429)(225,0.958571428571429)(226,0.958571428571429)(227,0.958571428571429)(228,0.9578270192995)(229,0.957142857142857)(230,0.957142857142857)(231,0.956397426733381)(232,0.956397426733381)(233,0.956397426733381)(234,0.956397426733381)(235,0.957081545064378)(236,0.957081545064378)(237,0.957766642806013)(238,0.955523672883788)(239,0.956272401433692)(240,0.95702005730659)(241,0.95702005730659)(242,0.95702005730659)(243,0.95702005730659)(244,0.95702005730659)(245,0.958393113342898)(246,0.958393113342898)(247,0.959139784946237)(248,0.958393113342898)(249,0.957645369705671)(250,0.958393113342898)(251,0.958393113342898)(252,0.958393113342898)(253,0.959139784946237)(254,0.956146657081237)(255,0.956146657081237)(256,0.956146657081237)(257,0.955395683453237)(258,0.955395683453237)(259,0.955395683453237)(260,0.956083513318934)(261,0.958333333333333)(262,0.957584471603163)(263,0.958333333333333)(264,0.959827833572453)(265,0.962061560486757)(266,0.962804005722461)(267,0.961428571428571)(268,0.961428571428571)(269,0.962169878658101)(270,0.961483594864479)(271,0.962169878658101)(272,0.962169878658101)(273,0.961483594864479)(274,0.961483594864479)(275,0.96074232690935)(276,0.961483594864479)(277,0.961483594864479)(278,0.961483594864479)(279,0.961483594864479)(280,0.96074232690935)(281,0.96074232690935)(282,0.961483594864479)(283,0.962169878658101)(284,0.962169878658101)(285,0.961428571428571)(286,0.961428571428571)(287,0.962169878658101)(288,0.962169878658101)(289,0.962857142857143)(290,0.963545389563974)(291,0.962857142857143)(292,0.962115796997855)(293,0.962115796997855)(294,0.962115796997855)(295,0.962115796997855)(296,0.963597430406852)(297,0.965074839629366)(298,0.965811965811966)(299,0.965811965811966)(300,0.965811965811966)(301,0.965811965811966)(302,0.965811965811966)(303,0.965811965811966)(304,0.965074839629366)(305,0.963597430406852)(306,0.962910128388017)(307,0.962910128388017)(308,0.962910128388017)(309,0.964285714285714)(310,0.966500356379187)(311,0.966500356379187)(312,0.967236467236467)(313,0.967236467236467)(314,0.96797153024911)(315,0.96797153024911)(316,0.96797153024911)(317,0.967329545454545)(318,0.967329545454545)(319,0.967329545454545)(320,0.967329545454545)(321,0.966595593461265)(322,0.965860597439545)(323,0.968794326241135)(324,0.967329545454545)(325,0.967329545454545)(326,0.9680624556423)(327,0.968107725017718)(328,0.968794326241135)(329,0.970254957507082)(330,0.970254957507082)(331,0.97029702970297)(332,0.97029702970297)(333,0.97029702970297)(334,0.97029702970297)(335,0.969611307420495)(336,0.968926553672316)(337,0.968242766407904)(338,0.969611307420495)(339,0.969611307420495)(340,0.971024734982332)(341,0.971024734982332)(342,0.970338983050847)(343,0.970338983050847)(344,0.969611307420495)(345,0.969611307420495)(346,0.969611307420495)(347,0.969611307420495)(348,0.968926553672316)(349,0.968926553672316)(350,0.969611307420495)(351,0.969611307420495)(352,0.969611307420495)(353,0.968152866242038)(354,0.968152866242038)(355,0.968152866242038)(356,0.968152866242038)(357,0.968838526912181)(358,0.968838526912181)(359,0.968107725017718)(360,0.967375886524823)(361,0.967375886524823)(362,0.967375886524823)(363,0.968107725017718)(364,0.968107725017718)(365,0.968152866242038)(366,0.968882602545969)(367,0.968882602545969)(368,0.968882602545969)(369,0.968882602545969)(370,0.968882602545969)(371,0.969611307420495)(372,0.969611307420495)(373,0.970338983050847)(374,0.972515856236786)(375,0.972515856236786)(376,0.969611307420495)(377,0.969611307420495)(378,0.969611307420495)(379,0.969654199011997)(380,0.968926553672316)(381,0.968197879858657)(382,0.968926553672316)(383,0.968197879858657)(384,0.968926553672316)(385,0.968926553672316)(386,0.970380818053596)(387,0.970380818053596)(388,0.970380818053596)(389,0.971106412966878)(390,0.971106412966878)(391,0.972554539057002)(392,0.972554539057002)(393,0.971830985915493)(394,0.972515856236786)(395,0.972515856236786)(396,0.972515856236786)(397,0.972515856236786)(398,0.972515856236786)(399,0.972515856236786)(400,0.972515856236786) 
};
\addplot [
color=red,
mark size=0.1pt,
only marks,
mark=*,
mark options={solid,fill=red},
forget plot
]
coordinates{
 (1,0)(2,0)(3,0.0693196405648267)(4,0.225766871165644)(5,0.710022953328233)(6,0.740943267259057)(7,0.745070422535211)(8,0.740088105726872)(9,0.787461773700306)(10,0.780262143407864)(11,0.781519185591229)(12,0.778471138845554)(13,0.776911076443058)(14,0.782131661442006)(15,0.78125)(16,0.784099766173032)(17,0.782540919719408)(18,0.787735849056604)(19,0.790551181102362)(20,0.8)(21,0.793130366900859)(22,0.792094861660079)(23,0.794670846394984)(24,0.792423046566693)(25,0.801276935355148)(26,0.801271860095389)(27,0.802223987291501)(28,0.802223987291501)(29,0.802547770700637)(30,0.803500397772474)(31,0.804140127388535)(32,0.801276935355148)(33,0.801593625498008)(34,0.800319488817891)(35,0.800319488817891)(36,0.804140127388535)(37,0.804780876494024)(38,0.804780876494024)(39,0.804780876494024)(40,0.807045636509207)(41,0.808)(42,0.815873015873016)(43,0.813694267515924)(44,0.816521048451152)(45,0.81687898089172)(46,0.817529880478088)(47,0.820349761526232)(48,0.820349761526232)(49,0.818834796488428)(50,0.819488817891374)(51,0.821371610845295)(52,0.826984126984127)(53,0.82791435368755)(54,0.826984126984127)(55,0.826984126984127)(56,0.82818685669042)(57,0.829113924050633)(58,0.829113924050633)(59,0.828843106180665)(60,0.828346456692913)(61,0.827965435978005)(62,0.827586206896551)(63,0.824726134585289)(64,0.827586206896551)(65,0.827586206896551)(66,0.829536527886881)(67,0.826018808777429)(68,0.826666666666666)(69,0.826666666666666)(70,0.827044025157233)(71,0.827694728560189)(72,0.826498422712934)(73,0.827694728560189)(74,0.828999211977935)(75,0.82992125984252)(76,0.831761006289308)(77,0.832678711704635)(78,0.833333333333333)(79,0.833333333333333)(80,0.833333333333333)(81,0.833988985051141)(82,0.833988985051141)(83,0.833988985051141)(84,0.833070866141732)(85,0.834905660377358)(86,0.833988985051141)(87,0.833988985051141)(88,0.832678711704635)(89,0.832025117739403)(90,0.832941176470588)(91,0.831372549019608)(92,0.831372549019608)(93,0.829536527886881)(94,0.831496062992126)(95,0.831496062992126)(96,0.830575256107171)(97,0.8318863456985)(98,0.8318863456985)(99,0.833860759493671)(100,0.835182250396196)(101,0.838504375497215)(102,0.84185303514377)(103,0.839171974522293)(104,0.84185303514377)(105,0.839171974522293)(106,0.836507936507936)(107,0.839840637450199)(108,0.839840637450199)(109,0.838504375497215)(110,0.839171974522293)(111,0.838504375497215)(112,0.840349483717236)(113,0.839171974522293)(114,0.840095465393795)(115,0.840095465393795)(116,0.840764331210191)(117,0.840764331210191)(118,0.841686555290374)(119,0.841686555290374)(120,0.843027888446215)(121,0.843700159489633)(122,0.841434262948207)(123,0.841434262948207)(124,0.842105263157895)(125,0.84)(126,0.84)(127,0.840255591054313)(128,0.840927258193445)(129,0.8416)(130,0.8416)(131,0.84185303514377)(132,0.84185303514377)(133,0.840927258193445)(134,0.840255591054313)(135,0.8384)(136,0.8384)(137,0.8384)(138,0.8384)(139,0.83974358974359)(140,0.84)(141,0.838141025641025)(142,0.835341365461847)(143,0.835341365461847)(144,0.835341365461847)(145,0.838141025641025)(146,0.838141025641025)(147,0.839071257005604)(148,0.838141025641025)(149,0.839071257005604)(150,0.839071257005604)(151,0.84)(152,0.840672538030424)(153,0.840672538030424)(154,0.840672538030424)(155,0.842696629213483)(156,0.842443729903537)(157,0.843373493975903)(158,0.842696629213483)(159,0.842696629213483)(160,0.842696629213483)(161,0.842696629213483)(162,0.842696629213483)(163,0.842696629213483)(164,0.842696629213483)(165,0.842696629213483)(166,0.842696629213483)(167,0.842696629213483)(168,0.842696629213483)(169,0.842696629213483)(170,0.842696629213483)(171,0.842696629213483)(172,0.842696629213483)(173,0.842696629213483)(174,0.842696629213483)(175,0.842696629213483)(176,0.841346153846154)(177,0.841346153846154)(178,0.842273819055244)(179,0.83855421686747)(180,0.839486356340289)(181,0.839486356340289)(182,0.837620578778135)(183,0.83855421686747)(184,0.840417000801925)(185,0.839486356340289)(186,0.839486356340289)(187,0.841346153846154)(188,0.841346153846154)(189,0.841346153846154)(190,0.841346153846154)(191,0.840672538030424)(192,0.840672538030424)(193,0.840672538030424)(194,0.840672538030424)(195,0.84)(196,0.839071257005604)(197,0.84)(198,0.84)(199,0.84)(200,0.84185303514377)(201,0.842777334397446)(202,0.843450479233227)(203,0.843450479233227)(204,0.842777334397446)(205,0.842777334397446)(206,0.842777334397446)(207,0.842777334397446)(208,0.843700159489633)(209,0.84185303514377)(210,0.840672538030424)(211,0.840927258193445)(212,0.842777334397446)(213,0.842777334397446)(214,0.842777334397446)(215,0.842777334397446)(216,0.842777334397446)(217,0.843700159489633)(218,0.843700159489633)(219,0.843700159489633)(220,0.843700159489633)(221,0.845541401273885)(222,0.843700159489633)(223,0.845541401273885)(224,0.845541401273885)(225,0.845541401273885)(226,0.845541401273885)(227,0.847133757961783)(228,0.847133757961783)(229,0.847133757961783)(230,0.847133757961783)(231,0.847133757961783)(232,0.84352660841938)(233,0.845360824742268)(234,0.844444444444444)(235,0.844444444444444)(236,0.84352660841938)(237,0.844197138314785)(238,0.844197138314785)(239,0.844868735083532)(240,0.844868735083532)(241,0.844868735083532)(242,0.847133757961783)(243,0.847133757961783)(244,0.847133757961783)(245,0.846215139442231)(246,0.846215139442231)(247,0.846215139442231)(248,0.846215139442231)(249,0.846215139442231)(250,0.846215139442231)(251,0.84688995215311)(252,0.84688995215311)(253,0.84688995215311)(254,0.84688995215311)(255,0.846915460776847)(256,0.859040590405904)(257,0.852962692026335)(258,0.852097130242826)(259,0.855869242199108)(260,0.86721680420105)(261,0.865558912386707)(262,0.866868381240544)(263,0.866868381240544)(264,0.865151515151515)(265,0.869630746043708)(266,0.869630746043708)(267,0.866010598031794)(268,0.867981790591806)(269,0.866666666666666)(270,0.867724867724868)(271,0.869499241274659)(272,0.870820668693009)(273,0.874242424242424)(274,0.874242424242424)(275,0.872727272727273)(276,0.875190258751903)(277,0.874333587204874)(278,0.876146788990826)(279,0.87681713848508)(280,0.878903274942879)(281,0.878903274942879)(282,0.878903274942879)(283,0.879573170731707)(284,0.88042650418888)(285,0.88042650418888)(286,0.88042650418888)(287,0.88042650418888)(288,0.88109756097561)(289,0.881769641495042)(290,0.880916030534351)(291,0.883472962680883)(292,0.883472962680883)(293,0.883472962680883)(294,0.883472962680883)(295,0.883472962680883)(296,0.882800608828006)(297,0.882800608828006)(298,0.882800608828006)(299,0.883472962680883)(300,0.880916030534351)(301,0.880916030534351)(302,0.880916030534351)(303,0.881407804131599)(304,0.881407804131599)(305,0.881407804131599)(306,0.881226053639847)(307,0.880368098159509)(308,0.88208269525268)(309,0.88208269525268)(310,0.884113584036838)(311,0.882758620689655)(312,0.883614088820827)(313,0.885998469778118)(314,0.886676875957121)(315,0.886676875957121)(316,0.886676875957121)(317,0.886676875957121)(318,0.886676875957121)(319,0.886503067484662)(320,0.886503067484662)(321,0.885648503453568)(322,0.888208269525268)(323,0.88735632183908)(324,0.886503067484662)(325,0.88735632183908)(326,0.886503067484662)(327,0.888208269525268)(328,0.885648503453568)(329,0.885648503453568)(330,0.885648503453568)(331,0.88735632183908)(332,0.885648503453568)(333,0.885648503453568)(334,0.888208269525268)(335,0.888208269525268)(336,0.888208269525268)(337,0.888208269525268)(338,0.888208269525268)(339,0.888208269525268)(340,0.888208269525268)(341,0.889228418640183)(342,0.888549618320611)(343,0.889228418640183)(344,0.889228418640183)(345,0.889908256880734)(346,0.889908256880734)(347,0.889228418640183)(348,0.889908256880734)(349,0.889228418640183)(350,0.889228418640183)(351,0.889228418640183)(352,0.889228418640183)(353,0.889228418640183)(354,0.889228418640183)(355,0.888379204892966)(356,0.887528691660291)(357,0.887528691660291)(358,0.88871834228703)(359,0.889400921658986)(360,0.888547271329746)(361,0.887864823348694)(362,0.88871834228703)(363,0.888547271329746)(364,0.889400921658986)(365,0.88871834228703)(366,0.88871834228703)(367,0.88871834228703)(368,0.88871834228703)(369,0.88871834228703)(370,0.88871834228703)(371,0.88871834228703)(372,0.88871834228703)(373,0.88871834228703)(374,0.88871834228703)(375,0.88871834228703)(376,0.88871834228703)(377,0.88871834228703)(378,0.88871834228703)(379,0.88871834228703)(380,0.88871834228703)(381,0.889400921658986)(382,0.89025326170376)(383,0.891621829362029)(384,0.890937019969278)(385,0.889570552147239)(386,0.889570552147239)(387,0.888547271329746)(388,0.888547271329746)(389,0.888547271329746)(390,0.888547271329746)(391,0.888547271329746)(392,0.887692307692308)(393,0.887692307692308)(394,0.886661526599846)(395,0.886661526599846)(396,0.886661526599846)(397,0.887345679012346)(398,0.887345679012346)(399,0.887345679012346)(400,0.887345679012346) 
};
\addplot [
color=red,
mark size=0.1pt,
only marks,
mark=*,
mark options={solid,fill=red},
forget plot
]
coordinates{
 (1,0)(2,0)(3,0)(4,0.132897603485839)(5,0.195175438596491)(6,0.196652719665272)(7,0.214432989690722)(8,0.242424242424242)(9,0.216505894962487)(10,0.202672605790646)(11,0.253699788583509)(12,0.252100840336134)(13,0.252100840336134)(14,0.25668449197861)(15,0.254817987152034)(16,0.259541984732824)(17,0.257360959651036)(18,0.256017505470459)(19,0.259423503325942)(20,0.260869565217391)(21,0.263513513513513)(22,0.264406779661017)(23,0.268518518518518)(24,0.268208092485549)(25,0.268208092485549)(26,0.264840182648402)(27,0.264538198403649)(28,0.265895953757225)(29,0.251748251748252)(30,0.251748251748252)(31,0.252927400468384)(32,0.263096623981374)(33,0.259953161592506)(34,0.259953161592506)(35,0.261792452830189)(36,0.263345195729537)(37,0.263345195729537)(38,0.263345195729537)(39,0.260143198090692)(40,0.243373493975904)(41,0.241254523522316)(42,0.241545893719807)(43,0.241545893719807)(44,0.243080625752106)(45,0.244946492271106)(46,0.247619047619048)(47,0.255621301775148)(48,0.25)(49,0.25)(50,0.249702734839477)(51,0.251184834123223)(52,0.250591016548463)(53,0.25)(54,0.250591016548463)(55,0.250591016548463)(56,0.256470588235294)(57,0.256470588235294)(58,0.257075471698113)(59,0.257075471698113)(60,0.257075471698113)(61,0.257075471698113)(62,0.257075471698113)(63,0.253855278766311)(64,0.254761904761905)(65,0.254761904761905)(66,0.258907363420428)(67,0.258064516129032)(68,0.257756563245823)(69,0.258064516129032)(70,0.259523809523809)(71,0.260972716488731)(72,0.259833134684148)(73,0.26397146254459)(74,0.26397146254459)(75,0.264285714285714)(76,0.264285714285714)(77,0.264285714285714)(78,0.26491646778043)(79,0.264600715137068)(80,0.264600715137068)(81,0.264600715137068)(82,0.26555023923445)(83,0.265868263473054)(84,0.266187050359712)(85,0.260240963855422)(86,0.260240963855422)(87,0.2647412755716)(88,0.2647412755716)(89,0.2647412755716)(90,0.264423076923077)(91,0.26378896882494)(92,0.272401433691756)(93,0.268585131894484)(94,0.266506602641056)(95,0.266506602641056)(96,0.266506602641056)(97,0.266506602641056)(98,0.266506602641056)(99,0.268585131894484)(100,0.268585131894484)(101,0.268585131894484)(102,0.272727272727273)(103,0.268585131894484)(104,0.268585131894484)(105,0.264423076923077)(106,0.262967430639324)(107,0.262967430639324)(108,0.262967430639324)(109,0.262967430639324)(110,0.262967430639324)(111,0.263285024154589)(112,0.263285024154589)(113,0.263285024154589)(114,0.262967430639324)(115,0.26360338573156)(116,0.26360338573156)(117,0.26360338573156)(118,0.26360338573156)(119,0.262650602409639)(120,0.263285024154589)(121,0.263285024154589)(122,0.267148014440433)(123,0.267148014440433)(124,0.265700483091787)(125,0.265700483091787)(126,0.265379975874548)(127,0.265700483091787)(128,0.269879518072289)(129,0.269879518072289)(130,0.263285024154589)(131,0.263285024154589)(132,0.26360338573156)(133,0.265700483091787)(134,0.265700483091787)(135,0.265700483091787)(136,0.265700483091787)(137,0.26360338573156)(138,0.263922518159806)(139,0.261818181818182)(140,0.261818181818182)(141,0.261818181818182)(142,0.311475409836065)(143,0.313725490196078)(144,0.337485843714609)(145,0.35016835016835)(146,0.357870894677237)(147,0.35972850678733)(148,0.357870894677237)(149,0.362204724409449)(150,0.36117381489842)(151,0.360953461975028)(152,0.358857142857143)(153,0.357630979498861)(154,0.359090909090909)(155,0.358683314415437)(156,0.359090909090909)(157,0.357224118316268)(158,0.356979405034325)(159,0.357224118316268)(160,0.357224118316268)(161,0.357224118316268)(162,0.356818181818182)(163,0.355353075170843)(164,0.357224118316268)(165,0.357224118316268)(166,0.357224118316268)(167,0.359090909090909)(168,0.358038768529076)(169,0.354691075514874)(170,0.354691075514874)(171,0.352806414662085)(172,0.350917431192661)(173,0.354691075514874)(174,0.355097365406644)(175,0.354691075514874)(176,0.353211009174312)(177,0.353211009174312)(178,0.354691075514874)(179,0.354691075514874)(180,0.352806414662085)(181,0.355097365406644)(182,0.351724137931034)(183,0.353211009174312)(184,0.353211009174312)(185,0.354691075514874)(186,0.354691075514874)(187,0.355097365406644)(188,0.355097365406644)(189,0.355097365406644)(190,0.355097365406644)(191,0.355097365406644)(192,0.355097365406644)(193,0.355097365406644)(194,0.355097365406644)(195,0.355097365406644)(196,0.355097365406644)(197,0.355097365406644)(198,0.356979405034325)(199,0.358857142857143)(200,0.358857142857143)(201,0.356979405034325)(202,0.359908883826879)(203,0.359499431171786)(204,0.359499431171786)(205,0.358038768529076)(206,0.356571428571429)(207,0.358447488584475)(208,0.358038768529076)(209,0.358038768529076)(210,0.361363636363636)(211,0.360953461975028)(212,0.360135900339751)(213,0.360135900339751)(214,0.360953461975028)(215,0.360953461975028)(216,0.360953461975028)(217,0.359908883826879)(218,0.361774744027304)(219,0.361774744027304)(220,0.359499431171786)(221,0.359499431171786)(222,0.361363636363636)(223,0.359908883826879)(224,0.359499431171786)(225,0.359908883826879)(226,0.361363636363636)(227,0.361774744027304)(228,0.361774744027304)(229,0.361363636363636)(230,0.361363636363636)(231,0.358038768529076)(232,0.358683314415437)(233,0.359090909090909)(234,0.359499431171786)(235,0.360953461975028)(236,0.360953461975028)(237,0.36281179138322)(238,0.36281179138322)(239,0.364665911664779)(240,0.36281179138322)(241,0.360953461975028)(242,0.360135900339751)(243,0.360135900339751)(244,0.361990950226244)(245,0.360135900339751)(246,0.360135900339751)(247,0.360135900339751)(248,0.360135900339751)(249,0.360953461975028)(250,0.360544217687075)(251,0.360544217687075)(252,0.360544217687075)(253,0.360544217687075)(254,0.360544217687075)(255,0.360544217687075)(256,0.360544217687075)(257,0.360953461975028)(258,0.360953461975028)(259,0.36281179138322)(260,0.36281179138322)(261,0.36281179138322)(262,0.36281179138322)(263,0.36281179138322)(264,0.36322360953462)(265,0.361990950226244)(266,0.361990950226244)(267,0.361990950226244)(268,0.361990950226244)(269,0.360135900339751)(270,0.360135900339751)(271,0.360135900339751)(272,0.360135900339751)(273,0.360135900339751)(274,0.360135900339751)(275,0.361363636363636)(276,0.361363636363636)(277,0.360544217687075)(278,0.360544217687075)(279,0.360544217687075)(280,0.360544217687075)(281,0.361363636363636)(282,0.360953461975028)(283,0.360953461975028)(284,0.36322360953462)(285,0.36322360953462)(286,0.36281179138322)(287,0.36281179138322)(288,0.36281179138322)(289,0.36281179138322)(290,0.362400906002265)(291,0.362400906002265)(292,0.362400906002265)(293,0.362400906002265)(294,0.36281179138322)(295,0.363636363636364)(296,0.36322360953462)(297,0.36322360953462)(298,0.362400906002265)(299,0.362400906002265)(300,0.362400906002265)(301,0.366101694915254)(302,0.366515837104072)(303,0.366515837104072)(304,0.366515837104072)(305,0.366515837104072)(306,0.366515837104072)(307,0.366515837104072)(308,0.366515837104072)(309,0.368361581920904)(310,0.37020316027088)(311,0.368361581920904)(312,0.37020316027088)(313,0.37020316027088)(314,0.37020316027088)(315,0.37020316027088)(316,0.37020316027088)(317,0.366930917327293)(318,0.366930917327293)(319,0.366930917327293)(320,0.368778280542986)(321,0.36734693877551)(322,0.36734693877551)(323,0.36734693877551)(324,0.365493757094211)(325,0.365493757094211)(326,0.369195922989807)(327,0.36734693877551)(328,0.36734693877551)(329,0.365493757094211)(330,0.363636363636364)(331,0.369195922989807)(332,0.36734693877551)(333,0.369195922989807)(334,0.369195922989807)(335,0.369195922989807)(336,0.369195922989807)(337,0.368778280542986)(338,0.368778280542986)(339,0.368778280542986)(340,0.366930917327293)(341,0.368778280542986)(342,0.368778280542986)(343,0.368778280542986)(344,0.368778280542986)(345,0.368778280542986)(346,0.368778280542986)(347,0.368361581920904)(348,0.368361581920904)(349,0.368361581920904)(350,0.368361581920904)(351,0.37020316027088)(352,0.37020316027088)(353,0.368361581920904)(354,0.368361581920904)(355,0.37020316027088)(356,0.37020316027088)(357,0.368361581920904)(358,0.368361581920904)(359,0.368361581920904)(360,0.368361581920904)(361,0.37020316027088)(362,0.37020316027088)(363,0.37570303712036)(364,0.373873873873874)(365,0.373873873873874)(366,0.373453318335208)(367,0.37570303712036)(368,0.37570303712036)(369,0.37570303712036)(370,0.373453318335208)(371,0.373033707865168)(372,0.373033707865168)(373,0.373033707865168)(374,0.374859708193041)(375,0.374859708193041)(376,0.374859708193041)(377,0.372615039281706)(378,0.373033707865168)(379,0.373033707865168)(380,0.372615039281706)(381,0.372615039281706)(382,0.372615039281706)(383,0.372615039281706)(384,0.372615039281706)(385,0.372615039281706)(386,0.372615039281706)(387,0.372615039281706)(388,0.372615039281706)(389,0.372615039281706)(390,0.373033707865168)(391,0.373033707865168)(392,0.373033707865168)(393,0.373033707865168)(394,0.373033707865168)(395,0.373033707865168)(396,0.373033707865168)(397,0.373033707865168)(398,0.373033707865168)(399,0.373033707865168)(400,0.373033707865168) 
};
\addplot [
color=red,
mark size=0.1pt,
only marks,
mark=*,
mark options={solid,fill=red},
forget plot
]
coordinates{
 (1,0)(2,0)(3,0)(4,0)(5,0)(6,0)(7,0)(8,0)(9,0)(10,0.0781990521327014)(11,0.0768277571251549)(12,0.102689486552567)(13,0.173611111111111)(14,0.210884353741497)(15,0.218302094818081)(16,0.218302094818081)(17,0.214364640883978)(18,0.232201533406353)(19,0.232201533406353)(20,0.231023102310231)(21,0.223204419889503)(22,0.21850613154961)(23,0.214765100671141)(24,0.213723284589426)(25,0.205649717514124)(26,0.205649717514124)(27,0.201357466063348)(28,0.200676437429538)(29,0.202702702702703)(30,0.208754208754209)(31,0.210762331838565)(32,0.211473565804274)(33,0.20814479638009)(34,0.206818181818182)(35,0.210526315789474)(36,0.21689497716895)(37,0.221208665906499)(38,0.222988505747126)(39,0.223502304147465)(40,0.229621125143513)(41,0.231651376146789)(42,0.237714285714286)(43,0.237714285714286)(44,0.238258877434135)(45,0.239907727797001)(46,0.242142025611176)(47,0.242142025611176)(48,0.244186046511628)(49,0.245040840140023)(50,0.246225319396051)(51,0.24390243902439)(52,0.245327102803738)(53,0.251184834123223)(54,0.249411764705882)(55,0.253223915592028)(56,0.257611241217799)(57,0.263710618436406)(58,0.257309941520468)(59,0.257309941520468)(60,0.263096623981374)(61,0.263096623981374)(62,0.263096623981374)(63,0.263403263403263)(64,0.262485481997677)(65,0.262485481997677)(66,0.263096623981374)(67,0.260969976905312)(68,0.262790697674419)(69,0.264018691588785)(70,0.265258215962441)(71,0.265258215962441)(72,0.263219741480611)(73,0.264150943396226)(74,0.264150943396226)(75,0.264150943396226)(76,0.26508875739645)(77,0.265402843601896)(78,0.265402843601896)(79,0.265402843601896)(80,0.265402843601896)(81,0.259215219976219)(82,0.265402843601896)(83,0.267455621301775)(84,0.27122641509434)(85,0.267455621301775)(86,0.267455621301775)(87,0.268408551068884)(88,0.269689737470167)(89,0.269689737470167)(90,0.269368295589988)(91,0.269689737470167)(92,0.269368295589988)(93,0.269368295589988)(94,0.269689737470167)(95,0.27065868263473)(96,0.27065868263473)(97,0.272401433691756)(98,0.272401433691756)(99,0.272401433691756)(100,0.272401433691756)(101,0.272401433691756)(102,0.270334928229665)(103,0.272076372315036)(104,0.272076372315036)(105,0.270011947431302)(106,0.270334928229665)(107,0.270334928229665)(108,0.270334928229665)(109,0.270334928229665)(110,0.270334928229665)(111,0.270334928229665)(112,0.270334928229665)(113,0.270334928229665)(114,0.270334928229665)(115,0.270334928229665)(116,0.27479091995221)(117,0.27479091995221)(118,0.282996432818074)(119,0.289099526066351)(120,0.31907514450867)(121,0.309495896834701)(122,0.315420560747664)(123,0.315420560747664)(124,0.332947976878613)(125,0.323255813953488)(126,0.319347319347319)(127,0.317386231038506)(128,0.319347319347319)(129,0.329084588644264)(130,0.33679354094579)(131,0.33679354094579)(132,0.342135476463835)(133,0.338709677419355)(134,0.340229885057471)(135,0.340621403912543)(136,0.344036697247706)(137,0.344036697247706)(138,0.344036697247706)(139,0.345933562428408)(140,0.347826086956522)(141,0.348224513172967)(142,0.356413166855846)(143,0.35827664399093)(144,0.352673492605233)(145,0.352)(146,0.348224513172967)(147,0.34443168771527)(148,0.342528735632184)(149,0.34443168771527)(150,0.34443168771527)(151,0.342528735632184)(152,0.347826086956522)(153,0.347826086956522)(154,0.348224513172967)(155,0.346636259977195)(156,0.34624145785877)(157,0.348122866894198)(158,0.349315068493151)(159,0.348916761687571)(160,0.348519362186788)(161,0.349315068493151)(162,0.345537757437071)(163,0.342528735632184)(164,0.340229885057471)(165,0.34443168771527)(166,0.34443168771527)(167,0.346330275229358)(168,0.34443168771527)(169,0.34443168771527)(170,0.34443168771527)(171,0.34331797235023)(172,0.344827586206896)(173,0.344827586206896)(174,0.346330275229358)(175,0.346330275229358)(176,0.345537757437071)(177,0.346636259977195)(178,0.347428571428571)(179,0.347826086956522)(180,0.347826086956522)(181,0.347428571428571)(182,0.346330275229358)(183,0.346330275229358)(184,0.346330275229358)(185,0.345537757437071)(186,0.345537757437071)(187,0.345537757437071)(188,0.346330275229358)(189,0.346330275229358)(190,0.346330275229358)(191,0.34443168771527)(192,0.342922899884925)(193,0.341013824884793)(194,0.341013824884793)(195,0.341013824884793)(196,0.337182448036951)(197,0.339100346020761)(198,0.339491916859122)(199,0.340621403912543)(200,0.346330275229358)(201,0.350114416475972)(202,0.350114416475972)(203,0.350114416475972)(204,0.352402745995423)(205,0.353881278538813)(206,0.354948805460751)(207,0.3557582668187)(208,0.353881278538813)(209,0.354285714285714)(210,0.354285714285714)(211,0.356164383561644)(212,0.356164383561644)(213,0.354285714285714)(214,0.356164383561644)(215,0.356164383561644)(216,0.348623853211009)(217,0.346727898966705)(218,0.346727898966705)(219,0.347826086956522)(220,0.346330275229358)(221,0.346330275229358)(222,0.346330275229358)(223,0.34443168771527)(224,0.342922899884925)(225,0.337962962962963)(226,0.339491916859122)(227,0.335648148148148)(228,0.339491916859122)(229,0.33757225433526)(230,0.339491916859122)(231,0.339491916859122)(232,0.34331797235023)(233,0.34331797235023)(234,0.341407151095732)(235,0.335648148148148)(236,0.337182448036951)(237,0.341013824884793)(238,0.341407151095732)(239,0.339491916859122)(240,0.33757225433526)(241,0.339491916859122)(242,0.345224395857307)(243,0.341407151095732)(244,0.341407151095732)(245,0.34331797235023)(246,0.34331797235023)(247,0.339491916859122)(248,0.341801385681293)(249,0.341407151095732)(250,0.339100346020761)(251,0.341407151095732)(252,0.343713956170703)(253,0.339884393063584)(254,0.339884393063584)(255,0.341801385681293)(256,0.341801385681293)(257,0.343713956170703)(258,0.343713956170703)(259,0.343713956170703)(260,0.345622119815668)(261,0.345622119815668)(262,0.345622119815668)(263,0.341801385681293)(264,0.339884393063584)(265,0.339884393063584)(266,0.339491916859122)(267,0.339491916859122)(268,0.33603707995365)(269,0.337962962962963)(270,0.339884393063584)(271,0.339884393063584)(272,0.340277777777778)(273,0.340277777777778)(274,0.340277777777778)(275,0.340277777777778)(276,0.338354577056779)(277,0.338354577056779)(278,0.338354577056779)(279,0.338354577056779)(280,0.338354577056779)(281,0.338354577056779)(282,0.340277777777778)(283,0.349827387802071)(284,0.349827387802071)(285,0.346020761245675)(286,0.344110854503464)(287,0.346020761245675)(288,0.347926267281106)(289,0.347926267281106)(290,0.349827387802071)(291,0.349827387802071)(292,0.349827387802071)(293,0.347926267281106)(294,0.349827387802071)(295,0.351724137931034)(296,0.351724137931034)(297,0.349827387802071)(298,0.346020761245675)(299,0.346020761245675)(300,0.346020761245675)(301,0.346020761245675)(302,0.346020761245675)(303,0.346020761245675)(304,0.347926267281106)(305,0.347926267281106)(306,0.346020761245675)(307,0.347926267281106)(308,0.347926267281106)(309,0.347926267281106)(310,0.347926267281106)(311,0.347926267281106)(312,0.347926267281106)(313,0.347926267281106)(314,0.347926267281106)(315,0.35361653272101)(316,0.35361653272101)(317,0.354691075514874)(318,0.355504587155963)(319,0.357388316151203)(320,0.357388316151203)(321,0.357388316151203)(322,0.357388316151203)(323,0.357388316151203)(324,0.357388316151203)(325,0.356979405034325)(326,0.357388316151203)(327,0.357388316151203)(328,0.357388316151203)(329,0.357388316151203)(330,0.357388316151203)(331,0.357388316151203)(332,0.357388316151203)(333,0.357388316151203)(334,0.357388316151203)(335,0.357388316151203)(336,0.357388316151203)(337,0.357388316151203)(338,0.357388316151203)(339,0.357388316151203)(340,0.357388316151203)(341,0.359267734553776)(342,0.357388316151203)(343,0.359267734553776)(344,0.359267734553776)(345,0.355097365406644)(346,0.355097365406644)(347,0.355097365406644)(348,0.356979405034325)(349,0.356979405034325)(350,0.356979405034325)(351,0.356979405034325)(352,0.356979405034325)(353,0.356979405034325)(354,0.356979405034325)(355,0.356979405034325)(356,0.356571428571429)(357,0.356979405034325)(358,0.356979405034325)(359,0.358447488584475)(360,0.360319270239453)(361,0.358447488584475)(362,0.360319270239453)(363,0.360319270239453)(364,0.360319270239453)(365,0.358447488584475)(366,0.358447488584475)(367,0.358447488584475)(368,0.358447488584475)(369,0.358447488584475)(370,0.358447488584475)(371,0.358857142857143)(372,0.354691075514874)(373,0.354691075514874)(374,0.354691075514874)(375,0.356571428571429)(376,0.354691075514874)(377,0.354285714285714)(378,0.354285714285714)(379,0.354285714285714)(380,0.354285714285714)(381,0.356164383561644)(382,0.356164383561644)(383,0.356164383561644)(384,0.357630979498861)(385,0.357630979498861)(386,0.357630979498861)(387,0.357630979498861)(388,0.359908883826879)(389,0.359908883826879)(390,0.361774744027304)(391,0.361774744027304)(392,0.361774744027304)(393,0.359908883826879)(394,0.359908883826879)(395,0.359908883826879)(396,0.359908883826879)(397,0.359908883826879)(398,0.359908883826879)(399,0.361774744027304)(400,0.361774744027304) 
};
\addplot [
color=red,
mark size=0.1pt,
only marks,
mark=*,
mark options={solid,fill=red},
forget plot
]
coordinates{
 (1,0)(2,0)(3,0)(4,0)(5,0)(6,0)(7,0)(8,0)(9,0.0379918588873813)(10,0.0353741496598639)(11,0.0808344198174706)(12,0.081151832460733)(13,0.0976863753213367)(14,0.0952380952380952)(15,0.111392405063291)(16,0.12453300124533)(17,0.124688279301746)(18,0.125)(19,0.124688279301746)(20,0.126742712294043)(21,0.128950695322377)(22,0.126582278481013)(23,0.126742712294043)(24,0.127388535031847)(25,0.12853470437018)(26,0.128700128700129)(27,0.133162612035851)(28,0.132992327365729)(29,0.133848133848134)(30,0.134193548387097)(31,0.134366925064599)(32,0.131953428201811)(33,0.131953428201811)(34,0.131953428201811)(35,0.131443298969072)(36,0.131953428201811)(37,0.132124352331606)(38,0.127272727272727)(39,0.122555410691004)(40,0.117647058823529)(41,0.117647058823529)(42,0.651198762567672)(43,0.702740688685875)(44,0.692753623188406)(45,0.712915129151291)(46,0.764621968616262)(47,0.753623188405797)(48,0.765705838876571)(49,0.774774774774775)(50,0.788432267884323)(51,0.791540785498489)(52,0.810126582278481)(53,0.800898203592814)(54,0.798798798798799)(55,0.811263318112633)(56,0.811550151975684)(57,0.813169984686064)(58,0.817270624518119)(59,0.821981424148607)(60,0.820433436532508)(61,0.817270624518119)(62,0.827480916030534)(63,0.827113480578827)(64,0.838212634822804)(65,0.834621329211746)(66,0.836641221374046)(67,0.836641221374046)(68,0.837423312883435)(69,0.837423312883435)(70,0.840336134453781)(71,0.847457627118644)(72,0.861937452326468)(73,0.865312264860797)(74,0.865807429871114)(75,0.865807429871114)(76,0.86910197869102)(77,0.866920152091255)(78,0.866057838660578)(79,0.864291129643669)(80,0.864253393665158)(81,0.865109269027882)(82,0.869038607115821)(83,0.869038607115821)(84,0.870356330553449)(85,0.872340425531915)(86,0.871678056188307)(87,0.870696250956389)(88,0.868702290076336)(89,0.868501529051988)(90,0.869166029074216)(91,0.869631901840491)(92,0.868965517241379)(93,0.868965517241379)(94,0.869631901840491)(95,0.871165644171779)(96,0.871362940275651)(97,0.872226472838561)(98,0.868501529051988)(99,0.868501529051988)(100,0.869166029074216)(101,0.868501529051988)(102,0.869166029074216)(103,0.869365928189457)(104,0.870498084291187)(105,0.868501529051988)(106,0.867838044308632)(107,0.867838044308632)(108,0.872503840245776)(109,0.873369148119724)(110,0.874039938556067)(111,0.874039938556067)(112,0.874039938556067)(113,0.875766871165644)(114,0.875095785440613)(115,0.873756694720734)(116,0.874425727411945)(117,0.874425727411945)(118,0.873756694720734)(119,0.875477463712758)(120,0.872146118721461)(121,0.872146118721461)(122,0.872146118721461)(123,0.87556904400607)(124,0.87556904400607)(125,0.875094625283876)(126,0.875094625283876)(127,0.87642153146323)(128,0.87642153146323)(129,0.87642153146323)(130,0.878122634367903)(131,0.877937831690675)(132,0.875379939209726)(133,0.878048780487805)(134,0.878048780487805)(135,0.879756468797565)(136,0.879756468797565)(137,0.878903274942879)(138,0.879271070615034)(139,0.878234398782344)(140,0.878234398782344)(141,0.877862595419847)(142,0.878718535469108)(143,0.878718535469108)(144,0.879877582249426)(145,0.880733944954128)(146,0.880733944954128)(147,0.882758620689655)(148,0.883435582822086)(149,0.886153846153846)(150,0.886153846153846)(151,0.886153846153846)(152,0.885119506553585)(153,0.885119506553585)(154,0.885119506553585)(155,0.885119506553585)(156,0.88597842835131)(157,0.88597842835131)(158,0.88597842835131)(159,0.886153846153846)(160,0.885296381832179)(161,0.884437596302003)(162,0.883577486507324)(163,0.88135593220339)(164,0.881538461538461)(165,0.884437596302003)(166,0.886153846153846)(167,0.885472713297463)(168,0.882758620689655)(169,0.884113584036838)(170,0.885472713297463)(171,0.885472713297463)(172,0.884615384615385)(173,0.884615384615385)(174,0.884615384615385)(175,0.884615384615385)(176,0.884615384615385)(177,0.884615384615385)(178,0.884437596302003)(179,0.884437596302003)(180,0.883256528417819)(181,0.883256528417819)(182,0.883256528417819)(183,0.882398155265181)(184,0.878764478764479)(185,0.878764478764479)(186,0.880677444187837)(187,0.882716049382716)(188,0.881853281853282)(189,0.881853281853282)(190,0.881853281853282)(191,0.881853281853282)(192,0.881853281853282)(193,0.880989180834621)(194,0.881853281853282)(195,0.882716049382716)(196,0.882716049382716)(197,0.883577486507324)(198,0.886153846153846)(199,0.885296381832179)(200,0.88597842835131)(201,0.88597842835131)(202,0.885802469135802)(203,0.884942084942085)(204,0.884080370942813)(205,0.883397683397683)(206,0.883397683397683)(207,0.883397683397683)(208,0.884259259259259)(209,0.884437596302003)(210,0.884437596302003)(211,0.884437596302003)(212,0.884437596302003)(213,0.884437596302003)(214,0.883577486507324)(215,0.885296381832179)(216,0.885296381832179)(217,0.885296381832179)(218,0.88597842835131)(219,0.886836027713626)(220,0.887519260400616)(221,0.887519260400616)(222,0.884942084942085)(223,0.886661526599846)(224,0.88597842835131)(225,0.88597842835131)(226,0.886153846153846)(227,0.885296381832179)(228,0.885296381832179)(229,0.886153846153846)(230,0.886153846153846)(231,0.886153846153846)(232,0.885296381832179)(233,0.883577486507324)(234,0.883577486507324)(235,0.883577486507324)(236,0.883397683397683)(237,0.883397683397683)(238,0.883397683397683)(239,0.883397683397683)(240,0.883397683397683)(241,0.883397683397683)(242,0.883397683397683)(243,0.883397683397683)(244,0.883397683397683)(245,0.883397683397683)(246,0.88597842835131)(247,0.88597842835131)(248,0.88597842835131)(249,0.886836027713626)(250,0.88597842835131)(251,0.88597842835131)(252,0.887692307692308)(253,0.887692307692308)(254,0.887692307692308)(255,0.887692307692308)(256,0.887692307692308)(257,0.887692307692308)(258,0.887692307692308)(259,0.886836027713626)(260,0.886836027713626)(261,0.88597842835131)(262,0.885802469135802)(263,0.890084550345888)(264,0.889400921658986)(265,0.889400921658986)(266,0.88871834228703)(267,0.889400921658986)(268,0.889400921658986)(269,0.889400921658986)(270,0.891271056661562)(271,0.891271056661562)(272,0.891104294478527)(273,0.891104294478527)(274,0.889058913542463)(275,0.889058913542463)(276,0.889058913542463)(277,0.889058913542463)(278,0.890421455938697)(279,0.89025326170376)(280,0.926793557833089)(281,0.922503725782414)(282,0.928094885100074)(283,0.939882697947214)(284,0.939882697947214)(285,0.948957584471603)(286,0.952722063037249)(287,0.955523672883788)(288,0.959942775393419)(289,0.96011396011396)(290,0.961702127659574)(291,0.964059196617336)(292,0.964739069111424)(293,0.966101694915254)(294,0.964639321074965)(295,0.965223562810504)(296,0.965223562810504)(297,0.965272856130404)(298,0.965272856130404)(299,0.964438122332859)(300,0.964438122332859)(301,0.963649322879544)(302,0.961428571428571)(303,0.965957446808511)(304,0.965957446808511)(305,0.965957446808511)(306,0.966595593461265)(307,0.968794326241135)(308,0.970859985785359)(309,0.970128022759602)(310,0.970128022759602)(311,0.970128022759602)(312,0.970128022759602)(313,0.970818505338078)(314,0.970859985785359)(315,0.970859985785359)(316,0.970128022759602)(317,0.970128022759602)(318,0.971590909090909)(319,0.972320794889993)(320,0.973049645390071)(321,0.973049645390071)(322,0.971590909090909)(323,0.971671388101983)(324,0.972320794889993)(325,0.970128022759602)(326,0.970128022759602)(327,0.971590909090909)(328,0.971590909090909)(329,0.971590909090909)(330,0.970859985785359)(331,0.970859985785359)(332,0.970942593905032)(333,0.970942593905032)(334,0.969568294409059)(335,0.970983722576079)(336,0.970983722576079)(337,0.970983722576079)(338,0.972438162544169)(339,0.971711456859972)(340,0.971711456859972)(341,0.971711456859972)(342,0.972438162544169)(343,0.973888496824276)(344,0.971711456859972)(345,0.973163841807909)(346,0.972438162544169)(347,0.972438162544169)(348,0.973125884016973)(349,0.972438162544169)(350,0.9723991507431)(351,0.9723991507431)(352,0.9723991507431)(353,0.9723991507431)(354,0.971711456859972)(355,0.971711456859972)(356,0.971711456859972)(357,0.970338983050847)(358,0.97106563161609)(359,0.97106563161609)(360,0.970338983050847)(361,0.971671388101983)(362,0.971671388101983)(363,0.971671388101983)(364,0.970901348474095)(365,0.970942593905032)(366,0.971671388101983)(367,0.971671388101983)(368,0.971671388101983)(369,0.972360028348689)(370,0.972360028348689)(371,0.972360028348689)(372,0.972360028348689)(373,0.973087818696884)(374,0.971671388101983)(375,0.971671388101983)(376,0.9723991507431)(377,0.9723991507431)(378,0.971671388101983)(379,0.971631205673759)(380,0.970901348474095)(381,0.970901348474095)(382,0.970901348474095)(383,0.970901348474095)(384,0.970983722576079)(385,0.97029702970297)(386,0.97029702970297)(387,0.970983722576079)(388,0.970983722576079)(389,0.971671388101983)(390,0.971024734982332)(391,0.971024734982332)(392,0.971751412429378)(393,0.971751412429378)(394,0.971751412429378)(395,0.971751412429378)(396,0.971751412429378)(397,0.971751412429378)(398,0.971751412429378)(399,0.972438162544169)(400,0.972438162544169) 
};
\addplot [
color=red,
mark size=0.1pt,
only marks,
mark=*,
mark options={solid,fill=red},
forget plot
]
coordinates{
 (1,0)(2,0.0291390728476821)(3,0.100502512562814)(4,0.12039312039312)(5,0.136585365853658)(6,0.484596169858451)(7,0.641434262948207)(8,0.744740532959327)(9,0.715555555555556)(10,0.758097863542385)(11,0.75705437026841)(12,0.764250527797326)(13,0.778723404255319)(14,0.782608695652174)(15,0.782913165266106)(16,0.782365290412876)(17,0.827184466019417)(18,0.825520833333333)(19,0.826058631921824)(20,0.828722002635046)(21,0.829590488771466)(22,0.829268292682927)(23,0.867132867132867)(24,0.86895585143658)(25,0.878962536023055)(26,0.879543834640057)(27,0.880629020729092)(28,0.886642599277978)(29,0.884753042233357)(30,0.88412017167382)(31,0.88412017167382)(32,0.883321403006442)(33,0.883154121863799)(34,0.885222381635581)(35,0.888418079096045)(36,0.890304317055909)(37,0.887949260042283)(38,0.892480674631061)(39,0.893108298171589)(40,0.894366197183099)(41,0.892932120363891)(42,0.891061452513967)(43,0.887353144436766)(44,0.887353144436766)(45,0.89727463312369)(46,0.899300699300699)(47,0.89873417721519)(48,0.89873417721519)(49,0.89873417721519)(50,0.899147727272727)(51,0.899786780383795)(52,0.899786780383795)(53,0.896600566572238)(54,0.893950177935943)(55,0.893950177935943)(56,0.89410092395167)(57,0.893314366998577)(58,0.893314366998577)(59,0.892198581560284)(60,0.892198581560284)(61,0.892983699503898)(62,0.892983699503898)(63,0.8954802259887)(64,0.8954802259887)(65,0.8954802259887)(66,0.897090134847409)(67,0.898653437278526)(68,0.898161244695898)(69,0.901988636363636)(70,0.899575671852899)(71,0.901348474095103)(72,0.901988636363636)(73,0.901988636363636)(74,0.901348474095103)(75,0.901988636363636)(76,0.905848787446505)(77,0.907142857142857)(78,0.911174785100286)(79,0.90948275862069)(80,0.90948275862069)(81,0.90752688172043)(82,0.90752688172043)(83,0.906876790830945)(84,0.906876790830945)(85,0.906227630637079)(86,0.906227630637079)(87,0.907142857142857)(88,0.910511363636363)(89,0.910511363636363)(90,0.911284599006387)(91,0.911428571428571)(92,0.91375623663578)(93,0.91375623663578)(94,0.915302491103203)(95,0.915302491103203)(96,0.919914953933381)(97,0.915954415954416)(98,0.917613636363636)(99,0.922967189728959)(100,0.922198429693076)(101,0.921316165951359)(102,0.920657612580415)(103,0.918918918918919)(104,0.918918918918919)(105,0.92022792022792)(106,0.92022792022792)(107,0.920883820384889)(108,0.920883820384889)(109,0.919572953736655)(110,0.918918918918919)(111,0.920454545454545)(112,0.920996441281139)(113,0.919687277896233)(114,0.920567375886525)(115,0.920567375886525)(116,0.920679886685552)(117,0.921332388377037)(118,0.921332388377037)(119,0.921985815602837)(120,0.92264017033357)(121,0.923295454545454)(122,0.925925925925926)(123,0.927246790299572)(124,0.926585887384177)(125,0.92494639027877)(126,0.924838940586972)(127,0.929597701149425)(128,0.928057553956834)(129,0.928725701943844)(130,0.930064888248017)(131,0.930935251798561)(132,0.929394812680115)(133,0.929963898916967)(134,0.929862617498192)(135,0.930535455861071)(136,0.930535455861071)(137,0.931884057971014)(138,0.932657494569153)(139,0.933526011560693)(140,0.934296028880866)(141,0.934296028880866)(142,0.933621933621934)(143,0.933621933621934)(144,0.934971098265896)(145,0.934971098265896)(146,0.932657494569153)(147,0.932657494569153)(148,0.932657494569153)(149,0.932657494569153)(150,0.929963898916967)(151,0.929963898916967)(152,0.926618705035971)(153,0.927953890489913)(154,0.929394812680115)(155,0.928057553956834)(156,0.928057553956834)(157,0.928057553956834)(158,0.930064888248017)(159,0.930064888248017)(160,0.930735930735931)(161,0.930064888248017)(162,0.929292929292929)(163,0.929190751445086)(164,0.929190751445086)(165,0.929862617498192)(166,0.929862617498192)(167,0.929862617498192)(168,0.929862617498192)(169,0.929862617498192)(170,0.929862617498192)(171,0.930535455861071)(172,0.930535455861071)(173,0.929862617498192)(174,0.929862617498192)(175,0.929862617498192)(176,0.931407942238267)(177,0.931407942238267)(178,0.931407942238267)(179,0.931407942238267)(180,0.930735930735931)(181,0.930835734870317)(182,0.932276657060519)(183,0.932276657060519)(184,0.932276657060519)(185,0.931982633863965)(186,0.931982633863965)(187,0.931209268645909)(188,0.930434782608696)(189,0.930434782608696)(190,0.931109499637418)(191,0.931109499637418)(192,0.931109499637418)(193,0.929088277858177)(194,0.929088277858177)(195,0.931109499637418)(196,0.930434782608696)(197,0.930434782608696)(198,0.929761042722665)(199,0.929088277858177)(200,0.929862617498192)(201,0.928985507246377)(202,0.929088277858177)(203,0.928208846990573)(204,0.928208846990573)(205,0.928985507246377)(206,0.928312816799421)(207,0.926970354302241)(208,0.928208846990573)(209,0.929088277858177)(210,0.929088277858177)(211,0.929088277858177)(212,0.928312816799421)(213,0.927641099855282)(214,0.927075812274368)(215,0.926406926406926)(216,0.927075812274368)(217,0.929088277858177)(218,0.929761042722665)(219,0.929761042722665)(220,0.929862617498192)(221,0.929862617498192)(222,0.929862617498192)(223,0.929862617498192)(224,0.929862617498192)(225,0.930535455861071)(226,0.929862617498192)(227,0.929862617498192)(228,0.929862617498192)(229,0.929862617498192)(230,0.929862617498192)(231,0.932178932178932)(232,0.93294881038212)(233,0.93294881038212)(234,0.93294881038212)(235,0.93294881038212)(236,0.93294881038212)(237,0.933621933621934)(238,0.933621933621934)(239,0.934296028880866)(240,0.935647143890094)(241,0.934296028880866)(242,0.93342981186686)(243,0.93342981186686)(244,0.933526011560693)(245,0.933526011560693)(246,0.932178932178932)(247,0.932851985559567)(248,0.932851985559567)(249,0.932851985559567)(250,0.932851985559567)(251,0.933621933621934)(252,0.931407942238267)(253,0.931506849315068)(254,0.932080924855491)(255,0.932080924855491)(256,0.933526011560693)(257,0.934201012292118)(258,0.934201012292118)(259,0.932851985559567)(260,0.932851985559567)(261,0.932851985559567)(262,0.932851985559567)(263,0.932851985559567)(264,0.932851985559567)(265,0.932851985559567)(266,0.932851985559567)(267,0.932851985559567)(268,0.932851985559567)(269,0.933526011560693)(270,0.935740072202166)(271,0.935740072202166)(272,0.935740072202166)(273,0.935740072202166)(274,0.937274693583273)(275,0.937274693583273)(276,0.937274693583273)(277,0.937274693583273)(278,0.9378612716763)(279,0.938628158844765)(280,0.938628158844765)(281,0.938628158844765)(282,0.9378612716763)(283,0.938539407086045)(284,0.937771345875543)(285,0.937093275488069)(286,0.9378612716763)(287,0.937002172338885)(288,0.937002172338885)(289,0.937002172338885)(290,0.937002172338885)(291,0.937002172338885)(292,0.937002172338885)(293,0.936139332365747)(294,0.936910804931109)(295,0.936231884057971)(296,0.936231884057971)(297,0.936231884057971)(298,0.936910804931109)(299,0.936910804931109)(300,0.939218523878437)(301,0.93768115942029)(302,0.93768115942029)(303,0.93768115942029)(304,0.93768115942029)(305,0.938361131254532)(306,0.938361131254532)(307,0.938361131254532)(308,0.938361131254532)(309,0.938361131254532)(310,0.940665701881331)(311,0.939898624185373)(312,0.939898624185373)(313,0.938361131254532)(314,0.937590711175617)(315,0.937590711175617)(316,0.937590711175617)(317,0.937590711175617)(318,0.937590711175617)(319,0.937590711175617)(320,0.93681917211329)(321,0.936139332365747)(322,0.93681917211329)(323,0.937590711175617)(324,0.93768115942029)(325,0.937002172338885)(326,0.937002172338885)(327,0.937002172338885)(328,0.93768115942029)(329,0.93768115942029)(330,0.938450398262129)(331,0.93768115942029)(332,0.938361131254532)(333,0.939042089985486)(334,0.939042089985486)(335,0.939042089985486)(336,0.939811457577955)(337,0.939042089985486)(338,0.939042089985486)(339,0.93681917211329)(340,0.9375)(341,0.936727272727273)(342,0.936727272727273)(343,0.936727272727273)(344,0.935953420669578)(345,0.935953420669578)(346,0.935178441369264)(347,0.935178441369264)(348,0.935178441369264)(349,0.935178441369264)(350,0.9375)(351,0.9375)(352,0.938271604938271)(353,0.938271604938271)(354,0.938271604938271)(355,0.938271604938271)(356,0.939724037763253)(357,0.939724037763253)(358,0.938953488372093)(359,0.939724037763253)(360,0.939724037763253)(361,0.939724037763253)(362,0.939724037763253)(363,0.939724037763253)(364,0.939724037763253)(365,0.940406976744186)(366,0.93768115942029)(367,0.938628158844765)(368,0.938628158844765)(369,0.937950937950938)(370,0.937950937950938)(371,0.941346850108617)(372,0.941346850108617)(373,0.941346850108617)(374,0.940665701881331)(375,0.940665701881331)(376,0.940665701881331)(377,0.940665701881331)(378,0.940665701881331)(379,0.940665701881331)(380,0.939898624185373)(381,0.939130434782609)(382,0.938450398262129)(383,0.93768115942029)(384,0.936910804931109)(385,0.936910804931109)(386,0.937590711175617)(387,0.934497816593886)(388,0.934497816593886)(389,0.935366739288308)(390,0.936139332365747)(391,0.936139332365747)(392,0.936139332365747)(393,0.936910804931109)(394,0.935366739288308)(395,0.935366739288308)(396,0.935366739288308)(397,0.935366739288308)(398,0.935553946415641)(399,0.935553946415641)(400,0.935553946415641) 
};
\addplot [
color=red,
mark size=0.1pt,
only marks,
mark=*,
mark options={solid,fill=red},
forget plot
]
coordinates{
 (1,0)(2,0)(3,0)(4,0.122292993630573)(5,0.136363636363636)(6,0.146453089244851)(7,0.140229885057471)(8,0.146118721461187)(9,0.25813221406086)(10,0.257383966244726)(11,0.258064516129032)(12,0.254736842105263)(13,0.246516613076099)(14,0.2497308934338)(15,0.25)(16,0.249723756906077)(17,0.239733629300777)(18,0.245175936435868)(19,0.246885617214043)(20,0.248587570621469)(21,0.241183162684869)(22,0.230149597238205)(23,0.225058004640371)(24,0.227114716106605)(25,0.225882352941176)(26,0.225882352941176)(27,0.226148409893993)(28,0.226148409893993)(29,0.230046948356807)(30,0.224056603773585)(31,0.224343675417661)(32,0.224096385542169)(33,0.226233453670277)(34,0.228571428571428)(35,0.229940119760479)(36,0.229665071770335)(37,0.229116945107399)(38,0.233333333333333)(39,0.233333333333333)(40,0.233333333333333)(41,0.233890214797136)(42,0.235995232419547)(43,0.248520710059172)(44,0.238095238095238)(45,0.246153846153846)(46,0.245862884160756)(47,0.239520958083832)(48,0.237980769230769)(49,0.237980769230769)(50,0.237980769230769)(51,0.237980769230769)(52,0.243436754176611)(53,0.247619047619048)(54,0.247619047619048)(55,0.248506571087216)(56,0.248506571087216)(57,0.248506571087216)(58,0.254761904761905)(59,0.254761904761905)(60,0.254761904761905)(61,0.254761904761905)(62,0.252983293556086)(63,0.253892215568862)(64,0.253892215568862)(65,0.254501800720288)(66,0.252403846153846)(67,0.252403846153846)(68,0.252707581227437)(69,0.254807692307692)(70,0.250602409638554)(71,0.251511487303507)(72,0.249394673123487)(73,0.245145631067961)(74,0.245145631067961)(75,0.241758241758242)(76,0.241758241758242)(77,0.233128834355828)(78,0.23500611995104)(79,0.243605359317905)(80,0.247873633049818)(81,0.246041412911084)(82,0.246041412911084)(83,0.25181598062954)(84,0.251511487303507)(85,0.252121212121212)(86,0.25)(87,0.253929866989117)(88,0.256038647342995)(89,0.258142340168878)(90,0.253929866989117)(91,0.252427184466019)(92,0.256348246674728)(93,0.256348246674728)(94,0.262650602409639)(95,0.256348246674728)(96,0.256348246674728)(97,0.256348246674728)(98,0.256348246674728)(99,0.256348246674728)(100,0.256348246674728)(101,0.256348246674728)(102,0.275534441805226)(103,0.275534441805226)(104,0.273053892215569)(105,0.266826923076923)(106,0.28978622327791)(107,0.302107728337236)(108,0.296470588235294)(109,0.302107728337236)(110,0.298472385428907)(111,0.296470588235294)(112,0.297169811320755)(113,0.299175500588928)(114,0.298823529411765)(115,0.318604651162791)(116,0.316647264260768)(117,0.309133489461358)(118,0.306791569086651)(119,0.30715123094959)(120,0.311111111111111)(121,0.310385064177363)(122,0.310385064177363)(123,0.311990686845169)(124,0.312354312354312)(125,0.312354312354312)(126,0.316279069767442)(127,0.326767091541136)(128,0.338319907940161)(129,0.337931034482759)(130,0.337931034482759)(131,0.336405529953917)(132,0.336405529953917)(133,0.346330275229358)(134,0.342528735632184)(135,0.342528735632184)(136,0.351197263397947)(137,0.344036697247706)(138,0.342135476463835)(139,0.340229885057471)(140,0.340229885057471)(141,0.338319907940161)(142,0.338319907940161)(143,0.340229885057471)(144,0.347826086956522)(145,0.342135476463835)(146,0.347826086956522)(147,0.345933562428408)(148,0.349714285714286)(149,0.349714285714286)(150,0.346330275229358)(151,0.350114416475972)(152,0.353881278538813)(153,0.3557582668187)(154,0.353881278538813)(155,0.3557582668187)(156,0.3557582668187)(157,0.3557582668187)(158,0.354285714285714)(159,0.354285714285714)(160,0.354285714285714)(161,0.353881278538813)(162,0.353881278538813)(163,0.350114416475972)(164,0.352)(165,0.352)(166,0.352)(167,0.352)(168,0.350114416475972)(169,0.353881278538813)(170,0.354948805460751)(171,0.353075170842825)(172,0.352402745995423)(173,0.352402745995423)(174,0.348623853211009)(175,0.34902411021814)(176,0.350114416475972)(177,0.350114416475972)(178,0.34902411021814)(179,0.347126436781609)(180,0.347126436781609)(181,0.347126436781609)(182,0.348224513172967)(183,0.348224513172967)(184,0.347826086956522)(185,0.349714285714286)(186,0.347826086956522)(187,0.347826086956522)(188,0.347826086956522)(189,0.347826086956522)(190,0.347826086956522)(191,0.347826086956522)(192,0.345933562428408)(193,0.348224513172967)(194,0.346330275229358)(195,0.346727898966705)(196,0.346727898966705)(197,0.346727898966705)(198,0.346727898966705)(199,0.346727898966705)(200,0.346727898966705)(201,0.34443168771527)(202,0.34443168771527)(203,0.346727898966705)(204,0.346727898966705)(205,0.346727898966705)(206,0.344827586206896)(207,0.346727898966705)(208,0.346727898966705)(209,0.34331797235023)(210,0.34331797235023)(211,0.345622119815668)(212,0.343713956170703)(213,0.345622119815668)(214,0.345622119815668)(215,0.343713956170703)(216,0.345224395857307)(217,0.346727898966705)(218,0.348623853211009)(219,0.347126436781609)(220,0.350917431192661)(221,0.350917431192661)(222,0.350515463917526)(223,0.350917431192661)(224,0.348623853211009)(225,0.34902411021814)(226,0.34902411021814)(227,0.34902411021814)(228,0.352402745995423)(229,0.352)(230,0.352)(231,0.352)(232,0.352)(233,0.352)(234,0.352)(235,0.352)(236,0.352)(237,0.350114416475972)(238,0.350114416475972)(239,0.350114416475972)(240,0.350114416475972)(241,0.348623853211009)(242,0.348623853211009)(243,0.350515463917526)(244,0.352)(245,0.352)(246,0.353881278538813)(247,0.353881278538813)(248,0.353881278538813)(249,0.353881278538813)(250,0.353881278538813)(251,0.353477765108324)(252,0.355353075170843)(253,0.355353075170843)(254,0.353477765108324)(255,0.353075170842825)(256,0.353075170842825)(257,0.352673492605233)(258,0.352673492605233)(259,0.354545454545454)(260,0.354545454545454)(261,0.354545454545454)(262,0.354143019296254)(263,0.354545454545454)(264,0.354545454545454)(265,0.354545454545454)(266,0.352673492605233)(267,0.352673492605233)(268,0.354143019296254)(269,0.354948805460751)(270,0.358683314415437)(271,0.356818181818182)(272,0.356818181818182)(273,0.354948805460751)(274,0.353075170842825)(275,0.353075170842825)(276,0.354948805460751)(277,0.356818181818182)(278,0.356818181818182)(279,0.356818181818182)(280,0.356818181818182)(281,0.354948805460751)(282,0.354948805460751)(283,0.354948805460751)(284,0.354948805460751)(285,0.356818181818182)(286,0.354545454545454)(287,0.358683314415437)(288,0.358683314415437)(289,0.360544217687075)(290,0.35827664399093)(291,0.360544217687075)(292,0.360544217687075)(293,0.359090909090909)(294,0.360544217687075)(295,0.358683314415437)(296,0.360544217687075)(297,0.358683314415437)(298,0.358683314415437)(299,0.358683314415437)(300,0.358683314415437)(301,0.358683314415437)(302,0.358683314415437)(303,0.358683314415437)(304,0.358683314415437)(305,0.35827664399093)(306,0.357870894677237)(307,0.35972850678733)(308,0.35972850678733)(309,0.358683314415437)(310,0.35827664399093)(311,0.358683314415437)(312,0.358683314415437)(313,0.358683314415437)(314,0.358683314415437)(315,0.359090909090909)(316,0.359090909090909)(317,0.359090909090909)(318,0.359090909090909)(319,0.359090909090909)(320,0.359090909090909)(321,0.359090909090909)(322,0.359090909090909)(323,0.360953461975028)(324,0.360953461975028)(325,0.360953461975028)(326,0.359090909090909)(327,0.359090909090909)(328,0.359090909090909)(329,0.359090909090909)(330,0.359090909090909)(331,0.359090909090909)(332,0.359499431171786)(333,0.359499431171786)(334,0.359090909090909)(335,0.360953461975028)(336,0.360953461975028)(337,0.360953461975028)(338,0.360953461975028)(339,0.359090909090909)(340,0.359090909090909)(341,0.360953461975028)(342,0.360544217687075)(343,0.358683314415437)(344,0.358683314415437)(345,0.360544217687075)(346,0.360544217687075)(347,0.360544217687075)(348,0.360544217687075)(349,0.360544217687075)(350,0.360544217687075)(351,0.360544217687075)(352,0.360544217687075)(353,0.362400906002265)(354,0.362400906002265)(355,0.362400906002265)(356,0.362400906002265)(357,0.362400906002265)(358,0.362400906002265)(359,0.362400906002265)(360,0.362400906002265)(361,0.362400906002265)(362,0.364253393665158)(363,0.362400906002265)(364,0.364253393665158)(365,0.364253393665158)(366,0.363841807909604)(367,0.363841807909604)(368,0.363431151241535)(369,0.361581920903955)(370,0.363431151241535)(371,0.361581920903955)(372,0.361581920903955)(373,0.361581920903955)(374,0.361581920903955)(375,0.361581920903955)(376,0.361581920903955)(377,0.361581920903955)(378,0.361581920903955)(379,0.361581920903955)(380,0.363431151241535)(381,0.361581920903955)(382,0.361990950226244)(383,0.361990950226244)(384,0.361990950226244)(385,0.361990950226244)(386,0.361990950226244)(387,0.361581920903955)(388,0.361581920903955)(389,0.363431151241535)(390,0.363841807909604)(391,0.363841807909604)(392,0.363841807909604)(393,0.363841807909604)(394,0.363841807909604)(395,0.363841807909604)(396,0.363841807909604)(397,0.363841807909604)(398,0.363841807909604)(399,0.363841807909604)(400,0.363841807909604) 
};
\addplot [
color=red,
mark size=0.1pt,
only marks,
mark=*,
mark options={solid,fill=red},
forget plot
]
coordinates{
 (1,0)(2,0)(3,0)(4,0)(5,0.532062391681109)(6,0.584103512014787)(7,0.621931260229132)(8,0.682124158563949)(9,0.750392464678179)(10,0.745035742652899)(11,0.760736196319018)(12,0.772727272727273)(13,0.775413711583924)(14,0.802762854950115)(15,0.803966437833715)(16,0.802741812642803)(17,0.801515151515152)(18,0.799083269671505)(19,0.798775822494262)(20,0.798775822494262)(21,0.801850424055513)(22,0.8)(23,0.800618238021638)(24,0.80123743232792)(25,0.80030959752322)(26,0.803392444101773)(27,0.800307219662058)(28,0.802469135802469)(29,0.803088803088803)(30,0.801850424055513)(31,0.801232665639445)(32,0.802155504234026)(33,0.801232665639445)(34,0.801232665639445)(35,0.803088803088803)(36,0.807153965785381)(37,0.805900621118012)(38,0.811571540265833)(39,0.813427010148321)(40,0.812792511700468)(41,0.814814814814815)(42,0.814814814814815)(43,0.820031298904538)(44,0.821960784313725)(45,0.822239624119029)(46,0.821316614420063)(47,0.828729281767956)(48,0.82807570977918)(49,0.827150749802683)(50,0.826224328593997)(51,0.825572217837411)(52,0.824271079590228)(53,0.825572217837411)(54,0.827150749802683)(55,0.826498422712934)(56,0.825847123719464)(57,0.826498422712934)(58,0.826498422712934)(59,0.825572217837411)(60,0.826498422712934)(61,0.826498422712934)(62,0.826498422712934)(63,0.826498422712934)(64,0.826498422712934)(65,0.8274231678487)(66,0.829383886255924)(67,0.828729281767956)(68,0.828729281767956)(69,0.829652996845426)(70,0.831496062992126)(71,0.832415420928403)(72,0.832415420928403)(73,0.833070866141732)(74,0.832415420928403)(75,0.832415420928403)(76,0.832415420928403)(77,0.832678711704635)(78,0.834509803921568)(79,0.834509803921568)(80,0.834509803921568)(81,0.837245696400626)(82,0.837647058823529)(83,0.839467501957713)(84,0.838154808444097)(85,0.838810641627543)(86,0.838810641627543)(87,0.838810641627543)(88,0.838810641627543)(89,0.838810641627543)(90,0.838810641627543)(91,0.838810641627543)(92,0.838810641627543)(93,0.839467501957713)(94,0.839467501957713)(95,0.839467501957713)(96,0.84078431372549)(97,0.84012539184953)(98,0.84012539184953)(99,0.84012539184953)(100,0.84012539184953)(101,0.84078431372549)(102,0.841692789968652)(103,0.841692789968652)(104,0.841692789968652)(105,0.841692789968652)(106,0.841692789968652)(107,0.841692789968652)(108,0.841692789968652)(109,0.841692789968652)(110,0.841692789968652)(111,0.841692789968652)(112,0.841692789968652)(113,0.842352941176471)(114,0.842352941176471)(115,0.840534171249018)(116,0.841444270015699)(117,0.841444270015699)(118,0.841444270015699)(119,0.841444270015699)(120,0.842105263157895)(121,0.841444270015699)(122,0.842352941176471)(123,0.842352941176471)(124,0.842352941176471)(125,0.842352941176471)(126,0.841033672670321)(127,0.841692789968652)(128,0.841033672670321)(129,0.838407494145199)(130,0.838407494145199)(131,0.8375)(132,0.8375)(133,0.836846213895394)(134,0.836193447737909)(135,0.835541699142634)(136,0.836193447737909)(137,0.836193447737909)(138,0.836193447737909)(139,0.836193447737909)(140,0.836193447737909)(141,0.836193447737909)(142,0.835284933645589)(143,0.835284933645589)(144,0.837245696400626)(145,0.837245696400626)(146,0.837245696400626)(147,0.837901331245106)(148,0.837901331245106)(149,0.838557993730407)(150,0.838557993730407)(151,0.83921568627451)(152,0.83921568627451)(153,0.83921568627451)(154,0.83921568627451)(155,0.84012539184953)(156,0.84012539184953)(157,0.84078431372549)(158,0.839874411302983)(159,0.839874411302983)(160,0.839874411302983)(161,0.841194968553459)(162,0.841194968553459)(163,0.841444270015699)(164,0.842105263157895)(165,0.842105263157895)(166,0.842105263157895)(167,0.841444270015699)(168,0.841444270015699)(169,0.841444270015699)(170,0.842105263157895)(171,0.842105263157895)(172,0.842105263157895)(173,0.841444270015699)(174,0.84078431372549)(175,0.84078431372549)(176,0.84078431372549)(177,0.84078431372549)(178,0.84078431372549)(179,0.84078431372549)(180,0.84078431372549)(181,0.84078431372549)(182,0.84078431372549)(183,0.84078431372549)(184,0.84078431372549)(185,0.842105263157895)(186,0.842105263157895)(187,0.841444270015699)(188,0.841444270015699)(189,0.841444270015699)(190,0.841444270015699)(191,0.843014128728414)(192,0.843014128728414)(193,0.843014128728414)(194,0.843014128728414)(195,0.843676355066771)(196,0.843676355066771)(197,0.843014128728414)(198,0.842352941176471)(199,0.843676355066771)(200,0.843676355066771)(201,0.843676355066771)(202,0.843014128728414)(203,0.843014128728414)(204,0.843676355066771)(205,0.843676355066771)(206,0.843676355066771)(207,0.843676355066771)(208,0.843676355066771)(209,0.843676355066771)(210,0.843676355066771)(211,0.843676355066771)(212,0.843676355066771)(213,0.843676355066771)(214,0.843676355066771)(215,0.843676355066771)(216,0.843676355066771)(217,0.844339622641509)(218,0.844339622641509)(219,0.844339622641509)(220,0.845003933910307)(221,0.845003933910307)(222,0.845669291338583)(223,0.845669291338583)(224,0.845669291338583)(225,0.845669291338583)(226,0.846577498033045)(227,0.846577498033045)(228,0.846577498033045)(229,0.846577498033045)(230,0.846577498033045)(231,0.845669291338583)(232,0.845003933910307)(233,0.845669291338583)(234,0.845669291338583)(235,0.845669291338583)(236,0.845669291338583)(237,0.845669291338583)(238,0.846335697399527)(239,0.846335697399527)(240,0.846335697399527)(241,0.845003933910307)(242,0.844339622641509)(243,0.843676355066771)(244,0.843676355066771)(245,0.84458398744113)(246,0.84458398744113)(247,0.84458398744113)(248,0.84458398744113)(249,0.84458398744113)(250,0.84458398744113)(251,0.845247446975648)(252,0.845247446975648)(253,0.845247446975648)(254,0.845247446975648)(255,0.845247446975648)(256,0.84458398744113)(257,0.845247446975648)(258,0.84458398744113)(259,0.84458398744113)(260,0.84458398744113)(261,0.84458398744113)(262,0.84458398744113)(263,0.84458398744113)(264,0.84458398744113)(265,0.84458398744113)(266,0.84458398744113)(267,0.84458398744113)(268,0.84458398744113)(269,0.84458398744113)(270,0.84458398744113)(271,0.84458398744113)(272,0.84458398744113)(273,0.84458398744113)(274,0.84458398744113)(275,0.84458398744113)(276,0.84458398744113)(277,0.84458398744113)(278,0.84458398744113)(279,0.84458398744113)(280,0.84458398744113)(281,0.843676355066771)(282,0.84458398744113)(283,0.84458398744113)(284,0.84458398744113)(285,0.84458398744113)(286,0.84458398744113)(287,0.84458398744113)(288,0.84458398744113)(289,0.84458398744113)(290,0.84458398744113)(291,0.843921568627451)(292,0.843921568627451)(293,0.84458398744113)(294,0.845247446975648)(295,0.845247446975648)(296,0.845247446975648)(297,0.844339622641509)(298,0.844339622641509)(299,0.844339622641509)(300,0.844339622641509)(301,0.844339622641509)(302,0.844339622641509)(303,0.843676355066771)(304,0.844339622641509)(305,0.844339622641509)(306,0.843014128728414)(307,0.843014128728414)(308,0.843014128728414)(309,0.843014128728414)(310,0.843014128728414)(311,0.844827586206896)(312,0.845732184808144)(313,0.845732184808144)(314,0.845732184808144)(315,0.845732184808144)(316,0.847537138389367)(317,0.846635367762128)(318,0.847537138389367)(319,0.847537138389367)(320,0.847537138389367)(321,0.847537138389367)(322,0.847537138389367)(323,0.846635367762128)(324,0.846635367762128)(325,0.846635367762128)(326,0.846635367762128)(327,0.846635367762128)(328,0.846635367762128)(329,0.846635367762128)(330,0.846635367762128)(331,0.846635367762128)(332,0.847537138389367)(333,0.847537138389367)(334,0.847537138389367)(335,0.848200312989045)(336,0.848200312989045)(337,0.848200312989045)(338,0.847537138389367)(339,0.847537138389367)(340,0.847537138389367)(341,0.847537138389367)(342,0.847537138389367)(343,0.847537138389367)(344,0.847537138389367)(345,0.847537138389367)(346,0.847537138389367)(347,0.847537138389367)(348,0.847537138389367)(349,0.847537138389367)(350,0.847537138389367)(351,0.847537138389367)(352,0.847537138389367)(353,0.847537138389367)(354,0.848200312989045)(355,0.848200312989045)(356,0.848200312989045)(357,0.849529780564263)(358,0.849529780564263)(359,0.849529780564263)(360,0.849529780564263)(361,0.849529780564263)(362,0.850196078431372)(363,0.850196078431372)(364,0.850196078431372)(365,0.850196078431372)(366,0.850196078431372)(367,0.850196078431372)(368,0.850196078431372)(369,0.850196078431372)(370,0.850196078431372)(371,0.850863422291994)(372,0.850196078431372)(373,0.849529780564263)(374,0.848627450980392)(375,0.848627450980392)(376,0.848627450980392)(377,0.848627450980392)(378,0.848627450980392)(379,0.848627450980392)(380,0.848627450980392)(381,0.848627450980392)(382,0.848627450980392)(383,0.848627450980392)(384,0.848627450980392)(385,0.848627450980392)(386,0.848627450980392)(387,0.847723704866562)(388,0.847723704866562)(389,0.849529780564263)(390,0.848627450980392)(391,0.848627450980392)(392,0.848627450980392)(393,0.848627450980392)(394,0.848627450980392)(395,0.849529780564263)(396,0.849529780564263)(397,0.849529780564263)(398,0.849529780564263)(399,0.849529780564263)(400,0.850430696945967) 
};
\addplot [
color=red,
mark size=0.1pt,
only marks,
mark=*,
mark options={solid,fill=red},
forget plot
]
coordinates{
 (1,0)(2,0)(3,0)(4,0.608695652173913)(5,0.649289099526066)(6,0.706081081081081)(7,0.695890410958904)(8,0.712842712842713)(9,0.751698113207547)(10,0.748872180451128)(11,0.754545454545454)(12,0.753601213040182)(13,0.780715396578538)(14,0.793235972328978)(15,0.800933125972006)(16,0.801247077162899)(17,0.798761609907121)(18,0.806451612903226)(19,0.805832693783576)(20,0.810185185185185)(21,0.81588999236058)(22,0.814024390243902)(23,0.815668202764977)(24,0.815325670498084)(25,0.816793893129771)(26,0.82)(27,0.81578947368421)(28,0.81832298136646)(29,0.820872274143302)(30,0.822794691647151)(31,0.82307092751364)(32,0.82051282051282)(33,0.825617283950617)(34,0.826893353941267)(35,0.827265685515104)(36,0.827265685515104)(37,0.827265685515104)(38,0.825986078886311)(39,0.826625386996904)(40,0.824074074074074)(41,0.824074074074074)(42,0.822892498066512)(43,0.824980724749422)(44,0.826893353941267)(45,0.825986078886311)(46,0.823255813953488)(47,0.821981424148607)(48,0.822618125484121)(49,0.822981366459627)(50,0.824902723735408)(51,0.824261275272162)(52,0.825174825174825)(53,0.825174825174825)(54,0.824261275272162)(55,0.823620823620824)(56,0.823620823620824)(57,0.824261275272162)(58,0.823346303501945)(59,0.82398753894081)(60,0.822706065318818)(61,0.82398753894081)(62,0.824629773967264)(63,0.823712948517941)(64,0.824629773967264)(65,0.825273010920437)(66,0.82591725214676)(67,0.827856025039124)(68,0.82591725214676)(69,0.826833073322933)(70,0.82591725214676)(71,0.827208756841282)(72,0.837025316455696)(73,0.841686555290374)(74,0.842356687898089)(75,0.842356687898089)(76,0.841686555290374)(77,0.841686555290374)(78,0.841686555290374)(79,0.841686555290374)(80,0.841686555290374)(81,0.842607313195548)(82,0.84352660841938)(83,0.84352660841938)(84,0.84352660841938)(85,0.845541401273885)(86,0.845541401273885)(87,0.844868735083532)(88,0.844197138314785)(89,0.844197138314785)(90,0.844197138314785)(91,0.844868735083532)(92,0.844868735083532)(93,0.844197138314785)(94,0.844197138314785)(95,0.844197138314785)(96,0.844868735083532)(97,0.844197138314785)(98,0.84352660841938)(99,0.842607313195548)(100,0.84352660841938)(101,0.842607313195548)(102,0.842607313195548)(103,0.842607313195548)(104,0.842607313195548)(105,0.841938046068308)(106,0.841269841269841)(107,0.841269841269841)(108,0.840602696272799)(109,0.840602696272799)(110,0.840602696272799)(111,0.840602696272799)(112,0.842438638163104)(113,0.841772151898734)(114,0.843774781919112)(115,0.842438638163104)(116,0.84310618066561)(117,0.84310618066561)(118,0.844514601420679)(119,0.845849802371541)(120,0.845849802371541)(121,0.845849802371541)(122,0.845360824742268)(123,0.846031746031746)(124,0.846031746031746)(125,0.846031746031746)(126,0.845115170770453)(127,0.845115170770453)(128,0.845115170770453)(129,0.845115170770453)(130,0.846031746031746)(131,0.846031746031746)(132,0.845115170770453)(133,0.845115170770453)(134,0.845115170770453)(135,0.845115170770453)(136,0.845115170770453)(137,0.844444444444444)(138,0.844444444444444)(139,0.844444444444444)(140,0.843774781919112)(141,0.845605700712589)(142,0.845849802371541)(143,0.845849802371541)(144,0.845849802371541)(145,0.845849802371541)(146,0.845849802371541)(147,0.846761453396524)(148,0.846761453396524)(149,0.846761453396524)(150,0.846761453396524)(151,0.847430830039526)(152,0.846761453396524)(153,0.846761453396524)(154,0.846761453396524)(155,0.846761453396524)(156,0.846761453396524)(157,0.846761453396524)(158,0.846761453396524)(159,0.847671665351223)(160,0.848341232227488)(161,0.848341232227488)(162,0.848341232227488)(163,0.848341232227488)(164,0.84901185770751)(165,0.84901185770751)(166,0.849250197316495)(167,0.849250197316495)(168,0.848580441640378)(169,0.849250197316495)(170,0.850157728706624)(171,0.850157728706624)(172,0.850157728706624)(173,0.850157728706624)(174,0.850157728706624)(175,0.850157728706624)(176,0.850157728706624)(177,0.849250197316495)(178,0.849250197316495)(179,0.849250197316495)(180,0.848580441640378)(181,0.849250197316495)(182,0.848580441640378)(183,0.848580441640378)(184,0.848580441640378)(185,0.848580441640378)(186,0.848580441640378)(187,0.848580441640378)(188,0.847671665351223)(189,0.848580441640378)(190,0.848580441640378)(191,0.848580441640378)(192,0.848580441640378)(193,0.848580441640378)(194,0.848580441640378)(195,0.848580441640378)(196,0.849250197316495)(197,0.849250197316495)(198,0.849921011058452)(199,0.849921011058452)(200,0.849921011058452)(201,0.849921011058452)(202,0.85126582278481)(203,0.85126582278481)(204,0.85193982581156)(205,0.85193982581156)(206,0.85193982581156)(207,0.85193982581156)(208,0.85193982581156)(209,0.85193982581156)(210,0.85193982581156)(211,0.85193982581156)(212,0.85193982581156)(213,0.85193982581156)(214,0.85126582278481)(215,0.85126582278481)(216,0.85126582278481)(217,0.85126582278481)(218,0.852173913043478)(219,0.852173913043478)(220,0.852173913043478)(221,0.852173913043478)(222,0.852173913043478)(223,0.852173913043478)(224,0.852848101265823)(225,0.85193982581156)(226,0.85193982581156)(227,0.85193982581156)(228,0.85193982581156)(229,0.85193982581156)(230,0.85193982581156)(231,0.852848101265823)(232,0.85193982581156)(233,0.85193982581156)(234,0.85193982581156)(235,0.85193982581156)(236,0.852614896988906)(237,0.852614896988906)(238,0.852614896988906)(239,0.852614896988906)(240,0.852614896988906)(241,0.852614896988906)(242,0.852614896988906)(243,0.853291038858049)(244,0.853291038858049)(245,0.854199683042789)(246,0.854199683042789)(247,0.854199683042789)(248,0.854199683042789)(249,0.856916996047431)(250,0.856916996047431)(251,0.856916996047431)(252,0.85781990521327)(253,0.85781990521327)(254,0.858498023715415)(255,0.858498023715415)(256,0.859399684044234)(257,0.859399684044234)(258,0.859399684044234)(259,0.858721389108129)(260,0.858721389108129)(261,0.858721389108129)(262,0.858044164037855)(263,0.857368006304176)(264,0.857368006304176)(265,0.857368006304176)(266,0.857368006304176)(267,0.857368006304176)(268,0.858044164037855)(269,0.858044164037855)(270,0.858044164037855)(271,0.857368006304176)(272,0.857368006304176)(273,0.857368006304176)(274,0.856692913385827)(275,0.856692913385827)(276,0.856692913385827)(277,0.856692913385827)(278,0.856692913385827)(279,0.856692913385827)(280,0.857368006304176)(281,0.857368006304176)(282,0.856692913385827)(283,0.856692913385827)(284,0.856692913385827)(285,0.858267716535433)(286,0.858267716535433)(287,0.858267716535433)(288,0.858267716535433)(289,0.858267716535433)(290,0.858944050433412)(291,0.858944050433412)(292,0.858944050433412)(293,0.858944050433412)(294,0.858267716535433)(295,0.858267716535433)(296,0.858267716535433)(297,0.858267716535433)(298,0.858267716535433)(299,0.858944050433412)(300,0.858944050433412)(301,0.858944050433412)(302,0.859621451104101)(303,0.859621451104101)(304,0.859621451104101)(305,0.859621451104101)(306,0.859621451104101)(307,0.859621451104101)(308,0.859621451104101)(309,0.859621451104101)(310,0.859621451104101)(311,0.859621451104101)(312,0.859621451104101)(313,0.859621451104101)(314,0.860299921073402)(315,0.859621451104101)(316,0.859621451104101)(317,0.859621451104101)(318,0.859621451104101)(319,0.860299921073402)(320,0.860299921073402)(321,0.860299921073402)(322,0.860299921073402)(323,0.860299921073402)(324,0.860299921073402)(325,0.860299921073402)(326,0.860299921073402)(327,0.860299921073402)(328,0.860299921073402)(329,0.860299921073402)(330,0.860299921073402)(331,0.860299921073402)(332,0.860299921073402)(333,0.860299921073402)(334,0.860299921073402)(335,0.860299921073402)(336,0.860299921073402)(337,0.860299921073402)(338,0.860979462875197)(339,0.860979462875197)(340,0.860979462875197)(341,0.860299921073402)(342,0.860979462875197)(343,0.860979462875197)(344,0.860979462875197)(345,0.860979462875197)(346,0.860979462875197)(347,0.860979462875197)(348,0.860979462875197)(349,0.860979462875197)(350,0.860979462875197)(351,0.860079051383399)(352,0.859399684044234)(353,0.859399684044234)(354,0.860299921073402)(355,0.860299921073402)(356,0.860299921073402)(357,0.860299921073402)(358,0.860299921073402)(359,0.860299921073402)(360,0.860299921073402)(361,0.860299921073402)(362,0.860299921073402)(363,0.860299921073402)(364,0.860979462875197)(365,0.860979462875197)(366,0.860979462875197)(367,0.860979462875197)(368,0.860979462875197)(369,0.860979462875197)(370,0.860979462875197)(371,0.860299921073402)(372,0.860299921073402)(373,0.860299921073402)(374,0.860079051383399)(375,0.859399684044234)(376,0.859399684044234)(377,0.857368006304176)(378,0.857368006304176)(379,0.857368006304176)(380,0.857368006304176)(381,0.856692913385827)(382,0.856692913385827)(383,0.856692913385827)(384,0.856692913385827)(385,0.856692913385827)(386,0.856692913385827)(387,0.856692913385827)(388,0.857368006304176)(389,0.857368006304176)(390,0.857368006304176)(391,0.856692913385827)(392,0.856692913385827)(393,0.856692913385827)(394,0.856018882769473)(395,0.856018882769473)(396,0.856018882769473)(397,0.856018882769473)(398,0.856018882769473)(399,0.856018882769473)(400,0.856018882769473) 
};
\addplot [
color=red,
mark size=0.1pt,
only marks,
mark=*,
mark options={solid,fill=red},
forget plot
]
coordinates{
 (1,0)(2,0)(3,0)(4,0.119953863898501)(5,0.119953863898501)(6,0.122497055359246)(7,0.115684093437152)(8,0.120231213872832)(9,0.120231213872832)(10,0.122931442080378)(11,0.123486682808717)(12,0.119221411192214)(13,0.235817575083426)(14,0.232704402515723)(15,0.204255319148936)(16,0.36983842010772)(17,0.669978708303761)(18,0.706908115358819)(19,0.762295081967213)(20,0.748432055749129)(21,0.745454545454545)(22,0.761379310344827)(23,0.745614035087719)(24,0.743849493487699)(25,0.749283667621776)(26,0.748740100791937)(27,0.750539180445722)(28,0.749106504646176)(29,0.762105263157895)(30,0.765570328901329)(31,0.769659011830202)(32,0.803452855245684)(33,0.805298013245033)(34,0.809619238476954)(35,0.810059563203177)(36,0.818537859007833)(37,0.819607843137255)(38,0.82313829787234)(39,0.818241903502974)(40,0.841095890410959)(41,0.842681258549931)(42,0.840799448656099)(43,0.849897189856066)(44,0.843771507226428)(45,0.844566712517194)(46,0.855944055944056)(47,0.856745479833101)(48,0.864827586206896)(49,0.869325997248968)(50,0.871056241426612)(51,0.872252747252747)(52,0.870103092783505)(53,0.879944482997918)(54,0.879166666666667)(55,0.880276816608996)(56,0.8790601243953)(57,0.879668049792531)(58,0.881379310344828)(59,0.881051175656985)(60,0.881496881496881)(61,0.880886426592798)(62,0.880442294402211)(63,0.880886426592798)(64,0.881051175656985)(65,0.873641304347826)(66,0.871584699453552)(67,0.872037914691943)(68,0.870923913043478)(69,0.870923913043478)(70,0.874576271186441)(71,0.881147540983606)(72,0.881309686221009)(73,0.882993197278911)(74,0.88048615800135)(75,0.884408602150538)(76,0.884408602150538)(77,0.885157824042982)(78,0.885157824042982)(79,0.886653252850436)(80,0.88739946380697)(81,0.890087660148348)(82,0.889636608344549)(83,0.894915254237288)(84,0.894021739130435)(85,0.892057026476578)(86,0.898785425101215)(87,0.897297297297297)(88,0.896551724137931)(89,0.895805142083897)(90,0.897018970189702)(91,0.897903989181879)(92,0.898179366149697)(93,0.901828029790115)(94,0.901828029790115)(95,0.910702113156101)(96,0.912925170068027)(97,0.910702113156101)(98,0.911080711354309)(99,0.912208504801097)(100,0.907840440165062)(101,0.907840440165062)(102,0.909215955983494)(103,0.908465244322092)(104,0.909215955983494)(105,0.904696132596685)(106,0.903181189488243)(107,0.903314917127072)(108,0.903314917127072)(109,0.904432132963989)(110,0.904432132963989)(111,0.903314917127072)(112,0.906944444444444)(113,0.906944444444444)(114,0.906944444444444)(115,0.902370990237099)(116,0.903765690376569)(117,0.903000697836706)(118,0.903000697836706)(119,0.903135888501742)(120,0.903270702853166)(121,0.902506963788301)(122,0.902506963788301)(123,0.90187891440501)(124,0.90187891440501)(125,0.898813677599442)(126,0.903589021815623)(127,0.913105413105413)(128,0.915422885572139)(129,0.915302491103203)(130,0.915422885572139)(131,0.915422885572139)(132,0.915542938254081)(133,0.919458303635068)(134,0.92022792022792)(135,0.92022792022792)(136,0.92022792022792)(137,0.92022792022792)(138,0.92022792022792)(139,0.918918918918919)(140,0.920454545454545)(141,0.921108742004264)(142,0.921332388377037)(143,0.921108742004264)(144,0.920567375886525)(145,0.921220723917672)(146,0.922749822820695)(147,0.925266903914591)(148,0.924501424501424)(149,0.928317955997161)(150,0.928317955997161)(151,0.926794598436389)(152,0.927350427350427)(153,0.931156848828957)(154,0.932671863926293)(155,0.932671863926293)(156,0.932671863926293)(157,0.932671863926293)(158,0.934751773049645)(159,0.938078291814947)(160,0.938078291814947)(161,0.938078291814947)(162,0.935046395431834)(163,0.932761087267525)(164,0.932761087267525)(165,0.932094353109364)(166,0.934472934472934)(167,0.933901918976546)(168,0.934472934472934)(169,0.933713471133286)(170,0.934472934472934)(171,0.933238636363636)(172,0.933995741660752)(173,0.935506732813607)(174,0.935506732813607)(175,0.935506732813607)(176,0.934844192634561)(177,0.934844192634561)(178,0.934751773049645)(179,0.931721194879089)(180,0.931721194879089)(181,0.935506732813607)(182,0.935506732813607)(183,0.936260623229462)(184,0.936170212765957)(185,0.937013446567587)(186,0.936260623229462)(187,0.936260623229462)(188,0.936260623229462)(189,0.935598018400566)(190,0.936079545454545)(191,0.938746438746439)(192,0.940756602426838)(193,0.940841054882395)(194,0.943100995732575)(195,0.94468085106383)(196,0.942008486562942)(197,0.942756183745583)(198,0.941092973740241)(199,0.942512420156139)(200,0.941176470588235)(201,0.942512420156139)(202,0.942675159235669)(203,0.941425546930134)(204,0.941425546930134)(205,0.942172073342736)(206,0.942172073342736)(207,0.942836979534227)(208,0.941342756183746)(209,0.940594059405941)(210,0.942090395480226)(211,0.942756183745583)(212,0.944992947813822)(213,0.946478873239436)(214,0.945736434108527)(215,0.945736434108527)(216,0.946554149085794)(217,0.945736434108527)(218,0.945736434108527)(219,0.944405348346235)(220,0.944405348346235)(221,0.944405348346235)(222,0.945888966971188)(223,0.945888966971188)(224,0.945888966971188)(225,0.945888966971188)(226,0.950035186488388)(227,0.950842696629213)(228,0.950842696629213)(229,0.951510892480674)(230,0.951510892480674)(231,0.951510892480674)(232,0.950035186488388)(233,0.950773558368495)(234,0.950773558368495)(235,0.949367088607595)(236,0.949367088607595)(237,0.94862772695285)(238,0.94862772695285)(239,0.950105411103303)(240,0.950842696629213)(241,0.950842696629213)(242,0.950773558368495)(243,0.950773558368495)(244,0.951510892480674)(245,0.952982456140351)(246,0.952982456140351)(247,0.954385964912281)(248,0.954385964912281)(249,0.951048951048951)(250,0.949057920446615)(251,0.949720670391061)(252,0.951714485654304)(253,0.951578947368421)(254,0.952247191011236)(255,0.951510892480674)(256,0.954866008462623)(257,0.957507082152974)(258,0.957507082152974)(259,0.957507082152974)(260,0.956152758132956)(261,0.954674220963173)(262,0.955287437899219)(263,0.955965909090909)(264,0.956028368794326)(265,0.955350815024805)(266,0.956152758132956)(267,0.956090651558073)(268,0.955350815024805)(269,0.954609929078014)(270,0.955414012738853)(271,0.956152758132956)(272,0.956152758132956)(273,0.95547703180212)(274,0.959887403237157)(275,0.95859649122807)(276,0.958538299367533)(277,0.958421423537702)(278,0.959097320169252)(279,0.957746478873239)(280,0.957746478873239)(281,0.957746478873239)(282,0.957746478873239)(283,0.957805907172996)(284,0.958538299367533)(285,0.958479943701618)(286,0.959830866807611)(287,0.959830866807611)(288,0.959830866807611)(289,0.959830866807611)(290,0.959830866807611)(291,0.959830866807611)(292,0.959039548022599)(293,0.960507757404795)(294,0.958303886925795)(295,0.959039548022599)(296,0.959039548022599)(297,0.959717314487632)(298,0.957627118644068)(299,0.959039548022599)(300,0.958362738179252)(301,0.958362738179252)(302,0.958362738179252)(303,0.958981612446959)(304,0.959097320169252)(305,0.958362738179252)(306,0.959097320169252)(307,0.961240310077519)(308,0.96056338028169)(309,0.959830866807611)(310,0.960507757404795)(311,0.961185603387438)(312,0.961185603387438)(313,0.961185603387438)(314,0.961185603387438)(315,0.96045197740113)(316,0.958923512747875)(317,0.959660297239915)(318,0.961864406779661)(319,0.96113074204947)(320,0.961075725406935)(321,0.959717314487632)(322,0.959039548022599)(323,0.959830866807611)(324,0.961294862772695)(325,0.961294862772695)(326,0.960618846694796)(327,0.961294862772695)(328,0.961294862772695)(329,0.961294862772695)(330,0.958421423537702)(331,0.959154929577465)(332,0.959154929577465)(333,0.959154929577465)(334,0.960507757404795)(335,0.959039548022599)(336,0.961240310077519)(337,0.961240310077519)(338,0.961240310077519)(339,0.961240310077519)(340,0.961918194640338)(341,0.961918194640338)(342,0.962649753347428)(343,0.962649753347428)(344,0.962649753347428)(345,0.962025316455696)(346,0.962649753347428)(347,0.963328631875881)(348,0.962649753347428)(349,0.961918194640338)(350,0.961240310077519)(351,0.96056338028169)(352,0.96056338028169)(353,0.96056338028169)(354,0.96056338028169)(355,0.96056338028169)(356,0.96056338028169)(357,0.96056338028169)(358,0.959887403237157)(359,0.962649753347428)(360,0.962649753347428)(361,0.961971830985915)(362,0.962702322308233)(363,0.963431786216596)(364,0.962702322308233)(365,0.961918194640338)(366,0.961918194640338)(367,0.961918194640338)(368,0.961918194640338)(369,0.961918194640338)(370,0.961240310077519)(371,0.961240310077519)(372,0.961240310077519)(373,0.961240310077519)(374,0.961240310077519)(375,0.96056338028169)(376,0.96056338028169)(377,0.959830866807611)(378,0.959830866807611)(379,0.958479943701618)(380,0.959830866807611)(381,0.959830866807611)(382,0.959830866807611)(383,0.961294862772695)(384,0.961294862772695)(385,0.961294862772695)(386,0.961971830985915)(387,0.962702322308233)(388,0.961240310077519)(389,0.961240310077519)(390,0.961240310077519)(391,0.961240310077519)(392,0.961240310077519)(393,0.960507757404795)(394,0.961240310077519)(395,0.961971830985915)(396,0.964109781843772)(397,0.964109781843772)(398,0.963380281690141)(399,0.962702322308233)(400,0.962702322308233) 
};
\addplot [
color=red,
mark size=0.1pt,
only marks,
mark=*,
mark options={solid,fill=red},
forget plot
]
coordinates{
 (1,0)(2,0.108247422680412)(3,0.640963855421687)(4,0.62063227953411)(5,0.603432700993677)(6,0.650421743205248)(7,0.653631284916201)(8,0.747422680412371)(9,0.744463373083475)(10,0.732940185341196)(11,0.745621351125938)(12,0.745424292845258)(13,0.741235392320534)(14,0.758223684210526)(15,0.745)(16,0.754045307443366)(17,0.756493506493506)(18,0.77504105090312)(19,0.789297658862876)(20,0.794338051623647)(21,0.798340248962656)(22,0.799336650082919)(23,0.79734219269103)(24,0.796680497925311)(25,0.797678275290216)(26,0.791979949874687)(27,0.792988313856427)(28,0.792642140468227)(29,0.8)(30,0.803970223325062)(31,0.802980132450331)(32,0.799670239076669)(33,0.798685291700904)(34,0.800982800982801)(35,0.800656275635767)(36,0.801639344262295)(37,0.802296964725185)(38,0.811074918566775)(39,0.806214227309894)(40,0.80358598207009)(41,0.800986842105263)(42,0.802306425041186)(43,0.802306425041186)(44,0.802306425041186)(45,0.804276315789473)(46,0.804276315789473)(47,0.805258833196384)(48,0.805258833196384)(49,0.80623973727422)(50,0.810147299509002)(51,0.810147299509002)(52,0.810147299509002)(53,0.821717990275527)(54,0.822672064777328)(55,0.823624595469256)(56,0.822672064777328)(57,0.824575586095392)(58,0.826472962066182)(59,0.830249396621078)(60,0.829307568438003)(61,0.831189710610933)(62,0.829032258064516)(63,0.829032258064516)(64,0.829032258064516)(65,0.830371567043618)(66,0.831315577078289)(67,0.831315577078289)(68,0.831315577078289)(69,0.830917874396135)(70,0.830917874396135)(71,0.832128514056225)(72,0.830249396621078)(73,0.830249396621078)(74,0.830249396621078)(75,0.829307568438003)(76,0.831189710610933)(77,0.831189710610933)(78,0.829975825946817)(79,0.831858407079646)(80,0.831858407079646)(81,0.830522088353414)(82,0.82864038616251)(83,0.826752618855761)(84,0.825422365245374)(85,0.827140549273021)(86,0.825910931174089)(87,0.825910931174089)(88,0.825910931174089)(89,0.826860841423948)(90,0.82875605815832)(91,0.825910931174089)(92,0.826860841423948)(93,0.82875605815832)(94,0.82875605815832)(95,0.827809215844786)(96,0.827809215844786)(97,0.827809215844786)(98,0.827809215844786)(99,0.828478964401294)(100,0.82914979757085)(101,0.828200972447326)(102,0.826298701298701)(103,0.826016260162601)(104,0.826969943135662)(105,0.826969943135662)(106,0.826969943135662)(107,0.826969943135662)(108,0.828872668288727)(109,0.827922077922078)(110,0.826688364524003)(111,0.825732899022801)(112,0.826688364524003)(113,0.82859463850528)(114,0.82859463850528)(115,0.830494728304947)(116,0.830769230769231)(117,0.830769230769231)(118,0.829821717990275)(119,0.829821717990275)(120,0.831715210355987)(121,0.833602584814216)(122,0.833602584814216)(123,0.833602584814216)(124,0.833602584814216)(125,0.833333333333333)(126,0.834276475343573)(127,0.834276475343573)(128,0.834276475343573)(129,0.834276475343573)(130,0.833602584814216)(131,0.833602584814216)(132,0.834276475343573)(133,0.834276475343573)(134,0.834276475343573)(135,0.834276475343573)(136,0.834276475343573)(137,0.834276475343573)(138,0.834276475343573)(139,0.833333333333333)(140,0.834276475343573)(141,0.835218093699515)(142,0.838033843674456)(143,0.838969404186795)(144,0.841767068273092)(145,0.842948717948718)(146,0.842273819055244)(147,0.842273819055244)(148,0.842273819055244)(149,0.842948717948718)(150,0.842948717948718)(151,0.842696629213483)(152,0.840836012861736)(153,0.840836012861736)(154,0.840836012861736)(155,0.840160642570281)(156,0.840836012861736)(157,0.839903459372486)(158,0.839903459372486)(159,0.840836012861736)(160,0.841767068273092)(161,0.841767068273092)(162,0.842696629213483)(163,0.842696629213483)(164,0.842696629213483)(165,0.842696629213483)(166,0.842696629213483)(167,0.842696629213483)(168,0.842696629213483)(169,0.842696629213483)(170,0.843373493975903)(171,0.843373493975903)(172,0.843373493975903)(173,0.843373493975903)(174,0.844051446945338)(175,0.843121480289622)(176,0.843121480289622)(177,0.843121480289622)(178,0.843121480289622)(179,0.843121480289622)(180,0.842190016103059)(181,0.842443729903537)(182,0.842443729903537)(183,0.842443729903537)(184,0.842443729903537)(185,0.842443729903537)(186,0.841512469831054)(187,0.841512469831054)(188,0.842443729903537)(189,0.842443729903537)(190,0.842443729903537)(191,0.842443729903537)(192,0.841512469831054)(193,0.841512469831054)(194,0.841512469831054)(195,0.842443729903537)(196,0.842443729903537)(197,0.842443729903537)(198,0.842443729903537)(199,0.842443729903537)(200,0.843373493975903)(201,0.843373493975903)(202,0.845228548516439)(203,0.845228548516439)(204,0.845228548516439)(205,0.846153846153846)(206,0.846153846153846)(207,0.845228548516439)(208,0.845228548516439)(209,0.845228548516439)(210,0.845228548516439)(211,0.845906902086677)(212,0.846153846153846)(213,0.846153846153846)(214,0.845228548516439)(215,0.845228548516439)(216,0.845228548516439)(217,0.845228548516439)(218,0.845228548516439)(219,0.845228548516439)(220,0.845228548516439)(221,0.84430176565008)(222,0.84430176565008)(223,0.84430176565008)(224,0.842443729903537)(225,0.842443729903537)(226,0.842443729903537)(227,0.843373493975903)(228,0.843373493975903)(229,0.843373493975903)(230,0.845228548516439)(231,0.842443729903537)(232,0.843373493975903)(233,0.842443729903537)(234,0.842443729903537)(235,0.842443729903537)(236,0.845228548516439)(237,0.845228548516439)(238,0.845228548516439)(239,0.845228548516439)(240,0.84430176565008)(241,0.845228548516439)(242,0.845228548516439)(243,0.845228548516439)(244,0.845228548516439)(245,0.845228548516439)(246,0.845228548516439)(247,0.845228548516439)(248,0.84430176565008)(249,0.846153846153846)(250,0.846153846153846)(251,0.846153846153846)(252,0.846153846153846)(253,0.84430176565008)(254,0.84430176565008)(255,0.84430176565008)(256,0.846153846153846)(257,0.846153846153846)(258,0.846153846153846)(259,0.845228548516439)(260,0.845228548516439)(261,0.845228548516439)(262,0.845228548516439)(263,0.845228548516439)(264,0.845228548516439)(265,0.845228548516439)(266,0.845228548516439)(267,0.84430176565008)(268,0.84430176565008)(269,0.845228548516439)(270,0.845228548516439)(271,0.846153846153846)(272,0.847077662129704)(273,0.846153846153846)(274,0.846153846153846)(275,0.846153846153846)(276,0.846153846153846)(277,0.846153846153846)(278,0.846153846153846)(279,0.846153846153846)(280,0.846153846153846)(281,0.846153846153846)(282,0.846153846153846)(283,0.846153846153846)(284,0.848)(285,0.848)(286,0.848)(287,0.848)(288,0.848)(289,0.848)(290,0.848)(291,0.848)(292,0.848)(293,0.848)(294,0.848)(295,0.848)(296,0.848)(297,0.848)(298,0.848)(299,0.848)(300,0.848)(301,0.848)(302,0.848)(303,0.848)(304,0.848)(305,0.848)(306,0.847077662129704)(307,0.847077662129704)(308,0.847077662129704)(309,0.847077662129704)(310,0.847077662129704)(311,0.847077662129704)(312,0.848920863309352)(313,0.848)(314,0.849840255591054)(315,0.849840255591054)(316,0.849840255591054)(317,0.849840255591054)(318,0.849840255591054)(319,0.849840255591054)(320,0.849840255591054)(321,0.849840255591054)(322,0.848920863309352)(323,0.848920863309352)(324,0.848920863309352)(325,0.848920863309352)(326,0.848920863309352)(327,0.848920863309352)(328,0.848920863309352)(329,0.848920863309352)(330,0.849840255591054)(331,0.849840255591054)(332,0.849840255591054)(333,0.849840255591054)(334,0.849840255591054)(335,0.848920863309352)(336,0.848920863309352)(337,0.848920863309352)(338,0.848920863309352)(339,0.848920863309352)(340,0.848920863309352)(341,0.848920863309352)(342,0.848920863309352)(343,0.848920863309352)(344,0.848920863309352)(345,0.848920863309352)(346,0.848920863309352)(347,0.848920863309352)(348,0.848920863309352)(349,0.849840255591054)(350,0.849840255591054)(351,0.849840255591054)(352,0.849840255591054)(353,0.850758180367119)(354,0.850758180367119)(355,0.850758180367119)(356,0.849840255591054)(357,0.849840255591054)(358,0.849840255591054)(359,0.848920863309352)(360,0.849840255591054)(361,0.849840255591054)(362,0.849840255591054)(363,0.849840255591054)(364,0.849840255591054)(365,0.849840255591054)(366,0.849840255591054)(367,0.849840255591054)(368,0.849840255591054)(369,0.849840255591054)(370,0.849840255591054)(371,0.850758180367119)(372,0.850758180367119)(373,0.850758180367119)(374,0.850758180367119)(375,0.850758180367119)(376,0.850758180367119)(377,0.850758180367119)(378,0.850758180367119)(379,0.850758180367119)(380,0.850758180367119)(381,0.850758180367119)(382,0.850758180367119)(383,0.850758180367119)(384,0.849840255591054)(385,0.850758180367119)(386,0.849840255591054)(387,0.849840255591054)(388,0.850758180367119)(389,0.850758180367119)(390,0.850758180367119)(391,0.850758180367119)(392,0.850758180367119)(393,0.850758180367119)(394,0.850758180367119)(395,0.849840255591054)(396,0.849840255591054)(397,0.849840255591054)(398,0.849840255591054)(399,0.849840255591054)(400,0.849840255591054) 
};

\end{axis}
\end{tikzpicture}%

%% This file was created by matlab2tikz v0.2.3.
% Copyright (c) 2008--2012, Nico Schlömer <nico.schloemer@gmail.com>
% All rights reserved.
% 
% 
% 
\begin{tikzpicture}

\begin{axis}[%
tick label style={font=\tiny},
label style={font=\tiny},
label shift={-4pt},
xlabel shift={-6pt},
legend style={font=\tiny},
view={0}{90},
width=\figurewidth,
height=\figureheight,
scale only axis,
xmin=0, xmax=400,
xlabel={Samples},
ymin=0, ymax=1,
ylabel={$F_1$-score},
axis lines*=left,
legend cell align=left,
legend style={at={(1.03,0)},anchor=south east,fill=none,draw=none,align=left,row sep=-0.2em},
clip=false]

\addplot [
color=blue,
mark size=0.1pt,
only marks,
mark=*,
mark options={solid,fill=blue},
forget plot
]
coordinates{
 (4,0)(5,0)(6,0)(8,0.574817518248175)(8,0)(8,0)(8,0.66517524235645)(8,0)(8,0.713068181818182)(8,0.711732711732712)(8,0.448579823702253)(9,0.635904830569574)(9,0.0684410646387832)(10,0.661804922515953)(10,0)(10,0)(10,0.239256678281068)(10,0)(11,0.669590643274854)(11,0.408474576271186)(11,0)(11,0.75975975975976)(11,0.051006711409396)(11,0.562264150943396)(11,0.186440677966102)(11,0.654810024252223)(12,0.738013698630137)(12,0.0609271523178808)(12,0.701598579040853)(12,0.247787610619469)(12,0.720774385703648)(12,0.733238231098431)(12,0.705364995602462)(12,0.0770186335403726)(13,0.743589743589744)(13,0.771103896103896)(13,0.2)(13,0.710526315789474)(13,0.529605263157895)(13,0.239224137931034)(13,0.769230769230769)(13,0.419672131147541)(13,0.461951656222023)(14,0.779794790844515)(14,0.780982073265783)(14,0.711488250652741)(14,0.740313272877164)(14,0.769454545454546)(14,0.757936507936508)(14,0.783402489626556)(14,0.593266606005459)(14,0.727524204702628)(15,0.702660406885759)(15,0.689655172413793)(15,0.786092214663643)(15,0.718023255813953)(15,0.259770114942529)(15,0.502472799208704)(15,0.175879396984925)(15,0.598337950138504)(15,0.604838709677419)(15,0.755192878338279)(15,0.703632887189292)(15,0.743570903747245)(16,0.768781911013858)(16,0.175246440306681)(16,0.733001658374793)(16,0.680124223602484)(16,0.736332179930796)(16,0.764341085271318)(16,0.809748427672956)(16,0.776729559748428)(16,0.730229120473023)(16,0.738842975206612)(16,0.794001578531965)(16,0.774803149606299)(16,0.727710843373494)(16,0.771908017402113)(16,0.734306569343066)(16,0.205464480874317)(16,0.760853568800589)(16,0.807613868116927)(16,0.782071097372488)(16,0.775018811136192)(16,0.790490341753343)(17,0.793650793650794)(17,0.788124156545209)(17,0.756329113924051)(17,0.683501683501683)(17,0.757009345794393)(17,0.795023696682464)(17,0.784283513097072)(17,0.733218588640275)(17,0.728176011355571)(17,0.744292237442922)(17,0.547101449275362)(17,0.789197299324831)(17,0.72)(17,0.542199488491048)(17,0.740617180984153)(17,0.760233918128655)(17,0.73216885007278)(17,0.253869969040248)(17,0.767191383595692)(17,0.767511177347243)(17,0.757328990228013)(17,0.771761480466073)(17,0.681159420289855)(18,0.750418760469012)(18,0.780487804878049)(18,0.792998477929985)(18,0.803738317757009)(18,0.728971962616822)(18,0.749615975422427)(18,0.768860353130016)(18,0.765156794425087)(18,0.722222222222222)(18,0.765110941086458)(18,0.789313904068002)(18,0.793002915451895)(18,0.719685039370079)(18,0.80336648814078)(18,0.793076317859953)(18,0.748378728923476)(18,0.734306569343066)(18,0.788856304985337)(18,0.621818181818182)(18,0.777364110201042)(18,0.7408)(18,0.792880258899676)(18,0.779470729751403)(18,0.785773366418528)(18,0.178098676293622)(18,0.749504296100463)(18,0.730207532667179)(19,0.787675820495646)(19,0.7936)(19,0.759581881533101)(19,0.776627218934911)(19,0.702111024237686)(19,0.782186234817814)(19,0.707587382779199)(19,0.791243158717748)(19,0.791569086651054)(19,0.76797385620915)(19,0.750952018278751)(19,0.772727272727273)(19,0.686976389946687)(19,0.803100775193798)(19,0.732463295269168)(19,0.750830564784053)(19,0.775426621160409)(19,0.752537080405933)(19,0.78419452887538)(19,0.780715396578538)(19,0.807412309728656)(19,0.758142758142758)(19,0.794165316045381)(19,0.730623306233062)(19,0.699174793698424)(19,0.788535031847134)(19,0.771263418662263)(19,0.670111972437554)(19,0.80026455026455)(20,0.684210526315789)(20,0.787066246056782)(20,0.751027115858669)(20,0.815503875968992)(20,0.549053356282272)(20,0.744763382467029)(20,0.692633361558001)(20,0.765751211631664)(20,0.807994289793005)(20,0.773694390715667)(20,0.83861459100958)(20,0.758327427356485)(20,0.780377358490566)(20,0.759197324414716)(20,0.793763288447909)(20,0.788906009244992)(20,0.806853582554517)(20,0.788755020080321)(20,0.8003108003108)(20,0.741998693664272)(20,0.770452740270056)(20,0.764341085271318)(20,0.788732394366197)(20,0.775477707006369)(20,0.789905362776025)(20,0.779295470884256)(20,0.808510638297872)(20,0.783477681545636)(20,0.774907749077491)(20,0.788447653429603)(20,0.723221936589546)(20,0.783805668016194)(20,0.796238244514106)(20,0.799398948159279)(20,0.808446455505279)(20,0.779661016949152)(20,0.705787781350482)(20,0.804878048780488)(20,0.699149265274555)(21,0.740279937791602)(21,0.786657400972898)(21,0.760451227604512)(21,0.796208530805687)(21,0.800951625693894)(21,0.822641509433962)(21,0.794788273615635)(21,0.781990521327014)(21,0.582037996545768)(21,0.786500366837858)(21,0.804289544235925)(21,0.798473282442748)(21,0.751381215469613)(21,0.720465890183028)(21,0.764389869531849)(21,0.777611940298507)(21,0.794117647058823)(21,0.74320652173913)(21,0.743089430894309)(21,0.771123872026251)(21,0.774876499647142)(21,0.764957264957265)(21,0.75243081525804)(21,0.820948336871904)(21,0.739565943238731)(21,0.731871838111299)(21,0.796636085626911)(21,0.769811320754717)(21,0.756843800322061)(21,0.763786008230453)(21,0.781710914454277)(21,0.734600760456274)(21,0.773841961852861)(21,0.753488372093023)(21,0.78440366972477)(21,0.784674329501916)(21,0.80610298792117)(21,0.795304475421863)(21,0.777777777777778)(21,0.821081830790569)(21,0.775862068965517)(21,0.743251726302574)(21,0.804091266719119)(21,0.777869529314616)(21,0.786499215070643)(22,0.782951854775059)(22,0.78841512469831)(22,0.765690376569038)(22,0.773869346733668)(22,0.77184841453983)(22,0.809638554216867)(22,0.739197530864197)(22,0.806880375293198)(22,0.784135240572171)(22,0.758508914100486)(22,0.769817073170732)(22,0.735135135135135)(22,0.775729646697389)(22,0.782022471910112)(22,0.794813119755911)(22,0.788688138256088)(22,0.772795216741405)(22,0.753726046841732)(22,0.783861671469741)(22,0.747206703910614)(22,0.808952837729816)(22,0.685906040268456)(22,0.689545091779729)(22,0.796223446105429)(22,0.807069219440353)(22,0.814594192107223)(22,0.851632047477745)(22,0.757267441860465)(22,0.777244688142563)(22,0.769102990033222)(22,0.782040816326531)(22,0.752585521081941)(22,0.758465011286682)(22,0.733124018838304)(22,0.777777777777778)(22,0.771451483560545)(22,0.7875)(22,0.788751714677641)(22,0.736842105263158)(22,0.809375)(22,0.796474358974359)(22,0.804651162790698)(23,0.780130293159609)(23,0.787592717464599)(23,0.793375394321767)(23,0.778286461739699)(23,0.802091112770724)(23,0.76356050069541)(23,0.784126984126984)(23,0.782113821138211)(23,0.778588807785888)(23,0.773928361714621)(23,0.767741935483871)(23,0.783225806451613)(23,0.794277929155313)(23,0.825305535585909)(23,0.747149564050972)(23,0.801635991820041)(23,0.713073005093379)(23,0.76523151909017)(23,0.810086682427108)(23,0.798418972332016)(23,0.788624135280553)(23,0.775143090760425)(23,0.792079207920792)(23,0.754032258064516)(23,0.786786786786787)(23,0.786450662739322)(23,0.789173789173789)(23,0.782540919719408)(23,0.730737365368683)(23,0.76734693877551)(23,0.794912559618442)(23,0.782185107863605)(23,0.79612992398065)(23,0.783539094650206)(23,0.809893992932862)(23,0.803849000740192)(23,0.77643504531722)(23,0.797797010228167)(23,0.768982229402262)(23,0.807613868116927)(23,0.780923994038748)(23,0.767511177347243)(23,0.788169464428457)(23,0.784155214227971)(23,0.793183578621224)(23,0.779299847792998)(23,0.783473603672532)(23,0.786299329858526)(23,0.697358490566037)(24,0.820113314447592)(24,0.797743755036261)(24,0.807017543859649)(24,0.783821478382148)(24,0.770247933884297)(24,0.818855218855219)(24,0.792884371029225)(24,0.794235388310648)(24,0.735271013354281)(24,0.808811959087333)(24,0.786259541984733)(24,0.782894736842105)(24,0.791935483870968)(24,0.8095952023988)(24,0.786838340486409)(24,0.786723163841808)(24,0.739357729648992)(24,0.801582069874753)(24,0.752921535893155)(24,0.811881188118812)(24,0.791277258566978)(24,0.774721189591078)(24,0.7900466562986)(24,0.773426573426573)(24,0.782060266292922)(24,0.673469387755102)(24,0.785299806576402)(24,0.770114942528735)(24,0.79967689822294)(24,0.752163650668765)(24,0.812297734627832)(24,0.732327243844321)(24,0.764984227129337)(24,0.775267538644471)(24,0.743676222596965)(24,0.720853858784893)(24,0.784602784602785)(24,0.797987059669303)(24,0.804027885360186)(24,0.799693016116654)(24,0.722967640094712)(24,0.783715012722646)(24,0.763582966226138)(24,0.779282868525896)(24,0.804560260586319)(24,0.803676953381484)(24,0.828660436137072)(24,0.814336917562724)(24,0.814186584425597)(24,0.790513833992095)(24,0.74223341729639)(25,0.788598574821853)(25,0.797583081570997)(25,0.810611643330877)(25,0.83298392732355)(25,0.768588137009189)(25,0.855051244509517)(25,0.783038869257951)(25,0.814610613370089)(25,0.772696245733788)(25,0.808314087759815)(25,0.76120619396903)(25,0.804838709677419)(25,0.806177606177606)(25,0.781684382665576)(25,0.806615776081425)(25,0.810567010309278)(25,0.776119402985074)(25,0.824943651389932)(25,0.820948336871904)(25,0.75540765391015)(25,0.769750168804861)(25,0.826923076923077)(25,0.815173527037934)(25,0.777531411677753)(25,0.796484110885733)(25,0.820775420629115)(25,0.771491550330639)(25,0.817859673990078)(25,0.804676753782668)(25,0.785627283800243)(25,0.787365177195686)(25,0.745595489781536)(25,0.789514263685428)(25,0.77030162412993)(25,0.794392523364486)(25,0.771900826446281)(25,0.762288477034649)(25,0.805486284289277)(25,0.756756756756757)(25,0.7552)(26,0.803328290468986)(26,0.814411964649898)(26,0.789223454833597)(26,0.76219512195122)(26,0.773662551440329)(26,0.782208588957055)(26,0.790996784565916)(26,0.735918068763716)(26,0.790159189580318)(26,0.784140969162995)(26,0.783707865168539)(26,0.829530201342282)(26,0.805389221556886)(26,0.813271604938271)(26,0.800944138473643)(26,0.795330739299611)(26,0.843324250681199)(26,0.797129810828441)(26,0.811932555123216)(26,0.808510638297872)(26,0.855869242199108)(26,0.78284547311096)(26,0.845375722543352)(26,0.734411085450346)(26,0.806774441878368)(26,0.807854137447405)(26,0.781874039938556)(26,0.825118163403106)(26,0.769475357710652)(26,0.776923076923077)(26,0.744884038199181)(26,0.805617147080562)(26,0.796435915010281)(26,0.804819277108434)(26,0.818001323626737)(26,0.810419681620839)(26,0.728862973760933)(26,0.819776714513556)(26,0.788659793814433)(26,0.804364770070148)(26,0.782040816326531)(27,0.828298887122416)(27,0.765243902439024)(27,0.811046511627907)(27,0.805071315372424)(27,0.825235678027556)(27,0.798392498325519)(27,0.806774441878368)(27,0.782477341389728)(27,0.800599700149925)(27,0.779009608277901)(27,0.812720848056537)(27,0.808429118773946)(27,0.822304832713754)(27,0.771341463414634)(27,0.771034482758621)(27,0.803515379786566)(27,0.81704628949302)(27,0.790627362055933)(27,0.770491803278688)(27,0.803680981595092)(27,0.81687898089172)(27,0.805337519623234)(27,0.719219219219219)(27,0.794348508634223)(27,0.808842652795839)(27,0.754268745360059)(27,0.804616652926628)(27,0.855172413793103)(27,0.796825396825397)(27,0.780337941628264)(27,0.801247077162899)(27,0.773780975219824)(27,0.722925457102672)(27,0.797886393659181)(27,0.819698173153296)(27,0.815100154083205)(27,0.819649122807017)(27,0.780447390223695)(27,0.769867549668874)(28,0.806746310611384)(28,0.815457413249211)(28,0.802130898021309)(28,0.817719680464779)(28,0.861189801699716)(28,0.823618470855413)(28,0.775181305398872)(28,0.832599118942731)(28,0.792792792792793)(28,0.77246963562753)(28,0.785150078988941)(28,0.755714285714286)(28,0.787249814677539)(28,0.7703125)(28,0.748397435897436)(28,0.79631053036126)(28,0.816521048451152)(28,0.774593338497289)(28,0.766265060240964)(28,0.803041825095057)(28,0.810263335584065)(28,0.849018280297901)(28,0.789514263685428)(28,0.732732732732733)(28,0.814229249011858)(28,0.775849056603773)(28,0.790103750997606)(28,0.855507868383405)(28,0.797564687975647)(28,0.762247838616715)(28,0.790106951871658)(28,0.80327868852459)(28,0.824081313526192)(28,0.818247646632875)(28,0.775)(28,0.797120418848168)(28,0.829268292682927)(28,0.774834437086093)(28,0.783558124598587)(29,0.838351822503962)(29,0.775510204081633)(29,0.756155679110405)(29,0.846208362863218)(29,0.763371150729335)(29,0.758732737611698)(29,0.817601135557133)(29,0.818652849740932)(29,0.843609022556391)(29,0.790986790986791)(29,0.792514239218877)(29,0.774287801314828)(29,0.802853437094682)(29,0.777408637873754)(29,0.753822629969419)(29,0.802427511800404)(29,0.811683320522675)(29,0.781701444622793)(29,0.801932367149758)(29,0.776886035313001)(29,0.807313642756681)(29,0.767123287671233)(29,0.823346303501945)(29,0.802244039270687)(29,0.793355481727575)(29,0.793103448275862)(29,0.802172226532195)(29,0.782080485952923)(29,0.833223467369809)(29,0.786948176583493)(29,0.809559372666169)(29,0.752655538694992)(29,0.774143302180685)(29,0.758461538461538)(29,0.813745704467354)(29,0.835862068965517)(29,0.81629392971246)(30,0.766666666666667)(30,0.822429906542056)(30,0.796958174904943)(30,0.784034212401996)(30,0.77190332326284)(30,0.796699174793698)(30,0.802045288531775)(30,0.81799591002045)(30,0.769953051643192)(30,0.803519061583578)(30,0.805483028720626)(30,0.827586206896551)(30,0.789638932496075)(30,0.793242156074014)(30,0.847364818617385)(30,0.812196812196812)(30,0.811437403400309)(30,0.806686046511628)(30,0.790143084260731)(30,0.780619111709287)(30,0.837473385379702)(30,0.811094452773613)(30,0.804453723034099)(30,0.77751756440281)(30,0.803341902313625)(30,0.815047021943574)(30,0.834890965732087)(30,0.848129126925899)(30,0.790322580645161)(31,0.871272727272727)(31,0.853884533143264)(31,0.779527559055118)(31,0.824644549763033)(31,0.780998389694042)(31,0.811849479583667)(31,0.796516231195566)(31,0.807482462977397)(31,0.773413897280967)(31,0.818802122820318)(31,0.778156996587031)(31,0.7889634601044)(31,0.826016260162601)(31,0.831043956043956)(31,0.804428044280443)(31,0.739622641509434)(31,0.791354945968412)(31,0.845490196078431)(31,0.798307475317348)(31,0.795786061588331)(31,0.782894736842105)(31,0.793625498007968)(31,0.819117647058824)(31,0.816835541699143)(31,0.799452429842573)(31,0.800313234142522)(31,0.806818181818182)(31,0.781860828772478)(31,0.808444096950743)(31,0.791258477769405)(31,0.850460666194188)(31,0.814117647058823)(31,0.840175953079179)(31,0.748349229640499)(31,0.817749603803486)(32,0.840188806473365)(32,0.778801843317972)(32,0.817232375979112)(32,0.777327935222672)(32,0.827389443651926)(32,0.804224207961007)(32,0.812792511700468)(32,0.820883534136546)(32,0.803500397772474)(32,0.807354116706635)(32,0.844349680170576)(32,0.78030303030303)(32,0.795580110497237)(32,0.822157434402332)(32,0.815950920245399)(32,0.796623177283192)(32,0.810263335584065)(32,0.826199740596628)(32,0.835761589403973)(32,0.842032011134308)(32,0.862518089725036)(32,0.8558888076079)(32,0.825688073394495)(32,0.784756527875794)(32,0.789333333333333)(32,0.782745098039216)(32,0.806646525679758)(32,0.829787234042553)(32,0.824140168721609)(32,0.813803019410496)(32,0.810810810810811)(32,0.798380566801619)(32,0.826725403817915)(32,0.808605341246291)(32,0.786677240285488)(32,0.812593703148426)(32,0.742320819112628)(33,0.86789667896679)(33,0.814701378254211)(33,0.813153961136024)(33,0.795162509448224)(33,0.836078431372549)(33,0.829717291255753)(33,0.816020025031289)(33,0.762869198312236)(33,0.822988505747126)(33,0.787535410764872)(33,0.825418060200669)(33,0.813151563753007)(33,0.806149732620321)(33,0.798129384255651)(33,0.782677165354331)(33,0.748822605965463)(33,0.79384203480589)(33,0.801955990220049)(33,0.872752420470263)(33,0.846969696969697)(33,0.791666666666667)(33,0.805152979066022)(33,0.791272727272727)(33,0.84479092841956)(33,0.844989185291997)(33,0.818688981868898)(33,0.786296900489396)(33,0.832848837209302)(33,0.851288830138797)(33,0.80672268907563)(33,0.811573747353564)(33,0.799431009957326)(33,0.844599303135888)(33,0.820659971305595)(33,0.786248131539611)(34,0.826984126984127)(34,0.811437403400309)(34,0.836363636363636)(34,0.812894183601962)(34,0.797043010752688)(34,0.786153846153846)(34,0.790880503144654)(34,0.806372549019608)(34,0.794498381877022)(34,0.803738317757009)(34,0.78939617083947)(34,0.848439821693908)(34,0.79073482428115)(34,0.822766570605187)(34,0.814968814968815)(34,0.843338213762811)(34,0.788216560509554)(34,0.839506172839506)(34,0.815868263473054)(34,0.772619984264359)(34,0.755583918315252)(34,0.827965435978005)(34,0.834249803613511)(34,0.850666666666666)(34,0.859375)(34,0.854950115118956)(34,0.786290322580645)(34,0.811188811188811)(35,0.869325997248968)(35,0.845051194539249)(35,0.825077399380805)(35,0.853623188405797)(35,0.794745484400657)(35,0.828034682080925)(35,0.854355400696864)(35,0.831114225648213)(35,0.821646341463415)(35,0.813404417364813)(35,0.794747753973739)(35,0.806833114323259)(35,0.795003673769287)(35,0.783661119515885)(35,0.845050215208034)(35,0.812982998454405)(35,0.793248945147679)(35,0.776218001651528)(35,0.814705882352941)(35,0.856531049250535)(35,0.788530465949821)(35,0.82)(36,0.813204508856683)(36,0.802588996763754)(36,0.865424430641822)(36,0.832624113475177)(36,0.774834437086093)(36,0.796352583586626)(36,0.787077826725404)(36,0.845090909090909)(36,0.793157076205288)(36,0.848751835535977)(36,0.830496936691627)(36,0.799719887955182)(36,0.810457516339869)(36,0.84761281883584)(36,0.82798833819242)(36,0.838038632986627)(36,0.815193571950329)(36,0.804265041888804)(36,0.81639085894405)(36,0.800650935720097)(36,0.851216022889843)(36,0.858585858585859)(36,0.805687203791469)(37,0.813261163734777)(37,0.827740492170022)(37,0.82859338970023)(37,0.833453496755587)(37,0.865979381443299)(37,0.847133757961783)(37,0.823445318084346)(37,0.791935483870968)(37,0.81437125748503)(37,0.826498422712934)(37,0.77502001601281)(37,0.785511363636364)(37,0.788253477588871)(37,0.816923076923077)(37,0.852819807427785)(37,0.801574803149606)(37,0.818897637795276)(37,0.813633067440174)(37,0.799344799344799)(37,0.844280442804428)(37,0.807849550286181)(37,0.869204152249135)(37,0.785087719298246)(37,0.856164383561644)(37,0.824902723735408)(37,0.870748299319728)(38,0.798807749627422)(38,0.836220472440945)(38,0.81046676096181)(38,0.850271528316524)(38,0.781778104335048)(38,0.807461692205196)(38,0.84907063197026)(38,0.87250172294969)(38,0.832797427652733)(38,0.815112540192926)(38,0.836477987421384)(38,0.814132104454685)(38,0.873744619799139)(38,0.803266518188567)(38,0.789544235924933)(38,0.82689556509299)(38,0.82258064516129)(38,0.816640986132511)(38,0.855994641661085)(38,0.831203765971755)(38,0.811244979919679)(38,0.821727019498607)(38,0.873381294964029)(38,0.826178747361013)(38,0.807663964627856)(39,0.784198975859546)(39,0.836030964109782)(39,0.793856103476152)(39,0.818181818181818)(39,0.87836853605244)(39,0.799396681749623)(39,0.831426392067124)(39,0.824345146379045)(39,0.817977528089888)(39,0.804190169218372)(39,0.828181164629763)(39,0.828458498023715)(39,0.807106598984771)(39,0.826025459688826)(39,0.868640850417616)(39,0.81145584725537)(39,0.848232848232848)(39,0.856946354883081)(39,0.847873107426099)(39,0.801948051948052)(39,0.821298322392414)(39,0.840082361015786)(39,0.854814814814815)(39,0.8)(39,0.840858623242043)(39,0.846834581347856)(39,0.802250803858521)(40,0.859320859320859)(40,0.837070254110613)(40,0.825284090909091)(40,0.834246575342466)(40,0.888070692194403)(40,0.867692307692308)(40,0.844827586206896)(40,0.817204301075269)(40,0.827785817655572)(40,0.809309309309309)(40,0.85657104736491)(40,0.833823529411765)(40,0.854395604395604)(40,0.788177339901478)(40,0.860433604336043)(40,0.7968)(40,0.847505270555165)(40,0.843355605048255)(40,0.824184566428003)(40,0.836150845253576)(41,0.830278884462151)(41,0.841302841302841)(41,0.831712789827973)(41,0.865627319970304)(41,0.848951048951049)(41,0.875798438608942)(41,0.821068938807126)(41,0.839584996009577)(41,0.837107377647918)(41,0.874820143884892)(41,0.845801526717557)(41,0.850382741823243)(41,0.805598755832037)(41,0.807888970051132)(41,0.820233463035019)(41,0.835286009648518)(41,0.835179153094462)(41,0.845090909090909)(41,0.792270531400966)(41,0.796173657100809)(41,0.808064516129032)(41,0.841328413284133)(42,0.814921090387374)(42,0.827225130890052)(42,0.828341855368882)(42,0.856145251396648)(42,0.85243553008596)(42,0.865807429871114)(42,0.825768667642752)(42,0.792642140468227)(42,0.841379310344827)(42,0.871553463349025)(42,0.847873107426099)(42,0.850208044382802)(42,0.785980611483967)(42,0.85258358662614)(42,0.851063829787234)(42,0.786938775510204)(43,0.829305135951662)(43,0.82807570977918)(43,0.826865671641791)(43,0.852631578947368)(43,0.841419261404779)(43,0.846538187009279)(43,0.835971223021583)(43,0.816232771822358)(43,0.863570391872278)(43,0.795069337442219)(43,0.817987152034261)(43,0.827265685515104)(43,0.834030683403068)(43,0.802907915993538)(43,0.845188284518828)(44,0.89556724267468)(44,0.806078147612156)(44,0.810855263157895)(44,0.811683320522675)(44,0.837177747625509)(44,0.805687203791469)(44,0.832690824980725)(44,0.836935166994106)(44,0.809266409266409)(44,0.865248226950355)(44,0.809663250366032)(44,0.852505292872265)(44,0.847192608386638)(44,0.817741935483871)(44,0.877410468319559)(44,0.860198624904507)(44,0.826018808777429)(44,0.821282401091405)(44,0.871345029239766)(44,0.793304221251819)(44,0.868628858578607)(44,0.823712948517941)(44,0.863463005339435)(44,0.810853950518755)(44,0.838268792710706)(44,0.835370237239396)(45,0.826459143968871)(45,0.829662261380323)(45,0.879151943462897)(45,0.852784134248665)(45,0.849426063470628)(45,0.836940836940837)(45,0.872727272727273)(45,0.833893745796906)(45,0.838383838383838)(45,0.828065739570164)(45,0.862629246676514)(45,0.861324041811847)(45,0.824295010845987)(45,0.852899575671853)(45,0.828464419475655)(45,0.841095890410959)(45,0.809968847352025)(45,0.803639120545868)(45,0.859353023909986)(45,0.858929812369701)(45,0.805722891566265)(45,0.813899613899614)(45,0.832440703902066)(46,0.83641975308642)(46,0.823184152604549)(46,0.812703583061889)(46,0.88412017167382)(46,0.844667143879742)(46,0.884473877851361)(46,0.871398453970485)(46,0.817610062893082)(46,0.881057268722467)(46,0.845132743362832)(46,0.885674931129476)(46,0.836245954692557)(46,0.882440476190476)(46,0.854619565217391)(46,0.826018808777429)(46,0.853420195439739)(46,0.844101123595505)(46,0.881597717546362)(46,0.839887640449438)(46,0.844036697247706)(47,0.813717848791894)(47,0.819650937297996)(47,0.833452466047176)(47,0.822924320352682)(47,0.828865979381443)(47,0.803889789303079)(47,0.845637583892617)(47,0.873758865248227)(47,0.804395604395604)(47,0.865671641791045)(47,0.818885448916409)(47,0.846258503401361)(47,0.824345146379045)(47,0.834269662921348)(47,0.871532846715328)(47,0.813920454545454)(47,0.804331013147718)(47,0.869902912621359)(47,0.857532379004772)(47,0.830100853374709)(47,0.858646616541353)(47,0.85)(48,0.816693944353519)(48,0.827150749802683)(48,0.833709556057186)(48,0.891737891737892)(48,0.863791923340178)(48,0.823343848580441)(48,0.821981424148607)(48,0.850184501845018)(48,0.841710427606902)(49,0.833727344365642)(49,0.826472962066182)(49,0.875636363636364)(49,0.877142857142857)(49,0.869125090383225)(49,0.837144837144837)(49,0.844117647058823)(49,0.802896218825422)(49,0.840875912408759)(49,0.842578710644677)(49,0.837938760268857)(49,0.866851595006935)(49,0.860927152317881)(49,0.821321321321321)(49,0.823343848580441)(49,0.832347140039448)(49,0.843167701863354)(49,0.825018615040953)(49,0.820973782771535)(49,0.824521072796935)(50,0.81897233201581)(50,0.833855799373041)(50,0.876068376068376)(50,0.831530139103555)(50,0.830383480825959)(50,0.87836853605244)(50,0.810207336523126)(50,0.843725943033102)(50,0.871903750884643)(50,0.871866295264624)(50,0.84761281883584)(50,0.857979502196193)(50,0.81767955801105)(50,0.890804597701149)(50,0.848354792560801)(50,0.849187935034803)(50,0.897378277153558)(50,0.842748091603053)(50,0.831730769230769)(51,0.823889739663093)(51,0.829599456890699)(51,0.836309523809524)(51,0.870007262164125)(51,0.823529411764706)(51,0.842889054355919)(51,0.847691247415575)(51,0.850299401197605)(51,0.828592268417214)(51,0.838910505836576)(51,0.843445692883895)(51,0.839871382636656)(51,0.837801608579088)(51,0.860606060606061)(52,0.88243831640058)(52,0.842599843382929)(52,0.8171875)(52,0.847058823529412)(52,0.828926905132193)(52,0.782903663500678)(52,0.833965125094769)(52,0.883754512635379)(52,0.864394488759971)(52,0.817404817404817)(52,0.865979381443299)(52,0.886551465063862)(52,0.883856829802776)(52,0.863670982482864)(52,0.893175074183976)(52,0.885857860732232)(53,0.874405974202308)(53,0.881281864530226)(53,0.847870182555781)(53,0.867453157529493)(53,0.83021483021483)(53,0.855623100303951)(53,0.831845238095238)(54,0.821981424148607)(54,0.834048640915594)(54,0.847873107426099)(54,0.823699421965318)(54,0.834488067744419)(54,0.884856943475227)(54,0.896296296296296)(54,0.827321565617805)(54,0.878980891719745)(54,0.856704980842912)(54,0.883040935672515)(54,0.831029185867896)(54,0.819032761310452)(54,0.828921885087153)(54,0.883417813178856)(55,0.883820384889522)(55,0.893961708394698)(55,0.832989690721649)(55,0.888571428571428)(55,0.856382978723404)(55,0.852411807055436)(55,0.802675585284281)(55,0.895565092989986)(55,0.82089552238806)(55,0.835752482811306)(55,0.834355828220859)(55,0.87252897068848)(55,0.886750555144337)(55,0.86987270155587)(55,0.885100074128984)(55,0.851332398316971)(55,0.824501032346868)(55,0.813639968279144)(56,0.853370396108408)(56,0.888264230498946)(56,0.801213960546282)(56,0.930888575458392)(56,0.841049382716049)(56,0.87218045112782)(56,0.820710973724884)(56,0.906340057636887)(56,0.868983957219251)(56,0.839846743295019)(56,0.867269984917044)(56,0.878744650499287)(56,0.892728581713463)(57,0.850837138508371)(57,0.911655530809206)(57,0.833723653395784)(57,0.890500362581581)(57,0.905385735080058)(57,0.871830985915493)(57,0.857784431137724)(57,0.859916782246879)(57,0.854519774011299)(57,0.824925816023739)(57,0.822683706070288)(57,0.872727272727273)(57,0.823064770932069)(57,0.865384615384615)(58,0.889978976874562)(58,0.89863407620417)(58,0.819411296738266)(58,0.895364238410596)(58,0.866903914590747)(58,0.828787878787879)(58,0.884531590413943)(58,0.857142857142857)(58,0.890974729241877)(58,0.844711177794448)(58,0.900926585887384)(58,0.863363363363363)(58,0.842687747035573)(58,0.887931034482758)(58,0.807324840764331)(59,0.88243831640058)(59,0.819411296738266)(59,0.893871449925261)(59,0.879884225759768)(59,0.877840909090909)(59,0.844604316546763)(59,0.858375269590223)(59,0.868745467730239)(59,0.832558139534884)(59,0.84053651266766)(60,0.853025936599424)(60,0.890718562874251)(60,0.890052356020942)(60,0.862385321100917)(60,0.903649635036496)(60,0.869502523431867)(60,0.889795918367347)(60,0.887582659808964)(60,0.856273764258555)(60,0.876673713883016)(60,0.838174273858921)(60,0.884839650145773)(61,0.87819110138585)(61,0.831831831831832)(61,0.841328413284133)(61,0.826849733028223)(61,0.845151953690304)(61,0.828025477707006)(61,0.874571624400274)(61,0.88174982911825)(61,0.865934065934066)(61,0.871687587168759)(61,0.831230283911672)(61,0.841949778434269)(61,0.883190883190883)(61,0.874815905743741)(61,0.851095993953137)(61,0.891812865497076)(61,0.846504559270517)(61,0.874163319946452)(62,0.874651810584958)(62,0.900420757363254)(62,0.817391304347826)(62,0.890198968312454)(62,0.866517524235645)(62,0.900515843773029)(62,0.857359635811836)(62,0.877521613832853)(62,0.888086642599278)(62,0.849733028222731)(62,0.8756148981026)(63,0.882749326145552)(63,0.88872936109117)(63,0.866010598031794)(63,0.878013148283418)(63,0.835266821345708)(63,0.854918612880396)(64,0.8421875)(64,0.904486251808972)(64,0.821292775665399)(64,0.87211093990755)(64,0.874909616775126)(64,0.872252747252747)(64,0.856191744340879)(64,0.879831342234715)(64,0.873475609756097)(64,0.844089920232052)(64,0.833860759493671)(64,0.878787878787879)(64,0.851499634235552)(64,0.839140103780578)(65,0.853474320241692)(65,0.889985895627644)(65,0.855380397532556)(65,0.85145482388974)(65,0.872473867595819)(65,0.856531049250535)(65,0.860916860916861)(65,0.85432473444613)(65,0.880984952120383)(65,0.861224489795918)(65,0.862248696947133)(65,0.893333333333333)(66,0.864214992927864)(66,0.885964912280702)(66,0.901989683124539)(66,0.917060013486177)(66,0.890459363957597)(66,0.871910112359551)(66,0.904795991410165)(66,0.853846153846154)(66,0.887608069164265)(66,0.864253393665158)(67,0.908689248895434)(67,0.836309523809524)(67,0.896935933147632)(67,0.865889212827988)(67,0.899563318777293)(67,0.834394904458599)(67,0.927953890489913)(67,0.874465049928673)(67,0.905444126074499)(67,0.898572501878287)(67,0.853474320241692)(67,0.87378640776699)(67,0.865051903114187)(67,0.876438986953185)(67,0.850735809390329)(68,0.889043963712491)(68,0.910212474297464)(68,0.889044943820225)(68,0.844444444444444)(68,0.883097542814594)(68,0.880056777856636)(68,0.861450107681263)(68,0.842433697347894)(69,0.880882352941176)(69,0.824271079590228)(69,0.87762490948588)(69,0.840255591054313)(69,0.900209351011863)(69,0.851159311892296)(69,0.878490566037736)(69,0.872700515084621)(69,0.880771881461061)(69,0.862491000719942)(70,0.892757660167131)(70,0.85096870342772)(70,0.829707426856714)(70,0.877944325481799)(70,0.871758934828311)(70,0.877565463552725)(70,0.887084870848708)(70,0.874115983026874)(70,0.825572217837411)(70,0.887482419127989)(70,0.880165289256198)(70,0.912542372881356)(70,0.845454545454545)(70,0.861514919663351)(70,0.891756272401434)(71,0.86997957794418)(71,0.88385269121813)(71,0.882513661202186)(71,0.833200319233839)(71,0.874576271186441)(71,0.908701854493581)(71,0.905172413793103)(71,0.853080568720379)(71,0.864457831325301)(71,0.880669923237962)(71,0.8983174835406)(71,0.901709401709402)(71,0.890520694259012)(71,0.89067055393586)(71,0.879543834640057)(71,0.913528591352859)(71,0.889526542324247)(71,0.901960784313725)(72,0.883401920438957)(72,0.891034482758621)(72,0.878448918717375)(72,0.869252873563218)(72,0.897418927862343)(72,0.880747126436781)(72,0.90096798212956)(72,0.841781874039939)(72,0.86697247706422)(72,0.888888888888889)(72,0.860706860706861)(73,0.867309117865085)(73,0.895565092989986)(73,0.87218045112782)(73,0.876847290640394)(74,0.907420494699647)(74,0.880512091038407)(74,0.86377245508982)(74,0.822327044025157)(74,0.907501820830299)(74,0.822652757078986)(74,0.891710231516057)(75,0.87098976109215)(75,0.874651810584958)(75,0.85500340367597)(75,0.907681465821001)(75,0.872058823529412)(75,0.860869565217391)(75,0.904285714285714)(75,0.87072808320951)(75,0.911871813546977)(76,0.906045156591406)(76,0.861172976985894)(76,0.8268156424581)(76,0.907026259758694)(76,0.902386117136659)(76,0.903086862885858)(76,0.9056875449964)(76,0.856313497822932)(77,0.873180873180873)(77,0.900072939460248)(77,0.882938026013772)(77,0.850430696945967)(77,0.909485879797248)(77,0.885017421602787)(77,0.848214285714285)(77,0.913841807909604)(77,0.875270367700072)(77,0.817529880478088)(77,0.889518413597734)(78,0.910522548317824)(78,0.894401133947555)(78,0.870217554388597)(78,0.903765690376569)(78,0.867820069204152)(78,0.919402985074627)(78,0.892395982783357)(78,0.901531728665208)(78,0.898761835396941)(78,0.894436519258202)(79,0.872803935347857)(79,0.895818568391212)(79,0.872570194384449)(79,0.835130970724191)(79,0.891380518570427)(79,0.912435614422369)(79,0.853550295857988)(79,0.874476987447699)(80,0.928571428571428)(80,0.82398753894081)(80,0.840873634945398)(80,0.884422110552764)(80,0.908707865168539)(80,0.833832335329341)(80,0.897360703812317)(80,0.880608365019011)(80,0.842185128983308)(80,0.875346260387812)(80,0.879882092851879)(81,0.839936608557845)(81,0.841181165203511)(81,0.855200543847723)(81,0.886750555144337)(81,0.83476898981989)(81,0.860587792012057)(82,0.908692476260044)(82,0.902787219578518)(82,0.898975109809663)(82,0.881944444444444)(82,0.94468085106383)(82,0.879501385041551)(83,0.872875594833446)(83,0.911535739561217)(83,0.898102600140548)(83,0.875952875952876)(83,0.902816901408451)(83,0.894545454545454)(84,0.904522613065326)(84,0.860759493670886)(84,0.905579399141631)(84,0.884333821376281)(84,0.895908111988514)(84,0.878529218647406)(84,0.891176470588235)(84,0.890995260663507)(84,0.916138125440451)(84,0.855842185128983)(85,0.902842035690681)(85,0.917018284106892)(85,0.823343848580441)(85,0.889048991354467)(85,0.897869213813372)(85,0.872680742162508)(85,0.859467455621302)(85,0.91844232182219)(85,0.853295535081502)(86,0.88536409516943)(86,0.868717179294824)(86,0.900634249471459)(86,0.86651411136537)(86,0.893706293706294)(86,0.914205344585091)(86,0.885732730794061)(86,0.889358703312192)(86,0.901875901875902)(87,0.908955223880597)(87,0.880706921944035)(87,0.904109589041096)(87,0.904795991410165)(87,0.869831546707504)(87,0.906705539358601)(87,0.861580013504389)(87,0.879526977087953)(87,0.877719429857464)(87,0.887303851640513)(87,0.869369369369369)(88,0.864544781643227)(88,0.906030855539972)(88,0.892465252377469)(88,0.910425844346549)(88,0.833594976452119)(88,0.850539291217257)(88,0.867362146050671)(89,0.897343862167983)(89,0.86286594761171)(89,0.913597733711048)(89,0.911516853932584)(89,0.923184357541899)(89,0.917325664989216)(89,0.891672807663965)(90,0.873452544704264)(90,0.830745341614907)(90,0.91170136396267)(90,0.860888565185725)(90,0.858849887976102)(90,0.885989010989011)(90,0.889695210449927)(91,0.857357357357357)(91,0.918996415770609)(91,0.837136113296617)(91,0.899135446685879)(91,0.904593639575972)(91,0.908303249097473)(91,0.920770877944325)(91,0.843797856049004)(91,0.884615384615385)(91,0.896854425749817)(92,0.90154711673699)(92,0.874910650464617)(92,0.921052631578947)(92,0.917510853835022)(92,0.886411149825784)(92,0.850724637681159)(92,0.903313049357674)(92,0.876506024096385)(92,0.884671532846715)(92,0.863448275862069)(92,0.911485003657644)(93,0.852107279693487)(93,0.932309839497557)(93,0.925608011444921)(93,0.858461538461538)(93,0.934903047091413)(93,0.894662921348314)(93,0.922419928825623)(93,0.868580060422961)(93,0.879019908116386)(93,0.915229885057471)(93,0.851485148514851)(94,0.853333333333333)(94,0.899178491411501)(94,0.908701854493581)(94,0.871687587168759)(94,0.865947611710324)(94,0.842268842268842)(94,0.888731396172927)(94,0.908852459016393)(94,0.849673202614379)(95,0.873261205564142)(95,0.868285504047093)(95,0.877061469265367)(95,0.8735976065819)(95,0.874906367041198)(95,0.90096798212956)(95,0.891461649782923)(95,0.901591895803184)(95,0.903314917127072)(95,0.918842625264644)(96,0.9059955588453)(96,0.842188739095955)(96,0.873408239700374)(96,0.877428998505231)(96,0.898529411764706)(96,0.837837837837838)(97,0.891191709844559)(97,0.855193328278999)(97,0.881491344873502)(97,0.895787139689579)(97,0.899213724088635)(97,0.875856816450876)(97,0.879581151832461)(97,0.915422885572139)(97,0.908339076498966)(97,0.866520787746171)(98,0.920567375886525)(98,0.901561439239647)(98,0.880875724404379)(98,0.868039664378337)(98,0.83452098178939)(98,0.899638989169675)(98,0.882768361581921)(99,0.906020558002937)(99,0.900414937759336)(99,0.870820668693009)(99,0.90547263681592)(99,0.834394904458599)(100,0.914201183431952)(100,0.89792899408284)(100,0.923185341789993)(100,0.915443745632425)(100,0.900221729490022)(100,0.867362146050671)(100,0.865109269027882)(100,0.9)(100,0.902158273381295)(100,0.876693766937669)(100,0.870535714285714)(100,0.93646408839779)(100,0.915229885057471)(100,0.903443619176232)(101,0.914446002805049)(101,0.865558912386707)(101,0.873684210526316)(101,0.907142857142857)(102,0.856259659969088)(102,0.910354412786657)(102,0.826328310864393)(102,0.870919881305638)(102,0.905848787446505)(103,0.928467153284671)(103,0.892978868438991)(103,0.894508670520231)(103,0.876258992805755)(103,0.849765258215962)(103,0.917080085046067)(103,0.904486251808972)(104,0.872292755787901)(104,0.930735930735931)(104,0.879888268156424)(104,0.920454545454545)(104,0.905551550108147)(104,0.908430232558139)(104,0.846749226006192)(104,0.876524390243902)(104,0.897869213813372)(105,0.909896602658789)(105,0.872226472838561)(105,0.928671328671329)(105,0.932778932778933)(105,0.90986515259049)(105,0.923303834808259)(105,0.911094783247612)(106,0.917303683113273)(106,0.863976083707025)(106,0.895482130815914)(107,0.900876601483479)(107,0.923076923076923)(107,0.882800608828006)(107,0.914862914862915)(107,0.89732770745429)(107,0.913653136531365)(107,0.863320463320463)(108,0.869305451829724)(108,0.904727272727273)(108,0.916833000665336)(108,0.897841726618705)(108,0.906876790830945)(109,0.909090909090909)(109,0.932559825960841)(109,0.883468834688347)(109,0.87238644556597)(109,0.853563038371182)(110,0.902857142857143)(110,0.917216117216117)(111,0.890198968312454)(111,0.888888888888889)(111,0.915593705293276)(111,0.905908096280087)(111,0.877937831690675)(111,0.908029197080292)(111,0.857361963190184)(111,0.917325664989216)(112,0.907780979827089)(112,0.874626865671642)(112,0.866415094339622)(112,0.903039073806078)(112,0.903735632183908)(112,0.928671328671329)(113,0.868965517241379)(113,0.873239436619718)(113,0.912230215827338)(113,0.914492753623188)(113,0.915714285714286)(113,0.864457831325301)(114,0.894083272461651)(114,0.910425844346549)(114,0.842948717948718)(114,0.924608819345661)(114,0.915060670949322)(114,0.896011396011396)(114,0.90150032615786)(114,0.930458970792768)(115,0.904347826086956)(115,0.915782024062279)(115,0.931801866475233)(115,0.898550724637681)(115,0.897988505747126)(115,0.945007235890014)(115,0.928208846990573)(115,0.950175438596491)(115,0.897751994198695)(116,0.919739696312364)(116,0.87816091954023)(116,0.899928005759539)(116,0.934722222222222)(116,0.926470588235294)(117,0.90436005625879)(117,0.920341394025604)(117,0.934553131597467)(117,0.916780354706685)(117,0.86404833836858)(117,0.827140549273021)(117,0.856702619414484)(118,0.928467153284671)(118,0.911095796002757)(118,0.905263157894737)(118,0.912607449856734)(118,0.869499241274659)(119,0.858895705521472)(119,0.91493924231594)(119,0.863741339491917)(119,0.91411501120239)(119,0.913475177304964)(119,0.869436201780415)(120,0.902244750181028)(120,0.853189853958493)(120,0.930265995686556)(120,0.888446215139442)(121,0.925104022191401)(121,0.877937831690675)(121,0.943754565376187)(121,0.911218169304886)(121,0.945833333333333)(122,0.920086393088553)(122,0.934285714285714)(122,0.935191637630662)(122,0.926253687315634)(122,0.915107913669065)(122,0.933988764044944)(123,0.862509391435011)(123,0.91970802919708)(123,0.900568181818182)(123,0.914163090128755)(123,0.943237561317449)(123,0.911306702775897)(124,0.908430232558139)(124,0.875562218890555)(124,0.899416909620991)(124,0.925287356321839)(124,0.920612813370474)(124,0.914033798677443)(124,0.93062368605466)(125,0.868421052631579)(125,0.925352112676056)(125,0.89142091152815)(125,0.908823529411765)(125,0.867415730337079)(125,0.936551724137931)(125,0.912023460410557)(125,0.911337209302325)(125,0.888106966924701)(126,0.905035971223021)(126,0.910931174089069)(126,0.912948061448427)(126,0.909221902017291)(126,0.895478131949592)(127,0.933797909407665)(127,0.863705972434916)(127,0.855631141345427)(127,0.936170212765957)(127,0.893246187363834)(127,0.882706766917293)(127,0.902366863905325)(127,0.916008614501077)(128,0.907231555880204)(128,0.905766526019691)(128,0.901933701657458)(128,0.92296918767507)(128,0.931082981715893)(128,0.916063675832127)(128,0.924929178470255)(129,0.899777282850779)(129,0.916119620714807)(129,0.926375982844889)(129,0.933988764044944)(129,0.860836501901141)(129,0.952515946137491)(130,0.872372372372372)(130,0.863602110022607)(130,0.931133428981348)(130,0.90407358738502)(130,0.903179190751445)(130,0.876438986953185)(130,0.831513260530421)(131,0.920842411038489)(131,0.928620928620928)(131,0.916008614501077)(132,0.929133858267716)(132,0.929482874412357)(132,0.906272022551092)(132,0.862537764350453)(132,0.960111966410077)(132,0.908321579689704)(132,0.91493924231594)(133,0.913925822253324)(133,0.907647907647907)(133,0.91340782122905)(133,0.907501820830299)(133,0.91358024691358)(133,0.942240779401531)(134,0.895238095238095)(134,0.918646508279338)(134,0.914285714285714)(134,0.932657494569153)(134,0.908290535583272)(135,0.871987951807229)(135,0.943181818181818)(135,0.919556171983356)(135,0.888888888888889)(135,0.909952606635071)(135,0.919263456090651)(135,0.912255257432922)(135,0.899575671852899)(136,0.912579957356077)(136,0.919609316303531)(136,0.926007326007326)(136,0.901428571428571)(136,0.930763740185582)(137,0.896451846488052)(137,0.905494505494505)(137,0.866965620328849)(137,0.924791086350975)(138,0.866464339908953)(138,0.919748778785764)(138,0.942079553384508)(138,0.949367088607595)(138,0.915542938254081)(138,0.893465909090909)(138,0.95615866388309)(138,0.919377652050919)(138,0.916292974588939)(139,0.922964049889949)(139,0.920567375886525)(139,0.912761355443403)(139,0.93969144460028)(139,0.917888563049853)(139,0.931868131868132)(140,0.911094783247612)(140,0.903271692745377)(140,0.934437543133195)(140,0.920883820384889)(141,0.928317955997161)(141,0.896551724137931)(141,0.911174785100286)(141,0.941678520625889)(141,0.934592043155765)(141,0.915697674418605)(141,0.899649122807017)(141,0.906077348066298)(142,0.874349442379182)(142,0.869767441860465)(142,0.913454545454545)(142,0.894814814814815)(142,0.925373134328358)(142,0.91220556745182)(143,0.923189465983906)(143,0.950213371266003)(143,0.903225806451613)(143,0.924393723252496)(143,0.894133513149022)(143,0.920698924731183)(143,0.914205344585091)(143,0.94061433447099)(143,0.913419913419913)(144,0.921397379912664)(144,0.920544022906228)(144,0.929953302201468)(144,0.920727272727273)(145,0.877519379844961)(145,0.942798070296347)(145,0.911337209302325)(145,0.921161825726141)(145,0.933995741660752)(145,0.893557422969188)(145,0.869172932330827)(145,0.929761042722665)(146,0.929577464788732)(146,0.914407988587732)(146,0.915451895043732)(146,0.87940074906367)(147,0.905141202027516)(147,0.926144756277696)(147,0.931618144888287)(147,0.882308276385725)(147,0.889041095890411)(148,0.915492957746479)(148,0.896296296296296)(148,0.877428998505231)(148,0.863298662704309)(148,0.875)(148,0.942675159235669)(148,0.869565217391304)(148,0.922048997772828)(149,0.871287128712871)(149,0.934449093444909)(149,0.912951167728238)(149,0.935418082936778)(149,0.940256045519203)(149,0.941747572815534)(150,0.911849710982659)(150,0.867749419953596)(150,0.95014245014245)(150,0.919389978213508)(150,0.86232980332829)(150,0.940689655172414)(150,0.873151750972762)(150,0.921071687183201)(150,0.937979094076655)(151,0.927598566308244)(151,0.886131386861314)(151,0.888719512195122)(152,0.941908713692946)(152,0.927789934354486)(152,0.862043251304996)(152,0.921965317919075)(152,0.931308749096168)(152,0.911032028469751)(152,0.960382513661202)(152,0.865269461077844)(152,0.88042650418888)(152,0.956276445698166)(153,0.944915254237288)(153,0.942148760330578)(153,0.925608011444921)(153,0.923413566739606)(153,0.867614061331339)(154,0.872066616199849)(154,0.944521497919556)(154,0.926694329183956)(154,0.910905203136137)(154,0.918220946915351)(154,0.914033798677443)(154,0.909992912827782)(154,0.924068767908309)(155,0.877351392024078)(155,0.919112383679313)(155,0.912398921832884)(155,0.925820256776034)(155,0.940277777777778)(155,0.921896792189679)(155,0.928675400291121)(156,0.929971988795518)(156,0.862684251357641)(156,0.864085041761579)(156,0.924623115577889)(156,0.867256637168141)(156,0.928876244665718)(157,0.878971255673222)(157,0.921090387374462)(157,0.950035186488388)(157,0.936017253774263)(157,0.925461254612546)(157,0.912382331643736)(158,0.907158043940468)(158,0.920863309352518)(158,0.939844302901628)(158,0.927789934354486)(158,0.933333333333333)(158,0.923631123919308)(158,0.94468085106383)(159,0.93629205440229)(159,0.867635807192043)(159,0.903225806451613)(159,0.924838940586972)(159,0.941260744985673)(159,0.952313167259786)(159,0.921975662133142)(159,0.9382373351839)(160,0.883256528417819)(160,0.922519913106445)(160,0.957686882933709)(160,0.91063829787234)(160,0.930774503084304)(161,0.919354838709677)(161,0.93040293040293)(161,0.867378048780488)(161,0.917798427448177)(161,0.93304535637149)(161,0.879508825786646)(161,0.926080892608089)(161,0.939142461964039)(161,0.914565826330532)(162,0.923520923520923)(162,0.924778761061947)(162,0.917378917378917)(162,0.917444364680545)(162,0.931884057971014)(162,0.942622950819672)(162,0.86676875957121)(162,0.850308641975309)(163,0.906993511175198)(163,0.875555555555555)(163,0.936761640027797)(164,0.925)(164,0.926618705035971)(164,0.952654232424677)(164,0.921511627906977)(164,0.862042088854248)(164,0.91340782122905)(165,0.93644996347699)(165,0.911062906724512)(165,0.912103746397695)(165,0.950634696755994)(165,0.942090395480226)(166,0.9375)(166,0.937142857142857)(166,0.921625544267053)(166,0.871987951807229)(166,0.955801104972376)(166,0.918248175182482)(166,0.940041350792557)(167,0.864823348694316)(167,0.883935434281322)(167,0.948915325402379)(167,0.895077720207254)(167,0.922401171303075)(167,0.945714285714286)(167,0.943681318681319)(167,0.905559276624246)(168,0.913868613138686)(168,0.958041958041958)(168,0.930198383541513)(168,0.937325905292479)(168,0.919398907103825)(169,0.9056875449964)(169,0.904665314401623)(169,0.936781609195402)(169,0.938746438746439)(169,0.920485175202156)(169,0.94424841213832)(169,0.94774011299435)(169,0.888382687927107)(169,0.873684210526316)(170,0.934210526315789)(170,0.92533147243545)(170,0.932107496463932)(170,0.876233864844343)(170,0.943502824858757)(170,0.930465949820788)(170,0.947224126113776)(170,0.875286916602907)(170,0.935843793584379)(171,0.943237561317449)(171,0.917818181818182)(171,0.914119359534206)(171,0.916008614501077)(171,0.941585535465925)(171,0.929761042722665)(171,0.931967812728603)(172,0.874904067536454)(172,0.927641099855282)(172,0.931686046511628)(172,0.932678821879383)(172,0.923711340206186)(172,0.911485003657644)(172,0.919971160778659)(172,0.938483547925608)(172,0.879297732260424)(173,0.921625544267053)(173,0.941091954022988)(173,0.93016558675306)(173,0.930167597765363)(173,0.923741007194245)(174,0.925898752751284)(174,0.963597430406852)(174,0.903939184519696)(175,0.920957215373459)(175,0.869047619047619)(175,0.879518072289157)(175,0.89792899408284)(175,0.928622927180966)(176,0.916847433116413)(176,0.920727272727273)(176,0.935933147632312)(176,0.912280701754386)(176,0.935281837160751)(176,0.879573170731707)(176,0.900763358778626)(177,0.917391304347826)(177,0.878718535469108)(177,0.872340425531915)(178,0.935389133627019)(178,0.922850844966936)(178,0.920634920634921)(178,0.928152492668622)(178,0.939093484419263)(178,0.920749279538905)(178,0.927113702623907)(179,0.919748778785764)(179,0.884322678843227)(179,0.891437308868501)(180,0.955414012738853)(180,0.919343326195574)(180,0.951977401129943)(180,0.939093484419263)(180,0.926253687315634)(180,0.923631123919308)(181,0.93342981186686)(181,0.946251768033946)(181,0.945704467353952)(181,0.923395445134576)(181,0.934201012292118)(181,0.911909795630726)(181,0.879154078549849)(182,0.911634756995582)(182,0.932287954383464)(182,0.876712328767123)(182,0.928469241773963)(183,0.861068702290076)(183,0.883333333333333)(183,0.916184971098266)(183,0.918732782369146)(183,0.881949733434882)(183,0.927966101694915)(183,0.926086956521739)(183,0.939242315939957)(183,0.92882818116463)(183,0.927374301675977)(183,0.924791086350975)(184,0.93002915451895)(184,0.872472783825816)(184,0.931506849315068)(184,0.926512968299712)(184,0.923413566739606)(184,0.931330472103004)(184,0.943422913719943)(184,0.92507204610951)(185,0.944723618090452)(185,0.946206896551724)(185,0.876146788990826)(185,0.931982633863965)(185,0.953998584571833)(185,0.871559633027523)(186,0.936910804931109)(186,0.954038997214484)(186,0.922848664688427)(186,0.946543121881682)(186,0.948126801152738)(186,0.918561995597946)(186,0.922509225092251)(187,0.939541348158443)(187,0.928675400291121)(187,0.945659844742413)(187,0.933139534883721)(187,0.959430604982206)(188,0.949152542372881)(188,0.936170212765957)(188,0.942148760330578)(188,0.952783650458069)(188,0.950796950796951)(188,0.966878083157153)(188,0.931967812728603)(188,0.922749822820695)(189,0.932846715328467)(189,0.941508104298802)(189,0.937321937321937)(189,0.90922619047619)(189,0.936170212765957)(189,0.87667161961367)(189,0.936986301369863)(189,0.917510853835022)(189,0.928362573099415)(190,0.938218390804598)(190,0.902816901408451)(190,0.954193093727977)(190,0.936170212765957)(191,0.930763740185582)(191,0.926618705035971)(191,0.882308276385725)(191,0.870514820592824)(192,0.881226053639847)(192,0.924590163934426)(192,0.880608365019011)(192,0.936507936507936)(192,0.946927374301676)(192,0.926212227687983)(193,0.924981791697014)(193,0.950034223134839)(193,0.956583629893238)(193,0.936376210235131)(193,0.880774962742176)(193,0.927453769559033)(193,0.951646811492642)(193,0.933528122717312)(193,0.945714285714286)(193,0.933975240715268)(194,0.871678056188307)(194,0.883076923076923)(194,0.960227272727273)(195,0.879508825786646)(195,0.931642001409443)(195,0.94468085106383)(195,0.943554006968641)(195,0.917680744452398)(196,0.875766871165644)(196,0.950728660652325)(196,0.914492753623188)(196,0.935640138408305)(196,0.950865051903114)(196,0.930888575458392)(196,0.880844645550528)(196,0.931129476584022)(196,0.941261783901378)(196,0.884730538922155)(197,0.948409893992933)(197,0.876608629825889)(197,0.87789799072643)(197,0.928362573099415)(197,0.953360768175583)(197,0.922519913106445)(197,0.909090909090909)(198,0.925287356321839)(198,0.935437589670014)(198,0.875379939209726)(198,0.873684210526316)(198,0.90450204638472)(198,0.942266571632217)(198,0.943157894736842)(198,0.912594631796283)(198,0.921863799283154)(198,0.929698708751793)(198,0.917867435158501)(198,0.877566539923954)(199,0.940098661028894)(199,0.95049504950495)(199,0.938288920056101)(199,0.938539407086045)(199,0.875856816450876)(199,0.874446085672083)(199,0.956521739130435)(199,0.91493924231594)(199,0.950373895309313)(200,0.91562729273661)(200,0.926193921852388)(200,0.883509833585477)(200,0.940848990953375)(200,0.87603305785124)(200,0.860700389105058)(200,0.925219941348973)(201,0.948957584471603)(201,0.940848990953375)(201,0.94608195542775)(201,0.945224719101123)(201,0.927325581395349)(201,0.942496493688639)(201,0.929454545454545)(201,0.941176470588235)(201,0.922857142857143)(201,0.932559825960841)(202,0.926253687315634)(202,0.87816091954023)(202,0.919025674786043)(202,0.939203354297694)(202,0.947662247034194)(203,0.933139534883721)(203,0.923188405797101)(203,0.943157894736842)(203,0.866057838660578)(203,0.930909090909091)(203,0.941908713692946)(204,0.923187365398421)(204,0.923829489867226)(204,0.9164265129683)(204,0.939775910364146)(204,0.940925266903915)(205,0.924746743849494)(205,0.924623115577889)(205,0.937950937950938)(205,0.959943780744905)(205,0.918763479511143)(205,0.923753665689149)(205,0.950314905528341)(206,0.938775510204082)(206,0.959154929577465)(206,0.933618843683083)(206,0.88109756097561)(206,0.948350071736011)(206,0.938254080908446)(206,0.864343958487769)(206,0.965860597439545)(206,0.931884057971014)(206,0.942430703624733)(207,0.933333333333333)(207,0.941176470588235)(207,0.905797101449275)(207,0.909625275532696)(207,0.930458970792768)(207,0.923410404624277)(207,0.95676429567643)(207,0.926406926406926)(208,0.867692307692308)(208,0.873282442748091)(208,0.936200716845878)(208,0.934201012292118)(208,0.946018893387314)(208,0.880804953560371)(209,0.881407804131599)(209,0.940594059405941)(209,0.914285714285714)(209,0.864944649446494)(209,0.938040345821326)(210,0.940845070422535)(210,0.944639103013315)(210,0.952772073921971)(210,0.879389312977099)(210,0.926345609065156)(210,0.884528301886792)(210,0.943502824858757)(210,0.930935251798561)(210,0.93809176984705)(211,0.929721815519766)(211,0.86890243902439)(211,0.86976389946687)(211,0.952108649035025)(211,0.93935119887165)(211,0.890234670704012)(211,0.948439620081411)(211,0.875667429443173)(211,0.939306358381503)(212,0.954088952654232)(212,0.949057920446615)(212,0.932551319648094)(212,0.92511013215859)(212,0.947517730496454)(212,0.935191637630662)(212,0.939457202505219)(212,0.951923076923077)(212,0.931754874651811)(212,0.919885550786838)(212,0.87815750371471)(212,0.871194379391101)(213,0.874425727411945)(213,0.900409276944065)(213,0.9553264604811)(213,0.953615279672578)(213,0.948936170212766)(213,0.933721777130371)(213,0.881226053639847)(213,0.922966162706983)(214,0.88208269525268)(214,0.94019471488178)(214,0.936708860759493)(214,0.960111966410077)(214,0.932432432432432)(215,0.957960027567195)(215,0.924608819345661)(215,0.919191919191919)(215,0.951578947368421)(215,0.93745704467354)(215,0.936910804931109)(215,0.917613636363636)(216,0.926618705035971)(216,0.930296756383713)(216,0.869565217391304)(216,0.944326990838619)(216,0.954038997214484)(216,0.927789934354486)(216,0.885844748858447)(216,0.884615384615385)(216,0.895306859205776)(216,0.922619047619048)(216,0.906841339155749)(216,0.945606694560669)(217,0.952646239554317)(217,0.942657342657343)(217,0.934182590233546)(217,0.963423050379572)(217,0.937812723373838)(217,0.93644996347699)(218,0.877752467729689)(218,0.91907514450867)(218,0.931686046511628)(218,0.947592067988668)(218,0.95676429567643)(218,0.926481084939329)(219,0.931486880466472)(219,0.868702290076336)(219,0.943722943722944)(220,0.87556904400607)(220,0.936886395511921)(220,0.954063604240283)(220,0.947443181818182)(220,0.955509924709103)(220,0.961810466760962)(220,0.931468531468531)(221,0.965662228451296)(221,0.941585535465925)(221,0.889055472263868)(222,0.925207756232687)(222,0.958654519971969)(222,0.917630057803468)(223,0.941091954022988)(223,0.934593023255814)(223,0.93541518807665)(223,0.867924528301887)(223,0.959943780744905)(223,0.947294448348559)(223,0.934296028880866)(223,0.951098511693834)(223,0.958479943701618)(223,0.942415730337079)(224,0.953136265320836)(224,0.88855421686747)(224,0.95850622406639)(224,0.946846208362863)(225,0.924889543446244)(225,0.957567185289957)(225,0.948771929824561)(225,0.940771349862259)(225,0.945301542776998)(225,0.898107714701601)(225,0.955974842767296)(225,0.884528301886792)(225,0.947515745276417)(225,0.919236417033774)(226,0.959097320169252)(226,0.952448545067424)(226,0.926300578034682)(226,0.96113074204947)(226,0.934497816593886)(226,0.930635838150289)(226,0.874233128834356)(226,0.944564434845212)(226,0.943475226796929)(227,0.933139534883721)(227,0.914454277286136)(227,0.885119506553585)(227,0.892586989409985)(227,0.93016558675306)(227,0.941346850108617)(227,0.934673366834171)(227,0.931330472103004)(227,0.95039539899353)(227,0.93033381712627)(227,0.94017094017094)(228,0.957805907172996)(228,0.92296511627907)(228,0.93304535637149)(228,0.953125)(228,0.927350427350427)(228,0.940934065934066)(228,0.933139534883721)(228,0.875856816450876)(229,0.962025316455696)(229,0.933333333333333)(229,0.96105702364395)(229,0.925845932325414)(229,0.877828054298642)(229,0.956829440905874)(229,0.950911640953717)(229,0.932944606413994)(229,0.95177304964539)(229,0.887537993920972)(230,0.916485112563544)(230,0.932568149210904)(230,0.892612338156893)(230,0.937590711175617)(230,0.953257790368272)(230,0.878603945371775)(230,0.966548042704626)(230,0.934989043097151)(231,0.936139332365747)(231,0.943019943019943)(231,0.955307262569832)(231,0.879154078549849)(231,0.879204892966361)(231,0.942008486562942)(232,0.957431960921144)(232,0.875)(232,0.922967189728959)(232,0.924667651403249)(233,0.944915254237288)(233,0.926829268292683)(233,0.959016393442623)(233,0.967787114845938)(233,0.876876876876877)(233,0.955079474775397)(233,0.946236559139785)(234,0.879038317054846)(234,0.881818181818182)(234,0.932461873638344)(234,0.920634920634921)(234,0.937723693629205)(235,0.95886524822695)(235,0.961618981158409)(235,0.954992967651195)(235,0.93681917211329)(235,0.948824343015214)(235,0.950937950937951)(235,0.929351784413692)(236,0.954577218728162)(236,0.907470870459218)(236,0.94661921708185)(236,0.945606694560669)(236,0.934919524142757)(236,0.946313528990694)(237,0.940848990953375)(237,0.874622356495468)(237,0.932664756446991)(237,0.964137931034483)(237,0.92040520984081)(237,0.878378378378378)(237,0.963431786216596)(237,0.921282798833819)(237,0.936017253774263)(238,0.879636638909917)(238,0.885022692889561)(238,0.960167714884696)(238,0.955801104972376)(238,0.924198250728863)(238,0.875667429443173)(238,0.951374207188161)(238,0.886850152905199)(239,0.95676429567643)(239,0.884820747520976)(239,0.952380952380952)(239,0.94)(239,0.94639027877055)(240,0.885171102661597)(240,0.945086705202312)(240,0.946191474493361)(240,0.950175438596491)(240,0.946638946638947)(240,0.949681077250177)(240,0.94503925767309)(240,0.869166029074216)(240,0.946022727272727)(241,0.954063604240283)(241,0.952580195258019)(241,0.942196531791907)(241,0.927848954821308)(241,0.949828178694158)(241,0.877351392024078)(241,0.87148288973384)(241,0.958244869072894)(241,0.867614061331339)(241,0.939393939393939)(241,0.944)(241,0.947368421052631)(242,0.957252978276104)(242,0.924872355944566)(242,0.946221919673247)(242,0.947294448348559)(242,0.96078431372549)(242,0.938395415472779)(242,0.957417582417582)(242,0.931703810208483)(243,0.956043956043956)(243,0.954954954954955)(243,0.941843971631206)(243,0.957686882933709)(243,0.946564885496183)(243,0.947441217150761)(243,0.942577030812325)(243,0.92)(243,0.938804895608351)(243,0.876146788990826)(244,0.93768115942029)(244,0.936263736263736)(244,0.877061469265367)(244,0.941590429275158)(244,0.891254752851711)(244,0.933042212518195)(244,0.935553946415641)(244,0.921995783555868)(245,0.944162436548223)(245,0.957372466806429)(245,0.948884089272858)(245,0.954257565095003)(246,0.954892435808466)(246,0.974323386537127)(247,0.947368421052631)(247,0.928571428571428)(247,0.875562218890555)(247,0.922407541696882)(247,0.923525127458121)(247,0.960608154803041)(247,0.936017253774263)(247,0.951841359773371)(247,0.936416184971098)(247,0.936507936507936)(247,0.938181818181818)(247,0.874809160305343)(247,0.941428571428571)(247,0.94759511844939)(247,0.86046511627907)(248,0.932203389830508)(248,0.931486880466472)(248,0.929292929292929)(248,0.931967812728603)(248,0.879761015683346)(248,0.96575821104123)(248,0.937325905292479)(248,0.934593023255814)(248,0.928362573099415)(248,0.889058913542463)(248,0.953310104529617)(249,0.945533769063181)(249,0.916109873793615)(249,0.971988795518207)(249,0.953974895397489)(249,0.888212927756654)(249,0.880664652567976)(249,0.959440559440559)(249,0.937812723373838)(249,0.928728875826598)(249,0.935860058309038)(249,0.88078967350038)(250,0.885345482156416)(250,0.932857142857143)(250,0.892465252377469)(250,0.944947735191637)(250,0.950564971751412)(250,0.949152542372881)(250,0.952850105559465)(250,0.937274693583273)(251,0.955650929899857)(251,0.946846208362863)(251,0.893455098934551)(251,0.948965517241379)(251,0.927953890489913)(251,0.875460574797347)(251,0.951135581555402)(252,0.920315865039483)(252,0.891402714932127)(252,0.961725817675713)(252,0.929292929292929)(252,0.938202247191011)(252,0.876249039200615)(252,0.942240779401531)(252,0.879815100154083)(252,0.889565879664889)(253,0.946927374301676)(253,0.92507204610951)(253,0.939306358381503)(253,0.954513645906228)(254,0.947441217150761)(254,0.922413793103448)(254,0.951724137931034)(254,0.926086956521739)(254,0.954022988505747)(254,0.960111966410077)(254,0.964360587002096)(254,0.872372372372372)(254,0.955414012738853)(255,0.956090651558073)(255,0.951578947368421)(255,0.885145482388974)(255,0.96105702364395)(255,0.956951305575159)(255,0.873369148119724)(256,0.953504510756419)(256,0.965325936199723)(256,0.958999305072967)(256,0.960839160839161)(256,0.960893854748603)(256,0.959269662921348)(256,0.926720947446336)(256,0.925979680696662)(256,0.949752300070771)(257,0.888382687927107)(257,0.952722063037249)(257,0.951374207188161)(257,0.968858131487889)(257,0.949720670391061)(257,0.946075085324232)(257,0.945606694560669)(257,0.91820580474934)(257,0.955307262569832)(257,0.910436713545522)(257,0.96280701754386)(257,0.919801277501774)(257,0.938053097345133)(257,0.884291187739464)(258,0.95448275862069)(258,0.944639103013315)(258,0.910948905109489)(259,0.869296210363496)(259,0.951236749116608)(259,0.875375375375375)(259,0.943866943866944)(259,0.954609929078014)(259,0.957004160887656)(259,0.948201438848921)(260,0.938571428571429)(260,0.950522648083624)(260,0.922058823529412)(260,0.946996466431095)(261,0.958133150308854)(261,0.936231884057971)(261,0.955617198335645)(261,0.961538461538461)(262,0.953651685393258)(262,0.937637564196625)(262,0.885196374622356)(262,0.957746478873239)(262,0.969654199011997)(262,0.921203438395415)(262,0.941595441595441)(263,0.960227272727273)(263,0.946778711484594)(263,0.929659173313996)(263,0.960334029227557)(263,0.958712386284115)(263,0.959440559440559)(263,0.958512160228898)(263,0.877786318216756)(263,0.938307030129125)(263,0.950749464668094)(263,0.885321100917431)(263,0.95859649122807)(264,0.94639027877055)(264,0.961904761904762)(264,0.95676429567643)(265,0.931667891256429)(265,0.878388845855926)(265,0.955862068965517)(265,0.9278951201748)(265,0.884702336096458)(265,0.955096222380613)(265,0.960167714884696)(266,0.884003032600455)(266,0.922852983988355)(266,0.932178932178932)(266,0.949057920446615)(266,0.932461873638344)(266,0.964739069111424)(266,0.967063770147162)(266,0.953802416488984)(266,0.944639103013315)(267,0.969359331476323)(267,0.953910614525139)(267,0.950911640953717)(267,0.958152958152958)(267,0.903469079939668)(268,0.932551319648094)(268,0.881720430107527)(268,0.928675400291121)(268,0.952045133991537)(268,0.886363636363636)(268,0.94201861130995)(269,0.957004160887656)(269,0.956399437412096)(269,0.887377173091459)(269,0.962702322308233)(269,0.96078431372549)(269,0.883233532934132)(269,0.884969325153374)(269,0.938101788170564)(269,0.934379457917261)(269,0.953114065780266)(269,0.951566951566951)(270,0.880551301684533)(270,0.955350815024805)(270,0.938947368421053)(270,0.950865051903114)(270,0.948771929824561)(270,0.943582510578279)(270,0.948753462603878)(270,0.931156848828957)(271,0.936324167872648)(271,0.955965909090909)(271,0.946712802768166)(271,0.96140350877193)(271,0.905011219147345)(271,0.93809176984705)(271,0.94661921708185)(271,0.941595441595441)(271,0.957567185289957)(271,0.951359084406295)(271,0.950591510090466)(271,0.959056210964608)(272,0.934201012292118)(272,0.950865051903114)(272,0.884644766997708)(272,0.956703910614525)(272,0.939460247994165)(272,0.918220946915351)(272,0.956036287508723)(272,0.925608011444921)(273,0.941600576784427)(273,0.92955700798838)(273,0.96105702364395)(273,0.952712100139082)(273,0.964336661911555)(273,0.952991452991453)(273,0.948644793152639)(274,0.961565338923829)(274,0.932559825960841)(274,0.955350815024805)(274,0.929676511954993)(274,0.938746438746439)(274,0.951578947368421)(274,0.918194640338505)(275,0.953257790368272)(275,0.881736526946108)(275,0.943262411347518)(275,0.959666203059805)(275,0.950035186488388)(275,0.955223880597015)(275,0.87681713848508)(275,0.947148817802503)(276,0.958217270194986)(276,0.964739069111424)(276,0.94364161849711)(276,0.944367176634214)(276,0.951590594744122)(276,0.957567185289957)(276,0.960170697012802)(276,0.964438122332859)(276,0.932568149210904)(277,0.954128440366972)(277,0.91441111923921)(277,0.961593172119488)(277,0.930035335689046)(277,0.945378151260504)(277,0.931884057971014)(277,0.947521865889213)(278,0.92557111274871)(278,0.931428571428571)(279,0.93541518807665)(279,0.961832061068702)(279,0.93998553868402)(279,0.958391123439667)(279,0.942857142857143)(279,0.970254957507082)(280,0.934201012292118)(280,0.956460674157303)(280,0.945378151260504)(280,0.968553459119497)(280,0.946354883081155)(280,0.872372372372372)(281,0.951510892480674)(281,0.957765667574932)(281,0.955801104972376)(281,0.946403385049365)(281,0.925898752751284)(281,0.928104575163399)(282,0.93974175035868)(282,0.929824561403509)(282,0.944767441860465)(282,0.940665701881331)(282,0.94331700489853)(283,0.937812723373838)(283,0.950819672131147)(283,0.933428775948461)(283,0.904477611940298)(283,0.950704225352113)(283,0.965706447187929)(283,0.951657458563536)(283,0.956152758132956)(283,0.907738095238095)(283,0.93949694085656)(284,0.966136834830684)(284,0.94364161849711)(284,0.954738330975955)(284,0.956397426733381)(284,0.884113584036838)(284,0.945964912280702)(285,0.95859649122807)(285,0.959603118355776)(285,0.938001458789205)(285,0.966690290574061)(285,0.941843971631206)(286,0.944365192582026)(286,0.96652719665272)(286,0.947887323943662)(287,0.953867991483321)(287,0.926512968299712)(287,0.963788300835654)(287,0.951510892480674)(287,0.957507082152974)(287,0.946454413892909)(287,0.950252343186734)(288,0.970670391061452)(288,0.9432918395574)(288,0.948396094839609)(288,0.952313167259786)(289,0.92129963898917)(289,0.928571428571429)(289,0.92836676217765)(290,0.956276445698166)(290,0.949270326615705)(290,0.94982332155477)(290,0.939218523878437)(291,0.924762600438276)(291,0.927835051546392)(291,0.939655172413793)(291,0.955223880597015)(291,0.962603878116343)(291,0.935553946415641)(291,0.953586497890295)(291,0.949790794979079)(292,0.928876244665718)(292,0.949008498583569)(292,0.947901591895803)(293,0.923410404624277)(293,0.958217270194986)(293,0.890895410082769)(293,0.953179594689029)(293,0.962237762237762)(293,0.957123098201936)(293,0.960893854748603)(293,0.961111111111111)(293,0.971830985915493)(293,0.950608446671439)(293,0.94550601556971)(294,0.933139534883721)(294,0.947873799725652)(294,0.942079553384508)(295,0.939501779359431)(295,0.880733944954128)(295,0.956399437412096)(295,0.945945945945946)(295,0.920379839298758)(296,0.96045197740113)(296,0.94683908045977)(296,0.930397727272727)(296,0.959326788218794)(296,0.925979680696662)(296,0.935272727272727)(296,0.927536231884058)(296,0.944126074498567)(297,0.969401947148818)(297,0.95928226363009)(297,0.954257565095003)(298,0.957081545064378)(298,0.947589098532495)(298,0.949964763918252)(298,0.964360587002096)(298,0.969057665260197)(298,0.933721777130371)(298,0.956214689265537)(298,0.93411420204978)(299,0.944211994421199)(299,0.930434782608696)(299,0.938980617372577)(299,0.938271604938271)(299,0.947813822284908)(299,0.96678445229682)(299,0.96969696969697)(300,0.965811965811966)(300,0.961185603387438)(300,0.963276836158192)(300,0.954892435808466)(300,0.935272727272727)(300,0.932761087267525)(300,0.936834634492548)(300,0.952050034746352)(300,0.96140350877193)(300,0.95373665480427)(300,0.955244755244755)(301,0.943965517241379)(301,0.943977591036415)(301,0.965181058495822)(301,0.929963898916967)(301,0.948771929824561)(301,0.930232558139535)(301,0.959372772630078)(301,0.950537634408602)(302,0.95578947368421)(302,0.932191291934333)(302,0.97581202487906)(302,0.9593837535014)(302,0.942415730337079)(302,0.957252978276104)(302,0.962804005722461)(302,0.956521739130435)(302,0.940256045519203)(302,0.968421052631579)(303,0.957865168539326)(303,0.957241379310345)(303,0.945165945165945)(303,0.946466809421842)(303,0.954257565095003)(303,0.961971830985915)(303,0.943267259056733)(303,0.966643009226402)(303,0.950522648083624)(304,0.972593113141251)(304,0.955431754874652)(304,0.933812949640288)(304,0.928362573099415)(304,0.966690290574061)(305,0.959553695955369)(305,0.93681917211329)(305,0.957843814789219)(305,0.949367088607595)(305,0.950796950796951)(305,0.968332160450387)(305,0.961185603387438)(305,0.96398891966759)(305,0.958017894012388)(305,0.933333333333333)(305,0.949720670391061)(306,0.956276445698166)(306,0.947148817802503)(306,0.954738330975955)(306,0.968465311843027)(306,0.956768249468462)(306,0.960662525879917)(306,0.952722063037249)(307,0.957312806158153)(307,0.96011396011396)(307,0.953191489361702)(307,0.953651685393258)(307,0.949964763918252)(307,0.961725817675713)(307,0.93304535637149)(307,0.96575821104123)(307,0.94364161849711)(307,0.965322009907997)(308,0.962859145059565)(308,0.94683908045977)(308,0.965753424657534)(308,0.954609929078014)(308,0.951624548736462)(308,0.972554539057002)(308,0.961020552799433)(309,0.962910128388017)(309,0.968188105117566)(309,0.950617283950617)(309,0.960839160839161)(309,0.952380952380952)(310,0.952315134761576)(310,0.96105702364395)(310,0.964059196617336)(310,0.957627118644068)(310,0.962343096234309)(310,0.956944444444444)(310,0.94862772695285)(310,0.961511546536039)(310,0.943502824858757)(311,0.961618981158409)(311,0.961672473867596)(311,0.946502057613169)(312,0.958185683912119)(312,0.967063770147162)(312,0.937544867193108)(312,0.962199312714777)(312,0.942028985507246)(312,0.970254957507082)(312,0.942877801879971)(312,0.956582633053221)(313,0.963431786216596)(313,0.957431960921144)(313,0.973163841807909)(313,0.920315865039483)(313,0.955244755244755)(314,0.940330697340043)(314,0.957004160887656)(314,0.955307262569832)(314,0.957609451007644)(314,0.968465311843027)(315,0.967017543859649)(315,0.968465311843027)(315,0.960167714884696)(315,0.94452347083926)(315,0.960839160839161)(315,0.96814404432133)(315,0.966987620357634)(315,0.967741935483871)(315,0.976421636615811)(315,0.956219596942321)(315,0.947750362844702)(316,0.96398891966759)(316,0.95983379501385)(316,0.969909027291812)(316,0.955160142348754)(316,0.956028368794326)(316,0.932363636363636)(317,0.960674157303371)(317,0.956760466712423)(317,0.966971187631764)(317,0.938659058487874)(317,0.948116560056859)(317,0.958630527817404)(317,0.960553633217993)(317,0.967787114845938)(317,0.967696629213483)(317,0.977746870653685)(318,0.964589235127479)(318,0.95676429567643)(318,0.954954954954955)(318,0.94413407821229)(319,0.948042704626334)(319,0.962758620689655)(319,0.971870604781997)(319,0.954979536152797)(319,0.952515946137491)(319,0.956583629893238)(319,0.959112959112959)(320,0.955602536997886)(320,0.96)(320,0.964438122332859)(321,0.956036287508723)(321,0.95578947368421)(321,0.962395543175487)(321,0.94361170592434)(321,0.951442646023927)(321,0.957264957264957)(321,0.950108459869848)(322,0.958479943701618)(322,0.974358974358974)(322,0.965092402464066)(322,0.953179594689029)(322,0.954230235783634)(322,0.951601908657123)(322,0.953998584571833)(323,0.935178441369264)(323,0.952108649035025)(323,0.960674157303371)(323,0.95707248416608)(324,0.96850944716585)(324,0.934645115952214)(324,0.95707248416608)(324,0.945244956772334)(324,0.957960027567195)(325,0.958303886925795)(325,0.938953488372093)(325,0.947443181818182)(325,0.949080622347949)(326,0.96113074204947)(326,0.957491289198606)(326,0.962910128388017)(326,0.951153324287653)(326,0.964589235127479)(326,0.955903271692745)(326,0.960055096418733)(326,0.949321912919343)(327,0.951578947368421)(327,0.954738330975955)(327,0.95807560137457)(327,0.963534361851332)(327,0.961672473867596)(327,0.961725817675713)(327,0.974719101123595)(327,0.946251768033946)(328,0.948275862068965)(328,0.964838255977496)(328,0.955539872971066)(328,0.949410949410949)(328,0.959269662921348)(328,0.96678445229682)(328,0.947589098532495)(328,0.9587852494577)(328,0.96969696969697)(328,0.956090651558073)(329,0.956768249468462)(329,0.959039548022599)(329,0.955264969029594)(329,0.962649753347428)(330,0.959545777146913)(330,0.966339410939691)(330,0.973389355742297)(330,0.961971830985915)(330,0.95373665480427)(330,0.958806818181818)(330,0.967787114845938)(330,0.959666203059805)(330,0.953191489361702)(330,0.958904109589041)(330,0.94232105628909)(331,0.956276445698166)(331,0.956028368794326)(331,0.970076548364649)(331,0.951374207188161)(331,0.954230235783634)(331,0.974143955276031)(332,0.972784368457781)(332,0.964639321074965)(332,0.963224893917963)(332,0.964739069111424)(333,0.959139784946237)(333,0.975103734439834)(333,0.966690290574061)(333,0.96045197740113)(333,0.974107767669699)(333,0.957924263674614)(334,0.965092402464066)(334,0.937184115523466)(334,0.964387464387464)(334,0.968465311843027)(335,0.964788732394366)(335,0.955965909090909)(335,0.962237762237762)(335,0.966386554621849)(335,0.9614576033637)(335,0.96416022487702)(335,0.970464135021097)(335,0.97907949790795)(336,0.939587775408671)(336,0.963957597173145)(336,0.961511546536039)(336,0.967787114845938)(337,0.953802416488984)(337,0.967257844474761)(337,0.939265536723164)(337,0.953867991483321)(337,0.945454545454545)(338,0.977194194885971)(338,0.964607911172797)(338,0.964438122332859)(338,0.96140350877193)(339,0.959545777146913)(339,0.96416022487702)(339,0.971626297577855)(339,0.955539872971066)(339,0.954545454545454)(340,0.957203994293866)(340,0.959497206703911)(340,0.961185603387438)(341,0.948863636363636)(341,0.96575821104123)(341,0.938307030129125)(341,0.973756906077348)(341,0.963838664812239)(341,0.958362738179252)(341,0.97396199859254)(341,0.962290502793296)(342,0.971547536433032)(342,0.969866853538893)(342,0.963483146067416)(343,0.971751412429378)(343,0.964539007092199)(343,0.967468175388967)(343,0.960334029227557)(344,0.957004160887656)(344,0.971187631763879)(344,0.960283687943262)(344,0.955414012738853)(344,0.957372466806429)(344,0.96551724137931)(344,0.969230769230769)(344,0.97029702970297)(345,0.963224893917963)(345,0.965322009907997)(345,0.970818505338078)(346,0.963888888888889)(346,0.948179271708683)(346,0.964459930313589)(346,0.957372466806429)(347,0.960716747070985)(347,0.958126330731015)(347,0.965034965034965)(347,0.950207468879668)(347,0.958391123439667)(347,0.964539007092199)(347,0.97106563161609)(347,0.957132817990161)(347,0.956944444444444)(347,0.959943780744905)(348,0.96301465457083)(349,0.96218487394958)(349,0.967201674808095)(349,0.959430604982206)(349,0.95707248416608)(349,0.96600566572238)(350,0.956641431520991)(351,0.968728283530229)(351,0.957475994513031)(351,0.965706447187929)(351,0.97350069735007)(351,0.963906581740976)(352,0.94683908045977)(352,0.959603118355776)(352,0.969101123595505)(352,0.956338028169014)(352,0.96600566572238)(353,0.966595593461265)(353,0.969654199011997)(353,0.964639321074965)(353,0.966971187631764)(353,0.956884561891516)(353,0.967605633802817)(354,0.958421423537702)(354,0.964851826326671)(354,0.971388695045359)(354,0.966971187631764)(354,0.968794326241135)(355,0.965034965034965)(355,0.957668285912561)(355,0.963573883161512)(355,0.964788732394366)(355,0.959666203059805)(356,0.953651685393258)(356,0.966831333803811)(356,0.956706884315117)(356,0.949494949494949)(357,0.958981612446959)(357,0.962544169611307)(357,0.962702322308233)(357,0.962962962962963)(357,0.96575821104123)(357,0.9614576033637)(357,0.959660297239915)(357,0.962649753347428)(358,0.951983298538622)(358,0.966804979253112)(358,0.969782150386507)(359,0.974535443909153)(359,0.97106563161609)(359,0.96078431372549)(359,0.960111966410077)(360,0.969993021632938)(360,0.95345557122708)(360,0.956824512534819)(360,0.961725817675713)(360,0.95589298626175)(360,0.956461644782308)(360,0.960283687943262)(361,0.967428967428967)(361,0.97106563161609)(361,0.952312138728324)(361,0.969571230982019)(362,0.956824512534819)(362,0.956760466712423)(363,0.962702322308233)(363,0.965565706254392)(363,0.969438521677327)(363,0.959256611865618)(363,0.961756373937677)(364,0.966924700914849)(364,0.96074232690935)(364,0.952850105559465)(364,0.959488272921109)(365,0.955862068965517)(365,0.963224893917963)(365,0.959039548022599)(365,0.967063770147162)(366,0.966149506346968)(366,0.968794326241135)(366,0.950959488272921)(366,0.953125)(366,0.961349262122277)(367,0.96850944716585)(367,0.972554539057002)(368,0.980446927374302)(368,0.967201674808095)(368,0.960932145305003)(368,0.976256983240223)(368,0.971147079521464)(369,0.964187327823691)(369,0.954954954954955)(370,0.976223776223776)(370,0.95751854349292)(370,0.959440559440559)(370,0.957805907172996)(372,0.96)(372,0.963321799307958)(372,0.966712898751734)(372,0.9553264604811)(373,0.959256611865618)(374,0.96575821104123)(374,0.96831955922865)(374,0.959595959595959)(374,0.950354609929078)(375,0.969101123595505)(375,0.952850105559465)(376,0.966690290574061)(376,0.965024982155603)(377,0.970034843205575)(377,0.956036287508723)(378,0.950522648083624)(379,0.96551724137931)(379,0.949784791965567)(379,0.958333333333333)(380,0.972822299651568)(380,0.969568294409059)(381,0.962384669978708)(381,0.956036287508723)(382,0.962544169611307)(382,0.964137931034483)(382,0.975541579315164)(384,0.963431786216596)(384,0.971228070175438)(384,0.968597348220516)(384,0.97191011235955)(386,0.960339943342776)(386,0.962655601659751)(386,0.95707248416608)(388,0.968771686328938)(388,0.970792767732962)(388,0.966666666666666)(389,0.968107725017718)(390,0.96995108315863)(394,0.975677553856845)(394,0.974143955276031)(395,0.967832167832168)(395,0.959610027855153)(399,0.954703832752613) 
};

\end{axis}
\end{tikzpicture}%

%% This file was created by matlab2tikz v0.2.3.
% Copyright (c) 2008--2012, Nico Schlömer <nico.schloemer@gmail.com>
% All rights reserved.
% 
% 
%

\definecolor{locol}{rgb}{0.26, 0.45, 0.65}

\begin{tikzpicture}

\begin{axis}[%
tick label style={font=\tiny},
label style={font=\tiny},
xlabel shift={-10pt},
ylabel shift={-17pt},
legend style={font=\tiny},
view={0}{90},
width=\figurewidth,
height=\figureheight,
scale only axis,
xmin=0, xmax=1478,
xtick={0, 400, 1000, 1400},
xlabel={Length (m)},
ymin=-18, ymax=0,
ytick={0, -4, -14, -18},
ylabel={Depth (m)},
name=plot1,
axis lines*=box,
tickwidth=0.1cm,
clip=false
]

\addplot [fill=locol,draw=none,forget plot] coordinates{ (1478,0)(1478,-0.181818181818182)(1478,-0.363636363636364)(1478,-0.545454545454545)(1478,-0.727272727272727)(1478,-0.909090909090909)(1478,-1.09090909090909)(1478,-1.27272727272727)(1478,-1.45454545454545)(1478,-1.63636363636364)(1478,-1.81818181818182)(1478,-2)(1478,-2.18181818181818)(1478,-2.36363636363636)(1478,-2.54545454545455)(1478,-2.72727272727273)(1478,-2.90909090909091)(1478,-3.09090909090909)(1478,-3.27272727272727)(1478,-3.45454545454545)(1478,-3.63636363636364)(1478,-3.81818181818182)(1478,-4)(1478,-4.18181818181818)(1478,-4.36363636363636)(1478,-4.54545454545455)(1478,-4.72727272727273)(1478,-4.90909090909091)(1478,-5.09090909090909)(1478,-5.27272727272727)(1478,-5.45454545454545)(1478,-5.63636363636364)(1478,-5.81818181818182)(1478,-6)(1478,-6.18181818181818)(1478,-6.36363636363636)(1478,-6.54545454545455)(1478,-6.72727272727273)(1478,-6.90909090909091)(1478,-7.09090909090909)(1478,-7.27272727272727)(1478,-7.45454545454545)(1478,-7.63636363636364)(1478,-7.81818181818182)(1478,-8)(1478,-8.18181818181818)(1478,-8.36363636363636)(1478,-8.54545454545455)(1478,-8.72727272727273)(1478,-8.90909090909091)(1478,-9.09090909090909)(1478,-9.27272727272727)(1478,-9.45454545454546)(1478,-9.63636363636364)(1478,-9.81818181818182)(1478,-10)(1478,-10.1818181818182)(1478,-10.3636363636364)(1478,-10.5454545454545)(1478,-10.7272727272727)(1478,-10.9090909090909)(1478,-11.0909090909091)(1478,-11.2727272727273)(1478,-11.4545454545455)(1478,-11.6363636363636)(1478,-11.8181818181818)(1478,-12)(1478,-12.1818181818182)(1478,-12.3636363636364)(1478,-12.5454545454545)(1478,-12.7272727272727)(1478,-12.9090909090909)(1478,-13.0909090909091)(1478,-13.2727272727273)(1478,-13.4545454545455)(1478,-13.6363636363636)(1478,-13.8181818181818)(1478,-14)(1478,-14.1818181818182)(1478,-14.3636363636364)(1478,-14.5454545454545)(1478,-14.7272727272727)(1478,-14.9090909090909)(1478,-15.0909090909091)(1478,-15.2727272727273)(1478,-15.4545454545455)(1478,-15.6363636363636)(1478,-15.8181818181818)(1478,-16)(1478,-16.1818181818182)(1478,-16.3636363636364)(1478,-16.5454545454545)(1478,-16.7272727272727)(1478,-16.9090909090909)(1478,-17.0909090909091)(1478,-17.2727272727273)(1478,-17.4545454545455)(1478,-17.6363636363636)(1478,-17.8181818181818)(1478,-18)(1463.07070707071,-18)(1448.14141414141,-18)(1433.21212121212,-18)(1418.28282828283,-18)(1403.35353535354,-18)(1388.42424242424,-18)(1373.49494949495,-18)(1358.56565656566,-18)(1343.63636363636,-18)(1328.70707070707,-18)(1313.77777777778,-18)(1298.84848484848,-18)(1283.91919191919,-18)(1268.9898989899,-18)(1254.06060606061,-18)(1239.13131313131,-18)(1224.20202020202,-18)(1209.27272727273,-18)(1194.34343434343,-18)(1179.41414141414,-18)(1164.48484848485,-18)(1149.55555555556,-18)(1134.62626262626,-18)(1119.69696969697,-18)(1104.76767676768,-18)(1089.83838383838,-18)(1074.90909090909,-18)(1059.9797979798,-18)(1045.05050505051,-18)(1030.12121212121,-18)(1015.19191919192,-18)(1000.26262626263,-18)(985.333333333333,-18)(970.40404040404,-18)(955.474747474747,-18)(940.545454545455,-18)(925.616161616162,-18)(910.686868686869,-18)(895.757575757576,-18)(880.828282828283,-18)(865.89898989899,-18)(850.969696969697,-18)(836.040404040404,-18)(821.111111111111,-18)(806.181818181818,-18)(791.252525252525,-18)(776.323232323232,-18)(761.393939393939,-18)(746.464646464646,-18)(731.535353535354,-18)(716.606060606061,-18)(701.676767676768,-18)(686.747474747475,-18)(671.818181818182,-18)(656.888888888889,-18)(641.959595959596,-18)(627.030303030303,-18)(612.10101010101,-18)(597.171717171717,-18)(582.242424242424,-18)(567.313131313131,-18)(552.383838383838,-18)(537.454545454546,-18)(522.525252525253,-18)(507.59595959596,-18)(492.666666666667,-18)(477.737373737374,-18)(462.808080808081,-18)(447.878787878788,-18)(432.949494949495,-18)(418.020202020202,-18)(403.090909090909,-18)(388.161616161616,-18)(373.232323232323,-18)(358.30303030303,-18)(343.373737373737,-18)(328.444444444444,-18)(313.515151515152,-18)(298.585858585859,-18)(283.656565656566,-18)(268.727272727273,-18)(253.79797979798,-18)(238.868686868687,-18)(223.939393939394,-18)(209.010101010101,-18)(194.080808080808,-18)(179.151515151515,-18)(164.222222222222,-18)(149.292929292929,-18)(134.363636363636,-18)(119.434343434343,-18)(104.505050505051,-18)(89.5757575757576,-18)(74.6464646464647,-18)(59.7171717171717,-18)(44.7878787878788,-18)(29.8585858585859,-18)(14.9292929292929,-18)(0,-18)(0,-17.8181818181818)(0,-17.6363636363636)(0,-17.4545454545455)(0,-17.2727272727273)(0,-17.0909090909091)(0,-16.9090909090909)(0,-16.7272727272727)(0,-16.5454545454545)(0,-16.3636363636364)(0,-16.1818181818182)(0,-16)(0,-15.8181818181818)(0,-15.6363636363636)(0,-15.4545454545455)(0,-15.2727272727273)(0,-15.0909090909091)(0,-14.9090909090909)(0,-14.7272727272727)(0,-14.5454545454545)(0,-14.3636363636364)(0,-14.1818181818182)(0,-14)(0,-13.8181818181818)(0,-13.6363636363636)(0,-13.4545454545455)(0,-13.2727272727273)(0,-13.0909090909091)(0,-12.9090909090909)(0,-12.7272727272727)(0,-12.5454545454545)(0,-12.3636363636364)(0,-12.1818181818182)(0,-12)(0,-11.8181818181818)(0,-11.6363636363636)(0,-11.4545454545455)(0,-11.2727272727273)(0,-11.0909090909091)(0,-10.9090909090909)(0,-10.7272727272727)(0,-10.5454545454545)(0,-10.3636363636364)(0,-10.1818181818182)(0,-10)(0,-9.81818181818182)(0,-9.63636363636364)(0,-9.45454545454546)(0,-9.27272727272727)(0,-9.09090909090909)(0,-8.90909090909091)(0,-8.72727272727273)(0,-8.54545454545455)(0,-8.36363636363636)(0,-8.18181818181818)(0,-8)(0,-7.81818181818182)(0,-7.63636363636364)(0,-7.45454545454545)(0,-7.27272727272727)(0,-7.09090909090909)(0,-6.90909090909091)(0,-6.72727272727273)(0,-6.54545454545455)(0,-6.36363636363636)(0,-6.18181818181818)(0,-6)(0,-5.81818181818182)(0,-5.63636363636364)(0,-5.45454545454545)(0,-5.27272727272727)(0,-5.09090909090909)(0,-4.90909090909091)(0,-4.72727272727273)(0,-4.54545454545455)(0,-4.36363636363636)(0,-4.18181818181818)(0,-4)(0,-3.81818181818182)(0,-3.63636363636364)(0,-3.45454545454545)(0,-3.27272727272727)(0,-3.09090909090909)(0,-2.90909090909091)(0,-2.72727272727273)(0,-2.54545454545455)(0,-2.36363636363636)(0,-2.18181818181818)(0,-2)(0,-1.81818181818182)(0,-1.63636363636364)(0,-1.45454545454545)(0,-1.27272727272727)(0,-1.09090909090909)(0,-0.909090909090909)(0,-0.727272727272727)(0,-0.545454545454545)(0,-0.363636363636364)(0,-0.181818181818182)(0,0)(14.9292929292929,0)(29.8585858585859,0)(44.7878787878788,0)(59.7171717171717,0)(74.6464646464647,0)(89.5757575757576,0)(104.505050505051,0)(119.434343434343,0)(134.363636363636,0)(149.292929292929,0)(164.222222222222,0)(179.151515151515,0)(194.080808080808,0)(209.010101010101,0)(223.939393939394,0)(238.868686868687,0)(253.79797979798,0)(268.727272727273,0)(283.656565656566,0)(298.585858585859,0)(313.515151515152,0)(328.444444444444,0)(343.373737373737,0)(358.30303030303,0)(373.232323232323,0)(388.161616161616,0)(403.090909090909,0)(418.020202020202,0)(432.949494949495,0)(447.878787878788,0)(462.808080808081,0)(477.737373737374,0)(492.666666666667,0)(507.59595959596,0)(522.525252525253,0)(537.454545454546,0)(552.383838383838,0)(567.313131313131,0)(582.242424242424,0)(597.171717171717,0)(612.10101010101,0)(627.030303030303,0)(641.959595959596,0)(656.888888888889,0)(671.818181818182,0)(686.747474747475,0)(701.676767676768,0)(716.606060606061,0)(731.535353535354,0)(746.464646464646,0)(761.393939393939,0)(776.323232323232,0)(791.252525252525,0)(806.181818181818,0)(821.111111111111,0)(836.040404040404,0)(850.969696969697,0)(865.89898989899,0)(880.828282828283,0)(895.757575757576,0)(910.686868686869,0)(925.616161616162,0)(940.545454545455,0)(955.474747474747,0)(970.40404040404,0)(985.333333333333,0)(1000.26262626263,0)(1015.19191919192,0)(1030.12121212121,0)(1045.05050505051,0)(1059.9797979798,0)(1074.90909090909,0)(1089.83838383838,0)(1104.76767676768,0)(1119.69696969697,0)(1134.62626262626,0)(1149.55555555556,0)(1164.48484848485,0)(1179.41414141414,0)(1194.34343434343,0)(1209.27272727273,0)(1224.20202020202,0)(1239.13131313131,0)(1254.06060606061,0)(1268.9898989899,0)(1283.91919191919,0)(1298.84848484848,0)(1313.77777777778,0)(1328.70707070707,0)(1343.63636363636,0)(1358.56565656566,0)(1373.49494949495,0)(1388.42424242424,0)(1403.35353535354,0)(1418.28282828283,0)(1433.21212121212,0)(1448.14141414141,0)(1463.07070707071,0)(1478,0)};

\addplot [fill=red!40!yellow,draw=none,forget plot] coordinates{ (574.777777777778,-7.27272727272727)(582.242424242424,-7.3030303030303)(597.171717171717,-7.42424242424242)(600.90404040404,-7.45454545454545)(612.10101010101,-7.54545454545455)(619.565656565657,-7.63636363636364)(627.030303030303,-7.77272727272727)(629.518518518518,-7.81818181818182)(634.494949494949,-8)(634.494949494949,-8.18181818181818)(629.518518518518,-8.36363636363636)(627.030303030303,-8.40909090909091)(619.565656565657,-8.54545454545455)(612.10101010101,-8.63636363636364)(604.636363636364,-8.72727272727273)(597.171717171717,-8.81818181818182)(585.974747474747,-8.90909090909091)(582.242424242424,-8.93939393939394)(567.313131313131,-9.06060606060606)(559.848484848485,-9.09090909090909)(552.383838383838,-9.12121212121212)(537.454545454546,-9.18181818181818)(522.525252525253,-9.18181818181818)(507.59595959596,-9.18181818181818)(492.666666666667,-9.12121212121212)(485.20202020202,-9.09090909090909)(477.737373737374,-9.06060606060606)(462.808080808081,-8.93939393939394)(459.075757575758,-8.90909090909091)(447.878787878788,-8.77272727272727)(445.390572390572,-8.72727272727273)(435.43771043771,-8.54545454545455)(432.949494949495,-8.45454545454545)(430.461279461279,-8.36363636363636)(420.508417508418,-8.18181818181818)(418.020202020202,-8.09090909090909)(415.531986531986,-8)(415.531986531986,-7.81818181818182)(418.020202020202,-7.77272727272727)(425.484848484849,-7.63636363636364)(432.949494949495,-7.54545454545455)(444.146464646465,-7.45454545454545)(447.878787878788,-7.42424242424242)(462.808080808081,-7.36363636363636)(477.737373737374,-7.3030303030303)(485.20202020202,-7.27272727272727)(492.666666666667,-7.24242424242424)(507.59595959596,-7.18181818181818)(522.525252525253,-7.18181818181818)(537.454545454546,-7.18181818181818)(552.383838383838,-7.18181818181818)(567.313131313131,-7.24242424242424)(574.777777777778,-7.27272727272727)};

\addplot [fill=red!40!yellow,draw=none,forget plot] coordinates{ (231.40404040404,-7.81818181818182)(238.868686868687,-7.84848484848485)(253.79797979798,-7.90909090909091)(268.727272727273,-7.96969696969697)(272.459595959596,-8)(283.656565656566,-8.13636363636364)(286.144781144781,-8.18181818181818)(286.144781144781,-8.36363636363636)(283.656565656566,-8.40909090909091)(276.191919191919,-8.54545454545455)(268.727272727273,-8.63636363636364)(257.530303030303,-8.72727272727273)(253.79797979798,-8.75757575757576)(238.868686868687,-8.81818181818182)(223.939393939394,-8.81818181818182)(209.010101010101,-8.81818181818182)(194.080808080808,-8.81818181818182)(179.151515151515,-8.75757575757576)(171.686868686869,-8.72727272727273)(164.222222222222,-8.6969696969697)(149.292929292929,-8.57575757575757)(145.560606060606,-8.54545454545455)(134.363636363636,-8.45454545454545)(126.89898989899,-8.36363636363636)(119.434343434343,-8.22727272727273)(116.946127946128,-8.18181818181818)(119.434343434343,-8.09090909090909)(121.922558922559,-8)(134.363636363636,-7.84848484848485)(141.828282828283,-7.81818181818182)(149.292929292929,-7.78787878787879)(164.222222222222,-7.72727272727273)(179.151515151515,-7.72727272727273)(194.080808080808,-7.72727272727273)(209.010101010101,-7.72727272727273)(223.939393939394,-7.78787878787879)(231.40404040404,-7.81818181818182)};

\addplot [
color=white,
draw=white,
only marks,
mark=x,
mark options={solid},
mark size=1.8pt,
line width=0.2pt,
forget plot
]
coordinates{
 (821.111111111111,-13.6363636363636)(1224.20202020202,-3.27272727272727)(0,-18)(0,-8)(0,-6)(253.79797979798,-8.90909090909091)(104.505050505051,-10.1818181818182)(388.161616161616,-7.63636363636364)(492.666666666667,-8.90909090909091)(209.010101010101,-7.63636363636364)(656.888888888889,-7.81818181818182)(388.161616161616,-9.27272727272727)(507.59595959596,-8)(507.59595959596,-7.27272727272727)(567.313131313131,-8.72727272727273)(612.10101010101,-8.90909090909091)(238.868686868687,-8.54545454545455)(552.383838383838,-7.27272727272727)(418.020202020202,-7.27272727272727)(209.010101010101,-8.36363636363636)(89.5757575757576,-8.54545454545455)(134.363636363636,-8)(119.434343434343,-8.72727272727273)(328.444444444444,-7.63636363636364)(119.434343434343,-7.81818181818182)(89.5757575757576,-7.63636363636364)(627.030303030303,-8.18181818181818)(477.737373737374,-9.09090909090909)(537.454545454546,-7.27272727272727)(223.939393939394,-8.90909090909091)(462.808080808081,-9.09090909090909)(74.6464646464647,-7.63636363636364)(194.080808080808,-7.63636363636364)(597.171717171717,-7.45454545454545)(612.10101010101,-7.27272727272727)(612.10101010101,-8.54545454545455)(194.080808080808,-8.90909090909091)(343.373737373737,-8.90909090909091)(477.737373737374,-7.27272727272727)(597.171717171717,-7.27272727272727)(373.232323232323,-7.63636363636364)(89.5757575757576,-8.36363636363636)(567.313131313131,-8.90909090909091)(522.525252525253,-9.09090909090909)(567.313131313131,-9.09090909090909)(328.444444444444,-8.90909090909091)(627.030303030303,-7.81818181818182)(388.161616161616,-8.90909090909091)(313.515151515152,-7.81818181818182)(223.939393939394,-7.63636363636364)(119.434343434343,-7.63636363636364)(582.242424242424,-9.09090909090909)(627.030303030303,-8.36363636363636)(462.808080808081,-7.27272727272727)(403.090909090909,-8.90909090909091)(358.30303030303,-8.72727272727273)(149.292929292929,-8.72727272727273)(388.161616161616,-8.72727272727273)(328.444444444444,-7.81818181818182)(238.868686868687,-7.63636363636364)(358.30303030303,-8.54545454545455)(612.10101010101,-8.72727272727273)(582.242424242424,-7.27272727272727)(373.232323232323,-8.54545454545455)(373.232323232323,-8.36363636363636)(627.030303030303,-8)(388.161616161616,-8.36363636363636)(432.949494949495,-8.54545454545455)(432.949494949495,-7.45454545454545)(597.171717171717,-8.90909090909091)(313.515151515152,-8.36363636363636)(104.505050505051,-7.63636363636364)(74.6464646464647,-8)(134.363636363636,-8.54545454545455)(238.868686868687,-8.72727272727273)(522.525252525253,-7.09090909090909)(298.585858585859,-7.81818181818182)(403.090909090909,-7.63636363636364)(447.878787878788,-8.72727272727273)(492.666666666667,-9.09090909090909)(552.383838383838,-9.09090909090909)(209.010101010101,-8.72727272727273)(104.505050505051,-8.18181818181818)(179.151515151515,-8.72727272727273)(627.030303030303,-8.54545454545455)(388.161616161616,-8.18181818181818)(298.585858585859,-8.54545454545455)(283.656565656566,-7.81818181818182)(298.585858585859,-8.36363636363636)(164.222222222222,-7.45454545454545)(388.161616161616,-7.81818181818182)(388.161616161616,-8)(418.020202020202,-8.36363636363636)(537.454545454546,-9.27272727272727)(552.383838383838,-9.27272727272727)(432.949494949495,-8.36363636363636)(89.5757575757576,-8)(119.434343434343,-8.36363636363636)(104.505050505051,-8.36363636363636)(298.585858585859,-8.18181818181818)(567.313131313131,-7.27272727272727)(403.090909090909,-8)(179.151515151515,-7.45454545454545)(507.59595959596,-9.09090909090909)(627.030303030303,-7.63636363636364)(612.10101010101,-7.45454545454545)(418.020202020202,-7.63636363636364)(89.5757575757576,-8.18181818181818)(268.727272727273,-7.81818181818182)(403.090909090909,-7.81818181818182)(164.222222222222,-8.72727272727273)(253.79797979798,-8.72727272727273)(283.656565656566,-8.54545454545455)(447.878787878788,-7.27272727272727)(298.585858585859,-8)(134.363636363636,-7.63636363636364)(104.505050505051,-7.81818181818182)(492.666666666667,-7.27272727272727)(209.010101010101,-8.90909090909091)(418.020202020202,-8.18181818181818)(119.434343434343,-8.54545454545455)(641.959595959596,-8)(641.959595959596,-8.18181818181818)(477.737373737374,-8.90909090909091)(641.959595959596,-7.81818181818182)(89.5757575757576,-7.81818181818182)(268.727272727273,-8.72727272727273)(283.656565656566,-8)(462.808080808081,-8.90909090909091)(447.878787878788,-7.45454545454545)(418.020202020202,-7.81818181818182)(432.949494949495,-7.63636363636364)(462.808080808081,-7.45454545454545)(522.525252525253,-7.27272727272727)(418.020202020202,-8)(432.949494949495,-8.18181818181818)(432.949494949495,-7.81818181818182)(253.79797979798,-7.63636363636364)(149.292929292929,-7.63636363636364)(179.151515151515,-7.63636363636364)(164.222222222222,-7.63636363636364)(253.79797979798,-7.81818181818182)(283.656565656566,-8.36363636363636)(447.878787878788,-8.54545454545455)(522.525252525253,-9.27272727272727)(641.959595959596,-8.36363636363636)(238.868686868687,-7.81818181818182)(432.949494949495,-8)(104.505050505051,-8)(238.868686868687,-8.90909090909091)(283.656565656566,-8.18181818181818)(582.242424242424,-8.90909090909091)(447.878787878788,-7.63636363636364)(268.727272727273,-8.54545454545455)(552.383838383838,-7.09090909090909)(537.454545454546,-7.09090909090909)(462.808080808081,-8.72727272727273)(447.878787878788,-8.90909090909091)(627.030303030303,-7.45454545454545)(537.454545454546,-9.09090909090909)(134.363636363636,-7.81818181818182)(612.10101010101,-7.63636363636364)(313.515151515152,-8.18181818181818)(627.030303030303,-8.72727272727273)(477.737373737374,-7.45454545454545)(194.080808080808,-8.72727272727273)(223.939393939394,-8.72727272727273)(313.515151515152,-8)(268.727272727273,-8)(597.171717171717,-8.72727272727273)(223.939393939394,-7.81818181818182)(567.313131313131,-7.09090909090909)(507.59595959596,-9.27272727272727)(447.878787878788,-8.36363636363636)(432.949494949495,-8.72727272727273)(149.292929292929,-8.54545454545455)(567.313131313131,-9.27272727272727)(283.656565656566,-8.72727272727273)(582.242424242424,-7.45454545454545)(492.666666666667,-9.27272727272727)(403.090909090909,-8.18181818181818)(119.434343434343,-8)(268.727272727273,-7.63636363636364)(74.6464646464647,-8.18181818181818)(597.171717171717,-9.09090909090909)(268.727272727273,-8.36363636363636)(134.363636363636,-8.72727272727273)(104.505050505051,-8.54545454545455)(179.151515151515,-8.90909090909091)(447.878787878788,-7.81818181818182)(507.59595959596,-7.09090909090909)(612.10101010101,-8.36363636363636)(209.010101010101,-7.81818181818182)(328.444444444444,-8.18181818181818)(268.727272727273,-8.18181818181818)(418.020202020202,-8.54545454545455)(74.6464646464647,-7.81818181818182)(612.10101010101,-7.81818181818182)(641.959595959596,-7.63636363636364)(328.444444444444,-8)(462.808080808081,-8.54545454545455)(403.090909090909,-8.36363636363636)(253.79797979798,-8.54545454545455)(149.292929292929,-7.81818181818182)(194.080808080808,-7.81818181818182)(253.79797979798,-8)(164.222222222222,-7.81818181818182)(179.151515151515,-7.81818181818182)(119.434343434343,-8.18181818181818)(134.363636363636,-8.36363636363636)(418.020202020202,-7.45454545454545)(492.666666666667,-7.09090909090909)(164.222222222222,-8.90909090909091)(164.222222222222,-8.54545454545455)(373.232323232323,-7.81818181818182)(313.515151515152,-8.54545454545455)(373.232323232323,-8)(268.727272727273,-8.90909090909091)(298.585858585859,-8.72727272727273)(238.868686868687,-8)(59.7171717171717,-8)(134.363636363636,-8.18181818181818)(328.444444444444,-8.36363636363636)(477.737373737374,-9.27272727272727)(432.949494949495,-7.27272727272727)(403.090909090909,-7.45454545454545)(477.737373737374,-7.09090909090909)(462.808080808081,-7.09090909090909)(447.878787878788,-7.09090909090909)(283.656565656566,-7.63636363636364)(432.949494949495,-8.90909090909091)(447.878787878788,-8.18181818181818)(388.161616161616,-7.45454545454545)(612.10101010101,-8.18181818181818)(343.373737373737,-8)(358.30303030303,-7.81818181818182)(343.373737373737,-7.81818181818182)(582.242424242424,-7.09090909090909)(597.171717171717,-7.63636363636364)(567.313131313131,-7.45454545454545)(253.79797979798,-8.18181818181818)(477.737373737374,-8.72727272727273)(149.292929292929,-8)(149.292929292929,-8.36363636363636)(223.939393939394,-8)(253.79797979798,-8.36363636363636)(283.656565656566,-8.90909090909091)(358.30303030303,-8)(641.959595959596,-8.54545454545455)(179.151515151515,-8.54545454545455)(343.373737373737,-8.18181818181818)(447.878787878788,-8)(447.878787878788,-9.09090909090909)(149.292929292929,-7.45454545454545)(403.090909090909,-7.27272727272727)(358.30303030303,-7.63636363636364)(373.232323232323,-7.45454545454545)(612.10101010101,-8)(656.888888888889,-8)(164.222222222222,-8)(298.585858585859,-7.63636363636364)(74.6464646464647,-8.36363636363636)(492.666666666667,-7.45454545454545)(552.383838383838,-8.90909090909091)(507.59595959596,-8.90909090909091)(582.242424242424,-8.72727272727273)(418.020202020202,-8.72727272727273)(209.010101010101,-8)(373.232323232323,-8.18181818181818)(149.292929292929,-8.18181818181818)(179.151515151515,-8)(194.080808080808,-8)(343.373737373737,-7.63636363636364)(313.515151515152,-7.63636363636364)(656.888888888889,-8.18181818181818)(134.363636363636,-7.45454545454545)(119.434343434343,-7.45454545454545)(388.161616161616,-7.27272727272727)(358.30303030303,-7.45454545454545)(223.939393939394,-8.54545454545455)(358.30303030303,-8.18181818181818)(403.090909090909,-8.54545454545455)(432.949494949495,-7.09090909090909)(462.808080808081,-7.63636363636364)(238.868686868687,-8.18181818181818)(194.080808080808,-8.54545454545455)(597.171717171717,-8.54545454545455)(164.222222222222,-8.36363636363636)(164.222222222222,-8.18181818181818)(209.010101010101,-8.54545454545455)(238.868686868687,-8.36363636363636)(194.080808080808,-7.45454545454545)(462.808080808081,-8.36363636363636)(223.939393939394,-9.09090909090909)(343.373737373737,-7.45454545454545)(522.525252525253,-9.45454545454546)(223.939393939394,-8.18181818181818)(462.808080808081,-9.27272727272727)(343.373737373737,-8.36363636363636)(59.7171717171717,-7.81818181818182)(328.444444444444,-8.54545454545455)(179.151515151515,-8.18181818181818)(179.151515151515,-8.36363636363636)(522.525252525253,-8.90909090909091)(537.454545454546,-8.90909090909091)(462.808080808081,-8.18181818181818)(641.959595959596,-7.45454545454545)(627.030303030303,-7.27272727272727)(656.888888888889,-7.63636363636364)(597.171717171717,-7.09090909090909)(612.10101010101,-7.09090909090909)(641.959595959596,-7.27272727272727)(507.59595959596,-7.45454545454545)(552.383838383838,-7.45454545454545)(462.808080808081,-7.81818181818182)(522.525252525253,-7.45454545454545)(537.454545454546,-7.45454545454545)(477.737373737374,-7.63636363636364)(418.020202020202,-7.09090909090909)(582.242424242424,-9.27272727272727)(627.030303030303,-8.90909090909091)(492.666666666667,-8.72727272727273)(358.30303030303,-8.36363636363636)(537.454545454546,-9.45454545454546)(313.515151515152,-8.72727272727273)(209.010101010101,-8.18181818181818)(223.939393939394,-8.36363636363636)(194.080808080808,-8.18181818181818)(477.737373737374,-8.54545454545455)(149.292929292929,-8.90909090909091)(209.010101010101,-7.45454545454545)(656.888888888889,-7.45454545454545)(507.59595959596,-9.45454545454546)(552.383838383838,-9.45454545454546)(104.505050505051,-7.45454545454545)(89.5757575757576,-7.45454545454545)(223.939393939394,-7.45454545454545)(328.444444444444,-7.45454545454545)(238.868686868687,-7.45454545454545)(194.080808080808,-8.36363636363636)(492.666666666667,-9.45454545454546)(537.454545454546,-6.90909090909091)(522.525252525253,-6.90909090909091)(552.383838383838,-6.90909090909091)(507.59595959596,-6.90909090909091)(567.313131313131,-6.90909090909091)(567.313131313131,-4.72727272727273)(492.666666666667,-6.90909090909091)(462.808080808081,-8)(388.161616161616,-8.54545454545455)(343.373737373737,-8.54545454545455)(403.090909090909,-8.72727272727273)(104.505050505051,-8.72727272727273)(59.7171717171717,-7.63636363636364)(656.888888888889,-8.36363636363636)(582.242424242424,-7.63636363636364)(597.171717171717,-7.81818181818182)(238.868686868687,-9.09090909090909)(209.010101010101,-9.09090909090909)(253.79797979798,-9.09090909090909)(194.080808080808,-9.09090909090909)(298.585858585859,-8.90909090909091)(268.727272727273,-9.09090909090909)(418.020202020202,-8.90909090909091)(597.171717171717,-8.36363636363636)(582.242424242424,-6.90909090909091)(627.030303030303,-7.09090909090909)(328.444444444444,-8.72727272727273)(612.10101010101,-9.09090909090909)(373.232323232323,-7.27272727272727)(432.949494949495,-9.09090909090909)(492.666666666667,-7.63636363636364)(477.737373737374,-6.90909090909091)(313.515151515152,-8.90909090909091)(283.656565656566,-9.09090909090909)(179.151515151515,-9.09090909090909)(582.242424242424,-8.54545454545455)(597.171717171717,-8)(313.515151515152,-7.45454545454545)(134.363636363636,-8.90909090909091)(447.878787878788,-9.27272727272727)(343.373737373737,-8.72727272727273)(477.737373737374,-9.45454545454546)(477.737373737374,-7.81818181818182)(641.959595959596,-8.72727272727273)(298.585858585859,-7.45454545454545)(253.79797979798,-7.45454545454545)(283.656565656566,-7.45454545454545)(59.7171717171717,-8.18181818181818)(567.313131313131,-7.63636363636364)(597.171717171717,-8.18181818181818)(477.737373737374,-8.36363636363636)(462.808080808081,-6.90909090909091)(507.59595959596,-7.63636363636364)(507.59595959596,-8.72727272727273)(597.171717171717,-6.90909090909091)(298.585858585859,-9.09090909090909)(552.383838383838,-8.72727272727273)(418.020202020202,-9.09090909090909)(373.232323232323,-8.72727272727273) 
};

%\node at (axis cs:50, -2.5) [shape=circle,fill=white,draw=black,inner sep=0pt,anchor=south west] {\scriptsize\color{locol}$\*L_{\*t}$};
%\node at (axis cs:460, -5.9) [shape=circle,fill=white,draw=black,inner sep=0pt,anchor=south west] {\scriptsize\color{orange!50!yellow}$\*H_{\*t}$};
%\node at (axis cs:160, -5.2) [shape=circle,fill=white,draw=black,inner sep=0pt,anchor=south west] {\scriptsize\color{darkgray}$\*U_{\*t}$};

\node at (axis cs:1155, -17) [shape=circle,fill=red!40!yellow,draw=black,inner sep=0.2pt,anchor=south west,minimum size=16pt]
  {\scriptsize\color{white}$\hat{\*H}$};
\node at (axis cs:1330, -17) [shape=circle,fill=locol,draw=black,inner sep=0.2pt,anchor=south west,minimum size=16pt]
  {\scriptsize\color{white}$\hat{\*L}$};

\end{axis}
\end{tikzpicture}%

%% This file was created by matlab2tikz v0.2.3.
% Copyright (c) 2008--2012, Nico Schlömer <nico.schloemer@gmail.com>
% All rights reserved.
% 
% 
%

\definecolor{locol}{rgb}{0.26, 0.45, 0.65}

\begin{tikzpicture}

\begin{axis}[%
tick label style={font=\tiny},
label style={font=\tiny},
xlabel shift={-10pt},
ylabel shift={-17pt},
legend style={font=\tiny},
view={0}{90},
width=\figurewidth,
height=\figureheight,
scale only axis,
xmin=0, xmax=1478,
xtick={0, 400, 1000, 1400},
xlabel={Length (m)},
ymin=-18, ymax=0,
ytick={0, -4, -14, -18},
ylabel={Depth (m)},
name=plot1,
axis lines*=box,
tickwidth=0.1cm,
clip=false
]

\addplot [fill=locol,draw=none,forget plot] coordinates{ (1478,0)(1478,-0.181818181818182)(1478,-0.363636363636364)(1478,-0.545454545454545)(1478,-0.727272727272727)(1478,-0.909090909090909)(1478,-1.09090909090909)(1478,-1.27272727272727)(1478,-1.45454545454545)(1478,-1.63636363636364)(1478,-1.81818181818182)(1478,-2)(1478,-2.18181818181818)(1478,-2.36363636363636)(1478,-2.54545454545455)(1478,-2.72727272727273)(1478,-2.90909090909091)(1478,-3.09090909090909)(1478,-3.27272727272727)(1478,-3.45454545454545)(1478,-3.63636363636364)(1478,-3.81818181818182)(1478,-4)(1478,-4.18181818181818)(1478,-4.36363636363636)(1478,-4.54545454545455)(1478,-4.72727272727273)(1478,-4.90909090909091)(1478,-5.09090909090909)(1478,-5.27272727272727)(1478,-5.45454545454545)(1478,-5.63636363636364)(1478,-5.81818181818182)(1478,-6)(1478,-6.18181818181818)(1478,-6.36363636363636)(1478,-6.54545454545455)(1478,-6.72727272727273)(1478,-6.90909090909091)(1478,-7.09090909090909)(1478,-7.27272727272727)(1478,-7.45454545454545)(1478,-7.63636363636364)(1478,-7.81818181818182)(1478,-8)(1478,-8.18181818181818)(1478,-8.36363636363636)(1478,-8.54545454545455)(1478,-8.72727272727273)(1478,-8.90909090909091)(1478,-9.09090909090909)(1478,-9.27272727272727)(1478,-9.45454545454546)(1478,-9.63636363636364)(1478,-9.81818181818182)(1478,-10)(1478,-10.1818181818182)(1478,-10.3636363636364)(1478,-10.5454545454545)(1478,-10.7272727272727)(1478,-10.9090909090909)(1478,-11.0909090909091)(1478,-11.2727272727273)(1478,-11.4545454545455)(1478,-11.6363636363636)(1478,-11.8181818181818)(1478,-12)(1478,-12.1818181818182)(1478,-12.3636363636364)(1478,-12.5454545454545)(1478,-12.7272727272727)(1478,-12.9090909090909)(1478,-13.0909090909091)(1478,-13.2727272727273)(1478,-13.4545454545455)(1478,-13.6363636363636)(1478,-13.8181818181818)(1478,-14)(1478,-14.1818181818182)(1478,-14.3636363636364)(1478,-14.5454545454545)(1478,-14.7272727272727)(1478,-14.9090909090909)(1478,-15.0909090909091)(1478,-15.2727272727273)(1478,-15.4545454545455)(1478,-15.6363636363636)(1478,-15.8181818181818)(1478,-16)(1478,-16.1818181818182)(1478,-16.3636363636364)(1478,-16.5454545454545)(1478,-16.7272727272727)(1478,-16.9090909090909)(1478,-17.0909090909091)(1478,-17.2727272727273)(1478,-17.4545454545455)(1478,-17.6363636363636)(1478,-17.8181818181818)(1478,-18)(1463.07070707071,-18)(1448.14141414141,-18)(1433.21212121212,-18)(1418.28282828283,-18)(1403.35353535354,-18)(1388.42424242424,-18)(1373.49494949495,-18)(1358.56565656566,-18)(1343.63636363636,-18)(1328.70707070707,-18)(1313.77777777778,-18)(1298.84848484848,-18)(1283.91919191919,-18)(1268.9898989899,-18)(1254.06060606061,-18)(1239.13131313131,-18)(1224.20202020202,-18)(1209.27272727273,-18)(1194.34343434343,-18)(1179.41414141414,-18)(1164.48484848485,-18)(1149.55555555556,-18)(1134.62626262626,-18)(1119.69696969697,-18)(1104.76767676768,-18)(1089.83838383838,-18)(1074.90909090909,-18)(1059.9797979798,-18)(1045.05050505051,-18)(1030.12121212121,-18)(1015.19191919192,-18)(1000.26262626263,-18)(985.333333333333,-18)(970.40404040404,-18)(955.474747474747,-18)(940.545454545455,-18)(925.616161616162,-18)(910.686868686869,-18)(895.757575757576,-18)(880.828282828283,-18)(865.89898989899,-18)(850.969696969697,-18)(836.040404040404,-18)(821.111111111111,-18)(806.181818181818,-18)(791.252525252525,-18)(776.323232323232,-18)(761.393939393939,-18)(746.464646464646,-18)(731.535353535354,-18)(716.606060606061,-18)(701.676767676768,-18)(686.747474747475,-18)(671.818181818182,-18)(656.888888888889,-18)(641.959595959596,-18)(627.030303030303,-18)(612.10101010101,-18)(597.171717171717,-18)(582.242424242424,-18)(567.313131313131,-18)(552.383838383838,-18)(537.454545454546,-18)(522.525252525253,-18)(507.59595959596,-18)(492.666666666667,-18)(477.737373737374,-18)(462.808080808081,-18)(447.878787878788,-18)(432.949494949495,-18)(418.020202020202,-18)(403.090909090909,-18)(388.161616161616,-18)(373.232323232323,-18)(358.30303030303,-18)(343.373737373737,-18)(328.444444444444,-18)(313.515151515152,-18)(298.585858585859,-18)(283.656565656566,-18)(268.727272727273,-18)(253.79797979798,-18)(238.868686868687,-18)(223.939393939394,-18)(209.010101010101,-18)(194.080808080808,-18)(179.151515151515,-18)(164.222222222222,-18)(149.292929292929,-18)(134.363636363636,-18)(119.434343434343,-18)(104.505050505051,-18)(89.5757575757576,-18)(74.6464646464647,-18)(59.7171717171717,-18)(44.7878787878788,-18)(29.8585858585859,-18)(14.9292929292929,-18)(0,-18)(0,-17.8181818181818)(0,-17.6363636363636)(0,-17.4545454545455)(0,-17.2727272727273)(0,-17.0909090909091)(0,-16.9090909090909)(0,-16.7272727272727)(0,-16.5454545454545)(0,-16.3636363636364)(0,-16.1818181818182)(0,-16)(0,-15.8181818181818)(0,-15.6363636363636)(0,-15.4545454545455)(0,-15.2727272727273)(0,-15.0909090909091)(0,-14.9090909090909)(0,-14.7272727272727)(0,-14.5454545454545)(0,-14.3636363636364)(0,-14.1818181818182)(0,-14)(0,-13.8181818181818)(0,-13.6363636363636)(0,-13.4545454545455)(0,-13.2727272727273)(0,-13.0909090909091)(0,-12.9090909090909)(0,-12.7272727272727)(0,-12.5454545454545)(0,-12.3636363636364)(0,-12.1818181818182)(0,-12)(0,-11.8181818181818)(0,-11.6363636363636)(0,-11.4545454545455)(0,-11.2727272727273)(0,-11.0909090909091)(0,-10.9090909090909)(0,-10.7272727272727)(0,-10.5454545454545)(0,-10.3636363636364)(0,-10.1818181818182)(0,-10)(0,-9.81818181818182)(0,-9.63636363636364)(0,-9.45454545454546)(0,-9.27272727272727)(0,-9.09090909090909)(0,-8.90909090909091)(0,-8.72727272727273)(0,-8.54545454545455)(0,-8.36363636363636)(0,-8.18181818181818)(0,-8)(0,-7.81818181818182)(0,-7.63636363636364)(0,-7.45454545454545)(0,-7.27272727272727)(0,-7.09090909090909)(0,-6.90909090909091)(0,-6.72727272727273)(0,-6.54545454545455)(0,-6.36363636363636)(0,-6.18181818181818)(0,-6)(0,-5.81818181818182)(0,-5.63636363636364)(0,-5.45454545454545)(0,-5.27272727272727)(0,-5.09090909090909)(0,-4.90909090909091)(0,-4.72727272727273)(0,-4.54545454545455)(0,-4.36363636363636)(0,-4.18181818181818)(0,-4)(0,-3.81818181818182)(0,-3.63636363636364)(0,-3.45454545454545)(0,-3.27272727272727)(0,-3.09090909090909)(0,-2.90909090909091)(0,-2.72727272727273)(0,-2.54545454545455)(0,-2.36363636363636)(0,-2.18181818181818)(0,-2)(0,-1.81818181818182)(0,-1.63636363636364)(0,-1.45454545454545)(0,-1.27272727272727)(0,-1.09090909090909)(0,-0.909090909090909)(0,-0.727272727272727)(0,-0.545454545454545)(0,-0.363636363636364)(0,-0.181818181818182)(0,0)(14.9292929292929,0)(29.8585858585859,0)(44.7878787878788,0)(59.7171717171717,0)(74.6464646464647,0)(89.5757575757576,0)(104.505050505051,0)(119.434343434343,0)(134.363636363636,0)(149.292929292929,0)(164.222222222222,0)(179.151515151515,0)(194.080808080808,0)(209.010101010101,0)(223.939393939394,0)(238.868686868687,0)(253.79797979798,0)(268.727272727273,0)(283.656565656566,0)(298.585858585859,0)(313.515151515152,0)(328.444444444444,0)(343.373737373737,0)(358.30303030303,0)(373.232323232323,0)(388.161616161616,0)(403.090909090909,0)(418.020202020202,0)(432.949494949495,0)(447.878787878788,0)(462.808080808081,0)(477.737373737374,0)(492.666666666667,0)(507.59595959596,0)(522.525252525253,0)(537.454545454546,0)(552.383838383838,0)(567.313131313131,0)(582.242424242424,0)(597.171717171717,0)(612.10101010101,0)(627.030303030303,0)(641.959595959596,0)(656.888888888889,0)(671.818181818182,0)(686.747474747475,0)(701.676767676768,0)(716.606060606061,0)(731.535353535354,0)(746.464646464646,0)(761.393939393939,0)(776.323232323232,0)(791.252525252525,0)(806.181818181818,0)(821.111111111111,0)(836.040404040404,0)(850.969696969697,0)(865.89898989899,0)(880.828282828283,0)(895.757575757576,0)(910.686868686869,0)(925.616161616162,0)(940.545454545455,0)(955.474747474747,0)(970.40404040404,0)(985.333333333333,0)(1000.26262626263,0)(1015.19191919192,0)(1030.12121212121,0)(1045.05050505051,0)(1059.9797979798,0)(1074.90909090909,0)(1089.83838383838,0)(1104.76767676768,0)(1119.69696969697,0)(1134.62626262626,0)(1149.55555555556,0)(1164.48484848485,0)(1179.41414141414,0)(1194.34343434343,0)(1209.27272727273,0)(1224.20202020202,0)(1239.13131313131,0)(1254.06060606061,0)(1268.9898989899,0)(1283.91919191919,0)(1298.84848484848,0)(1313.77777777778,0)(1328.70707070707,0)(1343.63636363636,0)(1358.56565656566,0)(1373.49494949495,0)(1388.42424242424,0)(1403.35353535354,0)(1418.28282828283,0)(1433.21212121212,0)(1448.14141414141,0)(1463.07070707071,0)(1478,0)};

\addplot [fill=red!40!yellow,draw=none,forget plot] coordinates{ (1007.72727272727,-7.27272727272727)(1015.19191919192,-7.3030303030303)(1030.12121212121,-7.36363636363636)(1045.05050505051,-7.36363636363636)(1059.9797979798,-7.36363636363636)(1074.90909090909,-7.36363636363636)(1089.83838383838,-7.42424242424242)(1097.30303030303,-7.45454545454545)(1104.76767676768,-7.48484848484848)(1119.69696969697,-7.54545454545455)(1134.62626262626,-7.54545454545455)(1149.55555555556,-7.54545454545455)(1164.48484848485,-7.54545454545455)(1179.41414141414,-7.54545454545455)(1194.34343434343,-7.60606060606061)(1201.80808080808,-7.63636363636364)(1209.27272727273,-7.66666666666667)(1224.20202020202,-7.72727272727273)(1239.13131313131,-7.72727272727273)(1254.06060606061,-7.72727272727273)(1268.9898989899,-7.72727272727273)(1283.91919191919,-7.72727272727273)(1298.84848484848,-7.72727272727273)(1313.77777777778,-7.78787878787879)(1321.24242424242,-7.81818181818182)(1328.70707070707,-7.84848484848485)(1343.63636363636,-7.96969696969697)(1347.36868686869,-8)(1358.56565656566,-8.09090909090909)(1366.0303030303,-8.18181818181818)(1373.49494949495,-8.27272727272727)(1380.9595959596,-8.36363636363636)(1388.42424242424,-8.45454545454545)(1395.88888888889,-8.54545454545455)(1403.35353535354,-8.63636363636364)(1410.81818181818,-8.72727272727273)(1418.28282828283,-8.81818181818182)(1425.74747474747,-8.90909090909091)(1433.21212121212,-9)(1440.67676767677,-9.09090909090909)(1448.14141414141,-9.22727272727273)(1450.62962962963,-9.27272727272727)(1455.60606060606,-9.45454545454546)(1455.60606060606,-9.63636363636364)(1448.14141414141,-9.77272727272727)(1445.6531986532,-9.81818181818182)(1433.21212121212,-9.96969696969697)(1425.74747474747,-10)(1418.28282828283,-10.030303030303)(1403.35353535354,-10.0909090909091)(1388.42424242424,-10.0909090909091)(1373.49494949495,-10.0909090909091)(1358.56565656566,-10.0909090909091)(1343.63636363636,-10.0909090909091)(1328.70707070707,-10.0909090909091)(1313.77777777778,-10.0909090909091)(1298.84848484848,-10.0909090909091)(1283.91919191919,-10.0909090909091)(1268.9898989899,-10.0909090909091)(1254.06060606061,-10.0909090909091)(1239.13131313131,-10.0909090909091)(1224.20202020202,-10.0909090909091)(1209.27272727273,-10.0909090909091)(1194.34343434343,-10.0909090909091)(1179.41414141414,-10.0909090909091)(1164.48484848485,-10.0909090909091)(1149.55555555556,-10.0909090909091)(1134.62626262626,-10.0909090909091)(1119.69696969697,-10.0909090909091)(1104.76767676768,-10.0909090909091)(1089.83838383838,-10.0909090909091)(1074.90909090909,-10.0909090909091)(1059.9797979798,-10.0909090909091)(1045.05050505051,-10.0909090909091)(1030.12121212121,-10.0909090909091)(1015.19191919192,-10.0909090909091)(1000.26262626263,-10.0909090909091)(985.333333333333,-10.0909090909091)(970.40404040404,-10.030303030303)(962.939393939394,-10)(955.474747474747,-9.96969696969697)(940.545454545455,-9.90909090909091)(925.616161616162,-9.90909090909091)(910.686868686869,-9.90909090909091)(895.757575757576,-9.84848484848485)(888.292929292929,-9.81818181818182)(880.828282828283,-9.78787878787879)(865.89898989899,-9.72727272727273)(850.969696969697,-9.66666666666667)(843.505050505051,-9.63636363636364)(836.040404040404,-9.60606060606061)(821.111111111111,-9.48484848484848)(817.378787878788,-9.45454545454546)(806.181818181818,-9.36363636363636)(798.717171717172,-9.27272727272727)(791.252525252525,-9.13636363636364)(788.76430976431,-9.09090909090909)(783.787878787879,-8.90909090909091)(783.787878787879,-8.72727272727273)(788.76430976431,-8.54545454545455)(791.252525252525,-8.45454545454545)(793.740740740741,-8.36363636363636)(806.181818181818,-8.21212121212121)(808.670033670034,-8.18181818181818)(821.111111111111,-8.03030303030303)(824.843434343434,-8)(836.040404040404,-7.90909090909091)(843.505050505051,-7.81818181818182)(850.969696969697,-7.72727272727273)(862.166666666667,-7.63636363636364)(865.89898989899,-7.60606060606061)(880.828282828283,-7.48484848484848)(888.292929292929,-7.45454545454545)(895.757575757576,-7.42424242424242)(910.686868686869,-7.36363636363636)(925.616161616162,-7.36363636363636)(940.545454545455,-7.3030303030303)(948.010101010101,-7.27272727272727)(955.474747474747,-7.24242424242424)(970.40404040404,-7.18181818181818)(985.333333333333,-7.18181818181818)(1000.26262626263,-7.24242424242424)(1007.72727272727,-7.27272727272727)};

\addplot [fill=red!40!yellow,draw=none,forget plot] coordinates{ (589.707070707071,-7.27272727272727)(597.171717171717,-7.3030303030303)(612.10101010101,-7.42424242424242)(615.833333333333,-7.45454545454545)(627.030303030303,-7.59090909090909)(629.518518518518,-7.63636363636364)(634.494949494949,-7.81818181818182)(634.494949494949,-8)(634.494949494949,-8.18181818181818)(629.518518518518,-8.36363636363636)(627.030303030303,-8.40909090909091)(619.565656565657,-8.54545454545455)(612.10101010101,-8.63636363636364)(604.636363636364,-8.72727272727273)(597.171717171717,-8.81818181818182)(585.974747474747,-8.90909090909091)(582.242424242424,-8.93939393939394)(567.313131313131,-9)(552.383838383838,-9.06060606060606)(544.919191919192,-9.09090909090909)(537.454545454546,-9.12121212121212)(522.525252525253,-9.18181818181818)(507.59595959596,-9.12121212121212)(500.131313131313,-9.09090909090909)(492.666666666667,-9.06060606060606)(477.737373737374,-8.93939393939394)(470.272727272727,-8.90909090909091)(462.808080808081,-8.87878787878788)(447.878787878788,-8.75757575757576)(444.146464646465,-8.72727272727273)(432.949494949495,-8.59090909090909)(430.461279461279,-8.54545454545455)(420.508417508418,-8.36363636363636)(418.020202020202,-8.27272727272727)(415.531986531986,-8.18181818181818)(410.555555555556,-8)(415.531986531986,-7.81818181818182)(418.020202020202,-7.77272727272727)(425.484848484849,-7.63636363636364)(432.949494949495,-7.54545454545455)(440.414141414141,-7.45454545454545)(447.878787878788,-7.36363636363636)(459.075757575758,-7.27272727272727)(462.808080808081,-7.24242424242424)(477.737373737374,-7.18181818181818)(492.666666666667,-7.18181818181818)(507.59595959596,-7.18181818181818)(522.525252525253,-7.18181818181818)(537.454545454546,-7.18181818181818)(552.383838383838,-7.18181818181818)(567.313131313131,-7.18181818181818)(582.242424242424,-7.24242424242424)(589.707070707071,-7.27272727272727)};

\addplot [fill=red!40!yellow,draw=none,forget plot] coordinates{ (227.671717171717,-7.81818181818182)(238.868686868687,-7.90909090909091)(246.333333333333,-8)(253.79797979798,-8.13636363636364)(256.286195286195,-8.18181818181818)(261.262626262626,-8.36363636363636)(253.79797979798,-8.5)(251.309764309764,-8.54545454545455)(238.868686868687,-8.6969696969697)(231.40404040404,-8.72727272727273)(223.939393939394,-8.75757575757576)(209.010101010101,-8.81818181818182)(194.080808080808,-8.81818181818182)(179.151515151515,-8.81818181818182)(164.222222222222,-8.75757575757576)(160.489898989899,-8.72727272727273)(149.292929292929,-8.63636363636364)(141.828282828283,-8.54545454545455)(134.363636363636,-8.45454545454545)(126.89898989899,-8.36363636363636)(119.434343434343,-8.22727272727273)(116.946127946128,-8.18181818181818)(111.969696969697,-8)(119.434343434343,-7.86363636363636)(123.166666666667,-7.81818181818182)(134.363636363636,-7.72727272727273)(149.292929292929,-7.72727272727273)(164.222222222222,-7.72727272727273)(179.151515151515,-7.72727272727273)(194.080808080808,-7.72727272727273)(209.010101010101,-7.72727272727273)(223.939393939394,-7.78787878787879)(227.671717171717,-7.81818181818182)};

\addplot [
color=white,
draw=white,
only marks,
mark=x,
mark options={solid},
mark size=2.0pt,
line width=0.3pt,
forget plot
]
coordinates{
 (1343.63636363636,-12.5454545454545)(328.444444444444,-12)(701.676767676768,-17.8181818181818)(821.111111111111,-6.72727272727273)(298.585858585859,-5.63636363636364)(865.89898989899,-3.09090909090909)(1343.63636363636,-7.81818181818182)(1478,-5.81818181818182)(1059.9797979798,-9.27272727272727)(836.040404040404,-9.81818181818182)(1313.77777777778,-9.63636363636364)(1059.9797979798,-10.7272727272727)(1074.90909090909,-7.45454545454545)(836.040404040404,-8.36363636363636)(477.737373737374,-8.72727272727273)(134.363636363636,-8.54545454545455)(328.444444444444,-9.45454545454546)(0,-7.09090909090909)(358.30303030303,-7.63636363636364)(1478,-9.09090909090909)(0,-9.45454545454546)(641.959595959596,-8)(627.030303030303,-9.45454545454546)(0,0)(1478,0)(0,-18)(1224.20202020202,-7.27272727272727)(1478,-18)(1478,-10.1818181818182)(1194.34343434343,-10.1818181818182)(179.151515151515,-7.63636363636364)(1478,-8.18181818181818)(925.616161616162,-7.45454545454545)(492.666666666667,-7.81818181818182)(313.515151515152,-8.54545454545455)(955.474747474747,-10)(1313.77777777778,-10.1818181818182)(746.464646464646,-8.90909090909091)(1358.56565656566,-8.36363636363636)(537.454545454546,-7.63636363636364)(104.505050505051,-7.81818181818182)(29.8585858585859,-8)(149.292929292929,-7.27272727272727)(791.252525252525,-7.81818181818182)(1089.83838383838,-7.27272727272727)(1239.13131313131,-7.81818181818182)(582.242424242424,-8.72727272727273)(1119.69696969697,-10.1818181818182)(821.111111111111,-9.27272727272727)(1433.21212121212,-9.27272727272727)(462.808080808081,-8.90909090909091)(238.868686868687,-8.18181818181818)(910.686868686869,-7.45454545454545)(403.090909090909,-7.81818181818182)(1030.12121212121,-10)(1478,-3.45454545454545)(701.676767676768,-8.36363636363636)(1239.13131313131,-10.1818181818182)(1388.42424242424,-8.54545454545455)(1388.42424242424,-9.81818181818182)(164.222222222222,-8.36363636363636)(388.161616161616,-8.72727272727273)(895.757575757576,-9.63636363636364)(223.939393939394,-7.63636363636364)(1119.69696969697,-7.45454545454545)(552.383838383838,-7.81818181818182)(447.878787878788,-7.63636363636364)(1463.07070707071,-9.45454545454546)(1224.20202020202,-7.63636363636364)(671.818181818182,-8.72727272727273)(776.323232323232,-8)(985.333333333333,-7.27272727272727)(74.6464646464647,-8.18181818181818)(522.525252525253,-8.90909090909091)(776.323232323232,-9.27272727272727)(1030.12121212121,-7.27272727272727)(1388.42424242424,-10)(1059.9797979798,-10)(895.757575757576,-9.81818181818182)(597.171717171717,-7.81818181818182)(582.242424242424,-7.63636363636364)(238.868686868687,-8.36363636363636)(0,-4.54545454545455)(268.727272727273,-7.63636363636364)(1373.49494949495,-8.36363636363636)(253.79797979798,-8.54545454545455)(149.292929292929,-8.54545454545455)(850.969696969697,-7.63636363636364)(1194.34343434343,-7.63636363636364)(89.5757575757576,-7.63636363636364)(656.888888888889,-8.90909090909091)(388.161616161616,-8.36363636363636)(1403.35353535354,-10)(1358.56565656566,-10.1818181818182)(1463.07070707071,-9.63636363636364)(1478,-9.63636363636364)(1478,-9.81818181818182)(806.181818181818,-9.45454545454546)(1268.9898989899,-7.81818181818182)(507.59595959596,-8.90909090909091)(1015.19191919192,-10)(627.030303030303,-7.81818181818182)(1433.21212121212,-8.90909090909091)(462.808080808081,-7.63636363636364)(776.323232323232,-8.18181818181818)(880.828282828283,-9.81818181818182)(104.505050505051,-8.36363636363636)(283.656565656566,-7.81818181818182)(612.10101010101,-7.63636363636364)(582.242424242424,-8.90909090909091)(104.505050505051,-7.63636363636364)(418.020202020202,-8.54545454545455)(821.111111111111,-8)(880.828282828283,-7.63636363636364)(970.40404040404,-7.27272727272727)(1089.83838383838,-7.45454545454545)(537.454545454546,-7.45454545454545)(567.313131313131,-7.45454545454545)(477.737373737374,-7.45454545454545)(597.171717171717,-7.45454545454545)(537.454545454546,-7.27272727272727)(985.333333333333,-10)(1328.70707070707,-8)(1418.28282828283,-8.72727272727273)(761.393939393939,-8.90909090909091)(1373.49494949495,-10.1818181818182)(1328.70707070707,-10.1818181818182)(671.818181818182,-7.81818181818182)(776.323232323232,-8.72727272727273)(194.080808080808,-8.54545454545455)(836.040404040404,-9.63636363636364)(1388.42424242424,-10.1818181818182)(403.090909090909,-8.18181818181818)(850.969696969697,-7.81818181818182)(1164.48484848485,-10.1818181818182)(1298.84848484848,-7.81818181818182)(1313.77777777778,-7.81818181818182)(1209.27272727273,-7.63636363636364)(641.959595959596,-8.72727272727273)(164.222222222222,-7.63636363636364)(656.888888888889,-8.36363636363636)(522.525252525253,-7.27272727272727)(791.252525252525,-8.54545454545455)(462.808080808081,-8.72727272727273)(1463.07070707071,-9.27272727272727)(970.40404040404,-10)(432.949494949495,-7.63636363636364)(104.505050505051,-8.18181818181818)(223.939393939394,-7.81818181818182)(1448.14141414141,-9.09090909090909)(1463.07070707071,-9.81818181818182)(1463.07070707071,-10)(821.111111111111,-9.45454545454546)(1343.63636363636,-8)(537.454545454546,-8.90909090909091)(403.090909090909,-8.36363636363636)(462.808080808081,-7.45454545454545)(507.59595959596,-7.27272727272727)(492.666666666667,-7.27272727272727)(477.737373737374,-7.27272727272727)(612.10101010101,-7.45454545454545)(1104.76767676768,-7.45454545454545)(806.181818181818,-8.36363636363636)(223.939393939394,-8.36363636363636)(1403.35353535354,-8.54545454545455)(1388.42424242424,-8.36363636363636)(627.030303030303,-8.54545454545455)(552.383838383838,-8.90909090909091)(492.666666666667,-8.90909090909091)(418.020202020202,-7.63636363636364)(179.151515151515,-8.54545454545455)(164.222222222222,-8.54545454545455)(238.868686868687,-8.54545454545455)(1239.13131313131,-7.63636363636364)(477.737373737374,-8.90909090909091)(1209.27272727273,-10)(641.959595959596,-8.18181818181818)(89.5757575757576,-8)(194.080808080808,-8.72727272727273)(403.090909090909,-8)(776.323232323232,-8.90909090909091)(209.010101010101,-7.81818181818182)(268.727272727273,-8.18181818181818)(1298.84848484848,-10.1818181818182)(1000.26262626263,-10)(1343.63636363636,-10.1818181818182)(612.10101010101,-8.72727272727273)(656.888888888889,0)(627.030303030303,-7.63636363636364)(104.505050505051,-8)(447.878787878788,-7.45454545454545)(1179.41414141414,-10.1818181818182)(865.89898989899,-7.63636363636364)(806.181818181818,-8.18181818181818)(1000.26262626263,-7.27272727272727)(791.252525252525,-9.09090909090909)(1448.14141414141,-10)(1373.49494949495,-8.18181818181818)(253.79797979798,-8.36363636363636)(522.525252525253,-9.09090909090909)(537.454545454546,-9.09090909090909)(507.59595959596,-9.09090909090909)(238.868686868687,-8)(179.151515151515,-7.81818181818182)(119.434343434343,-8.18181818181818)(462.808080808081,-7.27272727272727)(955.474747474747,-7.27272727272727)(1015.19191919192,-7.27272727272727)(209.010101010101,-8.54545454545455)(179.151515151515,-8.72727272727273)(149.292929292929,-7.81818181818182)(1478,-9.45454545454546)(1433.21212121212,-10)(865.89898989899,-9.81818181818182)(1283.91919191919,-10.1818181818182)(134.363636363636,-7.81818181818182)(641.959595959596,-8.36363636363636)(418.020202020202,-8.36363636363636)(447.878787878788,-8.72727272727273)(940.545454545455,-7.27272727272727)(597.171717171717,-8.90909090909091)(432.949494949495,-7.45454545454545)(1254.06060606061,-7.63636363636364)(432.949494949495,-8.54545454545455)(223.939393939394,-8)(791.252525252525,-9.27272727272727)(223.939393939394,-8.54545454545455)(209.010101010101,-8.72727272727273)(164.222222222222,-8.72727272727273)(149.292929292929,-8.72727272727273)(223.939393939394,-8.72727272727273)(940.545454545455,-10)(119.434343434343,-8)(119.434343434343,-8.36363636363636)(0,-14.1818181818182)(1328.70707070707,-7.81818181818182)(1358.56565656566,-8.18181818181818)(791.252525252525,-8.36363636363636)(447.878787878788,-7.27272727272727)(432.949494949495,-7.27272727272727)(1134.62626262626,-7.45454545454545)(641.959595959596,-7.81818181818182)(164.222222222222,-7.81818181818182)(194.080808080808,-7.81818181818182)(253.79797979798,-8.18181818181818)(1268.9898989899,-10.1818181818182)(552.383838383838,-9.09090909090909)(492.666666666667,-9.09090909090909)(836.040404040404,-7.81818181818182)(836.040404040404,-8)(119.434343434343,-7.81818181818182)(776.323232323232,-9.09090909090909)(641.959595959596,-8.54545454545455)(432.949494949495,-8.72727272727273)(418.020202020202,-7.45454545454545)(418.020202020202,-8.18181818181818)(567.313131313131,-9.09090909090909)(418.020202020202,-7.81818181818182)(821.111111111111,-9.63636363636364)(1448.14141414141,-9.27272727272727)(1448.14141414141,-9.81818181818182)(1433.21212121212,-9.09090909090909)(149.292929292929,-7.63636363636364)(134.363636363636,-7.63636363636364)(119.434343434343,-7.63636363636364)(1134.62626262626,-10.1818181818182)(1149.55555555556,-10.1818181818182)(850.969696969697,-9.63636363636364)(1418.28282828283,-8.90909090909091)(1045.05050505051,-7.27272727272727)(641.959595959596,-7.63636363636364)(1403.35353535354,-10.1818181818182)(238.868686868687,-8.72727272727273)(194.080808080808,-8.90909090909091)(821.111111111111,-8.18181818181818)(776.323232323232,-8.54545454545455)(477.737373737374,-9.09090909090909)(447.878787878788,-8.90909090909091)(1418.28282828283,-10)(1045.05050505051,-10)(418.020202020202,-8)(1268.9898989899,-7.63636363636364)(925.616161616162,-10)(134.363636363636,-7.45454545454545)(268.727272727273,-8.36363636363636)(268.727272727273,-8.54545454545455)(1403.35353535354,-8.72727272727273)(627.030303030303,-8.72727272727273)(388.161616161616,-8)(134.363636363636,-8.36363636363636)(119.434343434343,-8.54545454545455)(89.5757575757576,-7.81818181818182)(761.393939393939,-14)(1224.20202020202,-10)(1104.76767676768,-10.1818181818182)(179.151515151515,-8.90909090909091)(1448.14141414141,-9.45454545454546)(194.080808080808,-7.63636363636364)(253.79797979798,-8)(238.868686868687,-7.81818181818182)(403.090909090909,-7.63636363636364)(582.242424242424,-9.09090909090909)(656.888888888889,-8)(791.252525252525,-8.72727272727273)(1179.41414141414,-7.63636363636364)(1254.06060606061,-10.1818181818182)(89.5757575757576,-8.18181818181818)(597.171717171717,-8.72727272727273)(209.010101010101,-8.90909090909091)(253.79797979798,-8.72727272727273)(223.939393939394,-8.90909090909091)(164.222222222222,-8.90909090909091)(432.949494949495,-8.36363636363636)(283.656565656566,-8.36363636363636)(134.363636363636,-8.72727272727273)(1283.91919191919,-7.81818181818182)(656.888888888889,-8.18181818181818)(1239.13131313131,-10)(1194.34343434343,-10)(388.161616161616,-8.18181818181818)(850.969696969697,-9.81818181818182)(806.181818181818,-9.27272727272727)(910.686868686869,-9.81818181818182)(1418.28282828283,-10.1818181818182)(1448.14141414141,-9.63636363636364)(1164.48484848485,-7.63636363636364)(1149.55555555556,-7.63636363636364)(1358.56565656566,-8)(1059.9797979798,-7.45454545454545)(209.010101010101,-7.63636363636364)(149.292929292929,-7.45454545454545)(119.434343434343,-7.45454545454545)(761.393939393939,-8.72727272727273)(164.222222222222,-7.45454545454545)(1089.83838383838,-10.1818181818182)(268.727272727273,-8)(283.656565656566,-8.18181818181818)(388.161616161616,-7.81818181818182)(373.232323232323,-8)(1254.06060606061,-10)(418.020202020202,-7.27272727272727)(403.090909090909,-7.45454545454545)(268.727272727273,-8.72727272727273)(612.10101010101,-8.90909090909091)(656.888888888889,-7.81818181818182)(791.252525252525,-9.45454545454546)(283.656565656566,-8.54545454545455)(253.79797979798,-7.81818181818182)(283.656565656566,-8)(447.878787878788,-8.54545454545455)(582.242424242424,-7.45454545454545)(776.323232323232,-8.36363636363636)(1074.90909090909,-10.1818181818182)(1059.9797979798,-10.1818181818182)(567.313131313131,-8.90909090909091)(761.393939393939,-9.09090909090909)(791.252525252525,-8.90909090909091)(656.888888888889,-7.63636363636364)(1268.9898989899,-10)(1283.91919191919,-10)(432.949494949495,-8.18181818181818)(403.090909090909,-8.54545454545455)(388.161616161616,-7.63636363636364)(1418.28282828283,-8.54545454545455)(865.89898989899,-7.81818181818182)(791.252525252525,-8.18181818181818)(74.6464646464647,-7.81818181818182)(1298.84848484848,-10)(1313.77777777778,-10)(1283.91919191919,-7.63636363636364)(1179.41414141414,-10)(1164.48484848485,-10)(761.393939393939,-8.54545454545455)(104.505050505051,-8.54545454545455) 
};

%\node at (axis cs:50, -2.5) [shape=circle,fill=white,draw=black,inner sep=0pt,anchor=south west] {\scriptsize\color{locol}$\*L_{\*t}$};
%\node at (axis cs:460, -5.9) [shape=circle,fill=white,draw=black,inner sep=0pt,anchor=south west] {\scriptsize\color{orange!50!yellow}$\*H_{\*t}$};
%\node at (axis cs:160, -5.2) [shape=circle,fill=white,draw=black,inner sep=0pt,anchor=south west] {\scriptsize\color{darkgray}$\*U_{\*t}$};

\node at (axis cs:1155, -17) [shape=circle,fill=red!40!yellow,draw=black,inner sep=0.2pt,anchor=south west,minimum size=16pt]
  {\scriptsize\color{white}$\hat{\*H}$};
\node at (axis cs:1330, -17) [shape=circle,fill=locol,draw=black,inner sep=0.2pt,anchor=south west,minimum size=16pt]
  {\scriptsize\color{white}$\hat{\*L}$};

\end{axis}
\end{tikzpicture}%

%% This file was created by matlab2tikz v0.2.3.
% Copyright (c) 2008--2012, Nico Schlömer <nico.schloemer@gmail.com>
% All rights reserved.
% 
% 
%

\definecolor{locol}{rgb}{0.26, 0.45, 0.65}

\begin{tikzpicture}

\begin{axis}[%
tick label style={font=\tiny},
label style={font=\tiny},
xlabel shift={-10pt},
ylabel shift={-17pt},
legend style={font=\tiny},
view={0}{90},
width=\figurewidth,
height=\figureheight,
scale only axis,
xmin=0, xmax=1478,
xtick={0, 400, 1000, 1400},
xlabel={Length (m)},
ymin=-18, ymax=0,
ytick={0, -4, -14, -18},
ylabel={Depth (m)},
name=plot1,
axis lines*=box,
tickwidth=0.1cm,
clip=false
]

\addplot [fill=locol,draw=none,forget plot] coordinates{ (1478,0)(1478,-0.181818181818182)(1478,-0.363636363636364)(1478,-0.545454545454545)(1478,-0.727272727272727)(1478,-0.909090909090909)(1478,-1.09090909090909)(1478,-1.27272727272727)(1478,-1.45454545454545)(1478,-1.63636363636364)(1478,-1.81818181818182)(1478,-2)(1478,-2.18181818181818)(1478,-2.36363636363636)(1478,-2.54545454545455)(1478,-2.72727272727273)(1478,-2.90909090909091)(1478,-3.09090909090909)(1478,-3.27272727272727)(1478,-3.45454545454545)(1478,-3.63636363636364)(1478,-3.81818181818182)(1478,-4)(1478,-4.18181818181818)(1478,-4.36363636363636)(1478,-4.54545454545455)(1478,-4.72727272727273)(1478,-4.90909090909091)(1478,-5.09090909090909)(1478,-5.27272727272727)(1478,-5.45454545454545)(1478,-5.63636363636364)(1478,-5.81818181818182)(1478,-6)(1478,-6.18181818181818)(1478,-6.36363636363636)(1478,-6.54545454545455)(1478,-6.72727272727273)(1478,-6.90909090909091)(1478,-7.09090909090909)(1478,-7.27272727272727)(1478,-7.45454545454545)(1478,-7.63636363636364)(1478,-7.81818181818182)(1478,-8)(1478,-8.18181818181818)(1478,-8.36363636363636)(1478,-8.54545454545455)(1478,-8.72727272727273)(1478,-8.90909090909091)(1478.00016587919,-9.09090909090909)(1478.00033175469,-9.27272727272727)(1478.00033175469,-9.45454545454546)(1478.00016587919,-9.63636363636364)(1478,-9.81818181818182)(1478,-10)(1478,-10.1818181818182)(1478,-10.3636363636364)(1478,-10.5454545454545)(1478,-10.7272727272727)(1478,-10.9090909090909)(1478,-11.0909090909091)(1478,-11.2727272727273)(1478,-11.4545454545455)(1478,-11.6363636363636)(1478,-11.8181818181818)(1478,-12)(1478,-12.1818181818182)(1478,-12.3636363636364)(1478,-12.5454545454545)(1478,-12.7272727272727)(1478,-12.9090909090909)(1478,-13.0909090909091)(1478,-13.2727272727273)(1478,-13.4545454545455)(1478,-13.6363636363636)(1478,-13.8181818181818)(1478,-14)(1478,-14.1818181818182)(1478,-14.3636363636364)(1478,-14.5454545454545)(1478,-14.7272727272727)(1478,-14.9090909090909)(1478,-15.0909090909091)(1478,-15.2727272727273)(1478,-15.4545454545455)(1478,-15.6363636363636)(1478,-15.8181818181818)(1478,-16)(1478,-16.1818181818182)(1478,-16.3636363636364)(1478,-16.5454545454545)(1478,-16.7272727272727)(1478,-16.9090909090909)(1478,-17.0909090909091)(1478,-17.2727272727273)(1478,-17.4545454545455)(1478,-17.6363636363636)(1478,-17.8181818181818)(1478,-18)(1463.07070707071,-18)(1448.14141414141,-18)(1433.21212121212,-18)(1418.28282828283,-18)(1403.35353535354,-18)(1388.42424242424,-18)(1373.49494949495,-18)(1358.56565656566,-18)(1343.63636363636,-18)(1328.70707070707,-18)(1313.77777777778,-18)(1298.84848484848,-18)(1283.91919191919,-18)(1268.9898989899,-18)(1254.06060606061,-18)(1239.13131313131,-18)(1224.20202020202,-18)(1209.27272727273,-18)(1194.34343434343,-18)(1179.41414141414,-18)(1164.48484848485,-18)(1149.55555555556,-18)(1134.62626262626,-18)(1119.69696969697,-18)(1104.76767676768,-18)(1089.83838383838,-18)(1074.90909090909,-18)(1059.9797979798,-18)(1045.05050505051,-18)(1030.12121212121,-18)(1015.19191919192,-18)(1000.26262626263,-18)(985.333333333333,-18)(970.40404040404,-18)(955.474747474747,-18)(940.545454545455,-18)(925.616161616162,-18)(910.686868686869,-18)(895.757575757576,-18)(880.828282828283,-18)(865.89898989899,-18)(850.969696969697,-18)(836.040404040404,-18)(821.111111111111,-18)(806.181818181818,-18)(791.252525252525,-18)(776.323232323232,-18)(761.393939393939,-18)(746.464646464646,-18)(731.535353535354,-18)(716.606060606061,-18)(701.676767676768,-18)(686.747474747475,-18)(671.818181818182,-18)(656.888888888889,-18)(641.959595959596,-18)(627.030303030303,-18)(612.10101010101,-18)(597.171717171717,-18)(582.242424242424,-18)(567.313131313131,-18)(552.383838383838,-18)(537.454545454546,-18)(522.525252525253,-18)(507.59595959596,-18)(492.666666666667,-18)(477.737373737374,-18)(462.808080808081,-18)(447.878787878788,-18)(432.949494949495,-18)(418.020202020202,-18)(403.090909090909,-18)(388.161616161616,-18)(373.232323232323,-18)(358.30303030303,-18)(343.373737373737,-18)(328.444444444444,-18)(313.515151515152,-18)(298.585858585859,-18)(283.656565656566,-18)(268.727272727273,-18)(253.79797979798,-18)(238.868686868687,-18)(223.939393939394,-18)(209.010101010101,-18)(194.080808080808,-18)(179.151515151515,-18)(164.222222222222,-18)(149.292929292929,-18)(134.363636363636,-18)(119.434343434343,-18)(104.505050505051,-18)(89.5757575757576,-18)(74.6464646464647,-18)(59.7171717171717,-18)(44.7878787878788,-18)(29.8585858585859,-18)(14.9292929292929,-18)(0,-18)(0,-17.8181818181818)(0,-17.6363636363636)(0,-17.4545454545455)(0,-17.2727272727273)(0,-17.0909090909091)(0,-16.9090909090909)(0,-16.7272727272727)(0,-16.5454545454545)(0,-16.3636363636364)(0,-16.1818181818182)(0,-16)(0,-15.8181818181818)(0,-15.6363636363636)(0,-15.4545454545455)(0,-15.2727272727273)(0,-15.0909090909091)(0,-14.9090909090909)(0,-14.7272727272727)(0,-14.5454545454545)(0,-14.3636363636364)(0,-14.1818181818182)(0,-14)(0,-13.8181818181818)(0,-13.6363636363636)(0,-13.4545454545455)(0,-13.2727272727273)(0,-13.0909090909091)(0,-12.9090909090909)(0,-12.7272727272727)(0,-12.5454545454545)(0,-12.3636363636364)(0,-12.1818181818182)(0,-12)(0,-11.8181818181818)(0,-11.6363636363636)(0,-11.4545454545455)(0,-11.2727272727273)(0,-11.0909090909091)(0,-10.9090909090909)(0,-10.7272727272727)(0,-10.5454545454545)(0,-10.3636363636364)(0,-10.1818181818182)(0,-10)(0,-9.81818181818182)(0,-9.63636363636364)(0,-9.45454545454546)(0,-9.27272727272727)(0,-9.09090909090909)(0,-8.90909090909091)(0,-8.72727272727273)(0,-8.54545454545455)(0,-8.36363636363636)(0,-8.18181818181818)(0,-8)(0,-7.81818181818182)(0,-7.63636363636364)(0,-7.45454545454545)(0,-7.27272727272727)(0,-7.09090909090909)(0,-6.90909090909091)(0,-6.72727272727273)(0,-6.54545454545455)(0,-6.36363636363636)(0,-6.18181818181818)(0,-6)(0,-5.81818181818182)(0,-5.63636363636364)(0,-5.45454545454545)(0,-5.27272727272727)(0,-5.09090909090909)(0,-4.90909090909091)(0,-4.72727272727273)(0,-4.54545454545455)(0,-4.36363636363636)(0,-4.18181818181818)(0,-4)(0,-3.81818181818182)(0,-3.63636363636364)(0,-3.45454545454545)(0,-3.27272727272727)(0,-3.09090909090909)(0,-2.90909090909091)(0,-2.72727272727273)(0,-2.54545454545455)(0,-2.36363636363636)(0,-2.18181818181818)(0,-2)(0,-1.81818181818182)(0,-1.63636363636364)(0,-1.45454545454545)(0,-1.27272727272727)(0,-1.09090909090909)(0,-0.909090909090909)(0,-0.727272727272727)(0,-0.545454545454545)(0,-0.363636363636364)(0,-0.181818181818182)(0,0)(14.9292929292929,0)(29.8585858585859,0)(44.7878787878788,0)(59.7171717171717,0)(74.6464646464647,0)(89.5757575757576,0)(104.505050505051,0)(119.434343434343,0)(134.363636363636,0)(149.292929292929,0)(164.222222222222,0)(179.151515151515,0)(194.080808080808,0)(209.010101010101,0)(223.939393939394,0)(238.868686868687,0)(253.79797979798,0)(268.727272727273,0)(283.656565656566,0)(298.585858585859,0)(313.515151515152,0)(328.444444444444,0)(343.373737373737,0)(358.30303030303,0)(373.232323232323,0)(388.161616161616,0)(403.090909090909,0)(418.020202020202,0)(432.949494949495,0)(447.878787878788,0)(462.808080808081,0)(477.737373737374,0)(492.666666666667,0)(507.59595959596,0)(522.525252525253,0)(537.454545454546,0)(552.383838383838,0)(567.313131313131,0)(582.242424242424,0)(597.171717171717,0)(612.10101010101,0)(627.030303030303,0)(641.959595959596,0)(656.888888888889,0)(671.818181818182,0)(686.747474747475,0)(701.676767676768,0)(716.606060606061,0)(731.535353535354,0)(746.464646464646,0)(761.393939393939,0)(776.323232323232,0)(791.252525252525,0)(806.181818181818,0)(821.111111111111,0)(836.040404040404,0)(850.969696969697,0)(865.89898989899,0)(880.828282828283,0)(895.757575757576,0)(910.686868686869,0)(925.616161616162,0)(940.545454545455,0)(955.474747474747,0)(970.40404040404,0)(985.333333333333,0)(1000.26262626263,0)(1015.19191919192,0)(1030.12121212121,0)(1045.05050505051,0)(1059.9797979798,0)(1074.90909090909,0)(1089.83838383838,0)(1104.76767676768,0)(1119.69696969697,0)(1134.62626262626,0)(1149.55555555556,0)(1164.48484848485,0)(1179.41414141414,0)(1194.34343434343,0)(1209.27272727273,0)(1224.20202020202,0)(1239.13131313131,0)(1254.06060606061,0)(1268.9898989899,0)(1283.91919191919,0)(1298.84848484848,0)(1313.77777777778,0)(1328.70707070707,0)(1343.63636363636,0)(1358.56565656566,0)(1373.49494949495,0)(1388.42424242424,0)(1403.35353535354,0)(1418.28282828283,0)(1433.21212121212,0)(1448.14141414141,0)(1463.07070707071,0)(1478,0)};

\addplot [fill=red!40!yellow,draw=none,forget plot] coordinates{ (1216.73737373737,-7.63636363636364)(1224.20202020202,-7.66666666666667)(1239.13131313131,-7.72727272727273)(1254.06060606061,-7.72727272727273)(1268.9898989899,-7.72727272727273)(1283.91919191919,-7.78787878787879)(1291.38383838384,-7.81818181818182)(1298.84848484848,-7.84848484848485)(1313.77777777778,-7.90909090909091)(1328.70707070707,-7.96969696969697)(1336.17171717172,-8)(1343.63636363636,-8.03030303030303)(1358.56565656566,-8.15151515151515)(1362.29797979798,-8.18181818181818)(1373.49494949495,-8.27272727272727)(1384.69191919192,-8.36363636363636)(1388.42424242424,-8.39393939393939)(1403.35353535354,-8.51515151515152)(1407.08585858586,-8.54545454545455)(1418.28282828283,-8.63636363636364)(1425.74747474747,-8.72727272727273)(1433.21212121212,-8.81818181818182)(1440.67676767677,-8.90909090909091)(1448.14141414141,-9)(1455.60606060606,-9.09090909090909)(1463.07070707071,-9.22727272727273)(1465.55892255892,-9.27272727272727)(1465.55892255892,-9.45454545454546)(1463.07070707071,-9.5)(1455.60606060606,-9.63636363636364)(1448.14141414141,-9.72727272727273)(1436.94444444444,-9.81818181818182)(1433.21212121212,-9.84848484848485)(1418.28282828283,-9.96969696969697)(1410.81818181818,-10)(1403.35353535354,-10.030303030303)(1388.42424242424,-10.0909090909091)(1373.49494949495,-10.0909090909091)(1358.56565656566,-10.0909090909091)(1343.63636363636,-10.0909090909091)(1328.70707070707,-10.0909090909091)(1313.77777777778,-10.0909090909091)(1298.84848484848,-10.0909090909091)(1283.91919191919,-10.0909090909091)(1268.9898989899,-10.0909090909091)(1254.06060606061,-10.0909090909091)(1239.13131313131,-10.0909090909091)(1224.20202020202,-10.0909090909091)(1209.27272727273,-10.0909090909091)(1194.34343434343,-10.0909090909091)(1179.41414141414,-10.0909090909091)(1164.48484848485,-10.0909090909091)(1149.55555555556,-10.0909090909091)(1134.62626262626,-10.0909090909091)(1119.69696969697,-10.0909090909091)(1104.76767676768,-10.0909090909091)(1089.83838383838,-10.0909090909091)(1074.90909090909,-10.0909090909091)(1059.9797979798,-10.0909090909091)(1045.05050505051,-10.0909090909091)(1030.12121212121,-10.0909090909091)(1015.19191919192,-10.0909090909091)(1000.26262626263,-10.0909090909091)(985.333333333333,-10.0909090909091)(970.40404040404,-10.030303030303)(962.939393939394,-10)(955.474747474747,-9.96969696969697)(940.545454545455,-9.90909090909091)(925.616161616162,-9.90909090909091)(910.686868686869,-9.84848484848485)(903.222222222222,-9.81818181818182)(895.757575757576,-9.78787878787879)(880.828282828283,-9.72727272727273)(865.89898989899,-9.66666666666667)(858.434343434343,-9.63636363636364)(850.969696969697,-9.60606060606061)(836.040404040404,-9.48484848484848)(832.308080808081,-9.45454545454546)(821.111111111111,-9.36363636363636)(813.646464646465,-9.27272727272727)(806.181818181818,-9.18181818181818)(798.717171717172,-9.09090909090909)(791.252525252525,-9)(783.787878787879,-8.90909090909091)(776.323232323232,-8.77272727272727)(773.835016835017,-8.72727272727273)(773.835016835017,-8.54545454545455)(776.323232323232,-8.45454545454545)(778.811447811448,-8.36363636363636)(791.252525252525,-8.21212121212121)(793.740740740741,-8.18181818181818)(806.181818181818,-8.03030303030303)(813.646464646465,-8)(821.111111111111,-7.96969696969697)(836.040404040404,-7.84848484848485)(843.505050505051,-7.81818181818182)(850.969696969697,-7.78787878787879)(865.89898989899,-7.72727272727273)(880.828282828283,-7.66666666666667)(888.292929292929,-7.63636363636364)(895.757575757576,-7.60606060606061)(910.686868686869,-7.54545454545455)(925.616161616162,-7.54545454545455)(940.545454545455,-7.54545454545455)(955.474747474747,-7.54545454545455)(970.40404040404,-7.54545454545455)(985.333333333333,-7.54545454545455)(1000.26262626263,-7.54545454545455)(1015.19191919192,-7.54545454545455)(1030.12121212121,-7.54545454545455)(1045.05050505051,-7.54545454545455)(1059.9797979798,-7.54545454545455)(1074.90909090909,-7.54545454545455)(1089.83838383838,-7.54545454545455)(1104.76767676768,-7.54545454545455)(1119.69696969697,-7.54545454545455)(1134.62626262626,-7.54545454545455)(1149.55555555556,-7.54545454545455)(1164.48484848485,-7.54545454545455)(1179.41414141414,-7.54545454545455)(1194.34343434343,-7.54545454545455)(1209.27272727273,-7.60606060606061)(1216.73737373737,-7.63636363636364)};

\addplot [fill=red!40!yellow,draw=none,forget plot] coordinates{ (559.848484848485,-7.45454545454545)(567.313131313131,-7.48484848484848)(582.242424242424,-7.60606060606061)(585.974747474747,-7.63636363636364)(597.171717171717,-7.77272727272727)(599.659932659933,-7.81818181818182)(609.612794612795,-8)(609.612794612795,-8.18181818181818)(604.636363636364,-8.36363636363636)(597.171717171717,-8.45454545454545)(585.974747474747,-8.54545454545455)(582.242424242424,-8.57575757575757)(567.313131313131,-8.63636363636364)(552.383838383838,-8.63636363636364)(537.454545454546,-8.63636363636364)(522.525252525253,-8.63636363636364)(507.59595959596,-8.63636363636364)(492.666666666667,-8.63636363636364)(477.737373737374,-8.63636363636364)(462.808080808081,-8.63636363636364)(447.878787878788,-8.63636363636364)(432.949494949495,-8.63636363636364)(418.020202020202,-8.63636363636364)(403.090909090909,-8.63636363636364)(388.161616161616,-8.57575757575757)(380.69696969697,-8.54545454545455)(373.232323232323,-8.51515151515152)(358.30303030303,-8.45454545454545)(343.373737373737,-8.45454545454545)(328.444444444444,-8.45454545454545)(313.515151515152,-8.45454545454545)(298.585858585859,-8.45454545454545)(283.656565656566,-8.45454545454545)(268.727272727273,-8.51515151515152)(261.262626262626,-8.54545454545455)(253.79797979798,-8.57575757575757)(238.868686868687,-8.57575757575757)(231.40404040404,-8.54545454545455)(223.939393939394,-8.51515151515152)(209.010101010101,-8.45454545454545)(194.080808080808,-8.39393939393939)(190.348484848485,-8.36363636363636)(179.151515151515,-8.22727272727273)(176.6632996633,-8.18181818181818)(179.151515151515,-8.09090909090909)(182.883838383838,-8)(194.080808080808,-7.90909090909091)(209.010101010101,-7.90909090909091)(223.939393939394,-7.90909090909091)(238.868686868687,-7.90909090909091)(253.79797979798,-7.90909090909091)(268.727272727273,-7.90909090909091)(283.656565656566,-7.90909090909091)(298.585858585859,-7.90909090909091)(313.515151515152,-7.90909090909091)(328.444444444444,-7.90909090909091)(343.373737373737,-7.90909090909091)(358.30303030303,-7.84848484848485)(365.767676767677,-7.81818181818182)(373.232323232323,-7.78787878787879)(388.161616161616,-7.66666666666667)(395.626262626263,-7.63636363636364)(403.090909090909,-7.60606060606061)(418.020202020202,-7.54545454545455)(432.949494949495,-7.48484848484848)(440.414141414141,-7.45454545454545)(447.878787878788,-7.42424242424242)(462.808080808081,-7.36363636363636)(477.737373737374,-7.36363636363636)(492.666666666667,-7.36363636363636)(507.59595959596,-7.36363636363636)(522.525252525253,-7.36363636363636)(537.454545454546,-7.36363636363636)(552.383838383838,-7.42424242424242)(559.848484848485,-7.45454545454545)};

\addplot [
color=white,
draw=white,
only marks,
mark=x,
mark options={solid},
mark size=2.0pt,
line width=0.3pt,
forget plot
]
coordinates{
 (477.737373737374,-17.4545454545455)(1478,0)(0,-5.81818181818182)(1478,-10.1818181818182)(1478,-18)(418.020202020202,0)(0,-12.1818181818182)(940.545454545455,-5.45454545454545)(791.252525252525,-12.1818181818182)(1478,-4.72727272727273)(0,-18)(1478,-14.1818181818182)(462.808080808081,-8.54545454545455)(0,-1.81818181818182)(940.545454545455,-1.27272727272727)(970.40404040404,-18)(462.808080808081,-4)(1015.19191919192,-8.90909090909091)(358.30303030303,-14.3636363636364)(0,-9.09090909090909)(985.333333333333,-14.9090909090909)(1478,-7.45454545454545)(0,-15.2727272727273)(0,0)(388.161616161616,-11.2727272727273)(1194.34343434343,-12.1818181818182)(1283.91919191919,-2.54545454545455)(567.313131313131,-6.36363636363636)(1134.62626262626,0)(1283.91919191919,-16.1818181818182)(731.535353535354,0)(671.818181818182,-15.4545454545455)(776.323232323232,-3.27272727272727)(731.535353535354,-9.81818181818182)(358.30303030303,-2)(0,-3.81818181818182)(1209.27272727273,-6.54545454545455)(238.868686868687,-6.90909090909091)(1478,-12.1818181818182)(1478,-2.18181818181818)(223.939393939394,-16.5454545454545)(821.111111111111,-7.45454545454545)(716.606060606061,-18)(1149.55555555556,-4.18181818181818)(582.242424242424,-13.2727272727273)(1179.41414141414,-10.3636363636364)(1224.20202020202,-18)(209.010101010101,-10)(238.868686868687,-4.90909090909091)(268.727272727273,-18)(627.030303030303,-1.63636363636364)(1478,-16.1818181818182)(194.080808080808,-13.0909090909091)(1283.91919191919,-8.54545454545455)(1209.27272727273,-14)(0,-7.45454545454545)(910.686868686869,-16.5454545454545)(0,-13.8181818181818)(209.010101010101,0)(940.545454545455,-13.2727272727273)(955.474747474747,-10.9090909090909)(0,-10.7272727272727)(701.676767676768,-4.90909090909091)(1015.19191919192,-2.90909090909091)(194.080808080808,-3.09090909090909)(1283.91919191919,-1.09090909090909)(1478,-6)(597.171717171717,-11.0909090909091)(0,-16.7272727272727)(447.878787878788,-15.8181818181818)(223.939393939394,-8.36363636363636)(1030.12121212121,-7.27272727272727)(1478,-8.90909090909091)(925.616161616162,0)(761.393939393939,-14.1818181818182)(1478,-3.45454545454545)(462.808080808081,-9.81818181818182)(671.818181818182,-8.36363636363636)(1283.91919191919,-5.09090909090909)(179.151515151515,-1.27272727272727)(179.151515151515,-14.9090909090909)(403.090909090909,-5.63636363636364)(1104.76767676768,-16.5454545454545)(194.080808080808,-11.6363636363636)(403.090909090909,-12.7272727272727)(686.747474747475,-16.7272727272727)(552.383838383838,-2.90909090909091)(1343.63636363636,-11.0909090909091)(1313.77777777778,0)(925.616161616162,-4.18181818181818)(432.949494949495,-7.27272727272727)(1358.56565656566,-13.0909090909091)(1313.77777777778,-15.0909090909091)(776.323232323232,-6.18181818181818)(1478,-1.09090909090909)(1119.69696969697,-1.63636363636364)(567.313131313131,0)(522.525252525253,-18)(806.181818181818,-2)(1015.19191919192,-12.1818181818182)(537.454545454546,-14.5454545454545)(850.969696969697,-8.90909090909091)(447.878787878788,-1.09090909090909)(0,-4.72727272727273)(1358.56565656566,-17.2727272727273)(1104.76767676768,-5.63636363636364)(1313.77777777778,-7.27272727272727)(0,-2.72727272727273)(1313.77777777778,-3.81818181818182)(1313.77777777778,-9.63636363636364)(149.292929292929,-6)(134.363636363636,-18)(791.252525252525,-10.9090909090909)(1478,-15.2727272727273)(597.171717171717,-12.1818181818182)(1134.62626262626,-15.2727272727273)(836.040404040404,-15.4545454545455)(313.515151515152,-3.63636363636364)(0,-0.909090909090909)(1134.62626262626,-8.18181818181818)(552.383838383838,-4.90909090909091)(1478,-11.2727272727273)(1000.26262626263,-10)(627.030303030303,-7.27272727272727)(1089.83838383838,-13.4545454545455)(1478,-13.2727272727273)(761.393939393939,-0.909090909090909)(836.040404040404,-18)(1104.76767676768,-18)(313.515151515152,-9.27272727272727)(373.232323232323,-16.7272727272727)(1478,-17.0909090909091)(656.888888888889,-3.81818181818182)(89.5757575757576,-8.18181818181818)(134.363636363636,-4.18181818181818)(597.171717171717,-9.45454545454546)(298.585858585859,-15.4545454545455)(761.393939393939,-13.0909090909091)(0,-9.81818181818182)(940.545454545455,-6.54545454545455)(0,-12.9090909090909)(0,-6.72727272727273)(1119.69696969697,-11.2727272727273)(1149.55555555556,-3.09090909090909)(164.222222222222,-14)(1343.63636363636,-6)(104.505050505051,-16)(164.222222222222,-2.18181818181818)(1343.63636363636,-18)(373.232323232323,-18)(149.292929292929,-10.7272727272727)(567.313131313131,-16.3636363636364)(298.585858585859,-0.727272727272727)(1149.55555555556,-9.27272727272727)(1074.90909090909,-0.727272727272727)(418.020202020202,-2.72727272727273)(403.090909090909,-13.6363636363636)(343.373737373737,-10.5454545454545)(1478,-8.18181818181818)(119.434343434343,0)(910.686868686869,-14.1818181818182)(1328.70707070707,-12.1818181818182)(343.373737373737,-6.36363636363636)(910.686868686869,-2.54545454545455)(925.616161616162,-8)(0,-11.4545454545455)(0,-14.5454545454545)(298.585858585859,-12.1818181818182)(821.111111111111,-4.72727272727273)(821.111111111111,-17.0909090909091)(1478,-6.72727272727273)(134.363636363636,-17.0909090909091)(328.444444444444,-7.81818181818182)(1194.34343434343,-17.0909090909091)(1343.63636363636,-14.1818181818182)(1045.05050505051,-4.54545454545455)(134.363636363636,-9.27272727272727)(0,-8.36363636363636)(1358.56565656566,-1.81818181818182)(671.818181818182,-2.54545454545455)(373.232323232323,-4.72727272727273)(582.242424242424,-0.909090909090909)(1000.26262626263,-15.8181818181818)(865.89898989899,-9.81818181818182)(134.363636363636,-7.27272727272727)(656.888888888889,-5.81818181818182)(1015.19191919192,0)(895.757575757576,-11.8181818181818)(1478,-2.90909090909091)(627.030303030303,-18)(477.737373737374,-11.8181818181818)(0,-16)(1015.19191919192,-17.0909090909091)(582.242424242424,-10.3636363636364)(641.959595959596,-14.1818181818182)(1478,-5.27272727272727)(328.444444444444,0)(1478,-9.63636363636364)(119.434343434343,-12.3636363636364)(821.111111111111,0)(1463.07070707071,0)(1209.27272727273,-13.0909090909091)(1164.48484848485,-7.27272727272727)(0,-17.4545454545455)(552.383838383838,-8)(1089.83838383838,-14.3636363636364)(1239.13131313131,0)(119.434343434343,-5.27272727272727)(1478,-4.18181818181818)(477.737373737374,-15.0909090909091)(701.676767676768,-11.6363636363636)(507.59595959596,-2)(1074.90909090909,-2.18181818181818)(1328.70707070707,-10.3636363636364)(910.686868686869,-3.45454545454545)(1343.63636363636,-8)(1463.07070707071,-18)(731.535353535354,-7.09090909090909)(0,-3.27272727272727)(1059.9797979798,-6.36363636363636)(776.323232323232,-16)(1328.70707070707,-4.54545454545455)(761.393939393939,-8.54545454545455)(1343.63636363636,-3.09090909090909)(14.9292929292929,0)(283.656565656566,-13.4545454545455)(1478,-14.7272727272727)(1478,-0.909090909090909)(1373.49494949495,-16)(104.505050505051,-0.909090909090909)(627.030303030303,0)(1179.41414141414,-5.27272727272727)(283.656565656566,-17.2727272727273)(1478,-12.7272727272727)(283.656565656566,-2.54545454545455)(597.171717171717,-17.0909090909091)(477.737373737374,-6.36363636363636)(1239.13131313131,-11.2727272727273)(0,-5.27272727272727)(880.828282828283,-0.909090909090909)(268.727272727273,-5.81818181818182)(1478,-10.9090909090909)(836.040404040404,-5.63636363636364)(1179.41414141414,-15.8181818181818)(1194.34343434343,-1.09090909090909)(552.383838383838,-3.81818181818182)(1059.9797979798,-12.7272727272727)(776.323232323232,-14.9090909090909)(14.9292929292929,-18)(388.161616161616,-8.90909090909091)(507.59595959596,-12.9090909090909)(865.89898989899,-12.9090909090909)(477.737373737374,-10.7272727272727)(1074.90909090909,-10.3636363636364)(134.363636363636,-3.45454545454545)(761.393939393939,-4)(1358.56565656566,-0.727272727272727)(1059.9797979798,-3.63636363636364)(194.080808080808,-15.6363636363636)(1358.56565656566,-9.09090909090909)(298.585858585859,-1.45454545454545)(925.616161616162,-18)(253.79797979798,-11.0909090909091)(716.606060606061,-10.5454545454545)(0,-2)(1478,-16.7272727272727)(671.818181818182,-12.9090909090909)(1358.56565656566,-6.72727272727273)(522.525252525253,-5.45454545454545)(597.171717171717,-15.4545454545455)(1209.27272727273,-2.18181818181818)(0,-13.4545454545455)(0,-10.1818181818182)(582.242424242424,-8.90909090909091)(268.727272727273,-4.18181818181818)(1030.12121212121,-8.18181818181818)(477.737373737374,0)(1478,-1.81818181818182)(1224.20202020202,-14.7272727272727)(179.151515151515,-18)(283.656565656566,-14.5454545454545)(0,-6.36363636363636)(731.535353535354,-1.63636363636364)(970.40404040404,-1.81818181818182)(104.505050505051,-14.5454545454545)(432.949494949495,-18)(1030.12121212121,-11.4545454545455)(880.828282828283,-6.90909090909091)(1119.69696969697,-18)(492.666666666667,-14)(358.30303030303,-16)(1298.84848484848,-16.9090909090909)(1224.20202020202,-3.63636363636364)(104.505050505051,-11.4545454545455)(403.090909090909,-3.27272727272727)(1000.26262626263,-5.09090909090909)(1194.34343434343,-9.45454545454546)(880.828282828283,-10.5454545454545)(343.373737373737,-10)(0,-7.81818181818182)(1478,-13.8181818181818)(134.363636363636,-6.54545454545455)(119.434343434343,-13.2727272727273)(0,-4.36363636363636)(955.474747474747,-9.27272727272727)(238.868686868687,-7.63636363636364)(507.59595959596,-7.27272727272727)(477.737373737374,-16.7272727272727)(0,-1.09090909090909)(194.080808080808,-8.90909090909091)(1313.77777777778,-13.4545454545455)(1239.13131313131,-6)(1298.84848484848,-18)(1000.26262626263,-14)(746.464646464646,-17.2727272727273)(104.505050505051,-2.36363636363636)(1478,-11.8181818181818)(776.323232323232,-2.72727272727273)(1209.27272727273,-8)(1478,-15.6363636363636)(731.535353535354,-9.27272727272727)(119.434343434343,-10)(925.616161616162,-15.4545454545455)(821.111111111111,-13.8181818181818)(1358.56565656566,0)(1373.49494949495,-5.27272727272727)(1478,-7.81818181818182)(0,-16.9090909090909)(671.818181818182,-0.727272727272727)(1104.76767676768,0)(283.656565656566,-12.5454545454545)(432.949494949495,-0.909090909090909)(0,-15.0909090909091)(627.030303030303,-4.54545454545455)(238.868686868687,0)(1343.63636363636,-11.8181818181818)(746.464646464646,-7.81818181818182)(1478,-6.36363636363636)(940.545454545455,-17.0909090909091)(671.818181818182,-6.54545454545455)(0,-12)(0,-8.90909090909091)(776.323232323232,-18)(432.949494949495,-4.72727272727273)(1478,-9.09090909090909)(403.090909090909,-8)(1358.56565656566,-15.0909090909091)(627.030303030303,-3.09090909090909)(1478,-3.81818181818182)(343.373737373737,-6.90909090909091)(880.828282828283,0)(836.040404040404,-11.4545454545455)(418.020202020202,-12)(1015.19191919192,-0.909090909090909)(1149.55555555556,-12.1818181818182)(388.161616161616,-15.0909090909091)(1194.34343434343,-4.54545454545455)(134.363636363636,-4.72727272727273)(656.888888888889,-14.7272727272727)(1119.69696969697,-17.0909090909091)(582.242424242424,-11.6363636363636)(597.171717171717,-18)(731.535353535354,-5.45454545454545)(1059.9797979798,-16)(492.666666666667,-9.45454545454546)(1373.49494949495,-2.54545454545455)(134.363636363636,-16.7272727272727)(686.747474747475,-16.1818181818182)(985.333333333333,-6)(552.383838383838,-2)(880.828282828283,-8.18181818181818)(940.545454545455,-12.5454545454545)(671.818181818182,-13.6363636363636)(1478,-10.3636363636364)(1104.76767676768,-6.90909090909091)(0,-10.9090909090909)(298.585858585859,-5.27272727272727)(1478,-17.4545454545455)(1239.13131313131,-10.5454545454545)(880.828282828283,-4.54545454545455)(179.151515151515,-0.727272727272727)(1074.90909090909,-9.45454545454546)(716.606060606061,0)(313.515151515152,-18)(701.676767676768,-12.3636363636364)(880.828282828283,-14.7272727272727)(850.969696969697,-16.1818181818182)(1478,-5.45454545454545)(0,-15.8181818181818)(1283.91919191919,-1.63636363636364)(1149.55555555556,-2.72727272727273)(0,-0.181818181818182)(1089.83838383838,-14.9090909090909)(1373.49494949495,-4)(1030.12121212121,-18)(1478,-0.181818181818182)(865.89898989899,-1.81818181818182)(0,-2.90909090909091)(0,-14.1818181818182)(358.30303030303,-17.0909090909091) 
};

%\node at (axis cs:50, -2.5) [shape=circle,fill=white,draw=black,inner sep=0pt,anchor=south west] {\scriptsize\color{locol}$\*L_{\*t}$};
%\node at (axis cs:460, -5.9) [shape=circle,fill=white,draw=black,inner sep=0pt,anchor=south west] {\scriptsize\color{orange!50!yellow}$\*H_{\*t}$};
%\node at (axis cs:160, -5.2) [shape=circle,fill=white,draw=black,inner sep=0pt,anchor=south west] {\scriptsize\color{darkgray}$\*U_{\*t}$};

\node at (axis cs:1155, -17) [shape=circle,fill=red!40!yellow,draw=black,inner sep=0.2pt,anchor=south west,minimum size=16pt]
  {\scriptsize\color{white}$\hat{\*H}$};
\node at (axis cs:1330, -17) [shape=circle,fill=locol,draw=black,inner sep=0.2pt,anchor=south west,minimum size=16pt]
  {\scriptsize\color{white}$\hat{\*L}$};

\end{axis}
\end{tikzpicture}%

\renewcommand\trimlen{2pt}
\begin{figure}[htbp]
  \begin{subfigure}[b]{0.49\textwidth}
    \centering
    \adjincludegraphics[width=\linewidth,clip=true,trim=\trimlen{} \trimlen{} \trimlen{} \trimlen{}]{figures/ev_ping_seq}
    \vspace{-18pt}
    \caption{\textsf{[N]} Mean performance}
	  \label{fig:ping-seq}
  \end{subfigure}
  \hfill
  \begin{subfigure}[b]{0.49\textwidth}
    \centering
    \adjincludegraphics[width=\linewidth,clip=true,trim=\trimlen{} \trimlen{} \trimlen{} \trimlen{}]{figures/ev_chl_seq}
    \vspace{-18pt}
    \caption{\textsf{[C]} Mean performance}
	\label{fig:chl-seq}
  \end{subfigure}

  \begin{subfigure}[b]{0.49\textwidth}
    \vspace{8pt} % space of this row from above captions
    \centering
    \adjincludegraphics[width=\linewidth,clip=true,trim=\trimlen{} \trimlen{} \trimlen{} \trimlen{}]{figures/ev_bgape_seq}
    \vspace{-18pt}
    \caption{\textsf{[A]} Mean performance}
	  \label{fig:bgape-seq}
  \end{subfigure}
  \hfill
  \begin{subfigure}[b]{0.49\textwidth}
    \centering
    \adjincludegraphics[width=\linewidth,clip=true,trim=\trimlen{} \trimlen{} \trimlen{} \trimlen{}]{figures/limno_bgape_str_class400}
    \vspace{-18pt}
    \caption{\textsf{[A]} \str after $t=400$ iterations}
	\label{fig:bgape-str-class}
  \end{subfigure}

  \begin{subfigure}[b]{0.49\textwidth}
    \centering
    \vspace{8pt} % space of this row from above captions
    \adjincludegraphics[width=\linewidth,clip=true,trim=\trimlen{} \trimlen{} \trimlen{} \trimlen{}]{figures/limno_bgape_acl_class374}
    \vspace{-18pt}
    \caption{\textsf{[A]} \acl after $t=374$ iterations}
	  \label{fig:bgape-acl-class}
  \end{subfigure}
  \hfill
  \begin{subfigure}[b]{0.49\textwidth}
    \centering
    \adjincludegraphics[width=\linewidth,clip=true,trim=\trimlen{} \trimlen{} \trimlen{} \trimlen{}]{figures/limno_bgape_var_class400}
    \vspace{-18pt}
    \caption{\textsf{[A]} \var after $t=400$ iterations}
	  \label{fig:bgape-var-class}
  \end{subfigure}

  \begin{subfigure}[b]{0.49\textwidth}
    \centering
    \vspace{8pt} % space of this row from above captions
    \adjincludegraphics[width=\linewidth,clip=true,trim=\trimlen{} \trimlen{} \trimlen{} \trimlen{}]{figures/ev_bgape_seq_str_sc}
    \vspace{-18pt}
    \caption{\textsf{[A]} \str scatter plot}
	  \label{fig:bgape-seq-str-sc}
  \end{subfigure}
  \hfill
  \begin{subfigure}[b]{0.49\textwidth}
    \centering
    \adjincludegraphics[width=\linewidth,clip=true,trim=\trimlen{} \trimlen{} \trimlen{} \trimlen{}]{figures/ev_bgape_seq_acl_sc}
    \vspace{-18pt}
    \caption{\textsf{[A]} \acl scatter plot}
	  \label{fig:bgape-seq-acl-sc}
  \end{subfigure}

  \caption{Performance of explicit threshold sequential algorithms on the
           network latency \textsf{[N]},
           chlorophyll concentration \textsf{[C]},
           and algae concentration \textsf{[A]} datasets.
           \textbf{(a), (b)} \acl and \str are comparable and both clearly outperform
           \var.
           \textbf{(c)} \str performs poorly due to limited exploration.
           \textbf{(d)} An example of a \str execution getting stuck: even after $400$
           iterations it has only achieved an $F_1$-score of $0.37$.
           \textbf{(e)} An example of a \acl execution without any issues.
           \textbf{(f)} An example of a \var execution: the sample space is sampled
           rather uniformly.
           \textbf{(g), (h)} The results of (c) in more detail: about a third of \str's
           executions achieve an $F_1$-score of less than $0.4$, while \acl
           always achieves an $F_1$-score of at least $0.95$.
           }
  \label{fig:exp-seq}
\end{figure}

\section{Results I:\hspace{0.33em}Explicit threshold level}

\paragraph{Sequential sampling}
\figsref{fig:ping-seq}--\ref{fig:bgape-seq} compare the performance of the
strictly sequential algorithms on the three datasets.
On the first two datasets \acl and \str are comparable in performance, with
\acl being slightly better on the chorophyll dataset and slightly worse
on the latency dataset. However, the situation is different in the third
dataset, where \str performs extremely poorly. The reason is that the
algorithm can get stuck in situations where it is favorable according to
the straddle score to continue sampling points near the currently inferred
level set, instead of sampling points of high variance at yet unexplored
regions that lie far away from the currently inferred level set.

\figref{fig:bgape-str-class} shows an example of such an execution, where
\str has heavily sampled the left region of the level set, but has
completely missed the part of the level set that lies on the right of the
transect (cf. \figref{fig:limno_bgape}).
In \figref{fig:bgape-seq-str-sc} we depict as a scatter plot the
detailed results of the $50$ \str executions. Note that about a third
of the executions do not manage to achieve an $F_1$-score of 0.4, even
after $400$ iterations, i.e. they miss the right part of the level set
as explained above, and another third only achieve an $F_1$-score of 0.8,
i.e. they miss the left part of the level set.
On the other hand, as shown in \figref{fig:bgape-seq-acl-sc}, this problem
never occurs in \acl, because of its classification regime, which implicitly
forces exploration by excluding points that have already been classified from
being sampled.

Although \var is commonly used for estimating functions over their entire
domain, it is clearly outperformed in our experiments, because it does not
sufficiently focus on accurately estimating the desired level set, but
rather samples almost uniformly over the sample space.

\paragraph{The effect of $\*\epsilon$}
To illustrate how the choice of $\epsilon$ affects the results of \acl,
we present in \figref{fig:exp-eps} two scatter plots, similar to the one in
\figref{fig:bgape-seq-acl-sc}, that depict the results of running \acl
on the two environmental monitoring datasets for different values of
$\epsilon$. As before, each point in the plot corresponds to the result of
one execution of \acl, but now we have colored the points according to the
value of $\epsilon$ used in that execution. In our theoretical analysis
we have seen that $\epsilon$ represents a tradeoff parameter between
classification accuracy and sampling cost. The two scatter plots
give experimental evidence that this is indeed the case and, moreover,
that the transition when varying $\epsilon$ is fairly smooth.

%\setlength\figureheight{1.3in}\setlength\figurewidth{2.1in}
%% This file was created by matlab2tikz v0.2.3.
% Copyright (c) 2008--2012, Nico Schlömer <nico.schloemer@gmail.com>
% All rights reserved.
% 
% 
% 

% defining custom colors
\definecolor{mycolor1}{rgb}{1,0.04,0}
\definecolor{mycolor2}{rgb}{1,0.08,0}
\definecolor{mycolor3}{rgb}{1,0.12,0}
\definecolor{mycolor4}{rgb}{1,0.16,0}
\definecolor{mycolor5}{rgb}{1,0.2,0}
\definecolor{mycolor6}{rgb}{1,0.24,0}
\definecolor{mycolor7}{rgb}{1,0.28,0}
\definecolor{mycolor8}{rgb}{1,0.32,0}
\definecolor{mycolor9}{rgb}{1,0.36,0}
\definecolor{mycolor10}{rgb}{1,0.4,0}
\definecolor{mycolor11}{rgb}{1,0.44,0}
\definecolor{mycolor12}{rgb}{1,0.48,0}
\definecolor{mycolor13}{rgb}{1,0.52,0}
\definecolor{mycolor14}{rgb}{1,0.56,0}
\definecolor{mycolor15}{rgb}{1,0.6,0}
\definecolor{mycolor16}{rgb}{1,0.68,0}
\definecolor{mycolor17}{rgb}{1,0.72,0}
\definecolor{mycolor18}{rgb}{1,0.76,0}
\definecolor{mycolor19}{rgb}{1,0.84,0}
\definecolor{mycolor20}{rgb}{1,0.88,0}
\definecolor{mycolor21}{rgb}{1,0.96,0}
\definecolor{mycolor22}{rgb}{1,1,0}
\definecolor{mycolor23}{rgb}{0.92,1,0.08}
\definecolor{mycolor24}{rgb}{0.84,1,0.16}
\definecolor{mycolor25}{rgb}{0.76,1,0.24}
\definecolor{mycolor26}{rgb}{0.68,1,0.32}
\definecolor{mycolor27}{rgb}{0.6,1,0.4}
\definecolor{mycolor28}{rgb}{0.52,1,0.48}
\definecolor{mycolor29}{rgb}{0.44,1,0.56}
\definecolor{mycolor30}{rgb}{0.36,1,0.64}
\definecolor{mycolor31}{rgb}{0.24,1,0.76}
\definecolor{mycolor32}{rgb}{0.16,1,0.84}
\definecolor{mycolor33}{rgb}{0.04,1,0.96}
\definecolor{mycolor34}{rgb}{0,0.92,1}
\definecolor{mycolor35}{rgb}{0,0.8,1}
\definecolor{mycolor36}{rgb}{0,0.68,1}
\definecolor{mycolor37}{rgb}{0,0.56,1}
\definecolor{mycolor38}{rgb}{0,0.44,1}
\definecolor{mycolor39}{rgb}{0,0.28,1}
\definecolor{mycolor40}{rgb}{0,0.12,1}
\definecolor{mycolor41}{rgb}{0,0.52,1}
\definecolor{mycolor42}{rgb}{0,0.48,1}
\definecolor{mycolor43}{rgb}{0,0.4,1}
\definecolor{mycolor44}{rgb}{0,0.36,1}
\definecolor{mycolor45}{rgb}{0,0.32,1}
\definecolor{mycolor46}{rgb}{0,0.24,1}
\definecolor{mycolor47}{rgb}{0,0.2,1}
\definecolor{mycolor48}{rgb}{0,0.16,1}
\definecolor{mycolor49}{rgb}{0,0.08,1}
\definecolor{mycolor50}{rgb}{0,0.04,1}
\definecolor{mycolor51}{rgb}{1,0.64,0}
\definecolor{mycolor52}{rgb}{0.96,1,0.04}
\definecolor{mycolor53}{rgb}{0.88,1,0.12}
\definecolor{mycolor54}{rgb}{0.8,1,0.2}
\definecolor{mycolor55}{rgb}{0.72,1,0.28}

\begin{tikzpicture}

\begin{axis}[%
tick label style={font=\tiny},
label style={font=\tiny},
xlabel shift={-6pt},
ylabel shift={-4pt},
legend style={font=\tiny},
view={0}{90},
width=\figurewidth,
height=\figureheight,
scale only axis,
xmin=0, xmax=400,
xlabel={Samples},
ymin=0, ymax=1,
ylabel={$F_1$-score},
name=plot1,
axis lines*=left,
colormap={mymap}{[1pt] rgb(0pt)=(0.5,0,0); rgb(8pt)=(1,0,0); rgb(24pt)=(1,1,0); rgb(40pt)=(0,1,1); rgb(56pt)=(0,0,1); rgb(63pt)=(0,0,0.5625)},
colorbar right,
colorbar style={
  tickwidth=0.03cm,
  at={(1.06, 0)},
  anchor=south east,
  yticklabel pos=right,
  yticklabel shift={-2pt}
},
tickwidth=0.1cm,
colorbar/width=0.15cm,
point meta min=0,
point meta max=2
]

\addplot [
color=blue,
mark size=0.9pt,
only marks,
mark=*,
mark options={solid,fill=red!68!black,draw=black,line width=0.1pt},
forget plot
]
coordinates{
 (294,0.930862437633642) 
};
\addplot [
color=blue,
mark size=0.9pt,
only marks,
mark=*,
mark options={solid,fill=red!68!black,draw=black,line width=0.1pt},
forget plot
]
coordinates{
 (352,0.974216027874564) 
};
\addplot [
color=blue,
mark size=0.9pt,
only marks,
mark=*,
mark options={solid,fill=red!68!black,draw=black,line width=0.1pt},
forget plot
]
coordinates{
 (308,0.960111966410077) 
};
\addplot [
color=blue,
mark size=0.9pt,
only marks,
mark=*,
mark options={solid,fill=red!68!black,draw=black,line width=0.1pt},
forget plot
]
coordinates{
 (258,0.885196374622356) 
};
\addplot [
color=blue,
mark size=0.9pt,
only marks,
mark=*,
mark options={solid,fill=red!72!black,draw=black,line width=0.1pt},
forget plot
]
coordinates{
 (328,0.942997888810697) 
};
\addplot [
color=blue,
mark size=0.9pt,
only marks,
mark=*,
mark options={solid,fill=red!72!black,draw=black,line width=0.1pt},
forget plot
]
coordinates{
 (350,0.965034965034965) 
};
\addplot [
color=blue,
mark size=0.9pt,
only marks,
mark=*,
mark options={solid,fill=red!72!black,draw=black,line width=0.1pt},
forget plot
]
coordinates{
 (281,0.945454545454545) 
};
\addplot [
color=blue,
mark size=0.9pt,
only marks,
mark=*,
mark options={solid,fill=red!72!black,draw=black,line width=0.1pt},
forget plot
]
coordinates{
 (360,0.974107767669699) 
};
\addplot [
color=blue,
mark size=0.9pt,
only marks,
mark=*,
mark options={solid,fill=red!76!black,draw=black,line width=0.1pt},
forget plot
]
coordinates{
 (298,0.951219512195122) 
};
\addplot [
color=blue,
mark size=0.9pt,
only marks,
mark=*,
mark options={solid,fill=red!76!black,draw=black,line width=0.1pt},
forget plot
]
coordinates{
 (305,0.972515856236786) 
};
\addplot [
color=blue,
mark size=0.9pt,
only marks,
mark=*,
mark options={solid,fill=red!76!black,draw=black,line width=0.1pt},
forget plot
]
coordinates{
 (230,0.867078825347759) 
};
\addplot [
color=blue,
mark size=0.9pt,
only marks,
mark=*,
mark options={solid,fill=red!80!black,draw=black,line width=0.1pt},
forget plot
]
coordinates{
 (226,0.878971255673222) 
};
\addplot [
color=blue,
mark size=0.9pt,
only marks,
mark=*,
mark options={solid,fill=red!80!black,draw=black,line width=0.1pt},
forget plot
]
coordinates{
 (316,0.953439888811675) 
};
\addplot [
color=blue,
mark size=0.9pt,
only marks,
mark=*,
mark options={solid,fill=red!80!black,draw=black,line width=0.1pt},
forget plot
]
coordinates{
 (214,0.876712328767123) 
};
\addplot [
color=blue,
mark size=0.9pt,
only marks,
mark=*,
mark options={solid,fill=red!84!black,draw=black,line width=0.1pt},
forget plot
]
coordinates{
 (276,0.956951305575159) 
};
\addplot [
color=blue,
mark size=0.9pt,
only marks,
mark=*,
mark options={solid,fill=red!84!black,draw=black,line width=0.1pt},
forget plot
]
coordinates{
 (286,0.947220267417312) 
};
\addplot [
color=blue,
mark size=0.9pt,
only marks,
mark=*,
mark options={solid,fill=red!84!black,draw=black,line width=0.1pt},
forget plot
]
coordinates{
 (286,0.952843273231623) 
};
\addplot [
color=blue,
mark size=0.9pt,
only marks,
mark=*,
mark options={solid,fill=red!88!black,draw=black,line width=0.1pt},
forget plot
]
coordinates{
 (285,0.961884961884962) 
};
\addplot [
color=blue,
mark size=0.9pt,
only marks,
mark=*,
mark options={solid,fill=red!88!black,draw=black,line width=0.1pt},
forget plot
]
coordinates{
 (189,0.879818594104308) 
};
\addplot [
color=blue,
mark size=0.9pt,
only marks,
mark=*,
mark options={solid,fill=red!92!black,draw=black,line width=0.1pt},
forget plot
]
coordinates{
 (261,0.945404284727021) 
};
\addplot [
color=blue,
mark size=0.9pt,
only marks,
mark=*,
mark options={solid,fill=red!92!black,draw=black,line width=0.1pt},
forget plot
]
coordinates{
 (233,0.951269732326699) 
};
\addplot [
color=blue,
mark size=0.9pt,
only marks,
mark=*,
mark options={solid,fill=red!92!black,draw=black,line width=0.1pt},
forget plot
]
coordinates{
 (195,0.896860986547085) 
};
\addplot [
color=blue,
mark size=0.9pt,
only marks,
mark=*,
mark options={solid,fill=red!96!black,draw=black,line width=0.1pt},
forget plot
]
coordinates{
 (251,0.963838664812239) 
};
\addplot [
color=blue,
mark size=0.9pt,
only marks,
mark=*,
mark options={solid,fill=red!96!black,draw=black,line width=0.1pt},
forget plot
]
coordinates{
 (186,0.926898509581263) 
};
\addplot [
color=blue,
mark size=0.9pt,
only marks,
mark=*,
mark options={solid,fill=red,draw=black,line width=0.1pt},
forget plot
]
coordinates{
 (213,0.952247191011236) 
};
\addplot [
color=blue,
mark size=0.9pt,
only marks,
mark=*,
mark options={solid,fill=mycolor1,draw=black,line width=0.1pt},
forget plot
]
coordinates{
 (159,0.877488514548239) 
};
\addplot [
color=blue,
mark size=0.9pt,
only marks,
mark=*,
mark options={solid,fill=mycolor1,draw=black,line width=0.1pt},
forget plot
]
coordinates{
 (195,0.936319104268719) 
};
\addplot [
color=blue,
mark size=0.9pt,
only marks,
mark=*,
mark options={solid,fill=mycolor2,draw=black,line width=0.1pt},
forget plot
]
coordinates{
 (190,0.939958592132505) 
};
\addplot [
color=blue,
mark size=0.9pt,
only marks,
mark=*,
mark options={solid,fill=mycolor2,draw=black,line width=0.1pt},
forget plot
]
coordinates{
 (175,0.913105413105413) 
};
\addplot [
color=blue,
mark size=0.9pt,
only marks,
mark=*,
mark options={solid,fill=mycolor3,draw=black,line width=0.1pt},
forget plot
]
coordinates{
 (128,0.86697247706422) 
};
\addplot [
color=blue,
mark size=0.9pt,
only marks,
mark=*,
mark options={solid,fill=mycolor4,draw=black,line width=0.1pt},
forget plot
]
coordinates{
 (135,0.856278366111951) 
};
\addplot [
color=blue,
mark size=0.9pt,
only marks,
mark=*,
mark options={solid,fill=mycolor5,draw=black,line width=0.1pt},
forget plot
]
coordinates{
 (143,0.917018284106892) 
};
\addplot [
color=blue,
mark size=0.9pt,
only marks,
mark=*,
mark options={solid,fill=mycolor5,draw=black,line width=0.1pt},
forget plot
]
coordinates{
 (108,0.862147753236862) 
};
\addplot [
color=blue,
mark size=0.9pt,
only marks,
mark=*,
mark options={solid,fill=mycolor6,draw=black,line width=0.1pt},
forget plot
]
coordinates{
 (102,0.842105263157895) 
};
\addplot [
color=blue,
mark size=0.9pt,
only marks,
mark=*,
mark options={solid,fill=mycolor7,draw=black,line width=0.1pt},
forget plot
]
coordinates{
 (121,0.922865013774105) 
};
\addplot [
color=blue,
mark size=0.9pt,
only marks,
mark=*,
mark options={solid,fill=mycolor8,draw=black,line width=0.1pt},
forget plot
]
coordinates{
 (92,0.908821349147517) 
};
\addplot [
color=blue,
mark size=0.9pt,
only marks,
mark=*,
mark options={solid,fill=mycolor9,draw=black,line width=0.1pt},
forget plot
]
coordinates{
 (98,0.864216054013503) 
};
\addplot [
color=blue,
mark size=0.9pt,
only marks,
mark=*,
mark options={solid,fill=mycolor10,draw=black,line width=0.1pt},
forget plot
]
coordinates{
 (99,0.885409252669039) 
};
\addplot [
color=blue,
mark size=0.9pt,
only marks,
mark=*,
mark options={solid,fill=mycolor11,draw=black,line width=0.1pt},
forget plot
]
coordinates{
 (82,0.849438202247191) 
};
\addplot [
color=blue,
mark size=0.9pt,
only marks,
mark=*,
mark options={solid,fill=mycolor12,draw=black,line width=0.1pt},
forget plot
]
coordinates{
 (85,0.930428671820098) 
};
\addplot [
color=blue,
mark size=0.9pt,
only marks,
mark=*,
mark options={solid,fill=mycolor13,draw=black,line width=0.1pt},
forget plot
]
coordinates{
 (70,0.879219804951238) 
};
\addplot [
color=blue,
mark size=0.9pt,
only marks,
mark=*,
mark options={solid,fill=mycolor14,draw=black,line width=0.1pt},
forget plot
]
coordinates{
 (79,0.874243443174176) 
};
\addplot [
color=blue,
mark size=0.9pt,
only marks,
mark=*,
mark options={solid,fill=mycolor15,draw=black,line width=0.1pt},
forget plot
]
coordinates{
 (68,0.883755588673621) 
};
\addplot [
color=blue,
mark size=0.9pt,
only marks,
mark=*,
mark options={solid,fill=mycolor16,draw=black,line width=0.1pt},
forget plot
]
coordinates{
 (55,0.815052041633307) 
};
\addplot [
color=blue,
mark size=0.9pt,
only marks,
mark=*,
mark options={solid,fill=mycolor17,draw=black,line width=0.1pt},
forget plot
]
coordinates{
 (63,0.85912560721721) 
};
\addplot [
color=blue,
mark size=0.9pt,
only marks,
mark=*,
mark options={solid,fill=mycolor18,draw=black,line width=0.1pt},
forget plot
]
coordinates{
 (52,0.851851851851852) 
};
\addplot [
color=blue,
mark size=0.9pt,
only marks,
mark=*,
mark options={solid,fill=mycolor19,draw=black,line width=0.1pt},
forget plot
]
coordinates{
 (55,0.823275862068965) 
};
\addplot [
color=blue,
mark size=0.9pt,
only marks,
mark=*,
mark options={solid,fill=mycolor20,draw=black,line width=0.1pt},
forget plot
]
coordinates{
 (47,0.803951367781155) 
};
\addplot [
color=blue,
mark size=0.9pt,
only marks,
mark=*,
mark options={solid,fill=mycolor21,draw=black,line width=0.1pt},
forget plot
]
coordinates{
 (44,0.834905660377358) 
};
\addplot [
color=blue,
mark size=0.9pt,
only marks,
mark=*,
mark options={solid,fill=mycolor22,draw=black,line width=0.1pt},
forget plot
]
coordinates{
 (43,0.847107438016529) 
};
\addplot [
color=blue,
mark size=0.9pt,
only marks,
mark=*,
mark options={solid,fill=mycolor23,draw=black,line width=0.1pt},
forget plot
]
coordinates{
 (34,0.789259560618389) 
};
\addplot [
color=blue,
mark size=0.9pt,
only marks,
mark=*,
mark options={solid,fill=mycolor24,draw=black,line width=0.1pt},
forget plot
]
coordinates{
 (39,0.832298136645963) 
};
\addplot [
color=blue,
mark size=0.9pt,
only marks,
mark=*,
mark options={solid,fill=mycolor25,draw=black,line width=0.1pt},
forget plot
]
coordinates{
 (36,0.858564321250888) 
};
\addplot [
color=blue,
mark size=0.9pt,
only marks,
mark=*,
mark options={solid,fill=mycolor26,draw=black,line width=0.1pt},
forget plot
]
coordinates{
 (32,0.859408795962509) 
};
\addplot [
color=blue,
mark size=0.9pt,
only marks,
mark=*,
mark options={solid,fill=mycolor27,draw=black,line width=0.1pt},
forget plot
]
coordinates{
 (33,0.813404417364813) 
};
\addplot [
color=blue,
mark size=0.9pt,
only marks,
mark=*,
mark options={solid,fill=mycolor28,draw=black,line width=0.1pt},
forget plot
]
coordinates{
 (32,0.771747211895911) 
};
\addplot [
color=blue,
mark size=0.9pt,
only marks,
mark=*,
mark options={solid,fill=mycolor29,draw=black,line width=0.1pt},
forget plot
]
coordinates{
 (28,0.78893280632411) 
};
\addplot [
color=blue,
mark size=0.9pt,
only marks,
mark=*,
mark options={solid,fill=mycolor30,draw=black,line width=0.1pt},
forget plot
]
coordinates{
 (28,0.769230769230769) 
};
\addplot [
color=blue,
mark size=0.9pt,
only marks,
mark=*,
mark options={solid,fill=mycolor31,draw=black,line width=0.1pt},
forget plot
]
coordinates{
 (29,0.763840224246671) 
};
\addplot [
color=blue,
mark size=0.9pt,
only marks,
mark=*,
mark options={solid,fill=mycolor32,draw=black,line width=0.1pt},
forget plot
]
coordinates{
 (25,0.776377952755905) 
};
\addplot [
color=blue,
mark size=0.9pt,
only marks,
mark=*,
mark options={solid,fill=mycolor33,draw=black,line width=0.1pt},
forget plot
]
coordinates{
 (24,0.760746147607461) 
};
\addplot [
color=blue,
mark size=0.9pt,
only marks,
mark=*,
mark options={solid,fill=mycolor34,draw=black,line width=0.1pt},
forget plot
]
coordinates{
 (22,0.760655737704918) 
};
\addplot [
color=blue,
mark size=0.9pt,
only marks,
mark=*,
mark options={solid,fill=mycolor35,draw=black,line width=0.1pt},
forget plot
]
coordinates{
 (21,0.768333333333333) 
};
\addplot [
color=blue,
mark size=0.9pt,
only marks,
mark=*,
mark options={solid,fill=mycolor36,draw=black,line width=0.1pt},
forget plot
]
coordinates{
 (19,0.784873949579832) 
};
\addplot [
color=blue,
mark size=0.9pt,
only marks,
mark=*,
mark options={solid,fill=mycolor37,draw=black,line width=0.1pt},
forget plot
]
coordinates{
 (19,0.73195020746888) 
};
\addplot [
color=blue,
mark size=0.9pt,
only marks,
mark=*,
mark options={solid,fill=mycolor38,draw=black,line width=0.1pt},
forget plot
]
coordinates{
 (20,0.728129205921938) 
};
\addplot [
color=blue,
mark size=0.9pt,
only marks,
mark=*,
mark options={solid,fill=mycolor39,draw=black,line width=0.1pt},
forget plot
]
coordinates{
 (19,0.745874587458746) 
};
\addplot [
color=blue,
mark size=0.9pt,
only marks,
mark=*,
mark options={solid,fill=mycolor40,draw=black,line width=0.1pt},
forget plot
]
coordinates{
 (17,0.766265060240964) 
};
\addplot [
color=blue,
mark size=0.9pt,
only marks,
mark=*,
mark options={solid,fill=red!68!black,draw=black,line width=0.1pt},
forget plot
]
coordinates{
 (355,0.966433566433566) 
};
\addplot [
color=blue,
mark size=0.9pt,
only marks,
mark=*,
mark options={solid,fill=red!68!black,draw=black,line width=0.1pt},
forget plot
]
coordinates{
 (372,0.970159611380985) 
};
\addplot [
color=blue,
mark size=0.9pt,
only marks,
mark=*,
mark options={solid,fill=red!68!black,draw=black,line width=0.1pt},
forget plot
]
coordinates{
 (364,0.962095106822881) 
};
\addplot [
color=blue,
mark size=0.9pt,
only marks,
mark=*,
mark options={solid,fill=red!68!black,draw=black,line width=0.1pt},
forget plot
]
coordinates{
 (354,0.967877094972067) 
};
\addplot [
color=blue,
mark size=0.9pt,
only marks,
mark=*,
mark options={solid,fill=red!72!black,draw=black,line width=0.1pt},
forget plot
]
coordinates{
 (330,0.964754664823773) 
};
\addplot [
color=blue,
mark size=0.9pt,
only marks,
mark=*,
mark options={solid,fill=red!72!black,draw=black,line width=0.1pt},
forget plot
]
coordinates{
 (327,0.968728283530229) 
};
\addplot [
color=blue,
mark size=0.9pt,
only marks,
mark=*,
mark options={solid,fill=red!72!black,draw=black,line width=0.1pt},
forget plot
]
coordinates{
 (371,0.966461327857632) 
};
\addplot [
color=blue,
mark size=0.9pt,
only marks,
mark=*,
mark options={solid,fill=red!72!black,draw=black,line width=0.1pt},
forget plot
]
coordinates{
 (331,0.964509394572025) 
};
\addplot [
color=blue,
mark size=0.9pt,
only marks,
mark=*,
mark options={solid,fill=red!76!black,draw=black,line width=0.1pt},
forget plot
]
coordinates{
 (237,0.881559220389805) 
};
\addplot [
color=blue,
mark size=0.9pt,
only marks,
mark=*,
mark options={solid,fill=red!76!black,draw=black,line width=0.1pt},
forget plot
]
coordinates{
 (353,0.959666203059805) 
};
\addplot [
color=blue,
mark size=0.9pt,
only marks,
mark=*,
mark options={solid,fill=red!76!black,draw=black,line width=0.1pt},
forget plot
]
coordinates{
 (301,0.944793850454228) 
};
\addplot [
color=blue,
mark size=0.9pt,
only marks,
mark=*,
mark options={solid,fill=red!80!black,draw=black,line width=0.1pt},
forget plot
]
coordinates{
 (325,0.959778085991678) 
};
\addplot [
color=blue,
mark size=0.9pt,
only marks,
mark=*,
mark options={solid,fill=red!80!black,draw=black,line width=0.1pt},
forget plot
]
coordinates{
 (325,0.948986722571628) 
};
\addplot [
color=blue,
mark size=0.9pt,
only marks,
mark=*,
mark options={solid,fill=red!80!black,draw=black,line width=0.1pt},
forget plot
]
coordinates{
 (280,0.945086705202312) 
};
\addplot [
color=blue,
mark size=0.9pt,
only marks,
mark=*,
mark options={solid,fill=red!84!black,draw=black,line width=0.1pt},
forget plot
]
coordinates{
 (221,0.874536005939124) 
};
\addplot [
color=blue,
mark size=0.9pt,
only marks,
mark=*,
mark options={solid,fill=red!84!black,draw=black,line width=0.1pt},
forget plot
]
coordinates{
 (226,0.928011404133999) 
};
\addplot [
color=blue,
mark size=0.9pt,
only marks,
mark=*,
mark options={solid,fill=red!84!black,draw=black,line width=0.1pt},
forget plot
]
coordinates{
 (266,0.946938775510204) 
};
\addplot [
color=blue,
mark size=0.9pt,
only marks,
mark=*,
mark options={solid,fill=red!88!black,draw=black,line width=0.1pt},
forget plot
]
coordinates{
 (200,0.933914306463326) 
};
\addplot [
color=blue,
mark size=0.9pt,
only marks,
mark=*,
mark options={solid,fill=red!88!black,draw=black,line width=0.1pt},
forget plot
]
coordinates{
 (166,0.919540229885057) 
};
\addplot [
color=blue,
mark size=0.9pt,
only marks,
mark=*,
mark options={solid,fill=red!92!black,draw=black,line width=0.1pt},
forget plot
]
coordinates{
 (183,0.907246376811594) 
};
\addplot [
color=blue,
mark size=0.9pt,
only marks,
mark=*,
mark options={solid,fill=red!92!black,draw=black,line width=0.1pt},
forget plot
]
coordinates{
 (187,0.877643504531722) 
};
\addplot [
color=blue,
mark size=0.9pt,
only marks,
mark=*,
mark options={solid,fill=red!92!black,draw=black,line width=0.1pt},
forget plot
]
coordinates{
 (215,0.945908460471567) 
};
\addplot [
color=blue,
mark size=0.9pt,
only marks,
mark=*,
mark options={solid,fill=red!96!black,draw=black,line width=0.1pt},
forget plot
]
coordinates{
 (230,0.948251748251748) 
};
\addplot [
color=blue,
mark size=0.9pt,
only marks,
mark=*,
mark options={solid,fill=red!96!black,draw=black,line width=0.1pt},
forget plot
]
coordinates{
 (226,0.953179594689029) 
};
\addplot [
color=blue,
mark size=0.9pt,
only marks,
mark=*,
mark options={solid,fill=red,draw=black,line width=0.1pt},
forget plot
]
coordinates{
 (213,0.938186813186813) 
};
\addplot [
color=blue,
mark size=0.9pt,
only marks,
mark=*,
mark options={solid,fill=mycolor1,draw=black,line width=0.1pt},
forget plot
]
coordinates{
 (184,0.943181818181818) 
};
\addplot [
color=blue,
mark size=0.9pt,
only marks,
mark=*,
mark options={solid,fill=mycolor1,draw=black,line width=0.1pt},
forget plot
]
coordinates{
 (147,0.907246376811594) 
};
\addplot [
color=blue,
mark size=0.9pt,
only marks,
mark=*,
mark options={solid,fill=mycolor2,draw=black,line width=0.1pt},
forget plot
]
coordinates{
 (176,0.936140350877193) 
};
\addplot [
color=blue,
mark size=0.9pt,
only marks,
mark=*,
mark options={solid,fill=mycolor2,draw=black,line width=0.1pt},
forget plot
]
coordinates{
 (165,0.929190751445086) 
};
\addplot [
color=blue,
mark size=0.9pt,
only marks,
mark=*,
mark options={solid,fill=mycolor3,draw=black,line width=0.1pt},
forget plot
]
coordinates{
 (160,0.929235167977126) 
};
\addplot [
color=blue,
mark size=0.9pt,
only marks,
mark=*,
mark options={solid,fill=mycolor4,draw=black,line width=0.1pt},
forget plot
]
coordinates{
 (163,0.950102529049897) 
};
\addplot [
color=blue,
mark size=0.9pt,
only marks,
mark=*,
mark options={solid,fill=mycolor5,draw=black,line width=0.1pt},
forget plot
]
coordinates{
 (168,0.930679478380233) 
};
\addplot [
color=blue,
mark size=0.9pt,
only marks,
mark=*,
mark options={solid,fill=mycolor5,draw=black,line width=0.1pt},
forget plot
]
coordinates{
 (133,0.917915774446823) 
};
\addplot [
color=blue,
mark size=0.9pt,
only marks,
mark=*,
mark options={solid,fill=mycolor6,draw=black,line width=0.1pt},
forget plot
]
coordinates{
 (114,0.881638846737481) 
};
\addplot [
color=blue,
mark size=0.9pt,
only marks,
mark=*,
mark options={solid,fill=mycolor7,draw=black,line width=0.1pt},
forget plot
]
coordinates{
 (119,0.930826057756884) 
};
\addplot [
color=blue,
mark size=0.9pt,
only marks,
mark=*,
mark options={solid,fill=mycolor8,draw=black,line width=0.1pt},
forget plot
]
coordinates{
 (80,0.818110850897736) 
};
\addplot [
color=blue,
mark size=0.9pt,
only marks,
mark=*,
mark options={solid,fill=mycolor9,draw=black,line width=0.1pt},
forget plot
]
coordinates{
 (86,0.876371616678859) 
};
\addplot [
color=blue,
mark size=0.9pt,
only marks,
mark=*,
mark options={solid,fill=mycolor10,draw=black,line width=0.1pt},
forget plot
]
coordinates{
 (80,0.867415730337079) 
};
\addplot [
color=blue,
mark size=0.9pt,
only marks,
mark=*,
mark options={solid,fill=mycolor11,draw=black,line width=0.1pt},
forget plot
]
coordinates{
 (94,0.93183415319747) 
};
\addplot [
color=blue,
mark size=0.9pt,
only marks,
mark=*,
mark options={solid,fill=mycolor12,draw=black,line width=0.1pt},
forget plot
]
coordinates{
 (76,0.905606813342796) 
};
\addplot [
color=blue,
mark size=0.9pt,
only marks,
mark=*,
mark options={solid,fill=mycolor13,draw=black,line width=0.1pt},
forget plot
]
coordinates{
 (75,0.899470899470899) 
};
\addplot [
color=blue,
mark size=0.9pt,
only marks,
mark=*,
mark options={solid,fill=mycolor14,draw=black,line width=0.1pt},
forget plot
]
coordinates{
 (68,0.90146750524109) 
};
\addplot [
color=blue,
mark size=0.9pt,
only marks,
mark=*,
mark options={solid,fill=mycolor15,draw=black,line width=0.1pt},
forget plot
]
coordinates{
 (68,0.830135039090263) 
};
\addplot [
color=blue,
mark size=0.9pt,
only marks,
mark=*,
mark options={solid,fill=mycolor16,draw=black,line width=0.1pt},
forget plot
]
coordinates{
 (59,0.85348506401138) 
};
\addplot [
color=blue,
mark size=0.9pt,
only marks,
mark=*,
mark options={solid,fill=mycolor17,draw=black,line width=0.1pt},
forget plot
]
coordinates{
 (46,0.829268292682927) 
};
\addplot [
color=blue,
mark size=0.9pt,
only marks,
mark=*,
mark options={solid,fill=mycolor18,draw=black,line width=0.1pt},
forget plot
]
coordinates{
 (53,0.820705176294073) 
};
\addplot [
color=blue,
mark size=0.9pt,
only marks,
mark=*,
mark options={solid,fill=mycolor19,draw=black,line width=0.1pt},
forget plot
]
coordinates{
 (51,0.845814977973568) 
};
\addplot [
color=blue,
mark size=0.9pt,
only marks,
mark=*,
mark options={solid,fill=mycolor20,draw=black,line width=0.1pt},
forget plot
]
coordinates{
 (50,0.836473247927656) 
};
\addplot [
color=blue,
mark size=0.9pt,
only marks,
mark=*,
mark options={solid,fill=mycolor21,draw=black,line width=0.1pt},
forget plot
]
coordinates{
 (46,0.833333333333333) 
};
\addplot [
color=blue,
mark size=0.9pt,
only marks,
mark=*,
mark options={solid,fill=mycolor22,draw=black,line width=0.1pt},
forget plot
]
coordinates{
 (40,0.835481425322214) 
};
\addplot [
color=blue,
mark size=0.9pt,
only marks,
mark=*,
mark options={solid,fill=mycolor23,draw=black,line width=0.1pt},
forget plot
]
coordinates{
 (37,0.813008130081301) 
};
\addplot [
color=blue,
mark size=0.9pt,
only marks,
mark=*,
mark options={solid,fill=mycolor24,draw=black,line width=0.1pt},
forget plot
]
coordinates{
 (38,0.845986984815618) 
};
\addplot [
color=blue,
mark size=0.9pt,
only marks,
mark=*,
mark options={solid,fill=mycolor25,draw=black,line width=0.1pt},
forget plot
]
coordinates{
 (40,0.852805280528053) 
};
\addplot [
color=blue,
mark size=0.9pt,
only marks,
mark=*,
mark options={solid,fill=mycolor26,draw=black,line width=0.1pt},
forget plot
]
coordinates{
 (35,0.810566037735849) 
};
\addplot [
color=blue,
mark size=0.9pt,
only marks,
mark=*,
mark options={solid,fill=mycolor27,draw=black,line width=0.1pt},
forget plot
]
coordinates{
 (38,0.802684563758389) 
};
\addplot [
color=blue,
mark size=0.9pt,
only marks,
mark=*,
mark options={solid,fill=mycolor28,draw=black,line width=0.1pt},
forget plot
]
coordinates{
 (33,0.81413612565445) 
};
\addplot [
color=blue,
mark size=0.9pt,
only marks,
mark=*,
mark options={solid,fill=mycolor29,draw=black,line width=0.1pt},
forget plot
]
coordinates{
 (30,0.817219477769936) 
};
\addplot [
color=blue,
mark size=0.9pt,
only marks,
mark=*,
mark options={solid,fill=mycolor30,draw=black,line width=0.1pt},
forget plot
]
coordinates{
 (27,0.814555633310007) 
};
\addplot [
color=blue,
mark size=0.9pt,
only marks,
mark=*,
mark options={solid,fill=mycolor31,draw=black,line width=0.1pt},
forget plot
]
coordinates{
 (26,0.789515488482923) 
};
\addplot [
color=blue,
mark size=0.9pt,
only marks,
mark=*,
mark options={solid,fill=mycolor32,draw=black,line width=0.1pt},
forget plot
]
coordinates{
 (25,0.79409594095941) 
};
\addplot [
color=blue,
mark size=0.9pt,
only marks,
mark=*,
mark options={solid,fill=mycolor33,draw=black,line width=0.1pt},
forget plot
]
coordinates{
 (24,0.769356153219234) 
};
\addplot [
color=blue,
mark size=0.9pt,
only marks,
mark=*,
mark options={solid,fill=mycolor34,draw=black,line width=0.1pt},
forget plot
]
coordinates{
 (24,0.79288256227758) 
};
\addplot [
color=blue,
mark size=0.9pt,
only marks,
mark=*,
mark options={solid,fill=mycolor35,draw=black,line width=0.1pt},
forget plot
]
coordinates{
 (25,0.776957163958641) 
};
\addplot [
color=blue,
mark size=0.9pt,
only marks,
mark=*,
mark options={solid,fill=mycolor36,draw=black,line width=0.1pt},
forget plot
]
coordinates{
 (22,0.724842767295597) 
};
\addplot [
color=blue,
mark size=0.9pt,
only marks,
mark=*,
mark options={solid,fill=mycolor37,draw=black,line width=0.1pt},
forget plot
]
coordinates{
 (20,0.789598108747045) 
};
\addplot [
color=blue,
mark size=0.9pt,
only marks,
mark=*,
mark options={solid,fill=mycolor38,draw=black,line width=0.1pt},
forget plot
]
coordinates{
 (21,0.763212079615649) 
};
\addplot [
color=blue,
mark size=0.9pt,
only marks,
mark=*,
mark options={solid,fill=mycolor39,draw=black,line width=0.1pt},
forget plot
]
coordinates{
 (18,0.819464033850493) 
};
\addplot [
color=blue,
mark size=0.9pt,
only marks,
mark=*,
mark options={solid,fill=mycolor40,draw=black,line width=0.1pt},
forget plot
]
coordinates{
 (8,0.54728370221328) 
};
\addplot [
color=blue,
mark size=0.9pt,
only marks,
mark=*,
mark options={solid,fill=red!68!black,draw=black,line width=0.1pt},
forget plot
]
coordinates{
 (387,0.95859649122807) 
};
\addplot [
color=blue,
mark size=0.9pt,
only marks,
mark=*,
mark options={solid,fill=red!68!black,draw=black,line width=0.1pt},
forget plot
]
coordinates{
 (348,0.961702127659574) 
};
\addplot [
color=blue,
mark size=0.9pt,
only marks,
mark=*,
mark options={solid,fill=red!68!black,draw=black,line width=0.1pt},
forget plot
]
coordinates{
 (275,0.876233864844343) 
};
\addplot [
color=blue,
mark size=0.9pt,
only marks,
mark=*,
mark options={solid,fill=red!68!black,draw=black,line width=0.1pt},
forget plot
]
coordinates{
 (286,0.932559825960841) 
};
\addplot [
color=blue,
mark size=0.9pt,
only marks,
mark=*,
mark options={solid,fill=red!72!black,draw=black,line width=0.1pt},
forget plot
]
coordinates{
 (358,0.965034965034965) 
};
\addplot [
color=blue,
mark size=0.9pt,
only marks,
mark=*,
mark options={solid,fill=red!72!black,draw=black,line width=0.1pt},
forget plot
]
coordinates{
 (283,0.946191474493361) 
};
\addplot [
color=blue,
mark size=0.9pt,
only marks,
mark=*,
mark options={solid,fill=red!72!black,draw=black,line width=0.1pt},
forget plot
]
coordinates{
 (263,0.92463768115942) 
};
\addplot [
color=blue,
mark size=0.9pt,
only marks,
mark=*,
mark options={solid,fill=red!72!black,draw=black,line width=0.1pt},
forget plot
]
coordinates{
 (328,0.956829440905874) 
};
\addplot [
color=blue,
mark size=0.9pt,
only marks,
mark=*,
mark options={solid,fill=red!76!black,draw=black,line width=0.1pt},
forget plot
]
coordinates{
 (314,0.954766875434934) 
};
\addplot [
color=blue,
mark size=0.9pt,
only marks,
mark=*,
mark options={solid,fill=red!76!black,draw=black,line width=0.1pt},
forget plot
]
coordinates{
 (323,0.953389830508474) 
};
\addplot [
color=blue,
mark size=0.9pt,
only marks,
mark=*,
mark options={solid,fill=red!76!black,draw=black,line width=0.1pt},
forget plot
]
coordinates{
 (263,0.926300578034682) 
};
\addplot [
color=blue,
mark size=0.9pt,
only marks,
mark=*,
mark options={solid,fill=red!80!black,draw=black,line width=0.1pt},
forget plot
]
coordinates{
 (329,0.96056338028169) 
};
\addplot [
color=blue,
mark size=0.9pt,
only marks,
mark=*,
mark options={solid,fill=red!80!black,draw=black,line width=0.1pt},
forget plot
]
coordinates{
 (328,0.959056210964608) 
};
\addplot [
color=blue,
mark size=0.9pt,
only marks,
mark=*,
mark options={solid,fill=red!80!black,draw=black,line width=0.1pt},
forget plot
]
coordinates{
 (299,0.960339943342776) 
};
\addplot [
color=blue,
mark size=0.9pt,
only marks,
mark=*,
mark options={solid,fill=red!84!black,draw=black,line width=0.1pt},
forget plot
]
coordinates{
 (308,0.957182320441989) 
};
\addplot [
color=blue,
mark size=0.9pt,
only marks,
mark=*,
mark options={solid,fill=red!84!black,draw=black,line width=0.1pt},
forget plot
]
coordinates{
 (285,0.954321855235418) 
};
\addplot [
color=blue,
mark size=0.9pt,
only marks,
mark=*,
mark options={solid,fill=red!84!black,draw=black,line width=0.1pt},
forget plot
]
coordinates{
 (247,0.953191489361702) 
};
\addplot [
color=blue,
mark size=0.9pt,
only marks,
mark=*,
mark options={solid,fill=red!88!black,draw=black,line width=0.1pt},
forget plot
]
coordinates{
 (204,0.918681318681319) 
};
\addplot [
color=blue,
mark size=0.9pt,
only marks,
mark=*,
mark options={solid,fill=red!88!black,draw=black,line width=0.1pt},
forget plot
]
coordinates{
 (258,0.940925266903915) 
};
\addplot [
color=blue,
mark size=0.9pt,
only marks,
mark=*,
mark options={solid,fill=red!92!black,draw=black,line width=0.1pt},
forget plot
]
coordinates{
 (186,0.876795162509448) 
};
\addplot [
color=blue,
mark size=0.9pt,
only marks,
mark=*,
mark options={solid,fill=red!92!black,draw=black,line width=0.1pt},
forget plot
]
coordinates{
 (247,0.947001394700139) 
};
\addplot [
color=blue,
mark size=0.9pt,
only marks,
mark=*,
mark options={solid,fill=red!92!black,draw=black,line width=0.1pt},
forget plot
]
coordinates{
 (203,0.924652523774689) 
};
\addplot [
color=blue,
mark size=0.9pt,
only marks,
mark=*,
mark options={solid,fill=red!96!black,draw=black,line width=0.1pt},
forget plot
]
coordinates{
 (187,0.917888563049853) 
};
\addplot [
color=blue,
mark size=0.9pt,
only marks,
mark=*,
mark options={solid,fill=red!96!black,draw=black,line width=0.1pt},
forget plot
]
coordinates{
 (213,0.952112676056338) 
};
\addplot [
color=blue,
mark size=0.9pt,
only marks,
mark=*,
mark options={solid,fill=red,draw=black,line width=0.1pt},
forget plot
]
coordinates{
 (230,0.948936170212766) 
};
\addplot [
color=blue,
mark size=0.9pt,
only marks,
mark=*,
mark options={solid,fill=mycolor1,draw=black,line width=0.1pt},
forget plot
]
coordinates{
 (207,0.938628158844765) 
};
\addplot [
color=blue,
mark size=0.9pt,
only marks,
mark=*,
mark options={solid,fill=mycolor1,draw=black,line width=0.1pt},
forget plot
]
coordinates{
 (194,0.944674965421853) 
};
\addplot [
color=blue,
mark size=0.9pt,
only marks,
mark=*,
mark options={solid,fill=mycolor2,draw=black,line width=0.1pt},
forget plot
]
coordinates{
 (148,0.921511627906977) 
};
\addplot [
color=blue,
mark size=0.9pt,
only marks,
mark=*,
mark options={solid,fill=mycolor2,draw=black,line width=0.1pt},
forget plot
]
coordinates{
 (166,0.92600422832981) 
};
\addplot [
color=blue,
mark size=0.9pt,
only marks,
mark=*,
mark options={solid,fill=mycolor3,draw=black,line width=0.1pt},
forget plot
]
coordinates{
 (137,0.879939209726444) 
};
\addplot [
color=blue,
mark size=0.9pt,
only marks,
mark=*,
mark options={solid,fill=mycolor4,draw=black,line width=0.1pt},
forget plot
]
coordinates{
 (182,0.944979367262723) 
};
\addplot [
color=blue,
mark size=0.9pt,
only marks,
mark=*,
mark options={solid,fill=mycolor5,draw=black,line width=0.1pt},
forget plot
]
coordinates{
 (149,0.905109489051095) 
};
\addplot [
color=blue,
mark size=0.9pt,
only marks,
mark=*,
mark options={solid,fill=mycolor5,draw=black,line width=0.1pt},
forget plot
]
coordinates{
 (129,0.913868613138686) 
};
\addplot [
color=blue,
mark size=0.9pt,
only marks,
mark=*,
mark options={solid,fill=mycolor6,draw=black,line width=0.1pt},
forget plot
]
coordinates{
 (132,0.937588652482269) 
};
\addplot [
color=blue,
mark size=0.9pt,
only marks,
mark=*,
mark options={solid,fill=mycolor7,draw=black,line width=0.1pt},
forget plot
]
coordinates{
 (121,0.90050107372942) 
};
\addplot [
color=blue,
mark size=0.9pt,
only marks,
mark=*,
mark options={solid,fill=mycolor8,draw=black,line width=0.1pt},
forget plot
]
coordinates{
 (97,0.841221374045801) 
};
\addplot [
color=blue,
mark size=0.9pt,
only marks,
mark=*,
mark options={solid,fill=mycolor9,draw=black,line width=0.1pt},
forget plot
]
coordinates{
 (93,0.853374709076804) 
};
\addplot [
color=blue,
mark size=0.9pt,
only marks,
mark=*,
mark options={solid,fill=mycolor10,draw=black,line width=0.1pt},
forget plot
]
coordinates{
 (88,0.843822843822844) 
};
\addplot [
color=blue,
mark size=0.9pt,
only marks,
mark=*,
mark options={solid,fill=mycolor11,draw=black,line width=0.1pt},
forget plot
]
coordinates{
 (89,0.921052631578947) 
};
\addplot [
color=blue,
mark size=0.9pt,
only marks,
mark=*,
mark options={solid,fill=mycolor12,draw=black,line width=0.1pt},
forget plot
]
coordinates{
 (87,0.885838150289017) 
};
\addplot [
color=blue,
mark size=0.9pt,
only marks,
mark=*,
mark options={solid,fill=mycolor13,draw=black,line width=0.1pt},
forget plot
]
coordinates{
 (74,0.884241971620612) 
};
\addplot [
color=blue,
mark size=0.9pt,
only marks,
mark=*,
mark options={solid,fill=mycolor14,draw=black,line width=0.1pt},
forget plot
]
coordinates{
 (79,0.903724525650035) 
};
\addplot [
color=blue,
mark size=0.9pt,
only marks,
mark=*,
mark options={solid,fill=mycolor15,draw=black,line width=0.1pt},
forget plot
]
coordinates{
 (63,0.863768115942029) 
};
\addplot [
color=blue,
mark size=0.9pt,
only marks,
mark=*,
mark options={solid,fill=mycolor16,draw=black,line width=0.1pt},
forget plot
]
coordinates{
 (66,0.875280059746079) 
};
\addplot [
color=blue,
mark size=0.9pt,
only marks,
mark=*,
mark options={solid,fill=mycolor17,draw=black,line width=0.1pt},
forget plot
]
coordinates{
 (53,0.877167630057803) 
};
\addplot [
color=blue,
mark size=0.9pt,
only marks,
mark=*,
mark options={solid,fill=mycolor18,draw=black,line width=0.1pt},
forget plot
]
coordinates{
 (60,0.844137931034483) 
};
\addplot [
color=blue,
mark size=0.9pt,
only marks,
mark=*,
mark options={solid,fill=mycolor19,draw=black,line width=0.1pt},
forget plot
]
coordinates{
 (56,0.888730385164051) 
};
\addplot [
color=blue,
mark size=0.9pt,
only marks,
mark=*,
mark options={solid,fill=mycolor20,draw=black,line width=0.1pt},
forget plot
]
coordinates{
 (49,0.869433962264151) 
};
\addplot [
color=blue,
mark size=0.9pt,
only marks,
mark=*,
mark options={solid,fill=mycolor21,draw=black,line width=0.1pt},
forget plot
]
coordinates{
 (42,0.818722139673105) 
};
\addplot [
color=blue,
mark size=0.9pt,
only marks,
mark=*,
mark options={solid,fill=mycolor22,draw=black,line width=0.1pt},
forget plot
]
coordinates{
 (44,0.875524475524475) 
};
\addplot [
color=blue,
mark size=0.9pt,
only marks,
mark=*,
mark options={solid,fill=mycolor23,draw=black,line width=0.1pt},
forget plot
]
coordinates{
 (39,0.840245775729647) 
};
\addplot [
color=blue,
mark size=0.9pt,
only marks,
mark=*,
mark options={solid,fill=mycolor24,draw=black,line width=0.1pt},
forget plot
]
coordinates{
 (36,0.825590251332825) 
};
\addplot [
color=blue,
mark size=0.9pt,
only marks,
mark=*,
mark options={solid,fill=mycolor25,draw=black,line width=0.1pt},
forget plot
]
coordinates{
 (37,0.813559322033898) 
};
\addplot [
color=blue,
mark size=0.9pt,
only marks,
mark=*,
mark options={solid,fill=mycolor26,draw=black,line width=0.1pt},
forget plot
]
coordinates{
 (29,0.79126213592233) 
};
\addplot [
color=blue,
mark size=0.9pt,
only marks,
mark=*,
mark options={solid,fill=mycolor27,draw=black,line width=0.1pt},
forget plot
]
coordinates{
 (36,0.809160305343511) 
};
\addplot [
color=blue,
mark size=0.9pt,
only marks,
mark=*,
mark options={solid,fill=mycolor28,draw=black,line width=0.1pt},
forget plot
]
coordinates{
 (27,0.802896218825422) 
};
\addplot [
color=blue,
mark size=0.9pt,
only marks,
mark=*,
mark options={solid,fill=mycolor29,draw=black,line width=0.1pt},
forget plot
]
coordinates{
 (26,0.805687203791469) 
};
\addplot [
color=blue,
mark size=0.9pt,
only marks,
mark=*,
mark options={solid,fill=mycolor30,draw=black,line width=0.1pt},
forget plot
]
coordinates{
 (24,0.7808) 
};
\addplot [
color=blue,
mark size=0.9pt,
only marks,
mark=*,
mark options={solid,fill=mycolor31,draw=black,line width=0.1pt},
forget plot
]
coordinates{
 (25,0.786610878661088) 
};
\addplot [
color=blue,
mark size=0.9pt,
only marks,
mark=*,
mark options={solid,fill=mycolor32,draw=black,line width=0.1pt},
forget plot
]
coordinates{
 (24,0.764350453172205) 
};
\addplot [
color=blue,
mark size=0.9pt,
only marks,
mark=*,
mark options={solid,fill=mycolor33,draw=black,line width=0.1pt},
forget plot
]
coordinates{
 (23,0.78475935828877) 
};
\addplot [
color=blue,
mark size=0.9pt,
only marks,
mark=*,
mark options={solid,fill=mycolor34,draw=black,line width=0.1pt},
forget plot
]
coordinates{
 (21,0.785046728971962) 
};
\addplot [
color=blue,
mark size=0.9pt,
only marks,
mark=*,
mark options={solid,fill=mycolor35,draw=black,line width=0.1pt},
forget plot
]
coordinates{
 (25,0.788083953960731) 
};
\addplot [
color=blue,
mark size=0.9pt,
only marks,
mark=*,
mark options={solid,fill=mycolor36,draw=black,line width=0.1pt},
forget plot
]
coordinates{
 (22,0.811755361397935) 
};
\addplot [
color=blue,
mark size=0.9pt,
only marks,
mark=*,
mark options={solid,fill=mycolor37,draw=black,line width=0.1pt},
forget plot
]
coordinates{
 (20,0.696673189823875) 
};
\addplot [
color=blue,
mark size=0.9pt,
only marks,
mark=*,
mark options={solid,fill=mycolor38,draw=black,line width=0.1pt},
forget plot
]
coordinates{
 (22,0.806646525679758) 
};
\addplot [
color=blue,
mark size=0.9pt,
only marks,
mark=*,
mark options={solid,fill=mycolor39,draw=black,line width=0.1pt},
forget plot
]
coordinates{
 (21,0.788888888888889) 
};
\addplot [
color=blue,
mark size=0.9pt,
only marks,
mark=*,
mark options={solid,fill=mycolor40,draw=black,line width=0.1pt},
forget plot
]
coordinates{
 (13,0.785829307568438) 
};
\addplot [
color=blue,
mark size=0.9pt,
only marks,
mark=*,
mark options={solid,fill=red!68!black,draw=black,line width=0.1pt},
forget plot
]
coordinates{
 (374,0.962649753347428) 
};
\addplot [
color=blue,
mark size=0.9pt,
only marks,
mark=*,
mark options={solid,fill=red!68!black,draw=black,line width=0.1pt},
forget plot
]
coordinates{
 (338,0.97194950911641) 
};
\addplot [
color=blue,
mark size=0.9pt,
only marks,
mark=*,
mark options={solid,fill=red!68!black,draw=black,line width=0.1pt},
forget plot
]
coordinates{
 (345,0.963380281690141) 
};
\addplot [
color=blue,
mark size=0.9pt,
only marks,
mark=*,
mark options={solid,fill=red!68!black,draw=black,line width=0.1pt},
forget plot
]
coordinates{
 (362,0.956645344705046) 
};
\addplot [
color=blue,
mark size=0.9pt,
only marks,
mark=*,
mark options={solid,fill=red!72!black,draw=black,line width=0.1pt},
forget plot
]
coordinates{
 (365,0.968197879858657) 
};
\addplot [
color=blue,
mark size=0.9pt,
only marks,
mark=*,
mark options={solid,fill=red!72!black,draw=black,line width=0.1pt},
forget plot
]
coordinates{
 (340,0.962754743499649) 
};
\addplot [
color=blue,
mark size=0.9pt,
only marks,
mark=*,
mark options={solid,fill=red!72!black,draw=black,line width=0.1pt},
forget plot
]
coordinates{
 (343,0.959212376933896) 
};
\addplot [
color=blue,
mark size=0.9pt,
only marks,
mark=*,
mark options={solid,fill=red!72!black,draw=black,line width=0.1pt},
forget plot
]
coordinates{
 (334,0.959497206703911) 
};
\addplot [
color=blue,
mark size=0.9pt,
only marks,
mark=*,
mark options={solid,fill=red!76!black,draw=black,line width=0.1pt},
forget plot
]
coordinates{
 (247,0.933621933621934) 
};
\addplot [
color=blue,
mark size=0.9pt,
only marks,
mark=*,
mark options={solid,fill=red!76!black,draw=black,line width=0.1pt},
forget plot
]
coordinates{
 (296,0.926007326007326) 
};
\addplot [
color=blue,
mark size=0.9pt,
only marks,
mark=*,
mark options={solid,fill=red!76!black,draw=black,line width=0.1pt},
forget plot
]
coordinates{
 (266,0.942628903413217) 
};
\addplot [
color=blue,
mark size=0.9pt,
only marks,
mark=*,
mark options={solid,fill=red!80!black,draw=black,line width=0.1pt},
forget plot
]
coordinates{
 (283,0.93371757925072) 
};
\addplot [
color=blue,
mark size=0.9pt,
only marks,
mark=*,
mark options={solid,fill=red!80!black,draw=black,line width=0.1pt},
forget plot
]
coordinates{
 (313,0.960839160839161) 
};
\addplot [
color=blue,
mark size=0.9pt,
only marks,
mark=*,
mark options={solid,fill=red!80!black,draw=black,line width=0.1pt},
forget plot
]
coordinates{
 (244,0.938040345821326) 
};
\addplot [
color=blue,
mark size=0.9pt,
only marks,
mark=*,
mark options={solid,fill=red!84!black,draw=black,line width=0.1pt},
forget plot
]
coordinates{
 (283,0.961379310344827) 
};
\addplot [
color=blue,
mark size=0.9pt,
only marks,
mark=*,
mark options={solid,fill=red!84!black,draw=black,line width=0.1pt},
forget plot
]
coordinates{
 (255,0.934296028880866) 
};
\addplot [
color=blue,
mark size=0.9pt,
only marks,
mark=*,
mark options={solid,fill=red!84!black,draw=black,line width=0.1pt},
forget plot
]
coordinates{
 (232,0.935740072202166) 
};
\addplot [
color=blue,
mark size=0.9pt,
only marks,
mark=*,
mark options={solid,fill=red!88!black,draw=black,line width=0.1pt},
forget plot
]
coordinates{
 (235,0.951983298538622) 
};
\addplot [
color=blue,
mark size=0.9pt,
only marks,
mark=*,
mark options={solid,fill=red!88!black,draw=black,line width=0.1pt},
forget plot
]
coordinates{
 (204,0.880061115355233) 
};
\addplot [
color=blue,
mark size=0.9pt,
only marks,
mark=*,
mark options={solid,fill=red!92!black,draw=black,line width=0.1pt},
forget plot
]
coordinates{
 (259,0.941761363636364) 
};
\addplot [
color=blue,
mark size=0.9pt,
only marks,
mark=*,
mark options={solid,fill=red!92!black,draw=black,line width=0.1pt},
forget plot
]
coordinates{
 (218,0.897744360902256) 
};
\addplot [
color=blue,
mark size=0.9pt,
only marks,
mark=*,
mark options={solid,fill=red!92!black,draw=black,line width=0.1pt},
forget plot
]
coordinates{
 (275,0.964260686755431) 
};
\addplot [
color=blue,
mark size=0.9pt,
only marks,
mark=*,
mark options={solid,fill=red!96!black,draw=black,line width=0.1pt},
forget plot
]
coordinates{
 (211,0.903846153846154) 
};
\addplot [
color=blue,
mark size=0.9pt,
only marks,
mark=*,
mark options={solid,fill=red!96!black,draw=black,line width=0.1pt},
forget plot
]
coordinates{
 (195,0.938804895608351) 
};
\addplot [
color=blue,
mark size=0.9pt,
only marks,
mark=*,
mark options={solid,fill=red,draw=black,line width=0.1pt},
forget plot
]
coordinates{
 (233,0.956036287508723) 
};
\addplot [
color=blue,
mark size=0.9pt,
only marks,
mark=*,
mark options={solid,fill=mycolor1,draw=black,line width=0.1pt},
forget plot
]
coordinates{
 (172,0.923076923076923) 
};
\addplot [
color=blue,
mark size=0.9pt,
only marks,
mark=*,
mark options={solid,fill=mycolor1,draw=black,line width=0.1pt},
forget plot
]
coordinates{
 (210,0.961779013203613) 
};
\addplot [
color=blue,
mark size=0.9pt,
only marks,
mark=*,
mark options={solid,fill=mycolor2,draw=black,line width=0.1pt},
forget plot
]
coordinates{
 (181,0.909479686386315) 
};
\addplot [
color=blue,
mark size=0.9pt,
only marks,
mark=*,
mark options={solid,fill=mycolor2,draw=black,line width=0.1pt},
forget plot
]
coordinates{
 (188,0.949224259520451) 
};
\addplot [
color=blue,
mark size=0.9pt,
only marks,
mark=*,
mark options={solid,fill=mycolor3,draw=black,line width=0.1pt},
forget plot
]
coordinates{
 (169,0.950071326676177) 
};
\addplot [
color=blue,
mark size=0.9pt,
only marks,
mark=*,
mark options={solid,fill=mycolor4,draw=black,line width=0.1pt},
forget plot
]
coordinates{
 (171,0.96078431372549) 
};
\addplot [
color=blue,
mark size=0.9pt,
only marks,
mark=*,
mark options={solid,fill=mycolor5,draw=black,line width=0.1pt},
forget plot
]
coordinates{
 (161,0.945736434108527) 
};
\addplot [
color=blue,
mark size=0.9pt,
only marks,
mark=*,
mark options={solid,fill=mycolor5,draw=black,line width=0.1pt},
forget plot
]
coordinates{
 (148,0.936376210235131) 
};
\addplot [
color=blue,
mark size=0.9pt,
only marks,
mark=*,
mark options={solid,fill=mycolor6,draw=black,line width=0.1pt},
forget plot
]
coordinates{
 (146,0.935817805383023) 
};
\addplot [
color=blue,
mark size=0.9pt,
only marks,
mark=*,
mark options={solid,fill=mycolor7,draw=black,line width=0.1pt},
forget plot
]
coordinates{
 (116,0.925631768953068) 
};
\addplot [
color=blue,
mark size=0.9pt,
only marks,
mark=*,
mark options={solid,fill=mycolor8,draw=black,line width=0.1pt},
forget plot
]
coordinates{
 (105,0.914487632508834) 
};
\addplot [
color=blue,
mark size=0.9pt,
only marks,
mark=*,
mark options={solid,fill=mycolor9,draw=black,line width=0.1pt},
forget plot
]
coordinates{
 (94,0.90961262553802) 
};
\addplot [
color=blue,
mark size=0.9pt,
only marks,
mark=*,
mark options={solid,fill=mycolor10,draw=black,line width=0.1pt},
forget plot
]
coordinates{
 (97,0.905003625815808) 
};
\addplot [
color=blue,
mark size=0.9pt,
only marks,
mark=*,
mark options={solid,fill=mycolor11,draw=black,line width=0.1pt},
forget plot
]
coordinates{
 (90,0.895774647887324) 
};
\addplot [
color=blue,
mark size=0.9pt,
only marks,
mark=*,
mark options={solid,fill=mycolor12,draw=black,line width=0.1pt},
forget plot
]
coordinates{
 (90,0.900808229243203) 
};
\addplot [
color=blue,
mark size=0.9pt,
only marks,
mark=*,
mark options={solid,fill=mycolor13,draw=black,line width=0.1pt},
forget plot
]
coordinates{
 (76,0.836573074154068) 
};
\addplot [
color=blue,
mark size=0.9pt,
only marks,
mark=*,
mark options={solid,fill=mycolor14,draw=black,line width=0.1pt},
forget plot
]
coordinates{
 (75,0.835666912306559) 
};
\addplot [
color=blue,
mark size=0.9pt,
only marks,
mark=*,
mark options={solid,fill=mycolor15,draw=black,line width=0.1pt},
forget plot
]
coordinates{
 (68,0.853675945753034) 
};
\addplot [
color=blue,
mark size=0.9pt,
only marks,
mark=*,
mark options={solid,fill=mycolor16,draw=black,line width=0.1pt},
forget plot
]
coordinates{
 (68,0.884798909338787) 
};
\addplot [
color=blue,
mark size=0.9pt,
only marks,
mark=*,
mark options={solid,fill=mycolor17,draw=black,line width=0.1pt},
forget plot
]
coordinates{
 (53,0.823346303501945) 
};
\addplot [
color=blue,
mark size=0.9pt,
only marks,
mark=*,
mark options={solid,fill=mycolor18,draw=black,line width=0.1pt},
forget plot
]
coordinates{
 (59,0.831426392067124) 
};
\addplot [
color=blue,
mark size=0.9pt,
only marks,
mark=*,
mark options={solid,fill=mycolor19,draw=black,line width=0.1pt},
forget plot
]
coordinates{
 (44,0.826291079812207) 
};
\addplot [
color=blue,
mark size=0.9pt,
only marks,
mark=*,
mark options={solid,fill=mycolor20,draw=black,line width=0.1pt},
forget plot
]
coordinates{
 (49,0.839524517087667) 
};
\addplot [
color=blue,
mark size=0.9pt,
only marks,
mark=*,
mark options={solid,fill=mycolor21,draw=black,line width=0.1pt},
forget plot
]
coordinates{
 (48,0.826706676669167) 
};
\addplot [
color=blue,
mark size=0.9pt,
only marks,
mark=*,
mark options={solid,fill=mycolor22,draw=black,line width=0.1pt},
forget plot
]
coordinates{
 (41,0.843465045592705) 
};
\addplot [
color=blue,
mark size=0.9pt,
only marks,
mark=*,
mark options={solid,fill=mycolor23,draw=black,line width=0.1pt},
forget plot
]
coordinates{
 (37,0.813450760608487) 
};
\addplot [
color=blue,
mark size=0.9pt,
only marks,
mark=*,
mark options={solid,fill=mycolor24,draw=black,line width=0.1pt},
forget plot
]
coordinates{
 (34,0.804705882352941) 
};
\addplot [
color=blue,
mark size=0.9pt,
only marks,
mark=*,
mark options={solid,fill=mycolor25,draw=black,line width=0.1pt},
forget plot
]
coordinates{
 (36,0.808176100628931) 
};
\addplot [
color=blue,
mark size=0.9pt,
only marks,
mark=*,
mark options={solid,fill=mycolor26,draw=black,line width=0.1pt},
forget plot
]
coordinates{
 (33,0.793774319066148) 
};
\addplot [
color=blue,
mark size=0.9pt,
only marks,
mark=*,
mark options={solid,fill=mycolor27,draw=black,line width=0.1pt},
forget plot
]
coordinates{
 (31,0.810879190385832) 
};
\addplot [
color=blue,
mark size=0.9pt,
only marks,
mark=*,
mark options={solid,fill=mycolor28,draw=black,line width=0.1pt},
forget plot
]
coordinates{
 (30,0.82641252552757) 
};
\addplot [
color=blue,
mark size=0.9pt,
only marks,
mark=*,
mark options={solid,fill=mycolor29,draw=black,line width=0.1pt},
forget plot
]
coordinates{
 (30,0.805194805194805) 
};
\addplot [
color=blue,
mark size=0.9pt,
only marks,
mark=*,
mark options={solid,fill=mycolor30,draw=black,line width=0.1pt},
forget plot
]
coordinates{
 (25,0.813559322033898) 
};
\addplot [
color=blue,
mark size=0.9pt,
only marks,
mark=*,
mark options={solid,fill=mycolor31,draw=black,line width=0.1pt},
forget plot
]
coordinates{
 (28,0.761835396941005) 
};
\addplot [
color=blue,
mark size=0.9pt,
only marks,
mark=*,
mark options={solid,fill=mycolor32,draw=black,line width=0.1pt},
forget plot
]
coordinates{
 (25,0.810727969348659) 
};
\addplot [
color=blue,
mark size=0.9pt,
only marks,
mark=*,
mark options={solid,fill=mycolor33,draw=black,line width=0.1pt},
forget plot
]
coordinates{
 (26,0.809930178432894) 
};
\addplot [
color=blue,
mark size=0.9pt,
only marks,
mark=*,
mark options={solid,fill=mycolor34,draw=black,line width=0.1pt},
forget plot
]
coordinates{
 (24,0.793080505655356) 
};
\addplot [
color=blue,
mark size=0.9pt,
only marks,
mark=*,
mark options={solid,fill=mycolor35,draw=black,line width=0.1pt},
forget plot
]
coordinates{
 (24,0.822238478419898) 
};
\addplot [
color=blue,
mark size=0.9pt,
only marks,
mark=*,
mark options={solid,fill=mycolor36,draw=black,line width=0.1pt},
forget plot
]
coordinates{
 (23,0.805890227576974) 
};
\addplot [
color=blue,
mark size=0.9pt,
only marks,
mark=*,
mark options={solid,fill=mycolor37,draw=black,line width=0.1pt},
forget plot
]
coordinates{
 (16,0.789350039154268) 
};
\addplot [
color=blue,
mark size=0.9pt,
only marks,
mark=*,
mark options={solid,fill=mycolor38,draw=black,line width=0.1pt},
forget plot
]
coordinates{
 (22,0.785581106277191) 
};
\addplot [
color=blue,
mark size=0.9pt,
only marks,
mark=*,
mark options={solid,fill=mycolor39,draw=black,line width=0.1pt},
forget plot
]
coordinates{
 (15,0.759136212624585) 
};
\addplot [
color=blue,
mark size=0.9pt,
only marks,
mark=*,
mark options={solid,fill=mycolor40,draw=black,line width=0.1pt},
forget plot
]
coordinates{
 (13,0.766476810414971) 
};
\addplot [
color=blue,
mark size=0.9pt,
only marks,
mark=*,
mark options={solid,fill=red!68!black,draw=black,line width=0.1pt},
forget plot
]
coordinates{
 (343,0.964234620886981) 
};
\addplot [
color=blue,
mark size=0.9pt,
only marks,
mark=*,
mark options={solid,fill=red!68!black,draw=black,line width=0.1pt},
forget plot
]
coordinates{
 (328,0.954220314735336) 
};
\addplot [
color=blue,
mark size=0.9pt,
only marks,
mark=*,
mark options={solid,fill=red!68!black,draw=black,line width=0.1pt},
forget plot
]
coordinates{
 (311,0.936635105608157) 
};
\addplot [
color=blue,
mark size=0.9pt,
only marks,
mark=*,
mark options={solid,fill=red!68!black,draw=black,line width=0.1pt},
forget plot
]
coordinates{
 (306,0.967329545454545) 
};
\addplot [
color=blue,
mark size=0.9pt,
only marks,
mark=*,
mark options={solid,fill=red!72!black,draw=black,line width=0.1pt},
forget plot
]
coordinates{
 (307,0.94201861130995) 
};
\addplot [
color=blue,
mark size=0.9pt,
only marks,
mark=*,
mark options={solid,fill=red!72!black,draw=black,line width=0.1pt},
forget plot
]
coordinates{
 (336,0.94592645998558) 
};
\addplot [
color=blue,
mark size=0.9pt,
only marks,
mark=*,
mark options={solid,fill=red!72!black,draw=black,line width=0.1pt},
forget plot
]
coordinates{
 (323,0.960507757404795) 
};
\addplot [
color=blue,
mark size=0.9pt,
only marks,
mark=*,
mark options={solid,fill=red!72!black,draw=black,line width=0.1pt},
forget plot
]
coordinates{
 (262,0.938271604938271) 
};
\addplot [
color=blue,
mark size=0.9pt,
only marks,
mark=*,
mark options={solid,fill=red!76!black,draw=black,line width=0.1pt},
forget plot
]
coordinates{
 (251,0.917325664989216) 
};
\addplot [
color=blue,
mark size=0.9pt,
only marks,
mark=*,
mark options={solid,fill=red!76!black,draw=black,line width=0.1pt},
forget plot
]
coordinates{
 (314,0.954022988505747) 
};
\addplot [
color=blue,
mark size=0.9pt,
only marks,
mark=*,
mark options={solid,fill=red!76!black,draw=black,line width=0.1pt},
forget plot
]
coordinates{
 (287,0.94608195542775) 
};
\addplot [
color=blue,
mark size=0.9pt,
only marks,
mark=*,
mark options={solid,fill=red!80!black,draw=black,line width=0.1pt},
forget plot
]
coordinates{
 (287,0.946280991735537) 
};
\addplot [
color=blue,
mark size=0.9pt,
only marks,
mark=*,
mark options={solid,fill=red!80!black,draw=black,line width=0.1pt},
forget plot
]
coordinates{
 (247,0.941007194244604) 
};
\addplot [
color=blue,
mark size=0.9pt,
only marks,
mark=*,
mark options={solid,fill=red!80!black,draw=black,line width=0.1pt},
forget plot
]
coordinates{
 (290,0.961918194640338) 
};
\addplot [
color=blue,
mark size=0.9pt,
only marks,
mark=*,
mark options={solid,fill=red!84!black,draw=black,line width=0.1pt},
forget plot
]
coordinates{
 (283,0.950819672131147) 
};
\addplot [
color=blue,
mark size=0.9pt,
only marks,
mark=*,
mark options={solid,fill=red!84!black,draw=black,line width=0.1pt},
forget plot
]
coordinates{
 (291,0.962078651685393) 
};
\addplot [
color=blue,
mark size=0.9pt,
only marks,
mark=*,
mark options={solid,fill=red!84!black,draw=black,line width=0.1pt},
forget plot
]
coordinates{
 (231,0.933235509904622) 
};
\addplot [
color=blue,
mark size=0.9pt,
only marks,
mark=*,
mark options={solid,fill=red!88!black,draw=black,line width=0.1pt},
forget plot
]
coordinates{
 (231,0.929824561403509) 
};
\addplot [
color=blue,
mark size=0.9pt,
only marks,
mark=*,
mark options={solid,fill=red!88!black,draw=black,line width=0.1pt},
forget plot
]
coordinates{
 (209,0.918639053254438) 
};
\addplot [
color=blue,
mark size=0.9pt,
only marks,
mark=*,
mark options={solid,fill=red!92!black,draw=black,line width=0.1pt},
forget plot
]
coordinates{
 (286,0.951902368987796) 
};
\addplot [
color=blue,
mark size=0.9pt,
only marks,
mark=*,
mark options={solid,fill=red!92!black,draw=black,line width=0.1pt},
forget plot
]
coordinates{
 (233,0.953439888811675) 
};
\addplot [
color=blue,
mark size=0.9pt,
only marks,
mark=*,
mark options={solid,fill=red!92!black,draw=black,line width=0.1pt},
forget plot
]
coordinates{
 (262,0.957142857142857) 
};
\addplot [
color=blue,
mark size=0.9pt,
only marks,
mark=*,
mark options={solid,fill=red!96!black,draw=black,line width=0.1pt},
forget plot
]
coordinates{
 (236,0.948699929725931) 
};
\addplot [
color=blue,
mark size=0.9pt,
only marks,
mark=*,
mark options={solid,fill=red!96!black,draw=black,line width=0.1pt},
forget plot
]
coordinates{
 (249,0.957507082152974) 
};
\addplot [
color=blue,
mark size=0.9pt,
only marks,
mark=*,
mark options={solid,fill=red,draw=black,line width=0.1pt},
forget plot
]
coordinates{
 (178,0.886706948640483) 
};
\addplot [
color=blue,
mark size=0.9pt,
only marks,
mark=*,
mark options={solid,fill=mycolor1,draw=black,line width=0.1pt},
forget plot
]
coordinates{
 (168,0.920704845814978) 
};
\addplot [
color=blue,
mark size=0.9pt,
only marks,
mark=*,
mark options={solid,fill=mycolor1,draw=black,line width=0.1pt},
forget plot
]
coordinates{
 (160,0.924198250728863) 
};
\addplot [
color=blue,
mark size=0.9pt,
only marks,
mark=*,
mark options={solid,fill=mycolor2,draw=black,line width=0.1pt},
forget plot
]
coordinates{
 (180,0.930635838150289) 
};
\addplot [
color=blue,
mark size=0.9pt,
only marks,
mark=*,
mark options={solid,fill=mycolor2,draw=black,line width=0.1pt},
forget plot
]
coordinates{
 (163,0.930664760543245) 
};
\addplot [
color=blue,
mark size=0.9pt,
only marks,
mark=*,
mark options={solid,fill=mycolor3,draw=black,line width=0.1pt},
forget plot
]
coordinates{
 (143,0.92557111274871) 
};
\addplot [
color=blue,
mark size=0.9pt,
only marks,
mark=*,
mark options={solid,fill=mycolor4,draw=black,line width=0.1pt},
forget plot
]
coordinates{
 (162,0.932871972318339) 
};
\addplot [
color=blue,
mark size=0.9pt,
only marks,
mark=*,
mark options={solid,fill=mycolor5,draw=black,line width=0.1pt},
forget plot
]
coordinates{
 (138,0.920634920634921) 
};
\addplot [
color=blue,
mark size=0.9pt,
only marks,
mark=*,
mark options={solid,fill=mycolor5,draw=black,line width=0.1pt},
forget plot
]
coordinates{
 (139,0.943342776203966) 
};
\addplot [
color=blue,
mark size=0.9pt,
only marks,
mark=*,
mark options={solid,fill=mycolor6,draw=black,line width=0.1pt},
forget plot
]
coordinates{
 (134,0.928208846990573) 
};
\addplot [
color=blue,
mark size=0.9pt,
only marks,
mark=*,
mark options={solid,fill=mycolor7,draw=black,line width=0.1pt},
forget plot
]
coordinates{
 (100,0.864446165762975) 
};
\addplot [
color=blue,
mark size=0.9pt,
only marks,
mark=*,
mark options={solid,fill=mycolor8,draw=black,line width=0.1pt},
forget plot
]
coordinates{
 (101,0.841221374045801) 
};
\addplot [
color=blue,
mark size=0.9pt,
only marks,
mark=*,
mark options={solid,fill=mycolor9,draw=black,line width=0.1pt},
forget plot
]
coordinates{
 (101,0.869629629629629) 
};
\addplot [
color=blue,
mark size=0.9pt,
only marks,
mark=*,
mark options={solid,fill=mycolor10,draw=black,line width=0.1pt},
forget plot
]
coordinates{
 (89,0.893862815884476) 
};
\addplot [
color=blue,
mark size=0.9pt,
only marks,
mark=*,
mark options={solid,fill=mycolor11,draw=black,line width=0.1pt},
forget plot
]
coordinates{
 (84,0.901387874360847) 
};
\addplot [
color=blue,
mark size=0.9pt,
only marks,
mark=*,
mark options={solid,fill=mycolor12,draw=black,line width=0.1pt},
forget plot
]
coordinates{
 (81,0.874051593323217) 
};
\addplot [
color=blue,
mark size=0.9pt,
only marks,
mark=*,
mark options={solid,fill=mycolor13,draw=black,line width=0.1pt},
forget plot
]
coordinates{
 (87,0.913528591352859) 
};
\addplot [
color=blue,
mark size=0.9pt,
only marks,
mark=*,
mark options={solid,fill=mycolor14,draw=black,line width=0.1pt},
forget plot
]
coordinates{
 (77,0.897079276773296) 
};
\addplot [
color=blue,
mark size=0.9pt,
only marks,
mark=*,
mark options={solid,fill=mycolor15,draw=black,line width=0.1pt},
forget plot
]
coordinates{
 (60,0.880239520958084) 
};
\addplot [
color=blue,
mark size=0.9pt,
only marks,
mark=*,
mark options={solid,fill=mycolor16,draw=black,line width=0.1pt},
forget plot
]
coordinates{
 (57,0.87832973362131) 
};
\addplot [
color=blue,
mark size=0.9pt,
only marks,
mark=*,
mark options={solid,fill=mycolor17,draw=black,line width=0.1pt},
forget plot
]
coordinates{
 (54,0.797720797720798) 
};
\addplot [
color=blue,
mark size=0.9pt,
only marks,
mark=*,
mark options={solid,fill=mycolor18,draw=black,line width=0.1pt},
forget plot
]
coordinates{
 (54,0.88) 
};
\addplot [
color=blue,
mark size=0.9pt,
only marks,
mark=*,
mark options={solid,fill=mycolor19,draw=black,line width=0.1pt},
forget plot
]
coordinates{
 (47,0.814132104454685) 
};
\addplot [
color=blue,
mark size=0.9pt,
only marks,
mark=*,
mark options={solid,fill=mycolor20,draw=black,line width=0.1pt},
forget plot
]
coordinates{
 (43,0.796516231195566) 
};
\addplot [
color=blue,
mark size=0.9pt,
only marks,
mark=*,
mark options={solid,fill=mycolor21,draw=black,line width=0.1pt},
forget plot
]
coordinates{
 (47,0.890921885995777) 
};
\addplot [
color=blue,
mark size=0.9pt,
only marks,
mark=*,
mark options={solid,fill=mycolor22,draw=black,line width=0.1pt},
forget plot
]
coordinates{
 (44,0.872135994087213) 
};
\addplot [
color=blue,
mark size=0.9pt,
only marks,
mark=*,
mark options={solid,fill=mycolor23,draw=black,line width=0.1pt},
forget plot
]
coordinates{
 (42,0.832535885167464) 
};
\addplot [
color=blue,
mark size=0.9pt,
only marks,
mark=*,
mark options={solid,fill=mycolor24,draw=black,line width=0.1pt},
forget plot
]
coordinates{
 (37,0.827862289831865) 
};
\addplot [
color=blue,
mark size=0.9pt,
only marks,
mark=*,
mark options={solid,fill=mycolor25,draw=black,line width=0.1pt},
forget plot
]
coordinates{
 (35,0.880056777856636) 
};
\addplot [
color=blue,
mark size=0.9pt,
only marks,
mark=*,
mark options={solid,fill=mycolor26,draw=black,line width=0.1pt},
forget plot
]
coordinates{
 (35,0.81037037037037) 
};
\addplot [
color=blue,
mark size=0.9pt,
only marks,
mark=*,
mark options={solid,fill=mycolor27,draw=black,line width=0.1pt},
forget plot
]
coordinates{
 (30,0.81259842519685) 
};
\addplot [
color=blue,
mark size=0.9pt,
only marks,
mark=*,
mark options={solid,fill=mycolor28,draw=black,line width=0.1pt},
forget plot
]
coordinates{
 (30,0.778089887640449) 
};
\addplot [
color=blue,
mark size=0.9pt,
only marks,
mark=*,
mark options={solid,fill=mycolor29,draw=black,line width=0.1pt},
forget plot
]
coordinates{
 (29,0.791666666666667) 
};
\addplot [
color=blue,
mark size=0.9pt,
only marks,
mark=*,
mark options={solid,fill=mycolor30,draw=black,line width=0.1pt},
forget plot
]
coordinates{
 (27,0.809523809523809) 
};
\addplot [
color=blue,
mark size=0.9pt,
only marks,
mark=*,
mark options={solid,fill=mycolor31,draw=black,line width=0.1pt},
forget plot
]
coordinates{
 (27,0.84012539184953) 
};
\addplot [
color=blue,
mark size=0.9pt,
only marks,
mark=*,
mark options={solid,fill=mycolor32,draw=black,line width=0.1pt},
forget plot
]
coordinates{
 (25,0.811614730878187) 
};
\addplot [
color=blue,
mark size=0.9pt,
only marks,
mark=*,
mark options={solid,fill=mycolor33,draw=black,line width=0.1pt},
forget plot
]
coordinates{
 (25,0.827169811320755) 
};
\addplot [
color=blue,
mark size=0.9pt,
only marks,
mark=*,
mark options={solid,fill=mycolor34,draw=black,line width=0.1pt},
forget plot
]
coordinates{
 (24,0.815450643776824) 
};
\addplot [
color=blue,
mark size=0.9pt,
only marks,
mark=*,
mark options={solid,fill=mycolor35,draw=black,line width=0.1pt},
forget plot
]
coordinates{
 (24,0.843490304709141) 
};
\addplot [
color=blue,
mark size=0.9pt,
only marks,
mark=*,
mark options={solid,fill=mycolor36,draw=black,line width=0.1pt},
forget plot
]
coordinates{
 (21,0.733676975945017) 
};
\addplot [
color=blue,
mark size=0.9pt,
only marks,
mark=*,
mark options={solid,fill=mycolor37,draw=black,line width=0.1pt},
forget plot
]
coordinates{
 (22,0.817365269461078) 
};
\addplot [
color=blue,
mark size=0.9pt,
only marks,
mark=*,
mark options={solid,fill=mycolor38,draw=black,line width=0.1pt},
forget plot
]
coordinates{
 (22,0.76030534351145) 
};
\addplot [
color=blue,
mark size=0.9pt,
only marks,
mark=*,
mark options={solid,fill=mycolor39,draw=black,line width=0.1pt},
forget plot
]
coordinates{
 (13,0.503267973856209) 
};
\addplot [
color=blue,
mark size=0.9pt,
only marks,
mark=*,
mark options={solid,fill=mycolor40,draw=black,line width=0.1pt},
forget plot
]
coordinates{
 (11,0.701214574898785) 
};
\addplot [
color=blue,
mark size=0.9pt,
only marks,
mark=*,
mark options={solid,fill=red!68!black,draw=black,line width=0.1pt},
forget plot
]
coordinates{
 (350,0.964936886395512) 
};
\addplot [
color=blue,
mark size=0.9pt,
only marks,
mark=*,
mark options={solid,fill=red!68!black,draw=black,line width=0.1pt},
forget plot
]
coordinates{
 (269,0.885022692889561) 
};
\addplot [
color=blue,
mark size=0.9pt,
only marks,
mark=*,
mark options={solid,fill=red!68!black,draw=black,line width=0.1pt},
forget plot
]
coordinates{
 (360,0.967559943582511) 
};
\addplot [
color=blue,
mark size=0.9pt,
only marks,
mark=*,
mark options={solid,fill=red!68!black,draw=black,line width=0.1pt},
forget plot
]
coordinates{
 (248,0.885196374622356) 
};
\addplot [
color=blue,
mark size=0.9pt,
only marks,
mark=*,
mark options={solid,fill=red!72!black,draw=black,line width=0.1pt},
forget plot
]
coordinates{
 (202,0.900221729490022) 
};
\addplot [
color=blue,
mark size=0.9pt,
only marks,
mark=*,
mark options={solid,fill=red!72!black,draw=black,line width=0.1pt},
forget plot
]
coordinates{
 (333,0.96575821104123) 
};
\addplot [
color=blue,
mark size=0.9pt,
only marks,
mark=*,
mark options={solid,fill=red!72!black,draw=black,line width=0.1pt},
forget plot
]
coordinates{
 (352,0.963687150837989) 
};
\addplot [
color=blue,
mark size=0.9pt,
only marks,
mark=*,
mark options={solid,fill=red!72!black,draw=black,line width=0.1pt},
forget plot
]
coordinates{
 (302,0.963585434173669) 
};
\addplot [
color=blue,
mark size=0.9pt,
only marks,
mark=*,
mark options={solid,fill=red!76!black,draw=black,line width=0.1pt},
forget plot
]
coordinates{
 (323,0.959717314487632) 
};
\addplot [
color=blue,
mark size=0.9pt,
only marks,
mark=*,
mark options={solid,fill=red!76!black,draw=black,line width=0.1pt},
forget plot
]
coordinates{
 (246,0.951167728237792) 
};
\addplot [
color=blue,
mark size=0.9pt,
only marks,
mark=*,
mark options={solid,fill=red!76!black,draw=black,line width=0.1pt},
forget plot
]
coordinates{
 (217,0.929248723559446) 
};
\addplot [
color=blue,
mark size=0.9pt,
only marks,
mark=*,
mark options={solid,fill=red!80!black,draw=black,line width=0.1pt},
forget plot
]
coordinates{
 (268,0.934971098265896) 
};
\addplot [
color=blue,
mark size=0.9pt,
only marks,
mark=*,
mark options={solid,fill=red!80!black,draw=black,line width=0.1pt},
forget plot
]
coordinates{
 (256,0.940841054882395) 
};
\addplot [
color=blue,
mark size=0.9pt,
only marks,
mark=*,
mark options={solid,fill=red!80!black,draw=black,line width=0.1pt},
forget plot
]
coordinates{
 (331,0.97071129707113) 
};
\addplot [
color=blue,
mark size=0.9pt,
only marks,
mark=*,
mark options={solid,fill=red!84!black,draw=black,line width=0.1pt},
forget plot
]
coordinates{
 (269,0.972554539057002) 
};
\addplot [
color=blue,
mark size=0.9pt,
only marks,
mark=*,
mark options={solid,fill=red!84!black,draw=black,line width=0.1pt},
forget plot
]
coordinates{
 (284,0.960334029227557) 
};
\addplot [
color=blue,
mark size=0.9pt,
only marks,
mark=*,
mark options={solid,fill=red!84!black,draw=black,line width=0.1pt},
forget plot
]
coordinates{
 (280,0.962237762237762) 
};
\addplot [
color=blue,
mark size=0.9pt,
only marks,
mark=*,
mark options={solid,fill=red!88!black,draw=black,line width=0.1pt},
forget plot
]
coordinates{
 (274,0.955602536997886) 
};
\addplot [
color=blue,
mark size=0.9pt,
only marks,
mark=*,
mark options={solid,fill=red!88!black,draw=black,line width=0.1pt},
forget plot
]
coordinates{
 (220,0.925952552120776) 
};
\addplot [
color=blue,
mark size=0.9pt,
only marks,
mark=*,
mark options={solid,fill=red!92!black,draw=black,line width=0.1pt},
forget plot
]
coordinates{
 (220,0.932374100719424) 
};
\addplot [
color=blue,
mark size=0.9pt,
only marks,
mark=*,
mark options={solid,fill=red!92!black,draw=black,line width=0.1pt},
forget plot
]
coordinates{
 (217,0.946695095948827) 
};
\addplot [
color=blue,
mark size=0.9pt,
only marks,
mark=*,
mark options={solid,fill=red!92!black,draw=black,line width=0.1pt},
forget plot
]
coordinates{
 (238,0.965901183020181) 
};
\addplot [
color=blue,
mark size=0.9pt,
only marks,
mark=*,
mark options={solid,fill=red!96!black,draw=black,line width=0.1pt},
forget plot
]
coordinates{
 (163,0.884498480243161) 
};
\addplot [
color=blue,
mark size=0.9pt,
only marks,
mark=*,
mark options={solid,fill=red!96!black,draw=black,line width=0.1pt},
forget plot
]
coordinates{
 (218,0.948699929725931) 
};
\addplot [
color=blue,
mark size=0.9pt,
only marks,
mark=*,
mark options={solid,fill=red,draw=black,line width=0.1pt},
forget plot
]
coordinates{
 (207,0.946022727272727) 
};
\addplot [
color=blue,
mark size=0.9pt,
only marks,
mark=*,
mark options={solid,fill=mycolor1,draw=black,line width=0.1pt},
forget plot
]
coordinates{
 (198,0.951236749116608) 
};
\addplot [
color=blue,
mark size=0.9pt,
only marks,
mark=*,
mark options={solid,fill=mycolor1,draw=black,line width=0.1pt},
forget plot
]
coordinates{
 (194,0.942512420156139) 
};
\addplot [
color=blue,
mark size=0.9pt,
only marks,
mark=*,
mark options={solid,fill=mycolor2,draw=black,line width=0.1pt},
forget plot
]
coordinates{
 (154,0.934876989869754) 
};
\addplot [
color=blue,
mark size=0.9pt,
only marks,
mark=*,
mark options={solid,fill=mycolor2,draw=black,line width=0.1pt},
forget plot
]
coordinates{
 (154,0.944323933477946) 
};
\addplot [
color=blue,
mark size=0.9pt,
only marks,
mark=*,
mark options={solid,fill=mycolor3,draw=black,line width=0.1pt},
forget plot
]
coordinates{
 (190,0.956583629893238) 
};
\addplot [
color=blue,
mark size=0.9pt,
only marks,
mark=*,
mark options={solid,fill=mycolor4,draw=black,line width=0.1pt},
forget plot
]
coordinates{
 (168,0.952447552447552) 
};
\addplot [
color=blue,
mark size=0.9pt,
only marks,
mark=*,
mark options={solid,fill=mycolor5,draw=black,line width=0.1pt},
forget plot
]
coordinates{
 (117,0.893961708394698) 
};
\addplot [
color=blue,
mark size=0.9pt,
only marks,
mark=*,
mark options={solid,fill=mycolor5,draw=black,line width=0.1pt},
forget plot
]
coordinates{
 (93,0.894930198383541) 
};
\addplot [
color=blue,
mark size=0.9pt,
only marks,
mark=*,
mark options={solid,fill=mycolor6,draw=black,line width=0.1pt},
forget plot
]
coordinates{
 (123,0.915523465703971) 
};
\addplot [
color=blue,
mark size=0.9pt,
only marks,
mark=*,
mark options={solid,fill=mycolor7,draw=black,line width=0.1pt},
forget plot
]
coordinates{
 (102,0.867017280240421) 
};
\addplot [
color=blue,
mark size=0.9pt,
only marks,
mark=*,
mark options={solid,fill=mycolor8,draw=black,line width=0.1pt},
forget plot
]
coordinates{
 (103,0.888272033310201) 
};
\addplot [
color=blue,
mark size=0.9pt,
only marks,
mark=*,
mark options={solid,fill=mycolor9,draw=black,line width=0.1pt},
forget plot
]
coordinates{
 (92,0.869899923017706) 
};
\addplot [
color=blue,
mark size=0.9pt,
only marks,
mark=*,
mark options={solid,fill=mycolor10,draw=black,line width=0.1pt},
forget plot
]
coordinates{
 (96,0.913597733711048) 
};
\addplot [
color=blue,
mark size=0.9pt,
only marks,
mark=*,
mark options={solid,fill=mycolor11,draw=black,line width=0.1pt},
forget plot
]
coordinates{
 (79,0.840513983371126) 
};
\addplot [
color=blue,
mark size=0.9pt,
only marks,
mark=*,
mark options={solid,fill=mycolor12,draw=black,line width=0.1pt},
forget plot
]
coordinates{
 (66,0.824561403508772) 
};
\addplot [
color=blue,
mark size=0.9pt,
only marks,
mark=*,
mark options={solid,fill=mycolor13,draw=black,line width=0.1pt},
forget plot
]
coordinates{
 (74,0.884502923976608) 
};
\addplot [
color=blue,
mark size=0.9pt,
only marks,
mark=*,
mark options={solid,fill=mycolor14,draw=black,line width=0.1pt},
forget plot
]
coordinates{
 (72,0.895144264602393) 
};
\addplot [
color=blue,
mark size=0.9pt,
only marks,
mark=*,
mark options={solid,fill=mycolor15,draw=black,line width=0.1pt},
forget plot
]
coordinates{
 (75,0.885072655217966) 
};
\addplot [
color=blue,
mark size=0.9pt,
only marks,
mark=*,
mark options={solid,fill=mycolor16,draw=black,line width=0.1pt},
forget plot
]
coordinates{
 (64,0.876534296028881) 
};
\addplot [
color=blue,
mark size=0.9pt,
only marks,
mark=*,
mark options={solid,fill=mycolor17,draw=black,line width=0.1pt},
forget plot
]
coordinates{
 (62,0.888106966924701) 
};
\addplot [
color=blue,
mark size=0.9pt,
only marks,
mark=*,
mark options={solid,fill=mycolor18,draw=black,line width=0.1pt},
forget plot
]
coordinates{
 (54,0.847250509164969) 
};
\addplot [
color=blue,
mark size=0.9pt,
only marks,
mark=*,
mark options={solid,fill=mycolor19,draw=black,line width=0.1pt},
forget plot
]
coordinates{
 (43,0.801872074882995) 
};
\addplot [
color=blue,
mark size=0.9pt,
only marks,
mark=*,
mark options={solid,fill=mycolor20,draw=black,line width=0.1pt},
forget plot
]
coordinates{
 (48,0.851421188630491) 
};
\addplot [
color=blue,
mark size=0.9pt,
only marks,
mark=*,
mark options={solid,fill=mycolor21,draw=black,line width=0.1pt},
forget plot
]
coordinates{
 (42,0.802873104549082) 
};
\addplot [
color=blue,
mark size=0.9pt,
only marks,
mark=*,
mark options={solid,fill=mycolor22,draw=black,line width=0.1pt},
forget plot
]
coordinates{
 (40,0.834258524980174) 
};
\addplot [
color=blue,
mark size=0.9pt,
only marks,
mark=*,
mark options={solid,fill=mycolor23,draw=black,line width=0.1pt},
forget plot
]
coordinates{
 (40,0.814927646610815) 
};
\addplot [
color=blue,
mark size=0.9pt,
only marks,
mark=*,
mark options={solid,fill=mycolor24,draw=black,line width=0.1pt},
forget plot
]
coordinates{
 (38,0.861493836113125) 
};
\addplot [
color=blue,
mark size=0.9pt,
only marks,
mark=*,
mark options={solid,fill=mycolor25,draw=black,line width=0.1pt},
forget plot
]
coordinates{
 (33,0.818466353677621) 
};
\addplot [
color=blue,
mark size=0.9pt,
only marks,
mark=*,
mark options={solid,fill=mycolor26,draw=black,line width=0.1pt},
forget plot
]
coordinates{
 (34,0.833222591362126) 
};
\addplot [
color=blue,
mark size=0.9pt,
only marks,
mark=*,
mark options={solid,fill=mycolor27,draw=black,line width=0.1pt},
forget plot
]
coordinates{
 (29,0.804861580013504) 
};
\addplot [
color=blue,
mark size=0.9pt,
only marks,
mark=*,
mark options={solid,fill=mycolor28,draw=black,line width=0.1pt},
forget plot
]
coordinates{
 (30,0.828685258964143) 
};
\addplot [
color=blue,
mark size=0.9pt,
only marks,
mark=*,
mark options={solid,fill=mycolor29,draw=black,line width=0.1pt},
forget plot
]
coordinates{
 (28,0.773712737127371) 
};
\addplot [
color=blue,
mark size=0.9pt,
only marks,
mark=*,
mark options={solid,fill=mycolor30,draw=black,line width=0.1pt},
forget plot
]
coordinates{
 (31,0.767326732673267) 
};
\addplot [
color=blue,
mark size=0.9pt,
only marks,
mark=*,
mark options={solid,fill=mycolor31,draw=black,line width=0.1pt},
forget plot
]
coordinates{
 (28,0.809005083514887) 
};
\addplot [
color=blue,
mark size=0.9pt,
only marks,
mark=*,
mark options={solid,fill=mycolor32,draw=black,line width=0.1pt},
forget plot
]
coordinates{
 (26,0.774603174603175) 
};
\addplot [
color=blue,
mark size=0.9pt,
only marks,
mark=*,
mark options={solid,fill=mycolor33,draw=black,line width=0.1pt},
forget plot
]
coordinates{
 (24,0.7859375) 
};
\addplot [
color=blue,
mark size=0.9pt,
only marks,
mark=*,
mark options={solid,fill=mycolor34,draw=black,line width=0.1pt},
forget plot
]
coordinates{
 (22,0.804012345679012) 
};
\addplot [
color=blue,
mark size=0.9pt,
only marks,
mark=*,
mark options={solid,fill=mycolor35,draw=black,line width=0.1pt},
forget plot
]
coordinates{
 (23,0.802666666666667) 
};
\addplot [
color=blue,
mark size=0.9pt,
only marks,
mark=*,
mark options={solid,fill=mycolor36,draw=black,line width=0.1pt},
forget plot
]
coordinates{
 (18,0.749022673964034) 
};
\addplot [
color=blue,
mark size=0.9pt,
only marks,
mark=*,
mark options={solid,fill=mycolor37,draw=black,line width=0.1pt},
forget plot
]
coordinates{
 (20,0.724046140195208) 
};
\addplot [
color=blue,
mark size=0.9pt,
only marks,
mark=*,
mark options={solid,fill=mycolor38,draw=black,line width=0.1pt},
forget plot
]
coordinates{
 (16,0.689313517338995) 
};
\addplot [
color=blue,
mark size=0.9pt,
only marks,
mark=*,
mark options={solid,fill=mycolor39,draw=black,line width=0.1pt},
forget plot
]
coordinates{
 (15,0.725352112676056) 
};
\addplot [
color=blue,
mark size=0.9pt,
only marks,
mark=*,
mark options={solid,fill=mycolor40,draw=black,line width=0.1pt},
forget plot
]
coordinates{
 (13,0.414866032843561) 
};
\addplot [
color=blue,
mark size=0.9pt,
only marks,
mark=*,
mark options={solid,fill=red!68!black,draw=black,line width=0.1pt},
forget plot
]
coordinates{
 (279,0.928930366116296) 
};
\addplot [
color=blue,
mark size=0.9pt,
only marks,
mark=*,
mark options={solid,fill=red!68!black,draw=black,line width=0.1pt},
forget plot
]
coordinates{
 (356,0.954577218728162) 
};
\addplot [
color=blue,
mark size=0.9pt,
only marks,
mark=*,
mark options={solid,fill=red!68!black,draw=black,line width=0.1pt},
forget plot
]
coordinates{
 (362,0.955369595536959) 
};
\addplot [
color=blue,
mark size=0.9pt,
only marks,
mark=*,
mark options={solid,fill=red!68!black,draw=black,line width=0.1pt},
forget plot
]
coordinates{
 (357,0.955801104972376) 
};
\addplot [
color=blue,
mark size=0.9pt,
only marks,
mark=*,
mark options={solid,fill=red!72!black,draw=black,line width=0.1pt},
forget plot
]
coordinates{
 (385,0.956224350205198) 
};
\addplot [
color=blue,
mark size=0.9pt,
only marks,
mark=*,
mark options={solid,fill=red!72!black,draw=black,line width=0.1pt},
forget plot
]
coordinates{
 (309,0.938833570412518) 
};
\addplot [
color=blue,
mark size=0.9pt,
only marks,
mark=*,
mark options={solid,fill=red!72!black,draw=black,line width=0.1pt},
forget plot
]
coordinates{
 (300,0.944405348346235) 
};
\addplot [
color=blue,
mark size=0.9pt,
only marks,
mark=*,
mark options={solid,fill=red!72!black,draw=black,line width=0.1pt},
forget plot
]
coordinates{
 (274,0.926164874551971) 
};
\addplot [
color=blue,
mark size=0.9pt,
only marks,
mark=*,
mark options={solid,fill=red!76!black,draw=black,line width=0.1pt},
forget plot
]
coordinates{
 (329,0.945224719101123) 
};
\addplot [
color=blue,
mark size=0.9pt,
only marks,
mark=*,
mark options={solid,fill=red!76!black,draw=black,line width=0.1pt},
forget plot
]
coordinates{
 (299,0.931929824561403) 
};
\addplot [
color=blue,
mark size=0.9pt,
only marks,
mark=*,
mark options={solid,fill=red!76!black,draw=black,line width=0.1pt},
forget plot
]
coordinates{
 (355,0.956461644782308) 
};
\addplot [
color=blue,
mark size=0.9pt,
only marks,
mark=*,
mark options={solid,fill=red!80!black,draw=black,line width=0.1pt},
forget plot
]
coordinates{
 (257,0.924177396280401) 
};
\addplot [
color=blue,
mark size=0.9pt,
only marks,
mark=*,
mark options={solid,fill=red!80!black,draw=black,line width=0.1pt},
forget plot
]
coordinates{
 (335,0.955431754874652) 
};
\addplot [
color=blue,
mark size=0.9pt,
only marks,
mark=*,
mark options={solid,fill=red!80!black,draw=black,line width=0.1pt},
forget plot
]
coordinates{
 (266,0.935988620199146) 
};
\addplot [
color=blue,
mark size=0.9pt,
only marks,
mark=*,
mark options={solid,fill=red!84!black,draw=black,line width=0.1pt},
forget plot
]
coordinates{
 (328,0.949480968858131) 
};
\addplot [
color=blue,
mark size=0.9pt,
only marks,
mark=*,
mark options={solid,fill=red!84!black,draw=black,line width=0.1pt},
forget plot
]
coordinates{
 (262,0.957550452331246) 
};
\addplot [
color=blue,
mark size=0.9pt,
only marks,
mark=*,
mark options={solid,fill=red!84!black,draw=black,line width=0.1pt},
forget plot
]
coordinates{
 (281,0.948824343015214) 
};
\addplot [
color=blue,
mark size=0.9pt,
only marks,
mark=*,
mark options={solid,fill=red!88!black,draw=black,line width=0.1pt},
forget plot
]
coordinates{
 (298,0.96551724137931) 
};
\addplot [
color=blue,
mark size=0.9pt,
only marks,
mark=*,
mark options={solid,fill=red!88!black,draw=black,line width=0.1pt},
forget plot
]
coordinates{
 (249,0.943213296398892) 
};
\addplot [
color=blue,
mark size=0.9pt,
only marks,
mark=*,
mark options={solid,fill=red!92!black,draw=black,line width=0.1pt},
forget plot
]
coordinates{
 (234,0.941836019621584) 
};
\addplot [
color=blue,
mark size=0.9pt,
only marks,
mark=*,
mark options={solid,fill=red!92!black,draw=black,line width=0.1pt},
forget plot
]
coordinates{
 (256,0.948311509303928) 
};
\addplot [
color=blue,
mark size=0.9pt,
only marks,
mark=*,
mark options={solid,fill=red!92!black,draw=black,line width=0.1pt},
forget plot
]
coordinates{
 (254,0.942657342657343) 
};
\addplot [
color=blue,
mark size=0.9pt,
only marks,
mark=*,
mark options={solid,fill=red!96!black,draw=black,line width=0.1pt},
forget plot
]
coordinates{
 (192,0.868035190615836) 
};
\addplot [
color=blue,
mark size=0.9pt,
only marks,
mark=*,
mark options={solid,fill=red!96!black,draw=black,line width=0.1pt},
forget plot
]
coordinates{
 (228,0.943632567849687) 
};
\addplot [
color=blue,
mark size=0.9pt,
only marks,
mark=*,
mark options={solid,fill=red,draw=black,line width=0.1pt},
forget plot
]
coordinates{
 (178,0.923516797712652) 
};
\addplot [
color=blue,
mark size=0.9pt,
only marks,
mark=*,
mark options={solid,fill=mycolor1,draw=black,line width=0.1pt},
forget plot
]
coordinates{
 (212,0.923076923076923) 
};
\addplot [
color=blue,
mark size=0.9pt,
only marks,
mark=*,
mark options={solid,fill=mycolor1,draw=black,line width=0.1pt},
forget plot
]
coordinates{
 (161,0.922190201729106) 
};
\addplot [
color=blue,
mark size=0.9pt,
only marks,
mark=*,
mark options={solid,fill=mycolor2,draw=black,line width=0.1pt},
forget plot
]
coordinates{
 (181,0.922859164897381) 
};
\addplot [
color=blue,
mark size=0.9pt,
only marks,
mark=*,
mark options={solid,fill=mycolor2,draw=black,line width=0.1pt},
forget plot
]
coordinates{
 (186,0.947960618846695) 
};
\addplot [
color=blue,
mark size=0.9pt,
only marks,
mark=*,
mark options={solid,fill=mycolor3,draw=black,line width=0.1pt},
forget plot
]
coordinates{
 (151,0.915645277577505) 
};
\addplot [
color=blue,
mark size=0.9pt,
only marks,
mark=*,
mark options={solid,fill=mycolor4,draw=black,line width=0.1pt},
forget plot
]
coordinates{
 (148,0.865470852017937) 
};
\addplot [
color=blue,
mark size=0.9pt,
only marks,
mark=*,
mark options={solid,fill=mycolor5,draw=black,line width=0.1pt},
forget plot
]
coordinates{
 (139,0.928923293455313) 
};
\addplot [
color=blue,
mark size=0.9pt,
only marks,
mark=*,
mark options={solid,fill=mycolor5,draw=black,line width=0.1pt},
forget plot
]
coordinates{
 (143,0.92) 
};
\addplot [
color=blue,
mark size=0.9pt,
only marks,
mark=*,
mark options={solid,fill=mycolor6,draw=black,line width=0.1pt},
forget plot
]
coordinates{
 (144,0.934886908841672) 
};
\addplot [
color=blue,
mark size=0.9pt,
only marks,
mark=*,
mark options={solid,fill=mycolor7,draw=black,line width=0.1pt},
forget plot
]
coordinates{
 (105,0.888888888888889) 
};
\addplot [
color=blue,
mark size=0.9pt,
only marks,
mark=*,
mark options={solid,fill=mycolor8,draw=black,line width=0.1pt},
forget plot
]
coordinates{
 (98,0.889374090247453) 
};
\addplot [
color=blue,
mark size=0.9pt,
only marks,
mark=*,
mark options={solid,fill=mycolor9,draw=black,line width=0.1pt},
forget plot
]
coordinates{
 (108,0.930327868852459) 
};
\addplot [
color=blue,
mark size=0.9pt,
only marks,
mark=*,
mark options={solid,fill=mycolor10,draw=black,line width=0.1pt},
forget plot
]
coordinates{
 (90,0.908037653874004) 
};
\addplot [
color=blue,
mark size=0.9pt,
only marks,
mark=*,
mark options={solid,fill=mycolor11,draw=black,line width=0.1pt},
forget plot
]
coordinates{
 (84,0.893430656934307) 
};
\addplot [
color=blue,
mark size=0.9pt,
only marks,
mark=*,
mark options={solid,fill=mycolor12,draw=black,line width=0.1pt},
forget plot
]
coordinates{
 (69,0.835463258785942) 
};
\addplot [
color=blue,
mark size=0.9pt,
only marks,
mark=*,
mark options={solid,fill=mycolor13,draw=black,line width=0.1pt},
forget plot
]
coordinates{
 (71,0.887060583395662) 
};
\addplot [
color=blue,
mark size=0.9pt,
only marks,
mark=*,
mark options={solid,fill=mycolor14,draw=black,line width=0.1pt},
forget plot
]
coordinates{
 (64,0.816358024691358) 
};
\addplot [
color=blue,
mark size=0.9pt,
only marks,
mark=*,
mark options={solid,fill=mycolor15,draw=black,line width=0.1pt},
forget plot
]
coordinates{
 (65,0.858433734939759) 
};
\addplot [
color=blue,
mark size=0.9pt,
only marks,
mark=*,
mark options={solid,fill=mycolor16,draw=black,line width=0.1pt},
forget plot
]
coordinates{
 (60,0.831360946745562) 
};
\addplot [
color=blue,
mark size=0.9pt,
only marks,
mark=*,
mark options={solid,fill=mycolor17,draw=black,line width=0.1pt},
forget plot
]
coordinates{
 (58,0.821939586645469) 
};
\addplot [
color=blue,
mark size=0.9pt,
only marks,
mark=*,
mark options={solid,fill=mycolor18,draw=black,line width=0.1pt},
forget plot
]
coordinates{
 (53,0.836193447737909) 
};
\addplot [
color=blue,
mark size=0.9pt,
only marks,
mark=*,
mark options={solid,fill=mycolor19,draw=black,line width=0.1pt},
forget plot
]
coordinates{
 (54,0.84692417739628) 
};
\addplot [
color=blue,
mark size=0.9pt,
only marks,
mark=*,
mark options={solid,fill=mycolor20,draw=black,line width=0.1pt},
forget plot
]
coordinates{
 (45,0.802091112770724) 
};
\addplot [
color=blue,
mark size=0.9pt,
only marks,
mark=*,
mark options={solid,fill=mycolor21,draw=black,line width=0.1pt},
forget plot
]
coordinates{
 (44,0.794952681388013) 
};
\addplot [
color=blue,
mark size=0.9pt,
only marks,
mark=*,
mark options={solid,fill=mycolor22,draw=black,line width=0.1pt},
forget plot
]
coordinates{
 (42,0.78975131876413) 
};
\addplot [
color=blue,
mark size=0.9pt,
only marks,
mark=*,
mark options={solid,fill=mycolor23,draw=black,line width=0.1pt},
forget plot
]
coordinates{
 (36,0.795509222133119) 
};
\addplot [
color=blue,
mark size=0.9pt,
only marks,
mark=*,
mark options={solid,fill=mycolor24,draw=black,line width=0.1pt},
forget plot
]
coordinates{
 (40,0.838056680161943) 
};
\addplot [
color=blue,
mark size=0.9pt,
only marks,
mark=*,
mark options={solid,fill=mycolor25,draw=black,line width=0.1pt},
forget plot
]
coordinates{
 (35,0.802996254681648) 
};
\addplot [
color=blue,
mark size=0.9pt,
only marks,
mark=*,
mark options={solid,fill=mycolor26,draw=black,line width=0.1pt},
forget plot
]
coordinates{
 (30,0.825816485225505) 
};
\addplot [
color=blue,
mark size=0.9pt,
only marks,
mark=*,
mark options={solid,fill=mycolor27,draw=black,line width=0.1pt},
forget plot
]
coordinates{
 (31,0.836532097948378) 
};
\addplot [
color=blue,
mark size=0.9pt,
only marks,
mark=*,
mark options={solid,fill=mycolor28,draw=black,line width=0.1pt},
forget plot
]
coordinates{
 (33,0.781542898341745) 
};
\addplot [
color=blue,
mark size=0.9pt,
only marks,
mark=*,
mark options={solid,fill=mycolor29,draw=black,line width=0.1pt},
forget plot
]
coordinates{
 (29,0.811111111111111) 
};
\addplot [
color=blue,
mark size=0.9pt,
only marks,
mark=*,
mark options={solid,fill=mycolor30,draw=black,line width=0.1pt},
forget plot
]
coordinates{
 (30,0.806842480399144) 
};
\addplot [
color=blue,
mark size=0.9pt,
only marks,
mark=*,
mark options={solid,fill=mycolor31,draw=black,line width=0.1pt},
forget plot
]
coordinates{
 (30,0.813507290867229) 
};
\addplot [
color=blue,
mark size=0.9pt,
only marks,
mark=*,
mark options={solid,fill=mycolor32,draw=black,line width=0.1pt},
forget plot
]
coordinates{
 (27,0.817610062893082) 
};
\addplot [
color=blue,
mark size=0.9pt,
only marks,
mark=*,
mark options={solid,fill=mycolor33,draw=black,line width=0.1pt},
forget plot
]
coordinates{
 (22,0.768) 
};
\addplot [
color=blue,
mark size=0.9pt,
only marks,
mark=*,
mark options={solid,fill=mycolor34,draw=black,line width=0.1pt},
forget plot
]
coordinates{
 (23,0.802278275020342) 
};
\addplot [
color=blue,
mark size=0.9pt,
only marks,
mark=*,
mark options={solid,fill=mycolor35,draw=black,line width=0.1pt},
forget plot
]
coordinates{
 (24,0.782546494992847) 
};
\addplot [
color=blue,
mark size=0.9pt,
only marks,
mark=*,
mark options={solid,fill=mycolor36,draw=black,line width=0.1pt},
forget plot
]
coordinates{
 (23,0.810483870967742) 
};
\addplot [
color=blue,
mark size=0.9pt,
only marks,
mark=*,
mark options={solid,fill=mycolor37,draw=black,line width=0.1pt},
forget plot
]
coordinates{
 (18,0.764627659574468) 
};
\addplot [
color=blue,
mark size=0.9pt,
only marks,
mark=*,
mark options={solid,fill=mycolor38,draw=black,line width=0.1pt},
forget plot
]
coordinates{
 (16,0.772486772486773) 
};
\addplot [
color=blue,
mark size=0.9pt,
only marks,
mark=*,
mark options={solid,fill=mycolor39,draw=black,line width=0.1pt},
forget plot
]
coordinates{
 (16,0.757249378624689) 
};
\addplot [
color=blue,
mark size=0.9pt,
only marks,
mark=*,
mark options={solid,fill=mycolor40,draw=black,line width=0.1pt},
forget plot
]
coordinates{
 (10,0.081180811808118) 
};
\addplot [
color=blue,
mark size=0.9pt,
only marks,
mark=*,
mark options={solid,fill=red!68!black,draw=black,line width=0.1pt},
forget plot
]
coordinates{
 (348,0.959326788218794) 
};
\addplot [
color=blue,
mark size=0.9pt,
only marks,
mark=*,
mark options={solid,fill=red!68!black,draw=black,line width=0.1pt},
forget plot
]
coordinates{
 (359,0.961864406779661) 
};
\addplot [
color=blue,
mark size=0.9pt,
only marks,
mark=*,
mark options={solid,fill=red!68!black,draw=black,line width=0.1pt},
forget plot
]
coordinates{
 (378,0.968242766407904) 
};
\addplot [
color=blue,
mark size=0.9pt,
only marks,
mark=*,
mark options={solid,fill=red!68!black,draw=black,line width=0.1pt},
forget plot
]
coordinates{
 (364,0.959660297239915) 
};
\addplot [
color=blue,
mark size=0.9pt,
only marks,
mark=*,
mark options={solid,fill=red!72!black,draw=black,line width=0.1pt},
forget plot
]
coordinates{
 (304,0.951977401129943) 
};
\addplot [
color=blue,
mark size=0.9pt,
only marks,
mark=*,
mark options={solid,fill=red!72!black,draw=black,line width=0.1pt},
forget plot
]
coordinates{
 (350,0.960227272727273) 
};
\addplot [
color=blue,
mark size=0.9pt,
only marks,
mark=*,
mark options={solid,fill=red!72!black,draw=black,line width=0.1pt},
forget plot
]
coordinates{
 (314,0.948884089272858) 
};
\addplot [
color=blue,
mark size=0.9pt,
only marks,
mark=*,
mark options={solid,fill=red!72!black,draw=black,line width=0.1pt},
forget plot
]
coordinates{
 (268,0.93644996347699) 
};
\addplot [
color=blue,
mark size=0.9pt,
only marks,
mark=*,
mark options={solid,fill=red!76!black,draw=black,line width=0.1pt},
forget plot
]
coordinates{
 (328,0.955903271692745) 
};
\addplot [
color=blue,
mark size=0.9pt,
only marks,
mark=*,
mark options={solid,fill=red!76!black,draw=black,line width=0.1pt},
forget plot
]
coordinates{
 (246,0.932748538011696) 
};
\addplot [
color=blue,
mark size=0.9pt,
only marks,
mark=*,
mark options={solid,fill=red!76!black,draw=black,line width=0.1pt},
forget plot
]
coordinates{
 (230,0.928571428571428) 
};
\addplot [
color=blue,
mark size=0.9pt,
only marks,
mark=*,
mark options={solid,fill=red!80!black,draw=black,line width=0.1pt},
forget plot
]
coordinates{
 (290,0.959553695955369) 
};
\addplot [
color=blue,
mark size=0.9pt,
only marks,
mark=*,
mark options={solid,fill=red!80!black,draw=black,line width=0.1pt},
forget plot
]
coordinates{
 (195,0.923976608187134) 
};
\addplot [
color=blue,
mark size=0.9pt,
only marks,
mark=*,
mark options={solid,fill=red!80!black,draw=black,line width=0.1pt},
forget plot
]
coordinates{
 (316,0.963752665245202) 
};
\addplot [
color=blue,
mark size=0.9pt,
only marks,
mark=*,
mark options={solid,fill=red!84!black,draw=black,line width=0.1pt},
forget plot
]
coordinates{
 (299,0.96113074204947) 
};
\addplot [
color=blue,
mark size=0.9pt,
only marks,
mark=*,
mark options={solid,fill=red!84!black,draw=black,line width=0.1pt},
forget plot
]
coordinates{
 (249,0.941092973740241) 
};
\addplot [
color=blue,
mark size=0.9pt,
only marks,
mark=*,
mark options={solid,fill=red!84!black,draw=black,line width=0.1pt},
forget plot
]
coordinates{
 (211,0.878533231474408) 
};
\addplot [
color=blue,
mark size=0.9pt,
only marks,
mark=*,
mark options={solid,fill=red!88!black,draw=black,line width=0.1pt},
forget plot
]
coordinates{
 (237,0.934105720492397) 
};
\addplot [
color=blue,
mark size=0.9pt,
only marks,
mark=*,
mark options={solid,fill=red!88!black,draw=black,line width=0.1pt},
forget plot
]
coordinates{
 (237,0.936727272727273) 
};
\addplot [
color=blue,
mark size=0.9pt,
only marks,
mark=*,
mark options={solid,fill=red!92!black,draw=black,line width=0.1pt},
forget plot
]
coordinates{
 (187,0.878347360367253) 
};
\addplot [
color=blue,
mark size=0.9pt,
only marks,
mark=*,
mark options={solid,fill=red!92!black,draw=black,line width=0.1pt},
forget plot
]
coordinates{
 (265,0.9593837535014) 
};
\addplot [
color=blue,
mark size=0.9pt,
only marks,
mark=*,
mark options={solid,fill=red!92!black,draw=black,line width=0.1pt},
forget plot
]
coordinates{
 (235,0.952313167259786) 
};
\addplot [
color=blue,
mark size=0.9pt,
only marks,
mark=*,
mark options={solid,fill=red!96!black,draw=black,line width=0.1pt},
forget plot
]
coordinates{
 (218,0.941595441595441) 
};
\addplot [
color=blue,
mark size=0.9pt,
only marks,
mark=*,
mark options={solid,fill=red!96!black,draw=black,line width=0.1pt},
forget plot
]
coordinates{
 (206,0.934936350777935) 
};
\addplot [
color=blue,
mark size=0.9pt,
only marks,
mark=*,
mark options={solid,fill=red,draw=black,line width=0.1pt},
forget plot
]
coordinates{
 (164,0.877566539923954) 
};
\addplot [
color=blue,
mark size=0.9pt,
only marks,
mark=*,
mark options={solid,fill=mycolor1,draw=black,line width=0.1pt},
forget plot
]
coordinates{
 (211,0.949681077250177) 
};
\addplot [
color=blue,
mark size=0.9pt,
only marks,
mark=*,
mark options={solid,fill=mycolor1,draw=black,line width=0.1pt},
forget plot
]
coordinates{
 (174,0.92836676217765) 
};
\addplot [
color=blue,
mark size=0.9pt,
only marks,
mark=*,
mark options={solid,fill=mycolor2,draw=black,line width=0.1pt},
forget plot
]
coordinates{
 (204,0.946695095948827) 
};
\addplot [
color=blue,
mark size=0.9pt,
only marks,
mark=*,
mark options={solid,fill=mycolor2,draw=black,line width=0.1pt},
forget plot
]
coordinates{
 (142,0.859714928732183) 
};
\addplot [
color=blue,
mark size=0.9pt,
only marks,
mark=*,
mark options={solid,fill=mycolor3,draw=black,line width=0.1pt},
forget plot
]
coordinates{
 (134,0.922302158273381) 
};
\addplot [
color=blue,
mark size=0.9pt,
only marks,
mark=*,
mark options={solid,fill=mycolor4,draw=black,line width=0.1pt},
forget plot
]
coordinates{
 (173,0.940677966101695) 
};
\addplot [
color=blue,
mark size=0.9pt,
only marks,
mark=*,
mark options={solid,fill=mycolor5,draw=black,line width=0.1pt},
forget plot
]
coordinates{
 (156,0.933428775948461) 
};
\addplot [
color=blue,
mark size=0.9pt,
only marks,
mark=*,
mark options={solid,fill=mycolor5,draw=black,line width=0.1pt},
forget plot
]
coordinates{
 (152,0.934191702432046) 
};
\addplot [
color=blue,
mark size=0.9pt,
only marks,
mark=*,
mark options={solid,fill=mycolor6,draw=black,line width=0.1pt},
forget plot
]
coordinates{
 (88,0.83015873015873) 
};
\addplot [
color=blue,
mark size=0.9pt,
only marks,
mark=*,
mark options={solid,fill=mycolor7,draw=black,line width=0.1pt},
forget plot
]
coordinates{
 (112,0.896151053013798) 
};
\addplot [
color=blue,
mark size=0.9pt,
only marks,
mark=*,
mark options={solid,fill=mycolor8,draw=black,line width=0.1pt},
forget plot
]
coordinates{
 (92,0.881670533642691) 
};
\addplot [
color=blue,
mark size=0.9pt,
only marks,
mark=*,
mark options={solid,fill=mycolor9,draw=black,line width=0.1pt},
forget plot
]
coordinates{
 (112,0.935805991440799) 
};
\addplot [
color=blue,
mark size=0.9pt,
only marks,
mark=*,
mark options={solid,fill=mycolor10,draw=black,line width=0.1pt},
forget plot
]
coordinates{
 (78,0.892512972572276) 
};
\addplot [
color=blue,
mark size=0.9pt,
only marks,
mark=*,
mark options={solid,fill=mycolor11,draw=black,line width=0.1pt},
forget plot
]
coordinates{
 (90,0.915679442508711) 
};
\addplot [
color=blue,
mark size=0.9pt,
only marks,
mark=*,
mark options={solid,fill=mycolor12,draw=black,line width=0.1pt},
forget plot
]
coordinates{
 (77,0.897168405365127) 
};
\addplot [
color=blue,
mark size=0.9pt,
only marks,
mark=*,
mark options={solid,fill=mycolor13,draw=black,line width=0.1pt},
forget plot
]
coordinates{
 (74,0.881458966565349) 
};
\addplot [
color=blue,
mark size=0.9pt,
only marks,
mark=*,
mark options={solid,fill=mycolor14,draw=black,line width=0.1pt},
forget plot
]
coordinates{
 (66,0.883211678832117) 
};
\addplot [
color=blue,
mark size=0.9pt,
only marks,
mark=*,
mark options={solid,fill=mycolor15,draw=black,line width=0.1pt},
forget plot
]
coordinates{
 (65,0.877840909090909) 
};
\addplot [
color=blue,
mark size=0.9pt,
only marks,
mark=*,
mark options={solid,fill=mycolor16,draw=black,line width=0.1pt},
forget plot
]
coordinates{
 (68,0.898509581263307) 
};
\addplot [
color=blue,
mark size=0.9pt,
only marks,
mark=*,
mark options={solid,fill=mycolor17,draw=black,line width=0.1pt},
forget plot
]
coordinates{
 (57,0.861731843575419) 
};
\addplot [
color=blue,
mark size=0.9pt,
only marks,
mark=*,
mark options={solid,fill=mycolor18,draw=black,line width=0.1pt},
forget plot
]
coordinates{
 (61,0.871259568545581) 
};
\addplot [
color=blue,
mark size=0.9pt,
only marks,
mark=*,
mark options={solid,fill=mycolor19,draw=black,line width=0.1pt},
forget plot
]
coordinates{
 (54,0.869318181818182) 
};
\addplot [
color=blue,
mark size=0.9pt,
only marks,
mark=*,
mark options={solid,fill=mycolor20,draw=black,line width=0.1pt},
forget plot
]
coordinates{
 (47,0.847430830039526) 
};
\addplot [
color=blue,
mark size=0.9pt,
only marks,
mark=*,
mark options={solid,fill=mycolor21,draw=black,line width=0.1pt},
forget plot
]
coordinates{
 (40,0.814756671899529) 
};
\addplot [
color=blue,
mark size=0.9pt,
only marks,
mark=*,
mark options={solid,fill=mycolor22,draw=black,line width=0.1pt},
forget plot
]
coordinates{
 (43,0.814814814814815) 
};
\addplot [
color=blue,
mark size=0.9pt,
only marks,
mark=*,
mark options={solid,fill=mycolor23,draw=black,line width=0.1pt},
forget plot
]
coordinates{
 (38,0.800656275635767) 
};
\addplot [
color=blue,
mark size=0.9pt,
only marks,
mark=*,
mark options={solid,fill=mycolor24,draw=black,line width=0.1pt},
forget plot
]
coordinates{
 (31,0.795819935691318) 
};
\addplot [
color=blue,
mark size=0.9pt,
only marks,
mark=*,
mark options={solid,fill=mycolor25,draw=black,line width=0.1pt},
forget plot
]
coordinates{
 (32,0.813688212927757) 
};
\addplot [
color=blue,
mark size=0.9pt,
only marks,
mark=*,
mark options={solid,fill=mycolor26,draw=black,line width=0.1pt},
forget plot
]
coordinates{
 (30,0.845605700712589) 
};
\addplot [
color=blue,
mark size=0.9pt,
only marks,
mark=*,
mark options={solid,fill=mycolor27,draw=black,line width=0.1pt},
forget plot
]
coordinates{
 (32,0.831967213114754) 
};
\addplot [
color=blue,
mark size=0.9pt,
only marks,
mark=*,
mark options={solid,fill=mycolor28,draw=black,line width=0.1pt},
forget plot
]
coordinates{
 (32,0.821917808219178) 
};
\addplot [
color=blue,
mark size=0.9pt,
only marks,
mark=*,
mark options={solid,fill=mycolor29,draw=black,line width=0.1pt},
forget plot
]
coordinates{
 (26,0.808444096950743) 
};
\addplot [
color=blue,
mark size=0.9pt,
only marks,
mark=*,
mark options={solid,fill=mycolor30,draw=black,line width=0.1pt},
forget plot
]
coordinates{
 (29,0.7936) 
};
\addplot [
color=blue,
mark size=0.9pt,
only marks,
mark=*,
mark options={solid,fill=mycolor31,draw=black,line width=0.1pt},
forget plot
]
coordinates{
 (27,0.789712556732224) 
};
\addplot [
color=blue,
mark size=0.9pt,
only marks,
mark=*,
mark options={solid,fill=mycolor32,draw=black,line width=0.1pt},
forget plot
]
coordinates{
 (29,0.778435239973701) 
};
\addplot [
color=blue,
mark size=0.9pt,
only marks,
mark=*,
mark options={solid,fill=mycolor33,draw=black,line width=0.1pt},
forget plot
]
coordinates{
 (25,0.793650793650794) 
};
\addplot [
color=blue,
mark size=0.9pt,
only marks,
mark=*,
mark options={solid,fill=mycolor34,draw=black,line width=0.1pt},
forget plot
]
coordinates{
 (22,0.79874213836478) 
};
\addplot [
color=blue,
mark size=0.9pt,
only marks,
mark=*,
mark options={solid,fill=mycolor35,draw=black,line width=0.1pt},
forget plot
]
coordinates{
 (24,0.819007686932215) 
};
\addplot [
color=blue,
mark size=0.9pt,
only marks,
mark=*,
mark options={solid,fill=mycolor36,draw=black,line width=0.1pt},
forget plot
]
coordinates{
 (22,0.785333333333333) 
};
\addplot [
color=blue,
mark size=0.9pt,
only marks,
mark=*,
mark options={solid,fill=mycolor37,draw=black,line width=0.1pt},
forget plot
]
coordinates{
 (15,0.782292298362644) 
};
\addplot [
color=blue,
mark size=0.9pt,
only marks,
mark=*,
mark options={solid,fill=mycolor38,draw=black,line width=0.1pt},
forget plot
]
coordinates{
 (21,0.717832957110609) 
};
\addplot [
color=blue,
mark size=0.9pt,
only marks,
mark=*,
mark options={solid,fill=mycolor39,draw=black,line width=0.1pt},
forget plot
]
coordinates{
 (20,0.805946791862285) 
};
\addplot [
color=blue,
mark size=0.9pt,
only marks,
mark=*,
mark options={solid,fill=mycolor40,draw=black,line width=0.1pt},
forget plot
]
coordinates{
 (9,0.70697012802276) 
};
\addplot [
color=blue,
mark size=0.9pt,
only marks,
mark=*,
mark options={solid,fill=red!68!black,draw=black,line width=0.1pt},
forget plot
]
coordinates{
 (324,0.965322009907997) 
};
\addplot [
color=blue,
mark size=0.9pt,
only marks,
mark=*,
mark options={solid,fill=red!68!black,draw=black,line width=0.1pt},
forget plot
]
coordinates{
 (344,0.954967834167262) 
};
\addplot [
color=blue,
mark size=0.9pt,
only marks,
mark=*,
mark options={solid,fill=red!68!black,draw=black,line width=0.1pt},
forget plot
]
coordinates{
 (306,0.952116585704372) 
};
\addplot [
color=blue,
mark size=0.9pt,
only marks,
mark=*,
mark options={solid,fill=red!68!black,draw=black,line width=0.1pt},
forget plot
]
coordinates{
 (380,0.961702127659574) 
};
\addplot [
color=blue,
mark size=0.9pt,
only marks,
mark=*,
mark options={solid,fill=red!72!black,draw=black,line width=0.1pt},
forget plot
]
coordinates{
 (294,0.959603118355776) 
};
\addplot [
color=blue,
mark size=0.9pt,
only marks,
mark=*,
mark options={solid,fill=red!72!black,draw=black,line width=0.1pt},
forget plot
]
coordinates{
 (254,0.928312816799421) 
};
\addplot [
color=blue,
mark size=0.9pt,
only marks,
mark=*,
mark options={solid,fill=red!72!black,draw=black,line width=0.1pt},
forget plot
]
coordinates{
 (324,0.96218487394958) 
};
\addplot [
color=blue,
mark size=0.9pt,
only marks,
mark=*,
mark options={solid,fill=red!72!black,draw=black,line width=0.1pt},
forget plot
]
coordinates{
 (309,0.947444204463643) 
};
\addplot [
color=blue,
mark size=0.9pt,
only marks,
mark=*,
mark options={solid,fill=red!76!black,draw=black,line width=0.1pt},
forget plot
]
coordinates{
 (328,0.96011396011396) 
};
\addplot [
color=blue,
mark size=0.9pt,
only marks,
mark=*,
mark options={solid,fill=red!76!black,draw=black,line width=0.1pt},
forget plot
]
coordinates{
 (275,0.951359084406295) 
};
\addplot [
color=blue,
mark size=0.9pt,
only marks,
mark=*,
mark options={solid,fill=red!76!black,draw=black,line width=0.1pt},
forget plot
]
coordinates{
 (293,0.950749464668094) 
};
\addplot [
color=blue,
mark size=0.9pt,
only marks,
mark=*,
mark options={solid,fill=red!80!black,draw=black,line width=0.1pt},
forget plot
]
coordinates{
 (298,0.955523672883788) 
};
\addplot [
color=blue,
mark size=0.9pt,
only marks,
mark=*,
mark options={solid,fill=red!80!black,draw=black,line width=0.1pt},
forget plot
]
coordinates{
 (291,0.959603118355776) 
};
\addplot [
color=blue,
mark size=0.9pt,
only marks,
mark=*,
mark options={solid,fill=red!80!black,draw=black,line width=0.1pt},
forget plot
]
coordinates{
 (289,0.951236749116608) 
};
\addplot [
color=blue,
mark size=0.9pt,
only marks,
mark=*,
mark options={solid,fill=red!84!black,draw=black,line width=0.1pt},
forget plot
]
coordinates{
 (296,0.950634696755994) 
};
\addplot [
color=blue,
mark size=0.9pt,
only marks,
mark=*,
mark options={solid,fill=red!84!black,draw=black,line width=0.1pt},
forget plot
]
coordinates{
 (203,0.870109546165884) 
};
\addplot [
color=blue,
mark size=0.9pt,
only marks,
mark=*,
mark options={solid,fill=red!84!black,draw=black,line width=0.1pt},
forget plot
]
coordinates{
 (251,0.941672522839072) 
};
\addplot [
color=blue,
mark size=0.9pt,
only marks,
mark=*,
mark options={solid,fill=red!88!black,draw=black,line width=0.1pt},
forget plot
]
coordinates{
 (198,0.878274268104776) 
};
\addplot [
color=blue,
mark size=0.9pt,
only marks,
mark=*,
mark options={solid,fill=red!88!black,draw=black,line width=0.1pt},
forget plot
]
coordinates{
 (226,0.920883820384889) 
};
\addplot [
color=blue,
mark size=0.9pt,
only marks,
mark=*,
mark options={solid,fill=red!92!black,draw=black,line width=0.1pt},
forget plot
]
coordinates{
 (246,0.947519769949676) 
};
\addplot [
color=blue,
mark size=0.9pt,
only marks,
mark=*,
mark options={solid,fill=red!92!black,draw=black,line width=0.1pt},
forget plot
]
coordinates{
 (170,0.910029498525074) 
};
\addplot [
color=blue,
mark size=0.9pt,
only marks,
mark=*,
mark options={solid,fill=red!92!black,draw=black,line width=0.1pt},
forget plot
]
coordinates{
 (228,0.952924393723252) 
};
\addplot [
color=blue,
mark size=0.9pt,
only marks,
mark=*,
mark options={solid,fill=red!96!black,draw=black,line width=0.1pt},
forget plot
]
coordinates{
 (223,0.951841359773371) 
};
\addplot [
color=blue,
mark size=0.9pt,
only marks,
mark=*,
mark options={solid,fill=red!96!black,draw=black,line width=0.1pt},
forget plot
]
coordinates{
 (249,0.956583629893238) 
};
\addplot [
color=blue,
mark size=0.9pt,
only marks,
mark=*,
mark options={solid,fill=red,draw=black,line width=0.1pt},
forget plot
]
coordinates{
 (158,0.87024087024087) 
};
\addplot [
color=blue,
mark size=0.9pt,
only marks,
mark=*,
mark options={solid,fill=mycolor1,draw=black,line width=0.1pt},
forget plot
]
coordinates{
 (202,0.954738330975955) 
};
\addplot [
color=blue,
mark size=0.9pt,
only marks,
mark=*,
mark options={solid,fill=mycolor1,draw=black,line width=0.1pt},
forget plot
]
coordinates{
 (148,0.860182370820669) 
};
\addplot [
color=blue,
mark size=0.9pt,
only marks,
mark=*,
mark options={solid,fill=mycolor2,draw=black,line width=0.1pt},
forget plot
]
coordinates{
 (184,0.915032679738562) 
};
\addplot [
color=blue,
mark size=0.9pt,
only marks,
mark=*,
mark options={solid,fill=mycolor2,draw=black,line width=0.1pt},
forget plot
]
coordinates{
 (137,0.860046911649726) 
};
\addplot [
color=blue,
mark size=0.9pt,
only marks,
mark=*,
mark options={solid,fill=mycolor3,draw=black,line width=0.1pt},
forget plot
]
coordinates{
 (160,0.926007326007326) 
};
\addplot [
color=blue,
mark size=0.9pt,
only marks,
mark=*,
mark options={solid,fill=mycolor4,draw=black,line width=0.1pt},
forget plot
]
coordinates{
 (131,0.861154446177847) 
};
\addplot [
color=blue,
mark size=0.9pt,
only marks,
mark=*,
mark options={solid,fill=mycolor5,draw=black,line width=0.1pt},
forget plot
]
coordinates{
 (138,0.861989205859676) 
};
\addplot [
color=blue,
mark size=0.9pt,
only marks,
mark=*,
mark options={solid,fill=mycolor5,draw=black,line width=0.1pt},
forget plot
]
coordinates{
 (133,0.92138133725202) 
};
\addplot [
color=blue,
mark size=0.9pt,
only marks,
mark=*,
mark options={solid,fill=mycolor6,draw=black,line width=0.1pt},
forget plot
]
coordinates{
 (117,0.922855082912761) 
};
\addplot [
color=blue,
mark size=0.9pt,
only marks,
mark=*,
mark options={solid,fill=mycolor7,draw=black,line width=0.1pt},
forget plot
]
coordinates{
 (111,0.903597122302158) 
};
\addplot [
color=blue,
mark size=0.9pt,
only marks,
mark=*,
mark options={solid,fill=mycolor8,draw=black,line width=0.1pt},
forget plot
]
coordinates{
 (92,0.902616279069767) 
};
\addplot [
color=blue,
mark size=0.9pt,
only marks,
mark=*,
mark options={solid,fill=mycolor9,draw=black,line width=0.1pt},
forget plot
]
coordinates{
 (109,0.928923293455313) 
};
\addplot [
color=blue,
mark size=0.9pt,
only marks,
mark=*,
mark options={solid,fill=mycolor10,draw=black,line width=0.1pt},
forget plot
]
coordinates{
 (93,0.892392049348869) 
};
\addplot [
color=blue,
mark size=0.9pt,
only marks,
mark=*,
mark options={solid,fill=mycolor11,draw=black,line width=0.1pt},
forget plot
]
coordinates{
 (98,0.935323383084577) 
};
\addplot [
color=blue,
mark size=0.9pt,
only marks,
mark=*,
mark options={solid,fill=mycolor12,draw=black,line width=0.1pt},
forget plot
]
coordinates{
 (87,0.899042004421518) 
};
\addplot [
color=blue,
mark size=0.9pt,
only marks,
mark=*,
mark options={solid,fill=mycolor13,draw=black,line width=0.1pt},
forget plot
]
coordinates{
 (83,0.923829489867226) 
};
\addplot [
color=blue,
mark size=0.9pt,
only marks,
mark=*,
mark options={solid,fill=mycolor14,draw=black,line width=0.1pt},
forget plot
]
coordinates{
 (65,0.886750555144337) 
};
\addplot [
color=blue,
mark size=0.9pt,
only marks,
mark=*,
mark options={solid,fill=mycolor15,draw=black,line width=0.1pt},
forget plot
]
coordinates{
 (70,0.880116959064327) 
};
\addplot [
color=blue,
mark size=0.9pt,
only marks,
mark=*,
mark options={solid,fill=mycolor16,draw=black,line width=0.1pt},
forget plot
]
coordinates{
 (66,0.867292225201072) 
};
\addplot [
color=blue,
mark size=0.9pt,
only marks,
mark=*,
mark options={solid,fill=mycolor17,draw=black,line width=0.1pt},
forget plot
]
coordinates{
 (60,0.865609924190214) 
};
\addplot [
color=blue,
mark size=0.9pt,
only marks,
mark=*,
mark options={solid,fill=mycolor18,draw=black,line width=0.1pt},
forget plot
]
coordinates{
 (49,0.815642458100559) 
};
\addplot [
color=blue,
mark size=0.9pt,
only marks,
mark=*,
mark options={solid,fill=mycolor19,draw=black,line width=0.1pt},
forget plot
]
coordinates{
 (48,0.799365582870737) 
};
\addplot [
color=blue,
mark size=0.9pt,
only marks,
mark=*,
mark options={solid,fill=mycolor20,draw=black,line width=0.1pt},
forget plot
]
coordinates{
 (48,0.824742268041237) 
};
\addplot [
color=blue,
mark size=0.9pt,
only marks,
mark=*,
mark options={solid,fill=mycolor21,draw=black,line width=0.1pt},
forget plot
]
coordinates{
 (37,0.819512195121951) 
};
\addplot [
color=blue,
mark size=0.9pt,
only marks,
mark=*,
mark options={solid,fill=mycolor22,draw=black,line width=0.1pt},
forget plot
]
coordinates{
 (40,0.819018404907975) 
};
\addplot [
color=blue,
mark size=0.9pt,
only marks,
mark=*,
mark options={solid,fill=mycolor23,draw=black,line width=0.1pt},
forget plot
]
coordinates{
 (40,0.82648401826484) 
};
\addplot [
color=blue,
mark size=0.9pt,
only marks,
mark=*,
mark options={solid,fill=mycolor24,draw=black,line width=0.1pt},
forget plot
]
coordinates{
 (35,0.783387622149837) 
};
\addplot [
color=blue,
mark size=0.9pt,
only marks,
mark=*,
mark options={solid,fill=mycolor25,draw=black,line width=0.1pt},
forget plot
]
coordinates{
 (33,0.795962509012257) 
};
\addplot [
color=blue,
mark size=0.9pt,
only marks,
mark=*,
mark options={solid,fill=mycolor26,draw=black,line width=0.1pt},
forget plot
]
coordinates{
 (31,0.773246329526917) 
};
\addplot [
color=blue,
mark size=0.9pt,
only marks,
mark=*,
mark options={solid,fill=mycolor27,draw=black,line width=0.1pt},
forget plot
]
coordinates{
 (28,0.819334389857369) 
};
\addplot [
color=blue,
mark size=0.9pt,
only marks,
mark=*,
mark options={solid,fill=mycolor28,draw=black,line width=0.1pt},
forget plot
]
coordinates{
 (28,0.738095238095238) 
};
\addplot [
color=blue,
mark size=0.9pt,
only marks,
mark=*,
mark options={solid,fill=mycolor29,draw=black,line width=0.1pt},
forget plot
]
coordinates{
 (26,0.78010878010878) 
};
\addplot [
color=blue,
mark size=0.9pt,
only marks,
mark=*,
mark options={solid,fill=mycolor30,draw=black,line width=0.1pt},
forget plot
]
coordinates{
 (28,0.771565495207668) 
};
\addplot [
color=blue,
mark size=0.9pt,
only marks,
mark=*,
mark options={solid,fill=mycolor31,draw=black,line width=0.1pt},
forget plot
]
coordinates{
 (28,0.754601226993865) 
};
\addplot [
color=blue,
mark size=0.9pt,
only marks,
mark=*,
mark options={solid,fill=mycolor32,draw=black,line width=0.1pt},
forget plot
]
coordinates{
 (24,0.773480662983425) 
};
\addplot [
color=blue,
mark size=0.9pt,
only marks,
mark=*,
mark options={solid,fill=mycolor33,draw=black,line width=0.1pt},
forget plot
]
coordinates{
 (20,0.723842195540309) 
};
\addplot [
color=blue,
mark size=0.9pt,
only marks,
mark=*,
mark options={solid,fill=mycolor34,draw=black,line width=0.1pt},
forget plot
]
coordinates{
 (21,0.788663967611336) 
};
\addplot [
color=blue,
mark size=0.9pt,
only marks,
mark=*,
mark options={solid,fill=mycolor35,draw=black,line width=0.1pt},
forget plot
]
coordinates{
 (24,0.773584905660377) 
};
\addplot [
color=blue,
mark size=0.9pt,
only marks,
mark=*,
mark options={solid,fill=mycolor36,draw=black,line width=0.1pt},
forget plot
]
coordinates{
 (18,0.727117194183062) 
};
\addplot [
color=blue,
mark size=0.9pt,
only marks,
mark=*,
mark options={solid,fill=mycolor37,draw=black,line width=0.1pt},
forget plot
]
coordinates{
 (19,0.770359501100513) 
};
\addplot [
color=blue,
mark size=0.9pt,
only marks,
mark=*,
mark options={solid,fill=mycolor38,draw=black,line width=0.1pt},
forget plot
]
coordinates{
 (14,0.755212355212355) 
};
\addplot [
color=blue,
mark size=0.9pt,
only marks,
mark=*,
mark options={solid,fill=mycolor39,draw=black,line width=0.1pt},
forget plot
]
coordinates{
 (15,0.718446601941747) 
};
\addplot [
color=blue,
mark size=0.9pt,
only marks,
mark=*,
mark options={solid,fill=mycolor40,draw=black,line width=0.1pt},
forget plot
]
coordinates{
 (17,0.673060156931125) 
};
\addplot [
color=blue,
mark size=0.9pt,
only marks,
mark=*,
mark options={solid,fill=red!68!black,draw=black,line width=0.1pt},
forget plot
]
coordinates{
 (325,0.95591322603219) 
};
\addplot [
color=blue,
mark size=0.9pt,
only marks,
mark=*,
mark options={solid,fill=red!68!black,draw=black,line width=0.1pt},
forget plot
]
coordinates{
 (302,0.952654232424677) 
};
\addplot [
color=blue,
mark size=0.9pt,
only marks,
mark=*,
mark options={solid,fill=red!68!black,draw=black,line width=0.1pt},
forget plot
]
coordinates{
 (309,0.958512160228898) 
};
\addplot [
color=blue,
mark size=0.9pt,
only marks,
mark=*,
mark options={solid,fill=red!68!black,draw=black,line width=0.1pt},
forget plot
]
coordinates{
 (299,0.955287437899219) 
};
\addplot [
color=blue,
mark size=0.9pt,
only marks,
mark=*,
mark options={solid,fill=red!72!black,draw=black,line width=0.1pt},
forget plot
]
coordinates{
 (283,0.954415954415954) 
};
\addplot [
color=blue,
mark size=0.9pt,
only marks,
mark=*,
mark options={solid,fill=red!72!black,draw=black,line width=0.1pt},
forget plot
]
coordinates{
 (332,0.968421052631579) 
};
\addplot [
color=blue,
mark size=0.9pt,
only marks,
mark=*,
mark options={solid,fill=red!72!black,draw=black,line width=0.1pt},
forget plot
]
coordinates{
 (327,0.972593113141251) 
};
\addplot [
color=blue,
mark size=0.9pt,
only marks,
mark=*,
mark options={solid,fill=red!72!black,draw=black,line width=0.1pt},
forget plot
]
coordinates{
 (328,0.97191011235955) 
};
\addplot [
color=blue,
mark size=0.9pt,
only marks,
mark=*,
mark options={solid,fill=red!76!black,draw=black,line width=0.1pt},
forget plot
]
coordinates{
 (337,0.957887223411849) 
};
\addplot [
color=blue,
mark size=0.9pt,
only marks,
mark=*,
mark options={solid,fill=red!76!black,draw=black,line width=0.1pt},
forget plot
]
coordinates{
 (256,0.941944847605225) 
};
\addplot [
color=blue,
mark size=0.9pt,
only marks,
mark=*,
mark options={solid,fill=red!76!black,draw=black,line width=0.1pt},
forget plot
]
coordinates{
 (280,0.958185683912119) 
};
\addplot [
color=blue,
mark size=0.9pt,
only marks,
mark=*,
mark options={solid,fill=red!80!black,draw=black,line width=0.1pt},
forget plot
]
coordinates{
 (299,0.951029098651526) 
};
\addplot [
color=blue,
mark size=0.9pt,
only marks,
mark=*,
mark options={solid,fill=red!80!black,draw=black,line width=0.1pt},
forget plot
]
coordinates{
 (232,0.877030162412993) 
};
\addplot [
color=blue,
mark size=0.9pt,
only marks,
mark=*,
mark options={solid,fill=red!80!black,draw=black,line width=0.1pt},
forget plot
]
coordinates{
 (222,0.945165945165945) 
};
\addplot [
color=blue,
mark size=0.9pt,
only marks,
mark=*,
mark options={solid,fill=red!84!black,draw=black,line width=0.1pt},
forget plot
]
coordinates{
 (198,0.880860876249039) 
};
\addplot [
color=blue,
mark size=0.9pt,
only marks,
mark=*,
mark options={solid,fill=red!84!black,draw=black,line width=0.1pt},
forget plot
]
coordinates{
 (257,0.947444204463643) 
};
\addplot [
color=blue,
mark size=0.9pt,
only marks,
mark=*,
mark options={solid,fill=red!84!black,draw=black,line width=0.1pt},
forget plot
]
coordinates{
 (204,0.871439568899153) 
};
\addplot [
color=blue,
mark size=0.9pt,
only marks,
mark=*,
mark options={solid,fill=red!88!black,draw=black,line width=0.1pt},
forget plot
]
coordinates{
 (303,0.969187675070028) 
};
\addplot [
color=blue,
mark size=0.9pt,
only marks,
mark=*,
mark options={solid,fill=red!88!black,draw=black,line width=0.1pt},
forget plot
]
coordinates{
 (204,0.937226277372263) 
};
\addplot [
color=blue,
mark size=0.9pt,
only marks,
mark=*,
mark options={solid,fill=red!92!black,draw=black,line width=0.1pt},
forget plot
]
coordinates{
 (259,0.961294862772695) 
};
\addplot [
color=blue,
mark size=0.9pt,
only marks,
mark=*,
mark options={solid,fill=red!92!black,draw=black,line width=0.1pt},
forget plot
]
coordinates{
 (191,0.940493468795356) 
};
\addplot [
color=blue,
mark size=0.9pt,
only marks,
mark=*,
mark options={solid,fill=red!92!black,draw=black,line width=0.1pt},
forget plot
]
coordinates{
 (209,0.935272727272727) 
};
\addplot [
color=blue,
mark size=0.9pt,
only marks,
mark=*,
mark options={solid,fill=red!96!black,draw=black,line width=0.1pt},
forget plot
]
coordinates{
 (164,0.909090909090909) 
};
\addplot [
color=blue,
mark size=0.9pt,
only marks,
mark=*,
mark options={solid,fill=red!96!black,draw=black,line width=0.1pt},
forget plot
]
coordinates{
 (227,0.950508474576271) 
};
\addplot [
color=blue,
mark size=0.9pt,
only marks,
mark=*,
mark options={solid,fill=red,draw=black,line width=0.1pt},
forget plot
]
coordinates{
 (185,0.932851985559567) 
};
\addplot [
color=blue,
mark size=0.9pt,
only marks,
mark=*,
mark options={solid,fill=mycolor1,draw=black,line width=0.1pt},
forget plot
]
coordinates{
 (164,0.875576036866359) 
};
\addplot [
color=blue,
mark size=0.9pt,
only marks,
mark=*,
mark options={solid,fill=mycolor1,draw=black,line width=0.1pt},
forget plot
]
coordinates{
 (195,0.951305575158786) 
};
\addplot [
color=blue,
mark size=0.9pt,
only marks,
mark=*,
mark options={solid,fill=mycolor2,draw=black,line width=0.1pt},
forget plot
]
coordinates{
 (201,0.951497860199715) 
};
\addplot [
color=blue,
mark size=0.9pt,
only marks,
mark=*,
mark options={solid,fill=mycolor2,draw=black,line width=0.1pt},
forget plot
]
coordinates{
 (140,0.881996974281392) 
};
\addplot [
color=blue,
mark size=0.9pt,
only marks,
mark=*,
mark options={solid,fill=mycolor3,draw=black,line width=0.1pt},
forget plot
]
coordinates{
 (146,0.950980392156863) 
};
\addplot [
color=blue,
mark size=0.9pt,
only marks,
mark=*,
mark options={solid,fill=mycolor4,draw=black,line width=0.1pt},
forget plot
]
coordinates{
 (179,0.958827634333566) 
};
\addplot [
color=blue,
mark size=0.9pt,
only marks,
mark=*,
mark options={solid,fill=mycolor5,draw=black,line width=0.1pt},
forget plot
]
coordinates{
 (122,0.932153392330383) 
};
\addplot [
color=blue,
mark size=0.9pt,
only marks,
mark=*,
mark options={solid,fill=mycolor5,draw=black,line width=0.1pt},
forget plot
]
coordinates{
 (115,0.865900383141762) 
};
\addplot [
color=blue,
mark size=0.9pt,
only marks,
mark=*,
mark options={solid,fill=mycolor6,draw=black,line width=0.1pt},
forget plot
]
coordinates{
 (138,0.929264909847434) 
};
\addplot [
color=blue,
mark size=0.9pt,
only marks,
mark=*,
mark options={solid,fill=mycolor7,draw=black,line width=0.1pt},
forget plot
]
coordinates{
 (107,0.896408839779005) 
};
\addplot [
color=blue,
mark size=0.9pt,
only marks,
mark=*,
mark options={solid,fill=mycolor8,draw=black,line width=0.1pt},
forget plot
]
coordinates{
 (107,0.909336941813261) 
};
\addplot [
color=blue,
mark size=0.9pt,
only marks,
mark=*,
mark options={solid,fill=mycolor9,draw=black,line width=0.1pt},
forget plot
]
coordinates{
 (87,0.869961977186312) 
};
\addplot [
color=blue,
mark size=0.9pt,
only marks,
mark=*,
mark options={solid,fill=mycolor10,draw=black,line width=0.1pt},
forget plot
]
coordinates{
 (80,0.860700389105058) 
};
\addplot [
color=blue,
mark size=0.9pt,
only marks,
mark=*,
mark options={solid,fill=mycolor11,draw=black,line width=0.1pt},
forget plot
]
coordinates{
 (84,0.86969696969697) 
};
\addplot [
color=blue,
mark size=0.9pt,
only marks,
mark=*,
mark options={solid,fill=mycolor12,draw=black,line width=0.1pt},
forget plot
]
coordinates{
 (62,0.80521597392013) 
};
\addplot [
color=blue,
mark size=0.9pt,
only marks,
mark=*,
mark options={solid,fill=mycolor13,draw=black,line width=0.1pt},
forget plot
]
coordinates{
 (76,0.89900426742532) 
};
\addplot [
color=blue,
mark size=0.9pt,
only marks,
mark=*,
mark options={solid,fill=mycolor14,draw=black,line width=0.1pt},
forget plot
]
coordinates{
 (75,0.892957746478873) 
};
\addplot [
color=blue,
mark size=0.9pt,
only marks,
mark=*,
mark options={solid,fill=mycolor15,draw=black,line width=0.1pt},
forget plot
]
coordinates{
 (78,0.902571041948579) 
};
\addplot [
color=blue,
mark size=0.9pt,
only marks,
mark=*,
mark options={solid,fill=mycolor16,draw=black,line width=0.1pt},
forget plot
]
coordinates{
 (59,0.85949177877429) 
};
\addplot [
color=blue,
mark size=0.9pt,
only marks,
mark=*,
mark options={solid,fill=mycolor17,draw=black,line width=0.1pt},
forget plot
]
coordinates{
 (54,0.88967468175389) 
};
\addplot [
color=blue,
mark size=0.9pt,
only marks,
mark=*,
mark options={solid,fill=mycolor18,draw=black,line width=0.1pt},
forget plot
]
coordinates{
 (49,0.877611940298507) 
};
\addplot [
color=blue,
mark size=0.9pt,
only marks,
mark=*,
mark options={solid,fill=mycolor19,draw=black,line width=0.1pt},
forget plot
]
coordinates{
 (47,0.860973888496824) 
};
\addplot [
color=blue,
mark size=0.9pt,
only marks,
mark=*,
mark options={solid,fill=mycolor20,draw=black,line width=0.1pt},
forget plot
]
coordinates{
 (44,0.829975825946817) 
};
\addplot [
color=blue,
mark size=0.9pt,
only marks,
mark=*,
mark options={solid,fill=mycolor21,draw=black,line width=0.1pt},
forget plot
]
coordinates{
 (42,0.885174418604651) 
};
\addplot [
color=blue,
mark size=0.9pt,
only marks,
mark=*,
mark options={solid,fill=mycolor22,draw=black,line width=0.1pt},
forget plot
]
coordinates{
 (42,0.811945117029863) 
};
\addplot [
color=blue,
mark size=0.9pt,
only marks,
mark=*,
mark options={solid,fill=mycolor23,draw=black,line width=0.1pt},
forget plot
]
coordinates{
 (40,0.838709677419355) 
};
\addplot [
color=blue,
mark size=0.9pt,
only marks,
mark=*,
mark options={solid,fill=mycolor24,draw=black,line width=0.1pt},
forget plot
]
coordinates{
 (43,0.835409836065574) 
};
\addplot [
color=blue,
mark size=0.9pt,
only marks,
mark=*,
mark options={solid,fill=mycolor25,draw=black,line width=0.1pt},
forget plot
]
coordinates{
 (39,0.80250347705146) 
};
\addplot [
color=blue,
mark size=0.9pt,
only marks,
mark=*,
mark options={solid,fill=mycolor26,draw=black,line width=0.1pt},
forget plot
]
coordinates{
 (30,0.804043545878694) 
};
\addplot [
color=blue,
mark size=0.9pt,
only marks,
mark=*,
mark options={solid,fill=mycolor27,draw=black,line width=0.1pt},
forget plot
]
coordinates{
 (33,0.805673758865248) 
};
\addplot [
color=blue,
mark size=0.9pt,
only marks,
mark=*,
mark options={solid,fill=mycolor28,draw=black,line width=0.1pt},
forget plot
]
coordinates{
 (29,0.775038520801233) 
};
\addplot [
color=blue,
mark size=0.9pt,
only marks,
mark=*,
mark options={solid,fill=mycolor29,draw=black,line width=0.1pt},
forget plot
]
coordinates{
 (31,0.799139167862267) 
};
\addplot [
color=blue,
mark size=0.9pt,
only marks,
mark=*,
mark options={solid,fill=mycolor30,draw=black,line width=0.1pt},
forget plot
]
coordinates{
 (26,0.7546875) 
};
\addplot [
color=blue,
mark size=0.9pt,
only marks,
mark=*,
mark options={solid,fill=mycolor31,draw=black,line width=0.1pt},
forget plot
]
coordinates{
 (27,0.783258594917788) 
};
\addplot [
color=blue,
mark size=0.9pt,
only marks,
mark=*,
mark options={solid,fill=mycolor32,draw=black,line width=0.1pt},
forget plot
]
coordinates{
 (27,0.812089356110381) 
};
\addplot [
color=blue,
mark size=0.9pt,
only marks,
mark=*,
mark options={solid,fill=mycolor33,draw=black,line width=0.1pt},
forget plot
]
coordinates{
 (25,0.790844514601421) 
};
\addplot [
color=blue,
mark size=0.9pt,
only marks,
mark=*,
mark options={solid,fill=mycolor34,draw=black,line width=0.1pt},
forget plot
]
coordinates{
 (21,0.798069187449718) 
};
\addplot [
color=blue,
mark size=0.9pt,
only marks,
mark=*,
mark options={solid,fill=mycolor35,draw=black,line width=0.1pt},
forget plot
]
coordinates{
 (20,0.750402576489533) 
};
\addplot [
color=blue,
mark size=0.9pt,
only marks,
mark=*,
mark options={solid,fill=mycolor36,draw=black,line width=0.1pt},
forget plot
]
coordinates{
 (20,0.766584766584767) 
};
\addplot [
color=blue,
mark size=0.9pt,
only marks,
mark=*,
mark options={solid,fill=mycolor37,draw=black,line width=0.1pt},
forget plot
]
coordinates{
 (20,0.70631970260223) 
};
\addplot [
color=blue,
mark size=0.9pt,
only marks,
mark=*,
mark options={solid,fill=mycolor38,draw=black,line width=0.1pt},
forget plot
]
coordinates{
 (18,0.741100323624595) 
};
\addplot [
color=blue,
mark size=0.9pt,
only marks,
mark=*,
mark options={solid,fill=mycolor39,draw=black,line width=0.1pt},
forget plot
]
coordinates{
 (20,0.820405310971349) 
};
\addplot [
color=blue,
mark size=0.9pt,
only marks,
mark=*,
mark options={solid,fill=mycolor40,draw=black,line width=0.1pt},
forget plot
]
coordinates{
 (8,0.694850115295926) 
};
\addplot [
color=blue,
mark size=0.9pt,
only marks,
mark=*,
mark options={solid,fill=mycolor41,draw=black,line width=0.1pt},
forget plot
]
coordinates{
 (18,0.659279778393352) 
};
\addplot [
color=blue,
mark size=0.9pt,
only marks,
mark=*,
mark options={solid,fill=mycolor41,draw=black,line width=0.1pt},
forget plot
]
coordinates{
 (21,0.750405186385737) 
};
\addplot [
color=blue,
mark size=0.9pt,
only marks,
mark=*,
mark options={solid,fill=mycolor42,draw=black,line width=0.1pt},
forget plot
]
coordinates{
 (16,0.701298701298701) 
};
\addplot [
color=blue,
mark size=0.9pt,
only marks,
mark=*,
mark options={solid,fill=mycolor42,draw=black,line width=0.1pt},
forget plot
]
coordinates{
 (19,0.794076163610719) 
};
\addplot [
color=blue,
mark size=0.9pt,
only marks,
mark=*,
mark options={solid,fill=mycolor42,draw=black,line width=0.1pt},
forget plot
]
coordinates{
 (19,0.656502242152466) 
};
\addplot [
color=blue,
mark size=0.9pt,
only marks,
mark=*,
mark options={solid,fill=mycolor38,draw=black,line width=0.1pt},
forget plot
]
coordinates{
 (22,0.79048349961627) 
};
\addplot [
color=blue,
mark size=0.9pt,
only marks,
mark=*,
mark options={solid,fill=mycolor38,draw=black,line width=0.1pt},
forget plot
]
coordinates{
 (19,0.804532577903683) 
};
\addplot [
color=blue,
mark size=0.9pt,
only marks,
mark=*,
mark options={solid,fill=mycolor38,draw=black,line width=0.1pt},
forget plot
]
coordinates{
 (16,0.789808917197452) 
};
\addplot [
color=blue,
mark size=0.9pt,
only marks,
mark=*,
mark options={solid,fill=mycolor43,draw=black,line width=0.1pt},
forget plot
]
coordinates{
 (15,0.758827948910593) 
};
\addplot [
color=blue,
mark size=0.9pt,
only marks,
mark=*,
mark options={solid,fill=mycolor43,draw=black,line width=0.1pt},
forget plot
]
coordinates{
 (19,0.772635814889336) 
};
\addplot [
color=blue,
mark size=0.9pt,
only marks,
mark=*,
mark options={solid,fill=mycolor44,draw=black,line width=0.1pt},
forget plot
]
coordinates{
 (17,0.732796032238066) 
};
\addplot [
color=blue,
mark size=0.9pt,
only marks,
mark=*,
mark options={solid,fill=mycolor44,draw=black,line width=0.1pt},
forget plot
]
coordinates{
 (22,0.807417974322396) 
};
\addplot [
color=blue,
mark size=0.9pt,
only marks,
mark=*,
mark options={solid,fill=mycolor44,draw=black,line width=0.1pt},
forget plot
]
coordinates{
 (9,0) 
};
\addplot [
color=blue,
mark size=0.9pt,
only marks,
mark=*,
mark options={solid,fill=mycolor45,draw=black,line width=0.1pt},
forget plot
]
coordinates{
 (18,0.758152173913043) 
};
\addplot [
color=blue,
mark size=0.9pt,
only marks,
mark=*,
mark options={solid,fill=mycolor45,draw=black,line width=0.1pt},
forget plot
]
coordinates{
 (22,0.779639175257732) 
};
\addplot [
color=blue,
mark size=0.9pt,
only marks,
mark=*,
mark options={solid,fill=mycolor39,draw=black,line width=0.1pt},
forget plot
]
coordinates{
 (18,0.795505617977528) 
};
\addplot [
color=blue,
mark size=0.9pt,
only marks,
mark=*,
mark options={solid,fill=mycolor39,draw=black,line width=0.1pt},
forget plot
]
coordinates{
 (17,0.764754779717373) 
};
\addplot [
color=blue,
mark size=0.9pt,
only marks,
mark=*,
mark options={solid,fill=mycolor46,draw=black,line width=0.1pt},
forget plot
]
coordinates{
 (17,0.765809589993051) 
};
\addplot [
color=blue,
mark size=0.9pt,
only marks,
mark=*,
mark options={solid,fill=mycolor46,draw=black,line width=0.1pt},
forget plot
]
coordinates{
 (9,0) 
};
\addplot [
color=blue,
mark size=0.9pt,
only marks,
mark=*,
mark options={solid,fill=mycolor46,draw=black,line width=0.1pt},
forget plot
]
coordinates{
 (17,0.618556701030928) 
};
\addplot [
color=blue,
mark size=0.9pt,
only marks,
mark=*,
mark options={solid,fill=mycolor47,draw=black,line width=0.1pt},
forget plot
]
coordinates{
 (16,0.8140625) 
};
\addplot [
color=blue,
mark size=0.9pt,
only marks,
mark=*,
mark options={solid,fill=mycolor47,draw=black,line width=0.1pt},
forget plot
]
coordinates{
 (10,0.779050736497545) 
};
\addplot [
color=blue,
mark size=0.9pt,
only marks,
mark=*,
mark options={solid,fill=mycolor48,draw=black,line width=0.1pt},
forget plot
]
coordinates{
 (10,0.399612027158099) 
};
\addplot [
color=blue,
mark size=0.9pt,
only marks,
mark=*,
mark options={solid,fill=mycolor48,draw=black,line width=0.1pt},
forget plot
]
coordinates{
 (18,0.734463276836158) 
};
\addplot [
color=blue,
mark size=0.9pt,
only marks,
mark=*,
mark options={solid,fill=mycolor40,draw=black,line width=0.1pt},
forget plot
]
coordinates{
 (17,0.760072158749248) 
};
\addplot [
color=blue,
mark size=0.9pt,
only marks,
mark=*,
mark options={solid,fill=mycolor40,draw=black,line width=0.1pt},
forget plot
]
coordinates{
 (6,0) 
};
\addplot [
color=blue,
mark size=0.9pt,
only marks,
mark=*,
mark options={solid,fill=mycolor40,draw=black,line width=0.1pt},
forget plot
]
coordinates{
 (16,0.816816816816817) 
};
\addplot [
color=blue,
mark size=0.9pt,
only marks,
mark=*,
mark options={solid,fill=mycolor49,draw=black,line width=0.1pt},
forget plot
]
coordinates{
 (15,0.645510835913313) 
};
\addplot [
color=blue,
mark size=0.9pt,
only marks,
mark=*,
mark options={solid,fill=mycolor49,draw=black,line width=0.1pt},
forget plot
]
coordinates{
 (10,0.804580152671756) 
};
\addplot [
color=blue,
mark size=0.9pt,
only marks,
mark=*,
mark options={solid,fill=mycolor50,draw=black,line width=0.1pt},
forget plot
]
coordinates{
 (2,0) 
};
\addplot [
color=blue,
mark size=0.9pt,
only marks,
mark=*,
mark options={solid,fill=mycolor41,draw=black,line width=0.1pt},
forget plot
]
coordinates{
 (19,0.76523151909017) 
};
\addplot [
color=blue,
mark size=0.9pt,
only marks,
mark=*,
mark options={solid,fill=mycolor41,draw=black,line width=0.1pt},
forget plot
]
coordinates{
 (23,0.782808902532617) 
};
\addplot [
color=blue,
mark size=0.9pt,
only marks,
mark=*,
mark options={solid,fill=mycolor42,draw=black,line width=0.1pt},
forget plot
]
coordinates{
 (14,0.7) 
};
\addplot [
color=blue,
mark size=0.9pt,
only marks,
mark=*,
mark options={solid,fill=mycolor42,draw=black,line width=0.1pt},
forget plot
]
coordinates{
 (20,0.795768917819365) 
};
\addplot [
color=blue,
mark size=0.9pt,
only marks,
mark=*,
mark options={solid,fill=mycolor42,draw=black,line width=0.1pt},
forget plot
]
coordinates{
 (18,0.781575037147102) 
};
\addplot [
color=blue,
mark size=0.9pt,
only marks,
mark=*,
mark options={solid,fill=mycolor38,draw=black,line width=0.1pt},
forget plot
]
coordinates{
 (18,0.785104732350659) 
};
\addplot [
color=blue,
mark size=0.9pt,
only marks,
mark=*,
mark options={solid,fill=mycolor38,draw=black,line width=0.1pt},
forget plot
]
coordinates{
 (15,0.593448940269749) 
};
\addplot [
color=blue,
mark size=0.9pt,
only marks,
mark=*,
mark options={solid,fill=mycolor38,draw=black,line width=0.1pt},
forget plot
]
coordinates{
 (14,0.783402489626556) 
};
\addplot [
color=blue,
mark size=0.9pt,
only marks,
mark=*,
mark options={solid,fill=mycolor43,draw=black,line width=0.1pt},
forget plot
]
coordinates{
 (19,0.749190938511327) 
};
\addplot [
color=blue,
mark size=0.9pt,
only marks,
mark=*,
mark options={solid,fill=mycolor43,draw=black,line width=0.1pt},
forget plot
]
coordinates{
 (14,0.73890142964635) 
};
\addplot [
color=blue,
mark size=0.9pt,
only marks,
mark=*,
mark options={solid,fill=mycolor44,draw=black,line width=0.1pt},
forget plot
]
coordinates{
 (14,0.0355329949238578) 
};
\addplot [
color=blue,
mark size=0.9pt,
only marks,
mark=*,
mark options={solid,fill=mycolor44,draw=black,line width=0.1pt},
forget plot
]
coordinates{
 (22,0.746498599439776) 
};
\addplot [
color=blue,
mark size=0.9pt,
only marks,
mark=*,
mark options={solid,fill=mycolor44,draw=black,line width=0.1pt},
forget plot
]
coordinates{
 (20,0.768066070199587) 
};
\addplot [
color=blue,
mark size=0.9pt,
only marks,
mark=*,
mark options={solid,fill=mycolor45,draw=black,line width=0.1pt},
forget plot
]
coordinates{
 (14,0.233009708737864) 
};
\addplot [
color=blue,
mark size=0.9pt,
only marks,
mark=*,
mark options={solid,fill=mycolor45,draw=black,line width=0.1pt},
forget plot
]
coordinates{
 (15,0.771929824561403) 
};
\addplot [
color=blue,
mark size=0.9pt,
only marks,
mark=*,
mark options={solid,fill=mycolor39,draw=black,line width=0.1pt},
forget plot
]
coordinates{
 (16,0.689100219458669) 
};
\addplot [
color=blue,
mark size=0.9pt,
only marks,
mark=*,
mark options={solid,fill=mycolor39,draw=black,line width=0.1pt},
forget plot
]
coordinates{
 (15,0.76663993584603) 
};
\addplot [
color=blue,
mark size=0.9pt,
only marks,
mark=*,
mark options={solid,fill=mycolor46,draw=black,line width=0.1pt},
forget plot
]
coordinates{
 (10,0.262008733624454) 
};
\addplot [
color=blue,
mark size=0.9pt,
only marks,
mark=*,
mark options={solid,fill=mycolor46,draw=black,line width=0.1pt},
forget plot
]
coordinates{
 (13,0.740207833733013) 
};
\addplot [
color=blue,
mark size=0.9pt,
only marks,
mark=*,
mark options={solid,fill=mycolor46,draw=black,line width=0.1pt},
forget plot
]
coordinates{
 (18,0.794871794871795) 
};
\addplot [
color=blue,
mark size=0.9pt,
only marks,
mark=*,
mark options={solid,fill=mycolor47,draw=black,line width=0.1pt},
forget plot
]
coordinates{
 (10,0.207142857142857) 
};
\addplot [
color=blue,
mark size=0.9pt,
only marks,
mark=*,
mark options={solid,fill=mycolor47,draw=black,line width=0.1pt},
forget plot
]
coordinates{
 (18,0.797404470079308) 
};
\addplot [
color=blue,
mark size=0.9pt,
only marks,
mark=*,
mark options={solid,fill=mycolor48,draw=black,line width=0.1pt},
forget plot
]
coordinates{
 (14,0.763665594855305) 
};
\addplot [
color=blue,
mark size=0.9pt,
only marks,
mark=*,
mark options={solid,fill=mycolor48,draw=black,line width=0.1pt},
forget plot
]
coordinates{
 (13,0.783333333333333) 
};
\addplot [
color=blue,
mark size=0.9pt,
only marks,
mark=*,
mark options={solid,fill=mycolor40,draw=black,line width=0.1pt},
forget plot
]
coordinates{
 (11,0.779527559055118) 
};
\addplot [
color=blue,
mark size=0.9pt,
only marks,
mark=*,
mark options={solid,fill=mycolor40,draw=black,line width=0.1pt},
forget plot
]
coordinates{
 (14,0.227657572906867) 
};
\addplot [
color=blue,
mark size=0.9pt,
only marks,
mark=*,
mark options={solid,fill=mycolor40,draw=black,line width=0.1pt},
forget plot
]
coordinates{
 (16,0.761666666666667) 
};
\addplot [
color=blue,
mark size=0.9pt,
only marks,
mark=*,
mark options={solid,fill=mycolor49,draw=black,line width=0.1pt},
forget plot
]
coordinates{
 (9,0) 
};
\addplot [
color=blue,
mark size=0.9pt,
only marks,
mark=*,
mark options={solid,fill=mycolor49,draw=black,line width=0.1pt},
forget plot
]
coordinates{
 (4,0) 
};
\addplot [
color=blue,
mark size=0.9pt,
only marks,
mark=*,
mark options={solid,fill=mycolor50,draw=black,line width=0.1pt},
forget plot
]
coordinates{
 (7,0) 
};
\addplot [
color=blue,
mark size=0.9pt,
only marks,
mark=*,
mark options={solid,fill=mycolor41,draw=black,line width=0.1pt},
forget plot
]
coordinates{
 (18,0.800963081861958) 
};
\addplot [
color=blue,
mark size=0.9pt,
only marks,
mark=*,
mark options={solid,fill=mycolor41,draw=black,line width=0.1pt},
forget plot
]
coordinates{
 (14,0.236509758897819) 
};
\addplot [
color=blue,
mark size=0.9pt,
only marks,
mark=*,
mark options={solid,fill=mycolor42,draw=black,line width=0.1pt},
forget plot
]
coordinates{
 (20,0.782138024357239) 
};
\addplot [
color=blue,
mark size=0.9pt,
only marks,
mark=*,
mark options={solid,fill=mycolor42,draw=black,line width=0.1pt},
forget plot
]
coordinates{
 (23,0.782973621103117) 
};
\addplot [
color=blue,
mark size=0.9pt,
only marks,
mark=*,
mark options={solid,fill=mycolor42,draw=black,line width=0.1pt},
forget plot
]
coordinates{
 (18,0.770653514180025) 
};
\addplot [
color=blue,
mark size=0.9pt,
only marks,
mark=*,
mark options={solid,fill=mycolor38,draw=black,line width=0.1pt},
forget plot
]
coordinates{
 (25,0.790960451977401) 
};
\addplot [
color=blue,
mark size=0.9pt,
only marks,
mark=*,
mark options={solid,fill=mycolor38,draw=black,line width=0.1pt},
forget plot
]
coordinates{
 (21,0.78499106611078) 
};
\addplot [
color=blue,
mark size=0.9pt,
only marks,
mark=*,
mark options={solid,fill=mycolor38,draw=black,line width=0.1pt},
forget plot
]
coordinates{
 (21,0.75583596214511) 
};
\addplot [
color=blue,
mark size=0.9pt,
only marks,
mark=*,
mark options={solid,fill=mycolor43,draw=black,line width=0.1pt},
forget plot
]
coordinates{
 (22,0.790289952798382) 
};
\addplot [
color=blue,
mark size=0.9pt,
only marks,
mark=*,
mark options={solid,fill=mycolor43,draw=black,line width=0.1pt},
forget plot
]
coordinates{
 (20,0.774193548387097) 
};
\addplot [
color=blue,
mark size=0.9pt,
only marks,
mark=*,
mark options={solid,fill=mycolor44,draw=black,line width=0.1pt},
forget plot
]
coordinates{
 (15,0.72264631043257) 
};
\addplot [
color=blue,
mark size=0.9pt,
only marks,
mark=*,
mark options={solid,fill=mycolor44,draw=black,line width=0.1pt},
forget plot
]
coordinates{
 (19,0.76103714085494) 
};
\addplot [
color=blue,
mark size=0.9pt,
only marks,
mark=*,
mark options={solid,fill=mycolor44,draw=black,line width=0.1pt},
forget plot
]
coordinates{
 (19,0.763432237369687) 
};
\addplot [
color=blue,
mark size=0.9pt,
only marks,
mark=*,
mark options={solid,fill=mycolor45,draw=black,line width=0.1pt},
forget plot
]
coordinates{
 (8,0) 
};
\addplot [
color=blue,
mark size=0.9pt,
only marks,
mark=*,
mark options={solid,fill=mycolor45,draw=black,line width=0.1pt},
forget plot
]
coordinates{
 (16,0.782475019215988) 
};
\addplot [
color=blue,
mark size=0.9pt,
only marks,
mark=*,
mark options={solid,fill=mycolor39,draw=black,line width=0.1pt},
forget plot
]
coordinates{
 (17,0.730407523510972) 
};
\addplot [
color=blue,
mark size=0.9pt,
only marks,
mark=*,
mark options={solid,fill=mycolor39,draw=black,line width=0.1pt},
forget plot
]
coordinates{
 (19,0.773790951638065) 
};
\addplot [
color=blue,
mark size=0.9pt,
only marks,
mark=*,
mark options={solid,fill=mycolor46,draw=black,line width=0.1pt},
forget plot
]
coordinates{
 (14,0.761982128350934) 
};
\addplot [
color=blue,
mark size=0.9pt,
only marks,
mark=*,
mark options={solid,fill=mycolor46,draw=black,line width=0.1pt},
forget plot
]
coordinates{
 (12,0.783053323593864) 
};
\addplot [
color=blue,
mark size=0.9pt,
only marks,
mark=*,
mark options={solid,fill=mycolor46,draw=black,line width=0.1pt},
forget plot
]
coordinates{
 (13,0.788710907704043) 
};
\addplot [
color=blue,
mark size=0.9pt,
only marks,
mark=*,
mark options={solid,fill=mycolor47,draw=black,line width=0.1pt},
forget plot
]
coordinates{
 (14,0.689165186500888) 
};
\addplot [
color=blue,
mark size=0.9pt,
only marks,
mark=*,
mark options={solid,fill=mycolor47,draw=black,line width=0.1pt},
forget plot
]
coordinates{
 (19,0.772525849335303) 
};
\addplot [
color=blue,
mark size=0.9pt,
only marks,
mark=*,
mark options={solid,fill=mycolor48,draw=black,line width=0.1pt},
forget plot
]
coordinates{
 (13,0.76969696969697) 
};
\addplot [
color=blue,
mark size=0.9pt,
only marks,
mark=*,
mark options={solid,fill=mycolor48,draw=black,line width=0.1pt},
forget plot
]
coordinates{
 (17,0.758314855875831) 
};
\addplot [
color=blue,
mark size=0.9pt,
only marks,
mark=*,
mark options={solid,fill=mycolor40,draw=black,line width=0.1pt},
forget plot
]
coordinates{
 (18,0.724884080370943) 
};
\addplot [
color=blue,
mark size=0.9pt,
only marks,
mark=*,
mark options={solid,fill=mycolor40,draw=black,line width=0.1pt},
forget plot
]
coordinates{
 (9,0.651079136690647) 
};
\addplot [
color=blue,
mark size=0.9pt,
only marks,
mark=*,
mark options={solid,fill=mycolor40,draw=black,line width=0.1pt},
forget plot
]
coordinates{
 (15,0.764423076923077) 
};
\addplot [
color=blue,
mark size=0.9pt,
only marks,
mark=*,
mark options={solid,fill=mycolor49,draw=black,line width=0.1pt},
forget plot
]
coordinates{
 (16,0.774355751099937) 
};
\addplot [
color=blue,
mark size=0.9pt,
only marks,
mark=*,
mark options={solid,fill=mycolor49,draw=black,line width=0.1pt},
forget plot
]
coordinates{
 (3,0) 
};
\addplot [
color=blue,
mark size=0.9pt,
only marks,
mark=*,
mark options={solid,fill=mycolor50,draw=black,line width=0.1pt},
forget plot
]
coordinates{
 (13,0.795969773299748) 
};
\addplot [
color=blue,
mark size=0.9pt,
only marks,
mark=*,
mark options={solid,fill=mycolor41,draw=black,line width=0.1pt},
forget plot
]
coordinates{
 (20,0.708588957055215) 
};
\addplot [
color=blue,
mark size=0.9pt,
only marks,
mark=*,
mark options={solid,fill=mycolor41,draw=black,line width=0.1pt},
forget plot
]
coordinates{
 (19,0.762096774193548) 
};
\addplot [
color=blue,
mark size=0.9pt,
only marks,
mark=*,
mark options={solid,fill=mycolor42,draw=black,line width=0.1pt},
forget plot
]
coordinates{
 (23,0.762198888202594) 
};
\addplot [
color=blue,
mark size=0.9pt,
only marks,
mark=*,
mark options={solid,fill=mycolor42,draw=black,line width=0.1pt},
forget plot
]
coordinates{
 (19,0.766025641025641) 
};
\addplot [
color=blue,
mark size=0.9pt,
only marks,
mark=*,
mark options={solid,fill=mycolor42,draw=black,line width=0.1pt},
forget plot
]
coordinates{
 (15,0.740412979351032) 
};
\addplot [
color=blue,
mark size=0.9pt,
only marks,
mark=*,
mark options={solid,fill=mycolor38,draw=black,line width=0.1pt},
forget plot
]
coordinates{
 (22,0.762996941896024) 
};
\addplot [
color=blue,
mark size=0.9pt,
only marks,
mark=*,
mark options={solid,fill=mycolor38,draw=black,line width=0.1pt},
forget plot
]
coordinates{
 (16,0.637873754152824) 
};
\addplot [
color=blue,
mark size=0.9pt,
only marks,
mark=*,
mark options={solid,fill=mycolor38,draw=black,line width=0.1pt},
forget plot
]
coordinates{
 (14,0.568627450980392) 
};
\addplot [
color=blue,
mark size=0.9pt,
only marks,
mark=*,
mark options={solid,fill=mycolor43,draw=black,line width=0.1pt},
forget plot
]
coordinates{
 (19,0.801925722145804) 
};
\addplot [
color=blue,
mark size=0.9pt,
only marks,
mark=*,
mark options={solid,fill=mycolor43,draw=black,line width=0.1pt},
forget plot
]
coordinates{
 (13,0.778235779060181) 
};
\addplot [
color=blue,
mark size=0.9pt,
only marks,
mark=*,
mark options={solid,fill=mycolor44,draw=black,line width=0.1pt},
forget plot
]
coordinates{
 (12,0) 
};
\addplot [
color=blue,
mark size=0.9pt,
only marks,
mark=*,
mark options={solid,fill=mycolor44,draw=black,line width=0.1pt},
forget plot
]
coordinates{
 (13,0.456404736275565) 
};
\addplot [
color=blue,
mark size=0.9pt,
only marks,
mark=*,
mark options={solid,fill=mycolor44,draw=black,line width=0.1pt},
forget plot
]
coordinates{
 (21,0.790863668807994) 
};
\addplot [
color=blue,
mark size=0.9pt,
only marks,
mark=*,
mark options={solid,fill=mycolor45,draw=black,line width=0.1pt},
forget plot
]
coordinates{
 (20,0.774007220216606) 
};
\addplot [
color=blue,
mark size=0.9pt,
only marks,
mark=*,
mark options={solid,fill=mycolor45,draw=black,line width=0.1pt},
forget plot
]
coordinates{
 (16,0.778894472361809) 
};
\addplot [
color=blue,
mark size=0.9pt,
only marks,
mark=*,
mark options={solid,fill=mycolor39,draw=black,line width=0.1pt},
forget plot
]
coordinates{
 (17,0.706779661016949) 
};
\addplot [
color=blue,
mark size=0.9pt,
only marks,
mark=*,
mark options={solid,fill=mycolor39,draw=black,line width=0.1pt},
forget plot
]
coordinates{
 (16,0.758990053557766) 
};
\addplot [
color=blue,
mark size=0.9pt,
only marks,
mark=*,
mark options={solid,fill=mycolor46,draw=black,line width=0.1pt},
forget plot
]
coordinates{
 (16,0.746347941567065) 
};
\addplot [
color=blue,
mark size=0.9pt,
only marks,
mark=*,
mark options={solid,fill=mycolor46,draw=black,line width=0.1pt},
forget plot
]
coordinates{
 (14,0.726739926739927) 
};
\addplot [
color=blue,
mark size=0.9pt,
only marks,
mark=*,
mark options={solid,fill=mycolor46,draw=black,line width=0.1pt},
forget plot
]
coordinates{
 (15,0.758139534883721) 
};
\addplot [
color=blue,
mark size=0.9pt,
only marks,
mark=*,
mark options={solid,fill=mycolor47,draw=black,line width=0.1pt},
forget plot
]
coordinates{
 (15,0.702113156100886) 
};
\addplot [
color=blue,
mark size=0.9pt,
only marks,
mark=*,
mark options={solid,fill=mycolor47,draw=black,line width=0.1pt},
forget plot
]
coordinates{
 (17,0.738664468260511) 
};
\addplot [
color=blue,
mark size=0.9pt,
only marks,
mark=*,
mark options={solid,fill=mycolor48,draw=black,line width=0.1pt},
forget plot
]
coordinates{
 (13,0.793674698795181) 
};
\addplot [
color=blue,
mark size=0.9pt,
only marks,
mark=*,
mark options={solid,fill=mycolor48,draw=black,line width=0.1pt},
forget plot
]
coordinates{
 (18,0.759656652360515) 
};
\addplot [
color=blue,
mark size=0.9pt,
only marks,
mark=*,
mark options={solid,fill=mycolor40,draw=black,line width=0.1pt},
forget plot
]
coordinates{
 (8,0.210161662817552) 
};
\addplot [
color=blue,
mark size=0.9pt,
only marks,
mark=*,
mark options={solid,fill=mycolor40,draw=black,line width=0.1pt},
forget plot
]
coordinates{
 (5,0) 
};
\addplot [
color=blue,
mark size=0.9pt,
only marks,
mark=*,
mark options={solid,fill=mycolor40,draw=black,line width=0.1pt},
forget plot
]
coordinates{
 (15,0.790123456790123) 
};
\addplot [
color=blue,
mark size=0.9pt,
only marks,
mark=*,
mark options={solid,fill=mycolor49,draw=black,line width=0.1pt},
forget plot
]
coordinates{
 (17,0.781542898341745) 
};
\addplot [
color=blue,
mark size=0.9pt,
only marks,
mark=*,
mark options={solid,fill=mycolor49,draw=black,line width=0.1pt},
forget plot
]
coordinates{
 (10,0.516765285996055) 
};
\addplot [
color=blue,
mark size=0.9pt,
only marks,
mark=*,
mark options={solid,fill=mycolor50,draw=black,line width=0.1pt},
forget plot
]
coordinates{
 (9,0.671197960917587) 
};
\addplot [
color=blue,
mark size=0.9pt,
only marks,
mark=*,
mark options={solid,fill=mycolor41,draw=black,line width=0.1pt},
forget plot
]
coordinates{
 (17,0.748224151539068) 
};
\addplot [
color=blue,
mark size=0.9pt,
only marks,
mark=*,
mark options={solid,fill=mycolor41,draw=black,line width=0.1pt},
forget plot
]
coordinates{
 (21,0.701505757307352) 
};
\addplot [
color=blue,
mark size=0.9pt,
only marks,
mark=*,
mark options={solid,fill=mycolor42,draw=black,line width=0.1pt},
forget plot
]
coordinates{
 (19,0.744479495268139) 
};
\addplot [
color=blue,
mark size=0.9pt,
only marks,
mark=*,
mark options={solid,fill=mycolor42,draw=black,line width=0.1pt},
forget plot
]
coordinates{
 (20,0.806451612903226) 
};
\addplot [
color=blue,
mark size=0.9pt,
only marks,
mark=*,
mark options={solid,fill=mycolor42,draw=black,line width=0.1pt},
forget plot
]
coordinates{
 (22,0.776200135226504) 
};
\addplot [
color=blue,
mark size=0.9pt,
only marks,
mark=*,
mark options={solid,fill=mycolor38,draw=black,line width=0.1pt},
forget plot
]
coordinates{
 (23,0.782215523737754) 
};
\addplot [
color=blue,
mark size=0.9pt,
only marks,
mark=*,
mark options={solid,fill=mycolor38,draw=black,line width=0.1pt},
forget plot
]
coordinates{
 (21,0.744279946164199) 
};
\addplot [
color=blue,
mark size=0.9pt,
only marks,
mark=*,
mark options={solid,fill=mycolor38,draw=black,line width=0.1pt},
forget plot
]
coordinates{
 (18,0.775055679287305) 
};
\addplot [
color=blue,
mark size=0.9pt,
only marks,
mark=*,
mark options={solid,fill=mycolor43,draw=black,line width=0.1pt},
forget plot
]
coordinates{
 (17,0.800936768149883) 
};
\addplot [
color=blue,
mark size=0.9pt,
only marks,
mark=*,
mark options={solid,fill=mycolor43,draw=black,line width=0.1pt},
forget plot
]
coordinates{
 (18,0.797822706065319) 
};
\addplot [
color=blue,
mark size=0.9pt,
only marks,
mark=*,
mark options={solid,fill=mycolor44,draw=black,line width=0.1pt},
forget plot
]
coordinates{
 (18,0.785714285714286) 
};
\addplot [
color=blue,
mark size=0.9pt,
only marks,
mark=*,
mark options={solid,fill=mycolor44,draw=black,line width=0.1pt},
forget plot
]
coordinates{
 (19,0.822794691647151) 
};
\addplot [
color=blue,
mark size=0.9pt,
only marks,
mark=*,
mark options={solid,fill=mycolor44,draw=black,line width=0.1pt},
forget plot
]
coordinates{
 (16,0.791354945968412) 
};
\addplot [
color=blue,
mark size=0.9pt,
only marks,
mark=*,
mark options={solid,fill=mycolor45,draw=black,line width=0.1pt},
forget plot
]
coordinates{
 (17,0.775572519083969) 
};
\addplot [
color=blue,
mark size=0.9pt,
only marks,
mark=*,
mark options={solid,fill=mycolor45,draw=black,line width=0.1pt},
forget plot
]
coordinates{
 (17,0.699846860643185) 
};
\addplot [
color=blue,
mark size=0.9pt,
only marks,
mark=*,
mark options={solid,fill=mycolor39,draw=black,line width=0.1pt},
forget plot
]
coordinates{
 (17,0.734864300626305) 
};
\addplot [
color=blue,
mark size=0.9pt,
only marks,
mark=*,
mark options={solid,fill=mycolor39,draw=black,line width=0.1pt},
forget plot
]
coordinates{
 (20,0.668453976764969) 
};
\addplot [
color=blue,
mark size=0.9pt,
only marks,
mark=*,
mark options={solid,fill=mycolor46,draw=black,line width=0.1pt},
forget plot
]
coordinates{
 (14,0.798732171156894) 
};
\addplot [
color=blue,
mark size=0.9pt,
only marks,
mark=*,
mark options={solid,fill=mycolor46,draw=black,line width=0.1pt},
forget plot
]
coordinates{
 (16,0.713804713804714) 
};
\addplot [
color=blue,
mark size=0.9pt,
only marks,
mark=*,
mark options={solid,fill=mycolor46,draw=black,line width=0.1pt},
forget plot
]
coordinates{
 (17,0.780162842339008) 
};
\addplot [
color=blue,
mark size=0.9pt,
only marks,
mark=*,
mark options={solid,fill=mycolor47,draw=black,line width=0.1pt},
forget plot
]
coordinates{
 (8,0) 
};
\addplot [
color=blue,
mark size=0.9pt,
only marks,
mark=*,
mark options={solid,fill=mycolor47,draw=black,line width=0.1pt},
forget plot
]
coordinates{
 (15,0.750773993808049) 
};
\addplot [
color=blue,
mark size=0.9pt,
only marks,
mark=*,
mark options={solid,fill=mycolor48,draw=black,line width=0.1pt},
forget plot
]
coordinates{
 (13,0.780373831775701) 
};
\addplot [
color=blue,
mark size=0.9pt,
only marks,
mark=*,
mark options={solid,fill=mycolor48,draw=black,line width=0.1pt},
forget plot
]
coordinates{
 (14,0.73195020746888) 
};
\addplot [
color=blue,
mark size=0.9pt,
only marks,
mark=*,
mark options={solid,fill=mycolor40,draw=black,line width=0.1pt},
forget plot
]
coordinates{
 (14,0.795144157814871) 
};
\addplot [
color=blue,
mark size=0.9pt,
only marks,
mark=*,
mark options={solid,fill=mycolor40,draw=black,line width=0.1pt},
forget plot
]
coordinates{
 (18,0.747813411078717) 
};
\addplot [
color=blue,
mark size=0.9pt,
only marks,
mark=*,
mark options={solid,fill=mycolor40,draw=black,line width=0.1pt},
forget plot
]
coordinates{
 (15,0.743071161048689) 
};
\addplot [
color=blue,
mark size=0.9pt,
only marks,
mark=*,
mark options={solid,fill=mycolor49,draw=black,line width=0.1pt},
forget plot
]
coordinates{
 (11,0.77602523659306) 
};
\addplot [
color=blue,
mark size=0.9pt,
only marks,
mark=*,
mark options={solid,fill=mycolor49,draw=black,line width=0.1pt},
forget plot
]
coordinates{
 (13,0.771641791044776) 
};
\addplot [
color=blue,
mark size=0.9pt,
only marks,
mark=*,
mark options={solid,fill=mycolor50,draw=black,line width=0.1pt},
forget plot
]
coordinates{
 (9,0.700087950747581) 
};
\addplot [
color=blue,
mark size=0.9pt,
only marks,
mark=*,
mark options={solid,fill=mycolor41,draw=black,line width=0.1pt},
forget plot
]
coordinates{
 (23,0.804628999319265) 
};
\addplot [
color=blue,
mark size=0.9pt,
only marks,
mark=*,
mark options={solid,fill=mycolor41,draw=black,line width=0.1pt},
forget plot
]
coordinates{
 (21,0.748172757475083) 
};
\addplot [
color=blue,
mark size=0.9pt,
only marks,
mark=*,
mark options={solid,fill=mycolor42,draw=black,line width=0.1pt},
forget plot
]
coordinates{
 (19,0.786301369863013) 
};
\addplot [
color=blue,
mark size=0.9pt,
only marks,
mark=*,
mark options={solid,fill=mycolor42,draw=black,line width=0.1pt},
forget plot
]
coordinates{
 (16,0.774522292993631) 
};
\addplot [
color=blue,
mark size=0.9pt,
only marks,
mark=*,
mark options={solid,fill=mycolor42,draw=black,line width=0.1pt},
forget plot
]
coordinates{
 (17,0.777609682299546) 
};
\addplot [
color=blue,
mark size=0.9pt,
only marks,
mark=*,
mark options={solid,fill=mycolor38,draw=black,line width=0.1pt},
forget plot
]
coordinates{
 (17,0.789559543230016) 
};
\addplot [
color=blue,
mark size=0.9pt,
only marks,
mark=*,
mark options={solid,fill=mycolor38,draw=black,line width=0.1pt},
forget plot
]
coordinates{
 (18,0.788310762651461) 
};
\addplot [
color=blue,
mark size=0.9pt,
only marks,
mark=*,
mark options={solid,fill=mycolor38,draw=black,line width=0.1pt},
forget plot
]
coordinates{
 (20,0.744645799011532) 
};
\addplot [
color=blue,
mark size=0.9pt,
only marks,
mark=*,
mark options={solid,fill=mycolor43,draw=black,line width=0.1pt},
forget plot
]
coordinates{
 (16,0.811819595645412) 
};
\addplot [
color=blue,
mark size=0.9pt,
only marks,
mark=*,
mark options={solid,fill=mycolor43,draw=black,line width=0.1pt},
forget plot
]
coordinates{
 (21,0.777847702957835) 
};
\addplot [
color=blue,
mark size=0.9pt,
only marks,
mark=*,
mark options={solid,fill=mycolor44,draw=black,line width=0.1pt},
forget plot
]
coordinates{
 (16,0.673862760215883) 
};
\addplot [
color=blue,
mark size=0.9pt,
only marks,
mark=*,
mark options={solid,fill=mycolor44,draw=black,line width=0.1pt},
forget plot
]
coordinates{
 (20,0.765498652291105) 
};
\addplot [
color=blue,
mark size=0.9pt,
only marks,
mark=*,
mark options={solid,fill=mycolor44,draw=black,line width=0.1pt},
forget plot
]
coordinates{
 (14,0.611913357400722) 
};
\addplot [
color=blue,
mark size=0.9pt,
only marks,
mark=*,
mark options={solid,fill=mycolor45,draw=black,line width=0.1pt},
forget plot
]
coordinates{
 (17,0.757692307692308) 
};
\addplot [
color=blue,
mark size=0.9pt,
only marks,
mark=*,
mark options={solid,fill=mycolor45,draw=black,line width=0.1pt},
forget plot
]
coordinates{
 (21,0.797101449275362) 
};
\addplot [
color=blue,
mark size=0.9pt,
only marks,
mark=*,
mark options={solid,fill=mycolor39,draw=black,line width=0.1pt},
forget plot
]
coordinates{
 (19,0.804511278195489) 
};
\addplot [
color=blue,
mark size=0.9pt,
only marks,
mark=*,
mark options={solid,fill=mycolor39,draw=black,line width=0.1pt},
forget plot
]
coordinates{
 (17,0.780285035629454) 
};
\addplot [
color=blue,
mark size=0.9pt,
only marks,
mark=*,
mark options={solid,fill=mycolor46,draw=black,line width=0.1pt},
forget plot
]
coordinates{
 (11,0.0408163265306122) 
};
\addplot [
color=blue,
mark size=0.9pt,
only marks,
mark=*,
mark options={solid,fill=mycolor46,draw=black,line width=0.1pt},
forget plot
]
coordinates{
 (18,0.760104302477184) 
};
\addplot [
color=blue,
mark size=0.9pt,
only marks,
mark=*,
mark options={solid,fill=mycolor46,draw=black,line width=0.1pt},
forget plot
]
coordinates{
 (13,0.819672131147541) 
};
\addplot [
color=blue,
mark size=0.9pt,
only marks,
mark=*,
mark options={solid,fill=mycolor47,draw=black,line width=0.1pt},
forget plot
]
coordinates{
 (6,0) 
};
\addplot [
color=blue,
mark size=0.9pt,
only marks,
mark=*,
mark options={solid,fill=mycolor47,draw=black,line width=0.1pt},
forget plot
]
coordinates{
 (14,0.741312741312741) 
};
\addplot [
color=blue,
mark size=0.9pt,
only marks,
mark=*,
mark options={solid,fill=mycolor48,draw=black,line width=0.1pt},
forget plot
]
coordinates{
 (11,0.680456490727532) 
};
\addplot [
color=blue,
mark size=0.9pt,
only marks,
mark=*,
mark options={solid,fill=mycolor48,draw=black,line width=0.1pt},
forget plot
]
coordinates{
 (9,0.654418197725284) 
};
\addplot [
color=blue,
mark size=0.9pt,
only marks,
mark=*,
mark options={solid,fill=mycolor40,draw=black,line width=0.1pt},
forget plot
]
coordinates{
 (15,0.746172441579371) 
};
\addplot [
color=blue,
mark size=0.9pt,
only marks,
mark=*,
mark options={solid,fill=mycolor40,draw=black,line width=0.1pt},
forget plot
]
coordinates{
 (17,0.783279742765273) 
};
\addplot [
color=blue,
mark size=0.9pt,
only marks,
mark=*,
mark options={solid,fill=mycolor40,draw=black,line width=0.1pt},
forget plot
]
coordinates{
 (9,0.703216374269006) 
};
\addplot [
color=blue,
mark size=0.9pt,
only marks,
mark=*,
mark options={solid,fill=mycolor49,draw=black,line width=0.1pt},
forget plot
]
coordinates{
 (12,0.72249822569198) 
};
\addplot [
color=blue,
mark size=0.9pt,
only marks,
mark=*,
mark options={solid,fill=mycolor49,draw=black,line width=0.1pt},
forget plot
]
coordinates{
 (3,0) 
};
\addplot [
color=blue,
mark size=0.9pt,
only marks,
mark=*,
mark options={solid,fill=mycolor50,draw=black,line width=0.1pt},
forget plot
]
coordinates{
 (8,0.597320724980299) 
};
\addplot [
color=blue,
mark size=0.9pt,
only marks,
mark=*,
mark options={solid,fill=mycolor41,draw=black,line width=0.1pt},
forget plot
]
coordinates{
 (21,0.673196794300979) 
};
\addplot [
color=blue,
mark size=0.9pt,
only marks,
mark=*,
mark options={solid,fill=mycolor41,draw=black,line width=0.1pt},
forget plot
]
coordinates{
 (21,0.760330578512396) 
};
\addplot [
color=blue,
mark size=0.9pt,
only marks,
mark=*,
mark options={solid,fill=mycolor42,draw=black,line width=0.1pt},
forget plot
]
coordinates{
 (19,0.71701244813278) 
};
\addplot [
color=blue,
mark size=0.9pt,
only marks,
mark=*,
mark options={solid,fill=mycolor42,draw=black,line width=0.1pt},
forget plot
]
coordinates{
 (18,0.706355591311344) 
};
\addplot [
color=blue,
mark size=0.9pt,
only marks,
mark=*,
mark options={solid,fill=mycolor42,draw=black,line width=0.1pt},
forget plot
]
coordinates{
 (21,0.759316770186335) 
};
\addplot [
color=blue,
mark size=0.9pt,
only marks,
mark=*,
mark options={solid,fill=mycolor38,draw=black,line width=0.1pt},
forget plot
]
coordinates{
 (19,0.778294573643411) 
};
\addplot [
color=blue,
mark size=0.9pt,
only marks,
mark=*,
mark options={solid,fill=mycolor38,draw=black,line width=0.1pt},
forget plot
]
coordinates{
 (16,0.723646723646724) 
};
\addplot [
color=blue,
mark size=0.9pt,
only marks,
mark=*,
mark options={solid,fill=mycolor38,draw=black,line width=0.1pt},
forget plot
]
coordinates{
 (19,0.691455696202532) 
};
\addplot [
color=blue,
mark size=0.9pt,
only marks,
mark=*,
mark options={solid,fill=mycolor43,draw=black,line width=0.1pt},
forget plot
]
coordinates{
 (21,0.778248587570621) 
};
\addplot [
color=blue,
mark size=0.9pt,
only marks,
mark=*,
mark options={solid,fill=mycolor43,draw=black,line width=0.1pt},
forget plot
]
coordinates{
 (15,0.790697674418605) 
};
\addplot [
color=blue,
mark size=0.9pt,
only marks,
mark=*,
mark options={solid,fill=mycolor44,draw=black,line width=0.1pt},
forget plot
]
coordinates{
 (20,0.783634933123525) 
};
\addplot [
color=blue,
mark size=0.9pt,
only marks,
mark=*,
mark options={solid,fill=mycolor44,draw=black,line width=0.1pt},
forget plot
]
coordinates{
 (12,0.225058004640371) 
};
\addplot [
color=blue,
mark size=0.9pt,
only marks,
mark=*,
mark options={solid,fill=mycolor44,draw=black,line width=0.1pt},
forget plot
]
coordinates{
 (20,0.755555555555556) 
};
\addplot [
color=blue,
mark size=0.9pt,
only marks,
mark=*,
mark options={solid,fill=mycolor45,draw=black,line width=0.1pt},
forget plot
]
coordinates{
 (16,0.746196957566053) 
};
\addplot [
color=blue,
mark size=0.9pt,
only marks,
mark=*,
mark options={solid,fill=mycolor45,draw=black,line width=0.1pt},
forget plot
]
coordinates{
 (16,0.75984555984556) 
};
\addplot [
color=blue,
mark size=0.9pt,
only marks,
mark=*,
mark options={solid,fill=mycolor39,draw=black,line width=0.1pt},
forget plot
]
coordinates{
 (16,0.775732788002726) 
};
\addplot [
color=blue,
mark size=0.9pt,
only marks,
mark=*,
mark options={solid,fill=mycolor39,draw=black,line width=0.1pt},
forget plot
]
coordinates{
 (18,0.8) 
};
\addplot [
color=blue,
mark size=0.9pt,
only marks,
mark=*,
mark options={solid,fill=mycolor46,draw=black,line width=0.1pt},
forget plot
]
coordinates{
 (19,0.799467021985343) 
};
\addplot [
color=blue,
mark size=0.9pt,
only marks,
mark=*,
mark options={solid,fill=mycolor46,draw=black,line width=0.1pt},
forget plot
]
coordinates{
 (18,0.735372340425532) 
};
\addplot [
color=blue,
mark size=0.9pt,
only marks,
mark=*,
mark options={solid,fill=mycolor46,draw=black,line width=0.1pt},
forget plot
]
coordinates{
 (13,0.687105500450857) 
};
\addplot [
color=blue,
mark size=0.9pt,
only marks,
mark=*,
mark options={solid,fill=mycolor47,draw=black,line width=0.1pt},
forget plot
]
coordinates{
 (15,0.756134969325153) 
};
\addplot [
color=blue,
mark size=0.9pt,
only marks,
mark=*,
mark options={solid,fill=mycolor47,draw=black,line width=0.1pt},
forget plot
]
coordinates{
 (15,0.743388134381701) 
};
\addplot [
color=blue,
mark size=0.9pt,
only marks,
mark=*,
mark options={solid,fill=mycolor48,draw=black,line width=0.1pt},
forget plot
]
coordinates{
 (16,0.70028818443804) 
};
\addplot [
color=blue,
mark size=0.9pt,
only marks,
mark=*,
mark options={solid,fill=mycolor48,draw=black,line width=0.1pt},
forget plot
]
coordinates{
 (15,0.800623052959501) 
};
\addplot [
color=blue,
mark size=0.9pt,
only marks,
mark=*,
mark options={solid,fill=mycolor40,draw=black,line width=0.1pt},
forget plot
]
coordinates{
 (13,0.783171521035599) 
};
\addplot [
color=blue,
mark size=0.9pt,
only marks,
mark=*,
mark options={solid,fill=mycolor40,draw=black,line width=0.1pt},
forget plot
]
coordinates{
 (8,0.234866828087167) 
};
\addplot [
color=blue,
mark size=0.9pt,
only marks,
mark=*,
mark options={solid,fill=mycolor40,draw=black,line width=0.1pt},
forget plot
]
coordinates{
 (6,0) 
};
\addplot [
color=blue,
mark size=0.9pt,
only marks,
mark=*,
mark options={solid,fill=mycolor49,draw=black,line width=0.1pt},
forget plot
]
coordinates{
 (14,0.742351046698873) 
};
\addplot [
color=blue,
mark size=0.9pt,
only marks,
mark=*,
mark options={solid,fill=mycolor49,draw=black,line width=0.1pt},
forget plot
]
coordinates{
 (13,0.792971734148205) 
};
\addplot [
color=blue,
mark size=0.9pt,
only marks,
mark=*,
mark options={solid,fill=mycolor50,draw=black,line width=0.1pt},
forget plot
]
coordinates{
 (2,0) 
};
\addplot [
color=blue,
mark size=0.9pt,
only marks,
mark=*,
mark options={solid,fill=mycolor41,draw=black,line width=0.1pt},
forget plot
]
coordinates{
 (17,0.776906628652887) 
};
\addplot [
color=blue,
mark size=0.9pt,
only marks,
mark=*,
mark options={solid,fill=mycolor41,draw=black,line width=0.1pt},
forget plot
]
coordinates{
 (21,0.78129713423831) 
};
\addplot [
color=blue,
mark size=0.9pt,
only marks,
mark=*,
mark options={solid,fill=mycolor42,draw=black,line width=0.1pt},
forget plot
]
coordinates{
 (15,0.768768768768769) 
};
\addplot [
color=blue,
mark size=0.9pt,
only marks,
mark=*,
mark options={solid,fill=mycolor42,draw=black,line width=0.1pt},
forget plot
]
coordinates{
 (22,0.817864619678995) 
};
\addplot [
color=blue,
mark size=0.9pt,
only marks,
mark=*,
mark options={solid,fill=mycolor42,draw=black,line width=0.1pt},
forget plot
]
coordinates{
 (20,0.783004552352048) 
};
\addplot [
color=blue,
mark size=0.9pt,
only marks,
mark=*,
mark options={solid,fill=mycolor38,draw=black,line width=0.1pt},
forget plot
]
coordinates{
 (21,0.784167289021658) 
};
\addplot [
color=blue,
mark size=0.9pt,
only marks,
mark=*,
mark options={solid,fill=mycolor38,draw=black,line width=0.1pt},
forget plot
]
coordinates{
 (20,0.792805755395683) 
};
\addplot [
color=blue,
mark size=0.9pt,
only marks,
mark=*,
mark options={solid,fill=mycolor38,draw=black,line width=0.1pt},
forget plot
]
coordinates{
 (21,0.792421746293245) 
};
\addplot [
color=blue,
mark size=0.9pt,
only marks,
mark=*,
mark options={solid,fill=mycolor43,draw=black,line width=0.1pt},
forget plot
]
coordinates{
 (20,0.747051114023591) 
};
\addplot [
color=blue,
mark size=0.9pt,
only marks,
mark=*,
mark options={solid,fill=mycolor43,draw=black,line width=0.1pt},
forget plot
]
coordinates{
 (21,0.816434724983433) 
};
\addplot [
color=blue,
mark size=0.9pt,
only marks,
mark=*,
mark options={solid,fill=mycolor44,draw=black,line width=0.1pt},
forget plot
]
coordinates{
 (13,0.00262467191601047) 
};
\addplot [
color=blue,
mark size=0.9pt,
only marks,
mark=*,
mark options={solid,fill=mycolor44,draw=black,line width=0.1pt},
forget plot
]
coordinates{
 (18,0.779761904761905) 
};
\addplot [
color=blue,
mark size=0.9pt,
only marks,
mark=*,
mark options={solid,fill=mycolor44,draw=black,line width=0.1pt},
forget plot
]
coordinates{
 (16,0.258064516129032) 
};
\addplot [
color=blue,
mark size=0.9pt,
only marks,
mark=*,
mark options={solid,fill=mycolor45,draw=black,line width=0.1pt},
forget plot
]
coordinates{
 (10,0) 
};
\addplot [
color=blue,
mark size=0.9pt,
only marks,
mark=*,
mark options={solid,fill=mycolor45,draw=black,line width=0.1pt},
forget plot
]
coordinates{
 (14,0.807692307692308) 
};
\addplot [
color=blue,
mark size=0.9pt,
only marks,
mark=*,
mark options={solid,fill=mycolor39,draw=black,line width=0.1pt},
forget plot
]
coordinates{
 (20,0.808629088378566) 
};
\addplot [
color=blue,
mark size=0.9pt,
only marks,
mark=*,
mark options={solid,fill=mycolor39,draw=black,line width=0.1pt},
forget plot
]
coordinates{
 (16,0.800528401585205) 
};
\addplot [
color=blue,
mark size=0.9pt,
only marks,
mark=*,
mark options={solid,fill=mycolor46,draw=black,line width=0.1pt},
forget plot
]
coordinates{
 (18,0.757062146892655) 
};
\addplot [
color=blue,
mark size=0.9pt,
only marks,
mark=*,
mark options={solid,fill=mycolor46,draw=black,line width=0.1pt},
forget plot
]
coordinates{
 (11,0.703894195444526) 
};
\addplot [
color=blue,
mark size=0.9pt,
only marks,
mark=*,
mark options={solid,fill=mycolor46,draw=black,line width=0.1pt},
forget plot
]
coordinates{
 (17,0.694262890341322) 
};
\addplot [
color=blue,
mark size=0.9pt,
only marks,
mark=*,
mark options={solid,fill=mycolor47,draw=black,line width=0.1pt},
forget plot
]
coordinates{
 (21,0.802387267904509) 
};
\addplot [
color=blue,
mark size=0.9pt,
only marks,
mark=*,
mark options={solid,fill=mycolor47,draw=black,line width=0.1pt},
forget plot
]
coordinates{
 (17,0.751219512195122) 
};
\addplot [
color=blue,
mark size=0.9pt,
only marks,
mark=*,
mark options={solid,fill=mycolor48,draw=black,line width=0.1pt},
forget plot
]
coordinates{
 (17,0.766905330151153) 
};
\addplot [
color=blue,
mark size=0.9pt,
only marks,
mark=*,
mark options={solid,fill=mycolor48,draw=black,line width=0.1pt},
forget plot
]
coordinates{
 (14,0.664383561643836) 
};
\addplot [
color=blue,
mark size=0.9pt,
only marks,
mark=*,
mark options={solid,fill=mycolor40,draw=black,line width=0.1pt},
forget plot
]
coordinates{
 (14,0.748019017432647) 
};
\addplot [
color=blue,
mark size=0.9pt,
only marks,
mark=*,
mark options={solid,fill=mycolor40,draw=black,line width=0.1pt},
forget plot
]
coordinates{
 (16,0.717391304347826) 
};
\addplot [
color=blue,
mark size=0.9pt,
only marks,
mark=*,
mark options={solid,fill=mycolor40,draw=black,line width=0.1pt},
forget plot
]
coordinates{
 (19,0.754448398576512) 
};
\addplot [
color=blue,
mark size=0.9pt,
only marks,
mark=*,
mark options={solid,fill=mycolor49,draw=black,line width=0.1pt},
forget plot
]
coordinates{
 (13,0.35361216730038) 
};
\addplot [
color=blue,
mark size=0.9pt,
only marks,
mark=*,
mark options={solid,fill=mycolor49,draw=black,line width=0.1pt},
forget plot
]
coordinates{
 (4,0) 
};
\addplot [
color=blue,
mark size=0.9pt,
only marks,
mark=*,
mark options={solid,fill=mycolor50,draw=black,line width=0.1pt},
forget plot
]
coordinates{
 (19,0.795902285263987) 
};
\addplot [
color=blue,
mark size=0.9pt,
only marks,
mark=*,
mark options={solid,fill=mycolor41,draw=black,line width=0.1pt},
forget plot
]
coordinates{
 (22,0.73663101604278) 
};
\addplot [
color=blue,
mark size=0.9pt,
only marks,
mark=*,
mark options={solid,fill=mycolor41,draw=black,line width=0.1pt},
forget plot
]
coordinates{
 (19,0.793313069908815) 
};
\addplot [
color=blue,
mark size=0.9pt,
only marks,
mark=*,
mark options={solid,fill=mycolor42,draw=black,line width=0.1pt},
forget plot
]
coordinates{
 (15,0.75) 
};
\addplot [
color=blue,
mark size=0.9pt,
only marks,
mark=*,
mark options={solid,fill=mycolor42,draw=black,line width=0.1pt},
forget plot
]
coordinates{
 (20,0.755656108597285) 
};
\addplot [
color=blue,
mark size=0.9pt,
only marks,
mark=*,
mark options={solid,fill=mycolor42,draw=black,line width=0.1pt},
forget plot
]
coordinates{
 (16,0.776595744680851) 
};
\addplot [
color=blue,
mark size=0.9pt,
only marks,
mark=*,
mark options={solid,fill=mycolor38,draw=black,line width=0.1pt},
forget plot
]
coordinates{
 (22,0.78726483357453) 
};
\addplot [
color=blue,
mark size=0.9pt,
only marks,
mark=*,
mark options={solid,fill=mycolor38,draw=black,line width=0.1pt},
forget plot
]
coordinates{
 (20,0.736220472440945) 
};
\addplot [
color=blue,
mark size=0.9pt,
only marks,
mark=*,
mark options={solid,fill=mycolor38,draw=black,line width=0.1pt},
forget plot
]
coordinates{
 (20,0.76499647141849) 
};
\addplot [
color=blue,
mark size=0.9pt,
only marks,
mark=*,
mark options={solid,fill=mycolor43,draw=black,line width=0.1pt},
forget plot
]
coordinates{
 (17,0.768051118210863) 
};
\addplot [
color=blue,
mark size=0.9pt,
only marks,
mark=*,
mark options={solid,fill=mycolor43,draw=black,line width=0.1pt},
forget plot
]
coordinates{
 (20,0.666666666666667) 
};
\addplot [
color=blue,
mark size=0.9pt,
only marks,
mark=*,
mark options={solid,fill=mycolor44,draw=black,line width=0.1pt},
forget plot
]
coordinates{
 (18,0.76602086438152) 
};
\addplot [
color=blue,
mark size=0.9pt,
only marks,
mark=*,
mark options={solid,fill=mycolor44,draw=black,line width=0.1pt},
forget plot
]
coordinates{
 (10,0.695652173913043) 
};
\addplot [
color=blue,
mark size=0.9pt,
only marks,
mark=*,
mark options={solid,fill=mycolor44,draw=black,line width=0.1pt},
forget plot
]
coordinates{
 (11,0.626262626262626) 
};
\addplot [
color=blue,
mark size=0.9pt,
only marks,
mark=*,
mark options={solid,fill=mycolor45,draw=black,line width=0.1pt},
forget plot
]
coordinates{
 (21,0.785813630041725) 
};
\addplot [
color=blue,
mark size=0.9pt,
only marks,
mark=*,
mark options={solid,fill=mycolor45,draw=black,line width=0.1pt},
forget plot
]
coordinates{
 (18,0.708404802744425) 
};
\addplot [
color=blue,
mark size=0.9pt,
only marks,
mark=*,
mark options={solid,fill=mycolor39,draw=black,line width=0.1pt},
forget plot
]
coordinates{
 (13,0.629294755877034) 
};
\addplot [
color=blue,
mark size=0.9pt,
only marks,
mark=*,
mark options={solid,fill=mycolor39,draw=black,line width=0.1pt},
forget plot
]
coordinates{
 (20,0.802575107296137) 
};
\addplot [
color=blue,
mark size=0.9pt,
only marks,
mark=*,
mark options={solid,fill=mycolor46,draw=black,line width=0.1pt},
forget plot
]
coordinates{
 (14,0.789799072642967) 
};
\addplot [
color=blue,
mark size=0.9pt,
only marks,
mark=*,
mark options={solid,fill=mycolor46,draw=black,line width=0.1pt},
forget plot
]
coordinates{
 (9,0.619961612284069) 
};
\addplot [
color=blue,
mark size=0.9pt,
only marks,
mark=*,
mark options={solid,fill=mycolor46,draw=black,line width=0.1pt},
forget plot
]
coordinates{
 (12,0.713906111603189) 
};
\addplot [
color=blue,
mark size=0.9pt,
only marks,
mark=*,
mark options={solid,fill=mycolor47,draw=black,line width=0.1pt},
forget plot
]
coordinates{
 (14,0.783699059561129) 
};
\addplot [
color=blue,
mark size=0.9pt,
only marks,
mark=*,
mark options={solid,fill=mycolor47,draw=black,line width=0.1pt},
forget plot
]
coordinates{
 (16,0.779179810725552) 
};
\addplot [
color=blue,
mark size=0.9pt,
only marks,
mark=*,
mark options={solid,fill=mycolor48,draw=black,line width=0.1pt},
forget plot
]
coordinates{
 (10,0.670999187652315) 
};
\addplot [
color=blue,
mark size=0.9pt,
only marks,
mark=*,
mark options={solid,fill=mycolor48,draw=black,line width=0.1pt},
forget plot
]
coordinates{
 (14,0.428321678321678) 
};
\addplot [
color=blue,
mark size=0.9pt,
only marks,
mark=*,
mark options={solid,fill=mycolor40,draw=black,line width=0.1pt},
forget plot
]
coordinates{
 (6,0.56007226738934) 
};
\addplot [
color=blue,
mark size=0.9pt,
only marks,
mark=*,
mark options={solid,fill=mycolor40,draw=black,line width=0.1pt},
forget plot
]
coordinates{
 (8,0.633943427620632) 
};
\addplot [
color=blue,
mark size=0.9pt,
only marks,
mark=*,
mark options={solid,fill=mycolor40,draw=black,line width=0.1pt},
forget plot
]
coordinates{
 (16,0.75085910652921) 
};
\addplot [
color=blue,
mark size=0.9pt,
only marks,
mark=*,
mark options={solid,fill=mycolor49,draw=black,line width=0.1pt},
forget plot
]
coordinates{
 (11,0.765349032800673) 
};
\addplot [
color=blue,
mark size=0.9pt,
only marks,
mark=*,
mark options={solid,fill=mycolor49,draw=black,line width=0.1pt},
forget plot
]
coordinates{
 (15,0.779179810725552) 
};
\addplot [
color=blue,
mark size=0.9pt,
only marks,
mark=*,
mark options={solid,fill=mycolor50,draw=black,line width=0.1pt},
forget plot
]
coordinates{
 (2,0) 
};
\addplot [
color=blue,
mark size=0.9pt,
only marks,
mark=*,
mark options={solid,fill=mycolor41,draw=black,line width=0.1pt},
forget plot
]
coordinates{
 (12,0) 
};
\addplot [
color=blue,
mark size=0.9pt,
only marks,
mark=*,
mark options={solid,fill=mycolor41,draw=black,line width=0.1pt},
forget plot
]
coordinates{
 (20,0.784682080924855) 
};
\addplot [
color=blue,
mark size=0.9pt,
only marks,
mark=*,
mark options={solid,fill=mycolor42,draw=black,line width=0.1pt},
forget plot
]
coordinates{
 (20,0.774860779634049) 
};
\addplot [
color=blue,
mark size=0.9pt,
only marks,
mark=*,
mark options={solid,fill=mycolor42,draw=black,line width=0.1pt},
forget plot
]
coordinates{
 (23,0.780373831775701) 
};
\addplot [
color=blue,
mark size=0.9pt,
only marks,
mark=*,
mark options={solid,fill=mycolor42,draw=black,line width=0.1pt},
forget plot
]
coordinates{
 (14,0.244705882352941) 
};
\addplot [
color=blue,
mark size=0.9pt,
only marks,
mark=*,
mark options={solid,fill=mycolor38,draw=black,line width=0.1pt},
forget plot
]
coordinates{
 (20,0.785763175906913) 
};
\addplot [
color=blue,
mark size=0.9pt,
only marks,
mark=*,
mark options={solid,fill=mycolor38,draw=black,line width=0.1pt},
forget plot
]
coordinates{
 (24,0.837570621468926) 
};
\addplot [
color=blue,
mark size=0.9pt,
only marks,
mark=*,
mark options={solid,fill=mycolor38,draw=black,line width=0.1pt},
forget plot
]
coordinates{
 (20,0.7601246105919) 
};
\addplot [
color=blue,
mark size=0.9pt,
only marks,
mark=*,
mark options={solid,fill=mycolor43,draw=black,line width=0.1pt},
forget plot
]
coordinates{
 (19,0.784067085953878) 
};
\addplot [
color=blue,
mark size=0.9pt,
only marks,
mark=*,
mark options={solid,fill=mycolor43,draw=black,line width=0.1pt},
forget plot
]
coordinates{
 (21,0.762811127379209) 
};
\addplot [
color=blue,
mark size=0.9pt,
only marks,
mark=*,
mark options={solid,fill=mycolor44,draw=black,line width=0.1pt},
forget plot
]
coordinates{
 (16,0.776432606941081) 
};
\addplot [
color=blue,
mark size=0.9pt,
only marks,
mark=*,
mark options={solid,fill=mycolor44,draw=black,line width=0.1pt},
forget plot
]
coordinates{
 (18,0.753911806543385) 
};
\addplot [
color=blue,
mark size=0.9pt,
only marks,
mark=*,
mark options={solid,fill=mycolor44,draw=black,line width=0.1pt},
forget plot
]
coordinates{
 (17,0.744379683597002) 
};
\addplot [
color=blue,
mark size=0.9pt,
only marks,
mark=*,
mark options={solid,fill=mycolor45,draw=black,line width=0.1pt},
forget plot
]
coordinates{
 (20,0.748858447488584) 
};
\addplot [
color=blue,
mark size=0.9pt,
only marks,
mark=*,
mark options={solid,fill=mycolor45,draw=black,line width=0.1pt},
forget plot
]
coordinates{
 (11,0.710108604845447) 
};
\addplot [
color=blue,
mark size=0.9pt,
only marks,
mark=*,
mark options={solid,fill=mycolor39,draw=black,line width=0.1pt},
forget plot
]
coordinates{
 (16,0.723051409618574) 
};
\addplot [
color=blue,
mark size=0.9pt,
only marks,
mark=*,
mark options={solid,fill=mycolor39,draw=black,line width=0.1pt},
forget plot
]
coordinates{
 (15,0.775181305398872) 
};
\addplot [
color=blue,
mark size=0.9pt,
only marks,
mark=*,
mark options={solid,fill=mycolor46,draw=black,line width=0.1pt},
forget plot
]
coordinates{
 (17,0.800929512006197) 
};
\addplot [
color=blue,
mark size=0.9pt,
only marks,
mark=*,
mark options={solid,fill=mycolor46,draw=black,line width=0.1pt},
forget plot
]
coordinates{
 (15,0.726307808946171) 
};
\addplot [
color=blue,
mark size=0.9pt,
only marks,
mark=*,
mark options={solid,fill=mycolor46,draw=black,line width=0.1pt},
forget plot
]
coordinates{
 (18,0.743791641429437) 
};
\addplot [
color=blue,
mark size=0.9pt,
only marks,
mark=*,
mark options={solid,fill=mycolor47,draw=black,line width=0.1pt},
forget plot
]
coordinates{
 (19,0.712005551700208) 
};
\addplot [
color=blue,
mark size=0.9pt,
only marks,
mark=*,
mark options={solid,fill=mycolor47,draw=black,line width=0.1pt},
forget plot
]
coordinates{
 (20,0.763969974979149) 
};
\addplot [
color=blue,
mark size=0.9pt,
only marks,
mark=*,
mark options={solid,fill=mycolor48,draw=black,line width=0.1pt},
forget plot
]
coordinates{
 (13,0.793650793650794) 
};
\addplot [
color=blue,
mark size=0.9pt,
only marks,
mark=*,
mark options={solid,fill=mycolor48,draw=black,line width=0.1pt},
forget plot
]
coordinates{
 (13,0.733113673805601) 
};
\addplot [
color=blue,
mark size=0.9pt,
only marks,
mark=*,
mark options={solid,fill=mycolor40,draw=black,line width=0.1pt},
forget plot
]
coordinates{
 (6,0) 
};
\addplot [
color=blue,
mark size=0.9pt,
only marks,
mark=*,
mark options={solid,fill=mycolor40,draw=black,line width=0.1pt},
forget plot
]
coordinates{
 (6,0) 
};
\addplot [
color=blue,
mark size=0.9pt,
only marks,
mark=*,
mark options={solid,fill=mycolor40,draw=black,line width=0.1pt},
forget plot
]
coordinates{
 (7,0.702362204724409) 
};
\addplot [
color=blue,
mark size=0.9pt,
only marks,
mark=*,
mark options={solid,fill=mycolor49,draw=black,line width=0.1pt},
forget plot
]
coordinates{
 (8,0.655021834061135) 
};
\addplot [
color=blue,
mark size=0.9pt,
only marks,
mark=*,
mark options={solid,fill=mycolor49,draw=black,line width=0.1pt},
forget plot
]
coordinates{
 (16,0.72210953346856) 
};
\addplot [
color=blue,
mark size=0.9pt,
only marks,
mark=*,
mark options={solid,fill=mycolor50,draw=black,line width=0.1pt},
forget plot
]
coordinates{
 (7,0.686119873817035) 
};
\addplot [
color=blue,
mark size=0.9pt,
only marks,
mark=*,
mark options={solid,fill=mycolor41,draw=black,line width=0.1pt},
forget plot
]
coordinates{
 (19,0.770338372930165) 
};
\addplot [
color=blue,
mark size=0.9pt,
only marks,
mark=*,
mark options={solid,fill=mycolor41,draw=black,line width=0.1pt},
forget plot
]
coordinates{
 (21,0.788990825688073) 
};
\addplot [
color=blue,
mark size=0.9pt,
only marks,
mark=*,
mark options={solid,fill=mycolor42,draw=black,line width=0.1pt},
forget plot
]
coordinates{
 (23,0.799222797927461) 
};
\addplot [
color=blue,
mark size=0.9pt,
only marks,
mark=*,
mark options={solid,fill=mycolor42,draw=black,line width=0.1pt},
forget plot
]
coordinates{
 (17,0.732888146911519) 
};
\addplot [
color=blue,
mark size=0.9pt,
only marks,
mark=*,
mark options={solid,fill=mycolor42,draw=black,line width=0.1pt},
forget plot
]
coordinates{
 (17,0.768601798855274) 
};
\addplot [
color=blue,
mark size=0.9pt,
only marks,
mark=*,
mark options={solid,fill=mycolor38,draw=black,line width=0.1pt},
forget plot
]
coordinates{
 (20,0.80835734870317) 
};
\addplot [
color=blue,
mark size=0.9pt,
only marks,
mark=*,
mark options={solid,fill=mycolor38,draw=black,line width=0.1pt},
forget plot
]
coordinates{
 (23,0.750147666863556) 
};
\addplot [
color=blue,
mark size=0.9pt,
only marks,
mark=*,
mark options={solid,fill=mycolor38,draw=black,line width=0.1pt},
forget plot
]
coordinates{
 (20,0.796502384737679) 
};
\addplot [
color=blue,
mark size=0.9pt,
only marks,
mark=*,
mark options={solid,fill=mycolor43,draw=black,line width=0.1pt},
forget plot
]
coordinates{
 (20,0.780766096169519) 
};
\addplot [
color=blue,
mark size=0.9pt,
only marks,
mark=*,
mark options={solid,fill=mycolor43,draw=black,line width=0.1pt},
forget plot
]
coordinates{
 (15,0.275334608030593) 
};
\addplot [
color=blue,
mark size=0.9pt,
only marks,
mark=*,
mark options={solid,fill=mycolor44,draw=black,line width=0.1pt},
forget plot
]
coordinates{
 (23,0.767203513909224) 
};
\addplot [
color=blue,
mark size=0.9pt,
only marks,
mark=*,
mark options={solid,fill=mycolor44,draw=black,line width=0.1pt},
forget plot
]
coordinates{
 (19,0.754047802621434) 
};
\addplot [
color=blue,
mark size=0.9pt,
only marks,
mark=*,
mark options={solid,fill=mycolor44,draw=black,line width=0.1pt},
forget plot
]
coordinates{
 (19,0.748137108792846) 
};
\addplot [
color=blue,
mark size=0.9pt,
only marks,
mark=*,
mark options={solid,fill=mycolor45,draw=black,line width=0.1pt},
forget plot
]
coordinates{
 (11,0.0344827586206896) 
};
\addplot [
color=blue,
mark size=0.9pt,
only marks,
mark=*,
mark options={solid,fill=mycolor45,draw=black,line width=0.1pt},
forget plot
]
coordinates{
 (17,0.755020080321285) 
};
\addplot [
color=blue,
mark size=0.9pt,
only marks,
mark=*,
mark options={solid,fill=mycolor39,draw=black,line width=0.1pt},
forget plot
]
coordinates{
 (19,0.76253081347576) 
};
\addplot [
color=blue,
mark size=0.9pt,
only marks,
mark=*,
mark options={solid,fill=mycolor39,draw=black,line width=0.1pt},
forget plot
]
coordinates{
 (16,0.80253766851705) 
};
\addplot [
color=blue,
mark size=0.9pt,
only marks,
mark=*,
mark options={solid,fill=mycolor46,draw=black,line width=0.1pt},
forget plot
]
coordinates{
 (14,0.771293375394322) 
};
\addplot [
color=blue,
mark size=0.9pt,
only marks,
mark=*,
mark options={solid,fill=mycolor46,draw=black,line width=0.1pt},
forget plot
]
coordinates{
 (16,0.544268406337372) 
};
\addplot [
color=blue,
mark size=0.9pt,
only marks,
mark=*,
mark options={solid,fill=mycolor46,draw=black,line width=0.1pt},
forget plot
]
coordinates{
 (21,0.743790849673202) 
};
\addplot [
color=blue,
mark size=0.9pt,
only marks,
mark=*,
mark options={solid,fill=mycolor47,draw=black,line width=0.1pt},
forget plot
]
coordinates{
 (14,0.239651416122004) 
};
\addplot [
color=blue,
mark size=0.9pt,
only marks,
mark=*,
mark options={solid,fill=mycolor47,draw=black,line width=0.1pt},
forget plot
]
coordinates{
 (17,0.757695343330702) 
};
\addplot [
color=blue,
mark size=0.9pt,
only marks,
mark=*,
mark options={solid,fill=mycolor48,draw=black,line width=0.1pt},
forget plot
]
coordinates{
 (14,0.654993514915694) 
};
\addplot [
color=blue,
mark size=0.9pt,
only marks,
mark=*,
mark options={solid,fill=mycolor48,draw=black,line width=0.1pt},
forget plot
]
coordinates{
 (10,0) 
};
\addplot [
color=blue,
mark size=0.9pt,
only marks,
mark=*,
mark options={solid,fill=mycolor40,draw=black,line width=0.1pt},
forget plot
]
coordinates{
 (15,0.768979591836735) 
};
\addplot [
color=blue,
mark size=0.9pt,
only marks,
mark=*,
mark options={solid,fill=mycolor40,draw=black,line width=0.1pt},
forget plot
]
coordinates{
 (6,0) 
};
\addplot [
color=blue,
mark size=0.9pt,
only marks,
mark=*,
mark options={solid,fill=mycolor40,draw=black,line width=0.1pt},
forget plot
]
coordinates{
 (11,0.782407407407407) 
};
\addplot [
color=blue,
mark size=0.9pt,
only marks,
mark=*,
mark options={solid,fill=mycolor49,draw=black,line width=0.1pt},
forget plot
]
coordinates{
 (4,0) 
};
\addplot [
color=blue,
mark size=0.9pt,
only marks,
mark=*,
mark options={solid,fill=mycolor49,draw=black,line width=0.1pt},
forget plot
]
coordinates{
 (12,0.241918665276329) 
};
\addplot [
color=blue,
mark size=0.9pt,
only marks,
mark=*,
mark options={solid,fill=mycolor50,draw=black,line width=0.1pt},
forget plot
]
coordinates{
 (15,0.754613807245386) 
};
\addplot [
color=blue,
mark size=0.9pt,
only marks,
mark=*,
mark options={solid,fill=mycolor41,draw=black,line width=0.1pt},
forget plot
]
coordinates{
 (22,0.78587528174305) 
};
\addplot [
color=blue,
mark size=0.9pt,
only marks,
mark=*,
mark options={solid,fill=mycolor41,draw=black,line width=0.1pt},
forget plot
]
coordinates{
 (20,0.736842105263158) 
};
\addplot [
color=blue,
mark size=0.9pt,
only marks,
mark=*,
mark options={solid,fill=mycolor42,draw=black,line width=0.1pt},
forget plot
]
coordinates{
 (20,0.741827885256838) 
};
\addplot [
color=blue,
mark size=0.9pt,
only marks,
mark=*,
mark options={solid,fill=mycolor42,draw=black,line width=0.1pt},
forget plot
]
coordinates{
 (15,0.758620689655172) 
};
\addplot [
color=blue,
mark size=0.9pt,
only marks,
mark=*,
mark options={solid,fill=mycolor42,draw=black,line width=0.1pt},
forget plot
]
coordinates{
 (22,0.776173285198556) 
};
\addplot [
color=blue,
mark size=0.9pt,
only marks,
mark=*,
mark options={solid,fill=mycolor38,draw=black,line width=0.1pt},
forget plot
]
coordinates{
 (20,0.791192103264996) 
};
\addplot [
color=blue,
mark size=0.9pt,
only marks,
mark=*,
mark options={solid,fill=mycolor38,draw=black,line width=0.1pt},
forget plot
]
coordinates{
 (18,0.749185667752443) 
};
\addplot [
color=blue,
mark size=0.9pt,
only marks,
mark=*,
mark options={solid,fill=mycolor38,draw=black,line width=0.1pt},
forget plot
]
coordinates{
 (16,0.787037037037037) 
};
\addplot [
color=blue,
mark size=0.9pt,
only marks,
mark=*,
mark options={solid,fill=mycolor43,draw=black,line width=0.1pt},
forget plot
]
coordinates{
 (18,0.71850699844479) 
};
\addplot [
color=blue,
mark size=0.9pt,
only marks,
mark=*,
mark options={solid,fill=mycolor43,draw=black,line width=0.1pt},
forget plot
]
coordinates{
 (18,0.778635778635778) 
};
\addplot [
color=blue,
mark size=0.9pt,
only marks,
mark=*,
mark options={solid,fill=mycolor44,draw=black,line width=0.1pt},
forget plot
]
coordinates{
 (19,0.745198463508322) 
};
\addplot [
color=blue,
mark size=0.9pt,
only marks,
mark=*,
mark options={solid,fill=mycolor44,draw=black,line width=0.1pt},
forget plot
]
coordinates{
 (17,0.774877650897227) 
};
\addplot [
color=blue,
mark size=0.9pt,
only marks,
mark=*,
mark options={solid,fill=mycolor44,draw=black,line width=0.1pt},
forget plot
]
coordinates{
 (19,0.786499215070643) 
};
\addplot [
color=blue,
mark size=0.9pt,
only marks,
mark=*,
mark options={solid,fill=mycolor45,draw=black,line width=0.1pt},
forget plot
]
coordinates{
 (15,0.700348432055749) 
};
\addplot [
color=blue,
mark size=0.9pt,
only marks,
mark=*,
mark options={solid,fill=mycolor45,draw=black,line width=0.1pt},
forget plot
]
coordinates{
 (17,0.753462603878116) 
};
\addplot [
color=blue,
mark size=0.9pt,
only marks,
mark=*,
mark options={solid,fill=mycolor39,draw=black,line width=0.1pt},
forget plot
]
coordinates{
 (16,0.792626728110599) 
};
\addplot [
color=blue,
mark size=0.9pt,
only marks,
mark=*,
mark options={solid,fill=mycolor39,draw=black,line width=0.1pt},
forget plot
]
coordinates{
 (15,0.741176470588235) 
};
\addplot [
color=blue,
mark size=0.9pt,
only marks,
mark=*,
mark options={solid,fill=mycolor46,draw=black,line width=0.1pt},
forget plot
]
coordinates{
 (19,0.765957446808511) 
};
\addplot [
color=blue,
mark size=0.9pt,
only marks,
mark=*,
mark options={solid,fill=mycolor46,draw=black,line width=0.1pt},
forget plot
]
coordinates{
 (16,0.658008658008658) 
};
\addplot [
color=blue,
mark size=0.9pt,
only marks,
mark=*,
mark options={solid,fill=mycolor46,draw=black,line width=0.1pt},
forget plot
]
coordinates{
 (12,0.7765625) 
};
\addplot [
color=blue,
mark size=0.9pt,
only marks,
mark=*,
mark options={solid,fill=mycolor47,draw=black,line width=0.1pt},
forget plot
]
coordinates{
 (15,0.795050270688322) 
};
\addplot [
color=blue,
mark size=0.9pt,
only marks,
mark=*,
mark options={solid,fill=mycolor47,draw=black,line width=0.1pt},
forget plot
]
coordinates{
 (10,0.555133079847909) 
};
\addplot [
color=blue,
mark size=0.9pt,
only marks,
mark=*,
mark options={solid,fill=mycolor48,draw=black,line width=0.1pt},
forget plot
]
coordinates{
 (12,0.674740484429066) 
};
\addplot [
color=blue,
mark size=0.9pt,
only marks,
mark=*,
mark options={solid,fill=mycolor48,draw=black,line width=0.1pt},
forget plot
]
coordinates{
 (15,0.685673556664291) 
};
\addplot [
color=blue,
mark size=0.9pt,
only marks,
mark=*,
mark options={solid,fill=mycolor40,draw=black,line width=0.1pt},
forget plot
]
coordinates{
 (19,0.782824112303881) 
};
\addplot [
color=blue,
mark size=0.9pt,
only marks,
mark=*,
mark options={solid,fill=mycolor40,draw=black,line width=0.1pt},
forget plot
]
coordinates{
 (8,0.538314176245211) 
};
\addplot [
color=blue,
mark size=0.9pt,
only marks,
mark=*,
mark options={solid,fill=mycolor40,draw=black,line width=0.1pt},
forget plot
]
coordinates{
 (11,0.752365930599369) 
};
\addplot [
color=blue,
mark size=0.9pt,
only marks,
mark=*,
mark options={solid,fill=mycolor49,draw=black,line width=0.1pt},
forget plot
]
coordinates{
 (16,0.707638279192274) 
};
\addplot [
color=blue,
mark size=0.9pt,
only marks,
mark=*,
mark options={solid,fill=mycolor49,draw=black,line width=0.1pt},
forget plot
]
coordinates{
 (3,0) 
};
\addplot [
color=blue,
mark size=0.9pt,
only marks,
mark=*,
mark options={solid,fill=mycolor50,draw=black,line width=0.1pt},
forget plot
]
coordinates{
 (4,0) 
};
\addplot [
color=blue,
mark size=0.9pt,
only marks,
mark=*,
mark options={solid,fill=mycolor41,draw=black,line width=0.1pt},
forget plot
]
coordinates{
 (19,0.701612903225806) 
};
\addplot [
color=blue,
mark size=0.9pt,
only marks,
mark=*,
mark options={solid,fill=mycolor41,draw=black,line width=0.1pt},
forget plot
]
coordinates{
 (17,0.708487084870849) 
};
\addplot [
color=blue,
mark size=0.9pt,
only marks,
mark=*,
mark options={solid,fill=mycolor42,draw=black,line width=0.1pt},
forget plot
]
coordinates{
 (23,0.76875) 
};
\addplot [
color=blue,
mark size=0.9pt,
only marks,
mark=*,
mark options={solid,fill=mycolor42,draw=black,line width=0.1pt},
forget plot
]
coordinates{
 (20,0.618410700236035) 
};
\addplot [
color=blue,
mark size=0.9pt,
only marks,
mark=*,
mark options={solid,fill=mycolor42,draw=black,line width=0.1pt},
forget plot
]
coordinates{
 (14,0.154440154440154) 
};
\addplot [
color=blue,
mark size=0.9pt,
only marks,
mark=*,
mark options={solid,fill=mycolor38,draw=black,line width=0.1pt},
forget plot
]
coordinates{
 (16,0.803559206023272) 
};
\addplot [
color=blue,
mark size=0.9pt,
only marks,
mark=*,
mark options={solid,fill=mycolor38,draw=black,line width=0.1pt},
forget plot
]
coordinates{
 (22,0.768802228412256) 
};
\addplot [
color=blue,
mark size=0.9pt,
only marks,
mark=*,
mark options={solid,fill=mycolor38,draw=black,line width=0.1pt},
forget plot
]
coordinates{
 (18,0.707070707070707) 
};
\addplot [
color=blue,
mark size=0.9pt,
only marks,
mark=*,
mark options={solid,fill=mycolor43,draw=black,line width=0.1pt},
forget plot
]
coordinates{
 (15,0.705693664795509) 
};
\addplot [
color=blue,
mark size=0.9pt,
only marks,
mark=*,
mark options={solid,fill=mycolor43,draw=black,line width=0.1pt},
forget plot
]
coordinates{
 (21,0.792251169004676) 
};
\addplot [
color=blue,
mark size=0.9pt,
only marks,
mark=*,
mark options={solid,fill=mycolor44,draw=black,line width=0.1pt},
forget plot
]
coordinates{
 (20,0.747268754552076) 
};
\addplot [
color=blue,
mark size=0.9pt,
only marks,
mark=*,
mark options={solid,fill=mycolor44,draw=black,line width=0.1pt},
forget plot
]
coordinates{
 (19,0.776588151320485) 
};
\addplot [
color=blue,
mark size=0.9pt,
only marks,
mark=*,
mark options={solid,fill=mycolor44,draw=black,line width=0.1pt},
forget plot
]
coordinates{
 (14,0.766265060240964) 
};
\addplot [
color=blue,
mark size=0.9pt,
only marks,
mark=*,
mark options={solid,fill=mycolor45,draw=black,line width=0.1pt},
forget plot
]
coordinates{
 (14,0.163265306122449) 
};
\addplot [
color=blue,
mark size=0.9pt,
only marks,
mark=*,
mark options={solid,fill=mycolor45,draw=black,line width=0.1pt},
forget plot
]
coordinates{
 (11,0.579512195121951) 
};
\addplot [
color=blue,
mark size=0.9pt,
only marks,
mark=*,
mark options={solid,fill=mycolor39,draw=black,line width=0.1pt},
forget plot
]
coordinates{
 (17,0.742808798646362) 
};
\addplot [
color=blue,
mark size=0.9pt,
only marks,
mark=*,
mark options={solid,fill=mycolor39,draw=black,line width=0.1pt},
forget plot
]
coordinates{
 (12,0.715702479338843) 
};
\addplot [
color=blue,
mark size=0.9pt,
only marks,
mark=*,
mark options={solid,fill=mycolor46,draw=black,line width=0.1pt},
forget plot
]
coordinates{
 (19,0.73938679245283) 
};
\addplot [
color=blue,
mark size=0.9pt,
only marks,
mark=*,
mark options={solid,fill=mycolor46,draw=black,line width=0.1pt},
forget plot
]
coordinates{
 (13,0.80327868852459) 
};
\addplot [
color=blue,
mark size=0.9pt,
only marks,
mark=*,
mark options={solid,fill=mycolor46,draw=black,line width=0.1pt},
forget plot
]
coordinates{
 (18,0.769945778466305) 
};
\addplot [
color=blue,
mark size=0.9pt,
only marks,
mark=*,
mark options={solid,fill=mycolor47,draw=black,line width=0.1pt},
forget plot
]
coordinates{
 (16,0.787923416789396) 
};
\addplot [
color=blue,
mark size=0.9pt,
only marks,
mark=*,
mark options={solid,fill=mycolor47,draw=black,line width=0.1pt},
forget plot
]
coordinates{
 (14,0.600749063670412) 
};
\addplot [
color=blue,
mark size=0.9pt,
only marks,
mark=*,
mark options={solid,fill=mycolor48,draw=black,line width=0.1pt},
forget plot
]
coordinates{
 (10,0.748376623376623) 
};
\addplot [
color=blue,
mark size=0.9pt,
only marks,
mark=*,
mark options={solid,fill=mycolor48,draw=black,line width=0.1pt},
forget plot
]
coordinates{
 (15,0.622540250447227) 
};
\addplot [
color=blue,
mark size=0.9pt,
only marks,
mark=*,
mark options={solid,fill=mycolor40,draw=black,line width=0.1pt},
forget plot
]
coordinates{
 (17,0.707350901525659) 
};
\addplot [
color=blue,
mark size=0.9pt,
only marks,
mark=*,
mark options={solid,fill=mycolor40,draw=black,line width=0.1pt},
forget plot
]
coordinates{
 (11,0.778501628664495) 
};
\addplot [
color=blue,
mark size=0.9pt,
only marks,
mark=*,
mark options={solid,fill=mycolor40,draw=black,line width=0.1pt},
forget plot
]
coordinates{
 (13,0.767616191904048) 
};
\addplot [
color=blue,
mark size=0.9pt,
only marks,
mark=*,
mark options={solid,fill=mycolor49,draw=black,line width=0.1pt},
forget plot
]
coordinates{
 (16,0.783748361730013) 
};
\addplot [
color=blue,
mark size=0.9pt,
only marks,
mark=*,
mark options={solid,fill=mycolor49,draw=black,line width=0.1pt},
forget plot
]
coordinates{
 (12,0.785150078988941) 
};
\addplot [
color=blue,
mark size=0.9pt,
only marks,
mark=*,
mark options={solid,fill=mycolor50,draw=black,line width=0.1pt},
forget plot
]
coordinates{
 (14,0.778123057799875) 
};
\addplot [
color=blue,
mark size=0.9pt,
only marks,
mark=*,
mark options={solid,fill=mycolor41,draw=black,line width=0.1pt},
forget plot
]
coordinates{
 (18,0.699236641221374) 
};
\addplot [
color=blue,
mark size=0.9pt,
only marks,
mark=*,
mark options={solid,fill=mycolor41,draw=black,line width=0.1pt},
forget plot
]
coordinates{
 (18,0.796795338674435) 
};
\addplot [
color=blue,
mark size=0.9pt,
only marks,
mark=*,
mark options={solid,fill=mycolor42,draw=black,line width=0.1pt},
forget plot
]
coordinates{
 (19,0.764617691154423) 
};
\addplot [
color=blue,
mark size=0.9pt,
only marks,
mark=*,
mark options={solid,fill=mycolor42,draw=black,line width=0.1pt},
forget plot
]
coordinates{
 (13,0.745791245791246) 
};
\addplot [
color=blue,
mark size=0.9pt,
only marks,
mark=*,
mark options={solid,fill=mycolor42,draw=black,line width=0.1pt},
forget plot
]
coordinates{
 (16,0.815028901734104) 
};
\addplot [
color=blue,
mark size=0.9pt,
only marks,
mark=*,
mark options={solid,fill=mycolor38,draw=black,line width=0.1pt},
forget plot
]
coordinates{
 (18,0.784883720930232) 
};
\addplot [
color=blue,
mark size=0.9pt,
only marks,
mark=*,
mark options={solid,fill=mycolor38,draw=black,line width=0.1pt},
forget plot
]
coordinates{
 (18,0.784711388455538) 
};
\addplot [
color=blue,
mark size=0.9pt,
only marks,
mark=*,
mark options={solid,fill=mycolor38,draw=black,line width=0.1pt},
forget plot
]
coordinates{
 (14,0.754426481909161) 
};
\addplot [
color=blue,
mark size=0.9pt,
only marks,
mark=*,
mark options={solid,fill=mycolor43,draw=black,line width=0.1pt},
forget plot
]
coordinates{
 (16,0.755795981452859) 
};
\addplot [
color=blue,
mark size=0.9pt,
only marks,
mark=*,
mark options={solid,fill=mycolor43,draw=black,line width=0.1pt},
forget plot
]
coordinates{
 (20,0.812742812742813) 
};
\addplot [
color=blue,
mark size=0.9pt,
only marks,
mark=*,
mark options={solid,fill=mycolor44,draw=black,line width=0.1pt},
forget plot
]
coordinates{
 (15,0.673497267759563) 
};
\addplot [
color=blue,
mark size=0.9pt,
only marks,
mark=*,
mark options={solid,fill=mycolor44,draw=black,line width=0.1pt},
forget plot
]
coordinates{
 (13,0.761437908496732) 
};
\addplot [
color=blue,
mark size=0.9pt,
only marks,
mark=*,
mark options={solid,fill=mycolor44,draw=black,line width=0.1pt},
forget plot
]
coordinates{
 (17,0.461095100864553) 
};
\addplot [
color=blue,
mark size=0.9pt,
only marks,
mark=*,
mark options={solid,fill=mycolor45,draw=black,line width=0.1pt},
forget plot
]
coordinates{
 (19,0.774668630338733) 
};
\addplot [
color=blue,
mark size=0.9pt,
only marks,
mark=*,
mark options={solid,fill=mycolor45,draw=black,line width=0.1pt},
forget plot
]
coordinates{
 (13,0.759148936170213) 
};
\addplot [
color=blue,
mark size=0.9pt,
only marks,
mark=*,
mark options={solid,fill=mycolor39,draw=black,line width=0.1pt},
forget plot
]
coordinates{
 (13,0.774395003903201) 
};
\addplot [
color=blue,
mark size=0.9pt,
only marks,
mark=*,
mark options={solid,fill=mycolor39,draw=black,line width=0.1pt},
forget plot
]
coordinates{
 (16,0.783608914450036) 
};
\addplot [
color=blue,
mark size=0.9pt,
only marks,
mark=*,
mark options={solid,fill=mycolor46,draw=black,line width=0.1pt},
forget plot
]
coordinates{
 (18,0.757939308398024) 
};
\addplot [
color=blue,
mark size=0.9pt,
only marks,
mark=*,
mark options={solid,fill=mycolor46,draw=black,line width=0.1pt},
forget plot
]
coordinates{
 (15,0.687631027253669) 
};
\addplot [
color=blue,
mark size=0.9pt,
only marks,
mark=*,
mark options={solid,fill=mycolor46,draw=black,line width=0.1pt},
forget plot
]
coordinates{
 (13,0.718964204112719) 
};
\addplot [
color=blue,
mark size=0.9pt,
only marks,
mark=*,
mark options={solid,fill=mycolor47,draw=black,line width=0.1pt},
forget plot
]
coordinates{
 (17,0.722778891115564) 
};
\addplot [
color=blue,
mark size=0.9pt,
only marks,
mark=*,
mark options={solid,fill=mycolor47,draw=black,line width=0.1pt},
forget plot
]
coordinates{
 (16,0.793243243243243) 
};
\addplot [
color=blue,
mark size=0.9pt,
only marks,
mark=*,
mark options={solid,fill=mycolor48,draw=black,line width=0.1pt},
forget plot
]
coordinates{
 (9,0) 
};
\addplot [
color=blue,
mark size=0.9pt,
only marks,
mark=*,
mark options={solid,fill=mycolor48,draw=black,line width=0.1pt},
forget plot
]
coordinates{
 (19,0.750716332378223) 
};
\addplot [
color=blue,
mark size=0.9pt,
only marks,
mark=*,
mark options={solid,fill=mycolor40,draw=black,line width=0.1pt},
forget plot
]
coordinates{
 (18,0.76459741856177) 
};
\addplot [
color=blue,
mark size=0.9pt,
only marks,
mark=*,
mark options={solid,fill=mycolor40,draw=black,line width=0.1pt},
forget plot
]
coordinates{
 (7,0) 
};
\addplot [
color=blue,
mark size=0.9pt,
only marks,
mark=*,
mark options={solid,fill=mycolor40,draw=black,line width=0.1pt},
forget plot
]
coordinates{
 (19,0.739656269891788) 
};
\addplot [
color=blue,
mark size=0.9pt,
only marks,
mark=*,
mark options={solid,fill=mycolor49,draw=black,line width=0.1pt},
forget plot
]
coordinates{
 (15,0.746395250212044) 
};
\addplot [
color=blue,
mark size=0.9pt,
only marks,
mark=*,
mark options={solid,fill=mycolor49,draw=black,line width=0.1pt},
forget plot
]
coordinates{
 (15,0.767284991568297) 
};
\addplot [
color=blue,
mark size=0.9pt,
only marks,
mark=*,
mark options={solid,fill=mycolor50,draw=black,line width=0.1pt},
forget plot
]
coordinates{
 (3,0) 
};
\addplot [
color=blue,
mark size=0.9pt,
only marks,
mark=*,
mark options={solid,fill=mycolor41,draw=black,line width=0.1pt},
forget plot
]
coordinates{
 (22,0.798223350253807) 
};
\addplot [
color=blue,
mark size=0.9pt,
only marks,
mark=*,
mark options={solid,fill=mycolor41,draw=black,line width=0.1pt},
forget plot
]
coordinates{
 (24,0.770580296896086) 
};
\addplot [
color=blue,
mark size=0.9pt,
only marks,
mark=*,
mark options={solid,fill=mycolor42,draw=black,line width=0.1pt},
forget plot
]
coordinates{
 (19,0.788355625491739) 
};
\addplot [
color=blue,
mark size=0.9pt,
only marks,
mark=*,
mark options={solid,fill=mycolor42,draw=black,line width=0.1pt},
forget plot
]
coordinates{
 (15,0.729684908789386) 
};
\addplot [
color=blue,
mark size=0.9pt,
only marks,
mark=*,
mark options={solid,fill=mycolor42,draw=black,line width=0.1pt},
forget plot
]
coordinates{
 (16,0.709849157054126) 
};
\addplot [
color=blue,
mark size=0.9pt,
only marks,
mark=*,
mark options={solid,fill=mycolor38,draw=black,line width=0.1pt},
forget plot
]
coordinates{
 (16,0.803459119496855) 
};
\addplot [
color=blue,
mark size=0.9pt,
only marks,
mark=*,
mark options={solid,fill=mycolor38,draw=black,line width=0.1pt},
forget plot
]
coordinates{
 (19,0.753898305084746) 
};
\addplot [
color=blue,
mark size=0.9pt,
only marks,
mark=*,
mark options={solid,fill=mycolor38,draw=black,line width=0.1pt},
forget plot
]
coordinates{
 (16,0.785658612626656) 
};
\addplot [
color=blue,
mark size=0.9pt,
only marks,
mark=*,
mark options={solid,fill=mycolor43,draw=black,line width=0.1pt},
forget plot
]
coordinates{
 (15,0.55378858746492) 
};
\addplot [
color=blue,
mark size=0.9pt,
only marks,
mark=*,
mark options={solid,fill=mycolor43,draw=black,line width=0.1pt},
forget plot
]
coordinates{
 (20,0.792452830188679) 
};
\addplot [
color=blue,
mark size=0.9pt,
only marks,
mark=*,
mark options={solid,fill=mycolor44,draw=black,line width=0.1pt},
forget plot
]
coordinates{
 (16,0.689908256880734) 
};
\addplot [
color=blue,
mark size=0.9pt,
only marks,
mark=*,
mark options={solid,fill=mycolor44,draw=black,line width=0.1pt},
forget plot
]
coordinates{
 (15,0.771604938271605) 
};
\addplot [
color=blue,
mark size=0.9pt,
only marks,
mark=*,
mark options={solid,fill=mycolor44,draw=black,line width=0.1pt},
forget plot
]
coordinates{
 (19,0.807888970051132) 
};
\addplot [
color=blue,
mark size=0.9pt,
only marks,
mark=*,
mark options={solid,fill=mycolor45,draw=black,line width=0.1pt},
forget plot
]
coordinates{
 (15,0.24896265560166) 
};
\addplot [
color=blue,
mark size=0.9pt,
only marks,
mark=*,
mark options={solid,fill=mycolor45,draw=black,line width=0.1pt},
forget plot
]
coordinates{
 (15,0.803041825095057) 
};
\addplot [
color=blue,
mark size=0.9pt,
only marks,
mark=*,
mark options={solid,fill=mycolor39,draw=black,line width=0.1pt},
forget plot
]
coordinates{
 (15,0.8125) 
};
\addplot [
color=blue,
mark size=0.9pt,
only marks,
mark=*,
mark options={solid,fill=mycolor39,draw=black,line width=0.1pt},
forget plot
]
coordinates{
 (15,0.724003127443315) 
};
\addplot [
color=blue,
mark size=0.9pt,
only marks,
mark=*,
mark options={solid,fill=mycolor46,draw=black,line width=0.1pt},
forget plot
]
coordinates{
 (14,0.768353528153956) 
};
\addplot [
color=blue,
mark size=0.9pt,
only marks,
mark=*,
mark options={solid,fill=mycolor46,draw=black,line width=0.1pt},
forget plot
]
coordinates{
 (19,0.754257907542579) 
};
\addplot [
color=blue,
mark size=0.9pt,
only marks,
mark=*,
mark options={solid,fill=mycolor46,draw=black,line width=0.1pt},
forget plot
]
coordinates{
 (11,0.751152073732719) 
};
\addplot [
color=blue,
mark size=0.9pt,
only marks,
mark=*,
mark options={solid,fill=mycolor47,draw=black,line width=0.1pt},
forget plot
]
coordinates{
 (11,0.711920529801324) 
};
\addplot [
color=blue,
mark size=0.9pt,
only marks,
mark=*,
mark options={solid,fill=mycolor47,draw=black,line width=0.1pt},
forget plot
]
coordinates{
 (19,0.773333333333333) 
};
\addplot [
color=blue,
mark size=0.9pt,
only marks,
mark=*,
mark options={solid,fill=mycolor48,draw=black,line width=0.1pt},
forget plot
]
coordinates{
 (19,0.765990639625585) 
};
\addplot [
color=blue,
mark size=0.9pt,
only marks,
mark=*,
mark options={solid,fill=mycolor48,draw=black,line width=0.1pt},
forget plot
]
coordinates{
 (10,0.247058823529412) 
};
\addplot [
color=blue,
mark size=0.9pt,
only marks,
mark=*,
mark options={solid,fill=mycolor40,draw=black,line width=0.1pt},
forget plot
]
coordinates{
 (16,0.743548387096774) 
};
\addplot [
color=blue,
mark size=0.9pt,
only marks,
mark=*,
mark options={solid,fill=mycolor40,draw=black,line width=0.1pt},
forget plot
]
coordinates{
 (11,0.746177370030581) 
};
\addplot [
color=blue,
mark size=0.9pt,
only marks,
mark=*,
mark options={solid,fill=mycolor40,draw=black,line width=0.1pt},
forget plot
]
coordinates{
 (13,0.650392327811683) 
};
\addplot [
color=blue,
mark size=0.9pt,
only marks,
mark=*,
mark options={solid,fill=mycolor49,draw=black,line width=0.1pt},
forget plot
]
coordinates{
 (11,0.693706293706294) 
};
\addplot [
color=blue,
mark size=0.9pt,
only marks,
mark=*,
mark options={solid,fill=mycolor49,draw=black,line width=0.1pt},
forget plot
]
coordinates{
 (4,0) 
};
\addplot [
color=blue,
mark size=0.9pt,
only marks,
mark=*,
mark options={solid,fill=mycolor50,draw=black,line width=0.1pt},
forget plot
]
coordinates{
 (12,0.610915492957747) 
};
\addplot [
color=blue,
mark size=0.9pt,
only marks,
mark=*,
mark options={solid,fill=mycolor41,draw=black,line width=0.1pt},
forget plot
]
coordinates{
 (19,0.808646917534027) 
};
\addplot [
color=blue,
mark size=0.9pt,
only marks,
mark=*,
mark options={solid,fill=mycolor41,draw=black,line width=0.1pt},
forget plot
]
coordinates{
 (22,0.8218966846569) 
};
\addplot [
color=blue,
mark size=0.9pt,
only marks,
mark=*,
mark options={solid,fill=mycolor42,draw=black,line width=0.1pt},
forget plot
]
coordinates{
 (19,0.734099616858237) 
};
\addplot [
color=blue,
mark size=0.9pt,
only marks,
mark=*,
mark options={solid,fill=mycolor42,draw=black,line width=0.1pt},
forget plot
]
coordinates{
 (23,0.769911504424779) 
};
\addplot [
color=blue,
mark size=0.9pt,
only marks,
mark=*,
mark options={solid,fill=mycolor42,draw=black,line width=0.1pt},
forget plot
]
coordinates{
 (16,0.753184713375796) 
};
\addplot [
color=blue,
mark size=0.9pt,
only marks,
mark=*,
mark options={solid,fill=mycolor38,draw=black,line width=0.1pt},
forget plot
]
coordinates{
 (16,0.722488038277512) 
};
\addplot [
color=blue,
mark size=0.9pt,
only marks,
mark=*,
mark options={solid,fill=mycolor38,draw=black,line width=0.1pt},
forget plot
]
coordinates{
 (19,0.765990639625585) 
};
\addplot [
color=blue,
mark size=0.9pt,
only marks,
mark=*,
mark options={solid,fill=mycolor38,draw=black,line width=0.1pt},
forget plot
]
coordinates{
 (23,0.799199466310874) 
};
\addplot [
color=blue,
mark size=0.9pt,
only marks,
mark=*,
mark options={solid,fill=mycolor43,draw=black,line width=0.1pt},
forget plot
]
coordinates{
 (17,0.761389961389961) 
};
\addplot [
color=blue,
mark size=0.9pt,
only marks,
mark=*,
mark options={solid,fill=mycolor43,draw=black,line width=0.1pt},
forget plot
]
coordinates{
 (18,0.747336377473364) 
};
\addplot [
color=blue,
mark size=0.9pt,
only marks,
mark=*,
mark options={solid,fill=mycolor44,draw=black,line width=0.1pt},
forget plot
]
coordinates{
 (15,0.784810126582278) 
};
\addplot [
color=blue,
mark size=0.9pt,
only marks,
mark=*,
mark options={solid,fill=mycolor44,draw=black,line width=0.1pt},
forget plot
]
coordinates{
 (16,0.796570537802026) 
};
\addplot [
color=blue,
mark size=0.9pt,
only marks,
mark=*,
mark options={solid,fill=mycolor44,draw=black,line width=0.1pt},
forget plot
]
coordinates{
 (20,0.791758646063282) 
};
\addplot [
color=blue,
mark size=0.9pt,
only marks,
mark=*,
mark options={solid,fill=mycolor45,draw=black,line width=0.1pt},
forget plot
]
coordinates{
 (12,0.253521126760563) 
};
\addplot [
color=blue,
mark size=0.9pt,
only marks,
mark=*,
mark options={solid,fill=mycolor45,draw=black,line width=0.1pt},
forget plot
]
coordinates{
 (13,0.705432287681714) 
};
\addplot [
color=blue,
mark size=0.9pt,
only marks,
mark=*,
mark options={solid,fill=mycolor39,draw=black,line width=0.1pt},
forget plot
]
coordinates{
 (15,0.790584415584416) 
};
\addplot [
color=blue,
mark size=0.9pt,
only marks,
mark=*,
mark options={solid,fill=mycolor39,draw=black,line width=0.1pt},
forget plot
]
coordinates{
 (17,0.564727954971857) 
};
\addplot [
color=blue,
mark size=0.9pt,
only marks,
mark=*,
mark options={solid,fill=mycolor46,draw=black,line width=0.1pt},
forget plot
]
coordinates{
 (14,0.608775137111517) 
};
\addplot [
color=blue,
mark size=0.9pt,
only marks,
mark=*,
mark options={solid,fill=mycolor46,draw=black,line width=0.1pt},
forget plot
]
coordinates{
 (19,0.736022646850672) 
};
\addplot [
color=blue,
mark size=0.9pt,
only marks,
mark=*,
mark options={solid,fill=mycolor46,draw=black,line width=0.1pt},
forget plot
]
coordinates{
 (21,0.787456445993031) 
};
\addplot [
color=blue,
mark size=0.9pt,
only marks,
mark=*,
mark options={solid,fill=mycolor47,draw=black,line width=0.1pt},
forget plot
]
coordinates{
 (14,0.798114689709348) 
};
\addplot [
color=blue,
mark size=0.9pt,
only marks,
mark=*,
mark options={solid,fill=mycolor47,draw=black,line width=0.1pt},
forget plot
]
coordinates{
 (13,0.780337941628264) 
};
\addplot [
color=blue,
mark size=0.9pt,
only marks,
mark=*,
mark options={solid,fill=mycolor48,draw=black,line width=0.1pt},
forget plot
]
coordinates{
 (15,0.792018419033001) 
};
\addplot [
color=blue,
mark size=0.9pt,
only marks,
mark=*,
mark options={solid,fill=mycolor48,draw=black,line width=0.1pt},
forget plot
]
coordinates{
 (7,0) 
};
\addplot [
color=blue,
mark size=0.9pt,
only marks,
mark=*,
mark options={solid,fill=mycolor40,draw=black,line width=0.1pt},
forget plot
]
coordinates{
 (11,0.11070110701107) 
};
\addplot [
color=blue,
mark size=0.9pt,
only marks,
mark=*,
mark options={solid,fill=mycolor40,draw=black,line width=0.1pt},
forget plot
]
coordinates{
 (7,0) 
};
\addplot [
color=blue,
mark size=0.9pt,
only marks,
mark=*,
mark options={solid,fill=mycolor40,draw=black,line width=0.1pt},
forget plot
]
coordinates{
 (14,0.772200772200772) 
};
\addplot [
color=blue,
mark size=0.9pt,
only marks,
mark=*,
mark options={solid,fill=mycolor49,draw=black,line width=0.1pt},
forget plot
]
coordinates{
 (11,0.171683389074693) 
};
\addplot [
color=blue,
mark size=0.9pt,
only marks,
mark=*,
mark options={solid,fill=mycolor49,draw=black,line width=0.1pt},
forget plot
]
coordinates{
 (5,0) 
};
\addplot [
color=blue,
mark size=0.9pt,
only marks,
mark=*,
mark options={solid,fill=mycolor50,draw=black,line width=0.1pt},
forget plot
]
coordinates{
 (9,0.749797570850202) 
};
\addplot [
color=blue,
mark size=0.9pt,
only marks,
mark=*,
mark options={solid,fill=mycolor41,draw=black,line width=0.1pt},
forget plot
]
coordinates{
 (8,0) 
};
\addplot [
color=blue,
mark size=0.9pt,
only marks,
mark=*,
mark options={solid,fill=mycolor41,draw=black,line width=0.1pt},
forget plot
]
coordinates{
 (21,0.736688548504741) 
};
\addplot [
color=blue,
mark size=0.9pt,
only marks,
mark=*,
mark options={solid,fill=mycolor42,draw=black,line width=0.1pt},
forget plot
]
coordinates{
 (21,0.780790085205267) 
};
\addplot [
color=blue,
mark size=0.9pt,
only marks,
mark=*,
mark options={solid,fill=mycolor42,draw=black,line width=0.1pt},
forget plot
]
coordinates{
 (23,0.796758946657664) 
};
\addplot [
color=blue,
mark size=0.9pt,
only marks,
mark=*,
mark options={solid,fill=mycolor42,draw=black,line width=0.1pt},
forget plot
]
coordinates{
 (20,0.731966590736522) 
};
\addplot [
color=blue,
mark size=0.9pt,
only marks,
mark=*,
mark options={solid,fill=mycolor38,draw=black,line width=0.1pt},
forget plot
]
coordinates{
 (20,0.745393634840871) 
};
\addplot [
color=blue,
mark size=0.9pt,
only marks,
mark=*,
mark options={solid,fill=mycolor38,draw=black,line width=0.1pt},
forget plot
]
coordinates{
 (19,0.800959232613909) 
};
\addplot [
color=blue,
mark size=0.9pt,
only marks,
mark=*,
mark options={solid,fill=mycolor38,draw=black,line width=0.1pt},
forget plot
]
coordinates{
 (21,0.754349338900487) 
};
\addplot [
color=blue,
mark size=0.9pt,
only marks,
mark=*,
mark options={solid,fill=mycolor43,draw=black,line width=0.1pt},
forget plot
]
coordinates{
 (17,0.731297709923664) 
};
\addplot [
color=blue,
mark size=0.9pt,
only marks,
mark=*,
mark options={solid,fill=mycolor43,draw=black,line width=0.1pt},
forget plot
]
coordinates{
 (13,0.682512733446519) 
};
\addplot [
color=blue,
mark size=0.9pt,
only marks,
mark=*,
mark options={solid,fill=mycolor44,draw=black,line width=0.1pt},
forget plot
]
coordinates{
 (19,0.761020881670533) 
};
\addplot [
color=blue,
mark size=0.9pt,
only marks,
mark=*,
mark options={solid,fill=mycolor44,draw=black,line width=0.1pt},
forget plot
]
coordinates{
 (16,0.782870022539444) 
};
\addplot [
color=blue,
mark size=0.9pt,
only marks,
mark=*,
mark options={solid,fill=mycolor44,draw=black,line width=0.1pt},
forget plot
]
coordinates{
 (15,0.670618120237087) 
};
\addplot [
color=blue,
mark size=0.9pt,
only marks,
mark=*,
mark options={solid,fill=mycolor45,draw=black,line width=0.1pt},
forget plot
]
coordinates{
 (16,0.569811320754717) 
};
\addplot [
color=blue,
mark size=0.9pt,
only marks,
mark=*,
mark options={solid,fill=mycolor45,draw=black,line width=0.1pt},
forget plot
]
coordinates{
 (18,0.654594232059021) 
};
\addplot [
color=blue,
mark size=0.9pt,
only marks,
mark=*,
mark options={solid,fill=mycolor39,draw=black,line width=0.1pt},
forget plot
]
coordinates{
 (21,0.796403339755941) 
};
\addplot [
color=blue,
mark size=0.9pt,
only marks,
mark=*,
mark options={solid,fill=mycolor39,draw=black,line width=0.1pt},
forget plot
]
coordinates{
 (11,0) 
};
\addplot [
color=blue,
mark size=0.9pt,
only marks,
mark=*,
mark options={solid,fill=mycolor46,draw=black,line width=0.1pt},
forget plot
]
coordinates{
 (13,0.771084337349397) 
};
\addplot [
color=blue,
mark size=0.9pt,
only marks,
mark=*,
mark options={solid,fill=mycolor46,draw=black,line width=0.1pt},
forget plot
]
coordinates{
 (9,0.0164609053497942) 
};
\addplot [
color=blue,
mark size=0.9pt,
only marks,
mark=*,
mark options={solid,fill=mycolor46,draw=black,line width=0.1pt},
forget plot
]
coordinates{
 (13,0.729751403368083) 
};
\addplot [
color=blue,
mark size=0.9pt,
only marks,
mark=*,
mark options={solid,fill=mycolor47,draw=black,line width=0.1pt},
forget plot
]
coordinates{
 (15,0.72) 
};
\addplot [
color=blue,
mark size=0.9pt,
only marks,
mark=*,
mark options={solid,fill=mycolor47,draw=black,line width=0.1pt},
forget plot
]
coordinates{
 (16,0.753558052434457) 
};
\addplot [
color=blue,
mark size=0.9pt,
only marks,
mark=*,
mark options={solid,fill=mycolor48,draw=black,line width=0.1pt},
forget plot
]
coordinates{
 (15,0.713656387665198) 
};
\addplot [
color=blue,
mark size=0.9pt,
only marks,
mark=*,
mark options={solid,fill=mycolor48,draw=black,line width=0.1pt},
forget plot
]
coordinates{
 (14,0.772655007949125) 
};
\addplot [
color=blue,
mark size=0.9pt,
only marks,
mark=*,
mark options={solid,fill=mycolor40,draw=black,line width=0.1pt},
forget plot
]
coordinates{
 (15,0.732336956521739) 
};
\addplot [
color=blue,
mark size=0.9pt,
only marks,
mark=*,
mark options={solid,fill=mycolor40,draw=black,line width=0.1pt},
forget plot
]
coordinates{
 (14,0.781666666666667) 
};
\addplot [
color=blue,
mark size=0.9pt,
only marks,
mark=*,
mark options={solid,fill=mycolor40,draw=black,line width=0.1pt},
forget plot
]
coordinates{
 (15,0.762043795620438) 
};
\addplot [
color=blue,
mark size=0.9pt,
only marks,
mark=*,
mark options={solid,fill=mycolor49,draw=black,line width=0.1pt},
forget plot
]
coordinates{
 (12,0.738284066330209) 
};
\addplot [
color=blue,
mark size=0.9pt,
only marks,
mark=*,
mark options={solid,fill=mycolor49,draw=black,line width=0.1pt},
forget plot
]
coordinates{
 (14,0.716917922948074) 
};
\addplot [
color=blue,
mark size=0.9pt,
only marks,
mark=*,
mark options={solid,fill=mycolor50,draw=black,line width=0.1pt},
forget plot
]
coordinates{
 (3,0) 
};
\addplot [
color=blue,
mark size=0.9pt,
only marks,
mark=*,
mark options={solid,fill=mycolor41,draw=black,line width=0.1pt},
forget plot
]
coordinates{
 (21,0.692116182572614) 
};
\addplot [
color=blue,
mark size=0.9pt,
only marks,
mark=*,
mark options={solid,fill=mycolor41,draw=black,line width=0.1pt},
forget plot
]
coordinates{
 (21,0.774531835205992) 
};
\addplot [
color=blue,
mark size=0.9pt,
only marks,
mark=*,
mark options={solid,fill=mycolor42,draw=black,line width=0.1pt},
forget plot
]
coordinates{
 (14,0.746987951807229) 
};
\addplot [
color=blue,
mark size=0.9pt,
only marks,
mark=*,
mark options={solid,fill=mycolor42,draw=black,line width=0.1pt},
forget plot
]
coordinates{
 (14,0.716862745098039) 
};
\addplot [
color=blue,
mark size=0.9pt,
only marks,
mark=*,
mark options={solid,fill=mycolor42,draw=black,line width=0.1pt},
forget plot
]
coordinates{
 (17,0.652766639935846) 
};
\addplot [
color=blue,
mark size=0.9pt,
only marks,
mark=*,
mark options={solid,fill=mycolor38,draw=black,line width=0.1pt},
forget plot
]
coordinates{
 (19,0.749521988527725) 
};
\addplot [
color=blue,
mark size=0.9pt,
only marks,
mark=*,
mark options={solid,fill=mycolor38,draw=black,line width=0.1pt},
forget plot
]
coordinates{
 (23,0.789259560618389) 
};
\addplot [
color=blue,
mark size=0.9pt,
only marks,
mark=*,
mark options={solid,fill=mycolor38,draw=black,line width=0.1pt},
forget plot
]
coordinates{
 (18,0.770949720670391) 
};
\addplot [
color=blue,
mark size=0.9pt,
only marks,
mark=*,
mark options={solid,fill=mycolor43,draw=black,line width=0.1pt},
forget plot
]
coordinates{
 (16,0.708383961117861) 
};
\addplot [
color=blue,
mark size=0.9pt,
only marks,
mark=*,
mark options={solid,fill=mycolor43,draw=black,line width=0.1pt},
forget plot
]
coordinates{
 (17,0.792511700468019) 
};
\addplot [
color=blue,
mark size=0.9pt,
only marks,
mark=*,
mark options={solid,fill=mycolor44,draw=black,line width=0.1pt},
forget plot
]
coordinates{
 (17,0.697916666666667) 
};
\addplot [
color=blue,
mark size=0.9pt,
only marks,
mark=*,
mark options={solid,fill=mycolor44,draw=black,line width=0.1pt},
forget plot
]
coordinates{
 (19,0.695876288659794) 
};
\addplot [
color=blue,
mark size=0.9pt,
only marks,
mark=*,
mark options={solid,fill=mycolor44,draw=black,line width=0.1pt},
forget plot
]
coordinates{
 (20,0.727420667209113) 
};
\addplot [
color=blue,
mark size=0.9pt,
only marks,
mark=*,
mark options={solid,fill=mycolor45,draw=black,line width=0.1pt},
forget plot
]
coordinates{
 (15,0.747044917257683) 
};
\addplot [
color=blue,
mark size=0.9pt,
only marks,
mark=*,
mark options={solid,fill=mycolor45,draw=black,line width=0.1pt},
forget plot
]
coordinates{
 (17,0.762225969645868) 
};
\addplot [
color=blue,
mark size=0.9pt,
only marks,
mark=*,
mark options={solid,fill=mycolor39,draw=black,line width=0.1pt},
forget plot
]
coordinates{
 (21,0.77491961414791) 
};
\addplot [
color=blue,
mark size=0.9pt,
only marks,
mark=*,
mark options={solid,fill=mycolor39,draw=black,line width=0.1pt},
forget plot
]
coordinates{
 (19,0.752040175768989) 
};
\addplot [
color=blue,
mark size=0.9pt,
only marks,
mark=*,
mark options={solid,fill=mycolor46,draw=black,line width=0.1pt},
forget plot
]
coordinates{
 (18,0.767019667170953) 
};
\addplot [
color=blue,
mark size=0.9pt,
only marks,
mark=*,
mark options={solid,fill=mycolor46,draw=black,line width=0.1pt},
forget plot
]
coordinates{
 (20,0.790697674418605) 
};
\addplot [
color=blue,
mark size=0.9pt,
only marks,
mark=*,
mark options={solid,fill=mycolor46,draw=black,line width=0.1pt},
forget plot
]
coordinates{
 (18,0.75642965204236) 
};
\addplot [
color=blue,
mark size=0.9pt,
only marks,
mark=*,
mark options={solid,fill=mycolor47,draw=black,line width=0.1pt},
forget plot
]
coordinates{
 (6,0) 
};
\addplot [
color=blue,
mark size=0.9pt,
only marks,
mark=*,
mark options={solid,fill=mycolor47,draw=black,line width=0.1pt},
forget plot
]
coordinates{
 (17,0.788519637462236) 
};
\addplot [
color=blue,
mark size=0.9pt,
only marks,
mark=*,
mark options={solid,fill=mycolor48,draw=black,line width=0.1pt},
forget plot
]
coordinates{
 (17,0.774821544451655) 
};
\addplot [
color=blue,
mark size=0.9pt,
only marks,
mark=*,
mark options={solid,fill=mycolor48,draw=black,line width=0.1pt},
forget plot
]
coordinates{
 (8,0.240252897787144) 
};
\addplot [
color=blue,
mark size=0.9pt,
only marks,
mark=*,
mark options={solid,fill=mycolor40,draw=black,line width=0.1pt},
forget plot
]
coordinates{
 (13,0.696110744891233) 
};
\addplot [
color=blue,
mark size=0.9pt,
only marks,
mark=*,
mark options={solid,fill=mycolor40,draw=black,line width=0.1pt},
forget plot
]
coordinates{
 (16,0.759247242050616) 
};
\addplot [
color=blue,
mark size=0.9pt,
only marks,
mark=*,
mark options={solid,fill=mycolor40,draw=black,line width=0.1pt},
forget plot
]
coordinates{
 (17,0.772264631043257) 
};
\addplot [
color=blue,
mark size=0.9pt,
only marks,
mark=*,
mark options={solid,fill=mycolor49,draw=black,line width=0.1pt},
forget plot
]
coordinates{
 (4,0) 
};
\addplot [
color=blue,
mark size=0.9pt,
only marks,
mark=*,
mark options={solid,fill=mycolor49,draw=black,line width=0.1pt},
forget plot
]
coordinates{
 (10,0.65300727032386) 
};
\addplot [
color=blue,
mark size=0.9pt,
only marks,
mark=*,
mark options={solid,fill=mycolor50,draw=black,line width=0.1pt},
forget plot
]
coordinates{
 (18,0.790629575402635) 
};
\addplot [
color=blue,
mark size=0.9pt,
only marks,
mark=*,
mark options={solid,fill=mycolor41,draw=black,line width=0.1pt},
forget plot
]
coordinates{
 (13,0.259593679458239) 
};
\addplot [
color=blue,
mark size=0.9pt,
only marks,
mark=*,
mark options={solid,fill=mycolor41,draw=black,line width=0.1pt},
forget plot
]
coordinates{
 (17,0.755277560594214) 
};
\addplot [
color=blue,
mark size=0.9pt,
only marks,
mark=*,
mark options={solid,fill=mycolor42,draw=black,line width=0.1pt},
forget plot
]
coordinates{
 (18,0.765543071161049) 
};
\addplot [
color=blue,
mark size=0.9pt,
only marks,
mark=*,
mark options={solid,fill=mycolor42,draw=black,line width=0.1pt},
forget plot
]
coordinates{
 (21,0.762805358550039) 
};
\addplot [
color=blue,
mark size=0.9pt,
only marks,
mark=*,
mark options={solid,fill=mycolor42,draw=black,line width=0.1pt},
forget plot
]
coordinates{
 (19,0.746590093323762) 
};
\addplot [
color=blue,
mark size=0.9pt,
only marks,
mark=*,
mark options={solid,fill=mycolor38,draw=black,line width=0.1pt},
forget plot
]
coordinates{
 (19,0.822342901474011) 
};
\addplot [
color=blue,
mark size=0.9pt,
only marks,
mark=*,
mark options={solid,fill=mycolor38,draw=black,line width=0.1pt},
forget plot
]
coordinates{
 (19,0.711356466876972) 
};
\addplot [
color=blue,
mark size=0.9pt,
only marks,
mark=*,
mark options={solid,fill=mycolor38,draw=black,line width=0.1pt},
forget plot
]
coordinates{
 (20,0.793478260869565) 
};
\addplot [
color=blue,
mark size=0.9pt,
only marks,
mark=*,
mark options={solid,fill=mycolor43,draw=black,line width=0.1pt},
forget plot
]
coordinates{
 (14,0.634275618374558) 
};
\addplot [
color=blue,
mark size=0.9pt,
only marks,
mark=*,
mark options={solid,fill=mycolor43,draw=black,line width=0.1pt},
forget plot
]
coordinates{
 (17,0.650709219858156) 
};
\addplot [
color=blue,
mark size=0.9pt,
only marks,
mark=*,
mark options={solid,fill=mycolor44,draw=black,line width=0.1pt},
forget plot
]
coordinates{
 (19,0.773680404916847) 
};
\addplot [
color=blue,
mark size=0.9pt,
only marks,
mark=*,
mark options={solid,fill=mycolor44,draw=black,line width=0.1pt},
forget plot
]
coordinates{
 (16,0.73394495412844) 
};
\addplot [
color=blue,
mark size=0.9pt,
only marks,
mark=*,
mark options={solid,fill=mycolor44,draw=black,line width=0.1pt},
forget plot
]
coordinates{
 (13,0.515410958904109) 
};
\addplot [
color=blue,
mark size=0.9pt,
only marks,
mark=*,
mark options={solid,fill=mycolor45,draw=black,line width=0.1pt},
forget plot
]
coordinates{
 (11,0) 
};
\addplot [
color=blue,
mark size=0.9pt,
only marks,
mark=*,
mark options={solid,fill=mycolor45,draw=black,line width=0.1pt},
forget plot
]
coordinates{
 (17,0.754978354978355) 
};
\addplot [
color=blue,
mark size=0.9pt,
only marks,
mark=*,
mark options={solid,fill=mycolor39,draw=black,line width=0.1pt},
forget plot
]
coordinates{
 (17,0.773448773448773) 
};
\addplot [
color=blue,
mark size=0.9pt,
only marks,
mark=*,
mark options={solid,fill=mycolor39,draw=black,line width=0.1pt},
forget plot
]
coordinates{
 (22,0.768020969855832) 
};
\addplot [
color=blue,
mark size=0.9pt,
only marks,
mark=*,
mark options={solid,fill=mycolor46,draw=black,line width=0.1pt},
forget plot
]
coordinates{
 (22,0.786707882534776) 
};
\addplot [
color=blue,
mark size=0.9pt,
only marks,
mark=*,
mark options={solid,fill=mycolor46,draw=black,line width=0.1pt},
forget plot
]
coordinates{
 (16,0.722662440570523) 
};
\addplot [
color=blue,
mark size=0.9pt,
only marks,
mark=*,
mark options={solid,fill=mycolor46,draw=black,line width=0.1pt},
forget plot
]
coordinates{
 (17,0.766638584667228) 
};
\addplot [
color=blue,
mark size=0.9pt,
only marks,
mark=*,
mark options={solid,fill=mycolor47,draw=black,line width=0.1pt},
forget plot
]
coordinates{
 (10,0) 
};
\addplot [
color=blue,
mark size=0.9pt,
only marks,
mark=*,
mark options={solid,fill=mycolor47,draw=black,line width=0.1pt},
forget plot
]
coordinates{
 (20,0.72477693891558) 
};
\addplot [
color=blue,
mark size=0.9pt,
only marks,
mark=*,
mark options={solid,fill=mycolor48,draw=black,line width=0.1pt},
forget plot
]
coordinates{
 (16,0.734987990392314) 
};
\addplot [
color=blue,
mark size=0.9pt,
only marks,
mark=*,
mark options={solid,fill=mycolor48,draw=black,line width=0.1pt},
forget plot
]
coordinates{
 (16,0.786254295532646) 
};
\addplot [
color=blue,
mark size=0.9pt,
only marks,
mark=*,
mark options={solid,fill=mycolor40,draw=black,line width=0.1pt},
forget plot
]
coordinates{
 (16,0.641095890410959) 
};
\addplot [
color=blue,
mark size=0.9pt,
only marks,
mark=*,
mark options={solid,fill=mycolor40,draw=black,line width=0.1pt},
forget plot
]
coordinates{
 (18,0.806318681318681) 
};
\addplot [
color=blue,
mark size=0.9pt,
only marks,
mark=*,
mark options={solid,fill=mycolor40,draw=black,line width=0.1pt},
forget plot
]
coordinates{
 (17,0.779794790844515) 
};
\addplot [
color=blue,
mark size=0.9pt,
only marks,
mark=*,
mark options={solid,fill=mycolor49,draw=black,line width=0.1pt},
forget plot
]
coordinates{
 (10,0.727149627623561) 
};
\addplot [
color=blue,
mark size=0.9pt,
only marks,
mark=*,
mark options={solid,fill=mycolor49,draw=black,line width=0.1pt},
forget plot
]
coordinates{
 (16,0.799061767005473) 
};
\addplot [
color=blue,
mark size=0.9pt,
only marks,
mark=*,
mark options={solid,fill=mycolor50,draw=black,line width=0.1pt},
forget plot
]
coordinates{
 (3,0) 
};
\addplot [
color=blue,
mark size=0.9pt,
only marks,
mark=*,
mark options={solid,fill=mycolor41,draw=black,line width=0.1pt},
forget plot
]
coordinates{
 (22,0.811428571428571) 
};
\addplot [
color=blue,
mark size=0.9pt,
only marks,
mark=*,
mark options={solid,fill=mycolor41,draw=black,line width=0.1pt},
forget plot
]
coordinates{
 (21,0.758718861209964) 
};
\addplot [
color=blue,
mark size=0.9pt,
only marks,
mark=*,
mark options={solid,fill=mycolor42,draw=black,line width=0.1pt},
forget plot
]
coordinates{
 (22,0.831444759206799) 
};
\addplot [
color=blue,
mark size=0.9pt,
only marks,
mark=*,
mark options={solid,fill=mycolor42,draw=black,line width=0.1pt},
forget plot
]
coordinates{
 (17,0.786479802143446) 
};
\addplot [
color=blue,
mark size=0.9pt,
only marks,
mark=*,
mark options={solid,fill=mycolor42,draw=black,line width=0.1pt},
forget plot
]
coordinates{
 (18,0.756756756756757) 
};
\addplot [
color=blue,
mark size=0.9pt,
only marks,
mark=*,
mark options={solid,fill=mycolor38,draw=black,line width=0.1pt},
forget plot
]
coordinates{
 (22,0.760224538893344) 
};
\addplot [
color=blue,
mark size=0.9pt,
only marks,
mark=*,
mark options={solid,fill=mycolor38,draw=black,line width=0.1pt},
forget plot
]
coordinates{
 (22,0.799682034976153) 
};
\addplot [
color=blue,
mark size=0.9pt,
only marks,
mark=*,
mark options={solid,fill=mycolor38,draw=black,line width=0.1pt},
forget plot
]
coordinates{
 (16,0.705882352941176) 
};
\addplot [
color=blue,
mark size=0.9pt,
only marks,
mark=*,
mark options={solid,fill=mycolor43,draw=black,line width=0.1pt},
forget plot
]
coordinates{
 (18,0.764468371467025) 
};
\addplot [
color=blue,
mark size=0.9pt,
only marks,
mark=*,
mark options={solid,fill=mycolor43,draw=black,line width=0.1pt},
forget plot
]
coordinates{
 (21,0.756159728122345) 
};
\addplot [
color=blue,
mark size=0.9pt,
only marks,
mark=*,
mark options={solid,fill=mycolor44,draw=black,line width=0.1pt},
forget plot
]
coordinates{
 (18,0.755134281200632) 
};
\addplot [
color=blue,
mark size=0.9pt,
only marks,
mark=*,
mark options={solid,fill=mycolor44,draw=black,line width=0.1pt},
forget plot
]
coordinates{
 (18,0.758169934640523) 
};
\addplot [
color=blue,
mark size=0.9pt,
only marks,
mark=*,
mark options={solid,fill=mycolor44,draw=black,line width=0.1pt},
forget plot
]
coordinates{
 (19,0.74591381872214) 
};
\addplot [
color=blue,
mark size=0.9pt,
only marks,
mark=*,
mark options={solid,fill=mycolor45,draw=black,line width=0.1pt},
forget plot
]
coordinates{
 (18,0.755852842809364) 
};
\addplot [
color=blue,
mark size=0.9pt,
only marks,
mark=*,
mark options={solid,fill=mycolor45,draw=black,line width=0.1pt},
forget plot
]
coordinates{
 (13,0.785657370517928) 
};
\addplot [
color=blue,
mark size=0.9pt,
only marks,
mark=*,
mark options={solid,fill=mycolor39,draw=black,line width=0.1pt},
forget plot
]
coordinates{
 (15,0.786567164179104) 
};
\addplot [
color=blue,
mark size=0.9pt,
only marks,
mark=*,
mark options={solid,fill=mycolor39,draw=black,line width=0.1pt},
forget plot
]
coordinates{
 (14,0.611218568665377) 
};
\addplot [
color=blue,
mark size=0.9pt,
only marks,
mark=*,
mark options={solid,fill=mycolor46,draw=black,line width=0.1pt},
forget plot
]
coordinates{
 (12,0.752857142857143) 
};
\addplot [
color=blue,
mark size=0.9pt,
only marks,
mark=*,
mark options={solid,fill=mycolor46,draw=black,line width=0.1pt},
forget plot
]
coordinates{
 (18,0.756602426837973) 
};
\addplot [
color=blue,
mark size=0.9pt,
only marks,
mark=*,
mark options={solid,fill=mycolor46,draw=black,line width=0.1pt},
forget plot
]
coordinates{
 (16,0.697961704756022) 
};
\addplot [
color=blue,
mark size=0.9pt,
only marks,
mark=*,
mark options={solid,fill=mycolor47,draw=black,line width=0.1pt},
forget plot
]
coordinates{
 (14,0.802884615384615) 
};
\addplot [
color=blue,
mark size=0.9pt,
only marks,
mark=*,
mark options={solid,fill=mycolor47,draw=black,line width=0.1pt},
forget plot
]
coordinates{
 (15,0.765306122448979) 
};
\addplot [
color=blue,
mark size=0.9pt,
only marks,
mark=*,
mark options={solid,fill=mycolor48,draw=black,line width=0.1pt},
forget plot
]
coordinates{
 (17,0.749058025621703) 
};
\addplot [
color=blue,
mark size=0.9pt,
only marks,
mark=*,
mark options={solid,fill=mycolor48,draw=black,line width=0.1pt},
forget plot
]
coordinates{
 (19,0.79053583855254) 
};
\addplot [
color=blue,
mark size=0.9pt,
only marks,
mark=*,
mark options={solid,fill=mycolor40,draw=black,line width=0.1pt},
forget plot
]
coordinates{
 (11,0.798771121351766) 
};
\addplot [
color=blue,
mark size=0.9pt,
only marks,
mark=*,
mark options={solid,fill=mycolor40,draw=black,line width=0.1pt},
forget plot
]
coordinates{
 (16,0.800316957210777) 
};
\addplot [
color=blue,
mark size=0.9pt,
only marks,
mark=*,
mark options={solid,fill=mycolor40,draw=black,line width=0.1pt},
forget plot
]
coordinates{
 (8,0.643799472295514) 
};
\addplot [
color=blue,
mark size=0.9pt,
only marks,
mark=*,
mark options={solid,fill=mycolor49,draw=black,line width=0.1pt},
forget plot
]
coordinates{
 (14,0.763369963369963) 
};
\addplot [
color=blue,
mark size=0.9pt,
only marks,
mark=*,
mark options={solid,fill=mycolor49,draw=black,line width=0.1pt},
forget plot
]
coordinates{
 (11,0.738001573564123) 
};
\addplot [
color=blue,
mark size=0.9pt,
only marks,
mark=*,
mark options={solid,fill=mycolor50,draw=black,line width=0.1pt},
forget plot
]
coordinates{
 (4,0) 
};
\addplot [
color=blue,
mark size=0.9pt,
only marks,
mark=*,
mark options={solid,fill=red!68!black,draw=black,line width=0.1pt},
forget plot
]
coordinates{
 (302,0.935897435897436) 
};
\addplot [
color=blue,
mark size=0.9pt,
only marks,
mark=*,
mark options={solid,fill=red!68!black,draw=black,line width=0.1pt},
forget plot
]
coordinates{
 (262,0.930594900849858) 
};
\addplot [
color=blue,
mark size=0.9pt,
only marks,
mark=*,
mark options={solid,fill=red!72!black,draw=black,line width=0.1pt},
forget plot
]
coordinates{
 (373,0.958133150308854) 
};
\addplot [
color=blue,
mark size=0.9pt,
only marks,
mark=*,
mark options={solid,fill=red!72!black,draw=black,line width=0.1pt},
forget plot
]
coordinates{
 (383,0.970954356846473) 
};
\addplot [
color=blue,
mark size=0.9pt,
only marks,
mark=*,
mark options={solid,fill=red!72!black,draw=black,line width=0.1pt},
forget plot
]
coordinates{
 (309,0.916850625459897) 
};
\addplot [
color=blue,
mark size=0.9pt,
only marks,
mark=*,
mark options={solid,fill=red!72!black,draw=black,line width=0.1pt},
forget plot
]
coordinates{
 (283,0.943396226415094) 
};
\addplot [
color=blue,
mark size=0.9pt,
only marks,
mark=*,
mark options={solid,fill=red!76!black,draw=black,line width=0.1pt},
forget plot
]
coordinates{
 (323,0.95676429567643) 
};
\addplot [
color=blue,
mark size=0.9pt,
only marks,
mark=*,
mark options={solid,fill=red!76!black,draw=black,line width=0.1pt},
forget plot
]
coordinates{
 (254,0.932107496463932) 
};
\addplot [
color=blue,
mark size=0.9pt,
only marks,
mark=*,
mark options={solid,fill=red!76!black,draw=black,line width=0.1pt},
forget plot
]
coordinates{
 (299,0.959112959112959) 
};
\addplot [
color=blue,
mark size=0.9pt,
only marks,
mark=*,
mark options={solid,fill=red!80!black,draw=black,line width=0.1pt},
forget plot
]
coordinates{
 (308,0.961379310344827) 
};
\addplot [
color=blue,
mark size=0.9pt,
only marks,
mark=*,
mark options={solid,fill=red!80!black,draw=black,line width=0.1pt},
forget plot
]
coordinates{
 (248,0.938571428571429) 
};
\addplot [
color=blue,
mark size=0.9pt,
only marks,
mark=*,
mark options={solid,fill=red!80!black,draw=black,line width=0.1pt},
forget plot
]
coordinates{
 (286,0.963215258855586) 
};
\addplot [
color=blue,
mark size=0.9pt,
only marks,
mark=*,
mark options={solid,fill=red!84!black,draw=black,line width=0.1pt},
forget plot
]
coordinates{
 (303,0.969529085872576) 
};
\addplot [
color=blue,
mark size=0.9pt,
only marks,
mark=*,
mark options={solid,fill=red!84!black,draw=black,line width=0.1pt},
forget plot
]
coordinates{
 (295,0.954102920723227) 
};
\addplot [
color=blue,
mark size=0.9pt,
only marks,
mark=*,
mark options={solid,fill=red!88!black,draw=black,line width=0.1pt},
forget plot
]
coordinates{
 (207,0.884498480243161) 
};
\addplot [
color=blue,
mark size=0.9pt,
only marks,
mark=*,
mark options={solid,fill=red!88!black,draw=black,line width=0.1pt},
forget plot
]
coordinates{
 (235,0.92867332382311) 
};
\addplot [
color=blue,
mark size=0.9pt,
only marks,
mark=*,
mark options={solid,fill=red!92!black,draw=black,line width=0.1pt},
forget plot
]
coordinates{
 (221,0.906020558002937) 
};
\addplot [
color=blue,
mark size=0.9pt,
only marks,
mark=*,
mark options={solid,fill=red!92!black,draw=black,line width=0.1pt},
forget plot
]
coordinates{
 (243,0.952249134948097) 
};
\addplot [
color=blue,
mark size=0.9pt,
only marks,
mark=*,
mark options={solid,fill=red!96!black,draw=black,line width=0.1pt},
forget plot
]
coordinates{
 (235,0.949035812672176) 
};
\addplot [
color=blue,
mark size=0.9pt,
only marks,
mark=*,
mark options={solid,fill=red!96!black,draw=black,line width=0.1pt},
forget plot
]
coordinates{
 (236,0.954102920723227) 
};
\addplot [
color=blue,
mark size=0.9pt,
only marks,
mark=*,
mark options={solid,fill=red,draw=black,line width=0.1pt},
forget plot
]
coordinates{
 (183,0.924608819345661) 
};
\addplot [
color=blue,
mark size=0.9pt,
only marks,
mark=*,
mark options={solid,fill=red,draw=black,line width=0.1pt},
forget plot
]
coordinates{
 (198,0.946191474493361) 
};
\addplot [
color=blue,
mark size=0.9pt,
only marks,
mark=*,
mark options={solid,fill=mycolor1,draw=black,line width=0.1pt},
forget plot
]
coordinates{
 (206,0.939118264520644) 
};
\addplot [
color=blue,
mark size=0.9pt,
only marks,
mark=*,
mark options={solid,fill=mycolor2,draw=black,line width=0.1pt},
forget plot
]
coordinates{
 (163,0.92264017033357) 
};
\addplot [
color=blue,
mark size=0.9pt,
only marks,
mark=*,
mark options={solid,fill=mycolor3,draw=black,line width=0.1pt},
forget plot
]
coordinates{
 (171,0.922970159611381) 
};
\addplot [
color=blue,
mark size=0.9pt,
only marks,
mark=*,
mark options={solid,fill=mycolor3,draw=black,line width=0.1pt},
forget plot
]
coordinates{
 (137,0.887397464578673) 
};
\addplot [
color=blue,
mark size=0.9pt,
only marks,
mark=*,
mark options={solid,fill=mycolor4,draw=black,line width=0.1pt},
forget plot
]
coordinates{
 (145,0.943262411347518) 
};
\addplot [
color=blue,
mark size=0.9pt,
only marks,
mark=*,
mark options={solid,fill=mycolor5,draw=black,line width=0.1pt},
forget plot
]
coordinates{
 (144,0.926058865757358) 
};
\addplot [
color=blue,
mark size=0.9pt,
only marks,
mark=*,
mark options={solid,fill=mycolor6,draw=black,line width=0.1pt},
forget plot
]
coordinates{
 (136,0.919212491513917) 
};
\addplot [
color=blue,
mark size=0.9pt,
only marks,
mark=*,
mark options={solid,fill=mycolor7,draw=black,line width=0.1pt},
forget plot
]
coordinates{
 (109,0.902015288394718) 
};
\addplot [
color=blue,
mark size=0.9pt,
only marks,
mark=*,
mark options={solid,fill=mycolor8,draw=black,line width=0.1pt},
forget plot
]
coordinates{
 (102,0.857566765578635) 
};
\addplot [
color=blue,
mark size=0.9pt,
only marks,
mark=*,
mark options={solid,fill=mycolor9,draw=black,line width=0.1pt},
forget plot
]
coordinates{
 (113,0.949080622347949) 
};
\addplot [
color=blue,
mark size=0.9pt,
only marks,
mark=*,
mark options={solid,fill=mycolor10,draw=black,line width=0.1pt},
forget plot
]
coordinates{
 (63,0.831913245546088) 
};
\addplot [
color=blue,
mark size=0.9pt,
only marks,
mark=*,
mark options={solid,fill=mycolor11,draw=black,line width=0.1pt},
forget plot
]
coordinates{
 (79,0.894183601962158) 
};
\addplot [
color=blue,
mark size=0.9pt,
only marks,
mark=*,
mark options={solid,fill=mycolor12,draw=black,line width=0.1pt},
forget plot
]
coordinates{
 (76,0.891061452513967) 
};
\addplot [
color=blue,
mark size=0.9pt,
only marks,
mark=*,
mark options={solid,fill=mycolor13,draw=black,line width=0.1pt},
forget plot
]
coordinates{
 (64,0.84299674267101) 
};
\addplot [
color=blue,
mark size=0.9pt,
only marks,
mark=*,
mark options={solid,fill=mycolor15,draw=black,line width=0.1pt},
forget plot
]
coordinates{
 (64,0.841710427606902) 
};
\addplot [
color=blue,
mark size=0.9pt,
only marks,
mark=*,
mark options={solid,fill=mycolor51,draw=black,line width=0.1pt},
forget plot
]
coordinates{
 (62,0.873280231716148) 
};
\addplot [
color=blue,
mark size=0.9pt,
only marks,
mark=*,
mark options={solid,fill=mycolor17,draw=black,line width=0.1pt},
forget plot
]
coordinates{
 (63,0.874743326488706) 
};
\addplot [
color=blue,
mark size=0.9pt,
only marks,
mark=*,
mark options={solid,fill=mycolor18,draw=black,line width=0.1pt},
forget plot
]
coordinates{
 (56,0.830838323353293) 
};
\addplot [
color=blue,
mark size=0.9pt,
only marks,
mark=*,
mark options={solid,fill=mycolor19,draw=black,line width=0.1pt},
forget plot
]
coordinates{
 (55,0.822840409956076) 
};
\addplot [
color=blue,
mark size=0.9pt,
only marks,
mark=*,
mark options={solid,fill=mycolor20,draw=black,line width=0.1pt},
forget plot
]
coordinates{
 (47,0.857571214392803) 
};
\addplot [
color=blue,
mark size=0.9pt,
only marks,
mark=*,
mark options={solid,fill=mycolor21,draw=black,line width=0.1pt},
forget plot
]
coordinates{
 (39,0.821165438713999) 
};
\addplot [
color=blue,
mark size=0.9pt,
only marks,
mark=*,
mark options={solid,fill=mycolor52,draw=black,line width=0.1pt},
forget plot
]
coordinates{
 (42,0.843235504652827) 
};
\addplot [
color=blue,
mark size=0.9pt,
only marks,
mark=*,
mark options={solid,fill=mycolor53,draw=black,line width=0.1pt},
forget plot
]
coordinates{
 (37,0.838805970149254) 
};
\addplot [
color=blue,
mark size=0.9pt,
only marks,
mark=*,
mark options={solid,fill=mycolor54,draw=black,line width=0.1pt},
forget plot
]
coordinates{
 (35,0.804444444444444) 
};
\addplot [
color=blue,
mark size=0.9pt,
only marks,
mark=*,
mark options={solid,fill=mycolor55,draw=black,line width=0.1pt},
forget plot
]
coordinates{
 (35,0.798737174427782) 
};
\addplot [
color=blue,
mark size=0.9pt,
only marks,
mark=*,
mark options={solid,fill=mycolor27,draw=black,line width=0.1pt},
forget plot
]
coordinates{
 (28,0.807782101167315) 
};
\addplot [
color=blue,
mark size=0.9pt,
only marks,
mark=*,
mark options={solid,fill=mycolor28,draw=black,line width=0.1pt},
forget plot
]
coordinates{
 (27,0.747307373653687) 
};
\addplot [
color=blue,
mark size=0.9pt,
only marks,
mark=*,
mark options={solid,fill=red!68!black,draw=black,line width=0.1pt},
forget plot
]
coordinates{
 (329,0.950980392156863) 
};
\addplot [
color=blue,
mark size=0.9pt,
only marks,
mark=*,
mark options={solid,fill=red!68!black,draw=black,line width=0.1pt},
forget plot
]
coordinates{
 (349,0.960553633217993) 
};
\addplot [
color=blue,
mark size=0.9pt,
only marks,
mark=*,
mark options={solid,fill=red!68!black,draw=black,line width=0.1pt},
forget plot
]
coordinates{
 (338,0.949057920446615) 
};
\addplot [
color=blue,
mark size=0.9pt,
only marks,
mark=*,
mark options={solid,fill=red!72!black,draw=black,line width=0.1pt},
forget plot
]
coordinates{
 (262,0.934782608695652) 
};
\addplot [
color=blue,
mark size=0.9pt,
only marks,
mark=*,
mark options={solid,fill=red!72!black,draw=black,line width=0.1pt},
forget plot
]
coordinates{
 (279,0.935028248587571) 
};
\addplot [
color=blue,
mark size=0.9pt,
only marks,
mark=*,
mark options={solid,fill=red!72!black,draw=black,line width=0.1pt},
forget plot
]
coordinates{
 (347,0.958333333333333) 
};
\addplot [
color=blue,
mark size=0.9pt,
only marks,
mark=*,
mark options={solid,fill=red!72!black,draw=black,line width=0.1pt},
forget plot
]
coordinates{
 (357,0.970588235294118) 
};
\addplot [
color=blue,
mark size=0.9pt,
only marks,
mark=*,
mark options={solid,fill=red!76!black,draw=black,line width=0.1pt},
forget plot
]
coordinates{
 (275,0.945280437756498) 
};
\addplot [
color=blue,
mark size=0.9pt,
only marks,
mark=*,
mark options={solid,fill=red!76!black,draw=black,line width=0.1pt},
forget plot
]
coordinates{
 (214,0.8763197586727) 
};
\addplot [
color=blue,
mark size=0.9pt,
only marks,
mark=*,
mark options={solid,fill=red!76!black,draw=black,line width=0.1pt},
forget plot
]
coordinates{
 (236,0.935553946415641) 
};
\addplot [
color=blue,
mark size=0.9pt,
only marks,
mark=*,
mark options={solid,fill=red!80!black,draw=black,line width=0.1pt},
forget plot
]
coordinates{
 (292,0.954577218728162) 
};
\addplot [
color=blue,
mark size=0.9pt,
only marks,
mark=*,
mark options={solid,fill=red!80!black,draw=black,line width=0.1pt},
forget plot
]
coordinates{
 (329,0.971468336812804) 
};
\addplot [
color=blue,
mark size=0.9pt,
only marks,
mark=*,
mark options={solid,fill=red!80!black,draw=black,line width=0.1pt},
forget plot
]
coordinates{
 (255,0.946991404011461) 
};
\addplot [
color=blue,
mark size=0.9pt,
only marks,
mark=*,
mark options={solid,fill=red!84!black,draw=black,line width=0.1pt},
forget plot
]
coordinates{
 (281,0.959889349930844) 
};
\addplot [
color=blue,
mark size=0.9pt,
only marks,
mark=*,
mark options={solid,fill=red!84!black,draw=black,line width=0.1pt},
forget plot
]
coordinates{
 (274,0.940308517773306) 
};
\addplot [
color=blue,
mark size=0.9pt,
only marks,
mark=*,
mark options={solid,fill=red!88!black,draw=black,line width=0.1pt},
forget plot
]
coordinates{
 (188,0.919793966151582) 
};
\addplot [
color=blue,
mark size=0.9pt,
only marks,
mark=*,
mark options={solid,fill=red!88!black,draw=black,line width=0.1pt},
forget plot
]
coordinates{
 (228,0.918918918918919) 
};
\addplot [
color=blue,
mark size=0.9pt,
only marks,
mark=*,
mark options={solid,fill=red!92!black,draw=black,line width=0.1pt},
forget plot
]
coordinates{
 (237,0.94468085106383) 
};
\addplot [
color=blue,
mark size=0.9pt,
only marks,
mark=*,
mark options={solid,fill=red!92!black,draw=black,line width=0.1pt},
forget plot
]
coordinates{
 (235,0.945195729537366) 
};
\addplot [
color=blue,
mark size=0.9pt,
only marks,
mark=*,
mark options={solid,fill=red!96!black,draw=black,line width=0.1pt},
forget plot
]
coordinates{
 (209,0.948033707865168) 
};
\addplot [
color=blue,
mark size=0.9pt,
only marks,
mark=*,
mark options={solid,fill=red!96!black,draw=black,line width=0.1pt},
forget plot
]
coordinates{
 (241,0.954929577464789) 
};
\addplot [
color=blue,
mark size=0.9pt,
only marks,
mark=*,
mark options={solid,fill=red,draw=black,line width=0.1pt},
forget plot
]
coordinates{
 (174,0.922068463219228) 
};
\addplot [
color=blue,
mark size=0.9pt,
only marks,
mark=*,
mark options={solid,fill=red,draw=black,line width=0.1pt},
forget plot
]
coordinates{
 (172,0.924198250728863) 
};
\addplot [
color=blue,
mark size=0.9pt,
only marks,
mark=*,
mark options={solid,fill=mycolor1,draw=black,line width=0.1pt},
forget plot
]
coordinates{
 (180,0.925435540069686) 
};
\addplot [
color=blue,
mark size=0.9pt,
only marks,
mark=*,
mark options={solid,fill=mycolor2,draw=black,line width=0.1pt},
forget plot
]
coordinates{
 (175,0.954293628808864) 
};
\addplot [
color=blue,
mark size=0.9pt,
only marks,
mark=*,
mark options={solid,fill=mycolor3,draw=black,line width=0.1pt},
forget plot
]
coordinates{
 (159,0.929824561403509) 
};
\addplot [
color=blue,
mark size=0.9pt,
only marks,
mark=*,
mark options={solid,fill=mycolor3,draw=black,line width=0.1pt},
forget plot
]
coordinates{
 (136,0.870552611657835) 
};
\addplot [
color=blue,
mark size=0.9pt,
only marks,
mark=*,
mark options={solid,fill=mycolor4,draw=black,line width=0.1pt},
forget plot
]
coordinates{
 (121,0.87175572519084) 
};
\addplot [
color=blue,
mark size=0.9pt,
only marks,
mark=*,
mark options={solid,fill=mycolor5,draw=black,line width=0.1pt},
forget plot
]
coordinates{
 (153,0.92494639027877) 
};
\addplot [
color=blue,
mark size=0.9pt,
only marks,
mark=*,
mark options={solid,fill=mycolor6,draw=black,line width=0.1pt},
forget plot
]
coordinates{
 (112,0.82992125984252) 
};
\addplot [
color=blue,
mark size=0.9pt,
only marks,
mark=*,
mark options={solid,fill=mycolor7,draw=black,line width=0.1pt},
forget plot
]
coordinates{
 (121,0.915351506456241) 
};
\addplot [
color=blue,
mark size=0.9pt,
only marks,
mark=*,
mark options={solid,fill=mycolor8,draw=black,line width=0.1pt},
forget plot
]
coordinates{
 (133,0.922764227642276) 
};
\addplot [
color=blue,
mark size=0.9pt,
only marks,
mark=*,
mark options={solid,fill=mycolor9,draw=black,line width=0.1pt},
forget plot
]
coordinates{
 (96,0.917547568710359) 
};
\addplot [
color=blue,
mark size=0.9pt,
only marks,
mark=*,
mark options={solid,fill=mycolor10,draw=black,line width=0.1pt},
forget plot
]
coordinates{
 (91,0.894508670520231) 
};
\addplot [
color=blue,
mark size=0.9pt,
only marks,
mark=*,
mark options={solid,fill=mycolor11,draw=black,line width=0.1pt},
forget plot
]
coordinates{
 (78,0.854802680565897) 
};
\addplot [
color=blue,
mark size=0.9pt,
only marks,
mark=*,
mark options={solid,fill=mycolor12,draw=black,line width=0.1pt},
forget plot
]
coordinates{
 (80,0.914862914862915) 
};
\addplot [
color=blue,
mark size=0.9pt,
only marks,
mark=*,
mark options={solid,fill=mycolor13,draw=black,line width=0.1pt},
forget plot
]
coordinates{
 (62,0.863204096561814) 
};
\addplot [
color=blue,
mark size=0.9pt,
only marks,
mark=*,
mark options={solid,fill=mycolor15,draw=black,line width=0.1pt},
forget plot
]
coordinates{
 (64,0.893995552260934) 
};
\addplot [
color=blue,
mark size=0.9pt,
only marks,
mark=*,
mark options={solid,fill=mycolor51,draw=black,line width=0.1pt},
forget plot
]
coordinates{
 (58,0.850691915513474) 
};
\addplot [
color=blue,
mark size=0.9pt,
only marks,
mark=*,
mark options={solid,fill=mycolor17,draw=black,line width=0.1pt},
forget plot
]
coordinates{
 (60,0.888888888888889) 
};
\addplot [
color=blue,
mark size=0.9pt,
only marks,
mark=*,
mark options={solid,fill=mycolor18,draw=black,line width=0.1pt},
forget plot
]
coordinates{
 (56,0.849315068493151) 
};
\addplot [
color=blue,
mark size=0.9pt,
only marks,
mark=*,
mark options={solid,fill=mycolor19,draw=black,line width=0.1pt},
forget plot
]
coordinates{
 (50,0.825019794140934) 
};
\addplot [
color=blue,
mark size=0.9pt,
only marks,
mark=*,
mark options={solid,fill=mycolor20,draw=black,line width=0.1pt},
forget plot
]
coordinates{
 (48,0.848845867460908) 
};
\addplot [
color=blue,
mark size=0.9pt,
only marks,
mark=*,
mark options={solid,fill=mycolor21,draw=black,line width=0.1pt},
forget plot
]
coordinates{
 (40,0.77502001601281) 
};
\addplot [
color=blue,
mark size=0.9pt,
only marks,
mark=*,
mark options={solid,fill=mycolor52,draw=black,line width=0.1pt},
forget plot
]
coordinates{
 (47,0.830788804071247) 
};
\addplot [
color=blue,
mark size=0.9pt,
only marks,
mark=*,
mark options={solid,fill=mycolor53,draw=black,line width=0.1pt},
forget plot
]
coordinates{
 (37,0.784988272087568) 
};
\addplot [
color=blue,
mark size=0.9pt,
only marks,
mark=*,
mark options={solid,fill=mycolor54,draw=black,line width=0.1pt},
forget plot
]
coordinates{
 (37,0.819858156028369) 
};
\addplot [
color=blue,
mark size=0.9pt,
only marks,
mark=*,
mark options={solid,fill=mycolor55,draw=black,line width=0.1pt},
forget plot
]
coordinates{
 (33,0.822373393801965) 
};
\addplot [
color=blue,
mark size=0.9pt,
only marks,
mark=*,
mark options={solid,fill=mycolor27,draw=black,line width=0.1pt},
forget plot
]
coordinates{
 (33,0.832572298325723) 
};
\addplot [
color=blue,
mark size=0.9pt,
only marks,
mark=*,
mark options={solid,fill=mycolor28,draw=black,line width=0.1pt},
forget plot
]
coordinates{
 (32,0.823693379790941) 
};
\addplot [
color=blue,
mark size=0.9pt,
only marks,
mark=*,
mark options={solid,fill=red!68!black,draw=black,line width=0.1pt},
forget plot
]
coordinates{
 (360,0.958654519971969) 
};
\addplot [
color=blue,
mark size=0.9pt,
only marks,
mark=*,
mark options={solid,fill=red!68!black,draw=black,line width=0.1pt},
forget plot
]
coordinates{
 (355,0.953405017921147) 
};
\addplot [
color=blue,
mark size=0.9pt,
only marks,
mark=*,
mark options={solid,fill=red!68!black,draw=black,line width=0.1pt},
forget plot
]
coordinates{
 (357,0.962025316455696) 
};
\addplot [
color=blue,
mark size=0.9pt,
only marks,
mark=*,
mark options={solid,fill=red!72!black,draw=black,line width=0.1pt},
forget plot
]
coordinates{
 (289,0.934497816593886) 
};
\addplot [
color=blue,
mark size=0.9pt,
only marks,
mark=*,
mark options={solid,fill=red!72!black,draw=black,line width=0.1pt},
forget plot
]
coordinates{
 (269,0.86910197869102) 
};
\addplot [
color=blue,
mark size=0.9pt,
only marks,
mark=*,
mark options={solid,fill=red!72!black,draw=black,line width=0.1pt},
forget plot
]
coordinates{
 (317,0.948611111111111) 
};
\addplot [
color=blue,
mark size=0.9pt,
only marks,
mark=*,
mark options={solid,fill=red!72!black,draw=black,line width=0.1pt},
forget plot
]
coordinates{
 (328,0.953910614525139) 
};
\addplot [
color=blue,
mark size=0.9pt,
only marks,
mark=*,
mark options={solid,fill=red!76!black,draw=black,line width=0.1pt},
forget plot
]
coordinates{
 (259,0.875384615384615) 
};
\addplot [
color=blue,
mark size=0.9pt,
only marks,
mark=*,
mark options={solid,fill=red!76!black,draw=black,line width=0.1pt},
forget plot
]
coordinates{
 (292,0.945583038869258) 
};
\addplot [
color=blue,
mark size=0.9pt,
only marks,
mark=*,
mark options={solid,fill=red!76!black,draw=black,line width=0.1pt},
forget plot
]
coordinates{
 (300,0.950175438596491) 
};
\addplot [
color=blue,
mark size=0.9pt,
only marks,
mark=*,
mark options={solid,fill=red!80!black,draw=black,line width=0.1pt},
forget plot
]
coordinates{
 (278,0.948336871903751) 
};
\addplot [
color=blue,
mark size=0.9pt,
only marks,
mark=*,
mark options={solid,fill=red!80!black,draw=black,line width=0.1pt},
forget plot
]
coordinates{
 (258,0.932489451476793) 
};
\addplot [
color=blue,
mark size=0.9pt,
only marks,
mark=*,
mark options={solid,fill=red!80!black,draw=black,line width=0.1pt},
forget plot
]
coordinates{
 (328,0.957805907172996) 
};
\addplot [
color=blue,
mark size=0.9pt,
only marks,
mark=*,
mark options={solid,fill=red!84!black,draw=black,line width=0.1pt},
forget plot
]
coordinates{
 (300,0.963493199713672) 
};
\addplot [
color=blue,
mark size=0.9pt,
only marks,
mark=*,
mark options={solid,fill=red!84!black,draw=black,line width=0.1pt},
forget plot
]
coordinates{
 (318,0.964210526315789) 
};
\addplot [
color=blue,
mark size=0.9pt,
only marks,
mark=*,
mark options={solid,fill=red!88!black,draw=black,line width=0.1pt},
forget plot
]
coordinates{
 (238,0.927641099855282) 
};
\addplot [
color=blue,
mark size=0.9pt,
only marks,
mark=*,
mark options={solid,fill=red!88!black,draw=black,line width=0.1pt},
forget plot
]
coordinates{
 (287,0.956036287508723) 
};
\addplot [
color=blue,
mark size=0.9pt,
only marks,
mark=*,
mark options={solid,fill=red!92!black,draw=black,line width=0.1pt},
forget plot
]
coordinates{
 (222,0.921511627906977) 
};
\addplot [
color=blue,
mark size=0.9pt,
only marks,
mark=*,
mark options={solid,fill=red!92!black,draw=black,line width=0.1pt},
forget plot
]
coordinates{
 (211,0.925764192139738) 
};
\addplot [
color=blue,
mark size=0.9pt,
only marks,
mark=*,
mark options={solid,fill=red!96!black,draw=black,line width=0.1pt},
forget plot
]
coordinates{
 (257,0.953323903818953) 
};
\addplot [
color=blue,
mark size=0.9pt,
only marks,
mark=*,
mark options={solid,fill=red!96!black,draw=black,line width=0.1pt},
forget plot
]
coordinates{
 (221,0.939265536723164) 
};
\addplot [
color=blue,
mark size=0.9pt,
only marks,
mark=*,
mark options={solid,fill=red,draw=black,line width=0.1pt},
forget plot
]
coordinates{
 (215,0.927234927234927) 
};
\addplot [
color=blue,
mark size=0.9pt,
only marks,
mark=*,
mark options={solid,fill=red,draw=black,line width=0.1pt},
forget plot
]
coordinates{
 (176,0.914705882352941) 
};
\addplot [
color=blue,
mark size=0.9pt,
only marks,
mark=*,
mark options={solid,fill=mycolor1,draw=black,line width=0.1pt},
forget plot
]
coordinates{
 (203,0.947368421052632) 
};
\addplot [
color=blue,
mark size=0.9pt,
only marks,
mark=*,
mark options={solid,fill=mycolor2,draw=black,line width=0.1pt},
forget plot
]
coordinates{
 (158,0.920245398773006) 
};
\addplot [
color=blue,
mark size=0.9pt,
only marks,
mark=*,
mark options={solid,fill=mycolor3,draw=black,line width=0.1pt},
forget plot
]
coordinates{
 (148,0.911917098445596) 
};
\addplot [
color=blue,
mark size=0.9pt,
only marks,
mark=*,
mark options={solid,fill=mycolor3,draw=black,line width=0.1pt},
forget plot
]
coordinates{
 (169,0.922749822820695) 
};
\addplot [
color=blue,
mark size=0.9pt,
only marks,
mark=*,
mark options={solid,fill=mycolor4,draw=black,line width=0.1pt},
forget plot
]
coordinates{
 (151,0.90028901734104) 
};
\addplot [
color=blue,
mark size=0.9pt,
only marks,
mark=*,
mark options={solid,fill=mycolor5,draw=black,line width=0.1pt},
forget plot
]
coordinates{
 (165,0.930914166085136) 
};
\addplot [
color=blue,
mark size=0.9pt,
only marks,
mark=*,
mark options={solid,fill=mycolor6,draw=black,line width=0.1pt},
forget plot
]
coordinates{
 (136,0.914893617021276) 
};
\addplot [
color=blue,
mark size=0.9pt,
only marks,
mark=*,
mark options={solid,fill=mycolor7,draw=black,line width=0.1pt},
forget plot
]
coordinates{
 (137,0.912398921832884) 
};
\addplot [
color=blue,
mark size=0.9pt,
only marks,
mark=*,
mark options={solid,fill=mycolor8,draw=black,line width=0.1pt},
forget plot
]
coordinates{
 (101,0.922413793103448) 
};
\addplot [
color=blue,
mark size=0.9pt,
only marks,
mark=*,
mark options={solid,fill=mycolor9,draw=black,line width=0.1pt},
forget plot
]
coordinates{
 (101,0.89900426742532) 
};
\addplot [
color=blue,
mark size=0.9pt,
only marks,
mark=*,
mark options={solid,fill=mycolor10,draw=black,line width=0.1pt},
forget plot
]
coordinates{
 (100,0.926359256710254) 
};
\addplot [
color=blue,
mark size=0.9pt,
only marks,
mark=*,
mark options={solid,fill=mycolor11,draw=black,line width=0.1pt},
forget plot
]
coordinates{
 (86,0.888571428571428) 
};
\addplot [
color=blue,
mark size=0.9pt,
only marks,
mark=*,
mark options={solid,fill=mycolor12,draw=black,line width=0.1pt},
forget plot
]
coordinates{
 (73,0.86046511627907) 
};
\addplot [
color=blue,
mark size=0.9pt,
only marks,
mark=*,
mark options={solid,fill=mycolor13,draw=black,line width=0.1pt},
forget plot
]
coordinates{
 (65,0.822981366459627) 
};
\addplot [
color=blue,
mark size=0.9pt,
only marks,
mark=*,
mark options={solid,fill=mycolor15,draw=black,line width=0.1pt},
forget plot
]
coordinates{
 (78,0.875714285714286) 
};
\addplot [
color=blue,
mark size=0.9pt,
only marks,
mark=*,
mark options={solid,fill=mycolor51,draw=black,line width=0.1pt},
forget plot
]
coordinates{
 (72,0.884917798427448) 
};
\addplot [
color=blue,
mark size=0.9pt,
only marks,
mark=*,
mark options={solid,fill=mycolor17,draw=black,line width=0.1pt},
forget plot
]
coordinates{
 (61,0.909732016925247) 
};
\addplot [
color=blue,
mark size=0.9pt,
only marks,
mark=*,
mark options={solid,fill=mycolor18,draw=black,line width=0.1pt},
forget plot
]
coordinates{
 (57,0.86377245508982) 
};
\addplot [
color=blue,
mark size=0.9pt,
only marks,
mark=*,
mark options={solid,fill=mycolor19,draw=black,line width=0.1pt},
forget plot
]
coordinates{
 (54,0.893124574540504) 
};
\addplot [
color=blue,
mark size=0.9pt,
only marks,
mark=*,
mark options={solid,fill=mycolor20,draw=black,line width=0.1pt},
forget plot
]
coordinates{
 (46,0.795811518324607) 
};
\addplot [
color=blue,
mark size=0.9pt,
only marks,
mark=*,
mark options={solid,fill=mycolor21,draw=black,line width=0.1pt},
forget plot
]
coordinates{
 (43,0.866176470588235) 
};
\addplot [
color=blue,
mark size=0.9pt,
only marks,
mark=*,
mark options={solid,fill=mycolor52,draw=black,line width=0.1pt},
forget plot
]
coordinates{
 (40,0.825278810408922) 
};
\addplot [
color=blue,
mark size=0.9pt,
only marks,
mark=*,
mark options={solid,fill=mycolor53,draw=black,line width=0.1pt},
forget plot
]
coordinates{
 (43,0.839319470699433) 
};
\addplot [
color=blue,
mark size=0.9pt,
only marks,
mark=*,
mark options={solid,fill=mycolor54,draw=black,line width=0.1pt},
forget plot
]
coordinates{
 (38,0.814390842191333) 
};
\addplot [
color=blue,
mark size=0.9pt,
only marks,
mark=*,
mark options={solid,fill=mycolor55,draw=black,line width=0.1pt},
forget plot
]
coordinates{
 (37,0.791124713083397) 
};
\addplot [
color=blue,
mark size=0.9pt,
only marks,
mark=*,
mark options={solid,fill=mycolor27,draw=black,line width=0.1pt},
forget plot
]
coordinates{
 (29,0.85878962536023) 
};
\addplot [
color=blue,
mark size=0.9pt,
only marks,
mark=*,
mark options={solid,fill=mycolor28,draw=black,line width=0.1pt},
forget plot
]
coordinates{
 (32,0.811361981063365) 
};
\addplot [
color=blue,
mark size=0.9pt,
only marks,
mark=*,
mark options={solid,fill=red!68!black,draw=black,line width=0.1pt},
forget plot
]
coordinates{
 (361,0.961990324809951) 
};
\addplot [
color=blue,
mark size=0.9pt,
only marks,
mark=*,
mark options={solid,fill=red!68!black,draw=black,line width=0.1pt},
forget plot
]
coordinates{
 (400,0.974180041870202) 
};
\addplot [
color=blue,
mark size=0.9pt,
only marks,
mark=*,
mark options={solid,fill=red!68!black,draw=black,line width=0.1pt},
forget plot
]
coordinates{
 (384,0.973121984838043) 
};
\addplot [
color=blue,
mark size=0.9pt,
only marks,
mark=*,
mark options={solid,fill=red!72!black,draw=black,line width=0.1pt},
forget plot
]
coordinates{
 (356,0.973756906077348) 
};
\addplot [
color=blue,
mark size=0.9pt,
only marks,
mark=*,
mark options={solid,fill=red!72!black,draw=black,line width=0.1pt},
forget plot
]
coordinates{
 (357,0.975103734439834) 
};
\addplot [
color=blue,
mark size=0.9pt,
only marks,
mark=*,
mark options={solid,fill=red!72!black,draw=black,line width=0.1pt},
forget plot
]
coordinates{
 (338,0.962025316455696) 
};
\addplot [
color=blue,
mark size=0.9pt,
only marks,
mark=*,
mark options={solid,fill=red!72!black,draw=black,line width=0.1pt},
forget plot
]
coordinates{
 (369,0.973463687150838) 
};
\addplot [
color=blue,
mark size=0.9pt,
only marks,
mark=*,
mark options={solid,fill=red!76!black,draw=black,line width=0.1pt},
forget plot
]
coordinates{
 (234,0.890566037735849) 
};
\addplot [
color=blue,
mark size=0.9pt,
only marks,
mark=*,
mark options={solid,fill=red!76!black,draw=black,line width=0.1pt},
forget plot
]
coordinates{
 (272,0.964459930313589) 
};
\addplot [
color=blue,
mark size=0.9pt,
only marks,
mark=*,
mark options={solid,fill=red!76!black,draw=black,line width=0.1pt},
forget plot
]
coordinates{
 (340,0.969739619985925) 
};
\addplot [
color=blue,
mark size=0.9pt,
only marks,
mark=*,
mark options={solid,fill=red!80!black,draw=black,line width=0.1pt},
forget plot
]
coordinates{
 (299,0.964607911172797) 
};
\addplot [
color=blue,
mark size=0.9pt,
only marks,
mark=*,
mark options={solid,fill=red!80!black,draw=black,line width=0.1pt},
forget plot
]
coordinates{
 (297,0.970954356846473) 
};
\addplot [
color=blue,
mark size=0.9pt,
only marks,
mark=*,
mark options={solid,fill=red!80!black,draw=black,line width=0.1pt},
forget plot
]
coordinates{
 (307,0.970034843205575) 
};
\addplot [
color=blue,
mark size=0.9pt,
only marks,
mark=*,
mark options={solid,fill=red!84!black,draw=black,line width=0.1pt},
forget plot
]
coordinates{
 (291,0.966850828729282) 
};
\addplot [
color=blue,
mark size=0.9pt,
only marks,
mark=*,
mark options={solid,fill=red!84!black,draw=black,line width=0.1pt},
forget plot
]
coordinates{
 (273,0.969571230982019) 
};
\addplot [
color=blue,
mark size=0.9pt,
only marks,
mark=*,
mark options={solid,fill=red!88!black,draw=black,line width=0.1pt},
forget plot
]
coordinates{
 (286,0.97464016449623) 
};
\addplot [
color=blue,
mark size=0.9pt,
only marks,
mark=*,
mark options={solid,fill=red!88!black,draw=black,line width=0.1pt},
forget plot
]
coordinates{
 (224,0.943078004216444) 
};
\addplot [
color=blue,
mark size=0.9pt,
only marks,
mark=*,
mark options={solid,fill=red!92!black,draw=black,line width=0.1pt},
forget plot
]
coordinates{
 (264,0.967474048442907) 
};
\addplot [
color=blue,
mark size=0.9pt,
only marks,
mark=*,
mark options={solid,fill=red!92!black,draw=black,line width=0.1pt},
forget plot
]
coordinates{
 (234,0.961618981158409) 
};
\addplot [
color=blue,
mark size=0.9pt,
only marks,
mark=*,
mark options={solid,fill=red!96!black,draw=black,line width=0.1pt},
forget plot
]
coordinates{
 (237,0.957627118644068) 
};
\addplot [
color=blue,
mark size=0.9pt,
only marks,
mark=*,
mark options={solid,fill=red!96!black,draw=black,line width=0.1pt},
forget plot
]
coordinates{
 (205,0.967559943582511) 
};
\addplot [
color=blue,
mark size=0.9pt,
only marks,
mark=*,
mark options={solid,fill=red,draw=black,line width=0.1pt},
forget plot
]
coordinates{
 (212,0.934579439252336) 
};
\addplot [
color=blue,
mark size=0.9pt,
only marks,
mark=*,
mark options={solid,fill=red,draw=black,line width=0.1pt},
forget plot
]
coordinates{
 (184,0.953103448275862) 
};
\addplot [
color=blue,
mark size=0.9pt,
only marks,
mark=*,
mark options={solid,fill=mycolor1,draw=black,line width=0.1pt},
forget plot
]
coordinates{
 (188,0.936886395511921) 
};
\addplot [
color=blue,
mark size=0.9pt,
only marks,
mark=*,
mark options={solid,fill=mycolor2,draw=black,line width=0.1pt},
forget plot
]
coordinates{
 (168,0.952908587257618) 
};
\addplot [
color=blue,
mark size=0.9pt,
only marks,
mark=*,
mark options={solid,fill=mycolor3,draw=black,line width=0.1pt},
forget plot
]
coordinates{
 (126,0.870769230769231) 
};
\addplot [
color=blue,
mark size=0.9pt,
only marks,
mark=*,
mark options={solid,fill=mycolor3,draw=black,line width=0.1pt},
forget plot
]
coordinates{
 (120,0.879154078549849) 
};
\addplot [
color=blue,
mark size=0.9pt,
only marks,
mark=*,
mark options={solid,fill=mycolor4,draw=black,line width=0.1pt},
forget plot
]
coordinates{
 (147,0.952836637047163) 
};
\addplot [
color=blue,
mark size=0.9pt,
only marks,
mark=*,
mark options={solid,fill=mycolor5,draw=black,line width=0.1pt},
forget plot
]
coordinates{
 (147,0.93408929836995) 
};
\addplot [
color=blue,
mark size=0.9pt,
only marks,
mark=*,
mark options={solid,fill=mycolor6,draw=black,line width=0.1pt},
forget plot
]
coordinates{
 (130,0.930808448652586) 
};
\addplot [
color=blue,
mark size=0.9pt,
only marks,
mark=*,
mark options={solid,fill=mycolor7,draw=black,line width=0.1pt},
forget plot
]
coordinates{
 (119,0.912778904665314) 
};
\addplot [
color=blue,
mark size=0.9pt,
only marks,
mark=*,
mark options={solid,fill=mycolor8,draw=black,line width=0.1pt},
forget plot
]
coordinates{
 (99,0.91399416909621) 
};
\addplot [
color=blue,
mark size=0.9pt,
only marks,
mark=*,
mark options={solid,fill=mycolor9,draw=black,line width=0.1pt},
forget plot
]
coordinates{
 (116,0.922330097087379) 
};
\addplot [
color=blue,
mark size=0.9pt,
only marks,
mark=*,
mark options={solid,fill=mycolor10,draw=black,line width=0.1pt},
forget plot
]
coordinates{
 (87,0.871064467766117) 
};
\addplot [
color=blue,
mark size=0.9pt,
only marks,
mark=*,
mark options={solid,fill=mycolor11,draw=black,line width=0.1pt},
forget plot
]
coordinates{
 (86,0.901387874360847) 
};
\addplot [
color=blue,
mark size=0.9pt,
only marks,
mark=*,
mark options={solid,fill=mycolor12,draw=black,line width=0.1pt},
forget plot
]
coordinates{
 (85,0.88626907073509) 
};
\addplot [
color=blue,
mark size=0.9pt,
only marks,
mark=*,
mark options={solid,fill=mycolor13,draw=black,line width=0.1pt},
forget plot
]
coordinates{
 (72,0.890792291220557) 
};
\addplot [
color=blue,
mark size=0.9pt,
only marks,
mark=*,
mark options={solid,fill=mycolor15,draw=black,line width=0.1pt},
forget plot
]
coordinates{
 (63,0.868827160493827) 
};
\addplot [
color=blue,
mark size=0.9pt,
only marks,
mark=*,
mark options={solid,fill=mycolor51,draw=black,line width=0.1pt},
forget plot
]
coordinates{
 (50,0.844514601420679) 
};
\addplot [
color=blue,
mark size=0.9pt,
only marks,
mark=*,
mark options={solid,fill=mycolor17,draw=black,line width=0.1pt},
forget plot
]
coordinates{
 (53,0.843226788432268) 
};
\addplot [
color=blue,
mark size=0.9pt,
only marks,
mark=*,
mark options={solid,fill=mycolor18,draw=black,line width=0.1pt},
forget plot
]
coordinates{
 (55,0.858630952380952) 
};
\addplot [
color=blue,
mark size=0.9pt,
only marks,
mark=*,
mark options={solid,fill=mycolor19,draw=black,line width=0.1pt},
forget plot
]
coordinates{
 (57,0.869446343130553) 
};
\addplot [
color=blue,
mark size=0.9pt,
only marks,
mark=*,
mark options={solid,fill=mycolor20,draw=black,line width=0.1pt},
forget plot
]
coordinates{
 (43,0.837430610626487) 
};
\addplot [
color=blue,
mark size=0.9pt,
only marks,
mark=*,
mark options={solid,fill=mycolor21,draw=black,line width=0.1pt},
forget plot
]
coordinates{
 (45,0.853087295954578) 
};
\addplot [
color=blue,
mark size=0.9pt,
only marks,
mark=*,
mark options={solid,fill=mycolor52,draw=black,line width=0.1pt},
forget plot
]
coordinates{
 (39,0.821917808219178) 
};
\addplot [
color=blue,
mark size=0.9pt,
only marks,
mark=*,
mark options={solid,fill=mycolor53,draw=black,line width=0.1pt},
forget plot
]
coordinates{
 (33,0.802241793434748) 
};
\addplot [
color=blue,
mark size=0.9pt,
only marks,
mark=*,
mark options={solid,fill=mycolor54,draw=black,line width=0.1pt},
forget plot
]
coordinates{
 (36,0.827933765298776) 
};
\addplot [
color=blue,
mark size=0.9pt,
only marks,
mark=*,
mark options={solid,fill=mycolor55,draw=black,line width=0.1pt},
forget plot
]
coordinates{
 (34,0.842654735272185) 
};
\addplot [
color=blue,
mark size=0.9pt,
only marks,
mark=*,
mark options={solid,fill=mycolor27,draw=black,line width=0.1pt},
forget plot
]
coordinates{
 (34,0.816577129700691) 
};
\addplot [
color=blue,
mark size=0.9pt,
only marks,
mark=*,
mark options={solid,fill=mycolor28,draw=black,line width=0.1pt},
forget plot
]
coordinates{
 (29,0.814296814296814) 
};
\addplot [
color=blue,
mark size=0.9pt,
only marks,
mark=*,
mark options={solid,fill=red!68!black,draw=black,line width=0.1pt},
forget plot
]
coordinates{
 (336,0.957431960921144) 
};
\addplot [
color=blue,
mark size=0.9pt,
only marks,
mark=*,
mark options={solid,fill=red!68!black,draw=black,line width=0.1pt},
forget plot
]
coordinates{
 (344,0.953310104529617) 
};
\addplot [
color=blue,
mark size=0.9pt,
only marks,
mark=*,
mark options={solid,fill=red!68!black,draw=black,line width=0.1pt},
forget plot
]
coordinates{
 (317,0.956645344705046) 
};
\addplot [
color=blue,
mark size=0.9pt,
only marks,
mark=*,
mark options={solid,fill=red!72!black,draw=black,line width=0.1pt},
forget plot
]
coordinates{
 (375,0.963066202090592) 
};
\addplot [
color=blue,
mark size=0.9pt,
only marks,
mark=*,
mark options={solid,fill=red!72!black,draw=black,line width=0.1pt},
forget plot
]
coordinates{
 (340,0.960283687943262) 
};
\addplot [
color=blue,
mark size=0.9pt,
only marks,
mark=*,
mark options={solid,fill=red!72!black,draw=black,line width=0.1pt},
forget plot
]
coordinates{
 (244,0.882175226586103) 
};
\addplot [
color=blue,
mark size=0.9pt,
only marks,
mark=*,
mark options={solid,fill=red!72!black,draw=black,line width=0.1pt},
forget plot
]
coordinates{
 (353,0.966480446927374) 
};
\addplot [
color=blue,
mark size=0.9pt,
only marks,
mark=*,
mark options={solid,fill=red!76!black,draw=black,line width=0.1pt},
forget plot
]
coordinates{
 (322,0.959774170783345) 
};
\addplot [
color=blue,
mark size=0.9pt,
only marks,
mark=*,
mark options={solid,fill=red!76!black,draw=black,line width=0.1pt},
forget plot
]
coordinates{
 (328,0.958017894012388) 
};
\addplot [
color=blue,
mark size=0.9pt,
only marks,
mark=*,
mark options={solid,fill=red!76!black,draw=black,line width=0.1pt},
forget plot
]
coordinates{
 (292,0.958066808813077) 
};
\addplot [
color=blue,
mark size=0.9pt,
only marks,
mark=*,
mark options={solid,fill=red!80!black,draw=black,line width=0.1pt},
forget plot
]
coordinates{
 (304,0.964838255977496) 
};
\addplot [
color=blue,
mark size=0.9pt,
only marks,
mark=*,
mark options={solid,fill=red!80!black,draw=black,line width=0.1pt},
forget plot
]
coordinates{
 (297,0.950608446671439) 
};
\addplot [
color=blue,
mark size=0.9pt,
only marks,
mark=*,
mark options={solid,fill=red!80!black,draw=black,line width=0.1pt},
forget plot
]
coordinates{
 (228,0.878012048192771) 
};
\addplot [
color=blue,
mark size=0.9pt,
only marks,
mark=*,
mark options={solid,fill=red!84!black,draw=black,line width=0.1pt},
forget plot
]
coordinates{
 (255,0.940406976744186) 
};
\addplot [
color=blue,
mark size=0.9pt,
only marks,
mark=*,
mark options={solid,fill=red!84!black,draw=black,line width=0.1pt},
forget plot
]
coordinates{
 (287,0.964985994397759) 
};
\addplot [
color=blue,
mark size=0.9pt,
only marks,
mark=*,
mark options={solid,fill=red!88!black,draw=black,line width=0.1pt},
forget plot
]
coordinates{
 (264,0.963117606123869) 
};
\addplot [
color=blue,
mark size=0.9pt,
only marks,
mark=*,
mark options={solid,fill=red!88!black,draw=black,line width=0.1pt},
forget plot
]
coordinates{
 (223,0.93915533285612) 
};
\addplot [
color=blue,
mark size=0.9pt,
only marks,
mark=*,
mark options={solid,fill=red!92!black,draw=black,line width=0.1pt},
forget plot
]
coordinates{
 (183,0.925872093023256) 
};
\addplot [
color=blue,
mark size=0.9pt,
only marks,
mark=*,
mark options={solid,fill=red!92!black,draw=black,line width=0.1pt},
forget plot
]
coordinates{
 (266,0.949856733524355) 
};
\addplot [
color=blue,
mark size=0.9pt,
only marks,
mark=*,
mark options={solid,fill=red!96!black,draw=black,line width=0.1pt},
forget plot
]
coordinates{
 (247,0.951856946354883) 
};
\addplot [
color=blue,
mark size=0.9pt,
only marks,
mark=*,
mark options={solid,fill=red!96!black,draw=black,line width=0.1pt},
forget plot
]
coordinates{
 (219,0.927075812274368) 
};
\addplot [
color=blue,
mark size=0.9pt,
only marks,
mark=*,
mark options={solid,fill=red,draw=black,line width=0.1pt},
forget plot
]
coordinates{
 (211,0.946206896551724) 
};
\addplot [
color=blue,
mark size=0.9pt,
only marks,
mark=*,
mark options={solid,fill=red,draw=black,line width=0.1pt},
forget plot
]
coordinates{
 (179,0.928312816799421) 
};
\addplot [
color=blue,
mark size=0.9pt,
only marks,
mark=*,
mark options={solid,fill=mycolor1,draw=black,line width=0.1pt},
forget plot
]
coordinates{
 (192,0.944793850454228) 
};
\addplot [
color=blue,
mark size=0.9pt,
only marks,
mark=*,
mark options={solid,fill=mycolor2,draw=black,line width=0.1pt},
forget plot
]
coordinates{
 (178,0.955096222380613) 
};
\addplot [
color=blue,
mark size=0.9pt,
only marks,
mark=*,
mark options={solid,fill=mycolor3,draw=black,line width=0.1pt},
forget plot
]
coordinates{
 (178,0.926724137931034) 
};
\addplot [
color=blue,
mark size=0.9pt,
only marks,
mark=*,
mark options={solid,fill=mycolor3,draw=black,line width=0.1pt},
forget plot
]
coordinates{
 (178,0.929618768328446) 
};
\addplot [
color=blue,
mark size=0.9pt,
only marks,
mark=*,
mark options={solid,fill=mycolor4,draw=black,line width=0.1pt},
forget plot
]
coordinates{
 (161,0.935574229691877) 
};
\addplot [
color=blue,
mark size=0.9pt,
only marks,
mark=*,
mark options={solid,fill=mycolor5,draw=black,line width=0.1pt},
forget plot
]
coordinates{
 (155,0.952249134948097) 
};
\addplot [
color=blue,
mark size=0.9pt,
only marks,
mark=*,
mark options={solid,fill=mycolor6,draw=black,line width=0.1pt},
forget plot
]
coordinates{
 (121,0.912152269399707) 
};
\addplot [
color=blue,
mark size=0.9pt,
only marks,
mark=*,
mark options={solid,fill=mycolor7,draw=black,line width=0.1pt},
forget plot
]
coordinates{
 (119,0.910791366906475) 
};
\addplot [
color=blue,
mark size=0.9pt,
only marks,
mark=*,
mark options={solid,fill=mycolor8,draw=black,line width=0.1pt},
forget plot
]
coordinates{
 (90,0.836335160532498) 
};
\addplot [
color=blue,
mark size=0.9pt,
only marks,
mark=*,
mark options={solid,fill=mycolor9,draw=black,line width=0.1pt},
forget plot
]
coordinates{
 (97,0.900293255131965) 
};
\addplot [
color=blue,
mark size=0.9pt,
only marks,
mark=*,
mark options={solid,fill=mycolor10,draw=black,line width=0.1pt},
forget plot
]
coordinates{
 (88,0.896746817538897) 
};
\addplot [
color=blue,
mark size=0.9pt,
only marks,
mark=*,
mark options={solid,fill=mycolor11,draw=black,line width=0.1pt},
forget plot
]
coordinates{
 (84,0.905011219147345) 
};
\addplot [
color=blue,
mark size=0.9pt,
only marks,
mark=*,
mark options={solid,fill=mycolor12,draw=black,line width=0.1pt},
forget plot
]
coordinates{
 (77,0.850152905198777) 
};
\addplot [
color=blue,
mark size=0.9pt,
only marks,
mark=*,
mark options={solid,fill=mycolor13,draw=black,line width=0.1pt},
forget plot
]
coordinates{
 (68,0.887943262411347) 
};
\addplot [
color=blue,
mark size=0.9pt,
only marks,
mark=*,
mark options={solid,fill=mycolor15,draw=black,line width=0.1pt},
forget plot
]
coordinates{
 (67,0.882058613295211) 
};
\addplot [
color=blue,
mark size=0.9pt,
only marks,
mark=*,
mark options={solid,fill=mycolor51,draw=black,line width=0.1pt},
forget plot
]
coordinates{
 (73,0.902404526166902) 
};
\addplot [
color=blue,
mark size=0.9pt,
only marks,
mark=*,
mark options={solid,fill=mycolor17,draw=black,line width=0.1pt},
forget plot
]
coordinates{
 (65,0.871794871794872) 
};
\addplot [
color=blue,
mark size=0.9pt,
only marks,
mark=*,
mark options={solid,fill=mycolor18,draw=black,line width=0.1pt},
forget plot
]
coordinates{
 (51,0.820109976433621) 
};
\addplot [
color=blue,
mark size=0.9pt,
only marks,
mark=*,
mark options={solid,fill=mycolor19,draw=black,line width=0.1pt},
forget plot
]
coordinates{
 (45,0.853582554517134) 
};
\addplot [
color=blue,
mark size=0.9pt,
only marks,
mark=*,
mark options={solid,fill=mycolor20,draw=black,line width=0.1pt},
forget plot
]
coordinates{
 (49,0.836781609195402) 
};
\addplot [
color=blue,
mark size=0.9pt,
only marks,
mark=*,
mark options={solid,fill=mycolor21,draw=black,line width=0.1pt},
forget plot
]
coordinates{
 (43,0.839452843772498) 
};
\addplot [
color=blue,
mark size=0.9pt,
only marks,
mark=*,
mark options={solid,fill=mycolor52,draw=black,line width=0.1pt},
forget plot
]
coordinates{
 (41,0.819314641744548) 
};
\addplot [
color=blue,
mark size=0.9pt,
only marks,
mark=*,
mark options={solid,fill=mycolor53,draw=black,line width=0.1pt},
forget plot
]
coordinates{
 (37,0.808208366219416) 
};
\addplot [
color=blue,
mark size=0.9pt,
only marks,
mark=*,
mark options={solid,fill=mycolor54,draw=black,line width=0.1pt},
forget plot
]
coordinates{
 (42,0.824727272727273) 
};
\addplot [
color=blue,
mark size=0.9pt,
only marks,
mark=*,
mark options={solid,fill=mycolor55,draw=black,line width=0.1pt},
forget plot
]
coordinates{
 (30,0.791967871485944) 
};
\addplot [
color=blue,
mark size=0.9pt,
only marks,
mark=*,
mark options={solid,fill=mycolor27,draw=black,line width=0.1pt},
forget plot
]
coordinates{
 (36,0.81650700073692) 
};
\addplot [
color=blue,
mark size=0.9pt,
only marks,
mark=*,
mark options={solid,fill=mycolor28,draw=black,line width=0.1pt},
forget plot
]
coordinates{
 (35,0.792297111416781) 
};
\addplot [
color=blue,
mark size=0.9pt,
only marks,
mark=*,
mark options={solid,fill=red!68!black,draw=black,line width=0.1pt},
forget plot
]
coordinates{
 (380,0.967428967428967) 
};
\addplot [
color=blue,
mark size=0.9pt,
only marks,
mark=*,
mark options={solid,fill=red!68!black,draw=black,line width=0.1pt},
forget plot
]
coordinates{
 (354,0.970446735395189) 
};
\addplot [
color=blue,
mark size=0.9pt,
only marks,
mark=*,
mark options={solid,fill=red!68!black,draw=black,line width=0.1pt},
forget plot
]
coordinates{
 (295,0.965948575399583) 
};
\addplot [
color=blue,
mark size=0.9pt,
only marks,
mark=*,
mark options={solid,fill=red!72!black,draw=black,line width=0.1pt},
forget plot
]
coordinates{
 (267,0.92790863668808) 
};
\addplot [
color=blue,
mark size=0.9pt,
only marks,
mark=*,
mark options={solid,fill=red!72!black,draw=black,line width=0.1pt},
forget plot
]
coordinates{
 (367,0.964754664823773) 
};
\addplot [
color=blue,
mark size=0.9pt,
only marks,
mark=*,
mark options={solid,fill=red!72!black,draw=black,line width=0.1pt},
forget plot
]
coordinates{
 (313,0.948682385575589) 
};
\addplot [
color=blue,
mark size=0.9pt,
only marks,
mark=*,
mark options={solid,fill=red!72!black,draw=black,line width=0.1pt},
forget plot
]
coordinates{
 (259,0.881608339538347) 
};
\addplot [
color=blue,
mark size=0.9pt,
only marks,
mark=*,
mark options={solid,fill=red!76!black,draw=black,line width=0.1pt},
forget plot
]
coordinates{
 (349,0.965469613259669) 
};
\addplot [
color=blue,
mark size=0.9pt,
only marks,
mark=*,
mark options={solid,fill=red!76!black,draw=black,line width=0.1pt},
forget plot
]
coordinates{
 (329,0.955570745044429) 
};
\addplot [
color=blue,
mark size=0.9pt,
only marks,
mark=*,
mark options={solid,fill=red!76!black,draw=black,line width=0.1pt},
forget plot
]
coordinates{
 (313,0.965181058495822) 
};
\addplot [
color=blue,
mark size=0.9pt,
only marks,
mark=*,
mark options={solid,fill=red!80!black,draw=black,line width=0.1pt},
forget plot
]
coordinates{
 (294,0.962131837307153) 
};
\addplot [
color=blue,
mark size=0.9pt,
only marks,
mark=*,
mark options={solid,fill=red!80!black,draw=black,line width=0.1pt},
forget plot
]
coordinates{
 (304,0.95807560137457) 
};
\addplot [
color=blue,
mark size=0.9pt,
only marks,
mark=*,
mark options={solid,fill=red!80!black,draw=black,line width=0.1pt},
forget plot
]
coordinates{
 (223,0.880418535127055) 
};
\addplot [
color=blue,
mark size=0.9pt,
only marks,
mark=*,
mark options={solid,fill=red!84!black,draw=black,line width=0.1pt},
forget plot
]
coordinates{
 (228,0.881688018085908) 
};
\addplot [
color=blue,
mark size=0.9pt,
only marks,
mark=*,
mark options={solid,fill=red!84!black,draw=black,line width=0.1pt},
forget plot
]
coordinates{
 (287,0.95807560137457) 
};
\addplot [
color=blue,
mark size=0.9pt,
only marks,
mark=*,
mark options={solid,fill=red!88!black,draw=black,line width=0.1pt},
forget plot
]
coordinates{
 (222,0.931623931623932) 
};
\addplot [
color=blue,
mark size=0.9pt,
only marks,
mark=*,
mark options={solid,fill=red!88!black,draw=black,line width=0.1pt},
forget plot
]
coordinates{
 (264,0.942934782608695) 
};
\addplot [
color=blue,
mark size=0.9pt,
only marks,
mark=*,
mark options={solid,fill=red!92!black,draw=black,line width=0.1pt},
forget plot
]
coordinates{
 (208,0.926690391459075) 
};
\addplot [
color=blue,
mark size=0.9pt,
only marks,
mark=*,
mark options={solid,fill=red!92!black,draw=black,line width=0.1pt},
forget plot
]
coordinates{
 (205,0.879761015683346) 
};
\addplot [
color=blue,
mark size=0.9pt,
only marks,
mark=*,
mark options={solid,fill=red!96!black,draw=black,line width=0.1pt},
forget plot
]
coordinates{
 (239,0.935135135135135) 
};
\addplot [
color=blue,
mark size=0.9pt,
only marks,
mark=*,
mark options={solid,fill=red!96!black,draw=black,line width=0.1pt},
forget plot
]
coordinates{
 (179,0.882572924457741) 
};
\addplot [
color=blue,
mark size=0.9pt,
only marks,
mark=*,
mark options={solid,fill=red,draw=black,line width=0.1pt},
forget plot
]
coordinates{
 (201,0.926375982844889) 
};
\addplot [
color=blue,
mark size=0.9pt,
only marks,
mark=*,
mark options={solid,fill=red,draw=black,line width=0.1pt},
forget plot
]
coordinates{
 (160,0.869179600886918) 
};
\addplot [
color=blue,
mark size=0.9pt,
only marks,
mark=*,
mark options={solid,fill=mycolor1,draw=black,line width=0.1pt},
forget plot
]
coordinates{
 (165,0.875748502994012) 
};
\addplot [
color=blue,
mark size=0.9pt,
only marks,
mark=*,
mark options={solid,fill=mycolor2,draw=black,line width=0.1pt},
forget plot
]
coordinates{
 (161,0.92972972972973) 
};
\addplot [
color=blue,
mark size=0.9pt,
only marks,
mark=*,
mark options={solid,fill=mycolor3,draw=black,line width=0.1pt},
forget plot
]
coordinates{
 (176,0.946116165150455) 
};
\addplot [
color=blue,
mark size=0.9pt,
only marks,
mark=*,
mark options={solid,fill=mycolor3,draw=black,line width=0.1pt},
forget plot
]
coordinates{
 (186,0.947079037800687) 
};
\addplot [
color=blue,
mark size=0.9pt,
only marks,
mark=*,
mark options={solid,fill=mycolor4,draw=black,line width=0.1pt},
forget plot
]
coordinates{
 (152,0.918497519489723) 
};
\addplot [
color=blue,
mark size=0.9pt,
only marks,
mark=*,
mark options={solid,fill=mycolor5,draw=black,line width=0.1pt},
forget plot
]
coordinates{
 (163,0.951523545706371) 
};
\addplot [
color=blue,
mark size=0.9pt,
only marks,
mark=*,
mark options={solid,fill=mycolor6,draw=black,line width=0.1pt},
forget plot
]
coordinates{
 (111,0.88138030194105) 
};
\addplot [
color=blue,
mark size=0.9pt,
only marks,
mark=*,
mark options={solid,fill=mycolor7,draw=black,line width=0.1pt},
forget plot
]
coordinates{
 (124,0.917142857142857) 
};
\addplot [
color=blue,
mark size=0.9pt,
only marks,
mark=*,
mark options={solid,fill=mycolor8,draw=black,line width=0.1pt},
forget plot
]
coordinates{
 (112,0.902266288951841) 
};
\addplot [
color=blue,
mark size=0.9pt,
only marks,
mark=*,
mark options={solid,fill=mycolor9,draw=black,line width=0.1pt},
forget plot
]
coordinates{
 (96,0.864253393665158) 
};
\addplot [
color=blue,
mark size=0.9pt,
only marks,
mark=*,
mark options={solid,fill=mycolor10,draw=black,line width=0.1pt},
forget plot
]
coordinates{
 (86,0.919945725915875) 
};
\addplot [
color=blue,
mark size=0.9pt,
only marks,
mark=*,
mark options={solid,fill=mycolor11,draw=black,line width=0.1pt},
forget plot
]
coordinates{
 (83,0.88569374550683) 
};
\addplot [
color=blue,
mark size=0.9pt,
only marks,
mark=*,
mark options={solid,fill=mycolor12,draw=black,line width=0.1pt},
forget plot
]
coordinates{
 (88,0.918518518518518) 
};
\addplot [
color=blue,
mark size=0.9pt,
only marks,
mark=*,
mark options={solid,fill=mycolor13,draw=black,line width=0.1pt},
forget plot
]
coordinates{
 (75,0.856512141280353) 
};
\addplot [
color=blue,
mark size=0.9pt,
only marks,
mark=*,
mark options={solid,fill=mycolor15,draw=black,line width=0.1pt},
forget plot
]
coordinates{
 (73,0.896232650363516) 
};
\addplot [
color=blue,
mark size=0.9pt,
only marks,
mark=*,
mark options={solid,fill=mycolor51,draw=black,line width=0.1pt},
forget plot
]
coordinates{
 (63,0.885359116022099) 
};
\addplot [
color=blue,
mark size=0.9pt,
only marks,
mark=*,
mark options={solid,fill=mycolor17,draw=black,line width=0.1pt},
forget plot
]
coordinates{
 (51,0.842911877394636) 
};
\addplot [
color=blue,
mark size=0.9pt,
only marks,
mark=*,
mark options={solid,fill=mycolor18,draw=black,line width=0.1pt},
forget plot
]
coordinates{
 (52,0.853658536585366) 
};
\addplot [
color=blue,
mark size=0.9pt,
only marks,
mark=*,
mark options={solid,fill=mycolor19,draw=black,line width=0.1pt},
forget plot
]
coordinates{
 (46,0.870481927710843) 
};
\addplot [
color=blue,
mark size=0.9pt,
only marks,
mark=*,
mark options={solid,fill=mycolor20,draw=black,line width=0.1pt},
forget plot
]
coordinates{
 (44,0.875636363636364) 
};
\addplot [
color=blue,
mark size=0.9pt,
only marks,
mark=*,
mark options={solid,fill=mycolor21,draw=black,line width=0.1pt},
forget plot
]
coordinates{
 (45,0.84558327714093) 
};
\addplot [
color=blue,
mark size=0.9pt,
only marks,
mark=*,
mark options={solid,fill=mycolor52,draw=black,line width=0.1pt},
forget plot
]
coordinates{
 (43,0.856948228882834) 
};
\addplot [
color=blue,
mark size=0.9pt,
only marks,
mark=*,
mark options={solid,fill=mycolor53,draw=black,line width=0.1pt},
forget plot
]
coordinates{
 (40,0.843326885880077) 
};
\addplot [
color=blue,
mark size=0.9pt,
only marks,
mark=*,
mark options={solid,fill=mycolor54,draw=black,line width=0.1pt},
forget plot
]
coordinates{
 (35,0.811102544333076) 
};
\addplot [
color=blue,
mark size=0.9pt,
only marks,
mark=*,
mark options={solid,fill=mycolor55,draw=black,line width=0.1pt},
forget plot
]
coordinates{
 (34,0.832151300236407) 
};
\addplot [
color=blue,
mark size=0.9pt,
only marks,
mark=*,
mark options={solid,fill=mycolor27,draw=black,line width=0.1pt},
forget plot
]
coordinates{
 (30,0.776772247360482) 
};
\addplot [
color=blue,
mark size=0.9pt,
only marks,
mark=*,
mark options={solid,fill=mycolor28,draw=black,line width=0.1pt},
forget plot
]
coordinates{
 (32,0.862800565770863) 
};
\addplot [
color=blue,
mark size=0.9pt,
only marks,
mark=*,
mark options={solid,fill=red!68!black,draw=black,line width=0.1pt},
forget plot
]
coordinates{
 (288,0.893129770992366) 
};
\addplot [
color=blue,
mark size=0.9pt,
only marks,
mark=*,
mark options={solid,fill=red!68!black,draw=black,line width=0.1pt},
forget plot
]
coordinates{
 (316,0.942266571632217) 
};
\addplot [
color=blue,
mark size=0.9pt,
only marks,
mark=*,
mark options={solid,fill=red!68!black,draw=black,line width=0.1pt},
forget plot
]
coordinates{
 (380,0.973163841807909) 
};
\addplot [
color=blue,
mark size=0.9pt,
only marks,
mark=*,
mark options={solid,fill=red!72!black,draw=black,line width=0.1pt},
forget plot
]
coordinates{
 (368,0.976256983240223) 
};
\addplot [
color=blue,
mark size=0.9pt,
only marks,
mark=*,
mark options={solid,fill=red!72!black,draw=black,line width=0.1pt},
forget plot
]
coordinates{
 (359,0.968332160450387) 
};
\addplot [
color=blue,
mark size=0.9pt,
only marks,
mark=*,
mark options={solid,fill=red!72!black,draw=black,line width=0.1pt},
forget plot
]
coordinates{
 (240,0.891603053435114) 
};
\addplot [
color=blue,
mark size=0.9pt,
only marks,
mark=*,
mark options={solid,fill=red!72!black,draw=black,line width=0.1pt},
forget plot
]
coordinates{
 (322,0.962290502793296) 
};
\addplot [
color=blue,
mark size=0.9pt,
only marks,
mark=*,
mark options={solid,fill=red!76!black,draw=black,line width=0.1pt},
forget plot
]
coordinates{
 (335,0.966386554621849) 
};
\addplot [
color=blue,
mark size=0.9pt,
only marks,
mark=*,
mark options={solid,fill=red!76!black,draw=black,line width=0.1pt},
forget plot
]
coordinates{
 (340,0.968287526427061) 
};
\addplot [
color=blue,
mark size=0.9pt,
only marks,
mark=*,
mark options={solid,fill=red!76!black,draw=black,line width=0.1pt},
forget plot
]
coordinates{
 (342,0.977591036414566) 
};
\addplot [
color=blue,
mark size=0.9pt,
only marks,
mark=*,
mark options={solid,fill=red!80!black,draw=black,line width=0.1pt},
forget plot
]
coordinates{
 (273,0.967247386759582) 
};
\addplot [
color=blue,
mark size=0.9pt,
only marks,
mark=*,
mark options={solid,fill=red!80!black,draw=black,line width=0.1pt},
forget plot
]
coordinates{
 (293,0.971187631763879) 
};
\addplot [
color=blue,
mark size=0.9pt,
only marks,
mark=*,
mark options={solid,fill=red!80!black,draw=black,line width=0.1pt},
forget plot
]
coordinates{
 (303,0.977746870653685) 
};
\addplot [
color=blue,
mark size=0.9pt,
only marks,
mark=*,
mark options={solid,fill=red!84!black,draw=black,line width=0.1pt},
forget plot
]
coordinates{
 (255,0.9593837535014) 
};
\addplot [
color=blue,
mark size=0.9pt,
only marks,
mark=*,
mark options={solid,fill=red!84!black,draw=black,line width=0.1pt},
forget plot
]
coordinates{
 (250,0.936344969199179) 
};
\addplot [
color=blue,
mark size=0.9pt,
only marks,
mark=*,
mark options={solid,fill=red!88!black,draw=black,line width=0.1pt},
forget plot
]
coordinates{
 (227,0.93768115942029) 
};
\addplot [
color=blue,
mark size=0.9pt,
only marks,
mark=*,
mark options={solid,fill=red!88!black,draw=black,line width=0.1pt},
forget plot
]
coordinates{
 (290,0.968011126564673) 
};
\addplot [
color=blue,
mark size=0.9pt,
only marks,
mark=*,
mark options={solid,fill=red!92!black,draw=black,line width=0.1pt},
forget plot
]
coordinates{
 (189,0.893262679788039) 
};
\addplot [
color=blue,
mark size=0.9pt,
only marks,
mark=*,
mark options={solid,fill=red!92!black,draw=black,line width=0.1pt},
forget plot
]
coordinates{
 (264,0.960111966410077) 
};
\addplot [
color=blue,
mark size=0.9pt,
only marks,
mark=*,
mark options={solid,fill=red!96!black,draw=black,line width=0.1pt},
forget plot
]
coordinates{
 (191,0.93103448275862) 
};
\addplot [
color=blue,
mark size=0.9pt,
only marks,
mark=*,
mark options={solid,fill=red!96!black,draw=black,line width=0.1pt},
forget plot
]
coordinates{
 (232,0.956276445698166) 
};
\addplot [
color=blue,
mark size=0.9pt,
only marks,
mark=*,
mark options={solid,fill=red,draw=black,line width=0.1pt},
forget plot
]
coordinates{
 (132,0.856477889837083) 
};
\addplot [
color=blue,
mark size=0.9pt,
only marks,
mark=*,
mark options={solid,fill=red,draw=black,line width=0.1pt},
forget plot
]
coordinates{
 (177,0.927745664739884) 
};
\addplot [
color=blue,
mark size=0.9pt,
only marks,
mark=*,
mark options={solid,fill=mycolor1,draw=black,line width=0.1pt},
forget plot
]
coordinates{
 (205,0.958479943701618) 
};
\addplot [
color=blue,
mark size=0.9pt,
only marks,
mark=*,
mark options={solid,fill=mycolor2,draw=black,line width=0.1pt},
forget plot
]
coordinates{
 (175,0.93768115942029) 
};
\addplot [
color=blue,
mark size=0.9pt,
only marks,
mark=*,
mark options={solid,fill=mycolor3,draw=black,line width=0.1pt},
forget plot
]
coordinates{
 (134,0.882664647993944) 
};
\addplot [
color=blue,
mark size=0.9pt,
only marks,
mark=*,
mark options={solid,fill=mycolor3,draw=black,line width=0.1pt},
forget plot
]
coordinates{
 (127,0.925872093023256) 
};
\addplot [
color=blue,
mark size=0.9pt,
only marks,
mark=*,
mark options={solid,fill=mycolor4,draw=black,line width=0.1pt},
forget plot
]
coordinates{
 (151,0.913169319826339) 
};
\addplot [
color=blue,
mark size=0.9pt,
only marks,
mark=*,
mark options={solid,fill=mycolor5,draw=black,line width=0.1pt},
forget plot
]
coordinates{
 (122,0.906724511930586) 
};
\addplot [
color=blue,
mark size=0.9pt,
only marks,
mark=*,
mark options={solid,fill=mycolor6,draw=black,line width=0.1pt},
forget plot
]
coordinates{
 (142,0.93173821252639) 
};
\addplot [
color=blue,
mark size=0.9pt,
only marks,
mark=*,
mark options={solid,fill=mycolor7,draw=black,line width=0.1pt},
forget plot
]
coordinates{
 (109,0.858461538461538) 
};
\addplot [
color=blue,
mark size=0.9pt,
only marks,
mark=*,
mark options={solid,fill=mycolor8,draw=black,line width=0.1pt},
forget plot
]
coordinates{
 (121,0.909708078750849) 
};
\addplot [
color=blue,
mark size=0.9pt,
only marks,
mark=*,
mark options={solid,fill=mycolor9,draw=black,line width=0.1pt},
forget plot
]
coordinates{
 (85,0.884241971620612) 
};
\addplot [
color=blue,
mark size=0.9pt,
only marks,
mark=*,
mark options={solid,fill=mycolor10,draw=black,line width=0.1pt},
forget plot
]
coordinates{
 (89,0.900785153461813) 
};
\addplot [
color=blue,
mark size=0.9pt,
only marks,
mark=*,
mark options={solid,fill=mycolor11,draw=black,line width=0.1pt},
forget plot
]
coordinates{
 (87,0.896451846488052) 
};
\addplot [
color=blue,
mark size=0.9pt,
only marks,
mark=*,
mark options={solid,fill=mycolor12,draw=black,line width=0.1pt},
forget plot
]
coordinates{
 (84,0.894623655913978) 
};
\addplot [
color=blue,
mark size=0.9pt,
only marks,
mark=*,
mark options={solid,fill=mycolor13,draw=black,line width=0.1pt},
forget plot
]
coordinates{
 (62,0.853435114503817) 
};
\addplot [
color=blue,
mark size=0.9pt,
only marks,
mark=*,
mark options={solid,fill=mycolor15,draw=black,line width=0.1pt},
forget plot
]
coordinates{
 (62,0.832558139534884) 
};
\addplot [
color=blue,
mark size=0.9pt,
only marks,
mark=*,
mark options={solid,fill=mycolor51,draw=black,line width=0.1pt},
forget plot
]
coordinates{
 (62,0.835886214442013) 
};
\addplot [
color=blue,
mark size=0.9pt,
only marks,
mark=*,
mark options={solid,fill=mycolor17,draw=black,line width=0.1pt},
forget plot
]
coordinates{
 (51,0.866914498141264) 
};
\addplot [
color=blue,
mark size=0.9pt,
only marks,
mark=*,
mark options={solid,fill=mycolor18,draw=black,line width=0.1pt},
forget plot
]
coordinates{
 (50,0.852281515854602) 
};
\addplot [
color=blue,
mark size=0.9pt,
only marks,
mark=*,
mark options={solid,fill=mycolor19,draw=black,line width=0.1pt},
forget plot
]
coordinates{
 (46,0.858654572940287) 
};
\addplot [
color=blue,
mark size=0.9pt,
only marks,
mark=*,
mark options={solid,fill=mycolor20,draw=black,line width=0.1pt},
forget plot
]
coordinates{
 (47,0.829411764705882) 
};
\addplot [
color=blue,
mark size=0.9pt,
only marks,
mark=*,
mark options={solid,fill=mycolor21,draw=black,line width=0.1pt},
forget plot
]
coordinates{
 (49,0.862637362637362) 
};
\addplot [
color=blue,
mark size=0.9pt,
only marks,
mark=*,
mark options={solid,fill=mycolor52,draw=black,line width=0.1pt},
forget plot
]
coordinates{
 (42,0.804453723034099) 
};
\addplot [
color=blue,
mark size=0.9pt,
only marks,
mark=*,
mark options={solid,fill=mycolor53,draw=black,line width=0.1pt},
forget plot
]
coordinates{
 (35,0.809220985691574) 
};
\addplot [
color=blue,
mark size=0.9pt,
only marks,
mark=*,
mark options={solid,fill=mycolor54,draw=black,line width=0.1pt},
forget plot
]
coordinates{
 (36,0.811361981063365) 
};
\addplot [
color=blue,
mark size=0.9pt,
only marks,
mark=*,
mark options={solid,fill=mycolor55,draw=black,line width=0.1pt},
forget plot
]
coordinates{
 (33,0.888059701492537) 
};
\addplot [
color=blue,
mark size=0.9pt,
only marks,
mark=*,
mark options={solid,fill=mycolor27,draw=black,line width=0.1pt},
forget plot
]
coordinates{
 (33,0.826830937713894) 
};
\addplot [
color=blue,
mark size=0.9pt,
only marks,
mark=*,
mark options={solid,fill=mycolor28,draw=black,line width=0.1pt},
forget plot
]
coordinates{
 (28,0.823620823620824) 
};
\addplot [
color=blue,
mark size=0.9pt,
only marks,
mark=*,
mark options={solid,fill=red!68!black,draw=black,line width=0.1pt},
forget plot
]
coordinates{
 (363,0.963380281690141) 
};
\addplot [
color=blue,
mark size=0.9pt,
only marks,
mark=*,
mark options={solid,fill=red!68!black,draw=black,line width=0.1pt},
forget plot
]
coordinates{
 (343,0.967651195499297) 
};
\addplot [
color=blue,
mark size=0.9pt,
only marks,
mark=*,
mark options={solid,fill=red!68!black,draw=black,line width=0.1pt},
forget plot
]
coordinates{
 (363,0.963276836158192) 
};
\addplot [
color=blue,
mark size=0.9pt,
only marks,
mark=*,
mark options={solid,fill=red!72!black,draw=black,line width=0.1pt},
forget plot
]
coordinates{
 (352,0.967109867039888) 
};
\addplot [
color=blue,
mark size=0.9pt,
only marks,
mark=*,
mark options={solid,fill=red!72!black,draw=black,line width=0.1pt},
forget plot
]
coordinates{
 (232,0.87044220325834) 
};
\addplot [
color=blue,
mark size=0.9pt,
only marks,
mark=*,
mark options={solid,fill=red!72!black,draw=black,line width=0.1pt},
forget plot
]
coordinates{
 (231,0.867132867132867) 
};
\addplot [
color=blue,
mark size=0.9pt,
only marks,
mark=*,
mark options={solid,fill=red!72!black,draw=black,line width=0.1pt},
forget plot
]
coordinates{
 (349,0.962702322308233) 
};
\addplot [
color=blue,
mark size=0.9pt,
only marks,
mark=*,
mark options={solid,fill=red!76!black,draw=black,line width=0.1pt},
forget plot
]
coordinates{
 (344,0.96629213483146) 
};
\addplot [
color=blue,
mark size=0.9pt,
only marks,
mark=*,
mark options={solid,fill=red!76!black,draw=black,line width=0.1pt},
forget plot
]
coordinates{
 (339,0.97752808988764) 
};
\addplot [
color=blue,
mark size=0.9pt,
only marks,
mark=*,
mark options={solid,fill=red!76!black,draw=black,line width=0.1pt},
forget plot
]
coordinates{
 (285,0.940836940836941) 
};
\addplot [
color=blue,
mark size=0.9pt,
only marks,
mark=*,
mark options={solid,fill=red!80!black,draw=black,line width=0.1pt},
forget plot
]
coordinates{
 (288,0.959887403237157) 
};
\addplot [
color=blue,
mark size=0.9pt,
only marks,
mark=*,
mark options={solid,fill=red!80!black,draw=black,line width=0.1pt},
forget plot
]
coordinates{
 (219,0.868421052631579) 
};
\addplot [
color=blue,
mark size=0.9pt,
only marks,
mark=*,
mark options={solid,fill=red!80!black,draw=black,line width=0.1pt},
forget plot
]
coordinates{
 (247,0.934687953555878) 
};
\addplot [
color=blue,
mark size=0.9pt,
only marks,
mark=*,
mark options={solid,fill=red!84!black,draw=black,line width=0.1pt},
forget plot
]
coordinates{
 (291,0.946554149085794) 
};
\addplot [
color=blue,
mark size=0.9pt,
only marks,
mark=*,
mark options={solid,fill=red!84!black,draw=black,line width=0.1pt},
forget plot
]
coordinates{
 (272,0.938775510204082) 
};
\addplot [
color=blue,
mark size=0.9pt,
only marks,
mark=*,
mark options={solid,fill=red!88!black,draw=black,line width=0.1pt},
forget plot
]
coordinates{
 (259,0.950749464668094) 
};
\addplot [
color=blue,
mark size=0.9pt,
only marks,
mark=*,
mark options={solid,fill=red!88!black,draw=black,line width=0.1pt},
forget plot
]
coordinates{
 (245,0.927641099855282) 
};
\addplot [
color=blue,
mark size=0.9pt,
only marks,
mark=*,
mark options={solid,fill=red!92!black,draw=black,line width=0.1pt},
forget plot
]
coordinates{
 (195,0.92882818116463) 
};
\addplot [
color=blue,
mark size=0.9pt,
only marks,
mark=*,
mark options={solid,fill=red!92!black,draw=black,line width=0.1pt},
forget plot
]
coordinates{
 (242,0.945351312987935) 
};
\addplot [
color=blue,
mark size=0.9pt,
only marks,
mark=*,
mark options={solid,fill=red!96!black,draw=black,line width=0.1pt},
forget plot
]
coordinates{
 (189,0.873174481168332) 
};
\addplot [
color=blue,
mark size=0.9pt,
only marks,
mark=*,
mark options={solid,fill=red!96!black,draw=black,line width=0.1pt},
forget plot
]
coordinates{
 (169,0.86910197869102) 
};
\addplot [
color=blue,
mark size=0.9pt,
only marks,
mark=*,
mark options={solid,fill=red,draw=black,line width=0.1pt},
forget plot
]
coordinates{
 (178,0.917510853835022) 
};
\addplot [
color=blue,
mark size=0.9pt,
only marks,
mark=*,
mark options={solid,fill=red,draw=black,line width=0.1pt},
forget plot
]
coordinates{
 (203,0.936139332365747) 
};
\addplot [
color=blue,
mark size=0.9pt,
only marks,
mark=*,
mark options={solid,fill=mycolor1,draw=black,line width=0.1pt},
forget plot
]
coordinates{
 (185,0.942737430167598) 
};
\addplot [
color=blue,
mark size=0.9pt,
only marks,
mark=*,
mark options={solid,fill=mycolor2,draw=black,line width=0.1pt},
forget plot
]
coordinates{
 (176,0.919308357348703) 
};
\addplot [
color=blue,
mark size=0.9pt,
only marks,
mark=*,
mark options={solid,fill=mycolor3,draw=black,line width=0.1pt},
forget plot
]
coordinates{
 (179,0.941342756183746) 
};
\addplot [
color=blue,
mark size=0.9pt,
only marks,
mark=*,
mark options={solid,fill=mycolor3,draw=black,line width=0.1pt},
forget plot
]
coordinates{
 (128,0.910384068278805) 
};
\addplot [
color=blue,
mark size=0.9pt,
only marks,
mark=*,
mark options={solid,fill=mycolor4,draw=black,line width=0.1pt},
forget plot
]
coordinates{
 (156,0.942172073342736) 
};
\addplot [
color=blue,
mark size=0.9pt,
only marks,
mark=*,
mark options={solid,fill=mycolor5,draw=black,line width=0.1pt},
forget plot
]
coordinates{
 (144,0.943632567849687) 
};
\addplot [
color=blue,
mark size=0.9pt,
only marks,
mark=*,
mark options={solid,fill=mycolor6,draw=black,line width=0.1pt},
forget plot
]
coordinates{
 (115,0.921052631578947) 
};
\addplot [
color=blue,
mark size=0.9pt,
only marks,
mark=*,
mark options={solid,fill=mycolor7,draw=black,line width=0.1pt},
forget plot
]
coordinates{
 (119,0.87832973362131) 
};
\addplot [
color=blue,
mark size=0.9pt,
only marks,
mark=*,
mark options={solid,fill=mycolor8,draw=black,line width=0.1pt},
forget plot
]
coordinates{
 (98,0.871910112359551) 
};
\addplot [
color=blue,
mark size=0.9pt,
only marks,
mark=*,
mark options={solid,fill=mycolor9,draw=black,line width=0.1pt},
forget plot
]
coordinates{
 (101,0.911485003657644) 
};
\addplot [
color=blue,
mark size=0.9pt,
only marks,
mark=*,
mark options={solid,fill=mycolor10,draw=black,line width=0.1pt},
forget plot
]
coordinates{
 (75,0.838862559241706) 
};
\addplot [
color=blue,
mark size=0.9pt,
only marks,
mark=*,
mark options={solid,fill=mycolor11,draw=black,line width=0.1pt},
forget plot
]
coordinates{
 (87,0.924285714285714) 
};
\addplot [
color=blue,
mark size=0.9pt,
only marks,
mark=*,
mark options={solid,fill=mycolor12,draw=black,line width=0.1pt},
forget plot
]
coordinates{
 (73,0.884588804422944) 
};
\addplot [
color=blue,
mark size=0.9pt,
only marks,
mark=*,
mark options={solid,fill=mycolor13,draw=black,line width=0.1pt},
forget plot
]
coordinates{
 (69,0.877037037037037) 
};
\addplot [
color=blue,
mark size=0.9pt,
only marks,
mark=*,
mark options={solid,fill=mycolor15,draw=black,line width=0.1pt},
forget plot
]
coordinates{
 (68,0.883227176220807) 
};
\addplot [
color=blue,
mark size=0.9pt,
only marks,
mark=*,
mark options={solid,fill=mycolor51,draw=black,line width=0.1pt},
forget plot
]
coordinates{
 (61,0.85439763488544) 
};
\addplot [
color=blue,
mark size=0.9pt,
only marks,
mark=*,
mark options={solid,fill=mycolor17,draw=black,line width=0.1pt},
forget plot
]
coordinates{
 (58,0.87598944591029) 
};
\addplot [
color=blue,
mark size=0.9pt,
only marks,
mark=*,
mark options={solid,fill=mycolor18,draw=black,line width=0.1pt},
forget plot
]
coordinates{
 (51,0.819466248037676) 
};
\addplot [
color=blue,
mark size=0.9pt,
only marks,
mark=*,
mark options={solid,fill=mycolor19,draw=black,line width=0.1pt},
forget plot
]
coordinates{
 (51,0.836003051106026) 
};
\addplot [
color=blue,
mark size=0.9pt,
only marks,
mark=*,
mark options={solid,fill=mycolor20,draw=black,line width=0.1pt},
forget plot
]
coordinates{
 (46,0.856711915535445) 
};
\addplot [
color=blue,
mark size=0.9pt,
only marks,
mark=*,
mark options={solid,fill=mycolor21,draw=black,line width=0.1pt},
forget plot
]
coordinates{
 (40,0.786011656952539) 
};
\addplot [
color=blue,
mark size=0.9pt,
only marks,
mark=*,
mark options={solid,fill=mycolor52,draw=black,line width=0.1pt},
forget plot
]
coordinates{
 (40,0.808340016038492) 
};
\addplot [
color=blue,
mark size=0.9pt,
only marks,
mark=*,
mark options={solid,fill=mycolor53,draw=black,line width=0.1pt},
forget plot
]
coordinates{
 (32,0.838810641627543) 
};
\addplot [
color=blue,
mark size=0.9pt,
only marks,
mark=*,
mark options={solid,fill=mycolor54,draw=black,line width=0.1pt},
forget plot
]
coordinates{
 (35,0.87454677302393) 
};
\addplot [
color=blue,
mark size=0.9pt,
only marks,
mark=*,
mark options={solid,fill=mycolor55,draw=black,line width=0.1pt},
forget plot
]
coordinates{
 (36,0.784345047923323) 
};
\addplot [
color=blue,
mark size=0.9pt,
only marks,
mark=*,
mark options={solid,fill=mycolor27,draw=black,line width=0.1pt},
forget plot
]
coordinates{
 (31,0.805212620027435) 
};
\addplot [
color=blue,
mark size=0.9pt,
only marks,
mark=*,
mark options={solid,fill=mycolor28,draw=black,line width=0.1pt},
forget plot
]
coordinates{
 (27,0.752553024351925) 
};
\addplot [
color=blue,
mark size=0.9pt,
only marks,
mark=*,
mark options={solid,fill=red!68!black,draw=black,line width=0.1pt},
forget plot
]
coordinates{
 (395,0.962251201098147) 
};
\addplot [
color=blue,
mark size=0.9pt,
only marks,
mark=*,
mark options={solid,fill=red!68!black,draw=black,line width=0.1pt},
forget plot
]
coordinates{
 (281,0.934296028880866) 
};
\addplot [
color=blue,
mark size=0.9pt,
only marks,
mark=*,
mark options={solid,fill=red!68!black,draw=black,line width=0.1pt},
forget plot
]
coordinates{
 (381,0.963673749143249) 
};
\addplot [
color=blue,
mark size=0.9pt,
only marks,
mark=*,
mark options={solid,fill=red!72!black,draw=black,line width=0.1pt},
forget plot
]
coordinates{
 (331,0.949128919860627) 
};
\addplot [
color=blue,
mark size=0.9pt,
only marks,
mark=*,
mark options={solid,fill=red!72!black,draw=black,line width=0.1pt},
forget plot
]
coordinates{
 (316,0.937238493723849) 
};
\addplot [
color=blue,
mark size=0.9pt,
only marks,
mark=*,
mark options={solid,fill=red!72!black,draw=black,line width=0.1pt},
forget plot
]
coordinates{
 (271,0.937544867193108) 
};
\addplot [
color=blue,
mark size=0.9pt,
only marks,
mark=*,
mark options={solid,fill=red!72!black,draw=black,line width=0.1pt},
forget plot
]
coordinates{
 (252,0.878877400295421) 
};
\addplot [
color=blue,
mark size=0.9pt,
only marks,
mark=*,
mark options={solid,fill=red!76!black,draw=black,line width=0.1pt},
forget plot
]
coordinates{
 (287,0.937845303867403) 
};
\addplot [
color=blue,
mark size=0.9pt,
only marks,
mark=*,
mark options={solid,fill=red!76!black,draw=black,line width=0.1pt},
forget plot
]
coordinates{
 (357,0.96986301369863) 
};
\addplot [
color=blue,
mark size=0.9pt,
only marks,
mark=*,
mark options={solid,fill=red!76!black,draw=black,line width=0.1pt},
forget plot
]
coordinates{
 (312,0.960055096418733) 
};
\addplot [
color=blue,
mark size=0.9pt,
only marks,
mark=*,
mark options={solid,fill=red!80!black,draw=black,line width=0.1pt},
forget plot
]
coordinates{
 (301,0.9648033126294) 
};
\addplot [
color=blue,
mark size=0.9pt,
only marks,
mark=*,
mark options={solid,fill=red!80!black,draw=black,line width=0.1pt},
forget plot
]
coordinates{
 (302,0.950105411103303) 
};
\addplot [
color=blue,
mark size=0.9pt,
only marks,
mark=*,
mark options={solid,fill=red!80!black,draw=black,line width=0.1pt},
forget plot
]
coordinates{
 (323,0.964137931034483) 
};
\addplot [
color=blue,
mark size=0.9pt,
only marks,
mark=*,
mark options={solid,fill=red!84!black,draw=black,line width=0.1pt},
forget plot
]
coordinates{
 (272,0.9637234770705) 
};
\addplot [
color=blue,
mark size=0.9pt,
only marks,
mark=*,
mark options={solid,fill=red!84!black,draw=black,line width=0.1pt},
forget plot
]
coordinates{
 (294,0.954607977991747) 
};
\addplot [
color=blue,
mark size=0.9pt,
only marks,
mark=*,
mark options={solid,fill=red!88!black,draw=black,line width=0.1pt},
forget plot
]
coordinates{
 (260,0.939353099730458) 
};
\addplot [
color=blue,
mark size=0.9pt,
only marks,
mark=*,
mark options={solid,fill=red!88!black,draw=black,line width=0.1pt},
forget plot
]
coordinates{
 (259,0.962042788129744) 
};
\addplot [
color=blue,
mark size=0.9pt,
only marks,
mark=*,
mark options={solid,fill=red!92!black,draw=black,line width=0.1pt},
forget plot
]
coordinates{
 (225,0.929336188436831) 
};
\addplot [
color=blue,
mark size=0.9pt,
only marks,
mark=*,
mark options={solid,fill=red!92!black,draw=black,line width=0.1pt},
forget plot
]
coordinates{
 (269,0.953232462173315) 
};
\addplot [
color=blue,
mark size=0.9pt,
only marks,
mark=*,
mark options={solid,fill=red!96!black,draw=black,line width=0.1pt},
forget plot
]
coordinates{
 (211,0.938833570412518) 
};
\addplot [
color=blue,
mark size=0.9pt,
only marks,
mark=*,
mark options={solid,fill=red!96!black,draw=black,line width=0.1pt},
forget plot
]
coordinates{
 (230,0.943189596167009) 
};
\addplot [
color=blue,
mark size=0.9pt,
only marks,
mark=*,
mark options={solid,fill=red,draw=black,line width=0.1pt},
forget plot
]
coordinates{
 (186,0.884241971620612) 
};
\addplot [
color=blue,
mark size=0.9pt,
only marks,
mark=*,
mark options={solid,fill=red,draw=black,line width=0.1pt},
forget plot
]
coordinates{
 (171,0.919540229885057) 
};
\addplot [
color=blue,
mark size=0.9pt,
only marks,
mark=*,
mark options={solid,fill=mycolor1,draw=black,line width=0.1pt},
forget plot
]
coordinates{
 (169,0.908296943231441) 
};
\addplot [
color=blue,
mark size=0.9pt,
only marks,
mark=*,
mark options={solid,fill=mycolor2,draw=black,line width=0.1pt},
forget plot
]
coordinates{
 (186,0.934871099050203) 
};
\addplot [
color=blue,
mark size=0.9pt,
only marks,
mark=*,
mark options={solid,fill=mycolor3,draw=black,line width=0.1pt},
forget plot
]
coordinates{
 (183,0.95407813570939) 
};
\addplot [
color=blue,
mark size=0.9pt,
only marks,
mark=*,
mark options={solid,fill=mycolor3,draw=black,line width=0.1pt},
forget plot
]
coordinates{
 (137,0.883582089552239) 
};
\addplot [
color=blue,
mark size=0.9pt,
only marks,
mark=*,
mark options={solid,fill=mycolor4,draw=black,line width=0.1pt},
forget plot
]
coordinates{
 (151,0.941828254847645) 
};
\addplot [
color=blue,
mark size=0.9pt,
only marks,
mark=*,
mark options={solid,fill=mycolor5,draw=black,line width=0.1pt},
forget plot
]
coordinates{
 (130,0.91027496382055) 
};
\addplot [
color=blue,
mark size=0.9pt,
only marks,
mark=*,
mark options={solid,fill=mycolor6,draw=black,line width=0.1pt},
forget plot
]
coordinates{
 (152,0.907518296739854) 
};
\addplot [
color=blue,
mark size=0.9pt,
only marks,
mark=*,
mark options={solid,fill=mycolor7,draw=black,line width=0.1pt},
forget plot
]
coordinates{
 (117,0.909698996655518) 
};
\addplot [
color=blue,
mark size=0.9pt,
only marks,
mark=*,
mark options={solid,fill=mycolor8,draw=black,line width=0.1pt},
forget plot
]
coordinates{
 (118,0.886111111111111) 
};
\addplot [
color=blue,
mark size=0.9pt,
only marks,
mark=*,
mark options={solid,fill=mycolor9,draw=black,line width=0.1pt},
forget plot
]
coordinates{
 (99,0.913454545454545) 
};
\addplot [
color=blue,
mark size=0.9pt,
only marks,
mark=*,
mark options={solid,fill=mycolor10,draw=black,line width=0.1pt},
forget plot
]
coordinates{
 (106,0.891628812459442) 
};
\addplot [
color=blue,
mark size=0.9pt,
only marks,
mark=*,
mark options={solid,fill=mycolor11,draw=black,line width=0.1pt},
forget plot
]
coordinates{
 (79,0.875798438608942) 
};
\addplot [
color=blue,
mark size=0.9pt,
only marks,
mark=*,
mark options={solid,fill=mycolor12,draw=black,line width=0.1pt},
forget plot
]
coordinates{
 (82,0.887142857142857) 
};
\addplot [
color=blue,
mark size=0.9pt,
only marks,
mark=*,
mark options={solid,fill=mycolor13,draw=black,line width=0.1pt},
forget plot
]
coordinates{
 (76,0.877840909090909) 
};
\addplot [
color=blue,
mark size=0.9pt,
only marks,
mark=*,
mark options={solid,fill=mycolor15,draw=black,line width=0.1pt},
forget plot
]
coordinates{
 (66,0.888230940044412) 
};
\addplot [
color=blue,
mark size=0.9pt,
only marks,
mark=*,
mark options={solid,fill=mycolor51,draw=black,line width=0.1pt},
forget plot
]
coordinates{
 (64,0.885793871866295) 
};
\addplot [
color=blue,
mark size=0.9pt,
only marks,
mark=*,
mark options={solid,fill=mycolor17,draw=black,line width=0.1pt},
forget plot
]
coordinates{
 (49,0.808868501529052) 
};
\addplot [
color=blue,
mark size=0.9pt,
only marks,
mark=*,
mark options={solid,fill=mycolor18,draw=black,line width=0.1pt},
forget plot
]
coordinates{
 (49,0.825) 
};
\addplot [
color=blue,
mark size=0.9pt,
only marks,
mark=*,
mark options={solid,fill=mycolor19,draw=black,line width=0.1pt},
forget plot
]
coordinates{
 (49,0.860411899313501) 
};
\addplot [
color=blue,
mark size=0.9pt,
only marks,
mark=*,
mark options={solid,fill=mycolor20,draw=black,line width=0.1pt},
forget plot
]
coordinates{
 (47,0.836913285600636) 
};
\addplot [
color=blue,
mark size=0.9pt,
only marks,
mark=*,
mark options={solid,fill=mycolor21,draw=black,line width=0.1pt},
forget plot
]
coordinates{
 (45,0.874285714285714) 
};
\addplot [
color=blue,
mark size=0.9pt,
only marks,
mark=*,
mark options={solid,fill=mycolor52,draw=black,line width=0.1pt},
forget plot
]
coordinates{
 (39,0.791862284820031) 
};
\addplot [
color=blue,
mark size=0.9pt,
only marks,
mark=*,
mark options={solid,fill=mycolor53,draw=black,line width=0.1pt},
forget plot
]
coordinates{
 (39,0.827208756841282) 
};
\addplot [
color=blue,
mark size=0.9pt,
only marks,
mark=*,
mark options={solid,fill=mycolor54,draw=black,line width=0.1pt},
forget plot
]
coordinates{
 (37,0.793242156074014) 
};
\addplot [
color=blue,
mark size=0.9pt,
only marks,
mark=*,
mark options={solid,fill=mycolor55,draw=black,line width=0.1pt},
forget plot
]
coordinates{
 (38,0.859479553903346) 
};
\addplot [
color=blue,
mark size=0.9pt,
only marks,
mark=*,
mark options={solid,fill=mycolor27,draw=black,line width=0.1pt},
forget plot
]
coordinates{
 (35,0.799363057324841) 
};
\addplot [
color=blue,
mark size=0.9pt,
only marks,
mark=*,
mark options={solid,fill=mycolor28,draw=black,line width=0.1pt},
forget plot
]
coordinates{
 (30,0.809266409266409) 
};
\addplot [
color=blue,
mark size=0.9pt,
only marks,
mark=*,
mark options={solid,fill=red!68!black,draw=black,line width=0.1pt},
forget plot
]
coordinates{
 (281,0.930935251798561) 
};
\addplot [
color=blue,
mark size=0.9pt,
only marks,
mark=*,
mark options={solid,fill=red!68!black,draw=black,line width=0.1pt},
forget plot
]
coordinates{
 (321,0.952646239554317) 
};
\addplot [
color=blue,
mark size=0.9pt,
only marks,
mark=*,
mark options={solid,fill=red!68!black,draw=black,line width=0.1pt},
forget plot
]
coordinates{
 (387,0.962131837307153) 
};
\addplot [
color=blue,
mark size=0.9pt,
only marks,
mark=*,
mark options={solid,fill=red!72!black,draw=black,line width=0.1pt},
forget plot
]
coordinates{
 (248,0.877458396369138) 
};
\addplot [
color=blue,
mark size=0.9pt,
only marks,
mark=*,
mark options={solid,fill=red!72!black,draw=black,line width=0.1pt},
forget plot
]
coordinates{
 (372,0.963117606123869) 
};
\addplot [
color=blue,
mark size=0.9pt,
only marks,
mark=*,
mark options={solid,fill=red!72!black,draw=black,line width=0.1pt},
forget plot
]
coordinates{
 (325,0.957805907172996) 
};
\addplot [
color=blue,
mark size=0.9pt,
only marks,
mark=*,
mark options={solid,fill=red!72!black,draw=black,line width=0.1pt},
forget plot
]
coordinates{
 (312,0.951135581555402) 
};
\addplot [
color=blue,
mark size=0.9pt,
only marks,
mark=*,
mark options={solid,fill=red!76!black,draw=black,line width=0.1pt},
forget plot
]
coordinates{
 (316,0.955182072829132) 
};
\addplot [
color=blue,
mark size=0.9pt,
only marks,
mark=*,
mark options={solid,fill=red!76!black,draw=black,line width=0.1pt},
forget plot
]
coordinates{
 (357,0.96) 
};
\addplot [
color=blue,
mark size=0.9pt,
only marks,
mark=*,
mark options={solid,fill=red!76!black,draw=black,line width=0.1pt},
forget plot
]
coordinates{
 (226,0.873369148119724) 
};
\addplot [
color=blue,
mark size=0.9pt,
only marks,
mark=*,
mark options={solid,fill=red!80!black,draw=black,line width=0.1pt},
forget plot
]
coordinates{
 (308,0.96455872133426) 
};
\addplot [
color=blue,
mark size=0.9pt,
only marks,
mark=*,
mark options={solid,fill=red!80!black,draw=black,line width=0.1pt},
forget plot
]
coordinates{
 (313,0.9721059972106) 
};
\addplot [
color=blue,
mark size=0.9pt,
only marks,
mark=*,
mark options={solid,fill=red!80!black,draw=black,line width=0.1pt},
forget plot
]
coordinates{
 (317,0.963636363636364) 
};
\addplot [
color=blue,
mark size=0.9pt,
only marks,
mark=*,
mark options={solid,fill=red!84!black,draw=black,line width=0.1pt},
forget plot
]
coordinates{
 (279,0.95049504950495) 
};
\addplot [
color=blue,
mark size=0.9pt,
only marks,
mark=*,
mark options={solid,fill=red!84!black,draw=black,line width=0.1pt},
forget plot
]
coordinates{
 (207,0.876438986953185) 
};
\addplot [
color=blue,
mark size=0.9pt,
only marks,
mark=*,
mark options={solid,fill=red!88!black,draw=black,line width=0.1pt},
forget plot
]
coordinates{
 (278,0.963380281690141) 
};
\addplot [
color=blue,
mark size=0.9pt,
only marks,
mark=*,
mark options={solid,fill=red!88!black,draw=black,line width=0.1pt},
forget plot
]
coordinates{
 (223,0.904129793510324) 
};
\addplot [
color=blue,
mark size=0.9pt,
only marks,
mark=*,
mark options={solid,fill=red!92!black,draw=black,line width=0.1pt},
forget plot
]
coordinates{
 (256,0.96140350877193) 
};
\addplot [
color=blue,
mark size=0.9pt,
only marks,
mark=*,
mark options={solid,fill=red!92!black,draw=black,line width=0.1pt},
forget plot
]
coordinates{
 (235,0.945659844742413) 
};
\addplot [
color=blue,
mark size=0.9pt,
only marks,
mark=*,
mark options={solid,fill=red!96!black,draw=black,line width=0.1pt},
forget plot
]
coordinates{
 (239,0.954732510288066) 
};
\addplot [
color=blue,
mark size=0.9pt,
only marks,
mark=*,
mark options={solid,fill=red!96!black,draw=black,line width=0.1pt},
forget plot
]
coordinates{
 (177,0.874905231235785) 
};
\addplot [
color=blue,
mark size=0.9pt,
only marks,
mark=*,
mark options={solid,fill=red,draw=black,line width=0.1pt},
forget plot
]
coordinates{
 (179,0.857798165137615) 
};
\addplot [
color=blue,
mark size=0.9pt,
only marks,
mark=*,
mark options={solid,fill=red,draw=black,line width=0.1pt},
forget plot
]
coordinates{
 (220,0.959212376933896) 
};
\addplot [
color=blue,
mark size=0.9pt,
only marks,
mark=*,
mark options={solid,fill=mycolor1,draw=black,line width=0.1pt},
forget plot
]
coordinates{
 (171,0.925287356321839) 
};
\addplot [
color=blue,
mark size=0.9pt,
only marks,
mark=*,
mark options={solid,fill=mycolor2,draw=black,line width=0.1pt},
forget plot
]
coordinates{
 (192,0.956036287508723) 
};
\addplot [
color=blue,
mark size=0.9pt,
only marks,
mark=*,
mark options={solid,fill=mycolor3,draw=black,line width=0.1pt},
forget plot
]
coordinates{
 (152,0.909752547307132) 
};
\addplot [
color=blue,
mark size=0.9pt,
only marks,
mark=*,
mark options={solid,fill=mycolor3,draw=black,line width=0.1pt},
forget plot
]
coordinates{
 (175,0.95575221238938) 
};
\addplot [
color=blue,
mark size=0.9pt,
only marks,
mark=*,
mark options={solid,fill=mycolor4,draw=black,line width=0.1pt},
forget plot
]
coordinates{
 (144,0.929394812680115) 
};
\addplot [
color=blue,
mark size=0.9pt,
only marks,
mark=*,
mark options={solid,fill=mycolor5,draw=black,line width=0.1pt},
forget plot
]
coordinates{
 (136,0.922651933701657) 
};
\addplot [
color=blue,
mark size=0.9pt,
only marks,
mark=*,
mark options={solid,fill=mycolor6,draw=black,line width=0.1pt},
forget plot
]
coordinates{
 (137,0.943342776203966) 
};
\addplot [
color=blue,
mark size=0.9pt,
only marks,
mark=*,
mark options={solid,fill=mycolor7,draw=black,line width=0.1pt},
forget plot
]
coordinates{
 (110,0.861132660977502) 
};
\addplot [
color=blue,
mark size=0.9pt,
only marks,
mark=*,
mark options={solid,fill=mycolor8,draw=black,line width=0.1pt},
forget plot
]
coordinates{
 (111,0.91114245416079) 
};
\addplot [
color=blue,
mark size=0.9pt,
only marks,
mark=*,
mark options={solid,fill=mycolor9,draw=black,line width=0.1pt},
forget plot
]
coordinates{
 (107,0.90144727773949) 
};
\addplot [
color=blue,
mark size=0.9pt,
only marks,
mark=*,
mark options={solid,fill=mycolor10,draw=black,line width=0.1pt},
forget plot
]
coordinates{
 (100,0.899786780383795) 
};
\addplot [
color=blue,
mark size=0.9pt,
only marks,
mark=*,
mark options={solid,fill=mycolor11,draw=black,line width=0.1pt},
forget plot
]
coordinates{
 (91,0.858571428571429) 
};
\addplot [
color=blue,
mark size=0.9pt,
only marks,
mark=*,
mark options={solid,fill=mycolor12,draw=black,line width=0.1pt},
forget plot
]
coordinates{
 (76,0.839694656488549) 
};
\addplot [
color=blue,
mark size=0.9pt,
only marks,
mark=*,
mark options={solid,fill=mycolor13,draw=black,line width=0.1pt},
forget plot
]
coordinates{
 (71,0.821321321321321) 
};
\addplot [
color=blue,
mark size=0.9pt,
only marks,
mark=*,
mark options={solid,fill=mycolor15,draw=black,line width=0.1pt},
forget plot
]
coordinates{
 (74,0.898813677599442) 
};
\addplot [
color=blue,
mark size=0.9pt,
only marks,
mark=*,
mark options={solid,fill=mycolor51,draw=black,line width=0.1pt},
forget plot
]
coordinates{
 (58,0.857988165680473) 
};
\addplot [
color=blue,
mark size=0.9pt,
only marks,
mark=*,
mark options={solid,fill=mycolor17,draw=black,line width=0.1pt},
forget plot
]
coordinates{
 (61,0.852897473997028) 
};
\addplot [
color=blue,
mark size=0.9pt,
only marks,
mark=*,
mark options={solid,fill=mycolor18,draw=black,line width=0.1pt},
forget plot
]
coordinates{
 (53,0.884968242766408) 
};
\addplot [
color=blue,
mark size=0.9pt,
only marks,
mark=*,
mark options={solid,fill=mycolor19,draw=black,line width=0.1pt},
forget plot
]
coordinates{
 (47,0.816064257028112) 
};
\addplot [
color=blue,
mark size=0.9pt,
only marks,
mark=*,
mark options={solid,fill=mycolor20,draw=black,line width=0.1pt},
forget plot
]
coordinates{
 (43,0.857798165137615) 
};
\addplot [
color=blue,
mark size=0.9pt,
only marks,
mark=*,
mark options={solid,fill=mycolor21,draw=black,line width=0.1pt},
forget plot
]
coordinates{
 (47,0.833212996389892) 
};
\addplot [
color=blue,
mark size=0.9pt,
only marks,
mark=*,
mark options={solid,fill=mycolor52,draw=black,line width=0.1pt},
forget plot
]
coordinates{
 (38,0.824508320726172) 
};
\addplot [
color=blue,
mark size=0.9pt,
only marks,
mark=*,
mark options={solid,fill=mycolor53,draw=black,line width=0.1pt},
forget plot
]
coordinates{
 (42,0.835666912306559) 
};
\addplot [
color=blue,
mark size=0.9pt,
only marks,
mark=*,
mark options={solid,fill=mycolor54,draw=black,line width=0.1pt},
forget plot
]
coordinates{
 (33,0.841622035195103) 
};
\addplot [
color=blue,
mark size=0.9pt,
only marks,
mark=*,
mark options={solid,fill=mycolor55,draw=black,line width=0.1pt},
forget plot
]
coordinates{
 (36,0.827823691460055) 
};
\addplot [
color=blue,
mark size=0.9pt,
only marks,
mark=*,
mark options={solid,fill=mycolor27,draw=black,line width=0.1pt},
forget plot
]
coordinates{
 (28,0.793388429752066) 
};
\addplot [
color=blue,
mark size=0.9pt,
only marks,
mark=*,
mark options={solid,fill=mycolor28,draw=black,line width=0.1pt},
forget plot
]
coordinates{
 (26,0.806826997672614) 
};
\end{axis}
\end{tikzpicture}%

\renewcommand\trimlen{2pt}
\begin{figure}[tbp]
  \begin{subfigure}[b]{0.49\textwidth}
    \centering
    \adjincludegraphics[width=\linewidth,clip=true,trim=\trimlen{} \trimlen{} \trimlen{} \trimlen{}]{figures/ev_bgape_eps}
    \caption{\textsf{[A]} LSE scatter plot}
	  \label{fig:chl-eps}
  \end{subfigure}
  \hfill
  \begin{subfigure}[b]{0.49\textwidth}
    \centering
    \adjincludegraphics[width=\linewidth,clip=true,trim=\trimlen{} \trimlen{} \trimlen{} \trimlen{}]{figures/ev_bgape_eps}
    \caption{\textsf{[C]} LSE scatter plot}
	\label{fig:bgape-eps}
  \end{subfigure}

  \caption{The effect of $\epsilon$ on the sampling cost and classification
           accuracy of \acl on the two environmental monitoring datasets.
           }
  \label{fig:exp-eps}
\end{figure}

%\setlength\figureheight{1.3in}\setlength\figurewidth{2.1in}
%% This file was created by matlab2tikz v0.2.3.
% Copyright (c) 2008--2012, Nico Schlömer <nico.schloemer@gmail.com>
% All rights reserved.
% 
% 
% 
\begin{tikzpicture}

\begin{axis}[%
tick label style={font=\tiny},
label style={font=\tiny},
label shift={-4pt},
xlabel shift={-6pt},
legend style={font=\tiny},
view={0}{90},
width=\figurewidth,
height=\figureheight,
scale only axis,
xmin=0, xmax=400,
xlabel={Samples},
ymin=0.48, ymax=0.85,
ylabel={$F_1$-score},
axis lines*=left,
legend cell align=left,
legend style={at={(1.03,0)},anchor=south east,fill=none,draw=none,align=left,row sep=-0.2em},
clip=false]

\addplot [
color=blue,
solid,
line width=1.0pt,
]
coordinates{
 (14,0.636477938729992)(15,0.654772268329889)(15,0.654772268329889)(15,0.654772268329889)(15,0.654772268329889)(15,0.654772268329889)(15,0.654772268329889)(15,0.654772268329889)(16,0.670026067539877)(16,0.670026067539877)(16,0.670026067539877)(16,0.670026067539877)(16,0.670026067539877)(16,0.670026067539877)(16,0.670026067539877)(16,0.670026067539877)(16,0.670026067539877)(16,0.670026067539877)(16,0.670026067539877)(17,0.682037354813215)(17,0.682037354813215)(17,0.682037354813215)(17,0.682037354813215)(17,0.682037354813215)(17,0.682037354813215)(17,0.682037354813215)(17,0.682037354813215)(17,0.682037354813215)(17,0.682037354813215)(17,0.682037354813215)(17,0.682037354813215)(17,0.682037354813215)(18,0.690440589197074)(18,0.690440589197074)(18,0.690440589197074)(18,0.690440589197074)(18,0.690440589197074)(19,0.695416731455047)(19,0.695416731455047)(19,0.695416731455047)(19,0.695416731455047)(19,0.695416731455047)(19,0.695416731455047)(19,0.695416731455047)(19,0.695416731455047)(19,0.695416731455047)(19,0.695416731455047)(19,0.695416731455047)(20,0.698725029995701)(20,0.698725029995701)(20,0.698725029995701)(20,0.698725029995701)(20,0.698725029995701)(20,0.698725029995701)(20,0.698725029995701)(20,0.698725029995701)(20,0.698725029995701)(21,0.700719778589398)(21,0.700719778589398)(21,0.700719778589398)(21,0.700719778589398)(21,0.700719778589398)(21,0.700719778589398)(21,0.700719778589398)(21,0.700719778589398)(21,0.700719778589398)(21,0.700719778589398)(21,0.700719778589398)(21,0.700719778589398)(21,0.700719778589398)(21,0.700719778589398)(21,0.700719778589398)(21,0.700719778589398)(22,0.701343800782098)(22,0.701343800782098)(22,0.701343800782098)(22,0.701343800782098)(22,0.701343800782098)(22,0.701343800782098)(22,0.701343800782098)(22,0.701343800782098)(22,0.701343800782098)(22,0.701343800782098)(22,0.701343800782098)(22,0.701343800782098)(22,0.701343800782098)(22,0.701343800782098)(22,0.701343800782098)(22,0.701343800782098)(22,0.701343800782098)(22,0.701343800782098)(22,0.701343800782098)(22,0.701343800782098)(22,0.701343800782098)(22,0.701343800782098)(22,0.701343800782098)(23,0.699113774083587)(23,0.699113774083587)(23,0.699113774083587)(23,0.699113774083587)(23,0.699113774083587)(23,0.699113774083587)(23,0.699113774083587)(23,0.699113774083587)(23,0.699113774083587)(23,0.699113774083587)(23,0.699113774083587)(23,0.699113774083587)(23,0.699113774083587)(23,0.699113774083587)(23,0.699113774083587)(23,0.699113774083587)(23,0.699113774083587)(23,0.699113774083587)(23,0.699113774083587)(24,0.695922038251522)(24,0.695922038251522)(24,0.695922038251522)(24,0.695922038251522)(24,0.695922038251522)(24,0.695922038251522)(24,0.695922038251522)(24,0.695922038251522)(24,0.695922038251522)(24,0.695922038251522)(24,0.695922038251522)(24,0.695922038251522)(24,0.695922038251522)(24,0.695922038251522)(24,0.695922038251522)(24,0.695922038251522)(24,0.695922038251522)(24,0.695922038251522)(24,0.695922038251522)(24,0.695922038251522)(24,0.695922038251522)(24,0.695922038251522)(24,0.695922038251522)(24,0.695922038251522)(24,0.695922038251522)(25,0.692112534200826)(25,0.692112534200826)(25,0.692112534200826)(25,0.692112534200826)(25,0.692112534200826)(25,0.692112534200826)(25,0.692112534200826)(25,0.692112534200826)(25,0.692112534200826)(25,0.692112534200826)(25,0.692112534200826)(25,0.692112534200826)(25,0.692112534200826)(25,0.692112534200826)(25,0.692112534200826)(25,0.692112534200826)(25,0.692112534200826)(26,0.690962428626295)(26,0.690962428626295)(26,0.690962428626295)(26,0.690962428626295)(26,0.690962428626295)(26,0.690962428626295)(26,0.690962428626295)(26,0.690962428626295)(26,0.690962428626295)(26,0.690962428626295)(26,0.690962428626295)(26,0.690962428626295)(27,0.692286111880954)(27,0.692286111880954)(27,0.692286111880954)(27,0.692286111880954)(27,0.692286111880954)(27,0.692286111880954)(27,0.692286111880954)(27,0.692286111880954)(27,0.692286111880954)(27,0.692286111880954)(27,0.692286111880954)(27,0.692286111880954)(27,0.692286111880954)(27,0.692286111880954)(27,0.692286111880954)(28,0.695325748992316)(28,0.695325748992316)(28,0.695325748992316)(28,0.695325748992316)(28,0.695325748992316)(28,0.695325748992316)(28,0.695325748992316)(28,0.695325748992316)(28,0.695325748992316)(28,0.695325748992316)(28,0.695325748992316)(28,0.695325748992316)(28,0.695325748992316)(28,0.695325748992316)(28,0.695325748992316)(28,0.695325748992316)(28,0.695325748992316)(28,0.695325748992316)(28,0.695325748992316)(29,0.701600982791518)(29,0.701600982791518)(29,0.701600982791518)(29,0.701600982791518)(29,0.701600982791518)(29,0.701600982791518)(29,0.701600982791518)(29,0.701600982791518)(29,0.701600982791518)(29,0.701600982791518)(29,0.701600982791518)(29,0.701600982791518)(29,0.701600982791518)(29,0.701600982791518)(29,0.701600982791518)(29,0.701600982791518)(29,0.701600982791518)(29,0.701600982791518)(30,0.710721280800668)(30,0.710721280800668)(30,0.710721280800668)(30,0.710721280800668)(30,0.710721280800668)(30,0.710721280800668)(30,0.710721280800668)(30,0.710721280800668)(30,0.710721280800668)(30,0.710721280800668)(30,0.710721280800668)(30,0.710721280800668)(30,0.710721280800668)(30,0.710721280800668)(30,0.710721280800668)(30,0.710721280800668)(30,0.710721280800668)(30,0.710721280800668)(30,0.710721280800668)(30,0.710721280800668)(30,0.710721280800668)(30,0.710721280800668)(31,0.719628236668913)(31,0.719628236668913)(31,0.719628236668913)(31,0.719628236668913)(31,0.719628236668913)(31,0.719628236668913)(31,0.719628236668913)(31,0.719628236668913)(31,0.719628236668913)(31,0.719628236668913)(31,0.719628236668913)(31,0.719628236668913)(31,0.719628236668913)(31,0.719628236668913)(31,0.719628236668913)(31,0.719628236668913)(31,0.719628236668913)(31,0.719628236668913)(32,0.724391712117678)(32,0.724391712117678)(32,0.724391712117678)(32,0.724391712117678)(32,0.724391712117678)(32,0.724391712117678)(32,0.724391712117678)(32,0.724391712117678)(32,0.724391712117678)(32,0.724391712117678)(32,0.724391712117678)(32,0.724391712117678)(32,0.724391712117678)(32,0.724391712117678)(32,0.724391712117678)(32,0.724391712117678)(32,0.724391712117678)(32,0.724391712117678)(32,0.724391712117678)(32,0.724391712117678)(33,0.726109859790085)(33,0.726109859790085)(33,0.726109859790085)(33,0.726109859790085)(33,0.726109859790085)(33,0.726109859790085)(33,0.726109859790085)(33,0.726109859790085)(33,0.726109859790085)(33,0.726109859790085)(33,0.726109859790085)(33,0.726109859790085)(33,0.726109859790085)(33,0.726109859790085)(34,0.725765048499857)(34,0.725765048499857)(34,0.725765048499857)(34,0.725765048499857)(34,0.725765048499857)(34,0.725765048499857)(34,0.725765048499857)(34,0.725765048499857)(34,0.725765048499857)(34,0.725765048499857)(34,0.725765048499857)(34,0.725765048499857)(34,0.725765048499857)(34,0.725765048499857)(34,0.725765048499857)(34,0.725765048499857)(34,0.725765048499857)(34,0.725765048499857)(34,0.725765048499857)(34,0.725765048499857)(34,0.725765048499857)(34,0.725765048499857)(34,0.725765048499857)(34,0.725765048499857)(34,0.725765048499857)(34,0.725765048499857)(34,0.725765048499857)(34,0.725765048499857)(35,0.723449785764037)(35,0.723449785764037)(35,0.723449785764037)(35,0.723449785764037)(35,0.723449785764037)(35,0.723449785764037)(35,0.723449785764037)(35,0.723449785764037)(35,0.723449785764037)(35,0.723449785764037)(35,0.723449785764037)(35,0.723449785764037)(35,0.723449785764037)(35,0.723449785764037)(35,0.723449785764037)(35,0.723449785764037)(35,0.723449785764037)(35,0.723449785764037)(35,0.723449785764037)(35,0.723449785764037)(35,0.723449785764037)(35,0.723449785764037)(36,0.720487075811149)(36,0.720487075811149)(36,0.720487075811149)(36,0.720487075811149)(36,0.720487075811149)(36,0.720487075811149)(36,0.720487075811149)(36,0.720487075811149)(36,0.720487075811149)(36,0.720487075811149)(36,0.720487075811149)(36,0.720487075811149)(36,0.720487075811149)(36,0.720487075811149)(36,0.720487075811149)(36,0.720487075811149)(36,0.720487075811149)(36,0.720487075811149)(36,0.720487075811149)(36,0.720487075811149)(36,0.720487075811149)(36,0.720487075811149)(36,0.720487075811149)(36,0.720487075811149)(37,0.718344571580359)(37,0.718344571580359)(37,0.718344571580359)(37,0.718344571580359)(37,0.718344571580359)(37,0.718344571580359)(37,0.718344571580359)(37,0.718344571580359)(37,0.718344571580359)(37,0.718344571580359)(37,0.718344571580359)(37,0.718344571580359)(37,0.718344571580359)(37,0.718344571580359)(37,0.718344571580359)(38,0.717571265199188)(38,0.717571265199188)(38,0.717571265199188)(38,0.717571265199188)(38,0.717571265199188)(38,0.717571265199188)(38,0.717571265199188)(38,0.717571265199188)(38,0.717571265199188)(38,0.717571265199188)(38,0.717571265199188)(38,0.717571265199188)(38,0.717571265199188)(38,0.717571265199188)(38,0.717571265199188)(38,0.717571265199188)(38,0.717571265199188)(38,0.717571265199188)(38,0.717571265199188)(38,0.717571265199188)(39,0.717853765510841)(39,0.717853765510841)(39,0.717853765510841)(39,0.717853765510841)(39,0.717853765510841)(39,0.717853765510841)(39,0.717853765510841)(39,0.717853765510841)(39,0.717853765510841)(39,0.717853765510841)(39,0.717853765510841)(40,0.719192894136691)(40,0.719192894136691)(40,0.719192894136691)(40,0.719192894136691)(40,0.719192894136691)(40,0.719192894136691)(40,0.719192894136691)(40,0.719192894136691)(40,0.719192894136691)(40,0.719192894136691)(40,0.719192894136691)(40,0.719192894136691)(40,0.719192894136691)(40,0.719192894136691)(40,0.719192894136691)(40,0.719192894136691)(40,0.719192894136691)(40,0.719192894136691)(40,0.719192894136691)(40,0.719192894136691)(40,0.719192894136691)(40,0.719192894136691)(40,0.719192894136691)(41,0.722269727350281)(41,0.722269727350281)(41,0.722269727350281)(41,0.722269727350281)(41,0.722269727350281)(41,0.722269727350281)(41,0.722269727350281)(41,0.722269727350281)(41,0.722269727350281)(41,0.722269727350281)(41,0.722269727350281)(41,0.722269727350281)(41,0.722269727350281)(41,0.722269727350281)(41,0.722269727350281)(41,0.722269727350281)(41,0.722269727350281)(41,0.722269727350281)(41,0.722269727350281)(41,0.722269727350281)(42,0.724844278479809)(42,0.724844278479809)(42,0.724844278479809)(42,0.724844278479809)(42,0.724844278479809)(42,0.724844278479809)(42,0.724844278479809)(42,0.724844278479809)(42,0.724844278479809)(42,0.724844278479809)(42,0.724844278479809)(42,0.724844278479809)(42,0.724844278479809)(43,0.727144143452796)(43,0.727144143452796)(43,0.727144143452796)(43,0.727144143452796)(43,0.727144143452796)(43,0.727144143452796)(43,0.727144143452796)(43,0.727144143452796)(43,0.727144143452796)(43,0.727144143452796)(43,0.727144143452796)(43,0.727144143452796)(43,0.727144143452796)(43,0.727144143452796)(43,0.727144143452796)(43,0.727144143452796)(43,0.727144143452796)(43,0.727144143452796)(43,0.727144143452796)(43,0.727144143452796)(43,0.727144143452796)(44,0.728605879347697)(44,0.728605879347697)(44,0.728605879347697)(44,0.728605879347697)(44,0.728605879347697)(44,0.728605879347697)(44,0.728605879347697)(44,0.728605879347697)(44,0.728605879347697)(44,0.728605879347697)(44,0.728605879347697)(44,0.728605879347697)(44,0.728605879347697)(44,0.728605879347697)(45,0.730181078547607)(45,0.730181078547607)(45,0.730181078547607)(45,0.730181078547607)(45,0.730181078547607)(45,0.730181078547607)(45,0.730181078547607)(45,0.730181078547607)(45,0.730181078547607)(45,0.730181078547607)(45,0.730181078547607)(45,0.730181078547607)(45,0.730181078547607)(46,0.730895014424358)(46,0.730895014424358)(46,0.730895014424358)(46,0.730895014424358)(46,0.730895014424358)(46,0.730895014424358)(46,0.730895014424358)(46,0.730895014424358)(46,0.730895014424358)(46,0.730895014424358)(46,0.730895014424358)(46,0.730895014424358)(46,0.730895014424358)(46,0.730895014424358)(46,0.730895014424358)(46,0.730895014424358)(46,0.730895014424358)(46,0.730895014424358)(46,0.730895014424358)(47,0.731104962985294)(47,0.731104962985294)(47,0.731104962985294)(47,0.731104962985294)(47,0.731104962985294)(47,0.731104962985294)(47,0.731104962985294)(47,0.731104962985294)(47,0.731104962985294)(47,0.731104962985294)(47,0.731104962985294)(47,0.731104962985294)(48,0.731491804230122)(48,0.731491804230122)(48,0.731491804230122)(48,0.731491804230122)(48,0.731491804230122)(48,0.731491804230122)(48,0.731491804230122)(48,0.731491804230122)(48,0.731491804230122)(48,0.731491804230122)(48,0.731491804230122)(48,0.731491804230122)(48,0.731491804230122)(48,0.731491804230122)(48,0.731491804230122)(48,0.731491804230122)(48,0.731491804230122)(48,0.731491804230122)(49,0.73310367144568)(49,0.73310367144568)(49,0.73310367144568)(49,0.73310367144568)(49,0.73310367144568)(49,0.73310367144568)(49,0.73310367144568)(49,0.73310367144568)(49,0.73310367144568)(49,0.73310367144568)(49,0.73310367144568)(49,0.73310367144568)(49,0.73310367144568)(50,0.735838837122779)(50,0.735838837122779)(50,0.735838837122779)(50,0.735838837122779)(50,0.735838837122779)(50,0.735838837122779)(50,0.735838837122779)(50,0.735838837122779)(50,0.735838837122779)(50,0.735838837122779)(50,0.735838837122779)(50,0.735838837122779)(51,0.738524823431281)(51,0.738524823431281)(51,0.738524823431281)(51,0.738524823431281)(51,0.738524823431281)(51,0.738524823431281)(51,0.738524823431281)(51,0.738524823431281)(51,0.738524823431281)(51,0.738524823431281)(51,0.738524823431281)(51,0.738524823431281)(51,0.738524823431281)(51,0.738524823431281)(51,0.738524823431281)(52,0.740807104762009)(52,0.740807104762009)(52,0.740807104762009)(52,0.740807104762009)(52,0.740807104762009)(52,0.740807104762009)(52,0.740807104762009)(52,0.740807104762009)(52,0.740807104762009)(52,0.740807104762009)(52,0.740807104762009)(52,0.740807104762009)(52,0.740807104762009)(52,0.740807104762009)(52,0.740807104762009)(52,0.740807104762009)(52,0.740807104762009)(53,0.742950533107053)(53,0.742950533107053)(53,0.742950533107053)(53,0.742950533107053)(53,0.742950533107053)(53,0.742950533107053)(53,0.742950533107053)(53,0.742950533107053)(53,0.742950533107053)(53,0.742950533107053)(53,0.742950533107053)(53,0.742950533107053)(53,0.742950533107053)(53,0.742950533107053)(53,0.742950533107053)(54,0.745108423396071)(54,0.745108423396071)(54,0.745108423396071)(54,0.745108423396071)(54,0.745108423396071)(54,0.745108423396071)(54,0.745108423396071)(54,0.745108423396071)(54,0.745108423396071)(54,0.745108423396071)(54,0.745108423396071)(54,0.745108423396071)(54,0.745108423396071)(54,0.745108423396071)(54,0.745108423396071)(55,0.747301597622396)(55,0.747301597622396)(55,0.747301597622396)(55,0.747301597622396)(55,0.747301597622396)(55,0.747301597622396)(55,0.747301597622396)(55,0.747301597622396)(55,0.747301597622396)(55,0.747301597622396)(55,0.747301597622396)(55,0.747301597622396)(55,0.747301597622396)(56,0.749358271829761)(56,0.749358271829761)(56,0.749358271829761)(56,0.749358271829761)(56,0.749358271829761)(56,0.749358271829761)(56,0.749358271829761)(56,0.749358271829761)(56,0.749358271829761)(56,0.749358271829761)(56,0.749358271829761)(56,0.749358271829761)(57,0.75085192025567)(57,0.75085192025567)(57,0.75085192025567)(57,0.75085192025567)(57,0.75085192025567)(57,0.75085192025567)(57,0.75085192025567)(57,0.75085192025567)(57,0.75085192025567)(57,0.75085192025567)(57,0.75085192025567)(57,0.75085192025567)(57,0.75085192025567)(57,0.75085192025567)(58,0.752244929434664)(58,0.752244929434664)(58,0.752244929434664)(58,0.752244929434664)(58,0.752244929434664)(58,0.752244929434664)(58,0.752244929434664)(58,0.752244929434664)(58,0.752244929434664)(58,0.752244929434664)(58,0.752244929434664)(58,0.752244929434664)(58,0.752244929434664)(58,0.752244929434664)(58,0.752244929434664)(58,0.752244929434664)(58,0.752244929434664)(59,0.753795239951061)(59,0.753795239951061)(59,0.753795239951061)(59,0.753795239951061)(59,0.753795239951061)(59,0.753795239951061)(59,0.753795239951061)(59,0.753795239951061)(59,0.753795239951061)(59,0.753795239951061)(59,0.753795239951061)(59,0.753795239951061)(59,0.753795239951061)(59,0.753795239951061)(59,0.753795239951061)(59,0.753795239951061)(59,0.753795239951061)(59,0.753795239951061)(60,0.755524721373829)(60,0.755524721373829)(60,0.755524721373829)(60,0.755524721373829)(60,0.755524721373829)(60,0.755524721373829)(60,0.755524721373829)(60,0.755524721373829)(60,0.755524721373829)(60,0.755524721373829)(60,0.755524721373829)(60,0.755524721373829)(60,0.755524721373829)(60,0.755524721373829)(60,0.755524721373829)(61,0.757020878391112)(61,0.757020878391112)(61,0.757020878391112)(61,0.757020878391112)(61,0.757020878391112)(61,0.757020878391112)(61,0.757020878391112)(61,0.757020878391112)(61,0.757020878391112)(62,0.75777806086647)(62,0.75777806086647)(62,0.75777806086647)(62,0.75777806086647)(62,0.75777806086647)(62,0.75777806086647)(62,0.75777806086647)(62,0.75777806086647)(62,0.75777806086647)(62,0.75777806086647)(62,0.75777806086647)(62,0.75777806086647)(62,0.75777806086647)(62,0.75777806086647)(63,0.757367283328744)(63,0.757367283328744)(63,0.757367283328744)(63,0.757367283328744)(63,0.757367283328744)(63,0.757367283328744)(63,0.757367283328744)(63,0.757367283328744)(63,0.757367283328744)(63,0.757367283328744)(64,0.756293942498833)(64,0.756293942498833)(64,0.756293942498833)(64,0.756293942498833)(64,0.756293942498833)(64,0.756293942498833)(64,0.756293942498833)(64,0.756293942498833)(64,0.756293942498833)(64,0.756293942498833)(64,0.756293942498833)(64,0.756293942498833)(64,0.756293942498833)(64,0.756293942498833)(64,0.756293942498833)(64,0.756293942498833)(64,0.756293942498833)(64,0.756293942498833)(65,0.755085050499534)(65,0.755085050499534)(65,0.755085050499534)(65,0.755085050499534)(65,0.755085050499534)(65,0.755085050499534)(65,0.755085050499534)(65,0.755085050499534)(65,0.755085050499534)(65,0.755085050499534)(66,0.753746083923868)(66,0.753746083923868)(66,0.753746083923868)(66,0.753746083923868)(66,0.753746083923868)(66,0.753746083923868)(66,0.753746083923868)(66,0.753746083923868)(66,0.753746083923868)(66,0.753746083923868)(66,0.753746083923868)(66,0.753746083923868)(66,0.753746083923868)(66,0.753746083923868)(67,0.753078067850201)(67,0.753078067850201)(67,0.753078067850201)(67,0.753078067850201)(67,0.753078067850201)(67,0.753078067850201)(67,0.753078067850201)(67,0.753078067850201)(67,0.753078067850201)(67,0.753078067850201)(67,0.753078067850201)(67,0.753078067850201)(67,0.753078067850201)(68,0.752572683187233)(68,0.752572683187233)(68,0.752572683187233)(68,0.752572683187233)(68,0.752572683187233)(68,0.752572683187233)(68,0.752572683187233)(68,0.752572683187233)(69,0.752798156232776)(69,0.752798156232776)(69,0.752798156232776)(69,0.752798156232776)(69,0.752798156232776)(69,0.752798156232776)(69,0.752798156232776)(69,0.752798156232776)(69,0.752798156232776)(69,0.752798156232776)(69,0.752798156232776)(70,0.75479220432969)(70,0.75479220432969)(70,0.75479220432969)(70,0.75479220432969)(70,0.75479220432969)(70,0.75479220432969)(70,0.75479220432969)(70,0.75479220432969)(70,0.75479220432969)(70,0.75479220432969)(70,0.75479220432969)(70,0.75479220432969)(70,0.75479220432969)(70,0.75479220432969)(70,0.75479220432969)(70,0.75479220432969)(71,0.756460154874228)(71,0.756460154874228)(71,0.756460154874228)(71,0.756460154874228)(71,0.756460154874228)(71,0.756460154874228)(71,0.756460154874228)(71,0.756460154874228)(71,0.756460154874228)(71,0.756460154874228)(71,0.756460154874228)(71,0.756460154874228)(71,0.756460154874228)(71,0.756460154874228)(72,0.758270636987615)(72,0.758270636987615)(72,0.758270636987615)(72,0.758270636987615)(72,0.758270636987615)(72,0.758270636987615)(72,0.758270636987615)(72,0.758270636987615)(72,0.758270636987615)(72,0.758270636987615)(72,0.758270636987615)(73,0.760136487610401)(73,0.760136487610401)(73,0.760136487610401)(73,0.760136487610401)(73,0.760136487610401)(73,0.760136487610401)(73,0.760136487610401)(73,0.760136487610401)(73,0.760136487610401)(73,0.760136487610401)(73,0.760136487610401)(73,0.760136487610401)(73,0.760136487610401)(73,0.760136487610401)(73,0.760136487610401)(74,0.761071341540674)(74,0.761071341540674)(74,0.761071341540674)(74,0.761071341540674)(74,0.761071341540674)(74,0.761071341540674)(74,0.761071341540674)(74,0.761071341540674)(74,0.761071341540674)(74,0.761071341540674)(74,0.761071341540674)(74,0.761071341540674)(75,0.763201953418607)(75,0.763201953418607)(75,0.763201953418607)(75,0.763201953418607)(75,0.763201953418607)(75,0.763201953418607)(76,0.763504822876822)(76,0.763504822876822)(76,0.763504822876822)(76,0.763504822876822)(76,0.763504822876822)(76,0.763504822876822)(76,0.763504822876822)(76,0.763504822876822)(77,0.764115476674599)(77,0.764115476674599)(77,0.764115476674599)(77,0.764115476674599)(77,0.764115476674599)(77,0.764115476674599)(78,0.764734188641589)(78,0.764734188641589)(78,0.764734188641589)(78,0.764734188641589)(78,0.764734188641589)(78,0.764734188641589)(78,0.764734188641589)(79,0.765151828134951)(79,0.765151828134951)(79,0.765151828134951)(79,0.765151828134951)(79,0.765151828134951)(79,0.765151828134951)(79,0.765151828134951)(79,0.765151828134951)(79,0.765151828134951)(80,0.76551811037793)(80,0.76551811037793)(80,0.76551811037793)(80,0.76551811037793)(80,0.76551811037793)(80,0.76551811037793)(80,0.76551811037793)(80,0.76551811037793)(80,0.76551811037793)(80,0.76551811037793)(80,0.76551811037793)(80,0.76551811037793)(80,0.76551811037793)(81,0.76595160027272)(81,0.76595160027272)(81,0.76595160027272)(81,0.76595160027272)(81,0.76595160027272)(81,0.76595160027272)(81,0.76595160027272)(81,0.76595160027272)(81,0.76595160027272)(81,0.76595160027272)(82,0.7665398630591)(82,0.7665398630591)(82,0.7665398630591)(83,0.767361989135475)(83,0.767361989135475)(83,0.767361989135475)(83,0.767361989135475)(83,0.767361989135475)(83,0.767361989135475)(83,0.767361989135475)(83,0.767361989135475)(83,0.767361989135475)(83,0.767361989135475)(83,0.767361989135475)(83,0.767361989135475)(83,0.767361989135475)(83,0.767361989135475)(84,0.768349382869096)(84,0.768349382869096)(84,0.768349382869096)(84,0.768349382869096)(84,0.768349382869096)(84,0.768349382869096)(85,0.769411860097113)(85,0.769411860097113)(85,0.769411860097113)(85,0.769411860097113)(85,0.769411860097113)(85,0.769411860097113)(85,0.769411860097113)(85,0.769411860097113)(86,0.770587572675908)(86,0.770587572675908)(86,0.770587572675908)(86,0.770587572675908)(86,0.770587572675908)(86,0.770587572675908)(86,0.770587572675908)(86,0.770587572675908)(86,0.770587572675908)(86,0.770587572675908)(86,0.770587572675908)(86,0.770587572675908)(86,0.770587572675908)(87,0.771742427461378)(87,0.771742427461378)(87,0.771742427461378)(87,0.771742427461378)(87,0.771742427461378)(87,0.771742427461378)(87,0.771742427461378)(87,0.771742427461378)(87,0.771742427461378)(87,0.771742427461378)(87,0.771742427461378)(88,0.772806286175185)(88,0.772806286175185)(88,0.772806286175185)(88,0.772806286175185)(88,0.772806286175185)(89,0.77368743679864)(89,0.77368743679864)(89,0.77368743679864)(89,0.77368743679864)(89,0.77368743679864)(89,0.77368743679864)(89,0.77368743679864)(90,0.774437304628583)(90,0.774437304628583)(91,0.774935046078453)(91,0.774935046078453)(91,0.774935046078453)(91,0.774935046078453)(91,0.774935046078453)(91,0.774935046078453)(91,0.774935046078453)(92,0.775252523245247)(92,0.775252523245247)(92,0.775252523245247)(92,0.775252523245247)(92,0.775252523245247)(92,0.775252523245247)(92,0.775252523245247)(92,0.775252523245247)(92,0.775252523245247)(92,0.775252523245247)(92,0.775252523245247)(92,0.775252523245247)(92,0.775252523245247)(92,0.775252523245247)(93,0.775517010915645)(93,0.775517010915645)(93,0.775517010915645)(93,0.775517010915645)(94,0.775793747814392)(94,0.775793747814392)(94,0.775793747814392)(94,0.775793747814392)(94,0.775793747814392)(94,0.775793747814392)(94,0.775793747814392)(94,0.775793747814392)(94,0.775793747814392)(94,0.775793747814392)(94,0.775793747814392)(94,0.775793747814392)(94,0.775793747814392)(95,0.776079709338669)(95,0.776079709338669)(95,0.776079709338669)(95,0.776079709338669)(95,0.776079709338669)(95,0.776079709338669)(95,0.776079709338669)(95,0.776079709338669)(95,0.776079709338669)(95,0.776079709338669)(95,0.776079709338669)(96,0.776287080735566)(96,0.776287080735566)(96,0.776287080735566)(97,0.776243296822905)(97,0.776243296822905)(97,0.776243296822905)(97,0.776243296822905)(97,0.776243296822905)(97,0.776243296822905)(97,0.776243296822905)(97,0.776243296822905)(98,0.776318176173035)(98,0.776318176173035)(98,0.776318176173035)(98,0.776318176173035)(98,0.776318176173035)(98,0.776318176173035)(98,0.776318176173035)(98,0.776318176173035)(98,0.776318176173035)(99,0.77636713362818)(99,0.77636713362818)(99,0.77636713362818)(99,0.77636713362818)(99,0.77636713362818)(99,0.77636713362818)(99,0.77636713362818)(99,0.77636713362818)(99,0.77636713362818)(99,0.77636713362818)(99,0.77636713362818)(100,0.776387151693633)(100,0.776387151693633)(100,0.776387151693633)(100,0.776387151693633)(100,0.776387151693633)(100,0.776387151693633)(100,0.776387151693633)(100,0.776387151693633)(101,0.776445850506844)(101,0.776445850506844)(101,0.776445850506844)(101,0.776445850506844)(101,0.776445850506844)(101,0.776445850506844)(101,0.776445850506844)(101,0.776445850506844)(101,0.776445850506844)(101,0.776445850506844)(101,0.776445850506844)(102,0.776667492264206)(102,0.776667492264206)(102,0.776667492264206)(102,0.776667492264206)(102,0.776667492264206)(103,0.777049725619106)(103,0.777049725619106)(103,0.777049725619106)(103,0.777049725619106)(103,0.777049725619106)(103,0.777049725619106)(103,0.777049725619106)(104,0.77761993714185)(104,0.77761993714185)(104,0.77761993714185)(104,0.77761993714185)(104,0.77761993714185)(105,0.778013470232384)(105,0.778013470232384)(105,0.778013470232384)(105,0.778013470232384)(105,0.778013470232384)(105,0.778013470232384)(105,0.778013470232384)(105,0.778013470232384)(106,0.778405218933141)(106,0.778405218933141)(106,0.778405218933141)(106,0.778405218933141)(106,0.778405218933141)(107,0.778920852596376)(107,0.778920852596376)(107,0.778920852596376)(107,0.778920852596376)(107,0.778920852596376)(107,0.778920852596376)(108,0.779320423964693)(108,0.779320423964693)(108,0.779320423964693)(109,0.779471512517092)(109,0.779471512517092)(109,0.779471512517092)(109,0.779471512517092)(109,0.779471512517092)(109,0.779471512517092)(109,0.779471512517092)(110,0.779819572551721)(110,0.779819572551721)(110,0.779819572551721)(110,0.779819572551721)(110,0.779819572551721)(110,0.779819572551721)(110,0.779819572551721)(110,0.779819572551721)(110,0.779819572551721)(110,0.779819572551721)(110,0.779819572551721)(110,0.779819572551721)(111,0.780201807346319)(111,0.780201807346319)(111,0.780201807346319)(111,0.780201807346319)(111,0.780201807346319)(111,0.780201807346319)(111,0.780201807346319)(111,0.780201807346319)(112,0.780652367904955)(112,0.780652367904955)(112,0.780652367904955)(112,0.780652367904955)(113,0.781178644166019)(113,0.781178644166019)(113,0.781178644166019)(113,0.781178644166019)(114,0.781704713682437)(114,0.781704713682437)(114,0.781704713682437)(114,0.781704713682437)(114,0.781704713682437)(114,0.781704713682437)(115,0.782195984210233)(115,0.782195984210233)(115,0.782195984210233)(115,0.782195984210233)(115,0.782195984210233)(115,0.782195984210233)(116,0.782651258541574)(116,0.782651258541574)(116,0.782651258541574)(116,0.782651258541574)(116,0.782651258541574)(116,0.782651258541574)(116,0.782651258541574)(116,0.782651258541574)(116,0.782651258541574)(116,0.782651258541574)(116,0.782651258541574)(116,0.782651258541574)(117,0.783106860483064)(117,0.783106860483064)(117,0.783106860483064)(117,0.783106860483064)(117,0.783106860483064)(117,0.783106860483064)(118,0.783621505694341)(118,0.783621505694341)(118,0.783621505694341)(118,0.783621505694341)(118,0.783621505694341)(118,0.783621505694341)(118,0.783621505694341)(118,0.783621505694341)(119,0.784060551149548)(119,0.784060551149548)(119,0.784060551149548)(119,0.784060551149548)(119,0.784060551149548)(119,0.784060551149548)(119,0.784060551149548)(119,0.784060551149548)(120,0.784472755851389)(120,0.784472755851389)(120,0.784472755851389)(120,0.784472755851389)(120,0.784472755851389)(120,0.784472755851389)(120,0.784472755851389)(121,0.784854613384409)(121,0.784854613384409)(121,0.784854613384409)(121,0.784854613384409)(121,0.784854613384409)(122,0.785221795143674)(122,0.785221795143674)(122,0.785221795143674)(122,0.785221795143674)(122,0.785221795143674)(123,0.785614492057036)(123,0.785614492057036)(123,0.785614492057036)(123,0.785614492057036)(123,0.785614492057036)(123,0.785614492057036)(123,0.785614492057036)(124,0.786071525600719)(124,0.786071525600719)(124,0.786071525600719)(124,0.786071525600719)(124,0.786071525600719)(124,0.786071525600719)(124,0.786071525600719)(125,0.786605124541834)(125,0.786605124541834)(125,0.786605124541834)(125,0.786605124541834)(125,0.786605124541834)(126,0.787179648126383)(126,0.787179648126383)(126,0.787179648126383)(126,0.787179648126383)(126,0.787179648126383)(126,0.787179648126383)(126,0.787179648126383)(127,0.787715194323728)(127,0.787715194323728)(127,0.787715194323728)(127,0.787715194323728)(127,0.787715194323728)(128,0.788145434155175)(128,0.788145434155175)(128,0.788145434155175)(129,0.788445136479525)(129,0.788445136479525)(129,0.788445136479525)(129,0.788445136479525)(129,0.788445136479525)(129,0.788445136479525)(129,0.788445136479525)(130,0.788633100417507)(130,0.788633100417507)(130,0.788633100417507)(131,0.788707895131752)(131,0.788707895131752)(131,0.788707895131752)(131,0.788707895131752)(131,0.788707895131752)(131,0.788707895131752)(131,0.788707895131752)(131,0.788707895131752)(131,0.788707895131752)(131,0.788707895131752)(131,0.788707895131752)(131,0.788707895131752)(132,0.788679518340727)(132,0.788679518340727)(132,0.788679518340727)(132,0.788679518340727)(132,0.788679518340727)(132,0.788679518340727)(132,0.788679518340727)(133,0.788582541107263)(133,0.788582541107263)(133,0.788582541107263)(133,0.788582541107263)(133,0.788582541107263)(133,0.788582541107263)(133,0.788582541107263)(133,0.788582541107263)(133,0.788582541107263)(134,0.788430385912743)(134,0.788430385912743)(135,0.788230335121965)(135,0.788230335121965)(135,0.788230335121965)(135,0.788230335121965)(135,0.788230335121965)(136,0.788008736850314)(136,0.788008736850314)(136,0.788008736850314)(137,0.787680850514211)(137,0.787680850514211)(137,0.787680850514211)(137,0.787680850514211)(137,0.787680850514211)(137,0.787680850514211)(137,0.787680850514211)(137,0.787680850514211)(137,0.787680850514211)(137,0.787680850514211)(137,0.787680850514211)(138,0.787607227218405)(138,0.787607227218405)(138,0.787607227218405)(138,0.787607227218405)(138,0.787607227218405)(138,0.787607227218405)(138,0.787607227218405)(138,0.787607227218405)(139,0.787461977472303)(139,0.787461977472303)(140,0.78745416367541)(140,0.78745416367541)(140,0.78745416367541)(140,0.78745416367541)(140,0.78745416367541)(140,0.78745416367541)(140,0.78745416367541)(140,0.78745416367541)(140,0.78745416367541)(141,0.78749296631432)(141,0.78749296631432)(142,0.78768693317891)(142,0.78768693317891)(142,0.78768693317891)(142,0.78768693317891)(142,0.78768693317891)(142,0.78768693317891)(142,0.78768693317891)(142,0.78768693317891)(143,0.787786796540107)(143,0.787786796540107)(143,0.787786796540107)(143,0.787786796540107)(143,0.787786796540107)(144,0.787909240204767)(144,0.787909240204767)(144,0.787909240204767)(144,0.787909240204767)(144,0.787909240204767)(144,0.787909240204767)(144,0.787909240204767)(144,0.787909240204767)(145,0.787938562521599)(145,0.787938562521599)(145,0.787938562521599)(145,0.787938562521599)(145,0.787938562521599)(145,0.787938562521599)(146,0.788313406411547)(146,0.788313406411547)(146,0.788313406411547)(146,0.788313406411547)(146,0.788313406411547)(146,0.788313406411547)(146,0.788313406411547)(147,0.788607481062003)(147,0.788607481062003)(147,0.788607481062003)(147,0.788607481062003)(147,0.788607481062003)(147,0.788607481062003)(148,0.788956594755567)(148,0.788956594755567)(148,0.788956594755567)(148,0.788956594755567)(148,0.788956594755567)(148,0.788956594755567)(148,0.788956594755567)(148,0.788956594755567)(149,0.789359747215664)(149,0.789359747215664)(149,0.789359747215664)(149,0.789359747215664)(150,0.789818961295827)(150,0.789818961295827)(150,0.789818961295827)(151,0.790322302032166)(151,0.790322302032166)(151,0.790322302032166)(151,0.790322302032166)(151,0.790322302032166)(151,0.790322302032166)(151,0.790322302032166)(151,0.790322302032166)(151,0.790322302032166)(152,0.790861136971882)(152,0.790861136971882)(152,0.790861136971882)(152,0.790861136971882)(152,0.790861136971882)(152,0.790861136971882)(152,0.790861136971882)(152,0.790861136971882)(152,0.790861136971882)(152,0.790861136971882)(152,0.790861136971882)(153,0.791326694103251)(153,0.791326694103251)(153,0.791326694103251)(154,0.791863743052031)(154,0.791863743052031)(154,0.791863743052031)(155,0.79239205701665)(155,0.79239205701665)(155,0.79239205701665)(155,0.79239205701665)(155,0.79239205701665)(155,0.79239205701665)(155,0.79239205701665)(155,0.79239205701665)(155,0.79239205701665)(156,0.792692422229209)(156,0.792692422229209)(156,0.792692422229209)(156,0.792692422229209)(157,0.793263264504294)(157,0.793263264504294)(157,0.793263264504294)(157,0.793263264504294)(157,0.793263264504294)(157,0.793263264504294)(157,0.793263264504294)(157,0.793263264504294)(157,0.793263264504294)(157,0.793263264504294)(157,0.793263264504294)(158,0.793414916687417)(158,0.793414916687417)(158,0.793414916687417)(158,0.793414916687417)(159,0.793619321120899)(159,0.793619321120899)(159,0.793619321120899)(159,0.793619321120899)(159,0.793619321120899)(159,0.793619321120899)(160,0.793706841885838)(161,0.793679586881133)(161,0.793679586881133)(161,0.793679586881133)(161,0.793679586881133)(161,0.793679586881133)(162,0.79363154826575)(162,0.79363154826575)(163,0.793534876822427)(163,0.793534876822427)(163,0.793534876822427)(163,0.793534876822427)(163,0.793534876822427)(164,0.793380718306483)(164,0.793380718306483)(164,0.793380718306483)(164,0.793380718306483)(164,0.793380718306483)(164,0.793380718306483)(164,0.793380718306483)(165,0.793205605203157)(165,0.793205605203157)(165,0.793205605203157)(165,0.793205605203157)(165,0.793205605203157)(165,0.793205605203157)(165,0.793205605203157)(165,0.793205605203157)(165,0.793205605203157)(166,0.793010831991746)(166,0.793010831991746)(166,0.793010831991746)(167,0.792809956476032)(168,0.792621279257884)(168,0.792621279257884)(168,0.792621279257884)(168,0.792621279257884)(168,0.792621279257884)(168,0.792621279257884)(168,0.792621279257884)(169,0.792467215524671)(169,0.792467215524671)(169,0.792467215524671)(169,0.792467215524671)(169,0.792467215524671)(169,0.792467215524671)(169,0.792467215524671)(170,0.792355535795305)(170,0.792355535795305)(170,0.792355535795305)(171,0.792282684481942)(171,0.792282684481942)(172,0.792242533799237)(172,0.792242533799237)(172,0.792242533799237)(172,0.792242533799237)(173,0.792229137423507)(173,0.792229137423507)(173,0.792229137423507)(174,0.792234940631804)(174,0.792234940631804)(174,0.792234940631804)(174,0.792234940631804)(174,0.792234940631804)(175,0.792259805822113)(175,0.792259805822113)(175,0.792259805822113)(175,0.792259805822113)(175,0.792259805822113)(175,0.792259805822113)(175,0.792259805822113)(176,0.792313276781714)(176,0.792313276781714)(176,0.792313276781714)(176,0.792313276781714)(177,0.792514946054491)(177,0.792514946054491)(177,0.792514946054491)(177,0.792514946054491)(178,0.792656857648136)(178,0.792656857648136)(178,0.792656857648136)(178,0.792656857648136)(179,0.792851965748149)(179,0.792851965748149)(179,0.792851965748149)(179,0.792851965748149)(179,0.792851965748149)(179,0.792851965748149)(180,0.793104080893804)(180,0.793104080893804)(180,0.793104080893804)(181,0.793406391960751)(181,0.793406391960751)(182,0.793742812849333)(182,0.793742812849333)(182,0.793742812849333)(183,0.794090894118941)(183,0.794090894118941)(183,0.794090894118941)(183,0.794090894118941)(183,0.794090894118941)(183,0.794090894118941)(184,0.794434912363909)(185,0.794732205000391)(185,0.794732205000391)(185,0.794732205000391)(186,0.795049579889292)(186,0.795049579889292)(186,0.795049579889292)(186,0.795049579889292)(187,0.795360600924138)(187,0.795360600924138)(188,0.795666600665873)(188,0.795666600665873)(188,0.795666600665873)(188,0.795666600665873)(188,0.795666600665873)(188,0.795666600665873)(188,0.795666600665873)(188,0.795666600665873)(188,0.795666600665873)(189,0.796013257788411)(189,0.796013257788411)(189,0.796013257788411)(189,0.796013257788411)(189,0.796013257788411)(190,0.79629718322374)(190,0.79629718322374)(190,0.79629718322374)(190,0.79629718322374)(190,0.79629718322374)(190,0.79629718322374)(190,0.79629718322374)(190,0.79629718322374)(191,0.79656264149465)(191,0.79656264149465)(191,0.79656264149465)(191,0.79656264149465)(191,0.79656264149465)(191,0.79656264149465)(192,0.796855698533092)(192,0.796855698533092)(192,0.796855698533092)(192,0.796855698533092)(192,0.796855698533092)(192,0.796855698533092)(193,0.797141813471621)(193,0.797141813471621)(193,0.797141813471621)(193,0.797141813471621)(193,0.797141813471621)(193,0.797141813471621)(194,0.797412877237889)(194,0.797412877237889)(194,0.797412877237889)(194,0.797412877237889)(194,0.797412877237889)(194,0.797412877237889)(194,0.797412877237889)(195,0.797664030856361)(195,0.797664030856361)(195,0.797664030856361)(195,0.797664030856361)(195,0.797664030856361)(195,0.797664030856361)(196,0.797898044577483)(196,0.797898044577483)(196,0.797898044577483)(196,0.797898044577483)(196,0.797898044577483)(196,0.797898044577483)(197,0.798100445123162)(197,0.798100445123162)(197,0.798100445123162)(197,0.798100445123162)(198,0.798280919270368)(198,0.798280919270368)(198,0.798280919270368)(198,0.798280919270368)(198,0.798280919270368)(198,0.798280919270368)(198,0.798280919270368)(198,0.798280919270368)(199,0.798432279672514)(199,0.798432279672514)(199,0.798432279672514)(199,0.798432279672514)(200,0.798553948412643)(200,0.798553948412643)(200,0.798553948412643)(200,0.798553948412643)(200,0.798553948412643)(200,0.798553948412643)(201,0.798649403621924)(201,0.798649403621924)(201,0.798649403621924)(202,0.798726258113607)(202,0.798726258113607)(202,0.798726258113607)(202,0.798726258113607)(202,0.798726258113607)(203,0.798788829500065)(203,0.798788829500065)(203,0.798788829500065)(203,0.798788829500065)(203,0.798788829500065)(203,0.798788829500065)(204,0.798838499403406)(204,0.798838499403406)(205,0.798854616374007)(205,0.798854616374007)(205,0.798854616374007)(205,0.798854616374007)(205,0.798854616374007)(206,0.798881556645547)(206,0.798881556645547)(206,0.798881556645547)(206,0.798881556645547)(206,0.798881556645547)(206,0.798881556645547)(206,0.798881556645547)(206,0.798881556645547)(206,0.798881556645547)(206,0.798881556645547)(206,0.798881556645547)(207,0.79888263669341)(207,0.79888263669341)(207,0.79888263669341)(207,0.79888263669341)(208,0.798897539447127)(208,0.798897539447127)(208,0.798897539447127)(209,0.798929724909657)(209,0.798929724909657)(209,0.798929724909657)(209,0.798929724909657)(210,0.79894426228479)(211,0.798973593425598)(211,0.798973593425598)(211,0.798973593425598)(212,0.798978368275439)(212,0.798978368275439)(213,0.798994054025001)(213,0.798994054025001)(213,0.798994054025001)(213,0.798994054025001)(213,0.798994054025001)(213,0.798994054025001)(213,0.798994054025001)(214,0.799007391283541)(214,0.799007391283541)(214,0.799007391283541)(214,0.799007391283541)(214,0.799007391283541)(214,0.799007391283541)(215,0.799018425384983)(215,0.799018425384983)(215,0.799018425384983)(215,0.799018425384983)(215,0.799018425384983)(215,0.799018425384983)(215,0.799018425384983)(216,0.799027898668864)(216,0.799027898668864)(217,0.799035939755435)(217,0.799035939755435)(217,0.799035939755435)(217,0.799035939755435)(217,0.799035939755435)(218,0.799078396699528)(218,0.799078396699528)(218,0.799078396699528)(218,0.799078396699528)(219,0.799086592012797)(219,0.799086592012797)(219,0.799086592012797)(220,0.799125666410174)(220,0.799125666410174)(220,0.799125666410174)(220,0.799125666410174)(220,0.799125666410174)(220,0.799125666410174)(220,0.799125666410174)(221,0.799133911777983)(221,0.799133911777983)(221,0.799133911777983)(221,0.799133911777983)(222,0.799146754226599)(223,0.799191442866421)(223,0.799191442866421)(223,0.799191442866421)(223,0.799191442866421)(223,0.799191442866421)(223,0.799191442866421)(223,0.799191442866421)(224,0.79922118009406)(224,0.79922118009406)(224,0.79922118009406)(224,0.79922118009406)(224,0.79922118009406)(225,0.799261000640842)(225,0.799261000640842)(226,0.799331268118522)(226,0.799331268118522)(226,0.799331268118522)(227,0.799391882720865)(227,0.799391882720865)(227,0.799391882720865)(227,0.799391882720865)(227,0.799391882720865)(227,0.799391882720865)(228,0.79947193107907)(228,0.79947193107907)(228,0.79947193107907)(228,0.79947193107907)(228,0.79947193107907)(229,0.799573737565038)(229,0.799573737565038)(230,0.799701771485071)(230,0.799701771485071)(230,0.799701771485071)(232,0.799993181300011)(233,0.800148344562103)(233,0.800148344562103)(233,0.800148344562103)(233,0.800148344562103)(233,0.800148344562103)(234,0.800293566295869)(234,0.800293566295869)(234,0.800293566295869)(235,0.800420130102874)(235,0.800420130102874)(235,0.800420130102874)(236,0.800524353404515)(236,0.800524353404515)(236,0.800524353404515)(236,0.800524353404515)(236,0.800524353404515)(236,0.800524353404515)(237,0.800606061699773)(237,0.800606061699773)(237,0.800606061699773)(237,0.800606061699773)(237,0.800606061699773)(237,0.800606061699773)(237,0.800606061699773)(238,0.800670015321568)(239,0.80069962017014)(239,0.80069962017014)(239,0.80069962017014)(240,0.800756724310219)(240,0.800756724310219)(240,0.800756724310219)(241,0.800813972270787)(241,0.800813972270787)(241,0.800813972270787)(241,0.800813972270787)(241,0.800813972270787)(241,0.800813972270787)(241,0.800813972270787)(242,0.800863161981167)(242,0.800863161981167)(242,0.800863161981167)(242,0.800863161981167)(243,0.800915492484444)(243,0.800915492484444)(243,0.800915492484444)(243,0.800915492484444)(243,0.800915492484444)(244,0.800975178099207)(244,0.800975178099207)(245,0.801044356847606)(245,0.801044356847606)(245,0.801044356847606)(245,0.801044356847606)(246,0.801123266902873)(246,0.801123266902873)(247,0.801212684577317)(247,0.801212684577317)(248,0.801330531114208)(248,0.801330531114208)(249,0.801431076492086)(249,0.801431076492086)(249,0.801431076492086)(249,0.801431076492086)(251,0.801644634718359)(251,0.801644634718359)(251,0.801644634718359)(251,0.801644634718359)(251,0.801644634718359)(252,0.801758021898403)(252,0.801758021898403)(252,0.801758021898403)(252,0.801758021898403)(253,0.8018754074391)(253,0.8018754074391)(253,0.8018754074391)(254,0.80201263823681)(255,0.802135020358374)(255,0.802135020358374)(255,0.802135020358374)(255,0.802135020358374)(256,0.802254021783483)(256,0.802254021783483)(256,0.802254021783483)(257,0.802365752012504)(257,0.802365752012504)(257,0.802365752012504)(257,0.802365752012504)(258,0.802466536449589)(258,0.802466536449589)(258,0.802466536449589)(258,0.802466536449589)(258,0.802466536449589)(259,0.802550805889554)(259,0.802550805889554)(259,0.802550805889554)(260,0.802613217578847)(260,0.802613217578847)(260,0.802613217578847)(260,0.802613217578847)(261,0.802648697242833)(261,0.802648697242833)(261,0.802648697242833)(261,0.802648697242833)(262,0.802614810097337)(262,0.802614810097337)(262,0.802614810097337)(262,0.802614810097337)(262,0.802614810097337)(263,0.802596100741603)(263,0.802596100741603)(264,0.80256481529864)(264,0.80256481529864)(264,0.80256481529864)(264,0.80256481529864)(264,0.80256481529864)(265,0.802526590333386)(265,0.802526590333386)(265,0.802526590333386)(265,0.802526590333386)(266,0.80252444660568)(266,0.80252444660568)(267,0.802478520961154)(267,0.802478520961154)(267,0.802478520961154)(268,0.802471734618046)(268,0.802471734618046)(268,0.802471734618046)(268,0.802471734618046)(268,0.802471734618046)(268,0.802471734618046)(268,0.802471734618046)(269,0.802399200383226)(269,0.802399200383226)(269,0.802399200383226)(270,0.8023705902911)(270,0.8023705902911)(270,0.8023705902911)(270,0.8023705902911)(270,0.8023705902911)(271,0.802350258948089)(271,0.802350258948089)(271,0.802350258948089)(271,0.802350258948089)(272,0.802346355424262)(272,0.802346355424262)(272,0.802346355424262)(273,0.80236043786121)(273,0.80236043786121)(273,0.80236043786121)(273,0.80236043786121)(273,0.80236043786121)(273,0.80236043786121)(274,0.802352307978845)(274,0.802352307978845)(274,0.802352307978845)(274,0.802352307978845)(274,0.802352307978845)(275,0.802361300710197)(276,0.802372131500644)(276,0.802372131500644)(276,0.802372131500644)(276,0.802372131500644)(276,0.802372131500644)(277,0.802400915165814)(277,0.802400915165814)(277,0.802400915165814)(277,0.802400915165814)(278,0.802410007936543)(278,0.802410007936543)(278,0.802410007936543)(278,0.802410007936543)(278,0.802410007936543)(278,0.802410007936543)(279,0.802416996190269)(279,0.802416996190269)(279,0.802416996190269)(280,0.802440839612253)(280,0.802440839612253)(280,0.802440839612253)(281,0.802440062247706)(281,0.802440062247706)(281,0.802440062247706)(281,0.802440062247706)(281,0.802440062247706)(282,0.802432368199326)(282,0.802432368199326)(282,0.802432368199326)(282,0.802432368199326)(283,0.802399584659817)(283,0.802399584659817)(284,0.802380980512759)(284,0.802380980512759)(284,0.802380980512759)(285,0.802338303690725)(285,0.802338303690725)(285,0.802338303690725)(286,0.802316324355453)(286,0.802316324355453)(287,0.802296646919315)(287,0.802296646919315)(287,0.802296646919315)(287,0.802296646919315)(287,0.802296646919315)(288,0.802284625874267)(289,0.80230118574595)(289,0.80230118574595)(289,0.80230118574595)(290,0.802303961709259)(290,0.802303961709259)(291,0.802339730150709)(291,0.802339730150709)(291,0.802339730150709)(291,0.802339730150709)(291,0.802339730150709)(291,0.802339730150709)(292,0.802391566358652)(292,0.802391566358652)(293,0.802455707604005)(293,0.802455707604005)(293,0.802455707604005)(293,0.802455707604005)(293,0.802455707604005)(294,0.802529453172727)(294,0.802529453172727)(294,0.802529453172727)(295,0.802609083493826)(295,0.802609083493826)(295,0.802609083493826)(296,0.802690667526869)(296,0.802690667526869)(296,0.802690667526869)(297,0.80277010115992)(297,0.80277010115992)(297,0.80277010115992)(297,0.80277010115992)(298,0.802843868562403)(299,0.802894252005454)(300,0.802954887297598)(300,0.802954887297598)(300,0.802954887297598)(300,0.802954887297598)(300,0.802954887297598)(300,0.802954887297598)(301,0.803009541343192)(301,0.803009541343192)(301,0.803009541343192)(301,0.803009541343192)(301,0.803009541343192)(301,0.803009541343192)(301,0.803009541343192)(301,0.803009541343192)(302,0.803084828668635)(302,0.803084828668635)(303,0.803104422171227)(303,0.803104422171227)(303,0.803104422171227)(303,0.803104422171227)(304,0.80314541003463)(304,0.80314541003463)(304,0.80314541003463)(305,0.803196401872749)(306,0.803215235461359)(306,0.803215235461359)(306,0.803215235461359)(308,0.803244724101007)(309,0.803286850783199)(309,0.803286850783199)(309,0.803286850783199)(309,0.803286850783199)(310,0.803308277515032)(310,0.803308277515032)(310,0.803308277515032)(310,0.803308277515032)(311,0.803278331068392)(311,0.803278331068392)(312,0.80329945753789)(312,0.80329945753789)(312,0.80329945753789)(312,0.80329945753789)(312,0.80329945753789)(313,0.803328468263958)(313,0.803328468263958)(313,0.803328468263958)(314,0.803368582025939)(315,0.803379887201525)(315,0.803379887201525)(316,0.803485499772746)(316,0.803485499772746)(317,0.803558549761424)(317,0.803558549761424)(317,0.803558549761424)(318,0.803635884651501)(318,0.803635884651501)(318,0.803635884651501)(318,0.803635884651501)(319,0.803714281051675)(319,0.803714281051675)(319,0.803714281051675)(319,0.803714281051675)(319,0.803714281051675)(319,0.803714281051675)(320,0.803790671660125)(320,0.803790671660125)(320,0.803790671660125)(320,0.803790671660125)(320,0.803790671660125)(321,0.803861740626074)(321,0.803861740626074)(321,0.803861740626074)(321,0.803861740626074)(322,0.80392379341322)(322,0.80392379341322)(322,0.80392379341322)(323,0.803974229956528)(323,0.803974229956528)(323,0.803974229956528)(323,0.803974229956528)(325,0.804031640143025)(325,0.804031640143025)(325,0.804031640143025)(325,0.804031640143025)(325,0.804031640143025)(326,0.804034185762234)(326,0.804034185762234)(326,0.804034185762234)(327,0.804036211866176)(327,0.804036211866176)(327,0.804036211866176)(328,0.804002646681645)(328,0.804002646681645)(328,0.804002646681645)(329,0.80395707087261)(329,0.80395707087261)(329,0.80395707087261)(329,0.80395707087261)(330,0.803901010477671)(330,0.803901010477671)(330,0.803901010477671)(330,0.803901010477671)(331,0.803849145794803)(331,0.803849145794803)(331,0.803849145794803)(331,0.803849145794803)(332,0.803777824755172)(332,0.803777824755172)(332,0.803777824755172)(332,0.803777824755172)(332,0.803777824755172)(332,0.803777824755172)(333,0.803703571960798)(333,0.803703571960798)(335,0.80354742342126)(335,0.80354742342126)(336,0.8034801012896)(336,0.8034801012896)(336,0.8034801012896)(336,0.8034801012896)(336,0.8034801012896)(337,0.803414560736037)(337,0.803414560736037)(337,0.803414560736037)(337,0.803414560736037)(337,0.803414560736037)(337,0.803414560736037)(338,0.803350856237132)(338,0.803350856237132)(339,0.803289308930685)(339,0.803289308930685)(339,0.803289308930685)(339,0.803289308930685)(339,0.803289308930685)(339,0.803289308930685)(340,0.803229618608601)(340,0.803229618608601)(340,0.803229618608601)(340,0.803229618608601)(340,0.803229618608601)(341,0.803162663998866)(341,0.803162663998866)(342,0.803103744401253)(342,0.803103744401253)(343,0.803042943757914)(343,0.803042943757914)(343,0.803042943757914)(343,0.803042943757914)(343,0.803042943757914)(343,0.803042943757914)(343,0.803042943757914)(343,0.803042943757914)(343,0.803042943757914)(344,0.802978288937618)(344,0.802978288937618)(344,0.802978288937618)(345,0.802907442574248)(345,0.802907442574248)(345,0.802907442574248)(345,0.802907442574248)(345,0.802907442574248)(346,0.802828811410819)(346,0.802828811410819)(346,0.802828811410819)(346,0.802828811410819)(346,0.802828811410819)(346,0.802828811410819)(346,0.802828811410819)(347,0.802742363922219)(347,0.802742363922219)(347,0.802742363922219)(347,0.802742363922219)(348,0.802649434142326)(348,0.802649434142326)(348,0.802649434142326)(348,0.802649434142326)(348,0.802649434142326)(348,0.802649434142326)(349,0.802538681081453)(349,0.802538681081453)(349,0.802538681081453)(349,0.802538681081453)(349,0.802538681081453)(349,0.802538681081453)(349,0.802538681081453)(350,0.80245867250033)(351,0.802368440498836)(351,0.802368440498836)(351,0.802368440498836)(351,0.802368440498836)(352,0.802286728277505)(352,0.802286728277505)(352,0.802286728277505)(352,0.802286728277505)(353,0.802225475574613)(353,0.802225475574613)(354,0.802158160505247)(354,0.802158160505247)(354,0.802158160505247)(354,0.802158160505247)(355,0.802110394421949)(355,0.802110394421949)(355,0.802110394421949)(356,0.802061657366213)(356,0.802061657366213)(357,0.802015946586481)(357,0.802015946586481)(357,0.802015946586481)(358,0.801973684571636)(358,0.801973684571636)(358,0.801973684571636)(359,0.801935572890254)(359,0.801935572890254)(359,0.801935572890254)(360,0.801900976049128)(360,0.801900976049128)(360,0.801900976049128)(360,0.801900976049128)(360,0.801900976049128)(361,0.801869553934012)(361,0.801869553934012)(361,0.801869553934012)(361,0.801869553934012)(361,0.801869553934012)(361,0.801869553934012)(361,0.801869553934012)(361,0.801869553934012)(361,0.801869553934012)(362,0.801838573086881)(363,0.801806376234952)(363,0.801806376234952)(363,0.801806376234952)(363,0.801806376234952)(363,0.801806376234952)(364,0.801773625162822)(364,0.801773625162822)(364,0.801773625162822)(364,0.801773625162822)(364,0.801773625162822)(364,0.801773625162822)(364,0.801773625162822)(365,0.801742619112101)(365,0.801742619112101)(366,0.801716840426105)(366,0.801716840426105)(366,0.801716840426105)(367,0.801708428662841)(367,0.801708428662841)(367,0.801708428662841)(367,0.801708428662841)(368,0.801718000998296)(368,0.801718000998296)(368,0.801718000998296)(368,0.801718000998296)(369,0.801738861519277)(369,0.801738861519277)(369,0.801738861519277)(371,0.801800668134917)(371,0.801800668134917)(371,0.801800668134917)(371,0.801800668134917)(371,0.801800668134917)(372,0.801838939025976)(372,0.801838939025976)(372,0.801838939025976)(373,0.801881645938667)(373,0.801881645938667)(373,0.801881645938667)(373,0.801881645938667)(374,0.801928830928355)(374,0.801928830928355)(374,0.801928830928355)(375,0.801980619903959)(375,0.801980619903959)(376,0.802037108172413)(376,0.802037108172413)(376,0.802037108172413)(377,0.802098397083482)(377,0.802098397083482)(377,0.802098397083482)(377,0.802098397083482)(377,0.802098397083482)(377,0.802098397083482)(378,0.802164556060323)(379,0.80223571344915)(379,0.80223571344915)(380,0.80231200309815)(380,0.80231200309815)(380,0.80231200309815)(380,0.80231200309815)(381,0.802393523982474)(381,0.802393523982474)(381,0.802393523982474)(381,0.802393523982474)(383,0.802572614227347)(383,0.802572614227347)(384,0.802670351740494)(384,0.802670351740494)(384,0.802670351740494)(384,0.802670351740494)(385,0.802773687299603)(385,0.802773687299603)(385,0.802773687299603)(386,0.802882743020236)(386,0.802882743020236)(386,0.802882743020236)(387,0.80299761459963)(387,0.80299761459963)(387,0.80299761459963)(387,0.80299761459963)(387,0.80299761459963)(387,0.80299761459963)(388,0.803118363443638)(389,0.803245007770237)(390,0.803377530557985)(390,0.803377530557985)(390,0.803377530557985)(391,0.803515894019132)(391,0.803515894019132)(391,0.803515894019132)(391,0.803515894019132)(391,0.803515894019132)(392,0.80366006775728)(392,0.80366006775728)(392,0.80366006775728)(392,0.80366006775728)(392,0.80366006775728)(394,0.803965716959713)(394,0.803965716959713)(394,0.803965716959713)(395,0.804127062517898)(395,0.804127062517898)(396,0.804293964621)(396,0.804293964621)(396,0.804293964621)(396,0.804293964621)(396,0.804293964621)(396,0.804293964621)(397,0.804466312340562)(397,0.804466312340562)(398,0.804644002910275)(398,0.804644002910275)(398,0.804644002910275)(398,0.804644002910275)(398,0.804644002910275)(399,0.804826935469178)(399,0.804826935469178)(399,0.804826935469178)(399,0.804826935469178)(400,0.805015070574332)(400,0.805015070574332)(400,0.805015070574332) 
};
\addlegendentry{\acl (max. ambiguity)};

\addplot [
color=red,
solid,
line width=1.0pt,
]
coordinates{
 (9,0.525205645713756)(12,0.602636701849244)(12,0.602636701849244)(12,0.602636701849244)(13,0.623171629308652)(14,0.641166664923774)(14,0.641166664923774)(14,0.641166664923774)(14,0.641166664923774)(14,0.641166664923774)(14,0.641166664923774)(14,0.641166664923774)(14,0.641166664923774)(14,0.641166664923774)(14,0.641166664923774)(14,0.641166664923774)(14,0.641166664923774)(15,0.656614023748866)(15,0.656614023748866)(15,0.656614023748866)(15,0.656614023748866)(16,0.669431916986052)(16,0.669431916986052)(16,0.669431916986052)(16,0.669431916986052)(17,0.679436753771893)(17,0.679436753771893)(17,0.679436753771893)(17,0.679436753771893)(17,0.679436753771893)(17,0.679436753771893)(17,0.679436753771893)(17,0.679436753771893)(17,0.679436753771893)(17,0.679436753771893)(18,0.686416339996961)(18,0.686416339996961)(18,0.686416339996961)(18,0.686416339996961)(18,0.686416339996961)(18,0.686416339996961)(18,0.686416339996961)(18,0.686416339996961)(18,0.686416339996961)(19,0.690800383634439)(19,0.690800383634439)(19,0.690800383634439)(19,0.690800383634439)(19,0.690800383634439)(19,0.690800383634439)(20,0.693745259578842)(20,0.693745259578842)(20,0.693745259578842)(20,0.693745259578842)(20,0.693745259578842)(20,0.693745259578842)(20,0.693745259578842)(20,0.693745259578842)(20,0.693745259578842)(20,0.693745259578842)(20,0.693745259578842)(20,0.693745259578842)(21,0.695192247059377)(21,0.695192247059377)(21,0.695192247059377)(21,0.695192247059377)(21,0.695192247059377)(21,0.695192247059377)(21,0.695192247059377)(21,0.695192247059377)(21,0.695192247059377)(21,0.695192247059377)(21,0.695192247059377)(21,0.695192247059377)(21,0.695192247059377)(22,0.694366275871842)(22,0.694366275871842)(22,0.694366275871842)(22,0.694366275871842)(22,0.694366275871842)(22,0.694366275871842)(22,0.694366275871842)(22,0.694366275871842)(22,0.694366275871842)(22,0.694366275871842)(22,0.694366275871842)(23,0.69291719903369)(23,0.69291719903369)(23,0.69291719903369)(23,0.69291719903369)(23,0.69291719903369)(23,0.69291719903369)(23,0.69291719903369)(23,0.69291719903369)(23,0.69291719903369)(23,0.69291719903369)(23,0.69291719903369)(23,0.69291719903369)(23,0.69291719903369)(23,0.69291719903369)(24,0.692033433562984)(24,0.692033433562984)(24,0.692033433562984)(24,0.692033433562984)(24,0.692033433562984)(24,0.692033433562984)(24,0.692033433562984)(24,0.692033433562984)(24,0.692033433562984)(24,0.692033433562984)(24,0.692033433562984)(24,0.692033433562984)(24,0.692033433562984)(24,0.692033433562984)(24,0.692033433562984)(25,0.691394532474178)(25,0.691394532474178)(25,0.691394532474178)(25,0.691394532474178)(25,0.691394532474178)(25,0.691394532474178)(25,0.691394532474178)(25,0.691394532474178)(25,0.691394532474178)(25,0.691394532474178)(25,0.691394532474178)(25,0.691394532474178)(25,0.691394532474178)(25,0.691394532474178)(25,0.691394532474178)(25,0.691394532474178)(25,0.691394532474178)(25,0.691394532474178)(25,0.691394532474178)(25,0.691394532474178)(25,0.691394532474178)(25,0.691394532474178)(25,0.691394532474178)(25,0.691394532474178)(25,0.691394532474178)(26,0.691168409259131)(26,0.691168409259131)(26,0.691168409259131)(26,0.691168409259131)(26,0.691168409259131)(26,0.691168409259131)(26,0.691168409259131)(26,0.691168409259131)(26,0.691168409259131)(26,0.691168409259131)(26,0.691168409259131)(26,0.691168409259131)(26,0.691168409259131)(26,0.691168409259131)(26,0.691168409259131)(26,0.691168409259131)(26,0.691168409259131)(26,0.691168409259131)(26,0.691168409259131)(26,0.691168409259131)(26,0.691168409259131)(26,0.691168409259131)(26,0.691168409259131)(26,0.691168409259131)(27,0.692037627976509)(27,0.692037627976509)(27,0.692037627976509)(27,0.692037627976509)(27,0.692037627976509)(27,0.692037627976509)(27,0.692037627976509)(27,0.692037627976509)(27,0.692037627976509)(27,0.692037627976509)(27,0.692037627976509)(27,0.692037627976509)(27,0.692037627976509)(27,0.692037627976509)(27,0.692037627976509)(27,0.692037627976509)(27,0.692037627976509)(27,0.692037627976509)(27,0.692037627976509)(27,0.692037627976509)(27,0.692037627976509)(27,0.692037627976509)(28,0.694127431609804)(28,0.694127431609804)(28,0.694127431609804)(28,0.694127431609804)(28,0.694127431609804)(28,0.694127431609804)(28,0.694127431609804)(28,0.694127431609804)(28,0.694127431609804)(28,0.694127431609804)(28,0.694127431609804)(28,0.694127431609804)(28,0.694127431609804)(28,0.694127431609804)(28,0.694127431609804)(29,0.697219436557152)(29,0.697219436557152)(29,0.697219436557152)(29,0.697219436557152)(29,0.697219436557152)(29,0.697219436557152)(29,0.697219436557152)(29,0.697219436557152)(29,0.697219436557152)(29,0.697219436557152)(29,0.697219436557152)(30,0.698399393752976)(30,0.698399393752976)(30,0.698399393752976)(30,0.698399393752976)(30,0.698399393752976)(30,0.698399393752976)(30,0.698399393752976)(30,0.698399393752976)(30,0.698399393752976)(30,0.698399393752976)(30,0.698399393752976)(30,0.698399393752976)(30,0.698399393752976)(31,0.698289608723415)(31,0.698289608723415)(31,0.698289608723415)(31,0.698289608723415)(31,0.698289608723415)(31,0.698289608723415)(31,0.698289608723415)(31,0.698289608723415)(31,0.698289608723415)(31,0.698289608723415)(31,0.698289608723415)(31,0.698289608723415)(31,0.698289608723415)(32,0.699115490510431)(32,0.699115490510431)(32,0.699115490510431)(32,0.699115490510431)(32,0.699115490510431)(32,0.699115490510431)(32,0.699115490510431)(32,0.699115490510431)(32,0.699115490510431)(32,0.699115490510431)(32,0.699115490510431)(32,0.699115490510431)(32,0.699115490510431)(32,0.699115490510431)(32,0.699115490510431)(32,0.699115490510431)(32,0.699115490510431)(32,0.699115490510431)(32,0.699115490510431)(32,0.699115490510431)(32,0.699115490510431)(32,0.699115490510431)(32,0.699115490510431)(32,0.699115490510431)(33,0.701454321331067)(33,0.701454321331067)(33,0.701454321331067)(33,0.701454321331067)(33,0.701454321331067)(33,0.701454321331067)(33,0.701454321331067)(33,0.701454321331067)(33,0.701454321331067)(33,0.701454321331067)(33,0.701454321331067)(33,0.701454321331067)(33,0.701454321331067)(33,0.701454321331067)(33,0.701454321331067)(33,0.701454321331067)(33,0.701454321331067)(33,0.701454321331067)(33,0.701454321331067)(33,0.701454321331067)(34,0.704966051695877)(34,0.704966051695877)(34,0.704966051695877)(34,0.704966051695877)(34,0.704966051695877)(34,0.704966051695877)(34,0.704966051695877)(34,0.704966051695877)(34,0.704966051695877)(34,0.704966051695877)(34,0.704966051695877)(34,0.704966051695877)(34,0.704966051695877)(34,0.704966051695877)(35,0.70827282330898)(35,0.70827282330898)(35,0.70827282330898)(35,0.70827282330898)(35,0.70827282330898)(35,0.70827282330898)(35,0.70827282330898)(35,0.70827282330898)(35,0.70827282330898)(35,0.70827282330898)(35,0.70827282330898)(35,0.70827282330898)(35,0.70827282330898)(35,0.70827282330898)(35,0.70827282330898)(35,0.70827282330898)(35,0.70827282330898)(35,0.70827282330898)(35,0.70827282330898)(35,0.70827282330898)(35,0.70827282330898)(36,0.712500441976689)(36,0.712500441976689)(36,0.712500441976689)(36,0.712500441976689)(36,0.712500441976689)(36,0.712500441976689)(36,0.712500441976689)(36,0.712500441976689)(36,0.712500441976689)(36,0.712500441976689)(36,0.712500441976689)(36,0.712500441976689)(36,0.712500441976689)(36,0.712500441976689)(36,0.712500441976689)(36,0.712500441976689)(36,0.712500441976689)(36,0.712500441976689)(36,0.712500441976689)(36,0.712500441976689)(37,0.715603774998771)(37,0.715603774998771)(37,0.715603774998771)(37,0.715603774998771)(37,0.715603774998771)(37,0.715603774998771)(37,0.715603774998771)(37,0.715603774998771)(37,0.715603774998771)(37,0.715603774998771)(37,0.715603774998771)(37,0.715603774998771)(37,0.715603774998771)(37,0.715603774998771)(37,0.715603774998771)(37,0.715603774998771)(37,0.715603774998771)(37,0.715603774998771)(37,0.715603774998771)(37,0.715603774998771)(37,0.715603774998771)(37,0.715603774998771)(37,0.715603774998771)(37,0.715603774998771)(37,0.715603774998771)(38,0.717336390754434)(38,0.717336390754434)(38,0.717336390754434)(38,0.717336390754434)(38,0.717336390754434)(38,0.717336390754434)(38,0.717336390754434)(38,0.717336390754434)(38,0.717336390754434)(38,0.717336390754434)(38,0.717336390754434)(38,0.717336390754434)(38,0.717336390754434)(38,0.717336390754434)(38,0.717336390754434)(38,0.717336390754434)(38,0.717336390754434)(38,0.717336390754434)(39,0.718264386222346)(39,0.718264386222346)(39,0.718264386222346)(39,0.718264386222346)(39,0.718264386222346)(39,0.718264386222346)(39,0.718264386222346)(39,0.718264386222346)(39,0.718264386222346)(39,0.718264386222346)(39,0.718264386222346)(39,0.718264386222346)(39,0.718264386222346)(39,0.718264386222346)(39,0.718264386222346)(39,0.718264386222346)(39,0.718264386222346)(39,0.718264386222346)(40,0.719583241737173)(40,0.719583241737173)(40,0.719583241737173)(40,0.719583241737173)(40,0.719583241737173)(40,0.719583241737173)(40,0.719583241737173)(40,0.719583241737173)(40,0.719583241737173)(40,0.719583241737173)(40,0.719583241737173)(40,0.719583241737173)(40,0.719583241737173)(40,0.719583241737173)(40,0.719583241737173)(40,0.719583241737173)(40,0.719583241737173)(40,0.719583241737173)(41,0.721249332519477)(41,0.721249332519477)(41,0.721249332519477)(41,0.721249332519477)(41,0.721249332519477)(41,0.721249332519477)(41,0.721249332519477)(41,0.721249332519477)(41,0.721249332519477)(41,0.721249332519477)(41,0.721249332519477)(41,0.721249332519477)(41,0.721249332519477)(41,0.721249332519477)(41,0.721249332519477)(41,0.721249332519477)(41,0.721249332519477)(42,0.723028024469385)(42,0.723028024469385)(42,0.723028024469385)(42,0.723028024469385)(42,0.723028024469385)(42,0.723028024469385)(42,0.723028024469385)(42,0.723028024469385)(42,0.723028024469385)(42,0.723028024469385)(42,0.723028024469385)(42,0.723028024469385)(42,0.723028024469385)(42,0.723028024469385)(42,0.723028024469385)(42,0.723028024469385)(42,0.723028024469385)(42,0.723028024469385)(43,0.725186859754477)(43,0.725186859754477)(43,0.725186859754477)(43,0.725186859754477)(43,0.725186859754477)(43,0.725186859754477)(43,0.725186859754477)(43,0.725186859754477)(43,0.725186859754477)(43,0.725186859754477)(43,0.725186859754477)(43,0.725186859754477)(43,0.725186859754477)(43,0.725186859754477)(44,0.726306616511295)(44,0.726306616511295)(44,0.726306616511295)(44,0.726306616511295)(44,0.726306616511295)(44,0.726306616511295)(44,0.726306616511295)(44,0.726306616511295)(44,0.726306616511295)(44,0.726306616511295)(44,0.726306616511295)(44,0.726306616511295)(44,0.726306616511295)(44,0.726306616511295)(44,0.726306616511295)(44,0.726306616511295)(45,0.726402506423307)(45,0.726402506423307)(45,0.726402506423307)(45,0.726402506423307)(45,0.726402506423307)(45,0.726402506423307)(45,0.726402506423307)(45,0.726402506423307)(45,0.726402506423307)(45,0.726402506423307)(45,0.726402506423307)(45,0.726402506423307)(45,0.726402506423307)(45,0.726402506423307)(45,0.726402506423307)(45,0.726402506423307)(45,0.726402506423307)(45,0.726402506423307)(46,0.727287317134748)(46,0.727287317134748)(46,0.727287317134748)(46,0.727287317134748)(46,0.727287317134748)(46,0.727287317134748)(46,0.727287317134748)(46,0.727287317134748)(46,0.727287317134748)(46,0.727287317134748)(46,0.727287317134748)(46,0.727287317134748)(46,0.727287317134748)(46,0.727287317134748)(46,0.727287317134748)(46,0.727287317134748)(46,0.727287317134748)(46,0.727287317134748)(46,0.727287317134748)(46,0.727287317134748)(46,0.727287317134748)(46,0.727287317134748)(46,0.727287317134748)(46,0.727287317134748)(46,0.727287317134748)(46,0.727287317134748)(46,0.727287317134748)(46,0.727287317134748)(46,0.727287317134748)(47,0.7284381522036)(47,0.7284381522036)(47,0.7284381522036)(47,0.7284381522036)(47,0.7284381522036)(47,0.7284381522036)(47,0.7284381522036)(47,0.7284381522036)(47,0.7284381522036)(47,0.7284381522036)(47,0.7284381522036)(47,0.7284381522036)(47,0.7284381522036)(47,0.7284381522036)(47,0.7284381522036)(47,0.7284381522036)(47,0.7284381522036)(47,0.7284381522036)(48,0.72927891935081)(48,0.72927891935081)(48,0.72927891935081)(48,0.72927891935081)(48,0.72927891935081)(48,0.72927891935081)(48,0.72927891935081)(48,0.72927891935081)(48,0.72927891935081)(48,0.72927891935081)(48,0.72927891935081)(48,0.72927891935081)(49,0.730232885889571)(49,0.730232885889571)(49,0.730232885889571)(49,0.730232885889571)(49,0.730232885889571)(49,0.730232885889571)(49,0.730232885889571)(49,0.730232885889571)(49,0.730232885889571)(49,0.730232885889571)(49,0.730232885889571)(49,0.730232885889571)(49,0.730232885889571)(49,0.730232885889571)(50,0.731717616592921)(50,0.731717616592921)(50,0.731717616592921)(50,0.731717616592921)(50,0.731717616592921)(50,0.731717616592921)(50,0.731717616592921)(50,0.731717616592921)(50,0.731717616592921)(50,0.731717616592921)(50,0.731717616592921)(50,0.731717616592921)(50,0.731717616592921)(50,0.731717616592921)(50,0.731717616592921)(51,0.733130545678087)(51,0.733130545678087)(51,0.733130545678087)(51,0.733130545678087)(51,0.733130545678087)(51,0.733130545678087)(51,0.733130545678087)(51,0.733130545678087)(51,0.733130545678087)(51,0.733130545678087)(51,0.733130545678087)(51,0.733130545678087)(51,0.733130545678087)(51,0.733130545678087)(51,0.733130545678087)(51,0.733130545678087)(51,0.733130545678087)(51,0.733130545678087)(52,0.733613400202641)(52,0.733613400202641)(52,0.733613400202641)(52,0.733613400202641)(52,0.733613400202641)(52,0.733613400202641)(52,0.733613400202641)(52,0.733613400202641)(52,0.733613400202641)(52,0.733613400202641)(52,0.733613400202641)(52,0.733613400202641)(52,0.733613400202641)(52,0.733613400202641)(52,0.733613400202641)(53,0.733269687301484)(53,0.733269687301484)(53,0.733269687301484)(53,0.733269687301484)(53,0.733269687301484)(53,0.733269687301484)(53,0.733269687301484)(53,0.733269687301484)(53,0.733269687301484)(53,0.733269687301484)(54,0.734850391271185)(54,0.734850391271185)(54,0.734850391271185)(54,0.734850391271185)(54,0.734850391271185)(54,0.734850391271185)(54,0.734850391271185)(54,0.734850391271185)(54,0.734850391271185)(54,0.734850391271185)(54,0.734850391271185)(54,0.734850391271185)(55,0.736260161966363)(55,0.736260161966363)(55,0.736260161966363)(55,0.736260161966363)(55,0.736260161966363)(55,0.736260161966363)(55,0.736260161966363)(55,0.736260161966363)(55,0.736260161966363)(55,0.736260161966363)(55,0.736260161966363)(55,0.736260161966363)(55,0.736260161966363)(55,0.736260161966363)(55,0.736260161966363)(55,0.736260161966363)(55,0.736260161966363)(55,0.736260161966363)(55,0.736260161966363)(55,0.736260161966363)(56,0.738744910824234)(56,0.738744910824234)(56,0.738744910824234)(56,0.738744910824234)(56,0.738744910824234)(56,0.738744910824234)(56,0.738744910824234)(56,0.738744910824234)(56,0.738744910824234)(56,0.738744910824234)(56,0.738744910824234)(56,0.738744910824234)(56,0.738744910824234)(56,0.738744910824234)(57,0.742127359212723)(57,0.742127359212723)(57,0.742127359212723)(57,0.742127359212723)(57,0.742127359212723)(57,0.742127359212723)(57,0.742127359212723)(57,0.742127359212723)(57,0.742127359212723)(57,0.742127359212723)(57,0.742127359212723)(57,0.742127359212723)(57,0.742127359212723)(57,0.742127359212723)(57,0.742127359212723)(57,0.742127359212723)(58,0.745646262952642)(58,0.745646262952642)(58,0.745646262952642)(58,0.745646262952642)(58,0.745646262952642)(58,0.745646262952642)(58,0.745646262952642)(58,0.745646262952642)(58,0.745646262952642)(58,0.745646262952642)(58,0.745646262952642)(58,0.745646262952642)(58,0.745646262952642)(58,0.745646262952642)(58,0.745646262952642)(59,0.749288049058517)(59,0.749288049058517)(59,0.749288049058517)(59,0.749288049058517)(59,0.749288049058517)(59,0.749288049058517)(59,0.749288049058517)(59,0.749288049058517)(59,0.749288049058517)(60,0.751151620039207)(60,0.751151620039207)(60,0.751151620039207)(60,0.751151620039207)(60,0.751151620039207)(60,0.751151620039207)(60,0.751151620039207)(60,0.751151620039207)(60,0.751151620039207)(60,0.751151620039207)(60,0.751151620039207)(60,0.751151620039207)(60,0.751151620039207)(60,0.751151620039207)(60,0.751151620039207)(61,0.752678561935622)(61,0.752678561935622)(61,0.752678561935622)(61,0.752678561935622)(61,0.752678561935622)(61,0.752678561935622)(61,0.752678561935622)(61,0.752678561935622)(61,0.752678561935622)(61,0.752678561935622)(61,0.752678561935622)(61,0.752678561935622)(62,0.753378582054)(62,0.753378582054)(62,0.753378582054)(62,0.753378582054)(62,0.753378582054)(62,0.753378582054)(62,0.753378582054)(62,0.753378582054)(62,0.753378582054)(63,0.754054047108793)(63,0.754054047108793)(63,0.754054047108793)(63,0.754054047108793)(63,0.754054047108793)(63,0.754054047108793)(63,0.754054047108793)(63,0.754054047108793)(63,0.754054047108793)(63,0.754054047108793)(63,0.754054047108793)(63,0.754054047108793)(63,0.754054047108793)(63,0.754054047108793)(64,0.754688054356151)(64,0.754688054356151)(64,0.754688054356151)(64,0.754688054356151)(64,0.754688054356151)(64,0.754688054356151)(64,0.754688054356151)(64,0.754688054356151)(64,0.754688054356151)(64,0.754688054356151)(64,0.754688054356151)(64,0.754688054356151)(64,0.754688054356151)(64,0.754688054356151)(64,0.754688054356151)(64,0.754688054356151)(64,0.754688054356151)(64,0.754688054356151)(64,0.754688054356151)(64,0.754688054356151)(65,0.755176546174823)(65,0.755176546174823)(65,0.755176546174823)(65,0.755176546174823)(65,0.755176546174823)(65,0.755176546174823)(65,0.755176546174823)(65,0.755176546174823)(65,0.755176546174823)(65,0.755176546174823)(65,0.755176546174823)(66,0.75542958164327)(66,0.75542958164327)(66,0.75542958164327)(66,0.75542958164327)(66,0.75542958164327)(66,0.75542958164327)(66,0.75542958164327)(66,0.75542958164327)(66,0.75542958164327)(66,0.75542958164327)(66,0.75542958164327)(67,0.756285953648817)(67,0.756285953648817)(67,0.756285953648817)(67,0.756285953648817)(67,0.756285953648817)(67,0.756285953648817)(67,0.756285953648817)(67,0.756285953648817)(67,0.756285953648817)(67,0.756285953648817)(67,0.756285953648817)(67,0.756285953648817)(67,0.756285953648817)(67,0.756285953648817)(68,0.756993941740984)(68,0.756993941740984)(68,0.756993941740984)(68,0.756993941740984)(68,0.756993941740984)(68,0.756993941740984)(68,0.756993941740984)(68,0.756993941740984)(68,0.756993941740984)(68,0.756993941740984)(69,0.757870404364812)(69,0.757870404364812)(69,0.757870404364812)(69,0.757870404364812)(69,0.757870404364812)(69,0.757870404364812)(69,0.757870404364812)(69,0.757870404364812)(69,0.757870404364812)(69,0.757870404364812)(69,0.757870404364812)(69,0.757870404364812)(69,0.757870404364812)(69,0.757870404364812)(69,0.757870404364812)(69,0.757870404364812)(69,0.757870404364812)(69,0.757870404364812)(70,0.758734447953089)(70,0.758734447953089)(70,0.758734447953089)(70,0.758734447953089)(70,0.758734447953089)(70,0.758734447953089)(70,0.758734447953089)(70,0.758734447953089)(70,0.758734447953089)(70,0.758734447953089)(70,0.758734447953089)(71,0.758956468560409)(71,0.758956468560409)(71,0.758956468560409)(71,0.758956468560409)(71,0.758956468560409)(71,0.758956468560409)(71,0.758956468560409)(72,0.759408905138609)(72,0.759408905138609)(72,0.759408905138609)(72,0.759408905138609)(72,0.759408905138609)(72,0.759408905138609)(72,0.759408905138609)(72,0.759408905138609)(72,0.759408905138609)(72,0.759408905138609)(72,0.759408905138609)(72,0.759408905138609)(73,0.759792285692596)(73,0.759792285692596)(73,0.759792285692596)(73,0.759792285692596)(73,0.759792285692596)(73,0.759792285692596)(73,0.759792285692596)(74,0.760028146365062)(74,0.760028146365062)(74,0.760028146365062)(74,0.760028146365062)(74,0.760028146365062)(74,0.760028146365062)(74,0.760028146365062)(74,0.760028146365062)(75,0.760180200229311)(75,0.760180200229311)(75,0.760180200229311)(75,0.760180200229311)(75,0.760180200229311)(75,0.760180200229311)(75,0.760180200229311)(75,0.760180200229311)(75,0.760180200229311)(75,0.760180200229311)(75,0.760180200229311)(75,0.760180200229311)(76,0.760349905407971)(76,0.760349905407971)(76,0.760349905407971)(76,0.760349905407971)(76,0.760349905407971)(76,0.760349905407971)(76,0.760349905407971)(76,0.760349905407971)(76,0.760349905407971)(76,0.760349905407971)(77,0.760704701742367)(77,0.760704701742367)(77,0.760704701742367)(77,0.760704701742367)(77,0.760704701742367)(77,0.760704701742367)(77,0.760704701742367)(77,0.760704701742367)(78,0.761162030109207)(78,0.761162030109207)(78,0.761162030109207)(78,0.761162030109207)(78,0.761162030109207)(78,0.761162030109207)(78,0.761162030109207)(78,0.761162030109207)(79,0.76179032794799)(79,0.76179032794799)(79,0.76179032794799)(79,0.76179032794799)(79,0.76179032794799)(79,0.76179032794799)(79,0.76179032794799)(79,0.76179032794799)(79,0.76179032794799)(79,0.76179032794799)(79,0.76179032794799)(80,0.762461487297838)(80,0.762461487297838)(80,0.762461487297838)(80,0.762461487297838)(80,0.762461487297838)(80,0.762461487297838)(81,0.763307902014868)(81,0.763307902014868)(81,0.763307902014868)(81,0.763307902014868)(81,0.763307902014868)(81,0.763307902014868)(81,0.763307902014868)(81,0.763307902014868)(81,0.763307902014868)(81,0.763307902014868)(81,0.763307902014868)(82,0.764274202585819)(82,0.764274202585819)(82,0.764274202585819)(82,0.764274202585819)(82,0.764274202585819)(82,0.764274202585819)(82,0.764274202585819)(82,0.764274202585819)(82,0.764274202585819)(82,0.764274202585819)(83,0.765236770528049)(83,0.765236770528049)(83,0.765236770528049)(83,0.765236770528049)(83,0.765236770528049)(83,0.765236770528049)(83,0.765236770528049)(84,0.766261414077535)(84,0.766261414077535)(84,0.766261414077535)(84,0.766261414077535)(84,0.766261414077535)(84,0.766261414077535)(84,0.766261414077535)(84,0.766261414077535)(84,0.766261414077535)(84,0.766261414077535)(85,0.767283941200381)(85,0.767283941200381)(85,0.767283941200381)(85,0.767283941200381)(85,0.767283941200381)(85,0.767283941200381)(85,0.767283941200381)(85,0.767283941200381)(85,0.767283941200381)(85,0.767283941200381)(85,0.767283941200381)(86,0.768268971743673)(86,0.768268971743673)(86,0.768268971743673)(86,0.768268971743673)(86,0.768268971743673)(86,0.768268971743673)(86,0.768268971743673)(86,0.768268971743673)(86,0.768268971743673)(86,0.768268971743673)(87,0.76928262616202)(87,0.76928262616202)(87,0.76928262616202)(87,0.76928262616202)(88,0.77039793929822)(88,0.77039793929822)(89,0.771646121362493)(89,0.771646121362493)(89,0.771646121362493)(89,0.771646121362493)(89,0.771646121362493)(89,0.771646121362493)(89,0.771646121362493)(90,0.772970278309229)(90,0.772970278309229)(90,0.772970278309229)(90,0.772970278309229)(90,0.772970278309229)(90,0.772970278309229)(91,0.774128380862722)(91,0.774128380862722)(91,0.774128380862722)(91,0.774128380862722)(91,0.774128380862722)(92,0.775124541304326)(92,0.775124541304326)(92,0.775124541304326)(92,0.775124541304326)(92,0.775124541304326)(92,0.775124541304326)(92,0.775124541304326)(92,0.775124541304326)(92,0.775124541304326)(92,0.775124541304326)(93,0.775906326103546)(93,0.775906326103546)(93,0.775906326103546)(93,0.775906326103546)(93,0.775906326103546)(93,0.775906326103546)(93,0.775906326103546)(94,0.776388289348741)(94,0.776388289348741)(94,0.776388289348741)(94,0.776388289348741)(94,0.776388289348741)(94,0.776388289348741)(94,0.776388289348741)(94,0.776388289348741)(95,0.776797847358438)(95,0.776797847358438)(95,0.776797847358438)(95,0.776797847358438)(95,0.776797847358438)(95,0.776797847358438)(96,0.777053167534367)(96,0.777053167534367)(96,0.777053167534367)(96,0.777053167534367)(96,0.777053167534367)(96,0.777053167534367)(97,0.77720082780053)(97,0.77720082780053)(97,0.77720082780053)(97,0.77720082780053)(97,0.77720082780053)(97,0.77720082780053)(97,0.77720082780053)(97,0.77720082780053)(98,0.777251708686851)(98,0.777251708686851)(98,0.777251708686851)(98,0.777251708686851)(98,0.777251708686851)(98,0.777251708686851)(99,0.777189176227128)(99,0.777189176227128)(99,0.777189176227128)(99,0.777189176227128)(99,0.777189176227128)(99,0.777189176227128)(99,0.777189176227128)(99,0.777189176227128)(99,0.777189176227128)(99,0.777189176227128)(100,0.777046458475109)(100,0.777046458475109)(100,0.777046458475109)(100,0.777046458475109)(100,0.777046458475109)(100,0.777046458475109)(100,0.777046458475109)(100,0.777046458475109)(101,0.777014902153977)(101,0.777014902153977)(101,0.777014902153977)(101,0.777014902153977)(101,0.777014902153977)(101,0.777014902153977)(101,0.777014902153977)(101,0.777014902153977)(101,0.777014902153977)(101,0.777014902153977)(102,0.777035658952827)(102,0.777035658952827)(102,0.777035658952827)(102,0.777035658952827)(102,0.777035658952827)(102,0.777035658952827)(102,0.777035658952827)(103,0.777115436549424)(103,0.777115436549424)(103,0.777115436549424)(103,0.777115436549424)(103,0.777115436549424)(103,0.777115436549424)(103,0.777115436549424)(103,0.777115436549424)(103,0.777115436549424)(103,0.777115436549424)(104,0.777352212848323)(104,0.777352212848323)(104,0.777352212848323)(104,0.777352212848323)(105,0.777265067317906)(105,0.777265067317906)(105,0.777265067317906)(105,0.777265067317906)(105,0.777265067317906)(105,0.777265067317906)(105,0.777265067317906)(105,0.777265067317906)(106,0.777286969087177)(106,0.777286969087177)(106,0.777286969087177)(106,0.777286969087177)(107,0.777288165641712)(107,0.777288165641712)(107,0.777288165641712)(107,0.777288165641712)(107,0.777288165641712)(107,0.777288165641712)(107,0.777288165641712)(108,0.777323285451961)(108,0.777323285451961)(108,0.777323285451961)(108,0.777323285451961)(109,0.777436392391237)(109,0.777436392391237)(109,0.777436392391237)(109,0.777436392391237)(109,0.777436392391237)(110,0.77758857446559)(110,0.77758857446559)(110,0.77758857446559)(110,0.77758857446559)(110,0.77758857446559)(110,0.77758857446559)(111,0.777876061216191)(111,0.777876061216191)(111,0.777876061216191)(111,0.777876061216191)(111,0.777876061216191)(111,0.777876061216191)(112,0.778224138695491)(112,0.778224138695491)(112,0.778224138695491)(112,0.778224138695491)(112,0.778224138695491)(112,0.778224138695491)(112,0.778224138695491)(112,0.778224138695491)(113,0.778623907049993)(113,0.778623907049993)(113,0.778623907049993)(113,0.778623907049993)(113,0.778623907049993)(114,0.77903409101388)(114,0.77903409101388)(114,0.77903409101388)(114,0.77903409101388)(114,0.77903409101388)(114,0.77903409101388)(114,0.77903409101388)(114,0.77903409101388)(114,0.77903409101388)(115,0.779445656646291)(115,0.779445656646291)(115,0.779445656646291)(115,0.779445656646291)(115,0.779445656646291)(115,0.779445656646291)(115,0.779445656646291)(115,0.779445656646291)(116,0.779863213084848)(116,0.779863213084848)(116,0.779863213084848)(116,0.779863213084848)(116,0.779863213084848)(116,0.779863213084848)(117,0.780259662593579)(117,0.780259662593579)(117,0.780259662593579)(117,0.780259662593579)(117,0.780259662593579)(117,0.780259662593579)(117,0.780259662593579)(118,0.780629279745177)(118,0.780629279745177)(118,0.780629279745177)(119,0.780868914435988)(119,0.780868914435988)(119,0.780868914435988)(119,0.780868914435988)(119,0.780868914435988)(119,0.780868914435988)(119,0.780868914435988)(120,0.781052527600504)(120,0.781052527600504)(120,0.781052527600504)(120,0.781052527600504)(120,0.781052527600504)(120,0.781052527600504)(120,0.781052527600504)(121,0.781180127142166)(121,0.781180127142166)(121,0.781180127142166)(121,0.781180127142166)(121,0.781180127142166)(122,0.781179436144817)(122,0.781179436144817)(122,0.781179436144817)(122,0.781179436144817)(122,0.781179436144817)(122,0.781179436144817)(123,0.781313083517063)(123,0.781313083517063)(123,0.781313083517063)(124,0.78152056984057)(124,0.78152056984057)(124,0.78152056984057)(124,0.78152056984057)(124,0.78152056984057)(125,0.781827424045308)(125,0.781827424045308)(125,0.781827424045308)(125,0.781827424045308)(125,0.781827424045308)(125,0.781827424045308)(125,0.781827424045308)(125,0.781827424045308)(125,0.781827424045308)(125,0.781827424045308)(126,0.782232042575874)(126,0.782232042575874)(126,0.782232042575874)(126,0.782232042575874)(126,0.782232042575874)(126,0.782232042575874)(126,0.782232042575874)(127,0.782720332705383)(127,0.782720332705383)(127,0.782720332705383)(127,0.782720332705383)(128,0.783265605051976)(128,0.783265605051976)(128,0.783265605051976)(128,0.783265605051976)(128,0.783265605051976)(129,0.783837728874143)(129,0.783837728874143)(129,0.783837728874143)(129,0.783837728874143)(129,0.783837728874143)(129,0.783837728874143)(130,0.784407637544554)(130,0.784407637544554)(130,0.784407637544554)(130,0.784407637544554)(130,0.784407637544554)(130,0.784407637544554)(130,0.784407637544554)(131,0.784952585550127)(131,0.784952585550127)(131,0.784952585550127)(131,0.784952585550127)(131,0.784952585550127)(131,0.784952585550127)(132,0.785470806454736)(132,0.785470806454736)(132,0.785470806454736)(132,0.785470806454736)(132,0.785470806454736)(132,0.785470806454736)(132,0.785470806454736)(133,0.785982183897919)(133,0.785982183897919)(133,0.785982183897919)(133,0.785982183897919)(133,0.785982183897919)(134,0.786514946274133)(134,0.786514946274133)(134,0.786514946274133)(134,0.786514946274133)(134,0.786514946274133)(134,0.786514946274133)(135,0.787169217203925)(135,0.787169217203925)(135,0.787169217203925)(136,0.787731187799629)(137,0.788273570216319)(137,0.788273570216319)(137,0.788273570216319)(137,0.788273570216319)(137,0.788273570216319)(137,0.788273570216319)(138,0.788980084768703)(138,0.788980084768703)(138,0.788980084768703)(138,0.788980084768703)(138,0.788980084768703)(138,0.788980084768703)(138,0.788980084768703)(138,0.788980084768703)(139,0.789448182738451)(139,0.789448182738451)(139,0.789448182738451)(139,0.789448182738451)(140,0.790010302767826)(140,0.790010302767826)(140,0.790010302767826)(140,0.790010302767826)(140,0.790010302767826)(140,0.790010302767826)(140,0.790010302767826)(140,0.790010302767826)(141,0.790501963392733)(141,0.790501963392733)(141,0.790501963392733)(141,0.790501963392733)(142,0.790904100240056)(142,0.790904100240056)(143,0.791221495858506)(143,0.791221495858506)(143,0.791221495858506)(143,0.791221495858506)(143,0.791221495858506)(144,0.791459511312027)(144,0.791459511312027)(144,0.791459511312027)(144,0.791459511312027)(144,0.791459511312027)(144,0.791459511312027)(144,0.791459511312027)(145,0.791557089692635)(145,0.791557089692635)(146,0.791670859646227)(146,0.791670859646227)(146,0.791670859646227)(146,0.791670859646227)(146,0.791670859646227)(146,0.791670859646227)(146,0.791670859646227)(146,0.791670859646227)(147,0.791699938030953)(147,0.791699938030953)(147,0.791699938030953)(147,0.791699938030953)(147,0.791699938030953)(147,0.791699938030953)(147,0.791699938030953)(147,0.791699938030953)(147,0.791699938030953)(148,0.791639691251652)(148,0.791639691251652)(148,0.791639691251652)(148,0.791639691251652)(148,0.791639691251652)(149,0.791503209195476)(149,0.791503209195476)(149,0.791503209195476)(149,0.791503209195476)(149,0.791503209195476)(150,0.791308453209402)(150,0.791308453209402)(150,0.791308453209402)(150,0.791308453209402)(150,0.791308453209402)(150,0.791308453209402)(151,0.79107544143214)(151,0.79107544143214)(151,0.79107544143214)(151,0.79107544143214)(152,0.790908200888716)(152,0.790908200888716)(152,0.790908200888716)(152,0.790908200888716)(152,0.790908200888716)(152,0.790908200888716)(153,0.790662285369341)(153,0.790662285369341)(153,0.790662285369341)(153,0.790662285369341)(153,0.790662285369341)(153,0.790662285369341)(153,0.790662285369341)(154,0.790445707243679)(154,0.790445707243679)(154,0.790445707243679)(154,0.790445707243679)(155,0.790270989724692)(156,0.790128409937597)(156,0.790128409937597)(156,0.790128409937597)(156,0.790128409937597)(157,0.789991635096452)(157,0.789991635096452)(157,0.789991635096452)(157,0.789991635096452)(158,0.789844397086426)(158,0.789844397086426)(158,0.789844397086426)(158,0.789844397086426)(158,0.789844397086426)(159,0.789654719940117)(159,0.789654719940117)(159,0.789654719940117)(159,0.789654719940117)(159,0.789654719940117)(159,0.789654719940117)(159,0.789654719940117)(159,0.789654719940117)(160,0.789515264440822)(160,0.789515264440822)(160,0.789515264440822)(160,0.789515264440822)(160,0.789515264440822)(161,0.789347246950803)(161,0.789347246950803)(162,0.789189113599678)(162,0.789189113599678)(162,0.789189113599678)(162,0.789189113599678)(162,0.789189113599678)(163,0.789052040390759)(163,0.789052040390759)(163,0.789052040390759)(163,0.789052040390759)(163,0.789052040390759)(163,0.789052040390759)(163,0.789052040390759)(164,0.788950787173604)(164,0.788950787173604)(164,0.788950787173604)(165,0.78894358565905)(165,0.78894358565905)(165,0.78894358565905)(166,0.788929106365566)(166,0.788929106365566)(166,0.788929106365566)(167,0.789008826025232)(167,0.789008826025232)(167,0.789008826025232)(168,0.788988231067022)(168,0.788988231067022)(168,0.788988231067022)(168,0.788988231067022)(168,0.788988231067022)(169,0.789048238708638)(170,0.789122309653294)(170,0.789122309653294)(170,0.789122309653294)(170,0.789122309653294)(170,0.789122309653294)(170,0.789122309653294)(171,0.789213307822428)(171,0.789213307822428)(171,0.789213307822428)(171,0.789213307822428)(172,0.789325003073717)(172,0.789325003073717)(172,0.789325003073717)(172,0.789325003073717)(172,0.789325003073717)(172,0.789325003073717)(172,0.789325003073717)(172,0.789325003073717)(173,0.789461166439284)(173,0.789461166439284)(173,0.789461166439284)(173,0.789461166439284)(174,0.789614328263301)(174,0.789614328263301)(174,0.789614328263301)(174,0.789614328263301)(174,0.789614328263301)(174,0.789614328263301)(175,0.789781364204599)(175,0.789781364204599)(176,0.789972021748432)(176,0.789972021748432)(176,0.789972021748432)(176,0.789972021748432)(176,0.789972021748432)(177,0.790178393646237)(177,0.790178393646237)(177,0.790178393646237)(177,0.790178393646237)(177,0.790178393646237)(177,0.790178393646237)(178,0.790384944883329)(178,0.790384944883329)(178,0.790384944883329)(179,0.790606762133362)(179,0.790606762133362)(179,0.790606762133362)(179,0.790606762133362)(179,0.790606762133362)(179,0.790606762133362)(180,0.790835145313628)(180,0.790835145313628)(180,0.790835145313628)(180,0.790835145313628)(180,0.790835145313628)(180,0.790835145313628)(180,0.790835145313628)(181,0.79107206837052)(181,0.79107206837052)(181,0.79107206837052)(181,0.79107206837052)(181,0.79107206837052)(182,0.791316239663877)(182,0.791316239663877)(182,0.791316239663877)(183,0.791563880604018)(183,0.791563880604018)(184,0.791811846940681)(184,0.791811846940681)(184,0.791811846940681)(184,0.791811846940681)(184,0.791811846940681)(184,0.791811846940681)(184,0.791811846940681)(185,0.792058711530069)(185,0.792058711530069)(185,0.792058711530069)(185,0.792058711530069)(185,0.792058711530069)(186,0.792302772207564)(186,0.792302772207564)(186,0.792302772207564)(186,0.792302772207564)(186,0.792302772207564)(187,0.792547642587408)(187,0.792547642587408)(187,0.792547642587408)(187,0.792547642587408)(188,0.792792602472097)(188,0.792792602472097)(189,0.79303804663946)(189,0.79303804663946)(190,0.793283065073696)(190,0.793283065073696)(191,0.793544711915523)(191,0.793544711915523)(191,0.793544711915523)(191,0.793544711915523)(192,0.793774079600519)(192,0.793774079600519)(192,0.793774079600519)(193,0.793995898970674)(193,0.793995898970674)(193,0.793995898970674)(193,0.793995898970674)(193,0.793995898970674)(193,0.793995898970674)(194,0.79419826572107)(194,0.79419826572107)(194,0.79419826572107)(194,0.79419826572107)(194,0.79419826572107)(194,0.79419826572107)(194,0.79419826572107)(195,0.794415886460242)(195,0.794415886460242)(195,0.794415886460242)(195,0.794415886460242)(196,0.794633376246766)(196,0.794633376246766)(196,0.794633376246766)(196,0.794633376246766)(196,0.794633376246766)(197,0.794846699877314)(197,0.794846699877314)(197,0.794846699877314)(197,0.794846699877314)(197,0.794846699877314)(198,0.795050171983874)(198,0.795050171983874)(198,0.795050171983874)(198,0.795050171983874)(198,0.795050171983874)(199,0.795243533042248)(199,0.795243533042248)(199,0.795243533042248)(199,0.795243533042248)(199,0.795243533042248)(199,0.795243533042248)(199,0.795243533042248)(199,0.795243533042248)(199,0.795243533042248)(200,0.79540681781462)(200,0.79540681781462)(201,0.795578677255051)(201,0.795578677255051)(201,0.795578677255051)(201,0.795578677255051)(201,0.795578677255051)(202,0.795742510035732)(202,0.795742510035732)(202,0.795742510035732)(203,0.795945373590851)(203,0.795945373590851)(205,0.796128269739387)(205,0.796128269739387)(205,0.796128269739387)(205,0.796128269739387)(205,0.796128269739387)(205,0.796128269739387)(206,0.796203691006725)(206,0.796203691006725)(206,0.796203691006725)(206,0.796203691006725)(206,0.796203691006725)(207,0.796253495879719)(207,0.796253495879719)(208,0.796283988827124)(209,0.796303105743204)(209,0.796303105743204)(209,0.796303105743204)(209,0.796303105743204)(209,0.796303105743204)(209,0.796303105743204)(210,0.796318372138264)(210,0.796318372138264)(210,0.796318372138264)(211,0.796336114064861)(211,0.796336114064861)(211,0.796336114064861)(212,0.796361244924253)(212,0.796361244924253)(212,0.796361244924253)(212,0.796361244924253)(212,0.796361244924253)(212,0.796361244924253)(213,0.796394205601431)(213,0.796394205601431)(213,0.796394205601431)(213,0.796394205601431)(213,0.796394205601431)(214,0.796427638590901)(214,0.796427638590901)(214,0.796427638590901)(214,0.796427638590901)(214,0.796427638590901)(214,0.796427638590901)(215,0.796465671353671)(215,0.796465671353671)(215,0.796465671353671)(215,0.796465671353671)(216,0.796508086809454)(216,0.796508086809454)(216,0.796508086809454)(217,0.796553375896316)(217,0.796553375896316)(218,0.796589337167425)(218,0.796589337167425)(218,0.796589337167425)(218,0.796589337167425)(218,0.796589337167425)(219,0.79663083278285)(220,0.796718947150556)(220,0.796718947150556)(220,0.796718947150556)(220,0.796718947150556)(221,0.796792105335032)(221,0.796792105335032)(221,0.796792105335032)(221,0.796792105335032)(221,0.796792105335032)(221,0.796792105335032)(221,0.796792105335032)(221,0.796792105335032)(221,0.796792105335032)(222,0.796828332559469)(222,0.796828332559469)(222,0.796828332559469)(223,0.79699029081655)(223,0.79699029081655)(223,0.79699029081655)(223,0.79699029081655)(223,0.79699029081655)(223,0.79699029081655)(224,0.797116541991322)(224,0.797116541991322)(225,0.797259021065925)(225,0.797259021065925)(225,0.797259021065925)(226,0.797416859114205)(226,0.797416859114205)(227,0.797587371853839)(227,0.797587371853839)(228,0.797770999830568)(228,0.797770999830568)(228,0.797770999830568)(228,0.797770999830568)(229,0.797945454273825)(229,0.797945454273825)(229,0.797945454273825)(229,0.797945454273825)(229,0.797945454273825)(229,0.797945454273825)(229,0.797945454273825)(230,0.798117182782565)(230,0.798117182782565)(230,0.798117182782565)(230,0.798117182782565)(230,0.798117182782565)(230,0.798117182782565)(230,0.798117182782565)(230,0.798117182782565)(230,0.798117182782565)(230,0.798117182782565)(230,0.798117182782565)(231,0.798281041811525)(231,0.798281041811525)(231,0.798281041811525)(232,0.798433537129353)(232,0.798433537129353)(232,0.798433537129353)(233,0.798574435274162)(234,0.798735221111905)(234,0.798735221111905)(235,0.79882973642654)(235,0.79882973642654)(235,0.79882973642654)(235,0.79882973642654)(235,0.79882973642654)(235,0.79882973642654)(236,0.798946672139705)(236,0.798946672139705)(237,0.799058245278571)(237,0.799058245278571)(237,0.799058245278571)(237,0.799058245278571)(237,0.799058245278571)(237,0.799058245278571)(238,0.799200959984889)(238,0.799200959984889)(238,0.799200959984889)(238,0.799200959984889)(238,0.799200959984889)(239,0.799271995826295)(239,0.799271995826295)(239,0.799271995826295)(239,0.799271995826295)(240,0.799374223645027)(240,0.799374223645027)(240,0.799374223645027)(240,0.799374223645027)(240,0.799374223645027)(241,0.799473379206959)(241,0.799473379206959)(241,0.799473379206959)(243,0.799662226910066)(243,0.799662226910066)(243,0.799662226910066)(243,0.799662226910066)(243,0.799662226910066)(244,0.79975247180751)(244,0.79975247180751)(244,0.79975247180751)(244,0.79975247180751)(244,0.79975247180751)(244,0.79975247180751)(244,0.79975247180751)(245,0.799841336486654)(245,0.799841336486654)(245,0.799841336486654)(246,0.799942321258086)(246,0.799942321258086)(246,0.799942321258086)(247,0.800026921577163)(247,0.800026921577163)(247,0.800026921577163)(248,0.800131606791171)(249,0.800246406540138)(249,0.800246406540138)(249,0.800246406540138)(250,0.800373011196988)(250,0.800373011196988)(250,0.800373011196988)(250,0.800373011196988)(250,0.800373011196988)(250,0.800373011196988)(250,0.800373011196988)(250,0.800373011196988)(250,0.800373011196988)(250,0.800373011196988)(250,0.800373011196988)(251,0.800512972624917)(252,0.800664312303416)(252,0.800664312303416)(253,0.800826629151128)(253,0.800826629151128)(253,0.800826629151128)(253,0.800826629151128)(253,0.800826629151128)(253,0.800826629151128)(254,0.800999275642214)(254,0.800999275642214)(254,0.800999275642214)(255,0.801220462050698)(255,0.801220462050698)(255,0.801220462050698)(256,0.801405953663143)(256,0.801405953663143)(256,0.801405953663143)(256,0.801405953663143)(256,0.801405953663143)(257,0.801568057783491)(258,0.801772116583192)(258,0.801772116583192)(259,0.801982176780979)(259,0.801982176780979)(259,0.801982176780979)(259,0.801982176780979)(259,0.801982176780979)(259,0.801982176780979)(259,0.801982176780979)(260,0.802195781573363)(260,0.802195781573363)(260,0.802195781573363)(261,0.802403757468926)(261,0.802403757468926)(262,0.8026147760569)(262,0.8026147760569)(262,0.8026147760569)(262,0.8026147760569)(262,0.8026147760569)(262,0.8026147760569)(262,0.8026147760569)(263,0.802803565614688)(263,0.802803565614688)(264,0.803004885355349)(264,0.803004885355349)(264,0.803004885355349)(264,0.803004885355349)(265,0.803246731179805)(265,0.803246731179805)(265,0.803246731179805)(266,0.803445213899602)(266,0.803445213899602)(266,0.803445213899602)(266,0.803445213899602)(267,0.803633612146842)(267,0.803633612146842)(267,0.803633612146842)(268,0.803809785369327)(268,0.803809785369327)(268,0.803809785369327)(269,0.80397427742347)(269,0.80397427742347)(270,0.804130130445851)(270,0.804130130445851)(271,0.804279545566788)(271,0.804279545566788)(271,0.804279545566788)(272,0.804425531537962)(272,0.804425531537962)(272,0.804425531537962)(272,0.804425531537962)(272,0.804425531537962)(273,0.804569066113205)(273,0.804569066113205)(273,0.804569066113205)(273,0.804569066113205)(273,0.804569066113205)(274,0.804708095834651)(274,0.804708095834651)(274,0.804708095834651)(274,0.804708095834651)(275,0.804838993759097)(275,0.804838993759097)(275,0.804838993759097)(275,0.804838993759097)(275,0.804838993759097)(276,0.804966852791516)(276,0.804966852791516)(276,0.804966852791516)(277,0.805067860140383)(277,0.805067860140383)(277,0.805067860140383)(277,0.805067860140383)(278,0.805155146369402)(278,0.805155146369402)(278,0.805155146369402)(278,0.805155146369402)(278,0.805155146369402)(278,0.805155146369402)(279,0.80522848719447)(279,0.80522848719447)(279,0.80522848719447)(280,0.805297271836688)(280,0.805297271836688)(280,0.805297271836688)(280,0.805297271836688)(280,0.805297271836688)(281,0.805348290836217)(281,0.805348290836217)(281,0.805348290836217)(281,0.805348290836217)(282,0.805396084987646)(282,0.805396084987646)(282,0.805396084987646)(282,0.805396084987646)(282,0.805396084987646)(282,0.805396084987646)(284,0.80548150722313)(284,0.80548150722313)(284,0.80548150722313)(284,0.80548150722313)(284,0.80548150722313)(284,0.80548150722313)(284,0.80548150722313)(286,0.805517588684773)(286,0.805517588684773)(286,0.805517588684773)(287,0.805517982893598)(287,0.805517982893598)(287,0.805517982893598)(287,0.805517982893598)(287,0.805517982893598)(287,0.805517982893598)(287,0.805517982893598)(287,0.805517982893598)(288,0.80550874922992)(288,0.80550874922992)(288,0.80550874922992)(288,0.80550874922992)(288,0.80550874922992)(289,0.805490060491376)(289,0.805490060491376)(289,0.805490060491376)(289,0.805490060491376)(290,0.805451903168003)(290,0.805451903168003)(291,0.805437749316314)(291,0.805437749316314)(291,0.805437749316314)(291,0.805437749316314)(292,0.805409116016001)(292,0.805409116016001)(293,0.8053719436931)(293,0.8053719436931)(294,0.805356963460166)(295,0.805334689105184)(295,0.805334689105184)(295,0.805334689105184)(296,0.80531110026016)(296,0.80531110026016)(296,0.80531110026016)(297,0.805283290381794)(297,0.805283290381794)(297,0.805283290381794)(297,0.805283290381794)(297,0.805283290381794)(297,0.805283290381794)(297,0.805283290381794)(298,0.80524005458282)(298,0.80524005458282)(298,0.80524005458282)(298,0.80524005458282)(298,0.80524005458282)(298,0.80524005458282)(298,0.80524005458282)(299,0.805208498337874)(299,0.805208498337874)(299,0.805208498337874)(300,0.805160343640082)(300,0.805160343640082)(300,0.805160343640082)(300,0.805160343640082)(300,0.805160343640082)(301,0.805105258608892)(301,0.805105258608892)(301,0.805105258608892)(302,0.805038598028123)(302,0.805038598028123)(302,0.805038598028123)(302,0.805038598028123)(303,0.804967274254759)(303,0.804967274254759)(303,0.804967274254759)(304,0.804897678096633)(304,0.804897678096633)(304,0.804897678096633)(304,0.804897678096633)(305,0.804864456228175)(305,0.804864456228175)(305,0.804864456228175)(305,0.804864456228175)(305,0.804864456228175)(306,0.804814457784406)(307,0.804801246360913)(307,0.804801246360913)(309,0.804733551189376)(310,0.804739952658623)(310,0.804739952658623)(310,0.804739952658623)(311,0.804699677776963)(311,0.804699677776963)(312,0.80472428533859)(312,0.80472428533859)(312,0.80472428533859)(313,0.804735372685576)(313,0.804735372685576)(313,0.804735372685576)(314,0.804759483091077)(314,0.804759483091077)(314,0.804759483091077)(315,0.804795678080118)(315,0.804795678080118)(315,0.804795678080118)(315,0.804795678080118)(315,0.804795678080118)(315,0.804795678080118)(316,0.804842870240419)(316,0.804842870240419)(316,0.804842870240419)(316,0.804842870240419)(316,0.804842870240419)(317,0.804899195840968)(317,0.804899195840968)(317,0.804899195840968)(318,0.804959742050401)(318,0.804959742050401)(318,0.804959742050401)(319,0.805023438950562)(320,0.805099369277049)(320,0.805099369277049)(320,0.805099369277049)(320,0.805099369277049)(321,0.805175490765287)(321,0.805175490765287)(322,0.805261720457389)(322,0.805261720457389)(323,0.805354137663916)(323,0.805354137663916)(323,0.805354137663916)(324,0.805452392784402)(324,0.805452392784402)(325,0.805555183728974)(325,0.805555183728974)(325,0.805555183728974)(325,0.805555183728974)(326,0.805660663869996)(326,0.805660663869996)(326,0.805660663869996)(327,0.805762326577288)(327,0.805762326577288)(327,0.805762326577288)(327,0.805762326577288)(327,0.805762326577288)(327,0.805762326577288)(327,0.805762326577288)(328,0.805875333250185)(328,0.805875333250185)(328,0.805875333250185)(328,0.805875333250185)(329,0.80598358214943)(329,0.80598358214943)(329,0.80598358214943)(330,0.806092748893992)(330,0.806092748893992)(331,0.806203741362687)(331,0.806203741362687)(331,0.806203741362687)(332,0.806319566161448)(332,0.806319566161448)(332,0.806319566161448)(332,0.806319566161448)(332,0.806319566161448)(332,0.806319566161448)(332,0.806319566161448)(333,0.80643546958712)(333,0.80643546958712)(333,0.80643546958712)(333,0.80643546958712)(333,0.80643546958712)(334,0.806556553303924)(334,0.806556553303924)(334,0.806556553303924)(334,0.806556553303924)(335,0.806681578683215)(335,0.806681578683215)(335,0.806681578683215)(336,0.806809710480722)(336,0.806809710480722)(336,0.806809710480722)(336,0.806809710480722)(338,0.80706928742283)(338,0.80706928742283)(338,0.80706928742283)(339,0.807195995393577)(340,0.80731922225358)(340,0.80731922225358)(340,0.80731922225358)(341,0.807438745869052)(341,0.807438745869052)(341,0.807438745869052)(341,0.807438745869052)(342,0.807555349598948)(342,0.807555349598948)(343,0.807671198720292)(344,0.807791458823729)(345,0.807905962345761)(345,0.807905962345761)(345,0.807905962345761)(345,0.807905962345761)(345,0.807905962345761)(346,0.808024848349947)(346,0.808024848349947)(347,0.808142423188507)(347,0.808142423188507)(347,0.808142423188507)(348,0.808270383640939)(348,0.808270383640939)(348,0.808270383640939)(348,0.808270383640939)(348,0.808270383640939)(348,0.808270383640939)(349,0.808373062917718)(349,0.808373062917718)(350,0.808466086221556)(350,0.808466086221556)(350,0.808466086221556)(350,0.808466086221556)(351,0.808549506822077)(351,0.808549506822077)(351,0.808549506822077)(353,0.808687497934117)(354,0.808738766867678)(354,0.808738766867678)(354,0.808738766867678)(355,0.808776416641566)(355,0.808776416641566)(355,0.808776416641566)(356,0.808779712175859)(357,0.808784963489641)(357,0.808784963489641)(357,0.808784963489641)(357,0.808784963489641)(358,0.808779475438704)(359,0.808763882767614)(359,0.808763882767614)(359,0.808763882767614)(359,0.808763882767614)(359,0.808763882767614)(360,0.808737968026595)(360,0.808737968026595)(360,0.808737968026595)(360,0.808737968026595)(361,0.808702138785378)(361,0.808702138785378)(361,0.808702138785378)(362,0.808656481196134)(362,0.808656481196134)(362,0.808656481196134)(362,0.808656481196134)(362,0.808656481196134)(362,0.808656481196134)(363,0.808577008752217)(363,0.808577008752217)(363,0.808577008752217)(363,0.808577008752217)(363,0.808577008752217)(363,0.808577008752217)(364,0.80850147093243)(365,0.808433526678667)(365,0.808433526678667)(365,0.808433526678667)(365,0.808433526678667)(366,0.808373394923167)(367,0.808319691347308)(367,0.808319691347308)(367,0.808319691347308)(368,0.808270702738653)(368,0.808270702738653)(368,0.808270702738653)(369,0.808224902298752)(369,0.808224902298752)(370,0.80818108856613)(370,0.80818108856613)(371,0.808138352085518)(371,0.808138352085518)(371,0.808138352085518)(371,0.808138352085518)(371,0.808138352085518)(371,0.808138352085518)(372,0.808095994405151)(373,0.808053505963466)(373,0.808053505963466)(374,0.808010511021084)(374,0.808010511021084)(375,0.807966720219297)(375,0.807966720219297)(375,0.807966720219297)(375,0.807966720219297)(376,0.80792191227759)(376,0.80792191227759)(376,0.80792191227759)(376,0.80792191227759)(376,0.80792191227759)(377,0.807875940515588)(377,0.807875940515588)(377,0.807875940515588)(378,0.807828751532224)(378,0.807828751532224)(378,0.807828751532224)(378,0.807828751532224)(379,0.807780341699929)(379,0.807780341699929)(379,0.807780341699929)(380,0.807730719361747)(380,0.807730719361747)(380,0.807730719361747)(380,0.807730719361747)(381,0.807679919008612)(381,0.807679919008612)(382,0.807628004148876)(382,0.807628004148876)(383,0.807575070929885)(383,0.807575070929885)(383,0.807575070929885)(384,0.80752123539152)(384,0.80752123539152)(385,0.807466627557409)(385,0.807466627557409)(385,0.807466627557409)(385,0.807466627557409)(385,0.807466627557409)(386,0.807411413051147)(386,0.807411413051147)(387,0.807355789713625)(387,0.807355789713625)(387,0.807355789713625)(387,0.807355789713625)(388,0.807299950113083)(388,0.807299950113083)(388,0.807299950113083)(389,0.807244071226544)(389,0.807244071226544)(389,0.807244071226544)(390,0.80718829734225)(390,0.80718829734225)(390,0.80718829734225)(390,0.80718829734225)(391,0.807132752277781)(391,0.807132752277781)(391,0.807132752277781)(391,0.807132752277781)(391,0.807132752277781)(392,0.807077572207695)(392,0.807077572207695)(392,0.807077572207695)(393,0.807022877498045)(395,0.806915377320397)(395,0.806915377320397)(395,0.806915377320397)(396,0.806862801699163)(396,0.806862801699163)(396,0.806862801699163)(397,0.80681111699222)(398,0.806760354414034)(398,0.806760354414034)(398,0.806760354414034)(399,0.806710539635911)(399,0.806710539635911)(400,0.806661680361804)(400,0.806661680361804) 
};
\addlegendentry{\acl (max. variance)};

\addplot [
color=green!50!black,
solid,
line width=1.0pt,
]
coordinates{
 (10,0.608986656053959)(14,0.678365437164853)(14,0.678365437164853)(14,0.678365437164853)(14,0.678365437164853)(15,0.689940989729436)(15,0.689940989729436)(15,0.689940989729436)(15,0.689940989729436)(15,0.689940989729436)(15,0.689940989729436)(16,0.699433013911079)(16,0.699433013911079)(16,0.699433013911079)(16,0.699433013911079)(16,0.699433013911079)(16,0.699433013911079)(16,0.699433013911079)(17,0.70694360583489)(17,0.70694360583489)(17,0.70694360583489)(17,0.70694360583489)(17,0.70694360583489)(17,0.70694360583489)(17,0.70694360583489)(17,0.70694360583489)(17,0.70694360583489)(18,0.712539120523593)(18,0.712539120523593)(18,0.712539120523593)(18,0.712539120523593)(18,0.712539120523593)(18,0.712539120523593)(18,0.712539120523593)(18,0.712539120523593)(19,0.716194470115051)(19,0.716194470115051)(19,0.716194470115051)(19,0.716194470115051)(19,0.716194470115051)(19,0.716194470115051)(19,0.716194470115051)(19,0.716194470115051)(19,0.716194470115051)(20,0.717908851807876)(20,0.717908851807876)(20,0.717908851807876)(20,0.717908851807876)(20,0.717908851807876)(20,0.717908851807876)(20,0.717908851807876)(20,0.717908851807876)(21,0.718095330812828)(21,0.718095330812828)(21,0.718095330812828)(21,0.718095330812828)(21,0.718095330812828)(21,0.718095330812828)(21,0.718095330812828)(21,0.718095330812828)(22,0.716735705294114)(22,0.716735705294114)(22,0.716735705294114)(22,0.716735705294114)(22,0.716735705294114)(22,0.716735705294114)(22,0.716735705294114)(22,0.716735705294114)(23,0.713965808864074)(23,0.713965808864074)(23,0.713965808864074)(23,0.713965808864074)(23,0.713965808864074)(23,0.713965808864074)(23,0.713965808864074)(23,0.713965808864074)(23,0.713965808864074)(23,0.713965808864074)(23,0.713965808864074)(23,0.713965808864074)(23,0.713965808864074)(23,0.713965808864074)(23,0.713965808864074)(23,0.713965808864074)(23,0.713965808864074)(24,0.713325168349988)(24,0.713325168349988)(24,0.713325168349988)(24,0.713325168349988)(24,0.713325168349988)(24,0.713325168349988)(24,0.713325168349988)(24,0.713325168349988)(24,0.713325168349988)(24,0.713325168349988)(24,0.713325168349988)(24,0.713325168349988)(24,0.713325168349988)(24,0.713325168349988)(24,0.713325168349988)(24,0.713325168349988)(24,0.713325168349988)(24,0.713325168349988)(24,0.713325168349988)(24,0.713325168349988)(25,0.713382002817811)(25,0.713382002817811)(25,0.713382002817811)(25,0.713382002817811)(25,0.713382002817811)(25,0.713382002817811)(25,0.713382002817811)(25,0.713382002817811)(25,0.713382002817811)(25,0.713382002817811)(25,0.713382002817811)(26,0.714140626578976)(26,0.714140626578976)(26,0.714140626578976)(26,0.714140626578976)(26,0.714140626578976)(26,0.714140626578976)(26,0.714140626578976)(26,0.714140626578976)(26,0.714140626578976)(26,0.714140626578976)(26,0.714140626578976)(26,0.714140626578976)(26,0.714140626578976)(27,0.713622269631928)(27,0.713622269631928)(27,0.713622269631928)(27,0.713622269631928)(27,0.713622269631928)(27,0.713622269631928)(27,0.713622269631928)(27,0.713622269631928)(27,0.713622269631928)(27,0.713622269631928)(27,0.713622269631928)(27,0.713622269631928)(27,0.713622269631928)(27,0.713622269631928)(27,0.713622269631928)(27,0.713622269631928)(27,0.713622269631928)(28,0.712961818031889)(28,0.712961818031889)(28,0.712961818031889)(28,0.712961818031889)(28,0.712961818031889)(28,0.712961818031889)(28,0.712961818031889)(28,0.712961818031889)(28,0.712961818031889)(28,0.712961818031889)(28,0.712961818031889)(28,0.712961818031889)(28,0.712961818031889)(28,0.712961818031889)(28,0.712961818031889)(28,0.712961818031889)(28,0.712961818031889)(28,0.712961818031889)(28,0.712961818031889)(29,0.713253656313159)(29,0.713253656313159)(29,0.713253656313159)(29,0.713253656313159)(29,0.713253656313159)(29,0.713253656313159)(29,0.713253656313159)(29,0.713253656313159)(29,0.713253656313159)(29,0.713253656313159)(29,0.713253656313159)(29,0.713253656313159)(29,0.713253656313159)(29,0.713253656313159)(29,0.713253656313159)(29,0.713253656313159)(29,0.713253656313159)(29,0.713253656313159)(29,0.713253656313159)(29,0.713253656313159)(29,0.713253656313159)(29,0.713253656313159)(30,0.712227206236694)(30,0.712227206236694)(30,0.712227206236694)(30,0.712227206236694)(30,0.712227206236694)(30,0.712227206236694)(30,0.712227206236694)(30,0.712227206236694)(30,0.712227206236694)(30,0.712227206236694)(30,0.712227206236694)(30,0.712227206236694)(30,0.712227206236694)(30,0.712227206236694)(31,0.712357153601962)(31,0.712357153601962)(31,0.712357153601962)(31,0.712357153601962)(31,0.712357153601962)(31,0.712357153601962)(31,0.712357153601962)(31,0.712357153601962)(31,0.712357153601962)(32,0.716404692964435)(32,0.716404692964435)(32,0.716404692964435)(32,0.716404692964435)(32,0.716404692964435)(32,0.716404692964435)(32,0.716404692964435)(32,0.716404692964435)(32,0.716404692964435)(32,0.716404692964435)(32,0.716404692964435)(32,0.716404692964435)(33,0.722617032364021)(33,0.722617032364021)(33,0.722617032364021)(33,0.722617032364021)(33,0.722617032364021)(33,0.722617032364021)(33,0.722617032364021)(33,0.722617032364021)(33,0.722617032364021)(33,0.722617032364021)(33,0.722617032364021)(33,0.722617032364021)(33,0.722617032364021)(33,0.722617032364021)(33,0.722617032364021)(33,0.722617032364021)(33,0.722617032364021)(33,0.722617032364021)(33,0.722617032364021)(33,0.722617032364021)(33,0.722617032364021)(33,0.722617032364021)(34,0.728582816630312)(34,0.728582816630312)(34,0.728582816630312)(34,0.728582816630312)(34,0.728582816630312)(34,0.728582816630312)(34,0.728582816630312)(34,0.728582816630312)(34,0.728582816630312)(34,0.728582816630312)(34,0.728582816630312)(34,0.728582816630312)(34,0.728582816630312)(34,0.728582816630312)(35,0.733201471799011)(35,0.733201471799011)(35,0.733201471799011)(35,0.733201471799011)(35,0.733201471799011)(35,0.733201471799011)(35,0.733201471799011)(35,0.733201471799011)(35,0.733201471799011)(35,0.733201471799011)(35,0.733201471799011)(35,0.733201471799011)(35,0.733201471799011)(36,0.73567884824689)(36,0.73567884824689)(36,0.73567884824689)(36,0.73567884824689)(36,0.73567884824689)(36,0.73567884824689)(36,0.73567884824689)(36,0.73567884824689)(36,0.73567884824689)(36,0.73567884824689)(36,0.73567884824689)(36,0.73567884824689)(36,0.73567884824689)(36,0.73567884824689)(36,0.73567884824689)(36,0.73567884824689)(36,0.73567884824689)(37,0.735265247201092)(37,0.735265247201092)(37,0.735265247201092)(37,0.735265247201092)(37,0.735265247201092)(37,0.735265247201092)(37,0.735265247201092)(37,0.735265247201092)(37,0.735265247201092)(37,0.735265247201092)(37,0.735265247201092)(37,0.735265247201092)(37,0.735265247201092)(37,0.735265247201092)(37,0.735265247201092)(38,0.733739478921226)(38,0.733739478921226)(38,0.733739478921226)(38,0.733739478921226)(38,0.733739478921226)(38,0.733739478921226)(38,0.733739478921226)(38,0.733739478921226)(38,0.733739478921226)(38,0.733739478921226)(38,0.733739478921226)(38,0.733739478921226)(38,0.733739478921226)(39,0.732600488283553)(39,0.732600488283553)(39,0.732600488283553)(39,0.732600488283553)(39,0.732600488283553)(39,0.732600488283553)(39,0.732600488283553)(39,0.732600488283553)(39,0.732600488283553)(39,0.732600488283553)(39,0.732600488283553)(39,0.732600488283553)(40,0.732687475489433)(40,0.732687475489433)(40,0.732687475489433)(40,0.732687475489433)(40,0.732687475489433)(40,0.732687475489433)(40,0.732687475489433)(40,0.732687475489433)(40,0.732687475489433)(40,0.732687475489433)(40,0.732687475489433)(40,0.732687475489433)(40,0.732687475489433)(40,0.732687475489433)(40,0.732687475489433)(40,0.732687475489433)(40,0.732687475489433)(41,0.734273797448215)(41,0.734273797448215)(41,0.734273797448215)(41,0.734273797448215)(41,0.734273797448215)(41,0.734273797448215)(41,0.734273797448215)(41,0.734273797448215)(41,0.734273797448215)(41,0.734273797448215)(41,0.734273797448215)(41,0.734273797448215)(41,0.734273797448215)(41,0.734273797448215)(41,0.734273797448215)(41,0.734273797448215)(41,0.734273797448215)(41,0.734273797448215)(41,0.734273797448215)(41,0.734273797448215)(41,0.734273797448215)(41,0.734273797448215)(41,0.734273797448215)(41,0.734273797448215)(41,0.734273797448215)(41,0.734273797448215)(41,0.734273797448215)(41,0.734273797448215)(41,0.734273797448215)(42,0.736241662517489)(42,0.736241662517489)(42,0.736241662517489)(42,0.736241662517489)(42,0.736241662517489)(42,0.736241662517489)(42,0.736241662517489)(42,0.736241662517489)(42,0.736241662517489)(42,0.736241662517489)(43,0.737609518981827)(43,0.737609518981827)(43,0.737609518981827)(43,0.737609518981827)(43,0.737609518981827)(43,0.737609518981827)(43,0.737609518981827)(43,0.737609518981827)(43,0.737609518981827)(43,0.737609518981827)(44,0.738112150966641)(44,0.738112150966641)(44,0.738112150966641)(44,0.738112150966641)(44,0.738112150966641)(44,0.738112150966641)(44,0.738112150966641)(44,0.738112150966641)(44,0.738112150966641)(44,0.738112150966641)(44,0.738112150966641)(44,0.738112150966641)(44,0.738112150966641)(44,0.738112150966641)(44,0.738112150966641)(45,0.73788201137545)(45,0.73788201137545)(45,0.73788201137545)(45,0.73788201137545)(45,0.73788201137545)(45,0.73788201137545)(45,0.73788201137545)(45,0.73788201137545)(45,0.73788201137545)(45,0.73788201137545)(45,0.73788201137545)(45,0.73788201137545)(45,0.73788201137545)(45,0.73788201137545)(45,0.73788201137545)(45,0.73788201137545)(45,0.73788201137545)(45,0.73788201137545)(45,0.73788201137545)(46,0.737493883073246)(46,0.737493883073246)(46,0.737493883073246)(46,0.737493883073246)(46,0.737493883073246)(46,0.737493883073246)(46,0.737493883073246)(46,0.737493883073246)(46,0.737493883073246)(46,0.737493883073246)(46,0.737493883073246)(46,0.737493883073246)(46,0.737493883073246)(47,0.737388768125982)(47,0.737388768125982)(47,0.737388768125982)(47,0.737388768125982)(47,0.737388768125982)(47,0.737388768125982)(47,0.737388768125982)(47,0.737388768125982)(47,0.737388768125982)(47,0.737388768125982)(47,0.737388768125982)(47,0.737388768125982)(47,0.737388768125982)(48,0.737990946733675)(48,0.737990946733675)(48,0.737990946733675)(48,0.737990946733675)(48,0.737990946733675)(48,0.737990946733675)(48,0.737990946733675)(48,0.737990946733675)(48,0.737990946733675)(48,0.737990946733675)(48,0.737990946733675)(48,0.737990946733675)(48,0.737990946733675)(48,0.737990946733675)(48,0.737990946733675)(48,0.737990946733675)(49,0.739024958456482)(49,0.739024958456482)(49,0.739024958456482)(49,0.739024958456482)(49,0.739024958456482)(49,0.739024958456482)(49,0.739024958456482)(49,0.739024958456482)(49,0.739024958456482)(49,0.739024958456482)(49,0.739024958456482)(49,0.739024958456482)(49,0.739024958456482)(49,0.739024958456482)(49,0.739024958456482)(49,0.739024958456482)(50,0.739536382201507)(50,0.739536382201507)(50,0.739536382201507)(50,0.739536382201507)(50,0.739536382201507)(50,0.739536382201507)(50,0.739536382201507)(50,0.739536382201507)(50,0.739536382201507)(50,0.739536382201507)(50,0.739536382201507)(50,0.739536382201507)(50,0.739536382201507)(50,0.739536382201507)(50,0.739536382201507)(51,0.740212663325151)(51,0.740212663325151)(51,0.740212663325151)(51,0.740212663325151)(51,0.740212663325151)(51,0.740212663325151)(51,0.740212663325151)(51,0.740212663325151)(51,0.740212663325151)(51,0.740212663325151)(51,0.740212663325151)(51,0.740212663325151)(51,0.740212663325151)(51,0.740212663325151)(52,0.740817435792715)(52,0.740817435792715)(52,0.740817435792715)(52,0.740817435792715)(52,0.740817435792715)(52,0.740817435792715)(52,0.740817435792715)(52,0.740817435792715)(52,0.740817435792715)(52,0.740817435792715)(52,0.740817435792715)(52,0.740817435792715)(52,0.740817435792715)(53,0.740978238711863)(53,0.740978238711863)(53,0.740978238711863)(53,0.740978238711863)(53,0.740978238711863)(53,0.740978238711863)(53,0.740978238711863)(53,0.740978238711863)(53,0.740978238711863)(53,0.740978238711863)(53,0.740978238711863)(53,0.740978238711863)(53,0.740978238711863)(54,0.74137406140589)(54,0.74137406140589)(54,0.74137406140589)(54,0.74137406140589)(54,0.74137406140589)(54,0.74137406140589)(54,0.74137406140589)(54,0.74137406140589)(54,0.74137406140589)(54,0.74137406140589)(54,0.74137406140589)(54,0.74137406140589)(54,0.74137406140589)(55,0.74200229469148)(55,0.74200229469148)(55,0.74200229469148)(55,0.74200229469148)(55,0.74200229469148)(55,0.74200229469148)(55,0.74200229469148)(55,0.74200229469148)(55,0.74200229469148)(55,0.74200229469148)(55,0.74200229469148)(55,0.74200229469148)(55,0.74200229469148)(56,0.743226400615754)(56,0.743226400615754)(56,0.743226400615754)(56,0.743226400615754)(56,0.743226400615754)(56,0.743226400615754)(56,0.743226400615754)(56,0.743226400615754)(56,0.743226400615754)(56,0.743226400615754)(56,0.743226400615754)(56,0.743226400615754)(56,0.743226400615754)(57,0.744552029318394)(57,0.744552029318394)(57,0.744552029318394)(57,0.744552029318394)(57,0.744552029318394)(57,0.744552029318394)(57,0.744552029318394)(57,0.744552029318394)(57,0.744552029318394)(57,0.744552029318394)(57,0.744552029318394)(57,0.744552029318394)(57,0.744552029318394)(58,0.746042304362147)(58,0.746042304362147)(58,0.746042304362147)(58,0.746042304362147)(58,0.746042304362147)(58,0.746042304362147)(58,0.746042304362147)(58,0.746042304362147)(58,0.746042304362147)(58,0.746042304362147)(58,0.746042304362147)(58,0.746042304362147)(58,0.746042304362147)(58,0.746042304362147)(58,0.746042304362147)(58,0.746042304362147)(59,0.747023053056623)(59,0.747023053056623)(59,0.747023053056623)(59,0.747023053056623)(59,0.747023053056623)(59,0.747023053056623)(59,0.747023053056623)(60,0.747433871143335)(60,0.747433871143335)(60,0.747433871143335)(60,0.747433871143335)(60,0.747433871143335)(60,0.747433871143335)(60,0.747433871143335)(60,0.747433871143335)(60,0.747433871143335)(60,0.747433871143335)(60,0.747433871143335)(60,0.747433871143335)(60,0.747433871143335)(60,0.747433871143335)(60,0.747433871143335)(60,0.747433871143335)(61,0.747788866401504)(61,0.747788866401504)(61,0.747788866401504)(61,0.747788866401504)(61,0.747788866401504)(61,0.747788866401504)(61,0.747788866401504)(61,0.747788866401504)(61,0.747788866401504)(61,0.747788866401504)(61,0.747788866401504)(62,0.749221383433777)(62,0.749221383433777)(62,0.749221383433777)(62,0.749221383433777)(62,0.749221383433777)(62,0.749221383433777)(62,0.749221383433777)(62,0.749221383433777)(62,0.749221383433777)(62,0.749221383433777)(62,0.749221383433777)(62,0.749221383433777)(62,0.749221383433777)(63,0.750609124413414)(63,0.750609124413414)(63,0.750609124413414)(63,0.750609124413414)(63,0.750609124413414)(63,0.750609124413414)(63,0.750609124413414)(64,0.752073605998092)(64,0.752073605998092)(64,0.752073605998092)(64,0.752073605998092)(64,0.752073605998092)(64,0.752073605998092)(64,0.752073605998092)(64,0.752073605998092)(64,0.752073605998092)(64,0.752073605998092)(64,0.752073605998092)(65,0.753236616497293)(65,0.753236616497293)(65,0.753236616497293)(65,0.753236616497293)(65,0.753236616497293)(65,0.753236616497293)(65,0.753236616497293)(65,0.753236616497293)(65,0.753236616497293)(65,0.753236616497293)(65,0.753236616497293)(65,0.753236616497293)(65,0.753236616497293)(65,0.753236616497293)(65,0.753236616497293)(66,0.753818214646085)(66,0.753818214646085)(66,0.753818214646085)(66,0.753818214646085)(66,0.753818214646085)(66,0.753818214646085)(66,0.753818214646085)(66,0.753818214646085)(66,0.753818214646085)(66,0.753818214646085)(66,0.753818214646085)(66,0.753818214646085)(66,0.753818214646085)(67,0.754002236792262)(67,0.754002236792262)(67,0.754002236792262)(67,0.754002236792262)(67,0.754002236792262)(67,0.754002236792262)(67,0.754002236792262)(67,0.754002236792262)(67,0.754002236792262)(67,0.754002236792262)(68,0.754006515929149)(68,0.754006515929149)(68,0.754006515929149)(68,0.754006515929149)(68,0.754006515929149)(68,0.754006515929149)(68,0.754006515929149)(69,0.753802947293)(69,0.753802947293)(69,0.753802947293)(69,0.753802947293)(69,0.753802947293)(69,0.753802947293)(69,0.753802947293)(69,0.753802947293)(69,0.753802947293)(69,0.753802947293)(70,0.753475824176379)(70,0.753475824176379)(70,0.753475824176379)(70,0.753475824176379)(70,0.753475824176379)(70,0.753475824176379)(70,0.753475824176379)(70,0.753475824176379)(70,0.753475824176379)(70,0.753475824176379)(70,0.753475824176379)(70,0.753475824176379)(70,0.753475824176379)(71,0.753703786514765)(71,0.753703786514765)(71,0.753703786514765)(71,0.753703786514765)(72,0.754043808752652)(72,0.754043808752652)(72,0.754043808752652)(72,0.754043808752652)(72,0.754043808752652)(72,0.754043808752652)(72,0.754043808752652)(72,0.754043808752652)(72,0.754043808752652)(72,0.754043808752652)(73,0.754957703475014)(73,0.754957703475014)(73,0.754957703475014)(73,0.754957703475014)(73,0.754957703475014)(73,0.754957703475014)(73,0.754957703475014)(73,0.754957703475014)(73,0.754957703475014)(74,0.756229468186346)(74,0.756229468186346)(74,0.756229468186346)(74,0.756229468186346)(74,0.756229468186346)(74,0.756229468186346)(74,0.756229468186346)(74,0.756229468186346)(74,0.756229468186346)(74,0.756229468186346)(74,0.756229468186346)(74,0.756229468186346)(74,0.756229468186346)(75,0.757882271811075)(75,0.757882271811075)(75,0.757882271811075)(75,0.757882271811075)(75,0.757882271811075)(75,0.757882271811075)(75,0.757882271811075)(75,0.757882271811075)(75,0.757882271811075)(75,0.757882271811075)(76,0.759902508694002)(76,0.759902508694002)(76,0.759902508694002)(76,0.759902508694002)(76,0.759902508694002)(76,0.759902508694002)(76,0.759902508694002)(76,0.759902508694002)(76,0.759902508694002)(77,0.761784220055159)(77,0.761784220055159)(77,0.761784220055159)(77,0.761784220055159)(77,0.761784220055159)(77,0.761784220055159)(77,0.761784220055159)(77,0.761784220055159)(77,0.761784220055159)(77,0.761784220055159)(77,0.761784220055159)(77,0.761784220055159)(77,0.761784220055159)(77,0.761784220055159)(77,0.761784220055159)(77,0.761784220055159)(78,0.762512533555921)(78,0.762512533555921)(78,0.762512533555921)(78,0.762512533555921)(78,0.762512533555921)(78,0.762512533555921)(78,0.762512533555921)(78,0.762512533555921)(78,0.762512533555921)(78,0.762512533555921)(78,0.762512533555921)(79,0.763303214763702)(79,0.763303214763702)(79,0.763303214763702)(79,0.763303214763702)(79,0.763303214763702)(79,0.763303214763702)(79,0.763303214763702)(79,0.763303214763702)(80,0.763673483251285)(80,0.763673483251285)(80,0.763673483251285)(80,0.763673483251285)(80,0.763673483251285)(80,0.763673483251285)(80,0.763673483251285)(81,0.763729757357387)(81,0.763729757357387)(81,0.763729757357387)(81,0.763729757357387)(81,0.763729757357387)(81,0.763729757357387)(81,0.763729757357387)(82,0.763758854169645)(82,0.763758854169645)(82,0.763758854169645)(82,0.763758854169645)(82,0.763758854169645)(82,0.763758854169645)(82,0.763758854169645)(82,0.763758854169645)(82,0.763758854169645)(82,0.763758854169645)(82,0.763758854169645)(82,0.763758854169645)(82,0.763758854169645)(82,0.763758854169645)(82,0.763758854169645)(82,0.763758854169645)(82,0.763758854169645)(83,0.764124793102506)(83,0.764124793102506)(83,0.764124793102506)(83,0.764124793102506)(83,0.764124793102506)(83,0.764124793102506)(83,0.764124793102506)(83,0.764124793102506)(83,0.764124793102506)(84,0.764855458577462)(84,0.764855458577462)(84,0.764855458577462)(84,0.764855458577462)(85,0.765665589202894)(85,0.765665589202894)(85,0.765665589202894)(85,0.765665589202894)(85,0.765665589202894)(85,0.765665589202894)(85,0.765665589202894)(85,0.765665589202894)(85,0.765665589202894)(86,0.766351283731875)(86,0.766351283731875)(86,0.766351283731875)(86,0.766351283731875)(86,0.766351283731875)(86,0.766351283731875)(86,0.766351283731875)(87,0.766537013295831)(87,0.766537013295831)(87,0.766537013295831)(87,0.766537013295831)(87,0.766537013295831)(87,0.766537013295831)(87,0.766537013295831)(87,0.766537013295831)(87,0.766537013295831)(87,0.766537013295831)(87,0.766537013295831)(88,0.766849834943376)(88,0.766849834943376)(88,0.766849834943376)(88,0.766849834943376)(88,0.766849834943376)(88,0.766849834943376)(88,0.766849834943376)(88,0.766849834943376)(88,0.766849834943376)(88,0.766849834943376)(89,0.76756372656697)(89,0.76756372656697)(89,0.76756372656697)(89,0.76756372656697)(89,0.76756372656697)(89,0.76756372656697)(89,0.76756372656697)(89,0.76756372656697)(90,0.767643257309705)(90,0.767643257309705)(90,0.767643257309705)(90,0.767643257309705)(90,0.767643257309705)(91,0.767882893124061)(91,0.767882893124061)(91,0.767882893124061)(91,0.767882893124061)(92,0.767942587825756)(92,0.767942587825756)(92,0.767942587825756)(92,0.767942587825756)(92,0.767942587825756)(92,0.767942587825756)(92,0.767942587825756)(92,0.767942587825756)(92,0.767942587825756)(93,0.767739710208308)(93,0.767739710208308)(93,0.767739710208308)(93,0.767739710208308)(93,0.767739710208308)(93,0.767739710208308)(93,0.767739710208308)(93,0.767739710208308)(93,0.767739710208308)(93,0.767739710208308)(93,0.767739710208308)(93,0.767739710208308)(94,0.767208546626568)(94,0.767208546626568)(94,0.767208546626568)(94,0.767208546626568)(94,0.767208546626568)(94,0.767208546626568)(94,0.767208546626568)(95,0.766506986647339)(95,0.766506986647339)(95,0.766506986647339)(95,0.766506986647339)(95,0.766506986647339)(95,0.766506986647339)(95,0.766506986647339)(95,0.766506986647339)(95,0.766506986647339)(96,0.765985605799003)(96,0.765985605799003)(96,0.765985605799003)(96,0.765985605799003)(97,0.765553425219108)(97,0.765553425219108)(97,0.765553425219108)(97,0.765553425219108)(97,0.765553425219108)(97,0.765553425219108)(98,0.765612273331315)(98,0.765612273331315)(98,0.765612273331315)(98,0.765612273331315)(98,0.765612273331315)(98,0.765612273331315)(98,0.765612273331315)(98,0.765612273331315)(98,0.765612273331315)(98,0.765612273331315)(98,0.765612273331315)(98,0.765612273331315)(98,0.765612273331315)(98,0.765612273331315)(98,0.765612273331315)(99,0.765587647851706)(99,0.765587647851706)(99,0.765587647851706)(100,0.766245787290185)(100,0.766245787290185)(100,0.766245787290185)(100,0.766245787290185)(100,0.766245787290185)(100,0.766245787290185)(100,0.766245787290185)(100,0.766245787290185)(100,0.766245787290185)(101,0.766977760107653)(101,0.766977760107653)(101,0.766977760107653)(101,0.766977760107653)(101,0.766977760107653)(101,0.766977760107653)(101,0.766977760107653)(101,0.766977760107653)(101,0.766977760107653)(101,0.766977760107653)(101,0.766977760107653)(101,0.766977760107653)(101,0.766977760107653)(101,0.766977760107653)(102,0.767961106268534)(102,0.767961106268534)(102,0.767961106268534)(102,0.767961106268534)(102,0.767961106268534)(102,0.767961106268534)(102,0.767961106268534)(102,0.767961106268534)(103,0.768929580968163)(103,0.768929580968163)(103,0.768929580968163)(103,0.768929580968163)(103,0.768929580968163)(103,0.768929580968163)(103,0.768929580968163)(103,0.768929580968163)(103,0.768929580968163)(103,0.768929580968163)(104,0.769854957713883)(104,0.769854957713883)(104,0.769854957713883)(104,0.769854957713883)(104,0.769854957713883)(104,0.769854957713883)(104,0.769854957713883)(104,0.769854957713883)(104,0.769854957713883)(104,0.769854957713883)(105,0.770777550150223)(105,0.770777550150223)(105,0.770777550150223)(105,0.770777550150223)(105,0.770777550150223)(105,0.770777550150223)(105,0.770777550150223)(105,0.770777550150223)(106,0.771768612447841)(106,0.771768612447841)(107,0.772838943041708)(107,0.772838943041708)(107,0.772838943041708)(108,0.77400692026436)(108,0.77400692026436)(108,0.77400692026436)(108,0.77400692026436)(108,0.77400692026436)(108,0.77400692026436)(108,0.77400692026436)(108,0.77400692026436)(109,0.775206897029619)(109,0.775206897029619)(109,0.775206897029619)(109,0.775206897029619)(110,0.775933764123649)(110,0.775933764123649)(110,0.775933764123649)(110,0.775933764123649)(110,0.775933764123649)(110,0.775933764123649)(110,0.775933764123649)(110,0.775933764123649)(111,0.776717605899338)(111,0.776717605899338)(111,0.776717605899338)(111,0.776717605899338)(111,0.776717605899338)(111,0.776717605899338)(111,0.776717605899338)(112,0.776990418442819)(112,0.776990418442819)(112,0.776990418442819)(112,0.776990418442819)(112,0.776990418442819)(113,0.777416143534332)(113,0.777416143534332)(113,0.777416143534332)(113,0.777416143534332)(113,0.777416143534332)(113,0.777416143534332)(113,0.777416143534332)(113,0.777416143534332)(113,0.777416143534332)(113,0.777416143534332)(114,0.77739130777572)(114,0.77739130777572)(114,0.77739130777572)(115,0.777182815347657)(115,0.777182815347657)(115,0.777182815347657)(115,0.777182815347657)(115,0.777182815347657)(115,0.777182815347657)(115,0.777182815347657)(115,0.777182815347657)(115,0.777182815347657)(115,0.777182815347657)(116,0.776970711807132)(116,0.776970711807132)(116,0.776970711807132)(116,0.776970711807132)(117,0.776753491401349)(117,0.776753491401349)(117,0.776753491401349)(117,0.776753491401349)(117,0.776753491401349)(117,0.776753491401349)(117,0.776753491401349)(117,0.776753491401349)(117,0.776753491401349)(118,0.776573174570692)(118,0.776573174570692)(118,0.776573174570692)(118,0.776573174570692)(118,0.776573174570692)(118,0.776573174570692)(118,0.776573174570692)(118,0.776573174570692)(119,0.776559605764593)(119,0.776559605764593)(119,0.776559605764593)(119,0.776559605764593)(120,0.776426031002174)(120,0.776426031002174)(120,0.776426031002174)(120,0.776426031002174)(120,0.776426031002174)(121,0.776350808668501)(121,0.776350808668501)(121,0.776350808668501)(121,0.776350808668501)(121,0.776350808668501)(121,0.776350808668501)(121,0.776350808668501)(121,0.776350808668501)(122,0.776374896361067)(122,0.776374896361067)(122,0.776374896361067)(122,0.776374896361067)(122,0.776374896361067)(122,0.776374896361067)(122,0.776374896361067)(122,0.776374896361067)(123,0.776438564704708)(123,0.776438564704708)(123,0.776438564704708)(123,0.776438564704708)(123,0.776438564704708)(123,0.776438564704708)(124,0.776589633789253)(124,0.776589633789253)(124,0.776589633789253)(124,0.776589633789253)(124,0.776589633789253)(124,0.776589633789253)(125,0.776716788127588)(125,0.776716788127588)(125,0.776716788127588)(125,0.776716788127588)(126,0.777080912612966)(126,0.777080912612966)(126,0.777080912612966)(126,0.777080912612966)(126,0.777080912612966)(127,0.777758559197373)(127,0.777758559197373)(128,0.77832785118691)(128,0.77832785118691)(128,0.77832785118691)(128,0.77832785118691)(129,0.778907661233965)(129,0.778907661233965)(129,0.778907661233965)(129,0.778907661233965)(129,0.778907661233965)(129,0.778907661233965)(129,0.778907661233965)(130,0.77947782459783)(130,0.77947782459783)(130,0.77947782459783)(130,0.77947782459783)(130,0.77947782459783)(131,0.780039295677299)(131,0.780039295677299)(131,0.780039295677299)(131,0.780039295677299)(131,0.780039295677299)(131,0.780039295677299)(131,0.780039295677299)(131,0.780039295677299)(132,0.780567867232089)(132,0.780567867232089)(132,0.780567867232089)(132,0.780567867232089)(132,0.780567867232089)(132,0.780567867232089)(132,0.780567867232089)(133,0.780916216059008)(133,0.780916216059008)(133,0.780916216059008)(133,0.780916216059008)(133,0.780916216059008)(133,0.780916216059008)(134,0.781441077141726)(134,0.781441077141726)(134,0.781441077141726)(134,0.781441077141726)(134,0.781441077141726)(135,0.781592666753051)(135,0.781592666753051)(135,0.781592666753051)(135,0.781592666753051)(135,0.781592666753051)(135,0.781592666753051)(135,0.781592666753051)(136,0.781778834306608)(136,0.781778834306608)(136,0.781778834306608)(136,0.781778834306608)(137,0.781897998430737)(137,0.781897998430737)(137,0.781897998430737)(137,0.781897998430737)(137,0.781897998430737)(137,0.781897998430737)(137,0.781897998430737)(138,0.781992524239363)(138,0.781992524239363)(138,0.781992524239363)(139,0.782038943018313)(139,0.782038943018313)(139,0.782038943018313)(140,0.782121806825756)(140,0.782121806825756)(140,0.782121806825756)(140,0.782121806825756)(140,0.782121806825756)(140,0.782121806825756)(140,0.782121806825756)(141,0.782179955854532)(141,0.782179955854532)(141,0.782179955854532)(142,0.78224337485933)(142,0.78224337485933)(142,0.78224337485933)(142,0.78224337485933)(142,0.78224337485933)(142,0.78224337485933)(142,0.78224337485933)(142,0.78224337485933)(142,0.78224337485933)(143,0.782332680975683)(143,0.782332680975683)(143,0.782332680975683)(143,0.782332680975683)(143,0.782332680975683)(143,0.782332680975683)(143,0.782332680975683)(143,0.782332680975683)(143,0.782332680975683)(144,0.782459627548788)(144,0.782459627548788)(144,0.782459627548788)(145,0.782742237197907)(145,0.782742237197907)(145,0.782742237197907)(145,0.782742237197907)(146,0.782981476099472)(146,0.782981476099472)(146,0.782981476099472)(147,0.783244318947654)(147,0.783244318947654)(147,0.783244318947654)(147,0.783244318947654)(147,0.783244318947654)(147,0.783244318947654)(148,0.783530214670253)(148,0.783530214670253)(148,0.783530214670253)(148,0.783530214670253)(148,0.783530214670253)(148,0.783530214670253)(149,0.783701461721686)(149,0.783701461721686)(149,0.783701461721686)(149,0.783701461721686)(149,0.783701461721686)(149,0.783701461721686)(150,0.784099132211003)(150,0.784099132211003)(150,0.784099132211003)(150,0.784099132211003)(151,0.784574541764254)(151,0.784574541764254)(151,0.784574541764254)(151,0.784574541764254)(151,0.784574541764254)(152,0.785117492236431)(152,0.785117492236431)(152,0.785117492236431)(153,0.785609274570725)(153,0.785609274570725)(153,0.785609274570725)(153,0.785609274570725)(153,0.785609274570725)(153,0.785609274570725)(153,0.785609274570725)(153,0.785609274570725)(154,0.786114039828424)(154,0.786114039828424)(154,0.786114039828424)(154,0.786114039828424)(154,0.786114039828424)(155,0.786595497142642)(155,0.786595497142642)(155,0.786595497142642)(155,0.786595497142642)(155,0.786595497142642)(156,0.786910267573875)(156,0.786910267573875)(156,0.786910267573875)(156,0.786910267573875)(157,0.78730074264201)(157,0.78730074264201)(157,0.78730074264201)(157,0.78730074264201)(157,0.78730074264201)(157,0.78730074264201)(158,0.787495285065303)(158,0.787495285065303)(158,0.787495285065303)(158,0.787495285065303)(159,0.787775682257692)(159,0.787775682257692)(159,0.787775682257692)(159,0.787775682257692)(159,0.787775682257692)(160,0.787996492779602)(160,0.787996492779602)(161,0.788154566603543)(161,0.788154566603543)(161,0.788154566603543)(161,0.788154566603543)(161,0.788154566603543)(161,0.788154566603543)(161,0.788154566603543)(162,0.788158541399003)(162,0.788158541399003)(162,0.788158541399003)(163,0.788210237754538)(163,0.788210237754538)(163,0.788210237754538)(163,0.788210237754538)(163,0.788210237754538)(163,0.788210237754538)(164,0.788275210379421)(164,0.788275210379421)(164,0.788275210379421)(165,0.788263161800976)(165,0.788263161800976)(165,0.788263161800976)(165,0.788263161800976)(165,0.788263161800976)(165,0.788263161800976)(166,0.788268786037901)(166,0.788268786037901)(166,0.788268786037901)(167,0.788393810304309)(167,0.788393810304309)(167,0.788393810304309)(168,0.788513321194605)(168,0.788513321194605)(168,0.788513321194605)(168,0.788513321194605)(168,0.788513321194605)(168,0.788513321194605)(168,0.788513321194605)(169,0.788653416788086)(169,0.788653416788086)(169,0.788653416788086)(169,0.788653416788086)(169,0.788653416788086)(169,0.788653416788086)(170,0.788812387486247)(170,0.788812387486247)(170,0.788812387486247)(170,0.788812387486247)(170,0.788812387486247)(170,0.788812387486247)(171,0.788994194687663)(171,0.788994194687663)(171,0.788994194687663)(171,0.788994194687663)(171,0.788994194687663)(171,0.788994194687663)(171,0.788994194687663)(172,0.789210981148079)(172,0.789210981148079)(172,0.789210981148079)(172,0.789210981148079)(172,0.789210981148079)(172,0.789210981148079)(172,0.789210981148079)(172,0.789210981148079)(173,0.78947342119388)(173,0.78947342119388)(173,0.78947342119388)(175,0.790130189045382)(175,0.790130189045382)(175,0.790130189045382)(176,0.790437309063971)(176,0.790437309063971)(176,0.790437309063971)(176,0.790437309063971)(177,0.790745657309972)(177,0.790745657309972)(177,0.790745657309972)(177,0.790745657309972)(178,0.791057651351311)(178,0.791057651351311)(178,0.791057651351311)(179,0.791383643366885)(179,0.791383643366885)(179,0.791383643366885)(180,0.79172171510835)(180,0.79172171510835)(181,0.791959920241933)(182,0.792245786141415)(182,0.792245786141415)(183,0.792492308744088)(183,0.792492308744088)(183,0.792492308744088)(184,0.792686039213119)(184,0.792686039213119)(184,0.792686039213119)(184,0.792686039213119)(184,0.792686039213119)(185,0.792820637558201)(185,0.792820637558201)(185,0.792820637558201)(185,0.792820637558201)(185,0.792820637558201)(185,0.792820637558201)(185,0.792820637558201)(186,0.792905799525866)(186,0.792905799525866)(187,0.792954071359319)(187,0.792954071359319)(187,0.792954071359319)(188,0.792970647362965)(188,0.792970647362965)(188,0.792970647362965)(188,0.792970647362965)(188,0.792970647362965)(189,0.792958088048318)(189,0.792958088048318)(189,0.792958088048318)(190,0.792917434784473)(190,0.792917434784473)(190,0.792917434784473)(190,0.792917434784473)(190,0.792917434784473)(190,0.792917434784473)(190,0.792917434784473)(191,0.792846885012946)(191,0.792846885012946)(191,0.792846885012946)(191,0.792846885012946)(191,0.792846885012946)(191,0.792846885012946)(192,0.792741178247772)(192,0.792741178247772)(193,0.792597689934038)(193,0.792597689934038)(193,0.792597689934038)(193,0.792597689934038)(194,0.792419283427795)(194,0.792419283427795)(194,0.792419283427795)(194,0.792419283427795)(194,0.792419283427795)(195,0.792204280756916)(195,0.792204280756916)(195,0.792204280756916)(195,0.792204280756916)(195,0.792204280756916)(195,0.792204280756916)(195,0.792204280756916)(196,0.791938212835765)(196,0.791938212835765)(196,0.791938212835765)(196,0.791938212835765)(196,0.791938212835765)(196,0.791938212835765)(196,0.791938212835765)(197,0.791705692412493)(197,0.791705692412493)(197,0.791705692412493)(198,0.791443987456728)(198,0.791443987456728)(198,0.791443987456728)(198,0.791443987456728)(198,0.791443987456728)(199,0.791191196860658)(199,0.791191196860658)(199,0.791191196860658)(199,0.791191196860658)(201,0.790790731794944)(201,0.790790731794944)(201,0.790790731794944)(201,0.790790731794944)(202,0.790668774881862)(202,0.790668774881862)(202,0.790668774881862)(203,0.79048306884299)(203,0.79048306884299)(203,0.79048306884299)(203,0.79048306884299)(203,0.79048306884299)(204,0.790398066151006)(204,0.790398066151006)(204,0.790398066151006)(204,0.790398066151006)(204,0.790398066151006)(204,0.790398066151006)(204,0.790398066151006)(204,0.790398066151006)(205,0.790365477348768)(206,0.790379527795225)(206,0.790379527795225)(206,0.790379527795225)(207,0.790429108049657)(207,0.790429108049657)(208,0.790505773779099)(208,0.790505773779099)(209,0.790643744020599)(209,0.790643744020599)(209,0.790643744020599)(210,0.79070156682037)(210,0.79070156682037)(210,0.79070156682037)(210,0.79070156682037)(210,0.79070156682037)(210,0.79070156682037)(210,0.79070156682037)(210,0.79070156682037)(211,0.790796741235632)(211,0.790796741235632)(211,0.790796741235632)(211,0.790796741235632)(212,0.790878482758452)(212,0.790878482758452)(212,0.790878482758452)(212,0.790878482758452)(212,0.790878482758452)(213,0.790943102857028)(213,0.790943102857028)(213,0.790943102857028)(213,0.790943102857028)(214,0.790995936746449)(214,0.790995936746449)(215,0.791043673676354)(215,0.791043673676354)(215,0.791043673676354)(215,0.791043673676354)(216,0.791130143619422)(216,0.791130143619422)(216,0.791130143619422)(216,0.791130143619422)(217,0.791131507386545)(218,0.791172192098115)(218,0.791172192098115)(218,0.791172192098115)(218,0.791172192098115)(218,0.791172192098115)(218,0.791172192098115)(219,0.791214244574525)(219,0.791214244574525)(219,0.791214244574525)(220,0.791264641076783)(220,0.791264641076783)(220,0.791264641076783)(221,0.791326637587454)(221,0.791326637587454)(221,0.791326637587454)(221,0.791326637587454)(221,0.791326637587454)(222,0.791402276651874)(222,0.791402276651874)(222,0.791402276651874)(223,0.791488552755603)(223,0.791488552755603)(223,0.791488552755603)(223,0.791488552755603)(223,0.791488552755603)(223,0.791488552755603)(223,0.791488552755603)(223,0.791488552755603)(224,0.791584260796035)(224,0.791584260796035)(224,0.791584260796035)(224,0.791584260796035)(225,0.791670596228202)(225,0.791670596228202)(225,0.791670596228202)(225,0.791670596228202)(225,0.791670596228202)(225,0.791670596228202)(226,0.791796484080094)(226,0.791796484080094)(226,0.791796484080094)(226,0.791796484080094)(228,0.792080794201251)(228,0.792080794201251)(229,0.792240477970254)(229,0.792240477970254)(229,0.792240477970254)(229,0.792240477970254)(230,0.792411014738806)(230,0.792411014738806)(231,0.792593587909671)(231,0.792593587909671)(231,0.792593587909671)(231,0.792593587909671)(232,0.792792161922309)(232,0.792792161922309)(232,0.792792161922309)(232,0.792792161922309)(233,0.793006401235755)(233,0.793006401235755)(233,0.793006401235755)(234,0.793233692331201)(234,0.793233692331201)(234,0.793233692331201)(234,0.793233692331201)(234,0.793233692331201)(235,0.793570691851716)(236,0.793728584626299)(236,0.793728584626299)(236,0.793728584626299)(237,0.793995184077979)(237,0.793995184077979)(238,0.79427477977406)(238,0.79427477977406)(239,0.794565872943399)(239,0.794565872943399)(239,0.794565872943399)(239,0.794565872943399)(240,0.794866049615254)(240,0.794866049615254)(241,0.795176455030166)(241,0.795176455030166)(241,0.795176455030166)(241,0.795176455030166)(241,0.795176455030166)(241,0.795176455030166)(241,0.795176455030166)(242,0.7954614227268)(242,0.7954614227268)(242,0.7954614227268)(242,0.7954614227268)(242,0.7954614227268)(242,0.7954614227268)(242,0.7954614227268)(243,0.795746078234577)(243,0.795746078234577)(244,0.795986426582883)(244,0.795986426582883)(244,0.795986426582883)(244,0.795986426582883)(244,0.795986426582883)(245,0.796254869324495)(246,0.796580814804043)(246,0.796580814804043)(246,0.796580814804043)(246,0.796580814804043)(247,0.796853878061423)(247,0.796853878061423)(247,0.796853878061423)(247,0.796853878061423)(247,0.796853878061423)(248,0.797194786722235)(248,0.797194786722235)(248,0.797194786722235)(249,0.79737040380668)(249,0.79737040380668)(249,0.79737040380668)(250,0.797672928988138)(250,0.797672928988138)(250,0.797672928988138)(250,0.797672928988138)(250,0.797672928988138)(251,0.797866526520467)(251,0.797866526520467)(251,0.797866526520467)(251,0.797866526520467)(251,0.797866526520467)(252,0.798012034422828)(252,0.798012034422828)(252,0.798012034422828)(253,0.798101184055129)(253,0.798101184055129)(253,0.798101184055129)(253,0.798101184055129)(254,0.798112312677352)(255,0.798110260706821)(255,0.798110260706821)(256,0.798062375447558)(256,0.798062375447558)(256,0.798062375447558)(256,0.798062375447558)(256,0.798062375447558)(257,0.797974402466393)(257,0.797974402466393)(257,0.797974402466393)(257,0.797974402466393)(258,0.797869771416084)(258,0.797869771416084)(258,0.797869771416084)(258,0.797869771416084)(259,0.797728175036331)(259,0.797728175036331)(259,0.797728175036331)(259,0.797728175036331)(259,0.797728175036331)(259,0.797728175036331)(259,0.797728175036331)(260,0.797567907243253)(260,0.797567907243253)(260,0.797567907243253)(261,0.797390389266776)(262,0.797158807953064)(262,0.797158807953064)(262,0.797158807953064)(262,0.797158807953064)(262,0.797158807953064)(264,0.796741955908033)(264,0.796741955908033)(264,0.796741955908033)(264,0.796741955908033)(264,0.796741955908033)(264,0.796741955908033)(264,0.796741955908033)(264,0.796741955908033)(265,0.796532448639976)(265,0.796532448639976)(265,0.796532448639976)(266,0.796325623359445)(266,0.796325623359445)(266,0.796325623359445)(266,0.796325623359445)(267,0.796122690337567)(268,0.79592735765172)(268,0.79592735765172)(268,0.79592735765172)(269,0.795749450343019)(269,0.795749450343019)(269,0.795749450343019)(270,0.795558632933266)(270,0.795558632933266)(270,0.795558632933266)(271,0.795442920248629)(271,0.795442920248629)(271,0.795442920248629)(271,0.795442920248629)(271,0.795442920248629)(272,0.795413919402723)(272,0.795413919402723)(273,0.795375476039259)(273,0.795375476039259)(274,0.795443475457189)(275,0.795390039916379)(275,0.795390039916379)(275,0.795390039916379)(275,0.795390039916379)(276,0.795511529645296)(276,0.795511529645296)(276,0.795511529645296)(277,0.795568964772588)(277,0.795568964772588)(277,0.795568964772588)(277,0.795568964772588)(277,0.795568964772588)(277,0.795568964772588)(277,0.795568964772588)(277,0.795568964772588)(277,0.795568964772588)(277,0.795568964772588)(278,0.795635954771233)(278,0.795635954771233)(278,0.795635954771233)(278,0.795635954771233)(278,0.795635954771233)(278,0.795635954771233)(279,0.795709414687943)(279,0.795709414687943)(280,0.795788668306405)(280,0.795788668306405)(280,0.795788668306405)(280,0.795788668306405)(281,0.795873100473567)(281,0.795873100473567)(281,0.795873100473567)(282,0.79597380099149)(282,0.79597380099149)(282,0.79597380099149)(282,0.79597380099149)(282,0.79597380099149)(282,0.79597380099149)(283,0.796067458010603)(283,0.796067458010603)(283,0.796067458010603)(283,0.796067458010603)(283,0.796067458010603)(285,0.796282612108557)(285,0.796282612108557)(285,0.796282612108557)(285,0.796282612108557)(285,0.796282612108557)(285,0.796282612108557)(285,0.796282612108557)(285,0.796282612108557)(286,0.796406519596189)(286,0.796406519596189)(286,0.796406519596189)(286,0.796406519596189)(286,0.796406519596189)(287,0.79654159909067)(288,0.796686524172997)(288,0.796686524172997)(288,0.796686524172997)(288,0.796686524172997)(288,0.796686524172997)(289,0.796874106967489)(289,0.796874106967489)(289,0.796874106967489)(289,0.796874106967489)(289,0.796874106967489)(290,0.797046645047689)(290,0.797046645047689)(291,0.797228613824763)(292,0.797347472229211)(292,0.797347472229211)(293,0.797527746436742)(293,0.797527746436742)(293,0.797527746436742)(293,0.797527746436742)(295,0.797867831081274)(295,0.797867831081274)(295,0.797867831081274)(295,0.797867831081274)(295,0.797867831081274)(296,0.798058498252237)(296,0.798058498252237)(296,0.798058498252237)(297,0.798128710610697)(297,0.798128710610697)(297,0.798128710610697)(297,0.798128710610697)(297,0.798128710610697)(298,0.798224038016853)(299,0.798297082649966)(299,0.798297082649966)(299,0.798297082649966)(300,0.798350047257236)(300,0.798350047257236)(300,0.798350047257236)(300,0.798350047257236)(300,0.798350047257236)(301,0.798367141296904)(301,0.798367141296904)(302,0.798365756427919)(302,0.798365756427919)(302,0.798365756427919)(302,0.798365756427919)(303,0.798350540835978)(303,0.798350540835978)(304,0.798352480267071)(304,0.798352480267071)(304,0.798352480267071)(304,0.798352480267071)(304,0.798352480267071)(305,0.798324355326704)(305,0.798324355326704)(306,0.798270674886644)(306,0.798270674886644)(306,0.798270674886644)(307,0.798244631019805)(307,0.798244631019805)(307,0.798244631019805)(308,0.798224215896922)(309,0.798214721238291)(309,0.798214721238291)(309,0.798214721238291)(310,0.798221464197896)(310,0.798221464197896)(310,0.798221464197896)(310,0.798221464197896)(311,0.798271299168801)(311,0.798271299168801)(312,0.798316156632238)(312,0.798316156632238)(312,0.798316156632238)(313,0.798374244404221)(313,0.798374244404221)(313,0.798374244404221)(314,0.798447373897239)(315,0.798534743783038)(315,0.798534743783038)(315,0.798534743783038)(315,0.798534743783038)(316,0.798631809307957)(316,0.798631809307957)(316,0.798631809307957)(316,0.798631809307957)(317,0.798726170775643)(317,0.798726170775643)(318,0.798833768261512)(318,0.798833768261512)(318,0.798833768261512)(318,0.798833768261512)(319,0.798925035555462)(319,0.798925035555462)(319,0.798925035555462)(320,0.799028310604209)(320,0.799028310604209)(320,0.799028310604209)(321,0.799132672422658)(321,0.799132672422658)(321,0.799132672422658)(321,0.799132672422658)(321,0.799132672422658)(321,0.799132672422658)(321,0.799132672422658)(321,0.799132672422658)(321,0.799132672422658)(322,0.799236593770725)(322,0.799236593770725)(322,0.799236593770725)(322,0.799236593770725)(322,0.799236593770725)(323,0.799338773497492)(323,0.799338773497492)(324,0.799440924721508)(325,0.799542161804417)(325,0.799542161804417)(326,0.799650191144582)(326,0.799650191144582)(327,0.799746503785896)(327,0.799746503785896)(327,0.799746503785896)(327,0.799746503785896)(328,0.799849778861849)(328,0.799849778861849)(328,0.799849778861849)(328,0.799849778861849)(328,0.799849778861849)(328,0.799849778861849)(328,0.799849778861849)(328,0.799849778861849)(329,0.799954486758544)(329,0.799954486758544)(329,0.799954486758544)(330,0.80006166074165)(331,0.800173352763028)(331,0.800173352763028)(331,0.800173352763028)(331,0.800173352763028)(331,0.800173352763028)(332,0.800291199074465)(332,0.800291199074465)(332,0.800291199074465)(332,0.800291199074465)(333,0.800415480702992)(333,0.800415480702992)(333,0.800415480702992)(333,0.800415480702992)(333,0.800415480702992)(334,0.800544968681394)(334,0.800544968681394)(334,0.800544968681394)(335,0.800678370711044)(336,0.800812793915223)(336,0.800812793915223)(336,0.800812793915223)(336,0.800812793915223)(337,0.800942790293142)(338,0.801062731533949)(338,0.801062731533949)(338,0.801062731533949)(338,0.801062731533949)(338,0.801062731533949)(338,0.801062731533949)(338,0.801062731533949)(339,0.801138650319161)(340,0.801227354885831)(340,0.801227354885831)(340,0.801227354885831)(340,0.801227354885831)(340,0.801227354885831)(341,0.801303370698485)(341,0.801303370698485)(341,0.801303370698485)(341,0.801303370698485)(341,0.801303370698485)(341,0.801303370698485)(341,0.801303370698485)(342,0.801366577581228)(342,0.801366577581228)(342,0.801366577581228)(343,0.801451283519831)(343,0.801451283519831)(343,0.801451283519831)(345,0.801530272967046)(345,0.801530272967046)(345,0.801530272967046)(345,0.801530272967046)(345,0.801530272967046)(346,0.801560023458871)(346,0.801560023458871)(346,0.801560023458871)(346,0.801560023458871)(346,0.801560023458871)(347,0.801586295997223)(347,0.801586295997223)(347,0.801586295997223)(348,0.801599829939425)(348,0.801599829939425)(348,0.801599829939425)(348,0.801599829939425)(350,0.801656448827499)(350,0.801656448827499)(351,0.801696204934133)(351,0.801696204934133)(351,0.801696204934133)(352,0.801731030384318)(352,0.801731030384318)(352,0.801731030384318)(352,0.801731030384318)(352,0.801731030384318)(353,0.801770611183601)(353,0.801770611183601)(353,0.801770611183601)(354,0.801816399260813)(354,0.801816399260813)(354,0.801816399260813)(355,0.801869156699878)(355,0.801869156699878)(356,0.801931133130102)(357,0.802003754346429)(357,0.802003754346429)(357,0.802003754346429)(357,0.802003754346429)(358,0.802087502680741)(358,0.802087502680741)(359,0.802182216985413)(360,0.802287495118908)(360,0.802287495118908)(360,0.802287495118908)(362,0.802498148156356)(362,0.802498148156356)(362,0.802498148156356)(362,0.802498148156356)(363,0.802632274994029)(363,0.802632274994029)(363,0.802632274994029)(364,0.802797655985797)(364,0.802797655985797)(365,0.802911625073855)(365,0.802911625073855)(365,0.802911625073855)(366,0.803052020246357)(366,0.803052020246357)(366,0.803052020246357)(367,0.803208060250011)(367,0.803208060250011)(367,0.803208060250011)(367,0.803208060250011)(367,0.803208060250011)(368,0.80334970523894)(368,0.80334970523894)(368,0.80334970523894)(368,0.80334970523894)(368,0.80334970523894)(369,0.803480416580085)(369,0.803480416580085)(369,0.803480416580085)(370,0.803604505420449)(371,0.803725193974536)(371,0.803725193974536)(371,0.803725193974536)(371,0.803725193974536)(372,0.80384450335638)(373,0.803963622743954)(373,0.803963622743954)(373,0.803963622743954)(374,0.804083206462945)(375,0.804203584795769)(375,0.804203584795769)(376,0.804324917450758)(376,0.804324917450758)(376,0.804324917450758)(376,0.804324917450758)(377,0.804447243992879)(377,0.804447243992879)(377,0.804447243992879)(378,0.804570493072554)(378,0.804570493072554)(378,0.804570493072554)(378,0.804570493072554)(378,0.804570493072554)(378,0.804570493072554)(378,0.804570493072554)(378,0.804570493072554)(378,0.804570493072554)(379,0.804694532572115)(380,0.804819251128101)(380,0.804819251128101)(381,0.804944531434792)(381,0.804944531434792)(382,0.805070269805505)(382,0.805070269805505)(383,0.805196356614702)(384,0.805322649756516)(385,0.805449009847791)(385,0.805449009847791)(385,0.805449009847791)(385,0.805449009847791)(385,0.805449009847791)(387,0.805701396260131)(387,0.805701396260131)(387,0.805701396260131)(388,0.805827212511286)(388,0.805827212511286)(388,0.805827212511286)(389,0.805952634251133)(389,0.805952634251133)(390,0.806077552095367)(390,0.806077552095367)(390,0.806077552095367)(390,0.806077552095367)(390,0.806077552095367)(391,0.806201856329716)(391,0.806201856329716)(391,0.806201856329716)(392,0.806325460597602)(392,0.806325460597602)(392,0.806325460597602)(392,0.806325460597602)(392,0.806325460597602)(392,0.806325460597602)(392,0.806325460597602)(393,0.806448282846077)(393,0.806448282846077)(393,0.806448282846077)(394,0.806570260724749)(394,0.806570260724749)(395,0.806691356197323)(395,0.806691356197323)(395,0.806691356197323)(395,0.806691356197323)(395,0.806691356197323)(396,0.806811536838544)(396,0.806811536838544)(396,0.806811536838544)(397,0.806930784977884)(397,0.806930784977884)(397,0.806930784977884)(397,0.806930784977884)(398,0.807049118956831)(398,0.807049118956831)(398,0.807049118956831)(398,0.807049118956831)(398,0.807049118956831)(399,0.807166589442893)(399,0.807166589442893)(399,0.807166589442893)(399,0.807166589442893)(399,0.807166589442893)(400,0.807283220809706)(400,0.807283220809706)(400,0.807283220809706)(400,0.807283220809706)(400,0.807283220809706)(400,0.807283220809706) 
};
\addlegendentry{\acl (random)};

\addplot [
color=orange,
densely dotted,
line width=1.0pt,
]
coordinates{
 (3,0.529885670577277)(4,0.529843588740064)(5,0.524114225843782)(6,0.5519040941856)(7,0.570614778741484)(8,0.597870621270671)(9,0.603900239804462)(10,0.605505889789711)(11,0.628012153331517)(12,0.635657902273215)(13,0.649414359157191)(14,0.645843347863916)(15,0.658689747947712)(16,0.664140357849611)(17,0.659407755089051)(18,0.66032728802541)(19,0.664065901611842)(20,0.661379778379683)(21,0.665654923156283)(22,0.665683641465307)(23,0.671935812122478)(24,0.675127384443139)(25,0.679248749498767)(26,0.682859587929769)(27,0.680034333799083)(28,0.679449506928499)(29,0.681464064140961)(30,0.683765967628062)(31,0.687507543237132)(32,0.688239264355242)(33,0.693167398743821)(34,0.693542460067808)(35,0.69437146586143)(36,0.694979151183891)(37,0.69440132837617)(38,0.688760986171016)(39,0.689787698386667)(40,0.687706569824396)(41,0.686786914423253)(42,0.687414142540112)(43,0.687625898890657)(44,0.685115567197897)(45,0.691250489029271)(46,0.696100058014429)(47,0.686539251745561)(48,0.689238260241043)(49,0.691980411096248)(50,0.695848791660666)(51,0.696044236412898)(52,0.705459178230364)(53,0.704214335294907)(54,0.708699392230451)(55,0.710071908938195)(56,0.708779080563867)(57,0.706750842471612)(58,0.708754144419424)(59,0.707655495309179)(60,0.709065761042602)(61,0.711381383036992)(62,0.711609955539584)(63,0.710853781121195)(64,0.712610143082108)(65,0.716975388572815)(66,0.716217562293735)(67,0.720426823461158)(68,0.724836311391687)(69,0.724436748215109)(70,0.718569332224268)(71,0.719057764032513)(72,0.720182828215946)(73,0.722481800118811)(74,0.724881353022124)(75,0.726812977394103)(76,0.727880686939858)(77,0.727635691692697)(78,0.727037366403538)(79,0.727379976975789)(80,0.726951536288008)(81,0.726877559383331)(82,0.7301967298349)(83,0.732714907001867)(84,0.732795517963862)(85,0.733831628722151)(86,0.736601476671502)(87,0.737954305432674)(88,0.744589620398604)(89,0.742465727200403)(90,0.742328539575824)(91,0.743091726054681)(92,0.743729501160541)(93,0.744507896923761)(94,0.745218527778483)(95,0.74677810907214)(96,0.746100857462296)(97,0.748712600387868)(98,0.74905888560449)(99,0.748522677261229)(100,0.748537038928545)(101,0.748537038928545)(102,0.748925808149851)(103,0.75318662974284)(104,0.75318662974284)(105,0.753448158881041)(106,0.753448158881041)(107,0.753448158881041)(108,0.753665196567531)(109,0.751990390020027)(110,0.750614630304116)(111,0.749732532255277)(112,0.749732532255277)(113,0.750352328972793)(114,0.750017407483551)(115,0.749510498864578)(116,0.752099716919845)(117,0.752594702522069)(118,0.75267708521145)(119,0.752338878573369)(120,0.75250152660266)(121,0.752392147378459)(122,0.752518845117274)(123,0.752544365638472)(124,0.752965235882305)(125,0.752746246463625)(126,0.753857071905308)(127,0.756276061809886)(128,0.757766464522244)(129,0.757057031807639)(130,0.758442385837553)(131,0.7583639239851)(132,0.760007490262129)(133,0.760390137114096)(134,0.760861164625323)(135,0.76132040367338)(136,0.76098341084395)(137,0.762058684455356)(138,0.7629191123568)(139,0.762919918803364)(140,0.765299518013368)(141,0.764613638580786)(142,0.764415856636005)(143,0.763832538548918)(144,0.764364522614635)(145,0.764451902218873)(146,0.763396273817034)(147,0.763464560322398)(148,0.763880205652354)(149,0.765766227936117)(150,0.767283980481775)(151,0.767416137163271)(152,0.767870942308001)(153,0.766173402688127)(154,0.76726774679801)(155,0.767695210281549)(156,0.767807719924637)(157,0.767253921248296)(158,0.767253921248296)(159,0.76800643901219)(160,0.766885882929815)(161,0.766885882929815)(162,0.767265257281685)(163,0.767265257281685)(164,0.769035549682838)(165,0.76935307002408)(166,0.769727430210032)(167,0.768863130319417)(168,0.768202368072665)(169,0.767431234107964)(170,0.767167227817826)(171,0.767430578738714)(172,0.766715082948637)(173,0.766351357915775)(174,0.766469492986171)(175,0.7689482925571)(176,0.769125182021664)(177,0.768661868877175)(178,0.76912184756066)(179,0.769445375935179)(180,0.769650161868317)(181,0.768892022283844)(182,0.768892022283844)(183,0.769713971764806)(184,0.769069769652524)(185,0.770778667770684)(186,0.770944682745329)(187,0.77186962242669)(188,0.77186962242669)(189,0.772122261820134)(190,0.772195645094862)(191,0.772231380091438)(192,0.772115010045484)(193,0.771776894967998)(194,0.772630518001133)(195,0.773017530817131)(196,0.77332077406715)(197,0.773064121400656)(198,0.772900101723903)(199,0.772557747885649)(200,0.772768234199896)
};
\addlegendentry{\var};

\end{axis}
\end{tikzpicture}%

%% This file was created by matlab2tikz v0.2.3.
% Copyright (c) 2008--2012, Nico Schlömer <nico.schloemer@gmail.com>
% All rights reserved.
% 
% 
% 
\begin{tikzpicture}

\begin{axis}[%
tick label style={font=\tiny},
label style={font=\tiny},
label shift={-4pt},
xlabel shift={-6pt},
legend style={font=\tiny},
view={0}{90},
width=\figurewidth,
height=\figureheight,
scale only axis,
xmin=0, xmax=400,
xlabel={Samples},
ymin=0.48, ymax=1,
ylabel={$F_1$-score},
axis lines*=left,
legend cell align=left,
legend style={at={(1.03,0)},anchor=south east,fill=none,draw=none,align=left,row sep=-0.2em},
clip=false]

\addplot [
color=blue,
solid,
line width=1.0pt,
]
coordinates{
 (12,0.517371826092692)(12,0.517371826092692)(12,0.517371826092692)(12,0.517371826092692)(12,0.517371826092692)(12,0.517371826092692)(12,0.517371826092692)(12,0.517371826092692)(12,0.517371826092692)(13,0.567975669248341)(13,0.567975669248341)(13,0.567975669248341)(13,0.567975669248341)(13,0.567975669248341)(13,0.567975669248341)(13,0.567975669248341)(13,0.567975669248341)(13,0.567975669248341)(14,0.569441294640862)(14,0.569441294640862)(14,0.569441294640862)(14,0.569441294640862)(14,0.569441294640862)(14,0.569441294640862)(14,0.569441294640862)(14,0.569441294640862)(15,0.588370596497455)(15,0.588370596497455)(15,0.588370596497455)(15,0.588370596497455)(15,0.588370596497455)(15,0.588370596497455)(15,0.588370596497455)(15,0.588370596497455)(15,0.588370596497455)(15,0.588370596497455)(15,0.588370596497455)(15,0.588370596497455)(15,0.588370596497455)(16,0.615577359956341)(16,0.615577359956341)(16,0.615577359956341)(16,0.615577359956341)(16,0.615577359956341)(16,0.615577359956341)(16,0.615577359956341)(16,0.615577359956341)(16,0.615577359956341)(16,0.615577359956341)(16,0.615577359956341)(16,0.615577359956341)(17,0.626135321543159)(17,0.626135321543159)(17,0.626135321543159)(17,0.626135321543159)(17,0.626135321543159)(17,0.626135321543159)(17,0.626135321543159)(17,0.626135321543159)(17,0.626135321543159)(17,0.626135321543159)(17,0.626135321543159)(18,0.653257593402505)(18,0.653257593402505)(18,0.653257593402505)(18,0.653257593402505)(18,0.653257593402505)(18,0.653257593402505)(18,0.653257593402505)(19,0.653676147192342)(19,0.653676147192342)(19,0.653676147192342)(19,0.653676147192342)(19,0.653676147192342)(19,0.653676147192342)(19,0.653676147192342)(19,0.653676147192342)(20,0.673926213836114)(20,0.673926213836114)(20,0.673926213836114)(20,0.673926213836114)(20,0.673926213836114)(20,0.673926213836114)(20,0.673926213836114)(20,0.673926213836114)(20,0.673926213836114)(20,0.673926213836114)(20,0.673926213836114)(20,0.673926213836114)(20,0.673926213836114)(20,0.673926213836114)(20,0.673926213836114)(20,0.673926213836114)(20,0.673926213836114)(20,0.673926213836114)(21,0.688932094258866)(21,0.688932094258866)(21,0.688932094258866)(21,0.688932094258866)(21,0.688932094258866)(21,0.688932094258866)(21,0.688932094258866)(21,0.688932094258866)(21,0.688932094258866)(21,0.688932094258866)(21,0.688932094258866)(22,0.699517139071227)(22,0.699517139071227)(22,0.699517139071227)(22,0.699517139071227)(22,0.699517139071227)(22,0.699517139071227)(22,0.699517139071227)(22,0.699517139071227)(22,0.699517139071227)(23,0.714538528120473)(23,0.714538528120473)(23,0.714538528120473)(23,0.714538528120473)(23,0.714538528120473)(23,0.714538528120473)(24,0.727233730141643)(24,0.727233730141643)(24,0.727233730141643)(24,0.727233730141643)(24,0.727233730141643)(24,0.727233730141643)(24,0.727233730141643)(24,0.727233730141643)(24,0.727233730141643)(24,0.727233730141643)(24,0.727233730141643)(25,0.730148443116798)(25,0.730148443116798)(25,0.730148443116798)(25,0.730148443116798)(25,0.730148443116798)(25,0.730148443116798)(25,0.730148443116798)(25,0.730148443116798)(25,0.730148443116798)(26,0.734937439862497)(26,0.734937439862497)(26,0.734937439862497)(26,0.734937439862497)(26,0.734937439862497)(26,0.734937439862497)(26,0.734937439862497)(27,0.740354710135151)(27,0.740354710135151)(27,0.740354710135151)(27,0.740354710135151)(27,0.740354710135151)(27,0.740354710135151)(27,0.740354710135151)(27,0.740354710135151)(28,0.743497222609591)(28,0.743497222609591)(28,0.743497222609591)(28,0.743497222609591)(28,0.743497222609591)(28,0.743497222609591)(28,0.743497222609591)(28,0.743497222609591)(29,0.748585867138197)(29,0.748585867138197)(29,0.748585867138197)(29,0.748585867138197)(29,0.748585867138197)(29,0.748585867138197)(29,0.748585867138197)(29,0.748585867138197)(29,0.748585867138197)(29,0.748585867138197)(29,0.748585867138197)(29,0.748585867138197)(29,0.748585867138197)(29,0.748585867138197)(30,0.754083660285682)(30,0.754083660285682)(30,0.754083660285682)(30,0.754083660285682)(30,0.754083660285682)(30,0.754083660285682)(30,0.754083660285682)(30,0.754083660285682)(30,0.754083660285682)(31,0.766814127780325)(31,0.766814127780325)(31,0.766814127780325)(31,0.766814127780325)(31,0.766814127780325)(31,0.766814127780325)(31,0.766814127780325)(31,0.766814127780325)(31,0.766814127780325)(31,0.766814127780325)(31,0.766814127780325)(31,0.766814127780325)(31,0.766814127780325)(32,0.775615396888595)(32,0.775615396888595)(32,0.775615396888595)(32,0.775615396888595)(33,0.77677136000213)(33,0.77677136000213)(33,0.77677136000213)(33,0.77677136000213)(33,0.77677136000213)(33,0.77677136000213)(33,0.77677136000213)(34,0.779809036305868)(34,0.779809036305868)(34,0.779809036305868)(34,0.779809036305868)(34,0.779809036305868)(34,0.779809036305868)(34,0.779809036305868)(35,0.779578387754309)(35,0.779578387754309)(35,0.779578387754309)(35,0.779578387754309)(35,0.779578387754309)(35,0.779578387754309)(35,0.779578387754309)(36,0.784137397375464)(36,0.784137397375464)(36,0.784137397375464)(36,0.784137397375464)(36,0.784137397375464)(36,0.784137397375464)(36,0.784137397375464)(37,0.798209756602057)(37,0.798209756602057)(37,0.798209756602057)(37,0.798209756602057)(37,0.798209756602057)(38,0.807148017815209)(38,0.807148017815209)(38,0.807148017815209)(38,0.807148017815209)(38,0.807148017815209)(38,0.807148017815209)(38,0.807148017815209)(38,0.807148017815209)(39,0.810705869757986)(39,0.810705869757986)(39,0.810705869757986)(39,0.810705869757986)(39,0.810705869757986)(39,0.810705869757986)(39,0.810705869757986)(39,0.810705869757986)(39,0.810705869757986)(40,0.811522816063507)(40,0.811522816063507)(40,0.811522816063507)(40,0.811522816063507)(40,0.811522816063507)(41,0.805037810831504)(41,0.805037810831504)(41,0.805037810831504)(41,0.805037810831504)(41,0.805037810831504)(41,0.805037810831504)(41,0.805037810831504)(42,0.799153943624503)(42,0.799153943624503)(42,0.799153943624503)(42,0.799153943624503)(43,0.800971264643013)(43,0.800971264643013)(43,0.800971264643013)(43,0.800971264643013)(43,0.800971264643013)(43,0.800971264643013)(43,0.800971264643013)(44,0.806293005981012)(44,0.806293005981012)(44,0.806293005981012)(44,0.806293005981012)(44,0.806293005981012)(45,0.811172181579444)(45,0.811172181579444)(45,0.811172181579444)(45,0.811172181579444)(45,0.811172181579444)(46,0.811521598324402)(46,0.811521598324402)(46,0.811521598324402)(46,0.811521598324402)(46,0.811521598324402)(46,0.811521598324402)(47,0.81330345297774)(47,0.81330345297774)(47,0.81330345297774)(47,0.81330345297774)(47,0.81330345297774)(47,0.81330345297774)(47,0.81330345297774)(47,0.81330345297774)(48,0.81439326107061)(48,0.81439326107061)(48,0.81439326107061)(48,0.81439326107061)(48,0.81439326107061)(49,0.814097864842097)(50,0.816815456421519)(50,0.816815456421519)(50,0.816815456421519)(50,0.816815456421519)(50,0.816815456421519)(51,0.824067478433674)(51,0.824067478433674)(51,0.824067478433674)(51,0.824067478433674)(51,0.824067478433674)(52,0.832165350680679)(52,0.832165350680679)(52,0.832165350680679)(52,0.832165350680679)(52,0.832165350680679)(52,0.832165350680679)(53,0.839949664834987)(53,0.839949664834987)(53,0.839949664834987)(54,0.845187124258775)(54,0.845187124258775)(54,0.845187124258775)(54,0.845187124258775)(54,0.845187124258775)(55,0.845774940153465)(55,0.845774940153465)(55,0.845774940153465)(55,0.845774940153465)(55,0.845774940153465)(55,0.845774940153465)(56,0.845845043226818)(56,0.845845043226818)(56,0.845845043226818)(56,0.845845043226818)(56,0.845845043226818)(56,0.845845043226818)(57,0.844464676435243)(57,0.844464676435243)(58,0.843376359101087)(58,0.843376359101087)(60,0.839484289578182)(60,0.839484289578182)(60,0.839484289578182)(60,0.839484289578182)(60,0.839484289578182)(61,0.838047628845545)(61,0.838047628845545)(61,0.838047628845545)(61,0.838047628845545)(62,0.839382550026277)(62,0.839382550026277)(62,0.839382550026277)(63,0.841307486618618)(63,0.841307486618618)(64,0.844213361976998)(64,0.844213361976998)(64,0.844213361976998)(64,0.844213361976998)(64,0.844213361976998)(65,0.847816506183108)(65,0.847816506183108)(66,0.85210313664524)(66,0.85210313664524)(66,0.85210313664524)(67,0.856195906360421)(67,0.856195906360421)(67,0.856195906360421)(68,0.861054484622592)(68,0.861054484622592)(68,0.861054484622592)(69,0.864930429126234)(70,0.867842011471245)(70,0.867842011471245)(71,0.870175053134461)(71,0.870175053134461)(71,0.870175053134461)(71,0.870175053134461)(71,0.870175053134461)(72,0.87339942673453)(73,0.873897390232136)(73,0.873897390232136)(73,0.873897390232136)(74,0.874683353352498)(74,0.874683353352498)(75,0.875924953653414)(75,0.875924953653414)(75,0.875924953653414)(76,0.87678836291943)(77,0.877837948509802)(77,0.877837948509802)(78,0.878231799698609)(78,0.878231799698609)(79,0.878364281453745)(79,0.878364281453745)(79,0.878364281453745)(80,0.878446593351745)(80,0.878446593351745)(80,0.878446593351745)(81,0.878499720990771)(81,0.878499720990771)(81,0.878499720990771)(82,0.878474288860171)(82,0.878474288860171)(83,0.878424848083669)(83,0.878424848083669)(83,0.878424848083669)(83,0.878424848083669)(83,0.878424848083669)(83,0.878424848083669)(83,0.878424848083669)(84,0.878381461367454)(84,0.878381461367454)(84,0.878381461367454)(84,0.878381461367454)(86,0.878589189924744)(86,0.878589189924744)(88,0.879441818262085)(88,0.879441818262085)(88,0.879441818262085)(89,0.879842747999526)(89,0.879842747999526)(90,0.880197277394905)(90,0.880197277394905)(92,0.88030095147247)(92,0.88030095147247)(93,0.880120142529739)(93,0.880120142529739)(94,0.879920732900203)(94,0.879920732900203)(94,0.879920732900203)(94,0.879920732900203)(96,0.880236642472777)(96,0.880236642472777)(97,0.880989502709565)(98,0.881816236501713)(98,0.881816236501713)(100,0.883975032320908)(100,0.883975032320908)(102,0.885722156725374)(102,0.885722156725374)(102,0.885722156725374)(103,0.886235950997552)(103,0.886235950997552)(103,0.886235950997552)(103,0.886235950997552)(103,0.886235950997552)(104,0.88693411274915)(105,0.887230263339305)(105,0.887230263339305)(106,0.887503950707928)(108,0.88816604350941)(108,0.88816604350941)(109,0.888476251906188)(109,0.888476251906188)(109,0.888476251906188)(110,0.888821551149089)(110,0.888821551149089)(112,0.88974806142248)(112,0.88974806142248)(113,0.890632663194523)(116,0.892976365320904)(117,0.893662020529488)(118,0.894226031126047)(118,0.894226031126047)(119,0.894708449948314)(120,0.895163669710538)(120,0.895163669710538)(122,0.896189271856706)(122,0.896189271856706)(122,0.896189271856706)(123,0.896859119514757)(124,0.897243317155141)(125,0.897526558802154)(125,0.897526558802154)(125,0.897526558802154)(126,0.897709440029269)(127,0.897796324693435)(129,0.897682729441037)(130,0.897149224552484)(130,0.897149224552484)(131,0.896946947639376)(132,0.896484853846013)(132,0.896484853846013)(132,0.896484853846013)(132,0.896484853846013)(133,0.896247764105581)(134,0.895972084672567)(134,0.895972084672567)(134,0.895972084672567)(134,0.895972084672567)(135,0.895643236271228)(136,0.895426834296262)(137,0.895241930333277)(139,0.895080258490093)(142,0.895582416621594)(142,0.895582416621594)(142,0.895582416621594)(143,0.896362154273284)(143,0.896362154273284)(144,0.896963998253276)(145,0.89722228105639)(146,0.898115586067986)(146,0.898115586067986)(148,0.900727941382498)(150,0.902991067750962)(151,0.904334243220642)(152,0.905614002075212)(153,0.906602278240499)(153,0.906602278240499)(154,0.907902165976781)(157,0.909590955947193)(159,0.910521527977045)(159,0.910521527977045)(159,0.910521527977045)(160,0.911130654813163)(161,0.911688087098191)(162,0.912167738199805)(162,0.912167738199805)(163,0.91254960477913)(164,0.912543006049185)(166,0.912763914829809)(167,0.912913128540545)(167,0.912913128540545)(167,0.912913128540545)(171,0.913401118289077)(171,0.913401118289077)(171,0.913401118289077)(172,0.913681936605912)(172,0.913681936605912)(173,0.913956423140596)(174,0.914011102429075)(175,0.914008288383391)(176,0.913959014615017)(179,0.913356229618085)(179,0.913356229618085)(180,0.913549190711578)(180,0.913549190711578)(182,0.9137722795398)(182,0.9137722795398)(183,0.91388759195039)(184,0.914036434266308)(184,0.914036434266308)(184,0.914036434266308)(185,0.914144146511166)(187,0.914306511044577)(189,0.914529976709891)(191,0.914773023240541)(192,0.914954913665108)(194,0.915500765007867)(195,0.915713857242268)(197,0.916079505013217)(199,0.91611029135959)(201,0.916209825523144)(203,0.91622927978628)(204,0.916232712402271)(206,0.916295797525066)(207,0.916445762052037)(208,0.916540187354036)(209,0.916653543571333)(212,0.917532797905504)(212,0.917532797905504)(212,0.917532797905504)(213,0.917735308177029)(214,0.917757710544923)(215,0.918184391568645)(216,0.918597756215992)(216,0.918597756215992)(217,0.918983127572772)(218,0.919135423022594)(221,0.920125353079086)(222,0.920471088792609)(223,0.920836164560169)(232,0.924225128523607)(234,0.924892518692524)(235,0.925016370937621)(236,0.92511332415285)(236,0.92511332415285)(238,0.925267970967042)(238,0.925267970967042)(240,0.925651061969459)(240,0.925651061969459)(243,0.925861580444398)(247,0.925916397948404)(247,0.925916397948404)(248,0.925894843962828)(249,0.925855532704524)(250,0.925802329802732)(253,0.925624034290622)(253,0.925624034290622)(255,0.925459228171197)(259,0.925070293830347)(261,0.925088605907656)(261,0.925088605907656)(262,0.924913895606897)(266,0.925234131159457)(267,0.925153185696188)(268,0.925311624173534)(268,0.925311624173534)(269,0.9256714002966)(271,0.926249559482701)(273,0.926819707964875)(274,0.926993906297149)(277,0.927812828281129)(279,0.92833456719493)(280,0.928607975426783)(281,0.928881977838785)(282,0.929112109041688)(285,0.930077421161193)(285,0.930077421161193)(287,0.930895126956495)(296,0.932158585323047)(296,0.932158585323047)(297,0.932289100022168)(299,0.932424672289773)(299,0.932424672289773)(302,0.932544431191571)(306,0.932625711704273)(308,0.932627364494221)(310,0.932749268169458)(310,0.932749268169458)(311,0.932767016844665)(312,0.932804231306121)(314,0.932864236049456)(314,0.932864236049456)(314,0.932864236049456)(317,0.932890054732473)(318,0.932867384314349)(318,0.932867384314349)(321,0.932709350199847)(322,0.932633054118763)(323,0.93254788703177)(324,0.932494728534316)(327,0.932364083959945)(328,0.932341034823261)(329,0.932330126887265)(332,0.931997389179474)(332,0.931997389179474)(334,0.9318941876314)(334,0.9318941876314)(335,0.931862692546468)(336,0.931934250863351)(338,0.932181197978842)(339,0.932344541474861)(342,0.932908069187092)(349,0.934271955807637)(351,0.934651076730208)(355,0.935401607262349)(355,0.935401607262349)(357,0.935775964756498)(359,0.936151182299288)(361,0.93652826492589)(365,0.937291536145376)(369,0.938071892740233)(373,0.938874143918584)(374,0.939078595045719)(375,0.939284672699879)(376,0.939492381869173)(376,0.939492381869173)(378,0.939912642961281)(380,0.940339161877791)(380,0.940339161877791)(382,0.940771602422241)(384,0.941209571102805)(385,0.941430474768971)(397,0.94414355163138)(398,0.944371514776906)(398,0.944371514776906)(399,0.944599267773748)(399,0.944599267773748)(400,0.944826733354737) 
};
\addlegendentry{\acl (max. ambiguity)};

\addplot [
color=red,
solid,
line width=1.0pt,
]
coordinates{
 (12,0.534623903867452)(12,0.534623903867452)(12,0.534623903867452)(12,0.534623903867452)(12,0.534623903867452)(12,0.534623903867452)(12,0.534623903867452)(13,0.571770707067159)(13,0.571770707067159)(13,0.571770707067159)(13,0.571770707067159)(13,0.571770707067159)(13,0.571770707067159)(13,0.571770707067159)(13,0.571770707067159)(14,0.564277030804336)(14,0.564277030804336)(14,0.564277030804336)(14,0.564277030804336)(14,0.564277030804336)(14,0.564277030804336)(14,0.564277030804336)(14,0.564277030804336)(14,0.564277030804336)(14,0.564277030804336)(14,0.564277030804336)(14,0.564277030804336)(15,0.579332574230299)(15,0.579332574230299)(15,0.579332574230299)(15,0.579332574230299)(15,0.579332574230299)(15,0.579332574230299)(15,0.579332574230299)(15,0.579332574230299)(16,0.600974025071856)(16,0.600974025071856)(16,0.600974025071856)(16,0.600974025071856)(16,0.600974025071856)(16,0.600974025071856)(16,0.600974025071856)(16,0.600974025071856)(16,0.600974025071856)(16,0.600974025071856)(16,0.600974025071856)(16,0.600974025071856)(17,0.614787342145323)(17,0.614787342145323)(17,0.614787342145323)(17,0.614787342145323)(17,0.614787342145323)(17,0.614787342145323)(17,0.614787342145323)(17,0.614787342145323)(17,0.614787342145323)(17,0.614787342145323)(17,0.614787342145323)(18,0.62228333889074)(18,0.62228333889074)(18,0.62228333889074)(18,0.62228333889074)(18,0.62228333889074)(18,0.62228333889074)(18,0.62228333889074)(18,0.62228333889074)(18,0.62228333889074)(18,0.62228333889074)(19,0.641900259293649)(19,0.641900259293649)(19,0.641900259293649)(19,0.641900259293649)(19,0.641900259293649)(19,0.641900259293649)(19,0.641900259293649)(19,0.641900259293649)(19,0.641900259293649)(19,0.641900259293649)(19,0.641900259293649)(19,0.641900259293649)(19,0.641900259293649)(19,0.641900259293649)(19,0.641900259293649)(19,0.641900259293649)(19,0.641900259293649)(19,0.641900259293649)(20,0.641710544634978)(20,0.641710544634978)(20,0.641710544634978)(20,0.641710544634978)(20,0.641710544634978)(20,0.641710544634978)(20,0.641710544634978)(21,0.660908373674607)(21,0.660908373674607)(21,0.660908373674607)(21,0.660908373674607)(21,0.660908373674607)(21,0.660908373674607)(21,0.660908373674607)(21,0.660908373674607)(21,0.660908373674607)(21,0.660908373674607)(22,0.686187945315087)(22,0.686187945315087)(22,0.686187945315087)(22,0.686187945315087)(22,0.686187945315087)(22,0.686187945315087)(22,0.686187945315087)(22,0.686187945315087)(22,0.686187945315087)(22,0.686187945315087)(22,0.686187945315087)(22,0.686187945315087)(23,0.68464976328686)(23,0.68464976328686)(23,0.68464976328686)(23,0.68464976328686)(23,0.68464976328686)(23,0.68464976328686)(23,0.68464976328686)(24,0.688682102268111)(24,0.688682102268111)(24,0.688682102268111)(24,0.688682102268111)(24,0.688682102268111)(24,0.688682102268111)(24,0.688682102268111)(24,0.688682102268111)(24,0.688682102268111)(24,0.688682102268111)(24,0.688682102268111)(24,0.688682102268111)(25,0.699769462129122)(25,0.699769462129122)(25,0.699769462129122)(25,0.699769462129122)(26,0.714509446317143)(26,0.714509446317143)(26,0.714509446317143)(26,0.714509446317143)(26,0.714509446317143)(26,0.714509446317143)(26,0.714509446317143)(26,0.714509446317143)(26,0.714509446317143)(27,0.719681931124406)(27,0.719681931124406)(27,0.719681931124406)(27,0.719681931124406)(27,0.719681931124406)(27,0.719681931124406)(27,0.719681931124406)(27,0.719681931124406)(27,0.719681931124406)(27,0.719681931124406)(27,0.719681931124406)(28,0.721885311590972)(28,0.721885311590972)(28,0.721885311590972)(28,0.721885311590972)(28,0.721885311590972)(29,0.740241622339497)(29,0.740241622339497)(29,0.740241622339497)(29,0.740241622339497)(29,0.740241622339497)(29,0.740241622339497)(29,0.740241622339497)(29,0.740241622339497)(29,0.740241622339497)(29,0.740241622339497)(29,0.740241622339497)(29,0.740241622339497)(29,0.740241622339497)(29,0.740241622339497)(29,0.740241622339497)(29,0.740241622339497)(29,0.740241622339497)(30,0.757195520780394)(30,0.757195520780394)(30,0.757195520780394)(30,0.757195520780394)(30,0.757195520780394)(30,0.757195520780394)(31,0.76013526707405)(31,0.76013526707405)(31,0.76013526707405)(31,0.76013526707405)(31,0.76013526707405)(31,0.76013526707405)(31,0.76013526707405)(32,0.76113576271947)(32,0.76113576271947)(32,0.76113576271947)(32,0.76113576271947)(32,0.76113576271947)(32,0.76113576271947)(32,0.76113576271947)(32,0.76113576271947)(32,0.76113576271947)(32,0.76113576271947)(32,0.76113576271947)(32,0.76113576271947)(32,0.76113576271947)(32,0.76113576271947)(33,0.764574399179485)(33,0.764574399179485)(33,0.764574399179485)(34,0.77193877658096)(34,0.77193877658096)(34,0.77193877658096)(34,0.77193877658096)(34,0.77193877658096)(34,0.77193877658096)(35,0.780107449004389)(35,0.780107449004389)(35,0.780107449004389)(35,0.780107449004389)(35,0.780107449004389)(35,0.780107449004389)(35,0.780107449004389)(35,0.780107449004389)(35,0.780107449004389)(35,0.780107449004389)(36,0.785273769626291)(36,0.785273769626291)(37,0.786425720473021)(37,0.786425720473021)(37,0.786425720473021)(37,0.786425720473021)(37,0.786425720473021)(37,0.786425720473021)(37,0.786425720473021)(38,0.78304020821871)(38,0.78304020821871)(38,0.78304020821871)(38,0.78304020821871)(38,0.78304020821871)(38,0.78304020821871)(39,0.785627632293217)(39,0.785627632293217)(39,0.785627632293217)(39,0.785627632293217)(39,0.785627632293217)(39,0.785627632293217)(40,0.789418977975112)(40,0.789418977975112)(40,0.789418977975112)(40,0.789418977975112)(40,0.789418977975112)(40,0.789418977975112)(40,0.789418977975112)(40,0.789418977975112)(40,0.789418977975112)(41,0.787610233874253)(41,0.787610233874253)(41,0.787610233874253)(41,0.787610233874253)(41,0.787610233874253)(41,0.787610233874253)(41,0.787610233874253)(42,0.785681884264557)(42,0.785681884264557)(42,0.785681884264557)(42,0.785681884264557)(42,0.785681884264557)(42,0.785681884264557)(42,0.785681884264557)(43,0.788417251716215)(43,0.788417251716215)(44,0.794662320080549)(44,0.794662320080549)(44,0.794662320080549)(45,0.799707373200212)(45,0.799707373200212)(45,0.799707373200212)(45,0.799707373200212)(45,0.799707373200212)(45,0.799707373200212)(46,0.804930397660651)(46,0.804930397660651)(46,0.804930397660651)(46,0.804930397660651)(46,0.804930397660651)(46,0.804930397660651)(46,0.804930397660651)(47,0.805816349770063)(47,0.805816349770063)(47,0.805816349770063)(47,0.805816349770063)(47,0.805816349770063)(47,0.805816349770063)(47,0.805816349770063)(47,0.805816349770063)(47,0.805816349770063)(48,0.810153619254058)(48,0.810153619254058)(48,0.810153619254058)(48,0.810153619254058)(48,0.810153619254058)(49,0.812640880930608)(49,0.812640880930608)(49,0.812640880930608)(49,0.812640880930608)(50,0.816638127358295)(50,0.816638127358295)(50,0.816638127358295)(51,0.820494309285338)(51,0.820494309285338)(51,0.820494309285338)(51,0.820494309285338)(51,0.820494309285338)(51,0.820494309285338)(51,0.820494309285338)(52,0.826792711941456)(52,0.826792711941456)(52,0.826792711941456)(52,0.826792711941456)(52,0.826792711941456)(52,0.826792711941456)(52,0.826792711941456)(53,0.836353314653743)(54,0.84542915969808)(54,0.84542915969808)(55,0.847565016090682)(55,0.847565016090682)(56,0.844386118899287)(56,0.844386118899287)(56,0.844386118899287)(56,0.844386118899287)(57,0.84355348356146)(57,0.84355348356146)(57,0.84355348356146)(57,0.84355348356146)(58,0.842424168857555)(58,0.842424168857555)(58,0.842424168857555)(59,0.840083571254818)(59,0.840083571254818)(59,0.840083571254818)(59,0.840083571254818)(60,0.836188181949374)(60,0.836188181949374)(60,0.836188181949374)(61,0.83242178277523)(61,0.83242178277523)(61,0.83242178277523)(61,0.83242178277523)(61,0.83242178277523)(61,0.83242178277523)(62,0.831413507762456)(62,0.831413507762456)(62,0.831413507762456)(63,0.832673570777589)(63,0.832673570777589)(63,0.832673570777589)(63,0.832673570777589)(63,0.832673570777589)(64,0.833599121315702)(64,0.833599121315702)(65,0.834970041536272)(65,0.834970041536272)(65,0.834970041536272)(65,0.834970041536272)(66,0.836449105709797)(66,0.836449105709797)(66,0.836449105709797)(67,0.837646594761949)(67,0.837646594761949)(67,0.837646594761949)(67,0.837646594761949)(68,0.838376608108894)(68,0.838376608108894)(68,0.838376608108894)(68,0.838376608108894)(69,0.839160093535176)(69,0.839160093535176)(70,0.840640066138065)(70,0.840640066138065)(71,0.841429454946298)(71,0.841429454946298)(71,0.841429454946298)(72,0.843354334337309)(72,0.843354334337309)(72,0.843354334337309)(73,0.844622656232333)(74,0.846639261632389)(74,0.846639261632389)(74,0.846639261632389)(75,0.846441638213828)(75,0.846441638213828)(75,0.846441638213828)(76,0.847735057596957)(76,0.847735057596957)(76,0.847735057596957)(77,0.848860693658591)(77,0.848860693658591)(78,0.85100029604774)(78,0.85100029604774)(78,0.85100029604774)(79,0.853334297638959)(79,0.853334297638959)(80,0.856576390230982)(80,0.856576390230982)(80,0.856576390230982)(81,0.860475241919968)(81,0.860475241919968)(82,0.864053762925544)(82,0.864053762925544)(82,0.864053762925544)(83,0.867238469826409)(83,0.867238469826409)(84,0.870024153544948)(84,0.870024153544948)(84,0.870024153544948)(85,0.871542768828492)(86,0.872885312365971)(86,0.872885312365971)(86,0.872885312365971)(87,0.873871522059172)(87,0.873871522059172)(87,0.873871522059172)(87,0.873871522059172)(88,0.874579477301065)(89,0.875221311642538)(89,0.875221311642538)(90,0.876019219113663)(90,0.876019219113663)(90,0.876019219113663)(91,0.876739876316859)(92,0.876825213143238)(93,0.87699257838675)(93,0.87699257838675)(93,0.87699257838675)(93,0.87699257838675)(93,0.87699257838675)(96,0.876417925670927)(97,0.876905469729035)(97,0.876905469729035)(98,0.876962268498202)(98,0.876962268498202)(98,0.876962268498202)(98,0.876962268498202)(99,0.877356628997204)(99,0.877356628997204)(101,0.879614385592051)(102,0.881327849032806)(103,0.882961116771058)(103,0.882961116771058)(103,0.882961116771058)(103,0.882961116771058)(104,0.884606293571415)(105,0.885860312671441)(105,0.885860312671441)(105,0.885860312671441)(106,0.887425531361399)(107,0.888886109449978)(107,0.888886109449978)(107,0.888886109449978)(107,0.888886109449978)(108,0.890172366079421)(109,0.890998542750858)(109,0.890998542750858)(110,0.892048024010458)(111,0.89310145799543)(112,0.894158641418422)(112,0.894158641418422)(113,0.894939921487899)(113,0.894939921487899)(116,0.896266933566638)(117,0.895844952881852)(117,0.895844952881852)(117,0.895844952881852)(117,0.895844952881852)(119,0.894707018722092)(119,0.894707018722092)(119,0.894707018722092)(120,0.894532554006804)(120,0.894532554006804)(121,0.89418905535608)(121,0.89418905535608)(124,0.893127634888259)(125,0.892978693606912)(126,0.892702289563643)(126,0.892702289563643)(128,0.89216952788685)(128,0.89216952788685)(129,0.891966420177024)(129,0.891966420177024)(130,0.89186282909751)(130,0.89186282909751)(131,0.892132995834702)(133,0.892670176188174)(133,0.892670176188174)(134,0.892898732948563)(136,0.893624432184937)(136,0.893624432184937)(137,0.894288212359177)(137,0.894288212359177)(138,0.894722430617974)(139,0.895504658353627)(141,0.896901597994739)(141,0.896901597994739)(141,0.896901597994739)(143,0.898338687049211)(145,0.899647371774108)(145,0.899647371774108)(145,0.899647371774108)(147,0.900836521683158)(148,0.901836852446135)(149,0.902477919204069)(152,0.903823078429499)(152,0.903823078429499)(153,0.904239210580631)(155,0.904971845711332)(155,0.904971845711332)(157,0.90585538969239)(157,0.90585538969239)(158,0.906176298808752)(159,0.906678810277054)(160,0.907078802242653)(162,0.908115924511535)(162,0.908115924511535)(162,0.908115924511535)(164,0.909056256194778)(165,0.909697880260889)(165,0.909697880260889)(166,0.910456780723791)(166,0.910456780723791)(167,0.911123417012663)(167,0.911123417012663)(168,0.911775513220493)(172,0.914040379024932)(172,0.914040379024932)(173,0.914470423918718)(174,0.914845482624934)(175,0.915246090288979)(179,0.916501280059333)(180,0.916776190083908)(181,0.91720443256708)(182,0.917192032440578)(183,0.917526598981504)(183,0.917526598981504)(184,0.917399262589129)(184,0.917399262589129)(185,0.917444432475332)(185,0.917444432475332)(187,0.917477811024257)(190,0.917438122735498)(192,0.917277376156195)(192,0.917277376156195)(194,0.917001937579814)(196,0.916799803560095)(198,0.916544691673699)(198,0.916544691673699)(199,0.916516380374342)(199,0.916516380374342)(200,0.916382195430375)(200,0.916382195430375)(201,0.916277954967188)(202,0.916398631781834)(204,0.916460706631435)(205,0.916765235791622)(205,0.916765235791622)(208,0.917939896287042)(210,0.918482375104436)(212,0.919539882250951)(214,0.920962087248365)(216,0.92205030257333)(217,0.922530794905726)(218,0.923051082603586)(218,0.923051082603586)(219,0.923587792827449)(219,0.923587792827449)(221,0.924498876045624)(222,0.924886816229363)(222,0.924886816229363)(224,0.925723617130035)(224,0.925723617130035)(227,0.926286883446421)(227,0.926286883446421)(228,0.926363996683086)(229,0.926395223749899)(229,0.926395223749899)(229,0.926395223749899)(230,0.926387098305155)(235,0.925379490032464)(238,0.924835098040017)(240,0.924363363065801)(242,0.923755946914696)(244,0.922989063392636)(248,0.921690486191582)(249,0.921548685249262)(254,0.921494327511028)(254,0.921494327511028)(255,0.921527382445893)(258,0.922066113881394)(262,0.922903741888548)(263,0.923158003565851)(263,0.923158003565851)(263,0.923158003565851)(264,0.923253974352401)(264,0.923253974352401)(266,0.923859509994854)(267,0.924308189068748)(269,0.92490015191252)(270,0.925285717937123)(273,0.926468029851929)(273,0.926468029851929)(274,0.926891299687375)(275,0.927239223529705)(276,0.927575426886735)(276,0.927575426886735)(277,0.927903135577317)(279,0.928531687726526)(281,0.929283485332427)(287,0.931873319074634)(288,0.932022664057466)(289,0.932128031269633)(290,0.93265680623852)(291,0.932903528957053)(291,0.932903528957053)(297,0.933196469282703)(301,0.933753219950665)(302,0.933796334316095)(304,0.93380755840596)(306,0.933837757927597)(308,0.933852968617066)(309,0.933781332278281)(310,0.933780139746624)(313,0.933713531314301)(313,0.933713531314301)(315,0.933788683392034)(317,0.933819645968513)(318,0.933899058317048)(318,0.933899058317048)(323,0.934445959305643)(323,0.934445959305643)(328,0.93539842522758)(329,0.935553079185167)(331,0.935887577317673)(340,0.936637713615409)(345,0.936941509419264)(346,0.937012739032561)(346,0.937012739032561)(347,0.937087927994225)(347,0.937087927994225)(348,0.937167069779173)(354,0.937717335526975)(355,0.937819964121331)(356,0.937925186843337)(357,0.938032785925962)(359,0.938254254620326)(361,0.938482747074988)(365,0.93895547439481)(365,0.93895547439481)(366,0.939076160017353)(372,0.939813595686386)(373,0.939938011290229)(374,0.940062724876136)(375,0.940187721045017)(377,0.940438578166642)(379,0.940690744283144)(380,0.94081741291942)(380,0.94081741291942)(386,0.941589410101099)(386,0.941589410101099)(387,0.941720736494179)(387,0.941720736494179)(388,0.941853049697288)(393,0.942532117957911)(395,0.942813293781014)(398,0.943247243156976) 
};
\addlegendentry{\acl (max. variance)};

\addplot [
color=green!50!black,
solid,
line width=1.0pt,
]
coordinates{
 (12,0.504655810608979)(12,0.504655810608979)(12,0.504655810608979)(12,0.504655810608979)(12,0.504655810608979)(12,0.504655810608979)(12,0.504655810608979)(12,0.504655810608979)(12,0.504655810608979)(13,0.549029853680621)(13,0.549029853680621)(13,0.549029853680621)(13,0.549029853680621)(13,0.549029853680621)(13,0.549029853680621)(13,0.549029853680621)(14,0.564288639176868)(14,0.564288639176868)(14,0.564288639176868)(14,0.564288639176868)(14,0.564288639176868)(14,0.564288639176868)(14,0.564288639176868)(14,0.564288639176868)(14,0.564288639176868)(15,0.5691409200567)(15,0.5691409200567)(15,0.5691409200567)(15,0.5691409200567)(15,0.5691409200567)(15,0.5691409200567)(15,0.5691409200567)(15,0.5691409200567)(15,0.5691409200567)(16,0.566046197316637)(16,0.566046197316637)(16,0.566046197316637)(16,0.566046197316637)(16,0.566046197316637)(16,0.566046197316637)(16,0.566046197316637)(16,0.566046197316637)(16,0.566046197316637)(16,0.566046197316637)(16,0.566046197316637)(17,0.594380610689661)(17,0.594380610689661)(17,0.594380610689661)(17,0.594380610689661)(17,0.594380610689661)(17,0.594380610689661)(17,0.594380610689661)(17,0.594380610689661)(17,0.594380610689661)(17,0.594380610689661)(17,0.594380610689661)(17,0.594380610689661)(18,0.649929575634757)(18,0.649929575634757)(18,0.649929575634757)(18,0.649929575634757)(18,0.649929575634757)(18,0.649929575634757)(18,0.649929575634757)(18,0.649929575634757)(18,0.649929575634757)(18,0.649929575634757)(19,0.671412825177504)(19,0.671412825177504)(19,0.671412825177504)(19,0.671412825177504)(19,0.671412825177504)(19,0.671412825177504)(19,0.671412825177504)(19,0.671412825177504)(20,0.671502127183108)(20,0.671502127183108)(20,0.671502127183108)(20,0.671502127183108)(20,0.671502127183108)(20,0.671502127183108)(20,0.671502127183108)(20,0.671502127183108)(21,0.664934156629718)(21,0.664934156629718)(21,0.664934156629718)(21,0.664934156629718)(21,0.664934156629718)(21,0.664934156629718)(21,0.664934156629718)(21,0.664934156629718)(21,0.664934156629718)(21,0.664934156629718)(21,0.664934156629718)(21,0.664934156629718)(22,0.678093092045147)(22,0.678093092045147)(22,0.678093092045147)(22,0.678093092045147)(22,0.678093092045147)(22,0.678093092045147)(23,0.706110178783894)(23,0.706110178783894)(23,0.706110178783894)(23,0.706110178783894)(23,0.706110178783894)(23,0.706110178783894)(23,0.706110178783894)(23,0.706110178783894)(23,0.706110178783894)(23,0.706110178783894)(24,0.715549464804417)(24,0.715549464804417)(24,0.715549464804417)(24,0.715549464804417)(24,0.715549464804417)(24,0.715549464804417)(24,0.715549464804417)(25,0.703470065560465)(25,0.703470065560465)(25,0.703470065560465)(25,0.703470065560465)(25,0.703470065560465)(25,0.703470065560465)(25,0.703470065560465)(25,0.703470065560465)(26,0.695346149147626)(26,0.695346149147626)(26,0.695346149147626)(26,0.695346149147626)(26,0.695346149147626)(26,0.695346149147626)(26,0.695346149147626)(26,0.695346149147626)(26,0.695346149147626)(27,0.710440936741634)(27,0.710440936741634)(27,0.710440936741634)(27,0.710440936741634)(27,0.710440936741634)(27,0.710440936741634)(28,0.742822743391468)(28,0.742822743391468)(28,0.742822743391468)(28,0.742822743391468)(28,0.742822743391468)(28,0.742822743391468)(28,0.742822743391468)(28,0.742822743391468)(28,0.742822743391468)(29,0.765240849915671)(29,0.765240849915671)(29,0.765240849915671)(29,0.765240849915671)(29,0.765240849915671)(29,0.765240849915671)(29,0.765240849915671)(29,0.765240849915671)(29,0.765240849915671)(30,0.77266236761402)(30,0.77266236761402)(30,0.77266236761402)(30,0.77266236761402)(30,0.77266236761402)(30,0.77266236761402)(30,0.77266236761402)(30,0.77266236761402)(31,0.767650096258883)(31,0.767650096258883)(31,0.767650096258883)(31,0.767650096258883)(31,0.767650096258883)(31,0.767650096258883)(32,0.76777429070563)(32,0.76777429070563)(32,0.76777429070563)(32,0.76777429070563)(32,0.76777429070563)(32,0.76777429070563)(33,0.773413076622885)(33,0.773413076622885)(33,0.773413076622885)(33,0.773413076622885)(33,0.773413076622885)(33,0.773413076622885)(34,0.78293147972564)(34,0.78293147972564)(34,0.78293147972564)(34,0.78293147972564)(34,0.78293147972564)(34,0.78293147972564)(35,0.793220380989265)(35,0.793220380989265)(35,0.793220380989265)(35,0.793220380989265)(35,0.793220380989265)(35,0.793220380989265)(35,0.793220380989265)(36,0.798576015571441)(36,0.798576015571441)(36,0.798576015571441)(36,0.798576015571441)(36,0.798576015571441)(36,0.798576015571441)(37,0.796192560902903)(37,0.796192560902903)(37,0.796192560902903)(37,0.796192560902903)(37,0.796192560902903)(37,0.796192560902903)(38,0.792166221579726)(38,0.792166221579726)(38,0.792166221579726)(38,0.792166221579726)(38,0.792166221579726)(38,0.792166221579726)(38,0.792166221579726)(38,0.792166221579726)(38,0.792166221579726)(38,0.792166221579726)(38,0.792166221579726)(39,0.788292348627405)(39,0.788292348627405)(39,0.788292348627405)(39,0.788292348627405)(39,0.788292348627405)(39,0.788292348627405)(40,0.785327808741443)(40,0.785327808741443)(40,0.785327808741443)(40,0.785327808741443)(40,0.785327808741443)(40,0.785327808741443)(40,0.785327808741443)(41,0.786538695295405)(41,0.786538695295405)(41,0.786538695295405)(41,0.786538695295405)(41,0.786538695295405)(41,0.786538695295405)(42,0.795201031548636)(42,0.795201031548636)(42,0.795201031548636)(42,0.795201031548636)(42,0.795201031548636)(42,0.795201031548636)(42,0.795201031548636)(42,0.795201031548636)(43,0.807776936728008)(43,0.807776936728008)(43,0.807776936728008)(43,0.807776936728008)(44,0.820469092192215)(44,0.820469092192215)(45,0.823532508673924)(45,0.823532508673924)(45,0.823532508673924)(45,0.823532508673924)(45,0.823532508673924)(46,0.822768828271062)(46,0.822768828271062)(47,0.821221606687684)(47,0.821221606687684)(47,0.821221606687684)(48,0.818996851485192)(48,0.818996851485192)(48,0.818996851485192)(48,0.818996851485192)(48,0.818996851485192)(48,0.818996851485192)(48,0.818996851485192)(49,0.814894637106058)(49,0.814894637106058)(49,0.814894637106058)(49,0.814894637106058)(50,0.812639041309493)(50,0.812639041309493)(50,0.812639041309493)(50,0.812639041309493)(50,0.812639041309493)(50,0.812639041309493)(50,0.812639041309493)(50,0.812639041309493)(50,0.812639041309493)(51,0.812621407299512)(51,0.812621407299512)(51,0.812621407299512)(51,0.812621407299512)(51,0.812621407299512)(51,0.812621407299512)(52,0.815028204738429)(52,0.815028204738429)(53,0.82002982707823)(53,0.82002982707823)(53,0.82002982707823)(54,0.82431809887986)(54,0.82431809887986)(54,0.82431809887986)(55,0.825594429150141)(55,0.825594429150141)(55,0.825594429150141)(55,0.825594429150141)(55,0.825594429150141)(56,0.82540767561367)(56,0.82540767561367)(56,0.82540767561367)(56,0.82540767561367)(56,0.82540767561367)(56,0.82540767561367)(56,0.82540767561367)(57,0.827171638262576)(57,0.827171638262576)(57,0.827171638262576)(57,0.827171638262576)(57,0.827171638262576)(58,0.831021693888099)(59,0.834947200652675)(59,0.834947200652675)(59,0.834947200652675)(60,0.837095520312145)(60,0.837095520312145)(60,0.837095520312145)(61,0.838319809573701)(62,0.840947533399651)(62,0.840947533399651)(62,0.840947533399651)(62,0.840947533399651)(62,0.840947533399651)(62,0.840947533399651)(63,0.843922982041661)(63,0.843922982041661)(63,0.843922982041661)(64,0.848037852317628)(64,0.848037852317628)(64,0.848037852317628)(64,0.848037852317628)(64,0.848037852317628)(65,0.852258978190784)(65,0.852258978190784)(65,0.852258978190784)(65,0.852258978190784)(65,0.852258978190784)(66,0.85535155785259)(66,0.85535155785259)(67,0.857914872628208)(67,0.857914872628208)(67,0.857914872628208)(68,0.859404237555026)(68,0.859404237555026)(69,0.860874500774898)(70,0.862229738649982)(70,0.862229738649982)(70,0.862229738649982)(71,0.863123300427935)(72,0.86252229167795)(72,0.86252229167795)(72,0.86252229167795)(73,0.862037511185285)(73,0.862037511185285)(73,0.862037511185285)(73,0.862037511185285)(74,0.862932145469708)(74,0.862932145469708)(74,0.862932145469708)(74,0.862932145469708)(75,0.864048093946265)(75,0.864048093946265)(75,0.864048093946265)(75,0.864048093946265)(76,0.86667352992152)(76,0.86667352992152)(76,0.86667352992152)(77,0.86855154483629)(77,0.86855154483629)(77,0.86855154483629)(77,0.86855154483629)(77,0.86855154483629)(78,0.871157217262865)(78,0.871157217262865)(78,0.871157217262865)(79,0.874125118675324)(81,0.877878283872895)(81,0.877878283872895)(81,0.877878283872895)(81,0.877878283872895)(81,0.877878283872895)(81,0.877878283872895)(81,0.877878283872895)(81,0.877878283872895)(82,0.87942476474735)(82,0.87942476474735)(82,0.87942476474735)(83,0.879796140791721)(84,0.881332351231715)(84,0.881332351231715)(85,0.881133460606895)(85,0.881133460606895)(85,0.881133460606895)(87,0.877435624151884)(88,0.876419580537439)(88,0.876419580537439)(90,0.874648063283751)(90,0.874648063283751)(90,0.874648063283751)(90,0.874648063283751)(90,0.874648063283751)(90,0.874648063283751)(90,0.874648063283751)(91,0.87499541658908)(92,0.876730649899584)(92,0.876730649899584)(92,0.876730649899584)(92,0.876730649899584)(92,0.876730649899584)(93,0.878253637741495)(95,0.882779976070294)(95,0.882779976070294)(97,0.886243017939964)(98,0.888984253507919)(98,0.888984253507919)(98,0.888984253507919)(99,0.890180959912251)(99,0.890180959912251)(99,0.890180959912251)(99,0.890180959912251)(100,0.8900728802586)(100,0.8900728802586)(100,0.8900728802586)(102,0.890024195806981)(102,0.890024195806981)(103,0.889287842166728)(103,0.889287842166728)(104,0.888946836610925)(104,0.888946836610925)(104,0.888946836610925)(105,0.888273938006782)(105,0.888273938006782)(105,0.888273938006782)(106,0.887443430962408)(107,0.885861210501305)(108,0.88426101144396)(109,0.882717205514794)(110,0.882175769175381)(111,0.881859749983965)(112,0.881176414099662)(112,0.881176414099662)(112,0.881176414099662)(114,0.880459441095897)(114,0.880459441095897)(116,0.88044453009623)(116,0.88044453009623)(116,0.88044453009623)(117,0.881380317636957)(117,0.881380317636957)(118,0.881988916120975)(119,0.883118700705313)(119,0.883118700705313)(119,0.883118700705313)(120,0.884341952507326)(120,0.884341952507326)(120,0.884341952507326)(120,0.884341952507326)(123,0.886868033212183)(123,0.886868033212183)(123,0.886868033212183)(124,0.887699027872634)(126,0.889053574440946)(126,0.889053574440946)(126,0.889053574440946)(127,0.889491241090413)(127,0.889491241090413)(127,0.889491241090413)(128,0.889808693064074)(129,0.890080698654117)(131,0.891110993880208)(132,0.891756418394005)(132,0.891756418394005)(133,0.892666939155198)(134,0.893493190026299)(135,0.894351470642421)(135,0.894351470642421)(135,0.894351470642421)(136,0.895025634995278)(136,0.895025634995278)(136,0.895025634995278)(136,0.895025634995278)(139,0.89705393979232)(140,0.89744560042225)(141,0.897739584837014)(142,0.897775657338266)(144,0.898031287381577)(145,0.898116124624226)(146,0.898103791547263)(147,0.898197909716701)(147,0.898197909716701)(148,0.898332834619298)(148,0.898332834619298)(150,0.898613366579551)(150,0.898613366579551)(151,0.898819784355857)(151,0.898819784355857)(152,0.899145404664413)(154,0.899850503889351)(154,0.899850503889351)(156,0.900664708151958)(156,0.900664708151958)(158,0.901773737670396)(158,0.901773737670396)(161,0.903615676950801)(161,0.903615676950801)(162,0.904134503563844)(162,0.904134503563844)(162,0.904134503563844)(165,0.906049640711867)(165,0.906049640711867)(167,0.907423140750128)(167,0.907423140750128)(169,0.909235880448007)(171,0.911323891877232)(171,0.911323891877232)(172,0.912212928779393)(173,0.913236408967633)(176,0.915330338054491)(176,0.915330338054491)(176,0.915330338054491)(179,0.916121565922656)(180,0.916522872101935)(180,0.916522872101935)(181,0.917059974514886)(181,0.917059974514886)(182,0.917274553103366)(182,0.917274553103366)(184,0.917416337918746)(185,0.917331988171994)(186,0.917223569474192)(186,0.917223569474192)(187,0.91704921887156)(190,0.916612283710318)(191,0.916505301596711)(193,0.916356723364263)(194,0.916187114377048)(195,0.916165478776143)(196,0.915889949114786)(200,0.915503284572237)(200,0.915503284572237)(201,0.915378856692198)(203,0.915971237589826)(203,0.915971237589826)(204,0.91645319768686)(204,0.91645319768686)(204,0.91645319768686)(205,0.916550371268983)(206,0.916869978074074)(206,0.916869978074074)(206,0.916869978074074)(208,0.917465796125492)(209,0.917837973960127)(210,0.918199097122013)(211,0.918538515331514)(217,0.920735910723494)(217,0.920735910723494)(217,0.920735910723494)(218,0.920901743183387)(219,0.921048407153013)(219,0.921048407153013)(224,0.922223951303682)(224,0.922223951303682)(225,0.922263837545501)(227,0.922499512026699)(233,0.923293788379794)(235,0.923598325069649)(238,0.923938387876209)(238,0.923938387876209)(238,0.923938387876209)(239,0.924001529147254)(240,0.924078878679716)(240,0.924078878679716)(241,0.924076546647755)(242,0.924057443765051)(243,0.924044270693137)(243,0.924044270693137)(246,0.924062807213879)(246,0.924062807213879)(246,0.924062807213879)(247,0.924042843802876)(247,0.924042843802876)(248,0.924015597180932)(252,0.923846822306944)(254,0.923796455366602)(255,0.923768605198312)(258,0.923852498414158)(258,0.923852498414158)(260,0.923955684421483)(262,0.924247397300205)(264,0.924454022440395)(264,0.924454022440395)(272,0.926365777994896)(272,0.926365777994896)(273,0.926699583872677)(275,0.927284000330371)(275,0.927284000330371)(276,0.927568301179212)(278,0.928080871832896)(289,0.930306933644532)(290,0.930539765462226)(291,0.930782454161335)(293,0.931246458472807)(293,0.931246458472807)(293,0.931246458472807)(295,0.931758990277246)(297,0.932275186487554)(298,0.932461202499203)(299,0.932647597268203)(303,0.933551737260504)(304,0.933737575257405)(305,0.933918283557852)(305,0.933918283557852)(305,0.933918283557852)(310,0.934881472031446)(312,0.935173718209986)(314,0.93546193238369)(314,0.93546193238369)(315,0.935628371170214)(317,0.935881105947605)(318,0.935966237595163)(319,0.936038926113575)(321,0.936155439292079)(321,0.936155439292079)(324,0.936241271837332)(325,0.936275156119255)(332,0.936337118178141)(332,0.936337118178141)(334,0.93643585481898)(338,0.936648619990161)(339,0.936702987191422)(343,0.936914326566584)(343,0.936914326566584)(344,0.936964602966107)(345,0.937013568221292)(346,0.937061113986152)(350,0.937235493173193)(351,0.937274906062835)(352,0.937312586292295)(353,0.937348517634781)(354,0.937382670933078)(357,0.937474264759098)(358,0.937501185068059)(360,0.937549700378819)(363,0.937609397310295)(364,0.937625860809616)(367,0.937665013604183)(368,0.937674642511512)(368,0.937674642511512)(372,0.937696348620857)(374,0.937697506208666)(377,0.937687444638199)(380,0.937663291943414)(380,0.937663291943414)(384,0.937608991452025)(387,0.937551344104631)(388,0.937528829481968)(395,0.93732484443829)(396,0.937289140050105)(400,0.937129569809946) 
};
\addlegendentry{\acl (random)};

\addplot [
color=orange,
densely dotted,
line width=1.0pt,
]
coordinates{
 %(1,0.0901587493534934)(2,0.0900748012020438)(3,0.0892803395326255)(4,0.0916412503137695)(5,0.090174888902831)(6,0.0897377998186099)(7,0.244312184953591)(8,0.32590727777754)(9,0.334411150495897)(10,0.319693879109412)(11,0.327767638081797)(12,0.33311298878122)(13,0.38207957285824)(14,0.402520144495342)
 (15,0.500289734309699)(16,0.505058659242646)(17,0.543121923531459)(18,0.562093770311441)(19,0.58310687640404)(20,0.596736271570605)(21,0.610846109832805)(22,0.627121600845919)(23,0.634808148551407)(24,0.652261794848822)(25,0.654230942576377)(26,0.652691681356409)(27,0.661267594496822)(28,0.666026556113219)(29,0.670447749827208)(30,0.678262259527085)(31,0.683782842091598)(32,0.683063051832541)(33,0.691515027636447)(34,0.689968538821971)(35,0.690763351237076)(36,0.698613326083938)(37,0.701842559663659)(38,0.709734251619398)(39,0.717032447503282)(40,0.71853439866215)(41,0.727469415234565)(42,0.73217394494008)(43,0.738053530200522)(44,0.74591776576775)(45,0.747901109561062)(46,0.752109588928953)(47,0.754080838080412)(48,0.755580577662902)(49,0.756759707487492)(50,0.762841140104268)(51,0.767428704579424)(52,0.77036120613865)(53,0.773160816548109)(54,0.77615967630826)(55,0.779342381581926)(56,0.778991000483797)(57,0.779234193765466)(58,0.779913851658857)(59,0.783032505771668)(60,0.78314464855202)(61,0.784060191813664)(62,0.787460032043844)(63,0.790469023606203)(64,0.795008159093047)(65,0.797045333224793)(66,0.79933433415023)(67,0.800362698156956)(68,0.802795495797156)(69,0.804809271911723)(70,0.805383684422646)(71,0.808275419731623)(72,0.80856063647696)(73,0.8094814780936)(74,0.811916373353693)(75,0.81191265599181)(76,0.8146299065773)(77,0.814841456762696)(78,0.81620855251128)(79,0.816055860637124)(80,0.816248368525684)(81,0.817708424532919)(82,0.818201908745002)(83,0.819982576493933)(84,0.822260118754294)(85,0.822262153964001)(86,0.823332414545528)(87,0.824974804053834)(88,0.825847293840544)(89,0.827069328709349)(90,0.828085649352832)(91,0.829509626570228)(92,0.829236485526413)(93,0.830178452235116)(94,0.829053559432376)(95,0.829495867000151)(96,0.830338501337359)(97,0.830572705036003)(98,0.83174843714687)(99,0.832247045658107)(100,0.833569523848459)(101,0.835022930806174)(102,0.835053051745905)(103,0.835496886694247)(104,0.837036088196142)(105,0.837445986462748)(106,0.837794424194012)(107,0.837424799474049)(108,0.839233618775491)(109,0.838821545264449)(110,0.840143502302919)(111,0.840372197409377)(112,0.839653888875833)(113,0.840040593504219)(114,0.840629999074072)(115,0.840914500577426)(116,0.841517785676703)(117,0.842137190938897)(118,0.841608568299888)(119,0.842124348539919)(120,0.842843328326743)(121,0.843844674428902)(122,0.843783787161877)(123,0.844415053531873)(124,0.844631510505892)(125,0.845603220450694)(126,0.846047337687214)(127,0.846344989592641)(128,0.846535769241502)(129,0.846938971822723)(130,0.848414900694974)(131,0.849320311323704)(132,0.850513570989798)(133,0.851037663500759)(134,0.851900729793438)(135,0.852372854008322)(136,0.851647131308903)(137,0.852202344358529)(138,0.8536050312729)(139,0.854054965063948)(140,0.854802864718816)(141,0.855286222958447)(142,0.855362523390185)(143,0.855884616448084)(144,0.855848427426202)(145,0.856269284825665)(146,0.856916709176067)(147,0.857211567482171)(148,0.857929632077518)(149,0.859445137943077)(150,0.859725292928075)(151,0.859314835623482)(152,0.859839327199628)(153,0.860089949064697)(154,0.860852383741569)(155,0.860940520784458)(156,0.861336407495848)(157,0.861408667591698)(158,0.862120752726614)(159,0.861967637870309)(160,0.863196402799767)(161,0.863819973034319)(162,0.863897848928837)(163,0.863601161116344)(164,0.863701530449172)(165,0.86334758361853)(166,0.863920763874104)(167,0.864741613121566)(168,0.86488977344174)(169,0.864826564055796)(170,0.865151006183508)(171,0.86449600278498)(172,0.864390705768339)(173,0.864870485146495)(174,0.865571644941946)(175,0.865652627876503)(176,0.866170296270691)(177,0.866541530650006)(178,0.866275055696487)(179,0.866660463303039)(180,0.867447115862694)(181,0.867773856594141)(182,0.867645354478136)(183,0.868046205867009)(184,0.867992479063755)(185,0.86787899192037)(186,0.867784053174963)(187,0.867779470493911)(188,0.86859379578648)(189,0.868254419635225)(190,0.868639240940447)(191,0.868715173583931)(192,0.869660114286786)(193,0.870023230262874)(194,0.870400516781203)(195,0.870627245704497)(196,0.871599900767852)(197,0.871904470567543)(198,0.872701115681872)(199,0.873365720465621)(200,0.873120415749292)(201,0.873968040845917)(202,0.874631977462253)(203,0.874499941932997)(204,0.874371087724295)(205,0.874877953451317)(206,0.875048473774004)(207,0.875447878123387)(208,0.876072636084575)(209,0.875904337588585)(210,0.876259652701822)(211,0.876498716933955)(212,0.876168493936881)(213,0.876329517393846)(214,0.876373100414916)(215,0.8767656831911)(216,0.876959000580441)(217,0.877324520208848)(218,0.877285490605624)(219,0.877186732483806)(220,0.87709167436728)(221,0.877287148154537)(222,0.87730999345433)(223,0.876984167611194)(224,0.877174957302645)(225,0.877771122578454)(226,0.877809720689945)(227,0.878421627065428)(228,0.878567857674815)(229,0.879329003241481)(230,0.88002086640024)(231,0.880567463017552)(232,0.88101943100321)(233,0.881229914439408)(234,0.881731809863213)(235,0.88165676996654)(236,0.882153247711277)(237,0.882291689559894)(238,0.882560606670665)(239,0.882941253128879)(240,0.883095626246573)(241,0.88311432473853)(242,0.883947300226963)(243,0.883783145230702)(244,0.883857385018303)(245,0.884233966834155)(246,0.884638749890142)(247,0.884339627202131)(248,0.884480739987809)(249,0.884514443269033)(250,0.88440300304647)(251,0.88442321585199)(252,0.885165661358225)(253,0.885834258923711)(254,0.88657734176995)(255,0.88694084911334)(256,0.88725309347334)(257,0.887594224188825)(258,0.887783889561116)(259,0.888178533613438)(260,0.888752228001536)(261,0.889191737310354)(262,0.889521869553389)(263,0.889614702195936)(264,0.890065152208919)(265,0.890016795753468)(266,0.890395055014949)(267,0.890161236370351)(268,0.890421302787562)(269,0.890560609832486)(270,0.890344604586066)(271,0.890712238592629)(272,0.890930351998264)(273,0.890668488688035)(274,0.890436079006867)(275,0.890684952653967)(276,0.891036034517177)(277,0.891229115362634)(278,0.891101593394153)(279,0.891335855838963)(280,0.891575166232214)(281,0.891887515019388)(282,0.891578298645951)(283,0.891740296774548)(284,0.892408375394168)(285,0.893109636084831)(286,0.893209531442126)(287,0.893196607998807)(288,0.893066250937401)(289,0.892918194612276)(290,0.892934132285525)(291,0.893039840707697)(292,0.893266332058801)(293,0.893195317083654)(294,0.893476741793334)(295,0.893614640651471)(296,0.893633841767342)(297,0.893834821292362)(298,0.893567481718594)(299,0.893373175408322)(300,0.89347954383789)(301,0.893673469752225)(302,0.893906953160943)(303,0.894211507521856)(304,0.894901197246414)(305,0.894894143912787)(306,0.894942290288044)(307,0.895036063105602)(308,0.895381388072068)(309,0.895400195049574)(310,0.895520524313272)(311,0.895472152865917)(312,0.895484452761701)(313,0.895530720177722)(314,0.895695581074694)(315,0.895741456370212)(316,0.895383226054102)(317,0.895765896997116)(318,0.895914601781099)(319,0.89601225971738)(320,0.895822639066704)(321,0.895897327255566)(322,0.895984220837649)(323,0.895926466210874)(324,0.89608155178006)(325,0.896109772674995)(326,0.896131864269665)(327,0.896322400069533)(328,0.896297745237207)(329,0.896434544951132)(330,0.896815586806484)(331,0.896949843988861)(332,0.896848845719006)(333,0.897179337838408)(334,0.897556442947812)(335,0.897650847515074)(336,0.898053140521045)(337,0.897959088074473)(338,0.898043135442993)(339,0.897894286549597)(340,0.898254000224167)(341,0.898473438334324)(342,0.898526969149518)(343,0.898566741655091)(344,0.899022223519436)(345,0.899185088151081)(346,0.899955768562668)(347,0.899779893728872)(348,0.899227580359294)(349,0.899239015159351)(350,0.899332694453285)(351,0.899329015057992)(352,0.89941998751251)(353,0.899338134916934)(354,0.899576751774622)(355,0.899711455550783)(356,0.899468435458987)(357,0.899438922851876)(358,0.899405542101104)(359,0.899264426000097)(360,0.899523959410296)(361,0.899611288264275)(362,0.899911984252751)(363,0.900006922692745)(364,0.899725137709551)(365,0.900302332010367)(366,0.900756053679175)(367,0.900690857148752)(368,0.900524121211805)(369,0.9006966784341)(370,0.900566529646529)(371,0.900472090437646)(372,0.900563554855705)(373,0.900957953014257)(374,0.900999553579077)(375,0.900995714760338)(376,0.901453948661611)(377,0.901654326055017)(378,0.901607204865952)(379,0.90145435909391)(380,0.901736422557936)(381,0.901879655719289)(382,0.902063203781939)(383,0.902435754387496)(384,0.902603421779432)(385,0.902865583501788)(386,0.90270359243086)(387,0.902925314842565)(388,0.903186408244921)(389,0.90324250877155)(390,0.902933606145831)(391,0.903052873272245)(392,0.903259458833385)(393,0.903333496981637)(394,0.903227728066434)(395,0.903244250808563)(396,0.903180623891756)(397,0.903346447843517)(398,0.903603239581998)(399,0.903571073906319)(400,0.903736484050599) 
};
\addlegendentry{\var};

\end{axis}
\end{tikzpicture}%

%% This file was created by matlab2tikz v0.2.3.
% Copyright (c) 2008--2012, Nico Schlömer <nico.schloemer@gmail.com>
% All rights reserved.
% 
% 
% 
\begin{tikzpicture}

\begin{axis}[%
tick label style={font=\tiny},
label style={font=\tiny},
label shift={-4pt},
xlabel shift={-6pt},
legend style={font=\tiny},
view={0}{90},
width=\figurewidth,
height=\figureheight,
scale only axis,
xmin=0, xmax=400,
xlabel={Samples},
ymin=0.48, ymax=1,
ylabel={$F_1$-score},
axis lines*=left,
legend cell align=left,
legend style={at={(1.03,0)},anchor=south east,fill=none,draw=none,align=left,row sep=-0.2em},
clip=false]

\addplot [
color=blue,
solid,
line width=1.0pt,
]
coordinates{
 (12,0.525389729238701)(12,0.525389729238701)(12,0.525389729238701)(12,0.525389729238701)(12,0.525389729238701)(12,0.525389729238701)(13,0.583584921790136)(13,0.583584921790136)(14,0.6452877248702)(14,0.6452877248702)(14,0.6452877248702)(14,0.6452877248702)(14,0.6452877248702)(14,0.6452877248702)(14,0.6452877248702)(14,0.6452877248702)(15,0.694141611127947)(15,0.694141611127947)(15,0.694141611127947)(15,0.694141611127947)(15,0.694141611127947)(15,0.694141611127947)(15,0.694141611127947)(16,0.722223931031106)(16,0.722223931031106)(16,0.722223931031106)(16,0.722223931031106)(16,0.722223931031106)(16,0.722223931031106)(16,0.722223931031106)(16,0.722223931031106)(16,0.722223931031106)(16,0.722223931031106)(17,0.737813157621791)(17,0.737813157621791)(17,0.737813157621791)(17,0.737813157621791)(17,0.737813157621791)(17,0.737813157621791)(17,0.737813157621791)(17,0.737813157621791)(17,0.737813157621791)(17,0.737813157621791)(18,0.752313481770937)(18,0.752313481770937)(18,0.752313481770937)(18,0.752313481770937)(18,0.752313481770937)(18,0.752313481770937)(18,0.752313481770937)(18,0.752313481770937)(18,0.752313481770937)(18,0.752313481770937)(18,0.752313481770937)(18,0.752313481770937)(18,0.752313481770937)(18,0.752313481770937)(18,0.752313481770937)(18,0.752313481770937)(18,0.752313481770937)(18,0.752313481770937)(18,0.752313481770937)(18,0.752313481770937)(18,0.752313481770937)(18,0.752313481770937)(18,0.752313481770937)(19,0.763171861648842)(19,0.763171861648842)(19,0.763171861648842)(19,0.763171861648842)(19,0.763171861648842)(19,0.763171861648842)(19,0.763171861648842)(19,0.763171861648842)(19,0.763171861648842)(19,0.763171861648842)(19,0.763171861648842)(19,0.763171861648842)(19,0.763171861648842)(19,0.763171861648842)(19,0.763171861648842)(19,0.763171861648842)(19,0.763171861648842)(19,0.763171861648842)(19,0.763171861648842)(19,0.763171861648842)(20,0.771681075504717)(20,0.771681075504717)(20,0.771681075504717)(20,0.771681075504717)(20,0.771681075504717)(20,0.771681075504717)(20,0.771681075504717)(20,0.771681075504717)(20,0.771681075504717)(20,0.771681075504717)(20,0.771681075504717)(20,0.771681075504717)(20,0.771681075504717)(20,0.771681075504717)(20,0.771681075504717)(20,0.771681075504717)(20,0.771681075504717)(20,0.771681075504717)(20,0.771681075504717)(21,0.775001484073553)(21,0.775001484073553)(21,0.775001484073553)(21,0.775001484073553)(21,0.775001484073553)(21,0.775001484073553)(21,0.775001484073553)(21,0.775001484073553)(21,0.775001484073553)(21,0.775001484073553)(21,0.775001484073553)(21,0.775001484073553)(21,0.775001484073553)(21,0.775001484073553)(21,0.775001484073553)(21,0.775001484073553)(21,0.775001484073553)(21,0.775001484073553)(21,0.775001484073553)(21,0.775001484073553)(22,0.777369759522117)(22,0.777369759522117)(22,0.777369759522117)(22,0.777369759522117)(22,0.777369759522117)(22,0.777369759522117)(22,0.777369759522117)(22,0.777369759522117)(22,0.777369759522117)(22,0.777369759522117)(22,0.777369759522117)(22,0.777369759522117)(22,0.777369759522117)(22,0.777369759522117)(22,0.777369759522117)(22,0.777369759522117)(22,0.777369759522117)(22,0.777369759522117)(22,0.777369759522117)(23,0.783651333731729)(23,0.783651333731729)(23,0.783651333731729)(23,0.783651333731729)(23,0.783651333731729)(23,0.783651333731729)(23,0.783651333731729)(23,0.783651333731729)(23,0.783651333731729)(23,0.783651333731729)(23,0.783651333731729)(23,0.783651333731729)(23,0.783651333731729)(23,0.783651333731729)(23,0.783651333731729)(23,0.783651333731729)(23,0.783651333731729)(23,0.783651333731729)(23,0.783651333731729)(23,0.783651333731729)(23,0.783651333731729)(23,0.783651333731729)(23,0.783651333731729)(23,0.783651333731729)(23,0.783651333731729)(23,0.783651333731729)(24,0.787253908009297)(24,0.787253908009297)(24,0.787253908009297)(24,0.787253908009297)(24,0.787253908009297)(24,0.787253908009297)(24,0.787253908009297)(24,0.787253908009297)(24,0.787253908009297)(24,0.787253908009297)(24,0.787253908009297)(24,0.787253908009297)(24,0.787253908009297)(24,0.787253908009297)(24,0.787253908009297)(24,0.787253908009297)(24,0.787253908009297)(24,0.787253908009297)(24,0.787253908009297)(24,0.787253908009297)(24,0.787253908009297)(24,0.787253908009297)(24,0.787253908009297)(24,0.787253908009297)(25,0.788536322951009)(25,0.788536322951009)(25,0.788536322951009)(25,0.788536322951009)(25,0.788536322951009)(25,0.788536322951009)(25,0.788536322951009)(25,0.788536322951009)(25,0.788536322951009)(25,0.788536322951009)(25,0.788536322951009)(25,0.788536322951009)(25,0.788536322951009)(25,0.788536322951009)(25,0.788536322951009)(25,0.788536322951009)(25,0.788536322951009)(25,0.788536322951009)(25,0.788536322951009)(25,0.788536322951009)(25,0.788536322951009)(25,0.788536322951009)(25,0.788536322951009)(25,0.788536322951009)(26,0.790528570525884)(26,0.790528570525884)(26,0.790528570525884)(26,0.790528570525884)(26,0.790528570525884)(26,0.790528570525884)(26,0.790528570525884)(26,0.790528570525884)(26,0.790528570525884)(26,0.790528570525884)(26,0.790528570525884)(26,0.790528570525884)(26,0.790528570525884)(26,0.790528570525884)(26,0.790528570525884)(26,0.790528570525884)(26,0.790528570525884)(26,0.790528570525884)(26,0.790528570525884)(26,0.790528570525884)(26,0.790528570525884)(26,0.790528570525884)(26,0.790528570525884)(26,0.790528570525884)(26,0.790528570525884)(27,0.791798162520511)(27,0.791798162520511)(27,0.791798162520511)(27,0.791798162520511)(27,0.791798162520511)(27,0.791798162520511)(27,0.791798162520511)(27,0.791798162520511)(27,0.791798162520511)(27,0.791798162520511)(27,0.791798162520511)(27,0.791798162520511)(27,0.791798162520511)(27,0.791798162520511)(27,0.791798162520511)(27,0.791798162520511)(27,0.791798162520511)(27,0.791798162520511)(27,0.791798162520511)(28,0.795860007162673)(28,0.795860007162673)(28,0.795860007162673)(28,0.795860007162673)(28,0.795860007162673)(28,0.795860007162673)(28,0.795860007162673)(28,0.795860007162673)(28,0.795860007162673)(28,0.795860007162673)(28,0.795860007162673)(28,0.795860007162673)(28,0.795860007162673)(28,0.795860007162673)(28,0.795860007162673)(28,0.795860007162673)(28,0.795860007162673)(28,0.795860007162673)(28,0.795860007162673)(28,0.795860007162673)(28,0.795860007162673)(28,0.795860007162673)(28,0.795860007162673)(29,0.799642437775613)(29,0.799642437775613)(29,0.799642437775613)(29,0.799642437775613)(29,0.799642437775613)(29,0.799642437775613)(29,0.799642437775613)(29,0.799642437775613)(29,0.799642437775613)(29,0.799642437775613)(29,0.799642437775613)(29,0.799642437775613)(29,0.799642437775613)(29,0.799642437775613)(29,0.799642437775613)(29,0.799642437775613)(29,0.799642437775613)(29,0.799642437775613)(29,0.799642437775613)(29,0.799642437775613)(29,0.799642437775613)(30,0.802931437577957)(30,0.802931437577957)(30,0.802931437577957)(30,0.802931437577957)(30,0.802931437577957)(30,0.802931437577957)(30,0.802931437577957)(30,0.802931437577957)(30,0.802931437577957)(30,0.802931437577957)(30,0.802931437577957)(30,0.802931437577957)(30,0.802931437577957)(30,0.802931437577957)(30,0.802931437577957)(30,0.802931437577957)(31,0.807754040512801)(31,0.807754040512801)(31,0.807754040512801)(31,0.807754040512801)(31,0.807754040512801)(31,0.807754040512801)(31,0.807754040512801)(31,0.807754040512801)(31,0.807754040512801)(31,0.807754040512801)(31,0.807754040512801)(31,0.807754040512801)(31,0.807754040512801)(31,0.807754040512801)(31,0.807754040512801)(31,0.807754040512801)(31,0.807754040512801)(31,0.807754040512801)(31,0.807754040512801)(31,0.807754040512801)(31,0.807754040512801)(31,0.807754040512801)(32,0.81348702983066)(32,0.81348702983066)(32,0.81348702983066)(32,0.81348702983066)(32,0.81348702983066)(32,0.81348702983066)(32,0.81348702983066)(32,0.81348702983066)(32,0.81348702983066)(32,0.81348702983066)(32,0.81348702983066)(32,0.81348702983066)(32,0.81348702983066)(32,0.81348702983066)(32,0.81348702983066)(32,0.81348702983066)(32,0.81348702983066)(32,0.81348702983066)(32,0.81348702983066)(32,0.81348702983066)(32,0.81348702983066)(33,0.817117453465487)(33,0.817117453465487)(33,0.817117453465487)(33,0.817117453465487)(33,0.817117453465487)(33,0.817117453465487)(33,0.817117453465487)(33,0.817117453465487)(33,0.817117453465487)(33,0.817117453465487)(33,0.817117453465487)(33,0.817117453465487)(33,0.817117453465487)(33,0.817117453465487)(33,0.817117453465487)(33,0.817117453465487)(33,0.817117453465487)(33,0.817117453465487)(33,0.817117453465487)(34,0.819121007285051)(34,0.819121007285051)(34,0.819121007285051)(34,0.819121007285051)(34,0.819121007285051)(34,0.819121007285051)(34,0.819121007285051)(34,0.819121007285051)(34,0.819121007285051)(34,0.819121007285051)(34,0.819121007285051)(34,0.819121007285051)(34,0.819121007285051)(34,0.819121007285051)(34,0.819121007285051)(34,0.819121007285051)(35,0.819721680603458)(35,0.819721680603458)(35,0.819721680603458)(35,0.819721680603458)(35,0.819721680603458)(35,0.819721680603458)(35,0.819721680603458)(35,0.819721680603458)(35,0.819721680603458)(35,0.819721680603458)(35,0.819721680603458)(35,0.819721680603458)(35,0.819721680603458)(35,0.819721680603458)(36,0.820654577668584)(36,0.820654577668584)(36,0.820654577668584)(36,0.820654577668584)(36,0.820654577668584)(36,0.820654577668584)(36,0.820654577668584)(36,0.820654577668584)(37,0.821854924500773)(37,0.821854924500773)(37,0.821854924500773)(37,0.821854924500773)(37,0.821854924500773)(37,0.821854924500773)(37,0.821854924500773)(37,0.821854924500773)(37,0.821854924500773)(37,0.821854924500773)(37,0.821854924500773)(38,0.824373927769619)(38,0.824373927769619)(38,0.824373927769619)(38,0.824373927769619)(38,0.824373927769619)(38,0.824373927769619)(38,0.824373927769619)(38,0.824373927769619)(38,0.824373927769619)(38,0.824373927769619)(39,0.827634206071317)(39,0.827634206071317)(39,0.827634206071317)(39,0.827634206071317)(39,0.827634206071317)(39,0.827634206071317)(39,0.827634206071317)(39,0.827634206071317)(39,0.827634206071317)(39,0.827634206071317)(39,0.827634206071317)(39,0.827634206071317)(39,0.827634206071317)(39,0.827634206071317)(40,0.830435351761447)(40,0.830435351761447)(40,0.830435351761447)(40,0.830435351761447)(40,0.830435351761447)(40,0.830435351761447)(40,0.830435351761447)(40,0.830435351761447)(40,0.830435351761447)(40,0.830435351761447)(40,0.830435351761447)(40,0.830435351761447)(40,0.830435351761447)(40,0.830435351761447)(41,0.834470507905048)(41,0.834470507905048)(41,0.834470507905048)(41,0.834470507905048)(41,0.834470507905048)(41,0.834470507905048)(41,0.834470507905048)(41,0.834470507905048)(41,0.834470507905048)(41,0.834470507905048)(41,0.834470507905048)(41,0.834470507905048)(41,0.834470507905048)(42,0.838205632935293)(42,0.838205632935293)(42,0.838205632935293)(42,0.838205632935293)(42,0.838205632935293)(42,0.838205632935293)(42,0.838205632935293)(43,0.841280283556246)(43,0.841280283556246)(43,0.841280283556246)(43,0.841280283556246)(43,0.841280283556246)(43,0.841280283556246)(43,0.841280283556246)(43,0.841280283556246)(43,0.841280283556246)(44,0.843688650544686)(44,0.843688650544686)(44,0.843688650544686)(44,0.843688650544686)(44,0.843688650544686)(44,0.843688650544686)(44,0.843688650544686)(44,0.843688650544686)(44,0.843688650544686)(44,0.843688650544686)(44,0.843688650544686)(44,0.843688650544686)(44,0.843688650544686)(44,0.843688650544686)(44,0.843688650544686)(45,0.845855175315059)(45,0.845855175315059)(45,0.845855175315059)(45,0.845855175315059)(45,0.845855175315059)(45,0.845855175315059)(45,0.845855175315059)(45,0.845855175315059)(45,0.845855175315059)(45,0.845855175315059)(46,0.847747171766387)(46,0.847747171766387)(46,0.847747171766387)(46,0.847747171766387)(46,0.847747171766387)(46,0.847747171766387)(46,0.847747171766387)(46,0.847747171766387)(46,0.847747171766387)(46,0.847747171766387)(47,0.848805256169488)(47,0.848805256169488)(47,0.848805256169488)(47,0.848805256169488)(47,0.848805256169488)(47,0.848805256169488)(47,0.848805256169488)(47,0.848805256169488)(48,0.849696011990664)(48,0.849696011990664)(48,0.849696011990664)(48,0.849696011990664)(48,0.849696011990664)(48,0.849696011990664)(48,0.849696011990664)(48,0.849696011990664)(48,0.849696011990664)(48,0.849696011990664)(48,0.849696011990664)(49,0.850081015673844)(49,0.850081015673844)(49,0.850081015673844)(49,0.850081015673844)(49,0.850081015673844)(49,0.850081015673844)(49,0.850081015673844)(49,0.850081015673844)(49,0.850081015673844)(49,0.850081015673844)(49,0.850081015673844)(49,0.850081015673844)(49,0.850081015673844)(50,0.850456509555821)(50,0.850456509555821)(50,0.850456509555821)(50,0.850456509555821)(50,0.850456509555821)(51,0.850601383085762)(51,0.850601383085762)(51,0.850601383085762)(51,0.850601383085762)(51,0.850601383085762)(51,0.850601383085762)(51,0.850601383085762)(51,0.850601383085762)(51,0.850601383085762)(51,0.850601383085762)(51,0.850601383085762)(51,0.850601383085762)(52,0.850856252027051)(52,0.850856252027051)(52,0.850856252027051)(52,0.850856252027051)(52,0.850856252027051)(52,0.850856252027051)(52,0.850856252027051)(52,0.850856252027051)(53,0.851568259580076)(53,0.851568259580076)(53,0.851568259580076)(53,0.851568259580076)(53,0.851568259580076)(53,0.851568259580076)(53,0.851568259580076)(54,0.85344860224575)(54,0.85344860224575)(54,0.85344860224575)(54,0.85344860224575)(54,0.85344860224575)(54,0.85344860224575)(54,0.85344860224575)(54,0.85344860224575)(54,0.85344860224575)(54,0.85344860224575)(54,0.85344860224575)(55,0.855302420436072)(55,0.855302420436072)(55,0.855302420436072)(56,0.857448257107884)(56,0.857448257107884)(56,0.857448257107884)(56,0.857448257107884)(56,0.857448257107884)(56,0.857448257107884)(57,0.858922512976792)(57,0.858922512976792)(57,0.858922512976792)(57,0.858922512976792)(58,0.86035041384041)(58,0.86035041384041)(58,0.86035041384041)(58,0.86035041384041)(58,0.86035041384041)(58,0.86035041384041)(59,0.861467963349322)(59,0.861467963349322)(59,0.861467963349322)(59,0.861467963349322)(59,0.861467963349322)(59,0.861467963349322)(60,0.862272339309017)(60,0.862272339309017)(60,0.862272339309017)(60,0.862272339309017)(60,0.862272339309017)(60,0.862272339309017)(60,0.862272339309017)(61,0.862811476718565)(61,0.862811476718565)(62,0.863111731637971)(62,0.863111731637971)(62,0.863111731637971)(62,0.863111731637971)(62,0.863111731637971)(62,0.863111731637971)(62,0.863111731637971)(62,0.863111731637971)(62,0.863111731637971)(62,0.863111731637971)(62,0.863111731637971)(63,0.863656186577192)(63,0.863656186577192)(63,0.863656186577192)(63,0.863656186577192)(63,0.863656186577192)(63,0.863656186577192)(63,0.863656186577192)(63,0.863656186577192)(64,0.864103605871065)(64,0.864103605871065)(64,0.864103605871065)(64,0.864103605871065)(64,0.864103605871065)(64,0.864103605871065)(64,0.864103605871065)(64,0.864103605871065)(64,0.864103605871065)(64,0.864103605871065)(64,0.864103605871065)(65,0.86507780963456)(65,0.86507780963456)(66,0.866009971566013)(66,0.866009971566013)(66,0.866009971566013)(66,0.866009971566013)(66,0.866009971566013)(66,0.866009971566013)(66,0.866009971566013)(67,0.867691929917099)(67,0.867691929917099)(67,0.867691929917099)(67,0.867691929917099)(67,0.867691929917099)(67,0.867691929917099)(68,0.869077532646006)(68,0.869077532646006)(68,0.869077532646006)(68,0.869077532646006)(68,0.869077532646006)(69,0.870703579607496)(69,0.870703579607496)(69,0.870703579607496)(69,0.870703579607496)(70,0.872801349112882)(70,0.872801349112882)(70,0.872801349112882)(70,0.872801349112882)(70,0.872801349112882)(70,0.872801349112882)(70,0.872801349112882)(70,0.872801349112882)(71,0.874732505293145)(71,0.874732505293145)(71,0.874732505293145)(71,0.874732505293145)(71,0.874732505293145)(71,0.874732505293145)(71,0.874732505293145)(71,0.874732505293145)(71,0.874732505293145)(72,0.876711121620475)(72,0.876711121620475)(72,0.876711121620475)(72,0.876711121620475)(72,0.876711121620475)(72,0.876711121620475)(72,0.876711121620475)(72,0.876711121620475)(72,0.876711121620475)(72,0.876711121620475)(73,0.878516503093929)(73,0.878516503093929)(73,0.878516503093929)(73,0.878516503093929)(74,0.880232969508431)(74,0.880232969508431)(74,0.880232969508431)(74,0.880232969508431)(75,0.881808746603159)(75,0.881808746603159)(75,0.881808746603159)(75,0.881808746603159)(75,0.881808746603159)(75,0.881808746603159)(76,0.883744610663888)(76,0.883744610663888)(76,0.883744610663888)(76,0.883744610663888)(76,0.883744610663888)(76,0.883744610663888)(77,0.88472909855246)(77,0.88472909855246)(77,0.88472909855246)(77,0.88472909855246)(78,0.884555463989538)(78,0.884555463989538)(78,0.884555463989538)(78,0.884555463989538)(78,0.884555463989538)(79,0.884646018613939)(80,0.884460241811596)(80,0.884460241811596)(80,0.884460241811596)(80,0.884460241811596)(81,0.884311078467911)(81,0.884311078467911)(81,0.884311078467911)(81,0.884311078467911)(82,0.884246791575917)(83,0.884377623245239)(83,0.884377623245239)(83,0.884377623245239)(83,0.884377623245239)(84,0.885106305773935)(84,0.885106305773935)(84,0.885106305773935)(84,0.885106305773935)(84,0.885106305773935)(84,0.885106305773935)(85,0.88552527238202)(85,0.88552527238202)(85,0.88552527238202)(85,0.88552527238202)(85,0.88552527238202)(86,0.885998800491969)(86,0.885998800491969)(86,0.885998800491969)(86,0.885998800491969)(86,0.885998800491969)(87,0.88681795530179)(87,0.88681795530179)(87,0.88681795530179)(87,0.88681795530179)(87,0.88681795530179)(88,0.887400696555468)(88,0.887400696555468)(88,0.887400696555468)(89,0.888317148402291)(89,0.888317148402291)(89,0.888317148402291)(89,0.888317148402291)(89,0.888317148402291)(90,0.889063410603678)(90,0.889063410603678)(91,0.8899389821265)(91,0.8899389821265)(91,0.8899389821265)(91,0.8899389821265)(91,0.8899389821265)(92,0.890961362449436)(92,0.890961362449436)(92,0.890961362449436)(92,0.890961362449436)(93,0.892098010108082)(93,0.892098010108082)(93,0.892098010108082)(93,0.892098010108082)(93,0.892098010108082)(93,0.892098010108082)(93,0.892098010108082)(93,0.892098010108082)(94,0.893197369981784)(94,0.893197369981784)(94,0.893197369981784)(94,0.893197369981784)(94,0.893197369981784)(94,0.893197369981784)(95,0.894366186406029)(95,0.894366186406029)(95,0.894366186406029)(96,0.895354692137565)(96,0.895354692137565)(96,0.895354692137565)(97,0.896465261801814)(97,0.896465261801814)(98,0.897536748184418)(98,0.897536748184418)(98,0.897536748184418)(98,0.897536748184418)(98,0.897536748184418)(99,0.898532320829218)(99,0.898532320829218)(99,0.898532320829218)(100,0.899447127983014)(100,0.899447127983014)(100,0.899447127983014)(100,0.899447127983014)(100,0.899447127983014)(100,0.899447127983014)(100,0.899447127983014)(101,0.900286935651722)(101,0.900286935651722)(102,0.901031861376198)(102,0.901031861376198)(102,0.901031861376198)(102,0.901031861376198)(103,0.901644069949881)(103,0.901644069949881)(103,0.901644069949881)(103,0.901644069949881)(103,0.901644069949881)(103,0.901644069949881)(103,0.901644069949881)(103,0.901644069949881)(104,0.902113279574632)(104,0.902113279574632)(104,0.902113279574632)(105,0.902459903960966)(105,0.902459903960966)(106,0.902725789110883)(106,0.902725789110883)(107,0.902957601537118)(108,0.90321076174511)(109,0.903643733720737)(109,0.903643733720737)(109,0.903643733720737)(110,0.903889390956576)(110,0.903889390956576)(111,0.904113483010714)(111,0.904113483010714)(111,0.904113483010714)(112,0.904293168029492)(112,0.904293168029492)(112,0.904293168029492)(112,0.904293168029492)(112,0.904293168029492)(113,0.904642509861186)(113,0.904642509861186)(113,0.904642509861186)(113,0.904642509861186)(114,0.904556110386201)(114,0.904556110386201)(115,0.904807985048791)(115,0.904807985048791)(115,0.904807985048791)(115,0.904807985048791)(115,0.904807985048791)(115,0.904807985048791)(116,0.904836106411204)(117,0.905024666813239)(117,0.905024666813239)(117,0.905024666813239)(117,0.905024666813239)(117,0.905024666813239)(118,0.905300784291066)(118,0.905300784291066)(118,0.905300784291066)(118,0.905300784291066)(119,0.905578299407887)(119,0.905578299407887)(119,0.905578299407887)(119,0.905578299407887)(120,0.905872483526514)(120,0.905872483526514)(121,0.906175554222719)(121,0.906175554222719)(121,0.906175554222719)(121,0.906175554222719)(122,0.906562433031515)(122,0.906562433031515)(122,0.906562433031515)(122,0.906562433031515)(122,0.906562433031515)(123,0.906854288775092)(123,0.906854288775092)(123,0.906854288775092)(124,0.907156222796794)(124,0.907156222796794)(124,0.907156222796794)(124,0.907156222796794)(124,0.907156222796794)(125,0.907454488925756)(125,0.907454488925756)(125,0.907454488925756)(125,0.907454488925756)(125,0.907454488925756)(125,0.907454488925756)(126,0.9078110620541)(126,0.9078110620541)(126,0.9078110620541)(126,0.9078110620541)(126,0.9078110620541)(126,0.9078110620541)(126,0.9078110620541)(127,0.908179854336624)(127,0.908179854336624)(127,0.908179854336624)(128,0.908568496633346)(129,0.908977492752343)(129,0.908977492752343)(129,0.908977492752343)(130,0.909447196316944)(130,0.909447196316944)(130,0.909447196316944)(130,0.909447196316944)(130,0.909447196316944)(131,0.909855645873166)(131,0.909855645873166)(132,0.910259004646389)(132,0.910259004646389)(132,0.910259004646389)(133,0.910692200586484)(133,0.910692200586484)(134,0.911186039777786)(134,0.911186039777786)(134,0.911186039777786)(134,0.911186039777786)(134,0.911186039777786)(135,0.911651986503027)(135,0.911651986503027)(135,0.911651986503027)(135,0.911651986503027)(135,0.911651986503027)(135,0.911651986503027)(135,0.911651986503027)(136,0.912108783936407)(136,0.912108783936407)(136,0.912108783936407)(136,0.912108783936407)(137,0.912543112079418)(137,0.912543112079418)(137,0.912543112079418)(137,0.912543112079418)(137,0.912543112079418)(138,0.913032223759058)(138,0.913032223759058)(138,0.913032223759058)(139,0.91339881069493)(139,0.91339881069493)(139,0.91339881069493)(139,0.91339881069493)(139,0.91339881069493)(141,0.913922883582713)(141,0.913922883582713)(142,0.914115921001855)(142,0.914115921001855)(142,0.914115921001855)(143,0.914295700900565)(143,0.914295700900565)(144,0.914488779405359)(144,0.914488779405359)(144,0.914488779405359)(144,0.914488779405359)(144,0.914488779405359)(144,0.914488779405359)(144,0.914488779405359)(145,0.914702316047473)(145,0.914702316047473)(145,0.914702316047473)(145,0.914702316047473)(147,0.915088035953189)(147,0.915088035953189)(147,0.915088035953189)(147,0.915088035953189)(147,0.915088035953189)(148,0.915213216229748)(149,0.915285591683416)(149,0.915285591683416)(150,0.915329941037372)(150,0.915329941037372)(150,0.915329941037372)(151,0.915305844631413)(151,0.915305844631413)(151,0.915305844631413)(151,0.915305844631413)(151,0.915305844631413)(151,0.915305844631413)(151,0.915305844631413)(152,0.915324993675669)(152,0.915324993675669)(152,0.915324993675669)(152,0.915324993675669)(152,0.915324993675669)(152,0.915324993675669)(152,0.915324993675669)(153,0.915369544343077)(153,0.915369544343077)(153,0.915369544343077)(154,0.915441996007408)(154,0.915441996007408)(155,0.915541182054157)(155,0.915541182054157)(155,0.915541182054157)(155,0.915541182054157)(156,0.915660282167294)(158,0.915850148452726)(158,0.915850148452726)(158,0.915850148452726)(158,0.915850148452726)(158,0.915850148452726)(159,0.915981196738957)(159,0.915981196738957)(159,0.915981196738957)(159,0.915981196738957)(159,0.915981196738957)(160,0.91621375305472)(160,0.91621375305472)(160,0.91621375305472)(160,0.91621375305472)(160,0.91621375305472)(161,0.916552190465666)(161,0.916552190465666)(161,0.916552190465666)(162,0.917224271499046)(162,0.917224271499046)(162,0.917224271499046)(163,0.917835370480429)(163,0.917835370480429)(163,0.917835370480429)(163,0.917835370480429)(163,0.917835370480429)(163,0.917835370480429)(163,0.917835370480429)(163,0.917835370480429)(163,0.917835370480429)(164,0.918538640387051)(164,0.918538640387051)(165,0.919299877057227)(165,0.919299877057227)(165,0.919299877057227)(165,0.919299877057227)(165,0.919299877057227)(165,0.919299877057227)(166,0.920242537714861)(166,0.920242537714861)(166,0.920242537714861)(167,0.920832687150019)(167,0.920832687150019)(167,0.920832687150019)(167,0.920832687150019)(167,0.920832687150019)(167,0.920832687150019)(168,0.921535659241272)(168,0.921535659241272)(168,0.921535659241272)(169,0.922155532426323)(169,0.922155532426323)(169,0.922155532426323)(169,0.922155532426323)(170,0.922683894405479)(170,0.922683894405479)(170,0.922683894405479)(170,0.922683894405479)(170,0.922683894405479)(171,0.923120523541776)(171,0.923120523541776)(171,0.923120523541776)(172,0.923679907387625)(172,0.923679907387625)(172,0.923679907387625)(173,0.924004498600568)(173,0.924004498600568)(173,0.924004498600568)(173,0.924004498600568)(173,0.924004498600568)(174,0.924184834412404)(174,0.924184834412404)(174,0.924184834412404)(174,0.924184834412404)(175,0.924167512659597)(176,0.923948639321016)(176,0.923948639321016)(176,0.923948639321016)(177,0.923559563698677)(177,0.923559563698677)(178,0.922632840716325)(178,0.922632840716325)(178,0.922632840716325)(178,0.922632840716325)(178,0.922632840716325)(179,0.922382518203478)(180,0.921640409794613)(180,0.921640409794613)(180,0.921640409794613)(180,0.921640409794613)(181,0.921120825923385)(181,0.921120825923385)(181,0.921120825923385)(181,0.921120825923385)(181,0.921120825923385)(181,0.921120825923385)(182,0.920145169175424)(182,0.920145169175424)(182,0.920145169175424)(182,0.920145169175424)(182,0.920145169175424)(183,0.919062712595055)(183,0.919062712595055)(183,0.919062712595055)(184,0.917912203888703)(184,0.917912203888703)(185,0.916733045590557)(185,0.916733045590557)(185,0.916733045590557)(185,0.916733045590557)(185,0.916733045590557)(186,0.915603785686091)(186,0.915603785686091)(186,0.915603785686091)(186,0.915603785686091)(186,0.915603785686091)(186,0.915603785686091)(187,0.91509533706204)(188,0.914371838720635)(188,0.914371838720635)(189,0.914448105197798)(189,0.914448105197798)(189,0.914448105197798)(190,0.913562953347711)(190,0.913562953347711)(190,0.913562953347711)(191,0.913512805162941)(191,0.913512805162941)(191,0.913512805162941)(192,0.913676781821202)(192,0.913676781821202)(192,0.913676781821202)(193,0.914010808376368)(193,0.914010808376368)(193,0.914010808376368)(193,0.914010808376368)(193,0.914010808376368)(194,0.9144626005762)(194,0.9144626005762)(195,0.914997751797744)(195,0.914997751797744)(195,0.914997751797744)(195,0.914997751797744)(195,0.914997751797744)(195,0.914997751797744)(196,0.915596581490284)(196,0.915596581490284)(196,0.915596581490284)(196,0.915596581490284)(196,0.915596581490284)(197,0.91624710911367)(197,0.91624710911367)(197,0.91624710911367)(197,0.91624710911367)(198,0.916947009198699)(198,0.916947009198699)(198,0.916947009198699)(198,0.916947009198699)(198,0.916947009198699)(199,0.917710224218834)(199,0.917710224218834)(199,0.917710224218834)(200,0.918557136782912)(200,0.918557136782912)(200,0.918557136782912)(200,0.918557136782912)(201,0.919742600375148)(201,0.919742600375148)(201,0.919742600375148)(202,0.920714894293072)(202,0.920714894293072)(202,0.920714894293072)(202,0.920714894293072)(203,0.921734276022895)(203,0.921734276022895)(203,0.921734276022895)(204,0.922773707578567)(204,0.922773707578567)(204,0.922773707578567)(204,0.922773707578567)(204,0.922773707578567)(204,0.922773707578567)(204,0.922773707578567)(205,0.923802947705102)(205,0.923802947705102)(205,0.923802947705102)(205,0.923802947705102)(205,0.923802947705102)(206,0.925042052909485)(206,0.925042052909485)(206,0.925042052909485)(207,0.926079343120051)(207,0.926079343120051)(207,0.926079343120051)(207,0.926079343120051)(207,0.926079343120051)(208,0.927011950514294)(208,0.927011950514294)(208,0.927011950514294)(208,0.927011950514294)(209,0.92781400541199)(209,0.92781400541199)(210,0.928471822962556)(210,0.928471822962556)(210,0.928471822962556)(211,0.928820410876947)(212,0.929112811818984)(212,0.929112811818984)(212,0.929112811818984)(213,0.92949187239593)(213,0.92949187239593)(213,0.92949187239593)(214,0.929546441001894)(214,0.929546441001894)(214,0.929546441001894)(215,0.929527176029044)(215,0.929527176029044)(215,0.929527176029044)(215,0.929527176029044)(215,0.929527176029044)(215,0.929527176029044)(216,0.92944832291568)(216,0.92944832291568)(216,0.92944832291568)(216,0.92944832291568)(217,0.929509813484465)(217,0.929509813484465)(217,0.929509813484465)(217,0.929509813484465)(218,0.9295785421427)(218,0.9295785421427)(218,0.9295785421427)(218,0.9295785421427)(218,0.9295785421427)(218,0.9295785421427)(219,0.92970212811327)(219,0.92970212811327)(219,0.92970212811327)(219,0.92970212811327)(219,0.92970212811327)(220,0.92985431455564)(220,0.92985431455564)(220,0.92985431455564)(221,0.930011971476033)(221,0.930011971476033)(221,0.930011971476033)(221,0.930011971476033)(222,0.930160340260124)(222,0.930160340260124)(222,0.930160340260124)(222,0.930160340260124)(222,0.930160340260124)(222,0.930160340260124)(223,0.930290348583938)(223,0.930290348583938)(223,0.930290348583938)(223,0.930290348583938)(224,0.93038188757714)(224,0.93038188757714)(224,0.93038188757714)(224,0.93038188757714)(225,0.930414922619177)(225,0.930414922619177)(225,0.930414922619177)(225,0.930414922619177)(225,0.930414922619177)(226,0.930449090934625)(226,0.930449090934625)(226,0.930449090934625)(227,0.930198026344866)(227,0.930198026344866)(227,0.930198026344866)(227,0.930198026344866)(228,0.929931177815425)(228,0.929931177815425)(229,0.929569332908737)(229,0.929569332908737)(230,0.929402847843951)(230,0.929402847843951)(230,0.929402847843951)(230,0.929402847843951)(230,0.929402847843951)(231,0.929008488282826)(231,0.929008488282826)(232,0.928695025687735)(232,0.928695025687735)(233,0.928520025100122)(233,0.928520025100122)(233,0.928520025100122)(234,0.928508072970373)(234,0.928508072970373)(235,0.928630240518544)(235,0.928630240518544)(235,0.928630240518544)(235,0.928630240518544)(236,0.92886057886726)(236,0.92886057886726)(237,0.929172679425103)(237,0.929172679425103)(237,0.929172679425103)(237,0.929172679425103)(237,0.929172679425103)(237,0.929172679425103)(238,0.929514974525893)(238,0.929514974525893)(238,0.929514974525893)(239,0.93005362916783)(239,0.93005362916783)(240,0.930344242259481)(240,0.930344242259481)(241,0.930291037353047)(242,0.93038431481666)(242,0.93038431481666)(242,0.93038431481666)(242,0.93038431481666)(242,0.93038431481666)(243,0.930358774653482)(243,0.930358774653482)(243,0.930358774653482)(243,0.930358774653482)(243,0.930358774653482)(244,0.930206990854459)(244,0.930206990854459)(244,0.930206990854459)(245,0.929941508223868)(245,0.929941508223868)(245,0.929941508223868)(245,0.929941508223868)(245,0.929941508223868)(246,0.929589465225094)(246,0.929589465225094)(246,0.929589465225094)(246,0.929589465225094)(246,0.929589465225094)(246,0.929589465225094)(247,0.929102380641841)(247,0.929102380641841)(247,0.929102380641841)(247,0.929102380641841)(247,0.929102380641841)(247,0.929102380641841)(248,0.928759988876752)(248,0.928759988876752)(249,0.928283185667378)(249,0.928283185667378)(249,0.928283185667378)(250,0.928021261890375)(250,0.928021261890375)(250,0.928021261890375)(250,0.928021261890375)(251,0.927906209726626)(252,0.92791856615552)(252,0.92791856615552)(252,0.92791856615552)(252,0.92791856615552)(252,0.92791856615552)(252,0.92791856615552)(253,0.928026814200965)(253,0.928026814200965)(253,0.928026814200965)(254,0.928186468741811)(254,0.928186468741811)(254,0.928186468741811)(254,0.928186468741811)(254,0.928186468741811)(254,0.928186468741811)(255,0.928352101421591)(255,0.928352101421591)(255,0.928352101421591)(256,0.928495480485068)(256,0.928495480485068)(256,0.928495480485068)(256,0.928495480485068)(257,0.928569947878862)(257,0.928569947878862)(257,0.928569947878862)(258,0.928694281285353)(258,0.928694281285353)(259,0.928811803123317)(259,0.928811803123317)(259,0.928811803123317)(260,0.928943258940779)(260,0.928943258940779)(261,0.929089647917669)(261,0.929089647917669)(261,0.929089647917669)(261,0.929089647917669)(261,0.929089647917669)(262,0.929260004193965)(262,0.929260004193965)(262,0.929260004193965)(263,0.929488684664558)(263,0.929488684664558)(263,0.929488684664558)(264,0.929803635821059)(264,0.929803635821059)(264,0.929803635821059)(264,0.929803635821059)(264,0.929803635821059)(264,0.929803635821059)(265,0.930123819709993)(265,0.930123819709993)(265,0.930123819709993)(266,0.930604886640738)(266,0.930604886640738)(266,0.930604886640738)(266,0.930604886640738)(267,0.931216661016392)(267,0.931216661016392)(268,0.931947115824418)(268,0.931947115824418)(268,0.931947115824418)(269,0.93277113832037)(269,0.93277113832037)(269,0.93277113832037)(269,0.93277113832037)(269,0.93277113832037)(270,0.933643985600584)(270,0.933643985600584)(270,0.933643985600584)(270,0.933643985600584)(270,0.933643985600584)(271,0.934500969783454)(271,0.934500969783454)(271,0.934500969783454)(271,0.934500969783454)(271,0.934500969783454)(272,0.935406747966525)(272,0.935406747966525)(272,0.935406747966525)(272,0.935406747966525)(272,0.935406747966525)(272,0.935406747966525)(272,0.935406747966525)(272,0.935406747966525)(272,0.935406747966525)(273,0.936315626824253)(273,0.936315626824253)(273,0.936315626824253)(274,0.937216027068923)(274,0.937216027068923)(274,0.937216027068923)(274,0.937216027068923)(274,0.937216027068923)(274,0.937216027068923)(275,0.938109520908912)(275,0.938109520908912)(275,0.938109520908912)(275,0.938109520908912)(275,0.938109520908912)(276,0.939000924702357)(276,0.939000924702357)(276,0.939000924702357)(277,0.939907566467175)(277,0.939907566467175)(277,0.939907566467175)(277,0.939907566467175)(277,0.939907566467175)(277,0.939907566467175)(278,0.94085153725404)(278,0.94085153725404)(279,0.94182439561487)(279,0.94182439561487)(279,0.94182439561487)(279,0.94182439561487)(279,0.94182439561487)(280,0.942800614536203)(280,0.942800614536203)(280,0.942800614536203)(281,0.943755064910015)(281,0.943755064910015)(281,0.943755064910015)(281,0.943755064910015)(282,0.944680948022372)(282,0.944680948022372)(282,0.944680948022372)(282,0.944680948022372)(282,0.944680948022372)(283,0.945732332786568)(283,0.945732332786568)(283,0.945732332786568)(283,0.945732332786568)(284,0.946620326309386)(284,0.946620326309386)(284,0.946620326309386)(284,0.946620326309386)(284,0.946620326309386)(285,0.947457147644098)(285,0.947457147644098)(285,0.947457147644098)(286,0.948245038460214)(286,0.948245038460214)(286,0.948245038460214)(287,0.948979942159313)(287,0.948979942159313)(287,0.948979942159313)(288,0.949651139962147)(288,0.949651139962147)(288,0.949651139962147)(288,0.949651139962147)(288,0.949651139962147)(288,0.949651139962147)(288,0.949651139962147)(289,0.950249181559428)(289,0.950249181559428)(289,0.950249181559428)(290,0.950772831254163)(290,0.950772831254163)(290,0.950772831254163)(291,0.951219153096142)(291,0.951219153096142)(291,0.951219153096142)(292,0.951600937843238)(292,0.951600937843238)(292,0.951600937843238)(292,0.951600937843238)(292,0.951600937843238)(292,0.951600937843238)(293,0.95204738363197)(293,0.95204738363197)(293,0.95204738363197)(293,0.95204738363197)(293,0.95204738363197)(293,0.95204738363197)(294,0.952349094477709)(294,0.952349094477709)(294,0.952349094477709)(295,0.952607166074643)(296,0.952851023716632)(296,0.952851023716632)(296,0.952851023716632)(296,0.952851023716632)(297,0.953116798624263)(297,0.953116798624263)(297,0.953116798624263)(297,0.953116798624263)(299,0.953610214900238)(299,0.953610214900238)(299,0.953610214900238)(300,0.953808286307453)(300,0.953808286307453)(300,0.953808286307453)(300,0.953808286307453)(300,0.953808286307453)(301,0.953954646940386)(301,0.953954646940386)(301,0.953954646940386)(301,0.953954646940386)(301,0.953954646940386)(301,0.953954646940386)(302,0.954039305500288)(302,0.954039305500288)(302,0.954039305500288)(302,0.954039305500288)(302,0.954039305500288)(303,0.954141179186431)(303,0.954141179186431)(304,0.954111272781074)(304,0.954111272781074)(305,0.95404627702617)(305,0.95404627702617)(305,0.95404627702617)(305,0.95404627702617)(305,0.95404627702617)(306,0.953964418262955)(306,0.953964418262955)(306,0.953964418262955)(306,0.953964418262955)(306,0.953964418262955)(306,0.953964418262955)(307,0.953876019294291)(307,0.953876019294291)(307,0.953876019294291)(307,0.953876019294291)(307,0.953876019294291)(307,0.953876019294291)(308,0.953800066967897)(308,0.953800066967897)(308,0.953800066967897)(309,0.953693315139493)(309,0.953693315139493)(309,0.953693315139493)(309,0.953693315139493)(310,0.953699762899594)(310,0.953699762899594)(310,0.953699762899594)(310,0.953699762899594)(310,0.953699762899594)(311,0.953689452528729)(311,0.953689452528729)(312,0.953694082502727)(312,0.953694082502727)(312,0.953694082502727)(312,0.953694082502727)(312,0.953694082502727)(313,0.953940401919052)(313,0.953940401919052)(314,0.954102211019587)(314,0.954102211019587)(314,0.954102211019587)(315,0.95431256153559)(315,0.95431256153559)(316,0.954578483953118)(317,0.954905151194747)(317,0.954905151194747)(318,0.955474912391424)(318,0.955474912391424)(318,0.955474912391424)(318,0.955474912391424)(318,0.955474912391424)(319,0.955882699066343)(319,0.955882699066343)(319,0.955882699066343)(319,0.955882699066343)(320,0.956340670808475)(320,0.956340670808475)(320,0.956340670808475)(320,0.956340670808475)(320,0.956340670808475)(321,0.956830951547149)(321,0.956830951547149)(321,0.956830951547149)(321,0.956830951547149)(321,0.956830951547149)(321,0.956830951547149)(322,0.957326748095118)(323,0.957804591018893)(323,0.957804591018893)(323,0.957804591018893)(323,0.957804591018893)(324,0.958170996369892)(324,0.958170996369892)(324,0.958170996369892)(324,0.958170996369892)(324,0.958170996369892)(325,0.958671820438734)(325,0.958671820438734)(325,0.958671820438734)(325,0.958671820438734)(326,0.959046734273986)(326,0.959046734273986)(326,0.959046734273986)(327,0.95925469167213)(327,0.95925469167213)(327,0.95925469167213)(327,0.95925469167213)(327,0.95925469167213)(327,0.95925469167213)(327,0.95925469167213)(327,0.95925469167213)(328,0.959550026297845)(328,0.959550026297845)(328,0.959550026297845)(328,0.959550026297845)(329,0.959819853562007)(329,0.959819853562007)(329,0.959819853562007)(329,0.959819853562007)(329,0.959819853562007)(330,0.960066596899867)(330,0.960066596899867)(330,0.960066596899867)(330,0.960066596899867)(331,0.960280695714221)(331,0.960280695714221)(331,0.960280695714221)(332,0.960445064296118)(332,0.960445064296118)(332,0.960445064296118)(332,0.960445064296118)(333,0.960549823802932)(334,0.960600472708755)(334,0.960600472708755)(334,0.960600472708755)(334,0.960600472708755)(334,0.960600472708755)(335,0.960591998197153)(335,0.960591998197153)(335,0.960591998197153)(335,0.960591998197153)(336,0.960624112940936)(336,0.960624112940936)(336,0.960624112940936)(336,0.960624112940936)(337,0.960630515819017)(337,0.960630515819017)(337,0.960630515819017)(338,0.960614295262187)(339,0.960583842099168)(339,0.960583842099168)(339,0.960583842099168)(339,0.960583842099168)(340,0.960548323888744)(340,0.960548323888744)(341,0.960519475832252)(341,0.960519475832252)(341,0.960519475832252)(342,0.960503138489656)(342,0.960503138489656)(343,0.960505312923436)(343,0.960505312923436)(343,0.960505312923436)(344,0.960677460853193)(344,0.960677460853193)(344,0.960677460853193)(344,0.960677460853193)(345,0.960602796458119)(345,0.960602796458119)(345,0.960602796458119)(345,0.960602796458119)(345,0.960602796458119)(346,0.960717048853635)(346,0.960717048853635)(347,0.96091660257035)(348,0.961079941601479)(348,0.961079941601479)(349,0.961263305510577)(349,0.961263305510577)(349,0.961263305510577)(350,0.961456121689508)(350,0.961456121689508)(350,0.961456121689508)(351,0.961623349370706)(351,0.961623349370706)(351,0.961623349370706)(351,0.961623349370706)(351,0.961623349370706)(352,0.961825138042103)(352,0.961825138042103)(353,0.961971685707644)(354,0.962110907637694)(355,0.962245322253164)(355,0.962245322253164)(355,0.962245322253164)(355,0.962245322253164)(356,0.962376799369781)(356,0.962376799369781)(356,0.962376799369781)(356,0.962376799369781)(356,0.962376799369781)(357,0.962558786463509)(357,0.962558786463509)(357,0.962558786463509)(358,0.962679534100503)(358,0.962679534100503)(358,0.962679534100503)(358,0.962679534100503)(359,0.962850816522897)(359,0.962850816522897)(360,0.962956334682654)(360,0.962956334682654)(360,0.962956334682654)(360,0.962956334682654)(360,0.962956334682654)(360,0.962956334682654)(361,0.963051624242333)(361,0.963051624242333)(361,0.963051624242333)(362,0.963191653097882)(362,0.963191653097882)(363,0.963276368654328)(363,0.963276368654328)(363,0.963276368654328)(364,0.963370881965764)(364,0.963370881965764)(364,0.963370881965764)(364,0.963370881965764)(364,0.963370881965764)(364,0.963370881965764)(365,0.963474275920867)(365,0.963474275920867)(365,0.963474275920867)(365,0.963474275920867)(366,0.963581221626612)(366,0.963581221626612)(366,0.963581221626612)(367,0.963686844647344)(367,0.963686844647344)(367,0.963686844647344)(368,0.963787783460679)(369,0.96388204896176)(369,0.96388204896176)(369,0.96388204896176)(370,0.963968530781009)(370,0.963968530781009)(371,0.964046668621471)(371,0.964046668621471)(372,0.964116232690347)(373,0.964177144555867)(373,0.964177144555867)(373,0.964177144555867)(373,0.964177144555867)(375,0.96427325686489)(375,0.96427325686489)(376,0.964308822291944)(376,0.964308822291944)(376,0.964308822291944)(377,0.964336328029162)(377,0.964336328029162)(378,0.964355979970708)(380,0.964372604979973)(382,0.964360404639197)(382,0.964360404639197)(383,0.964343933034723)(383,0.964343933034723)(383,0.964343933034723)(383,0.964343933034723)(387,0.964211846350302)(388,0.964162747146183)(389,0.964107359901669)(391,0.963977856013692)(392,0.963903728259635)(394,0.963736559407801)(394,0.963736559407801)(395,0.963643365681177)(396,0.963543568472361)(400,0.963072214646116)(400,0.963072214646116) 
};
\addlegendentry{\acl (max. ambiguity)};

\addplot [
color=red,
solid,
line width=1.0pt,
]
coordinates{
 (14,0.542515418618291)(14,0.542515418618291)(14,0.542515418618291)(15,0.586498920683904)(15,0.586498920683904)(15,0.586498920683904)(15,0.586498920683904)(15,0.586498920683904)(16,0.632573873012377)(16,0.632573873012377)(16,0.632573873012377)(16,0.632573873012377)(16,0.632573873012377)(17,0.673553229755534)(17,0.673553229755534)(17,0.673553229755534)(17,0.673553229755534)(17,0.673553229755534)(18,0.702305241636049)(18,0.702305241636049)(18,0.702305241636049)(19,0.722525821669099)(19,0.722525821669099)(19,0.722525821669099)(19,0.722525821669099)(19,0.722525821669099)(19,0.722525821669099)(19,0.722525821669099)(19,0.722525821669099)(19,0.722525821669099)(19,0.722525821669099)(19,0.722525821669099)(20,0.736345643654755)(20,0.736345643654755)(20,0.736345643654755)(20,0.736345643654755)(20,0.736345643654755)(21,0.745050463197471)(21,0.745050463197471)(21,0.745050463197471)(21,0.745050463197471)(21,0.745050463197471)(21,0.745050463197471)(21,0.745050463197471)(21,0.745050463197471)(22,0.748324428659585)(22,0.748324428659585)(22,0.748324428659585)(22,0.748324428659585)(22,0.748324428659585)(22,0.748324428659585)(22,0.748324428659585)(22,0.748324428659585)(22,0.748324428659585)(23,0.755015161747214)(23,0.755015161747214)(23,0.755015161747214)(23,0.755015161747214)(23,0.755015161747214)(23,0.755015161747214)(23,0.755015161747214)(23,0.755015161747214)(23,0.755015161747214)(23,0.755015161747214)(23,0.755015161747214)(24,0.761143927291288)(24,0.761143927291288)(24,0.761143927291288)(24,0.761143927291288)(24,0.761143927291288)(24,0.761143927291288)(24,0.761143927291288)(24,0.761143927291288)(24,0.761143927291288)(24,0.761143927291288)(24,0.761143927291288)(24,0.761143927291288)(24,0.761143927291288)(24,0.761143927291288)(24,0.761143927291288)(24,0.761143927291288)(24,0.761143927291288)(24,0.761143927291288)(24,0.761143927291288)(25,0.7669796733037)(25,0.7669796733037)(25,0.7669796733037)(25,0.7669796733037)(25,0.7669796733037)(25,0.7669796733037)(25,0.7669796733037)(25,0.7669796733037)(25,0.7669796733037)(25,0.7669796733037)(25,0.7669796733037)(25,0.7669796733037)(25,0.7669796733037)(26,0.773840077312004)(26,0.773840077312004)(26,0.773840077312004)(26,0.773840077312004)(26,0.773840077312004)(26,0.773840077312004)(26,0.773840077312004)(26,0.773840077312004)(26,0.773840077312004)(26,0.773840077312004)(26,0.773840077312004)(26,0.773840077312004)(26,0.773840077312004)(26,0.773840077312004)(26,0.773840077312004)(26,0.773840077312004)(26,0.773840077312004)(26,0.773840077312004)(26,0.773840077312004)(26,0.773840077312004)(26,0.773840077312004)(26,0.773840077312004)(26,0.773840077312004)(27,0.777946326875893)(27,0.777946326875893)(27,0.777946326875893)(27,0.777946326875893)(27,0.777946326875893)(27,0.777946326875893)(27,0.777946326875893)(27,0.777946326875893)(27,0.777946326875893)(27,0.777946326875893)(27,0.777946326875893)(27,0.777946326875893)(27,0.777946326875893)(27,0.777946326875893)(28,0.781037471517282)(28,0.781037471517282)(28,0.781037471517282)(28,0.781037471517282)(28,0.781037471517282)(28,0.781037471517282)(28,0.781037471517282)(28,0.781037471517282)(28,0.781037471517282)(28,0.781037471517282)(28,0.781037471517282)(28,0.781037471517282)(28,0.781037471517282)(28,0.781037471517282)(28,0.781037471517282)(28,0.781037471517282)(28,0.781037471517282)(28,0.781037471517282)(28,0.781037471517282)(28,0.781037471517282)(28,0.781037471517282)(28,0.781037471517282)(28,0.781037471517282)(28,0.781037471517282)(28,0.781037471517282)(29,0.782277658096885)(29,0.782277658096885)(29,0.782277658096885)(29,0.782277658096885)(29,0.782277658096885)(29,0.782277658096885)(29,0.782277658096885)(29,0.782277658096885)(29,0.782277658096885)(29,0.782277658096885)(29,0.782277658096885)(29,0.782277658096885)(29,0.782277658096885)(29,0.782277658096885)(29,0.782277658096885)(29,0.782277658096885)(29,0.782277658096885)(29,0.782277658096885)(29,0.782277658096885)(29,0.782277658096885)(29,0.782277658096885)(30,0.784713488497252)(30,0.784713488497252)(30,0.784713488497252)(30,0.784713488497252)(30,0.784713488497252)(30,0.784713488497252)(30,0.784713488497252)(30,0.784713488497252)(30,0.784713488497252)(30,0.784713488497252)(30,0.784713488497252)(30,0.784713488497252)(30,0.784713488497252)(30,0.784713488497252)(30,0.784713488497252)(30,0.784713488497252)(30,0.784713488497252)(30,0.784713488497252)(30,0.784713488497252)(30,0.784713488497252)(30,0.784713488497252)(30,0.784713488497252)(30,0.784713488497252)(31,0.79045536962032)(31,0.79045536962032)(31,0.79045536962032)(31,0.79045536962032)(31,0.79045536962032)(31,0.79045536962032)(31,0.79045536962032)(31,0.79045536962032)(31,0.79045536962032)(31,0.79045536962032)(31,0.79045536962032)(31,0.79045536962032)(31,0.79045536962032)(31,0.79045536962032)(31,0.79045536962032)(31,0.79045536962032)(31,0.79045536962032)(31,0.79045536962032)(31,0.79045536962032)(31,0.79045536962032)(31,0.79045536962032)(31,0.79045536962032)(32,0.795163095465287)(32,0.795163095465287)(32,0.795163095465287)(32,0.795163095465287)(32,0.795163095465287)(32,0.795163095465287)(32,0.795163095465287)(32,0.795163095465287)(32,0.795163095465287)(32,0.795163095465287)(32,0.795163095465287)(32,0.795163095465287)(32,0.795163095465287)(33,0.799714114126557)(33,0.799714114126557)(33,0.799714114126557)(33,0.799714114126557)(33,0.799714114126557)(33,0.799714114126557)(33,0.799714114126557)(33,0.799714114126557)(33,0.799714114126557)(33,0.799714114126557)(33,0.799714114126557)(33,0.799714114126557)(33,0.799714114126557)(33,0.799714114126557)(33,0.799714114126557)(33,0.799714114126557)(33,0.799714114126557)(33,0.799714114126557)(33,0.799714114126557)(33,0.799714114126557)(33,0.799714114126557)(33,0.799714114126557)(33,0.799714114126557)(33,0.799714114126557)(33,0.799714114126557)(33,0.799714114126557)(33,0.799714114126557)(33,0.799714114126557)(34,0.802312456780027)(34,0.802312456780027)(34,0.802312456780027)(34,0.802312456780027)(34,0.802312456780027)(34,0.802312456780027)(34,0.802312456780027)(34,0.802312456780027)(34,0.802312456780027)(35,0.803093466776013)(35,0.803093466776013)(35,0.803093466776013)(35,0.803093466776013)(35,0.803093466776013)(35,0.803093466776013)(35,0.803093466776013)(35,0.803093466776013)(35,0.803093466776013)(35,0.803093466776013)(35,0.803093466776013)(35,0.803093466776013)(35,0.803093466776013)(35,0.803093466776013)(35,0.803093466776013)(35,0.803093466776013)(35,0.803093466776013)(35,0.803093466776013)(35,0.803093466776013)(35,0.803093466776013)(35,0.803093466776013)(36,0.805542499016565)(36,0.805542499016565)(36,0.805542499016565)(36,0.805542499016565)(36,0.805542499016565)(36,0.805542499016565)(36,0.805542499016565)(36,0.805542499016565)(36,0.805542499016565)(36,0.805542499016565)(36,0.805542499016565)(36,0.805542499016565)(36,0.805542499016565)(36,0.805542499016565)(36,0.805542499016565)(36,0.805542499016565)(37,0.807975974573517)(37,0.807975974573517)(37,0.807975974573517)(37,0.807975974573517)(37,0.807975974573517)(37,0.807975974573517)(37,0.807975974573517)(37,0.807975974573517)(37,0.807975974573517)(37,0.807975974573517)(38,0.809530869984393)(38,0.809530869984393)(38,0.809530869984393)(38,0.809530869984393)(38,0.809530869984393)(38,0.809530869984393)(38,0.809530869984393)(38,0.809530869984393)(38,0.809530869984393)(38,0.809530869984393)(38,0.809530869984393)(38,0.809530869984393)(39,0.812961395392631)(39,0.812961395392631)(39,0.812961395392631)(39,0.812961395392631)(39,0.812961395392631)(39,0.812961395392631)(39,0.812961395392631)(39,0.812961395392631)(39,0.812961395392631)(39,0.812961395392631)(39,0.812961395392631)(39,0.812961395392631)(39,0.812961395392631)(39,0.812961395392631)(39,0.812961395392631)(39,0.812961395392631)(39,0.812961395392631)(39,0.812961395392631)(39,0.812961395392631)(39,0.812961395392631)(39,0.812961395392631)(40,0.815784941380892)(40,0.815784941380892)(40,0.815784941380892)(40,0.815784941380892)(40,0.815784941380892)(40,0.815784941380892)(40,0.815784941380892)(40,0.815784941380892)(40,0.815784941380892)(40,0.815784941380892)(40,0.815784941380892)(40,0.815784941380892)(40,0.815784941380892)(40,0.815784941380892)(41,0.816681474761662)(41,0.816681474761662)(41,0.816681474761662)(41,0.816681474761662)(41,0.816681474761662)(41,0.816681474761662)(41,0.816681474761662)(41,0.816681474761662)(41,0.816681474761662)(41,0.816681474761662)(41,0.816681474761662)(41,0.816681474761662)(42,0.816837787682589)(42,0.816837787682589)(42,0.816837787682589)(42,0.816837787682589)(42,0.816837787682589)(42,0.816837787682589)(42,0.816837787682589)(42,0.816837787682589)(42,0.816837787682589)(42,0.816837787682589)(42,0.816837787682589)(42,0.816837787682589)(42,0.816837787682589)(42,0.816837787682589)(42,0.816837787682589)(42,0.816837787682589)(42,0.816837787682589)(43,0.81940462546135)(43,0.81940462546135)(43,0.81940462546135)(43,0.81940462546135)(43,0.81940462546135)(43,0.81940462546135)(43,0.81940462546135)(43,0.81940462546135)(43,0.81940462546135)(43,0.81940462546135)(43,0.81940462546135)(43,0.81940462546135)(43,0.81940462546135)(43,0.81940462546135)(43,0.81940462546135)(43,0.81940462546135)(44,0.82304064905151)(44,0.82304064905151)(44,0.82304064905151)(44,0.82304064905151)(44,0.82304064905151)(44,0.82304064905151)(44,0.82304064905151)(44,0.82304064905151)(44,0.82304064905151)(44,0.82304064905151)(44,0.82304064905151)(45,0.826396723460098)(45,0.826396723460098)(45,0.826396723460098)(45,0.826396723460098)(45,0.826396723460098)(45,0.826396723460098)(45,0.826396723460098)(45,0.826396723460098)(46,0.829959039222099)(46,0.829959039222099)(46,0.829959039222099)(46,0.829959039222099)(46,0.829959039222099)(46,0.829959039222099)(46,0.829959039222099)(46,0.829959039222099)(46,0.829959039222099)(46,0.829959039222099)(46,0.829959039222099)(47,0.833140006190102)(47,0.833140006190102)(47,0.833140006190102)(47,0.833140006190102)(47,0.833140006190102)(47,0.833140006190102)(47,0.833140006190102)(47,0.833140006190102)(47,0.833140006190102)(47,0.833140006190102)(48,0.836199852913122)(48,0.836199852913122)(48,0.836199852913122)(48,0.836199852913122)(48,0.836199852913122)(48,0.836199852913122)(48,0.836199852913122)(48,0.836199852913122)(48,0.836199852913122)(48,0.836199852913122)(48,0.836199852913122)(48,0.836199852913122)(48,0.836199852913122)(48,0.836199852913122)(48,0.836199852913122)(48,0.836199852913122)(49,0.838792590566592)(49,0.838792590566592)(49,0.838792590566592)(49,0.838792590566592)(49,0.838792590566592)(49,0.838792590566592)(49,0.838792590566592)(49,0.838792590566592)(49,0.838792590566592)(49,0.838792590566592)(50,0.843492068955311)(50,0.843492068955311)(50,0.843492068955311)(50,0.843492068955311)(50,0.843492068955311)(50,0.843492068955311)(50,0.843492068955311)(51,0.847568756299401)(51,0.847568756299401)(51,0.847568756299401)(51,0.847568756299401)(51,0.847568756299401)(51,0.847568756299401)(52,0.851386103595482)(52,0.851386103595482)(52,0.851386103595482)(52,0.851386103595482)(52,0.851386103595482)(52,0.851386103595482)(52,0.851386103595482)(52,0.851386103595482)(52,0.851386103595482)(52,0.851386103595482)(52,0.851386103595482)(52,0.851386103595482)(53,0.853748894314695)(53,0.853748894314695)(53,0.853748894314695)(53,0.853748894314695)(53,0.853748894314695)(53,0.853748894314695)(53,0.853748894314695)(53,0.853748894314695)(53,0.853748894314695)(53,0.853748894314695)(53,0.853748894314695)(53,0.853748894314695)(53,0.853748894314695)(54,0.854676977267225)(54,0.854676977267225)(54,0.854676977267225)(54,0.854676977267225)(54,0.854676977267225)(54,0.854676977267225)(54,0.854676977267225)(54,0.854676977267225)(55,0.855831585625687)(55,0.855831585625687)(55,0.855831585625687)(55,0.855831585625687)(55,0.855831585625687)(55,0.855831585625687)(55,0.855831585625687)(55,0.855831585625687)(55,0.855831585625687)(55,0.855831585625687)(55,0.855831585625687)(55,0.855831585625687)(56,0.85607702973781)(56,0.85607702973781)(56,0.85607702973781)(56,0.85607702973781)(56,0.85607702973781)(56,0.85607702973781)(56,0.85607702973781)(56,0.85607702973781)(56,0.85607702973781)(56,0.85607702973781)(56,0.85607702973781)(56,0.85607702973781)(57,0.85636569904245)(57,0.85636569904245)(57,0.85636569904245)(57,0.85636569904245)(57,0.85636569904245)(57,0.85636569904245)(57,0.85636569904245)(57,0.85636569904245)(58,0.857866888681775)(58,0.857866888681775)(58,0.857866888681775)(58,0.857866888681775)(58,0.857866888681775)(59,0.859045069738945)(59,0.859045069738945)(59,0.859045069738945)(59,0.859045069738945)(59,0.859045069738945)(59,0.859045069738945)(59,0.859045069738945)(59,0.859045069738945)(59,0.859045069738945)(59,0.859045069738945)(59,0.859045069738945)(60,0.860799340128392)(60,0.860799340128392)(60,0.860799340128392)(60,0.860799340128392)(60,0.860799340128392)(60,0.860799340128392)(60,0.860799340128392)(60,0.860799340128392)(60,0.860799340128392)(61,0.863272761937464)(61,0.863272761937464)(61,0.863272761937464)(61,0.863272761937464)(61,0.863272761937464)(61,0.863272761937464)(61,0.863272761937464)(61,0.863272761937464)(61,0.863272761937464)(62,0.866257284974296)(62,0.866257284974296)(62,0.866257284974296)(63,0.869009100873387)(63,0.869009100873387)(63,0.869009100873387)(63,0.869009100873387)(63,0.869009100873387)(63,0.869009100873387)(64,0.87190043959688)(64,0.87190043959688)(64,0.87190043959688)(64,0.87190043959688)(64,0.87190043959688)(64,0.87190043959688)(64,0.87190043959688)(64,0.87190043959688)(64,0.87190043959688)(65,0.874201805619806)(65,0.874201805619806)(65,0.874201805619806)(65,0.874201805619806)(65,0.874201805619806)(65,0.874201805619806)(65,0.874201805619806)(65,0.874201805619806)(65,0.874201805619806)(66,0.876304106677518)(66,0.876304106677518)(66,0.876304106677518)(66,0.876304106677518)(66,0.876304106677518)(66,0.876304106677518)(66,0.876304106677518)(67,0.878048146618433)(67,0.878048146618433)(67,0.878048146618433)(67,0.878048146618433)(67,0.878048146618433)(68,0.880021238356322)(68,0.880021238356322)(68,0.880021238356322)(68,0.880021238356322)(69,0.881804592886741)(69,0.881804592886741)(69,0.881804592886741)(69,0.881804592886741)(70,0.883363637539335)(70,0.883363637539335)(70,0.883363637539335)(70,0.883363637539335)(70,0.883363637539335)(70,0.883363637539335)(70,0.883363637539335)(70,0.883363637539335)(70,0.883363637539335)(70,0.883363637539335)(70,0.883363637539335)(71,0.884283572085161)(71,0.884283572085161)(71,0.884283572085161)(71,0.884283572085161)(71,0.884283572085161)(72,0.88518417109657)(72,0.88518417109657)(72,0.88518417109657)(72,0.88518417109657)(73,0.88601757222616)(73,0.88601757222616)(73,0.88601757222616)(73,0.88601757222616)(73,0.88601757222616)(73,0.88601757222616)(73,0.88601757222616)(74,0.887011334295428)(74,0.887011334295428)(74,0.887011334295428)(74,0.887011334295428)(74,0.887011334295428)(74,0.887011334295428)(75,0.888153907703023)(75,0.888153907703023)(75,0.888153907703023)(76,0.88963483419148)(76,0.88963483419148)(76,0.88963483419148)(76,0.88963483419148)(76,0.88963483419148)(76,0.88963483419148)(77,0.890820325880799)(77,0.890820325880799)(77,0.890820325880799)(78,0.891891918254197)(78,0.891891918254197)(78,0.891891918254197)(79,0.892841039329002)(79,0.892841039329002)(79,0.892841039329002)(79,0.892841039329002)(79,0.892841039329002)(80,0.89374055584136)(80,0.89374055584136)(80,0.89374055584136)(80,0.89374055584136)(80,0.89374055584136)(80,0.89374055584136)(80,0.89374055584136)(80,0.89374055584136)(80,0.89374055584136)(81,0.894421340873108)(81,0.894421340873108)(81,0.894421340873108)(82,0.895181111596904)(82,0.895181111596904)(82,0.895181111596904)(82,0.895181111596904)(82,0.895181111596904)(82,0.895181111596904)(82,0.895181111596904)(83,0.895866877801811)(83,0.895866877801811)(83,0.895866877801811)(83,0.895866877801811)(83,0.895866877801811)(85,0.897532305706511)(85,0.897532305706511)(85,0.897532305706511)(86,0.898367838904173)(86,0.898367838904173)(86,0.898367838904173)(86,0.898367838904173)(86,0.898367838904173)(88,0.899935285488815)(88,0.899935285488815)(88,0.899935285488815)(88,0.899935285488815)(89,0.900523282796945)(89,0.900523282796945)(89,0.900523282796945)(89,0.900523282796945)(89,0.900523282796945)(89,0.900523282796945)(89,0.900523282796945)(89,0.900523282796945)(89,0.900523282796945)(90,0.900862348531719)(90,0.900862348531719)(90,0.900862348531719)(90,0.900862348531719)(90,0.900862348531719)(90,0.900862348531719)(90,0.900862348531719)(91,0.900961033172851)(91,0.900961033172851)(92,0.900843413693304)(92,0.900843413693304)(92,0.900843413693304)(92,0.900843413693304)(93,0.900573395558544)(93,0.900573395558544)(93,0.900573395558544)(93,0.900573395558544)(94,0.900145994100129)(94,0.900145994100129)(94,0.900145994100129)(94,0.900145994100129)(95,0.899892016802434)(95,0.899892016802434)(95,0.899892016802434)(95,0.899892016802434)(95,0.899892016802434)(96,0.89973363840898)(96,0.89973363840898)(96,0.89973363840898)(96,0.89973363840898)(97,0.899915483414613)(97,0.899915483414613)(97,0.899915483414613)(97,0.899915483414613)(98,0.900073340066791)(98,0.900073340066791)(98,0.900073340066791)(98,0.900073340066791)(98,0.900073340066791)(99,0.900376891166757)(99,0.900376891166757)(99,0.900376891166757)(99,0.900376891166757)(100,0.900808835583895)(100,0.900808835583895)(101,0.901315847913596)(101,0.901315847913596)(101,0.901315847913596)(101,0.901315847913596)(101,0.901315847913596)(102,0.901843026153423)(102,0.901843026153423)(102,0.901843026153423)(102,0.901843026153423)(103,0.902384671542846)(103,0.902384671542846)(103,0.902384671542846)(103,0.902384671542846)(103,0.902384671542846)(103,0.902384671542846)(104,0.903114779025495)(104,0.903114779025495)(104,0.903114779025495)(105,0.90367475264961)(105,0.90367475264961)(105,0.90367475264961)(105,0.90367475264961)(105,0.90367475264961)(106,0.904260656651088)(106,0.904260656651088)(106,0.904260656651088)(107,0.904868049695029)(107,0.904868049695029)(107,0.904868049695029)(107,0.904868049695029)(107,0.904868049695029)(107,0.904868049695029)(108,0.905690909646733)(108,0.905690909646733)(108,0.905690909646733)(109,0.906636869165414)(109,0.906636869165414)(109,0.906636869165414)(109,0.906636869165414)(109,0.906636869165414)(110,0.907644474340705)(110,0.907644474340705)(110,0.907644474340705)(111,0.908794087980939)(111,0.908794087980939)(111,0.908794087980939)(111,0.908794087980939)(112,0.90984128708822)(112,0.90984128708822)(112,0.90984128708822)(113,0.911038384263876)(113,0.911038384263876)(113,0.911038384263876)(113,0.911038384263876)(114,0.911992791600079)(114,0.911992791600079)(114,0.911992791600079)(115,0.913345741413443)(115,0.913345741413443)(115,0.913345741413443)(117,0.915300001134197)(117,0.915300001134197)(117,0.915300001134197)(118,0.916263643349794)(118,0.916263643349794)(118,0.916263643349794)(119,0.91711451599576)(119,0.91711451599576)(119,0.91711451599576)(120,0.917880034454444)(120,0.917880034454444)(120,0.917880034454444)(120,0.917880034454444)(121,0.918477156713087)(121,0.918477156713087)(121,0.918477156713087)(121,0.918477156713087)(121,0.918477156713087)(121,0.918477156713087)(121,0.918477156713087)(122,0.919023790834356)(122,0.919023790834356)(122,0.919023790834356)(122,0.919023790834356)(123,0.919571761833215)(123,0.919571761833215)(124,0.92003191250103)(124,0.92003191250103)(124,0.92003191250103)(125,0.920483238370908)(125,0.920483238370908)(126,0.920936221750095)(126,0.920936221750095)(126,0.920936221750095)(127,0.921361164881013)(127,0.921361164881013)(127,0.921361164881013)(127,0.921361164881013)(127,0.921361164881013)(128,0.921774596707742)(128,0.921774596707742)(128,0.921774596707742)(128,0.921774596707742)(128,0.921774596707742)(129,0.922167509864396)(129,0.922167509864396)(130,0.92235140131788)(130,0.92235140131788)(130,0.92235140131788)(130,0.92235140131788)(131,0.922602890143046)(132,0.922779252983132)(133,0.922871455755356)(133,0.922871455755356)(133,0.922871455755356)(133,0.922871455755356)(133,0.922871455755356)(134,0.922884965039863)(134,0.922884965039863)(134,0.922884965039863)(135,0.922849069541184)(135,0.922849069541184)(136,0.922800943352596)(136,0.922800943352596)(137,0.922851250310829)(137,0.922851250310829)(137,0.922851250310829)(138,0.922913514158111)(138,0.922913514158111)(138,0.922913514158111)(138,0.922913514158111)(138,0.922913514158111)(139,0.923017409412208)(139,0.923017409412208)(139,0.923017409412208)(139,0.923017409412208)(140,0.92329195743944)(141,0.923474051056554)(141,0.923474051056554)(142,0.923687185028323)(142,0.923687185028323)(143,0.92405466958043)(143,0.92405466958043)(143,0.92405466958043)(144,0.924335545632095)(144,0.924335545632095)(145,0.924637357922874)(145,0.924637357922874)(145,0.924637357922874)(145,0.924637357922874)(145,0.924637357922874)(145,0.924637357922874)(146,0.924954626876778)(146,0.924954626876778)(146,0.924954626876778)(147,0.925288880590409)(147,0.925288880590409)(147,0.925288880590409)(147,0.925288880590409)(148,0.925646312146655)(148,0.925646312146655)(149,0.926036187361479)(149,0.926036187361479)(149,0.926036187361479)(150,0.926454891939577)(151,0.926930050563123)(151,0.926930050563123)(151,0.926930050563123)(152,0.927431311402722)(152,0.927431311402722)(153,0.927938178046104)(153,0.927938178046104)(153,0.927938178046104)(153,0.927938178046104)(154,0.928445962151544)(154,0.928445962151544)(154,0.928445962151544)(154,0.928445962151544)(156,0.929445787551455)(156,0.929445787551455)(157,0.929931066450926)(157,0.929931066450926)(157,0.929931066450926)(158,0.930296794924889)(158,0.930296794924889)(158,0.930296794924889)(159,0.930760011124213)(159,0.930760011124213)(159,0.930760011124213)(159,0.930760011124213)(159,0.930760011124213)(160,0.931221988319267)(160,0.931221988319267)(160,0.931221988319267)(160,0.931221988319267)(161,0.931679695673682)(161,0.931679695673682)(161,0.931679695673682)(161,0.931679695673682)(162,0.932122771568469)(162,0.932122771568469)(163,0.932542739446245)(163,0.932542739446245)(163,0.932542739446245)(164,0.93294344570429)(165,0.933293142441574)(166,0.933620997106694)(166,0.933620997106694)(167,0.933920200602375)(168,0.934195035867833)(169,0.934444170185229)(169,0.934444170185229)(170,0.934664073803084)(170,0.934664073803084)(171,0.934863366533817)(171,0.934863366533817)(171,0.934863366533817)(172,0.935044367929149)(173,0.935230790912972)(173,0.935230790912972)(173,0.935230790912972)(173,0.935230790912972)(174,0.935394805650467)(174,0.935394805650467)(175,0.935507207351847)(176,0.935673312913624)(176,0.935673312913624)(176,0.935673312913624)(177,0.935858197914366)(177,0.935858197914366)(178,0.936064076353367)(178,0.936064076353367)(178,0.936064076353367)(179,0.936286150407234)(179,0.936286150407234)(179,0.936286150407234)(179,0.936286150407234)(179,0.936286150407234)(179,0.936286150407234)(179,0.936286150407234)(179,0.936286150407234)(180,0.936491703815972)(181,0.9367192517099)(182,0.936955199275342)(182,0.936955199275342)(182,0.936955199275342)(183,0.937199045288051)(184,0.937448001857339)(185,0.937699017105997)(185,0.937699017105997)(185,0.937699017105997)(185,0.937699017105997)(185,0.937699017105997)(186,0.937948278469554)(186,0.937948278469554)(187,0.938194094032295)(187,0.938194094032295)(188,0.938458478755028)(188,0.938458478755028)(190,0.938918309779193)(190,0.938918309779193)(190,0.938918309779193)(191,0.939173185209341)(191,0.939173185209341)(192,0.93950611115225)(192,0.93950611115225)(192,0.93950611115225)(193,0.939804383905851)(193,0.939804383905851)(193,0.939804383905851)(194,0.940122485883313)(194,0.940122485883313)(194,0.940122485883313)(194,0.940122485883313)(194,0.940122485883313)(194,0.940122485883313)(195,0.940457393053871)(195,0.940457393053871)(195,0.940457393053871)(195,0.940457393053871)(195,0.940457393053871)(197,0.941237510297467)(197,0.941237510297467)(197,0.941237510297467)(198,0.941621566340113)(199,0.941956767966951)(199,0.941956767966951)(200,0.942352270582027)(201,0.942710293461849)(201,0.942710293461849)(202,0.943111862012516)(202,0.943111862012516)(202,0.943111862012516)(203,0.943501765548652)(203,0.943501765548652)(204,0.94385574981318)(205,0.944189484587259)(205,0.944189484587259)(206,0.944501586491696)(207,0.944792039851983)(207,0.944792039851983)(207,0.944792039851983)(208,0.945064603485748)(208,0.945064603485748)(208,0.945064603485748)(208,0.945064603485748)(208,0.945064603485748)(209,0.945325826691094)(209,0.945325826691094)(209,0.945325826691094)(210,0.945663600369882)(210,0.945663600369882)(210,0.945663600369882)(210,0.945663600369882)(211,0.945907035836556)(211,0.945907035836556)(212,0.946117219356616)(212,0.946117219356616)(214,0.946408870534283)(214,0.946408870534283)(214,0.946408870534283)(214,0.946408870534283)(214,0.946408870534283)(215,0.946480302103764)(215,0.946480302103764)(217,0.946455883719292)(217,0.946455883719292)(217,0.946455883719292)(217,0.946455883719292)(218,0.946486217146843)(218,0.946486217146843)(219,0.946463686841728)(220,0.946449017784095)(220,0.946449017784095)(220,0.946449017784095)(221,0.946444997962012)(221,0.946444997962012)(222,0.946426387856845)(223,0.946425589780603)(223,0.946425589780603)(223,0.946425589780603)(223,0.946425589780603)(224,0.946441766665924)(224,0.946441766665924)(224,0.946441766665924)(225,0.946477693861537)(225,0.946477693861537)(226,0.946530244929179)(227,0.946592098006434)(229,0.946780234949612)(229,0.946780234949612)(229,0.946780234949612)(230,0.946821030608821)(230,0.946821030608821)(230,0.946821030608821)(230,0.946821030608821)(230,0.946821030608821)(230,0.946821030608821)(231,0.946855414748459)(231,0.946855414748459)(232,0.946846362282402)(232,0.946846362282402)(232,0.946846362282402)(233,0.946891792994846)(233,0.946891792994846)(233,0.946891792994846)(234,0.946921751880098)(234,0.946921751880098)(234,0.946921751880098)(234,0.946921751880098)(234,0.946921751880098)(235,0.946995830874894)(235,0.946995830874894)(235,0.946995830874894)(235,0.946995830874894)(236,0.947048642381701)(237,0.947140167219978)(240,0.947344722247993)(240,0.947344722247993)(240,0.947344722247993)(241,0.947413629021431)(242,0.947486058467608)(242,0.947486058467608)(242,0.947486058467608)(242,0.947486058467608)(242,0.947486058467608)(243,0.947567797628837)(243,0.947567797628837)(243,0.947567797628837)(244,0.947663438408987)(244,0.947663438408987)(244,0.947663438408987)(244,0.947663438408987)(245,0.947779036169748)(245,0.947779036169748)(245,0.947779036169748)(246,0.94791212852214)(247,0.948037150945855)(247,0.948037150945855)(247,0.948037150945855)(247,0.948037150945855)(247,0.948037150945855)(247,0.948037150945855)(247,0.948037150945855)(248,0.948174569098756)(248,0.948174569098756)(248,0.948174569098756)(248,0.948174569098756)(249,0.948317812837224)(250,0.948394783785757)(250,0.948394783785757)(252,0.948619512640765)(252,0.948619512640765)(252,0.948619512640765)(253,0.94869712883168)(253,0.94869712883168)(253,0.94869712883168)(254,0.948751544555489)(254,0.948751544555489)(254,0.948751544555489)(254,0.948751544555489)(255,0.948782962938762)(255,0.948782962938762)(255,0.948782962938762)(255,0.948782962938762)(256,0.948791860789148)(256,0.948791860789148)(257,0.948792389737797)(257,0.948792389737797)(257,0.948792389737797)(258,0.948784323672126)(258,0.948784323672126)(258,0.948784323672126)(259,0.948759630708294)(259,0.948759630708294)(260,0.948773533134016)(260,0.948773533134016)(260,0.948773533134016)(260,0.948773533134016)(261,0.948659103188167)(261,0.948659103188167)(263,0.9485540793173)(264,0.94845659667918)(264,0.94845659667918)(264,0.94845659667918)(264,0.94845659667918)(264,0.94845659667918)(264,0.94845659667918)(264,0.94845659667918)(265,0.948437672849643)(265,0.948437672849643)(266,0.948499604589459)(266,0.948499604589459)(266,0.948499604589459)(266,0.948499604589459)(266,0.948499604589459)(267,0.948520502660558)(268,0.948634496465279)(269,0.948690857028674)(269,0.948690857028674)(269,0.948690857028674)(269,0.948690857028674)(270,0.948756416789428)(270,0.948756416789428)(270,0.948756416789428)(271,0.948817713019191)(271,0.948817713019191)(271,0.948817713019191)(272,0.948925108325964)(272,0.948925108325964)(272,0.948925108325964)(273,0.949039411828724)(273,0.949039411828724)(274,0.949165188262858)(274,0.949165188262858)(274,0.949165188262858)(275,0.949312476873918)(275,0.949312476873918)(275,0.949312476873918)(275,0.949312476873918)(275,0.949312476873918)(276,0.949492630039149)(276,0.949492630039149)(277,0.949841591635355)(277,0.949841591635355)(277,0.949841591635355)(277,0.949841591635355)(277,0.949841591635355)(278,0.950085309679968)(278,0.950085309679968)(278,0.950085309679968)(278,0.950085309679968)(278,0.950085309679968)(279,0.950222737382489)(279,0.950222737382489)(280,0.950408552768516)(280,0.950408552768516)(280,0.950408552768516)(280,0.950408552768516)(281,0.950875450054383)(281,0.950875450054383)(281,0.950875450054383)(282,0.951262848984247)(282,0.951262848984247)(283,0.951684787587545)(284,0.95213379026251)(284,0.95213379026251)(284,0.95213379026251)(285,0.952596904687735)(285,0.952596904687735)(285,0.952596904687735)(286,0.953070490452783)(286,0.953070490452783)(286,0.953070490452783)(286,0.953070490452783)(286,0.953070490452783)(287,0.953514502119343)(287,0.953514502119343)(287,0.953514502119343)(287,0.953514502119343)(287,0.953514502119343)(287,0.953514502119343)(288,0.95399895440012)(288,0.95399895440012)(288,0.95399895440012)(288,0.95399895440012)(289,0.954489510606171)(290,0.95497723177142)(290,0.95497723177142)(290,0.95497723177142)(291,0.955450029050555)(292,0.955887310458996)(292,0.955887310458996)(292,0.955887310458996)(293,0.956403052639331)(293,0.956403052639331)(294,0.956555083766436)(294,0.956555083766436)(295,0.95676186127941)(295,0.95676186127941)(295,0.95676186127941)(295,0.95676186127941)(296,0.956774723860622)(296,0.956774723860622)(297,0.9568645048527)(297,0.9568645048527)(297,0.9568645048527)(297,0.9568645048527)(298,0.956968709064613)(300,0.956879284212771)(300,0.956879284212771)(300,0.956879284212771)(300,0.956879284212771)(302,0.956686172895233)(302,0.956686172895233)(302,0.956686172895233)(302,0.956686172895233)(303,0.956582401605558)(303,0.956582401605558)(303,0.956582401605558)(303,0.956582401605558)(304,0.956487344729481)(304,0.956487344729481)(304,0.956487344729481)(305,0.956405284722594)(306,0.956299062051927)(306,0.956299062051927)(306,0.956299062051927)(307,0.956265562035519)(307,0.956265562035519)(307,0.956265562035519)(307,0.956265562035519)(308,0.956197346123281)(308,0.956197346123281)(308,0.956197346123281)(308,0.956197346123281)(308,0.956197346123281)(309,0.956159069890839)(309,0.956159069890839)(310,0.956074023143578)(311,0.955980296152062)(311,0.955980296152062)(311,0.955980296152062)(311,0.955980296152062)(312,0.955879445578222)(312,0.955879445578222)(313,0.955771418414643)(314,0.955651903992868)(314,0.955651903992868)(314,0.955651903992868)(314,0.955651903992868)(315,0.955559632038078)(315,0.955559632038078)(315,0.955559632038078)(317,0.955386428325789)(318,0.955335594850305)(318,0.955335594850305)(318,0.955335594850305)(318,0.955335594850305)(318,0.955335594850305)(319,0.955305972259237)(320,0.955304868524587)(320,0.955304868524587)(321,0.955343337225143)(321,0.955343337225143)(321,0.955343337225143)(322,0.95537048193165)(322,0.95537048193165)(323,0.955557373505756)(323,0.955557373505756)(323,0.955557373505756)(324,0.955727895380232)(324,0.955727895380232)(324,0.955727895380232)(325,0.955930694667878)(326,0.956158050368089)(327,0.956349678187857)(327,0.956349678187857)(327,0.956349678187857)(328,0.956626660541835)(328,0.956626660541835)(328,0.956626660541835)(328,0.956626660541835)(329,0.956921669966478)(329,0.956921669966478)(330,0.95722875776911)(330,0.95722875776911)(330,0.95722875776911)(330,0.95722875776911)(330,0.95722875776911)(330,0.95722875776911)(330,0.95722875776911)(331,0.95750749680274)(332,0.957817655050421)(332,0.957817655050421)(332,0.957817655050421)(332,0.957817655050421)(332,0.957817655050421)(333,0.958150469309119)(334,0.95840573722515)(334,0.95840573722515)(334,0.95840573722515)(334,0.95840573722515)(334,0.95840573722515)(334,0.95840573722515)(334,0.95840573722515)(335,0.958672414916505)(335,0.958672414916505)(335,0.958672414916505)(335,0.958672414916505)(337,0.959129776221274)(337,0.959129776221274)(337,0.959129776221274)(337,0.959129776221274)(338,0.959328760533567)(338,0.959328760533567)(338,0.959328760533567)(340,0.959687426763624)(340,0.959687426763624)(341,0.959852654693859)(341,0.959852654693859)(341,0.959852654693859)(341,0.959852654693859)(341,0.959852654693859)(342,0.960007118105813)(342,0.960007118105813)(342,0.960007118105813)(343,0.960037537726118)(344,0.960264208116497)(344,0.960264208116497)(344,0.960264208116497)(344,0.960264208116497)(345,0.960354531955046)(345,0.960354531955046)(346,0.960503585350665)(346,0.960503585350665)(347,0.960413572613197)(348,0.960388616229773)(348,0.960388616229773)(348,0.960388616229773)(348,0.960388616229773)(348,0.960388616229773)(348,0.960388616229773)(348,0.960388616229773)(349,0.960340002109488)(349,0.960340002109488)(349,0.960340002109488)(350,0.960274463143427)(350,0.960274463143427)(350,0.960274463143427)(350,0.960274463143427)(350,0.960274463143427)(351,0.960170994588079)(351,0.960170994588079)(352,0.960053862993151)(352,0.960053862993151)(352,0.960053862993151)(353,0.959919503011558)(355,0.959494302747145)(355,0.959494302747145)(355,0.959494302747145)(356,0.959309494927677)(356,0.959309494927677)(356,0.959309494927677)(356,0.959309494927677)(356,0.959309494927677)(356,0.959309494927677)(357,0.95912172351714)(357,0.95912172351714)(357,0.95912172351714)(358,0.958969398010719)(358,0.958969398010719)(358,0.958969398010719)(358,0.958969398010719)(359,0.958782685312134)(359,0.958782685312134)(360,0.958570772781846)(360,0.958570772781846)(360,0.958570772781846)(360,0.958570772781846)(360,0.958570772781846)(361,0.958404198115068)(361,0.958404198115068)(362,0.95827058242937)(362,0.95827058242937)(363,0.958167901250893)(363,0.958167901250893)(363,0.958167901250893)(363,0.958167901250893)(363,0.958167901250893)(363,0.958167901250893)(364,0.958092984137073)(364,0.958092984137073)(364,0.958092984137073)(364,0.958092984137073)(365,0.958047817614294)(366,0.958036838686772)(366,0.958036838686772)(366,0.958036838686772)(366,0.958036838686772)(366,0.958036838686772)(366,0.958036838686772)(367,0.958043800622858)(368,0.958044982770562)(368,0.958044982770562)(369,0.958054613946264)(369,0.958054613946264)(369,0.958054613946264)(370,0.958067045355832)(370,0.958067045355832)(371,0.95804337200586)(371,0.95804337200586)(373,0.958030794605986)(373,0.958030794605986)(373,0.958030794605986)(373,0.958030794605986)(373,0.958030794605986)(374,0.958029263544541)(374,0.958029263544541)(374,0.958029263544541)(374,0.958029263544541)(374,0.958029263544541)(375,0.95802488312213)(375,0.95802488312213)(375,0.95802488312213)(376,0.958027528244165)(377,0.958000003724217)(378,0.957959207198427)(378,0.957959207198427)(378,0.957959207198427)(378,0.957959207198427)(378,0.957959207198427)(378,0.957959207198427)(378,0.957959207198427)(378,0.957959207198427)(379,0.957918165021275)(380,0.957883969259391)(380,0.957883969259391)(380,0.957883969259391)(381,0.957860177177125)(381,0.957860177177125)(382,0.957848212504885)(382,0.957848212504885)(382,0.957848212504885)(382,0.957848212504885)(383,0.957848416815301)(383,0.957848416815301)(383,0.957848416815301)(383,0.957848416815301)(384,0.957860329567704)(384,0.957860329567704)(384,0.957860329567704)(385,0.95788319330793)(385,0.95788319330793)(386,0.957916039447836)(386,0.957916039447836)(387,0.957957893634258)(388,0.958008042600016)(388,0.958008042600016)(388,0.958008042600016)(388,0.958008042600016)(388,0.958008042600016)(388,0.958008042600016)(389,0.958065941779167)(389,0.958065941779167)(390,0.958131243477321)(390,0.958131243477321)(390,0.958131243477321)(391,0.958203627747705)(391,0.958203627747705)(391,0.958203627747705)(391,0.958203627747705)(391,0.958203627747705)(391,0.958203627747705)(392,0.958282753953344)(392,0.958282753953344)(392,0.958282753953344)(393,0.958368479126603)(393,0.958368479126603)(393,0.958368479126603)(393,0.958368479126603)(393,0.958368479126603)(393,0.958368479126603)(393,0.958368479126603)(393,0.958368479126603)(394,0.958460702986279)(394,0.958460702986279)(394,0.958460702986279)(394,0.958460702986279)(394,0.958460702986279)(394,0.958460702986279)(394,0.958460702986279)(395,0.958559367106328)(395,0.958559367106328)(395,0.958559367106328)(395,0.958559367106328)(395,0.958559367106328)(396,0.958664491443435)(396,0.958664491443435)(396,0.958664491443435)(396,0.958664491443435)(396,0.958664491443435)(397,0.958776092321122)(397,0.958776092321122)(397,0.958776092321122)(397,0.958776092321122)(397,0.958776092321122)(398,0.958894423763262)(398,0.958894423763262)(399,0.959019893464228)(399,0.959019893464228)(399,0.959019893464228)(399,0.959019893464228)(400,0.959152890727919)(400,0.959152890727919)(400,0.959152890727919) 
};
\addlegendentry{\acl (max. variance)};

\addplot [
color=green!50!black,
solid,
line width=1.0pt,
]
coordinates{
 (12,0.520269868983676)(12,0.520269868983676)(14,0.627643211888186)(14,0.627643211888186)(14,0.627643211888186)(15,0.674883498850051)(15,0.674883498850051)(15,0.674883498850051)(16,0.710485999671765)(16,0.710485999671765)(16,0.710485999671765)(16,0.710485999671765)(16,0.710485999671765)(16,0.710485999671765)(16,0.710485999671765)(16,0.710485999671765)(17,0.732663782085289)(17,0.732663782085289)(17,0.732663782085289)(17,0.732663782085289)(17,0.732663782085289)(18,0.742962426594032)(18,0.742962426594032)(18,0.742962426594032)(18,0.742962426594032)(18,0.742962426594032)(18,0.742962426594032)(18,0.742962426594032)(19,0.749756619995662)(19,0.749756619995662)(19,0.749756619995662)(19,0.749756619995662)(19,0.749756619995662)(19,0.749756619995662)(20,0.757621719357678)(20,0.757621719357678)(20,0.757621719357678)(20,0.757621719357678)(20,0.757621719357678)(20,0.757621719357678)(20,0.757621719357678)(21,0.765411930840436)(21,0.765411930840436)(21,0.765411930840436)(21,0.765411930840436)(21,0.765411930840436)(21,0.765411930840436)(21,0.765411930840436)(21,0.765411930840436)(21,0.765411930840436)(21,0.765411930840436)(21,0.765411930840436)(21,0.765411930840436)(21,0.765411930840436)(21,0.765411930840436)(21,0.765411930840436)(22,0.772629665416525)(22,0.772629665416525)(22,0.772629665416525)(22,0.772629665416525)(22,0.772629665416525)(22,0.772629665416525)(22,0.772629665416525)(22,0.772629665416525)(22,0.772629665416525)(22,0.772629665416525)(22,0.772629665416525)(22,0.772629665416525)(22,0.772629665416525)(22,0.772629665416525)(22,0.772629665416525)(22,0.772629665416525)(22,0.772629665416525)(22,0.772629665416525)(22,0.772629665416525)(22,0.772629665416525)(22,0.772629665416525)(23,0.775116516816223)(23,0.775116516816223)(23,0.775116516816223)(23,0.775116516816223)(23,0.775116516816223)(23,0.775116516816223)(23,0.775116516816223)(23,0.775116516816223)(23,0.775116516816223)(23,0.775116516816223)(23,0.775116516816223)(23,0.775116516816223)(23,0.775116516816223)(23,0.775116516816223)(23,0.775116516816223)(23,0.775116516816223)(23,0.775116516816223)(23,0.775116516816223)(23,0.775116516816223)(23,0.775116516816223)(23,0.775116516816223)(23,0.775116516816223)(23,0.775116516816223)(24,0.776736166644763)(24,0.776736166644763)(24,0.776736166644763)(24,0.776736166644763)(24,0.776736166644763)(24,0.776736166644763)(24,0.776736166644763)(24,0.776736166644763)(24,0.776736166644763)(24,0.776736166644763)(24,0.776736166644763)(24,0.776736166644763)(24,0.776736166644763)(24,0.776736166644763)(24,0.776736166644763)(24,0.776736166644763)(24,0.776736166644763)(24,0.776736166644763)(24,0.776736166644763)(24,0.776736166644763)(24,0.776736166644763)(24,0.776736166644763)(24,0.776736166644763)(25,0.78146487866827)(25,0.78146487866827)(25,0.78146487866827)(25,0.78146487866827)(25,0.78146487866827)(25,0.78146487866827)(25,0.78146487866827)(25,0.78146487866827)(25,0.78146487866827)(25,0.78146487866827)(25,0.78146487866827)(25,0.78146487866827)(25,0.78146487866827)(25,0.78146487866827)(26,0.785320195162583)(26,0.785320195162583)(26,0.785320195162583)(26,0.785320195162583)(26,0.785320195162583)(26,0.785320195162583)(26,0.785320195162583)(26,0.785320195162583)(26,0.785320195162583)(26,0.785320195162583)(26,0.785320195162583)(26,0.785320195162583)(26,0.785320195162583)(26,0.785320195162583)(26,0.785320195162583)(26,0.785320195162583)(26,0.785320195162583)(26,0.785320195162583)(27,0.788793045105147)(27,0.788793045105147)(27,0.788793045105147)(27,0.788793045105147)(27,0.788793045105147)(27,0.788793045105147)(27,0.788793045105147)(27,0.788793045105147)(27,0.788793045105147)(27,0.788793045105147)(27,0.788793045105147)(27,0.788793045105147)(27,0.788793045105147)(27,0.788793045105147)(27,0.788793045105147)(27,0.788793045105147)(27,0.788793045105147)(27,0.788793045105147)(27,0.788793045105147)(27,0.788793045105147)(27,0.788793045105147)(28,0.79191988239061)(28,0.79191988239061)(28,0.79191988239061)(28,0.79191988239061)(28,0.79191988239061)(28,0.79191988239061)(28,0.79191988239061)(28,0.79191988239061)(28,0.79191988239061)(28,0.79191988239061)(28,0.79191988239061)(28,0.79191988239061)(28,0.79191988239061)(28,0.79191988239061)(28,0.79191988239061)(28,0.79191988239061)(28,0.79191988239061)(28,0.79191988239061)(28,0.79191988239061)(29,0.79578426681371)(29,0.79578426681371)(29,0.79578426681371)(29,0.79578426681371)(29,0.79578426681371)(29,0.79578426681371)(29,0.79578426681371)(29,0.79578426681371)(29,0.79578426681371)(29,0.79578426681371)(29,0.79578426681371)(29,0.79578426681371)(29,0.79578426681371)(29,0.79578426681371)(29,0.79578426681371)(29,0.79578426681371)(30,0.800515216470141)(30,0.800515216470141)(30,0.800515216470141)(30,0.800515216470141)(30,0.800515216470141)(30,0.800515216470141)(30,0.800515216470141)(30,0.800515216470141)(30,0.800515216470141)(30,0.800515216470141)(30,0.800515216470141)(30,0.800515216470141)(30,0.800515216470141)(30,0.800515216470141)(30,0.800515216470141)(30,0.800515216470141)(30,0.800515216470141)(31,0.803724788463443)(31,0.803724788463443)(31,0.803724788463443)(31,0.803724788463443)(31,0.803724788463443)(31,0.803724788463443)(31,0.803724788463443)(31,0.803724788463443)(31,0.803724788463443)(31,0.803724788463443)(31,0.803724788463443)(31,0.803724788463443)(31,0.803724788463443)(31,0.803724788463443)(31,0.803724788463443)(31,0.803724788463443)(31,0.803724788463443)(31,0.803724788463443)(31,0.803724788463443)(31,0.803724788463443)(31,0.803724788463443)(32,0.805114145392081)(32,0.805114145392081)(32,0.805114145392081)(32,0.805114145392081)(32,0.805114145392081)(32,0.805114145392081)(32,0.805114145392081)(32,0.805114145392081)(32,0.805114145392081)(32,0.805114145392081)(32,0.805114145392081)(32,0.805114145392081)(32,0.805114145392081)(32,0.805114145392081)(32,0.805114145392081)(32,0.805114145392081)(32,0.805114145392081)(33,0.805778827210335)(33,0.805778827210335)(33,0.805778827210335)(33,0.805778827210335)(33,0.805778827210335)(33,0.805778827210335)(33,0.805778827210335)(33,0.805778827210335)(33,0.805778827210335)(33,0.805778827210335)(33,0.805778827210335)(33,0.805778827210335)(33,0.805778827210335)(33,0.805778827210335)(34,0.807551272316851)(34,0.807551272316851)(34,0.807551272316851)(34,0.807551272316851)(34,0.807551272316851)(34,0.807551272316851)(34,0.807551272316851)(34,0.807551272316851)(34,0.807551272316851)(34,0.807551272316851)(34,0.807551272316851)(34,0.807551272316851)(34,0.807551272316851)(34,0.807551272316851)(34,0.807551272316851)(34,0.807551272316851)(35,0.810602346375291)(35,0.810602346375291)(35,0.810602346375291)(35,0.810602346375291)(35,0.810602346375291)(35,0.810602346375291)(35,0.810602346375291)(35,0.810602346375291)(35,0.810602346375291)(35,0.810602346375291)(35,0.810602346375291)(35,0.810602346375291)(35,0.810602346375291)(35,0.810602346375291)(35,0.810602346375291)(35,0.810602346375291)(35,0.810602346375291)(35,0.810602346375291)(35,0.810602346375291)(36,0.813726864473434)(36,0.813726864473434)(36,0.813726864473434)(36,0.813726864473434)(36,0.813726864473434)(36,0.813726864473434)(36,0.813726864473434)(36,0.813726864473434)(36,0.813726864473434)(36,0.813726864473434)(36,0.813726864473434)(36,0.813726864473434)(36,0.813726864473434)(36,0.813726864473434)(36,0.813726864473434)(36,0.813726864473434)(36,0.813726864473434)(36,0.813726864473434)(37,0.816892180223108)(37,0.816892180223108)(37,0.816892180223108)(37,0.816892180223108)(37,0.816892180223108)(37,0.816892180223108)(37,0.816892180223108)(37,0.816892180223108)(37,0.816892180223108)(37,0.816892180223108)(37,0.816892180223108)(37,0.816892180223108)(37,0.816892180223108)(37,0.816892180223108)(37,0.816892180223108)(37,0.816892180223108)(37,0.816892180223108)(37,0.816892180223108)(37,0.816892180223108)(37,0.816892180223108)(37,0.816892180223108)(38,0.819772703875216)(38,0.819772703875216)(38,0.819772703875216)(38,0.819772703875216)(38,0.819772703875216)(38,0.819772703875216)(38,0.819772703875216)(38,0.819772703875216)(38,0.819772703875216)(38,0.819772703875216)(39,0.823120329839772)(39,0.823120329839772)(39,0.823120329839772)(39,0.823120329839772)(39,0.823120329839772)(39,0.823120329839772)(39,0.823120329839772)(39,0.823120329839772)(39,0.823120329839772)(39,0.823120329839772)(39,0.823120329839772)(39,0.823120329839772)(39,0.823120329839772)(39,0.823120329839772)(39,0.823120329839772)(40,0.827034409220333)(40,0.827034409220333)(40,0.827034409220333)(40,0.827034409220333)(40,0.827034409220333)(40,0.827034409220333)(40,0.827034409220333)(40,0.827034409220333)(40,0.827034409220333)(40,0.827034409220333)(40,0.827034409220333)(40,0.827034409220333)(40,0.827034409220333)(41,0.829392300356047)(41,0.829392300356047)(41,0.829392300356047)(41,0.829392300356047)(41,0.829392300356047)(41,0.829392300356047)(41,0.829392300356047)(41,0.829392300356047)(41,0.829392300356047)(42,0.831422570533328)(42,0.831422570533328)(42,0.831422570533328)(42,0.831422570533328)(42,0.831422570533328)(42,0.831422570533328)(42,0.831422570533328)(42,0.831422570533328)(42,0.831422570533328)(42,0.831422570533328)(42,0.831422570533328)(42,0.831422570533328)(42,0.831422570533328)(42,0.831422570533328)(43,0.833666602182496)(43,0.833666602182496)(43,0.833666602182496)(43,0.833666602182496)(43,0.833666602182496)(43,0.833666602182496)(43,0.833666602182496)(43,0.833666602182496)(43,0.833666602182496)(43,0.833666602182496)(44,0.836183718860269)(44,0.836183718860269)(44,0.836183718860269)(44,0.836183718860269)(44,0.836183718860269)(44,0.836183718860269)(44,0.836183718860269)(44,0.836183718860269)(44,0.836183718860269)(44,0.836183718860269)(44,0.836183718860269)(44,0.836183718860269)(44,0.836183718860269)(44,0.836183718860269)(44,0.836183718860269)(45,0.838032803932643)(45,0.838032803932643)(45,0.838032803932643)(45,0.838032803932643)(45,0.838032803932643)(45,0.838032803932643)(45,0.838032803932643)(45,0.838032803932643)(45,0.838032803932643)(46,0.838807888578113)(46,0.838807888578113)(46,0.838807888578113)(46,0.838807888578113)(46,0.838807888578113)(46,0.838807888578113)(46,0.838807888578113)(46,0.838807888578113)(46,0.838807888578113)(46,0.838807888578113)(47,0.839607055980788)(47,0.839607055980788)(47,0.839607055980788)(47,0.839607055980788)(47,0.839607055980788)(47,0.839607055980788)(47,0.839607055980788)(47,0.839607055980788)(47,0.839607055980788)(47,0.839607055980788)(47,0.839607055980788)(47,0.839607055980788)(48,0.840893231333228)(48,0.840893231333228)(48,0.840893231333228)(48,0.840893231333228)(48,0.840893231333228)(48,0.840893231333228)(48,0.840893231333228)(48,0.840893231333228)(48,0.840893231333228)(48,0.840893231333228)(48,0.840893231333228)(48,0.840893231333228)(49,0.842970575412195)(49,0.842970575412195)(49,0.842970575412195)(49,0.842970575412195)(49,0.842970575412195)(50,0.845443586695015)(50,0.845443586695015)(50,0.845443586695015)(50,0.845443586695015)(50,0.845443586695015)(50,0.845443586695015)(50,0.845443586695015)(50,0.845443586695015)(50,0.845443586695015)(51,0.847937671241909)(51,0.847937671241909)(51,0.847937671241909)(51,0.847937671241909)(51,0.847937671241909)(51,0.847937671241909)(51,0.847937671241909)(51,0.847937671241909)(51,0.847937671241909)(51,0.847937671241909)(51,0.847937671241909)(51,0.847937671241909)(51,0.847937671241909)(51,0.847937671241909)(52,0.850833308162868)(52,0.850833308162868)(52,0.850833308162868)(52,0.850833308162868)(52,0.850833308162868)(52,0.850833308162868)(52,0.850833308162868)(52,0.850833308162868)(52,0.850833308162868)(53,0.853227534232459)(53,0.853227534232459)(53,0.853227534232459)(53,0.853227534232459)(53,0.853227534232459)(53,0.853227534232459)(53,0.853227534232459)(53,0.853227534232459)(53,0.853227534232459)(53,0.853227534232459)(54,0.855742091823922)(54,0.855742091823922)(54,0.855742091823922)(54,0.855742091823922)(54,0.855742091823922)(54,0.855742091823922)(54,0.855742091823922)(54,0.855742091823922)(54,0.855742091823922)(54,0.855742091823922)(54,0.855742091823922)(54,0.855742091823922)(55,0.857612110511561)(55,0.857612110511561)(55,0.857612110511561)(55,0.857612110511561)(55,0.857612110511561)(55,0.857612110511561)(55,0.857612110511561)(55,0.857612110511561)(55,0.857612110511561)(56,0.859759869087035)(56,0.859759869087035)(56,0.859759869087035)(56,0.859759869087035)(56,0.859759869087035)(56,0.859759869087035)(56,0.859759869087035)(57,0.861486565673413)(57,0.861486565673413)(57,0.861486565673413)(57,0.861486565673413)(57,0.861486565673413)(57,0.861486565673413)(58,0.862437154373006)(58,0.862437154373006)(58,0.862437154373006)(58,0.862437154373006)(58,0.862437154373006)(58,0.862437154373006)(58,0.862437154373006)(58,0.862437154373006)(58,0.862437154373006)(59,0.863179174310163)(59,0.863179174310163)(59,0.863179174310163)(59,0.863179174310163)(59,0.863179174310163)(59,0.863179174310163)(59,0.863179174310163)(59,0.863179174310163)(60,0.863534567030453)(60,0.863534567030453)(61,0.864041150493249)(61,0.864041150493249)(61,0.864041150493249)(61,0.864041150493249)(61,0.864041150493249)(61,0.864041150493249)(61,0.864041150493249)(61,0.864041150493249)(62,0.865088541713233)(62,0.865088541713233)(62,0.865088541713233)(62,0.865088541713233)(63,0.866955090368762)(63,0.866955090368762)(63,0.866955090368762)(63,0.866955090368762)(63,0.866955090368762)(63,0.866955090368762)(64,0.868725894771374)(64,0.868725894771374)(64,0.868725894771374)(64,0.868725894771374)(64,0.868725894771374)(64,0.868725894771374)(64,0.868725894771374)(64,0.868725894771374)(64,0.868725894771374)(65,0.870676556547044)(65,0.870676556547044)(65,0.870676556547044)(65,0.870676556547044)(65,0.870676556547044)(65,0.870676556547044)(66,0.872589762174424)(66,0.872589762174424)(66,0.872589762174424)(66,0.872589762174424)(66,0.872589762174424)(66,0.872589762174424)(66,0.872589762174424)(66,0.872589762174424)(67,0.874391509978051)(67,0.874391509978051)(67,0.874391509978051)(67,0.874391509978051)(67,0.874391509978051)(67,0.874391509978051)(67,0.874391509978051)(68,0.87599733371461)(68,0.87599733371461)(68,0.87599733371461)(68,0.87599733371461)(68,0.87599733371461)(69,0.877144646198532)(69,0.877144646198532)(70,0.878194711216158)(70,0.878194711216158)(70,0.878194711216158)(70,0.878194711216158)(70,0.878194711216158)(70,0.878194711216158)(70,0.878194711216158)(71,0.878872394257772)(71,0.878872394257772)(71,0.878872394257772)(71,0.878872394257772)(71,0.878872394257772)(71,0.878872394257772)(71,0.878872394257772)(72,0.878844595635794)(72,0.878844595635794)(72,0.878844595635794)(72,0.878844595635794)(72,0.878844595635794)(72,0.878844595635794)(72,0.878844595635794)(72,0.878844595635794)(72,0.878844595635794)(73,0.879104165689125)(73,0.879104165689125)(73,0.879104165689125)(73,0.879104165689125)(74,0.879271329388919)(74,0.879271329388919)(74,0.879271329388919)(74,0.879271329388919)(75,0.879460005003723)(75,0.879460005003723)(75,0.879460005003723)(76,0.879709510037971)(76,0.879709510037971)(76,0.879709510037971)(76,0.879709510037971)(76,0.879709510037971)(76,0.879709510037971)(76,0.879709510037971)(76,0.879709510037971)(77,0.880080747799966)(77,0.880080747799966)(78,0.88104893470025)(78,0.88104893470025)(78,0.88104893470025)(78,0.88104893470025)(78,0.88104893470025)(78,0.88104893470025)(78,0.88104893470025)(79,0.881568378474074)(79,0.881568378474074)(79,0.881568378474074)(79,0.881568378474074)(79,0.881568378474074)(79,0.881568378474074)(79,0.881568378474074)(79,0.881568378474074)(80,0.882103505992437)(80,0.882103505992437)(80,0.882103505992437)(80,0.882103505992437)(80,0.882103505992437)(80,0.882103505992437)(81,0.882700047873916)(81,0.882700047873916)(82,0.883414311539653)(82,0.883414311539653)(82,0.883414311539653)(82,0.883414311539653)(82,0.883414311539653)(82,0.883414311539653)(82,0.883414311539653)(83,0.88400160870352)(83,0.88400160870352)(83,0.88400160870352)(83,0.88400160870352)(84,0.884729886872847)(84,0.884729886872847)(84,0.884729886872847)(84,0.884729886872847)(84,0.884729886872847)(84,0.884729886872847)(85,0.885545348977976)(85,0.885545348977976)(85,0.885545348977976)(85,0.885545348977976)(86,0.88645304373566)(86,0.88645304373566)(86,0.88645304373566)(86,0.88645304373566)(86,0.88645304373566)(87,0.887221197396469)(87,0.887221197396469)(88,0.888189698749815)(88,0.888189698749815)(88,0.888189698749815)(88,0.888189698749815)(88,0.888189698749815)(88,0.888189698749815)(89,0.889133302374255)(89,0.889133302374255)(89,0.889133302374255)(89,0.889133302374255)(89,0.889133302374255)(90,0.890102988142895)(90,0.890102988142895)(90,0.890102988142895)(91,0.891290883013176)(91,0.891290883013176)(91,0.891290883013176)(91,0.891290883013176)(91,0.891290883013176)(91,0.891290883013176)(92,0.89237538442844)(92,0.89237538442844)(92,0.89237538442844)(92,0.89237538442844)(92,0.89237538442844)(92,0.89237538442844)(92,0.89237538442844)(93,0.893568462453115)(93,0.893568462453115)(94,0.894555599976602)(94,0.894555599976602)(94,0.894555599976602)(94,0.894555599976602)(94,0.894555599976602)(95,0.89551402686647)(96,0.896460915024518)(96,0.896460915024518)(96,0.896460915024518)(96,0.896460915024518)(97,0.897400576447082)(97,0.897400576447082)(97,0.897400576447082)(97,0.897400576447082)(97,0.897400576447082)(97,0.897400576447082)(97,0.897400576447082)(97,0.897400576447082)(98,0.898334088563716)(98,0.898334088563716)(98,0.898334088563716)(98,0.898334088563716)(99,0.899255923857692)(99,0.899255923857692)(99,0.899255923857692)(99,0.899255923857692)(100,0.899995998422457)(100,0.899995998422457)(100,0.899995998422457)(101,0.900811864383872)(101,0.900811864383872)(101,0.900811864383872)(101,0.900811864383872)(101,0.900811864383872)(101,0.900811864383872)(101,0.900811864383872)(101,0.900811864383872)(102,0.901294735270285)(102,0.901294735270285)(103,0.901981146860443)(103,0.901981146860443)(103,0.901981146860443)(104,0.902602255598636)(104,0.902602255598636)(104,0.902602255598636)(104,0.902602255598636)(104,0.902602255598636)(105,0.903114484710005)(106,0.903488193848119)(106,0.903488193848119)(107,0.903733567909487)(107,0.903733567909487)(107,0.903733567909487)(107,0.903733567909487)(107,0.903733567909487)(107,0.903733567909487)(107,0.903733567909487)(108,0.903885948856256)(108,0.903885948856256)(109,0.904038291323848)(109,0.904038291323848)(109,0.904038291323848)(110,0.90415550051358)(111,0.904261387236647)(111,0.904261387236647)(112,0.904392076972664)(112,0.904392076972664)(112,0.904392076972664)(112,0.904392076972664)(112,0.904392076972664)(113,0.904574404026299)(113,0.904574404026299)(113,0.904574404026299)(114,0.905013845065613)(114,0.905013845065613)(115,0.9053076775548)(115,0.9053076775548)(115,0.9053076775548)(115,0.9053076775548)(115,0.9053076775548)(115,0.9053076775548)(115,0.9053076775548)(116,0.905691293501544)(116,0.905691293501544)(116,0.905691293501544)(117,0.906155816141898)(117,0.906155816141898)(117,0.906155816141898)(118,0.906796150384041)(118,0.906796150384041)(118,0.906796150384041)(118,0.906796150384041)(119,0.907314831700378)(119,0.907314831700378)(119,0.907314831700378)(119,0.907314831700378)(120,0.908086160844043)(120,0.908086160844043)(121,0.908861884413057)(121,0.908861884413057)(121,0.908861884413057)(121,0.908861884413057)(122,0.909686003016568)(122,0.909686003016568)(123,0.910665019846352)(123,0.910665019846352)(123,0.910665019846352)(123,0.910665019846352)(124,0.911635806587959)(124,0.911635806587959)(124,0.911635806587959)(125,0.912488974985119)(125,0.912488974985119)(125,0.912488974985119)(125,0.912488974985119)(125,0.912488974985119)(126,0.913318808930024)(126,0.913318808930024)(126,0.913318808930024)(126,0.913318808930024)(127,0.91410566017481)(128,0.914845864201419)(128,0.914845864201419)(128,0.914845864201419)(129,0.915520147436468)(129,0.915520147436468)(129,0.915520147436468)(129,0.915520147436468)(130,0.915937927499014)(130,0.915937927499014)(130,0.915937927499014)(130,0.915937927499014)(130,0.915937927499014)(130,0.915937927499014)(130,0.915937927499014)(131,0.916607754235439)(131,0.916607754235439)(132,0.917043243975626)(132,0.917043243975626)(132,0.917043243975626)(133,0.917437621327246)(133,0.917437621327246)(133,0.917437621327246)(134,0.917808152687129)(134,0.917808152687129)(134,0.917808152687129)(134,0.917808152687129)(135,0.918166902650197)(135,0.918166902650197)(136,0.918500315145717)(136,0.918500315145717)(136,0.918500315145717)(136,0.918500315145717)(136,0.918500315145717)(136,0.918500315145717)(137,0.918850958575539)(137,0.918850958575539)(138,0.919193895930777)(138,0.919193895930777)(139,0.919510859645591)(139,0.919510859645591)(139,0.919510859645591)(140,0.919800272147825)(140,0.919800272147825)(140,0.919800272147825)(140,0.919800272147825)(141,0.920022377352594)(141,0.920022377352594)(141,0.920022377352594)(142,0.920312877101542)(143,0.920552884825081)(143,0.920552884825081)(143,0.920552884825081)(144,0.92079042102117)(144,0.92079042102117)(144,0.92079042102117)(144,0.92079042102117)(144,0.92079042102117)(145,0.921022139600342)(145,0.921022139600342)(146,0.921238681835597)(146,0.921238681835597)(146,0.921238681835597)(147,0.921443871511953)(147,0.921443871511953)(147,0.921443871511953)(147,0.921443871511953)(147,0.921443871511953)(147,0.921443871511953)(148,0.921693454021765)(148,0.921693454021765)(149,0.921904979661946)(149,0.921904979661946)(150,0.922161817711867)(151,0.922379652102805)(153,0.922826645924928)(153,0.922826645924928)(154,0.923056471480842)(154,0.923056471480842)(154,0.923056471480842)(155,0.923418084698919)(155,0.923418084698919)(155,0.923418084698919)(155,0.923418084698919)(155,0.923418084698919)(156,0.923799836801967)(156,0.923799836801967)(156,0.923799836801967)(156,0.923799836801967)(157,0.924234241890047)(157,0.924234241890047)(157,0.924234241890047)(157,0.924234241890047)(158,0.92471596639993)(159,0.92523699932357)(159,0.92523699932357)(159,0.92523699932357)(159,0.92523699932357)(160,0.925786481267882)(160,0.925786481267882)(160,0.925786481267882)(160,0.925786481267882)(160,0.925786481267882)(161,0.926348263879901)(161,0.926348263879901)(161,0.926348263879901)(163,0.927410947921351)(163,0.927410947921351)(163,0.927410947921351)(163,0.927410947921351)(165,0.928410481853862)(165,0.928410481853862)(165,0.928410481853862)(165,0.928410481853862)(167,0.929111888055332)(169,0.929715085087593)(169,0.929715085087593)(169,0.929715085087593)(169,0.929715085087593)(169,0.929715085087593)(170,0.929963491767681)(170,0.929963491767681)(170,0.929963491767681)(172,0.930374089427985)(172,0.930374089427985)(172,0.930374089427985)(173,0.930540213634555)(173,0.930540213634555)(173,0.930540213634555)(173,0.930540213634555)(173,0.930540213634555)(174,0.930646912779026)(174,0.930646912779026)(174,0.930646912779026)(174,0.930646912779026)(175,0.93081442251749)(176,0.930927519442725)(176,0.930927519442725)(176,0.930927519442725)(176,0.930927519442725)(176,0.930927519442725)(177,0.931018423033342)(177,0.931018423033342)(179,0.931089057763086)(179,0.931089057763086)(180,0.931047521598267)(181,0.930961161417289)(181,0.930961161417289)(182,0.930837885594349)(182,0.930837885594349)(183,0.930793637745141)(183,0.930793637745141)(183,0.930793637745141)(184,0.93058313558245)(184,0.93058313558245)(184,0.93058313558245)(184,0.93058313558245)(184,0.93058313558245)(185,0.930348700538818)(185,0.930348700538818)(185,0.930348700538818)(185,0.930348700538818)(186,0.930108308394849)(186,0.930108308394849)(186,0.930108308394849)(187,0.929877276243786)(187,0.929877276243786)(187,0.929877276243786)(188,0.929707941628889)(188,0.929707941628889)(189,0.92956420738326)(189,0.92956420738326)(190,0.929468721421599)(190,0.929468721421599)(190,0.929468721421599)(190,0.929468721421599)(191,0.929431887493645)(191,0.929431887493645)(191,0.929431887493645)(191,0.929431887493645)(192,0.929373794741194)(192,0.929373794741194)(192,0.929373794741194)(192,0.929373794741194)(192,0.929373794741194)(193,0.929601864107849)(193,0.929601864107849)(194,0.929810745006216)(194,0.929810745006216)(194,0.929810745006216)(194,0.929810745006216)(196,0.930436958468623)(196,0.930436958468623)(196,0.930436958468623)(196,0.930436958468623)(197,0.930840908850393)(197,0.930840908850393)(197,0.930840908850393)(198,0.931296584847464)(198,0.931296584847464)(199,0.931797701461422)(199,0.931797701461422)(199,0.931797701461422)(199,0.931797701461422)(200,0.932342672217992)(200,0.932342672217992)(200,0.932342672217992)(200,0.932342672217992)(200,0.932342672217992)(200,0.932342672217992)(201,0.932928077325319)(202,0.933549144639079)(202,0.933549144639079)(203,0.934202398798407)(203,0.934202398798407)(203,0.934202398798407)(203,0.934202398798407)(203,0.934202398798407)(204,0.934917095015184)(206,0.936273813978996)(206,0.936273813978996)(207,0.936960425317968)(207,0.936960425317968)(208,0.937591976467151)(208,0.937591976467151)(208,0.937591976467151)(208,0.937591976467151)(208,0.937591976467151)(209,0.938311996386395)(209,0.938311996386395)(210,0.938943662107938)(210,0.938943662107938)(210,0.938943662107938)(210,0.938943662107938)(211,0.939610194253841)(211,0.939610194253841)(211,0.939610194253841)(211,0.939610194253841)(211,0.939610194253841)(211,0.939610194253841)(213,0.940950303679807)(213,0.940950303679807)(214,0.94158442554064)(214,0.94158442554064)(214,0.94158442554064)(214,0.94158442554064)(214,0.94158442554064)(214,0.94158442554064)(214,0.94158442554064)(214,0.94158442554064)(215,0.942201650711502)(215,0.942201650711502)(216,0.942796584949243)(217,0.943362504710157)(217,0.943362504710157)(218,0.943771747252387)(218,0.943771747252387)(219,0.944260574865556)(219,0.944260574865556)(220,0.944707437473318)(220,0.944707437473318)(220,0.944707437473318)(222,0.945461834868778)(222,0.945461834868778)(223,0.945754674195944)(223,0.945754674195944)(223,0.945754674195944)(223,0.945754674195944)(223,0.945754674195944)(223,0.945754674195944)(223,0.945754674195944)(224,0.945984643198419)(224,0.945984643198419)(224,0.945984643198419)(224,0.945984643198419)(224,0.945984643198419)(224,0.945984643198419)(225,0.946155597065222)(225,0.946155597065222)(226,0.94627797188145)(226,0.94627797188145)(227,0.946412154971751)(227,0.946412154971751)(227,0.946412154971751)(228,0.946423248437525)(228,0.946423248437525)(229,0.946379728217505)(230,0.946420652855687)(231,0.946469453938715)(231,0.946469453938715)(231,0.946469453938715)(232,0.946522555250363)(234,0.946658105820131)(234,0.946658105820131)(234,0.946658105820131)(235,0.946705068461677)(235,0.946705068461677)(235,0.946705068461677)(235,0.946705068461677)(236,0.946773831540345)(236,0.946773831540345)(236,0.946773831540345)(237,0.946837948195794)(237,0.946837948195794)(238,0.946937634607218)(239,0.947006430643542)(239,0.947006430643542)(240,0.947065919276875)(240,0.947065919276875)(241,0.947067385333011)(241,0.947067385333011)(242,0.947138716443928)(242,0.947138716443928)(242,0.947138716443928)(242,0.947138716443928)(243,0.947249079789093)(243,0.947249079789093)(243,0.947249079789093)(244,0.947342734547416)(245,0.94743570853509)(246,0.94754947527078)(246,0.94754947527078)(246,0.94754947527078)(246,0.94754947527078)(247,0.947729267130778)(247,0.947729267130778)(247,0.947729267130778)(248,0.947861045479732)(248,0.947861045479732)(250,0.948124713056937)(250,0.948124713056937)(251,0.948257738345591)(251,0.948257738345591)(251,0.948257738345591)(251,0.948257738345591)(252,0.948392362521996)(252,0.948392362521996)(253,0.948533425981786)(253,0.948533425981786)(253,0.948533425981786)(254,0.948678836216827)(254,0.948678836216827)(254,0.948678836216827)(254,0.948678836216827)(256,0.949062245062867)(256,0.949062245062867)(256,0.949062245062867)(257,0.94926352510081)(257,0.94926352510081)(258,0.94947572798106)(258,0.94947572798106)(258,0.94947572798106)(258,0.94947572798106)(258,0.94947572798106)(258,0.94947572798106)(258,0.94947572798106)(259,0.949675844637854)(259,0.949675844637854)(260,0.949875995486064)(260,0.949875995486064)(260,0.949875995486064)(261,0.950066600639542)(262,0.950229710931364)(262,0.950229710931364)(262,0.950229710931364)(262,0.950229710931364)(262,0.950229710931364)(262,0.950229710931364)(262,0.950229710931364)(262,0.950229710931364)(262,0.950229710931364)(263,0.950406033899057)(263,0.950406033899057)(263,0.950406033899057)(263,0.950406033899057)(264,0.950557004833062)(264,0.950557004833062)(265,0.950704249662838)(265,0.950704249662838)(265,0.950704249662838)(266,0.950838049875828)(266,0.950838049875828)(266,0.950838049875828)(266,0.950838049875828)(267,0.950962342864133)(268,0.951092192255367)(268,0.951092192255367)(268,0.951092192255367)(268,0.951092192255367)(268,0.951092192255367)(269,0.951253756756559)(269,0.951253756756559)(269,0.951253756756559)(269,0.951253756756559)(269,0.951253756756559)(270,0.951399687155496)(270,0.951399687155496)(271,0.951535444666842)(271,0.951535444666842)(272,0.951697774698619)(272,0.951697774698619)(272,0.951697774698619)(272,0.951697774698619)(273,0.951779378583525)(273,0.951779378583525)(274,0.951854497821311)(274,0.951854497821311)(275,0.951857164206129)(275,0.951857164206129)(275,0.951857164206129)(275,0.951857164206129)(275,0.951857164206129)(275,0.951857164206129)(276,0.951824223786323)(276,0.951824223786323)(277,0.951779430172532)(277,0.951779430172532)(277,0.951779430172532)(277,0.951779430172532)(277,0.951779430172532)(278,0.951674599735774)(279,0.951556932155573)(279,0.951556932155573)(279,0.951556932155573)(280,0.951411880627244)(280,0.951411880627244)(281,0.951258846635482)(281,0.951258846635482)(282,0.951133334705418)(283,0.951021053396185)(283,0.951021053396185)(283,0.951021053396185)(284,0.950963328799072)(284,0.950963328799072)(285,0.95097263018278)(285,0.95097263018278)(286,0.951047504568656)(286,0.951047504568656)(286,0.951047504568656)(286,0.951047504568656)(287,0.951257138664675)(288,0.951333514907971)(289,0.951501835306191)(289,0.951501835306191)(289,0.951501835306191)(289,0.951501835306191)(289,0.951501835306191)(289,0.951501835306191)(289,0.951501835306191)(289,0.951501835306191)(290,0.951679848408326)(290,0.951679848408326)(290,0.951679848408326)(291,0.95178546610233)(292,0.951869453550467)(292,0.951869453550467)(293,0.95193278959159)(293,0.95193278959159)(293,0.95193278959159)(294,0.951975107297374)(294,0.951975107297374)(294,0.951975107297374)(295,0.952000890307052)(296,0.952019580013567)(296,0.952019580013567)(296,0.952019580013567)(297,0.952042187598348)(297,0.952042187598348)(297,0.952042187598348)(297,0.952042187598348)(297,0.952042187598348)(297,0.952042187598348)(297,0.952042187598348)(298,0.952075526232462)(298,0.952075526232462)(299,0.95212157066826)(299,0.95212157066826)(299,0.95212157066826)(300,0.952178655101273)(300,0.952178655101273)(300,0.952178655101273)(300,0.952178655101273)(301,0.952243354939875)(301,0.952243354939875)(301,0.952243354939875)(301,0.952243354939875)(302,0.952309264297747)(302,0.952309264297747)(302,0.952309264297747)(302,0.952309264297747)(302,0.952309264297747)(303,0.952430367353662)(303,0.952430367353662)(303,0.952430367353662)(303,0.952430367353662)(303,0.952430367353662)(303,0.952430367353662)(304,0.952488102891477)(304,0.952488102891477)(304,0.952488102891477)(304,0.952488102891477)(304,0.952488102891477)(305,0.952563163041006)(305,0.952563163041006)(305,0.952563163041006)(305,0.952563163041006)(305,0.952563163041006)(306,0.952660934439784)(306,0.952660934439784)(306,0.952660934439784)(306,0.952660934439784)(306,0.952660934439784)(306,0.952660934439784)(306,0.952660934439784)(307,0.952775828494102)(309,0.95297935330009)(309,0.95297935330009)(309,0.95297935330009)(309,0.95297935330009)(309,0.95297935330009)(310,0.953131719021858)(310,0.953131719021858)(310,0.953131719021858)(310,0.953131719021858)(311,0.953300571874854)(311,0.953300571874854)(312,0.953393731008102)(312,0.953393731008102)(312,0.953393731008102)(312,0.953393731008102)(313,0.953556325108611)(313,0.953556325108611)(313,0.953556325108611)(313,0.953556325108611)(314,0.953724065535977)(315,0.953888579661829)(315,0.953888579661829)(315,0.953888579661829)(315,0.953888579661829)(316,0.953972381831759)(317,0.954162802919416)(317,0.954162802919416)(317,0.954162802919416)(317,0.954162802919416)(317,0.954162802919416)(318,0.954363743162627)(318,0.954363743162627)(319,0.954585940701707)(319,0.954585940701707)(319,0.954585940701707)(319,0.954585940701707)(319,0.954585940701707)(319,0.954585940701707)(320,0.95494180197364)(320,0.95494180197364)(320,0.95494180197364)(320,0.95494180197364)(321,0.955112459774914)(321,0.955112459774914)(321,0.955112459774914)(322,0.955399628773132)(323,0.95569415438265)(323,0.95569415438265)(323,0.95569415438265)(324,0.955997037601244)(324,0.955997037601244)(325,0.95629837637581)(325,0.95629837637581)(325,0.95629837637581)(325,0.95629837637581)(325,0.95629837637581)(325,0.95629837637581)(326,0.956583331239619)(326,0.956583331239619)(327,0.956832556265932)(327,0.956832556265932)(327,0.956832556265932)(328,0.957038518253838)(328,0.957038518253838)(328,0.957038518253838)(328,0.957038518253838)(329,0.957207168757048)(329,0.957207168757048)(329,0.957207168757048)(330,0.957347726022316)(330,0.957347726022316)(330,0.957347726022316)(330,0.957347726022316)(330,0.957347726022316)(330,0.957347726022316)(330,0.957347726022316)(330,0.957347726022316)(330,0.957347726022316)(331,0.957474653610745)(331,0.957474653610745)(332,0.957610687914795)(332,0.957610687914795)(332,0.957610687914795)(332,0.957610687914795)(332,0.957610687914795)(333,0.957807959371852)(333,0.957807959371852)(334,0.957975660270775)(334,0.957975660270775)(334,0.957975660270775)(335,0.958123536451116)(335,0.958123536451116)(336,0.958296872273934)(336,0.958296872273934)(336,0.958296872273934)(336,0.958296872273934)(336,0.958296872273934)(336,0.958296872273934)(337,0.958404531397843)(337,0.958404531397843)(337,0.958404531397843)(337,0.958404531397843)(337,0.958404531397843)(338,0.958566366001063)(338,0.958566366001063)(338,0.958566366001063)(338,0.958566366001063)(338,0.958566366001063)(338,0.958566366001063)(339,0.958713345180689)(339,0.958713345180689)(339,0.958713345180689)(339,0.958713345180689)(340,0.958966227663651)(340,0.958966227663651)(340,0.958966227663651)(340,0.958966227663651)(340,0.958966227663651)(341,0.959243880793365)(341,0.959243880793365)(341,0.959243880793365)(341,0.959243880793365)(342,0.959471428667221)(343,0.959690666067391)(343,0.959690666067391)(343,0.959690666067391)(343,0.959690666067391)(343,0.959690666067391)(343,0.959690666067391)(343,0.959690666067391)(344,0.959871205941693)(344,0.959871205941693)(344,0.959871205941693)(345,0.960081943776548)(345,0.960081943776548)(346,0.96013726010616)(346,0.96013726010616)(346,0.96013726010616)(346,0.96013726010616)(347,0.960110074210331)(347,0.960110074210331)(347,0.960110074210331)(348,0.960010758689971)(348,0.960010758689971)(348,0.960010758689971)(348,0.960010758689971)(349,0.9598657629479)(349,0.9598657629479)(349,0.9598657629479)(350,0.959700109081295)(350,0.959700109081295)(350,0.959700109081295)(350,0.959700109081295)(350,0.959700109081295)(350,0.959700109081295)(350,0.959700109081295)(350,0.959700109081295)(351,0.959547047616648)(351,0.959547047616648)(351,0.959547047616648)(352,0.95937131821936)(352,0.95937131821936)(352,0.95937131821936)(353,0.959230015643351)(353,0.959230015643351)(353,0.959230015643351)(353,0.959230015643351)(354,0.959108800441606)(354,0.959108800441606)(354,0.959108800441606)(354,0.959108800441606)(354,0.959108800441606)(354,0.959108800441606)(354,0.959108800441606)(355,0.959042362488064)(355,0.959042362488064)(355,0.959042362488064)(356,0.958967407226903)(356,0.958967407226903)(356,0.958967407226903)(356,0.958967407226903)(356,0.958967407226903)(356,0.958967407226903)(356,0.958967407226903)(357,0.958894099939353)(357,0.958894099939353)(357,0.958894099939353)(358,0.95881324654098)(358,0.95881324654098)(359,0.958721534095355)(359,0.958721534095355)(359,0.958721534095355)(360,0.958638450998072)(360,0.958638450998072)(360,0.958638450998072)(360,0.958638450998072)(360,0.958638450998072)(360,0.958638450998072)(361,0.95858718472248)(361,0.95858718472248)(361,0.95858718472248)(361,0.95858718472248)(361,0.95858718472248)(361,0.95858718472248)(362,0.958576776896015)(362,0.958576776896015)(362,0.958576776896015)(362,0.958576776896015)(362,0.958576776896015)(363,0.958610088615092)(363,0.958610088615092)(363,0.958610088615092)(364,0.958693165284686)(364,0.958693165284686)(364,0.958693165284686)(365,0.95893071122153)(365,0.95893071122153)(366,0.959042328035867)(366,0.959042328035867)(366,0.959042328035867)(367,0.959303406883276)(367,0.959303406883276)(367,0.959303406883276)(367,0.959303406883276)(368,0.959544919072571)(368,0.959544919072571)(368,0.959544919072571)(368,0.959544919072571)(368,0.959544919072571)(368,0.959544919072571)(368,0.959544919072571)(368,0.959544919072571)(370,0.960092288456028)(370,0.960092288456028)(370,0.960092288456028)(370,0.960092288456028)(370,0.960092288456028)(371,0.96041207196768)(371,0.96041207196768)(371,0.96041207196768)(371,0.96041207196768)(371,0.96041207196768)(371,0.96041207196768)(372,0.960556509164708)(372,0.960556509164708)(372,0.960556509164708)(372,0.960556509164708)(373,0.960784548953414)(373,0.960784548953414)(373,0.960784548953414)(373,0.960784548953414)(373,0.960784548953414)(373,0.960784548953414)(374,0.960911905537961)(374,0.960911905537961)(375,0.961024589125919)(375,0.961024589125919)(375,0.961024589125919)(375,0.961024589125919)(375,0.961024589125919)(376,0.961136156069103)(376,0.961136156069103)(376,0.961136156069103)(376,0.961136156069103)(376,0.961136156069103)(376,0.961136156069103)(376,0.961136156069103)(377,0.961251264316326)(377,0.961251264316326)(377,0.961251264316326)(377,0.961251264316326)(377,0.961251264316326)(377,0.961251264316326)(377,0.961251264316326)(377,0.961251264316326)(378,0.961399535129155)(378,0.961399535129155)(378,0.961399535129155)(379,0.961536263566727)(379,0.961536263566727)(379,0.961536263566727)(379,0.961536263566727)(379,0.961536263566727)(379,0.961536263566727)(379,0.961536263566727)(380,0.961688357945919)(380,0.961688357945919)(380,0.961688357945919)(381,0.961820037390705)(382,0.96192961321861)(382,0.96192961321861)(382,0.96192961321861)(382,0.96192961321861)(382,0.96192961321861)(382,0.96192961321861)(383,0.962019247923974)(383,0.962019247923974)(383,0.962019247923974)(383,0.962019247923974)(383,0.962019247923974)(384,0.962091553982361)(384,0.962091553982361)(384,0.962091553982361)(385,0.962148400747989)(386,0.962191102076151)(386,0.962191102076151)(386,0.962191102076151)(387,0.962220530524025)(387,0.962220530524025)(387,0.962220530524025)(387,0.962220530524025)(387,0.962220530524025)(388,0.962237481794892)(388,0.962237481794892)(389,0.962242407864926)(390,0.962235643053333)(390,0.962235643053333)(390,0.962235643053333)(391,0.962217354245848)(391,0.962217354245848)(391,0.962217354245848)(391,0.962217354245848)(392,0.962187595254565)(392,0.962187595254565)(392,0.962187595254565)(392,0.962187595254565)(392,0.962187595254565)(393,0.962146136862857)(393,0.962146136862857)(393,0.962146136862857)(393,0.962146136862857)(393,0.962146136862857)(394,0.96209242097699)(394,0.96209242097699)(394,0.96209242097699)(394,0.96209242097699)(394,0.96209242097699)(394,0.96209242097699)(394,0.96209242097699)(394,0.96209242097699)(394,0.96209242097699)(395,0.962025746359863)(396,0.961945464963945)(396,0.961945464963945)(396,0.961945464963945)(396,0.961945464963945)(396,0.961945464963945)(397,0.961851010127989)(397,0.961851010127989)(397,0.961851010127989)(397,0.961851010127989)(397,0.961851010127989)(397,0.961851010127989)(398,0.961741886051835)(398,0.961741886051835)(399,0.961617773708692)(399,0.961617773708692)(399,0.961617773708692)(399,0.961617773708692)(399,0.961617773708692)(400,0.961478466568242)(400,0.961478466568242)(400,0.961478466568242) 
};
\addlegendentry{\acl (random)};

\addplot [
color=orange,
densely dotted,
line width=1.0pt,
]
coordinates{
 %(1,0.0168355194759798)(2,0.0168355194759798)(3,0.0168355194759798)(4,0.0168374610889054)(5,0.0167699930780963)(6,0.048208589049083)(7,0.0477307785738257)(8,0.0476460981374464)(9,0.0474410305321807)(10,0.0485550543005947)(11,0.0507299655269844)(12,0.0532599869110817)(13,0.0669352482510405)(14,0.0856605578885116)(15,0.123749024176474)(16,0.132778182367498)(17,0.173523899962934)(18,0.228390698777483)(19,0.267670379610717)(20,0.328391421207214)(21,0.35553555784785)(22,0.388809416890549)(23,0.390987890598007)(24,0.398668883538064)(25,0.398607087878894)(26,0.413139948110276)(27,0.418235643086597)(28,0.421605492316469)(29,0.423641706313654)(30,0.430336010974062)(31,0.463627316393779)(32,0.464837285908879)
 (33,0.501703648534391)(34,0.515453153799586)(35,0.539465051247664)(36,0.555947493149717)(37,0.57339876711497)(38,0.627532176488351)(39,0.651856985149348)(40,0.662254513534309)(41,0.66133725322749)(42,0.675204859726853)(43,0.683783026853696)(44,0.685788959960219)(45,0.70282708160731)(46,0.706140469390923)(47,0.707702871846997)(48,0.714364336393848)(49,0.7151327654894)(50,0.718845886658305)(51,0.720697686870771)(52,0.723220881976813)(53,0.722307575815388)(54,0.731157566241356)(55,0.73185815760204)(56,0.737072589368828)(57,0.741196582140359)(58,0.745478161918016)(59,0.74668888008618)(60,0.748434036728539)(61,0.74866006828087)(62,0.749375203100507)(63,0.748366076922985)(64,0.756285447780586)(65,0.756545733769362)(66,0.758999038373228)(67,0.760875728901419)(68,0.760511098667807)(69,0.763315523563978)(70,0.762907675019111)(71,0.766780101808984)(72,0.767574594140362)(73,0.768873861635267)(74,0.77132158968897)(75,0.773969779107824)(76,0.776988811797957)(77,0.777269579257164)(78,0.77873812748378)(79,0.779590427010662)(80,0.779770542255994)(81,0.779814316292224)(82,0.779773865949109)(83,0.780821478558642)(84,0.781316430697107)(85,0.780119438999362)(86,0.782783511154606)(87,0.782935221683918)(88,0.784722878402314)(89,0.784528601892085)(90,0.785755530764448)(91,0.786579202746182)(92,0.787541394366187)(93,0.787037131978506)(94,0.787796126417614)(95,0.788657082673002)(96,0.788866586595803)(97,0.79245415547999)(98,0.79299645313061)(99,0.793239263423599)(100,0.793410417040492)(101,0.793398485158353)(102,0.79504230683255)(103,0.79538670163636)(104,0.795433929052428)(105,0.796636136401633)(106,0.797395298683088)(107,0.797478768764061)(108,0.79873184209954)(109,0.798188612412934)(110,0.798158006197179)(111,0.798737463747929)(112,0.799981252856107)(113,0.79988901329743)(114,0.800639170469608)(115,0.801271163621793)(116,0.802610363789278)(117,0.803153598680265)(118,0.805186114721705)(119,0.805857404220269)(120,0.805855969142446)(121,0.806059302750794)(122,0.807082294191735)(123,0.807535786430194)(124,0.807935321562945)(125,0.808146105261737)(126,0.809740449986898)(127,0.81113816625602)(128,0.812492651952222)(129,0.812520183313194)(130,0.813667241844308)(131,0.814520012562363)(132,0.81547005237087)(133,0.816599866145591)(134,0.817372719028093)(135,0.818284061575476)(136,0.818782014650651)(137,0.820492189331931)(138,0.821052916225409)(139,0.821660943450204)(140,0.822346179095648)(141,0.823463028124937)(142,0.823975253142869)(143,0.823888218841833)(144,0.823604992068404)(145,0.824073479069842)(146,0.823936301680092)(147,0.824320049556736)(148,0.824838356442337)(149,0.82507894986564)(150,0.825613977232026)(151,0.826323183825629)(152,0.826962847313908)(153,0.828598820170037)(154,0.82985830017499)(155,0.830091612363561)(156,0.830469168697509)(157,0.830480264249874)(158,0.830985835212206)(159,0.831601569407353)(160,0.831969784089505)(161,0.833068392622128)(162,0.832680461999549)(163,0.832665677203082)(164,0.832503845548131)(165,0.832504585295565)(166,0.832349902077364)(167,0.832451006253437)(168,0.833279823733234)(169,0.833362393545589)(170,0.833740108485466)(171,0.833553881327365)(172,0.834186084568046)(173,0.834493014200965)(174,0.834425251195595)(175,0.834158024728102)(176,0.834716512131478)(177,0.836077174064374)(178,0.836360618691598)(179,0.837543084209139)(180,0.838204625710763)(181,0.838210529312491)(182,0.838833858819765)(183,0.839193669209591)(184,0.839575527917163)(185,0.839979112792977)(186,0.839694962095681)(187,0.839634934091879)(188,0.839907908839505)(189,0.839925215110266)(190,0.839570148194517)(191,0.839775667052517)(192,0.839771032628145)(193,0.840681861570861)(194,0.840707042102804)(195,0.840789249207689)(196,0.842234796288293)(197,0.841964361273978)(198,0.842725131134611)(199,0.843522617295772)(200,0.843943100392778)(201,0.844022305405221)(202,0.844581235753074)(203,0.844343308636547)(204,0.845484770318911)(205,0.845885106866485)(206,0.845833825786078)(207,0.846020622495698)(208,0.846504034140925)(209,0.846422434529814)(210,0.846357244773762)(211,0.847324221779135)(212,0.848073796411742)(213,0.847929383752023)(214,0.848110535671909)(215,0.848096479489859)(216,0.849233244979107)(217,0.849938734058634)(218,0.850329026327976)(219,0.850896478217075)(220,0.851873469501244)(221,0.852716046868096)(222,0.852740261467841)(223,0.852625704924477)(224,0.852708835540035)(225,0.852931651320368)(226,0.853418221977078)(227,0.853504187290808)(228,0.853618563902009)(229,0.853832051812884)(230,0.853746659263614)(231,0.853745845298752)(232,0.853849741689243)(233,0.853895308109015)(234,0.854093567819473)(235,0.854157679336484)(236,0.854933714308131)(237,0.854520929241889)(238,0.854680414970597)(239,0.856186910172796)(240,0.856461855092529)(241,0.856484273237266)(242,0.856581839465739)(243,0.856423170046712)(244,0.856148413712893)(245,0.85649616902178)(246,0.856743762563026)(247,0.857780261905147)(248,0.857918172575755)(249,0.85760562973369)(250,0.857593875818756)(251,0.858281078706888)(252,0.858809733905514)(253,0.858757182008126)(254,0.859223736175564)(255,0.859720795726467)(256,0.860129612462307)(257,0.860066164277161)(258,0.8602412857742)(259,0.860650812063371)(260,0.860919496049057)(261,0.861295476059137)(262,0.861291150914811)(263,0.861211916856695)(264,0.861233584896272)(265,0.861401795085125)(266,0.861692472791896)(267,0.862428046006644)(268,0.862398187435361)(269,0.862452123071314)(270,0.863261396924492)(271,0.863331417005527)(272,0.863587034960837)(273,0.863475169304848)(274,0.863686904227101)(275,0.864615257121548)(276,0.8649558583527)(277,0.865063161884179)(278,0.865298523771288)(279,0.865380557862139)(280,0.865818113016894)(281,0.865871485443187)(282,0.866198881028621)(283,0.865974049433974)(284,0.866383400350716)(285,0.866963561479466)(286,0.867030460957878)(287,0.867462326764827)(288,0.867565296934443)(289,0.867779957070339)(290,0.868098175012602)(291,0.867873072942303)(292,0.867871784402578)(293,0.867658087915904)(294,0.867657801654585)(295,0.868627435317739)(296,0.868807588071323)(297,0.869113551102529)(298,0.869051069554459)(299,0.869151219951336)(300,0.869517058151138)(301,0.869809701337033)(302,0.870036146648058)(303,0.870223460310499)(304,0.870495709497698)(305,0.870916483540594)(306,0.871311438940446)(307,0.871096734376483)(308,0.871082497470043)(309,0.871394714041076)(310,0.871662718036576)(311,0.872020896966391)(312,0.872094417370846)(313,0.872265891374552)(314,0.872380685120301)(315,0.873012924052392)(316,0.873159201400678)(317,0.873155344608866)(318,0.873146245838206)(319,0.873504941352877)(320,0.873570951318809)(321,0.873699738445901)(322,0.874054054609346)(323,0.874332868915781)(324,0.874523001002456)(325,0.874880592675815)(326,0.875031016390726)(327,0.875156231565368)(328,0.875390336627467)(329,0.875057547909804)(330,0.874953682735061)(331,0.874843933292625)(332,0.875346992212524)(333,0.875262431664718)(334,0.875245973009823)(335,0.875111736792106)(336,0.875256555961774)(337,0.875507707723524)(338,0.875667862243695)(339,0.875881150457632)(340,0.875915799800559)(341,0.875880545598203)(342,0.87551929626444)(343,0.875903339532271)(344,0.876755387162011)(345,0.876983903548128)(346,0.877287282257713)(347,0.877278833482363)(348,0.877578993172259)(349,0.87757437882039)(350,0.878041331443303)(351,0.878008560125416)(352,0.878947126559563)(353,0.878761743389138)(354,0.878727279059182)(355,0.878791126245787)(356,0.878789270274246)(357,0.878879369457419)(358,0.878929332879307)(359,0.878869288487378)(360,0.87889432516828)(361,0.879159686486771)(362,0.879363120582601)(363,0.879844176991724)(364,0.879873960354928)(365,0.880035159406256)(366,0.880175209002008)(367,0.880306951854499)(368,0.880402699118724)(369,0.880431631472416)(370,0.880226654793281)(371,0.880316828539134)(372,0.880423225535735)(373,0.880254638796996)(374,0.880252375022937)(375,0.880482707617788)(376,0.880957396108648)(377,0.880980920378297)(378,0.881475374942296)(379,0.881509432889557)(380,0.881538203795952)(381,0.88178829347817)(382,0.882028813151936)(383,0.882509888101808)(384,0.882617992606114)(385,0.882582759302722)(386,0.882864494641784)(387,0.882857438270651)(388,0.882929266247059)(389,0.883337517327456)(390,0.88332394745124)(391,0.883351528922512)(392,0.883450625693178)(393,0.883786359642393)(394,0.884406657062666)(395,0.884541154944127)(396,0.884563591161788)(397,0.884919352416905)(398,0.88491506952062)(399,0.884526098247838)(400,0.884542989925364) 
};
\addlegendentry{\var};

\end{axis}
\end{tikzpicture}%

%% This file was created by matlab2tikz v0.2.3.
% Copyright (c) 2008--2012, Nico Schlömer <nico.schloemer@gmail.com>
% All rights reserved.
% 
% 
% 
\begin{tikzpicture}

\begin{axis}[%
tick label style={font=\tiny},
label style={font=\tiny},
label shift={-4pt},
xlabel shift={-6pt},
legend style={font=\tiny},
view={0}{90},
width=\figurewidth,
height=\figureheight,
scale only axis,
xmin=0, xmax=400,
xlabel={Samples},
ymin=0.48, ymax=0.85,
ylabel={$F_1$-score},
axis lines*=left,
legend cell align=left,
legend style={at={(1.03,0)},anchor=south east,fill=none,draw=none,align=left,row sep=-0.2em},
clip=false]

\addplot [
color=blue,
solid,
line width=1.0pt,
]
coordinates{
 (1,0.50499353346612)(2,0.525460676803361)(3,0.533032074113085)(4,0.551223655394847)(5,0.570550889964546)(6,0.607989879446322)(7,0.646594233151857)(8,0.663748324213583)(9,0.674288790531702)(10,0.679656073009898)(11,0.680354894546303)(12,0.680334715950774)(13,0.681762626054374)(14,0.683588935134126)(15,0.684634747174899)(16,0.686956482806509)(17,0.689256930206646)(18,0.693239274144404)(19,0.694089640995396)(20,0.694340913663611)(21,0.695191236946799)(22,0.696842820384993)(23,0.698364502514991)(24,0.699195058853491)(25,0.701025574829494)(26,0.702208740416849)(27,0.702529806193124)(28,0.702462001532441)(29,0.704849593743049)(30,0.708636050947943)(31,0.712325989213624)(32,0.716755064440036)(33,0.719840305940062)(34,0.721934948789977)(35,0.721881378371049)(36,0.723654194062596)(37,0.723953011286592)(38,0.723973519556634)(39,0.724339422156323)(40,0.724839553807487)(41,0.723741669971963)(42,0.724921390240578)(43,0.727073825157176)(44,0.72692176225631)(45,0.727042898074881)(46,0.726580489555796)(47,0.727951468676105)(48,0.730373258682881)(49,0.731731100633092)(50,0.73368766987797)(51,0.734690251820091)(52,0.736423919534552)(53,0.737533464208091)(54,0.739550971209447)(55,0.740079083698857)(56,0.740129865296803)(57,0.74230046827069)(58,0.743283711535009)(59,0.744263785592474)(60,0.745740083061752)(61,0.747608312486841)(62,0.749490831920238)(63,0.75020085717409)(64,0.75088855582631)(65,0.751813353628292)(66,0.753543675240218)(67,0.753941933029519)(68,0.754624467483603)(69,0.756424655399996)(70,0.757609659053724)(71,0.759247485711727)(72,0.760839141815096)(73,0.761368469687671)(74,0.762892164439497)(75,0.763906397916687)(76,0.765428030234251)(77,0.766573285707525)(78,0.767217365623669)(79,0.768758312096696)(80,0.76942723946945)(81,0.770142660100854)(82,0.770680232034324)(83,0.770900204884578)(84,0.771678847654424)(85,0.77302843234232)(86,0.773966303040727)(87,0.774396644108859)(88,0.774572385421272)(89,0.775892558446823)(90,0.776090183854804)(91,0.776851788126392)(92,0.777777111046217)(93,0.778224358818571)(94,0.778918217489081)(95,0.779519120548117)(96,0.780961477415924)(97,0.781155040236319)(98,0.78125709559588)(99,0.782511208671616)(100,0.782736181476786)(101,0.782874442091136)(102,0.783421910059339)(103,0.783695829078149)(104,0.783909391432457)(105,0.784257370502102)(106,0.78501872587983)(107,0.785447795948377)(108,0.785240266655901)(109,0.78548395065192)(110,0.785782201576145)(111,0.786152125888585)(112,0.786424366012975)(113,0.786635362863093)(114,0.787200103478296)(115,0.787340770002639)(116,0.787245662034931)(117,0.78710959568157)(118,0.787351933437509)(119,0.787969785393655)(120,0.787915438406619)(121,0.787779010813884)(122,0.788208371114636)(123,0.788295402164598)(124,0.788772981906269)(125,0.789220315854422)(126,0.789485850685347)(127,0.789655021086843)(128,0.789585079576582)(129,0.790085263487461)(130,0.78982667564133)(131,0.790288993581374)(132,0.789882233080459)(133,0.790154975346559)(134,0.790557179725233)(135,0.790326661436231)(136,0.790209622446119)(137,0.789868689551842)(138,0.789456609991847)(139,0.789591243228488)(140,0.789849098936001)(141,0.790030673331191)(142,0.789805310891841)(143,0.790248203397493)(144,0.790672082151784)(145,0.791141134834778)(146,0.79064382627176)(147,0.790921068058866)(148,0.791007325797586)(149,0.790737624874385)(150,0.790645661458395)(151,0.791083060114493)(152,0.79116143454087)(153,0.791276163922626)(154,0.791624349703189)(155,0.79230575111453)(156,0.792455908829726)(157,0.792743359173706)(158,0.792448379527117)(159,0.792669519243893)(160,0.792788302957822)(161,0.793290758269341)(162,0.793664620344962)(163,0.793756782114368)(164,0.793583409234677)(165,0.793758190390482)(166,0.793863013170631)(167,0.794004502505873)(168,0.793905866249839)(169,0.794232990697674)(170,0.794356612599543)(171,0.794362191432686)(172,0.794609947573462)(173,0.794634642538662)(174,0.794691640253369)(175,0.794625372850886)(176,0.794737370641624)(177,0.794865758488719)(178,0.795116024239571)(179,0.795244251058676)(180,0.795307542293187)(181,0.795495358008249)(182,0.795541522764291)(183,0.795722840483126)(184,0.795730463611612)(185,0.795510584063165)(186,0.795800419817983)(187,0.795885144836598)(188,0.796022412176817)(189,0.796518399580019)(190,0.796325322152724)(191,0.796211132160841)(192,0.796342031821587)(193,0.796455864751561)(194,0.79629082682916)(195,0.796184391766139)(196,0.796329480125971)(197,0.796591181265792)(198,0.796293080142346)(199,0.796818964717715)(200,0.796677138654502)(201,0.796842248711626)(202,0.797164649707069)(203,0.797725259916369)(204,0.797811845667324)(205,0.798034994038346)(206,0.798099885079282)(207,0.79845906016813)(208,0.798867772776934)(209,0.798900380399536)(210,0.798844669795883)(211,0.798737329879565)(212,0.798793862546171)(213,0.799078602243049)(214,0.79906424947435)(215,0.799164751023827)(216,0.798937609144436)(217,0.799069331460905)(218,0.799324099069413)(219,0.799837533105725)(220,0.79992165421313)(221,0.799889025324387)(222,0.80020858430334)(223,0.800350359747513)(224,0.800664152830379)(225,0.800638812927623)(226,0.800897407223211)(227,0.800889204358133)(228,0.801024861828661)(229,0.800951522922266)(230,0.800920499258861)(231,0.801001003653377)(232,0.801061640813429)(233,0.801282961025795)(234,0.801336961177242)(235,0.801419755325223)(236,0.801098910118162)(237,0.801433135231298)(238,0.801469125174531)(239,0.801750096596213)(240,0.801656996987554)(241,0.801655582296755)(242,0.801796724690802)(243,0.802042190963239)(244,0.8016302082541)(245,0.801925649083471)(246,0.802013342221162)(247,0.802065399137712)(248,0.801956189206628)(249,0.801902388182104)(250,0.802038163587979)(251,0.802022431216253)(252,0.802314690088256)(253,0.802399774543451)(254,0.802443605085009)(255,0.802244261288507)(256,0.802500267512121)(257,0.802657442147138)(258,0.80253124114619)(259,0.802731168286396)(260,0.802874579913359)(261,0.802827126547619)(262,0.802946708208816)(263,0.802757370623331)(264,0.802698628402589)(265,0.802783967250068)(266,0.802761631257178)(267,0.802815314165584)(268,0.802790954816742)(269,0.803037618245776)(270,0.80313014374934)(271,0.803330879892833)(272,0.80332522719787)(273,0.803170369459306)(274,0.803526593777527)(275,0.803509505253379)(276,0.803474119440693)(277,0.803512009744932)(278,0.803437343185858)(279,0.80338848050761)(280,0.803155415041256)(281,0.803511737592874)(282,0.803515194734991)(283,0.803421708178673)(284,0.803050029945999)(285,0.803014998470785)(286,0.803393633388256)(287,0.803120567382542)(288,0.802917605823788)(289,0.803324225550335)(290,0.80304698255204)(291,0.80290801831411)(292,0.802983638597433)(293,0.802846250096666)(294,0.802920800822037)(295,0.802959856196722)(296,0.803098587921895)(297,0.803021268559427)(298,0.80308997720758)(299,0.802978769085395)(300,0.802839408815519)(301,0.802921884487237)(302,0.802909991652091)(303,0.803109095466314)(304,0.8029297688395)(305,0.80299255088653)(306,0.802744417449956)(307,0.802880169778362)(308,0.803083045249659)(309,0.803436787646817)(310,0.803302605104284)(311,0.80313573753249)(312,0.803102214448788)(313,0.803044457142479)(314,0.803110259116603)(315,0.803090889257532)(316,0.803155741444464)(317,0.803194750708097)(318,0.802963777031516)(319,0.802893391917051)(320,0.802850747735386)(321,0.803014836001587)(322,0.802892157459009)(323,0.802799619610829)(324,0.802820554369542)(325,0.802848205429965)(326,0.802818135549908)(327,0.802746503894384)(328,0.802705795970986)(329,0.802563338768301)(330,0.802379186581103)(331,0.802400423195515)(332,0.802426078604143)(333,0.802565225093997)(334,0.802605091389394)(335,0.802613007476151)(336,0.802503095754783)(337,0.802087678742676)(338,0.802113810587608)(339,0.802285147032445)(340,0.802207993526192)(341,0.802076989328828)(342,0.8019640851555)(343,0.801854044970059)(344,0.801564894836281)(345,0.801628566483444)(346,0.801621917506308)(347,0.801465499815411)(348,0.801504636185719)(349,0.801630184787666)(350,0.801666924877231)(351,0.801799180887629)(352,0.801880712485718)(353,0.802153927497071)(354,0.802075022313067)(355,0.802271989078454)(356,0.802318698107103)(357,0.802113589712096)(358,0.802162254464823)(359,0.802355348482377)(360,0.802305956725485)(361,0.802025818127585)(362,0.802088922341051)(363,0.802465240749233)(364,0.80247286631134)(365,0.802612152834847)(366,0.802534680876288)(367,0.802696780567799)(368,0.802555786059948)(369,0.802805476601857)(370,0.803128730488452)(371,0.802987418646993)(372,0.803286893548005)(373,0.80325749749195)(374,0.803247758826705)(375,0.803095692455272)(376,0.803545411650296)(377,0.803441392237288)(378,0.804009577505652)(379,0.80353682990582)(380,0.803441343036227)(381,0.803389674955383)(382,0.803466006616272)(383,0.803592266202819)(384,0.803767861907048)(385,0.803579684257487)(386,0.803788753744124)(387,0.803929710451125)(388,0.803798069860253)(389,0.803879502006676)(390,0.803804639022127)(391,0.804016840085167)(392,0.803758716203069)(393,0.804074613167402)(394,0.804135768626746)(395,0.804363844716566)(396,0.804318107854142)(397,0.804240674528807)(398,0.804562491500052)(399,0.80472303528316)(400,0.804299797173694)
 %(401,0.804671277914254)(402,0.804725384924145)(403,0.804472715022562)(404,0.804540694190523)(405,0.804226424491949)(406,0.804303077581748)(407,0.804284345612604)(408,0.804650793998914)(409,0.80440141922554)(410,0.80463248638951)(411,0.804678237386147)(412,0.804663621507265)(413,0.804797995716433)(414,0.804856125797282)(415,0.804634585060983)(416,0.804832247430895)(417,0.805050521793574)(418,0.804910660428954)(419,0.804549234009776)(420,0.804714467078961)(421,0.804665246145017)(422,0.805068402164796)(423,0.805308501289937)(424,0.805314933665748)(425,0.805512434478923)(426,0.805309465104258)(427,0.804989555254631)(428,0.805002970683121)(429,0.805330698732683)(430,0.805400847614977)(431,0.80518647926669)(432,0.80521324013507)(433,0.805280574421263)(434,0.805065087057154)(435,0.804755294147723)(436,0.805281339583509)(437,0.805631928695759)(438,0.805480746297812)(439,0.80548265207383)(440,0.805616556948907)(441,0.805687885587196)(442,0.805739720164925)(443,0.805956446246017)(444,0.80582944783199)(445,0.805845111346834)(446,0.806024388704062)(447,0.805981835107865)(448,0.805870644167966)(449,0.806301096853321)(450,0.806499497775447)(451,0.806309585354173)(452,0.805925022824774)(453,0.805663661790341)(454,0.806308401079323)(455,0.807050763560703)(456,0.807401563262201)(457,0.807331261247305)(458,0.807132263496165)(459,0.807030191693694)(460,0.80652519919981)(461,0.806764376307559)(462,0.807466588806981)(463,0.807508441149277)(464,0.807111275321521)(465,0.808000375037404)(466,0.808569676744244)(467,0.808156775584051)(468,0.807664653327946)(469,0.809257966424202)(470,0.810276116198453)(471,0.81036220536228)(472,0.810222496139039)(473,0.81020520888998)(474,0.810202916606695)(475,0.809985923101991)(476,0.809803800423472)(477,0.809834375343949)(478,0.809679268773844)(479,0.80973737543561)(480,0.80918668370767)(481,0.811845029862432)(482,0.811845029862432)(483,0.811845029862432)(484,0.809590368980613)(485,0.809590368980613)(486,0.809590368980613)(487,0.809590368980613)(488,0.809590368980613)(489,0.80969696969697)(490,0.811565304087737)(491,0.811565304087737)(492,0.811565304087737)(493,0.814070351758794)(494,0.814070351758794)(495,0.814070351758794)(496,0.814070351758794)(497,0.814070351758794)(498,0.814070351758794)(499,0.814070351758794)(500,0.813905930470348)(501,0.813905930470348)(502,0.814738996929376)(503,0.814738996929376) 
};
\addlegendentry{\acl (max. ambiguity)};

\addplot [
color=red,
solid,
line width=1.0pt,
]
coordinates{
 (1,0.513344117720014)(2,0.513256788062484)(3,0.530485745015086)(4,0.544961166484278)(5,0.556143554509287)(6,0.56571878396994)(7,0.590552056035094)(8,0.614549136896883)(9,0.624166489793731)(10,0.629384073848215)(11,0.634021215820784)(12,0.639431832606356)(13,0.649719015774977)(14,0.656984760245589)(15,0.656014925919952)(16,0.657800034821719)(17,0.659795561167579)(18,0.660104781045858)(19,0.668860973811307)(20,0.675329982555553)(21,0.681891615349372)(22,0.684347601690366)(23,0.687086457517943)(24,0.690876567193529)(25,0.694099071091508)(26,0.698934115736617)(27,0.695459342576443)(28,0.697474969895627)(29,0.695729675989747)(30,0.693296330740974)(31,0.695046841857721)(32,0.697415158039003)(33,0.700368619964056)(34,0.701993472531515)(35,0.701611195717747)(36,0.702233399301074)(37,0.703023459896114)(38,0.704368919788303)(39,0.703733804723706)(40,0.703773937566155)(41,0.704144115018862)(42,0.704890440037144)(43,0.70548498223701)(44,0.705204100029761)(45,0.71014028608222)(46,0.713643985386067)(47,0.715759927604947)(48,0.717935518508681)(49,0.719384753366424)(50,0.719853567096066)(51,0.720732830409861)(52,0.721479948690359)(53,0.721216412786104)(54,0.721904852026278)(55,0.721696528856303)(56,0.721996344542908)(57,0.72112577291912)(58,0.722686101808959)(59,0.723595645788868)(60,0.724392601698766)(61,0.72547316064036)(62,0.726522877830443)(63,0.727433317271635)(64,0.728507853031803)(65,0.729156558154237)(66,0.730783563595874)(67,0.732347591682473)(68,0.732955069236812)(69,0.734605556336755)(70,0.733992479584292)(71,0.734162533799601)(72,0.736449121297154)(73,0.738132925560513)(74,0.73931281778172)(75,0.739984726433518)(76,0.740174203755218)(77,0.74111660765021)(78,0.741940356362371)(79,0.7425488545058)(80,0.742637149138935)(81,0.744945038217539)(82,0.746288716385372)(83,0.7472039123443)(84,0.747035922015857)(85,0.747037655821383)(86,0.74798237444105)(87,0.748530206015954)(88,0.748743406993525)(89,0.749170688587051)(90,0.750129638635045)(91,0.750931256814586)(92,0.752128322198528)(93,0.75289465968979)(94,0.753236730953326)(95,0.754132784958287)(96,0.754940576113655)(97,0.755478600856704)(98,0.756200028775517)(99,0.755997098716572)(100,0.756548209390204)(101,0.756426226387181)(102,0.756569005843824)(103,0.756606276082603)(104,0.756364470933772)(105,0.756273543736985)(106,0.756655358353484)(107,0.75666930099687)(108,0.756967896134558)(109,0.757660100036112)(110,0.758260766793509)(111,0.758732768373502)(112,0.759479396551959)(113,0.760038711481436)(114,0.761207152066034)(115,0.761931945772858)(116,0.76285616868749)(117,0.763094873380726)(118,0.763969577404055)(119,0.763954719916662)(120,0.763839125259735)(121,0.764798667176487)(122,0.765131196587051)(123,0.765817571703258)(124,0.76596798380758)(125,0.76610397642519)(126,0.767005458713181)(127,0.767480931604772)(128,0.768037895931408)(129,0.768475823018515)(130,0.768368543825035)(131,0.768587077299468)(132,0.768771834317725)(133,0.768814135441609)(134,0.768250389155282)(135,0.768243060910414)(136,0.768263098805819)(137,0.76862878526275)(138,0.768511277329863)(139,0.768357424041819)(140,0.769001995697337)(141,0.769333646734397)(142,0.769713027784293)(143,0.770078634652365)(144,0.770362096921991)(145,0.770754658079858)(146,0.7709325196709)(147,0.771643805727264)(148,0.772099412535974)(149,0.772607060116319)(150,0.772833496648088)(151,0.773252022044384)(152,0.773390972833516)(153,0.773664543013903)(154,0.774235686290255)(155,0.774676056216351)(156,0.775100796551275)(157,0.775518840106517)(158,0.775762121042123)(159,0.775974797227012)(160,0.776272308433174)(161,0.777169415715506)(162,0.777595378457171)(163,0.777570326286164)(164,0.777575128226267)(165,0.778048340921362)(166,0.778143988756024)(167,0.778507061226594)(168,0.778642935333015)(169,0.778929717802756)(170,0.77910838584192)(171,0.77922944672396)(172,0.779436730902739)(173,0.779496877854579)(174,0.779683828364804)(175,0.779930726881159)(176,0.780097955914293)(177,0.780380998151022)(178,0.780639309154419)(179,0.780347353217668)(180,0.780716894196206)(181,0.780656076447477)(182,0.780630284761581)(183,0.78075243559465)(184,0.780989018106654)(185,0.780769585982802)(186,0.780711416311267)(187,0.780343126954088)(188,0.779860796915213)(189,0.780339751669026)(190,0.780323961456227)(191,0.779931361305589)(192,0.779507011141765)(193,0.779320331734458)(194,0.779683580110303)(195,0.779886440636221)(196,0.779760248831815)(197,0.780076933219982)(198,0.780563277630689)(199,0.780578821034111)(200,0.780557950292003)(201,0.780771608074408)(202,0.780882995280933)(203,0.780878840673066)(204,0.780819529640152)(205,0.780981875952411)(206,0.781232744931429)(207,0.781396921538053)(208,0.781779838395473)(209,0.78229153173031)(210,0.782418717468032)(211,0.782373621955454)(212,0.782411634374137)(213,0.782615468988793)(214,0.783040462131404)(215,0.782635363503608)(216,0.782551973739661)(217,0.782808028448273)(218,0.782724455276015)(219,0.782846134024301)(220,0.782893959973127)(221,0.783158471651714)(222,0.783518778350772)(223,0.784174734148522)(224,0.784519118478196)(225,0.785293875567397)(226,0.785434150237669)(227,0.785968264578999)(228,0.786534821901871)(229,0.787093899741777)(230,0.787790090579467)(231,0.788220651693178)(232,0.788317857471152)(233,0.788837821197002)(234,0.788918931071532)(235,0.789040206011981)(236,0.789252122068856)(237,0.788980708786719)(238,0.789028692012128)(239,0.788935692836282)(240,0.788971282210502)(241,0.788891261338394)(242,0.788865737511112)(243,0.7882914676143)(244,0.788213617192006)(245,0.787492914360131)(246,0.787173641441799)(247,0.786803657730512)(248,0.786245946286731)(249,0.786051292449747)(250,0.785486283765138)(251,0.785302034972345)(252,0.785263824850365)(253,0.784898470065263)(254,0.784900964617097)(255,0.784717767285487)(256,0.784930790054476)(257,0.784867917814266)(258,0.784920611233471)(259,0.785155662782341)(260,0.785377761569441)(261,0.785600065213954)(262,0.785713814536434)(263,0.785637337246843)(264,0.786029033719153)(265,0.786088804197246)(266,0.786295050237208)(267,0.78668341061784)(268,0.787239416516102)(269,0.787331794418461)(270,0.787378148071131)(271,0.78755221606681)(272,0.787875051155546)(273,0.787978761087834)(274,0.788461312030971)(275,0.788703519526033)(276,0.78892057300825)(277,0.789140891500959)(278,0.789734507083798)(279,0.790219908585741)(280,0.790497558358436)(281,0.790953437802823)(282,0.791165118758537)(283,0.79152007273228)(284,0.791915014751443)(285,0.792063406110344)(286,0.792036920624413)(287,0.792285329343188)(288,0.792678109254519)(289,0.793105292929599)(290,0.79342225684677)(291,0.793614193607524)(292,0.793793948153124)(293,0.794084185411765)(294,0.794550578171603)(295,0.794906851205911)(296,0.795210418434566)(297,0.795520111677761)(298,0.796131074289252)(299,0.796301354153159)(300,0.796492711923028)(301,0.796513355794119)(302,0.796942053558469)(303,0.797507713500926)(304,0.797715711409746)(305,0.798071944518461)(306,0.79826647727301)(307,0.798424897849315)(308,0.798637171548086)(309,0.798875234396583)(310,0.798875360588769)(311,0.798588986550054)(312,0.799093046544569)(313,0.799181091195108)(314,0.799379343306969)(315,0.799651080579818)(316,0.799832321055524)(317,0.800124862209421)(318,0.800124067745537)(319,0.800390890517882)(320,0.800585300536005)(321,0.800445665781044)(322,0.800602356278232)(323,0.80059531526392)(324,0.801084261730588)(325,0.801338603901211)(326,0.801032836013487)(327,0.801321782888092)(328,0.801322753131599)(329,0.801248587724724)(330,0.801260416877659)(331,0.801163863296344)(332,0.80130395161524)(333,0.801307506247228)(334,0.801042613050992)(335,0.801277911135931)(336,0.801437352178348)(337,0.801338030414472)(338,0.801026389767468)(339,0.80094738088956)(340,0.800970395797975)(341,0.800850932582132)(342,0.801044600296664)(343,0.80173213168453)(344,0.801887257725976)(345,0.801907686544275)(346,0.801962270251037)(347,0.802184196584319)(348,0.802324053760829)(349,0.802425596303446)(350,0.802965748709204)(351,0.803076953527701)(352,0.803270350591772)(353,0.803306484529885)(354,0.803297620381856)(355,0.803396164651487)(356,0.803356856933947)(357,0.803424111457272)(358,0.803429992752656)(359,0.803746410824801)(360,0.804009231180141)(361,0.80402032787978)(362,0.804392385163674)(363,0.804400733152509)(364,0.804592891543008)(365,0.804659698303857)(366,0.80478922409697)(367,0.80486311212001)(368,0.805086999823969)(369,0.805127029905666)(370,0.804844424812117)(371,0.804734439669987)(372,0.804899367755142)(373,0.805069555830133)(374,0.805178266042604)(375,0.804775761808911)(376,0.80485268139698)(377,0.805337345761446)(378,0.805323849341977)(379,0.805215074482664)(380,0.805535567249703)(381,0.805529290091967)(382,0.805635342284389)(383,0.805668990896496)(384,0.806029350145038)(385,0.805670070522443)(386,0.806017471084777)(387,0.806000588673245)(388,0.806049670153003)(389,0.806022135009723)(390,0.806144028756231)(391,0.805816864552783)(392,0.806192464616772)(393,0.806163657811485)(394,0.806076754151669)(395,0.806376331675622)(396,0.80677249129823)(397,0.807075824971115)(398,0.806719818437806)(399,0.806795289907873)(400,0.806757783315879)
 %(401,0.806837610145633)(402,0.807059671036425)(403,0.807259334179965)(404,0.807311420516381)(405,0.807706862771351)(406,0.80777960397547)(407,0.807538445694323)(408,0.807669323738218)(409,0.807846961375789)(410,0.807572823262709)(411,0.807934728036891)(412,0.807901059028692)(413,0.807731652376934)(414,0.807636818860173)(415,0.807658781329086)(416,0.807315684953602)(417,0.807207933096022)(418,0.807050333341463)(419,0.80706666816704)(420,0.806915921305216)(421,0.806679188527797)(422,0.8071561822131)(423,0.807252380488225)(424,0.807531118650959)(425,0.807220372283706)(426,0.807585928132468)(427,0.807900787729135)(428,0.808028417805556)(429,0.808220825948764)(430,0.80795772305342)(431,0.808263219187886)(432,0.808087406822635)(433,0.808290157074773)(434,0.807977325359807)(435,0.808022062113897)(436,0.808244930997154)(437,0.808235480514092)(438,0.807686221529322)(439,0.807496077761295)(440,0.807024552536252)(441,0.806898623836272)(442,0.806933762070359)(443,0.807389685658045)(444,0.807536348119724)(445,0.807253678626247)(446,0.807411673420067)(447,0.8074016245957)(448,0.806931154783516)(449,0.80724362759338)(450,0.807241989413497)(451,0.807350365151566)(452,0.807109129521695)(453,0.807569068099335)(454,0.807060500359226)(455,0.80747775553012)(456,0.807562372466225)(457,0.807455613242733)(458,0.807879803103563)(459,0.807997181991402)(460,0.808366439983935)(461,0.807640651702096)(462,0.80868063027165)(463,0.808264704368084)(464,0.808786370782706)(465,0.808674079236971)(466,0.808704766830683)(467,0.809320636905571)(468,0.808987211089566)(469,0.808834856336109)(470,0.808678278676336)(471,0.809340023179237)(472,0.809070029375171)(473,0.809310599053359)(474,0.808843539222953)(475,0.809283622658284)(476,0.809037632315709)(477,0.809873621659112)(478,0.810289991357013)(479,0.809132733948133)(480,0.809095742773279)(481,0.808426289202597)(482,0.809297874320208)(483,0.809802964233026)(484,0.809956075395355)(485,0.810499817844718)(486,0.808544082801712)(487,0.809820567781589)(488,0.808893501026764)(489,0.808683344925729)(490,0.80917950817904)(491,0.809897793142431)(492,0.808831169640688)(493,0.808267970865923)(494,0.809054107442447)(495,0.809188706483577)(496,0.810994950386066)(497,0.810522558232216)(498,0.811151872077138)(499,0.810552649118888)(500,0.810501202239274)(501,0.809191269939173)(502,0.809732031757893)(503,0.810052328911417)(504,0.810957610205635)(505,0.812000485400459)(506,0.811213976028068)(507,0.811451699946957)(508,0.809385139983072)(509,0.813187935496749)(510,0.808673676611174)(511,0.811100098081566)(512,0.814119966717164)(513,0.809674759918029)(514,0.811693545246772)(515,0.815114249590641)(516,0.813345424074128)(517,0.81016668445372) 
};
\addlegendentry{\acl (max. variance)};

\addplot [
color=green!50!black,
solid,
line width=1.0pt,
]
coordinates{
 (1,0.521382017055908)(2,0.559081416518246)(3,0.596107376651465)(4,0.617322127786944)(5,0.639167581382985)(6,0.6493131319898)(7,0.658253245324589)(8,0.673627310716104)(9,0.685529657065636)(10,0.693954888884298)(11,0.700166298346551)(12,0.702706140639495)(13,0.705432680432723)(14,0.708472612377262)(15,0.70942223806579)(16,0.714018161042844)(17,0.717892395686364)(18,0.718036384664661)(19,0.719257464573251)(20,0.719484779779256)(21,0.72040215711543)(22,0.722406920157926)(23,0.722825657946276)(24,0.721832864356324)(25,0.722315131805177)(26,0.724205032138468)(27,0.724288154755685)(28,0.724769733896454)(29,0.725134677051116)(30,0.725823214303665)(31,0.726383414543211)(32,0.726656855127914)(33,0.726840183916179)(34,0.72837982479709)(35,0.728972396286175)(36,0.728906468519063)(37,0.72927553101172)(38,0.730366447741952)(39,0.730000186627139)(40,0.730201112966864)(41,0.730905720857823)(42,0.730428258215644)(43,0.730682246029072)(44,0.731856462317642)(45,0.731850245665137)(46,0.732534089789258)(47,0.732787713558368)(48,0.7337265892746)(49,0.735340782615071)(50,0.735349212797597)(51,0.73659224614518)(52,0.737857898840462)(53,0.738775105861378)(54,0.738215174733083)(55,0.738823152087964)(56,0.738716905150139)(57,0.738831957311835)(58,0.738979206301906)(59,0.739486764812979)(60,0.739955762471976)(61,0.740451574247303)(62,0.740990352125236)(63,0.741821053639209)(64,0.74214191430194)(65,0.741709822367961)(66,0.74260889097954)(67,0.742467822493957)(68,0.743322113932317)(69,0.744233454702824)(70,0.745566283359607)(71,0.745650990131633)(72,0.74637786055464)(73,0.745520449993233)(74,0.745246117803307)(75,0.745552285914462)(76,0.745040465842709)(77,0.745321929225646)(78,0.745352410188909)(79,0.74481288969248)(80,0.745107398589553)(81,0.745640047300309)(82,0.745146965133306)(83,0.746288978441743)(84,0.747784824375715)(85,0.747199062022111)(86,0.747289463717599)(87,0.748146291499573)(88,0.748985198771934)(89,0.750471111193482)(90,0.751132717595438)(91,0.751393071107561)(92,0.751596418385434)(93,0.751863077165711)(94,0.752876573025824)(95,0.75346681464341)(96,0.753944110032096)(97,0.754320889755752)(98,0.754592842427989)(99,0.754231774628747)(100,0.754292825691304)(101,0.754512634880907)(102,0.754697333470871)(103,0.755721573634281)(104,0.755372165577242)(105,0.756380015024852)(106,0.756916114490752)(107,0.757900907790721)(108,0.758768346293579)(109,0.759175996246572)(110,0.760211888304355)(111,0.760361430499167)(112,0.760319478374149)(113,0.76036666629961)(114,0.760177646560997)(115,0.760018915873829)(116,0.760946946323009)(117,0.76152007372772)(118,0.762106083392292)(119,0.761934782214532)(120,0.762769935649856)(121,0.763478463972477)(122,0.763659268551787)(123,0.764387422531966)(124,0.7641454565066)(125,0.764063603065475)(126,0.764377589832108)(127,0.764233301456506)(128,0.764587067324764)(129,0.764825219511459)(130,0.764864754330055)(131,0.765269592800979)(132,0.765620800303048)(133,0.76584157877672)(134,0.765724318033679)(135,0.766000905543313)(136,0.766347428030168)(137,0.767156123579431)(138,0.767559152249023)(139,0.767681544631681)(140,0.767940135935535)(141,0.767664828268148)(142,0.767565710128051)(143,0.767814265279394)(144,0.768440903134378)(145,0.768917411309806)(146,0.768859928881188)(147,0.768351809097527)(148,0.768178477586945)(149,0.768425594800622)(150,0.768736158710939)(151,0.768849261675851)(152,0.769161352057474)(153,0.769818752312879)(154,0.770153684730346)(155,0.769920455409835)(156,0.770171778622537)(157,0.770143849875029)(158,0.771161629096233)(159,0.771343841863512)(160,0.771068496794139)(161,0.772123333708075)(162,0.772653375189722)(163,0.772882358695477)(164,0.77268780197279)(165,0.772456916792107)(166,0.77265774026295)(167,0.772816263848417)(168,0.772778020568144)(169,0.773162133074832)(170,0.77329351622971)(171,0.773370472239285)(172,0.773565455294906)(173,0.773779626300864)(174,0.773183632630597)(175,0.77352197850507)(176,0.773436545512744)(177,0.774053587106377)(178,0.774462419497109)(179,0.774486296384539)(180,0.774804383944404)(181,0.774896179828832)(182,0.775569324061675)(183,0.77528478841861)(184,0.775902090054397)(185,0.776348929501594)(186,0.776630378978504)(187,0.776858873093434)(188,0.776992463003411)(189,0.777166473043235)(190,0.777369852563854)(191,0.777019986734293)(192,0.777273962698825)(193,0.77729681559563)(194,0.777033354393936)(195,0.777078787164626)(196,0.777238137332912)(197,0.777469133234742)(198,0.778001226734413)(199,0.778242260807994)(200,0.778496092087569)(201,0.778829296386286)(202,0.778844207591465)(203,0.778337620982173)(204,0.778673089255001)(205,0.779008497689048)(206,0.778968158421476)(207,0.779231017151695)(208,0.779366947666246)(209,0.779818080467599)(210,0.779584661118942)(211,0.779617885552185)(212,0.78039490779393)(213,0.780797861667421)(214,0.780991287475295)(215,0.781329836763376)(216,0.781793513057682)(217,0.781155301357755)(218,0.781613508907331)(219,0.781600925309133)(220,0.781709993901885)(221,0.781842042910896)(222,0.782498893141627)(223,0.782480929754036)(224,0.782597481097348)(225,0.782279468863809)(226,0.782171807498342)(227,0.782540265085163)(228,0.782794546861928)(229,0.78318976422026)(230,0.782855725328963)(231,0.782866892634198)(232,0.783328359691557)(233,0.783354136370238)(234,0.7834538259399)(235,0.783568423669656)(236,0.783732310456004)(237,0.783891027932853)(238,0.783835930496299)(239,0.78437525196728)(240,0.784481222332451)(241,0.784521045512415)(242,0.784628834673019)(243,0.784765674268703)(244,0.785385980566893)(245,0.785847738562155)(246,0.785511377609827)(247,0.785793172024649)(248,0.786048689489589)(249,0.786252508597886)(250,0.786485451892607)(251,0.786367430027612)(252,0.786544491222249)(253,0.786820917763451)(254,0.786899108791108)(255,0.787168374945873)(256,0.787695888248605)(257,0.787897511901559)(258,0.787870370764177)(259,0.788216359508436)(260,0.788480608510164)(261,0.788696571698856)(262,0.788861337699519)(263,0.789251661604623)(264,0.789042242023557)(265,0.789364456813568)(266,0.78975075472767)(267,0.789831415939149)(268,0.789927763682676)(269,0.789704916128555)(270,0.789322592634654)(271,0.789378848541743)(272,0.789511595424458)(273,0.789682815774545)(274,0.789647661097547)(275,0.790027393995511)(276,0.789977935671618)(277,0.790364819901294)(278,0.790428110329867)(279,0.790506677067857)(280,0.790341375330559)(281,0.790542297443431)(282,0.790581061707023)(283,0.790306220595129)(284,0.79055899300243)(285,0.790313403573598)(286,0.790444955826831)(287,0.790132159918793)(288,0.790613750702618)(289,0.790628575636078)(290,0.790646824827409)(291,0.791086252030294)(292,0.791119345946063)(293,0.791006223458591)(294,0.791021120728587)(295,0.791071491995829)(296,0.791418661659898)(297,0.791605077779544)(298,0.791862808861101)(299,0.792012829412815)(300,0.792299042696058)(301,0.792580561055573)(302,0.792768774139145)(303,0.792613555539573)(304,0.792862240816442)(305,0.792951539725708)(306,0.792877621339789)(307,0.793006165719485)(308,0.793114416146317)(309,0.793291945933031)(310,0.793371451514737)(311,0.793240830274914)(312,0.793541466221879)(313,0.793440271448877)(314,0.793548446871709)(315,0.79357935836059)(316,0.794187900171033)(317,0.794294505910944)(318,0.794208671698867)(319,0.794433227075924)(320,0.794458905246678)(321,0.794415289669031)(322,0.794649219798455)(323,0.794939756020249)(324,0.794866053589222)(325,0.794662779499451)(326,0.794926174391764)(327,0.794668679082852)(328,0.794694818442767)(329,0.79458366366982)(330,0.795301146798587)(331,0.795751268438403)(332,0.795778618752875)(333,0.795614614129246)(334,0.795752111510383)(335,0.795975690552505)(336,0.796286500405081)(337,0.796071859638794)(338,0.796041215677337)(339,0.796278075353445)(340,0.795858169305624)(341,0.795883987477482)(342,0.796301953554989)(343,0.796501849181235)(344,0.796786587611992)(345,0.796928594028313)(346,0.79676319738805)(347,0.79685024812429)(348,0.797091848000258)(349,0.796932372746924)(350,0.7968980758027)(351,0.797066946650243)(352,0.797009084441855)(353,0.797217097864371)(354,0.797025897951229)(355,0.797238067144087)(356,0.797194548101642)(357,0.797402727581825)(358,0.797567819775273)(359,0.797618491082114)(360,0.797648119643955)(361,0.797549081937455)(362,0.797833144710713)(363,0.797957409429842)(364,0.797946423426042)(365,0.797953110793384)(366,0.797720043060797)(367,0.797770380122071)(368,0.797712193982072)(369,0.797953963848501)(370,0.797988165685305)(371,0.798136835920062)(372,0.798089308667372)(373,0.798222375458103)(374,0.798335834616684)(375,0.798251418711912)(376,0.798252323790149)(377,0.798381438125663)(378,0.79868685630379)(379,0.798888573601072)(380,0.7990785394755)(381,0.799209158115647)(382,0.799846503602531)(383,0.800224015175283)(384,0.800191644206492)(385,0.800291280631883)(386,0.80051075492588)(387,0.800633097213142)(388,0.800779852782812)(389,0.800887126830158)(390,0.800665840177731)(391,0.800727631563792)(392,0.800977821673098)(393,0.800953391640207)(394,0.801195381882104)(395,0.801175694250169)(396,0.801161477191222)(397,0.801200502976066)(398,0.801473523881045)(399,0.801491179564583)(400,0.801496482297972)
 %(401,0.801486232621271)(402,0.801393604051229)(403,0.801514949989715)(404,0.801691111884417)(405,0.802173013123884)(406,0.802064228547715)(407,0.801969877049906)(408,0.801979009367251)(409,0.801931361616847)(410,0.802028150457064)(411,0.802171300090612)(412,0.802002213914279)(413,0.801981349873105)(414,0.801986791111262)(415,0.802195103269025)(416,0.80216301192931)(417,0.80236423481497)(418,0.802435235640819)(419,0.802693804243497)(420,0.802952174459865)(421,0.803114748221716)(422,0.803200040879338)(423,0.8033118904085)(424,0.803440586888248)(425,0.803628345773495)(426,0.803644384973112)(427,0.803713601751264)(428,0.803594573120845)(429,0.80335990916829)(430,0.803488105658125)(431,0.803476225681356)(432,0.803389051767701)(433,0.803249002104407)(434,0.803273378702169)(435,0.803305602084496)(436,0.803279504369432)(437,0.803618852321946)(438,0.803830371228688)(439,0.804251266493337)(440,0.804373053982169)(441,0.804528552170616)(442,0.804502231215344)(443,0.804450810088309)(444,0.804529499446036)(445,0.804679346491636)(446,0.804699151397621)(447,0.804838012177842)(448,0.804580979946201)(449,0.804777005340401)(450,0.804790089787797)(451,0.80467967205928)(452,0.804852831523688)(453,0.804857837488502)(454,0.804912592275738)(455,0.805094314765846)(456,0.805190463128)(457,0.805147285446608)(458,0.805516322610621)(459,0.80566158702225)(460,0.805714583182644)(461,0.805706399783863)(462,0.805652974062971)(463,0.805647723954847)(464,0.805458458482324)(465,0.805356670248757)(466,0.805315821446184)(467,0.805188254137097)(468,0.80521011309631)(469,0.805143074164505)(470,0.805287267409441)(471,0.805390996814234)(472,0.805507838852197)(473,0.80549374116046)(474,0.805403499222838)(475,0.805738387805532)(476,0.805592363730479)(477,0.80565397858189)(478,0.805744869946118)(479,0.805749475405383)(480,0.805864864124419)(481,0.80580208448235)(482,0.805796281962035)(483,0.805604633369281)(484,0.805452050140835)(485,0.805516693358886)(486,0.805722951977678)(487,0.805803874086615)(488,0.80569003529939)(489,0.805753029860021)(490,0.80576247934239)(491,0.805696740772564)(492,0.806235843805747)(493,0.806305422135335)(494,0.806602887561274)(495,0.806817251761766)(496,0.806887985102586)(497,0.807098335333526)(498,0.807177617656136)(499,0.807060805463415)(500,0.807196331530932)(501,0.807503344180065)(502,0.807200293872167)(503,0.80753629093562)(504,0.80747949518118)(505,0.807666258296668)(506,0.807656750697979)(507,0.80745147249276)(508,0.807213250130821)(509,0.807142585573616)(510,0.807258776913737)(511,0.807135415746856)(512,0.807363358597654)(513,0.807606762563234)(514,0.807250174237884)(515,0.807605009496022)(516,0.80753528188266)(517,0.80735256903694)(518,0.807413938095954)(519,0.807950371376517)(520,0.807488819948679)(521,0.807396146321216)(522,0.807695583068445)(523,0.807503982153807)(524,0.80794706266291)(525,0.808190467925821)(526,0.808039212636698)(527,0.808674302350195)(528,0.808750384054744)(529,0.808790022649057)(530,0.809882769974781)(531,0.809431937716463)(532,0.810933813841114)(533,0.809784790201809)(534,0.809957960895211)(535,0.810048509771944)(536,0.809359434895095)(537,0.810249867474719)(538,0.811284830039891)(539,0.810185446535241)(540,0.811778458150022)(541,0.810854815794032)(542,0.810854815794032)(543,0.810854815794032)(544,0.812714689063981)(545,0.812714689063981)(546,0.81108491858999)(547,0.81108491858999)(548,0.812548262548263)(549,0.814178033022254)(550,0.814178033022254)(551,0.814178033022254)(552,0.814780398396415)(553,0.814780398396415)(554,0.814364522137842)(555,0.814364522137842)(556,0.814364522137842) 
};
\addlegendentry{\acl (random)};

\addplot [
color=orange,
densely dotted,
line width=1.0pt,
]
coordinates{
 %(1,0.470216301241586)(2,0.470539009624162)
 (3,0.529885670577277)(4,0.529843588740064)(5,0.524114225843782)(6,0.5519040941856)(7,0.570614778741484)(8,0.597870621270671)(9,0.603900239804462)(10,0.605505889789711)(11,0.628012153331517)(12,0.635657902273215)(13,0.649414359157191)(14,0.645843347863916)(15,0.658689747947712)(16,0.664140357849611)(17,0.659407755089051)(18,0.66032728802541)(19,0.664065901611842)(20,0.661379778379683)(21,0.665654923156283)(22,0.665683641465307)(23,0.671935812122478)(24,0.675127384443139)(25,0.679248749498767)(26,0.682859587929769)(27,0.680034333799083)(28,0.679449506928499)(29,0.681464064140961)(30,0.683765967628062)(31,0.687507543237132)(32,0.688239264355242)(33,0.693167398743821)(34,0.693542460067808)(35,0.69437146586143)(36,0.694979151183891)(37,0.69440132837617)(38,0.688760986171016)(39,0.689787698386667)(40,0.687706569824396)(41,0.686786914423253)(42,0.687414142540112)(43,0.687625898890657)(44,0.685115567197897)(45,0.691250489029271)(46,0.696100058014429)(47,0.686539251745561)(48,0.689238260241043)(49,0.691980411096248)(50,0.695848791660666)(51,0.696044236412898)(52,0.705459178230364)(53,0.704214335294907)(54,0.708699392230451)(55,0.710071908938195)(56,0.708779080563867)(57,0.706750842471612)(58,0.708754144419424)(59,0.707655495309179)(60,0.709065761042602)(61,0.711381383036992)(62,0.711609955539584)(63,0.710853781121195)(64,0.712610143082108)(65,0.716975388572815)(66,0.716217562293735)(67,0.720426823461158)(68,0.724836311391687)(69,0.724436748215109)(70,0.718569332224268)(71,0.719057764032513)(72,0.720182828215946)(73,0.722481800118811)(74,0.724881353022124)(75,0.726812977394103)(76,0.727880686939858)(77,0.727635691692697)(78,0.727037366403538)(79,0.727379976975789)(80,0.726951536288008)(81,0.726877559383331)(82,0.7301967298349)(83,0.732714907001867)(84,0.732795517963862)(85,0.733831628722151)(86,0.736601476671502)(87,0.737954305432674)(88,0.744589620398604)(89,0.742465727200403)(90,0.742328539575824)(91,0.743091726054681)(92,0.743729501160541)(93,0.744507896923761)(94,0.745218527778483)(95,0.74677810907214)(96,0.746100857462296)(97,0.748712600387868)(98,0.74905888560449)(99,0.748522677261229)(100,0.748537038928545)(101,0.748537038928545)(102,0.748925808149851)(103,0.75318662974284)(104,0.75318662974284)(105,0.753448158881041)(106,0.753448158881041)(107,0.753448158881041)(108,0.753665196567531)(109,0.751990390020027)(110,0.750614630304116)(111,0.749732532255277)(112,0.749732532255277)(113,0.750352328972793)(114,0.750017407483551)(115,0.749510498864578)(116,0.752099716919845)(117,0.752594702522069)(118,0.75267708521145)(119,0.752338878573369)(120,0.75250152660266)(121,0.752392147378459)(122,0.752518845117274)(123,0.752544365638472)(124,0.752965235882305)(125,0.752746246463625)(126,0.753857071905308)(127,0.756276061809886)(128,0.757766464522244)(129,0.757057031807639)(130,0.758442385837553)(131,0.7583639239851)(132,0.760007490262129)(133,0.760390137114096)(134,0.760861164625323)(135,0.76132040367338)(136,0.76098341084395)(137,0.762058684455356)(138,0.7629191123568)(139,0.762919918803364)(140,0.765299518013368)(141,0.764613638580786)(142,0.764415856636005)(143,0.763832538548918)(144,0.764364522614635)(145,0.764451902218873)(146,0.763396273817034)(147,0.763464560322398)(148,0.763880205652354)(149,0.765766227936117)(150,0.767283980481775)(151,0.767416137163271)(152,0.767870942308001)(153,0.766173402688127)(154,0.76726774679801)(155,0.767695210281549)(156,0.767807719924637)(157,0.767253921248296)(158,0.767253921248296)(159,0.76800643901219)(160,0.766885882929815)(161,0.766885882929815)(162,0.767265257281685)(163,0.767265257281685)(164,0.769035549682838)(165,0.76935307002408)(166,0.769727430210032)(167,0.768863130319417)(168,0.768202368072665)(169,0.767431234107964)(170,0.767167227817826)(171,0.767430578738714)(172,0.766715082948637)(173,0.766351357915775)(174,0.766469492986171)(175,0.7689482925571)(176,0.769125182021664)(177,0.768661868877175)(178,0.76912184756066)(179,0.769445375935179)(180,0.769650161868317)(181,0.768892022283844)(182,0.768892022283844)(183,0.769713971764806)(184,0.769069769652524)(185,0.770778667770684)(186,0.770944682745329)(187,0.77186962242669)(188,0.77186962242669)(189,0.772122261820134)(190,0.772195645094862)(191,0.772231380091438)(192,0.772115010045484)(193,0.771776894967998)(194,0.772630518001133)(195,0.773017530817131)(196,0.77332077406715)(197,0.773064121400656)(198,0.772900101723903)(199,0.772557747885649)(200,0.772768234199896)
 %(201,0.773873851433923)(202,0.774067978386628)(203,0.77475347298454)(204,0.775368783133615)(205,0.775447207633604)(206,0.77592146380891)(207,0.775951185586505)(208,0.777454539845965)(209,0.778241588197198)(210,0.779255727148367)(211,0.779513763044891)(212,0.779758650741205)(213,0.780550417861294)(214,0.779925591230452)(215,0.779495473362572)(216,0.779277997546227)(217,0.779942886160672)(218,0.779805240147773)(219,0.779805240147773)(220,0.779766629103217)(221,0.779651433794808)(222,0.779811013218976)(223,0.780068356919014)(224,0.779700418003666)(225,0.780080635840455)(226,0.779244435784164)(227,0.780120947631595)(228,0.780114080282301)(229,0.780082345771865)(230,0.781975083841414)(231,0.781961880380686)(232,0.782379036235233)(233,0.782053157810205)(234,0.782120067086282)(235,0.781355099048743)(236,0.781929997913674)(237,0.78224182400581)(238,0.782205879805358)(239,0.782437895464852)(240,0.782437895464852)(241,0.782801498200489)(242,0.782573697602261)(243,0.783040563436511)(244,0.782914604634416)(245,0.782737724334084)(246,0.783046986528504)(247,0.782947054459542)(248,0.783091723575825)(249,0.782274943638159)(250,0.781712601099892)(251,0.781596792309114)(252,0.781305967914229)(253,0.781234919718752)(254,0.781129784411894)(255,0.781461174411217)(256,0.781606383824545)(257,0.781614433965423)(258,0.781492824961792)(259,0.781437692877715)(260,0.781365479349481)(261,0.781294940467306)(262,0.781320999027398)(263,0.781169157435113)(264,0.780776342427092)(265,0.782111153157115)(266,0.782149251892512)(267,0.782193026789485)(268,0.782104439610324)(269,0.782104439610324)(270,0.782217110181049)(271,0.781783387669991)(272,0.781783387669991)(273,0.782113546341914)(274,0.781849497588806)(275,0.782384068271032)(276,0.782133352673154)(277,0.781459059734556)(278,0.781347478721261)(279,0.781347478721261)(280,0.781570429589626)(281,0.781885617364965)(282,0.782149072233622)(283,0.782714234959149)(284,0.783175369267201)(285,0.782724119656247)(286,0.783293366025207)(287,0.783323502499206)(288,0.783250242558466)(289,0.78453605058045)(290,0.784462070569286)(291,0.784543603095589)(292,0.784587253574301)(293,0.784587253574301)(294,0.784715427873521)(295,0.784715427873521)(296,0.784748692531432)(297,0.784904285286174)(298,0.784938289075093)(299,0.7849038618964)(300,0.785094835153176)(301,0.785094835153176)(302,0.785473982467933)(303,0.785473982467933)(304,0.785029707831604)(305,0.785253551412423)(306,0.785011768634357)(307,0.784938320104627)(308,0.784905172894414)(309,0.784905172894414)(310,0.784905172894414)(311,0.784162155829584)(312,0.783717025660044)(313,0.783493182079225)(314,0.783493182079225)(315,0.783445303690333)(316,0.783173215030269)(317,0.783202912007597)(318,0.782090790984597)(319,0.782555859624025)(320,0.782482045884805)(321,0.782515176369129)(322,0.781995006578724)(323,0.782542265622323)(324,0.78246878373642)(325,0.782788647349647)(326,0.782747803615006)(327,0.78277823933792)(328,0.782871187499167)(329,0.783250242641352)(330,0.783637458586075)(331,0.783944721155907)(332,0.784617770714583)(333,0.785303328822156)(334,0.786599909529275)(335,0.786672617810338)(336,0.787002441893015)(337,0.787038696163815)(338,0.788449780755196)(339,0.788412303913255)(340,0.788841950100513)(341,0.789242926657905)(342,0.789566618340548)(343,0.789646967732683)(344,0.789646967732683)(345,0.789646967732683)(346,0.789498577575414)(347,0.789675218467861)(348,0.789745248045408)(349,0.789636562519011)(350,0.789636562519011)(351,0.789614950726949)(352,0.789763015818088)(353,0.789614950726949)(354,0.790251251017696)(355,0.790032385913078)(356,0.788945319530755)(357,0.788945319530755)(358,0.788945319530755)(359,0.788945319530755)(360,0.788634768616483)(361,0.78843667857387)(362,0.788400532282463)(363,0.788440951112443)(364,0.789201101756949)(365,0.789201101756949)(366,0.789163216958857)(367,0.789277272087271)(368,0.789312707405039)(369,0.789090432133826)(370,0.788719126129283)(371,0.788373084206139)(372,0.788849297202953)(373,0.788783150344722)(374,0.788078332027527)(375,0.786702161239923)(376,0.784499984371322)(377,0.784274477452486)(378,0.784126980502334)(379,0.784126980502334)(380,0.783872353096913)(381,0.783643736660626)(382,0.783674425038241)(383,0.784315236570686)(384,0.783941636633794)(385,0.783902523565876)(386,0.784318150089652)(387,0.785506029977422)(388,0.785578980476512)(389,0.785856135513443)(390,0.785439654575634)(391,0.785440361896203)(392,0.786002226024854)(393,0.786143971346927)(394,0.786255764595981)(395,0.786255764595981)(396,0.786150097359782)(397,0.785726402103585)(398,0.785639157534425)(399,0.785639157534425)(400,0.785639157534425) 
};
\addlegendentry{\var};

\end{axis}
\end{tikzpicture}%

%% This file was created by matlab2tikz v0.2.3.
% Copyright (c) 2008--2012, Nico Schlömer <nico.schloemer@gmail.com>
% All rights reserved.
% 
% 
% 
\begin{tikzpicture}

\begin{axis}[%
tick label style={font=\tiny},
label style={font=\tiny},
label shift={-4pt},
xlabel shift={-6pt},
legend style={font=\tiny},
view={0}{90},
width=\figurewidth,
height=\figureheight,
scale only axis,
xmin=0, xmax=400,
xlabel={Samples},
ymin=0.48, ymax=1,
ylabel={$F_1$-score},
axis lines*=left,
legend cell align=left,
legend style={at={(1.03,0)},anchor=south east,fill=none,draw=none,align=left,row sep=-0.2em},
clip=false]

\addplot [
color=blue,
solid,
line width=1.0pt,
]
coordinates{
 %(1,0.0593332678556568)(2,0.0885011364033723)(3,0.125594009599034)(4,0.186251237347874)(5,0.249298953385746)(6,0.313857649980424)(7,0.394142591237241)(8,0.462439862832394)(9,0.495493212954451)
 (10,0.523087752740701)(11,0.545464279810858)(12,0.560798656703801)(13,0.577236500809535)(14,0.590790931532552)(15,0.600814583778963)(16,0.610989313009376)(17,0.628473216604642)(18,0.646762567084228)(19,0.651876968628447)(20,0.66070305141687)(21,0.675107790899812)(22,0.682121311009016)(23,0.693446183616421)(24,0.698314707298168)(25,0.712014928389044)(26,0.723538536040338)(27,0.736775168459233)(28,0.742737206044871)(29,0.75014605393293)(30,0.75817022006342)(31,0.768115527994348)(32,0.776328664558191)(33,0.78518272660053)(34,0.789335316088713)(35,0.797833061224615)(36,0.799871068699421)(37,0.80037175824133)(38,0.804016176539319)(39,0.808575939091297)(40,0.811292342442157)(41,0.815621481257978)(42,0.819377530880306)(43,0.821331235516986)(44,0.822901293978292)(45,0.827673886211427)(46,0.830106462840583)(47,0.834079398266667)(48,0.836933410148022)(49,0.839877736421554)(50,0.84006115134056)(51,0.841449600408849)(52,0.841886353067657)(53,0.841107191946106)(54,0.841320947621956)(55,0.844233879068463)(56,0.845401850262575)(57,0.847245687394201)(58,0.848062652507167)(59,0.849580052416247)(60,0.850127459757452)(61,0.850517173083623)(62,0.851024864963397)(63,0.852993690285548)(64,0.852982131787645)(65,0.853739622834479)(66,0.855242797232592)(67,0.855857096365744)(68,0.85716764278526)(69,0.857329122594874)(70,0.858684642300948)(71,0.859405638531875)(72,0.860377843930983)(73,0.861717863092491)(74,0.863505320412036)(75,0.864165857741254)(76,0.864563804520451)(77,0.864754281204907)(78,0.867287011378939)(79,0.869233656795549)(80,0.870479696565074)(81,0.871425899782389)(82,0.871130826016893)(83,0.872156680325934)(84,0.873560034802775)(85,0.874368445314941)(86,0.875257046631701)(87,0.875459348419878)(88,0.876344671912977)(89,0.877132084806781)(90,0.878223859705832)(91,0.878707736158021)(92,0.879458499021736)(93,0.87959678933229)(94,0.880847579552201)(95,0.881344487355528)(96,0.881766575456473)(97,0.882501824854552)(98,0.88293210723146)(99,0.884523507299586)(100,0.88492985364559)(101,0.88448730661308)(102,0.886129421960464)(103,0.886196833000273)(104,0.887064798133515)(105,0.887044512347013)(106,0.888189143221376)(107,0.889496458990325)(108,0.889276744747511)(109,0.88982844718883)(110,0.889995593508877)(111,0.891081781415364)(112,0.891185964857918)(113,0.891136146906303)(114,0.891527257848245)(115,0.892238825425101)(116,0.893001053381706)(117,0.893071053511471)(118,0.892671568784232)(119,0.893032242543782)(120,0.893219458873105)(121,0.893401963217769)(122,0.893815407851279)(123,0.894516805465114)(124,0.895094734211327)(125,0.895628260289255)(126,0.896208349466981)(127,0.897393010257737)(128,0.898558464759819)(129,0.898935174761247)(130,0.89966490199106)(131,0.899797793346735)(132,0.900429043559002)(133,0.900313440133067)(134,0.901201573154071)(135,0.90157234701631)(136,0.902142533279449)(137,0.902139164770004)(138,0.902406920686919)(139,0.902496240816506)(140,0.90205416478253)(141,0.902497913701415)(142,0.902705953263437)(143,0.902430091960104)(144,0.903167006387965)(145,0.904016587651977)(146,0.905149558253207)(147,0.905380485784615)(148,0.905844692933435)(149,0.906646190091394)(150,0.906876953824072)(151,0.907357698721602)(152,0.908186397796489)(153,0.908635450043624)(154,0.908686700373262)(155,0.908994202164222)(156,0.909734907563671)(157,0.910171993845924)(158,0.910058086755718)(159,0.910314540088486)(160,0.911238981188709)(161,0.911333482821817)(162,0.911783017906521)(163,0.912056583283522)(164,0.912563279116935)(165,0.912736280947766)(166,0.913097416573952)(167,0.913366881969438)(168,0.913431378119809)(169,0.913602528261145)(170,0.913929230074887)(171,0.914176205426966)(172,0.91435509729969)(173,0.914619025848416)(174,0.915092710700031)(175,0.915275337220869)(176,0.915265898820358)(177,0.915736866785841)(178,0.915906016248003)(179,0.915947248324037)(180,0.915632817961986)(181,0.915924778394618)(182,0.91621410724839)(183,0.916590146467275)(184,0.916934469415358)(185,0.917492086475263)(186,0.917818483905822)(187,0.917926551280875)(188,0.918475563810668)(189,0.918830726948686)(190,0.918741141402175)(191,0.919060295319952)(192,0.919231730651608)(193,0.919275079965973)(194,0.91939476542586)(195,0.919199031220273)(196,0.919382930728754)(197,0.919904738557772)(198,0.920146539373317)(199,0.920729318944147)(200,0.921207585999597)(201,0.921845430948843)(202,0.921784294330795)(203,0.921728187233401)(204,0.921876228941444)(205,0.922155045551984)(206,0.92250121971582)(207,0.922650037362891)(208,0.922739406098669)(209,0.922736015005444)(210,0.923028498464677)(211,0.922842364118918)(212,0.922994635045799)(213,0.923498934587875)(214,0.92349540893825)(215,0.923534568968737)(216,0.923573347868012)(217,0.923853290487459)(218,0.923600992418778)(219,0.923706854042569)(220,0.923941256014876)(221,0.924175869567514)(222,0.92384977677611)(223,0.924023635633941)(224,0.924685808730536)(225,0.924940214182973)(226,0.925133676082814)(227,0.925189391440293)(228,0.92549696240415)(229,0.925973841370608)(230,0.926579040384174)(231,0.926780003714041)(232,0.926866649992787)(233,0.926717102545336)(234,0.926510970398054)(235,0.926538814418585)(236,0.92680117888838)(237,0.927182400122259)(238,0.927395861896146)(239,0.927707299759889)(240,0.927531334082556)(241,0.927850955875763)(242,0.92765737149363)(243,0.927749647430673)(244,0.928115251808715)(245,0.928549136098552)(246,0.929233222775594)(247,0.929141228000022)(248,0.929066263204636)(249,0.929279239063295)(250,0.929603495448539)(251,0.92964599247369)(252,0.930028958585935)(253,0.930309230500756)(254,0.930370224540058)(255,0.930218941466676)(256,0.930107120020274)(257,0.930277572965104)(258,0.930721942639419)(259,0.930997084865097)(260,0.93122193125316)(261,0.931199686336509)(262,0.931476961164285)(263,0.931754543982024)(264,0.93186449061451)(265,0.932268525492077)(266,0.932445142099443)(267,0.932308574757772)(268,0.93250761060007)(269,0.932597474457111)(270,0.932790767145219)(271,0.932960204743139)(272,0.933178035104494)(273,0.933358481634018)(274,0.933606114876895)(275,0.933988912212342)(276,0.934100869910112)(277,0.934357548327882)(278,0.934813321271827)(279,0.934831644031121)(280,0.934879241855376)(281,0.935221138201651)(282,0.935284891906707)(283,0.935708373877416)(284,0.935761768280621)(285,0.935850696406866)(286,0.936067651873453)(287,0.936015865303805)(288,0.936228611607816)(289,0.936601844023585)(290,0.93694803222189)(291,0.937120196375218)(292,0.937295986637288)(293,0.93751683269647)(294,0.937762270871366)(295,0.937962207097653)(296,0.937985114548433)(297,0.938022053302764)(298,0.938162761648201)(299,0.938140512270738)(300,0.938206068782213)(301,0.938368352844845)(302,0.938739697012242)(303,0.938497459190464)(304,0.938670542155782)(305,0.938848527208965)(306,0.938984188546409)(307,0.939277635647428)(308,0.939320146661175)(309,0.939481473255068)(310,0.93975600792081)(311,0.939713116299617)(312,0.939794647360032)(313,0.939977159660887)(314,0.940048087581501)(315,0.940327926200342)(316,0.940375794292925)(317,0.940339569582667)(318,0.94054884620303)(319,0.940802260814874)(320,0.940963156559455)(321,0.941056632326216)(322,0.941155088877648)(323,0.941160735853999)(324,0.941372494173771)(325,0.941269072273834)(326,0.941081138459973)(327,0.940962901747419)(328,0.940966128708303)(329,0.940901320172559)(330,0.941170845765125)(331,0.941238437289276)(332,0.941275358467623)(333,0.94143006239962)(334,0.941678055412342)(335,0.941787163414355)(336,0.941756173657734)(337,0.941678381591428)(338,0.941858334396373)(339,0.941855902308276)(340,0.941941221929707)(341,0.941971314709597)(342,0.942085598479823)(343,0.942351273170459)(344,0.942570722877559)(345,0.942414632764098)(346,0.942567452300653)(347,0.942690155412216)(348,0.942764430981928)(349,0.942786446573865)(350,0.942853802633363)(351,0.94288454172959)(352,0.942952250042877)(353,0.943063549576199)(354,0.942898425570895)(355,0.942913384691445)(356,0.942953877579519)(357,0.94284971544336)(358,0.942727949270267)(359,0.942732695204693)(360,0.942736349344491)(361,0.942944575812769)(362,0.943088015783225)(363,0.943111771375771)(364,0.943202534884229)(365,0.943349028196548)(366,0.94345647450512)(367,0.94371561044183)(368,0.943784074659706)(369,0.943993551229741)(370,0.944181405270393)(371,0.944208281812126)(372,0.944142737325012)(373,0.944304762689065)(374,0.944421115316828)(375,0.944630238936909)(376,0.944651039095423)(377,0.94473030983982)(378,0.944892698378952)(379,0.944751499049173)(380,0.944955357415329)(381,0.945099997554157)(382,0.945194526898984)(383,0.94520289409277)(384,0.945390246465832)(385,0.945587291485331)(386,0.945788793278568)(387,0.945760444925851)(388,0.945877326974083)(389,0.945811825143496)(390,0.945754156554277)(391,0.945866532395726)(392,0.94595857740698)(393,0.946025893815141)(394,0.946087461115078)(395,0.946132649130883)(396,0.946021159695956)(397,0.946260435265607)(398,0.946530704998107)(399,0.946630677388369)(400,0.946692127050075)%(401,0.946761966066368)(402,0.946885902931551)(403,0.947037866697926)(404,0.947041112716616)(405,0.947182422200582)(406,0.947342600593879)(407,0.947465384688166)(408,0.947618568290148)(409,0.947763768728011)(410,0.947781532398517)(411,0.947960173850956)(412,0.947948829082889)(413,0.948086957170492)(414,0.948078181036757)(415,0.948239179252767)(416,0.94824726622365)(417,0.94830677251661)(418,0.948434408216486)(419,0.948524443080585)(420,0.948655426363958)(421,0.948667716864677)(422,0.948748591722257)(423,0.948705762061674)(424,0.948752857712682)(425,0.948800894756381)(426,0.948881907667512)(427,0.949024809030286)(428,0.949075618968638)(429,0.949090015649029)(430,0.949230403388241)(431,0.949288214752782)(432,0.949573874307498)(433,0.949628357094246)(434,0.94974969763759)(435,0.949781105034392)(436,0.949907406338242)(437,0.949947887190298)(438,0.949932217441461)(439,0.950055590221281)(440,0.950133261538808)(441,0.950205259732822)(442,0.950387103768834)(443,0.950396170858419)(444,0.950404023407104)(445,0.950539053938297)(446,0.95059294041457)(447,0.950678855969714)(448,0.950687915540977)(449,0.950622060621704)(450,0.950600104598329)(451,0.95059793601971)(452,0.950681962791113)(453,0.950790212298321)(454,0.950774779995129)(455,0.950727901664781)(456,0.950756073567614)(457,0.950806856401758)(458,0.950815256313646)(459,0.950727665514104)(460,0.950845690944365)(461,0.950785347268107)(462,0.95100436116344)(463,0.951063466915246)(464,0.951138120707518)(465,0.951163628351912)(466,0.951168323232583)(467,0.951275827480009)(468,0.951230063189555)(469,0.951229305922705)(470,0.951264338105343)(471,0.951292538404177)(472,0.951395286355613)(473,0.951445191147927)(474,0.951378734393084)(475,0.951321942446147)(476,0.951289656699212)(477,0.951308799843606)(478,0.951283126394562)(479,0.95129170119417)(480,0.951255008852963)(481,0.951467902598444)(482,0.95155945424113)(483,0.951476381047745)(484,0.951485285722994)(485,0.951474983235163)(486,0.951445935983984)(487,0.951623698674064)(488,0.95174240673736)(489,0.951747348416896)(490,0.95149400911259)(491,0.95160951525817)(492,0.95179358316214)(493,0.951964265001244)(494,0.952052500750655)(495,0.952182164860081)(496,0.952189935370066)(497,0.952162627507968)(498,0.952411247131722)(499,0.952381803522969)(500,0.952439947603491)(501,0.952518392676124)(502,0.952640338670804)(503,0.952735102075297)(504,0.952823765781565)(505,0.9528645915306)(506,0.95296364618171)(507,0.953138532628729)(508,0.953360033388627)(509,0.953405035239143)(510,0.953492556468229)(511,0.953586421590726)(512,0.953694945767476)(513,0.953615888824424)(514,0.953591114605673)(515,0.953620337908901)(516,0.953535143126412)(517,0.95364851084879)(518,0.953740595564285)(519,0.953731140813506)(520,0.953764206249953)(521,0.953701038206195)(522,0.953770771754639)(523,0.953833832561926)(524,0.953815088838153)(525,0.953878522382953)(526,0.954013736711299)(527,0.954056359607251)(528,0.954140962947649)(529,0.954161764998771)(530,0.954206380597259)(531,0.954268391654145)(532,0.954272964948862)(533,0.954357362201933)(534,0.954409142119705)(535,0.954314541928979)(536,0.95437020055551)(537,0.954367094333557)(538,0.95449509146518)(539,0.954661686234033)(540,0.954597413672401)(541,0.954604515818731)(542,0.954712597497059)(543,0.954698439026819)(544,0.954879220250175)(545,0.954799359224056)(546,0.954815197234358)(547,0.954891389464451)(548,0.954825131280304)(549,0.954757855640844)(550,0.954653639655894)(551,0.954577919606973)(552,0.954583114091933)(553,0.954665930845387)(554,0.954611983009633)(555,0.954537239711912)(556,0.954422276231151)(557,0.954453289553886)(558,0.954498801898776)(559,0.954547549944137)(560,0.954620690113738)(561,0.954471822640599)(562,0.954564176277198)(563,0.954587056879115)(564,0.954673752691031)(565,0.954738190662326)(566,0.954865287713734)(567,0.95491683205677)(568,0.954960712934602)(569,0.955021649811748)(570,0.954832415624887)(571,0.954939009825565)(572,0.954970916188005)(573,0.955074191642901)(574,0.955006797031724)(575,0.954983672887818)(576,0.954994553049723)(577,0.955070102608433)(578,0.955018700460199)(579,0.955039815155413)(580,0.955125884833075)(581,0.955173314119176)(582,0.954630403266642)(583,0.954775850262588)(584,0.954952956067795)(585,0.955102409337145)(586,0.955013357341754)(587,0.955134846806347)(588,0.955067465987869)(589,0.955120251087899)(590,0.955187275173337)(591,0.955161387566285)(592,0.955143724058867)(593,0.955288405239344)(594,0.955316869838342)(595,0.955458080927616)(596,0.955539960723164)(597,0.955528815739411)(598,0.955593561851073)(599,0.955515559027233)(600,0.955348986214436)(601,0.955910691848189)(602,0.95600357192129)(603,0.956269083256715)(604,0.956356191089602)(605,0.956413182949117)(606,0.956577967206993)(607,0.957209391986985)(608,0.957266420308019)(609,0.957328726359294)(610,0.957688377983801)(611,0.957739069968121)(612,0.958158267884618)(613,0.958191383009515)(614,0.958093450286846)(615,0.958066506734566)(616,0.957994073295511)(617,0.957867553592119)(618,0.958116423813528)(619,0.958053825584035)(620,0.957944869636615)(621,0.957865151246589)(622,0.957923140224258)(623,0.958246299273929)(624,0.95821549787803)(625,0.958246364585472)(626,0.958353316047589)(627,0.958388016753341)(628,0.958611145095886)(629,0.95816006917143)(630,0.958398144785032)(631,0.959268821258663)(632,0.959267432339841)(633,0.959364020041614)(634,0.959356286805231)(635,0.958421794410548)(636,0.958152615728668)(637,0.958197441098762)(638,0.959149796557599)(639,0.959212030932364)(640,0.959291058612791)(641,0.959135150624776)(642,0.959276258660068)(643,0.959356544144967)(644,0.964763374156279)(645,0.964874736499691)(646,0.964763264411158)(647,0.965200251133971)(648,0.964714338142066)(649,0.964714338142066)(650,0.964955509397665)(651,0.965178698485693)(652,0.964937619422277)(653,0.964919507303331)(654,0.964807886060294)(655,0.961431226765799)(656,0.96094839609484)(657,0.961431226765799)(658,0.961315728515173)(659,0.961672473867596)(660,0.961431226765799)(661,0.961207897793264)(662,0.960743321718931)(663,0.960743321718931)(664,0.960315618472964)(665,0.960074280408542)(666,0.960556844547564)(667,0.960556844547564)(668,0.961038961038961)(669,0.961038961038961)(670,0.961761297798378)(671,0.961315728515173)(672,0.962430426716141)(673,0.962430426716141)(674,0.962430426716141)(675,0.962207280315326)(676,0.962207280315326)(677,0.962207280315326)(678,0.962928637627433)(679,0.962705582580496)(680,0.961279851611407)(681,0.961279851611407)(682,0.961279851611407)(683,0.959832830276294)(684,0.959832830276294)(685,0.960074280408542)(686,0.960297190619921)(687,0.960966542750929)(688,0.96144914073386)(689,0.96144914073386) 
};
\addlegendentry{\acl (max. ambiguity)};

\addplot [
color=red,
solid,
line width=1.0pt,
]
coordinates{
 %(1,0.0303830036977273)(2,0.0303121305492827)(3,0.0350294598656475)(4,0.0794623034870431)(5,0.0916530952116797)(6,0.116973932010854)(7,0.173900151741611)(8,0.263783102306656)(9,0.32076860950769)(10,0.353745969167309)(11,0.375578149668077)(12,0.40210264434372)(13,0.433229021082751)(14,0.462041803181201)(15,0.492667122025231)
 (16,0.513322568525395)(17,0.549351393785487)(18,0.564433581811395)(19,0.566535168783442)(20,0.570339420384865)(21,0.586409513636958)(22,0.594352817738173)(23,0.604536119420554)(24,0.623020626870941)(25,0.629288049893869)(26,0.624494606088274)(27,0.630003354358937)(28,0.646551181893358)(29,0.651881397439648)(30,0.657841859536606)(31,0.669981329596966)(32,0.683062423863273)(33,0.702156776426017)(34,0.699986239205436)(35,0.709519912637862)(36,0.724388186273677)(37,0.737199064491561)(38,0.743355664832357)(39,0.748879283339903)(40,0.753415775935313)(41,0.75697447461135)(42,0.75832144914681)(43,0.764945307627698)(44,0.770318667367749)(45,0.774386176037249)(46,0.777669797531404)(47,0.77875625014474)(48,0.784875367866415)(49,0.787307297795426)(50,0.789995666439679)(51,0.794579251123972)(52,0.798356614385867)(53,0.801480654505976)(54,0.80411781548767)(55,0.805984341532486)(56,0.807126861153959)(57,0.809762484608454)(58,0.811238339509344)(59,0.812981196264913)(60,0.814947166911822)(61,0.817777561020912)(62,0.819392479393009)(63,0.822459474553628)(64,0.823047006342367)(65,0.823390370293819)(66,0.8249341310988)(67,0.826627280002868)(68,0.829466313761634)(69,0.831409495169557)(70,0.834054012914043)(71,0.834409405147177)(72,0.835570257958125)(73,0.839080033921804)(74,0.841748836130753)(75,0.841862793427873)(76,0.841870155135089)(77,0.84318424383728)(78,0.842520694296955)(79,0.844165906363907)(80,0.846006751351637)(81,0.84695292324411)(82,0.847785320893021)(83,0.84916434993096)(84,0.850728553211471)(85,0.851996485076453)(86,0.85353721576608)(87,0.855249692196717)(88,0.855946433733054)(89,0.857000801684062)(90,0.858515963230026)(91,0.859538106963021)(92,0.859922437981986)(93,0.860762247037217)(94,0.860171958417264)(95,0.86220288864605)(96,0.862430088225272)(97,0.862827007749166)(98,0.863769988710904)(99,0.864938600179142)(100,0.865397931154032)(101,0.865920079442954)(102,0.866603468572063)(103,0.866671851215193)(104,0.868022012085659)(105,0.868287682536479)(106,0.869190974635544)(107,0.869351387299617)(108,0.87070958134333)(109,0.871305784552225)(110,0.872410889556269)(111,0.871940655971427)(112,0.871912303207089)(113,0.872473955056341)(114,0.872701928543219)(115,0.873891000688323)(116,0.875130922686813)(117,0.876532321991148)(118,0.876709933071455)(119,0.877424205995494)(120,0.87781572199115)(121,0.879885308863455)(122,0.880777489632733)(123,0.881841807948775)(124,0.882350130453704)(125,0.883111197823026)(126,0.883203595198409)(127,0.883018753721688)(128,0.883279132340166)(129,0.883077523379645)(130,0.883769016460733)(131,0.884157084345888)(132,0.883758509168128)(133,0.884661354347215)(134,0.884405404327175)(135,0.884858181841853)(136,0.885842879114844)(137,0.886288285435432)(138,0.886246839502624)(139,0.886839484686008)(140,0.8871851452753)(141,0.888204878351408)(142,0.888296750324546)(143,0.888472721950602)(144,0.889269006953798)(145,0.889206377225333)(146,0.889312815131968)(147,0.889434060826295)(148,0.889220868656063)(149,0.888459430621886)(150,0.888575946245646)(151,0.88910689531611)(152,0.889247433663961)(153,0.889198835942356)(154,0.888777147570507)(155,0.888423137543949)(156,0.889357629331425)(157,0.889348458991987)(158,0.889892969983303)(159,0.890326583973884)(160,0.890671782403475)(161,0.890532751521054)(162,0.890304628650873)(163,0.89116255810984)(164,0.892102348009176)(165,0.892018264872717)(166,0.891794031944458)(167,0.892416779049715)(168,0.892900802885992)(169,0.892948560771951)(170,0.892729687634021)(171,0.892980319551629)(172,0.893019875259948)(173,0.893033295675401)(174,0.893352397530902)(175,0.893075877540427)(176,0.893440477681087)(177,0.894312863072444)(178,0.894658445853638)(179,0.895648035480863)(180,0.896293039170314)(181,0.896669109000736)(182,0.897743726758401)(183,0.898091209984735)(184,0.898856394957367)(185,0.899140296138161)(186,0.899328323738281)(187,0.89928107395098)(188,0.899420019280662)(189,0.899775068437397)(190,0.899384572486266)(191,0.900117078454888)(192,0.900450834616815)(193,0.900645813201997)(194,0.901370595489043)(195,0.902092353876414)(196,0.902526607385214)(197,0.902840392090077)(198,0.903004735844058)(199,0.903319562917502)(200,0.90274791069035)(201,0.903219805101156)(202,0.903469554792303)(203,0.90399397997437)(204,0.904038162256417)(205,0.904238515352758)(206,0.904606386543825)(207,0.904867830929781)(208,0.904746298359301)(209,0.904710799948017)(210,0.904771961237927)(211,0.905066790224594)(212,0.906131467852466)(213,0.906475019258613)(214,0.906812132264367)(215,0.906957610867399)(216,0.90701030223489)(217,0.90732563581511)(218,0.907313345922617)(219,0.907721510143389)(220,0.908270646244681)(221,0.908486703377622)(222,0.908597334016805)(223,0.908928804763785)(224,0.909492392450202)(225,0.909311052125977)(226,0.909334601270847)(227,0.909565979676647)(228,0.909715076491898)(229,0.909790263029982)(230,0.910078817865517)(231,0.909863504511333)(232,0.910107867909308)(233,0.910508616886615)(234,0.910839042767545)(235,0.910805575533862)(236,0.911465224498522)(237,0.911263902941005)(238,0.91171873549738)(239,0.911690132263435)(240,0.912167696027618)(241,0.912333634360502)(242,0.912706873480538)(243,0.912570017253258)(244,0.912706046837715)(245,0.912617821215748)(246,0.913256714349191)(247,0.913618404818439)(248,0.91412217685419)(249,0.914091666023591)(250,0.914049686568834)(251,0.913995000933399)(252,0.91425808155148)(253,0.914837663789443)(254,0.915598377646498)(255,0.915704209907223)(256,0.915745862513976)(257,0.915791235270544)(258,0.915774084635545)(259,0.915800444768878)(260,0.915916786656885)(261,0.91620635819356)(262,0.916149533120485)(263,0.916224968388232)(264,0.916624851156531)(265,0.916738458078548)(266,0.91677898990921)(267,0.916758156522288)(268,0.916809669692836)(269,0.916829640169897)(270,0.917111072116916)(271,0.917162591803776)(272,0.917573249722372)(273,0.917709550837696)(274,0.917685320627226)(275,0.917741282552164)(276,0.917809410598023)(277,0.917947401248935)(278,0.917983588790081)(279,0.918580427625443)(280,0.918655538382588)(281,0.918638631430048)(282,0.918704949041982)(283,0.918947450994967)(284,0.919147794223825)(285,0.91929897556802)(286,0.91945158416834)(287,0.919898757229296)(288,0.91972876743938)(289,0.919794739229824)(290,0.919751645300653)(291,0.919822860584153)(292,0.920131399343284)(293,0.920390302274928)(294,0.920565287948108)(295,0.920722411491185)(296,0.920985299707202)(297,0.921343740621638)(298,0.921559906127707)(299,0.921749769767596)(300,0.922055741280395)(301,0.922246853713096)(302,0.922544045300795)(303,0.922654366086105)(304,0.922735334447887)(305,0.923017479199262)(306,0.923024044572688)(307,0.923072007098587)(308,0.923248613215009)(309,0.923360607001515)(310,0.92356622621132)(311,0.923868498358884)(312,0.92423785105576)(313,0.924291347768614)(314,0.924548439379681)(315,0.924400306483522)(316,0.924489682713545)(317,0.924555358353296)(318,0.924670267923827)(319,0.925042508504666)(320,0.925183207356306)(321,0.925464573678037)(322,0.925399163621935)(323,0.925537592662724)(324,0.925746582769386)(325,0.9256774318427)(326,0.925736295861941)(327,0.925544023252163)(328,0.92571860283529)(329,0.92578027627442)(330,0.925860858243194)(331,0.926214330671242)(332,0.926237917089283)(333,0.926572038459687)(334,0.926684590362265)(335,0.92670694751485)(336,0.926567005461187)(337,0.926672507210744)(338,0.926563282101235)(339,0.926556607042831)(340,0.926636525621436)(341,0.926701811954159)(342,0.926616950516803)(343,0.926669089025282)(344,0.926747898461046)(345,0.926984308753187)(346,0.927144194216072)(347,0.927260522554944)(348,0.927277832570465)(349,0.927558175362956)(350,0.928111678276835)(351,0.928188010736528)(352,0.928173413577519)(353,0.928287580869322)(354,0.928472962638984)(355,0.928533774295476)(356,0.928778811271551)(357,0.928760669273862)(358,0.928797025038642)(359,0.928858425745311)(360,0.929035001981813)(361,0.92907293940946)(362,0.929299834826417)(363,0.929387962712452)(364,0.929514547743707)(365,0.929597473605244)(366,0.929637941504939)(367,0.929878685019909)(368,0.930450079018834)(369,0.930626657912327)(370,0.930746462789421)(371,0.930814024649548)(372,0.930984198301917)(373,0.930980624121807)(374,0.931268882321519)(375,0.931601834322868)(376,0.931796202317033)(377,0.931762117318217)(378,0.931762595669849)(379,0.931808216631055)(380,0.931811667338271)(381,0.93183115507135)(382,0.931863414523162)(383,0.932227581505323)(384,0.932289667963741)(385,0.932188506110595)(386,0.932416724234079)(387,0.932543956782042)(388,0.932682152048939)(389,0.932883209641024)(390,0.933186452656581)(391,0.932918445512465)(392,0.933289745981932)(393,0.933483714598428)(394,0.933837668165754)(395,0.933850828746432)(396,0.934138424928051)(397,0.934108512826098)(398,0.934133743006308)(399,0.934183792514056)(400,0.934405344588864)%(401,0.934453549289971)(402,0.934608863282275)(403,0.934509440756846)(404,0.934706616675661)(405,0.93484153227529)(406,0.934875229432456)(407,0.935050171974078)(408,0.935209139117949)(409,0.935439902985791)(410,0.935732154740479)(411,0.935859810791725)(412,0.936007699665035)(413,0.936024327875374)(414,0.936076254864891)(415,0.936431255581605)(416,0.936651363851869)(417,0.936806660519378)(418,0.936979595306175)(419,0.937030728659588)(420,0.937245800984732)(421,0.937135478425331)(422,0.937417893336771)(423,0.937666264209437)(424,0.937728818974794)(425,0.937861976319255)(426,0.937957856768187)(427,0.937910974272322)(428,0.938056820035282)(429,0.938241941553951)(430,0.938354849508176)(431,0.938533188823929)(432,0.938645486406325)(433,0.938710563973246)(434,0.939018727442057)(435,0.938995898436016)(436,0.938927046514195)(437,0.938979515465738)(438,0.938930309149383)(439,0.939157750024243)(440,0.939163173789639)(441,0.939310303663126)(442,0.939201731135956)(443,0.939454610164863)(444,0.939465426826799)(445,0.939509380141659)(446,0.939562134788938)(447,0.939595412437191)(448,0.939533500558507)(449,0.939912192449895)(450,0.939952911765775)(451,0.940081545560668)(452,0.940182473392764)(453,0.940223237505862)(454,0.940393228706816)(455,0.940514759803515)(456,0.940572608539587)(457,0.940626325765409)(458,0.940888067561684)(459,0.940939002722495)(460,0.940941420233369)(461,0.940976023004159)(462,0.94083892064706)(463,0.941158368071368)(464,0.941204132323184)(465,0.941378788716834)(466,0.941681490848005)(467,0.941702644429862)(468,0.941865051735559)(469,0.9419200911072)(470,0.941936139170756)(471,0.94195435771424)(472,0.942086966469966)(473,0.942298134124071)(474,0.942341999681674)(475,0.942187831563078)(476,0.942043508372487)(477,0.941970333542377)(478,0.942075116014872)(479,0.941974841179368)(480,0.942001238627534)(481,0.942330150461257)(482,0.942212180715298)(483,0.942329634931676)(484,0.942477281328186)(485,0.942603080262764)(486,0.942607745723544)(487,0.942739081043369)(488,0.942931193238262)(489,0.943051086768098)(490,0.943179382830019)(491,0.943289906767742)(492,0.943423318950078)(493,0.94351681183949)(494,0.943708480477387)(495,0.943923830605078)(496,0.944001323490651)(497,0.944047311705362)(498,0.944078565807381)(499,0.944198181595946)(500,0.944209074652308)(501,0.944183697128479)(502,0.944149035013733)(503,0.944175482202527)(504,0.944063297673669)(505,0.944310387064555)(506,0.94434645995869)(507,0.944298180638366)(508,0.944477463968517)(509,0.944577681216792)(510,0.944670164190369)(511,0.944647185413811)(512,0.944787417360259)(513,0.944815426053801)(514,0.944979353783811)(515,0.945022430265116)(516,0.945026975920677)(517,0.945206472006391)(518,0.945215471683016)(519,0.94521191163764)(520,0.945188634731144)(521,0.945175192552093)(522,0.945168891268583)(523,0.945315104211703)(524,0.945445780378115)(525,0.945499077034002)(526,0.945535033510499)(527,0.945763938372782)(528,0.945768307149606)(529,0.945829537801215)(530,0.945820155725217)(531,0.946006734583323)(532,0.946322945952935)(533,0.946243079779319)(534,0.946336318480395)(535,0.946193199804523)(536,0.946224022472642)(537,0.946256419745138)(538,0.946483590217613)(539,0.94649655391275)(540,0.946475413421611)(541,0.946328260853479)(542,0.946506243796944)(543,0.946634794355958)(544,0.946716536338702)(545,0.946705239713724)(546,0.946825485363381)(547,0.94672089553961)(548,0.946799889752696)(549,0.946824845244552)(550,0.946810787498248)(551,0.94687098923615)(552,0.946965229308036)(553,0.947182689506521)(554,0.947143389977211)(555,0.947124997058218)(556,0.947187734237945)(557,0.947224052180042)(558,0.947264028657919)(559,0.947107037719718)(560,0.947222911390399)(561,0.947289094977631)(562,0.947398732761967)(563,0.947409321162308)(564,0.947460579892105)(565,0.947515272926481)(566,0.947498016705548)(567,0.947621520960083)(568,0.947545013013752)(569,0.947711379053202)(570,0.947853925473411)(571,0.947868948352626)(572,0.948080078133818)(573,0.948268446500057)(574,0.948330309023679)(575,0.948262875658183)(576,0.948466084991139)(577,0.948526218779715)(578,0.948618256915025)(579,0.948627210889691)(580,0.948564725654207)(581,0.94859381633724)(582,0.948554909264496)(583,0.948459948875366)(584,0.948543296981504)(585,0.948636224510279)(586,0.948603174111123)(587,0.948601230245954)(588,0.948669904452015)(589,0.948664051710361)(590,0.948668554217195)(591,0.948735395313108)(592,0.948842256277411)(593,0.948983311100737)(594,0.949157990994964)(595,0.949115753693524)(596,0.949214260148519)(597,0.949271968243264)(598,0.949134006334543)(599,0.949020003104545)(600,0.949100784769771)(601,0.949142273735471)(602,0.949163650939602)(603,0.949281114703251)(604,0.949218279881173)(605,0.949198972645773)(606,0.949224083865103)(607,0.94918593812189)(608,0.949271747367442)(609,0.949319348191257)(610,0.949530553657436)(611,0.949649703680463)(612,0.949698294520553)(613,0.949701824349669)(614,0.949709288657317)(615,0.949687915230888)(616,0.949655440113014)(617,0.949780552673132)(618,0.949884882635593)(619,0.949912831942327)(620,0.950011298383021)(621,0.950008628634107)(622,0.949945293898954)(623,0.950104770458992)(624,0.950109483158817)(625,0.949991423215783)(626,0.950221150267323)(627,0.950234430042104)(628,0.950276658453365)(629,0.950326182222148)(630,0.950389386025078)(631,0.950439976694774)(632,0.950545128317483)(633,0.950432038559165)(634,0.950482264410219)(635,0.950517786404129)(636,0.950485314280631)(637,0.950672327787636)(638,0.950599963493565)(639,0.950561342881741)(640,0.950697469972144)(641,0.950878443793302)(642,0.950937824833967)(643,0.950969214241376)(644,0.9510364963197)(645,0.951192227874512)(646,0.951156214740755)(647,0.951193223849748)(648,0.95115051667713)(649,0.951080619283967)(650,0.950950026047647)(651,0.951001307378779)(652,0.951080092148486)(653,0.951145835816713)(654,0.951082617228113)(655,0.951026552289156)(656,0.95093387505084)(657,0.950920733490975)(658,0.950994224566183)(659,0.951001902817613)(660,0.951091023994476)(661,0.951120831080001)(662,0.951239634365856)(663,0.951381522710106)(664,0.951663371013869)(665,0.951596653777175)(666,0.95154309941921)(667,0.951480918909956)(668,0.95145967744109)(669,0.951397151380859)(670,0.951482313000535)(671,0.951487401880707)(672,0.951600196183837)(673,0.951640096680326)(674,0.95164162637547)(675,0.951688708888728)(676,0.951721263555141)(677,0.951857070733047)(678,0.951864367206454)(679,0.951903150699643)(680,0.952045468953854)(681,0.952005089109328)(682,0.952013070584541)(683,0.952120649732272)(684,0.952135599825177)(685,0.952131420529772)(686,0.952100484940974)(687,0.95216867299375)(688,0.952042781530314)(689,0.952132875782576)(690,0.952190179557819)(691,0.952302161306214)(692,0.952170003808691)(693,0.952297733208977)(694,0.952266393995346)(695,0.952267689190885)(696,0.952370635825393)(697,0.952377149406282)(698,0.952513929811309)(699,0.952561501126293)(700,0.952463062750667)(701,0.952657466317223)(702,0.95289336250221)(703,0.953039097643713)(704,0.953077666798472)(705,0.953169320495126)(706,0.953421692693986)(707,0.953468731298069)(708,0.953586349185795)(709,0.953673459695009)(710,0.954116324104125)(711,0.954236702629304)(712,0.954263024883734)(713,0.954288672180274)(714,0.954496964940367)(715,0.954313174601279)(716,0.95443583927804)(717,0.954437369915476)(718,0.954532429776201)(719,0.954587761382766)(720,0.954620688580629)(721,0.954110318394448)(722,0.954155368395388)(723,0.954088242650527)(724,0.954092609350056)(725,0.954086725468759)(726,0.954217359729113)(727,0.954201598106269)(728,0.954453279657335)(729,0.953503580375063)(730,0.953709759769462)(731,0.953657237086357)(732,0.953550887325844)(733,0.953455874950296)(734,0.953240578877628)(735,0.953400942942391)(736,0.953119072452077)(737,0.953155589868345)(738,0.953079705598287)(739,0.953036832534082)(740,0.953223726398414)(741,0.953371445679625)(742,0.953343511324065)(743,0.953171173818346)(744,0.953170351851448)(745,0.953216671846146)(746,0.953159564740496)(747,0.953381401665094)(748,0.953448078112679)(749,0.95332431165965)(750,0.953161190834313)(751,0.953197725961218)(752,0.953188891064773)(753,0.953345196607201)(754,0.95290519250604)(755,0.951910611604841)(756,0.95200218979006)(757,0.952216830225022)(758,0.951985715176743)(759,0.952205221175234)(760,0.950253889370635)(761,0.948898857033108)(762,0.950488387919135)(763,0.949159750120985)(764,0.94926116964814)(765,0.949174558898408)(766,0.948981457027212)(767,0.948980131313909)(768,0.948830115370228)(769,0.948680056581403)(770,0.948775393685495)(771,0.948930392225655)(772,0.948682146168965)(773,0.949401984704198)(774,0.949331198625796)(775,0.949204941473496)(776,0.951066590523421)(777,0.951546543536733)(778,0.951793177118714)(779,0.951845269015464)(780,0.951896322225541)(781,0.952457178146386)(782,0.952545127466125)(783,0.952615279272159)(784,0.951955054948276)(785,0.951865974505784)(786,0.952475253980577)(787,0.953324718302686)(788,0.953463118945154)(789,0.953160010999245)(790,0.952688004802273)(791,0.952575703022758)(792,0.952815263021759)(793,0.95277260330743)(794,0.950960504454664)(795,0.949942307617322)(796,0.950636813190165)(797,0.951175665774389)(798,0.951572937068853)(799,0.952789646454837)(800,0.953082635524357)(801,0.952585442753179)(802,0.952016595145336)(803,0.950911540236159)(804,0.956104252400549)(805,0.956302905513612)(806,0.9558655385319)(807,0.956084172003659)(808,0.956322890464212)(809,0.956541628545288)(810,0.956541628545288)(811,0.956740672922866)(812,0.956740672922866)(813,0.956740672922866)(814,0.956740672922866)(815,0.957617411225659)(816,0.95763682161667)(817,0.95763682161667)(818,0.95763682161667)(819,0.95763682161667)(820,0.958276020174232)(821,0.957617411225659) 
};
\addlegendentry{\acl (max. variance)};

\addplot [
color=green!50!black,
solid,
line width=1.0pt,
]
coordinates{
 %(1,0.0575824793217458)(2,0.0946847371752363)(3,0.147190694063385)(4,0.179069001568382)(5,0.24951221279727)(6,0.282231935905728)(7,0.314971268419609)(8,0.354948630409644)(9,0.384326523084702)(10,0.402554862320253)(11,0.410118793305596)(12,0.442311742598938)(13,0.459376201106324)(14,0.485234828714078)
 (15,0.51224405883981)(16,0.538684218778185)(17,0.555308975685496)(18,0.567403668135982)(19,0.580535799242952)(20,0.601521347446574)(21,0.609518776262414)(22,0.619262580966327)(23,0.632006245715135)(24,0.646558989981971)(25,0.65212879387976)(26,0.662651366613697)(27,0.677440830201811)(28,0.683742397579191)(29,0.691553377653132)(30,0.691571567748511)(31,0.698104954327309)(32,0.708206414381651)(33,0.712377427912414)(34,0.716071337671865)(35,0.727024254735095)(36,0.726536068484527)(37,0.731921014553502)(38,0.73617577758331)(39,0.734935701244165)(40,0.743608198161603)(41,0.751544637027729)(42,0.75408619861406)(43,0.753173947931747)(44,0.756476113334387)(45,0.761936924750028)(46,0.763826734927637)(47,0.767076964264746)(48,0.772620583771101)(49,0.780961906783015)(50,0.784843139166046)(51,0.78389918455074)(52,0.786562044919994)(53,0.789308374349735)(54,0.78957083887928)(55,0.79401808295036)(56,0.794859644290157)(57,0.797420993707836)(58,0.800934307588691)(59,0.801992203296261)(60,0.802060287830768)(61,0.801775183521841)(62,0.805294049636485)(63,0.80849245585389)(64,0.810433797789565)(65,0.81091575885236)(66,0.814826444905855)(67,0.819539705192853)(68,0.821453289641611)(69,0.822583757926574)(70,0.825838429376709)(71,0.826897471742775)(72,0.829196523659337)(73,0.832061955998522)(74,0.833048813908103)(75,0.834395258264941)(76,0.833791268893764)(77,0.835022870034018)(78,0.836602795924313)(79,0.838567624659901)(80,0.839209300132736)(81,0.840138396056761)(82,0.840016891086463)(83,0.84133406601396)(84,0.843955312042629)(85,0.845973316233197)(86,0.84623890226211)(87,0.847207505255998)(88,0.846844951606412)(89,0.849478068935907)(90,0.84955878401383)(91,0.850080116200564)(92,0.850365729350645)(93,0.850145022287943)(94,0.850357311738623)(95,0.852256796575515)(96,0.852887195295581)(97,0.853554277269902)(98,0.855416644067345)(99,0.856595742890654)(100,0.85806371612492)(101,0.858505024250637)(102,0.85859682761795)(103,0.858767793285984)(104,0.859913764577278)(105,0.859130753782471)(106,0.861069899271616)(107,0.861563992312091)(108,0.862044228303761)(109,0.862261391446205)(110,0.863388447405863)(111,0.864219672763899)(112,0.865353190733642)(113,0.866630063774564)(114,0.867259133152757)(115,0.867655795248851)(116,0.868064511434175)(117,0.870040276303657)(118,0.871142763824089)(119,0.873602331702769)(120,0.873921093298925)(121,0.873554434431154)(122,0.87464210643763)(123,0.875005309532482)(124,0.877083408363901)(125,0.878030441447227)(126,0.879583963859563)(127,0.880185601365543)(128,0.880162183637576)(129,0.88053801294562)(130,0.881168451054289)(131,0.881541201595012)(132,0.882064491962527)(133,0.883699098308984)(134,0.883632319736571)(135,0.884272797963808)(136,0.884264603063081)(137,0.884788647338957)(138,0.885016679861847)(139,0.88497040140843)(140,0.885727225086383)(141,0.886495764841799)(142,0.886832544811929)(143,0.887116999138708)(144,0.886567075400433)(145,0.887802716429048)(146,0.88883315292753)(147,0.888898441848329)(148,0.889763232448205)(149,0.890079122636717)(150,0.890765535328647)(151,0.891211396135277)(152,0.891550136098743)(153,0.891421775676792)(154,0.892247330565265)(155,0.893080392312766)(156,0.893391806673444)(157,0.893853839693191)(158,0.894518616866704)(159,0.894576827018819)(160,0.894856616717013)(161,0.895004679816234)(162,0.894669839837227)(163,0.894835069483575)(164,0.895284872489388)(165,0.895479092899532)(166,0.895941170882699)(167,0.896885829832715)(168,0.896466488323128)(169,0.898210879564072)(170,0.898235492634437)(171,0.898158973292896)(172,0.898333732972446)(173,0.899048515749806)(174,0.899992715457329)(175,0.90068028009385)(176,0.901701640144223)(177,0.901560546116856)(178,0.902978690562304)(179,0.903273387526964)(180,0.903139472701987)(181,0.903227407106181)(182,0.90356901609501)(183,0.903510305619486)(184,0.903620081846967)(185,0.904125319472429)(186,0.904371003435524)(187,0.905005055188018)(188,0.905222758709546)(189,0.905318530170932)(190,0.905538046295556)(191,0.906211495011671)(192,0.906581253597599)(193,0.906612904681008)(194,0.906933362124168)(195,0.907631209535918)(196,0.907810995008743)(197,0.908047856454621)(198,0.90825998985345)(199,0.908444364262419)(200,0.908372321999319)(201,0.909094087044567)(202,0.909740328526838)(203,0.910532881120764)(204,0.911181425753867)(205,0.91134935629821)(206,0.911948459385756)(207,0.912818155468648)(208,0.912866678783621)(209,0.912770508485388)(210,0.912783466983998)(211,0.91334684228567)(212,0.914139110458234)(213,0.914346818042316)(214,0.914731721220247)(215,0.915110454299272)(216,0.915503906780641)(217,0.915433532288092)(218,0.915823553159184)(219,0.915847888180397)(220,0.916187591807861)(221,0.916501035453609)(222,0.916723397674302)(223,0.916846674601528)(224,0.917283404781988)(225,0.917548102693933)(226,0.917891669846271)(227,0.918041070615773)(228,0.918096690683596)(229,0.918765555607299)(230,0.919036148524374)(231,0.919484492390912)(232,0.919535747025859)(233,0.919482576614745)(234,0.919572034694675)(235,0.919993355810633)(236,0.919942643799606)(237,0.92043549974874)(238,0.920690140830881)(239,0.920896403389033)(240,0.920985239319277)(241,0.921010989353576)(242,0.921018530162731)(243,0.921161681683487)(244,0.921727670595306)(245,0.921735174902209)(246,0.92157466713712)(247,0.92156823391831)(248,0.921828455100803)(249,0.922088083027023)(250,0.922559846754232)(251,0.922367554018705)(252,0.92246107107425)(253,0.92263538855427)(254,0.922436226885639)(255,0.92270159673087)(256,0.92325303230198)(257,0.923049161428117)(258,0.923458697760731)(259,0.923786203382195)(260,0.924106209188331)(261,0.924123262054243)(262,0.923981670817738)(263,0.923933993005226)(264,0.924323039930494)(265,0.924243621477168)(266,0.924629200524405)(267,0.924643065827682)(268,0.924750981327942)(269,0.924979313487021)(270,0.925178747719815)(271,0.925329576042969)(272,0.925843438032156)(273,0.926036612816889)(274,0.925948253756138)(275,0.92607470818671)(276,0.926360031086508)(277,0.92656922619593)(278,0.926720095125632)(279,0.926833769728691)(280,0.926814855603951)(281,0.927043538193551)(282,0.927088763468579)(283,0.927095833161582)(284,0.927165668759819)(285,0.927201477894162)(286,0.927261017171018)(287,0.927636646154986)(288,0.927726676368704)(289,0.927685702753414)(290,0.927837840957551)(291,0.9281785724581)(292,0.928044897516087)(293,0.928100392038339)(294,0.928230812860728)(295,0.927986587897978)(296,0.927666675117975)(297,0.927784790446129)(298,0.928026376557495)(299,0.928361096332229)(300,0.928705889239307)(301,0.928727537261426)(302,0.928813915133324)(303,0.928997866951303)(304,0.929305456677117)(305,0.929771188744062)(306,0.930266330800388)(307,0.930290903191736)(308,0.930279726772241)(309,0.930506426679929)(310,0.930859310791114)(311,0.931359007347516)(312,0.931355292160711)(313,0.931773633811565)(314,0.931782277719246)(315,0.932050720266715)(316,0.931951280006472)(317,0.932171256988492)(318,0.932217742728355)(319,0.932022126502325)(320,0.932214023956846)(321,0.932337535308215)(322,0.932482531769598)(323,0.932498331039336)(324,0.932544554335316)(325,0.93276503526416)(326,0.932828494362702)(327,0.932913103467254)(328,0.933113179980995)(329,0.933363301608757)(330,0.933476297366729)(331,0.933859862256745)(332,0.933755005487689)(333,0.933808897603525)(334,0.933813351910351)(335,0.933970495090999)(336,0.934178342211127)(337,0.934148377410377)(338,0.934348043646908)(339,0.934581754006336)(340,0.934782808391664)(341,0.93474388987136)(342,0.934745698327033)(343,0.934959869325917)(344,0.935033993098908)(345,0.935235230310342)(346,0.935311108629813)(347,0.93528920671928)(348,0.935304476487267)(349,0.93536673148228)(350,0.935675260099917)(351,0.935895704122119)(352,0.9358629364444)(353,0.935757171016536)(354,0.935732818937727)(355,0.935465392225858)(356,0.935591036166277)(357,0.935939601635789)(358,0.936033460972412)(359,0.936104699969154)(360,0.936285582352666)(361,0.936679548192875)(362,0.936940976775274)(363,0.937157435425817)(364,0.937287843837372)(365,0.93757329283812)(366,0.93751549278483)(367,0.937715519855461)(368,0.937851624910938)(369,0.937933437833348)(370,0.938164933003358)(371,0.938309334995246)(372,0.93849131748238)(373,0.93855414453955)(374,0.938733188622557)(375,0.938896571975283)(376,0.939067232961007)(377,0.939171829631281)(378,0.939142442887372)(379,0.939500257459069)(380,0.939745469292782)(381,0.940008985369002)(382,0.940120270457652)(383,0.940335079328505)(384,0.940372028072637)(385,0.940308639892103)(386,0.940306324143501)(387,0.940452081945402)(388,0.940413705361666)(389,0.94045314031624)(390,0.940809148445896)(391,0.941121112614196)(392,0.941306190638264)(393,0.941411150852064)(394,0.941395907407064)(395,0.941354685994151)(396,0.94156062789403)(397,0.941568719835117)(398,0.941749517783612)(399,0.941748956111991)(400,0.941683259270541)%(401,0.941752058305321)(402,0.941794735943167)(403,0.941867967068757)(404,0.941968174389555)(405,0.942248546381343)(406,0.942223844809026)(407,0.94226059888829)(408,0.942469900061013)(409,0.942499460028044)(410,0.942629517602045)(411,0.942737034064121)(412,0.942927477440702)(413,0.94300884890513)(414,0.94301714933909)(415,0.943171763078982)(416,0.943260793594871)(417,0.943372809974293)(418,0.943265625199951)(419,0.943235587196326)(420,0.943289431270021)(421,0.943169248507866)(422,0.943215420385911)(423,0.943228429380687)(424,0.943373452067562)(425,0.94334047516021)(426,0.943397138140442)(427,0.943500458512994)(428,0.943446132094689)(429,0.943566217029107)(430,0.943639377190684)(431,0.943742360062965)(432,0.943853980440471)(433,0.94388617725771)(434,0.944016579976688)(435,0.944171391288895)(436,0.944173865037744)(437,0.944261055346755)(438,0.94441561585442)(439,0.944282628707418)(440,0.94444435615896)(441,0.944472654948519)(442,0.944538579132356)(443,0.944642123125224)(444,0.944754890980374)(445,0.944922324766722)(446,0.944828641360896)(447,0.94481689317482)(448,0.945163542889787)(449,0.945298726213113)(450,0.945460860239154)(451,0.945474430148832)(452,0.945451912368018)(453,0.945554162954539)(454,0.945720256785746)(455,0.945959362212113)(456,0.946092300755757)(457,0.946136026126919)(458,0.946206156922506)(459,0.946075517498381)(460,0.945974135549508)(461,0.946080373049234)(462,0.946091328889019)(463,0.946134354640491)(464,0.94618639293996)(465,0.946145602037304)(466,0.946244266817098)(467,0.946211656787143)(468,0.946352180899905)(469,0.946397944599878)(470,0.946617593170059)(471,0.946694221295348)(472,0.946735450888825)(473,0.946779843016686)(474,0.946965263804475)(475,0.946894270833376)(476,0.947008582023777)(477,0.947120821636098)(478,0.947284179653013)(479,0.947385113520546)(480,0.947432398911797)(481,0.947557256513653)(482,0.947632391498811)(483,0.94754276074738)(484,0.947523646607502)(485,0.947755645732841)(486,0.94773932849141)(487,0.947719254147657)(488,0.947798265409926)(489,0.947742939191728)(490,0.947882107902297)(491,0.947980032531158)(492,0.948019534463372)(493,0.947893692611333)(494,0.948065704979832)(495,0.948117770840385)(496,0.948153200243757)(497,0.948235972545709)(498,0.948311463030676)(499,0.948304404846396)(500,0.948394575908151)(501,0.948471877690037)(502,0.948385709403106)(503,0.948423971944616)(504,0.948466639035723)(505,0.948505594217958)(506,0.948511222661338)(507,0.94846154978958)(508,0.948410035023717)(509,0.948444646727075)(510,0.94841030853583)(511,0.948388554900156)(512,0.948367867938542)(513,0.948399609580307)(514,0.948523153035155)(515,0.948384651283777)(516,0.948383847672694)(517,0.948520826975614)(518,0.948547726749207)(519,0.948476464589229)(520,0.948643423104776)(521,0.948720904914556)(522,0.948832605103876)(523,0.949018972786368)(524,0.949119321515416)(525,0.949303545561817)(526,0.949445368305278)(527,0.949489442068973)(528,0.949566337171617)(529,0.949559812906804)(530,0.949485463770575)(531,0.949554714167837)(532,0.949686294491428)(533,0.949745818099593)(534,0.949757806819526)(535,0.949616689095366)(536,0.949670590385537)(537,0.949795177202474)(538,0.949734965571566)(539,0.949820429703687)(540,0.949997189207571)(541,0.949991103251627)(542,0.95005373778429)(543,0.950088612572538)(544,0.950118468464117)(545,0.95030813032472)(546,0.950384524523761)(547,0.950544227129311)(548,0.950697693034422)(549,0.950960930869156)(550,0.950946858751347)(551,0.951133101742898)(552,0.951194298744066)(553,0.951393511656739)(554,0.951350696148206)(555,0.951477775178291)(556,0.951535197914705)(557,0.951463536399383)(558,0.95154239552681)(559,0.951664462635529)(560,0.951773720863463)(561,0.951658664544991)(562,0.951736835820377)(563,0.951837471902788)(564,0.951956184348674)(565,0.951914493150301)(566,0.951953502088811)(567,0.952005943688247)(568,0.952028672428626)(569,0.952072526111376)(570,0.952236971383312)(571,0.952187919690091)(572,0.952145197593139)(573,0.952286451344313)(574,0.952157278033308)(575,0.952133491577862)(576,0.952114407249573)(577,0.952253816431638)(578,0.952314005462586)(579,0.952245946065762)(580,0.952176537939641)(581,0.952176401870651)(582,0.952296205647776)(583,0.9522548715068)(584,0.952119499391585)(585,0.952164273705992)(586,0.952183425039683)(587,0.952279883457476)(588,0.952324093341983)(589,0.952376986143489)(590,0.952497334012405)(591,0.952470627777338)(592,0.952434695975133)(593,0.952527907656069)(594,0.952533923658725)(595,0.952520016678516)(596,0.95249958336379)(597,0.95252851960564)(598,0.95273481984981)(599,0.952793280926192)(600,0.95280889457286)(601,0.952804242118653)(602,0.952907152634682)(603,0.952879304154515)(604,0.95300383768426)(605,0.952969451746771)(606,0.952985090423628)(607,0.952928042887412)(608,0.952950166646992)(609,0.952956993921624)(610,0.953002763355413)(611,0.953092134345941)(612,0.953053190837768)(613,0.953182439869998)(614,0.953310802724686)(615,0.953392720166522)(616,0.953509970972177)(617,0.95354960915889)(618,0.95354793229474)(619,0.953595525436157)(620,0.953480486265691)(621,0.953512574048863)(622,0.95346318549478)(623,0.953538787041946)(624,0.953607068121137)(625,0.953799018187673)(626,0.953907135145334)(627,0.954037562751955)(628,0.953972067612268)(629,0.954032391757487)(630,0.953933388238873)(631,0.95399737948564)(632,0.953994027921668)(633,0.954040458905395)(634,0.954170007129442)(635,0.954065170140593)(636,0.954147357502159)(637,0.954235626821369)(638,0.954298057172112)(639,0.954438923165978)(640,0.954406579866675)(641,0.954444151863643)(642,0.954591085559516)(643,0.954651774362793)(644,0.954679203918019)(645,0.954786410574928)(646,0.954825269711898)(647,0.95494134206496)(648,0.954782568046111)(649,0.954821751688114)(650,0.954820271061645)(651,0.954744457479936)(652,0.954936503003974)(653,0.955107075815491)(654,0.955120030408434)(655,0.955183539386532)(656,0.955229583881557)(657,0.955248839876751)(658,0.95540490903514)(659,0.955533162821834)(660,0.955331545737229)(661,0.955251677547596)(662,0.955283645928144)(663,0.95530246295833)(664,0.955314684159047)(665,0.955242526449426)(666,0.955239155702543)(667,0.95525840308352)(668,0.955225708082099)(669,0.955740987798701)(670,0.955816144469887)(671,0.955797246570148)(672,0.955624348485713)(673,0.955644579545309)(674,0.955312331598478)(675,0.955354750181526)(676,0.95536038081838)(677,0.955286782586576)(678,0.95530638352062)(679,0.955379214861706)(680,0.955524507202675)(681,0.955622967441704)(682,0.955571391307088)(683,0.955609188854197)(684,0.95608785750768)(685,0.956049275822637)(686,0.956189552717442)(687,0.956205082152312)(688,0.956256494814484)(689,0.95600105315919)(690,0.956186807690527)(691,0.956183977787278)(692,0.956126965384216)(693,0.956230359212872)(694,0.956128699835135)(695,0.956233439492458)(696,0.956244452832522)(697,0.956497477604088)(698,0.956434827992612)(699,0.956531240558393)(700,0.956475839413718)(701,0.956861861488523)(702,0.956900797268415)(703,0.956908446617331)(704,0.957043733439108)(705,0.957165043971443)(706,0.95725536680696)(707,0.957049639791534)(708,0.956644677271821)(709,0.956575157363027)(710,0.956515684586147)(711,0.956216302033728)(712,0.955738646686318)(713,0.956047355026834)(714,0.955975874547265)(715,0.95602956198232)(716,0.955991677800968)(717,0.956124412508371)(718,0.955952932201644)(719,0.95587650230021)(720,0.956011933065596)(721,0.95569595770747)(722,0.955694830009489)(723,0.955662951882269)(724,0.955808596774936)(725,0.955797801265905)(726,0.955962923068535)(727,0.955817376848568)(728,0.955908826819144)(729,0.955918790173785)(730,0.955193780406043)(731,0.954891277588648)(732,0.955106813020782)(733,0.9548163482523)(734,0.954471133180319)(735,0.954268452613563)(736,0.954275548358478)(737,0.952987262538898)(738,0.953260679528817)(739,0.953576425283671)(740,0.953463183657674)(741,0.954464320818109)(742,0.954309273318492)(743,0.954363286804187)(744,0.953730971905364)(745,0.954399214628423)(746,0.95458748585418)(747,0.95458594784021)(748,0.954465182426702)(749,0.954465940116001)(750,0.955298219692658)(751,0.955411117902557)(752,0.955403154666512)(753,0.954813967514323)(754,0.954811113950649)(755,0.954898769754267)(756,0.954420662936712)(757,0.952682840153709)(758,0.95712428292055)(759,0.956632992679462)(760,0.956389322213254)(761,0.956389322213254)(762,0.956622661490816)(763,0.956744468284322)(764,0.95615671641791)(765,0.955933784098857)(766,0.955710955710956)(767,0.955710955710956)(768,0.955446699323536)(769,0.955202986467569)(770,0.954449894884373)(771,0.954896003739191)(772,0.954471165071212)(773,0.954471165071212)(774,0.954471165071212)(775,0.954980172614882)(776,0.954492415402567)(777,0.954248366013072)(778,0.954917075449661) 
};
\addlegendentry{\acl (random)};

\addplot [
color=orange,
densely dotted,
line width=1.0pt,
]
coordinates{
 %(1,0.0901587493534934)(2,0.0900748012020438)(3,0.0892803395326255)(4,0.0916412503137695)(5,0.090174888902831)(6,0.0897377998186099)(7,0.244312184953591)(8,0.32590727777754)(9,0.334411150495897)(10,0.319693879109412)(11,0.327767638081797)(12,0.33311298878122)(13,0.38207957285824)(14,0.402520144495342)
 (15,0.608258758487151)(16,0.611728701676047)(17,0.606388106239489)(18,0.571588091134913)(19,0.571036571904162)(20,0.580514653523327)(21,0.588676674810184)(22,0.58569463507376)(23,0.585207948636001)(24,0.596390146633837)(25,0.613755996445539)(26,0.617650653381088)(27,0.618656292023741)(28,0.628527218258956)(29,0.647178366588846)(30,0.652453047271827)(31,0.650713777889618)(32,0.651010566914916)(33,0.660096799474086)(34,0.665316280221752)(35,0.664282208788561)(36,0.675854884610198)(37,0.691417350942688)(38,0.694514373084067)(39,0.699032801957312)(40,0.710043724154461)(41,0.710991032414557)(42,0.711521164612414)(43,0.704605947535149)(44,0.728618790115421)(45,0.727708895157445)(46,0.730682570838418)(47,0.731421989868023)(48,0.734054405099678)(49,0.732372178847209)(50,0.73303511683211)(51,0.733983237820807)(52,0.734090544134287)(53,0.734379105796785)(54,0.731456524612405)(55,0.730422979780398)(56,0.740568674503316)(57,0.746398112074194)(58,0.75464566978456)(59,0.748491976269344)(60,0.746304003817651)(61,0.752221488296789)(62,0.743887213203617)(63,0.74935515459196)(64,0.752853255484232)(65,0.7531776936143)(66,0.75417506225466)(67,0.753725038477741)(68,0.757601156531315)(69,0.765214924923415)(70,0.765125939303927)(71,0.774272784403591)(72,0.773519012900117)(73,0.780035259598228)(74,0.781552060056605)(75,0.791926213987349)(76,0.791940296555778)(77,0.790508254457715)(78,0.791750194648075)(79,0.792521634266714)(80,0.793611949532884)(81,0.794120199166299)(82,0.794010233427442)(83,0.79317576694935)(84,0.794520736797696)(85,0.794835555719273)(86,0.795376180619894)(87,0.794298834925929)(88,0.795149437803164)(89,0.794008693802113)(90,0.80054306350594)(91,0.802750359683539)(92,0.80310635738028)(93,0.804861269521311)(94,0.81080872231824)(95,0.810589270595173)(96,0.800251268659579)(97,0.800431558217991)(98,0.801304072328396)(99,0.80275171808572)(100,0.802980109259287)(101,0.803035903440169)(102,0.803533679847461)(103,0.807736625670996)(104,0.808122797880837)(105,0.813505106851149)(106,0.813748956308596)(107,0.815885630922716)(108,0.808816755865177)(109,0.809749281802988)(110,0.813556765890864)(111,0.815207482367719)(112,0.8141528490058)(113,0.814485573196576)(114,0.816094980708252)(115,0.816193494375252)(116,0.815551167702579)(117,0.817505090603353)(118,0.817739207218794)(119,0.82459259208667)(120,0.825321651035014)(121,0.809134152550877)(122,0.809697815511757)(123,0.81066148100115)(124,0.809940762307051)(125,0.80940226664585)(126,0.807994486023634)(127,0.808924726251633)(128,0.809412529544586)(129,0.809111243626005)(130,0.808998239105329)(131,0.811122643719361)(132,0.806748198802564)(133,0.806196198258469)(134,0.808619623827921)(135,0.808332290029043)(136,0.809532264456691)(137,0.80959405177475)(138,0.812465869444411)(139,0.812050694286802)(140,0.813070280392057)(141,0.812377473147647)(142,0.810420142750837)(143,0.811594550359173)(144,0.811754879983867)(145,0.813036208609042)(146,0.813075852196898)(147,0.813007549618121)(148,0.813093477681436)(149,0.813288251134942)(150,0.813314546258104)(151,0.813801533944215)(152,0.824602809401907)(153,0.823458106305242)(154,0.823458106305242)(155,0.823585460351445)(156,0.823596657414797)(157,0.830891552533514)(158,0.830884733709235)(159,0.831002511964206)(160,0.830456573864463)(161,0.82938890543605)(162,0.830207920097877)(163,0.829805037446376)(164,0.831404972039117)(165,0.834623229691196)(166,0.835672448048064)(167,0.835575241888284)(168,0.835614797054166)(169,0.835702202382753)(170,0.832140252632996)(171,0.832140252632996)(172,0.831609172478952)(173,0.831946489180025)(174,0.830949443892708)(175,0.833234575805219)(176,0.831894945658354)(177,0.834222616055862)(178,0.834474304286105)(179,0.834213520883376)(180,0.835541279764748)(181,0.835470521908925)(182,0.834984957517253)(183,0.835834606269014)(184,0.83512487552642)(185,0.835224391467475)(186,0.835053386634483)(187,0.837403942638783)(188,0.837254530910837)(189,0.838412381907423)(190,0.838674462808652)(191,0.838762371632351)(192,0.838814092585494)(193,0.840441296285503)(194,0.840284277210241)(195,0.841563528833408)(196,0.849437443217071)(197,0.847635820759868)(198,0.848065067243126)(199,0.848065067243126)(200,0.847467325688966)(201,0.847549506835145)(202,0.847560839910238)(203,0.851626346084979)(204,0.852244648860034)(205,0.851334365922362)(206,0.852004075265695)(207,0.853160066192223)(208,0.852759416777195)(209,0.855870886934837)(210,0.855870886934837)(211,0.854606741986398)(212,0.855235612209839)(213,0.857156463722128)(214,0.859736604935877)(215,0.859534791006872)(216,0.859845425064323)(217,0.860055228073524)(218,0.859793708640376)(219,0.859714407860197)(220,0.859995172259154)(221,0.86032160751335)(222,0.869749947417862)(223,0.871154304704409)(224,0.869996661089025)(225,0.870146922420246)(226,0.870719409724317)(227,0.869775026539416)(228,0.869675967599347)(229,0.869645000354433)(230,0.871383577685595)(231,0.871729289654203)(232,0.871207221287355)(233,0.872214881879733)(234,0.872214881879733)(235,0.872434769380153)(236,0.873995952420965)(237,0.87397592056098)(238,0.874396640868158)(239,0.874513303523923)(240,0.874410189569896)(241,0.876244629085262)(242,0.876076978291483)(243,0.875446015285087)(244,0.873523052078398)(245,0.873671005006119)(246,0.873430626471803)(247,0.875661283978093)(248,0.875661283978093)(249,0.875723713155445)(250,0.879121176155718)(251,0.878987710828466)(252,0.879597867520918)(253,0.881853556189351)(254,0.881339234038547)(255,0.881474906339681)(256,0.882086399457522)(257,0.882224286431256)(258,0.882116130270497)(259,0.882884950049555)(260,0.883305444765879)(261,0.883250689349232)(262,0.882271361118177)(263,0.883140127745891)(264,0.883539930278763)(265,0.884825311159278)(266,0.884422396741754)(267,0.887536788599842)(268,0.886997539040478)(269,0.886235064620744)(270,0.886120515207188)(271,0.886120515207188)(272,0.886120515207188)(273,0.889718630510924)(274,0.88940743255315)(275,0.889732833470507)(276,0.889864381698972)(277,0.890132740660398)(278,0.890241973741931)(279,0.890284208998269)(280,0.891596992192132)(281,0.891512489581095)(282,0.892529912365215)(283,0.892631472504975)(284,0.891923477019724)(285,0.892265741158118)(286,0.892550027154187)(287,0.892447371344065)(288,0.892426361351544)(289,0.892338868359202)(290,0.892546207229734)(291,0.892063364163914)(292,0.892063364163914)(293,0.8921397878849)(294,0.893484749518898)(295,0.894298384365979)(296,0.894288053298679)(297,0.894572936915307)(298,0.896426018848943)(299,0.896817953663818)(300,0.89766456131145)(301,0.897881142694921)(302,0.898979092997038)(303,0.898407771442878)(304,0.898268848257067)(305,0.898257640848362)(306,0.899274019624922)(307,0.899490288481367)(308,0.900747582419074)(309,0.901021602619938)(310,0.903042682285418)(311,0.902585891866929)(312,0.902543957766919)(313,0.902577707063677)(314,0.902375858743446)(315,0.902557400580903)(316,0.904423338869309)(317,0.905348390021201)(318,0.905788365814003)(319,0.906143615960657)(320,0.905900196591709)(321,0.906702964392633)(322,0.906481215062957)(323,0.906523919676679)(324,0.906428185165101)(325,0.906480293806091)(326,0.909182068205075)(327,0.909124830080313)(328,0.908485113243544)(329,0.910223778559329)(330,0.909997264388645)(331,0.90990380644472)(332,0.909858258602137)(333,0.910480839351698)(334,0.912164222545254)(335,0.912678474087219)(336,0.91256129075447)(337,0.91233139109525)(338,0.911247009311276)(339,0.911421540423217)(340,0.911066810991407)(341,0.905684847644836)(342,0.905763349048171)(343,0.90772299487737)(344,0.90772299487737)(345,0.908021727667123)(346,0.908387173166026)(347,0.908849731994676)(348,0.90886370335339)(349,0.909224614530876)(350,0.908608120327107)(351,0.908124191626862)(352,0.907788386293329)(353,0.907810038670315)(354,0.907435559756029)(355,0.908007137935459)(356,0.908232359391581)(357,0.90832295697532)(358,0.908955656906318)(359,0.909650058287528)(360,0.909244305330763)(361,0.908833385941906)(362,0.908240310877347)(363,0.9088238016148)(364,0.908275635072596)(365,0.906447784655954)(366,0.906489501742466)(367,0.9065256120612)(368,0.90672046470446)(369,0.90706771534254)(370,0.90703056664194)(371,0.906897584187426)(372,0.906327653942846)(373,0.904871528714546)(374,0.905241897394976)(375,0.905609486870797)(376,0.90576477418317)(377,0.906279012220146)(378,0.906221000747823)(379,0.906264433932494)(380,0.906275438673434)(381,0.907143660253424)(382,0.907143660253424)(383,0.907020377523807)(384,0.905683620209576)(385,0.905434077115395)(386,0.905084275143775)(387,0.905131816523159)(388,0.904733212214719)(389,0.904464437511162)(390,0.903399190152627)(391,0.903693135162535)(392,0.908156071661088)(393,0.908251547722082)(394,0.90801809677216)(395,0.907912940149011)(396,0.907946588004027)(397,0.907810579020378)(398,0.907709714246644)(399,0.907080916296106)(400,0.907038631298525) 
};
\addlegendentry{\var};

\end{axis}
\end{tikzpicture}%

%% This file was created by matlab2tikz v0.2.3.
% Copyright (c) 2008--2012, Nico Schlömer <nico.schloemer@gmail.com>
% All rights reserved.
% 
% 
% 
\begin{tikzpicture}

\begin{axis}[%
tick label style={font=\tiny},
label style={font=\tiny},
label shift={-4pt},
xlabel shift={-6pt},
legend style={font=\tiny},
view={0}{90},
width=\figurewidth,
height=\figureheight,
scale only axis,
xmin=0, xmax=400,
xlabel={Samples},
ymin=0.48, ymax=1,
ylabel={$F_1$-score},
axis lines*=left,
legend cell align=left,
legend style={at={(1.03,0)},anchor=south east,fill=none,draw=none,align=left,row sep=-0.2em},
clip=false]

\addplot [
color=blue,
solid,
line width=1.0pt,
]
coordinates{
 %(1,0.119776628385918)(2,0.0630012509669314)(3,0.0866297769325603)(4,0.106288237787856)(5,0.1663088651006)(6,0.338855979961448)(7,0.407796690524303)(8,0.474636200585825)
 (9,0.518778284778947)(10,0.553260136507491)(11,0.591833674730681)(12,0.655035898736684)(13,0.720405447755606)(14,0.740333788221117)(15,0.743672817642789)(16,0.746670580958394)(17,0.752621016031454)(18,0.769656913536186)(19,0.773122236655634)(20,0.778172463469366)(21,0.780929398753087)(22,0.783119528436339)(23,0.785365372582416)(24,0.788038045957682)(25,0.789065369771292)(26,0.794139748361366)(27,0.797779881911234)(28,0.800152011563577)(29,0.80517063584964)(30,0.809778268527415)(31,0.814246243079442)(32,0.814848701219529)(33,0.818202423738493)(34,0.819348739468136)(35,0.82285086491449)(36,0.825680780304512)(37,0.828160351156116)(38,0.831191903629589)(39,0.834445140883633)(40,0.835466129045998)(41,0.836821003206706)(42,0.839112619341295)(43,0.84017044897452)(44,0.841449431397933)(45,0.844058462759176)(46,0.84675026025457)(47,0.848010676514641)(48,0.849612040474414)(49,0.851327187101238)(50,0.851934737227965)(51,0.853135732020148)(52,0.855484311997305)(53,0.855818406485894)(54,0.856263293563193)(55,0.857509215818192)(56,0.85811452014641)(57,0.859501375690299)(58,0.860171438881843)(59,0.861678289890824)(60,0.862657850832343)(61,0.863173324360027)(62,0.863724101438954)(63,0.865737843668042)(64,0.866414762298771)(65,0.868230558827204)(66,0.869354190763555)(67,0.871120207438848)(68,0.871708631092566)(69,0.873256418793053)(70,0.873897484346413)(71,0.875372511060823)(72,0.876316142859306)(73,0.877664005620856)(74,0.878480350184235)(75,0.879105985249362)(76,0.879895009511668)(77,0.880427132243413)(78,0.880922496793215)(79,0.881868921123112)(80,0.882882090306611)(81,0.883936915917155)(82,0.885286907295875)(83,0.885563369737139)(84,0.885794125771678)(85,0.886569410810888)(86,0.887462237185626)(87,0.887966662107567)(88,0.889662614029018)(89,0.891201585138564)(90,0.891992716061227)(91,0.892130070293891)(92,0.892827396376029)(93,0.893565993123189)(94,0.89467400696025)(95,0.895411699358127)(96,0.895870744484598)(97,0.896412286141949)(98,0.896667337203066)(99,0.896892002763723)(100,0.897792588718581)(101,0.898043137125818)(102,0.897927708645712)(103,0.897898394381254)(104,0.898926346239443)(105,0.899410460424425)(106,0.899765541405962)(107,0.900128970949169)(108,0.900514644636518)(109,0.900965342198739)(110,0.901464073482405)(111,0.90220865911439)(112,0.902856894796168)(113,0.902845095604675)(114,0.903470880018723)(115,0.903817594249579)(116,0.904519976864158)(117,0.905101412977946)(118,0.905158495903484)(119,0.905869957446198)(120,0.906400460259407)(121,0.906604046449329)(122,0.906917065940041)(123,0.907539389349507)(124,0.908313406118305)(125,0.908714121421488)(126,0.909343360781594)(127,0.909606291538753)(128,0.909825436631537)(129,0.910343865913933)(130,0.91039360390847)(131,0.910809059756185)(132,0.910898983740825)(133,0.911359639349069)(134,0.91186008941235)(135,0.912269852399273)(136,0.913023249841207)(137,0.913854074583764)(138,0.914651661650192)(139,0.915356399730626)(140,0.915615422906158)(141,0.916016883337503)(142,0.916846826708231)(143,0.916743454293736)(144,0.916798391715086)(145,0.917170877889547)(146,0.917676152266081)(147,0.918261429598075)(148,0.918620653592199)(149,0.918760521715806)(150,0.918909318605021)(151,0.919188183760176)(152,0.919550845270375)(153,0.919900389320592)(154,0.920248645277746)(155,0.92048511097228)(156,0.920958164611893)(157,0.9212090614033)(158,0.921533580779468)(159,0.921913633892783)(160,0.922088941943644)(161,0.922398347899158)(162,0.922689405118572)(163,0.922988310214724)(164,0.923158955616056)(165,0.923678497593724)(166,0.924132243162911)(167,0.924893345517726)(168,0.924916758921378)(169,0.9250240923025)(170,0.92532854245341)(171,0.925546736660522)(172,0.925515770499608)(173,0.925831815581992)(174,0.926351024488757)(175,0.926679014826991)(176,0.926689248092765)(177,0.9265891148764)(178,0.927214045045892)(179,0.927286339828951)(180,0.927570793293778)(181,0.927882185528697)(182,0.928421034231383)(183,0.929031107171323)(184,0.929170059038619)(185,0.929269717198334)(186,0.929008068045418)(187,0.929272611817848)(188,0.929044640306324)(189,0.929282266006523)(190,0.929321213058738)(191,0.929396771629186)(192,0.92957075964041)(193,0.929736029530951)(194,0.929794295354833)(195,0.930071674032707)(196,0.930344417447825)(197,0.930690504952596)(198,0.930943595258614)(199,0.930957333457811)(200,0.931262359608415)(201,0.932003057007128)(202,0.93175894973234)(203,0.932006015560305)(204,0.932069715264688)(205,0.932517521891247)(206,0.932974750395118)(207,0.933318244769634)(208,0.933479516929167)(209,0.933413987353636)(210,0.93357771102368)(211,0.933767067376481)(212,0.933841524010279)(213,0.934125212304039)(214,0.934396762761223)(215,0.935085436198803)(216,0.935014548304393)(217,0.935183619919219)(218,0.935323735409739)(219,0.935474845994641)(220,0.935571280959291)(221,0.935566170432932)(222,0.935536552396715)(223,0.935683438194334)(224,0.935841010529133)(225,0.93591239557616)(226,0.935630755538733)(227,0.935974992297335)(228,0.936119001020833)(229,0.936108648808702)(230,0.936079946241961)(231,0.93620360697738)(232,0.936227589167818)(233,0.936317080411626)(234,0.936382567740571)(235,0.936305770693609)(236,0.936313519603444)(237,0.936405703815839)(238,0.936198625620198)(239,0.936498746738909)(240,0.936701807394137)(241,0.936677335186723)(242,0.936831455907307)(243,0.936955438304193)(244,0.937608570900783)(245,0.937684100368896)(246,0.937873672731199)(247,0.93802035368613)(248,0.938137272590549)(249,0.938253854530541)(250,0.938425374463177)(251,0.938543021601491)(252,0.938662934344705)(253,0.94005741564855)(254,0.940204104127416)(255,0.940207812470693)(256,0.940332601712801)(257,0.940375298348335)(258,0.940507014041343)(259,0.940480047699903)(260,0.940659377304155)(261,0.940785039559251)(262,0.941024614085383)(263,0.940724200201592)(264,0.94079157671115)(265,0.940768814191604)(266,0.94086092112843)(267,0.940794201615126)(268,0.94110174460635)(269,0.941214683412643)(270,0.941532264869746)(271,0.941595134344502)(272,0.942178406802134)(273,0.942143453485706)(274,0.942309984143793)(275,0.942265983899402)(276,0.942228812403441)(277,0.942876355867722)(278,0.944398920490738)(279,0.946430293206511)(280,0.946617380380284)(281,0.948083664806839)(282,0.94832265095946)(283,0.948489001315316)(284,0.948566170459656)(285,0.94884072735013)(286,0.948834875372245)(287,0.951093655237439)(288,0.951344869829423)(289,0.951560247107044)(290,0.951661073777458)(291,0.95171567671374)(292,0.951850153487909)(293,0.952087985327627)(294,0.95206084854552)(295,0.952065410084266)(296,0.952323628116529)(297,0.954523682195416)(298,0.954423770801445)(299,0.954642695009613)(300,0.954702122682575)(301,0.954776644985707)(302,0.95488547689478)(303,0.955185582040475)(304,0.955017432486431)(305,0.954755812662662)(306,0.954634213130572)(307,0.955115962354866)(308,0.956147205207445)(309,0.956038101329054)(310,0.95588507419287)(311,0.956115045956424)(312,0.955930841267649)(313,0.955891259976608)(314,0.955775435857754)(315,0.955848486710192)(316,0.956615598111395)(317,0.956635680674756)(318,0.956759349479721)(319,0.956727074164133)(320,0.956901416712749)(321,0.956915772375775)(322,0.95774499113563)(323,0.957603354289098)(324,0.957516107058003)(325,0.957928473829951)(326,0.957926777835811)(327,0.95782670924903)(328,0.95792172111014)(329,0.957947258511685)(330,0.958629477519677)(331,0.958888844994548)(332,0.958939639369854)(333,0.958765296150448)(334,0.958719417144572)(335,0.958768751640767)(336,0.958798321488116)(337,0.958992555947602)(338,0.958787746343785)(339,0.958939290117318)(340,0.958954981659926)(341,0.959142633489217)(342,0.959142155724392)(343,0.959166564394081)(344,0.959385096521863)(345,0.960395301213891)(346,0.961006532426917)(347,0.96109464796732)(348,0.961128911113307)(349,0.960555063592965)(350,0.960345789334232)(351,0.960341002781052)(352,0.960260633228508)(353,0.960712127944236)(354,0.961817839200059)(355,0.960967218887447)(356,0.961122907827054)(357,0.961118238511179)(358,0.961274173623387)(359,0.961434608160434)(360,0.961741096393302)(361,0.961621478946724)(362,0.962186786417859)(363,0.962131173190402)(364,0.962511092850589)(365,0.962783136933266)(366,0.962698102776122)(367,0.96284431812905)(368,0.963098011499234)(369,0.963088665194954)(370,0.962902313486181)(371,0.963453079731804)(372,0.963763096920413)(373,0.96376686896296)(374,0.963991796884815)(375,0.962736088497833)(376,0.963096564663811)(377,0.962866291510658)(378,0.96319153016831)(379,0.963337210681864)(380,0.963325403196571)(381,0.963335646353354)(382,0.962907375136166)(383,0.963048404140191)(384,0.96318633175572)(385,0.963062632942645)(386,0.963208135701804)(387,0.96335456460296)(388,0.963760263680935)(389,0.963760263680935)(390,0.963929322378367)(391,0.963691504168568)(392,0.963874334840754)(393,0.964228699283673)(394,0.963682033303049)(395,0.964574552069003)(396,0.964566757574584)(397,0.962122669472395)(398,0.964256477270439)(399,0.966433566433566)(400,0.966433566433566)%(401,0.966433566433566)(402,0.964985994397759)(403,0.963534361851332)(404,0.964260686755431)(405,0.964260686755431)(406,0.965614035087719)(407,0.965614035087719)(408,0.965614035087719)(409,0.964936886395512)(410,0.964936886395512)(411,0.965614035087719)(412,0.967832167832168)(413,0.967832167832168)(414,0.968553459119497)(415,0.968553459119497) 
};
\addlegendentry{\acl (max. ambiguity)};

\addplot [
color=red,
solid,
line width=1.0pt,
]
coordinates{
 %(1,0.115908757576926)(2,0.0212253002445928)(3,0.021252333037184)(4,0.0212297912093676)(5,0.0238579063237192)(6,0.0466885536855931)(7,0.0717870375218616)(8,0.0715320795355291)(9,0.0720143458295947)(10,0.0756200349825623)(11,0.0785767810664322)(12,0.078821876710902)(13,0.0881380074880606)(14,0.203146761207263)(15,0.236863880680768)(16,0.253312722406794)(17,0.300427598593618)(18,0.338088223483635)(19,0.421476983997287)(20,0.43091739795632)(21,0.454041613735133)(22,0.482606134829538)
 (23,0.530200191622714)(24,0.558945187568102)(25,0.625285540858811)(26,0.674820481135258)(27,0.694799194675399)(28,0.70353558326729)(29,0.719816493330761)(30,0.72608226293726)(31,0.737147854478434)(32,0.743119442379365)(33,0.752703308875046)(34,0.763755695747619)(35,0.772558676579645)(36,0.77687136337568)(37,0.778519905046193)(38,0.780960455841003)(39,0.784310260100943)(40,0.790829878292016)(41,0.794889688503689)(42,0.798618565503852)(43,0.800356309124783)(44,0.805493889154718)(45,0.81051167856147)(46,0.813676933502288)(47,0.817143150299088)(48,0.819682448960781)(49,0.823039322867061)(50,0.827313376856211)(51,0.829627269490846)(52,0.831229741932715)(53,0.832016654187585)(54,0.834398789377378)(55,0.836289802857461)(56,0.839601457953505)(57,0.842997270342368)(58,0.848140990452827)(59,0.84984431239314)(60,0.850636838511647)(61,0.85252002984696)(62,0.853332418663103)(63,0.858221327618437)(64,0.860477856424232)(65,0.862438577948476)(66,0.863205653939726)(67,0.864358566489685)(68,0.865277390326146)(69,0.866984385035317)(70,0.86857648399646)(71,0.870002454008671)(72,0.871784764078776)(73,0.872837075510359)(74,0.876061326201969)(75,0.878215987681297)(76,0.880030717046872)(77,0.880537090630703)(78,0.881923685793611)(79,0.882355225236379)(80,0.884473523586481)(81,0.885738734667552)(82,0.88669065730305)(83,0.888023112966048)(84,0.888736157143831)(85,0.890040905737593)(86,0.891320092148463)(87,0.893315744098459)(88,0.894309926185664)(89,0.89541393050788)(90,0.895357615049023)(91,0.895722373067687)(92,0.896253260360597)(93,0.897221175266312)(94,0.897650375150678)(95,0.899319400431509)(96,0.900657198248183)(97,0.90113837070231)(98,0.901149719721549)(99,0.901069462191336)(100,0.901764064415157)(101,0.903258215947105)(102,0.903884297036844)(103,0.904248076205249)(104,0.90487227283641)(105,0.905349258226024)(106,0.905958275955989)(107,0.906117450107521)(108,0.906630212090351)(109,0.907130833143602)(110,0.907730746519583)(111,0.908480906521718)(112,0.909010581570024)(113,0.909111341532449)(114,0.910225401734166)(115,0.910889704780522)(116,0.911036418478938)(117,0.910796793798072)(118,0.911688426979552)(119,0.912706049971809)(120,0.913243175387004)(121,0.913225267763452)(122,0.913847673392836)(123,0.91413685930268)(124,0.914443081511085)(125,0.91509353765909)(126,0.915414141001731)(127,0.916018707021558)(128,0.9162927837256)(129,0.91637902210396)(130,0.917121278970468)(131,0.916781069096127)(132,0.917708659562249)(133,0.918325972650636)(134,0.918822088892229)(135,0.919305249104215)(136,0.920141635882856)(137,0.920900276591234)(138,0.921234340810967)(139,0.921370731256991)(140,0.922059725422821)(141,0.922700986597224)(142,0.923111800047539)(143,0.923329212464858)(144,0.923299068633879)(145,0.924029419361696)(146,0.924625241472602)(147,0.92443545287456)(148,0.924374485236843)(149,0.924865512098994)(150,0.924960782327568)(151,0.924814469294973)(152,0.925033287225176)(153,0.92518628443015)(154,0.925356939613139)(155,0.925581408674993)(156,0.926218487630475)(157,0.926963375136977)(158,0.926806936878349)(159,0.927277233055137)(160,0.927692143420914)(161,0.927737970599549)(162,0.927915693929606)(163,0.928234870342391)(164,0.928617049710389)(165,0.928491255689445)(166,0.928815045243117)(167,0.928936174835646)(168,0.929223262763945)(169,0.93000126383692)(170,0.931198946985546)(171,0.930997673415552)(172,0.931533331792871)(173,0.931756246110993)(174,0.931795863363455)(175,0.932373566381598)(176,0.932863429342547)(177,0.93297034738032)(178,0.933035639769376)(179,0.933196884852908)(180,0.933079776597814)(181,0.933213019278958)(182,0.933423373187469)(183,0.933753580195536)(184,0.934268800306281)(185,0.934442883813061)(186,0.934484121054888)(187,0.934699456236303)(188,0.935041456038169)(189,0.935221037629666)(190,0.935718537921693)(191,0.936245399518166)(192,0.936330228564581)(193,0.936725820821894)(194,0.936961083362003)(195,0.937603768257491)(196,0.937262793364563)(197,0.937802053263635)(198,0.937748811177773)(199,0.937829928537752)(200,0.938171179942808)(201,0.93857987534171)(202,0.938755119092357)(203,0.93903621676907)(204,0.939087813294762)(205,0.939612802969461)(206,0.939747791820846)(207,0.939961459187282)(208,0.940451385687398)(209,0.940553127653352)(210,0.940629245700619)(211,0.940917807282683)(212,0.941101560191818)(213,0.941409957830118)(214,0.94163395746753)(215,0.941696002001187)(216,0.942173485237048)(217,0.942472983971822)(218,0.942460828955341)(219,0.942643667057127)(220,0.942777968749763)(221,0.942896799173168)(222,0.942910489426058)(223,0.943137003524554)(224,0.943451232759897)(225,0.943201702400644)(226,0.943313646720483)(227,0.943660130392943)(228,0.943643071860304)(229,0.944015699295104)(230,0.944024765976292)(231,0.944071660234503)(232,0.944228643152208)(233,0.944318689193652)(234,0.944385864434261)(235,0.944613809459018)(236,0.944633532549119)(237,0.944904885847642)(238,0.945054884506885)(239,0.945245220718977)(240,0.945530289754608)(241,0.94552560564115)(242,0.945643317577049)(243,0.945853599622711)(244,0.945875890062721)(245,0.94620977044107)(246,0.94642240387161)(247,0.946511618847884)(248,0.94642997711285)(249,0.946330536261911)(250,0.946384258197727)(251,0.946308548703878)(252,0.946518932364815)(253,0.946759211766131)(254,0.94681006228108)(255,0.947175593862408)(256,0.947049826527845)(257,0.947777059350724)(258,0.947698502184285)(259,0.947660150908781)(260,0.947794197757146)(261,0.948013516166978)(262,0.948044445323645)(263,0.948262616628581)(264,0.948520656853029)(265,0.948425206379102)(266,0.948635591272244)(267,0.948923115900209)(268,0.948989090118931)(269,0.949174970832302)(270,0.94953814498098)(271,0.949667163165772)(272,0.950155445034478)(273,0.950145190371729)(274,0.950055677502495)(275,0.949986649437865)(276,0.950158281389616)(277,0.949934129115903)(278,0.950448185526635)(279,0.950370007032467)(280,0.950513876916211)(281,0.95065863068109)(282,0.950846299324745)(283,0.951255412070214)(284,0.951316759388717)(285,0.951431346155575)(286,0.951599320990021)(287,0.951748007475401)(288,0.951964217830874)(289,0.951910426256953)(290,0.951988666663095)(291,0.951970213224707)(292,0.952157469932154)(293,0.952146690808097)(294,0.952152165911516)(295,0.951939603084539)(296,0.95191210897683)(297,0.952079049588855)(298,0.952271532599642)(299,0.952496619776153)(300,0.952804099649227)(301,0.953248586412677)(302,0.953223006159852)(303,0.953229605544977)(304,0.95319502321019)(305,0.953209820369112)(306,0.953168835054142)(307,0.953510515737742)(308,0.953518265069015)(309,0.953631993031099)(310,0.95391364263517)(311,0.953942670096982)(312,0.953994294148137)(313,0.954136845219572)(314,0.954369246086912)(315,0.954570766889643)(316,0.954702635564158)(317,0.954935663862106)(318,0.955152515982327)(319,0.955268739297118)(320,0.955253324239653)(321,0.955199993897642)(322,0.955391210584681)(323,0.955511819194081)(324,0.955640355563608)(325,0.955640775257377)(326,0.955741009175862)(327,0.955906674093574)(328,0.956005781965347)(329,0.956162192380169)(330,0.956450296721401)(331,0.956648991333811)(332,0.956831302499523)(333,0.956909407155659)(334,0.956859848862391)(335,0.956984810821065)(336,0.957174913228565)(337,0.957041293766822)(338,0.9572636037187)(339,0.957317249398624)(340,0.95749744191278)(341,0.95768153835748)(342,0.957662292693121)(343,0.957590349404785)(344,0.95775509426056)(345,0.958050289250246)(346,0.958081855193406)(347,0.958261763926862)(348,0.958225221775669)(349,0.958219853233313)(350,0.958195258891427)(351,0.958287342713314)(352,0.958449120339559)(353,0.958321461883464)(354,0.958378092164504)(355,0.958522206325053)(356,0.958593567603397)(357,0.958691100637867)(358,0.958599854190498)(359,0.958714157251515)(360,0.958806944821677)(361,0.958992015012675)(362,0.958910792909288)(363,0.959055823703591)(364,0.959202414368755)(365,0.959107131096086)(366,0.959316446848248)(367,0.959455439083168)(368,0.959414710898381)(369,0.959637669099927)(370,0.959670947305488)(371,0.959764599294873)(372,0.959951333691995)(373,0.960116295627544)(374,0.960144777761018)(375,0.960282864177764)(376,0.960309525069614)(377,0.960440725319224)(378,0.960572030460145)(379,0.960691702677736)(380,0.960666195202754)(381,0.960440068255817)(382,0.960528936510954)(383,0.960812283243596)(384,0.960780414565169)(385,0.960779682322009)(386,0.960669309833016)(387,0.960751105679673)(388,0.960841176140468)(389,0.960933612655694)(390,0.961076264707049)(391,0.961319150442331)(392,0.961196123381444)(393,0.961216180961344)(394,0.961424982845765)(395,0.961490979636588)(396,0.961494151196364)(397,0.961470553836464)(398,0.9618949181443)(399,0.961938254390172)(400,0.961989668761057)%(401,0.962104115085924)(402,0.962217973521755)(403,0.962246012774114)(404,0.962193970411739)(405,0.962182933222853)(406,0.962315819101377)(407,0.962438997135946)(408,0.962487828213529)(409,0.962644927645826)(410,0.962705770149752)(411,0.962604419462477)(412,0.962808169710236)(413,0.962866686178224)(414,0.963092532719296)(415,0.963067726501674)(416,0.963011084929185)(417,0.963615333149292)(418,0.963661006477124)(419,0.963836168203566)(420,0.964062514400516)(421,0.964231074950977)(422,0.964155115040004)(423,0.964036785556412)(424,0.964090741104382)(425,0.964243199437232)(426,0.964060614973)(427,0.9641223522421)(428,0.964335925196197)(429,0.964364068067787)(430,0.964607580780803)(431,0.96489673979905)(432,0.964633078916599)(433,0.964779340001961)(434,0.964713227784506)(435,0.964735026281104)(436,0.964660435899453)(437,0.965170432121789)(438,0.965472885330447)(439,0.965431555136479)(440,0.965778189209395)(441,0.966291866778283)(442,0.966410132669645)(443,0.966414631002664)(444,0.966539791143682)(445,0.96689768700407)(446,0.966918963048715)(447,0.967277472482258)(448,0.967207196093927)(449,0.967173054856227)(450,0.967236843549231)(451,0.967075313348951)(452,0.967205102023907)(453,0.967089252925744)(454,0.967338799322021)(455,0.967189141248462)(456,0.967235422268307)(457,0.966973847370333)(458,0.966976919391907)(459,0.966696440408862)(460,0.966701904135998)(461,0.966834280757091)(462,0.966805501324122)(463,0.966601487315959)(464,0.966929171521)(465,0.966967029632994)(466,0.966842785355813)(467,0.967011838844217)(468,0.966928324288972)(469,0.966483977635478)(470,0.967175014679245)(471,0.967323188324869)(472,0.967353011067283)(473,0.967204816312617)(474,0.967817046482272)(475,0.967986073978205)(476,0.967659056913786)(477,0.96858523351456)(478,0.968725619536743)(479,0.968870709172469)(480,0.968737705505259)(481,0.968964731075619)(482,0.968962597026468)(483,0.968530835684028)(484,0.96848044443255)(485,0.968998964073343)(486,0.968661866091267)(487,0.968900569819903)(488,0.970080455994784)(489,0.970876669873382)(490,0.969349233959487)(491,0.968988875449767)(492,0.968806509472677)(493,0.968982849873905)(494,0.970831000098241)(495,0.970831000098241)(496,0.970831000098241)(497,0.966914511690631)(498,0.967280982905983)(499,0.966914511690631)(500,0.966915557625428)(501,0.966547519179098)(502,0.968017057569296)(503,0.968017057569296)(504,0.968705547652916)(505,0.965811965811966)(506,0.967880085653105)(507,0.967880085653105)(508,0.967880085653105)(509,0.967142857142857)(510,0.965763195435092)(511,0.965763195435092) 
};
\addlegendentry{\acl (max. variance)};

\addplot [
color=green!50!black,
solid,
line width=1.0pt,
]
coordinates{
 %(1,0.104490305287326)(2,0.0221479719484768)(3,0.0239978513933236)(4,0.0345156638653712)(5,0.0942081470343238)(6,0.100852032900276)(7,0.133040513228925)(8,0.156112878016003)(9,0.172131195588661)(10,0.192756961674802)(11,0.217500836062254)(12,0.286069988354857)(13,0.321981330322168)(14,0.37888387859104)(15,0.411832182739284)(16,0.432528841024235)(17,0.459350473716648)(18,0.482389743190953)
 (19,0.528690704041509)(20,0.549635757804234)(21,0.567839262676724)(22,0.592106938458859)(23,0.618290309693968)(24,0.658790547286456)(25,0.674108666195349)(26,0.681509669461246)(27,0.701000388084988)(28,0.716260976561856)(29,0.731429646647036)(30,0.738212471892179)(31,0.750776814003808)(32,0.754646336597746)(33,0.760500139310083)(34,0.764306630449407)(35,0.766071117544811)(36,0.777734197699248)(37,0.777362714507182)(38,0.781717953474516)(39,0.790552500294131)(40,0.795360529690819)(41,0.796990467123719)(42,0.800091997375004)(43,0.804365149489941)(44,0.803067812128858)(45,0.805386136562294)(46,0.807965253253851)(47,0.811617962724143)(48,0.814548167949257)(49,0.816406716168432)(50,0.821509053831162)(51,0.822176441445895)(52,0.824306192183748)(53,0.8288828518463)(54,0.831493884135667)(55,0.833674368347129)(56,0.836030272791193)(57,0.837484402849697)(58,0.839057542054222)(59,0.841071662276711)(60,0.842606687331454)(61,0.845641468609211)(62,0.847232459520847)(63,0.848741585542534)(64,0.849351395396005)(65,0.850792239674063)(66,0.852681212052726)(67,0.854643277452573)(68,0.857575444474596)(69,0.859158961834874)(70,0.860138391682243)(71,0.86277424128623)(72,0.864468543269156)(73,0.865577319071911)(74,0.866551727120251)(75,0.867391862782858)(76,0.867539351462238)(77,0.86988403804402)(78,0.871700067975441)(79,0.872665206292542)(80,0.873285057400938)(81,0.875713582722905)(82,0.877398537740872)(83,0.878037176054218)(84,0.87959324506158)(85,0.880901464922678)(86,0.882735652593173)(87,0.883353271439467)(88,0.884482141648332)(89,0.88575288955274)(90,0.887466139791076)(91,0.889143520760705)(92,0.889778339004382)(93,0.891064925101566)(94,0.891702814221494)(95,0.892740630416861)(96,0.892705676614251)(97,0.894507275170673)(98,0.895397121542416)(99,0.896515558244208)(100,0.897493619447483)(101,0.898585618279587)(102,0.899136059686523)(103,0.899708053795977)(104,0.900649987624833)(105,0.90092220947789)(106,0.901388679886049)(107,0.902308374995704)(108,0.903426248155605)(109,0.904285001701485)(110,0.904854024691155)(111,0.9048676481591)(112,0.905262504932824)(113,0.90478935460866)(114,0.905796221163719)(115,0.906068020853747)(116,0.906861253136851)(117,0.907614699196856)(118,0.907996155160441)(119,0.909636372397419)(120,0.910366580181332)(121,0.911904838126669)(122,0.913217897813777)(123,0.913824499369127)(124,0.914723839950616)(125,0.916176503658601)(126,0.916695930113282)(127,0.917320124401945)(128,0.918313903697007)(129,0.919064995873182)(130,0.919974372627528)(131,0.920346963512226)(132,0.920687276780893)(133,0.921162297045146)(134,0.921343990363394)(135,0.921081715327097)(136,0.921858106476322)(137,0.921739633639507)(138,0.922508260180483)(139,0.92279829017701)(140,0.923186753174195)(141,0.923818377918159)(142,0.923976126797192)(143,0.92451223004251)(144,0.924520379619863)(145,0.925029599794723)(146,0.92552397814439)(147,0.926514550231504)(148,0.927028795147992)(149,0.92739141136649)(150,0.927490164581523)(151,0.927648975804308)(152,0.927776294760086)(153,0.92852039623061)(154,0.928634334173534)(155,0.929111130538234)(156,0.929028463790322)(157,0.929354095491836)(158,0.929631814439886)(159,0.929764493957272)(160,0.930803988773859)(161,0.931043898876406)(162,0.931272958613625)(163,0.931132646469213)(164,0.931226815578856)(165,0.931423722217233)(166,0.931622670083179)(167,0.932390113868188)(168,0.933135826651152)(169,0.93376578645999)(170,0.934651982432454)(171,0.934400515825532)(172,0.934910791824099)(173,0.935470494031765)(174,0.935675986625437)(175,0.936431280715435)(176,0.93669783687376)(177,0.937318494655962)(178,0.937243278816012)(179,0.93729417279176)(180,0.93742438007777)(181,0.937120615176976)(182,0.937032644740925)(183,0.937043089087529)(184,0.936622616413079)(185,0.936576664447263)(186,0.937263613364433)(187,0.937047136630079)(188,0.937247369038688)(189,0.937732680728613)(190,0.938038531973632)(191,0.937990018009097)(192,0.938246953015701)(193,0.938795373614563)(194,0.939193846076857)(195,0.939390844156711)(196,0.939605501540521)(197,0.939797797251624)(198,0.939637826806854)(199,0.939772176392746)(200,0.940201751654981)(201,0.940223313579894)(202,0.940623256743523)(203,0.940767252389702)(204,0.941051040522712)(205,0.94096860501667)(206,0.941176097868102)(207,0.941304091093314)(208,0.941398909552417)(209,0.941798003425747)(210,0.941877291507969)(211,0.942072262488495)(212,0.942326378263377)(213,0.942513737585893)(214,0.942576461250836)(215,0.942971162839409)(216,0.943204768572484)(217,0.943331754487267)(218,0.943531061930888)(219,0.943606489543238)(220,0.943762572177887)(221,0.944207462854716)(222,0.944414986065254)(223,0.944890345137616)(224,0.945063651410268)(225,0.94503691538603)(226,0.945461816458058)(227,0.945447508402655)(228,0.945580522820791)(229,0.945713414713312)(230,0.945842811473065)(231,0.946045685533863)(232,0.946398848499915)(233,0.946616106269499)(234,0.946394039912657)(235,0.946555194004687)(236,0.946682476310406)(237,0.947093933243623)(238,0.947074503096649)(239,0.947275183796283)(240,0.94738150672469)(241,0.947607498416876)(242,0.947663440052375)(243,0.947806191302533)(244,0.947952118064495)(245,0.948175898992587)(246,0.948559995743745)(247,0.948712732689938)(248,0.948869729249366)(249,0.949034179372851)(250,0.948984645022914)(251,0.949006147009268)(252,0.949256307818387)(253,0.949163871017107)(254,0.949105772639909)(255,0.949466015332727)(256,0.949524210820968)(257,0.949940399871156)(258,0.950275959233065)(259,0.950241042438534)(260,0.950200548092729)(261,0.950320716702207)(262,0.950534889147822)(263,0.950823089361739)(264,0.951059445806623)(265,0.951282313981643)(266,0.951334332135404)(267,0.951471882470199)(268,0.951766146926874)(269,0.951951382598195)(270,0.95218729986632)(271,0.952443865054443)(272,0.952839081522213)(273,0.952932715201523)(274,0.953120869737897)(275,0.953243959152891)(276,0.953359997378481)(277,0.953270809505213)(278,0.953193098179394)(279,0.953425202612546)(280,0.953615805293521)(281,0.953926823955617)(282,0.954116281581295)(283,0.954416440046033)(284,0.954475499920772)(285,0.954523330460706)(286,0.954607623081447)(287,0.954562986736074)(288,0.954812655774228)(289,0.955009947358344)(290,0.955332873768323)(291,0.95551844361234)(292,0.955706056921273)(293,0.955915377157754)(294,0.955885727393443)(295,0.956027506318638)(296,0.95612104277754)(297,0.956166489645867)(298,0.956501280826351)(299,0.956663998923566)(300,0.956834169640732)(301,0.957143686866809)(302,0.957015798203273)(303,0.956852438301944)(304,0.957194050904346)(305,0.957216137463587)(306,0.957076684359593)(307,0.957263453408374)(308,0.957299773205084)(309,0.957408517571677)(310,0.95754307630505)(311,0.957703334053724)(312,0.958003080989441)(313,0.95826474352735)(314,0.958410911616003)(315,0.958441567614422)(316,0.95841034354882)(317,0.958384595445596)(318,0.958402806326435)(319,0.958212202009986)(320,0.95822411638479)(321,0.958229251091506)(322,0.958454121047995)(323,0.958627021682088)(324,0.95860385036169)(325,0.958799525353472)(326,0.958902453986446)(327,0.958776829837671)(328,0.958902178097025)(329,0.959358220520861)(330,0.959427460023388)(331,0.959631773943291)(332,0.95970105517323)(333,0.959738046139854)(334,0.959804775802039)(335,0.959835855251147)(336,0.960071361554033)(337,0.960183333710586)(338,0.959992274165993)(339,0.960038199429827)(340,0.959987666976131)(341,0.960087374862797)(342,0.96015750697779)(343,0.960502767217187)(344,0.960445473693654)(345,0.960726083389577)(346,0.960789223677035)(347,0.960945224050806)(348,0.960992525598089)(349,0.961135436269031)(350,0.961204011896704)(351,0.961230798363062)(352,0.96122497372671)(353,0.961298008585358)(354,0.96129860654)(355,0.961366974839613)(356,0.961573141497867)(357,0.961791501876689)(358,0.961787489805106)(359,0.961915654975619)(360,0.96182784334298)(361,0.961765164937404)(362,0.961778655284462)(363,0.962406556514025)(364,0.962361665756932)(365,0.96246090308282)(366,0.962747945194658)(367,0.962848123713054)(368,0.962859969825089)(369,0.96354057511349)(370,0.963921477708878)(371,0.963994097188785)(372,0.963860852232444)(373,0.963952501429659)(374,0.964006083857631)(375,0.963997358552949)(376,0.964238607238994)(377,0.964514779341416)(378,0.964375944374391)(379,0.964429643360751)(380,0.964299662079789)(381,0.964469784363638)(382,0.964540931895123)(383,0.964750433049803)(384,0.964775881560079)(385,0.965043493577015)(386,0.965313698306881)(387,0.965310878937942)(388,0.965421871158506)(389,0.9654404088387)(390,0.965442052986996)(391,0.965575987883711)(392,0.965786509228037)(393,0.965852435343084)(394,0.965744152485423)(395,0.965947495182502)(396,0.965907202004833)(397,0.966059272252056)(398,0.966125580225613)(399,0.96627890648439)(400,0.966434434759309)%(401,0.966537060606032)(402,0.966258670855995)(403,0.966425839468513)(404,0.966697895864808)(405,0.966419074295457)(406,0.966944465273211)(407,0.967111354911746)(408,0.96708643080664)(409,0.967647653494114)(410,0.967618596258527)(411,0.967320952124173)(412,0.967403699850303)(413,0.96730138020953)(414,0.967578070012314)(415,0.967492799429203)(416,0.968061702335364)(417,0.968783601157055)(418,0.968879506463032)(419,0.968882016145366)(420,0.968860995213496)(421,0.969296821690877)(422,0.969837626208168)(423,0.969894039285115)(424,0.970006742209753)(425,0.969840755126638)(426,0.969996768394233)(427,0.969800718218028)(428,0.969673874040868)(429,0.968688536061129)(430,0.968695589041772)(431,0.968143744842551)(432,0.968241067121177)(433,0.968402517167353)(434,0.968764619266998)(435,0.969340967966066)(436,0.969449015037135)(437,0.968874020658602)(438,0.969328954153797)(439,0.969327788726093)(440,0.967757172682888)(441,0.968098100063349)(442,0.968281612346505)(443,0.968663477912958)(444,0.968676186162653)(445,0.969940034815263)(446,0.96993883592027)(447,0.969914141152807)(448,0.970127325524735)(449,0.970117972456078)(450,0.969892888262158)(451,0.969899886287145)(452,0.968678655520723)(453,0.968920413058821)(454,0.969187731855965)(455,0.969187731855965)(456,0.971292029874213)(457,0.971292029874213)(458,0.973232669869595)(459,0.973232669869595) 
};
\addlegendentry{\acl (random)};

\addplot [
color=orange,
densely dotted,
line width=1.0pt,
]
coordinates{
 %(1,0.0168355194759798)(2,0.0168355194759798)(3,0.0168355194759798)(4,0.0168374610889054)(5,0.0167699930780963)(6,0.048208589049083)(7,0.0477307785738257)(8,0.0476460981374464)(9,0.0474410305321807)(10,0.0485550543005947)(11,0.0507299655269844)(12,0.0532599869110817)(13,0.0669352482510405)(14,0.0856605578885116)(15,0.123749024176474)(16,0.132778182367498)(17,0.173523899962934)(18,0.228390698777483)(19,0.267670379610717)(20,0.328391421207214)(21,0.35553555784785)(22,0.388809416890549)(23,0.390987890598007)(24,0.398668883538064)(25,0.398607087878894)(26,0.413139948110276)(27,0.418235643086597)(28,0.421605492316469)(29,0.423641706313654)(30,0.430336010974062)(31,0.463627316393779)(32,0.464837285908879)
 (33,0.501703648534391)(34,0.515453153799586)(35,0.539465051247664)(36,0.555947493149717)(37,0.57339876711497)(38,0.627532176488351)(39,0.651856985149348)(40,0.662254513534309)(41,0.66133725322749)(42,0.675204859726853)(43,0.683783026853696)(44,0.685788959960219)(45,0.70282708160731)(46,0.706140469390923)(47,0.707702871846997)(48,0.714364336393848)(49,0.7151327654894)(50,0.718845886658305)(51,0.720697686870771)(52,0.723220881976813)(53,0.722307575815388)(54,0.731157566241356)(55,0.73185815760204)(56,0.737072589368828)(57,0.741196582140359)(58,0.745478161918016)(59,0.74668888008618)(60,0.748434036728539)(61,0.74866006828087)(62,0.749375203100507)(63,0.748366076922985)(64,0.756285447780586)(65,0.756545733769362)(66,0.758999038373228)(67,0.760875728901419)(68,0.760511098667807)(69,0.763315523563978)(70,0.762907675019111)(71,0.766780101808984)(72,0.767574594140362)(73,0.768873861635267)(74,0.77132158968897)(75,0.773969779107824)(76,0.776988811797957)(77,0.777269579257164)(78,0.77873812748378)(79,0.779590427010662)(80,0.779770542255994)(81,0.779814316292224)(82,0.779773865949109)(83,0.780821478558642)(84,0.781316430697107)(85,0.780119438999362)(86,0.782783511154606)(87,0.782935221683918)(88,0.784722878402314)(89,0.784528601892085)(90,0.785755530764448)(91,0.786579202746182)(92,0.787541394366187)(93,0.787037131978506)(94,0.787796126417614)(95,0.788657082673002)(96,0.788866586595803)(97,0.79245415547999)(98,0.79299645313061)(99,0.793239263423599)(100,0.793410417040492)(101,0.793398485158353)(102,0.79504230683255)(103,0.79538670163636)(104,0.795433929052428)(105,0.796636136401633)(106,0.797395298683088)(107,0.797478768764061)(108,0.79873184209954)(109,0.798188612412934)(110,0.798158006197179)(111,0.798737463747929)(112,0.799981252856107)(113,0.79988901329743)(114,0.800639170469608)(115,0.801271163621793)(116,0.802610363789278)(117,0.803153598680265)(118,0.805186114721705)(119,0.805857404220269)(120,0.805855969142446)(121,0.806059302750794)(122,0.807082294191735)(123,0.807535786430194)(124,0.807935321562945)(125,0.808146105261737)(126,0.809740449986898)(127,0.81113816625602)(128,0.812492651952222)(129,0.812520183313194)(130,0.813667241844308)(131,0.814520012562363)(132,0.81547005237087)(133,0.816599866145591)(134,0.817372719028093)(135,0.818284061575476)(136,0.818782014650651)(137,0.820492189331931)(138,0.821052916225409)(139,0.821660943450204)(140,0.822346179095648)(141,0.823463028124937)(142,0.823975253142869)(143,0.823888218841833)(144,0.823604992068404)(145,0.824073479069842)(146,0.823936301680092)(147,0.824320049556736)(148,0.824838356442337)(149,0.82507894986564)(150,0.825613977232026)(151,0.826323183825629)(152,0.826962847313908)(153,0.828598820170037)(154,0.82985830017499)(155,0.830091612363561)(156,0.830469168697509)(157,0.830480264249874)(158,0.830985835212206)(159,0.831601569407353)(160,0.831969784089505)(161,0.833068392622128)(162,0.832680461999549)(163,0.832665677203082)(164,0.832503845548131)(165,0.832504585295565)(166,0.832349902077364)(167,0.832451006253437)(168,0.833279823733234)(169,0.833362393545589)(170,0.833740108485466)(171,0.833553881327365)(172,0.834186084568046)(173,0.834493014200965)(174,0.834425251195595)(175,0.834158024728102)(176,0.834716512131478)(177,0.836077174064374)(178,0.836360618691598)(179,0.837543084209139)(180,0.838204625710763)(181,0.838210529312491)(182,0.838833858819765)(183,0.839193669209591)(184,0.839575527917163)(185,0.839979112792977)(186,0.839694962095681)(187,0.839634934091879)(188,0.839907908839505)(189,0.839925215110266)(190,0.839570148194517)(191,0.839775667052517)(192,0.839771032628145)(193,0.840681861570861)(194,0.840707042102804)(195,0.840789249207689)(196,0.842234796288293)(197,0.841964361273978)(198,0.842725131134611)(199,0.843522617295772)(200,0.843943100392778)(201,0.844022305405221)(202,0.844581235753074)(203,0.844343308636547)(204,0.845484770318911)(205,0.845885106866485)(206,0.845833825786078)(207,0.846020622495698)(208,0.846504034140925)(209,0.846422434529814)(210,0.846357244773762)(211,0.847324221779135)(212,0.848073796411742)(213,0.847929383752023)(214,0.848110535671909)(215,0.848096479489859)(216,0.849233244979107)(217,0.849938734058634)(218,0.850329026327976)(219,0.850896478217075)(220,0.851873469501244)(221,0.852716046868096)(222,0.852740261467841)(223,0.852625704924477)(224,0.852708835540035)(225,0.852931651320368)(226,0.853418221977078)(227,0.853504187290808)(228,0.853618563902009)(229,0.853832051812884)(230,0.853746659263614)(231,0.853745845298752)(232,0.853849741689243)(233,0.853895308109015)(234,0.854093567819473)(235,0.854157679336484)(236,0.854933714308131)(237,0.854520929241889)(238,0.854680414970597)(239,0.856186910172796)(240,0.856461855092529)(241,0.856484273237266)(242,0.856581839465739)(243,0.856423170046712)(244,0.856148413712893)(245,0.85649616902178)(246,0.856743762563026)(247,0.857780261905147)(248,0.857918172575755)(249,0.85760562973369)(250,0.857593875818756)(251,0.858281078706888)(252,0.858809733905514)(253,0.858757182008126)(254,0.859223736175564)(255,0.859720795726467)(256,0.860129612462307)(257,0.860066164277161)(258,0.8602412857742)(259,0.860650812063371)(260,0.860919496049057)(261,0.861295476059137)(262,0.861291150914811)(263,0.861211916856695)(264,0.861233584896272)(265,0.861401795085125)(266,0.861692472791896)(267,0.862428046006644)(268,0.862398187435361)(269,0.862452123071314)(270,0.863261396924492)(271,0.863331417005527)(272,0.863587034960837)(273,0.863475169304848)(274,0.863686904227101)(275,0.864615257121548)(276,0.8649558583527)(277,0.865063161884179)(278,0.865298523771288)(279,0.865380557862139)(280,0.865818113016894)(281,0.865871485443187)(282,0.866198881028621)(283,0.865974049433974)(284,0.866383400350716)(285,0.866963561479466)(286,0.867030460957878)(287,0.867462326764827)(288,0.867565296934443)(289,0.867779957070339)(290,0.868098175012602)(291,0.867873072942303)(292,0.867871784402578)(293,0.867658087915904)(294,0.867657801654585)(295,0.868627435317739)(296,0.868807588071323)(297,0.869113551102529)(298,0.869051069554459)(299,0.869151219951336)(300,0.869517058151138)(301,0.869809701337033)(302,0.870036146648058)(303,0.870223460310499)(304,0.870495709497698)(305,0.870916483540594)(306,0.871311438940446)(307,0.871096734376483)(308,0.871082497470043)(309,0.871394714041076)(310,0.871662718036576)(311,0.872020896966391)(312,0.872094417370846)(313,0.872265891374552)(314,0.872380685120301)(315,0.873012924052392)(316,0.873159201400678)(317,0.873155344608866)(318,0.873146245838206)(319,0.873504941352877)(320,0.873570951318809)(321,0.873699738445901)(322,0.874054054609346)(323,0.874332868915781)(324,0.874523001002456)(325,0.874880592675815)(326,0.875031016390726)(327,0.875156231565368)(328,0.875390336627467)(329,0.875057547909804)(330,0.874953682735061)(331,0.874843933292625)(332,0.875346992212524)(333,0.875262431664718)(334,0.875245973009823)(335,0.875111736792106)(336,0.875256555961774)(337,0.875507707723524)(338,0.875667862243695)(339,0.875881150457632)(340,0.875915799800559)(341,0.875880545598203)(342,0.87551929626444)(343,0.875903339532271)(344,0.876755387162011)(345,0.876983903548128)(346,0.877287282257713)(347,0.877278833482363)(348,0.877578993172259)(349,0.87757437882039)(350,0.878041331443303)(351,0.878008560125416)(352,0.878947126559563)(353,0.878761743389138)(354,0.878727279059182)(355,0.878791126245787)(356,0.878789270274246)(357,0.878879369457419)(358,0.878929332879307)(359,0.878869288487378)(360,0.87889432516828)(361,0.879159686486771)(362,0.879363120582601)(363,0.879844176991724)(364,0.879873960354928)(365,0.880035159406256)(366,0.880175209002008)(367,0.880306951854499)(368,0.880402699118724)(369,0.880431631472416)(370,0.880226654793281)(371,0.880316828539134)(372,0.880423225535735)(373,0.880254638796996)(374,0.880252375022937)(375,0.880482707617788)(376,0.880957396108648)(377,0.880980920378297)(378,0.881475374942296)(379,0.881509432889557)(380,0.881538203795952)(381,0.88178829347817)(382,0.882028813151936)(383,0.882509888101808)(384,0.882617992606114)(385,0.882582759302722)(386,0.882864494641784)(387,0.882857438270651)(388,0.882929266247059)(389,0.883337517327456)(390,0.88332394745124)(391,0.883351528922512)(392,0.883450625693178)(393,0.883786359642393)(394,0.884406657062666)(395,0.884541154944127)(396,0.884563591161788)(397,0.884919352416905)(398,0.88491506952062)(399,0.884526098247838)(400,0.884542989925364) 
};
\addlegendentry{\var};

\end{axis}
\end{tikzpicture}%

\renewcommand\trimlen{2pt}
\begin{figure}[tbp]
  \begin{subfigure}[b]{0.49\textwidth}
    \centering
    \adjincludegraphics[width=\linewidth,clip=true,trim=\trimlen{} \trimlen{} \trimlen{} \trimlen{}]{figures/ev_ping_rule}
    \vspace{-15pt}
    \caption{\textsf{[N]} Varying $\epsilon$}
	  \label{fig:ping-rule}
  \end{subfigure}
  \hfill
  \begin{subfigure}[b]{0.49\textwidth}
    \centering
    \adjincludegraphics[width=\linewidth,clip=true,trim=\trimlen{} \trimlen{} \trimlen{} \trimlen{}]{figures/ev_ping_rule2}
    \vspace{-15pt}
    \caption{\textsf{[N]} Fixed $\epsilon$}
	\label{fig:ping-rule2}
  \end{subfigure}

  \begin{subfigure}[b]{0.49\textwidth}
    \centering
    \vspace{12pt} % space of this row from above captions
    \adjincludegraphics[width=\linewidth,clip=true,trim=\trimlen{} \trimlen{} \trimlen{} \trimlen{}]{figures/ev_chl_rule}
    \vspace{-15pt}
    \caption{\textsf{[C]} Varying $\epsilon$}
	  \label{fig:chl-rule}
  \end{subfigure}
  \hfill
  \begin{subfigure}[b]{0.49\textwidth}
    \centering
    \adjincludegraphics[width=\linewidth,clip=true,trim=\trimlen{} \trimlen{} \trimlen{} \trimlen{}]{figures/ev_chl_rule2}
    \vspace{-15pt}
    \caption{\textsf{[C]} Fixed $\epsilon$}
	\label{fig:chl-rule2}
  \end{subfigure}
  
  \begin{subfigure}[b]{0.49\textwidth}
    \centering
    \vspace{12pt} % space of this row from above captions
    \adjincludegraphics[width=\linewidth,clip=true,trim=\trimlen{} \trimlen{} \trimlen{} \trimlen{}]{figures/ev_bgape_rule}
    \vspace{-15pt}
    \caption{\textsf{[A]} Varying $\epsilon$}
	  \label{fig:bgape-rule}
  \end{subfigure}
  \hfill
  \begin{subfigure}[b]{0.49\textwidth}
    \centering
    \adjincludegraphics[width=\linewidth,clip=true,trim=\trimlen{} \trimlen{} \trimlen{} \trimlen{}]{figures/ev_bgape_rule2}
    \vspace{-15pt}
    \caption{\textsf{[A]} Fixed $\epsilon$}
	\label{fig:bgape-rule2}
  \end{subfigure}

  \caption{
           Performance of different \acl selection rules on the three datasets.
           \textbf{(a), (c), (e)} For varying $\epsilon$ there is no notable difference
           between the three selection rules.
           \textbf{(b), (d), (f)} When using a small fixed value of $\epsilon$ and
           evaluating during each \acl execution, the maximum ambiguity
           selection rule has an advantage over the other two rules.
           }
  \label{fig:exp-rule}
\end{figure}

\paragraph{Sample selection rules}
In \figsref{fig:ping-rule}, \ref{fig:chl-rule}, and \ref{fig:bgape-rule}
we compare the performance
of \acl using different sample selection rules (\lineref{lin:sel1} of
\algoref{alg:acl}). In particular, we compare the maximum ambiguity
rule (as presented in \acl), the maximum variance rule
(which we use in \iacl), and the rule of randomly choosing the next
sample from $U_t$.

To further investigate the potential effect of the sampling rules, we also
ran a different set of experiments, where, instead of varying $\epsilon$
and evaluating the $F_1$-score after termination, we fixed $\epsilon$ to a
small value ($h/50$) and
evaluated the different \acl variants during their execution using the
GP posterior mean to classify points into the super- and sublevel sets
(as explained before for \str and \var). In other words, we evaluated the
performance obtained when the tuning of $\epsilon$ for controlling the
required number of samples is ignored.

The results are presented in \figsref{fig:ping-rule2}, \ref{fig:chl-rule2},
and \ref{fig:bgape-rule2}. In this case, the maximum ambiguity rule
outperforms the other two rules on all three datasets. However, note
that in the two environmental monitoring datasets the difference is fairly
small and the other two rules still perform considerably better than \var.

To connect the above observations with the operation of \acl, recall
that the algorithm is largely based on the progressive classification of
points into $H_t$ and $L_t$, a process that focuses the selection of the next
sample at each step on a set of ``interesting'' (w.r.t. to the sought after
level set), yet unclassified points $U_t$. The
way in which the sample is selected from $U_t$ seems to be of secondary
importance and selecting using maximum ambiguity might provide a small
performance benefit.

%\setlength\figureheight{1.3in}\setlength\figurewidth{2.1in}
%% This file was created by matlab2tikz v0.2.3.
% Copyright (c) 2008--2012, Nico Schlömer <nico.schloemer@gmail.com>
% All rights reserved.
% 
% 
% 
\begin{tikzpicture}

\begin{axis}[%
tick label style={font=\tiny},
label style={font=\tiny},
label shift={-4pt},
xlabel shift={-6pt},
legend style={font=\tiny},
view={0}{90},
width=\figurewidth,
height=\figureheight,
scale only axis,
xmin=0, xmax=240,
xlabel={Samples},
ymin=0.48, ymax=0.85,
ylabel={$F_1$-score},
axis lines*=left,
legend cell align=left,
legend style={at={(1.03,0)},anchor=south east,fill=none,draw=none,align=left,row sep=-0.2em},
clip=false]

\addplot [
color=red,
densely dotted,
line width=1.0pt,
]
coordinates{
 (16,0.667733354091681)(16,0.667733354091681)(18,0.683897116669292)(18,0.683897116669292)(18,0.683897116669292)(19,0.689452322658859)(19,0.689452322658859)(19,0.689452322658859)(20,0.693702557466728)(20,0.693702557466728)(20,0.693702557466728)(20,0.693702557466728)(20,0.693702557466728)(21,0.696646748427684)(21,0.696646748427684)(21,0.696646748427684)(21,0.696646748427684)(22,0.697836300278999)(22,0.697836300278999)(22,0.697836300278999)(22,0.697836300278999)(22,0.697836300278999)(22,0.697836300278999)(22,0.697836300278999)(22,0.697836300278999)(22,0.697836300278999)(22,0.697836300278999)(23,0.696600635310183)(23,0.696600635310183)(23,0.696600635310183)(23,0.696600635310183)(23,0.696600635310183)(24,0.696375658072051)(24,0.696375658072051)(24,0.696375658072051)(24,0.696375658072051)(24,0.696375658072051)(24,0.696375658072051)(24,0.696375658072051)(24,0.696375658072051)(24,0.696375658072051)(25,0.70072688145614)(25,0.70072688145614)(25,0.70072688145614)(25,0.70072688145614)(25,0.70072688145614)(25,0.70072688145614)(25,0.70072688145614)(25,0.70072688145614)(26,0.706266702094706)(26,0.706266702094706)(26,0.706266702094706)(26,0.706266702094706)(27,0.711789463374266)(27,0.711789463374266)(27,0.711789463374266)(27,0.711789463374266)(27,0.711789463374266)(27,0.711789463374266)(27,0.711789463374266)(28,0.712089954701152)(28,0.712089954701152)(28,0.712089954701152)(28,0.712089954701152)(28,0.712089954701152)(28,0.712089954701152)(28,0.712089954701152)(28,0.712089954701152)(29,0.707067635272622)(29,0.707067635272622)(29,0.707067635272622)(30,0.702611638310362)(30,0.702611638310362)(30,0.702611638310362)(30,0.702611638310362)(30,0.702611638310362)(30,0.702611638310362)(30,0.702611638310362)(31,0.702511885042345)(31,0.702511885042345)(31,0.702511885042345)(31,0.702511885042345)(31,0.702511885042345)(31,0.702511885042345)(31,0.702511885042345)(31,0.702511885042345)(31,0.702511885042345)(31,0.702511885042345)(32,0.704909402849649)(32,0.704909402849649)(32,0.704909402849649)(32,0.704909402849649)(32,0.704909402849649)(32,0.704909402849649)(32,0.704909402849649)(33,0.710683722915031)(33,0.710683722915031)(33,0.710683722915031)(33,0.710683722915031)(33,0.710683722915031)(33,0.710683722915031)(33,0.710683722915031)(33,0.710683722915031)(34,0.719859622238175)(34,0.719859622238175)(34,0.719859622238175)(34,0.719859622238175)(34,0.719859622238175)(35,0.724480825718809)(35,0.724480825718809)(35,0.724480825718809)(35,0.724480825718809)(36,0.725202472667685)(36,0.725202472667685)(36,0.725202472667685)(36,0.725202472667685)(36,0.725202472667685)(36,0.725202472667685)(36,0.725202472667685)(36,0.725202472667685)(37,0.722316269423803)(37,0.722316269423803)(37,0.722316269423803)(37,0.722316269423803)(37,0.722316269423803)(37,0.722316269423803)(37,0.722316269423803)(37,0.722316269423803)(38,0.716203259755281)(38,0.716203259755281)(38,0.716203259755281)(38,0.716203259755281)(38,0.716203259755281)(38,0.716203259755281)(39,0.709972657685087)(39,0.709972657685087)(39,0.709972657685087)(39,0.709972657685087)(39,0.709972657685087)(39,0.709972657685087)(39,0.709972657685087)(40,0.712012642162277)(40,0.712012642162277)(40,0.712012642162277)(40,0.712012642162277)(40,0.712012642162277)(40,0.712012642162277)(41,0.713818793255306)(41,0.713818793255306)(41,0.713818793255306)(41,0.713818793255306)(41,0.713818793255306)(41,0.713818793255306)(42,0.717562502574405)(42,0.717562502574405)(42,0.717562502574405)(42,0.717562502574405)(42,0.717562502574405)(42,0.717562502574405)(43,0.71783798441135)(43,0.71783798441135)(43,0.71783798441135)(43,0.71783798441135)(43,0.71783798441135)(43,0.71783798441135)(43,0.71783798441135)(43,0.71783798441135)(43,0.71783798441135)(44,0.718432718937133)(44,0.718432718937133)(44,0.718432718937133)(44,0.718432718937133)(44,0.718432718937133)(45,0.718514367349046)(45,0.718514367349046)(45,0.718514367349046)(45,0.718514367349046)(45,0.718514367349046)(45,0.718514367349046)(45,0.718514367349046)(45,0.718514367349046)(45,0.718514367349046)(46,0.719592387403055)(47,0.720615295513403)(47,0.720615295513403)(47,0.720615295513403)(47,0.720615295513403)(47,0.720615295513403)(47,0.720615295513403)(48,0.722412766820268)(48,0.722412766820268)(48,0.722412766820268)(48,0.722412766820268)(48,0.722412766820268)(48,0.722412766820268)(48,0.722412766820268)(49,0.725527446987011)(49,0.725527446987011)(49,0.725527446987011)(49,0.725527446987011)(49,0.725527446987011)(50,0.72905569992371)(50,0.72905569992371)(51,0.731329274769408)(51,0.731329274769408)(51,0.731329274769408)(51,0.731329274769408)(52,0.732035382872626)(52,0.732035382872626)(53,0.733312034897521)(53,0.733312034897521)(53,0.733312034897521)(54,0.735686821177466)(54,0.735686821177466)(54,0.735686821177466)(54,0.735686821177466)(54,0.735686821177466)(54,0.735686821177466)(54,0.735686821177466)(54,0.735686821177466)(54,0.735686821177466)(55,0.736247249821717)(55,0.736247249821717)(55,0.736247249821717)(55,0.736247249821717)(55,0.736247249821717)(55,0.736247249821717)(56,0.737681759190814)(56,0.737681759190814)(56,0.737681759190814)(56,0.737681759190814)(56,0.737681759190814)(56,0.737681759190814)(56,0.737681759190814)(56,0.737681759190814)(57,0.738472944298363)(57,0.738472944298363)(57,0.738472944298363)(57,0.738472944298363)(57,0.738472944298363)(57,0.738472944298363)(57,0.738472944298363)(58,0.739915668913179)(58,0.739915668913179)(58,0.739915668913179)(58,0.739915668913179)(59,0.740099238453627)(59,0.740099238453627)(59,0.740099238453627)(60,0.739517002677836)(60,0.739517002677836)(60,0.739517002677836)(60,0.739517002677836)(60,0.739517002677836)(60,0.739517002677836)(60,0.739517002677836)(61,0.737967088593692)(61,0.737967088593692)(61,0.737967088593692)(61,0.737967088593692)(62,0.738858469857056)(62,0.738858469857056)(62,0.738858469857056)(63,0.742604432816141)(63,0.742604432816141)(63,0.742604432816141)(63,0.742604432816141)(63,0.742604432816141)(63,0.742604432816141)(64,0.748197418480488)(64,0.748197418480488)(64,0.748197418480488)(64,0.748197418480488)(64,0.748197418480488)(64,0.748197418480488)(65,0.752179350915116)(65,0.752179350915116)(66,0.755712198666309)(66,0.755712198666309)(66,0.755712198666309)(66,0.755712198666309)(67,0.757262880906437)(67,0.757262880906437)(67,0.757262880906437)(67,0.757262880906437)(67,0.757262880906437)(68,0.75893512931078)(68,0.75893512931078)(68,0.75893512931078)(68,0.75893512931078)(68,0.75893512931078)(68,0.75893512931078)(69,0.759896954419146)(69,0.759896954419146)(69,0.759896954419146)(69,0.759896954419146)(70,0.761004567521)(70,0.761004567521)(70,0.761004567521)(70,0.761004567521)(70,0.761004567521)(70,0.761004567521)(70,0.761004567521)(70,0.761004567521)(71,0.761314392374807)(71,0.761314392374807)(71,0.761314392374807)(71,0.761314392374807)(71,0.761314392374807)(71,0.761314392374807)(72,0.761123710513821)(72,0.761123710513821)(73,0.760343778515144)(73,0.760343778515144)(74,0.75949943339395)(74,0.75949943339395)(74,0.75949943339395)(75,0.760190556153384)(75,0.760190556153384)(75,0.760190556153384)(76,0.76151995468781)(77,0.763660617592232)(77,0.763660617592232)(77,0.763660617592232)(77,0.763660617592232)(77,0.763660617592232)(78,0.765940144057935)(79,0.767735186154637)(80,0.769047783988815)(80,0.769047783988815)(80,0.769047783988815)(80,0.769047783988815)(80,0.769047783988815)(80,0.769047783988815)(80,0.769047783988815)(80,0.769047783988815)(80,0.769047783988815)(81,0.769690425912887)(81,0.769690425912887)(82,0.770436529642019)(82,0.770436529642019)(83,0.770987274227462)(83,0.770987274227462)(83,0.770987274227462)(83,0.770987274227462)(83,0.770987274227462)(84,0.771906679644753)(85,0.772956655427339)(85,0.772956655427339)(85,0.772956655427339)(85,0.772956655427339)(86,0.773771250012881)(87,0.774033080865089)(87,0.774033080865089)(88,0.774294763867503)(89,0.774497577195663)(89,0.774497577195663)(89,0.774497577195663)(89,0.774497577195663)(90,0.774465583751633)(91,0.774664767877785)(91,0.774664767877785)(91,0.774664767877785)(91,0.774664767877785)(91,0.774664767877785)(92,0.775118173248884)(92,0.775118173248884)(92,0.775118173248884)(92,0.775118173248884)(92,0.775118173248884)(93,0.775556081387871)(93,0.775556081387871)(93,0.775556081387871)(94,0.776443274878899)(95,0.776654925694697)(95,0.776654925694697)(95,0.776654925694697)(96,0.776964868147717)(96,0.776964868147717)(97,0.776520210915009)(97,0.776520210915009)(98,0.776199048542742)(98,0.776199048542742)(98,0.776199048542742)(99,0.77566090118787)(99,0.77566090118787)(100,0.775256796526401)(100,0.775256796526401)(100,0.775256796526401)(100,0.775256796526401)(101,0.775576920393688)(101,0.775576920393688)(101,0.775576920393688)(103,0.77622064532544)(104,0.777221945509281)(104,0.777221945509281)(104,0.777221945509281)(105,0.777977755260091)(105,0.777977755260091)(105,0.777977755260091)(105,0.777977755260091)(105,0.777977755260091)(105,0.777977755260091)(105,0.777977755260091)(106,0.778964673067522)(106,0.778964673067522)(106,0.778964673067522)(107,0.77990629233716)(107,0.77990629233716)(107,0.77990629233716)(107,0.77990629233716)(108,0.78083876548677)(108,0.78083876548677)(108,0.78083876548677)(109,0.782006040669816)(110,0.783156342321489)(110,0.783156342321489)(110,0.783156342321489)(111,0.784512131873486)(111,0.784512131873486)(111,0.784512131873486)(113,0.787266711530008)(114,0.788702480975759)(114,0.788702480975759)(114,0.788702480975759)(114,0.788702480975759)(115,0.790065369093699)(116,0.790496184504916)(116,0.790496184504916)(116,0.790496184504916)(118,0.790436226585343)(118,0.790436226585343)(118,0.790436226585343)(118,0.790436226585343)(119,0.790143715672104)(119,0.790143715672104)(120,0.789760764415783)(121,0.789093970503513)(121,0.789093970503513)(121,0.789093970503513)(122,0.788610970578975)(123,0.788196248905252)(123,0.788196248905252)(124,0.787943039758245)(124,0.787943039758245)(124,0.787943039758245)(124,0.787943039758245)(124,0.787943039758245)(124,0.787943039758245)(125,0.787710814390734)(126,0.787513881464323)(126,0.787513881464323)(126,0.787513881464323)(126,0.787513881464323)(126,0.787513881464323)(127,0.787367912528375)(127,0.787367912528375)(127,0.787367912528375)(128,0.787168579917541)(129,0.786881945785787)(129,0.786881945785787)(130,0.786689273439725)(130,0.786689273439725)(131,0.786478567524394)(131,0.786478567524394)(131,0.786478567524394)(131,0.786478567524394)(131,0.786478567524394)(131,0.786478567524394)(132,0.786360901962584)(133,0.786284978624195)(134,0.786622695233259)(135,0.786947580818259)(136,0.787359727850614)(136,0.787359727850614)(136,0.787359727850614)(136,0.787359727850614)(136,0.787359727850614)(138,0.788370508870756)(138,0.788370508870756)(139,0.788816899556286)(139,0.788816899556286)(139,0.788816899556286)(139,0.788816899556286)(139,0.788816899556286)(140,0.789163985007055)(140,0.789163985007055)(140,0.789163985007055)(141,0.78952271169384)(141,0.78952271169384)(142,0.78992944207987)(143,0.790077963836642)(143,0.790077963836642)(144,0.790390693316187)(144,0.790390693316187)(145,0.790452430507669)(146,0.790230624496094)(148,0.789872163279454)(148,0.789872163279454)(149,0.78964932686879)(150,0.789467842673859)(151,0.78927853541045)(151,0.78927853541045)(152,0.789084717005457)(152,0.789084717005457)(152,0.789084717005457)(154,0.78874135003309)(154,0.78874135003309)(155,0.788638130103016)(155,0.788638130103016)(156,0.788616614960394)(157,0.788538970645238)(157,0.788538970645238)(158,0.788294682517534)(159,0.788202764801006)(161,0.788386687300345)(161,0.788386687300345)(162,0.788578695000899)(163,0.788610875890519)(163,0.788610875890519)(163,0.788610875890519)(165,0.789070543344757)(165,0.789070543344757)(165,0.789070543344757)(165,0.789070543344757)(165,0.789070543344757)(166,0.789351370870402)(166,0.789351370870402)(166,0.789351370870402)(167,0.789669363784956)(167,0.789669363784956)(168,0.790008740241868)(170,0.790683682951524)(170,0.790683682951524)(170,0.790683682951524)(171,0.791062946491539)(171,0.791062946491539)(172,0.791389477313675)(172,0.791389477313675)(174,0.792018098716952)(174,0.792018098716952)(174,0.792018098716952)(174,0.792018098716952)(174,0.792018098716952)(175,0.792224217678216)(176,0.792366825797504)(177,0.792495635185097)(177,0.792495635185097)(177,0.792495635185097)(180,0.792773661100279)(180,0.792773661100279)(181,0.792817197691661)(182,0.792813851087019)(182,0.792813851087019)(182,0.792813851087019)(183,0.79286340055075)(183,0.79286340055075)(184,0.792869011566158)(186,0.793009383574827)(187,0.793192307492358)(188,0.793332808087568)(188,0.793332808087568)(189,0.793593486338552)(190,0.793979769553074)(190,0.793979769553074)(191,0.794324386819211)(191,0.794324386819211)(191,0.794324386819211)(192,0.794680447955706)(192,0.794680447955706)(192,0.794680447955706)(193,0.79507188935576)(193,0.79507188935576)(194,0.795423167740793)(194,0.795423167740793)(195,0.795723405476347)(195,0.795723405476347)(195,0.795723405476347)(196,0.796028938210089)(196,0.796028938210089)(197,0.79633037842111)(197,0.79633037842111)(197,0.79633037842111)(197,0.79633037842111)(198,0.796636051529974)(198,0.796636051529974)(198,0.796636051529974)(199,0.796928014950214)(199,0.796928014950214)(201,0.800171095793963)(202,0.800139849950287)(203,0.799771895601775)(204,0.800065853067271)(205,0.799759408328664)(206,0.798438944517992)(207,0.798438944517992)(208,0.798438944517992)(209,0.798206793684613)(210,0.79728174035623)(211,0.79728174035623)(212,0.797668083419471)(213,0.798641088217355)(214,0.798641088217355)(215,0.798272006220718)(216,0.798708309481009)(217,0.798611384069556)(218,0.798908105771199)(219,0.799072064421035)(220,0.798511879736315)(221,0.799352333043826)(222,0.799942431794075)(223,0.799451935375807)(224,0.799201110109272)(225,0.799775520171277)(226,0.799289790290139)(227,0.7992004370879)(228,0.799412014469923)(229,0.799296894724831)(230,0.799212557016652)(231,0.79915923018881)(232,0.799451186192663)(233,0.799428392486258)(234,0.799052835457501)(235,0.799339349152359)(236,0.799380671466409)(237,0.799654677721753)(238,0.798982083561364)(239,0.799873688052868)(240,0.799795340139164)
};
\addlegendentry{\acl};

\addplot [
color=orange,
densely dotted,
line width=1.0pt,
]
coordinates{
 (30,0.647657841140529)(60,0.623198352779684)(90,0.66571018651363)(120,0.711076684740511)(150,0.728896103896104)(180,0.728896103896104)(210,0.735366859027205)(240,0.734323432343234)%(270,0.754553339115351)(300,0.75219683655536)(330,0.792592592592593)(360,0.797020484171322)(390,0.802303262955854) 
};
\addlegendentry{\rstr};

\addplot [
color=green!50!black,
densely dotted,
line width=1.0pt,
]
coordinates{
 (30,0.694486242590679)(60,0.749408554401963)(90,0.770615743994192)(120,0.785612242739348)(150,0.786331376093907)(180,0.795739617463711)(210,0.793364424017545)(240,0.79841426005683)%(270,0.794731293798327)(300,0.797415944987867)(330,0.798421440503015)(360,0.798059206436361)(390,0.797713888193401) 
};
\addlegendentry{\bstr};

\addplot [
color=blue,
solid,
line width=1.3pt,
]
coordinates{
 (30,0.692971922469215)(60,0.733619780984522)(90,0.770682763904843)(120,0.780680800007343)(150,0.786449033277445)(180,0.792162585413592)(210,0.793984099016559)(240,0.799271529958187)%(270,0.794826843119961)(300,0.798404722019013)(330,0.803261110112576)(360,0.800448288267692)(390,0.802726225565982) 
};
\addlegendentry{\bacl};

\end{axis}
\end{tikzpicture}%

%% This file was created by matlab2tikz v0.2.3.
% Copyright (c) 2008--2012, Nico Schlömer <nico.schloemer@gmail.com>
% All rights reserved.
% 
% 
% 
\begin{tikzpicture}

\begin{axis}[%
tick label style={font=\tiny},
label style={font=\tiny},
label shift={-4pt},
xlabel shift={-6pt},
legend style={font=\tiny},
view={0}{90},
width=\figurewidth,
height=\figureheight,
scale only axis,
xmin=0, xmax=400,
xlabel={Samples},
ymin=0.48, ymax=1,
ylabel={$F_1$-score},
axis lines*=left,
legend cell align=left,
legend style={at={(1.03,0)},anchor=south east,fill=none,draw=none,align=left,row sep=-0.2em},
clip=false]

\addplot [
color=red,
densely dotted,
line width=1.0pt,
]
coordinates{
 (12,0.504865078580825)(12,0.504865078580825)(12,0.504865078580825)(12,0.504865078580825)(12,0.504865078580825)(12,0.504865078580825)(12,0.504865078580825)(12,0.504865078580825)(12,0.504865078580825)(12,0.504865078580825)(12,0.504865078580825)(12,0.504865078580825)(12,0.504865078580825)(12,0.504865078580825)(12,0.504865078580825)(12,0.504865078580825)(12,0.504865078580825)(12,0.504865078580825)(12,0.504865078580825)(12,0.504865078580825)(12,0.504865078580825)(12,0.504865078580825)(12,0.504865078580825)(12,0.504865078580825)(12,0.504865078580825)(12,0.504865078580825)(12,0.504865078580825)(12,0.504865078580825)(12,0.504865078580825)(12,0.504865078580825)(12,0.504865078580825)(12,0.504865078580825)(12,0.504865078580825)(12,0.504865078580825)(12,0.504865078580825)(12,0.504865078580825)(12,0.504865078580825)(12,0.504865078580825)(12,0.504865078580825)(12,0.504865078580825)(12,0.504865078580825)(12,0.504865078580825)(12,0.504865078580825)(12,0.504865078580825)(12,0.504865078580825)(12,0.504865078580825)(12,0.504865078580825)(12,0.504865078580825)(12,0.504865078580825)(12,0.504865078580825)(12,0.504865078580825)(12,0.504865078580825)(13,0.523169792868839)(13,0.523169792868839)(13,0.523169792868839)(13,0.523169792868839)(13,0.523169792868839)(13,0.523169792868839)(13,0.523169792868839)(13,0.523169792868839)(13,0.523169792868839)(13,0.523169792868839)(13,0.523169792868839)(13,0.523169792868839)(13,0.523169792868839)(13,0.523169792868839)(13,0.523169792868839)(13,0.523169792868839)(13,0.523169792868839)(13,0.523169792868839)(13,0.523169792868839)(13,0.523169792868839)(13,0.523169792868839)(13,0.523169792868839)(13,0.523169792868839)(13,0.523169792868839)(13,0.523169792868839)(13,0.523169792868839)(13,0.523169792868839)(13,0.523169792868839)(13,0.523169792868839)(13,0.523169792868839)(13,0.523169792868839)(13,0.523169792868839)(13,0.523169792868839)(13,0.523169792868839)(13,0.523169792868839)(13,0.523169792868839)(13,0.523169792868839)(13,0.523169792868839)(13,0.523169792868839)(13,0.523169792868839)(13,0.523169792868839)(13,0.523169792868839)(13,0.523169792868839)(13,0.523169792868839)(13,0.523169792868839)(13,0.523169792868839)(14,0.546177091572435)(14,0.546177091572435)(14,0.546177091572435)(14,0.546177091572435)(14,0.546177091572435)(14,0.546177091572435)(14,0.546177091572435)(14,0.546177091572435)(14,0.546177091572435)(14,0.546177091572435)(14,0.546177091572435)(14,0.546177091572435)(14,0.546177091572435)(14,0.546177091572435)(14,0.546177091572435)(14,0.546177091572435)(14,0.546177091572435)(14,0.546177091572435)(14,0.546177091572435)(14,0.546177091572435)(14,0.546177091572435)(14,0.546177091572435)(14,0.546177091572435)(14,0.546177091572435)(14,0.546177091572435)(14,0.546177091572435)(14,0.546177091572435)(14,0.546177091572435)(14,0.546177091572435)(14,0.546177091572435)(14,0.546177091572435)(14,0.546177091572435)(14,0.546177091572435)(14,0.546177091572435)(14,0.546177091572435)(14,0.546177091572435)(14,0.546177091572435)(14,0.546177091572435)(14,0.546177091572435)(14,0.546177091572435)(14,0.546177091572435)(14,0.546177091572435)(14,0.546177091572435)(14,0.546177091572435)(14,0.546177091572435)(14,0.546177091572435)(14,0.546177091572435)(14,0.546177091572435)(14,0.546177091572435)(14,0.546177091572435)(14,0.546177091572435)(14,0.546177091572435)(14,0.546177091572435)(14,0.546177091572435)(14,0.546177091572435)(14,0.546177091572435)(14,0.546177091572435)(14,0.546177091572435)(14,0.546177091572435)(14,0.546177091572435)(14,0.546177091572435)(14,0.546177091572435)(14,0.546177091572435)(14,0.546177091572435)(14,0.546177091572435)(14,0.546177091572435)(15,0.573370808065954)(15,0.573370808065954)(15,0.573370808065954)(15,0.573370808065954)(15,0.573370808065954)(15,0.573370808065954)(15,0.573370808065954)(15,0.573370808065954)(15,0.573370808065954)(15,0.573370808065954)(15,0.573370808065954)(15,0.573370808065954)(15,0.573370808065954)(15,0.573370808065954)(15,0.573370808065954)(15,0.573370808065954)(15,0.573370808065954)(15,0.573370808065954)(15,0.573370808065954)(15,0.573370808065954)(15,0.573370808065954)(15,0.573370808065954)(15,0.573370808065954)(15,0.573370808065954)(15,0.573370808065954)(15,0.573370808065954)(15,0.573370808065954)(15,0.573370808065954)(15,0.573370808065954)(15,0.573370808065954)(15,0.573370808065954)(15,0.573370808065954)(15,0.573370808065954)(15,0.573370808065954)(15,0.573370808065954)(15,0.573370808065954)(15,0.573370808065954)(15,0.573370808065954)(15,0.573370808065954)(15,0.573370808065954)(15,0.573370808065954)(15,0.573370808065954)(15,0.573370808065954)(15,0.573370808065954)(15,0.573370808065954)(15,0.573370808065954)(15,0.573370808065954)(15,0.573370808065954)(15,0.573370808065954)(15,0.573370808065954)(16,0.599320887872199)(16,0.599320887872199)(16,0.599320887872199)(16,0.599320887872199)(16,0.599320887872199)(16,0.599320887872199)(16,0.599320887872199)(16,0.599320887872199)(16,0.599320887872199)(16,0.599320887872199)(16,0.599320887872199)(16,0.599320887872199)(16,0.599320887872199)(16,0.599320887872199)(16,0.599320887872199)(16,0.599320887872199)(16,0.599320887872199)(16,0.599320887872199)(16,0.599320887872199)(16,0.599320887872199)(16,0.599320887872199)(16,0.599320887872199)(16,0.599320887872199)(16,0.599320887872199)(16,0.599320887872199)(16,0.599320887872199)(16,0.599320887872199)(16,0.599320887872199)(16,0.599320887872199)(16,0.599320887872199)(16,0.599320887872199)(16,0.599320887872199)(16,0.599320887872199)(16,0.599320887872199)(16,0.599320887872199)(16,0.599320887872199)(16,0.599320887872199)(16,0.599320887872199)(16,0.599320887872199)(16,0.599320887872199)(16,0.599320887872199)(16,0.599320887872199)(16,0.599320887872199)(16,0.599320887872199)(16,0.599320887872199)(16,0.599320887872199)(16,0.599320887872199)(16,0.599320887872199)(16,0.599320887872199)(16,0.599320887872199)(16,0.599320887872199)(17,0.619901915077436)(17,0.619901915077436)(17,0.619901915077436)(17,0.619901915077436)(17,0.619901915077436)(17,0.619901915077436)(17,0.619901915077436)(17,0.619901915077436)(17,0.619901915077436)(17,0.619901915077436)(17,0.619901915077436)(17,0.619901915077436)(17,0.619901915077436)(17,0.619901915077436)(17,0.619901915077436)(17,0.619901915077436)(17,0.619901915077436)(17,0.619901915077436)(17,0.619901915077436)(17,0.619901915077436)(17,0.619901915077436)(17,0.619901915077436)(17,0.619901915077436)(17,0.619901915077436)(17,0.619901915077436)(17,0.619901915077436)(17,0.619901915077436)(17,0.619901915077436)(17,0.619901915077436)(17,0.619901915077436)(17,0.619901915077436)(17,0.619901915077436)(17,0.619901915077436)(17,0.619901915077436)(17,0.619901915077436)(17,0.619901915077436)(17,0.619901915077436)(17,0.619901915077436)(17,0.619901915077436)(17,0.619901915077436)(17,0.619901915077436)(17,0.619901915077436)(17,0.619901915077436)(17,0.619901915077436)(17,0.619901915077436)(17,0.619901915077436)(17,0.619901915077436)(17,0.619901915077436)(17,0.619901915077436)(17,0.619901915077436)(17,0.619901915077436)(17,0.619901915077436)(17,0.619901915077436)(17,0.619901915077436)(17,0.619901915077436)(17,0.619901915077436)(17,0.619901915077436)(17,0.619901915077436)(17,0.619901915077436)(17,0.619901915077436)(17,0.619901915077436)(17,0.619901915077436)(17,0.619901915077436)(17,0.619901915077436)(17,0.619901915077436)(18,0.632419995199887)(18,0.632419995199887)(18,0.632419995199887)(18,0.632419995199887)(18,0.632419995199887)(18,0.632419995199887)(18,0.632419995199887)(18,0.632419995199887)(18,0.632419995199887)(18,0.632419995199887)(18,0.632419995199887)(18,0.632419995199887)(18,0.632419995199887)(18,0.632419995199887)(18,0.632419995199887)(18,0.632419995199887)(18,0.632419995199887)(18,0.632419995199887)(18,0.632419995199887)(18,0.632419995199887)(18,0.632419995199887)(18,0.632419995199887)(18,0.632419995199887)(18,0.632419995199887)(18,0.632419995199887)(18,0.632419995199887)(18,0.632419995199887)(18,0.632419995199887)(18,0.632419995199887)(18,0.632419995199887)(18,0.632419995199887)(18,0.632419995199887)(18,0.632419995199887)(18,0.632419995199887)(18,0.632419995199887)(18,0.632419995199887)(18,0.632419995199887)(18,0.632419995199887)(18,0.632419995199887)(18,0.632419995199887)(18,0.632419995199887)(18,0.632419995199887)(18,0.632419995199887)(18,0.632419995199887)(18,0.632419995199887)(18,0.632419995199887)(18,0.632419995199887)(18,0.632419995199887)(18,0.632419995199887)(18,0.632419995199887)(18,0.632419995199887)(18,0.632419995199887)(18,0.632419995199887)(18,0.632419995199887)(19,0.646012458484744)(19,0.646012458484744)(19,0.646012458484744)(19,0.646012458484744)(19,0.646012458484744)(19,0.646012458484744)(19,0.646012458484744)(19,0.646012458484744)(19,0.646012458484744)(19,0.646012458484744)(19,0.646012458484744)(19,0.646012458484744)(19,0.646012458484744)(19,0.646012458484744)(19,0.646012458484744)(19,0.646012458484744)(19,0.646012458484744)(19,0.646012458484744)(19,0.646012458484744)(19,0.646012458484744)(19,0.646012458484744)(19,0.646012458484744)(19,0.646012458484744)(19,0.646012458484744)(19,0.646012458484744)(19,0.646012458484744)(19,0.646012458484744)(19,0.646012458484744)(19,0.646012458484744)(19,0.646012458484744)(19,0.646012458484744)(19,0.646012458484744)(19,0.646012458484744)(19,0.646012458484744)(19,0.646012458484744)(19,0.646012458484744)(19,0.646012458484744)(19,0.646012458484744)(19,0.646012458484744)(19,0.646012458484744)(19,0.646012458484744)(19,0.646012458484744)(19,0.646012458484744)(19,0.646012458484744)(19,0.646012458484744)(19,0.646012458484744)(19,0.646012458484744)(20,0.664717621965014)(20,0.664717621965014)(20,0.664717621965014)(20,0.664717621965014)(20,0.664717621965014)(20,0.664717621965014)(20,0.664717621965014)(20,0.664717621965014)(20,0.664717621965014)(20,0.664717621965014)(20,0.664717621965014)(20,0.664717621965014)(20,0.664717621965014)(20,0.664717621965014)(20,0.664717621965014)(20,0.664717621965014)(20,0.664717621965014)(20,0.664717621965014)(20,0.664717621965014)(20,0.664717621965014)(20,0.664717621965014)(20,0.664717621965014)(20,0.664717621965014)(20,0.664717621965014)(20,0.664717621965014)(20,0.664717621965014)(20,0.664717621965014)(20,0.664717621965014)(20,0.664717621965014)(20,0.664717621965014)(20,0.664717621965014)(20,0.664717621965014)(20,0.664717621965014)(20,0.664717621965014)(20,0.664717621965014)(20,0.664717621965014)(20,0.664717621965014)(20,0.664717621965014)(20,0.664717621965014)(20,0.664717621965014)(20,0.664717621965014)(20,0.664717621965014)(20,0.664717621965014)(21,0.681959792736889)(21,0.681959792736889)(21,0.681959792736889)(21,0.681959792736889)(21,0.681959792736889)(21,0.681959792736889)(21,0.681959792736889)(21,0.681959792736889)(21,0.681959792736889)(21,0.681959792736889)(21,0.681959792736889)(21,0.681959792736889)(21,0.681959792736889)(21,0.681959792736889)(21,0.681959792736889)(21,0.681959792736889)(21,0.681959792736889)(21,0.681959792736889)(21,0.681959792736889)(21,0.681959792736889)(21,0.681959792736889)(21,0.681959792736889)(21,0.681959792736889)(21,0.681959792736889)(21,0.681959792736889)(21,0.681959792736889)(21,0.681959792736889)(21,0.681959792736889)(21,0.681959792736889)(21,0.681959792736889)(21,0.681959792736889)(21,0.681959792736889)(21,0.681959792736889)(21,0.681959792736889)(21,0.681959792736889)(21,0.681959792736889)(21,0.681959792736889)(21,0.681959792736889)(21,0.681959792736889)(21,0.681959792736889)(21,0.681959792736889)(21,0.681959792736889)(21,0.681959792736889)(21,0.681959792736889)(21,0.681959792736889)(21,0.681959792736889)(21,0.681959792736889)(21,0.681959792736889)(21,0.681959792736889)(21,0.681959792736889)(21,0.681959792736889)(21,0.681959792736889)(21,0.681959792736889)(21,0.681959792736889)(21,0.681959792736889)(21,0.681959792736889)(22,0.69159436653001)(22,0.69159436653001)(22,0.69159436653001)(22,0.69159436653001)(22,0.69159436653001)(22,0.69159436653001)(22,0.69159436653001)(22,0.69159436653001)(22,0.69159436653001)(22,0.69159436653001)(22,0.69159436653001)(22,0.69159436653001)(22,0.69159436653001)(22,0.69159436653001)(22,0.69159436653001)(22,0.69159436653001)(22,0.69159436653001)(22,0.69159436653001)(22,0.69159436653001)(22,0.69159436653001)(22,0.69159436653001)(22,0.69159436653001)(22,0.69159436653001)(22,0.69159436653001)(22,0.69159436653001)(22,0.69159436653001)(22,0.69159436653001)(22,0.69159436653001)(22,0.69159436653001)(22,0.69159436653001)(22,0.69159436653001)(22,0.69159436653001)(22,0.69159436653001)(22,0.69159436653001)(22,0.69159436653001)(22,0.69159436653001)(22,0.69159436653001)(22,0.69159436653001)(22,0.69159436653001)(22,0.69159436653001)(22,0.69159436653001)(22,0.69159436653001)(22,0.69159436653001)(22,0.69159436653001)(22,0.69159436653001)(22,0.69159436653001)(22,0.69159436653001)(22,0.69159436653001)(22,0.69159436653001)(22,0.69159436653001)(22,0.69159436653001)(23,0.692386232195861)(23,0.692386232195861)(23,0.692386232195861)(23,0.692386232195861)(23,0.692386232195861)(23,0.692386232195861)(23,0.692386232195861)(23,0.692386232195861)(23,0.692386232195861)(23,0.692386232195861)(23,0.692386232195861)(23,0.692386232195861)(23,0.692386232195861)(23,0.692386232195861)(23,0.692386232195861)(23,0.692386232195861)(23,0.692386232195861)(23,0.692386232195861)(23,0.692386232195861)(23,0.692386232195861)(23,0.692386232195861)(23,0.692386232195861)(23,0.692386232195861)(23,0.692386232195861)(23,0.692386232195861)(23,0.692386232195861)(23,0.692386232195861)(23,0.692386232195861)(23,0.692386232195861)(23,0.692386232195861)(23,0.692386232195861)(23,0.692386232195861)(23,0.692386232195861)(23,0.692386232195861)(23,0.692386232195861)(23,0.692386232195861)(23,0.692386232195861)(23,0.692386232195861)(23,0.692386232195861)(23,0.692386232195861)(23,0.692386232195861)(23,0.692386232195861)(23,0.692386232195861)(23,0.692386232195861)(24,0.693233843470702)(24,0.693233843470702)(24,0.693233843470702)(24,0.693233843470702)(24,0.693233843470702)(24,0.693233843470702)(24,0.693233843470702)(24,0.693233843470702)(24,0.693233843470702)(24,0.693233843470702)(24,0.693233843470702)(24,0.693233843470702)(24,0.693233843470702)(24,0.693233843470702)(24,0.693233843470702)(24,0.693233843470702)(24,0.693233843470702)(24,0.693233843470702)(24,0.693233843470702)(24,0.693233843470702)(24,0.693233843470702)(24,0.693233843470702)(24,0.693233843470702)(24,0.693233843470702)(24,0.693233843470702)(24,0.693233843470702)(24,0.693233843470702)(24,0.693233843470702)(24,0.693233843470702)(24,0.693233843470702)(24,0.693233843470702)(24,0.693233843470702)(24,0.693233843470702)(24,0.693233843470702)(24,0.693233843470702)(24,0.693233843470702)(24,0.693233843470702)(24,0.693233843470702)(24,0.693233843470702)(24,0.693233843470702)(24,0.693233843470702)(24,0.693233843470702)(24,0.693233843470702)(24,0.693233843470702)(25,0.709406391904183)(25,0.709406391904183)(25,0.709406391904183)(25,0.709406391904183)(25,0.709406391904183)(25,0.709406391904183)(25,0.709406391904183)(25,0.709406391904183)(25,0.709406391904183)(25,0.709406391904183)(25,0.709406391904183)(25,0.709406391904183)(25,0.709406391904183)(25,0.709406391904183)(25,0.709406391904183)(25,0.709406391904183)(25,0.709406391904183)(25,0.709406391904183)(25,0.709406391904183)(25,0.709406391904183)(25,0.709406391904183)(25,0.709406391904183)(25,0.709406391904183)(25,0.709406391904183)(25,0.709406391904183)(25,0.709406391904183)(25,0.709406391904183)(25,0.709406391904183)(25,0.709406391904183)(25,0.709406391904183)(25,0.709406391904183)(25,0.709406391904183)(25,0.709406391904183)(25,0.709406391904183)(25,0.709406391904183)(25,0.709406391904183)(25,0.709406391904183)(25,0.709406391904183)(25,0.709406391904183)(25,0.709406391904183)(25,0.709406391904183)(25,0.709406391904183)(25,0.709406391904183)(25,0.709406391904183)(25,0.709406391904183)(25,0.709406391904183)(26,0.735220764383329)(26,0.735220764383329)(26,0.735220764383329)(26,0.735220764383329)(26,0.735220764383329)(26,0.735220764383329)(26,0.735220764383329)(26,0.735220764383329)(26,0.735220764383329)(26,0.735220764383329)(26,0.735220764383329)(26,0.735220764383329)(26,0.735220764383329)(26,0.735220764383329)(26,0.735220764383329)(26,0.735220764383329)(26,0.735220764383329)(26,0.735220764383329)(26,0.735220764383329)(26,0.735220764383329)(26,0.735220764383329)(26,0.735220764383329)(26,0.735220764383329)(26,0.735220764383329)(26,0.735220764383329)(26,0.735220764383329)(26,0.735220764383329)(26,0.735220764383329)(26,0.735220764383329)(26,0.735220764383329)(26,0.735220764383329)(26,0.735220764383329)(26,0.735220764383329)(26,0.735220764383329)(26,0.735220764383329)(26,0.735220764383329)(26,0.735220764383329)(26,0.735220764383329)(26,0.735220764383329)(26,0.735220764383329)(26,0.735220764383329)(26,0.735220764383329)(26,0.735220764383329)(26,0.735220764383329)(26,0.735220764383329)(27,0.74466161724681)(27,0.74466161724681)(27,0.74466161724681)(27,0.74466161724681)(27,0.74466161724681)(27,0.74466161724681)(27,0.74466161724681)(27,0.74466161724681)(27,0.74466161724681)(27,0.74466161724681)(27,0.74466161724681)(27,0.74466161724681)(27,0.74466161724681)(27,0.74466161724681)(27,0.74466161724681)(27,0.74466161724681)(27,0.74466161724681)(27,0.74466161724681)(27,0.74466161724681)(27,0.74466161724681)(27,0.74466161724681)(27,0.74466161724681)(27,0.74466161724681)(27,0.74466161724681)(27,0.74466161724681)(27,0.74466161724681)(27,0.74466161724681)(27,0.74466161724681)(27,0.74466161724681)(27,0.74466161724681)(27,0.74466161724681)(27,0.74466161724681)(27,0.74466161724681)(27,0.74466161724681)(27,0.74466161724681)(27,0.74466161724681)(27,0.74466161724681)(27,0.74466161724681)(27,0.74466161724681)(27,0.74466161724681)(27,0.74466161724681)(27,0.74466161724681)(27,0.74466161724681)(27,0.74466161724681)(27,0.74466161724681)(27,0.74466161724681)(27,0.74466161724681)(28,0.741313158867134)(28,0.741313158867134)(28,0.741313158867134)(28,0.741313158867134)(28,0.741313158867134)(28,0.741313158867134)(28,0.741313158867134)(28,0.741313158867134)(28,0.741313158867134)(28,0.741313158867134)(28,0.741313158867134)(28,0.741313158867134)(28,0.741313158867134)(28,0.741313158867134)(28,0.741313158867134)(28,0.741313158867134)(28,0.741313158867134)(28,0.741313158867134)(28,0.741313158867134)(28,0.741313158867134)(28,0.741313158867134)(28,0.741313158867134)(28,0.741313158867134)(28,0.741313158867134)(28,0.741313158867134)(28,0.741313158867134)(28,0.741313158867134)(28,0.741313158867134)(28,0.741313158867134)(28,0.741313158867134)(28,0.741313158867134)(28,0.741313158867134)(28,0.741313158867134)(28,0.741313158867134)(28,0.741313158867134)(28,0.741313158867134)(28,0.741313158867134)(28,0.741313158867134)(28,0.741313158867134)(28,0.741313158867134)(28,0.741313158867134)(28,0.741313158867134)(28,0.741313158867134)(28,0.741313158867134)(28,0.741313158867134)(28,0.741313158867134)(28,0.741313158867134)(28,0.741313158867134)(28,0.741313158867134)(29,0.747991022511293)(29,0.747991022511293)(29,0.747991022511293)(29,0.747991022511293)(29,0.747991022511293)(29,0.747991022511293)(29,0.747991022511293)(29,0.747991022511293)(29,0.747991022511293)(29,0.747991022511293)(29,0.747991022511293)(29,0.747991022511293)(29,0.747991022511293)(29,0.747991022511293)(29,0.747991022511293)(29,0.747991022511293)(29,0.747991022511293)(29,0.747991022511293)(29,0.747991022511293)(29,0.747991022511293)(29,0.747991022511293)(29,0.747991022511293)(29,0.747991022511293)(29,0.747991022511293)(29,0.747991022511293)(29,0.747991022511293)(29,0.747991022511293)(29,0.747991022511293)(29,0.747991022511293)(29,0.747991022511293)(30,0.75456187543932)(30,0.75456187543932)(30,0.75456187543932)(30,0.75456187543932)(30,0.75456187543932)(30,0.75456187543932)(30,0.75456187543932)(30,0.75456187543932)(30,0.75456187543932)(30,0.75456187543932)(30,0.75456187543932)(30,0.75456187543932)(30,0.75456187543932)(30,0.75456187543932)(30,0.75456187543932)(30,0.75456187543932)(30,0.75456187543932)(30,0.75456187543932)(30,0.75456187543932)(30,0.75456187543932)(30,0.75456187543932)(30,0.75456187543932)(30,0.75456187543932)(30,0.75456187543932)(30,0.75456187543932)(30,0.75456187543932)(30,0.75456187543932)(30,0.75456187543932)(30,0.75456187543932)(30,0.75456187543932)(30,0.75456187543932)(30,0.75456187543932)(30,0.75456187543932)(30,0.75456187543932)(30,0.75456187543932)(30,0.75456187543932)(30,0.75456187543932)(30,0.75456187543932)(30,0.75456187543932)(30,0.75456187543932)(30,0.75456187543932)(30,0.75456187543932)(30,0.75456187543932)(30,0.75456187543932)(30,0.75456187543932)(30,0.75456187543932)(30,0.75456187543932)(30,0.75456187543932)(30,0.75456187543932)(30,0.75456187543932)(31,0.766612408711424)(31,0.766612408711424)(31,0.766612408711424)(31,0.766612408711424)(31,0.766612408711424)(31,0.766612408711424)(31,0.766612408711424)(31,0.766612408711424)(31,0.766612408711424)(31,0.766612408711424)(31,0.766612408711424)(31,0.766612408711424)(31,0.766612408711424)(31,0.766612408711424)(31,0.766612408711424)(31,0.766612408711424)(31,0.766612408711424)(31,0.766612408711424)(31,0.766612408711424)(31,0.766612408711424)(31,0.766612408711424)(31,0.766612408711424)(31,0.766612408711424)(31,0.766612408711424)(31,0.766612408711424)(31,0.766612408711424)(31,0.766612408711424)(31,0.766612408711424)(31,0.766612408711424)(31,0.766612408711424)(31,0.766612408711424)(31,0.766612408711424)(31,0.766612408711424)(31,0.766612408711424)(31,0.766612408711424)(31,0.766612408711424)(31,0.766612408711424)(31,0.766612408711424)(31,0.766612408711424)(31,0.766612408711424)(31,0.766612408711424)(31,0.766612408711424)(31,0.766612408711424)(31,0.766612408711424)(31,0.766612408711424)(31,0.766612408711424)(31,0.766612408711424)(32,0.77759816601267)(32,0.77759816601267)(32,0.77759816601267)(32,0.77759816601267)(32,0.77759816601267)(32,0.77759816601267)(32,0.77759816601267)(32,0.77759816601267)(32,0.77759816601267)(32,0.77759816601267)(32,0.77759816601267)(32,0.77759816601267)(32,0.77759816601267)(32,0.77759816601267)(32,0.77759816601267)(32,0.77759816601267)(32,0.77759816601267)(32,0.77759816601267)(32,0.77759816601267)(32,0.77759816601267)(32,0.77759816601267)(32,0.77759816601267)(32,0.77759816601267)(32,0.77759816601267)(32,0.77759816601267)(32,0.77759816601267)(32,0.77759816601267)(32,0.77759816601267)(32,0.77759816601267)(32,0.77759816601267)(32,0.77759816601267)(32,0.77759816601267)(32,0.77759816601267)(32,0.77759816601267)(32,0.77759816601267)(32,0.77759816601267)(32,0.77759816601267)(32,0.77759816601267)(32,0.77759816601267)(32,0.77759816601267)(32,0.77759816601267)(33,0.781523990555022)(33,0.781523990555022)(33,0.781523990555022)(33,0.781523990555022)(33,0.781523990555022)(33,0.781523990555022)(33,0.781523990555022)(33,0.781523990555022)(33,0.781523990555022)(33,0.781523990555022)(33,0.781523990555022)(33,0.781523990555022)(33,0.781523990555022)(33,0.781523990555022)(33,0.781523990555022)(33,0.781523990555022)(33,0.781523990555022)(33,0.781523990555022)(33,0.781523990555022)(33,0.781523990555022)(33,0.781523990555022)(33,0.781523990555022)(33,0.781523990555022)(33,0.781523990555022)(33,0.781523990555022)(33,0.781523990555022)(33,0.781523990555022)(33,0.781523990555022)(33,0.781523990555022)(33,0.781523990555022)(33,0.781523990555022)(33,0.781523990555022)(34,0.784174255772276)(34,0.784174255772276)(34,0.784174255772276)(34,0.784174255772276)(34,0.784174255772276)(34,0.784174255772276)(34,0.784174255772276)(34,0.784174255772276)(34,0.784174255772276)(34,0.784174255772276)(34,0.784174255772276)(34,0.784174255772276)(34,0.784174255772276)(34,0.784174255772276)(34,0.784174255772276)(34,0.784174255772276)(34,0.784174255772276)(34,0.784174255772276)(34,0.784174255772276)(34,0.784174255772276)(34,0.784174255772276)(34,0.784174255772276)(34,0.784174255772276)(34,0.784174255772276)(34,0.784174255772276)(34,0.784174255772276)(34,0.784174255772276)(34,0.784174255772276)(34,0.784174255772276)(34,0.784174255772276)(34,0.784174255772276)(34,0.784174255772276)(34,0.784174255772276)(34,0.784174255772276)(34,0.784174255772276)(34,0.784174255772276)(34,0.784174255772276)(34,0.784174255772276)(34,0.784174255772276)(34,0.784174255772276)(34,0.784174255772276)(34,0.784174255772276)(34,0.784174255772276)(34,0.784174255772276)(34,0.784174255772276)(35,0.788378917354959)(35,0.788378917354959)(35,0.788378917354959)(35,0.788378917354959)(35,0.788378917354959)(35,0.788378917354959)(35,0.788378917354959)(35,0.788378917354959)(35,0.788378917354959)(35,0.788378917354959)(35,0.788378917354959)(35,0.788378917354959)(35,0.788378917354959)(35,0.788378917354959)(35,0.788378917354959)(35,0.788378917354959)(35,0.788378917354959)(35,0.788378917354959)(35,0.788378917354959)(35,0.788378917354959)(35,0.788378917354959)(35,0.788378917354959)(35,0.788378917354959)(35,0.788378917354959)(35,0.788378917354959)(35,0.788378917354959)(35,0.788378917354959)(35,0.788378917354959)(35,0.788378917354959)(35,0.788378917354959)(35,0.788378917354959)(36,0.792408633993642)(36,0.792408633993642)(36,0.792408633993642)(36,0.792408633993642)(36,0.792408633993642)(36,0.792408633993642)(36,0.792408633993642)(36,0.792408633993642)(36,0.792408633993642)(36,0.792408633993642)(36,0.792408633993642)(36,0.792408633993642)(36,0.792408633993642)(36,0.792408633993642)(36,0.792408633993642)(36,0.792408633993642)(36,0.792408633993642)(36,0.792408633993642)(36,0.792408633993642)(36,0.792408633993642)(36,0.792408633993642)(36,0.792408633993642)(36,0.792408633993642)(36,0.792408633993642)(36,0.792408633993642)(36,0.792408633993642)(36,0.792408633993642)(36,0.792408633993642)(37,0.795350542896441)(37,0.795350542896441)(37,0.795350542896441)(37,0.795350542896441)(37,0.795350542896441)(37,0.795350542896441)(37,0.795350542896441)(37,0.795350542896441)(37,0.795350542896441)(37,0.795350542896441)(37,0.795350542896441)(37,0.795350542896441)(37,0.795350542896441)(37,0.795350542896441)(37,0.795350542896441)(37,0.795350542896441)(37,0.795350542896441)(37,0.795350542896441)(37,0.795350542896441)(37,0.795350542896441)(37,0.795350542896441)(37,0.795350542896441)(37,0.795350542896441)(37,0.795350542896441)(37,0.795350542896441)(37,0.795350542896441)(37,0.795350542896441)(37,0.795350542896441)(37,0.795350542896441)(37,0.795350542896441)(37,0.795350542896441)(37,0.795350542896441)(37,0.795350542896441)(37,0.795350542896441)(37,0.795350542896441)(37,0.795350542896441)(37,0.795350542896441)(37,0.795350542896441)(37,0.795350542896441)(37,0.795350542896441)(37,0.795350542896441)(38,0.798843806543084)(38,0.798843806543084)(38,0.798843806543084)(38,0.798843806543084)(38,0.798843806543084)(38,0.798843806543084)(38,0.798843806543084)(38,0.798843806543084)(38,0.798843806543084)(38,0.798843806543084)(38,0.798843806543084)(38,0.798843806543084)(38,0.798843806543084)(38,0.798843806543084)(38,0.798843806543084)(38,0.798843806543084)(38,0.798843806543084)(38,0.798843806543084)(38,0.798843806543084)(38,0.798843806543084)(38,0.798843806543084)(38,0.798843806543084)(38,0.798843806543084)(38,0.798843806543084)(38,0.798843806543084)(38,0.798843806543084)(38,0.798843806543084)(38,0.798843806543084)(38,0.798843806543084)(38,0.798843806543084)(38,0.798843806543084)(38,0.798843806543084)(38,0.798843806543084)(38,0.798843806543084)(38,0.798843806543084)(39,0.801678933543835)(39,0.801678933543835)(39,0.801678933543835)(39,0.801678933543835)(39,0.801678933543835)(39,0.801678933543835)(39,0.801678933543835)(39,0.801678933543835)(39,0.801678933543835)(39,0.801678933543835)(39,0.801678933543835)(39,0.801678933543835)(39,0.801678933543835)(39,0.801678933543835)(39,0.801678933543835)(39,0.801678933543835)(39,0.801678933543835)(39,0.801678933543835)(39,0.801678933543835)(39,0.801678933543835)(39,0.801678933543835)(39,0.801678933543835)(39,0.801678933543835)(39,0.801678933543835)(39,0.801678933543835)(39,0.801678933543835)(39,0.801678933543835)(39,0.801678933543835)(39,0.801678933543835)(39,0.801678933543835)(40,0.803144416350204)(40,0.803144416350204)(40,0.803144416350204)(40,0.803144416350204)(40,0.803144416350204)(40,0.803144416350204)(40,0.803144416350204)(40,0.803144416350204)(40,0.803144416350204)(40,0.803144416350204)(40,0.803144416350204)(40,0.803144416350204)(40,0.803144416350204)(40,0.803144416350204)(40,0.803144416350204)(40,0.803144416350204)(40,0.803144416350204)(40,0.803144416350204)(40,0.803144416350204)(40,0.803144416350204)(40,0.803144416350204)(40,0.803144416350204)(40,0.803144416350204)(40,0.803144416350204)(40,0.803144416350204)(40,0.803144416350204)(40,0.803144416350204)(40,0.803144416350204)(40,0.803144416350204)(40,0.803144416350204)(40,0.803144416350204)(40,0.803144416350204)(41,0.803829870251427)(41,0.803829870251427)(41,0.803829870251427)(41,0.803829870251427)(41,0.803829870251427)(41,0.803829870251427)(41,0.803829870251427)(41,0.803829870251427)(41,0.803829870251427)(41,0.803829870251427)(41,0.803829870251427)(41,0.803829870251427)(41,0.803829870251427)(41,0.803829870251427)(41,0.803829870251427)(41,0.803829870251427)(41,0.803829870251427)(41,0.803829870251427)(41,0.803829870251427)(41,0.803829870251427)(41,0.803829870251427)(41,0.803829870251427)(41,0.803829870251427)(41,0.803829870251427)(41,0.803829870251427)(41,0.803829870251427)(41,0.803829870251427)(41,0.803829870251427)(41,0.803829870251427)(41,0.803829870251427)(41,0.803829870251427)(41,0.803829870251427)(41,0.803829870251427)(41,0.803829870251427)(42,0.80442832917967)(42,0.80442832917967)(42,0.80442832917967)(42,0.80442832917967)(42,0.80442832917967)(42,0.80442832917967)(42,0.80442832917967)(42,0.80442832917967)(42,0.80442832917967)(42,0.80442832917967)(42,0.80442832917967)(42,0.80442832917967)(42,0.80442832917967)(42,0.80442832917967)(42,0.80442832917967)(42,0.80442832917967)(42,0.80442832917967)(42,0.80442832917967)(42,0.80442832917967)(42,0.80442832917967)(42,0.80442832917967)(42,0.80442832917967)(42,0.80442832917967)(42,0.80442832917967)(42,0.80442832917967)(43,0.8057303756705)(43,0.8057303756705)(43,0.8057303756705)(43,0.8057303756705)(43,0.8057303756705)(43,0.8057303756705)(43,0.8057303756705)(43,0.8057303756705)(43,0.8057303756705)(43,0.8057303756705)(43,0.8057303756705)(43,0.8057303756705)(43,0.8057303756705)(43,0.8057303756705)(43,0.8057303756705)(43,0.8057303756705)(43,0.8057303756705)(43,0.8057303756705)(43,0.8057303756705)(43,0.8057303756705)(43,0.8057303756705)(43,0.8057303756705)(43,0.8057303756705)(43,0.8057303756705)(43,0.8057303756705)(43,0.8057303756705)(43,0.8057303756705)(43,0.8057303756705)(43,0.8057303756705)(43,0.8057303756705)(44,0.808721362654209)(44,0.808721362654209)(44,0.808721362654209)(44,0.808721362654209)(44,0.808721362654209)(44,0.808721362654209)(44,0.808721362654209)(44,0.808721362654209)(44,0.808721362654209)(44,0.808721362654209)(44,0.808721362654209)(44,0.808721362654209)(44,0.808721362654209)(44,0.808721362654209)(44,0.808721362654209)(44,0.808721362654209)(44,0.808721362654209)(44,0.808721362654209)(45,0.812839847238627)(45,0.812839847238627)(45,0.812839847238627)(45,0.812839847238627)(45,0.812839847238627)(45,0.812839847238627)(45,0.812839847238627)(45,0.812839847238627)(45,0.812839847238627)(45,0.812839847238627)(45,0.812839847238627)(45,0.812839847238627)(45,0.812839847238627)(45,0.812839847238627)(45,0.812839847238627)(45,0.812839847238627)(45,0.812839847238627)(45,0.812839847238627)(45,0.812839847238627)(45,0.812839847238627)(45,0.812839847238627)(45,0.812839847238627)(45,0.812839847238627)(45,0.812839847238627)(45,0.812839847238627)(45,0.812839847238627)(46,0.81604204531106)(46,0.81604204531106)(46,0.81604204531106)(46,0.81604204531106)(46,0.81604204531106)(46,0.81604204531106)(46,0.81604204531106)(46,0.81604204531106)(46,0.81604204531106)(46,0.81604204531106)(46,0.81604204531106)(46,0.81604204531106)(46,0.81604204531106)(46,0.81604204531106)(46,0.81604204531106)(46,0.81604204531106)(46,0.81604204531106)(46,0.81604204531106)(46,0.81604204531106)(47,0.818424730783829)(47,0.818424730783829)(47,0.818424730783829)(47,0.818424730783829)(47,0.818424730783829)(47,0.818424730783829)(47,0.818424730783829)(47,0.818424730783829)(47,0.818424730783829)(47,0.818424730783829)(47,0.818424730783829)(47,0.818424730783829)(47,0.818424730783829)(47,0.818424730783829)(47,0.818424730783829)(47,0.818424730783829)(47,0.818424730783829)(47,0.818424730783829)(47,0.818424730783829)(47,0.818424730783829)(47,0.818424730783829)(47,0.818424730783829)(47,0.818424730783829)(47,0.818424730783829)(47,0.818424730783829)(47,0.818424730783829)(47,0.818424730783829)(47,0.818424730783829)(47,0.818424730783829)(47,0.818424730783829)(47,0.818424730783829)(47,0.818424730783829)(47,0.818424730783829)(48,0.820481755133678)(48,0.820481755133678)(48,0.820481755133678)(48,0.820481755133678)(48,0.820481755133678)(48,0.820481755133678)(48,0.820481755133678)(48,0.820481755133678)(48,0.820481755133678)(48,0.820481755133678)(48,0.820481755133678)(48,0.820481755133678)(48,0.820481755133678)(48,0.820481755133678)(48,0.820481755133678)(48,0.820481755133678)(48,0.820481755133678)(48,0.820481755133678)(48,0.820481755133678)(48,0.820481755133678)(48,0.820481755133678)(48,0.820481755133678)(49,0.823217179919157)(49,0.823217179919157)(49,0.823217179919157)(49,0.823217179919157)(49,0.823217179919157)(49,0.823217179919157)(49,0.823217179919157)(49,0.823217179919157)(49,0.823217179919157)(49,0.823217179919157)(49,0.823217179919157)(49,0.823217179919157)(49,0.823217179919157)(49,0.823217179919157)(49,0.823217179919157)(49,0.823217179919157)(49,0.823217179919157)(49,0.823217179919157)(49,0.823217179919157)(50,0.82625159519024)(50,0.82625159519024)(50,0.82625159519024)(50,0.82625159519024)(50,0.82625159519024)(50,0.82625159519024)(50,0.82625159519024)(50,0.82625159519024)(50,0.82625159519024)(50,0.82625159519024)(50,0.82625159519024)(50,0.82625159519024)(50,0.82625159519024)(50,0.82625159519024)(50,0.82625159519024)(50,0.82625159519024)(50,0.82625159519024)(50,0.82625159519024)(50,0.82625159519024)(50,0.82625159519024)(50,0.82625159519024)(50,0.82625159519024)(50,0.82625159519024)(50,0.82625159519024)(50,0.82625159519024)(50,0.82625159519024)(50,0.82625159519024)(50,0.82625159519024)(50,0.82625159519024)(50,0.82625159519024)(51,0.829002119033889)(51,0.829002119033889)(51,0.829002119033889)(51,0.829002119033889)(51,0.829002119033889)(51,0.829002119033889)(51,0.829002119033889)(51,0.829002119033889)(51,0.829002119033889)(51,0.829002119033889)(51,0.829002119033889)(51,0.829002119033889)(51,0.829002119033889)(51,0.829002119033889)(51,0.829002119033889)(51,0.829002119033889)(51,0.829002119033889)(51,0.829002119033889)(51,0.829002119033889)(51,0.829002119033889)(51,0.829002119033889)(51,0.829002119033889)(51,0.829002119033889)(51,0.829002119033889)(52,0.831329859929671)(52,0.831329859929671)(52,0.831329859929671)(52,0.831329859929671)(52,0.831329859929671)(52,0.831329859929671)(52,0.831329859929671)(52,0.831329859929671)(52,0.831329859929671)(52,0.831329859929671)(52,0.831329859929671)(52,0.831329859929671)(52,0.831329859929671)(52,0.831329859929671)(52,0.831329859929671)(52,0.831329859929671)(52,0.831329859929671)(52,0.831329859929671)(52,0.831329859929671)(52,0.831329859929671)(52,0.831329859929671)(52,0.831329859929671)(52,0.831329859929671)(52,0.831329859929671)(53,0.833257141200067)(53,0.833257141200067)(53,0.833257141200067)(53,0.833257141200067)(53,0.833257141200067)(53,0.833257141200067)(53,0.833257141200067)(53,0.833257141200067)(53,0.833257141200067)(53,0.833257141200067)(53,0.833257141200067)(53,0.833257141200067)(53,0.833257141200067)(53,0.833257141200067)(53,0.833257141200067)(53,0.833257141200067)(53,0.833257141200067)(53,0.833257141200067)(53,0.833257141200067)(53,0.833257141200067)(54,0.8347663212136)(54,0.8347663212136)(54,0.8347663212136)(54,0.8347663212136)(54,0.8347663212136)(54,0.8347663212136)(54,0.8347663212136)(54,0.8347663212136)(54,0.8347663212136)(54,0.8347663212136)(54,0.8347663212136)(54,0.8347663212136)(54,0.8347663212136)(54,0.8347663212136)(54,0.8347663212136)(54,0.8347663212136)(54,0.8347663212136)(54,0.8347663212136)(54,0.8347663212136)(54,0.8347663212136)(54,0.8347663212136)(55,0.836501874697888)(55,0.836501874697888)(55,0.836501874697888)(55,0.836501874697888)(55,0.836501874697888)(55,0.836501874697888)(55,0.836501874697888)(55,0.836501874697888)(55,0.836501874697888)(55,0.836501874697888)(55,0.836501874697888)(55,0.836501874697888)(55,0.836501874697888)(55,0.836501874697888)(55,0.836501874697888)(55,0.836501874697888)(55,0.836501874697888)(55,0.836501874697888)(56,0.83843008927367)(56,0.83843008927367)(56,0.83843008927367)(56,0.83843008927367)(56,0.83843008927367)(56,0.83843008927367)(56,0.83843008927367)(56,0.83843008927367)(56,0.83843008927367)(56,0.83843008927367)(56,0.83843008927367)(56,0.83843008927367)(56,0.83843008927367)(57,0.840305038829841)(57,0.840305038829841)(57,0.840305038829841)(57,0.840305038829841)(57,0.840305038829841)(57,0.840305038829841)(57,0.840305038829841)(57,0.840305038829841)(57,0.840305038829841)(57,0.840305038829841)(57,0.840305038829841)(57,0.840305038829841)(57,0.840305038829841)(57,0.840305038829841)(57,0.840305038829841)(57,0.840305038829841)(57,0.840305038829841)(57,0.840305038829841)(57,0.840305038829841)(57,0.840305038829841)(58,0.842011018090292)(58,0.842011018090292)(58,0.842011018090292)(58,0.842011018090292)(58,0.842011018090292)(58,0.842011018090292)(58,0.842011018090292)(58,0.842011018090292)(58,0.842011018090292)(58,0.842011018090292)(58,0.842011018090292)(58,0.842011018090292)(58,0.842011018090292)(59,0.842902785153524)(59,0.842902785153524)(59,0.842902785153524)(59,0.842902785153524)(59,0.842902785153524)(59,0.842902785153524)(59,0.842902785153524)(59,0.842902785153524)(59,0.842902785153524)(59,0.842902785153524)(59,0.842902785153524)(59,0.842902785153524)(59,0.842902785153524)(60,0.843702198635364)(60,0.843702198635364)(60,0.843702198635364)(60,0.843702198635364)(60,0.843702198635364)(60,0.843702198635364)(60,0.843702198635364)(60,0.843702198635364)(60,0.843702198635364)(60,0.843702198635364)(60,0.843702198635364)(60,0.843702198635364)(60,0.843702198635364)(60,0.843702198635364)(60,0.843702198635364)(60,0.843702198635364)(60,0.843702198635364)(60,0.843702198635364)(60,0.843702198635364)(61,0.845226640196423)(61,0.845226640196423)(61,0.845226640196423)(61,0.845226640196423)(61,0.845226640196423)(61,0.845226640196423)(61,0.845226640196423)(61,0.845226640196423)(61,0.845226640196423)(61,0.845226640196423)(61,0.845226640196423)(61,0.845226640196423)(61,0.845226640196423)(61,0.845226640196423)(61,0.845226640196423)(62,0.846847553268688)(62,0.846847553268688)(62,0.846847553268688)(62,0.846847553268688)(62,0.846847553268688)(62,0.846847553268688)(62,0.846847553268688)(62,0.846847553268688)(62,0.846847553268688)(62,0.846847553268688)(62,0.846847553268688)(62,0.846847553268688)(62,0.846847553268688)(62,0.846847553268688)(62,0.846847553268688)(62,0.846847553268688)(62,0.846847553268688)(62,0.846847553268688)(62,0.846847553268688)(62,0.846847553268688)(62,0.846847553268688)(62,0.846847553268688)(62,0.846847553268688)(62,0.846847553268688)(63,0.848801731742995)(63,0.848801731742995)(63,0.848801731742995)(63,0.848801731742995)(63,0.848801731742995)(63,0.848801731742995)(63,0.848801731742995)(63,0.848801731742995)(63,0.848801731742995)(63,0.848801731742995)(63,0.848801731742995)(63,0.848801731742995)(63,0.848801731742995)(63,0.848801731742995)(63,0.848801731742995)(63,0.848801731742995)(63,0.848801731742995)(63,0.848801731742995)(64,0.850869353446309)(64,0.850869353446309)(64,0.850869353446309)(64,0.850869353446309)(64,0.850869353446309)(64,0.850869353446309)(64,0.850869353446309)(64,0.850869353446309)(64,0.850869353446309)(64,0.850869353446309)(64,0.850869353446309)(64,0.850869353446309)(64,0.850869353446309)(64,0.850869353446309)(64,0.850869353446309)(65,0.85229762254709)(65,0.85229762254709)(65,0.85229762254709)(65,0.85229762254709)(65,0.85229762254709)(65,0.85229762254709)(65,0.85229762254709)(65,0.85229762254709)(65,0.85229762254709)(65,0.85229762254709)(65,0.85229762254709)(65,0.85229762254709)(65,0.85229762254709)(65,0.85229762254709)(65,0.85229762254709)(65,0.85229762254709)(65,0.85229762254709)(66,0.853801137284702)(66,0.853801137284702)(66,0.853801137284702)(66,0.853801137284702)(66,0.853801137284702)(66,0.853801137284702)(66,0.853801137284702)(66,0.853801137284702)(66,0.853801137284702)(67,0.854824972133955)(67,0.854824972133955)(67,0.854824972133955)(67,0.854824972133955)(67,0.854824972133955)(67,0.854824972133955)(67,0.854824972133955)(67,0.854824972133955)(67,0.854824972133955)(67,0.854824972133955)(67,0.854824972133955)(67,0.854824972133955)(68,0.855669382763225)(68,0.855669382763225)(68,0.855669382763225)(68,0.855669382763225)(68,0.855669382763225)(68,0.855669382763225)(68,0.855669382763225)(68,0.855669382763225)(68,0.855669382763225)(68,0.855669382763225)(68,0.855669382763225)(68,0.855669382763225)(68,0.855669382763225)(68,0.855669382763225)(68,0.855669382763225)(68,0.855669382763225)(68,0.855669382763225)(68,0.855669382763225)(68,0.855669382763225)(68,0.855669382763225)(68,0.855669382763225)(69,0.856244783694022)(69,0.856244783694022)(69,0.856244783694022)(69,0.856244783694022)(69,0.856244783694022)(69,0.856244783694022)(69,0.856244783694022)(69,0.856244783694022)(69,0.856244783694022)(70,0.856736316181424)(70,0.856736316181424)(70,0.856736316181424)(70,0.856736316181424)(70,0.856736316181424)(70,0.856736316181424)(70,0.856736316181424)(70,0.856736316181424)(70,0.856736316181424)(70,0.856736316181424)(70,0.856736316181424)(70,0.856736316181424)(71,0.857306902409606)(71,0.857306902409606)(71,0.857306902409606)(71,0.857306902409606)(71,0.857306902409606)(71,0.857306902409606)(71,0.857306902409606)(71,0.857306902409606)(71,0.857306902409606)(71,0.857306902409606)(71,0.857306902409606)(71,0.857306902409606)(71,0.857306902409606)(71,0.857306902409606)(71,0.857306902409606)(71,0.857306902409606)(71,0.857306902409606)(71,0.857306902409606)(71,0.857306902409606)(72,0.858018887600214)(72,0.858018887600214)(72,0.858018887600214)(72,0.858018887600214)(72,0.858018887600214)(72,0.858018887600214)(72,0.858018887600214)(72,0.858018887600214)(72,0.858018887600214)(72,0.858018887600214)(73,0.858919987628094)(73,0.858919987628094)(73,0.858919987628094)(73,0.858919987628094)(73,0.858919987628094)(73,0.858919987628094)(73,0.858919987628094)(73,0.858919987628094)(73,0.858919987628094)(73,0.858919987628094)(73,0.858919987628094)(73,0.858919987628094)(73,0.858919987628094)(73,0.858919987628094)(74,0.860146484105036)(74,0.860146484105036)(74,0.860146484105036)(74,0.860146484105036)(74,0.860146484105036)(74,0.860146484105036)(74,0.860146484105036)(74,0.860146484105036)(74,0.860146484105036)(74,0.860146484105036)(74,0.860146484105036)(74,0.860146484105036)(74,0.860146484105036)(74,0.860146484105036)(74,0.860146484105036)(74,0.860146484105036)(75,0.861251507808856)(75,0.861251507808856)(75,0.861251507808856)(75,0.861251507808856)(75,0.861251507808856)(75,0.861251507808856)(75,0.861251507808856)(75,0.861251507808856)(75,0.861251507808856)(75,0.861251507808856)(76,0.862372887669196)(76,0.862372887669196)(76,0.862372887669196)(76,0.862372887669196)(76,0.862372887669196)(76,0.862372887669196)(76,0.862372887669196)(76,0.862372887669196)(76,0.862372887669196)(76,0.862372887669196)(76,0.862372887669196)(76,0.862372887669196)(76,0.862372887669196)(76,0.862372887669196)(76,0.862372887669196)(76,0.862372887669196)(77,0.863451524723784)(77,0.863451524723784)(77,0.863451524723784)(77,0.863451524723784)(77,0.863451524723784)(77,0.863451524723784)(77,0.863451524723784)(77,0.863451524723784)(78,0.864386264260367)(78,0.864386264260367)(78,0.864386264260367)(78,0.864386264260367)(78,0.864386264260367)(78,0.864386264260367)(78,0.864386264260367)(78,0.864386264260367)(78,0.864386264260367)(78,0.864386264260367)(78,0.864386264260367)(78,0.864386264260367)(78,0.864386264260367)(78,0.864386264260367)(79,0.865216942042594)(79,0.865216942042594)(79,0.865216942042594)(79,0.865216942042594)(79,0.865216942042594)(79,0.865216942042594)(79,0.865216942042594)(79,0.865216942042594)(79,0.865216942042594)(79,0.865216942042594)(80,0.866034467938662)(80,0.866034467938662)(80,0.866034467938662)(80,0.866034467938662)(80,0.866034467938662)(80,0.866034467938662)(80,0.866034467938662)(80,0.866034467938662)(80,0.866034467938662)(80,0.866034467938662)(81,0.866906591199006)(81,0.866906591199006)(81,0.866906591199006)(81,0.866906591199006)(81,0.866906591199006)(81,0.866906591199006)(81,0.866906591199006)(81,0.866906591199006)(81,0.866906591199006)(81,0.866906591199006)(81,0.866906591199006)(81,0.866906591199006)(81,0.866906591199006)(81,0.866906591199006)(81,0.866906591199006)(81,0.866906591199006)(82,0.867938511345117)(82,0.867938511345117)(82,0.867938511345117)(82,0.867938511345117)(82,0.867938511345117)(82,0.867938511345117)(82,0.867938511345117)(82,0.867938511345117)(82,0.867938511345117)(83,0.868964177720208)(83,0.868964177720208)(83,0.868964177720208)(83,0.868964177720208)(84,0.870115085683796)(84,0.870115085683796)(84,0.870115085683796)(84,0.870115085683796)(84,0.870115085683796)(84,0.870115085683796)(84,0.870115085683796)(84,0.870115085683796)(84,0.870115085683796)(84,0.870115085683796)(85,0.871263585094177)(85,0.871263585094177)(85,0.871263585094177)(85,0.871263585094177)(85,0.871263585094177)(85,0.871263585094177)(85,0.871263585094177)(86,0.872375869179939)(86,0.872375869179939)(86,0.872375869179939)(86,0.872375869179939)(86,0.872375869179939)(86,0.872375869179939)(86,0.872375869179939)(86,0.872375869179939)(86,0.872375869179939)(86,0.872375869179939)(86,0.872375869179939)(86,0.872375869179939)(86,0.872375869179939)(86,0.872375869179939)(87,0.873470104234195)(87,0.873470104234195)(87,0.873470104234195)(88,0.874765589139725)(88,0.874765589139725)(88,0.874765589139725)(88,0.874765589139725)(88,0.874765589139725)(88,0.874765589139725)(88,0.874765589139725)(88,0.874765589139725)(88,0.874765589139725)(88,0.874765589139725)(88,0.874765589139725)(89,0.875918063104193)(89,0.875918063104193)(89,0.875918063104193)(89,0.875918063104193)(89,0.875918063104193)(89,0.875918063104193)(90,0.87711695983655)(90,0.87711695983655)(90,0.87711695983655)(90,0.87711695983655)(90,0.87711695983655)(90,0.87711695983655)(90,0.87711695983655)(90,0.87711695983655)(90,0.87711695983655)(90,0.87711695983655)(90,0.87711695983655)(90,0.87711695983655)(90,0.87711695983655)(91,0.87815603020879)(91,0.87815603020879)(91,0.87815603020879)(91,0.87815603020879)(91,0.87815603020879)(91,0.87815603020879)(91,0.87815603020879)(91,0.87815603020879)(91,0.87815603020879)(91,0.87815603020879)(91,0.87815603020879)(92,0.879166842958229)(92,0.879166842958229)(92,0.879166842958229)(92,0.879166842958229)(92,0.879166842958229)(93,0.880149304947892)(93,0.880149304947892)(93,0.880149304947892)(93,0.880149304947892)(93,0.880149304947892)(93,0.880149304947892)(94,0.881024540013573)(94,0.881024540013573)(94,0.881024540013573)(94,0.881024540013573)(94,0.881024540013573)(94,0.881024540013573)(94,0.881024540013573)(94,0.881024540013573)(94,0.881024540013573)(94,0.881024540013573)(94,0.881024540013573)(94,0.881024540013573)(94,0.881024540013573)(94,0.881024540013573)(95,0.881827235416656)(95,0.881827235416656)(95,0.881827235416656)(95,0.881827235416656)(95,0.881827235416656)(95,0.881827235416656)(95,0.881827235416656)(95,0.881827235416656)(96,0.88245427481638)(96,0.88245427481638)(96,0.88245427481638)(96,0.88245427481638)(96,0.88245427481638)(96,0.88245427481638)(96,0.88245427481638)(96,0.88245427481638)(96,0.88245427481638)(96,0.88245427481638)(97,0.883095742961618)(97,0.883095742961618)(97,0.883095742961618)(97,0.883095742961618)(97,0.883095742961618)(97,0.883095742961618)(97,0.883095742961618)(97,0.883095742961618)(97,0.883095742961618)(97,0.883095742961618)(98,0.883696734759346)(98,0.883696734759346)(98,0.883696734759346)(98,0.883696734759346)(98,0.883696734759346)(98,0.883696734759346)(98,0.883696734759346)(98,0.883696734759346)(98,0.883696734759346)(98,0.883696734759346)(98,0.883696734759346)(99,0.88425078682894)(99,0.88425078682894)(99,0.88425078682894)(99,0.88425078682894)(99,0.88425078682894)(99,0.88425078682894)(99,0.88425078682894)(100,0.884548181783039)(100,0.884548181783039)(100,0.884548181783039)(100,0.884548181783039)(100,0.884548181783039)(100,0.884548181783039)(100,0.884548181783039)(101,0.885010330910515)(101,0.885010330910515)(101,0.885010330910515)(101,0.885010330910515)(101,0.885010330910515)(101,0.885010330910515)(101,0.885010330910515)(102,0.885426127861043)(102,0.885426127861043)(102,0.885426127861043)(102,0.885426127861043)(102,0.885426127861043)(102,0.885426127861043)(103,0.885769793922489)(103,0.885769793922489)(103,0.885769793922489)(103,0.885769793922489)(103,0.885769793922489)(103,0.885769793922489)(103,0.885769793922489)(103,0.885769793922489)(103,0.885769793922489)(104,0.886032314254541)(104,0.886032314254541)(104,0.886032314254541)(104,0.886032314254541)(104,0.886032314254541)(104,0.886032314254541)(104,0.886032314254541)(104,0.886032314254541)(105,0.88629880218148)(105,0.88629880218148)(105,0.88629880218148)(105,0.88629880218148)(105,0.88629880218148)(105,0.88629880218148)(105,0.88629880218148)(105,0.88629880218148)(105,0.88629880218148)(106,0.886517673825304)(106,0.886517673825304)(106,0.886517673825304)(106,0.886517673825304)(106,0.886517673825304)(106,0.886517673825304)(106,0.886517673825304)(107,0.886705004501407)(107,0.886705004501407)(107,0.886705004501407)(107,0.886705004501407)(107,0.886705004501407)(107,0.886705004501407)(107,0.886705004501407)(107,0.886705004501407)(107,0.886705004501407)(107,0.886705004501407)(108,0.886897695798863)(108,0.886897695798863)(109,0.887157082672705)(109,0.887157082672705)(109,0.887157082672705)(109,0.887157082672705)(109,0.887157082672705)(109,0.887157082672705)(109,0.887157082672705)(109,0.887157082672705)(109,0.887157082672705)(110,0.887783728467878)(110,0.887783728467878)(110,0.887783728467878)(110,0.887783728467878)(110,0.887783728467878)(110,0.887783728467878)(110,0.887783728467878)(111,0.888286966749207)(111,0.888286966749207)(111,0.888286966749207)(112,0.888886911520906)(112,0.888886911520906)(112,0.888886911520906)(112,0.888886911520906)(112,0.888886911520906)(112,0.888886911520906)(113,0.889574965531872)(113,0.889574965531872)(113,0.889574965531872)(113,0.889574965531872)(113,0.889574965531872)(113,0.889574965531872)(113,0.889574965531872)(114,0.890344667610861)(114,0.890344667610861)(114,0.890344667610861)(114,0.890344667610861)(114,0.890344667610861)(114,0.890344667610861)(114,0.890344667610861)(114,0.890344667610861)(114,0.890344667610861)(114,0.890344667610861)(114,0.890344667610861)(114,0.890344667610861)(114,0.890344667610861)(115,0.891178560933554)(115,0.891178560933554)(115,0.891178560933554)(116,0.892102859021432)(116,0.892102859021432)(116,0.892102859021432)(116,0.892102859021432)(116,0.892102859021432)(116,0.892102859021432)(116,0.892102859021432)(116,0.892102859021432)(116,0.892102859021432)(117,0.892952229672649)(117,0.892952229672649)(117,0.892952229672649)(117,0.892952229672649)(117,0.892952229672649)(118,0.893788445121733)(118,0.893788445121733)(118,0.893788445121733)(118,0.893788445121733)(118,0.893788445121733)(118,0.893788445121733)(118,0.893788445121733)(118,0.893788445121733)(119,0.8946594856061)(119,0.8946594856061)(119,0.8946594856061)(119,0.8946594856061)(119,0.8946594856061)(120,0.895567455089183)(120,0.895567455089183)(120,0.895567455089183)(120,0.895567455089183)(120,0.895567455089183)(120,0.895567455089183)(120,0.895567455089183)(120,0.895567455089183)(120,0.895567455089183)(120,0.895567455089183)(120,0.895567455089183)(120,0.895567455089183)(120,0.895567455089183)(120,0.895567455089183)(121,0.896512697148473)(121,0.896512697148473)(121,0.896512697148473)(121,0.896512697148473)(121,0.896512697148473)(121,0.896512697148473)(121,0.896512697148473)(121,0.896512697148473)(121,0.896512697148473)(122,0.897495209273577)(122,0.897495209273577)(122,0.897495209273577)(122,0.897495209273577)(122,0.897495209273577)(122,0.897495209273577)(122,0.897495209273577)(123,0.898461048864739)(123,0.898461048864739)(123,0.898461048864739)(123,0.898461048864739)(123,0.898461048864739)(123,0.898461048864739)(123,0.898461048864739)(123,0.898461048864739)(123,0.898461048864739)(124,0.899405460019621)(124,0.899405460019621)(124,0.899405460019621)(124,0.899405460019621)(124,0.899405460019621)(124,0.899405460019621)(125,0.900369640024093)(125,0.900369640024093)(125,0.900369640024093)(125,0.900369640024093)(125,0.900369640024093)(125,0.900369640024093)(125,0.900369640024093)(125,0.900369640024093)(126,0.901302491058131)(126,0.901302491058131)(126,0.901302491058131)(126,0.901302491058131)(126,0.901302491058131)(126,0.901302491058131)(126,0.901302491058131)(127,0.902186066874667)(127,0.902186066874667)(128,0.903003775901638)(128,0.903003775901638)(128,0.903003775901638)(129,0.903617772793256)(129,0.903617772793256)(129,0.903617772793256)(129,0.903617772793256)(129,0.903617772793256)(129,0.903617772793256)(129,0.903617772793256)(129,0.903617772793256)(129,0.903617772793256)(130,0.904248583082145)(130,0.904248583082145)(130,0.904248583082145)(130,0.904248583082145)(130,0.904248583082145)(130,0.904248583082145)(130,0.904248583082145)(130,0.904248583082145)(130,0.904248583082145)(130,0.904248583082145)(131,0.904788273064382)(131,0.904788273064382)(131,0.904788273064382)(131,0.904788273064382)(131,0.904788273064382)(131,0.904788273064382)(132,0.905241658172107)(132,0.905241658172107)(132,0.905241658172107)(132,0.905241658172107)(132,0.905241658172107)(132,0.905241658172107)(132,0.905241658172107)(132,0.905241658172107)(132,0.905241658172107)(132,0.905241658172107)(132,0.905241658172107)(132,0.905241658172107)(133,0.905617236178036)(133,0.905617236178036)(133,0.905617236178036)(133,0.905617236178036)(133,0.905617236178036)(133,0.905617236178036)(133,0.905617236178036)(133,0.905617236178036)(133,0.905617236178036)(134,0.905807896647698)(134,0.905807896647698)(134,0.905807896647698)(134,0.905807896647698)(134,0.905807896647698)(135,0.906079523665759)(135,0.906079523665759)(136,0.906303045632532)(136,0.906303045632532)(136,0.906303045632532)(136,0.906303045632532)(137,0.906454417995213)(137,0.906454417995213)(137,0.906454417995213)(137,0.906454417995213)(137,0.906454417995213)(137,0.906454417995213)(137,0.906454417995213)(137,0.906454417995213)(137,0.906454417995213)(137,0.906454417995213)(138,0.906631985471256)(138,0.906631985471256)(138,0.906631985471256)(138,0.906631985471256)(138,0.906631985471256)(138,0.906631985471256)(138,0.906631985471256)(138,0.906631985471256)(139,0.906776636442685)(139,0.906776636442685)(139,0.906776636442685)(139,0.906776636442685)(139,0.906776636442685)(139,0.906776636442685)(139,0.906776636442685)(139,0.906776636442685)(140,0.906897043796306)(141,0.907061352929401)(141,0.907061352929401)(141,0.907061352929401)(141,0.907061352929401)(141,0.907061352929401)(141,0.907061352929401)(141,0.907061352929401)(141,0.907061352929401)(142,0.907175595583422)(142,0.907175595583422)(142,0.907175595583422)(143,0.907286516589718)(143,0.907286516589718)(143,0.907286516589718)(143,0.907286516589718)(143,0.907286516589718)(144,0.907395999888178)(144,0.907395999888178)(144,0.907395999888178)(144,0.907395999888178)(145,0.907494813442418)(145,0.907494813442418)(145,0.907494813442418)(145,0.907494813442418)(146,0.907484436035112)(146,0.907484436035112)(146,0.907484436035112)(146,0.907484436035112)(146,0.907484436035112)(146,0.907484436035112)(146,0.907484436035112)(146,0.907484436035112)(147,0.907642245540492)(147,0.907642245540492)(148,0.907685200940031)(148,0.907685200940031)(148,0.907685200940031)(148,0.907685200940031)(148,0.907685200940031)(148,0.907685200940031)(148,0.907685200940031)(148,0.907685200940031)(149,0.907738156107916)(149,0.907738156107916)(149,0.907738156107916)(150,0.90775325923015)(150,0.90775325923015)(150,0.90775325923015)(150,0.90775325923015)(150,0.90775325923015)(151,0.907734883477113)(151,0.907734883477113)(152,0.907713328264532)(152,0.907713328264532)(152,0.907713328264532)(153,0.907693056293033)(153,0.907693056293033)(154,0.907681994991709)(154,0.907681994991709)(154,0.907681994991709)(154,0.907681994991709)(154,0.907681994991709)(154,0.907681994991709)(155,0.907695657140671)(155,0.907695657140671)(155,0.907695657140671)(155,0.907695657140671)(156,0.907715960999977)(156,0.907715960999977)(156,0.907715960999977)(156,0.907715960999977)(156,0.907715960999977)(156,0.907715960999977)(156,0.907715960999977)(156,0.907715960999977)(156,0.907715960999977)(156,0.907715960999977)(157,0.907761613154578)(157,0.907761613154578)(157,0.907761613154578)(157,0.907761613154578)(157,0.907761613154578)(157,0.907761613154578)(158,0.907861331972772)(158,0.907861331972772)(158,0.907861331972772)(158,0.907861331972772)(158,0.907861331972772)(158,0.907861331972772)(158,0.907861331972772)(158,0.907861331972772)(158,0.907861331972772)(159,0.908011023348507)(159,0.908011023348507)(159,0.908011023348507)(159,0.908011023348507)(159,0.908011023348507)(160,0.908156530215909)(160,0.908156530215909)(160,0.908156530215909)(160,0.908156530215909)(160,0.908156530215909)(160,0.908156530215909)(160,0.908156530215909)(161,0.908333975740979)(161,0.908333975740979)(161,0.908333975740979)(161,0.908333975740979)(161,0.908333975740979)(161,0.908333975740979)(162,0.908546046789197)(162,0.908546046789197)(162,0.908546046789197)(162,0.908546046789197)(162,0.908546046789197)(163,0.908868594655885)(163,0.908868594655885)(163,0.908868594655885)(163,0.908868594655885)(163,0.908868594655885)(163,0.908868594655885)(163,0.908868594655885)(164,0.90915005747973)(164,0.90915005747973)(164,0.90915005747973)(165,0.909454046203418)(166,0.90984683569755)(166,0.90984683569755)(166,0.90984683569755)(166,0.90984683569755)(166,0.90984683569755)(166,0.90984683569755)(167,0.910174702990907)(167,0.910174702990907)(167,0.910174702990907)(167,0.910174702990907)(168,0.910557116951432)(168,0.910557116951432)(168,0.910557116951432)(169,0.91089832006316)(169,0.91089832006316)(169,0.91089832006316)(169,0.91089832006316)(170,0.911235258854136)(170,0.911235258854136)(170,0.911235258854136)(170,0.911235258854136)(171,0.911617307473616)(171,0.911617307473616)(171,0.911617307473616)(171,0.911617307473616)(171,0.911617307473616)(171,0.911617307473616)(171,0.911617307473616)(172,0.911992454665332)(172,0.911992454665332)(172,0.911992454665332)(172,0.911992454665332)(172,0.911992454665332)(172,0.911992454665332)(173,0.912347590301615)(173,0.912347590301615)(174,0.912715246558631)(174,0.912715246558631)(174,0.912715246558631)(174,0.912715246558631)(174,0.912715246558631)(174,0.912715246558631)(175,0.913081150472102)(175,0.913081150472102)(175,0.913081150472102)(175,0.913081150472102)(176,0.913382632910585)(176,0.913382632910585)(176,0.913382632910585)(176,0.913382632910585)(176,0.913382632910585)(176,0.913382632910585)(177,0.91372689029414)(177,0.91372689029414)(177,0.91372689029414)(177,0.91372689029414)(177,0.91372689029414)(177,0.91372689029414)(177,0.91372689029414)(178,0.913999103572663)(178,0.913999103572663)(178,0.913999103572663)(178,0.913999103572663)(179,0.914317348500506)(179,0.914317348500506)(179,0.914317348500506)(179,0.914317348500506)(179,0.914317348500506)(179,0.914317348500506)(179,0.914317348500506)(179,0.914317348500506)(179,0.914317348500506)(180,0.914619319139855)(180,0.914619319139855)(180,0.914619319139855)(181,0.914902556182025)(181,0.914902556182025)(181,0.914902556182025)(181,0.914902556182025)(181,0.914902556182025)(182,0.915166794456473)(182,0.915166794456473)(182,0.915166794456473)(183,0.915385332665607)(183,0.915385332665607)(183,0.915385332665607)(183,0.915385332665607)(183,0.915385332665607)(184,0.91562595708624)(184,0.91562595708624)(185,0.915827279076721)(186,0.916048754554075)(186,0.916048754554075)(186,0.916048754554075)(187,0.916242923256125)(187,0.916242923256125)(187,0.916242923256125)(188,0.916458389493785)(188,0.916458389493785)(188,0.916458389493785)(188,0.916458389493785)(189,0.916672689017806)(189,0.916672689017806)(189,0.916672689017806)(189,0.916672689017806)(189,0.916672689017806)(189,0.916672689017806)(189,0.916672689017806)(190,0.916889106136478)(190,0.916889106136478)(191,0.917109578096285)(191,0.917109578096285)(191,0.917109578096285)(192,0.917334696007648)(192,0.917334696007648)(192,0.917334696007648)(192,0.917334696007648)(192,0.917334696007648)(192,0.917334696007648)(193,0.917542301696305)(193,0.917542301696305)(193,0.917542301696305)(193,0.917542301696305)(193,0.917542301696305)(193,0.917542301696305)(194,0.917771606718821)(194,0.917771606718821)(195,0.918002313045923)(195,0.918002313045923)(196,0.918232118289294)(196,0.918232118289294)(197,0.918458605850243)(197,0.918458605850243)(197,0.918458605850243)(198,0.918646977862196)(198,0.918646977862196)(198,0.918646977862196)(198,0.918646977862196)(198,0.918646977862196)(198,0.918646977862196)(198,0.918646977862196)(198,0.918646977862196)(198,0.918646977862196)(199,0.918857260391134)(199,0.918857260391134)(199,0.918857260391134)(199,0.918857260391134)(199,0.918857260391134)(200,0.919058522666762)(200,0.919058522666762)(200,0.919058522666762)(200,0.919058522666762)(200,0.919058522666762)(200,0.919058522666762)(201,0.919249682995987)(201,0.919249682995987)(202,0.91942956004163)(202,0.91942956004163)(202,0.91942956004163)(202,0.91942956004163)(203,0.919597109101008)(203,0.919597109101008)(204,0.919751795235801)(204,0.919751795235801)(204,0.919751795235801)(205,0.919894006618849)(205,0.919894006618849)(206,0.920024947876035)(206,0.920024947876035)(206,0.920024947876035)(206,0.920024947876035)(206,0.920024947876035)(207,0.92014824718155)(208,0.920300035318397)(208,0.920300035318397)(208,0.920300035318397)(208,0.920300035318397)(209,0.920428542650078)(209,0.920428542650078)(209,0.920428542650078)(210,0.920604183580779)(210,0.920604183580779)(210,0.920604183580779)(211,0.920743842124826)(211,0.920743842124826)(211,0.920743842124826)(211,0.920743842124826)(211,0.920743842124826)(211,0.920743842124826)(212,0.920887036235071)(212,0.920887036235071)(212,0.920887036235071)(213,0.921097619881995)(213,0.921097619881995)(213,0.921097619881995)(213,0.921097619881995)(213,0.921097619881995)(214,0.92125553380862)(214,0.92125553380862)(214,0.92125553380862)(215,0.921484089277178)(215,0.921484089277178)(215,0.921484089277178)(217,0.921895483300176)(217,0.921895483300176)(217,0.921895483300176)(217,0.921895483300176)(217,0.921895483300176)(217,0.921895483300176)(218,0.922079144242787)(218,0.922079144242787)(218,0.922079144242787)(218,0.922079144242787)(219,0.92226996325111)(220,0.922420728735662)(220,0.922420728735662)(220,0.922420728735662)(220,0.922420728735662)(220,0.922420728735662)(221,0.922677885796664)(221,0.922677885796664)(221,0.922677885796664)(221,0.922677885796664)(221,0.922677885796664)(222,0.922896585442007)(222,0.922896585442007)(223,0.923125057767663)(223,0.923125057767663)(223,0.923125057767663)(223,0.923125057767663)(223,0.923125057767663)(223,0.923125057767663)(223,0.923125057767663)(224,0.923363112046205)(225,0.923611820325899)(225,0.923611820325899)(225,0.923611820325899)(225,0.923611820325899)(225,0.923611820325899)(225,0.923611820325899)(225,0.923611820325899)(225,0.923611820325899)(226,0.923855942315075)(226,0.923855942315075)(226,0.923855942315075)(226,0.923855942315075)(227,0.924107806382789)(227,0.924107806382789)(227,0.924107806382789)(228,0.924365789877882)(228,0.924365789877882)(229,0.924627722999601)(229,0.924627722999601)(230,0.924892069324925)(230,0.924892069324925)(231,0.925128639310369)(231,0.925128639310369)(231,0.925128639310369)(232,0.925393743179381)(232,0.925393743179381)(233,0.925657397266281)(233,0.925657397266281)(233,0.925657397266281)(233,0.925657397266281)(233,0.925657397266281)(234,0.925895943308491)(235,0.926154158526316)(235,0.926154158526316)(236,0.926407650552276)(236,0.926407650552276)(236,0.926407650552276)(236,0.926407650552276)(237,0.926636602376951)(237,0.926636602376951)(237,0.926636602376951)(237,0.926636602376951)(237,0.926636602376951)(238,0.926879691061465)(238,0.926879691061465)(238,0.926879691061465)(239,0.927094470039503)(239,0.927094470039503)(239,0.927094470039503)(239,0.927094470039503)(239,0.927094470039503)(239,0.927094470039503)(240,0.927323915440798)(241,0.927546156459612)(242,0.927760505315576)(242,0.927760505315576)(242,0.927760505315576)(243,0.927966731102263)(243,0.927966731102263)(244,0.928140238652443)(244,0.928140238652443)(244,0.928140238652443)(245,0.928331059814702)(245,0.928331059814702)(245,0.928331059814702)(245,0.928331059814702)(245,0.928331059814702)(245,0.928331059814702)(245,0.928331059814702)(246,0.928536299114566)(246,0.928536299114566)(246,0.928536299114566)(246,0.928536299114566)(246,0.928536299114566)(247,0.928710366737071)(247,0.928710366737071)(248,0.928877348936947)(248,0.928877348936947)(248,0.928877348936947)(249,0.929037645036051)(249,0.929037645036051)(249,0.929037645036051)(249,0.929037645036051)(249,0.929037645036051)(251,0.92930811116282)(251,0.92930811116282)(251,0.92930811116282)(251,0.92930811116282)(252,0.929456013630432)(252,0.929456013630432)(252,0.929456013630432)(253,0.929613308238166)(253,0.929613308238166)(253,0.929613308238166)(253,0.929613308238166)(253,0.929613308238166)(253,0.929613308238166)(254,0.92974282664044)(254,0.92974282664044)(255,0.929882964510021)(255,0.929882964510021)(255,0.929882964510021)(255,0.929882964510021)(255,0.929882964510021)(255,0.929882964510021)(255,0.929882964510021)(255,0.929882964510021)(256,0.930021573920496)(256,0.930021573920496)(256,0.930021573920496)(256,0.930021573920496)(256,0.930021573920496)(257,0.930159284013752)(257,0.930159284013752)(258,0.930292813742559)(258,0.930292813742559)(258,0.930292813742559)(259,0.930433568956954)(259,0.930433568956954)(259,0.930433568956954)(259,0.930433568956954)(259,0.930433568956954)(260,0.930570280329598)(260,0.930570280329598)(261,0.930708082872402)(261,0.930708082872402)(262,0.930846788608287)(262,0.930846788608287)(262,0.930846788608287)(263,0.93099276907075)(263,0.93099276907075)(263,0.93099276907075)(263,0.93099276907075)(263,0.93099276907075)(264,0.931133057197957)(264,0.931133057197957)(266,0.931424920079992)(266,0.931424920079992)(267,0.9315646932364)(267,0.9315646932364)(267,0.9315646932364)(268,0.931704008997427)(268,0.931704008997427)(269,0.931843051998276)(269,0.931843051998276)(269,0.931843051998276)(270,0.931982019783776)(271,0.932120833300943)(271,0.932120833300943)(271,0.932120833300943)(271,0.932120833300943)(272,0.932259207445736)(272,0.932259207445736)(272,0.932259207445736)(272,0.932259207445736)(272,0.932259207445736)(272,0.932259207445736)(273,0.932396690431212)(273,0.932396690431212)(273,0.932396690431212)(273,0.932396690431212)(274,0.932532822535606)(274,0.932532822535606)(275,0.932667369785282)(275,0.932667369785282)(275,0.932667369785282)(275,0.932667369785282)(276,0.932816895484207)(276,0.932816895484207)(276,0.932816895484207)(276,0.932816895484207)(276,0.932816895484207)(277,0.932948002349668)(277,0.932948002349668)(279,0.933217467265927)(279,0.933217467265927)(279,0.933217467265927)(280,0.933342044172786)(280,0.933342044172786)(281,0.933472948435949)(281,0.933472948435949)(281,0.933472948435949)(282,0.933584937569411)(282,0.933584937569411)(282,0.933584937569411)(283,0.933709350251084)(283,0.933709350251084)(283,0.933709350251084)(284,0.933819739619485)(284,0.933819739619485)(284,0.933819739619485)(284,0.933819739619485)(285,0.933938769917531)(285,0.933938769917531)(285,0.933938769917531)(285,0.933938769917531)(285,0.933938769917531)(286,0.934050775633334)(286,0.934050775633334)(287,0.934160828637192)(287,0.934160828637192)(287,0.934160828637192)(288,0.934269087955371)(288,0.934269087955371)(289,0.934376850022737)(289,0.934376850022737)(289,0.934376850022737)(290,0.934481497274048)(290,0.934481497274048)(290,0.934481497274048)(291,0.934585090514012)(291,0.934585090514012)(293,0.934788235007227)(293,0.934788235007227)(293,0.934788235007227)(293,0.934788235007227)(293,0.934788235007227)(293,0.934788235007227)(293,0.934788235007227)(293,0.934788235007227)(294,0.934889336576926)(294,0.934889336576926)(294,0.934889336576926)(295,0.934989802627311)(295,0.934989802627311)(296,0.935086320396076)(296,0.935086320396076)(296,0.935086320396076)(297,0.935184529174303)(297,0.935184529174303)(299,0.935379295634482)(299,0.935379295634482)(300,0.935487569678589)(300,0.935487569678589)(300,0.935487569678589)(300,0.935487569678589)(300,0.935487569678589)(301,0.935586134078895)(301,0.935586134078895)(302,0.935683241432036)(302,0.935683241432036)(302,0.935683241432036)(302,0.935683241432036)(303,0.93577823558027)(303,0.93577823558027)(303,0.93577823558027)(303,0.93577823558027)(303,0.93577823558027)(304,0.935870874909296)(305,0.935981720856055)(305,0.935981720856055)(306,0.936048918303784)(306,0.936048918303784)(306,0.936048918303784)(306,0.936048918303784)(306,0.936048918303784)(307,0.936134149521146)(307,0.936134149521146)(307,0.936134149521146)(308,0.936216908739423)(308,0.936216908739423)(309,0.93629718545123)(309,0.93629718545123)(309,0.93629718545123)(310,0.936358460218166)(310,0.936358460218166)(311,0.936449499733632)(311,0.936449499733632)(312,0.936521415819782)(312,0.936521415819782)(313,0.936590806488635)(314,0.936658101032171)(314,0.936658101032171)(314,0.936658101032171)(314,0.936658101032171)(314,0.936658101032171)(315,0.936723851250188)(315,0.936723851250188)(315,0.936723851250188)(316,0.936788567381344)(316,0.936788567381344)(316,0.936788567381344)(316,0.936788567381344)(316,0.936788567381344)(317,0.936852845110949)(317,0.936852845110949)(318,0.936916937526551)(318,0.936916937526551)(319,0.936983875154879)(319,0.936983875154879)(320,0.937051515042851)(321,0.93712015260113)(321,0.93712015260113)(321,0.93712015260113)(321,0.93712015260113)(321,0.93712015260113)(322,0.937189843634425)(323,0.937260457595923)(323,0.937260457595923)(323,0.937260457595923)(323,0.937260457595923)(324,0.937331790616095)(324,0.937331790616095)(325,0.937403435304088)(325,0.937403435304088)(325,0.937403435304088)(326,0.937474840290392)(326,0.937474840290392)(326,0.937474840290392)(327,0.937552238131523)(327,0.937552238131523)(328,0.937626773371286)(329,0.937694461384792)(329,0.937694461384792)(330,0.937761183275321)(331,0.937827184712241)(331,0.937827184712241)(331,0.937827184712241)(332,0.937892562301913)(332,0.937892562301913)(332,0.937892562301913)(332,0.937892562301913)(332,0.937892562301913)(333,0.937962061626514)(333,0.937962061626514)(334,0.938027578756025)(334,0.938027578756025)(335,0.938093283481599)(335,0.938093283481599)(335,0.938093283481599)(336,0.938168166772309)(336,0.938168166772309)(336,0.938168166772309)(336,0.938168166772309)(337,0.938225694355226)(338,0.93830547151828)(338,0.93830547151828)(338,0.93830547151828)(339,0.938360580528777)(339,0.938360580528777)(339,0.938360580528777)(340,0.938429551618198)(340,0.938429551618198)(340,0.938429551618198)(340,0.938429551618198)(342,0.938571750493247)(343,0.938645665560288)(344,0.938697997744301)(344,0.938697997744301)(345,0.938801028221319)(345,0.938801028221319)(346,0.938883481310555)(346,0.938883481310555)(347,0.938994220632478)(347,0.938994220632478)(347,0.938994220632478)(347,0.938994220632478)(348,0.939083431342537)(348,0.939083431342537)(349,0.939176110254813)(349,0.939176110254813)(349,0.939176110254813)(349,0.939176110254813)(350,0.939272349987267)(350,0.939272349987267)(351,0.939372120140375)(351,0.939372120140375)(352,0.939475293040564)(353,0.939569232136106)(353,0.939569232136106)(353,0.939569232136106)(353,0.939569232136106)(354,0.93968184769229)(354,0.93968184769229)(354,0.93968184769229)(355,0.939797175801459)(355,0.939797175801459)(356,0.939914894987963)(356,0.939914894987963)(357,0.940036768326032)(357,0.940036768326032)(358,0.940156713854856)(359,0.940278442927191)(359,0.940278442927191)(360,0.940400909279599)(360,0.940400909279599)(360,0.940400909279599)(361,0.940523780944138)(362,0.940647140775341)(362,0.940647140775341)(363,0.940770672336747)(363,0.940770672336747)(364,0.940894111064379)(364,0.940894111064379)(364,0.940894111064379)(364,0.940894111064379)(365,0.941017124135091)(365,0.941017124135091)(365,0.941017124135091)(365,0.941017124135091)(366,0.941139401524147)(366,0.941139401524147)(367,0.941266638794667)(367,0.941266638794667)(367,0.941266638794667)(367,0.941266638794667)(367,0.941266638794667)(368,0.941387320226942)(368,0.941387320226942)(369,0.941499892982037)(369,0.941499892982037)(369,0.941499892982037)(370,0.941617096108515)(370,0.941617096108515)(370,0.941617096108515)(370,0.941617096108515)(370,0.941617096108515)(370,0.941617096108515)(370,0.941617096108515)(371,0.941740447152059)(371,0.941740447152059)(372,0.941846551666915)(373,0.941967009707575)(373,0.941967009707575)(374,0.942077368382727)(374,0.942077368382727)(374,0.942077368382727)(374,0.942077368382727)(374,0.942077368382727)(374,0.942077368382727)(374,0.942077368382727)(375,0.942185879642803)(375,0.942185879642803)(376,0.942292672818459)(376,0.942292672818459)(377,0.942397735332375)(377,0.942397735332375)(378,0.942501120623488)(378,0.942501120623488)(378,0.942501120623488)(378,0.942501120623488)(378,0.942501120623488)(378,0.942501120623488)(378,0.942501120623488)(379,0.942595519990944)(380,0.942696330384733)(380,0.942696330384733)(380,0.942696330384733)(381,0.942795822886535)(381,0.942795822886535)(381,0.942795822886535)(382,0.94289410686233)(382,0.94289410686233)(382,0.94289410686233)(383,0.942985492072449)(383,0.942985492072449)(383,0.942985492072449)(384,0.943082568416734)(384,0.943082568416734)(384,0.943082568416734)(385,0.94317910308381)(385,0.94317910308381)(385,0.94317910308381)(385,0.94317910308381)(385,0.94317910308381)(385,0.94317910308381)(386,0.943275198904385)(386,0.943275198904385)(387,0.943370872334121)(387,0.943370872334121)(387,0.943370872334121)(388,0.943466083869711)(389,0.943560824408739)(390,0.94365450012108)(390,0.94365450012108)(390,0.94365450012108)(391,0.943749044231695)(392,0.943842833015413)(392,0.943842833015413)(392,0.943842833015413)(392,0.943842833015413)(392,0.943842833015413)(392,0.943842833015413)(392,0.943842833015413)(393,0.943936917860752)(393,0.943936917860752)(394,0.944030271008052)(394,0.944030271008052)(394,0.944030271008052)(395,0.944122992027728)(395,0.944122992027728)(395,0.944122992027728)(395,0.944122992027728)(395,0.944122992027728)(396,0.944214917403092)(396,0.944214917403092)(397,0.944305802030547)(397,0.944305802030547)(398,0.944395397563303)(398,0.944395397563303)(399,0.944485693852462)(399,0.944485693852462)
 %(401,0.944657453334161)(401,0.944657453334161)(401,0.944657453334161)(401,0.944657453334161)(401,0.944657453334161)(401,0.944657453334161)(402,0.944736454167789)(402,0.944736454167789)(403,0.944821196927252)(403,0.944821196927252)(404,0.94490029870112)(405,0.944972583375254)(405,0.944972583375254)(406,0.945053144363382)(406,0.945053144363382)(407,0.94512684154692)(408,0.945198647277905)(408,0.945198647277905)(409,0.945268560331253)(409,0.945268560331253)(411,0.945408308977651)(411,0.945408308977651)(411,0.945408308977651)(411,0.945408308977651)(412,0.945468856781945)(412,0.945468856781945)(412,0.945468856781945)(412,0.945468856781945)(413,0.94553343926536)(413,0.94553343926536)(413,0.94553343926536)(414,0.945597322196766)(414,0.945597322196766)(415,0.945660747227194)(416,0.945723965366795)(416,0.945723965366795)(416,0.945723965366795)(417,0.945787237267002)(417,0.945787237267002)(417,0.945787237267002)(417,0.945787237267002)(417,0.945787237267002)(417,0.945787237267002)(418,0.9458508095406)(419,0.945914814077907)(419,0.945914814077907)(420,0.945978891317321)(420,0.945978891317321)(421,0.946043805315419)(421,0.946043805315419)(421,0.946043805315419)(422,0.9461096778952)(422,0.9461096778952)(423,0.946176574742381)(423,0.946176574742381)(423,0.946176574742381)(423,0.946176574742381)(424,0.946244493966012)(424,0.946244493966012)(425,0.946313381067814)(425,0.946313381067814)(426,0.94638311401538)(426,0.94638311401538)(426,0.94638311401538)(426,0.94638311401538)(427,0.946453685889465)(427,0.946453685889465)(427,0.946453685889465)(427,0.946453685889465)(427,0.946453685889465)(428,0.946525084372605)(428,0.946525084372605)(429,0.946597271631574)(429,0.946597271631574)(429,0.946597271631574)(430,0.946670184249663)(430,0.946670184249663)(431,0.94674364812043)(432,0.946817423248669)(433,0.946891187422726)(434,0.946964716842736)(434,0.946964716842736)(434,0.946964716842736)(435,0.947037839776952)(435,0.947037839776952)(435,0.947037839776952)(436,0.947110388733007)(436,0.947110388733007)(436,0.947110388733007)(436,0.947110388733007)(436,0.947110388733007)(437,0.947182163726667)(437,0.947182163726667)(438,0.947245076963646)(438,0.947245076963646)(439,0.947322967519866)(440,0.947392044664776)(440,0.947392044664776)(441,0.947460286903572)(441,0.947460286903572)(441,0.947460286903572)(441,0.947460286903572)(441,0.947460286903572)(441,0.947460286903572)(442,0.94752764723418)(442,0.94752764723418)(442,0.94752764723418)(443,0.94759413211329)(443,0.94759413211329)(443,0.94759413211329)(444,0.947664004512676)(444,0.947664004512676)(445,0.947728360207688)(445,0.947728360207688)(446,0.947792241081021)(446,0.947792241081021)(446,0.947792241081021)(446,0.947792241081021)(446,0.947792241081021)(446,0.947792241081021)(446,0.947792241081021)(446,0.947792241081021)(448,0.947919116602405)(448,0.947919116602405)(448,0.947919116602405)(449,0.947981056568098)(450,0.948043061875552)(450,0.948043061875552)(450,0.948043061875552)(451,0.948108349798987)(451,0.948108349798987)(451,0.948108349798987)(451,0.948108349798987)(452,0.94817137337366)(452,0.94817137337366)(453,0.948230859692371)(453,0.948230859692371)(453,0.948230859692371)(454,0.948294456929469)(454,0.948294456929469)(456,0.94842398873187)(456,0.94842398873187)(456,0.94842398873187)(457,0.948490191832307)(457,0.948490191832307)(457,0.948490191832307)(457,0.948490191832307)(458,0.948554281590503)(458,0.948554281590503)(460,0.948694611057989)(460,0.948694611057989)(460,0.948694611057989)(460,0.948694611057989)(461,0.948763728823843)(462,0.948828798903317)(463,0.948897238689887)(463,0.948897238689887)(464,0.948967506802842)(464,0.948967506802842)(464,0.948967506802842)(465,0.949032809026039)(465,0.949032809026039)(465,0.949032809026039)(467,0.949157307273251)(468,0.949215981626661)(468,0.949215981626661)(470,0.949327263074816)(471,0.94937991222194)(471,0.94937991222194)(471,0.94937991222194)(472,0.949431826455634)(472,0.949431826455634)(472,0.949431826455634)(473,0.949480843893816)(473,0.949480843893816)(473,0.949480843893816)(474,0.949529711653238)(474,0.949529711653238)(474,0.949529711653238)(474,0.949529711653238)(475,0.949577208119978)(476,0.949623985800978)(476,0.949623985800978)(476,0.949623985800978)(477,0.949672117128658)(477,0.949672117128658)(477,0.949672117128658)(478,0.949718710994607)(478,0.949718710994607)(478,0.949718710994607)(478,0.949718710994607)(478,0.949718710994607)(479,0.949764989001752)(480,0.949810908023758)(480,0.949810908023758)(481,0.949856452084605)(481,0.949856452084605)(481,0.949856452084605)(481,0.949856452084605)(481,0.949856452084605)(482,0.949901613088314)(482,0.949901613088314)(482,0.949901613088314)(484,0.949990330332869)(484,0.949990330332869)(484,0.949990330332869)(484,0.949990330332869)(485,0.950039763219163)(486,0.950082208469944)(486,0.950082208469944)(486,0.950082208469944)(486,0.950082208469944)(487,0.950123620462965)(487,0.950123620462965)(488,0.950170717821384)(488,0.950170717821384)(488,0.950170717821384)(488,0.950170717821384)(488,0.950170717821384)(488,0.950170717821384)(490,0.950248036299166)(491,0.950291766376354)(491,0.950291766376354)(491,0.950291766376354)(491,0.950291766376354)(491,0.950291766376354)(492,0.950327254677343)(493,0.950361522907843)(493,0.950361522907843)(494,0.950388170555405)(494,0.950388170555405)(494,0.950388170555405)(495,0.950420566169453)(495,0.950420566169453)(495,0.950420566169453)(495,0.950420566169453)(495,0.950420566169453)(496,0.95045210112978)(496,0.95045210112978)(496,0.95045210112978)(496,0.95045210112978)(497,0.950482939033348)(497,0.950482939033348)(498,0.950513164218836)(500,0.950572101114518)(500,0.950572101114518)(500,0.950572101114518)(500,0.950572101114518)(500,0.950572101114518)(501,0.950601102205993)(501,0.950601102205993)(501,0.950601102205993)(502,0.95063003055344)(502,0.95063003055344)(502,0.95063003055344)(503,0.950659104668244)(504,0.950688410828647)(504,0.950688410828647)(505,0.950714909064388)(505,0.950714909064388)(505,0.950714909064388)(505,0.950714909064388)(506,0.950745542352787)(506,0.950745542352787)(506,0.950745542352787)(507,0.950779196330858)(508,0.95081733033243)(508,0.95081733033243)(509,0.950854633271211)(509,0.950854633271211)(509,0.950854633271211)(510,0.95090051378696)(510,0.95090051378696)(511,0.950941334885675)(511,0.950941334885675)(511,0.950941334885675)(512,0.950982778319549)(512,0.950982778319549)(513,0.951011825508632)(513,0.951011825508632)(513,0.951011825508632)(514,0.951052919995166)(514,0.951052919995166)(515,0.95109474838894)(517,0.951197844164733)(518,0.951225288996182)(518,0.951225288996182)(518,0.951225288996182)(518,0.951225288996182)(518,0.951225288996182)(519,0.95127079677028)(519,0.95127079677028)(519,0.95127079677028)(520,0.951317531750138)(520,0.951317531750138)(520,0.951317531750138)(522,0.951448635209508)(523,0.951512067120886)(524,0.951573164678356)(524,0.951573164678356)(524,0.951573164678356)(524,0.951573164678356)(524,0.951573164678356)(525,0.951631651151461)(525,0.951631651151461)(526,0.951687494971741)(526,0.951687494971741)(526,0.951687494971741)(526,0.951687494971741)(527,0.951740818335252)(527,0.951740818335252)(527,0.951740818335252)(528,0.951791827235929)(528,0.951791827235929)(529,0.951840773880612)(529,0.951840773880612)(529,0.951840773880612)(530,0.951887914426481)(530,0.951887914426481)(530,0.951887914426481)(531,0.951933493174719)(531,0.951933493174719)(531,0.951933493174719)(531,0.951933493174719)(532,0.95197773294013)(533,0.952020829251745)(533,0.952020829251745)(533,0.952020829251745)(533,0.952020829251745)(535,0.952104254611248)(535,0.952104254611248)(535,0.952104254611248)(535,0.952104254611248)(536,0.95214486148952)(536,0.95214486148952)(536,0.95214486148952)(536,0.95214486148952)(538,0.952224403117952)(538,0.952224403117952)(538,0.952224403117952)(539,0.952263507560446)(539,0.952263507560446)(539,0.952263507560446)(540,0.952302260465378)(540,0.952302260465378)(540,0.952302260465378)(541,0.952340717716904)(541,0.952340717716904)(542,0.952378925822984)(542,0.952378925822984)(542,0.952378925822984)(543,0.952416923188126)(543,0.952416923188126)(543,0.952416923188126)(543,0.952416923188126)(544,0.952454740106917)(544,0.952454740106917)(544,0.952454740106917)(544,0.952454740106917)(544,0.952454740106917)(544,0.952454740106917)(545,0.952492400501982)(545,0.952492400501982)(545,0.952492400501982)(545,0.952492400501982)(545,0.952492400501982)(548,0.952604627185151)(548,0.952604627185151)(548,0.952604627185151)(549,0.952641827538437)(549,0.952641827538437)(550,0.952678937810328)(550,0.952678937810328)(550,0.952678937810328)(551,0.952715962914968)(551,0.952715962914968)(551,0.952715962914968)(552,0.952752907212379)(552,0.952752907212379)(552,0.952752907212379)(552,0.952752907212379)(552,0.952752907212379)(553,0.952789773657518)(553,0.952789773657518)(554,0.952826561948452)(554,0.952826561948452)(555,0.952863271378543)(556,0.952899902496912)(556,0.952899902496912)(556,0.952899902496912)(556,0.952899902496912)(556,0.952899902496912)(556,0.952899902496912)(556,0.952899902496912)(557,0.9529364548354)(557,0.9529364548354)(558,0.952972924677132)(559,0.953009307625925)(559,0.953009307625925)(559,0.953009307625925)(559,0.953009307625925)(560,0.953045599381395)(560,0.953045599381395)(560,0.953045599381395)(561,0.95308179478169)(561,0.95308179478169)(561,0.95308179478169)(561,0.95308179478169)(562,0.953117886561998)(563,0.953153865331655)(563,0.953153865331655)(563,0.953153865331655)(563,0.953153865331655)(563,0.953153865331655)(564,0.953189721435023)(564,0.953189721435023)(565,0.953225445103168)(565,0.953225445103168)(565,0.953225445103168)(565,0.953225445103168)(566,0.953261025205445)(566,0.953261025205445)(566,0.953261025205445)(567,0.953296449605817)(567,0.953296449605817)(567,0.953296449605817)(567,0.953296449605817)(567,0.953296449605817)(567,0.953296449605817)(567,0.953296449605817)(567,0.953296449605817)(568,0.953331706496799)(568,0.953331706496799)(569,0.953366785284404)(569,0.953366785284404)(569,0.953366785284404)(570,0.953401675273727)(570,0.953401675273727)(571,0.95343636443456)(571,0.95343636443456)(573,0.953505088042898)(573,0.953505088042898)(573,0.953505088042898)(575,0.953572846983547)(575,0.953572846983547)(575,0.953572846983547)(576,0.953606327529818)(576,0.953606327529818)(576,0.953606327529818)(577,0.953639522192884)(577,0.953639522192884)(577,0.953639522192884)(577,0.953639522192884)(578,0.953672417024098)(578,0.953672417024098)(578,0.953672417024098)(579,0.95370499861541)(579,0.95370499861541)(579,0.95370499861541)(579,0.95370499861541)(580,0.953737255942739)(581,0.95376917945238)(581,0.95376917945238)(583,0.95383199114179)(583,0.95383199114179)(584,0.953862864984462)(585,0.953893376702103)(586,0.953923523332726)(588,0.953982715761175)(591,0.954068752755272)(591,0.954068752755272)(592,0.954096704082241)(592,0.954096704082241)(593,0.954124293423111)(593,0.954124293423111)(593,0.954124293423111)(593,0.954124293423111)(593,0.954124293423111)(594,0.954151520191929)(595,0.954178382720145)(595,0.954178382720145)(595,0.954178382720145)(595,0.954178382720145)(597,0.95423101602886)(597,0.95423101602886)(597,0.95423101602886)(598,0.954256791974855)(598,0.954256791974855)(598,0.954256791974855)(598,0.954256791974855)(599,0.954282215485697)(599,0.954282215485697)(600,0.95430729189829)(600,0.95430729189829) 
};
\addlegendentry{\acl};

\addplot [
color=orange,
densely dotted,
line width=1.0pt,
]
coordinates{
 %(30,0)(60,0.191144411068378)(90,0.390199197570515)(120,0.436736391085857)(150,0.456985349116281)
 (180,0.498680610667282)(210,0.54585241494589)(240,0.607455834959923)(270,0.646894602147306)(300,0.694320343659307)(330,0.743956174841398)(360,0.779347815356398)(390,0.795141604828387) 
};
\addlegendentry{\rstr};

\addplot [
color=green!50!black,
densely dotted,
line width=1.0pt,
]
coordinates{
 (30,0.680883568275444)(60,0.826542286763675)(90,0.85769466223284)(120,0.888551982117022)(150,0.902110189887598)(180,0.906775763173944)(210,0.915166612050159)(240,0.922350503323697)(270,0.928855704228774)(300,0.931342053869797)(330,0.934476828901283)(360,0.937312490495522)(390,0.940393578536127) 
};
\addlegendentry{\bstr};

\addplot [
color=blue,
solid,
line width=1.3pt,
]
coordinates{
 (30,0.685158283421578)(60,0.8040881097992)(90,0.857371126439146)(120,0.884366584067808)(150,0.9017811628251)(180,0.915767739391697)(210,0.913041954378934)(240,0.926176913912651)(270,0.938366106239837)(300,0.932453824179565)(330,0.935643971973499)(360,0.941335022538945)(390,0.943129606867063) 
};
\addlegendentry{\bacl};

\end{axis}
\end{tikzpicture}%

%% This file was created by matlab2tikz v0.2.3.
% Copyright (c) 2008--2012, Nico Schlömer <nico.schloemer@gmail.com>
% All rights reserved.
% 
% 
% 
\begin{tikzpicture}

\begin{axis}[%
tick label style={font=\tiny},
label style={font=\tiny},
label shift={-4pt},
xlabel shift={-6pt},
legend style={font=\tiny},
view={0}{90},
width=\figurewidth,
height=\figureheight,
scale only axis,
xmin=0, xmax=400,
xlabel={Samples},
ymin=0.37, ymax=1,
ylabel={$F_1$-score},
axis lines*=left,
legend cell align=left,
legend style={at={(1.03,0)},anchor=south east,fill=none,draw=none,align=left,row sep=-0.2em},
clip=false]

\addplot [
color=red,
densely dotted,
line width=1.0pt,
]
coordinates{
 (12,0.513242753000793)(12,0.513242753000793)(12,0.513242753000793)(12,0.513242753000793)(12,0.513242753000793)(12,0.513242753000793)(12,0.513242753000793)(12,0.513242753000793)(13,0.563071802960045)(13,0.563071802960045)(13,0.563071802960045)(13,0.563071802960045)(13,0.563071802960045)(13,0.563071802960045)(13,0.563071802960045)(13,0.563071802960045)(13,0.563071802960045)(14,0.609412089731185)(14,0.609412089731185)(14,0.609412089731185)(14,0.609412089731185)(14,0.609412089731185)(14,0.609412089731185)(14,0.609412089731185)(14,0.609412089731185)(14,0.609412089731185)(15,0.654045205749223)(15,0.654045205749223)(15,0.654045205749223)(15,0.654045205749223)(15,0.654045205749223)(15,0.654045205749223)(15,0.654045205749223)(15,0.654045205749223)(15,0.654045205749223)(15,0.654045205749223)(15,0.654045205749223)(15,0.654045205749223)(16,0.693800447635594)(16,0.693800447635594)(16,0.693800447635594)(16,0.693800447635594)(16,0.693800447635594)(16,0.693800447635594)(16,0.693800447635594)(16,0.693800447635594)(16,0.693800447635594)(16,0.693800447635594)(16,0.693800447635594)(16,0.693800447635594)(16,0.693800447635594)(16,0.693800447635594)(16,0.693800447635594)(16,0.693800447635594)(16,0.693800447635594)(16,0.693800447635594)(16,0.693800447635594)(16,0.693800447635594)(16,0.693800447635594)(17,0.719964994412613)(17,0.719964994412613)(17,0.719964994412613)(17,0.719964994412613)(17,0.719964994412613)(17,0.719964994412613)(17,0.719964994412613)(17,0.719964994412613)(17,0.719964994412613)(17,0.719964994412613)(17,0.719964994412613)(17,0.719964994412613)(17,0.719964994412613)(17,0.719964994412613)(17,0.719964994412613)(17,0.719964994412613)(17,0.719964994412613)(17,0.719964994412613)(17,0.719964994412613)(17,0.719964994412613)(17,0.719964994412613)(17,0.719964994412613)(17,0.719964994412613)(18,0.738956378279693)(18,0.738956378279693)(18,0.738956378279693)(18,0.738956378279693)(18,0.738956378279693)(18,0.738956378279693)(18,0.738956378279693)(18,0.738956378279693)(18,0.738956378279693)(18,0.738956378279693)(18,0.738956378279693)(18,0.738956378279693)(18,0.738956378279693)(18,0.738956378279693)(18,0.738956378279693)(18,0.738956378279693)(18,0.738956378279693)(18,0.738956378279693)(18,0.738956378279693)(18,0.738956378279693)(18,0.738956378279693)(18,0.738956378279693)(18,0.738956378279693)(18,0.738956378279693)(18,0.738956378279693)(18,0.738956378279693)(18,0.738956378279693)(19,0.756578089042498)(19,0.756578089042498)(19,0.756578089042498)(19,0.756578089042498)(19,0.756578089042498)(19,0.756578089042498)(19,0.756578089042498)(19,0.756578089042498)(19,0.756578089042498)(19,0.756578089042498)(19,0.756578089042498)(19,0.756578089042498)(19,0.756578089042498)(19,0.756578089042498)(19,0.756578089042498)(19,0.756578089042498)(19,0.756578089042498)(19,0.756578089042498)(19,0.756578089042498)(19,0.756578089042498)(19,0.756578089042498)(19,0.756578089042498)(19,0.756578089042498)(19,0.756578089042498)(19,0.756578089042498)(19,0.756578089042498)(19,0.756578089042498)(19,0.756578089042498)(19,0.756578089042498)(20,0.766154555098519)(20,0.766154555098519)(20,0.766154555098519)(20,0.766154555098519)(20,0.766154555098519)(20,0.766154555098519)(20,0.766154555098519)(20,0.766154555098519)(20,0.766154555098519)(20,0.766154555098519)(20,0.766154555098519)(20,0.766154555098519)(20,0.766154555098519)(20,0.766154555098519)(20,0.766154555098519)(20,0.766154555098519)(20,0.766154555098519)(20,0.766154555098519)(20,0.766154555098519)(20,0.766154555098519)(20,0.766154555098519)(20,0.766154555098519)(20,0.766154555098519)(20,0.766154555098519)(20,0.766154555098519)(20,0.766154555098519)(20,0.766154555098519)(20,0.766154555098519)(20,0.766154555098519)(20,0.766154555098519)(20,0.766154555098519)(20,0.766154555098519)(20,0.766154555098519)(20,0.766154555098519)(20,0.766154555098519)(20,0.766154555098519)(20,0.766154555098519)(20,0.766154555098519)(20,0.766154555098519)(21,0.772168694536424)(21,0.772168694536424)(21,0.772168694536424)(21,0.772168694536424)(21,0.772168694536424)(21,0.772168694536424)(21,0.772168694536424)(21,0.772168694536424)(21,0.772168694536424)(21,0.772168694536424)(21,0.772168694536424)(21,0.772168694536424)(21,0.772168694536424)(21,0.772168694536424)(21,0.772168694536424)(21,0.772168694536424)(21,0.772168694536424)(21,0.772168694536424)(21,0.772168694536424)(21,0.772168694536424)(21,0.772168694536424)(21,0.772168694536424)(21,0.772168694536424)(21,0.772168694536424)(21,0.772168694536424)(21,0.772168694536424)(21,0.772168694536424)(21,0.772168694536424)(21,0.772168694536424)(21,0.772168694536424)(21,0.772168694536424)(21,0.772168694536424)(21,0.772168694536424)(21,0.772168694536424)(21,0.772168694536424)(21,0.772168694536424)(21,0.772168694536424)(21,0.772168694536424)(21,0.772168694536424)(21,0.772168694536424)(21,0.772168694536424)(21,0.772168694536424)(21,0.772168694536424)(21,0.772168694536424)(21,0.772168694536424)(22,0.775246122095812)(22,0.775246122095812)(22,0.775246122095812)(22,0.775246122095812)(22,0.775246122095812)(22,0.775246122095812)(22,0.775246122095812)(22,0.775246122095812)(22,0.775246122095812)(22,0.775246122095812)(22,0.775246122095812)(22,0.775246122095812)(22,0.775246122095812)(22,0.775246122095812)(22,0.775246122095812)(22,0.775246122095812)(22,0.775246122095812)(22,0.775246122095812)(22,0.775246122095812)(22,0.775246122095812)(22,0.775246122095812)(22,0.775246122095812)(22,0.775246122095812)(22,0.775246122095812)(22,0.775246122095812)(22,0.775246122095812)(22,0.775246122095812)(22,0.775246122095812)(22,0.775246122095812)(22,0.775246122095812)(22,0.775246122095812)(22,0.775246122095812)(22,0.775246122095812)(22,0.775246122095812)(22,0.775246122095812)(22,0.775246122095812)(22,0.775246122095812)(22,0.775246122095812)(22,0.775246122095812)(22,0.775246122095812)(22,0.775246122095812)(22,0.775246122095812)(23,0.779159487806811)(23,0.779159487806811)(23,0.779159487806811)(23,0.779159487806811)(23,0.779159487806811)(23,0.779159487806811)(23,0.779159487806811)(23,0.779159487806811)(23,0.779159487806811)(23,0.779159487806811)(23,0.779159487806811)(23,0.779159487806811)(23,0.779159487806811)(23,0.779159487806811)(23,0.779159487806811)(23,0.779159487806811)(23,0.779159487806811)(23,0.779159487806811)(23,0.779159487806811)(23,0.779159487806811)(23,0.779159487806811)(23,0.779159487806811)(23,0.779159487806811)(23,0.779159487806811)(23,0.779159487806811)(23,0.779159487806811)(23,0.779159487806811)(23,0.779159487806811)(23,0.779159487806811)(23,0.779159487806811)(23,0.779159487806811)(23,0.779159487806811)(23,0.779159487806811)(23,0.779159487806811)(23,0.779159487806811)(23,0.779159487806811)(23,0.779159487806811)(23,0.779159487806811)(23,0.779159487806811)(23,0.779159487806811)(23,0.779159487806811)(23,0.779159487806811)(23,0.779159487806811)(23,0.779159487806811)(23,0.779159487806811)(23,0.779159487806811)(23,0.779159487806811)(23,0.779159487806811)(23,0.779159487806811)(24,0.784632419978595)(24,0.784632419978595)(24,0.784632419978595)(24,0.784632419978595)(24,0.784632419978595)(24,0.784632419978595)(24,0.784632419978595)(24,0.784632419978595)(24,0.784632419978595)(24,0.784632419978595)(24,0.784632419978595)(24,0.784632419978595)(24,0.784632419978595)(24,0.784632419978595)(24,0.784632419978595)(24,0.784632419978595)(24,0.784632419978595)(24,0.784632419978595)(24,0.784632419978595)(24,0.784632419978595)(24,0.784632419978595)(24,0.784632419978595)(24,0.784632419978595)(24,0.784632419978595)(24,0.784632419978595)(24,0.784632419978595)(24,0.784632419978595)(24,0.784632419978595)(24,0.784632419978595)(24,0.784632419978595)(24,0.784632419978595)(24,0.784632419978595)(24,0.784632419978595)(24,0.784632419978595)(24,0.784632419978595)(24,0.784632419978595)(24,0.784632419978595)(24,0.784632419978595)(24,0.784632419978595)(24,0.784632419978595)(24,0.784632419978595)(24,0.784632419978595)(24,0.784632419978595)(24,0.784632419978595)(24,0.784632419978595)(24,0.784632419978595)(24,0.784632419978595)(24,0.784632419978595)(24,0.784632419978595)(24,0.784632419978595)(24,0.784632419978595)(25,0.790027700330571)(25,0.790027700330571)(25,0.790027700330571)(25,0.790027700330571)(25,0.790027700330571)(25,0.790027700330571)(25,0.790027700330571)(25,0.790027700330571)(25,0.790027700330571)(25,0.790027700330571)(25,0.790027700330571)(25,0.790027700330571)(25,0.790027700330571)(25,0.790027700330571)(25,0.790027700330571)(25,0.790027700330571)(25,0.790027700330571)(25,0.790027700330571)(25,0.790027700330571)(25,0.790027700330571)(25,0.790027700330571)(25,0.790027700330571)(25,0.790027700330571)(25,0.790027700330571)(25,0.790027700330571)(25,0.790027700330571)(25,0.790027700330571)(25,0.790027700330571)(25,0.790027700330571)(25,0.790027700330571)(25,0.790027700330571)(25,0.790027700330571)(25,0.790027700330571)(25,0.790027700330571)(25,0.790027700330571)(25,0.790027700330571)(25,0.790027700330571)(25,0.790027700330571)(25,0.790027700330571)(25,0.790027700330571)(26,0.79447497049185)(26,0.79447497049185)(26,0.79447497049185)(26,0.79447497049185)(26,0.79447497049185)(26,0.79447497049185)(26,0.79447497049185)(26,0.79447497049185)(26,0.79447497049185)(26,0.79447497049185)(26,0.79447497049185)(26,0.79447497049185)(26,0.79447497049185)(26,0.79447497049185)(26,0.79447497049185)(26,0.79447497049185)(26,0.79447497049185)(26,0.79447497049185)(26,0.79447497049185)(26,0.79447497049185)(26,0.79447497049185)(26,0.79447497049185)(26,0.79447497049185)(26,0.79447497049185)(26,0.79447497049185)(26,0.79447497049185)(26,0.79447497049185)(26,0.79447497049185)(26,0.79447497049185)(26,0.79447497049185)(26,0.79447497049185)(26,0.79447497049185)(26,0.79447497049185)(26,0.79447497049185)(26,0.79447497049185)(26,0.79447497049185)(26,0.79447497049185)(26,0.79447497049185)(26,0.79447497049185)(26,0.79447497049185)(26,0.79447497049185)(27,0.796074899938778)(27,0.796074899938778)(27,0.796074899938778)(27,0.796074899938778)(27,0.796074899938778)(27,0.796074899938778)(27,0.796074899938778)(27,0.796074899938778)(27,0.796074899938778)(27,0.796074899938778)(27,0.796074899938778)(27,0.796074899938778)(27,0.796074899938778)(27,0.796074899938778)(27,0.796074899938778)(27,0.796074899938778)(27,0.796074899938778)(27,0.796074899938778)(27,0.796074899938778)(27,0.796074899938778)(27,0.796074899938778)(27,0.796074899938778)(27,0.796074899938778)(27,0.796074899938778)(27,0.796074899938778)(27,0.796074899938778)(27,0.796074899938778)(27,0.796074899938778)(27,0.796074899938778)(27,0.796074899938778)(27,0.796074899938778)(27,0.796074899938778)(27,0.796074899938778)(27,0.796074899938778)(27,0.796074899938778)(27,0.796074899938778)(27,0.796074899938778)(27,0.796074899938778)(27,0.796074899938778)(28,0.795999379478335)(28,0.795999379478335)(28,0.795999379478335)(28,0.795999379478335)(28,0.795999379478335)(28,0.795999379478335)(28,0.795999379478335)(28,0.795999379478335)(28,0.795999379478335)(28,0.795999379478335)(28,0.795999379478335)(28,0.795999379478335)(28,0.795999379478335)(28,0.795999379478335)(28,0.795999379478335)(28,0.795999379478335)(28,0.795999379478335)(28,0.795999379478335)(28,0.795999379478335)(28,0.795999379478335)(28,0.795999379478335)(28,0.795999379478335)(28,0.795999379478335)(28,0.795999379478335)(28,0.795999379478335)(28,0.795999379478335)(28,0.795999379478335)(28,0.795999379478335)(28,0.795999379478335)(28,0.795999379478335)(28,0.795999379478335)(28,0.795999379478335)(28,0.795999379478335)(28,0.795999379478335)(28,0.795999379478335)(28,0.795999379478335)(28,0.795999379478335)(28,0.795999379478335)(28,0.795999379478335)(29,0.798409856541205)(29,0.798409856541205)(29,0.798409856541205)(29,0.798409856541205)(29,0.798409856541205)(29,0.798409856541205)(29,0.798409856541205)(29,0.798409856541205)(29,0.798409856541205)(29,0.798409856541205)(29,0.798409856541205)(29,0.798409856541205)(29,0.798409856541205)(29,0.798409856541205)(29,0.798409856541205)(29,0.798409856541205)(29,0.798409856541205)(29,0.798409856541205)(29,0.798409856541205)(29,0.798409856541205)(29,0.798409856541205)(29,0.798409856541205)(29,0.798409856541205)(29,0.798409856541205)(29,0.798409856541205)(29,0.798409856541205)(29,0.798409856541205)(29,0.798409856541205)(29,0.798409856541205)(29,0.798409856541205)(29,0.798409856541205)(29,0.798409856541205)(29,0.798409856541205)(29,0.798409856541205)(29,0.798409856541205)(29,0.798409856541205)(29,0.798409856541205)(30,0.801907554024585)(30,0.801907554024585)(30,0.801907554024585)(30,0.801907554024585)(30,0.801907554024585)(30,0.801907554024585)(30,0.801907554024585)(30,0.801907554024585)(30,0.801907554024585)(30,0.801907554024585)(30,0.801907554024585)(30,0.801907554024585)(30,0.801907554024585)(30,0.801907554024585)(30,0.801907554024585)(30,0.801907554024585)(30,0.801907554024585)(30,0.801907554024585)(30,0.801907554024585)(30,0.801907554024585)(30,0.801907554024585)(30,0.801907554024585)(30,0.801907554024585)(30,0.801907554024585)(30,0.801907554024585)(30,0.801907554024585)(30,0.801907554024585)(30,0.801907554024585)(30,0.801907554024585)(31,0.806226472417736)(31,0.806226472417736)(31,0.806226472417736)(31,0.806226472417736)(31,0.806226472417736)(31,0.806226472417736)(31,0.806226472417736)(31,0.806226472417736)(31,0.806226472417736)(31,0.806226472417736)(31,0.806226472417736)(31,0.806226472417736)(31,0.806226472417736)(31,0.806226472417736)(31,0.806226472417736)(31,0.806226472417736)(31,0.806226472417736)(31,0.806226472417736)(31,0.806226472417736)(31,0.806226472417736)(31,0.806226472417736)(31,0.806226472417736)(31,0.806226472417736)(31,0.806226472417736)(31,0.806226472417736)(31,0.806226472417736)(31,0.806226472417736)(31,0.806226472417736)(31,0.806226472417736)(31,0.806226472417736)(31,0.806226472417736)(31,0.806226472417736)(31,0.806226472417736)(31,0.806226472417736)(31,0.806226472417736)(32,0.810075821596545)(32,0.810075821596545)(32,0.810075821596545)(32,0.810075821596545)(32,0.810075821596545)(32,0.810075821596545)(32,0.810075821596545)(32,0.810075821596545)(32,0.810075821596545)(32,0.810075821596545)(32,0.810075821596545)(32,0.810075821596545)(32,0.810075821596545)(32,0.810075821596545)(32,0.810075821596545)(32,0.810075821596545)(32,0.810075821596545)(32,0.810075821596545)(32,0.810075821596545)(32,0.810075821596545)(32,0.810075821596545)(32,0.810075821596545)(32,0.810075821596545)(32,0.810075821596545)(32,0.810075821596545)(32,0.810075821596545)(32,0.810075821596545)(32,0.810075821596545)(32,0.810075821596545)(32,0.810075821596545)(32,0.810075821596545)(32,0.810075821596545)(32,0.810075821596545)(32,0.810075821596545)(32,0.810075821596545)(32,0.810075821596545)(32,0.810075821596545)(33,0.81300247138301)(33,0.81300247138301)(33,0.81300247138301)(33,0.81300247138301)(33,0.81300247138301)(33,0.81300247138301)(33,0.81300247138301)(33,0.81300247138301)(33,0.81300247138301)(33,0.81300247138301)(33,0.81300247138301)(33,0.81300247138301)(33,0.81300247138301)(33,0.81300247138301)(33,0.81300247138301)(33,0.81300247138301)(33,0.81300247138301)(33,0.81300247138301)(33,0.81300247138301)(33,0.81300247138301)(33,0.81300247138301)(33,0.81300247138301)(33,0.81300247138301)(33,0.81300247138301)(33,0.81300247138301)(33,0.81300247138301)(33,0.81300247138301)(33,0.81300247138301)(33,0.81300247138301)(33,0.81300247138301)(33,0.81300247138301)(33,0.81300247138301)(33,0.81300247138301)(33,0.81300247138301)(33,0.81300247138301)(34,0.815634494830405)(34,0.815634494830405)(34,0.815634494830405)(34,0.815634494830405)(34,0.815634494830405)(34,0.815634494830405)(34,0.815634494830405)(34,0.815634494830405)(34,0.815634494830405)(34,0.815634494830405)(34,0.815634494830405)(34,0.815634494830405)(34,0.815634494830405)(34,0.815634494830405)(34,0.815634494830405)(34,0.815634494830405)(34,0.815634494830405)(34,0.815634494830405)(34,0.815634494830405)(34,0.815634494830405)(34,0.815634494830405)(34,0.815634494830405)(34,0.815634494830405)(34,0.815634494830405)(34,0.815634494830405)(34,0.815634494830405)(34,0.815634494830405)(34,0.815634494830405)(35,0.817866616393611)(35,0.817866616393611)(35,0.817866616393611)(35,0.817866616393611)(35,0.817866616393611)(35,0.817866616393611)(35,0.817866616393611)(35,0.817866616393611)(35,0.817866616393611)(35,0.817866616393611)(35,0.817866616393611)(35,0.817866616393611)(35,0.817866616393611)(35,0.817866616393611)(35,0.817866616393611)(35,0.817866616393611)(35,0.817866616393611)(35,0.817866616393611)(35,0.817866616393611)(35,0.817866616393611)(35,0.817866616393611)(35,0.817866616393611)(36,0.820093974807631)(36,0.820093974807631)(36,0.820093974807631)(36,0.820093974807631)(36,0.820093974807631)(36,0.820093974807631)(36,0.820093974807631)(36,0.820093974807631)(36,0.820093974807631)(36,0.820093974807631)(36,0.820093974807631)(36,0.820093974807631)(36,0.820093974807631)(36,0.820093974807631)(36,0.820093974807631)(36,0.820093974807631)(36,0.820093974807631)(36,0.820093974807631)(36,0.820093974807631)(36,0.820093974807631)(36,0.820093974807631)(36,0.820093974807631)(36,0.820093974807631)(37,0.823267231828805)(37,0.823267231828805)(37,0.823267231828805)(37,0.823267231828805)(37,0.823267231828805)(37,0.823267231828805)(37,0.823267231828805)(37,0.823267231828805)(37,0.823267231828805)(37,0.823267231828805)(37,0.823267231828805)(37,0.823267231828805)(37,0.823267231828805)(37,0.823267231828805)(37,0.823267231828805)(37,0.823267231828805)(37,0.823267231828805)(37,0.823267231828805)(37,0.823267231828805)(37,0.823267231828805)(37,0.823267231828805)(37,0.823267231828805)(37,0.823267231828805)(37,0.823267231828805)(37,0.823267231828805)(37,0.823267231828805)(38,0.826497293362125)(38,0.826497293362125)(38,0.826497293362125)(38,0.826497293362125)(38,0.826497293362125)(38,0.826497293362125)(38,0.826497293362125)(38,0.826497293362125)(38,0.826497293362125)(38,0.826497293362125)(38,0.826497293362125)(38,0.826497293362125)(38,0.826497293362125)(38,0.826497293362125)(38,0.826497293362125)(38,0.826497293362125)(38,0.826497293362125)(38,0.826497293362125)(38,0.826497293362125)(38,0.826497293362125)(38,0.826497293362125)(38,0.826497293362125)(38,0.826497293362125)(38,0.826497293362125)(38,0.826497293362125)(39,0.82935832921807)(39,0.82935832921807)(39,0.82935832921807)(39,0.82935832921807)(39,0.82935832921807)(39,0.82935832921807)(39,0.82935832921807)(39,0.82935832921807)(39,0.82935832921807)(39,0.82935832921807)(39,0.82935832921807)(39,0.82935832921807)(39,0.82935832921807)(39,0.82935832921807)(39,0.82935832921807)(39,0.82935832921807)(39,0.82935832921807)(39,0.82935832921807)(39,0.82935832921807)(39,0.82935832921807)(39,0.82935832921807)(39,0.82935832921807)(39,0.82935832921807)(39,0.82935832921807)(39,0.82935832921807)(39,0.82935832921807)(39,0.82935832921807)(40,0.83140553472798)(40,0.83140553472798)(40,0.83140553472798)(40,0.83140553472798)(40,0.83140553472798)(40,0.83140553472798)(40,0.83140553472798)(40,0.83140553472798)(40,0.83140553472798)(40,0.83140553472798)(40,0.83140553472798)(40,0.83140553472798)(40,0.83140553472798)(40,0.83140553472798)(40,0.83140553472798)(40,0.83140553472798)(40,0.83140553472798)(40,0.83140553472798)(40,0.83140553472798)(40,0.83140553472798)(41,0.832979516062166)(41,0.832979516062166)(41,0.832979516062166)(41,0.832979516062166)(41,0.832979516062166)(41,0.832979516062166)(41,0.832979516062166)(41,0.832979516062166)(41,0.832979516062166)(41,0.832979516062166)(41,0.832979516062166)(41,0.832979516062166)(41,0.832979516062166)(41,0.832979516062166)(41,0.832979516062166)(41,0.832979516062166)(41,0.832979516062166)(41,0.832979516062166)(41,0.832979516062166)(41,0.832979516062166)(41,0.832979516062166)(41,0.832979516062166)(42,0.835016811547283)(42,0.835016811547283)(42,0.835016811547283)(42,0.835016811547283)(42,0.835016811547283)(42,0.835016811547283)(42,0.835016811547283)(42,0.835016811547283)(42,0.835016811547283)(42,0.835016811547283)(42,0.835016811547283)(42,0.835016811547283)(42,0.835016811547283)(42,0.835016811547283)(42,0.835016811547283)(42,0.835016811547283)(43,0.836782158554706)(43,0.836782158554706)(43,0.836782158554706)(43,0.836782158554706)(43,0.836782158554706)(43,0.836782158554706)(43,0.836782158554706)(43,0.836782158554706)(43,0.836782158554706)(43,0.836782158554706)(43,0.836782158554706)(43,0.836782158554706)(43,0.836782158554706)(43,0.836782158554706)(43,0.836782158554706)(44,0.838048687172316)(44,0.838048687172316)(44,0.838048687172316)(44,0.838048687172316)(44,0.838048687172316)(44,0.838048687172316)(44,0.838048687172316)(44,0.838048687172316)(44,0.838048687172316)(44,0.838048687172316)(44,0.838048687172316)(44,0.838048687172316)(44,0.838048687172316)(44,0.838048687172316)(44,0.838048687172316)(44,0.838048687172316)(44,0.838048687172316)(44,0.838048687172316)(44,0.838048687172316)(44,0.838048687172316)(44,0.838048687172316)(44,0.838048687172316)(44,0.838048687172316)(44,0.838048687172316)(44,0.838048687172316)(44,0.838048687172316)(45,0.83911360475653)(45,0.83911360475653)(45,0.83911360475653)(45,0.83911360475653)(45,0.83911360475653)(45,0.83911360475653)(45,0.83911360475653)(45,0.83911360475653)(45,0.83911360475653)(45,0.83911360475653)(45,0.83911360475653)(45,0.83911360475653)(45,0.83911360475653)(45,0.83911360475653)(45,0.83911360475653)(45,0.83911360475653)(45,0.83911360475653)(45,0.83911360475653)(45,0.83911360475653)(45,0.83911360475653)(45,0.83911360475653)(45,0.83911360475653)(45,0.83911360475653)(46,0.840165489288512)(46,0.840165489288512)(46,0.840165489288512)(46,0.840165489288512)(46,0.840165489288512)(46,0.840165489288512)(46,0.840165489288512)(46,0.840165489288512)(46,0.840165489288512)(46,0.840165489288512)(46,0.840165489288512)(46,0.840165489288512)(46,0.840165489288512)(46,0.840165489288512)(46,0.840165489288512)(46,0.840165489288512)(46,0.840165489288512)(46,0.840165489288512)(46,0.840165489288512)(46,0.840165489288512)(47,0.841519534640496)(47,0.841519534640496)(47,0.841519534640496)(47,0.841519534640496)(47,0.841519534640496)(47,0.841519534640496)(47,0.841519534640496)(47,0.841519534640496)(47,0.841519534640496)(47,0.841519534640496)(47,0.841519534640496)(47,0.841519534640496)(47,0.841519534640496)(47,0.841519534640496)(47,0.841519534640496)(47,0.841519534640496)(47,0.841519534640496)(47,0.841519534640496)(47,0.841519534640496)(47,0.841519534640496)(47,0.841519534640496)(47,0.841519534640496)(48,0.842937547168656)(48,0.842937547168656)(48,0.842937547168656)(48,0.842937547168656)(48,0.842937547168656)(48,0.842937547168656)(48,0.842937547168656)(48,0.842937547168656)(48,0.842937547168656)(49,0.844463742087862)(49,0.844463742087862)(49,0.844463742087862)(49,0.844463742087862)(49,0.844463742087862)(49,0.844463742087862)(49,0.844463742087862)(49,0.844463742087862)(49,0.844463742087862)(49,0.844463742087862)(49,0.844463742087862)(49,0.844463742087862)(49,0.844463742087862)(49,0.844463742087862)(49,0.844463742087862)(49,0.844463742087862)(49,0.844463742087862)(49,0.844463742087862)(49,0.844463742087862)(49,0.844463742087862)(50,0.845997992207294)(50,0.845997992207294)(50,0.845997992207294)(50,0.845997992207294)(50,0.845997992207294)(50,0.845997992207294)(50,0.845997992207294)(50,0.845997992207294)(50,0.845997992207294)(50,0.845997992207294)(50,0.845997992207294)(50,0.845997992207294)(50,0.845997992207294)(50,0.845997992207294)(50,0.845997992207294)(50,0.845997992207294)(50,0.845997992207294)(50,0.845997992207294)(50,0.845997992207294)(51,0.847649750433022)(51,0.847649750433022)(51,0.847649750433022)(51,0.847649750433022)(51,0.847649750433022)(51,0.847649750433022)(51,0.847649750433022)(51,0.847649750433022)(51,0.847649750433022)(51,0.847649750433022)(51,0.847649750433022)(51,0.847649750433022)(51,0.847649750433022)(51,0.847649750433022)(52,0.850004427720027)(52,0.850004427720027)(52,0.850004427720027)(52,0.850004427720027)(52,0.850004427720027)(52,0.850004427720027)(52,0.850004427720027)(52,0.850004427720027)(52,0.850004427720027)(52,0.850004427720027)(52,0.850004427720027)(52,0.850004427720027)(52,0.850004427720027)(52,0.850004427720027)(52,0.850004427720027)(52,0.850004427720027)(53,0.852261417456051)(53,0.852261417456051)(53,0.852261417456051)(53,0.852261417456051)(53,0.852261417456051)(53,0.852261417456051)(53,0.852261417456051)(54,0.854751441376187)(54,0.854751441376187)(54,0.854751441376187)(54,0.854751441376187)(54,0.854751441376187)(54,0.854751441376187)(54,0.854751441376187)(54,0.854751441376187)(54,0.854751441376187)(54,0.854751441376187)(54,0.854751441376187)(54,0.854751441376187)(54,0.854751441376187)(54,0.854751441376187)(54,0.854751441376187)(55,0.857406982603522)(55,0.857406982603522)(55,0.857406982603522)(55,0.857406982603522)(55,0.857406982603522)(55,0.857406982603522)(55,0.857406982603522)(55,0.857406982603522)(55,0.857406982603522)(55,0.857406982603522)(55,0.857406982603522)(55,0.857406982603522)(55,0.857406982603522)(55,0.857406982603522)(55,0.857406982603522)(55,0.857406982603522)(55,0.857406982603522)(55,0.857406982603522)(56,0.859328278945926)(56,0.859328278945926)(56,0.859328278945926)(56,0.859328278945926)(56,0.859328278945926)(56,0.859328278945926)(56,0.859328278945926)(56,0.859328278945926)(56,0.859328278945926)(56,0.859328278945926)(56,0.859328278945926)(56,0.859328278945926)(56,0.859328278945926)(57,0.861063335311418)(57,0.861063335311418)(57,0.861063335311418)(57,0.861063335311418)(57,0.861063335311418)(57,0.861063335311418)(57,0.861063335311418)(57,0.861063335311418)(57,0.861063335311418)(57,0.861063335311418)(57,0.861063335311418)(57,0.861063335311418)(57,0.861063335311418)(57,0.861063335311418)(58,0.862397066154262)(58,0.862397066154262)(58,0.862397066154262)(58,0.862397066154262)(58,0.862397066154262)(58,0.862397066154262)(58,0.862397066154262)(58,0.862397066154262)(58,0.862397066154262)(58,0.862397066154262)(58,0.862397066154262)(58,0.862397066154262)(58,0.862397066154262)(58,0.862397066154262)(58,0.862397066154262)(59,0.863641311211628)(59,0.863641311211628)(59,0.863641311211628)(59,0.863641311211628)(59,0.863641311211628)(59,0.863641311211628)(59,0.863641311211628)(59,0.863641311211628)(59,0.863641311211628)(59,0.863641311211628)(60,0.864923564406168)(60,0.864923564406168)(60,0.864923564406168)(60,0.864923564406168)(60,0.864923564406168)(60,0.864923564406168)(60,0.864923564406168)(60,0.864923564406168)(60,0.864923564406168)(60,0.864923564406168)(60,0.864923564406168)(60,0.864923564406168)(61,0.866069771229191)(61,0.866069771229191)(61,0.866069771229191)(61,0.866069771229191)(61,0.866069771229191)(61,0.866069771229191)(61,0.866069771229191)(61,0.866069771229191)(61,0.866069771229191)(61,0.866069771229191)(61,0.866069771229191)(61,0.866069771229191)(61,0.866069771229191)(61,0.866069771229191)(61,0.866069771229191)(61,0.866069771229191)(61,0.866069771229191)(61,0.866069771229191)(62,0.86714028889457)(62,0.86714028889457)(62,0.86714028889457)(62,0.86714028889457)(62,0.86714028889457)(62,0.86714028889457)(62,0.86714028889457)(62,0.86714028889457)(62,0.86714028889457)(62,0.86714028889457)(62,0.86714028889457)(63,0.868379210237588)(63,0.868379210237588)(63,0.868379210237588)(63,0.868379210237588)(63,0.868379210237588)(63,0.868379210237588)(64,0.86945236496554)(64,0.86945236496554)(64,0.86945236496554)(64,0.86945236496554)(64,0.86945236496554)(64,0.86945236496554)(64,0.86945236496554)(64,0.86945236496554)(64,0.86945236496554)(64,0.86945236496554)(64,0.86945236496554)(64,0.86945236496554)(64,0.86945236496554)(64,0.86945236496554)(65,0.870630399980386)(65,0.870630399980386)(65,0.870630399980386)(65,0.870630399980386)(65,0.870630399980386)(65,0.870630399980386)(65,0.870630399980386)(65,0.870630399980386)(65,0.870630399980386)(65,0.870630399980386)(65,0.870630399980386)(65,0.870630399980386)(66,0.871661181225682)(66,0.871661181225682)(66,0.871661181225682)(66,0.871661181225682)(66,0.871661181225682)(66,0.871661181225682)(66,0.871661181225682)(66,0.871661181225682)(66,0.871661181225682)(66,0.871661181225682)(67,0.872777656661999)(67,0.872777656661999)(67,0.872777656661999)(67,0.872777656661999)(67,0.872777656661999)(67,0.872777656661999)(67,0.872777656661999)(67,0.872777656661999)(67,0.872777656661999)(67,0.872777656661999)(67,0.872777656661999)(67,0.872777656661999)(67,0.872777656661999)(67,0.872777656661999)(67,0.872777656661999)(68,0.874094440732809)(68,0.874094440732809)(68,0.874094440732809)(68,0.874094440732809)(68,0.874094440732809)(68,0.874094440732809)(68,0.874094440732809)(68,0.874094440732809)(69,0.875274710403873)(69,0.875274710403873)(69,0.875274710403873)(69,0.875274710403873)(69,0.875274710403873)(69,0.875274710403873)(69,0.875274710403873)(69,0.875274710403873)(69,0.875274710403873)(69,0.875274710403873)(70,0.876213199750608)(70,0.876213199750608)(70,0.876213199750608)(70,0.876213199750608)(70,0.876213199750608)(70,0.876213199750608)(70,0.876213199750608)(70,0.876213199750608)(70,0.876213199750608)(70,0.876213199750608)(70,0.876213199750608)(70,0.876213199750608)(70,0.876213199750608)(70,0.876213199750608)(70,0.876213199750608)(71,0.877107762438385)(71,0.877107762438385)(71,0.877107762438385)(71,0.877107762438385)(71,0.877107762438385)(71,0.877107762438385)(71,0.877107762438385)(71,0.877107762438385)(71,0.877107762438385)(71,0.877107762438385)(71,0.877107762438385)(71,0.877107762438385)(71,0.877107762438385)(71,0.877107762438385)(71,0.877107762438385)(71,0.877107762438385)(71,0.877107762438385)(71,0.877107762438385)(72,0.877849086914781)(72,0.877849086914781)(72,0.877849086914781)(72,0.877849086914781)(72,0.877849086914781)(72,0.877849086914781)(72,0.877849086914781)(72,0.877849086914781)(72,0.877849086914781)(72,0.877849086914781)(72,0.877849086914781)(73,0.878339661501157)(73,0.878339661501157)(73,0.878339661501157)(73,0.878339661501157)(74,0.87895365306575)(74,0.87895365306575)(74,0.87895365306575)(74,0.87895365306575)(74,0.87895365306575)(74,0.87895365306575)(74,0.87895365306575)(75,0.879497079213324)(75,0.879497079213324)(75,0.879497079213324)(75,0.879497079213324)(75,0.879497079213324)(75,0.879497079213324)(75,0.879497079213324)(75,0.879497079213324)(75,0.879497079213324)(76,0.879983979577299)(76,0.879983979577299)(76,0.879983979577299)(76,0.879983979577299)(76,0.879983979577299)(76,0.879983979577299)(76,0.879983979577299)(76,0.879983979577299)(77,0.880428644612614)(77,0.880428644612614)(77,0.880428644612614)(77,0.880428644612614)(77,0.880428644612614)(77,0.880428644612614)(77,0.880428644612614)(77,0.880428644612614)(77,0.880428644612614)(77,0.880428644612614)(77,0.880428644612614)(78,0.880884441235036)(78,0.880884441235036)(78,0.880884441235036)(78,0.880884441235036)(78,0.880884441235036)(78,0.880884441235036)(78,0.880884441235036)(78,0.880884441235036)(78,0.880884441235036)(78,0.880884441235036)(79,0.881525322533769)(79,0.881525322533769)(79,0.881525322533769)(79,0.881525322533769)(79,0.881525322533769)(79,0.881525322533769)(79,0.881525322533769)(79,0.881525322533769)(80,0.882069292779447)(80,0.882069292779447)(80,0.882069292779447)(80,0.882069292779447)(80,0.882069292779447)(80,0.882069292779447)(80,0.882069292779447)(80,0.882069292779447)(80,0.882069292779447)(80,0.882069292779447)(80,0.882069292779447)(81,0.882661011436284)(81,0.882661011436284)(81,0.882661011436284)(81,0.882661011436284)(81,0.882661011436284)(81,0.882661011436284)(82,0.883283510096945)(82,0.883283510096945)(82,0.883283510096945)(82,0.883283510096945)(82,0.883283510096945)(82,0.883283510096945)(83,0.883868112059667)(83,0.883868112059667)(83,0.883868112059667)(83,0.883868112059667)(83,0.883868112059667)(83,0.883868112059667)(84,0.884270575794794)(84,0.884270575794794)(84,0.884270575794794)(84,0.884270575794794)(84,0.884270575794794)(84,0.884270575794794)(84,0.884270575794794)(84,0.884270575794794)(84,0.884270575794794)(84,0.884270575794794)(85,0.884624618568871)(85,0.884624618568871)(85,0.884624618568871)(85,0.884624618568871)(85,0.884624618568871)(85,0.884624618568871)(85,0.884624618568871)(85,0.884624618568871)(85,0.884624618568871)(86,0.884937108886295)(86,0.884937108886295)(86,0.884937108886295)(86,0.884937108886295)(86,0.884937108886295)(86,0.884937108886295)(86,0.884937108886295)(86,0.884937108886295)(86,0.884937108886295)(87,0.885080252494528)(87,0.885080252494528)(87,0.885080252494528)(87,0.885080252494528)(87,0.885080252494528)(87,0.885080252494528)(87,0.885080252494528)(87,0.885080252494528)(87,0.885080252494528)(87,0.885080252494528)(87,0.885080252494528)(88,0.885296675031833)(88,0.885296675031833)(88,0.885296675031833)(88,0.885296675031833)(88,0.885296675031833)(88,0.885296675031833)(88,0.885296675031833)(89,0.885504136120846)(89,0.885504136120846)(89,0.885504136120846)(89,0.885504136120846)(89,0.885504136120846)(89,0.885504136120846)(89,0.885504136120846)(90,0.885617405637599)(90,0.885617405637599)(90,0.885617405637599)(90,0.885617405637599)(90,0.885617405637599)(90,0.885617405637599)(90,0.885617405637599)(91,0.885896246301623)(91,0.885896246301623)(91,0.885896246301623)(91,0.885896246301623)(91,0.885896246301623)(91,0.885896246301623)(91,0.885896246301623)(91,0.885896246301623)(91,0.885896246301623)(91,0.885896246301623)(92,0.88621120038927)(92,0.88621120038927)(92,0.88621120038927)(92,0.88621120038927)(92,0.88621120038927)(92,0.88621120038927)(92,0.88621120038927)(92,0.88621120038927)(92,0.88621120038927)(92,0.88621120038927)(92,0.88621120038927)(93,0.8866810990757)(93,0.8866810990757)(93,0.8866810990757)(93,0.8866810990757)(93,0.8866810990757)(93,0.8866810990757)(93,0.8866810990757)(93,0.8866810990757)(93,0.8866810990757)(93,0.8866810990757)(93,0.8866810990757)(94,0.887144966388669)(94,0.887144966388669)(94,0.887144966388669)(94,0.887144966388669)(94,0.887144966388669)(94,0.887144966388669)(94,0.887144966388669)(94,0.887144966388669)(94,0.887144966388669)(95,0.887618801414262)(95,0.887618801414262)(95,0.887618801414262)(95,0.887618801414262)(95,0.887618801414262)(95,0.887618801414262)(95,0.887618801414262)(95,0.887618801414262)(95,0.887618801414262)(95,0.887618801414262)(96,0.888085972647855)(96,0.888085972647855)(96,0.888085972647855)(96,0.888085972647855)(96,0.888085972647855)(96,0.888085972647855)(97,0.888555526147879)(97,0.888555526147879)(97,0.888555526147879)(97,0.888555526147879)(97,0.888555526147879)(97,0.888555526147879)(97,0.888555526147879)(97,0.888555526147879)(97,0.888555526147879)(97,0.888555526147879)(98,0.889203931927193)(98,0.889203931927193)(98,0.889203931927193)(98,0.889203931927193)(98,0.889203931927193)(98,0.889203931927193)(98,0.889203931927193)(99,0.889779126406695)(99,0.889779126406695)(99,0.889779126406695)(99,0.889779126406695)(99,0.889779126406695)(100,0.890410517944253)(100,0.890410517944253)(100,0.890410517944253)(100,0.890410517944253)(100,0.890410517944253)(100,0.890410517944253)(100,0.890410517944253)(100,0.890410517944253)(100,0.890410517944253)(100,0.890410517944253)(100,0.890410517944253)(100,0.890410517944253)(100,0.890410517944253)(100,0.890410517944253)(101,0.891098887988349)(101,0.891098887988349)(101,0.891098887988349)(101,0.891098887988349)(102,0.891861261990652)(102,0.891861261990652)(102,0.891861261990652)(102,0.891861261990652)(102,0.891861261990652)(103,0.89262418989616)(103,0.89262418989616)(103,0.89262418989616)(103,0.89262418989616)(103,0.89262418989616)(103,0.89262418989616)(103,0.89262418989616)(104,0.893483787450481)(104,0.893483787450481)(104,0.893483787450481)(104,0.893483787450481)(104,0.893483787450481)(104,0.893483787450481)(104,0.893483787450481)(104,0.893483787450481)(104,0.893483787450481)(105,0.894412274850794)(105,0.894412274850794)(105,0.894412274850794)(105,0.894412274850794)(105,0.894412274850794)(105,0.894412274850794)(105,0.894412274850794)(106,0.895370698323643)(106,0.895370698323643)(106,0.895370698323643)(107,0.896344780150986)(107,0.896344780150986)(107,0.896344780150986)(107,0.896344780150986)(107,0.896344780150986)(107,0.896344780150986)(107,0.896344780150986)(108,0.897336442355504)(108,0.897336442355504)(108,0.897336442355504)(108,0.897336442355504)(108,0.897336442355504)(109,0.89832614256793)(109,0.89832614256793)(109,0.89832614256793)(109,0.89832614256793)(109,0.89832614256793)(110,0.899215723414439)(110,0.899215723414439)(111,0.900091671229955)(111,0.900091671229955)(111,0.900091671229955)(111,0.900091671229955)(111,0.900091671229955)(111,0.900091671229955)(111,0.900091671229955)(111,0.900091671229955)(112,0.900895387171251)(112,0.900895387171251)(112,0.900895387171251)(112,0.900895387171251)(112,0.900895387171251)(112,0.900895387171251)(113,0.901621042139023)(113,0.901621042139023)(113,0.901621042139023)(113,0.901621042139023)(113,0.901621042139023)(113,0.901621042139023)(114,0.902265862459695)(114,0.902265862459695)(114,0.902265862459695)(114,0.902265862459695)(114,0.902265862459695)(114,0.902265862459695)(114,0.902265862459695)(114,0.902265862459695)(115,0.90282856860764)(115,0.90282856860764)(115,0.90282856860764)(115,0.90282856860764)(115,0.90282856860764)(115,0.90282856860764)(115,0.90282856860764)(115,0.90282856860764)(115,0.90282856860764)(116,0.903302394529468)(116,0.903302394529468)(116,0.903302394529468)(116,0.903302394529468)(116,0.903302394529468)(117,0.90366445589929)(117,0.90366445589929)(117,0.90366445589929)(117,0.90366445589929)(117,0.90366445589929)(117,0.90366445589929)(117,0.90366445589929)(118,0.904052072986014)(118,0.904052072986014)(118,0.904052072986014)(118,0.904052072986014)(118,0.904052072986014)(119,0.90442940929921)(119,0.90442940929921)(119,0.90442940929921)(119,0.90442940929921)(119,0.90442940929921)(119,0.90442940929921)(120,0.904820989336109)(120,0.904820989336109)(120,0.904820989336109)(120,0.904820989336109)(121,0.905290476626895)(121,0.905290476626895)(121,0.905290476626895)(121,0.905290476626895)(121,0.905290476626895)(122,0.905762975609968)(122,0.905762975609968)(122,0.905762975609968)(122,0.905762975609968)(122,0.905762975609968)(122,0.905762975609968)(123,0.906288210386602)(123,0.906288210386602)(123,0.906288210386602)(123,0.906288210386602)(123,0.906288210386602)(123,0.906288210386602)(124,0.906857921242519)(124,0.906857921242519)(124,0.906857921242519)(124,0.906857921242519)(124,0.906857921242519)(124,0.906857921242519)(124,0.906857921242519)(125,0.907458971897663)(125,0.907458971897663)(125,0.907458971897663)(125,0.907458971897663)(125,0.907458971897663)(125,0.907458971897663)(125,0.907458971897663)(125,0.907458971897663)(125,0.907458971897663)(126,0.908073697422993)(126,0.908073697422993)(126,0.908073697422993)(126,0.908073697422993)(126,0.908073697422993)(127,0.908682117119428)(127,0.908682117119428)(127,0.908682117119428)(127,0.908682117119428)(127,0.908682117119428)(127,0.908682117119428)(127,0.908682117119428)(127,0.908682117119428)(128,0.909266995551112)(128,0.909266995551112)(128,0.909266995551112)(128,0.909266995551112)(128,0.909266995551112)(128,0.909266995551112)(128,0.909266995551112)(129,0.909804698477688)(129,0.909804698477688)(129,0.909804698477688)(129,0.909804698477688)(129,0.909804698477688)(129,0.909804698477688)(130,0.910271424406664)(130,0.910271424406664)(130,0.910271424406664)(130,0.910271424406664)(130,0.910271424406664)(130,0.910271424406664)(130,0.910271424406664)(131,0.910666087030618)(131,0.910666087030618)(131,0.910666087030618)(132,0.910994912722979)(132,0.910994912722979)(132,0.910994912722979)(132,0.910994912722979)(132,0.910994912722979)(132,0.910994912722979)(132,0.910994912722979)(133,0.911265224144584)(133,0.911265224144584)(133,0.911265224144584)(133,0.911265224144584)(133,0.911265224144584)(133,0.911265224144584)(134,0.911479562692085)(134,0.911479562692085)(134,0.911479562692085)(134,0.911479562692085)(134,0.911479562692085)(135,0.911647070464452)(135,0.911647070464452)(135,0.911647070464452)(135,0.911647070464452)(135,0.911647070464452)(135,0.911647070464452)(135,0.911647070464452)(135,0.911647070464452)(136,0.911782068522749)(136,0.911782068522749)(136,0.911782068522749)(136,0.911782068522749)(136,0.911782068522749)(137,0.911902289743262)(137,0.911902289743262)(137,0.911902289743262)(137,0.911902289743262)(138,0.912025538806745)(138,0.912025538806745)(138,0.912025538806745)(138,0.912025538806745)(138,0.912025538806745)(138,0.912025538806745)(138,0.912025538806745)(138,0.912025538806745)(138,0.912025538806745)(139,0.912157017168855)(139,0.912157017168855)(139,0.912157017168855)(139,0.912157017168855)(139,0.912157017168855)(139,0.912157017168855)(140,0.912288966028396)(140,0.912288966028396)(140,0.912288966028396)(140,0.912288966028396)(141,0.912411892748327)(141,0.912411892748327)(141,0.912411892748327)(141,0.912411892748327)(141,0.912411892748327)(141,0.912411892748327)(141,0.912411892748327)(141,0.912411892748327)(142,0.912531494087763)(142,0.912531494087763)(142,0.912531494087763)(142,0.912531494087763)(142,0.912531494087763)(142,0.912531494087763)(143,0.912659214864361)(143,0.912659214864361)(143,0.912659214864361)(143,0.912659214864361)(143,0.912659214864361)(143,0.912659214864361)(143,0.912659214864361)(143,0.912659214864361)(143,0.912659214864361)(144,0.912797002438885)(144,0.912797002438885)(144,0.912797002438885)(144,0.912797002438885)(145,0.912970339199763)(145,0.912970339199763)(145,0.912970339199763)(145,0.912970339199763)(145,0.912970339199763)(145,0.912970339199763)(145,0.912970339199763)(145,0.912970339199763)(146,0.913049215410552)(146,0.913049215410552)(146,0.913049215410552)(146,0.913049215410552)(147,0.913085221356563)(147,0.913085221356563)(147,0.913085221356563)(147,0.913085221356563)(147,0.913085221356563)(148,0.913100331670334)(148,0.913100331670334)(148,0.913100331670334)(148,0.913100331670334)(148,0.913100331670334)(148,0.913100331670334)(148,0.913100331670334)(148,0.913100331670334)(149,0.91312152334848)(149,0.91312152334848)(149,0.91312152334848)(149,0.91312152334848)(149,0.91312152334848)(149,0.91312152334848)(150,0.913170983171252)(150,0.913170983171252)(150,0.913170983171252)(150,0.913170983171252)(150,0.913170983171252)(150,0.913170983171252)(150,0.913170983171252)(150,0.913170983171252)(150,0.913170983171252)(151,0.913453143866604)(151,0.913453143866604)(151,0.913453143866604)(152,0.913646758974046)(152,0.913646758974046)(152,0.913646758974046)(152,0.913646758974046)(152,0.913646758974046)(152,0.913646758974046)(152,0.913646758974046)(152,0.913646758974046)(152,0.913646758974046)(152,0.913646758974046)(153,0.913917254304116)(153,0.913917254304116)(153,0.913917254304116)(153,0.913917254304116)(153,0.913917254304116)(154,0.914265043301814)(154,0.914265043301814)(154,0.914265043301814)(154,0.914265043301814)(154,0.914265043301814)(154,0.914265043301814)(154,0.914265043301814)(154,0.914265043301814)(155,0.914675968349094)(155,0.914675968349094)(155,0.914675968349094)(155,0.914675968349094)(155,0.914675968349094)(155,0.914675968349094)(155,0.914675968349094)(156,0.915138446186076)(156,0.915138446186076)(156,0.915138446186076)(156,0.915138446186076)(156,0.915138446186076)(156,0.915138446186076)(157,0.915640033422324)(157,0.915640033422324)(157,0.915640033422324)(157,0.915640033422324)(157,0.915640033422324)(157,0.915640033422324)(158,0.916175342762926)(158,0.916175342762926)(158,0.916175342762926)(158,0.916175342762926)(158,0.916175342762926)(158,0.916175342762926)(158,0.916175342762926)(159,0.916696768984411)(159,0.916696768984411)(159,0.916696768984411)(159,0.916696768984411)(159,0.916696768984411)(159,0.916696768984411)(159,0.916696768984411)(159,0.916696768984411)(160,0.917326693682256)(160,0.917326693682256)(160,0.917326693682256)(160,0.917326693682256)(160,0.917326693682256)(161,0.917867490864265)(161,0.917867490864265)(161,0.917867490864265)(161,0.917867490864265)(161,0.917867490864265)(161,0.917867490864265)(161,0.917867490864265)(161,0.917867490864265)(161,0.917867490864265)(162,0.918341190066385)(162,0.918341190066385)(162,0.918341190066385)(162,0.918341190066385)(162,0.918341190066385)(162,0.918341190066385)(162,0.918341190066385)(162,0.918341190066385)(163,0.918717931029981)(163,0.918717931029981)(163,0.918717931029981)(164,0.918974578453551)(164,0.918974578453551)(164,0.918974578453551)(164,0.918974578453551)(164,0.918974578453551)(164,0.918974578453551)(165,0.918892313269438)(165,0.918892313269438)(165,0.918892313269438)(165,0.918892313269438)(165,0.918892313269438)(166,0.919004286565191)(166,0.919004286565191)(166,0.919004286565191)(166,0.919004286565191)(166,0.919004286565191)(166,0.919004286565191)(166,0.919004286565191)(167,0.919208466633211)(167,0.919208466633211)(167,0.919208466633211)(167,0.919208466633211)(167,0.919208466633211)(167,0.919208466633211)(167,0.919208466633211)(167,0.919208466633211)(168,0.919209525369934)(168,0.919209525369934)(168,0.919209525369934)(168,0.919209525369934)(168,0.919209525369934)(169,0.919227589419664)(169,0.919227589419664)(169,0.919227589419664)(169,0.919227589419664)(169,0.919227589419664)(169,0.919227589419664)(169,0.919227589419664)(169,0.919227589419664)(169,0.919227589419664)(170,0.919228938879919)(170,0.919228938879919)(170,0.919228938879919)(170,0.919228938879919)(170,0.919228938879919)(170,0.919228938879919)(170,0.919228938879919)(170,0.919228938879919)(170,0.919228938879919)(171,0.919302490922463)(171,0.919302490922463)(171,0.919302490922463)(171,0.919302490922463)(171,0.919302490922463)(171,0.919302490922463)(171,0.919302490922463)(172,0.919447206153786)(172,0.919447206153786)(172,0.919447206153786)(172,0.919447206153786)(172,0.919447206153786)(172,0.919447206153786)(172,0.919447206153786)(172,0.919447206153786)(172,0.919447206153786)(173,0.919678970076311)(173,0.919678970076311)(173,0.919678970076311)(173,0.919678970076311)(173,0.919678970076311)(174,0.920003216764254)(174,0.920003216764254)(174,0.920003216764254)(175,0.920147093836222)(175,0.920147093836222)(175,0.920147093836222)(175,0.920147093836222)(175,0.920147093836222)(176,0.920589996613363)(176,0.920589996613363)(176,0.920589996613363)(176,0.920589996613363)(176,0.920589996613363)(176,0.920589996613363)(176,0.920589996613363)(177,0.92106950351799)(177,0.92106950351799)(177,0.92106950351799)(178,0.921693998148347)(178,0.921693998148347)(178,0.921693998148347)(178,0.921693998148347)(178,0.921693998148347)(178,0.921693998148347)(178,0.921693998148347)(179,0.922058867817332)(179,0.922058867817332)(179,0.922058867817332)(180,0.922369993301596)(180,0.922369993301596)(180,0.922369993301596)(180,0.922369993301596)(180,0.922369993301596)(180,0.922369993301596)(181,0.9226598000701)(181,0.9226598000701)(181,0.9226598000701)(181,0.9226598000701)(181,0.9226598000701)(181,0.9226598000701)(181,0.9226598000701)(182,0.922879363294031)(182,0.922879363294031)(182,0.922879363294031)(182,0.922879363294031)(183,0.923010624099443)(183,0.923010624099443)(183,0.923010624099443)(183,0.923010624099443)(183,0.923010624099443)(183,0.923010624099443)(183,0.923010624099443)(183,0.923010624099443)(183,0.923010624099443)(183,0.923010624099443)(183,0.923010624099443)(184,0.923081888901675)(184,0.923081888901675)(184,0.923081888901675)(184,0.923081888901675)(184,0.923081888901675)(184,0.923081888901675)(184,0.923081888901675)(184,0.923081888901675)(185,0.923142748599112)(185,0.923142748599112)(185,0.923142748599112)(185,0.923142748599112)(185,0.923142748599112)(185,0.923142748599112)(186,0.923233115335294)(186,0.923233115335294)(186,0.923233115335294)(186,0.923233115335294)(186,0.923233115335294)(186,0.923233115335294)(186,0.923233115335294)(187,0.923366134857737)(187,0.923366134857737)(187,0.923366134857737)(187,0.923366134857737)(187,0.923366134857737)(188,0.923539277155194)(188,0.923539277155194)(188,0.923539277155194)(188,0.923539277155194)(188,0.923539277155194)(188,0.923539277155194)(188,0.923539277155194)(188,0.923539277155194)(189,0.92387069681485)(189,0.92387069681485)(189,0.92387069681485)(189,0.92387069681485)(189,0.92387069681485)(189,0.92387069681485)(189,0.92387069681485)(189,0.92387069681485)(189,0.92387069681485)(190,0.924032343083996)(190,0.924032343083996)(190,0.924032343083996)(190,0.924032343083996)(191,0.924164300915139)(191,0.924164300915139)(191,0.924164300915139)(191,0.924164300915139)(192,0.924230284390243)(192,0.924230284390243)(192,0.924230284390243)(192,0.924230284390243)(192,0.924230284390243)(192,0.924230284390243)(193,0.924206051309703)(193,0.924206051309703)(193,0.924206051309703)(193,0.924206051309703)(193,0.924206051309703)(193,0.924206051309703)(193,0.924206051309703)(193,0.924206051309703)(193,0.924206051309703)(193,0.924206051309703)(194,0.92408993528439)(194,0.92408993528439)(194,0.92408993528439)(195,0.923921577475897)(195,0.923921577475897)(195,0.923921577475897)(195,0.923921577475897)(195,0.923921577475897)(196,0.923686836089653)(196,0.923686836089653)(196,0.923686836089653)(196,0.923686836089653)(196,0.923686836089653)(196,0.923686836089653)(196,0.923686836089653)(196,0.923686836089653)(196,0.923686836089653)(196,0.923686836089653)(197,0.92340494913761)(197,0.92340494913761)(197,0.92340494913761)(197,0.92340494913761)(197,0.92340494913761)(197,0.92340494913761)(197,0.92340494913761)(198,0.923096399506605)(198,0.923096399506605)(198,0.923096399506605)(198,0.923096399506605)(198,0.923096399506605)(198,0.923096399506605)(198,0.923096399506605)(198,0.923096399506605)(198,0.923096399506605)(198,0.923096399506605)(198,0.923096399506605)(198,0.923096399506605)(199,0.922788035664489)(199,0.922788035664489)(199,0.922788035664489)(199,0.922788035664489)(199,0.922788035664489)(199,0.922788035664489)(199,0.922788035664489)(199,0.922788035664489)(199,0.922788035664489)(200,0.922501962114206)(200,0.922501962114206)(200,0.922501962114206)(200,0.922501962114206)(200,0.922501962114206)(200,0.922501962114206)(200,0.922501962114206)(201,0.922252395594164)(201,0.922252395594164)(201,0.922252395594164)(201,0.922252395594164)(201,0.922252395594164)(201,0.922252395594164)(201,0.922252395594164)(201,0.922252395594164)(201,0.922252395594164)(201,0.922252395594164)(202,0.921987535368166)(202,0.921987535368166)(202,0.921987535368166)(202,0.921987535368166)(202,0.921987535368166)(203,0.921866979232156)(203,0.921866979232156)(203,0.921866979232156)(203,0.921866979232156)(203,0.921866979232156)(203,0.921866979232156)(204,0.921862204062474)(204,0.921862204062474)(204,0.921862204062474)(204,0.921862204062474)(204,0.921862204062474)(205,0.922018885850143)(205,0.922018885850143)(205,0.922018885850143)(205,0.922018885850143)(205,0.922018885850143)(205,0.922018885850143)(205,0.922018885850143)(206,0.922404613114075)(206,0.922404613114075)(206,0.922404613114075)(206,0.922404613114075)(206,0.922404613114075)(206,0.922404613114075)(206,0.922404613114075)(206,0.922404613114075)(206,0.922404613114075)(206,0.922404613114075)(207,0.92287352706863)(207,0.92287352706863)(207,0.92287352706863)(207,0.92287352706863)(207,0.92287352706863)(207,0.92287352706863)(207,0.92287352706863)(207,0.92287352706863)(208,0.923395728059796)(208,0.923395728059796)(208,0.923395728059796)(208,0.923395728059796)(208,0.923395728059796)(208,0.923395728059796)(209,0.923946230294363)(209,0.923946230294363)(209,0.923946230294363)(209,0.923946230294363)(209,0.923946230294363)(210,0.924521795051165)(210,0.924521795051165)(210,0.924521795051165)(210,0.924521795051165)(210,0.924521795051165)(210,0.924521795051165)(210,0.924521795051165)(210,0.924521795051165)(210,0.924521795051165)(211,0.92507066762613)(211,0.92507066762613)(211,0.92507066762613)(211,0.92507066762613)(211,0.92507066762613)(211,0.92507066762613)(211,0.92507066762613)(211,0.92507066762613)(211,0.92507066762613)(212,0.925593336605208)(212,0.925593336605208)(212,0.925593336605208)(212,0.925593336605208)(212,0.925593336605208)(212,0.925593336605208)(212,0.925593336605208)(212,0.925593336605208)(212,0.925593336605208)(212,0.925593336605208)(212,0.925593336605208)(212,0.925593336605208)(213,0.92609027066118)(213,0.92609027066118)(213,0.92609027066118)(213,0.92609027066118)(213,0.92609027066118)(213,0.92609027066118)(213,0.92609027066118)(213,0.92609027066118)(214,0.926557249575485)(214,0.926557249575485)(214,0.926557249575485)(214,0.926557249575485)(214,0.926557249575485)(215,0.926909574017588)(215,0.926909574017588)(215,0.926909574017588)(215,0.926909574017588)(215,0.926909574017588)(215,0.926909574017588)(215,0.926909574017588)(216,0.9273384048181)(216,0.9273384048181)(216,0.9273384048181)(216,0.9273384048181)(216,0.9273384048181)(216,0.9273384048181)(216,0.9273384048181)(216,0.9273384048181)(216,0.9273384048181)(216,0.9273384048181)(216,0.9273384048181)(216,0.9273384048181)(217,0.927795265164317)(217,0.927795265164317)(217,0.927795265164317)(217,0.927795265164317)(217,0.927795265164317)(217,0.927795265164317)(218,0.928300989943222)(218,0.928300989943222)(218,0.928300989943222)(218,0.928300989943222)(218,0.928300989943222)(218,0.928300989943222)(219,0.928858880520682)(219,0.928858880520682)(219,0.928858880520682)(220,0.929448392740186)(220,0.929448392740186)(220,0.929448392740186)(220,0.929448392740186)(220,0.929448392740186)(220,0.929448392740186)(220,0.929448392740186)(221,0.930009018963276)(221,0.930009018963276)(221,0.930009018963276)(222,0.930486980244315)(222,0.930486980244315)(222,0.930486980244315)(223,0.930590901576675)(223,0.930590901576675)(223,0.930590901576675)(223,0.930590901576675)(223,0.930590901576675)(223,0.930590901576675)(223,0.930590901576675)(223,0.930590901576675)(223,0.930590901576675)(223,0.930590901576675)(224,0.93114736483215)(224,0.93114736483215)(224,0.93114736483215)(224,0.93114736483215)(225,0.931330264124905)(225,0.931330264124905)(225,0.931330264124905)(225,0.931330264124905)(225,0.931330264124905)(225,0.931330264124905)(225,0.931330264124905)(225,0.931330264124905)(225,0.931330264124905)(225,0.931330264124905)(226,0.931425315510514)(226,0.931425315510514)(226,0.931425315510514)(226,0.931425315510514)(226,0.931425315510514)(226,0.931425315510514)(226,0.931425315510514)(226,0.931425315510514)(226,0.931425315510514)(227,0.931485377316195)(227,0.931485377316195)(227,0.931485377316195)(227,0.931485377316195)(227,0.931485377316195)(227,0.931485377316195)(227,0.931485377316195)(227,0.931485377316195)(227,0.931485377316195)(227,0.931485377316195)(227,0.931485377316195)(228,0.93135741700128)(228,0.93135741700128)(228,0.93135741700128)(228,0.93135741700128)(228,0.93135741700128)(228,0.93135741700128)(228,0.93135741700128)(228,0.93135741700128)(229,0.931229067391955)(229,0.931229067391955)(229,0.931229067391955)(229,0.931229067391955)(229,0.931229067391955)(229,0.931229067391955)(229,0.931229067391955)(229,0.931229067391955)(229,0.931229067391955)(229,0.931229067391955)(230,0.931123285836686)(230,0.931123285836686)(230,0.931123285836686)(230,0.931123285836686)(230,0.931123285836686)(230,0.931123285836686)(230,0.931123285836686)(230,0.931123285836686)(231,0.931040292263274)(231,0.931040292263274)(231,0.931040292263274)(231,0.931040292263274)(231,0.931040292263274)(231,0.931040292263274)(232,0.930990176843429)(232,0.930990176843429)(232,0.930990176843429)(232,0.930990176843429)(233,0.930969949307365)(233,0.930969949307365)(233,0.930969949307365)(233,0.930969949307365)(233,0.930969949307365)(233,0.930969949307365)(233,0.930969949307365)(234,0.93095587475416)(234,0.93095587475416)(234,0.93095587475416)(234,0.93095587475416)(234,0.93095587475416)(235,0.930932180950164)(235,0.930932180950164)(235,0.930932180950164)(235,0.930932180950164)(235,0.930932180950164)(235,0.930932180950164)(235,0.930932180950164)(236,0.930864021882447)(236,0.930864021882447)(236,0.930864021882447)(236,0.930864021882447)(236,0.930864021882447)(236,0.930864021882447)(237,0.930785214149691)(237,0.930785214149691)(237,0.930785214149691)(237,0.930785214149691)(237,0.930785214149691)(237,0.930785214149691)(237,0.930785214149691)(237,0.930785214149691)(237,0.930785214149691)(238,0.930545805270038)(238,0.930545805270038)(238,0.930545805270038)(238,0.930545805270038)(238,0.930545805270038)(238,0.930545805270038)(238,0.930545805270038)(238,0.930545805270038)(239,0.930439951570301)(239,0.930439951570301)(239,0.930439951570301)(239,0.930439951570301)(239,0.930439951570301)(240,0.930370388221364)(240,0.930370388221364)(240,0.930370388221364)(240,0.930370388221364)(240,0.930370388221364)(240,0.930370388221364)(240,0.930370388221364)(240,0.930370388221364)(240,0.930370388221364)(241,0.930344219139873)(241,0.930344219139873)(241,0.930344219139873)(241,0.930344219139873)(241,0.930344219139873)(241,0.930344219139873)(241,0.930344219139873)(241,0.930344219139873)(241,0.930344219139873)(241,0.930344219139873)(241,0.930344219139873)(241,0.930344219139873)(242,0.930359789430444)(242,0.930359789430444)(242,0.930359789430444)(242,0.930359789430444)(242,0.930359789430444)(242,0.930359789430444)(242,0.930359789430444)(242,0.930359789430444)(243,0.930414236149959)(243,0.930414236149959)(243,0.930414236149959)(243,0.930414236149959)(243,0.930414236149959)(243,0.930414236149959)(243,0.930414236149959)(243,0.930414236149959)(243,0.930414236149959)(243,0.930414236149959)(244,0.930556630074303)(244,0.930556630074303)(244,0.930556630074303)(244,0.930556630074303)(244,0.930556630074303)(244,0.930556630074303)(244,0.930556630074303)(244,0.930556630074303)(245,0.930647427813468)(245,0.930647427813468)(245,0.930647427813468)(245,0.930647427813468)(246,0.930785229644968)(246,0.930785229644968)(247,0.930972597381396)(247,0.930972597381396)(247,0.930972597381396)(247,0.930972597381396)(247,0.930972597381396)(247,0.930972597381396)(247,0.930972597381396)(247,0.930972597381396)(247,0.930972597381396)(247,0.930972597381396)(247,0.930972597381396)(247,0.930972597381396)(247,0.930972597381396)(247,0.930972597381396)(247,0.930972597381396)(248,0.931226570515829)(248,0.931226570515829)(248,0.931226570515829)(248,0.931226570515829)(248,0.931226570515829)(248,0.931226570515829)(248,0.931226570515829)(248,0.931226570515829)(248,0.931226570515829)(248,0.931226570515829)(248,0.931226570515829)(249,0.931564331868745)(249,0.931564331868745)(249,0.931564331868745)(249,0.931564331868745)(249,0.931564331868745)(249,0.931564331868745)(249,0.931564331868745)(249,0.931564331868745)(249,0.931564331868745)(249,0.931564331868745)(249,0.931564331868745)(250,0.931971380568781)(250,0.931971380568781)(250,0.931971380568781)(250,0.931971380568781)(250,0.931971380568781)(250,0.931971380568781)(250,0.931971380568781)(250,0.931971380568781)(251,0.932419799014061)(251,0.932419799014061)(251,0.932419799014061)(251,0.932419799014061)(251,0.932419799014061)(251,0.932419799014061)(251,0.932419799014061)(252,0.932887537332004)(252,0.932887537332004)(252,0.932887537332004)(252,0.932887537332004)(252,0.932887537332004)(252,0.932887537332004)(252,0.932887537332004)(252,0.932887537332004)(252,0.932887537332004)(253,0.933346342165302)(253,0.933346342165302)(253,0.933346342165302)(253,0.933346342165302)(254,0.933785813449489)(254,0.933785813449489)(254,0.933785813449489)(254,0.933785813449489)(254,0.933785813449489)(254,0.933785813449489)(254,0.933785813449489)(254,0.933785813449489)(254,0.933785813449489)(255,0.934212384311136)(255,0.934212384311136)(255,0.934212384311136)(255,0.934212384311136)(255,0.934212384311136)(255,0.934212384311136)(256,0.934631342762802)(256,0.934631342762802)(256,0.934631342762802)(256,0.934631342762802)(256,0.934631342762802)(256,0.934631342762802)(256,0.934631342762802)(256,0.934631342762802)(256,0.934631342762802)(257,0.935052139100331)(257,0.935052139100331)(257,0.935052139100331)(257,0.935052139100331)(257,0.935052139100331)(257,0.935052139100331)(257,0.935052139100331)(257,0.935052139100331)(257,0.935052139100331)(257,0.935052139100331)(257,0.935052139100331)(257,0.935052139100331)(257,0.935052139100331)(257,0.935052139100331)(258,0.935508809929941)(258,0.935508809929941)(258,0.935508809929941)(259,0.936016701477708)(259,0.936016701477708)(259,0.936016701477708)(259,0.936016701477708)(259,0.936016701477708)(259,0.936016701477708)(259,0.936016701477708)(260,0.93633319614183)(260,0.93633319614183)(260,0.93633319614183)(260,0.93633319614183)(261,0.937102954477552)(261,0.937102954477552)(261,0.937102954477552)(261,0.937102954477552)(262,0.937606462772912)(262,0.937606462772912)(262,0.937606462772912)(262,0.937606462772912)(262,0.937606462772912)(262,0.937606462772912)(262,0.937606462772912)(263,0.937912104505906)(263,0.937912104505906)(263,0.937912104505906)(263,0.937912104505906)(263,0.937912104505906)(263,0.937912104505906)(263,0.937912104505906)(263,0.937912104505906)(263,0.937912104505906)(263,0.937912104505906)(263,0.937912104505906)(263,0.937912104505906)(264,0.938342346329133)(264,0.938342346329133)(264,0.938342346329133)(265,0.938622825548547)(265,0.938622825548547)(265,0.938622825548547)(265,0.938622825548547)(265,0.938622825548547)(265,0.938622825548547)(265,0.938622825548547)(266,0.93891634941725)(266,0.93891634941725)(266,0.93891634941725)(266,0.93891634941725)(266,0.93891634941725)(266,0.93891634941725)(266,0.93891634941725)(266,0.93891634941725)(266,0.93891634941725)(267,0.939141462624407)(267,0.939141462624407)(267,0.939141462624407)(267,0.939141462624407)(267,0.939141462624407)(268,0.939370326124655)(268,0.939370326124655)(268,0.939370326124655)(268,0.939370326124655)(268,0.939370326124655)(268,0.939370326124655)(269,0.939452165348286)(269,0.939452165348286)(269,0.939452165348286)(269,0.939452165348286)(269,0.939452165348286)(269,0.939452165348286)(269,0.939452165348286)(269,0.939452165348286)(269,0.939452165348286)(269,0.939452165348286)(269,0.939452165348286)(270,0.939579828203442)(270,0.939579828203442)(270,0.939579828203442)(270,0.939579828203442)(270,0.939579828203442)(270,0.939579828203442)(270,0.939579828203442)(270,0.939579828203442)(271,0.939735728780331)(271,0.939735728780331)(271,0.939735728780331)(271,0.939735728780331)(271,0.939735728780331)(271,0.939735728780331)(271,0.939735728780331)(271,0.939735728780331)(271,0.939735728780331)(271,0.939735728780331)(271,0.939735728780331)(271,0.939735728780331)(272,0.940097573071384)(272,0.940097573071384)(272,0.940097573071384)(272,0.940097573071384)(272,0.940097573071384)(272,0.940097573071384)(272,0.940097573071384)(272,0.940097573071384)(273,0.940360968649988)(273,0.940360968649988)(273,0.940360968649988)(273,0.940360968649988)(273,0.940360968649988)(273,0.940360968649988)(273,0.940360968649988)(274,0.940679299029566)(274,0.940679299029566)(274,0.940679299029566)(274,0.940679299029566)(274,0.940679299029566)(274,0.940679299029566)(274,0.940679299029566)(275,0.941044841474603)(275,0.941044841474603)(275,0.941044841474603)(275,0.941044841474603)(275,0.941044841474603)(275,0.941044841474603)(275,0.941044841474603)(275,0.941044841474603)(276,0.941449229898842)(276,0.941449229898842)(276,0.941449229898842)(276,0.941449229898842)(276,0.941449229898842)(276,0.941449229898842)(276,0.941449229898842)(276,0.941449229898842)(276,0.941449229898842)(277,0.941873046097793)(277,0.941873046097793)(277,0.941873046097793)(277,0.941873046097793)(277,0.941873046097793)(277,0.941873046097793)(277,0.941873046097793)(278,0.942285337021387)(278,0.942285337021387)(279,0.942664781164883)(279,0.942664781164883)(279,0.942664781164883)(279,0.942664781164883)(279,0.942664781164883)(279,0.942664781164883)(280,0.943000623995104)(280,0.943000623995104)(280,0.943000623995104)(280,0.943000623995104)(280,0.943000623995104)(280,0.943000623995104)(281,0.943291953907347)(281,0.943291953907347)(281,0.943291953907347)(281,0.943291953907347)(281,0.943291953907347)(281,0.943291953907347)(282,0.943552364856462)(282,0.943552364856462)(282,0.943552364856462)(282,0.943552364856462)(282,0.943552364856462)(283,0.943784962514867)(283,0.943784962514867)(283,0.943784962514867)(283,0.943784962514867)(283,0.943784962514867)(283,0.943784962514867)(283,0.943784962514867)(283,0.943784962514867)(283,0.943784962514867)(283,0.943784962514867)(284,0.943984390365138)(284,0.943984390365138)(284,0.943984390365138)(284,0.943984390365138)(284,0.943984390365138)(284,0.943984390365138)(285,0.944144980860743)(285,0.944144980860743)(285,0.944144980860743)(285,0.944144980860743)(285,0.944144980860743)(286,0.944286228332822)(286,0.944286228332822)(286,0.944286228332822)(287,0.944394054141991)(287,0.944394054141991)(287,0.944394054141991)(287,0.944394054141991)(287,0.944394054141991)(287,0.944394054141991)(287,0.944394054141991)(288,0.94452553944118)(288,0.94452553944118)(288,0.94452553944118)(288,0.94452553944118)(289,0.944692690010736)(289,0.944692690010736)(289,0.944692690010736)(290,0.944917019878289)(290,0.944917019878289)(290,0.944917019878289)(290,0.944917019878289)(291,0.945207576196414)(291,0.945207576196414)(291,0.945207576196414)(291,0.945207576196414)(291,0.945207576196414)(291,0.945207576196414)(291,0.945207576196414)(291,0.945207576196414)(292,0.94557134831913)(292,0.94557134831913)(292,0.94557134831913)(293,0.94589533748509)(293,0.94589533748509)(293,0.94589533748509)(293,0.94589533748509)(293,0.94589533748509)(293,0.94589533748509)(293,0.94589533748509)(293,0.94589533748509)(293,0.94589533748509)(293,0.94589533748509)(293,0.94589533748509)(294,0.946519065796333)(294,0.946519065796333)(294,0.946519065796333)(295,0.94706717789734)(295,0.94706717789734)(295,0.94706717789734)(295,0.94706717789734)(295,0.94706717789734)(296,0.947644294913962)(296,0.947644294913962)(296,0.947644294913962)(296,0.947644294913962)(296,0.947644294913962)(296,0.947644294913962)(296,0.947644294913962)(296,0.947644294913962)(297,0.948322053791683)(297,0.948322053791683)(297,0.948322053791683)(298,0.948937412765218)(298,0.948937412765218)(298,0.948937412765218)(298,0.948937412765218)(298,0.948937412765218)(298,0.948937412765218)(298,0.948937412765218)(298,0.948937412765218)(299,0.949572832647687)(299,0.949572832647687)(299,0.949572832647687)(299,0.949572832647687)(299,0.949572832647687)(299,0.949572832647687)(299,0.949572832647687)(300,0.950228249802406)(300,0.950228249802406)(300,0.950228249802406)(300,0.950228249802406)(300,0.950228249802406)(300,0.950228249802406)(300,0.950228249802406)(300,0.950228249802406)(300,0.950228249802406)(300,0.950228249802406)(300,0.950228249802406)(301,0.95090035956721)(301,0.95090035956721)(301,0.95090035956721)(301,0.95090035956721)(301,0.95090035956721)(301,0.95090035956721)(301,0.95090035956721)(301,0.95090035956721)(302,0.951583610289534)(302,0.951583610289534)(302,0.951583610289534)(302,0.951583610289534)(302,0.951583610289534)(302,0.951583610289534)(302,0.951583610289534)(302,0.951583610289534)(302,0.951583610289534)(302,0.951583610289534)(303,0.952278762226028)(303,0.952278762226028)(303,0.952278762226028)(303,0.952278762226028)(303,0.952278762226028)(303,0.952278762226028)(303,0.952278762226028)(303,0.952278762226028)(303,0.952278762226028)(304,0.952978886687073)(304,0.952978886687073)(304,0.952978886687073)(304,0.952978886687073)(304,0.952978886687073)(305,0.953669302637451)(305,0.953669302637451)(305,0.953669302637451)(305,0.953669302637451)(305,0.953669302637451)(305,0.953669302637451)(305,0.953669302637451)(305,0.953669302637451)(305,0.953669302637451)(305,0.953669302637451)(305,0.953669302637451)(306,0.954336361550128)(306,0.954336361550128)(306,0.954336361550128)(306,0.954336361550128)(306,0.954336361550128)(306,0.954336361550128)(306,0.954336361550128)(307,0.954961117376979)(307,0.954961117376979)(307,0.954961117376979)(307,0.954961117376979)(307,0.954961117376979)(307,0.954961117376979)(307,0.954961117376979)(307,0.954961117376979)(307,0.954961117376979)(307,0.954961117376979)(308,0.955524529771646)(308,0.955524529771646)(308,0.955524529771646)(308,0.955524529771646)(308,0.955524529771646)(308,0.955524529771646)(308,0.955524529771646)(309,0.956009312602022)(309,0.956009312602022)(309,0.956009312602022)(309,0.956009312602022)(309,0.956009312602022)(310,0.956406909840839)(310,0.956406909840839)(310,0.956406909840839)(310,0.956406909840839)(310,0.956406909840839)(310,0.956406909840839)(310,0.956406909840839)(310,0.956406909840839)(310,0.956406909840839)(311,0.956712479644112)(311,0.956712479644112)(311,0.956712479644112)(312,0.956933905690378)(312,0.956933905690378)(312,0.956933905690378)(312,0.956933905690378)(312,0.956933905690378)(312,0.956933905690378)(312,0.956933905690378)(312,0.956933905690378)(313,0.957087205412894)(313,0.957087205412894)(313,0.957087205412894)(313,0.957087205412894)(313,0.957087205412894)(314,0.957153415696524)(314,0.957153415696524)(314,0.957153415696524)(314,0.957153415696524)(314,0.957153415696524)(315,0.957257115405326)(315,0.957257115405326)(315,0.957257115405326)(315,0.957257115405326)(315,0.957257115405326)(315,0.957257115405326)(315,0.957257115405326)(315,0.957257115405326)(315,0.957257115405326)(315,0.957257115405326)(315,0.957257115405326)(316,0.957299632416401)(316,0.957299632416401)(316,0.957299632416401)(316,0.957299632416401)(316,0.957299632416401)(316,0.957299632416401)(317,0.957332779610485)(317,0.957332779610485)(317,0.957332779610485)(317,0.957332779610485)(317,0.957332779610485)(317,0.957332779610485)(317,0.957332779610485)(317,0.957332779610485)(317,0.957332779610485)(317,0.957332779610485)(318,0.957369601128973)(318,0.957369601128973)(318,0.957369601128973)(318,0.957369601128973)(319,0.957419501599356)(319,0.957419501599356)(319,0.957419501599356)(319,0.957419501599356)(319,0.957419501599356)(319,0.957419501599356)(319,0.957419501599356)(320,0.957554578418521)(320,0.957554578418521)(320,0.957554578418521)(321,0.95757200454874)(321,0.95757200454874)(321,0.95757200454874)(321,0.95757200454874)(321,0.95757200454874)(321,0.95757200454874)(321,0.95757200454874)(322,0.957744224569694)(322,0.957744224569694)(322,0.957744224569694)(322,0.957744224569694)(322,0.957744224569694)(322,0.957744224569694)(322,0.957744224569694)(323,0.957865192539705)(323,0.957865192539705)(323,0.957865192539705)(323,0.957865192539705)(324,0.957997338079995)(324,0.957997338079995)(324,0.957997338079995)(324,0.957997338079995)(324,0.957997338079995)(325,0.958144410462858)(325,0.958144410462858)(325,0.958144410462858)(325,0.958144410462858)(326,0.958309372346608)(326,0.958309372346608)(326,0.958309372346608)(326,0.958309372346608)(326,0.958309372346608)(326,0.958309372346608)(326,0.958309372346608)(326,0.958309372346608)(327,0.958565900893205)(327,0.958565900893205)(327,0.958565900893205)(327,0.958565900893205)(327,0.958565900893205)(327,0.958565900893205)(327,0.958565900893205)(327,0.958565900893205)(328,0.958757008128417)(328,0.958757008128417)(328,0.958757008128417)(328,0.958757008128417)(328,0.958757008128417)(328,0.958757008128417)(328,0.958757008128417)(328,0.958757008128417)(328,0.958757008128417)(328,0.958757008128417)(329,0.958965442289426)(329,0.958965442289426)(329,0.958965442289426)(329,0.958965442289426)(330,0.959192098622117)(330,0.959192098622117)(330,0.959192098622117)(330,0.959192098622117)(330,0.959192098622117)(330,0.959192098622117)(330,0.959192098622117)(330,0.959192098622117)(330,0.959192098622117)(330,0.959192098622117)(330,0.959192098622117)(331,0.959461168890595)(331,0.959461168890595)(331,0.959461168890595)(331,0.959461168890595)(331,0.959461168890595)(331,0.959461168890595)(332,0.959712129129329)(332,0.959712129129329)(332,0.959712129129329)(332,0.959712129129329)(333,0.959975800674616)(333,0.959975800674616)(333,0.959975800674616)(333,0.959975800674616)(333,0.959975800674616)(333,0.959975800674616)(334,0.960239914606844)(334,0.960239914606844)(334,0.960239914606844)(334,0.960239914606844)(335,0.960511767772069)(335,0.960511767772069)(335,0.960511767772069)(335,0.960511767772069)(335,0.960511767772069)(335,0.960511767772069)(335,0.960511767772069)(335,0.960511767772069)(336,0.960787048933944)(336,0.960787048933944)(336,0.960787048933944)(336,0.960787048933944)(337,0.961016494061257)(337,0.961016494061257)(337,0.961016494061257)(337,0.961016494061257)(337,0.961016494061257)(338,0.961277467829354)(338,0.961277467829354)(338,0.961277467829354)(338,0.961277467829354)(339,0.961460023501389)(339,0.961460023501389)(339,0.961460023501389)(339,0.961460023501389)(339,0.961460023501389)(340,0.961682983258959)(340,0.961682983258959)(340,0.961682983258959)(341,0.9618102486193)(341,0.9618102486193)(341,0.9618102486193)(341,0.9618102486193)(341,0.9618102486193)(341,0.9618102486193)(341,0.9618102486193)(341,0.9618102486193)(342,0.961996971644293)(342,0.961996971644293)(342,0.961996971644293)(343,0.962094787604586)(343,0.962094787604586)(343,0.962094787604586)(343,0.962094787604586)(344,0.962252116891891)(344,0.962252116891891)(344,0.962252116891891)(344,0.962252116891891)(344,0.962252116891891)(344,0.962252116891891)(344,0.962252116891891)(344,0.962252116891891)(345,0.962333738978964)(345,0.962333738978964)(345,0.962333738978964)(346,0.962415743467908)(346,0.962415743467908)(346,0.962415743467908)(346,0.962415743467908)(347,0.962533489062679)(347,0.962533489062679)(347,0.962533489062679)(347,0.962533489062679)(347,0.962533489062679)(347,0.962533489062679)(347,0.962533489062679)(347,0.962533489062679)(347,0.962533489062679)(347,0.962533489062679)(348,0.962628978097089)(349,0.962678182208823)(349,0.962678182208823)(349,0.962678182208823)(349,0.962678182208823)(349,0.962678182208823)(350,0.962713468270638)(351,0.962756014630323)(351,0.962756014630323)(351,0.962756014630323)(351,0.962756014630323)(351,0.962756014630323)(352,0.962771307621729)(352,0.962771307621729)(352,0.962771307621729)(352,0.962771307621729)(352,0.962771307621729)(353,0.962781379915339)(353,0.962781379915339)(353,0.962781379915339)(353,0.962781379915339)(353,0.962781379915339)(353,0.962781379915339)(354,0.962791240004302)(354,0.962791240004302)(354,0.962791240004302)(354,0.962791240004302)(354,0.962791240004302)(355,0.962795724513849)(355,0.962795724513849)(355,0.962795724513849)(355,0.962795724513849)(355,0.962795724513849)(356,0.962803454805453)(356,0.962803454805453)(356,0.962803454805453)(356,0.962803454805453)(357,0.962814412623668)(357,0.962814412623668)(357,0.962814412623668)(357,0.962814412623668)(357,0.962814412623668)(357,0.962814412623668)(357,0.962814412623668)(357,0.962814412623668)(358,0.96283229362331)(358,0.96283229362331)(358,0.96283229362331)(359,0.962859207368946)(359,0.962859207368946)(359,0.962859207368946)(359,0.962859207368946)(360,0.96289508040129)(360,0.96289508040129)(360,0.96289508040129)(360,0.96289508040129)(360,0.96289508040129)(360,0.96289508040129)(360,0.96289508040129)(361,0.962938760176549)(361,0.962938760176549)(361,0.962938760176549)(361,0.962938760176549)(362,0.962988834705437)(362,0.962988834705437)(363,0.963043960579464)(363,0.963043960579464)(363,0.963043960579464)(363,0.963043960579464)(363,0.963043960579464)(364,0.96310301669742)(364,0.96310301669742)(364,0.96310301669742)(364,0.96310301669742)(365,0.963165115333559)(365,0.963165115333559)(365,0.963165115333559)(365,0.963165115333559)(366,0.963229596987339)(366,0.963229596987339)(366,0.963229596987339)(366,0.963229596987339)(366,0.963229596987339)(367,0.963295992720769)(367,0.963295992720769)(368,0.96336398418692)(368,0.96336398418692)(368,0.96336398418692)(368,0.96336398418692)(368,0.96336398418692)(369,0.963433325115759)(369,0.963433325115759)(370,0.963503798971485)(370,0.963503798971485)(370,0.963503798971485)(370,0.963503798971485)(372,0.963647623799517)(372,0.963647623799517)(372,0.963647623799517)(372,0.963647623799517)(373,0.963720855700045)(374,0.963794965505029)(374,0.963794965505029)(374,0.963794965505029)(374,0.963794965505029)(375,0.963869999387688)(375,0.963869999387688)(376,0.963946037131751)(376,0.963946037131751)(377,0.964023179324354)(377,0.964023179324354)(378,0.964101532858321)(379,0.964181216008743)(379,0.964181216008743)(379,0.964181216008743)(380,0.964262373287412)(380,0.964262373287412)(381,0.964345173191453)(381,0.964345173191453)(382,0.964429775192126)(382,0.964429775192126)(382,0.964429775192126)(384,0.96460503106239)(384,0.96460503106239)(384,0.96460503106239)(384,0.96460503106239)(386,0.964789293406043)(386,0.964789293406043)(386,0.964789293406043)(388,0.96498346986617)(388,0.96498346986617)(388,0.96498346986617)(389,0.965084580603826)(390,0.965188508045022)(394,0.965633587009639)(394,0.965633587009639)(395,0.965752382541782)(395,0.965752382541782)(399,0.966256640961473) 
};
\addlegendentry{\acl};

\addplot [
color=orange,
densely dotted,
line width=1.0pt,
]
coordinates{
 (150,0.532036902138964)(180,0.602608566774702)(210,0.633963047798765)(240,0.660807778419082)(270,0.685588632108525)(300,0.70585197702025)(330,0.719465170478611)(360,0.732502909286226)(390,0.752246081215268) 
};
\addlegendentry{\rstr};

\addplot [
color=green!50!black,
densely dotted,
line width=1.0pt,
]
coordinates{
 (30,0.39893135601704)(60,0.79973692648923)(90,0.872314920693221)(120,0.898898134654891)(150,0.914745869254766)(180,0.923560926215837)(210,0.929092453124412)(240,0.933016127800257)(270,0.934140366160171)(300,0.938436063179169)(330,0.93996532136056)(360,0.940748532930402)(390,0.944894583030356) 
};
\addlegendentry{\bstr};

\addplot [
color=blue,
solid,
line width=1.3pt,
]
coordinates{
% (30,0.4789819770986)(60,0.805361694081093)(90,0.865419156380707)(120,0.896720598930792)(150,0.92202519257533)(180,0.934695526879322)(210,0.949455475528735)(240,0.944876813973243)(270,0.961483811692617)(300,0.961011710636837)(330,0.959987630648086)(360,0.9652294181368)(390,0.966393052820146) 
  (30,0.450348138232051)(60,0.810595756687757)(90,0.867002361073163)(120,0.905620188229337)(150,0.924485510663702)(180,0.936665255675441)(210,0.94754365168575)(240,0.95523197520678)(270,0.955738989742935)(300,0.967293168689706)(330,0.964703094899903)(360,0.962867083879745)(390,0.964710684709887) 
};
\addlegendentry{\bacl};

\end{axis}
\end{tikzpicture}%

%% This file was created by matlab2tikz v0.2.3.
% Copyright (c) 2008--2012, Nico Schlömer <nico.schloemer@gmail.com>
% All rights reserved.
% 
% 
%

\definecolor{locol}{rgb}{0.26, 0.45, 0.65}

\begin{tikzpicture}

\begin{axis}[%
tick label style={font=\tiny},
label style={font=\tiny},
xlabel shift={-10pt},
ylabel shift={-17pt},
legend style={font=\tiny},
view={0}{90},
width=\figurewidth,
height=\figureheight,
scale only axis,
xmin=0, xmax=1478,
xtick={0, 400, 1000, 1400},
xlabel={Length (m)},
ymin=-18, ymax=0,
ytick={0, -4, -14, -18},
ylabel={Depth (m)},
name=plot1,
axis lines*=box,
tickwidth=0.1cm,
clip=false
]

\addplot [fill=locol,draw=none,forget plot] coordinates{ (1478,0)(1478,-0.181818181818182)(1478,-0.363636363636364)(1478,-0.545454545454545)(1478,-0.727272727272727)(1478,-0.909090909090909)(1478,-1.09090909090909)(1478,-1.27272727272727)(1478,-1.45454545454545)(1478,-1.63636363636364)(1478,-1.81818181818182)(1478,-2)(1478,-2.18181818181818)(1478,-2.36363636363636)(1478,-2.54545454545455)(1478,-2.72727272727273)(1478,-2.90909090909091)(1478,-3.09090909090909)(1478,-3.27272727272727)(1478,-3.45454545454545)(1478,-3.63636363636364)(1478,-3.81818181818182)(1478,-4)(1478,-4.18181818181818)(1478,-4.36363636363636)(1478,-4.54545454545455)(1478,-4.72727272727273)(1478,-4.90909090909091)(1478,-5.09090909090909)(1478,-5.27272727272727)(1478,-5.45454545454545)(1478,-5.63636363636364)(1478,-5.81818181818182)(1478,-6)(1478,-6.18181818181818)(1478,-6.36363636363636)(1478,-6.54545454545455)(1478,-6.72727272727273)(1478,-6.90909090909091)(1478,-7.09090909090909)(1478,-7.27272727272727)(1478,-7.45454545454545)(1478,-7.63636363636364)(1478,-7.81818181818182)(1478,-8)(1478,-8.18181818181818)(1478,-8.36363636363636)(1478,-8.54545454545455)(1478,-8.72727272727273)(1478,-8.90909090909091)(1478,-9.09090909090909)(1478,-9.27272727272727)(1478,-9.45454545454546)(1478,-9.63636363636364)(1478,-9.81818181818182)(1478,-10)(1478,-10.1818181818182)(1478,-10.3636363636364)(1478,-10.5454545454545)(1478,-10.7272727272727)(1478,-10.9090909090909)(1478,-11.0909090909091)(1478,-11.2727272727273)(1478,-11.4545454545455)(1478,-11.6363636363636)(1478,-11.8181818181818)(1478,-12)(1478,-12.1818181818182)(1478,-12.3636363636364)(1478,-12.5454545454545)(1478,-12.7272727272727)(1478,-12.9090909090909)(1478,-13.0909090909091)(1478,-13.2727272727273)(1478,-13.4545454545455)(1478,-13.6363636363636)(1478,-13.8181818181818)(1478,-14)(1478,-14.1818181818182)(1478,-14.3636363636364)(1478,-14.5454545454545)(1478,-14.7272727272727)(1478,-14.9090909090909)(1478,-15.0909090909091)(1478,-15.2727272727273)(1478,-15.4545454545455)(1478,-15.6363636363636)(1478,-15.8181818181818)(1478,-16)(1478,-16.1818181818182)(1478,-16.3636363636364)(1478,-16.5454545454545)(1478,-16.7272727272727)(1478,-16.9090909090909)(1478,-17.0909090909091)(1478,-17.2727272727273)(1478,-17.4545454545455)(1478,-17.6363636363636)(1478,-17.8181818181818)(1478,-18)(1463.07070707071,-18)(1448.14141414141,-18)(1433.21212121212,-18)(1418.28282828283,-18)(1403.35353535354,-18)(1388.42424242424,-18)(1373.49494949495,-18)(1358.56565656566,-18)(1343.63636363636,-18)(1328.70707070707,-18)(1313.77777777778,-18)(1298.84848484848,-18)(1283.91919191919,-18)(1268.9898989899,-18)(1254.06060606061,-18)(1239.13131313131,-18)(1224.20202020202,-18)(1209.27272727273,-18)(1194.34343434343,-18)(1179.41414141414,-18)(1164.48484848485,-18)(1149.55555555556,-18)(1134.62626262626,-18)(1119.69696969697,-18)(1104.76767676768,-18)(1089.83838383838,-18)(1074.90909090909,-18)(1059.9797979798,-18)(1045.05050505051,-18)(1030.12121212121,-18)(1015.19191919192,-18)(1000.26262626263,-18)(985.333333333333,-18)(970.40404040404,-18)(955.474747474747,-18)(940.545454545455,-18)(925.616161616162,-18)(910.686868686869,-18)(895.757575757576,-18)(880.828282828283,-18)(865.89898989899,-18)(850.969696969697,-18)(836.040404040404,-18)(821.111111111111,-18)(806.181818181818,-18)(791.252525252525,-18)(776.323232323232,-18)(761.393939393939,-18)(746.464646464646,-18)(731.535353535354,-18)(716.606060606061,-18)(701.676767676768,-18)(686.747474747475,-18)(671.818181818182,-18)(656.888888888889,-18)(641.959595959596,-18)(627.030303030303,-18)(612.10101010101,-18)(597.171717171717,-18)(582.242424242424,-18)(567.313131313131,-18)(552.383838383838,-18)(537.454545454546,-18)(522.525252525253,-18)(507.59595959596,-18)(492.666666666667,-18)(477.737373737374,-18)(462.808080808081,-18)(447.878787878788,-18)(432.949494949495,-18)(418.020202020202,-18)(403.090909090909,-18)(388.161616161616,-18)(373.232323232323,-18)(358.30303030303,-18)(343.373737373737,-18)(328.444444444444,-18)(313.515151515152,-18)(298.585858585859,-18)(283.656565656566,-18)(268.727272727273,-18)(253.79797979798,-18)(238.868686868687,-18)(223.939393939394,-18)(209.010101010101,-18)(194.080808080808,-18)(179.151515151515,-18)(164.222222222222,-18)(149.292929292929,-18)(134.363636363636,-18)(119.434343434343,-18)(104.505050505051,-18)(89.5757575757576,-18)(74.6464646464647,-18)(59.7171717171717,-18)(44.7878787878788,-18)(29.8585858585859,-18)(14.9292929292929,-18)(0,-18)(0,-17.8181818181818)(0,-17.6363636363636)(0,-17.4545454545455)(0,-17.2727272727273)(0,-17.0909090909091)(0,-16.9090909090909)(0,-16.7272727272727)(0,-16.5454545454545)(0,-16.3636363636364)(0,-16.1818181818182)(0,-16)(0,-15.8181818181818)(0,-15.6363636363636)(0,-15.4545454545455)(0,-15.2727272727273)(0,-15.0909090909091)(0,-14.9090909090909)(0,-14.7272727272727)(0,-14.5454545454545)(0,-14.3636363636364)(0,-14.1818181818182)(0,-14)(0,-13.8181818181818)(0,-13.6363636363636)(0,-13.4545454545455)(0,-13.2727272727273)(0,-13.0909090909091)(0,-12.9090909090909)(0,-12.7272727272727)(0,-12.5454545454545)(0,-12.3636363636364)(0,-12.1818181818182)(0,-12)(0,-11.8181818181818)(0,-11.6363636363636)(0,-11.4545454545455)(0,-11.2727272727273)(0,-11.0909090909091)(0,-10.9090909090909)(0,-10.7272727272727)(0,-10.5454545454545)(0,-10.3636363636364)(0,-10.1818181818182)(0,-10)(0,-9.81818181818182)(0,-9.63636363636364)(0,-9.45454545454546)(-0.000497626510092015,-9.27272727272727)(-0.000995219847296375,-9.09090909090909)(-0.00149278001492805,-8.90909090909091)(-0.00149278001492805,-8.72727272727273)(-0.00149278001492805,-8.54545454545455)(-0.00149278001492805,-8.36363636363636)(-0.00149278001492805,-8.18181818181818)(-0.00149278001492805,-8)(-0.00149278001492805,-7.81818181818182)(-0.00149278001492805,-7.63636363636364)(-0.00149278001492805,-7.45454545454545)(-0.00149278001492805,-7.27272727272727)(-0.00149278001492805,-7.09090909090909)(-0.00149278001492805,-6.90909090909091)(-0.00149278001492805,-6.72727272727273)(-0.000995219847296375,-6.54545454545455)(-0.000497626510092015,-6.36363636363636)(0,-6.18181818181818)(0,-6)(0,-5.81818181818182)(0,-5.63636363636364)(0,-5.45454545454545)(0,-5.27272727272727)(0,-5.09090909090909)(0,-4.90909090909091)(0,-4.72727272727273)(0,-4.54545454545455)(0,-4.36363636363636)(0,-4.18181818181818)(0,-4)(0,-3.81818181818182)(0,-3.63636363636364)(0,-3.45454545454545)(0,-3.27272727272727)(0,-3.09090909090909)(0,-2.90909090909091)(0,-2.72727272727273)(0,-2.54545454545455)(0,-2.36363636363636)(0,-2.18181818181818)(0,-2)(0,-1.81818181818182)(0,-1.63636363636364)(0,-1.45454545454545)(0,-1.27272727272727)(0,-1.09090909090909)(0,-0.909090909090909)(0,-0.727272727272727)(0,-0.545454545454545)(0,-0.363636363636364)(0,-0.181818181818182)(0,0)(14.9292929292929,0)(29.8585858585859,0)(44.7878787878788,0)(59.7171717171717,0)(74.6464646464647,0)(89.5757575757576,0)(104.505050505051,0)(119.434343434343,0)(134.363636363636,0)(149.292929292929,0)(164.222222222222,0)(179.151515151515,0)(194.080808080808,0)(209.010101010101,0)(223.939393939394,0)(238.868686868687,0)(253.79797979798,0)(268.727272727273,0)(283.656565656566,0)(298.585858585859,0)(313.515151515152,0)(328.444444444444,0)(343.373737373737,0)(358.30303030303,0)(373.232323232323,0)(388.161616161616,0)(403.090909090909,0)(418.020202020202,0)(432.949494949495,0)(447.878787878788,0)(462.808080808081,0)(477.737373737374,0)(492.666666666667,0)(507.59595959596,0)(522.525252525253,0)(537.454545454546,0)(552.383838383838,0)(567.313131313131,0)(582.242424242424,0)(597.171717171717,0)(612.10101010101,0)(627.030303030303,0)(641.959595959596,0)(656.888888888889,0)(671.818181818182,0)(686.747474747475,0)(701.676767676768,0)(716.606060606061,0)(731.535353535354,0)(746.464646464646,0)(761.393939393939,0)(776.323232323232,0)(791.252525252525,0)(806.181818181818,0)(821.111111111111,0)(836.040404040404,0)(850.969696969697,0)(865.89898989899,0)(880.828282828283,0)(895.757575757576,0)(910.686868686869,0)(925.616161616162,0)(940.545454545455,0)(955.474747474747,0)(970.40404040404,0)(985.333333333333,0)(1000.26262626263,0)(1015.19191919192,0)(1030.12121212121,0)(1045.05050505051,0)(1059.9797979798,0)(1074.90909090909,0)(1089.83838383838,0)(1104.76767676768,0)(1119.69696969697,0)(1134.62626262626,0)(1149.55555555556,0)(1164.48484848485,0)(1179.41414141414,0)(1194.34343434343,0)(1209.27272727273,0)(1224.20202020202,0)(1239.13131313131,0)(1254.06060606061,0)(1268.9898989899,0)(1283.91919191919,0)(1298.84848484848,0)(1313.77777777778,0)(1328.70707070707,0)(1343.63636363636,0)(1358.56565656566,0)(1373.49494949495,0)(1388.42424242424,0)(1403.35353535354,0)(1418.28282828283,0)(1433.21212121212,0)(1448.14141414141,0)(1463.07070707071,0)(1478,0)};

\addplot [fill=red!40!yellow,draw=none,forget plot] coordinates{ (798.717171717172,-3.81818181818182)(806.181818181818,-3.84848484848485)(821.111111111111,-3.90909090909091)(832.308080808081,-4)(836.040404040404,-4.09090909090909)(838.528619528619,-4.18181818181818)(836.040404040404,-4.22727272727273)(828.575757575758,-4.36363636363636)(821.111111111111,-4.45454545454546)(809.914141414141,-4.54545454545455)(806.181818181818,-4.57575757575758)(791.252525252525,-4.6969696969697)(783.787878787879,-4.72727272727273)(776.323232323232,-4.75757575757576)(761.393939393939,-4.87878787878788)(753.929292929293,-4.90909090909091)(746.464646464646,-4.93939393939394)(731.535353535354,-5.06060606060606)(724.070707070707,-5.09090909090909)(716.606060606061,-5.12121212121212)(701.676767676768,-5.24242424242424)(694.212121212121,-5.27272727272727)(686.747474747475,-5.3030303030303)(671.818181818182,-5.42424242424242)(664.353535353535,-5.45454545454545)(656.888888888889,-5.48484848484848)(641.959595959596,-5.54545454545455)(627.030303030303,-5.60606060606061)(619.565656565657,-5.63636363636364)(612.10101010101,-5.66666666666667)(597.171717171717,-5.72727272727273)(582.242424242424,-5.78787878787879)(574.777777777778,-5.81818181818182)(567.313131313131,-5.84848484848485)(552.383838383838,-5.90909090909091)(537.454545454546,-5.90909090909091)(522.525252525253,-5.90909090909091)(507.59595959596,-5.90909090909091)(492.666666666667,-5.90909090909091)(477.737373737374,-5.90909090909091)(462.808080808081,-5.90909090909091)(447.878787878788,-5.90909090909091)(432.949494949495,-5.90909090909091)(418.020202020202,-5.90909090909091)(403.090909090909,-5.90909090909091)(388.161616161616,-5.90909090909091)(373.232323232323,-5.84848484848485)(365.767676767677,-5.81818181818182)(358.30303030303,-5.78787878787879)(343.373737373737,-5.72727272727273)(328.444444444444,-5.66666666666667)(324.712121212121,-5.63636363636364)(313.515151515152,-5.5)(311.026936026936,-5.45454545454545)(306.050505050505,-5.27272727272727)(311.026936026936,-5.09090909090909)(313.515151515152,-5.04545454545455)(320.979797979798,-4.90909090909091)(328.444444444444,-4.81818181818182)(335.909090909091,-4.72727272727273)(343.373737373737,-4.63636363636364)(354.570707070707,-4.54545454545455)(358.30303030303,-4.51515151515152)(373.232323232323,-4.39393939393939)(380.69696969697,-4.36363636363636)(388.161616161616,-4.33333333333333)(403.090909090909,-4.27272727272727)(418.020202020202,-4.21212121212121)(425.484848484849,-4.18181818181818)(432.949494949495,-4.15151515151515)(447.878787878788,-4.09090909090909)(462.808080808081,-4.09090909090909)(477.737373737374,-4.03030303030303)(485.20202020202,-4)(492.666666666667,-3.96969696969697)(507.59595959596,-3.90909090909091)(522.525252525253,-3.90909090909091)(537.454545454546,-3.84848484848485)(544.919191919192,-3.81818181818182)(552.383838383838,-3.78787878787879)(567.313131313131,-3.72727272727273)(582.242424242424,-3.72727272727273)(597.171717171717,-3.72727272727273)(612.10101010101,-3.72727272727273)(627.030303030303,-3.72727272727273)(641.959595959596,-3.72727272727273)(656.888888888889,-3.72727272727273)(671.818181818182,-3.72727272727273)(686.747474747475,-3.72727272727273)(701.676767676768,-3.72727272727273)(716.606060606061,-3.72727272727273)(731.535353535354,-3.72727272727273)(746.464646464646,-3.72727272727273)(761.393939393939,-3.72727272727273)(776.323232323232,-3.72727272727273)(791.252525252525,-3.78787878787879)(798.717171717172,-3.81818181818182)};

\addplot [fill=red!40!yellow,draw=none,forget plot] coordinates{ (82.1111111111111,-6.54545454545455)(89.5757575757576,-6.57575757575758)(104.505050505051,-6.63636363636364)(119.434343434343,-6.6969696969697)(123.166666666667,-6.72727272727273)(134.363636363636,-6.81818181818182)(141.828282828283,-6.90909090909091)(149.292929292929,-7.04545454545455)(151.781144781145,-7.09090909090909)(156.757575757576,-7.27272727272727)(156.757575757576,-7.45454545454545)(156.757575757576,-7.63636363636364)(156.757575757576,-7.81818181818182)(151.781144781145,-8)(149.292929292929,-8.04545454545455)(141.828282828283,-8.18181818181818)(134.363636363636,-8.27272727272727)(126.89898989899,-8.36363636363636)(119.434343434343,-8.45454545454545)(111.969696969697,-8.54545454545455)(104.505050505051,-8.63636363636364)(93.3080808080808,-8.72727272727273)(89.5757575757576,-8.75757575757576)(74.6464646464647,-8.87878787878788)(67.1818181818182,-8.90909090909091)(59.7171717171717,-8.93939393939394)(44.7878787878788,-9)(29.8585858585859,-9.06060606060606)(22.3939393939394,-9.09090909090909)(14.9292929292929,-9.12121212121212)(0,-9.18181818181818)(-0.000248804961825751,-9.09090909090909)(-0.000746390007464024,-8.90909090909091)(-0.000746390007464024,-8.72727272727273)(-0.000746390007464024,-8.54545454545455)(-0.000746390007464024,-8.36363636363636)(-0.000746390007464024,-8.18181818181818)(-0.000746390007464024,-8)(-0.000746390007464024,-7.81818181818182)(-0.000746390007464024,-7.63636363636364)(-0.000746390007464024,-7.45454545454545)(-0.000746390007464024,-7.27272727272727)(-0.000746390007464024,-7.09090909090909)(-0.000746390007464024,-6.90909090909091)(-0.000746390007464024,-6.72727272727273)(-0.000248804961825751,-6.54545454545455)(0,-6.45454545454546)(14.9292929292929,-6.45454545454546)(29.8585858585859,-6.45454545454546)(44.7878787878788,-6.45454545454546)(59.7171717171717,-6.45454545454546)(74.6464646464647,-6.51515151515152)(82.1111111111111,-6.54545454545455)};

\addplot [fill=red!40!yellow,draw=none,forget plot] coordinates{ (1157.0202020202,-8.36363636363636)(1164.48484848485,-8.39393939393939)(1179.41414141414,-8.51515151515152)(1183.14646464646,-8.54545454545455)(1194.34343434343,-8.68181818181818)(1196.83164983165,-8.72727272727273)(1196.83164983165,-8.90909090909091)(1194.34343434343,-8.95454545454546)(1183.14646464646,-9.09090909090909)(1179.41414141414,-9.12121212121212)(1164.48484848485,-9.18181818181818)(1149.55555555556,-9.18181818181818)(1134.62626262626,-9.24242424242424)(1127.16161616162,-9.27272727272727)(1119.69696969697,-9.3030303030303)(1104.76767676768,-9.36363636363636)(1089.83838383838,-9.36363636363636)(1074.90909090909,-9.36363636363636)(1059.9797979798,-9.36363636363636)(1045.05050505051,-9.36363636363636)(1030.12121212121,-9.3030303030303)(1022.65656565657,-9.27272727272727)(1015.19191919192,-9.24242424242424)(1000.26262626263,-9.18181818181818)(985.333333333333,-9.18181818181818)(970.40404040404,-9.18181818181818)(955.474747474747,-9.12121212121212)(948.010101010101,-9.09090909090909)(940.545454545455,-9.06060606060606)(928.104377104377,-8.90909090909091)(925.616161616162,-8.86363636363636)(918.151515151515,-8.72727272727273)(923.127946127946,-8.54545454545455)(925.616161616162,-8.5)(936.813131313131,-8.36363636363636)(940.545454545455,-8.33333333333333)(955.474747474747,-8.27272727272727)(970.40404040404,-8.27272727272727)(985.333333333333,-8.27272727272727)(1000.26262626263,-8.27272727272727)(1015.19191919192,-8.27272727272727)(1030.12121212121,-8.27272727272727)(1045.05050505051,-8.27272727272727)(1059.9797979798,-8.27272727272727)(1074.90909090909,-8.27272727272727)(1089.83838383838,-8.27272727272727)(1104.76767676768,-8.27272727272727)(1119.69696969697,-8.27272727272727)(1134.62626262626,-8.27272727272727)(1149.55555555556,-8.33333333333333)(1157.0202020202,-8.36363636363636)};

\addplot [
color=white,
draw=white,
only marks,
mark=x,
mark options={solid},
mark size=2.0pt,
line width=0.3pt,
forget plot
]
coordinates{
 (0,0)(0,-0.181818181818182)(0,-0.363636363636364)(0,-0.545454545454545)(0,-0.727272727272727)(0,-0.909090909090909)(0,-1.09090909090909)(0,-1.27272727272727)(0,-1.45454545454545)(0,-1.63636363636364)(0,-1.81818181818182)(0,-2)(0,-2.18181818181818)(0,-2.36363636363636)(0,-2.54545454545455)(0,-2.72727272727273)(0,-2.90909090909091)(0,-3.09090909090909)(0,-3.27272727272727)(0,-3.45454545454545)(0,-3.63636363636364)(0,-3.81818181818182)(0,-4)(0,-4.18181818181818)(0,-4.36363636363636)(0,-4.54545454545455)(0,-4.72727272727273)(0,-4.90909090909091)(0,-5.09090909090909)(0,-5.27272727272727)(104.505050505051,-10.1818181818182)(119.434343434343,-10.1818181818182)(89.5757575757576,-10.1818181818182)(134.363636363636,-10.1818181818182)(74.6464646464647,-10.1818181818182)(59.7171717171717,-10.1818181818182)(149.292929292929,-10.1818181818182)(44.7878787878788,-10.1818181818182)(29.8585858585859,-10.1818181818182)(14.9292929292929,-10.1818181818182)(0,-10.1818181818182)(164.222222222222,-10.1818181818182)(209.010101010101,-10)(223.939393939394,-10)(194.080808080808,-10)(238.868686868687,-10)(179.151515151515,-10.1818181818182)(179.151515151515,-10)(0,-10.3636363636364)(14.9292929292929,-10.3636363636364)(29.8585858585859,-10.3636363636364)(44.7878787878788,-10.3636363636364)(253.79797979798,-10)(59.7171717171717,-10.3636363636364)(283.656565656566,-9.81818181818182)(194.080808080808,-10.1818181818182)(164.222222222222,-10)(268.727272727273,-9.81818181818182)(74.6464646464647,-10.3636363636364)(298.585858585859,-9.81818181818182)(1478,-18)(1463.07070707071,-18)(1478,-17.8181818181818)(1448.14141414141,-18)(1463.07070707071,-17.8181818181818)(1433.21212121212,-18)(1478,-17.6363636363636)(1448.14141414141,-17.8181818181818)(1418.28282828283,-18)(1463.07070707071,-17.6363636363636)(1478,-5.45454545454545)(1478,-5.27272727272727)(1433.21212121212,-17.8181818181818)(1478,-17.4545454545455)(1478,-5.63636363636364)(1478,-5.09090909090909)(1403.35353535354,-18)(1448.14141414141,-17.6363636363636)(1478,-5.81818181818182)(1463.07070707071,-5.45454545454545)(1463.07070707071,-5.27272727272727)(1478,-4.90909090909091)(1418.28282828283,-17.8181818181818)(1463.07070707071,-5.63636363636364)(1463.07070707071,-5.09090909090909)(1463.07070707071,-17.4545454545455)(1388.42424242424,-18)(1478,-6)(1433.21212121212,-17.6363636363636)(1478,-17.2727272727273)(940.545454545455,-4.54545454545455)(925.616161616162,-4.54545454545455)(940.545454545455,-4.36363636363636)(925.616161616162,-4.72727272727273)(955.474747474747,-4.36363636363636)(955.474747474747,-4.54545454545455)(925.616161616162,-4.36363636363636)(940.545454545455,-4.72727272727273)(910.686868686869,-4.54545454545455)(910.686868686869,-4.72727272727273)(955.474747474747,-4.18181818181818)(940.545454545455,-4.18181818181818)(910.686868686869,-4.36363636363636)(970.40404040404,-4.36363636363636)(955.474747474747,-4.72727272727273)(895.757575757576,-4.72727272727273)(970.40404040404,-4.54545454545455)(925.616161616162,-4.90909090909091)(895.757575757576,-4.54545454545455)(925.616161616162,-4.18181818181818)(970.40404040404,-4.18181818181818)(910.686868686869,-4.90909090909091)(940.545454545455,-4.90909090909091)(895.757575757576,-4.90909090909091)(955.474747474747,-4)(895.757575757576,-4.36363636363636)(970.40404040404,-4.72727272727273)(910.686868686869,-4.18181818181818)(940.545454545455,-4)(970.40404040404,-4)(627.030303030303,-2.90909090909091)(627.030303030303,-2.72727272727273)(641.959595959596,-2.72727272727273)(612.10101010101,-2.90909090909091)(641.959595959596,-2.54545454545455)(656.888888888889,-2.54545454545455)(612.10101010101,-3.09090909090909)(597.171717171717,-3.09090909090909)(641.959595959596,-2.90909090909091)(612.10101010101,-2.72727272727273)(656.888888888889,-2.72727272727273)(597.171717171717,-2.90909090909091)(627.030303030303,-2.54545454545455)(627.030303030303,-3.09090909090909)(656.888888888889,-2.36363636363636)(671.818181818182,-2.36363636363636)(671.818181818182,-2.54545454545455)(597.171717171717,-3.27272727272727)(582.242424242424,-3.09090909090909)(641.959595959596,-2.36363636363636)(686.747474747475,-2.36363636363636)(582.242424242424,-3.27272727272727)(612.10101010101,-3.27272727272727)(686.747474747475,-2.18181818181818)(671.818181818182,-2.18181818181818)(656.888888888889,-2.90909090909091)(671.818181818182,-2.72727272727273)(597.171717171717,-2.72727272727273)(612.10101010101,-2.54545454545455)(641.959595959596,-3.09090909090909)(477.737373737374,-5.45454545454545)(492.666666666667,-5.45454545454545)(462.808080808081,-5.45454545454545)(492.666666666667,-5.27272727272727)(477.737373737374,-5.27272727272727)(477.737373737374,-5.63636363636364)(507.59595959596,-5.45454545454545)(462.808080808081,-5.63636363636364)(492.666666666667,-5.63636363636364)(507.59595959596,-5.27272727272727)(447.878787878788,-5.45454545454545)(462.808080808081,-5.27272727272727)(447.878787878788,-5.63636363636364)(507.59595959596,-5.63636363636364)(522.525252525253,-5.45454545454545)(447.878787878788,-5.27272727272727)(432.949494949495,-5.45454545454545)(477.737373737374,-5.81818181818182)(432.949494949495,-5.63636363636364)(462.808080808081,-5.81818181818182)(522.525252525253,-5.63636363636364)(492.666666666667,-5.81818181818182)(447.878787878788,-5.81818181818182)(522.525252525253,-5.27272727272727)(537.454545454546,-5.45454545454545)(432.949494949495,-5.27272727272727)(462.808080808081,-5.09090909090909)(507.59595959596,-5.81818181818182)(432.949494949495,-5.81818181818182)(418.020202020202,-5.45454545454545)(1478,-1.09090909090909)(1478,-1.27272727272727)(1478,-0.909090909090909)(1478,-1.45454545454545)(1478,-0.727272727272727)(1463.07070707071,-1.09090909090909)(1463.07070707071,-1.27272727272727)(1463.07070707071,-0.909090909090909)(1478,-0.545454545454545)(1463.07070707071,-1.45454545454545)(1478,-1.63636363636364)(1463.07070707071,-0.727272727272727)(1448.14141414141,-1.09090909090909)(1448.14141414141,-1.27272727272727)(1478,-0.363636363636364)(1448.14141414141,-0.909090909090909)(1463.07070707071,-0.545454545454545)(1448.14141414141,-1.45454545454545)(1463.07070707071,-1.63636363636364)(1448.14141414141,-0.727272727272727)(1478,-0.181818181818182)(1463.07070707071,-0.363636363636364)(1433.21212121212,-1.27272727272727)(1433.21212121212,-1.09090909090909)(1478,-1.81818181818182)(1448.14141414141,-1.63636363636364)(1448.14141414141,-0.545454545454545)(1433.21212121212,-0.909090909090909)(1433.21212121212,-1.45454545454545)(1478,0)(0,-18)(14.9292929292929,-18)(0,-17.8181818181818)(29.8585858585859,-18)(14.9292929292929,-17.8181818181818)(44.7878787878788,-18)(0,-17.6363636363636)(29.8585858585859,-17.8181818181818)(59.7171717171717,-18)(14.9292929292929,-17.6363636363636)(44.7878787878788,-17.8181818181818)(74.6464646464647,-18)(0,-17.4545454545455)(29.8585858585859,-17.6363636363636)(59.7171717171717,-17.8181818181818)(89.5757575757576,-18)(14.9292929292929,-17.4545454545455)(44.7878787878788,-17.6363636363636)(74.6464646464647,-17.8181818181818)(104.505050505051,-18)(29.8585858585859,-17.4545454545455)(59.7171717171717,-17.6363636363636)(0,-17.2727272727273)(89.5757575757576,-17.8181818181818)(119.434343434343,-18)(44.7878787878788,-17.4545454545455)(74.6464646464647,-17.6363636363636)(14.9292929292929,-17.2727272727273)(104.505050505051,-17.8181818181818)(134.363636363636,-18)(1478,-10.5454545454545)(1478,-10.7272727272727)(1478,-10.3636363636364)(1463.07070707071,-10.5454545454545)(1463.07070707071,-10.7272727272727)(1478,-10.9090909090909)(1463.07070707071,-10.3636363636364)(1448.14141414141,-10.7272727272727)(1448.14141414141,-10.5454545454545)(1463.07070707071,-10.9090909090909)(1478,-10.1818181818182)(1433.21212121212,-10.7272727272727)(1433.21212121212,-10.5454545454545)(1448.14141414141,-10.9090909090909)(1448.14141414141,-10.3636363636364)(1418.28282828283,-10.7272727272727)(1478,-11.0909090909091)(1433.21212121212,-10.9090909090909)(1418.28282828283,-10.5454545454545)(1463.07070707071,-10.1818181818182)(1433.21212121212,-10.3636363636364)(1463.07070707071,-11.0909090909091)(1403.35353535354,-10.7272727272727)(1418.28282828283,-10.9090909090909)(1403.35353535354,-10.5454545454545)(1448.14141414141,-11.0909090909091)(1403.35353535354,-10.9090909090909)(1418.28282828283,-10.3636363636364)(1388.42424242424,-10.7272727272727)(1448.14141414141,-10.1818181818182)(1478,-8.54545454545455)(1478,-8.36363636363636)(1463.07070707071,-8.54545454545455)(1463.07070707071,-8.36363636363636)(1448.14141414141,-8.36363636363636)(1448.14141414141,-8.54545454545455)(1478,-8.18181818181818)(1433.21212121212,-8.36363636363636)(1478,-8.72727272727273)(1433.21212121212,-8.54545454545455)(1418.28282828283,-8.36363636363636)(1418.28282828283,-8.54545454545455)(1403.35353535354,-8.36363636363636)(1403.35353535354,-8.54545454545455)(1388.42424242424,-8.36363636363636)(1313.77777777778,-8.36363636363636)(1328.70707070707,-8.36363636363636)(1343.63636363636,-8.36363636363636)(1298.84848484848,-8.36363636363636)(1373.49494949495,-8.36363636363636)(1358.56565656566,-8.36363636363636)(1313.77777777778,-8.54545454545455)(1463.07070707071,-8.18181818181818)(1298.84848484848,-8.54545454545455)(1328.70707070707,-8.54545454545455)(1388.42424242424,-8.54545454545455)(1283.91919191919,-8.36363636363636)(1343.63636363636,-8.54545454545455)(1283.91919191919,-8.54545454545455)(1373.49494949495,-8.54545454545455)(940.545454545455,-9.09090909090909)(925.616161616162,-9.09090909090909)(940.545454545455,-9.27272727272727)(925.616161616162,-9.27272727272727)(955.474747474747,-9.09090909090909)(910.686868686869,-9.09090909090909)(955.474747474747,-9.27272727272727)(910.686868686869,-9.27272727272727)(925.616161616162,-8.90909090909091)(940.545454545455,-8.90909090909091)(910.686868686869,-8.90909090909091)(955.474747474747,-8.90909090909091)(895.757575757576,-9.09090909090909)(970.40404040404,-9.09090909090909)(925.616161616162,-9.45454545454546)(940.545454545455,-9.45454545454546)(970.40404040404,-9.27272727272727)(895.757575757576,-9.27272727272727)(910.686868686869,-9.45454545454546)(955.474747474747,-9.45454545454546)(895.757575757576,-8.90909090909091)(970.40404040404,-8.90909090909091)(880.828282828283,-9.09090909090909)(925.616161616162,-8.72727272727273)(895.757575757576,-9.45454545454546)(985.333333333333,-9.09090909090909)(940.545454545455,-8.72727272727273)(970.40404040404,-9.45454545454546)(985.333333333333,-9.27272727272727)(880.828282828283,-9.27272727272727)(1045.05050505051,-7.09090909090909)(1030.12121212121,-7.09090909090909)(1059.9797979798,-7.09090909090909)(1015.19191919192,-7.09090909090909)(1000.26262626263,-7.09090909090909)(1074.90909090909,-7.09090909090909)(985.333333333333,-7.09090909090909)(1089.83838383838,-7.09090909090909)(970.40404040404,-7.09090909090909)(1104.76767676768,-7.09090909090909)(955.474747474747,-7.09090909090909)(880.828282828283,-7.27272727272727)(940.545454545455,-7.09090909090909)(1134.62626262626,-7.27272727272727)(1119.69696969697,-7.09090909090909)(865.89898989899,-7.27272727272727)(925.616161616162,-7.09090909090909)(1119.69696969697,-7.27272727272727)(910.686868686869,-7.09090909090909)(1104.76767676768,-7.27272727272727)(895.757575757576,-7.27272727272727)(850.969696969697,-7.27272727272727)(836.040404040404,-7.45454545454545)(1059.9797979798,-6.90909090909091)(1134.62626262626,-7.09090909090909)(1045.05050505051,-6.90909090909091)(1074.90909090909,-6.90909090909091)(895.757575757576,-7.09090909090909)(1089.83838383838,-7.27272727272727)(1030.12121212121,-6.90909090909091)(358.30303030303,-4.18181818181818)(343.373737373737,-4.36363636363636)(343.373737373737,-4.18181818181818)(388.161616161616,-4)(328.444444444444,-4.36363636363636)(373.232323232323,-4)(328.444444444444,-4.54545454545455)(328.444444444444,-4.18181818181818)(373.232323232323,-4.18181818181818)(358.30303030303,-4)(403.090909090909,-4)(313.515151515152,-4.54545454545455)(313.515151515152,-4.36363636363636)(343.373737373737,-4)(313.515151515152,-4.72727272727273)(313.515151515152,-4.18181818181818)(298.585858585859,-4.54545454545455)(298.585858585859,-4.36363636363636)(328.444444444444,-4)(298.585858585859,-4.72727272727273)(358.30303030303,-4.36363636363636)(403.090909090909,-3.81818181818182)(388.161616161616,-3.81818181818182)(0,-7.81818181818182)(418.020202020202,-3.81818181818182)(313.515151515152,-4.90909090909091)(0,-7.63636363636364)(298.585858585859,-4.18181818181818)(373.232323232323,-3.81818181818182)(432.949494949495,-3.81818181818182)(238.868686868687,-7.63636363636364)(253.79797979798,-7.45454545454545)(238.868686868687,-7.81818181818182)(253.79797979798,-7.27272727272727)(253.79797979798,-7.63636363636364)(238.868686868687,-7.45454545454545)(223.939393939394,-8)(253.79797979798,-7.09090909090909)(238.868686868687,-7.27272727272727)(268.727272727273,-7.45454545454545) 
};

%\node at (axis cs:50, -2.5) [shape=circle,fill=white,draw=black,inner sep=0pt,anchor=south west] {\scriptsize\color{locol}$\*L_{\*t}$};
%\node at (axis cs:460, -5.9) [shape=circle,fill=white,draw=black,inner sep=0pt,anchor=south west] {\scriptsize\color{orange!50!yellow}$\*H_{\*t}$};
%\node at (axis cs:160, -5.2) [shape=circle,fill=white,draw=black,inner sep=0pt,anchor=south west] {\scriptsize\color{darkgray}$\*U_{\*t}$};

\node at (axis cs:1155, -17) [shape=circle,fill=red!40!yellow,draw=black,inner sep=0.2pt,anchor=south west,minimum size=16pt]
  {\scriptsize\color{white}$\hat{\*H}$};
\node at (axis cs:1330, -17) [shape=circle,fill=locol,draw=black,inner sep=0.2pt,anchor=south west,minimum size=16pt]
  {\scriptsize\color{white}$\hat{\*L}$};

\end{axis}
\end{tikzpicture}%

\renewcommand\trimlen{2pt}
\begin{figure}[tb]
  \begin{subfigure}[b]{0.49\textwidth}
    \centering
    \adjincludegraphics[width=\linewidth,clip=true,trim=\trimlen{} \trimlen{} \trimlen{} \trimlen{}]{figures/ev_ping_batch}
    \caption{\textsf{[N]} Mean performance}
	  \label{fig:ping-batch}
  \end{subfigure}
  \hfill
  \begin{subfigure}[b]{0.49\textwidth}
    \centering
    \adjincludegraphics[width=\linewidth,clip=true,trim=\trimlen{} \trimlen{} \trimlen{} \trimlen{}]{figures/ev_chl_batch}
    \caption{\textsf{[C]} Mean performance}
	\label{fig:chl-batch}
  \end{subfigure}

  \begin{subfigure}[b]{0.49\textwidth}
    \centering
    \vspace{12pt} % space of this row from above captions
    \adjincludegraphics[width=\linewidth,clip=true,trim=\trimlen{} \trimlen{} \trimlen{} \trimlen{}]{figures/ev_bgape_batch}
    \caption{\textsf{[A]} Mean performance}
	  \label{fig:bgape-batch}
  \end{subfigure}
  \hfill
  \begin{subfigure}[b]{0.49\textwidth}
    \centering
    \adjincludegraphics[width=\linewidth,clip=true,trim=\trimlen{} \trimlen{} \trimlen{} \trimlen{}]{figures/limno_bgape_rstr_class400}
    \caption{\textsf{[C]} \rstr after $t=400$ iterations}
	\label{fig:bgape-rstr-class}
  \end{subfigure}

  \caption{Performance of explicit threshold batch algorithms on the
           three datasets.
           \textbf{(a)--(c)} \bacl and \bstr achieve comparable performance,
           which is slightly worse than that of their sequential counterparts,
           but significantly better than that of the naive \rstr algorithm.
           \textbf{(d)} The redundant sampling behavior of \rstr resulting
           from not using up-to-date variance estimates.
           }
  \label{fig:exp-batch}
\end{figure}

\paragraph{Batch sampling}
In \figsref{fig:ping-batch}--\ref{fig:bgape-batch} we show the performance
of the explicit threshold batch algorithms on the three datasets. The \bacl
and \bstr algorithms,
which use the always up-to-date variance estimates for selecting batches,
achieve similar performance. Furthermore, there is only a slight performance
penalty when compared to the strictly sequential \str, which can easily be
outweighed by the benefits of batch point selection (e.g. in the network
latency application, the batch algorithms would have about $B = 30$ times
higher throughput). On the other hand, as expected, the \rstr algorithm
performs significantly worse than the other two batch algorithms, since it
selects a lot of redundant samples in areas of high straddle score, as depicted
in \figref{fig:bgape-rstr-class} (cf. \sectref{sect:bacl}).

\section{Results II:\hspace{0.33em}Implicit threshold level}
In \figsref{fig:chl-imp} and~\ref{fig:bgape-imp} we present the results of
executing the implicit threshold algorithms on the two
environmental monitoring datasets. The difficulty of estimating the
function maximum at the same time as performing classification with respect to
the implicit threshold level is manifested in the notably larger sampling cost
of \iacl required to achieve high accuracy compared to the explicit threshold
experiments. As before, the batch version of
\iacl is only slightly worse that its sequential counterpart.

More importantly, the naive \istr algorithm
fails to achieve high accuracy, as it mostly infers wrong estimates of
the function maximum and never recovers, since the (modified) straddle rule
gives no incentive for sampling near the maximum.
This is not very clear in the case of the chlorophyll dataset, because
the function is fairly smooth in the vicinity of its maxima and, thus,
\istr is able to obtain a rather accurate estimate of the function
maximum.
However, in the case of the algae dataset,
as can be seen in \figref{fig:bgape-imp-istr-sc}, most
of the executions of \istr achieve an $F_1$-score of only about
0.8, even after $400$ samples. The inability of \istr to recover from a wrong
threshold estimate is illustrated by the fact that most of these executions
have already achieved this $F_1$-score after less than $100$ samples, but show
no tendency for improvement thereafter. The behavior of \iacl depicted in
more detail in \figref{fig:bgape-imp-istr-sc} is very different
in that it starts considerably slower, but always achieves $F_1$-scores of
at least $0.9$ after $400$ samples.

\figref{fig:bgape-imp-istr-h} shows the estimated implicit threshold
level for an example execution of \istr on the algae concentration
dataset. Note that the inferred level
is smaller than the true level $h = 7$ and, even worse, seems to
deviate further away from the true level over time.
In contrast, as shown in \figref{fig:bgape-imp-iacl-h}, the optimistic
and pessimistic estimates used by \iacl correctly bound the true level
from above and below respectively and get more accurate over time
(as predicted by theory; cf. \lemmaref{lem:hdif}
and~\lemmasref{lem:hopt}--\ref{lem:hpes}),
thus allowing for better classification results.

Our \iacl simulations have shown that, as one might expect, the sooner
the maximum of the function is accurately estimated the faster the
algorithm converges to an accurate classification. Some preliminary
experimentation with simple heuristics, such as sampling according
to the \gpucb rule every $k_t$ steps (with $k_t$ starting small and
increasing with time), leads to improved overall sampling costs and
would be an interesting direction to explore theoretically.

%\setlength\figureheight{1.3in}\setlength\figurewidth{2.1in}
%% This file was created by matlab2tikz v0.2.3.
% Copyright (c) 2008--2012, Nico Schlömer <nico.schloemer@gmail.com>
% All rights reserved.
% 
% 
% 
\begin{tikzpicture}

\begin{axis}[%
tick label style={font=\tiny},
label style={font=\tiny},
label shift={-4pt},
xlabel shift={-6pt},
legend style={font=\tiny},
view={0}{90},
width=\figurewidth,
height=\figureheight,
scale only axis,
xmin=0, xmax=600,
xlabel={Samples},
ymin=0.5, ymax=1,
ylabel={$F_1$-score},
axis lines*=left,
legend cell align=left,
legend style={at={(1.03,0)},anchor=south east,fill=none,draw=none,align=left,row sep=-0.2em},
clip=false]

\addplot [
color=red,
densely dotted,
line width=1.0pt,
]
coordinates{
 (12,0.504865078580825)(12,0.504865078580825)(12,0.504865078580825)(12,0.504865078580825)(12,0.504865078580825)(12,0.504865078580825)(12,0.504865078580825)(12,0.504865078580825)(12,0.504865078580825)(12,0.504865078580825)(12,0.504865078580825)(12,0.504865078580825)(12,0.504865078580825)(12,0.504865078580825)(12,0.504865078580825)(12,0.504865078580825)(12,0.504865078580825)(12,0.504865078580825)(12,0.504865078580825)(12,0.504865078580825)(12,0.504865078580825)(12,0.504865078580825)(12,0.504865078580825)(12,0.504865078580825)(12,0.504865078580825)(12,0.504865078580825)(12,0.504865078580825)(12,0.504865078580825)(12,0.504865078580825)(12,0.504865078580825)(12,0.504865078580825)(12,0.504865078580825)(12,0.504865078580825)(12,0.504865078580825)(12,0.504865078580825)(12,0.504865078580825)(12,0.504865078580825)(12,0.504865078580825)(12,0.504865078580825)(12,0.504865078580825)(12,0.504865078580825)(12,0.504865078580825)(12,0.504865078580825)(12,0.504865078580825)(12,0.504865078580825)(12,0.504865078580825)(12,0.504865078580825)(12,0.504865078580825)(12,0.504865078580825)(12,0.504865078580825)(12,0.504865078580825)(12,0.504865078580825)(13,0.523169792868839)(13,0.523169792868839)(13,0.523169792868839)(13,0.523169792868839)(13,0.523169792868839)(13,0.523169792868839)(13,0.523169792868839)(13,0.523169792868839)(13,0.523169792868839)(13,0.523169792868839)(13,0.523169792868839)(13,0.523169792868839)(13,0.523169792868839)(13,0.523169792868839)(13,0.523169792868839)(13,0.523169792868839)(13,0.523169792868839)(13,0.523169792868839)(13,0.523169792868839)(13,0.523169792868839)(13,0.523169792868839)(13,0.523169792868839)(13,0.523169792868839)(13,0.523169792868839)(13,0.523169792868839)(13,0.523169792868839)(13,0.523169792868839)(13,0.523169792868839)(13,0.523169792868839)(13,0.523169792868839)(13,0.523169792868839)(13,0.523169792868839)(13,0.523169792868839)(13,0.523169792868839)(13,0.523169792868839)(13,0.523169792868839)(13,0.523169792868839)(13,0.523169792868839)(13,0.523169792868839)(13,0.523169792868839)(13,0.523169792868839)(13,0.523169792868839)(13,0.523169792868839)(13,0.523169792868839)(13,0.523169792868839)(13,0.523169792868839)(14,0.546177091572435)(14,0.546177091572435)(14,0.546177091572435)(14,0.546177091572435)(14,0.546177091572435)(14,0.546177091572435)(14,0.546177091572435)(14,0.546177091572435)(14,0.546177091572435)(14,0.546177091572435)(14,0.546177091572435)(14,0.546177091572435)(14,0.546177091572435)(14,0.546177091572435)(14,0.546177091572435)(14,0.546177091572435)(14,0.546177091572435)(14,0.546177091572435)(14,0.546177091572435)(14,0.546177091572435)(14,0.546177091572435)(14,0.546177091572435)(14,0.546177091572435)(14,0.546177091572435)(14,0.546177091572435)(14,0.546177091572435)(14,0.546177091572435)(14,0.546177091572435)(14,0.546177091572435)(14,0.546177091572435)(14,0.546177091572435)(14,0.546177091572435)(14,0.546177091572435)(14,0.546177091572435)(14,0.546177091572435)(14,0.546177091572435)(14,0.546177091572435)(14,0.546177091572435)(14,0.546177091572435)(14,0.546177091572435)(14,0.546177091572435)(14,0.546177091572435)(14,0.546177091572435)(14,0.546177091572435)(14,0.546177091572435)(14,0.546177091572435)(14,0.546177091572435)(14,0.546177091572435)(14,0.546177091572435)(14,0.546177091572435)(14,0.546177091572435)(14,0.546177091572435)(14,0.546177091572435)(14,0.546177091572435)(14,0.546177091572435)(14,0.546177091572435)(14,0.546177091572435)(14,0.546177091572435)(14,0.546177091572435)(14,0.546177091572435)(14,0.546177091572435)(14,0.546177091572435)(14,0.546177091572435)(14,0.546177091572435)(14,0.546177091572435)(14,0.546177091572435)(15,0.573370808065954)(15,0.573370808065954)(15,0.573370808065954)(15,0.573370808065954)(15,0.573370808065954)(15,0.573370808065954)(15,0.573370808065954)(15,0.573370808065954)(15,0.573370808065954)(15,0.573370808065954)(15,0.573370808065954)(15,0.573370808065954)(15,0.573370808065954)(15,0.573370808065954)(15,0.573370808065954)(15,0.573370808065954)(15,0.573370808065954)(15,0.573370808065954)(15,0.573370808065954)(15,0.573370808065954)(15,0.573370808065954)(15,0.573370808065954)(15,0.573370808065954)(15,0.573370808065954)(15,0.573370808065954)(15,0.573370808065954)(15,0.573370808065954)(15,0.573370808065954)(15,0.573370808065954)(15,0.573370808065954)(15,0.573370808065954)(15,0.573370808065954)(15,0.573370808065954)(15,0.573370808065954)(15,0.573370808065954)(15,0.573370808065954)(15,0.573370808065954)(15,0.573370808065954)(15,0.573370808065954)(15,0.573370808065954)(15,0.573370808065954)(15,0.573370808065954)(15,0.573370808065954)(15,0.573370808065954)(15,0.573370808065954)(15,0.573370808065954)(15,0.573370808065954)(15,0.573370808065954)(15,0.573370808065954)(15,0.573370808065954)(16,0.599320887872199)(16,0.599320887872199)(16,0.599320887872199)(16,0.599320887872199)(16,0.599320887872199)(16,0.599320887872199)(16,0.599320887872199)(16,0.599320887872199)(16,0.599320887872199)(16,0.599320887872199)(16,0.599320887872199)(16,0.599320887872199)(16,0.599320887872199)(16,0.599320887872199)(16,0.599320887872199)(16,0.599320887872199)(16,0.599320887872199)(16,0.599320887872199)(16,0.599320887872199)(16,0.599320887872199)(16,0.599320887872199)(16,0.599320887872199)(16,0.599320887872199)(16,0.599320887872199)(16,0.599320887872199)(16,0.599320887872199)(16,0.599320887872199)(16,0.599320887872199)(16,0.599320887872199)(16,0.599320887872199)(16,0.599320887872199)(16,0.599320887872199)(16,0.599320887872199)(16,0.599320887872199)(16,0.599320887872199)(16,0.599320887872199)(16,0.599320887872199)(16,0.599320887872199)(16,0.599320887872199)(16,0.599320887872199)(16,0.599320887872199)(16,0.599320887872199)(16,0.599320887872199)(16,0.599320887872199)(16,0.599320887872199)(16,0.599320887872199)(16,0.599320887872199)(16,0.599320887872199)(16,0.599320887872199)(16,0.599320887872199)(16,0.599320887872199)(17,0.619901915077436)(17,0.619901915077436)(17,0.619901915077436)(17,0.619901915077436)(17,0.619901915077436)(17,0.619901915077436)(17,0.619901915077436)(17,0.619901915077436)(17,0.619901915077436)(17,0.619901915077436)(17,0.619901915077436)(17,0.619901915077436)(17,0.619901915077436)(17,0.619901915077436)(17,0.619901915077436)(17,0.619901915077436)(17,0.619901915077436)(17,0.619901915077436)(17,0.619901915077436)(17,0.619901915077436)(17,0.619901915077436)(17,0.619901915077436)(17,0.619901915077436)(17,0.619901915077436)(17,0.619901915077436)(17,0.619901915077436)(17,0.619901915077436)(17,0.619901915077436)(17,0.619901915077436)(17,0.619901915077436)(17,0.619901915077436)(17,0.619901915077436)(17,0.619901915077436)(17,0.619901915077436)(17,0.619901915077436)(17,0.619901915077436)(17,0.619901915077436)(17,0.619901915077436)(17,0.619901915077436)(17,0.619901915077436)(17,0.619901915077436)(17,0.619901915077436)(17,0.619901915077436)(17,0.619901915077436)(17,0.619901915077436)(17,0.619901915077436)(17,0.619901915077436)(17,0.619901915077436)(17,0.619901915077436)(17,0.619901915077436)(17,0.619901915077436)(17,0.619901915077436)(17,0.619901915077436)(17,0.619901915077436)(17,0.619901915077436)(17,0.619901915077436)(17,0.619901915077436)(17,0.619901915077436)(17,0.619901915077436)(17,0.619901915077436)(17,0.619901915077436)(17,0.619901915077436)(17,0.619901915077436)(17,0.619901915077436)(17,0.619901915077436)(18,0.632419995199887)(18,0.632419995199887)(18,0.632419995199887)(18,0.632419995199887)(18,0.632419995199887)(18,0.632419995199887)(18,0.632419995199887)(18,0.632419995199887)(18,0.632419995199887)(18,0.632419995199887)(18,0.632419995199887)(18,0.632419995199887)(18,0.632419995199887)(18,0.632419995199887)(18,0.632419995199887)(18,0.632419995199887)(18,0.632419995199887)(18,0.632419995199887)(18,0.632419995199887)(18,0.632419995199887)(18,0.632419995199887)(18,0.632419995199887)(18,0.632419995199887)(18,0.632419995199887)(18,0.632419995199887)(18,0.632419995199887)(18,0.632419995199887)(18,0.632419995199887)(18,0.632419995199887)(18,0.632419995199887)(18,0.632419995199887)(18,0.632419995199887)(18,0.632419995199887)(18,0.632419995199887)(18,0.632419995199887)(18,0.632419995199887)(18,0.632419995199887)(18,0.632419995199887)(18,0.632419995199887)(18,0.632419995199887)(18,0.632419995199887)(18,0.632419995199887)(18,0.632419995199887)(18,0.632419995199887)(18,0.632419995199887)(18,0.632419995199887)(18,0.632419995199887)(18,0.632419995199887)(18,0.632419995199887)(18,0.632419995199887)(18,0.632419995199887)(18,0.632419995199887)(18,0.632419995199887)(18,0.632419995199887)(19,0.646012458484744)(19,0.646012458484744)(19,0.646012458484744)(19,0.646012458484744)(19,0.646012458484744)(19,0.646012458484744)(19,0.646012458484744)(19,0.646012458484744)(19,0.646012458484744)(19,0.646012458484744)(19,0.646012458484744)(19,0.646012458484744)(19,0.646012458484744)(19,0.646012458484744)(19,0.646012458484744)(19,0.646012458484744)(19,0.646012458484744)(19,0.646012458484744)(19,0.646012458484744)(19,0.646012458484744)(19,0.646012458484744)(19,0.646012458484744)(19,0.646012458484744)(19,0.646012458484744)(19,0.646012458484744)(19,0.646012458484744)(19,0.646012458484744)(19,0.646012458484744)(19,0.646012458484744)(19,0.646012458484744)(19,0.646012458484744)(19,0.646012458484744)(19,0.646012458484744)(19,0.646012458484744)(19,0.646012458484744)(19,0.646012458484744)(19,0.646012458484744)(19,0.646012458484744)(19,0.646012458484744)(19,0.646012458484744)(19,0.646012458484744)(19,0.646012458484744)(19,0.646012458484744)(19,0.646012458484744)(19,0.646012458484744)(19,0.646012458484744)(19,0.646012458484744)(20,0.664717621965014)(20,0.664717621965014)(20,0.664717621965014)(20,0.664717621965014)(20,0.664717621965014)(20,0.664717621965014)(20,0.664717621965014)(20,0.664717621965014)(20,0.664717621965014)(20,0.664717621965014)(20,0.664717621965014)(20,0.664717621965014)(20,0.664717621965014)(20,0.664717621965014)(20,0.664717621965014)(20,0.664717621965014)(20,0.664717621965014)(20,0.664717621965014)(20,0.664717621965014)(20,0.664717621965014)(20,0.664717621965014)(20,0.664717621965014)(20,0.664717621965014)(20,0.664717621965014)(20,0.664717621965014)(20,0.664717621965014)(20,0.664717621965014)(20,0.664717621965014)(20,0.664717621965014)(20,0.664717621965014)(20,0.664717621965014)(20,0.664717621965014)(20,0.664717621965014)(20,0.664717621965014)(20,0.664717621965014)(20,0.664717621965014)(20,0.664717621965014)(20,0.664717621965014)(20,0.664717621965014)(20,0.664717621965014)(20,0.664717621965014)(20,0.664717621965014)(20,0.664717621965014)(21,0.681959792736889)(21,0.681959792736889)(21,0.681959792736889)(21,0.681959792736889)(21,0.681959792736889)(21,0.681959792736889)(21,0.681959792736889)(21,0.681959792736889)(21,0.681959792736889)(21,0.681959792736889)(21,0.681959792736889)(21,0.681959792736889)(21,0.681959792736889)(21,0.681959792736889)(21,0.681959792736889)(21,0.681959792736889)(21,0.681959792736889)(21,0.681959792736889)(21,0.681959792736889)(21,0.681959792736889)(21,0.681959792736889)(21,0.681959792736889)(21,0.681959792736889)(21,0.681959792736889)(21,0.681959792736889)(21,0.681959792736889)(21,0.681959792736889)(21,0.681959792736889)(21,0.681959792736889)(21,0.681959792736889)(21,0.681959792736889)(21,0.681959792736889)(21,0.681959792736889)(21,0.681959792736889)(21,0.681959792736889)(21,0.681959792736889)(21,0.681959792736889)(21,0.681959792736889)(21,0.681959792736889)(21,0.681959792736889)(21,0.681959792736889)(21,0.681959792736889)(21,0.681959792736889)(21,0.681959792736889)(21,0.681959792736889)(21,0.681959792736889)(21,0.681959792736889)(21,0.681959792736889)(21,0.681959792736889)(21,0.681959792736889)(21,0.681959792736889)(21,0.681959792736889)(21,0.681959792736889)(21,0.681959792736889)(21,0.681959792736889)(21,0.681959792736889)(22,0.69159436653001)(22,0.69159436653001)(22,0.69159436653001)(22,0.69159436653001)(22,0.69159436653001)(22,0.69159436653001)(22,0.69159436653001)(22,0.69159436653001)(22,0.69159436653001)(22,0.69159436653001)(22,0.69159436653001)(22,0.69159436653001)(22,0.69159436653001)(22,0.69159436653001)(22,0.69159436653001)(22,0.69159436653001)(22,0.69159436653001)(22,0.69159436653001)(22,0.69159436653001)(22,0.69159436653001)(22,0.69159436653001)(22,0.69159436653001)(22,0.69159436653001)(22,0.69159436653001)(22,0.69159436653001)(22,0.69159436653001)(22,0.69159436653001)(22,0.69159436653001)(22,0.69159436653001)(22,0.69159436653001)(22,0.69159436653001)(22,0.69159436653001)(22,0.69159436653001)(22,0.69159436653001)(22,0.69159436653001)(22,0.69159436653001)(22,0.69159436653001)(22,0.69159436653001)(22,0.69159436653001)(22,0.69159436653001)(22,0.69159436653001)(22,0.69159436653001)(22,0.69159436653001)(22,0.69159436653001)(22,0.69159436653001)(22,0.69159436653001)(22,0.69159436653001)(22,0.69159436653001)(22,0.69159436653001)(22,0.69159436653001)(22,0.69159436653001)(23,0.692386232195861)(23,0.692386232195861)(23,0.692386232195861)(23,0.692386232195861)(23,0.692386232195861)(23,0.692386232195861)(23,0.692386232195861)(23,0.692386232195861)(23,0.692386232195861)(23,0.692386232195861)(23,0.692386232195861)(23,0.692386232195861)(23,0.692386232195861)(23,0.692386232195861)(23,0.692386232195861)(23,0.692386232195861)(23,0.692386232195861)(23,0.692386232195861)(23,0.692386232195861)(23,0.692386232195861)(23,0.692386232195861)(23,0.692386232195861)(23,0.692386232195861)(23,0.692386232195861)(23,0.692386232195861)(23,0.692386232195861)(23,0.692386232195861)(23,0.692386232195861)(23,0.692386232195861)(23,0.692386232195861)(23,0.692386232195861)(23,0.692386232195861)(23,0.692386232195861)(23,0.692386232195861)(23,0.692386232195861)(23,0.692386232195861)(23,0.692386232195861)(23,0.692386232195861)(23,0.692386232195861)(23,0.692386232195861)(23,0.692386232195861)(23,0.692386232195861)(23,0.692386232195861)(23,0.692386232195861)(24,0.693233843470702)(24,0.693233843470702)(24,0.693233843470702)(24,0.693233843470702)(24,0.693233843470702)(24,0.693233843470702)(24,0.693233843470702)(24,0.693233843470702)(24,0.693233843470702)(24,0.693233843470702)(24,0.693233843470702)(24,0.693233843470702)(24,0.693233843470702)(24,0.693233843470702)(24,0.693233843470702)(24,0.693233843470702)(24,0.693233843470702)(24,0.693233843470702)(24,0.693233843470702)(24,0.693233843470702)(24,0.693233843470702)(24,0.693233843470702)(24,0.693233843470702)(24,0.693233843470702)(24,0.693233843470702)(24,0.693233843470702)(24,0.693233843470702)(24,0.693233843470702)(24,0.693233843470702)(24,0.693233843470702)(24,0.693233843470702)(24,0.693233843470702)(24,0.693233843470702)(24,0.693233843470702)(24,0.693233843470702)(24,0.693233843470702)(24,0.693233843470702)(24,0.693233843470702)(24,0.693233843470702)(24,0.693233843470702)(24,0.693233843470702)(24,0.693233843470702)(24,0.693233843470702)(24,0.693233843470702)(25,0.709406391904183)(25,0.709406391904183)(25,0.709406391904183)(25,0.709406391904183)(25,0.709406391904183)(25,0.709406391904183)(25,0.709406391904183)(25,0.709406391904183)(25,0.709406391904183)(25,0.709406391904183)(25,0.709406391904183)(25,0.709406391904183)(25,0.709406391904183)(25,0.709406391904183)(25,0.709406391904183)(25,0.709406391904183)(25,0.709406391904183)(25,0.709406391904183)(25,0.709406391904183)(25,0.709406391904183)(25,0.709406391904183)(25,0.709406391904183)(25,0.709406391904183)(25,0.709406391904183)(25,0.709406391904183)(25,0.709406391904183)(25,0.709406391904183)(25,0.709406391904183)(25,0.709406391904183)(25,0.709406391904183)(25,0.709406391904183)(25,0.709406391904183)(25,0.709406391904183)(25,0.709406391904183)(25,0.709406391904183)(25,0.709406391904183)(25,0.709406391904183)(25,0.709406391904183)(25,0.709406391904183)(25,0.709406391904183)(25,0.709406391904183)(25,0.709406391904183)(25,0.709406391904183)(25,0.709406391904183)(25,0.709406391904183)(25,0.709406391904183)(26,0.735220764383329)(26,0.735220764383329)(26,0.735220764383329)(26,0.735220764383329)(26,0.735220764383329)(26,0.735220764383329)(26,0.735220764383329)(26,0.735220764383329)(26,0.735220764383329)(26,0.735220764383329)(26,0.735220764383329)(26,0.735220764383329)(26,0.735220764383329)(26,0.735220764383329)(26,0.735220764383329)(26,0.735220764383329)(26,0.735220764383329)(26,0.735220764383329)(26,0.735220764383329)(26,0.735220764383329)(26,0.735220764383329)(26,0.735220764383329)(26,0.735220764383329)(26,0.735220764383329)(26,0.735220764383329)(26,0.735220764383329)(26,0.735220764383329)(26,0.735220764383329)(26,0.735220764383329)(26,0.735220764383329)(26,0.735220764383329)(26,0.735220764383329)(26,0.735220764383329)(26,0.735220764383329)(26,0.735220764383329)(26,0.735220764383329)(26,0.735220764383329)(26,0.735220764383329)(26,0.735220764383329)(26,0.735220764383329)(26,0.735220764383329)(26,0.735220764383329)(26,0.735220764383329)(26,0.735220764383329)(26,0.735220764383329)(27,0.74466161724681)(27,0.74466161724681)(27,0.74466161724681)(27,0.74466161724681)(27,0.74466161724681)(27,0.74466161724681)(27,0.74466161724681)(27,0.74466161724681)(27,0.74466161724681)(27,0.74466161724681)(27,0.74466161724681)(27,0.74466161724681)(27,0.74466161724681)(27,0.74466161724681)(27,0.74466161724681)(27,0.74466161724681)(27,0.74466161724681)(27,0.74466161724681)(27,0.74466161724681)(27,0.74466161724681)(27,0.74466161724681)(27,0.74466161724681)(27,0.74466161724681)(27,0.74466161724681)(27,0.74466161724681)(27,0.74466161724681)(27,0.74466161724681)(27,0.74466161724681)(27,0.74466161724681)(27,0.74466161724681)(27,0.74466161724681)(27,0.74466161724681)(27,0.74466161724681)(27,0.74466161724681)(27,0.74466161724681)(27,0.74466161724681)(27,0.74466161724681)(27,0.74466161724681)(27,0.74466161724681)(27,0.74466161724681)(27,0.74466161724681)(27,0.74466161724681)(27,0.74466161724681)(27,0.74466161724681)(27,0.74466161724681)(27,0.74466161724681)(27,0.74466161724681)(28,0.741313158867134)(28,0.741313158867134)(28,0.741313158867134)(28,0.741313158867134)(28,0.741313158867134)(28,0.741313158867134)(28,0.741313158867134)(28,0.741313158867134)(28,0.741313158867134)(28,0.741313158867134)(28,0.741313158867134)(28,0.741313158867134)(28,0.741313158867134)(28,0.741313158867134)(28,0.741313158867134)(28,0.741313158867134)(28,0.741313158867134)(28,0.741313158867134)(28,0.741313158867134)(28,0.741313158867134)(28,0.741313158867134)(28,0.741313158867134)(28,0.741313158867134)(28,0.741313158867134)(28,0.741313158867134)(28,0.741313158867134)(28,0.741313158867134)(28,0.741313158867134)(28,0.741313158867134)(28,0.741313158867134)(28,0.741313158867134)(28,0.741313158867134)(28,0.741313158867134)(28,0.741313158867134)(28,0.741313158867134)(28,0.741313158867134)(28,0.741313158867134)(28,0.741313158867134)(28,0.741313158867134)(28,0.741313158867134)(28,0.741313158867134)(28,0.741313158867134)(28,0.741313158867134)(28,0.741313158867134)(28,0.741313158867134)(28,0.741313158867134)(28,0.741313158867134)(28,0.741313158867134)(28,0.741313158867134)(29,0.747991022511293)(29,0.747991022511293)(29,0.747991022511293)(29,0.747991022511293)(29,0.747991022511293)(29,0.747991022511293)(29,0.747991022511293)(29,0.747991022511293)(29,0.747991022511293)(29,0.747991022511293)(29,0.747991022511293)(29,0.747991022511293)(29,0.747991022511293)(29,0.747991022511293)(29,0.747991022511293)(29,0.747991022511293)(29,0.747991022511293)(29,0.747991022511293)(29,0.747991022511293)(29,0.747991022511293)(29,0.747991022511293)(29,0.747991022511293)(29,0.747991022511293)(29,0.747991022511293)(29,0.747991022511293)(29,0.747991022511293)(29,0.747991022511293)(29,0.747991022511293)(29,0.747991022511293)(29,0.747991022511293)(30,0.75456187543932)(30,0.75456187543932)(30,0.75456187543932)(30,0.75456187543932)(30,0.75456187543932)(30,0.75456187543932)(30,0.75456187543932)(30,0.75456187543932)(30,0.75456187543932)(30,0.75456187543932)(30,0.75456187543932)(30,0.75456187543932)(30,0.75456187543932)(30,0.75456187543932)(30,0.75456187543932)(30,0.75456187543932)(30,0.75456187543932)(30,0.75456187543932)(30,0.75456187543932)(30,0.75456187543932)(30,0.75456187543932)(30,0.75456187543932)(30,0.75456187543932)(30,0.75456187543932)(30,0.75456187543932)(30,0.75456187543932)(30,0.75456187543932)(30,0.75456187543932)(30,0.75456187543932)(30,0.75456187543932)(30,0.75456187543932)(30,0.75456187543932)(30,0.75456187543932)(30,0.75456187543932)(30,0.75456187543932)(30,0.75456187543932)(30,0.75456187543932)(30,0.75456187543932)(30,0.75456187543932)(30,0.75456187543932)(30,0.75456187543932)(30,0.75456187543932)(30,0.75456187543932)(30,0.75456187543932)(30,0.75456187543932)(30,0.75456187543932)(30,0.75456187543932)(30,0.75456187543932)(30,0.75456187543932)(30,0.75456187543932)(31,0.766612408711424)(31,0.766612408711424)(31,0.766612408711424)(31,0.766612408711424)(31,0.766612408711424)(31,0.766612408711424)(31,0.766612408711424)(31,0.766612408711424)(31,0.766612408711424)(31,0.766612408711424)(31,0.766612408711424)(31,0.766612408711424)(31,0.766612408711424)(31,0.766612408711424)(31,0.766612408711424)(31,0.766612408711424)(31,0.766612408711424)(31,0.766612408711424)(31,0.766612408711424)(31,0.766612408711424)(31,0.766612408711424)(31,0.766612408711424)(31,0.766612408711424)(31,0.766612408711424)(31,0.766612408711424)(31,0.766612408711424)(31,0.766612408711424)(31,0.766612408711424)(31,0.766612408711424)(31,0.766612408711424)(31,0.766612408711424)(31,0.766612408711424)(31,0.766612408711424)(31,0.766612408711424)(31,0.766612408711424)(31,0.766612408711424)(31,0.766612408711424)(31,0.766612408711424)(31,0.766612408711424)(31,0.766612408711424)(31,0.766612408711424)(31,0.766612408711424)(31,0.766612408711424)(31,0.766612408711424)(31,0.766612408711424)(31,0.766612408711424)(31,0.766612408711424)(32,0.77759816601267)(32,0.77759816601267)(32,0.77759816601267)(32,0.77759816601267)(32,0.77759816601267)(32,0.77759816601267)(32,0.77759816601267)(32,0.77759816601267)(32,0.77759816601267)(32,0.77759816601267)(32,0.77759816601267)(32,0.77759816601267)(32,0.77759816601267)(32,0.77759816601267)(32,0.77759816601267)(32,0.77759816601267)(32,0.77759816601267)(32,0.77759816601267)(32,0.77759816601267)(32,0.77759816601267)(32,0.77759816601267)(32,0.77759816601267)(32,0.77759816601267)(32,0.77759816601267)(32,0.77759816601267)(32,0.77759816601267)(32,0.77759816601267)(32,0.77759816601267)(32,0.77759816601267)(32,0.77759816601267)(32,0.77759816601267)(32,0.77759816601267)(32,0.77759816601267)(32,0.77759816601267)(32,0.77759816601267)(32,0.77759816601267)(32,0.77759816601267)(32,0.77759816601267)(32,0.77759816601267)(32,0.77759816601267)(32,0.77759816601267)(33,0.781523990555022)(33,0.781523990555022)(33,0.781523990555022)(33,0.781523990555022)(33,0.781523990555022)(33,0.781523990555022)(33,0.781523990555022)(33,0.781523990555022)(33,0.781523990555022)(33,0.781523990555022)(33,0.781523990555022)(33,0.781523990555022)(33,0.781523990555022)(33,0.781523990555022)(33,0.781523990555022)(33,0.781523990555022)(33,0.781523990555022)(33,0.781523990555022)(33,0.781523990555022)(33,0.781523990555022)(33,0.781523990555022)(33,0.781523990555022)(33,0.781523990555022)(33,0.781523990555022)(33,0.781523990555022)(33,0.781523990555022)(33,0.781523990555022)(33,0.781523990555022)(33,0.781523990555022)(33,0.781523990555022)(33,0.781523990555022)(33,0.781523990555022)(34,0.784174255772276)(34,0.784174255772276)(34,0.784174255772276)(34,0.784174255772276)(34,0.784174255772276)(34,0.784174255772276)(34,0.784174255772276)(34,0.784174255772276)(34,0.784174255772276)(34,0.784174255772276)(34,0.784174255772276)(34,0.784174255772276)(34,0.784174255772276)(34,0.784174255772276)(34,0.784174255772276)(34,0.784174255772276)(34,0.784174255772276)(34,0.784174255772276)(34,0.784174255772276)(34,0.784174255772276)(34,0.784174255772276)(34,0.784174255772276)(34,0.784174255772276)(34,0.784174255772276)(34,0.784174255772276)(34,0.784174255772276)(34,0.784174255772276)(34,0.784174255772276)(34,0.784174255772276)(34,0.784174255772276)(34,0.784174255772276)(34,0.784174255772276)(34,0.784174255772276)(34,0.784174255772276)(34,0.784174255772276)(34,0.784174255772276)(34,0.784174255772276)(34,0.784174255772276)(34,0.784174255772276)(34,0.784174255772276)(34,0.784174255772276)(34,0.784174255772276)(34,0.784174255772276)(34,0.784174255772276)(34,0.784174255772276)(35,0.788378917354959)(35,0.788378917354959)(35,0.788378917354959)(35,0.788378917354959)(35,0.788378917354959)(35,0.788378917354959)(35,0.788378917354959)(35,0.788378917354959)(35,0.788378917354959)(35,0.788378917354959)(35,0.788378917354959)(35,0.788378917354959)(35,0.788378917354959)(35,0.788378917354959)(35,0.788378917354959)(35,0.788378917354959)(35,0.788378917354959)(35,0.788378917354959)(35,0.788378917354959)(35,0.788378917354959)(35,0.788378917354959)(35,0.788378917354959)(35,0.788378917354959)(35,0.788378917354959)(35,0.788378917354959)(35,0.788378917354959)(35,0.788378917354959)(35,0.788378917354959)(35,0.788378917354959)(35,0.788378917354959)(35,0.788378917354959)(36,0.792408633993642)(36,0.792408633993642)(36,0.792408633993642)(36,0.792408633993642)(36,0.792408633993642)(36,0.792408633993642)(36,0.792408633993642)(36,0.792408633993642)(36,0.792408633993642)(36,0.792408633993642)(36,0.792408633993642)(36,0.792408633993642)(36,0.792408633993642)(36,0.792408633993642)(36,0.792408633993642)(36,0.792408633993642)(36,0.792408633993642)(36,0.792408633993642)(36,0.792408633993642)(36,0.792408633993642)(36,0.792408633993642)(36,0.792408633993642)(36,0.792408633993642)(36,0.792408633993642)(36,0.792408633993642)(36,0.792408633993642)(36,0.792408633993642)(36,0.792408633993642)(37,0.795350542896441)(37,0.795350542896441)(37,0.795350542896441)(37,0.795350542896441)(37,0.795350542896441)(37,0.795350542896441)(37,0.795350542896441)(37,0.795350542896441)(37,0.795350542896441)(37,0.795350542896441)(37,0.795350542896441)(37,0.795350542896441)(37,0.795350542896441)(37,0.795350542896441)(37,0.795350542896441)(37,0.795350542896441)(37,0.795350542896441)(37,0.795350542896441)(37,0.795350542896441)(37,0.795350542896441)(37,0.795350542896441)(37,0.795350542896441)(37,0.795350542896441)(37,0.795350542896441)(37,0.795350542896441)(37,0.795350542896441)(37,0.795350542896441)(37,0.795350542896441)(37,0.795350542896441)(37,0.795350542896441)(37,0.795350542896441)(37,0.795350542896441)(37,0.795350542896441)(37,0.795350542896441)(37,0.795350542896441)(37,0.795350542896441)(37,0.795350542896441)(37,0.795350542896441)(37,0.795350542896441)(37,0.795350542896441)(37,0.795350542896441)(38,0.798843806543084)(38,0.798843806543084)(38,0.798843806543084)(38,0.798843806543084)(38,0.798843806543084)(38,0.798843806543084)(38,0.798843806543084)(38,0.798843806543084)(38,0.798843806543084)(38,0.798843806543084)(38,0.798843806543084)(38,0.798843806543084)(38,0.798843806543084)(38,0.798843806543084)(38,0.798843806543084)(38,0.798843806543084)(38,0.798843806543084)(38,0.798843806543084)(38,0.798843806543084)(38,0.798843806543084)(38,0.798843806543084)(38,0.798843806543084)(38,0.798843806543084)(38,0.798843806543084)(38,0.798843806543084)(38,0.798843806543084)(38,0.798843806543084)(38,0.798843806543084)(38,0.798843806543084)(38,0.798843806543084)(38,0.798843806543084)(38,0.798843806543084)(38,0.798843806543084)(38,0.798843806543084)(38,0.798843806543084)(39,0.801678933543835)(39,0.801678933543835)(39,0.801678933543835)(39,0.801678933543835)(39,0.801678933543835)(39,0.801678933543835)(39,0.801678933543835)(39,0.801678933543835)(39,0.801678933543835)(39,0.801678933543835)(39,0.801678933543835)(39,0.801678933543835)(39,0.801678933543835)(39,0.801678933543835)(39,0.801678933543835)(39,0.801678933543835)(39,0.801678933543835)(39,0.801678933543835)(39,0.801678933543835)(39,0.801678933543835)(39,0.801678933543835)(39,0.801678933543835)(39,0.801678933543835)(39,0.801678933543835)(39,0.801678933543835)(39,0.801678933543835)(39,0.801678933543835)(39,0.801678933543835)(39,0.801678933543835)(39,0.801678933543835)(40,0.803144416350204)(40,0.803144416350204)(40,0.803144416350204)(40,0.803144416350204)(40,0.803144416350204)(40,0.803144416350204)(40,0.803144416350204)(40,0.803144416350204)(40,0.803144416350204)(40,0.803144416350204)(40,0.803144416350204)(40,0.803144416350204)(40,0.803144416350204)(40,0.803144416350204)(40,0.803144416350204)(40,0.803144416350204)(40,0.803144416350204)(40,0.803144416350204)(40,0.803144416350204)(40,0.803144416350204)(40,0.803144416350204)(40,0.803144416350204)(40,0.803144416350204)(40,0.803144416350204)(40,0.803144416350204)(40,0.803144416350204)(40,0.803144416350204)(40,0.803144416350204)(40,0.803144416350204)(40,0.803144416350204)(40,0.803144416350204)(40,0.803144416350204)(41,0.803829870251427)(41,0.803829870251427)(41,0.803829870251427)(41,0.803829870251427)(41,0.803829870251427)(41,0.803829870251427)(41,0.803829870251427)(41,0.803829870251427)(41,0.803829870251427)(41,0.803829870251427)(41,0.803829870251427)(41,0.803829870251427)(41,0.803829870251427)(41,0.803829870251427)(41,0.803829870251427)(41,0.803829870251427)(41,0.803829870251427)(41,0.803829870251427)(41,0.803829870251427)(41,0.803829870251427)(41,0.803829870251427)(41,0.803829870251427)(41,0.803829870251427)(41,0.803829870251427)(41,0.803829870251427)(41,0.803829870251427)(41,0.803829870251427)(41,0.803829870251427)(41,0.803829870251427)(41,0.803829870251427)(41,0.803829870251427)(41,0.803829870251427)(41,0.803829870251427)(41,0.803829870251427)(42,0.80442832917967)(42,0.80442832917967)(42,0.80442832917967)(42,0.80442832917967)(42,0.80442832917967)(42,0.80442832917967)(42,0.80442832917967)(42,0.80442832917967)(42,0.80442832917967)(42,0.80442832917967)(42,0.80442832917967)(42,0.80442832917967)(42,0.80442832917967)(42,0.80442832917967)(42,0.80442832917967)(42,0.80442832917967)(42,0.80442832917967)(42,0.80442832917967)(42,0.80442832917967)(42,0.80442832917967)(42,0.80442832917967)(42,0.80442832917967)(42,0.80442832917967)(42,0.80442832917967)(42,0.80442832917967)(43,0.8057303756705)(43,0.8057303756705)(43,0.8057303756705)(43,0.8057303756705)(43,0.8057303756705)(43,0.8057303756705)(43,0.8057303756705)(43,0.8057303756705)(43,0.8057303756705)(43,0.8057303756705)(43,0.8057303756705)(43,0.8057303756705)(43,0.8057303756705)(43,0.8057303756705)(43,0.8057303756705)(43,0.8057303756705)(43,0.8057303756705)(43,0.8057303756705)(43,0.8057303756705)(43,0.8057303756705)(43,0.8057303756705)(43,0.8057303756705)(43,0.8057303756705)(43,0.8057303756705)(43,0.8057303756705)(43,0.8057303756705)(43,0.8057303756705)(43,0.8057303756705)(43,0.8057303756705)(43,0.8057303756705)(44,0.808721362654209)(44,0.808721362654209)(44,0.808721362654209)(44,0.808721362654209)(44,0.808721362654209)(44,0.808721362654209)(44,0.808721362654209)(44,0.808721362654209)(44,0.808721362654209)(44,0.808721362654209)(44,0.808721362654209)(44,0.808721362654209)(44,0.808721362654209)(44,0.808721362654209)(44,0.808721362654209)(44,0.808721362654209)(44,0.808721362654209)(44,0.808721362654209)(45,0.812839847238627)(45,0.812839847238627)(45,0.812839847238627)(45,0.812839847238627)(45,0.812839847238627)(45,0.812839847238627)(45,0.812839847238627)(45,0.812839847238627)(45,0.812839847238627)(45,0.812839847238627)(45,0.812839847238627)(45,0.812839847238627)(45,0.812839847238627)(45,0.812839847238627)(45,0.812839847238627)(45,0.812839847238627)(45,0.812839847238627)(45,0.812839847238627)(45,0.812839847238627)(45,0.812839847238627)(45,0.812839847238627)(45,0.812839847238627)(45,0.812839847238627)(45,0.812839847238627)(45,0.812839847238627)(45,0.812839847238627)(46,0.81604204531106)(46,0.81604204531106)(46,0.81604204531106)(46,0.81604204531106)(46,0.81604204531106)(46,0.81604204531106)(46,0.81604204531106)(46,0.81604204531106)(46,0.81604204531106)(46,0.81604204531106)(46,0.81604204531106)(46,0.81604204531106)(46,0.81604204531106)(46,0.81604204531106)(46,0.81604204531106)(46,0.81604204531106)(46,0.81604204531106)(46,0.81604204531106)(46,0.81604204531106)(47,0.818424730783829)(47,0.818424730783829)(47,0.818424730783829)(47,0.818424730783829)(47,0.818424730783829)(47,0.818424730783829)(47,0.818424730783829)(47,0.818424730783829)(47,0.818424730783829)(47,0.818424730783829)(47,0.818424730783829)(47,0.818424730783829)(47,0.818424730783829)(47,0.818424730783829)(47,0.818424730783829)(47,0.818424730783829)(47,0.818424730783829)(47,0.818424730783829)(47,0.818424730783829)(47,0.818424730783829)(47,0.818424730783829)(47,0.818424730783829)(47,0.818424730783829)(47,0.818424730783829)(47,0.818424730783829)(47,0.818424730783829)(47,0.818424730783829)(47,0.818424730783829)(47,0.818424730783829)(47,0.818424730783829)(47,0.818424730783829)(47,0.818424730783829)(47,0.818424730783829)(48,0.820481755133678)(48,0.820481755133678)(48,0.820481755133678)(48,0.820481755133678)(48,0.820481755133678)(48,0.820481755133678)(48,0.820481755133678)(48,0.820481755133678)(48,0.820481755133678)(48,0.820481755133678)(48,0.820481755133678)(48,0.820481755133678)(48,0.820481755133678)(48,0.820481755133678)(48,0.820481755133678)(48,0.820481755133678)(48,0.820481755133678)(48,0.820481755133678)(48,0.820481755133678)(48,0.820481755133678)(48,0.820481755133678)(48,0.820481755133678)(49,0.823217179919157)(49,0.823217179919157)(49,0.823217179919157)(49,0.823217179919157)(49,0.823217179919157)(49,0.823217179919157)(49,0.823217179919157)(49,0.823217179919157)(49,0.823217179919157)(49,0.823217179919157)(49,0.823217179919157)(49,0.823217179919157)(49,0.823217179919157)(49,0.823217179919157)(49,0.823217179919157)(49,0.823217179919157)(49,0.823217179919157)(49,0.823217179919157)(49,0.823217179919157)(50,0.82625159519024)(50,0.82625159519024)(50,0.82625159519024)(50,0.82625159519024)(50,0.82625159519024)(50,0.82625159519024)(50,0.82625159519024)(50,0.82625159519024)(50,0.82625159519024)(50,0.82625159519024)(50,0.82625159519024)(50,0.82625159519024)(50,0.82625159519024)(50,0.82625159519024)(50,0.82625159519024)(50,0.82625159519024)(50,0.82625159519024)(50,0.82625159519024)(50,0.82625159519024)(50,0.82625159519024)(50,0.82625159519024)(50,0.82625159519024)(50,0.82625159519024)(50,0.82625159519024)(50,0.82625159519024)(50,0.82625159519024)(50,0.82625159519024)(50,0.82625159519024)(50,0.82625159519024)(50,0.82625159519024)(51,0.829002119033889)(51,0.829002119033889)(51,0.829002119033889)(51,0.829002119033889)(51,0.829002119033889)(51,0.829002119033889)(51,0.829002119033889)(51,0.829002119033889)(51,0.829002119033889)(51,0.829002119033889)(51,0.829002119033889)(51,0.829002119033889)(51,0.829002119033889)(51,0.829002119033889)(51,0.829002119033889)(51,0.829002119033889)(51,0.829002119033889)(51,0.829002119033889)(51,0.829002119033889)(51,0.829002119033889)(51,0.829002119033889)(51,0.829002119033889)(51,0.829002119033889)(51,0.829002119033889)(52,0.831329859929671)(52,0.831329859929671)(52,0.831329859929671)(52,0.831329859929671)(52,0.831329859929671)(52,0.831329859929671)(52,0.831329859929671)(52,0.831329859929671)(52,0.831329859929671)(52,0.831329859929671)(52,0.831329859929671)(52,0.831329859929671)(52,0.831329859929671)(52,0.831329859929671)(52,0.831329859929671)(52,0.831329859929671)(52,0.831329859929671)(52,0.831329859929671)(52,0.831329859929671)(52,0.831329859929671)(52,0.831329859929671)(52,0.831329859929671)(52,0.831329859929671)(52,0.831329859929671)(53,0.833257141200067)(53,0.833257141200067)(53,0.833257141200067)(53,0.833257141200067)(53,0.833257141200067)(53,0.833257141200067)(53,0.833257141200067)(53,0.833257141200067)(53,0.833257141200067)(53,0.833257141200067)(53,0.833257141200067)(53,0.833257141200067)(53,0.833257141200067)(53,0.833257141200067)(53,0.833257141200067)(53,0.833257141200067)(53,0.833257141200067)(53,0.833257141200067)(53,0.833257141200067)(53,0.833257141200067)(54,0.8347663212136)(54,0.8347663212136)(54,0.8347663212136)(54,0.8347663212136)(54,0.8347663212136)(54,0.8347663212136)(54,0.8347663212136)(54,0.8347663212136)(54,0.8347663212136)(54,0.8347663212136)(54,0.8347663212136)(54,0.8347663212136)(54,0.8347663212136)(54,0.8347663212136)(54,0.8347663212136)(54,0.8347663212136)(54,0.8347663212136)(54,0.8347663212136)(54,0.8347663212136)(54,0.8347663212136)(54,0.8347663212136)(55,0.836501874697888)(55,0.836501874697888)(55,0.836501874697888)(55,0.836501874697888)(55,0.836501874697888)(55,0.836501874697888)(55,0.836501874697888)(55,0.836501874697888)(55,0.836501874697888)(55,0.836501874697888)(55,0.836501874697888)(55,0.836501874697888)(55,0.836501874697888)(55,0.836501874697888)(55,0.836501874697888)(55,0.836501874697888)(55,0.836501874697888)(55,0.836501874697888)(56,0.83843008927367)(56,0.83843008927367)(56,0.83843008927367)(56,0.83843008927367)(56,0.83843008927367)(56,0.83843008927367)(56,0.83843008927367)(56,0.83843008927367)(56,0.83843008927367)(56,0.83843008927367)(56,0.83843008927367)(56,0.83843008927367)(56,0.83843008927367)(57,0.840305038829841)(57,0.840305038829841)(57,0.840305038829841)(57,0.840305038829841)(57,0.840305038829841)(57,0.840305038829841)(57,0.840305038829841)(57,0.840305038829841)(57,0.840305038829841)(57,0.840305038829841)(57,0.840305038829841)(57,0.840305038829841)(57,0.840305038829841)(57,0.840305038829841)(57,0.840305038829841)(57,0.840305038829841)(57,0.840305038829841)(57,0.840305038829841)(57,0.840305038829841)(57,0.840305038829841)(58,0.842011018090292)(58,0.842011018090292)(58,0.842011018090292)(58,0.842011018090292)(58,0.842011018090292)(58,0.842011018090292)(58,0.842011018090292)(58,0.842011018090292)(58,0.842011018090292)(58,0.842011018090292)(58,0.842011018090292)(58,0.842011018090292)(58,0.842011018090292)(59,0.842902785153524)(59,0.842902785153524)(59,0.842902785153524)(59,0.842902785153524)(59,0.842902785153524)(59,0.842902785153524)(59,0.842902785153524)(59,0.842902785153524)(59,0.842902785153524)(59,0.842902785153524)(59,0.842902785153524)(59,0.842902785153524)(59,0.842902785153524)(60,0.843702198635364)(60,0.843702198635364)(60,0.843702198635364)(60,0.843702198635364)(60,0.843702198635364)(60,0.843702198635364)(60,0.843702198635364)(60,0.843702198635364)(60,0.843702198635364)(60,0.843702198635364)(60,0.843702198635364)(60,0.843702198635364)(60,0.843702198635364)(60,0.843702198635364)(60,0.843702198635364)(60,0.843702198635364)(60,0.843702198635364)(60,0.843702198635364)(60,0.843702198635364)(61,0.845226640196423)(61,0.845226640196423)(61,0.845226640196423)(61,0.845226640196423)(61,0.845226640196423)(61,0.845226640196423)(61,0.845226640196423)(61,0.845226640196423)(61,0.845226640196423)(61,0.845226640196423)(61,0.845226640196423)(61,0.845226640196423)(61,0.845226640196423)(61,0.845226640196423)(61,0.845226640196423)(62,0.846847553268688)(62,0.846847553268688)(62,0.846847553268688)(62,0.846847553268688)(62,0.846847553268688)(62,0.846847553268688)(62,0.846847553268688)(62,0.846847553268688)(62,0.846847553268688)(62,0.846847553268688)(62,0.846847553268688)(62,0.846847553268688)(62,0.846847553268688)(62,0.846847553268688)(62,0.846847553268688)(62,0.846847553268688)(62,0.846847553268688)(62,0.846847553268688)(62,0.846847553268688)(62,0.846847553268688)(62,0.846847553268688)(62,0.846847553268688)(62,0.846847553268688)(62,0.846847553268688)(63,0.848801731742995)(63,0.848801731742995)(63,0.848801731742995)(63,0.848801731742995)(63,0.848801731742995)(63,0.848801731742995)(63,0.848801731742995)(63,0.848801731742995)(63,0.848801731742995)(63,0.848801731742995)(63,0.848801731742995)(63,0.848801731742995)(63,0.848801731742995)(63,0.848801731742995)(63,0.848801731742995)(63,0.848801731742995)(63,0.848801731742995)(63,0.848801731742995)(64,0.850869353446309)(64,0.850869353446309)(64,0.850869353446309)(64,0.850869353446309)(64,0.850869353446309)(64,0.850869353446309)(64,0.850869353446309)(64,0.850869353446309)(64,0.850869353446309)(64,0.850869353446309)(64,0.850869353446309)(64,0.850869353446309)(64,0.850869353446309)(64,0.850869353446309)(64,0.850869353446309)(65,0.85229762254709)(65,0.85229762254709)(65,0.85229762254709)(65,0.85229762254709)(65,0.85229762254709)(65,0.85229762254709)(65,0.85229762254709)(65,0.85229762254709)(65,0.85229762254709)(65,0.85229762254709)(65,0.85229762254709)(65,0.85229762254709)(65,0.85229762254709)(65,0.85229762254709)(65,0.85229762254709)(65,0.85229762254709)(65,0.85229762254709)(66,0.853801137284702)(66,0.853801137284702)(66,0.853801137284702)(66,0.853801137284702)(66,0.853801137284702)(66,0.853801137284702)(66,0.853801137284702)(66,0.853801137284702)(66,0.853801137284702)(67,0.854824972133955)(67,0.854824972133955)(67,0.854824972133955)(67,0.854824972133955)(67,0.854824972133955)(67,0.854824972133955)(67,0.854824972133955)(67,0.854824972133955)(67,0.854824972133955)(67,0.854824972133955)(67,0.854824972133955)(67,0.854824972133955)(68,0.855669382763225)(68,0.855669382763225)(68,0.855669382763225)(68,0.855669382763225)(68,0.855669382763225)(68,0.855669382763225)(68,0.855669382763225)(68,0.855669382763225)(68,0.855669382763225)(68,0.855669382763225)(68,0.855669382763225)(68,0.855669382763225)(68,0.855669382763225)(68,0.855669382763225)(68,0.855669382763225)(68,0.855669382763225)(68,0.855669382763225)(68,0.855669382763225)(68,0.855669382763225)(68,0.855669382763225)(68,0.855669382763225)(69,0.856244783694022)(69,0.856244783694022)(69,0.856244783694022)(69,0.856244783694022)(69,0.856244783694022)(69,0.856244783694022)(69,0.856244783694022)(69,0.856244783694022)(69,0.856244783694022)(70,0.856736316181424)(70,0.856736316181424)(70,0.856736316181424)(70,0.856736316181424)(70,0.856736316181424)(70,0.856736316181424)(70,0.856736316181424)(70,0.856736316181424)(70,0.856736316181424)(70,0.856736316181424)(70,0.856736316181424)(70,0.856736316181424)(71,0.857306902409606)(71,0.857306902409606)(71,0.857306902409606)(71,0.857306902409606)(71,0.857306902409606)(71,0.857306902409606)(71,0.857306902409606)(71,0.857306902409606)(71,0.857306902409606)(71,0.857306902409606)(71,0.857306902409606)(71,0.857306902409606)(71,0.857306902409606)(71,0.857306902409606)(71,0.857306902409606)(71,0.857306902409606)(71,0.857306902409606)(71,0.857306902409606)(71,0.857306902409606)(72,0.858018887600214)(72,0.858018887600214)(72,0.858018887600214)(72,0.858018887600214)(72,0.858018887600214)(72,0.858018887600214)(72,0.858018887600214)(72,0.858018887600214)(72,0.858018887600214)(72,0.858018887600214)(73,0.858919987628094)(73,0.858919987628094)(73,0.858919987628094)(73,0.858919987628094)(73,0.858919987628094)(73,0.858919987628094)(73,0.858919987628094)(73,0.858919987628094)(73,0.858919987628094)(73,0.858919987628094)(73,0.858919987628094)(73,0.858919987628094)(73,0.858919987628094)(73,0.858919987628094)(74,0.860146484105036)(74,0.860146484105036)(74,0.860146484105036)(74,0.860146484105036)(74,0.860146484105036)(74,0.860146484105036)(74,0.860146484105036)(74,0.860146484105036)(74,0.860146484105036)(74,0.860146484105036)(74,0.860146484105036)(74,0.860146484105036)(74,0.860146484105036)(74,0.860146484105036)(74,0.860146484105036)(74,0.860146484105036)(75,0.861251507808856)(75,0.861251507808856)(75,0.861251507808856)(75,0.861251507808856)(75,0.861251507808856)(75,0.861251507808856)(75,0.861251507808856)(75,0.861251507808856)(75,0.861251507808856)(75,0.861251507808856)(76,0.862372887669196)(76,0.862372887669196)(76,0.862372887669196)(76,0.862372887669196)(76,0.862372887669196)(76,0.862372887669196)(76,0.862372887669196)(76,0.862372887669196)(76,0.862372887669196)(76,0.862372887669196)(76,0.862372887669196)(76,0.862372887669196)(76,0.862372887669196)(76,0.862372887669196)(76,0.862372887669196)(76,0.862372887669196)(77,0.863451524723784)(77,0.863451524723784)(77,0.863451524723784)(77,0.863451524723784)(77,0.863451524723784)(77,0.863451524723784)(77,0.863451524723784)(77,0.863451524723784)(78,0.864386264260367)(78,0.864386264260367)(78,0.864386264260367)(78,0.864386264260367)(78,0.864386264260367)(78,0.864386264260367)(78,0.864386264260367)(78,0.864386264260367)(78,0.864386264260367)(78,0.864386264260367)(78,0.864386264260367)(78,0.864386264260367)(78,0.864386264260367)(78,0.864386264260367)(79,0.865216942042594)(79,0.865216942042594)(79,0.865216942042594)(79,0.865216942042594)(79,0.865216942042594)(79,0.865216942042594)(79,0.865216942042594)(79,0.865216942042594)(79,0.865216942042594)(79,0.865216942042594)(80,0.866034467938662)(80,0.866034467938662)(80,0.866034467938662)(80,0.866034467938662)(80,0.866034467938662)(80,0.866034467938662)(80,0.866034467938662)(80,0.866034467938662)(80,0.866034467938662)(80,0.866034467938662)(81,0.866906591199006)(81,0.866906591199006)(81,0.866906591199006)(81,0.866906591199006)(81,0.866906591199006)(81,0.866906591199006)(81,0.866906591199006)(81,0.866906591199006)(81,0.866906591199006)(81,0.866906591199006)(81,0.866906591199006)(81,0.866906591199006)(81,0.866906591199006)(81,0.866906591199006)(81,0.866906591199006)(81,0.866906591199006)(82,0.867938511345117)(82,0.867938511345117)(82,0.867938511345117)(82,0.867938511345117)(82,0.867938511345117)(82,0.867938511345117)(82,0.867938511345117)(82,0.867938511345117)(82,0.867938511345117)(83,0.868964177720208)(83,0.868964177720208)(83,0.868964177720208)(83,0.868964177720208)(84,0.870115085683796)(84,0.870115085683796)(84,0.870115085683796)(84,0.870115085683796)(84,0.870115085683796)(84,0.870115085683796)(84,0.870115085683796)(84,0.870115085683796)(84,0.870115085683796)(84,0.870115085683796)(85,0.871263585094177)(85,0.871263585094177)(85,0.871263585094177)(85,0.871263585094177)(85,0.871263585094177)(85,0.871263585094177)(85,0.871263585094177)(86,0.872375869179939)(86,0.872375869179939)(86,0.872375869179939)(86,0.872375869179939)(86,0.872375869179939)(86,0.872375869179939)(86,0.872375869179939)(86,0.872375869179939)(86,0.872375869179939)(86,0.872375869179939)(86,0.872375869179939)(86,0.872375869179939)(86,0.872375869179939)(86,0.872375869179939)(87,0.873470104234195)(87,0.873470104234195)(87,0.873470104234195)(88,0.874765589139725)(88,0.874765589139725)(88,0.874765589139725)(88,0.874765589139725)(88,0.874765589139725)(88,0.874765589139725)(88,0.874765589139725)(88,0.874765589139725)(88,0.874765589139725)(88,0.874765589139725)(88,0.874765589139725)(89,0.875918063104193)(89,0.875918063104193)(89,0.875918063104193)(89,0.875918063104193)(89,0.875918063104193)(89,0.875918063104193)(90,0.87711695983655)(90,0.87711695983655)(90,0.87711695983655)(90,0.87711695983655)(90,0.87711695983655)(90,0.87711695983655)(90,0.87711695983655)(90,0.87711695983655)(90,0.87711695983655)(90,0.87711695983655)(90,0.87711695983655)(90,0.87711695983655)(90,0.87711695983655)(91,0.87815603020879)(91,0.87815603020879)(91,0.87815603020879)(91,0.87815603020879)(91,0.87815603020879)(91,0.87815603020879)(91,0.87815603020879)(91,0.87815603020879)(91,0.87815603020879)(91,0.87815603020879)(91,0.87815603020879)(92,0.879166842958229)(92,0.879166842958229)(92,0.879166842958229)(92,0.879166842958229)(92,0.879166842958229)(93,0.880149304947892)(93,0.880149304947892)(93,0.880149304947892)(93,0.880149304947892)(93,0.880149304947892)(93,0.880149304947892)(94,0.881024540013573)(94,0.881024540013573)(94,0.881024540013573)(94,0.881024540013573)(94,0.881024540013573)(94,0.881024540013573)(94,0.881024540013573)(94,0.881024540013573)(94,0.881024540013573)(94,0.881024540013573)(94,0.881024540013573)(94,0.881024540013573)(94,0.881024540013573)(94,0.881024540013573)(95,0.881827235416656)(95,0.881827235416656)(95,0.881827235416656)(95,0.881827235416656)(95,0.881827235416656)(95,0.881827235416656)(95,0.881827235416656)(95,0.881827235416656)(96,0.88245427481638)(96,0.88245427481638)(96,0.88245427481638)(96,0.88245427481638)(96,0.88245427481638)(96,0.88245427481638)(96,0.88245427481638)(96,0.88245427481638)(96,0.88245427481638)(96,0.88245427481638)(97,0.883095742961618)(97,0.883095742961618)(97,0.883095742961618)(97,0.883095742961618)(97,0.883095742961618)(97,0.883095742961618)(97,0.883095742961618)(97,0.883095742961618)(97,0.883095742961618)(97,0.883095742961618)(98,0.883696734759346)(98,0.883696734759346)(98,0.883696734759346)(98,0.883696734759346)(98,0.883696734759346)(98,0.883696734759346)(98,0.883696734759346)(98,0.883696734759346)(98,0.883696734759346)(98,0.883696734759346)(98,0.883696734759346)(99,0.88425078682894)(99,0.88425078682894)(99,0.88425078682894)(99,0.88425078682894)(99,0.88425078682894)(99,0.88425078682894)(99,0.88425078682894)(100,0.884548181783039)(100,0.884548181783039)(100,0.884548181783039)(100,0.884548181783039)(100,0.884548181783039)(100,0.884548181783039)(100,0.884548181783039)(101,0.885010330910515)(101,0.885010330910515)(101,0.885010330910515)(101,0.885010330910515)(101,0.885010330910515)(101,0.885010330910515)(101,0.885010330910515)(102,0.885426127861043)(102,0.885426127861043)(102,0.885426127861043)(102,0.885426127861043)(102,0.885426127861043)(102,0.885426127861043)(103,0.885769793922489)(103,0.885769793922489)(103,0.885769793922489)(103,0.885769793922489)(103,0.885769793922489)(103,0.885769793922489)(103,0.885769793922489)(103,0.885769793922489)(103,0.885769793922489)(104,0.886032314254541)(104,0.886032314254541)(104,0.886032314254541)(104,0.886032314254541)(104,0.886032314254541)(104,0.886032314254541)(104,0.886032314254541)(104,0.886032314254541)(105,0.88629880218148)(105,0.88629880218148)(105,0.88629880218148)(105,0.88629880218148)(105,0.88629880218148)(105,0.88629880218148)(105,0.88629880218148)(105,0.88629880218148)(105,0.88629880218148)(106,0.886517673825304)(106,0.886517673825304)(106,0.886517673825304)(106,0.886517673825304)(106,0.886517673825304)(106,0.886517673825304)(106,0.886517673825304)(107,0.886705004501407)(107,0.886705004501407)(107,0.886705004501407)(107,0.886705004501407)(107,0.886705004501407)(107,0.886705004501407)(107,0.886705004501407)(107,0.886705004501407)(107,0.886705004501407)(107,0.886705004501407)(108,0.886897695798863)(108,0.886897695798863)(109,0.887157082672705)(109,0.887157082672705)(109,0.887157082672705)(109,0.887157082672705)(109,0.887157082672705)(109,0.887157082672705)(109,0.887157082672705)(109,0.887157082672705)(109,0.887157082672705)(110,0.887783728467878)(110,0.887783728467878)(110,0.887783728467878)(110,0.887783728467878)(110,0.887783728467878)(110,0.887783728467878)(110,0.887783728467878)(111,0.888286966749207)(111,0.888286966749207)(111,0.888286966749207)(112,0.888886911520906)(112,0.888886911520906)(112,0.888886911520906)(112,0.888886911520906)(112,0.888886911520906)(112,0.888886911520906)(113,0.889574965531872)(113,0.889574965531872)(113,0.889574965531872)(113,0.889574965531872)(113,0.889574965531872)(113,0.889574965531872)(113,0.889574965531872)(114,0.890344667610861)(114,0.890344667610861)(114,0.890344667610861)(114,0.890344667610861)(114,0.890344667610861)(114,0.890344667610861)(114,0.890344667610861)(114,0.890344667610861)(114,0.890344667610861)(114,0.890344667610861)(114,0.890344667610861)(114,0.890344667610861)(114,0.890344667610861)(115,0.891178560933554)(115,0.891178560933554)(115,0.891178560933554)(116,0.892102859021432)(116,0.892102859021432)(116,0.892102859021432)(116,0.892102859021432)(116,0.892102859021432)(116,0.892102859021432)(116,0.892102859021432)(116,0.892102859021432)(116,0.892102859021432)(117,0.892952229672649)(117,0.892952229672649)(117,0.892952229672649)(117,0.892952229672649)(117,0.892952229672649)(118,0.893788445121733)(118,0.893788445121733)(118,0.893788445121733)(118,0.893788445121733)(118,0.893788445121733)(118,0.893788445121733)(118,0.893788445121733)(118,0.893788445121733)(119,0.8946594856061)(119,0.8946594856061)(119,0.8946594856061)(119,0.8946594856061)(119,0.8946594856061)(120,0.895567455089183)(120,0.895567455089183)(120,0.895567455089183)(120,0.895567455089183)(120,0.895567455089183)(120,0.895567455089183)(120,0.895567455089183)(120,0.895567455089183)(120,0.895567455089183)(120,0.895567455089183)(120,0.895567455089183)(120,0.895567455089183)(120,0.895567455089183)(120,0.895567455089183)(121,0.896512697148473)(121,0.896512697148473)(121,0.896512697148473)(121,0.896512697148473)(121,0.896512697148473)(121,0.896512697148473)(121,0.896512697148473)(121,0.896512697148473)(121,0.896512697148473)(122,0.897495209273577)(122,0.897495209273577)(122,0.897495209273577)(122,0.897495209273577)(122,0.897495209273577)(122,0.897495209273577)(122,0.897495209273577)(123,0.898461048864739)(123,0.898461048864739)(123,0.898461048864739)(123,0.898461048864739)(123,0.898461048864739)(123,0.898461048864739)(123,0.898461048864739)(123,0.898461048864739)(123,0.898461048864739)(124,0.899405460019621)(124,0.899405460019621)(124,0.899405460019621)(124,0.899405460019621)(124,0.899405460019621)(124,0.899405460019621)(125,0.900369640024093)(125,0.900369640024093)(125,0.900369640024093)(125,0.900369640024093)(125,0.900369640024093)(125,0.900369640024093)(125,0.900369640024093)(125,0.900369640024093)(126,0.901302491058131)(126,0.901302491058131)(126,0.901302491058131)(126,0.901302491058131)(126,0.901302491058131)(126,0.901302491058131)(126,0.901302491058131)(127,0.902186066874667)(127,0.902186066874667)(128,0.903003775901638)(128,0.903003775901638)(128,0.903003775901638)(129,0.903617772793256)(129,0.903617772793256)(129,0.903617772793256)(129,0.903617772793256)(129,0.903617772793256)(129,0.903617772793256)(129,0.903617772793256)(129,0.903617772793256)(129,0.903617772793256)(130,0.904248583082145)(130,0.904248583082145)(130,0.904248583082145)(130,0.904248583082145)(130,0.904248583082145)(130,0.904248583082145)(130,0.904248583082145)(130,0.904248583082145)(130,0.904248583082145)(130,0.904248583082145)(131,0.904788273064382)(131,0.904788273064382)(131,0.904788273064382)(131,0.904788273064382)(131,0.904788273064382)(131,0.904788273064382)(132,0.905241658172107)(132,0.905241658172107)(132,0.905241658172107)(132,0.905241658172107)(132,0.905241658172107)(132,0.905241658172107)(132,0.905241658172107)(132,0.905241658172107)(132,0.905241658172107)(132,0.905241658172107)(132,0.905241658172107)(132,0.905241658172107)(133,0.905617236178036)(133,0.905617236178036)(133,0.905617236178036)(133,0.905617236178036)(133,0.905617236178036)(133,0.905617236178036)(133,0.905617236178036)(133,0.905617236178036)(133,0.905617236178036)(134,0.905807896647698)(134,0.905807896647698)(134,0.905807896647698)(134,0.905807896647698)(134,0.905807896647698)(135,0.906079523665759)(135,0.906079523665759)(136,0.906303045632532)(136,0.906303045632532)(136,0.906303045632532)(136,0.906303045632532)(137,0.906454417995213)(137,0.906454417995213)(137,0.906454417995213)(137,0.906454417995213)(137,0.906454417995213)(137,0.906454417995213)(137,0.906454417995213)(137,0.906454417995213)(137,0.906454417995213)(137,0.906454417995213)(138,0.906631985471256)(138,0.906631985471256)(138,0.906631985471256)(138,0.906631985471256)(138,0.906631985471256)(138,0.906631985471256)(138,0.906631985471256)(138,0.906631985471256)(139,0.906776636442685)(139,0.906776636442685)(139,0.906776636442685)(139,0.906776636442685)(139,0.906776636442685)(139,0.906776636442685)(139,0.906776636442685)(139,0.906776636442685)(140,0.906897043796306)(141,0.907061352929401)(141,0.907061352929401)(141,0.907061352929401)(141,0.907061352929401)(141,0.907061352929401)(141,0.907061352929401)(141,0.907061352929401)(141,0.907061352929401)(142,0.907175595583422)(142,0.907175595583422)(142,0.907175595583422)(143,0.907286516589718)(143,0.907286516589718)(143,0.907286516589718)(143,0.907286516589718)(143,0.907286516589718)(144,0.907395999888178)(144,0.907395999888178)(144,0.907395999888178)(144,0.907395999888178)(145,0.907494813442418)(145,0.907494813442418)(145,0.907494813442418)(145,0.907494813442418)(146,0.907484436035112)(146,0.907484436035112)(146,0.907484436035112)(146,0.907484436035112)(146,0.907484436035112)(146,0.907484436035112)(146,0.907484436035112)(146,0.907484436035112)(147,0.907642245540492)(147,0.907642245540492)(148,0.907685200940031)(148,0.907685200940031)(148,0.907685200940031)(148,0.907685200940031)(148,0.907685200940031)(148,0.907685200940031)(148,0.907685200940031)(148,0.907685200940031)(149,0.907738156107916)(149,0.907738156107916)(149,0.907738156107916)(150,0.90775325923015)(150,0.90775325923015)(150,0.90775325923015)(150,0.90775325923015)(150,0.90775325923015)(151,0.907734883477113)(151,0.907734883477113)(152,0.907713328264532)(152,0.907713328264532)(152,0.907713328264532)(153,0.907693056293033)(153,0.907693056293033)(154,0.907681994991709)(154,0.907681994991709)(154,0.907681994991709)(154,0.907681994991709)(154,0.907681994991709)(154,0.907681994991709)(155,0.907695657140671)(155,0.907695657140671)(155,0.907695657140671)(155,0.907695657140671)(156,0.907715960999977)(156,0.907715960999977)(156,0.907715960999977)(156,0.907715960999977)(156,0.907715960999977)(156,0.907715960999977)(156,0.907715960999977)(156,0.907715960999977)(156,0.907715960999977)(156,0.907715960999977)(157,0.907761613154578)(157,0.907761613154578)(157,0.907761613154578)(157,0.907761613154578)(157,0.907761613154578)(157,0.907761613154578)(158,0.907861331972772)(158,0.907861331972772)(158,0.907861331972772)(158,0.907861331972772)(158,0.907861331972772)(158,0.907861331972772)(158,0.907861331972772)(158,0.907861331972772)(158,0.907861331972772)(159,0.908011023348507)(159,0.908011023348507)(159,0.908011023348507)(159,0.908011023348507)(159,0.908011023348507)(160,0.908156530215909)(160,0.908156530215909)(160,0.908156530215909)(160,0.908156530215909)(160,0.908156530215909)(160,0.908156530215909)(160,0.908156530215909)(161,0.908333975740979)(161,0.908333975740979)(161,0.908333975740979)(161,0.908333975740979)(161,0.908333975740979)(161,0.908333975740979)(162,0.908546046789197)(162,0.908546046789197)(162,0.908546046789197)(162,0.908546046789197)(162,0.908546046789197)(163,0.908868594655885)(163,0.908868594655885)(163,0.908868594655885)(163,0.908868594655885)(163,0.908868594655885)(163,0.908868594655885)(163,0.908868594655885)(164,0.90915005747973)(164,0.90915005747973)(164,0.90915005747973)(165,0.909454046203418)(166,0.90984683569755)(166,0.90984683569755)(166,0.90984683569755)(166,0.90984683569755)(166,0.90984683569755)(166,0.90984683569755)(167,0.910174702990907)(167,0.910174702990907)(167,0.910174702990907)(167,0.910174702990907)(168,0.910557116951432)(168,0.910557116951432)(168,0.910557116951432)(169,0.91089832006316)(169,0.91089832006316)(169,0.91089832006316)(169,0.91089832006316)(170,0.911235258854136)(170,0.911235258854136)(170,0.911235258854136)(170,0.911235258854136)(171,0.911617307473616)(171,0.911617307473616)(171,0.911617307473616)(171,0.911617307473616)(171,0.911617307473616)(171,0.911617307473616)(171,0.911617307473616)(172,0.911992454665332)(172,0.911992454665332)(172,0.911992454665332)(172,0.911992454665332)(172,0.911992454665332)(172,0.911992454665332)(173,0.912347590301615)(173,0.912347590301615)(174,0.912715246558631)(174,0.912715246558631)(174,0.912715246558631)(174,0.912715246558631)(174,0.912715246558631)(174,0.912715246558631)(175,0.913081150472102)(175,0.913081150472102)(175,0.913081150472102)(175,0.913081150472102)(176,0.913382632910585)(176,0.913382632910585)(176,0.913382632910585)(176,0.913382632910585)(176,0.913382632910585)(176,0.913382632910585)(177,0.91372689029414)(177,0.91372689029414)(177,0.91372689029414)(177,0.91372689029414)(177,0.91372689029414)(177,0.91372689029414)(177,0.91372689029414)(178,0.913999103572663)(178,0.913999103572663)(178,0.913999103572663)(178,0.913999103572663)(179,0.914317348500506)(179,0.914317348500506)(179,0.914317348500506)(179,0.914317348500506)(179,0.914317348500506)(179,0.914317348500506)(179,0.914317348500506)(179,0.914317348500506)(179,0.914317348500506)(180,0.914619319139855)(180,0.914619319139855)(180,0.914619319139855)(181,0.914902556182025)(181,0.914902556182025)(181,0.914902556182025)(181,0.914902556182025)(181,0.914902556182025)(182,0.915166794456473)(182,0.915166794456473)(182,0.915166794456473)(183,0.915385332665607)(183,0.915385332665607)(183,0.915385332665607)(183,0.915385332665607)(183,0.915385332665607)(184,0.91562595708624)(184,0.91562595708624)(185,0.915827279076721)(186,0.916048754554075)(186,0.916048754554075)(186,0.916048754554075)(187,0.916242923256125)(187,0.916242923256125)(187,0.916242923256125)(188,0.916458389493785)(188,0.916458389493785)(188,0.916458389493785)(188,0.916458389493785)(189,0.916672689017806)(189,0.916672689017806)(189,0.916672689017806)(189,0.916672689017806)(189,0.916672689017806)(189,0.916672689017806)(189,0.916672689017806)(190,0.916889106136478)(190,0.916889106136478)(191,0.917109578096285)(191,0.917109578096285)(191,0.917109578096285)(192,0.917334696007648)(192,0.917334696007648)(192,0.917334696007648)(192,0.917334696007648)(192,0.917334696007648)(192,0.917334696007648)(193,0.917542301696305)(193,0.917542301696305)(193,0.917542301696305)(193,0.917542301696305)(193,0.917542301696305)(193,0.917542301696305)(194,0.917771606718821)(194,0.917771606718821)(195,0.918002313045923)(195,0.918002313045923)(196,0.918232118289294)(196,0.918232118289294)(197,0.918458605850243)(197,0.918458605850243)(197,0.918458605850243)(198,0.918646977862196)(198,0.918646977862196)(198,0.918646977862196)(198,0.918646977862196)(198,0.918646977862196)(198,0.918646977862196)(198,0.918646977862196)(198,0.918646977862196)(198,0.918646977862196)(199,0.918857260391134)(199,0.918857260391134)(199,0.918857260391134)(199,0.918857260391134)(199,0.918857260391134)(200,0.919058522666762)(200,0.919058522666762)(200,0.919058522666762)(200,0.919058522666762)(200,0.919058522666762)(200,0.919058522666762)(201,0.919249682995987)(201,0.919249682995987)(202,0.91942956004163)(202,0.91942956004163)(202,0.91942956004163)(202,0.91942956004163)(203,0.919597109101008)(203,0.919597109101008)(204,0.919751795235801)(204,0.919751795235801)(204,0.919751795235801)(205,0.919894006618849)(205,0.919894006618849)(206,0.920024947876035)(206,0.920024947876035)(206,0.920024947876035)(206,0.920024947876035)(206,0.920024947876035)(207,0.92014824718155)(208,0.920300035318397)(208,0.920300035318397)(208,0.920300035318397)(208,0.920300035318397)(209,0.920428542650078)(209,0.920428542650078)(209,0.920428542650078)(210,0.920604183580779)(210,0.920604183580779)(210,0.920604183580779)(211,0.920743842124826)(211,0.920743842124826)(211,0.920743842124826)(211,0.920743842124826)(211,0.920743842124826)(211,0.920743842124826)(212,0.920887036235071)(212,0.920887036235071)(212,0.920887036235071)(213,0.921097619881995)(213,0.921097619881995)(213,0.921097619881995)(213,0.921097619881995)(213,0.921097619881995)(214,0.92125553380862)(214,0.92125553380862)(214,0.92125553380862)(215,0.921484089277178)(215,0.921484089277178)(215,0.921484089277178)(217,0.921895483300176)(217,0.921895483300176)(217,0.921895483300176)(217,0.921895483300176)(217,0.921895483300176)(217,0.921895483300176)(218,0.922079144242787)(218,0.922079144242787)(218,0.922079144242787)(218,0.922079144242787)(219,0.92226996325111)(220,0.922420728735662)(220,0.922420728735662)(220,0.922420728735662)(220,0.922420728735662)(220,0.922420728735662)(221,0.922677885796664)(221,0.922677885796664)(221,0.922677885796664)(221,0.922677885796664)(221,0.922677885796664)(222,0.922896585442007)(222,0.922896585442007)(223,0.923125057767663)(223,0.923125057767663)(223,0.923125057767663)(223,0.923125057767663)(223,0.923125057767663)(223,0.923125057767663)(223,0.923125057767663)(224,0.923363112046205)(225,0.923611820325899)(225,0.923611820325899)(225,0.923611820325899)(225,0.923611820325899)(225,0.923611820325899)(225,0.923611820325899)(225,0.923611820325899)(225,0.923611820325899)(226,0.923855942315075)(226,0.923855942315075)(226,0.923855942315075)(226,0.923855942315075)(227,0.924107806382789)(227,0.924107806382789)(227,0.924107806382789)(228,0.924365789877882)(228,0.924365789877882)(229,0.924627722999601)(229,0.924627722999601)(230,0.924892069324925)(230,0.924892069324925)(231,0.925128639310369)(231,0.925128639310369)(231,0.925128639310369)(232,0.925393743179381)(232,0.925393743179381)(233,0.925657397266281)(233,0.925657397266281)(233,0.925657397266281)(233,0.925657397266281)(233,0.925657397266281)(234,0.925895943308491)(235,0.926154158526316)(235,0.926154158526316)(236,0.926407650552276)(236,0.926407650552276)(236,0.926407650552276)(236,0.926407650552276)(237,0.926636602376951)(237,0.926636602376951)(237,0.926636602376951)(237,0.926636602376951)(237,0.926636602376951)(238,0.926879691061465)(238,0.926879691061465)(238,0.926879691061465)(239,0.927094470039503)(239,0.927094470039503)(239,0.927094470039503)(239,0.927094470039503)(239,0.927094470039503)(239,0.927094470039503)(240,0.927323915440798)(241,0.927546156459612)(242,0.927760505315576)(242,0.927760505315576)(242,0.927760505315576)(243,0.927966731102263)(243,0.927966731102263)(244,0.928140238652443)(244,0.928140238652443)(244,0.928140238652443)(245,0.928331059814702)(245,0.928331059814702)(245,0.928331059814702)(245,0.928331059814702)(245,0.928331059814702)(245,0.928331059814702)(245,0.928331059814702)(246,0.928536299114566)(246,0.928536299114566)(246,0.928536299114566)(246,0.928536299114566)(246,0.928536299114566)(247,0.928710366737071)(247,0.928710366737071)(248,0.928877348936947)(248,0.928877348936947)(248,0.928877348936947)(249,0.929037645036051)(249,0.929037645036051)(249,0.929037645036051)(249,0.929037645036051)(249,0.929037645036051)(251,0.92930811116282)(251,0.92930811116282)(251,0.92930811116282)(251,0.92930811116282)(252,0.929456013630432)(252,0.929456013630432)(252,0.929456013630432)(253,0.929613308238166)(253,0.929613308238166)(253,0.929613308238166)(253,0.929613308238166)(253,0.929613308238166)(253,0.929613308238166)(254,0.92974282664044)(254,0.92974282664044)(255,0.929882964510021)(255,0.929882964510021)(255,0.929882964510021)(255,0.929882964510021)(255,0.929882964510021)(255,0.929882964510021)(255,0.929882964510021)(255,0.929882964510021)(256,0.930021573920496)(256,0.930021573920496)(256,0.930021573920496)(256,0.930021573920496)(256,0.930021573920496)(257,0.930159284013752)(257,0.930159284013752)(258,0.930292813742559)(258,0.930292813742559)(258,0.930292813742559)(259,0.930433568956954)(259,0.930433568956954)(259,0.930433568956954)(259,0.930433568956954)(259,0.930433568956954)(260,0.930570280329598)(260,0.930570280329598)(261,0.930708082872402)(261,0.930708082872402)(262,0.930846788608287)(262,0.930846788608287)(262,0.930846788608287)(263,0.93099276907075)(263,0.93099276907075)(263,0.93099276907075)(263,0.93099276907075)(263,0.93099276907075)(264,0.931133057197957)(264,0.931133057197957)(266,0.931424920079992)(266,0.931424920079992)(267,0.9315646932364)(267,0.9315646932364)(267,0.9315646932364)(268,0.931704008997427)(268,0.931704008997427)(269,0.931843051998276)(269,0.931843051998276)(269,0.931843051998276)(270,0.931982019783776)(271,0.932120833300943)(271,0.932120833300943)(271,0.932120833300943)(271,0.932120833300943)(272,0.932259207445736)(272,0.932259207445736)(272,0.932259207445736)(272,0.932259207445736)(272,0.932259207445736)(272,0.932259207445736)(273,0.932396690431212)(273,0.932396690431212)(273,0.932396690431212)(273,0.932396690431212)(274,0.932532822535606)(274,0.932532822535606)(275,0.932667369785282)(275,0.932667369785282)(275,0.932667369785282)(275,0.932667369785282)(276,0.932816895484207)(276,0.932816895484207)(276,0.932816895484207)(276,0.932816895484207)(276,0.932816895484207)(277,0.932948002349668)(277,0.932948002349668)(279,0.933217467265927)(279,0.933217467265927)(279,0.933217467265927)(280,0.933342044172786)(280,0.933342044172786)(281,0.933472948435949)(281,0.933472948435949)(281,0.933472948435949)(282,0.933584937569411)(282,0.933584937569411)(282,0.933584937569411)(283,0.933709350251084)(283,0.933709350251084)(283,0.933709350251084)(284,0.933819739619485)(284,0.933819739619485)(284,0.933819739619485)(284,0.933819739619485)(285,0.933938769917531)(285,0.933938769917531)(285,0.933938769917531)(285,0.933938769917531)(285,0.933938769917531)(286,0.934050775633334)(286,0.934050775633334)(287,0.934160828637192)(287,0.934160828637192)(287,0.934160828637192)(288,0.934269087955371)(288,0.934269087955371)(289,0.934376850022737)(289,0.934376850022737)(289,0.934376850022737)(290,0.934481497274048)(290,0.934481497274048)(290,0.934481497274048)(291,0.934585090514012)(291,0.934585090514012)(293,0.934788235007227)(293,0.934788235007227)(293,0.934788235007227)(293,0.934788235007227)(293,0.934788235007227)(293,0.934788235007227)(293,0.934788235007227)(293,0.934788235007227)(294,0.934889336576926)(294,0.934889336576926)(294,0.934889336576926)(295,0.934989802627311)(295,0.934989802627311)(296,0.935086320396076)(296,0.935086320396076)(296,0.935086320396076)(297,0.935184529174303)(297,0.935184529174303)(299,0.935379295634482)(299,0.935379295634482)(300,0.935487569678589)(300,0.935487569678589)(300,0.935487569678589)(300,0.935487569678589)(300,0.935487569678589)(301,0.935586134078895)(301,0.935586134078895)(302,0.935683241432036)(302,0.935683241432036)(302,0.935683241432036)(302,0.935683241432036)(303,0.93577823558027)(303,0.93577823558027)(303,0.93577823558027)(303,0.93577823558027)(303,0.93577823558027)(304,0.935870874909296)(305,0.935981720856055)(305,0.935981720856055)(306,0.936048918303784)(306,0.936048918303784)(306,0.936048918303784)(306,0.936048918303784)(306,0.936048918303784)(307,0.936134149521146)(307,0.936134149521146)(307,0.936134149521146)(308,0.936216908739423)(308,0.936216908739423)(309,0.93629718545123)(309,0.93629718545123)(309,0.93629718545123)(310,0.936358460218166)(310,0.936358460218166)(311,0.936449499733632)(311,0.936449499733632)(312,0.936521415819782)(312,0.936521415819782)(313,0.936590806488635)(314,0.936658101032171)(314,0.936658101032171)(314,0.936658101032171)(314,0.936658101032171)(314,0.936658101032171)(315,0.936723851250188)(315,0.936723851250188)(315,0.936723851250188)(316,0.936788567381344)(316,0.936788567381344)(316,0.936788567381344)(316,0.936788567381344)(316,0.936788567381344)(317,0.936852845110949)(317,0.936852845110949)(318,0.936916937526551)(318,0.936916937526551)(319,0.936983875154879)(319,0.936983875154879)(320,0.937051515042851)(321,0.93712015260113)(321,0.93712015260113)(321,0.93712015260113)(321,0.93712015260113)(321,0.93712015260113)(322,0.937189843634425)(323,0.937260457595923)(323,0.937260457595923)(323,0.937260457595923)(323,0.937260457595923)(324,0.937331790616095)(324,0.937331790616095)(325,0.937403435304088)(325,0.937403435304088)(325,0.937403435304088)(326,0.937474840290392)(326,0.937474840290392)(326,0.937474840290392)(327,0.937552238131523)(327,0.937552238131523)(328,0.937626773371286)(329,0.937694461384792)(329,0.937694461384792)(330,0.937761183275321)(331,0.937827184712241)(331,0.937827184712241)(331,0.937827184712241)(332,0.937892562301913)(332,0.937892562301913)(332,0.937892562301913)(332,0.937892562301913)(332,0.937892562301913)(333,0.937962061626514)(333,0.937962061626514)(334,0.938027578756025)(334,0.938027578756025)(335,0.938093283481599)(335,0.938093283481599)(335,0.938093283481599)(336,0.938168166772309)(336,0.938168166772309)(336,0.938168166772309)(336,0.938168166772309)(337,0.938225694355226)(338,0.93830547151828)(338,0.93830547151828)(338,0.93830547151828)(339,0.938360580528777)(339,0.938360580528777)(339,0.938360580528777)(340,0.938429551618198)(340,0.938429551618198)(340,0.938429551618198)(340,0.938429551618198)(342,0.938571750493247)(343,0.938645665560288)(344,0.938697997744301)(344,0.938697997744301)(345,0.938801028221319)(345,0.938801028221319)(346,0.938883481310555)(346,0.938883481310555)(347,0.938994220632478)(347,0.938994220632478)(347,0.938994220632478)(347,0.938994220632478)(348,0.939083431342537)(348,0.939083431342537)(349,0.939176110254813)(349,0.939176110254813)(349,0.939176110254813)(349,0.939176110254813)(350,0.939272349987267)(350,0.939272349987267)(351,0.939372120140375)(351,0.939372120140375)(352,0.939475293040564)(353,0.939569232136106)(353,0.939569232136106)(353,0.939569232136106)(353,0.939569232136106)(354,0.93968184769229)(354,0.93968184769229)(354,0.93968184769229)(355,0.939797175801459)(355,0.939797175801459)(356,0.939914894987963)(356,0.939914894987963)(357,0.940036768326032)(357,0.940036768326032)(358,0.940156713854856)(359,0.940278442927191)(359,0.940278442927191)(360,0.940400909279599)(360,0.940400909279599)(360,0.940400909279599)(361,0.940523780944138)(362,0.940647140775341)(362,0.940647140775341)(363,0.940770672336747)(363,0.940770672336747)(364,0.940894111064379)(364,0.940894111064379)(364,0.940894111064379)(364,0.940894111064379)(365,0.941017124135091)(365,0.941017124135091)(365,0.941017124135091)(365,0.941017124135091)(366,0.941139401524147)(366,0.941139401524147)(367,0.941266638794667)(367,0.941266638794667)(367,0.941266638794667)(367,0.941266638794667)(367,0.941266638794667)(368,0.941387320226942)(368,0.941387320226942)(369,0.941499892982037)(369,0.941499892982037)(369,0.941499892982037)(370,0.941617096108515)(370,0.941617096108515)(370,0.941617096108515)(370,0.941617096108515)(370,0.941617096108515)(370,0.941617096108515)(370,0.941617096108515)(371,0.941740447152059)(371,0.941740447152059)(372,0.941846551666915)(373,0.941967009707575)(373,0.941967009707575)(374,0.942077368382727)(374,0.942077368382727)(374,0.942077368382727)(374,0.942077368382727)(374,0.942077368382727)(374,0.942077368382727)(374,0.942077368382727)(375,0.942185879642803)(375,0.942185879642803)(376,0.942292672818459)(376,0.942292672818459)(377,0.942397735332375)(377,0.942397735332375)(378,0.942501120623488)(378,0.942501120623488)(378,0.942501120623488)(378,0.942501120623488)(378,0.942501120623488)(378,0.942501120623488)(378,0.942501120623488)(379,0.942595519990944)(380,0.942696330384733)(380,0.942696330384733)(380,0.942696330384733)(381,0.942795822886535)(381,0.942795822886535)(381,0.942795822886535)(382,0.94289410686233)(382,0.94289410686233)(382,0.94289410686233)(383,0.942985492072449)(383,0.942985492072449)(383,0.942985492072449)(384,0.943082568416734)(384,0.943082568416734)(384,0.943082568416734)(385,0.94317910308381)(385,0.94317910308381)(385,0.94317910308381)(385,0.94317910308381)(385,0.94317910308381)(385,0.94317910308381)(386,0.943275198904385)(386,0.943275198904385)(387,0.943370872334121)(387,0.943370872334121)(387,0.943370872334121)(388,0.943466083869711)(389,0.943560824408739)(390,0.94365450012108)(390,0.94365450012108)(390,0.94365450012108)(391,0.943749044231695)(392,0.943842833015413)(392,0.943842833015413)(392,0.943842833015413)(392,0.943842833015413)(392,0.943842833015413)(392,0.943842833015413)(392,0.943842833015413)(393,0.943936917860752)(393,0.943936917860752)(394,0.944030271008052)(394,0.944030271008052)(394,0.944030271008052)(395,0.944122992027728)(395,0.944122992027728)(395,0.944122992027728)(395,0.944122992027728)(395,0.944122992027728)(396,0.944214917403092)(396,0.944214917403092)(397,0.944305802030547)(397,0.944305802030547)(398,0.944395397563303)(398,0.944395397563303)(399,0.944485693852462)(399,0.944485693852462)(401,0.944657453334161)(401,0.944657453334161)(401,0.944657453334161)(401,0.944657453334161)(401,0.944657453334161)(401,0.944657453334161)(402,0.944736454167789)(402,0.944736454167789)(403,0.944821196927252)(403,0.944821196927252)(404,0.94490029870112)(405,0.944972583375254)(405,0.944972583375254)(406,0.945053144363382)(406,0.945053144363382)(407,0.94512684154692)(408,0.945198647277905)(408,0.945198647277905)(409,0.945268560331253)(409,0.945268560331253)(411,0.945408308977651)(411,0.945408308977651)(411,0.945408308977651)(411,0.945408308977651)(412,0.945468856781945)(412,0.945468856781945)(412,0.945468856781945)(412,0.945468856781945)(413,0.94553343926536)(413,0.94553343926536)(413,0.94553343926536)(414,0.945597322196766)(414,0.945597322196766)(415,0.945660747227194)(416,0.945723965366795)(416,0.945723965366795)(416,0.945723965366795)(417,0.945787237267002)(417,0.945787237267002)(417,0.945787237267002)(417,0.945787237267002)(417,0.945787237267002)(417,0.945787237267002)(418,0.9458508095406)(419,0.945914814077907)(419,0.945914814077907)(420,0.945978891317321)(420,0.945978891317321)(421,0.946043805315419)(421,0.946043805315419)(421,0.946043805315419)(422,0.9461096778952)(422,0.9461096778952)(423,0.946176574742381)(423,0.946176574742381)(423,0.946176574742381)(423,0.946176574742381)(424,0.946244493966012)(424,0.946244493966012)(425,0.946313381067814)(425,0.946313381067814)(426,0.94638311401538)(426,0.94638311401538)(426,0.94638311401538)(426,0.94638311401538)(427,0.946453685889465)(427,0.946453685889465)(427,0.946453685889465)(427,0.946453685889465)(427,0.946453685889465)(428,0.946525084372605)(428,0.946525084372605)(429,0.946597271631574)(429,0.946597271631574)(429,0.946597271631574)(430,0.946670184249663)(430,0.946670184249663)(431,0.94674364812043)(432,0.946817423248669)(433,0.946891187422726)(434,0.946964716842736)(434,0.946964716842736)(434,0.946964716842736)(435,0.947037839776952)(435,0.947037839776952)(435,0.947037839776952)(436,0.947110388733007)(436,0.947110388733007)(436,0.947110388733007)(436,0.947110388733007)(436,0.947110388733007)(437,0.947182163726667)(437,0.947182163726667)(438,0.947245076963646)(438,0.947245076963646)(439,0.947322967519866)(440,0.947392044664776)(440,0.947392044664776)(441,0.947460286903572)(441,0.947460286903572)(441,0.947460286903572)(441,0.947460286903572)(441,0.947460286903572)(441,0.947460286903572)(442,0.94752764723418)(442,0.94752764723418)(442,0.94752764723418)(443,0.94759413211329)(443,0.94759413211329)(443,0.94759413211329)(444,0.947664004512676)(444,0.947664004512676)(445,0.947728360207688)(445,0.947728360207688)(446,0.947792241081021)(446,0.947792241081021)(446,0.947792241081021)(446,0.947792241081021)(446,0.947792241081021)(446,0.947792241081021)(446,0.947792241081021)(446,0.947792241081021)(448,0.947919116602405)(448,0.947919116602405)(448,0.947919116602405)(449,0.947981056568098)(450,0.948043061875552)(450,0.948043061875552)(450,0.948043061875552)(451,0.948108349798987)(451,0.948108349798987)(451,0.948108349798987)(451,0.948108349798987)(452,0.94817137337366)(452,0.94817137337366)(453,0.948230859692371)(453,0.948230859692371)(453,0.948230859692371)(454,0.948294456929469)(454,0.948294456929469)(456,0.94842398873187)(456,0.94842398873187)(456,0.94842398873187)(457,0.948490191832307)(457,0.948490191832307)(457,0.948490191832307)(457,0.948490191832307)(458,0.948554281590503)(458,0.948554281590503)(460,0.948694611057989)(460,0.948694611057989)(460,0.948694611057989)(460,0.948694611057989)(461,0.948763728823843)(462,0.948828798903317)(463,0.948897238689887)(463,0.948897238689887)(464,0.948967506802842)(464,0.948967506802842)(464,0.948967506802842)(465,0.949032809026039)(465,0.949032809026039)(465,0.949032809026039)(467,0.949157307273251)(468,0.949215981626661)(468,0.949215981626661)(470,0.949327263074816)(471,0.94937991222194)(471,0.94937991222194)(471,0.94937991222194)(472,0.949431826455634)(472,0.949431826455634)(472,0.949431826455634)(473,0.949480843893816)(473,0.949480843893816)(473,0.949480843893816)(474,0.949529711653238)(474,0.949529711653238)(474,0.949529711653238)(474,0.949529711653238)(475,0.949577208119978)(476,0.949623985800978)(476,0.949623985800978)(476,0.949623985800978)(477,0.949672117128658)(477,0.949672117128658)(477,0.949672117128658)(478,0.949718710994607)(478,0.949718710994607)(478,0.949718710994607)(478,0.949718710994607)(478,0.949718710994607)(479,0.949764989001752)(480,0.949810908023758)(480,0.949810908023758)(481,0.949856452084605)(481,0.949856452084605)(481,0.949856452084605)(481,0.949856452084605)(481,0.949856452084605)(482,0.949901613088314)(482,0.949901613088314)(482,0.949901613088314)(484,0.949990330332869)(484,0.949990330332869)(484,0.949990330332869)(484,0.949990330332869)(485,0.950039763219163)(486,0.950082208469944)(486,0.950082208469944)(486,0.950082208469944)(486,0.950082208469944)(487,0.950123620462965)(487,0.950123620462965)(488,0.950170717821384)(488,0.950170717821384)(488,0.950170717821384)(488,0.950170717821384)(488,0.950170717821384)(488,0.950170717821384)(490,0.950248036299166)(491,0.950291766376354)(491,0.950291766376354)(491,0.950291766376354)(491,0.950291766376354)(491,0.950291766376354)(492,0.950327254677343)(493,0.950361522907843)(493,0.950361522907843)(494,0.950388170555405)(494,0.950388170555405)(494,0.950388170555405)(495,0.950420566169453)(495,0.950420566169453)(495,0.950420566169453)(495,0.950420566169453)(495,0.950420566169453)(496,0.95045210112978)(496,0.95045210112978)(496,0.95045210112978)(496,0.95045210112978)(497,0.950482939033348)(497,0.950482939033348)(498,0.950513164218836)(500,0.950572101114518)(500,0.950572101114518)(500,0.950572101114518)(500,0.950572101114518)(500,0.950572101114518)(501,0.950601102205993)(501,0.950601102205993)(501,0.950601102205993)(502,0.95063003055344)(502,0.95063003055344)(502,0.95063003055344)(503,0.950659104668244)(504,0.950688410828647)(504,0.950688410828647)(505,0.950714909064388)(505,0.950714909064388)(505,0.950714909064388)(505,0.950714909064388)(506,0.950745542352787)(506,0.950745542352787)(506,0.950745542352787)(507,0.950779196330858)(508,0.95081733033243)(508,0.95081733033243)(509,0.950854633271211)(509,0.950854633271211)(509,0.950854633271211)(510,0.95090051378696)(510,0.95090051378696)(511,0.950941334885675)(511,0.950941334885675)(511,0.950941334885675)(512,0.950982778319549)(512,0.950982778319549)(513,0.951011825508632)(513,0.951011825508632)(513,0.951011825508632)(514,0.951052919995166)(514,0.951052919995166)(515,0.95109474838894)(517,0.951197844164733)(518,0.951225288996182)(518,0.951225288996182)(518,0.951225288996182)(518,0.951225288996182)(518,0.951225288996182)(519,0.95127079677028)(519,0.95127079677028)(519,0.95127079677028)(520,0.951317531750138)(520,0.951317531750138)(520,0.951317531750138)(522,0.951448635209508)(523,0.951512067120886)(524,0.951573164678356)(524,0.951573164678356)(524,0.951573164678356)(524,0.951573164678356)(524,0.951573164678356)(525,0.951631651151461)(525,0.951631651151461)(526,0.951687494971741)(526,0.951687494971741)(526,0.951687494971741)(526,0.951687494971741)(527,0.951740818335252)(527,0.951740818335252)(527,0.951740818335252)(528,0.951791827235929)(528,0.951791827235929)(529,0.951840773880612)(529,0.951840773880612)(529,0.951840773880612)(530,0.951887914426481)(530,0.951887914426481)(530,0.951887914426481)(531,0.951933493174719)(531,0.951933493174719)(531,0.951933493174719)(531,0.951933493174719)(532,0.95197773294013)(533,0.952020829251745)(533,0.952020829251745)(533,0.952020829251745)(533,0.952020829251745)(535,0.952104254611248)(535,0.952104254611248)(535,0.952104254611248)(535,0.952104254611248)(536,0.95214486148952)(536,0.95214486148952)(536,0.95214486148952)(536,0.95214486148952)(538,0.952224403117952)(538,0.952224403117952)(538,0.952224403117952)(539,0.952263507560446)(539,0.952263507560446)(539,0.952263507560446)(540,0.952302260465378)(540,0.952302260465378)(540,0.952302260465378)(541,0.952340717716904)(541,0.952340717716904)(542,0.952378925822984)(542,0.952378925822984)(542,0.952378925822984)(543,0.952416923188126)(543,0.952416923188126)(543,0.952416923188126)(543,0.952416923188126)(544,0.952454740106917)(544,0.952454740106917)(544,0.952454740106917)(544,0.952454740106917)(544,0.952454740106917)(544,0.952454740106917)(545,0.952492400501982)(545,0.952492400501982)(545,0.952492400501982)(545,0.952492400501982)(545,0.952492400501982)(548,0.952604627185151)(548,0.952604627185151)(548,0.952604627185151)(549,0.952641827538437)(549,0.952641827538437)(550,0.952678937810328)(550,0.952678937810328)(550,0.952678937810328)(551,0.952715962914968)(551,0.952715962914968)(551,0.952715962914968)(552,0.952752907212379)(552,0.952752907212379)(552,0.952752907212379)(552,0.952752907212379)(552,0.952752907212379)(553,0.952789773657518)(553,0.952789773657518)(554,0.952826561948452)(554,0.952826561948452)(555,0.952863271378543)(556,0.952899902496912)(556,0.952899902496912)(556,0.952899902496912)(556,0.952899902496912)(556,0.952899902496912)(556,0.952899902496912)(556,0.952899902496912)(557,0.9529364548354)(557,0.9529364548354)(558,0.952972924677132)(559,0.953009307625925)(559,0.953009307625925)(559,0.953009307625925)(559,0.953009307625925)(560,0.953045599381395)(560,0.953045599381395)(560,0.953045599381395)(561,0.95308179478169)(561,0.95308179478169)(561,0.95308179478169)(561,0.95308179478169)(562,0.953117886561998)(563,0.953153865331655)(563,0.953153865331655)(563,0.953153865331655)(563,0.953153865331655)(563,0.953153865331655)(564,0.953189721435023)(564,0.953189721435023)(565,0.953225445103168)(565,0.953225445103168)(565,0.953225445103168)(565,0.953225445103168)(566,0.953261025205445)(566,0.953261025205445)(566,0.953261025205445)(567,0.953296449605817)(567,0.953296449605817)(567,0.953296449605817)(567,0.953296449605817)(567,0.953296449605817)(567,0.953296449605817)(567,0.953296449605817)(567,0.953296449605817)(568,0.953331706496799)(568,0.953331706496799)(569,0.953366785284404)(569,0.953366785284404)(569,0.953366785284404)(570,0.953401675273727)(570,0.953401675273727)(571,0.95343636443456)(571,0.95343636443456)(573,0.953505088042898)(573,0.953505088042898)(573,0.953505088042898)(575,0.953572846983547)(575,0.953572846983547)(575,0.953572846983547)(576,0.953606327529818)(576,0.953606327529818)(576,0.953606327529818)(577,0.953639522192884)(577,0.953639522192884)(577,0.953639522192884)(577,0.953639522192884)(578,0.953672417024098)(578,0.953672417024098)(578,0.953672417024098)(579,0.95370499861541)(579,0.95370499861541)(579,0.95370499861541)(579,0.95370499861541)(580,0.953737255942739)(581,0.95376917945238)(581,0.95376917945238)(583,0.95383199114179)(583,0.95383199114179)(584,0.953862864984462)(585,0.953893376702103)(586,0.953923523332726)(588,0.953982715761175)(591,0.954068752755272)(591,0.954068752755272)(592,0.954096704082241)(592,0.954096704082241)(593,0.954124293423111)(593,0.954124293423111)(593,0.954124293423111)(593,0.954124293423111)(593,0.954124293423111)(594,0.954151520191929)(595,0.954178382720145)(595,0.954178382720145)(595,0.954178382720145)(595,0.954178382720145)(597,0.95423101602886)(597,0.95423101602886)(597,0.95423101602886)(598,0.954256791974855)(598,0.954256791974855)(598,0.954256791974855)(598,0.954256791974855)(599,0.954282215485697)(599,0.954282215485697)(600,0.95430729189829)(600,0.95430729189829) 
};
\addlegendentry{\acl};

\addplot [
color=orange,
densely dotted,
line width=1.0pt,
]
coordinates{
 %(1,0.392177909307569)(2,0.434305607433622)(3,0.460896247045768)(4,0.48846519867748)
 (5,0.510603202590168)(6,0.541759958909996)(7,0.559395017477202)(8,0.567534632806055)(9,0.561334774230373)(10,0.553952424163672)(11,0.586401404088778)(12,0.610368715486238)(13,0.621513657591476)(14,0.637342435480346)(15,0.652084051335159)(16,0.66028829887245)(17,0.663178019536025)(18,0.664502923673692)(19,0.681566409849065)(20,0.687189429850302)(21,0.694130919647457)(22,0.706965821521425)(23,0.715065154439404)(24,0.719098612754838)(25,0.724454822341628)(26,0.724323663300594)(27,0.730642973950263)(28,0.735550673047947)(29,0.737472477556679)(30,0.741973244857011)(31,0.744263412501384)(32,0.747827693107535)(33,0.750563421671531)(34,0.754850832889301)(35,0.757405302707012)(36,0.76261421638441)(37,0.771595438713839)(38,0.774484716467004)(39,0.776412578288743)(40,0.777715245709549)(41,0.782137493482581)(42,0.785340345277127)(43,0.790183615909202)(44,0.792254932662845)(45,0.79588922958548)(46,0.798594722213691)(47,0.800889870735558)(48,0.801988852673127)(49,0.804602080848541)(50,0.807071343802153)(51,0.808765821100399)(52,0.812148126360336)(53,0.812787350105096)(54,0.812893448473761)(55,0.816605727878638)(56,0.818643367714006)(57,0.821173298259702)(58,0.821303158148494)(59,0.822678383313494)(60,0.823627102161218)(61,0.824988603400808)(62,0.827413515192519)(63,0.828118960464225)(64,0.830454563637208)(65,0.830816205633827)(66,0.831258807751198)(67,0.83282708006963)(68,0.833305570628797)(69,0.835967529086521)(70,0.837527647337614)(71,0.837944210487354)(72,0.838495246585828)(73,0.839382437805333)(74,0.83962049806619)(75,0.839436421114712)(76,0.840523566771127)(77,0.840488839044777)(78,0.840911829283312)(79,0.8433214036321)(80,0.843553650183411)(81,0.843814237845753)(82,0.8449628503591)(83,0.845791115960526)(84,0.846248538147691)(85,0.847881806820399)(86,0.847861270100337)(87,0.848148751367538)(88,0.847794459701455)(89,0.850205334778999)(90,0.850352028327293)(91,0.850561839404417)(92,0.851226053785681)(93,0.851750938570967)(94,0.851141494633922)(95,0.851578347103454)(96,0.851807236807806)(97,0.852012782518477)(98,0.853115636816115)(99,0.853842546098837)(100,0.853293333733195)(101,0.852790980433476)(102,0.852404853047144)(103,0.851811174163624)(104,0.852286960458259)(105,0.852328183275976)(106,0.853509509850928)(107,0.85446260750531)(108,0.855095869084598)(109,0.856068660403046)(110,0.856343950872113)(111,0.856384052507733)(112,0.859674559990213)(113,0.859686869606565)(114,0.86094058064997)(115,0.860987872693964)(116,0.861671143518224)(117,0.861911159149054)(118,0.862189386602573)(119,0.862143083462875)(120,0.862516048094474)(121,0.86271640021711)(122,0.862897418064781)(123,0.863246865004816)(124,0.863632642651843)(125,0.863646045481476)(126,0.86447411477076)(127,0.864344754131238)(128,0.865518826130816)(129,0.865319047383617)(130,0.865920763600936)(131,0.865414387526887)(132,0.865366690588121)(133,0.865578272449321)(134,0.864612541811356)(135,0.864896914016978)(136,0.865417446222026)(137,0.865730939729039)(138,0.866704215789558)(139,0.866429438866189)(140,0.866564381190482)(141,0.866901301908505)(142,0.865826191502138)(143,0.865951626648741)(144,0.866330750367837)(145,0.866639563165819)(146,0.867113573355419)(147,0.867513497270724)(148,0.867610786756046)(149,0.867561668215731)(150,0.867963343539587)(151,0.868404520615245)(152,0.86812235429625)(153,0.868565484675432)(154,0.868459244335659)(155,0.868889940586563)(156,0.869235332815046)(157,0.870488884646859)(158,0.871298569764229)(159,0.871539528157036)(160,0.871596127859458)(161,0.871988839619362)(162,0.872302583053398)(163,0.873254843478197)(164,0.873714203381859)(165,0.87367408854292)(166,0.873848946577713)(167,0.874127735868758)(168,0.87456764629588)(169,0.875113902752167)(170,0.875335289003982)(171,0.875178912303321)(172,0.875143098930017)(173,0.875349727726284)(174,0.87564996953362)(175,0.875644041630002)(176,0.876197658262335)(177,0.876064184092992)(178,0.875424001327038)(179,0.87573713150353)(180,0.875548635640144)(181,0.875326409333511)(182,0.875748802028017)(183,0.876158516785436)(184,0.876326667388851)(185,0.876460438053816)(186,0.876641161473296)(187,0.876808401883723)(188,0.876500694509223)(189,0.876791794422052)(190,0.87670730076034)(191,0.877487939216201)(192,0.877525237176353)(193,0.87758537879455)(194,0.877833409582984)(195,0.878189107685786)(196,0.878371775852341)(197,0.878344147372301)(198,0.878179806928612)(199,0.878422252824072)(200,0.878674859120912)(201,0.879052604011007)(202,0.879206890564204)(203,0.879372848411761)(204,0.87920243099779)(205,0.879552416517159)(206,0.880016546183501)(207,0.880228704164601)(208,0.880143633841616)(209,0.880010354466422)(210,0.879898403667828)(211,0.880067943730996)(212,0.880070596681998)(213,0.879438161916072)(214,0.878867564826086)(215,0.878933718747763)(216,0.878795872599082)(217,0.879043022690725)(218,0.879140153662427)(219,0.879386013304915)(220,0.880042323683274)(221,0.880073015090081)(222,0.880375369235307)(223,0.880502776526734)(224,0.880408121948301)(225,0.880681976203209)(226,0.88061039846341)(227,0.88073230665122)(228,0.880657057286917)(229,0.880761268493161)(230,0.880786223044232)(231,0.880768847575946)(232,0.880703006405412)(233,0.880439653135534)(234,0.880656409879115)(235,0.881102242455584)(236,0.881855467421049)(237,0.882337635199699)(238,0.882156589141116)(239,0.881837406296399)(240,0.881710330467841)(241,0.881994735827437)(242,0.881724609397933)(243,0.881682807084348)(244,0.882021731392521)(245,0.882185846300988)(246,0.882303969030763)(247,0.882569492212847)(248,0.882321935673501)(249,0.88252345528708)(250,0.882773677880689)(251,0.883273377113342)(252,0.882904045476928)(253,0.883554621513736)(254,0.883813892607702)(255,0.884138186055733)(256,0.884239142867972)(257,0.884399336390763)(258,0.884594393823889)(259,0.884739420503078)(260,0.88481958088532)(261,0.885318402824425)(262,0.885501740547411)(263,0.886061535604216)(264,0.886447624247986)(265,0.886621602968066)(266,0.887086442217922)(267,0.887320514644856)(268,0.88706610168339)(269,0.887353865184737)(270,0.887213642293424)(271,0.88730828383702)(272,0.887469252744991)(273,0.887292598333699)(274,0.887507292463619)(275,0.888047472290935)(276,0.888151901882842)(277,0.888377924682568)(278,0.888390147386794)(279,0.88830800464181)(280,0.888333406007914)(281,0.888486412305535)(282,0.888647409343968)(283,0.888917886379454)(284,0.888834150789662)(285,0.888855490322091)(286,0.88900737947079)(287,0.888892248066503)(288,0.88860176966624)(289,0.888512479383079)(290,0.88861964555797)(291,0.889500013899261)(292,0.889740259024171)(293,0.889617173840253)(294,0.889857590155014)(295,0.889741237605197)(296,0.889956451948332)(297,0.890294360824646)(298,0.890485646145969)(299,0.890728050311036)(300,0.890880879596515)(301,0.891069073976573)(302,0.891326663243706)(303,0.89166649329334)(304,0.891815226200105)(305,0.892016458699443)(306,0.8919694578632)(307,0.892141623135134)(308,0.892498593450117)(309,0.892530880588495)(310,0.892988095245665)(311,0.892958979210057)(312,0.893269094067054)(313,0.893538490736538)(314,0.893615746691374)(315,0.893827336140206)(316,0.893844191172967)(317,0.894367383066641)(318,0.894515335815894)(319,0.894535986380688)(320,0.894299629474141)(321,0.894418605829355)(322,0.894432520643552)(323,0.894524402349229)(324,0.894613118850749)(325,0.894646880907919)(326,0.894985729578886)(327,0.894960688533839)(328,0.895319673811744)(329,0.895127345274135)(330,0.895121393650125)(331,0.895413068090575)(332,0.895331791262932)(333,0.895551615392741)(334,0.895909414547586)(335,0.895811676961937)(336,0.89609305151717)(337,0.896342925540334)(338,0.896513339791038)(339,0.896193501103016)(340,0.896306653964714)(341,0.896575796618769)(342,0.896559273881444)(343,0.896408572098799)(344,0.896504438028272)(345,0.896654464474392)(346,0.896692709228924)(347,0.897298667513502)(348,0.897124886836194)(349,0.897261547084508)(350,0.897129371690448)(351,0.896658726849181)(352,0.896838836408849)(353,0.896595932217809)(354,0.896664941892105)(355,0.896739928009266)(356,0.896720532686958)(357,0.897121694922424)(358,0.897128345537141)(359,0.897264270890217)(360,0.897548436261553)(361,0.897378046161669)(362,0.897683167338406)(363,0.897887157051013)(364,0.897922244161419)(365,0.89811860238332)(366,0.897857687875592)(367,0.897849158542264)(368,0.897901395440447)(369,0.897814230779597)(370,0.897696845362466)(371,0.897375541430808)(372,0.89718777234211)(373,0.897309330744772)(374,0.897400477718954)(375,0.897506042265313)(376,0.897494178411415)(377,0.897496505950808)(378,0.897351316907233)(379,0.897351434971993)(380,0.897607254857857)(381,0.897962319910413)(382,0.898404461461519)(383,0.89863873140715)(384,0.898285449788929)(385,0.898220425305483)(386,0.898044351179498)(387,0.897696499816465)(388,0.897657537865114)(389,0.897832100676653)(390,0.89782678527717)(391,0.897747741833213)(392,0.897778731650063)(393,0.89793493385191)(394,0.898006378250964)(395,0.898052258548962)(396,0.897847608359694)(397,0.897670765935783)(398,0.898051329785823)(399,0.898157678871445)(400,0.89808453626214)(401,0.898208068123512)(402,0.898166366885353)(403,0.898304922084512)(404,0.898484447899436)(405,0.899069151639118)(406,0.899150198909894)(407,0.899162848755018)(408,0.899299599467651)(409,0.899587088053389)(410,0.899509644068459)(411,0.899614722965379)(412,0.899231762013753)(413,0.899330447712135)(414,0.899622603469486)(415,0.899448846040407)(416,0.899713589410095)(417,0.89970676008646)(418,0.89978058226618)(419,0.899933292180417)(420,0.899620876297905)(421,0.899483152970332)(422,0.899558816593914)(423,0.899843235874174)(424,0.899760658604647)(425,0.899808437822254)(426,0.899915533098587)(427,0.899944268199698)(428,0.899821575467557)(429,0.900010956850296)(430,0.900149120038707)(431,0.900502829211847)(432,0.900837949289552)(433,0.900617116803233)(434,0.900236707451713)(435,0.900262162496491)(436,0.900315240206601)(437,0.900363030152768)(438,0.900469210323377)(439,0.900271432935223)(440,0.900404390241825)(441,0.900030640850881)(442,0.899961185801576)(443,0.900052291523566)(444,0.900104403415678)(445,0.899854724756821)(446,0.900227604988063)(447,0.900120270373915)(448,0.900182112005288)(449,0.900258628757174)(450,0.899972072568175)(451,0.899740181083451)(452,0.899996847932102)(453,0.900216744923499)(454,0.899985171962201)(455,0.900027386114433)(456,0.900117463790876)(457,0.899881983458284)(458,0.899834166129996)(459,0.899561106029835)(460,0.899891319412978)(461,0.899245063643144)(462,0.899066335818916)(463,0.899137158599375)(464,0.899300300051189)(465,0.899351649247134)(466,0.899473747595197)(467,0.899094832773853)(468,0.899133942430166)(469,0.89927353277556)(470,0.899170806273099)(471,0.899368984375073)(472,0.899715044911987)(473,0.900098990079628)(474,0.899989196141663)(475,0.899788227732729)(476,0.899707899473417)(477,0.899618168722195)(478,0.899794523041557)(479,0.900462790713662)(480,0.900424397217152)(481,0.900727801194096)(482,0.900439232018412)(483,0.900712953289604)(484,0.900834185350256)(485,0.900790218928108)(486,0.900614295001561)(487,0.900578044379653)(488,0.900574634474427)(489,0.900695518660356)(490,0.901020057300416)(491,0.901127121161068)(492,0.90083033611786)(493,0.901376936171782)(494,0.90143075846066)(495,0.90156058263186)(496,0.901786824718936)(497,0.901426634727424)(498,0.901628228486151)(499,0.902065982100204)(500,0.902231955603461)(501,0.902290572459501)(502,0.902138569857982)(503,0.902500714609051)(504,0.902766066634375)(505,0.902795670990294)(506,0.902336692086272)(507,0.902275062928381)(508,0.90229322473946)(509,0.902103868276061)(510,0.902125712095977)(511,0.902034495405979)(512,0.902137318960395)(513,0.901907243139582)(514,0.901918184159879)(515,0.901918878271214)(516,0.901826549348278)(517,0.901680697833655)(518,0.901683316368526)(519,0.90150736373766)(520,0.9016599200173)(521,0.901910913288225)(522,0.902038549916329)(523,0.901558799238413)(524,0.901686121611663)(525,0.901468256314563)(526,0.901314430036563)(527,0.901089104783613)(528,0.901263864113018)(529,0.901256561481617)(530,0.901380389710179)(531,0.901369418504379)(532,0.90142406476739)(533,0.901516053104477)(534,0.901473183953659)(535,0.901274615707697)(536,0.901143193010359)(537,0.90136723024555)(538,0.901344705930869)(539,0.901324250151275)(540,0.901407644436537)(541,0.90140269842847)(542,0.901420587336458)(543,0.901621868248814)(544,0.901454432883193)(545,0.901463958845521)(546,0.901462905737503)(547,0.901094531751937)(548,0.900605960049099)(549,0.900792911670881)(550,0.900853914786773)(551,0.90092981891631)(552,0.900743609237195)(553,0.900906559732703)(554,0.901020181882249)(555,0.900950854332279)(556,0.900875995586544)(557,0.90060779771057)(558,0.90056780174116)(559,0.900499999890109)(560,0.900894202698622)(561,0.900830145606716)(562,0.900811894173883)(563,0.900835628669912)(564,0.900995167022325)(565,0.900543268082297)(566,0.900260440244259)(567,0.900379457811686)(568,0.900152836077625)(569,0.900064989899621)(570,0.89990862173802)(571,0.900062542436201)(572,0.900072708033508)(573,0.900383426979451)(574,0.900400022775944)(575,0.900354908712577)(576,0.900335223935472)(577,0.900230212202332)(578,0.900505021093346)(579,0.900761113377068)(580,0.900721975990385)(581,0.900746717796348)(582,0.90103313193588)(583,0.901038907991304)(584,0.901022021130113)(585,0.900738449420243)(586,0.900122756972683)(587,0.899945573162621)(588,0.899860518413832)(589,0.899894784036314)(590,0.899906427787618)(591,0.899984971016089)(592,0.899983272196588)(593,0.900366513874376)(594,0.900407873421129)(595,0.900221229232447)(596,0.899992744890097)(597,0.900239432779329)(598,0.900398656099371)(599,0.900376216490753)(600,0.900584047072352) 
};
\addlegendentry{\istr};

\addplot [
color=blue,
solid,
line width=1.3pt,
]
coordinates{
 (19,0.674453440090695)(19,0.674453440090695)(19,0.674453440090695)(20,0.677699964848883)(20,0.677699964848883)(20,0.677699964848883)(20,0.677699964848883)(20,0.677699964848883)(20,0.677699964848883)(21,0.680977560662371)(21,0.680977560662371)(21,0.680977560662371)(21,0.680977560662371)(21,0.680977560662371)(21,0.680977560662371)(22,0.684288894163283)(22,0.684288894163283)(22,0.684288894163283)(22,0.684288894163283)(22,0.684288894163283)(22,0.684288894163283)(22,0.684288894163283)(22,0.684288894163283)(22,0.684288894163283)(22,0.684288894163283)(22,0.684288894163283)(22,0.684288894163283)(22,0.684288894163283)(22,0.684288894163283)(23,0.68763080056962)(23,0.68763080056962)(23,0.68763080056962)(23,0.68763080056962)(23,0.68763080056962)(23,0.68763080056962)(24,0.690996091380526)(24,0.690996091380526)(24,0.690996091380526)(24,0.690996091380526)(24,0.690996091380526)(24,0.690996091380526)(24,0.690996091380526)(24,0.690996091380526)(24,0.690996091380526)(24,0.690996091380526)(24,0.690996091380526)(24,0.690996091380526)(24,0.690996091380526)(24,0.690996091380526)(24,0.690996091380526)(24,0.690996091380526)(25,0.694383049474102)(25,0.694383049474102)(25,0.694383049474102)(25,0.694383049474102)(25,0.694383049474102)(25,0.694383049474102)(25,0.694383049474102)(25,0.694383049474102)(25,0.694383049474102)(26,0.697787906186677)(26,0.697787906186677)(26,0.697787906186677)(26,0.697787906186677)(26,0.697787906186677)(27,0.701207871177154)(27,0.701207871177154)(27,0.701207871177154)(27,0.701207871177154)(27,0.701207871177154)(27,0.701207871177154)(27,0.701207871177154)(27,0.701207871177154)(27,0.701207871177154)(28,0.704637661525807)(28,0.704637661525807)(28,0.704637661525807)(28,0.704637661525807)(28,0.704637661525807)(28,0.704637661525807)(28,0.704637661525807)(28,0.704637661525807)(29,0.708071392591006)(29,0.708071392591006)(29,0.708071392591006)(29,0.708071392591006)(29,0.708071392591006)(29,0.708071392591006)(30,0.711511494054162)(30,0.711511494054162)(30,0.711511494054162)(30,0.711511494054162)(30,0.711511494054162)(30,0.711511494054162)(30,0.711511494054162)(30,0.711511494054162)(31,0.714965330086416)(31,0.714965330086416)(31,0.714965330086416)(31,0.714965330086416)(31,0.714965330086416)(31,0.714965330086416)(31,0.714965330086416)(31,0.714965330086416)(32,0.718468909638097)(32,0.718468909638097)(32,0.718468909638097)(32,0.718468909638097)(32,0.718468909638097)(32,0.718468909638097)(32,0.718468909638097)(32,0.718468909638097)(32,0.718468909638097)(32,0.718468909638097)(32,0.718468909638097)(33,0.722493522344897)(33,0.722493522344897)(33,0.722493522344897)(33,0.722493522344897)(33,0.722493522344897)(33,0.722493522344897)(33,0.722493522344897)(33,0.722493522344897)(33,0.722493522344897)(33,0.722493522344897)(34,0.726642044711796)(34,0.726642044711796)(34,0.726642044711796)(34,0.726642044711796)(34,0.726642044711796)(34,0.726642044711796)(34,0.726642044711796)(35,0.730823138224753)(35,0.730823138224753)(35,0.730823138224753)(35,0.730823138224753)(35,0.730823138224753)(35,0.730823138224753)(35,0.730823138224753)(35,0.730823138224753)(35,0.730823138224753)(35,0.730823138224753)(36,0.735094829070367)(36,0.735094829070367)(36,0.735094829070367)(36,0.735094829070367)(36,0.735094829070367)(36,0.735094829070367)(36,0.735094829070367)(36,0.735094829070367)(36,0.735094829070367)(36,0.735094829070367)(37,0.739180608091518)(37,0.739180608091518)(37,0.739180608091518)(37,0.739180608091518)(37,0.739180608091518)(37,0.739180608091518)(37,0.739180608091518)(37,0.739180608091518)(37,0.739180608091518)(37,0.739180608091518)(37,0.739180608091518)(37,0.739180608091518)(37,0.739180608091518)(37,0.739180608091518)(37,0.739180608091518)(38,0.743076531017204)(38,0.743076531017204)(38,0.743076531017204)(38,0.743076531017204)(38,0.743076531017204)(38,0.743076531017204)(38,0.743076531017204)(38,0.743076531017204)(38,0.743076531017204)(38,0.743076531017204)(38,0.743076531017204)(38,0.743076531017204)(38,0.743076531017204)(38,0.743076531017204)(39,0.746374724933858)(39,0.746374724933858)(39,0.746374724933858)(39,0.746374724933858)(39,0.746374724933858)(39,0.746374724933858)(39,0.746374724933858)(39,0.746374724933858)(39,0.746374724933858)(39,0.746374724933858)(39,0.746374724933858)(39,0.746374724933858)(40,0.749839870631291)(40,0.749839870631291)(40,0.749839870631291)(40,0.749839870631291)(40,0.749839870631291)(40,0.749839870631291)(40,0.749839870631291)(40,0.749839870631291)(40,0.749839870631291)(40,0.749839870631291)(40,0.749839870631291)(41,0.753211659696352)(41,0.753211659696352)(41,0.753211659696352)(41,0.753211659696352)(41,0.753211659696352)(41,0.753211659696352)(41,0.753211659696352)(41,0.753211659696352)(42,0.756549310910998)(42,0.756549310910998)(42,0.756549310910998)(42,0.756549310910998)(42,0.756549310910998)(42,0.756549310910998)(42,0.756549310910998)(43,0.759898296704403)(43,0.759898296704403)(43,0.759898296704403)(43,0.759898296704403)(43,0.759898296704403)(43,0.759898296704403)(43,0.759898296704403)(43,0.759898296704403)(44,0.763266183038931)(44,0.763266183038931)(44,0.763266183038931)(44,0.763266183038931)(44,0.763266183038931)(44,0.763266183038931)(45,0.766604917105248)(45,0.766604917105248)(45,0.766604917105248)(46,0.76953816259403)(46,0.76953816259403)(46,0.76953816259403)(46,0.76953816259403)(46,0.76953816259403)(46,0.76953816259403)(46,0.76953816259403)(46,0.76953816259403)(47,0.772498993825033)(47,0.772498993825033)(47,0.772498993825033)(47,0.772498993825033)(47,0.772498993825033)(47,0.772498993825033)(48,0.775224890655789)(48,0.775224890655789)(48,0.775224890655789)(48,0.775224890655789)(48,0.775224890655789)(48,0.775224890655789)(48,0.775224890655789)(48,0.775224890655789)(48,0.775224890655789)(49,0.777663804180379)(49,0.777663804180379)(49,0.777663804180379)(49,0.777663804180379)(49,0.777663804180379)(49,0.777663804180379)(50,0.779126061664587)(50,0.779126061664587)(50,0.779126061664587)(50,0.779126061664587)(50,0.779126061664587)(50,0.779126061664587)(50,0.779126061664587)(50,0.779126061664587)(50,0.779126061664587)(51,0.780977298337612)(51,0.780977298337612)(51,0.780977298337612)(52,0.782616660372839)(52,0.782616660372839)(52,0.782616660372839)(52,0.782616660372839)(52,0.782616660372839)(52,0.782616660372839)(53,0.783802109584786)(53,0.783802109584786)(53,0.783802109584786)(53,0.783802109584786)(53,0.783802109584786)(53,0.783802109584786)(53,0.783802109584786)(53,0.783802109584786)(54,0.785385846158879)(54,0.785385846158879)(54,0.785385846158879)(54,0.785385846158879)(54,0.785385846158879)(54,0.785385846158879)(54,0.785385846158879)(54,0.785385846158879)(54,0.785385846158879)(54,0.785385846158879)(55,0.786815824638529)(55,0.786815824638529)(55,0.786815824638529)(56,0.788365852375526)(56,0.788365852375526)(56,0.788365852375526)(56,0.788365852375526)(57,0.789705551733683)(57,0.789705551733683)(57,0.789705551733683)(57,0.789705551733683)(57,0.789705551733683)(57,0.789705551733683)(57,0.789705551733683)(57,0.789705551733683)(57,0.789705551733683)(57,0.789705551733683)(58,0.791120103135555)(58,0.791120103135555)(58,0.791120103135555)(59,0.792260565533962)(59,0.792260565533962)(59,0.792260565533962)(59,0.792260565533962)(60,0.793586758274726)(60,0.793586758274726)(60,0.793586758274726)(60,0.793586758274726)(60,0.793586758274726)(60,0.793586758274726)(60,0.793586758274726)(60,0.793586758274726)(61,0.794761792375211)(61,0.794761792375211)(61,0.794761792375211)(61,0.794761792375211)(61,0.794761792375211)(62,0.796049439375086)(62,0.796049439375086)(62,0.796049439375086)(62,0.796049439375086)(63,0.797214587322273)(63,0.797214587322273)(63,0.797214587322273)(63,0.797214587322273)(63,0.797214587322273)(63,0.797214587322273)(64,0.798371477092781)(64,0.798371477092781)(64,0.798371477092781)(64,0.798371477092781)(64,0.798371477092781)(64,0.798371477092781)(65,0.799569079005914)(65,0.799569079005914)(65,0.799569079005914)(65,0.799569079005914)(65,0.799569079005914)(66,0.800724872334555)(66,0.800724872334555)(66,0.800724872334555)(66,0.800724872334555)(66,0.800724872334555)(66,0.800724872334555)(66,0.800724872334555)(66,0.800724872334555)(67,0.801860768978134)(67,0.801860768978134)(67,0.801860768978134)(67,0.801860768978134)(67,0.801860768978134)(68,0.803052412131144)(68,0.803052412131144)(68,0.803052412131144)(68,0.803052412131144)(68,0.803052412131144)(68,0.803052412131144)(69,0.804232371111876)(69,0.804232371111876)(69,0.804232371111876)(70,0.805432428555284)(70,0.805432428555284)(71,0.806650115034251)(71,0.806650115034251)(71,0.806650115034251)(72,0.808050264450221)(72,0.808050264450221)(72,0.808050264450221)(73,0.809325682111939)(73,0.809325682111939)(74,0.810820828319305)(74,0.810820828319305)(74,0.810820828319305)(75,0.812179667990661)(75,0.812179667990661)(75,0.812179667990661)(75,0.812179667990661)(75,0.812179667990661)(75,0.812179667990661)(75,0.812179667990661)(75,0.812179667990661)(76,0.813756923938875)(76,0.813756923938875)(76,0.813756923938875)(76,0.813756923938875)(76,0.813756923938875)(77,0.815342338507385)(77,0.815342338507385)(77,0.815342338507385)(77,0.815342338507385)(77,0.815342338507385)(78,0.816785028438318)(78,0.816785028438318)(78,0.816785028438318)(78,0.816785028438318)(79,0.818252480333769)(79,0.818252480333769)(79,0.818252480333769)(79,0.818252480333769)(79,0.818252480333769)(79,0.818252480333769)(80,0.81981431857685)(80,0.81981431857685)(80,0.81981431857685)(81,0.821260876285586)(81,0.821260876285586)(81,0.821260876285586)(81,0.821260876285586)(82,0.822628466750489)(82,0.822628466750489)(82,0.822628466750489)(82,0.822628466750489)(83,0.823972489157672)(83,0.823972489157672)(83,0.823972489157672)(83,0.823972489157672)(83,0.823972489157672)(84,0.825289666537978)(84,0.825289666537978)(85,0.826427655710159)(85,0.826427655710159)(85,0.826427655710159)(87,0.82875479009027)(87,0.82875479009027)(88,0.82981517452713)(88,0.82981517452713)(88,0.82981517452713)(88,0.82981517452713)(89,0.830862464154905)(89,0.830862464154905)(89,0.830862464154905)(89,0.830862464154905)(89,0.830862464154905)(89,0.830862464154905)(90,0.832031155042986)(90,0.832031155042986)(90,0.832031155042986)(91,0.833177129235281)(91,0.833177129235281)(91,0.833177129235281)(91,0.833177129235281)(91,0.833177129235281)(91,0.833177129235281)(92,0.834295361515534)(92,0.834295361515534)(92,0.834295361515534)(92,0.834295361515534)(93,0.835203461758739)(93,0.835203461758739)(93,0.835203461758739)(94,0.836250299990985)(94,0.836250299990985)(94,0.836250299990985)(95,0.837079748063787)(95,0.837079748063787)(96,0.838071841864321)(96,0.838071841864321)(96,0.838071841864321)(96,0.838071841864321)(97,0.839042497628649)(98,0.839816473203692)(98,0.839816473203692)(98,0.839816473203692)(99,0.840600930488731)(99,0.840600930488731)(99,0.840600930488731)(100,0.841546850052976)(100,0.841546850052976)(101,0.842345646039647)(101,0.842345646039647)(101,0.842345646039647)(101,0.842345646039647)(102,0.843277973080683)(102,0.843277973080683)(102,0.843277973080683)(103,0.844077378082485)(104,0.844992638978261)(104,0.844992638978261)(105,0.845781020809966)(105,0.845781020809966)(106,0.846548108343828)(106,0.846548108343828)(106,0.846548108343828)(106,0.846548108343828)(106,0.846548108343828)(107,0.847408416788352)(108,0.848111412424177)(108,0.848111412424177)(108,0.848111412424177)(109,0.848912586845812)(109,0.848912586845812)(109,0.848912586845812)(110,0.849534077496212)(110,0.849534077496212)(110,0.849534077496212)(112,0.850824139937532)(113,0.851355889349202)(113,0.851355889349202)(113,0.851355889349202)(113,0.851355889349202)(114,0.85201993156695)(114,0.85201993156695)(114,0.85201993156695)(114,0.85201993156695)(114,0.85201993156695)(115,0.852534023627697)(117,0.853702673515281)(117,0.853702673515281)(118,0.854335051666197)(118,0.854335051666197)(119,0.85489287780009)(119,0.85489287780009)(120,0.855524398368221)(120,0.855524398368221)(120,0.855524398368221)(120,0.855524398368221)(120,0.855524398368221)(120,0.855524398368221)(122,0.856764874238118)(122,0.856764874238118)(123,0.857372774086232)(124,0.857971953656984)(124,0.857971953656984)(124,0.857971953656984)(125,0.858561869780786)(125,0.858561869780786)(125,0.858561869780786)(125,0.858561869780786)(125,0.858561869780786)(126,0.859142060422411)(126,0.859142060422411)(126,0.859142060422411)(126,0.859142060422411)(129,0.860797793983993)(129,0.860797793983993)(129,0.860797793983993)(130,0.861293417011583)(130,0.861293417011583)(130,0.861293417011583)(130,0.861293417011583)(131,0.861800533179164)(133,0.862696227324257)(133,0.862696227324257)(134,0.863142526665702)(134,0.863142526665702)(134,0.863142526665702)(134,0.863142526665702)(135,0.863573696154416)(136,0.863904896526776)(137,0.864309166800621)(137,0.864309166800621)(137,0.864309166800621)(138,0.86470535987699)(138,0.86470535987699)(139,0.865186298309703)(140,0.865571265857905)(141,0.865953048005332)(141,0.865953048005332)(142,0.866247153495957)(142,0.866247153495957)(144,0.866945389189527)(145,0.867337800925585)(145,0.867337800925585)(145,0.867337800925585)(146,0.867733906345426)(147,0.86813432939422)(148,0.868539036553336)(149,0.868902917900531)(149,0.868902917900531)(149,0.868902917900531)(150,0.869317366964179)(150,0.869317366964179)(151,0.869734155410472)(152,0.870152020772529)(153,0.870536101400658)(153,0.870536101400658)(153,0.870536101400658)(156,0.871792051345945)(157,0.872225204178759)(157,0.872225204178759)(158,0.872633113441247)(158,0.872633113441247)(158,0.872633113441247)(159,0.873027064944668)(160,0.873431149310365)(160,0.873431149310365)(160,0.873431149310365)(161,0.87383042815016)(161,0.87383042815016)(162,0.874224829055887)(162,0.874224829055887)(162,0.874224829055887)(163,0.874615185718417)(163,0.874615185718417)(163,0.874615185718417)(164,0.875002946636755)(164,0.875002946636755)(165,0.87538919562711)(165,0.87538919562711)(166,0.875774734444944)(166,0.875774734444944)(168,0.876543654710632)(169,0.87693567191922)(169,0.87693567191922)(170,0.877314663687342)(171,0.877677084686097)(171,0.877677084686097)(171,0.877677084686097)(171,0.877677084686097)(172,0.878030369676711)(173,0.878391720700497)(173,0.878391720700497)(173,0.878391720700497)(174,0.878760667941973)(174,0.878760667941973)(175,0.879095234666651)(175,0.879095234666651)(175,0.879095234666651)(175,0.879095234666651)(176,0.879451550025973)(176,0.879451550025973)(176,0.879451550025973)(176,0.879451550025973)(176,0.879451550025973)(177,0.879787683630161)(177,0.879787683630161)(177,0.879787683630161)(178,0.880104075390894)(179,0.880442243864374)(179,0.880442243864374)(179,0.880442243864374)(180,0.880761780388827)(180,0.880761780388827)(180,0.880761780388827)(181,0.881076698783184)(181,0.881076698783184)(182,0.881386992425141)(182,0.881386992425141)(182,0.881386992425141)(182,0.881386992425141)(183,0.881669397542979)(184,0.881945962340387)(184,0.881945962340387)(185,0.882241746426151)(186,0.882559939664275)(186,0.882559939664275)(187,0.882851025648837)(187,0.882851025648837)(187,0.882851025648837)(187,0.882851025648837)(187,0.882851025648837)(188,0.883139512496243)(189,0.883425842092288)(190,0.88371014196211)(190,0.88371014196211)(190,0.88371014196211)(191,0.883992414396604)(191,0.883992414396604)(191,0.883992414396604)(192,0.884272731705278)(192,0.884272731705278)(192,0.884272731705278)(192,0.884272731705278)(192,0.884272731705278)(192,0.884272731705278)(194,0.884813301894178)(194,0.884813301894178)(195,0.885097574910392)(195,0.885097574910392)(195,0.885097574910392)(196,0.885381424657317)(196,0.885381424657317)(196,0.885381424657317)(197,0.88566522263262)(197,0.88566522263262)(198,0.885951709523519)(199,0.886238943735584)(199,0.886238943735584)(199,0.886238943735584)(200,0.886527138790488)(200,0.886527138790488)(200,0.886527138790488)(200,0.886527138790488)(200,0.886527138790488)(201,0.886816639998952)(201,0.886816639998952)(202,0.887097292777554)(202,0.887097292777554)(202,0.887097292777554)(203,0.887389217337839)(204,0.887684492391921)(204,0.887684492391921)(205,0.88797117333439)(205,0.88797117333439)(205,0.88797117333439)(207,0.88859963110977)(207,0.88859963110977)(207,0.88859963110977)(207,0.88859963110977)(208,0.888904446781207)(209,0.88922768578988)(210,0.889556086709923)(210,0.889556086709923)(210,0.889556086709923)(211,0.889887523564214)(211,0.889887523564214)(211,0.889887523564214)(212,0.890226404608792)(213,0.890574106065839)(213,0.890574106065839)(213,0.890574106065839)(213,0.890574106065839)(214,0.890918564537442)(214,0.890918564537442)(214,0.890918564537442)(216,0.891605008140284)(216,0.891605008140284)(216,0.891605008140284)(217,0.891931994130899)(218,0.892267475213929)(218,0.892267475213929)(218,0.892267475213929)(219,0.892582492569033)(219,0.892582492569033)(219,0.892582492569033)(220,0.892907813159101)(220,0.892907813159101)(220,0.892907813159101)(222,0.893539699882923)(222,0.893539699882923)(222,0.893539699882923)(223,0.893843935174645)(223,0.893843935174645)(223,0.893843935174645)(224,0.894139232272662)(224,0.894139232272662)(225,0.894400527238666)(225,0.894400527238666)(225,0.894400527238666)(226,0.894645077216617)(226,0.894645077216617)(226,0.894645077216617)(226,0.894645077216617)(228,0.895220867059537)(228,0.895220867059537)(228,0.895220867059537)(229,0.895465685945898)(231,0.895925453700809)(231,0.895925453700809)(233,0.896348556128809)(233,0.896348556128809)(233,0.896348556128809)(233,0.896348556128809)(234,0.896547884142375)(234,0.896547884142375)(235,0.896715539285873)(235,0.896715539285873)(236,0.896901767676642)(236,0.896901767676642)(236,0.896901767676642)(236,0.896901767676642)(237,0.897083064705636)(237,0.897083064705636)(238,0.897238892273073)(238,0.897238892273073)(239,0.897437215499967)(239,0.897437215499967)(239,0.897437215499967)(239,0.897437215499967)(241,0.897770493256431)(241,0.897770493256431)(243,0.898128226710772)(243,0.898128226710772)(243,0.898128226710772)(243,0.898128226710772)(244,0.898291222002207)(244,0.898291222002207)(244,0.898291222002207)(245,0.89845500013102)(245,0.89845500013102)(245,0.89845500013102)(245,0.89845500013102)(246,0.89863738892611)(248,0.898984583727742)(249,0.899195120302528)(250,0.899384034479661)(251,0.899574090811934)(252,0.899764751947876)(253,0.899919122402648)(255,0.900294827547968)(255,0.900294827547968)(256,0.900481835400154)(257,0.900625377779575)(257,0.900625377779575)(259,0.900953085350487)(260,0.901137034835918)(260,0.901137034835918)(261,0.901320501755426)(262,0.901503256576544)(262,0.901503256576544)(262,0.901503256576544)(263,0.90164636259753)(263,0.90164636259753)(264,0.90182801983095)(264,0.90182801983095)(264,0.90182801983095)(264,0.90182801983095)(265,0.902045964685469)(267,0.902330490771785)(267,0.902330490771785)(267,0.902330490771785)(267,0.902330490771785)(267,0.902330490771785)(268,0.90250680097039)(268,0.90250680097039)(269,0.902648033485004)(269,0.902648033485004)(269,0.902648033485004)(269,0.902648033485004)(270,0.902820811675143)(270,0.902820811675143)(271,0.9029916413202)(271,0.9029916413202)(272,0.903130372633558)(273,0.903297957289494)(273,0.903297957289494)(275,0.903600938519526)(277,0.903903081455149)(277,0.903903081455149)(279,0.904222101184656)(279,0.904222101184656)(279,0.904222101184656)(280,0.904361863381583)(281,0.904517034227847)(281,0.904517034227847)(282,0.904669407811078)(282,0.904669407811078)(282,0.904669407811078)(282,0.904669407811078)(283,0.90481891953189)(283,0.90481891953189)(287,0.905376217155974)(287,0.905376217155974)(287,0.905376217155974)(287,0.905376217155974)(288,0.905513722656677)(288,0.905513722656677)(289,0.905642896906461)(289,0.905642896906461)(289,0.905642896906461)(289,0.905642896906461)(290,0.905776121339917)(290,0.905776121339917)(290,0.905776121339917)(291,0.90590130490304)(292,0.906034356273475)(292,0.906034356273475)(293,0.906155387814424)(293,0.906155387814424)(293,0.906155387814424)(295,0.906397579210474)(295,0.906397579210474)(295,0.906397579210474)(297,0.906624767601888)(297,0.906624767601888)(297,0.906624767601888)(297,0.906624767601888)(298,0.906735819411332)(299,0.906846161908544)(299,0.906846161908544)(299,0.906846161908544)(301,0.907054587019594)(301,0.907054587019594)(301,0.907054587019594)(301,0.907054587019594)(302,0.907154988785399)(303,0.907247530187911)(303,0.907247530187911)(304,0.907340906020946)(306,0.907495330829499)(307,0.907590696200495)(308,0.907658161213539)(309,0.907736776842831)(309,0.907736776842831)(309,0.907736776842831)(312,0.907966259583901)(312,0.907966259583901)(312,0.907966259583901)(313,0.908042119199341)(313,0.908042119199341)(313,0.908042119199341)(314,0.908108179455613)(314,0.908108179455613)(314,0.908108179455613)(315,0.908186832330368)(317,0.908349737892079)(317,0.908349737892079)(318,0.908432060488023)(319,0.908521221522504)(319,0.908521221522504)(319,0.908521221522504)(320,0.908613068036725)(323,0.908903908841363)(323,0.908903908841363)(324,0.909005585910315)(325,0.909109450623178)(325,0.909109450623178)(325,0.909109450623178)(326,0.909221549868803)(326,0.909221549868803)(326,0.909221549868803)(327,0.909329812405272)(328,0.909446907133233)(329,0.909558874629939)(330,0.909679631911186)(330,0.909679631911186)(330,0.909679631911186)(331,0.909801453000455)(331,0.909801453000455)(333,0.910028224840699)(334,0.91014660432626)(335,0.910265692105217)(335,0.910265692105217)(337,0.910505238912701)(337,0.910505238912701)(338,0.910624402536759)(338,0.910624402536759)(339,0.910742932542849)(339,0.910742932542849)(339,0.910742932542849)(341,0.910979172748598)(342,0.91109660946133)(344,0.911329534259381)(345,0.911444664996699)(346,0.911558630321751)(346,0.911558630321751)(346,0.911558630321751)(350,0.911989628897148)(351,0.912096512197352)(352,0.912191311875011)(352,0.912191311875011)(354,0.912408998034245)(355,0.912500051809796)(356,0.912589768067626)(357,0.912698879118599)(358,0.912786475386432)(358,0.912786475386432)(358,0.912786475386432)(361,0.913056577567082)(362,0.913144257358124)(363,0.913239028736973)(363,0.913239028736973)(363,0.913239028736973)(364,0.913327563789841)(365,0.913421660820816)(367,0.913608049678219)(368,0.91370271650426)(368,0.91370271650426)(369,0.913794011158335)(369,0.913794011158335)(370,0.913884282937795)(372,0.914067460292374)(373,0.914158177771616)(373,0.914158177771616)(374,0.914249511060912)(375,0.914343767274813)(375,0.914343767274813)(375,0.914343767274813)(375,0.914343767274813)(375,0.914343767274813)(376,0.914436739514167)(376,0.914436739514167)(376,0.914436739514167)(377,0.91453496332322)(377,0.91453496332322)(378,0.914635909371292)(378,0.914635909371292)(378,0.914635909371292)(379,0.914725650903957)(381,0.914919205984067)(382,0.915025937520546)(382,0.915025937520546)(383,0.915134471049987)(384,0.915243969102905)(385,0.915344678239337)(386,0.915454569438943)(387,0.915555646197604)(388,0.915665188555637)(389,0.9157664799961)(389,0.9157664799961)(392,0.916079012770618)(392,0.916079012770618)(393,0.916180948442467)(394,0.916290475875759)(394,0.916290475875759)(395,0.91639978066366)(396,0.916508968920624)(396,0.916508968920624)(397,0.916604893776469)(397,0.916604893776469)(398,0.916714698350151)(398,0.916714698350151)(399,0.916817838310901)(400,0.916921217467868)(401,0.917032354914969)(402,0.917136685618921)(402,0.917136685618921)(402,0.917136685618921)(402,0.917136685618921)(404,0.917338814702703)(404,0.917338814702703)(405,0.917452944424764)(406,0.917568163179328)(406,0.917568163179328)(406,0.917568163179328)(407,0.917675854956944)(409,0.917893810021448)(409,0.917893810021448)(411,0.918115501751803)(412,0.918219376308471)(413,0.918341127868253)(414,0.918455474214085)(415,0.918562788564974)(416,0.918679376810133)(417,0.918796998095607)(417,0.918796998095607)(418,0.918915632318948)(420,0.919171086727122)(420,0.919171086727122)(421,0.919292503910603)(424,0.919662646292541)(425,0.919787836571833)(426,0.919921973085124)(427,0.920048671843864)(428,0.92016809388871)(429,0.920296229388024)(430,0.92042492199746)(431,0.920554083628267)(431,0.920554083628267)(435,0.921089560692496)(436,0.921222548179397)(436,0.921222548179397)(437,0.921354020340326)(437,0.921354020340326)(438,0.921483915969196)(438,0.921483915969196)(439,0.92161035359106)(439,0.92161035359106)(441,0.921862702524999)(441,0.921862702524999)(441,0.921862702524999)(441,0.921862702524999)(442,0.921989401042533)(442,0.921989401042533)(444,0.922240770122475)(445,0.922366120056453)(445,0.922366120056453)(445,0.922366120056453)(449,0.922866851658591)(450,0.922991783163993)(450,0.922991783163993)(451,0.923115711138648)(453,0.923362673243379)(455,0.923609069502915)(456,0.923731443246755)(457,0.923850002364664)(458,0.923974177626871)(459,0.924094434176189)(460,0.924213858723586)(460,0.924213858723586)(464,0.924681982748506)(464,0.924681982748506)(464,0.924681982748506)(464,0.924681982748506)(465,0.924802396341329)(465,0.924802396341329)(465,0.924802396341329)(465,0.924802396341329)(466,0.924922040359104)(468,0.925132463206388)(468,0.925132463206388)(468,0.925132463206388)(468,0.925132463206388)(473,0.925667138764828)(477,0.926053737526306)(479,0.926205871567194)(480,0.926275190166037)(482,0.926401152787408)(483,0.926458446716797)(484,0.926512453868997)(484,0.926512453868997)(485,0.926563562673289)(488,0.926703127985197)(488,0.926703127985197)(490,0.926787821224272)(490,0.926787821224272)(490,0.926787821224272)(492,0.926868137249751)(492,0.926868137249751)(492,0.926868137249751)(496,0.927021019409395)(499,0.927132345531125)(499,0.927132345531125)(501,0.927205977339676)(501,0.927205977339676)(502,0.927242737043756)(503,0.92727949606933)(503,0.92727949606933)(504,0.927316275081449)(505,0.927353089375235)(506,0.927389949792399)(507,0.927426864728638)(507,0.927426864728638)(507,0.927426864728638)(511,0.927575166662085)(511,0.927575166662085)(512,0.927612412044907)(512,0.927612412044907)(512,0.927612412044907)(512,0.927612412044907)(514,0.927687096649818)(515,0.927724528810012)(516,0.927762014731477)(516,0.927762014731477)(518,0.927837125744072)(518,0.927837125744072)(519,0.92787473770111)(519,0.92787473770111)(519,0.92787473770111)(519,0.92787473770111)(520,0.927912377526157)(520,0.927912377526157)(521,0.927950037981431)(522,0.927987710670681)(523,0.928025386806389)(523,0.928025386806389)(526,0.928138335757126)(527,0.928175922527511)(529,0.928250938704609)(530,0.928288348541859)(531,0.928325679634336)(531,0.928325679634336)(532,0.928362921980142)(533,0.928400065737625)(534,0.928437101425221)(534,0.928437101425221)(535,0.928474020866402)(536,0.92851081597998)(538,0.928583999699801)(542,0.928728519368405)(543,0.928764220154498)(549,0.928974313223259)(551,0.929042631243637)(551,0.929042631243637)(552,0.929076439218097)(553,0.929110001394714)(557,0.9292415902125)(557,0.9292415902125)(558,0.929273772703137)(559,0.929305649089847)(562,0.929399343491966)(562,0.929399343491966)(566,0.929519435404618)(569,0.929605625822789)(570,0.929633583553051)(571,0.929661146471326)(572,0.929688310624041)(573,0.929715074344636)(574,0.929741435463086)(576,0.92979293459324)(576,0.92979293459324)(576,0.92979293459324)(576,0.92979293459324)(577,0.929818063843923)(578,0.929842773956752)(578,0.929842773956752)(580,0.929890919249697)(581,0.929914345119828)(581,0.929914345119828)(582,0.929937334100292)(586,0.930024862151809)(587,0.930045620855535)(587,0.930045620855535)(590,0.930105154130997)(590,0.930105154130997)(591,0.930124079825509)(594,0.930178102820392)(595,0.93019519320586)(595,0.93019519320586)(595,0.93019519320586) 
};
\addlegendentry{\iacl};

\addplot [
color=green!50!black,
solid,
line width=1.3pt,
]
coordinates{
 (30,0.713547123924086)(30,0.713547123924086)(30,0.713547123924086)(30,0.713547123924086)(30,0.713547123924086)(30,0.713547123924086)(30,0.713547123924086)(30,0.713547123924086)(30,0.713547123924086)(30,0.713547123924086)(30,0.713547123924086)(30,0.713547123924086)(30,0.713547123924086)(30,0.713547123924086)(30,0.713547123924086)(30,0.713547123924086)(30,0.713547123924086)(30,0.713547123924086)(30,0.713547123924086)(30,0.713547123924086)(30,0.713547123924086)(30,0.713547123924086)(30,0.713547123924086)(30,0.713547123924086)(30,0.713547123924086)(30,0.713547123924086)(30,0.713547123924086)(30,0.713547123924086)(30,0.713547123924086)(30,0.713547123924086)(30,0.713547123924086)(30,0.713547123924086)(30,0.713547123924086)(30,0.713547123924086)(30,0.713547123924086)(30,0.713547123924086)(30,0.713547123924086)(30,0.713547123924086)(30,0.713547123924086)(30,0.713547123924086)(30,0.713547123924086)(30,0.713547123924086)(30,0.713547123924086)(30,0.713547123924086)(30,0.713547123924086)(30,0.713547123924086)(30,0.713547123924086)(30,0.713547123924086)(30,0.713547123924086)(30,0.713547123924086)(30,0.713547123924086)(30,0.713547123924086)(30,0.713547123924086)(30,0.713547123924086)(30,0.713547123924086)(30,0.713547123924086)(30,0.713547123924086)(30,0.713547123924086)(30,0.713547123924086)(30,0.713547123924086)(30,0.713547123924086)(30,0.713547123924086)(30,0.713547123924086)(30,0.713547123924086)(30,0.713547123924086)(30,0.713547123924086)(30,0.713547123924086)(30,0.713547123924086)(30,0.713547123924086)(30,0.713547123924086)(30,0.713547123924086)(30,0.713547123924086)(30,0.713547123924086)(30,0.713547123924086)(30,0.713547123924086)(30,0.713547123924086)(30,0.713547123924086)(30,0.713547123924086)(30,0.713547123924086)(30,0.713547123924086)(30,0.713547123924086)(30,0.713547123924086)(30,0.713547123924086)(30,0.713547123924086)(30,0.713547123924086)(30,0.713547123924086)(30,0.713547123924086)(30,0.713547123924086)(30,0.713547123924086)(30,0.713547123924086)(30,0.713547123924086)(30,0.713547123924086)(30,0.713547123924086)(30,0.713547123924086)(30,0.713547123924086)(30,0.713547123924086)(30,0.713547123924086)(30,0.713547123924086)(30,0.713547123924086)(30,0.713547123924086)(30,0.713547123924086)(30,0.713547123924086)(30,0.713547123924086)(30,0.713547123924086)(30,0.713547123924086)(30,0.713547123924086)(30,0.713547123924086)(30,0.713547123924086)(30,0.713547123924086)(30,0.713547123924086)(30,0.713547123924086)(30,0.713547123924086)(30,0.713547123924086)(30,0.713547123924086)(30,0.713547123924086)(30,0.713547123924086)(30,0.713547123924086)(30,0.713547123924086)(30,0.713547123924086)(30,0.713547123924086)(30,0.713547123924086)(30,0.713547123924086)(30,0.713547123924086)(30,0.713547123924086)(30,0.713547123924086)(30,0.713547123924086)(30,0.713547123924086)(30,0.713547123924086)(30,0.713547123924086)(30,0.713547123924086)(30,0.713547123924086)(30,0.713547123924086)(30,0.713547123924086)(30,0.713547123924086)(30,0.713547123924086)(30,0.713547123924086)(30,0.713547123924086)(30,0.713547123924086)(30,0.713547123924086)(30,0.713547123924086)(30,0.713547123924086)(30,0.713547123924086)(30,0.713547123924086)(30,0.713547123924086)(30,0.713547123924086)(30,0.713547123924086)(30,0.713547123924086)(30,0.713547123924086)(30,0.713547123924086)(30,0.713547123924086)(30,0.713547123924086)(30,0.713547123924086)(30,0.713547123924086)(30,0.713547123924086)(30,0.713547123924086)(30,0.713547123924086)(30,0.713547123924086)(30,0.713547123924086)(30,0.713547123924086)(30,0.713547123924086)(30,0.713547123924086)(30,0.713547123924086)(30,0.713547123924086)(30,0.713547123924086)(30,0.713547123924086)(30,0.713547123924086)(30,0.713547123924086)(30,0.713547123924086)(30,0.713547123924086)(30,0.713547123924086)(30,0.713547123924086)(30,0.713547123924086)(30,0.713547123924086)(30,0.713547123924086)(30,0.713547123924086)(30,0.713547123924086)(30,0.713547123924086)(30,0.713547123924086)(30,0.713547123924086)(30,0.713547123924086)(30,0.713547123924086)(30,0.713547123924086)(30,0.713547123924086)(30,0.713547123924086)(30,0.713547123924086)(30,0.713547123924086)(30,0.713547123924086)(30,0.713547123924086)(30,0.713547123924086)(30,0.713547123924086)(30,0.713547123924086)(30,0.713547123924086)(30,0.713547123924086)(30,0.713547123924086)(30,0.713547123924086)(30,0.713547123924086)(30,0.713547123924086)(30,0.713547123924086)(30,0.713547123924086)(30,0.713547123924086)(30,0.713547123924086)(30,0.713547123924086)(30,0.713547123924086)(30,0.713547123924086)(30,0.713547123924086)(30,0.713547123924086)(30,0.713547123924086)(30,0.713547123924086)(30,0.713547123924086)(30,0.713547123924086)(30,0.713547123924086)(30,0.713547123924086)(60,0.777176631484691)(60,0.777176631484691)(60,0.777176631484691)(60,0.777176631484691)(60,0.777176631484691)(60,0.777176631484691)(60,0.777176631484691)(60,0.777176631484691)(60,0.777176631484691)(60,0.777176631484691)(60,0.777176631484691)(60,0.777176631484691)(60,0.777176631484691)(60,0.777176631484691)(60,0.777176631484691)(60,0.777176631484691)(60,0.777176631484691)(60,0.777176631484691)(60,0.777176631484691)(60,0.777176631484691)(60,0.777176631484691)(60,0.777176631484691)(60,0.777176631484691)(60,0.777176631484691)(60,0.777176631484691)(60,0.777176631484691)(60,0.777176631484691)(60,0.777176631484691)(60,0.777176631484691)(60,0.777176631484691)(60,0.777176631484691)(60,0.777176631484691)(60,0.777176631484691)(60,0.777176631484691)(60,0.777176631484691)(60,0.777176631484691)(60,0.777176631484691)(60,0.777176631484691)(60,0.777176631484691)(60,0.777176631484691)(60,0.777176631484691)(60,0.777176631484691)(60,0.777176631484691)(60,0.777176631484691)(60,0.777176631484691)(60,0.777176631484691)(60,0.777176631484691)(60,0.777176631484691)(60,0.777176631484691)(60,0.777176631484691)(60,0.777176631484691)(60,0.777176631484691)(60,0.777176631484691)(60,0.777176631484691)(60,0.777176631484691)(60,0.777176631484691)(60,0.777176631484691)(60,0.777176631484691)(60,0.777176631484691)(60,0.777176631484691)(60,0.777176631484691)(60,0.777176631484691)(60,0.777176631484691)(60,0.777176631484691)(60,0.777176631484691)(60,0.777176631484691)(60,0.777176631484691)(60,0.777176631484691)(60,0.777176631484691)(60,0.777176631484691)(60,0.777176631484691)(60,0.777176631484691)(60,0.777176631484691)(60,0.777176631484691)(60,0.777176631484691)(60,0.777176631484691)(60,0.777176631484691)(60,0.777176631484691)(60,0.777176631484691)(60,0.777176631484691)(60,0.777176631484691)(60,0.777176631484691)(60,0.777176631484691)(60,0.777176631484691)(60,0.777176631484691)(60,0.777176631484691)(60,0.777176631484691)(60,0.777176631484691)(60,0.777176631484691)(60,0.777176631484691)(60,0.777176631484691)(60,0.777176631484691)(60,0.777176631484691)(60,0.777176631484691)(60,0.777176631484691)(60,0.777176631484691)(60,0.777176631484691)(60,0.777176631484691)(60,0.777176631484691)(60,0.777176631484691)(60,0.777176631484691)(60,0.777176631484691)(60,0.777176631484691)(60,0.777176631484691)(60,0.777176631484691)(60,0.777176631484691)(60,0.777176631484691)(60,0.777176631484691)(60,0.777176631484691)(60,0.777176631484691)(60,0.777176631484691)(60,0.777176631484691)(60,0.777176631484691)(60,0.777176631484691)(60,0.777176631484691)(60,0.777176631484691)(60,0.777176631484691)(60,0.777176631484691)(60,0.777176631484691)(60,0.777176631484691)(60,0.777176631484691)(60,0.777176631484691)(60,0.777176631484691)(60,0.777176631484691)(60,0.777176631484691)(60,0.777176631484691)(60,0.777176631484691)(60,0.777176631484691)(60,0.777176631484691)(60,0.777176631484691)(60,0.777176631484691)(60,0.777176631484691)(60,0.777176631484691)(60,0.777176631484691)(60,0.777176631484691)(60,0.777176631484691)(60,0.777176631484691)(60,0.777176631484691)(60,0.777176631484691)(60,0.777176631484691)(60,0.777176631484691)(60,0.777176631484691)(60,0.777176631484691)(60,0.777176631484691)(60,0.777176631484691)(60,0.777176631484691)(60,0.777176631484691)(60,0.777176631484691)(60,0.777176631484691)(60,0.777176631484691)(60,0.777176631484691)(60,0.777176631484691)(60,0.777176631484691)(60,0.777176631484691)(60,0.777176631484691)(60,0.777176631484691)(60,0.777176631484691)(60,0.777176631484691)(60,0.777176631484691)(60,0.777176631484691)(60,0.777176631484691)(60,0.777176631484691)(60,0.777176631484691)(60,0.777176631484691)(60,0.777176631484691)(60,0.777176631484691)(60,0.777176631484691)(60,0.777176631484691)(90,0.820638980854524)(90,0.820638980854524)(90,0.820638980854524)(90,0.820638980854524)(90,0.820638980854524)(90,0.820638980854524)(90,0.820638980854524)(90,0.820638980854524)(90,0.820638980854524)(90,0.820638980854524)(90,0.820638980854524)(90,0.820638980854524)(90,0.820638980854524)(90,0.820638980854524)(90,0.820638980854524)(90,0.820638980854524)(90,0.820638980854524)(90,0.820638980854524)(90,0.820638980854524)(90,0.820638980854524)(90,0.820638980854524)(90,0.820638980854524)(90,0.820638980854524)(90,0.820638980854524)(90,0.820638980854524)(90,0.820638980854524)(90,0.820638980854524)(90,0.820638980854524)(90,0.820638980854524)(90,0.820638980854524)(90,0.820638980854524)(90,0.820638980854524)(90,0.820638980854524)(90,0.820638980854524)(90,0.820638980854524)(90,0.820638980854524)(90,0.820638980854524)(90,0.820638980854524)(90,0.820638980854524)(90,0.820638980854524)(90,0.820638980854524)(90,0.820638980854524)(90,0.820638980854524)(90,0.820638980854524)(90,0.820638980854524)(90,0.820638980854524)(90,0.820638980854524)(90,0.820638980854524)(90,0.820638980854524)(90,0.820638980854524)(90,0.820638980854524)(90,0.820638980854524)(90,0.820638980854524)(90,0.820638980854524)(90,0.820638980854524)(90,0.820638980854524)(90,0.820638980854524)(90,0.820638980854524)(90,0.820638980854524)(90,0.820638980854524)(90,0.820638980854524)(90,0.820638980854524)(90,0.820638980854524)(90,0.820638980854524)(90,0.820638980854524)(90,0.820638980854524)(90,0.820638980854524)(90,0.820638980854524)(90,0.820638980854524)(90,0.820638980854524)(90,0.820638980854524)(90,0.820638980854524)(90,0.820638980854524)(90,0.820638980854524)(90,0.820638980854524)(90,0.820638980854524)(90,0.820638980854524)(90,0.820638980854524)(90,0.820638980854524)(90,0.820638980854524)(90,0.820638980854524)(90,0.820638980854524)(90,0.820638980854524)(90,0.820638980854524)(90,0.820638980854524)(90,0.820638980854524)(90,0.820638980854524)(90,0.820638980854524)(90,0.820638980854524)(90,0.820638980854524)(90,0.820638980854524)(90,0.820638980854524)(90,0.820638980854524)(90,0.820638980854524)(90,0.820638980854524)(90,0.820638980854524)(90,0.820638980854524)(90,0.820638980854524)(90,0.820638980854524)(90,0.820638980854524)(120,0.841183504468059)(120,0.841183504468059)(120,0.841183504468059)(120,0.841183504468059)(120,0.841183504468059)(120,0.841183504468059)(120,0.841183504468059)(120,0.841183504468059)(120,0.841183504468059)(120,0.841183504468059)(120,0.841183504468059)(120,0.841183504468059)(120,0.841183504468059)(120,0.841183504468059)(120,0.841183504468059)(120,0.841183504468059)(120,0.841183504468059)(120,0.841183504468059)(120,0.841183504468059)(120,0.841183504468059)(120,0.841183504468059)(120,0.841183504468059)(120,0.841183504468059)(120,0.841183504468059)(120,0.841183504468059)(120,0.841183504468059)(120,0.841183504468059)(120,0.841183504468059)(120,0.841183504468059)(120,0.841183504468059)(120,0.841183504468059)(120,0.841183504468059)(120,0.841183504468059)(120,0.841183504468059)(120,0.841183504468059)(120,0.841183504468059)(120,0.841183504468059)(120,0.841183504468059)(120,0.841183504468059)(120,0.841183504468059)(120,0.841183504468059)(120,0.841183504468059)(120,0.841183504468059)(120,0.841183504468059)(120,0.841183504468059)(120,0.841183504468059)(120,0.841183504468059)(120,0.841183504468059)(120,0.841183504468059)(120,0.841183504468059)(120,0.841183504468059)(120,0.841183504468059)(120,0.841183504468059)(120,0.841183504468059)(120,0.841183504468059)(120,0.841183504468059)(120,0.841183504468059)(120,0.841183504468059)(120,0.841183504468059)(120,0.841183504468059)(120,0.841183504468059)(120,0.841183504468059)(120,0.841183504468059)(121,0.841869870602578)(150,0.858675571752741)(150,0.858675571752741)(150,0.858675571752741)(150,0.858675571752741)(150,0.858675571752741)(150,0.858675571752741)(150,0.858675571752741)(150,0.858675571752741)(150,0.858675571752741)(150,0.858675571752741)(150,0.858675571752741)(150,0.858675571752741)(150,0.858675571752741)(150,0.858675571752741)(150,0.858675571752741)(150,0.858675571752741)(150,0.858675571752741)(150,0.858675571752741)(150,0.858675571752741)(150,0.858675571752741)(150,0.858675571752741)(150,0.858675571752741)(150,0.858675571752741)(150,0.858675571752741)(150,0.858675571752741)(150,0.858675571752741)(150,0.858675571752741)(150,0.858675571752741)(150,0.858675571752741)(150,0.858675571752741)(150,0.858675571752741)(150,0.858675571752741)(150,0.858675571752741)(150,0.858675571752741)(150,0.858675571752741)(150,0.858675571752741)(150,0.858675571752741)(150,0.858675571752741)(150,0.858675571752741)(150,0.858675571752741)(150,0.858675571752741)(150,0.858675571752741)(150,0.858675571752741)(150,0.858675571752741)(180,0.870195941901372)(180,0.870195941901372)(180,0.870195941901372)(180,0.870195941901372)(180,0.870195941901372)(180,0.870195941901372)(180,0.870195941901372)(180,0.870195941901372)(180,0.870195941901372)(180,0.870195941901372)(180,0.870195941901372)(180,0.870195941901372)(180,0.870195941901372)(180,0.870195941901372)(180,0.870195941901372)(180,0.870195941901372)(180,0.870195941901372)(180,0.870195941901372)(180,0.870195941901372)(180,0.870195941901372)(180,0.870195941901372)(180,0.870195941901372)(180,0.870195941901372)(180,0.870195941901372)(180,0.870195941901372)(180,0.870195941901372)(180,0.870195941901372)(180,0.870195941901372)(180,0.870195941901372)(180,0.870195941901372)(180,0.870195941901372)(180,0.870195941901372)(180,0.870195941901372)(180,0.870195941901372)(180,0.870195941901372)(180,0.870195941901372)(180,0.870195941901372)(180,0.870195941901372)(180,0.870195941901372)(180,0.870195941901372)(180,0.870195941901372)(180,0.870195941901372)(180,0.870195941901372)(180,0.870195941901372)(183,0.870809227000777)(210,0.877475515736301)(210,0.877475515736301)(210,0.877475515736301)(210,0.877475515736301)(210,0.877475515736301)(210,0.877475515736301)(210,0.877475515736301)(210,0.877475515736301)(210,0.877475515736301)(210,0.877475515736301)(210,0.877475515736301)(210,0.877475515736301)(210,0.877475515736301)(210,0.877475515736301)(210,0.877475515736301)(210,0.877475515736301)(210,0.877475515736301)(210,0.877475515736301)(210,0.877475515736301)(210,0.877475515736301)(210,0.877475515736301)(210,0.877475515736301)(210,0.877475515736301)(210,0.877475515736301)(210,0.877475515736301)(210,0.877475515736301)(210,0.877475515736301)(210,0.877475515736301)(210,0.877475515736301)(210,0.877475515736301)(210,0.877475515736301)(240,0.8862161737606)(240,0.8862161737606)(240,0.8862161737606)(240,0.8862161737606)(240,0.8862161737606)(240,0.8862161737606)(240,0.8862161737606)(240,0.8862161737606)(240,0.8862161737606)(240,0.8862161737606)(240,0.8862161737606)(240,0.8862161737606)(240,0.8862161737606)(240,0.8862161737606)(240,0.8862161737606)(240,0.8862161737606)(240,0.8862161737606)(240,0.8862161737606)(240,0.8862161737606)(240,0.8862161737606)(240,0.8862161737606)(240,0.8862161737606)(240,0.8862161737606)(240,0.8862161737606)(240,0.8862161737606)(240,0.8862161737606)(240,0.8862161737606)(240,0.8862161737606)(240,0.8862161737606)(240,0.8862161737606)(240,0.8862161737606)(240,0.8862161737606)(240,0.8862161737606)(240,0.8862161737606)(240,0.8862161737606)(240,0.8862161737606)(240,0.8862161737606)(240,0.8862161737606)(240,0.8862161737606)(240,0.8862161737606)(240,0.8862161737606)(240,0.8862161737606)(240,0.8862161737606)(240,0.8862161737606)(240,0.8862161737606)(240,0.8862161737606)(240,0.8862161737606)(240,0.8862161737606)(240,0.8862161737606)(240,0.8862161737606)(240,0.8862161737606)(240,0.8862161737606)(240,0.8862161737606)(240,0.8862161737606)(240,0.8862161737606)(240,0.8862161737606)(240,0.8862161737606)(240,0.8862161737606)(240,0.8862161737606)(240,0.8862161737606)(240,0.8862161737606)(240,0.8862161737606)(240,0.8862161737606)(240,0.8862161737606)(240,0.8862161737606)(240,0.8862161737606)(240,0.8862161737606)(240,0.8862161737606)(240,0.8862161737606)(270,0.896725493090125)(270,0.896725493090125)(270,0.896725493090125)(270,0.896725493090125)(270,0.896725493090125)(270,0.896725493090125)(270,0.896725493090125)(270,0.896725493090125)(270,0.896725493090125)(270,0.896725493090125)(270,0.896725493090125)(270,0.896725493090125)(270,0.896725493090125)(270,0.896725493090125)(270,0.896725493090125)(270,0.896725493090125)(270,0.896725493090125)(270,0.896725493090125)(270,0.896725493090125)(270,0.896725493090125)(270,0.896725493090125)(270,0.896725493090125)(270,0.896725493090125)(270,0.896725493090125)(270,0.896725493090125)(270,0.896725493090125)(270,0.896725493090125)(270,0.896725493090125)(270,0.896725493090125)(270,0.896725493090125)(270,0.896725493090125)(270,0.896725493090125)(270,0.896725493090125)(270,0.896725493090125)(270,0.896725493090125)(270,0.896725493090125)(270,0.896725493090125)(270,0.896725493090125)(270,0.896725493090125)(270,0.896725493090125)(270,0.896725493090125)(270,0.896725493090125)(270,0.896725493090125)(270,0.896725493090125)(270,0.896725493090125)(270,0.896725493090125)(270,0.896725493090125)(270,0.896725493090125)(270,0.896725493090125)(270,0.896725493090125)(270,0.896725493090125)(270,0.896725493090125)(270,0.896725493090125)(270,0.896725493090125)(270,0.896725493090125)(270,0.896725493090125)(270,0.896725493090125)(270,0.896725493090125)(270,0.896725493090125)(270,0.896725493090125)(270,0.896725493090125)(270,0.896725493090125)(270,0.896725493090125)(270,0.896725493090125)(270,0.896725493090125)(270,0.896725493090125)(270,0.896725493090125)(276,0.898746111883518)(300,0.901613640276182)(300,0.901613640276182)(300,0.901613640276182)(300,0.901613640276182)(300,0.901613640276182)(300,0.901613640276182)(300,0.901613640276182)(300,0.901613640276182)(300,0.901613640276182)(300,0.901613640276182)(300,0.901613640276182)(300,0.901613640276182)(300,0.901613640276182)(300,0.901613640276182)(300,0.901613640276182)(300,0.901613640276182)(300,0.901613640276182)(300,0.901613640276182)(300,0.901613640276182)(300,0.901613640276182)(300,0.901613640276182)(300,0.901613640276182)(300,0.901613640276182)(300,0.901613640276182)(300,0.901613640276182)(300,0.901613640276182)(300,0.901613640276182)(300,0.901613640276182)(300,0.901613640276182)(300,0.901613640276182)(300,0.901613640276182)(300,0.901613640276182)(300,0.901613640276182)(300,0.901613640276182)(300,0.901613640276182)(300,0.901613640276182)(300,0.901613640276182)(300,0.901613640276182)(300,0.901613640276182)(300,0.901613640276182)(300,0.901613640276182)(300,0.901613640276182)(300,0.901613640276182)(300,0.901613640276182)(300,0.901613640276182)(300,0.901613640276182)(300,0.901613640276182)(300,0.901613640276182)(300,0.901613640276182)(300,0.901613640276182)(300,0.901613640276182)(300,0.901613640276182)(300,0.901613640276182)(300,0.901613640276182)(300,0.901613640276182)(300,0.901613640276182)(300,0.901613640276182)(300,0.901613640276182)(300,0.901613640276182)(300,0.901613640276182)(300,0.901613640276182)(300,0.901613640276182)(300,0.901613640276182)(330,0.901938531294537)(330,0.901938531294537)(330,0.901938531294537)(330,0.901938531294537)(330,0.901938531294537)(330,0.901938531294537)(330,0.901938531294537)(330,0.901938531294537)(330,0.901938531294537)(330,0.901938531294537)(330,0.901938531294537)(330,0.901938531294537)(330,0.901938531294537)(330,0.901938531294537)(330,0.901938531294537)(330,0.901938531294537)(330,0.901938531294537)(330,0.901938531294537)(330,0.901938531294537)(330,0.901938531294537)(330,0.901938531294537)(330,0.901938531294537)(330,0.901938531294537)(330,0.901938531294537)(330,0.901938531294537)(330,0.901938531294537)(330,0.901938531294537)(330,0.901938531294537)(330,0.901938531294537)(330,0.901938531294537)(330,0.901938531294537)(330,0.901938531294537)(330,0.901938531294537)(330,0.901938531294537)(330,0.901938531294537)(330,0.901938531294537)(330,0.901938531294537)(330,0.901938531294537)(330,0.901938531294537)(330,0.901938531294537)(330,0.901938531294537)(330,0.901938531294537)(330,0.901938531294537)(330,0.901938531294537)(330,0.901938531294537)(330,0.901938531294537)(330,0.901938531294537)(330,0.901938531294537)(330,0.901938531294537)(330,0.901938531294537)(331,0.901973891594738)(333,0.902103229640601)(334,0.902205026281043)(344,0.903585147708995)(360,0.905371775878201)(360,0.905371775878201)(360,0.905371775878201)(360,0.905371775878201)(360,0.905371775878201)(360,0.905371775878201)(360,0.905371775878201)(360,0.905371775878201)(360,0.905371775878201)(360,0.905371775878201)(360,0.905371775878201)(360,0.905371775878201)(360,0.905371775878201)(360,0.905371775878201)(360,0.905371775878201)(360,0.905371775878201)(360,0.905371775878201)(360,0.905371775878201)(360,0.905371775878201)(360,0.905371775878201)(360,0.905371775878201)(360,0.905371775878201)(360,0.905371775878201)(360,0.905371775878201)(360,0.905371775878201)(360,0.905371775878201)(360,0.905371775878201)(360,0.905371775878201)(360,0.905371775878201)(360,0.905371775878201)(360,0.905371775878201)(360,0.905371775878201)(360,0.905371775878201)(360,0.905371775878201)(360,0.905371775878201)(360,0.905371775878201)(360,0.905371775878201)(361,0.905449451238246)(365,0.905760960720121)(390,0.908051076614414)(390,0.908051076614414)(390,0.908051076614414)(390,0.908051076614414)(390,0.908051076614414)(390,0.908051076614414)(390,0.908051076614414)(390,0.908051076614414)(390,0.908051076614414)(390,0.908051076614414)(390,0.908051076614414)(390,0.908051076614414)(390,0.908051076614414)(390,0.908051076614414)(390,0.908051076614414)(390,0.908051076614414)(390,0.908051076614414)(390,0.908051076614414)(390,0.908051076614414)(390,0.908051076614414)(390,0.908051076614414)(390,0.908051076614414)(390,0.908051076614414)(390,0.908051076614414)(390,0.908051076614414)(390,0.908051076614414)(390,0.908051076614414)(390,0.908051076614414)(390,0.908051076614414)(390,0.908051076614414)(390,0.908051076614414)(390,0.908051076614414)(390,0.908051076614414)(390,0.908051076614414)(390,0.908051076614414)(390,0.908051076614414)(390,0.908051076614414)(390,0.908051076614414)(390,0.908051076614414)(390,0.908051076614414)(390,0.908051076614414)(420,0.911889440103302)(420,0.911889440103302)(420,0.911889440103302)(420,0.911889440103302)(420,0.911889440103302)(420,0.911889440103302)(420,0.911889440103302)(420,0.911889440103302)(420,0.911889440103302)(420,0.911889440103302)(420,0.911889440103302)(420,0.911889440103302)(420,0.911889440103302)(420,0.911889440103302)(420,0.911889440103302)(420,0.911889440103302)(420,0.911889440103302)(420,0.911889440103302)(420,0.911889440103302)(420,0.911889440103302)(420,0.911889440103302)(420,0.911889440103302)(420,0.911889440103302)(420,0.911889440103302)(420,0.911889440103302)(420,0.911889440103302)(420,0.911889440103302)(420,0.911889440103302)(420,0.911889440103302)(420,0.911889440103302)(420,0.911889440103302)(424,0.912500948311094)(438,0.914301785181643)(450,0.915182924238946)(450,0.915182924238946)(450,0.915182924238946)(450,0.915182924238946)(450,0.915182924238946)(450,0.915182924238946)(450,0.915182924238946)(450,0.915182924238946)(450,0.915182924238946)(450,0.915182924238946)(450,0.915182924238946)(450,0.915182924238946)(450,0.915182924238946)(450,0.915182924238946)(450,0.915182924238946)(450,0.915182924238946)(450,0.915182924238946)(450,0.915182924238946)(450,0.915182924238946)(450,0.915182924238946)(450,0.915182924238946)(450,0.915182924238946)(450,0.915182924238946)(450,0.915182924238946)(450,0.915182924238946)(450,0.915182924238946)(450,0.915182924238946)(450,0.915182924238946)(450,0.915182924238946)(450,0.915182924238946)(450,0.915182924238946)(451,0.915285336819475)(452,0.915386014205949)(453,0.91548488615177)(454,0.91558192797045)(460,0.9161295900006)(462,0.916300867825127)(469,0.916662222525509)(480,0.917213716213425)(480,0.917213716213425)(480,0.917213716213425)(480,0.917213716213425)(480,0.917213716213425)(480,0.917213716213425)(480,0.917213716213425)(480,0.917213716213425)(480,0.917213716213425)(480,0.917213716213425)(480,0.917213716213425)(480,0.917213716213425)(480,0.917213716213425)(480,0.917213716213425)(480,0.917213716213425)(480,0.917213716213425)(480,0.917213716213425)(480,0.917213716213425)(480,0.917213716213425)(480,0.917213716213425)(480,0.917213716213425)(480,0.917213716213425)(480,0.917213716213425)(480,0.917213716213425)(480,0.917213716213425)(480,0.917213716213425)(480,0.917213716213425)(480,0.917213716213425)(480,0.917213716213425)(486,0.91761306561436)(510,0.918769915668662)(510,0.918769915668662)(510,0.918769915668662)(510,0.918769915668662)(510,0.918769915668662)(510,0.918769915668662)(510,0.918769915668662)(510,0.918769915668662)(510,0.918769915668662)(510,0.918769915668662)(510,0.918769915668662)(510,0.918769915668662)(510,0.918769915668662)(510,0.918769915668662)(510,0.918769915668662)(510,0.918769915668662)(510,0.918769915668662)(510,0.918769915668662)(510,0.918769915668662)(512,0.918850398301604)(517,0.919046541668753)(522,0.919236160767289)(523,0.919273360003902)(525,0.919347067498085)(526,0.919383582173779)(540,0.919872019743474)(540,0.919872019743474)(540,0.919872019743474)(540,0.919872019743474)(540,0.919872019743474)(540,0.919872019743474)(540,0.919872019743474)(540,0.919872019743474)(540,0.919872019743474)(540,0.919872019743474)(540,0.919872019743474)(540,0.919872019743474)(540,0.919872019743474)(540,0.919872019743474)(540,0.919872019743474)(540,0.919872019743474)(540,0.919872019743474)(540,0.919872019743474)(540,0.919872019743474)(540,0.919872019743474)(540,0.919872019743474)(552,0.920260421976783)(558,0.920447313979799)(570,0.920812154222695)(570,0.920812154222695)(570,0.920812154222695)(570,0.920812154222695)(570,0.920812154222695)(570,0.920812154222695)(570,0.920812154222695)(570,0.920812154222695)(570,0.920812154222695)(570,0.920812154222695)(570,0.920812154222695)(570,0.920812154222695)(570,0.920812154222695)(570,0.920812154222695)(570,0.920812154222695)(570,0.920812154222695)(570,0.920812154222695)(570,0.920812154222695)(570,0.920812154222695)(570,0.920812154222695)(570,0.920812154222695)(570,0.920812154222695)(570,0.920812154222695)(570,0.920812154222695)(570,0.920812154222695)(571,0.920842174104741)(571,0.920842174104741)(577,0.921020968068828)(588,0.921339690212484)(595,0.921533729675335)(600,0.92166696233089)(600,0.92166696233089)(600,0.92166696233089)(600,0.92166696233089)(600,0.92166696233089)(600,0.92166696233089)(600,0.92166696233089)(600,0.92166696233089)(600,0.92166696233089)(600,0.92166696233089)(600,0.92166696233089)(600,0.92166696233089)(600,0.92166696233089)(600,0.92166696233089)(600,0.92166696233089) 
};
\addlegendentry{\ibacl};

\end{axis}
\end{tikzpicture}%

%% This file was created by matlab2tikz v0.2.3.
% Copyright (c) 2008--2012, Nico Schlömer <nico.schloemer@gmail.com>
% All rights reserved.
% 
% 
% 
\begin{tikzpicture}

\begin{axis}[%
tick label style={font=\tiny},
label style={font=\tiny},
label shift={-4pt},
xlabel shift={-6pt},
legend style={font=\tiny},
view={0}{90},
width=\figurewidth,
height=\figureheight,
scale only axis,
xmin=0, xmax=400,
xlabel={Samples},
ymin=0.5, ymax=1,
ylabel={$F_1$-score},
axis lines*=left,
legend cell align=left,
legend style={at={(1.03,0)},anchor=south east,fill=none,draw=none,align=left,row sep=-0.2em},
clip=false]

\addplot [
color=red,
densely dotted,
line width=1.0pt,
]
coordinates{
 (12,0.513242753000793)(12,0.513242753000793)(12,0.513242753000793)(12,0.513242753000793)(12,0.513242753000793)(12,0.513242753000793)(12,0.513242753000793)(12,0.513242753000793)(13,0.563071802960045)(13,0.563071802960045)(13,0.563071802960045)(13,0.563071802960045)(13,0.563071802960045)(13,0.563071802960045)(13,0.563071802960045)(13,0.563071802960045)(13,0.563071802960045)(14,0.609412089731185)(14,0.609412089731185)(14,0.609412089731185)(14,0.609412089731185)(14,0.609412089731185)(14,0.609412089731185)(14,0.609412089731185)(14,0.609412089731185)(14,0.609412089731185)(15,0.654045205749223)(15,0.654045205749223)(15,0.654045205749223)(15,0.654045205749223)(15,0.654045205749223)(15,0.654045205749223)(15,0.654045205749223)(15,0.654045205749223)(15,0.654045205749223)(15,0.654045205749223)(15,0.654045205749223)(15,0.654045205749223)(16,0.693800447635594)(16,0.693800447635594)(16,0.693800447635594)(16,0.693800447635594)(16,0.693800447635594)(16,0.693800447635594)(16,0.693800447635594)(16,0.693800447635594)(16,0.693800447635594)(16,0.693800447635594)(16,0.693800447635594)(16,0.693800447635594)(16,0.693800447635594)(16,0.693800447635594)(16,0.693800447635594)(16,0.693800447635594)(16,0.693800447635594)(16,0.693800447635594)(16,0.693800447635594)(16,0.693800447635594)(16,0.693800447635594)(17,0.719964994412613)(17,0.719964994412613)(17,0.719964994412613)(17,0.719964994412613)(17,0.719964994412613)(17,0.719964994412613)(17,0.719964994412613)(17,0.719964994412613)(17,0.719964994412613)(17,0.719964994412613)(17,0.719964994412613)(17,0.719964994412613)(17,0.719964994412613)(17,0.719964994412613)(17,0.719964994412613)(17,0.719964994412613)(17,0.719964994412613)(17,0.719964994412613)(17,0.719964994412613)(17,0.719964994412613)(17,0.719964994412613)(17,0.719964994412613)(17,0.719964994412613)(18,0.738956378279693)(18,0.738956378279693)(18,0.738956378279693)(18,0.738956378279693)(18,0.738956378279693)(18,0.738956378279693)(18,0.738956378279693)(18,0.738956378279693)(18,0.738956378279693)(18,0.738956378279693)(18,0.738956378279693)(18,0.738956378279693)(18,0.738956378279693)(18,0.738956378279693)(18,0.738956378279693)(18,0.738956378279693)(18,0.738956378279693)(18,0.738956378279693)(18,0.738956378279693)(18,0.738956378279693)(18,0.738956378279693)(18,0.738956378279693)(18,0.738956378279693)(18,0.738956378279693)(18,0.738956378279693)(18,0.738956378279693)(18,0.738956378279693)(19,0.756578089042498)(19,0.756578089042498)(19,0.756578089042498)(19,0.756578089042498)(19,0.756578089042498)(19,0.756578089042498)(19,0.756578089042498)(19,0.756578089042498)(19,0.756578089042498)(19,0.756578089042498)(19,0.756578089042498)(19,0.756578089042498)(19,0.756578089042498)(19,0.756578089042498)(19,0.756578089042498)(19,0.756578089042498)(19,0.756578089042498)(19,0.756578089042498)(19,0.756578089042498)(19,0.756578089042498)(19,0.756578089042498)(19,0.756578089042498)(19,0.756578089042498)(19,0.756578089042498)(19,0.756578089042498)(19,0.756578089042498)(19,0.756578089042498)(19,0.756578089042498)(19,0.756578089042498)(20,0.766154555098519)(20,0.766154555098519)(20,0.766154555098519)(20,0.766154555098519)(20,0.766154555098519)(20,0.766154555098519)(20,0.766154555098519)(20,0.766154555098519)(20,0.766154555098519)(20,0.766154555098519)(20,0.766154555098519)(20,0.766154555098519)(20,0.766154555098519)(20,0.766154555098519)(20,0.766154555098519)(20,0.766154555098519)(20,0.766154555098519)(20,0.766154555098519)(20,0.766154555098519)(20,0.766154555098519)(20,0.766154555098519)(20,0.766154555098519)(20,0.766154555098519)(20,0.766154555098519)(20,0.766154555098519)(20,0.766154555098519)(20,0.766154555098519)(20,0.766154555098519)(20,0.766154555098519)(20,0.766154555098519)(20,0.766154555098519)(20,0.766154555098519)(20,0.766154555098519)(20,0.766154555098519)(20,0.766154555098519)(20,0.766154555098519)(20,0.766154555098519)(20,0.766154555098519)(20,0.766154555098519)(21,0.772168694536424)(21,0.772168694536424)(21,0.772168694536424)(21,0.772168694536424)(21,0.772168694536424)(21,0.772168694536424)(21,0.772168694536424)(21,0.772168694536424)(21,0.772168694536424)(21,0.772168694536424)(21,0.772168694536424)(21,0.772168694536424)(21,0.772168694536424)(21,0.772168694536424)(21,0.772168694536424)(21,0.772168694536424)(21,0.772168694536424)(21,0.772168694536424)(21,0.772168694536424)(21,0.772168694536424)(21,0.772168694536424)(21,0.772168694536424)(21,0.772168694536424)(21,0.772168694536424)(21,0.772168694536424)(21,0.772168694536424)(21,0.772168694536424)(21,0.772168694536424)(21,0.772168694536424)(21,0.772168694536424)(21,0.772168694536424)(21,0.772168694536424)(21,0.772168694536424)(21,0.772168694536424)(21,0.772168694536424)(21,0.772168694536424)(21,0.772168694536424)(21,0.772168694536424)(21,0.772168694536424)(21,0.772168694536424)(21,0.772168694536424)(21,0.772168694536424)(21,0.772168694536424)(21,0.772168694536424)(21,0.772168694536424)(22,0.775246122095812)(22,0.775246122095812)(22,0.775246122095812)(22,0.775246122095812)(22,0.775246122095812)(22,0.775246122095812)(22,0.775246122095812)(22,0.775246122095812)(22,0.775246122095812)(22,0.775246122095812)(22,0.775246122095812)(22,0.775246122095812)(22,0.775246122095812)(22,0.775246122095812)(22,0.775246122095812)(22,0.775246122095812)(22,0.775246122095812)(22,0.775246122095812)(22,0.775246122095812)(22,0.775246122095812)(22,0.775246122095812)(22,0.775246122095812)(22,0.775246122095812)(22,0.775246122095812)(22,0.775246122095812)(22,0.775246122095812)(22,0.775246122095812)(22,0.775246122095812)(22,0.775246122095812)(22,0.775246122095812)(22,0.775246122095812)(22,0.775246122095812)(22,0.775246122095812)(22,0.775246122095812)(22,0.775246122095812)(22,0.775246122095812)(22,0.775246122095812)(22,0.775246122095812)(22,0.775246122095812)(22,0.775246122095812)(22,0.775246122095812)(22,0.775246122095812)(23,0.779159487806811)(23,0.779159487806811)(23,0.779159487806811)(23,0.779159487806811)(23,0.779159487806811)(23,0.779159487806811)(23,0.779159487806811)(23,0.779159487806811)(23,0.779159487806811)(23,0.779159487806811)(23,0.779159487806811)(23,0.779159487806811)(23,0.779159487806811)(23,0.779159487806811)(23,0.779159487806811)(23,0.779159487806811)(23,0.779159487806811)(23,0.779159487806811)(23,0.779159487806811)(23,0.779159487806811)(23,0.779159487806811)(23,0.779159487806811)(23,0.779159487806811)(23,0.779159487806811)(23,0.779159487806811)(23,0.779159487806811)(23,0.779159487806811)(23,0.779159487806811)(23,0.779159487806811)(23,0.779159487806811)(23,0.779159487806811)(23,0.779159487806811)(23,0.779159487806811)(23,0.779159487806811)(23,0.779159487806811)(23,0.779159487806811)(23,0.779159487806811)(23,0.779159487806811)(23,0.779159487806811)(23,0.779159487806811)(23,0.779159487806811)(23,0.779159487806811)(23,0.779159487806811)(23,0.779159487806811)(23,0.779159487806811)(23,0.779159487806811)(23,0.779159487806811)(23,0.779159487806811)(23,0.779159487806811)(24,0.784632419978595)(24,0.784632419978595)(24,0.784632419978595)(24,0.784632419978595)(24,0.784632419978595)(24,0.784632419978595)(24,0.784632419978595)(24,0.784632419978595)(24,0.784632419978595)(24,0.784632419978595)(24,0.784632419978595)(24,0.784632419978595)(24,0.784632419978595)(24,0.784632419978595)(24,0.784632419978595)(24,0.784632419978595)(24,0.784632419978595)(24,0.784632419978595)(24,0.784632419978595)(24,0.784632419978595)(24,0.784632419978595)(24,0.784632419978595)(24,0.784632419978595)(24,0.784632419978595)(24,0.784632419978595)(24,0.784632419978595)(24,0.784632419978595)(24,0.784632419978595)(24,0.784632419978595)(24,0.784632419978595)(24,0.784632419978595)(24,0.784632419978595)(24,0.784632419978595)(24,0.784632419978595)(24,0.784632419978595)(24,0.784632419978595)(24,0.784632419978595)(24,0.784632419978595)(24,0.784632419978595)(24,0.784632419978595)(24,0.784632419978595)(24,0.784632419978595)(24,0.784632419978595)(24,0.784632419978595)(24,0.784632419978595)(24,0.784632419978595)(24,0.784632419978595)(24,0.784632419978595)(24,0.784632419978595)(24,0.784632419978595)(24,0.784632419978595)(25,0.790027700330571)(25,0.790027700330571)(25,0.790027700330571)(25,0.790027700330571)(25,0.790027700330571)(25,0.790027700330571)(25,0.790027700330571)(25,0.790027700330571)(25,0.790027700330571)(25,0.790027700330571)(25,0.790027700330571)(25,0.790027700330571)(25,0.790027700330571)(25,0.790027700330571)(25,0.790027700330571)(25,0.790027700330571)(25,0.790027700330571)(25,0.790027700330571)(25,0.790027700330571)(25,0.790027700330571)(25,0.790027700330571)(25,0.790027700330571)(25,0.790027700330571)(25,0.790027700330571)(25,0.790027700330571)(25,0.790027700330571)(25,0.790027700330571)(25,0.790027700330571)(25,0.790027700330571)(25,0.790027700330571)(25,0.790027700330571)(25,0.790027700330571)(25,0.790027700330571)(25,0.790027700330571)(25,0.790027700330571)(25,0.790027700330571)(25,0.790027700330571)(25,0.790027700330571)(25,0.790027700330571)(25,0.790027700330571)(26,0.79447497049185)(26,0.79447497049185)(26,0.79447497049185)(26,0.79447497049185)(26,0.79447497049185)(26,0.79447497049185)(26,0.79447497049185)(26,0.79447497049185)(26,0.79447497049185)(26,0.79447497049185)(26,0.79447497049185)(26,0.79447497049185)(26,0.79447497049185)(26,0.79447497049185)(26,0.79447497049185)(26,0.79447497049185)(26,0.79447497049185)(26,0.79447497049185)(26,0.79447497049185)(26,0.79447497049185)(26,0.79447497049185)(26,0.79447497049185)(26,0.79447497049185)(26,0.79447497049185)(26,0.79447497049185)(26,0.79447497049185)(26,0.79447497049185)(26,0.79447497049185)(26,0.79447497049185)(26,0.79447497049185)(26,0.79447497049185)(26,0.79447497049185)(26,0.79447497049185)(26,0.79447497049185)(26,0.79447497049185)(26,0.79447497049185)(26,0.79447497049185)(26,0.79447497049185)(26,0.79447497049185)(26,0.79447497049185)(26,0.79447497049185)(27,0.796074899938778)(27,0.796074899938778)(27,0.796074899938778)(27,0.796074899938778)(27,0.796074899938778)(27,0.796074899938778)(27,0.796074899938778)(27,0.796074899938778)(27,0.796074899938778)(27,0.796074899938778)(27,0.796074899938778)(27,0.796074899938778)(27,0.796074899938778)(27,0.796074899938778)(27,0.796074899938778)(27,0.796074899938778)(27,0.796074899938778)(27,0.796074899938778)(27,0.796074899938778)(27,0.796074899938778)(27,0.796074899938778)(27,0.796074899938778)(27,0.796074899938778)(27,0.796074899938778)(27,0.796074899938778)(27,0.796074899938778)(27,0.796074899938778)(27,0.796074899938778)(27,0.796074899938778)(27,0.796074899938778)(27,0.796074899938778)(27,0.796074899938778)(27,0.796074899938778)(27,0.796074899938778)(27,0.796074899938778)(27,0.796074899938778)(27,0.796074899938778)(27,0.796074899938778)(27,0.796074899938778)(28,0.795999379478335)(28,0.795999379478335)(28,0.795999379478335)(28,0.795999379478335)(28,0.795999379478335)(28,0.795999379478335)(28,0.795999379478335)(28,0.795999379478335)(28,0.795999379478335)(28,0.795999379478335)(28,0.795999379478335)(28,0.795999379478335)(28,0.795999379478335)(28,0.795999379478335)(28,0.795999379478335)(28,0.795999379478335)(28,0.795999379478335)(28,0.795999379478335)(28,0.795999379478335)(28,0.795999379478335)(28,0.795999379478335)(28,0.795999379478335)(28,0.795999379478335)(28,0.795999379478335)(28,0.795999379478335)(28,0.795999379478335)(28,0.795999379478335)(28,0.795999379478335)(28,0.795999379478335)(28,0.795999379478335)(28,0.795999379478335)(28,0.795999379478335)(28,0.795999379478335)(28,0.795999379478335)(28,0.795999379478335)(28,0.795999379478335)(28,0.795999379478335)(28,0.795999379478335)(28,0.795999379478335)(29,0.798409856541205)(29,0.798409856541205)(29,0.798409856541205)(29,0.798409856541205)(29,0.798409856541205)(29,0.798409856541205)(29,0.798409856541205)(29,0.798409856541205)(29,0.798409856541205)(29,0.798409856541205)(29,0.798409856541205)(29,0.798409856541205)(29,0.798409856541205)(29,0.798409856541205)(29,0.798409856541205)(29,0.798409856541205)(29,0.798409856541205)(29,0.798409856541205)(29,0.798409856541205)(29,0.798409856541205)(29,0.798409856541205)(29,0.798409856541205)(29,0.798409856541205)(29,0.798409856541205)(29,0.798409856541205)(29,0.798409856541205)(29,0.798409856541205)(29,0.798409856541205)(29,0.798409856541205)(29,0.798409856541205)(29,0.798409856541205)(29,0.798409856541205)(29,0.798409856541205)(29,0.798409856541205)(29,0.798409856541205)(29,0.798409856541205)(29,0.798409856541205)(30,0.801907554024585)(30,0.801907554024585)(30,0.801907554024585)(30,0.801907554024585)(30,0.801907554024585)(30,0.801907554024585)(30,0.801907554024585)(30,0.801907554024585)(30,0.801907554024585)(30,0.801907554024585)(30,0.801907554024585)(30,0.801907554024585)(30,0.801907554024585)(30,0.801907554024585)(30,0.801907554024585)(30,0.801907554024585)(30,0.801907554024585)(30,0.801907554024585)(30,0.801907554024585)(30,0.801907554024585)(30,0.801907554024585)(30,0.801907554024585)(30,0.801907554024585)(30,0.801907554024585)(30,0.801907554024585)(30,0.801907554024585)(30,0.801907554024585)(30,0.801907554024585)(30,0.801907554024585)(31,0.806226472417736)(31,0.806226472417736)(31,0.806226472417736)(31,0.806226472417736)(31,0.806226472417736)(31,0.806226472417736)(31,0.806226472417736)(31,0.806226472417736)(31,0.806226472417736)(31,0.806226472417736)(31,0.806226472417736)(31,0.806226472417736)(31,0.806226472417736)(31,0.806226472417736)(31,0.806226472417736)(31,0.806226472417736)(31,0.806226472417736)(31,0.806226472417736)(31,0.806226472417736)(31,0.806226472417736)(31,0.806226472417736)(31,0.806226472417736)(31,0.806226472417736)(31,0.806226472417736)(31,0.806226472417736)(31,0.806226472417736)(31,0.806226472417736)(31,0.806226472417736)(31,0.806226472417736)(31,0.806226472417736)(31,0.806226472417736)(31,0.806226472417736)(31,0.806226472417736)(31,0.806226472417736)(31,0.806226472417736)(32,0.810075821596545)(32,0.810075821596545)(32,0.810075821596545)(32,0.810075821596545)(32,0.810075821596545)(32,0.810075821596545)(32,0.810075821596545)(32,0.810075821596545)(32,0.810075821596545)(32,0.810075821596545)(32,0.810075821596545)(32,0.810075821596545)(32,0.810075821596545)(32,0.810075821596545)(32,0.810075821596545)(32,0.810075821596545)(32,0.810075821596545)(32,0.810075821596545)(32,0.810075821596545)(32,0.810075821596545)(32,0.810075821596545)(32,0.810075821596545)(32,0.810075821596545)(32,0.810075821596545)(32,0.810075821596545)(32,0.810075821596545)(32,0.810075821596545)(32,0.810075821596545)(32,0.810075821596545)(32,0.810075821596545)(32,0.810075821596545)(32,0.810075821596545)(32,0.810075821596545)(32,0.810075821596545)(32,0.810075821596545)(32,0.810075821596545)(32,0.810075821596545)(33,0.81300247138301)(33,0.81300247138301)(33,0.81300247138301)(33,0.81300247138301)(33,0.81300247138301)(33,0.81300247138301)(33,0.81300247138301)(33,0.81300247138301)(33,0.81300247138301)(33,0.81300247138301)(33,0.81300247138301)(33,0.81300247138301)(33,0.81300247138301)(33,0.81300247138301)(33,0.81300247138301)(33,0.81300247138301)(33,0.81300247138301)(33,0.81300247138301)(33,0.81300247138301)(33,0.81300247138301)(33,0.81300247138301)(33,0.81300247138301)(33,0.81300247138301)(33,0.81300247138301)(33,0.81300247138301)(33,0.81300247138301)(33,0.81300247138301)(33,0.81300247138301)(33,0.81300247138301)(33,0.81300247138301)(33,0.81300247138301)(33,0.81300247138301)(33,0.81300247138301)(33,0.81300247138301)(33,0.81300247138301)(34,0.815634494830405)(34,0.815634494830405)(34,0.815634494830405)(34,0.815634494830405)(34,0.815634494830405)(34,0.815634494830405)(34,0.815634494830405)(34,0.815634494830405)(34,0.815634494830405)(34,0.815634494830405)(34,0.815634494830405)(34,0.815634494830405)(34,0.815634494830405)(34,0.815634494830405)(34,0.815634494830405)(34,0.815634494830405)(34,0.815634494830405)(34,0.815634494830405)(34,0.815634494830405)(34,0.815634494830405)(34,0.815634494830405)(34,0.815634494830405)(34,0.815634494830405)(34,0.815634494830405)(34,0.815634494830405)(34,0.815634494830405)(34,0.815634494830405)(34,0.815634494830405)(35,0.817866616393611)(35,0.817866616393611)(35,0.817866616393611)(35,0.817866616393611)(35,0.817866616393611)(35,0.817866616393611)(35,0.817866616393611)(35,0.817866616393611)(35,0.817866616393611)(35,0.817866616393611)(35,0.817866616393611)(35,0.817866616393611)(35,0.817866616393611)(35,0.817866616393611)(35,0.817866616393611)(35,0.817866616393611)(35,0.817866616393611)(35,0.817866616393611)(35,0.817866616393611)(35,0.817866616393611)(35,0.817866616393611)(35,0.817866616393611)(36,0.820093974807631)(36,0.820093974807631)(36,0.820093974807631)(36,0.820093974807631)(36,0.820093974807631)(36,0.820093974807631)(36,0.820093974807631)(36,0.820093974807631)(36,0.820093974807631)(36,0.820093974807631)(36,0.820093974807631)(36,0.820093974807631)(36,0.820093974807631)(36,0.820093974807631)(36,0.820093974807631)(36,0.820093974807631)(36,0.820093974807631)(36,0.820093974807631)(36,0.820093974807631)(36,0.820093974807631)(36,0.820093974807631)(36,0.820093974807631)(36,0.820093974807631)(37,0.823267231828805)(37,0.823267231828805)(37,0.823267231828805)(37,0.823267231828805)(37,0.823267231828805)(37,0.823267231828805)(37,0.823267231828805)(37,0.823267231828805)(37,0.823267231828805)(37,0.823267231828805)(37,0.823267231828805)(37,0.823267231828805)(37,0.823267231828805)(37,0.823267231828805)(37,0.823267231828805)(37,0.823267231828805)(37,0.823267231828805)(37,0.823267231828805)(37,0.823267231828805)(37,0.823267231828805)(37,0.823267231828805)(37,0.823267231828805)(37,0.823267231828805)(37,0.823267231828805)(37,0.823267231828805)(37,0.823267231828805)(38,0.826497293362125)(38,0.826497293362125)(38,0.826497293362125)(38,0.826497293362125)(38,0.826497293362125)(38,0.826497293362125)(38,0.826497293362125)(38,0.826497293362125)(38,0.826497293362125)(38,0.826497293362125)(38,0.826497293362125)(38,0.826497293362125)(38,0.826497293362125)(38,0.826497293362125)(38,0.826497293362125)(38,0.826497293362125)(38,0.826497293362125)(38,0.826497293362125)(38,0.826497293362125)(38,0.826497293362125)(38,0.826497293362125)(38,0.826497293362125)(38,0.826497293362125)(38,0.826497293362125)(38,0.826497293362125)(39,0.82935832921807)(39,0.82935832921807)(39,0.82935832921807)(39,0.82935832921807)(39,0.82935832921807)(39,0.82935832921807)(39,0.82935832921807)(39,0.82935832921807)(39,0.82935832921807)(39,0.82935832921807)(39,0.82935832921807)(39,0.82935832921807)(39,0.82935832921807)(39,0.82935832921807)(39,0.82935832921807)(39,0.82935832921807)(39,0.82935832921807)(39,0.82935832921807)(39,0.82935832921807)(39,0.82935832921807)(39,0.82935832921807)(39,0.82935832921807)(39,0.82935832921807)(39,0.82935832921807)(39,0.82935832921807)(39,0.82935832921807)(39,0.82935832921807)(40,0.83140553472798)(40,0.83140553472798)(40,0.83140553472798)(40,0.83140553472798)(40,0.83140553472798)(40,0.83140553472798)(40,0.83140553472798)(40,0.83140553472798)(40,0.83140553472798)(40,0.83140553472798)(40,0.83140553472798)(40,0.83140553472798)(40,0.83140553472798)(40,0.83140553472798)(40,0.83140553472798)(40,0.83140553472798)(40,0.83140553472798)(40,0.83140553472798)(40,0.83140553472798)(40,0.83140553472798)(41,0.832979516062166)(41,0.832979516062166)(41,0.832979516062166)(41,0.832979516062166)(41,0.832979516062166)(41,0.832979516062166)(41,0.832979516062166)(41,0.832979516062166)(41,0.832979516062166)(41,0.832979516062166)(41,0.832979516062166)(41,0.832979516062166)(41,0.832979516062166)(41,0.832979516062166)(41,0.832979516062166)(41,0.832979516062166)(41,0.832979516062166)(41,0.832979516062166)(41,0.832979516062166)(41,0.832979516062166)(41,0.832979516062166)(41,0.832979516062166)(42,0.835016811547283)(42,0.835016811547283)(42,0.835016811547283)(42,0.835016811547283)(42,0.835016811547283)(42,0.835016811547283)(42,0.835016811547283)(42,0.835016811547283)(42,0.835016811547283)(42,0.835016811547283)(42,0.835016811547283)(42,0.835016811547283)(42,0.835016811547283)(42,0.835016811547283)(42,0.835016811547283)(42,0.835016811547283)(43,0.836782158554706)(43,0.836782158554706)(43,0.836782158554706)(43,0.836782158554706)(43,0.836782158554706)(43,0.836782158554706)(43,0.836782158554706)(43,0.836782158554706)(43,0.836782158554706)(43,0.836782158554706)(43,0.836782158554706)(43,0.836782158554706)(43,0.836782158554706)(43,0.836782158554706)(43,0.836782158554706)(44,0.838048687172316)(44,0.838048687172316)(44,0.838048687172316)(44,0.838048687172316)(44,0.838048687172316)(44,0.838048687172316)(44,0.838048687172316)(44,0.838048687172316)(44,0.838048687172316)(44,0.838048687172316)(44,0.838048687172316)(44,0.838048687172316)(44,0.838048687172316)(44,0.838048687172316)(44,0.838048687172316)(44,0.838048687172316)(44,0.838048687172316)(44,0.838048687172316)(44,0.838048687172316)(44,0.838048687172316)(44,0.838048687172316)(44,0.838048687172316)(44,0.838048687172316)(44,0.838048687172316)(44,0.838048687172316)(44,0.838048687172316)(45,0.83911360475653)(45,0.83911360475653)(45,0.83911360475653)(45,0.83911360475653)(45,0.83911360475653)(45,0.83911360475653)(45,0.83911360475653)(45,0.83911360475653)(45,0.83911360475653)(45,0.83911360475653)(45,0.83911360475653)(45,0.83911360475653)(45,0.83911360475653)(45,0.83911360475653)(45,0.83911360475653)(45,0.83911360475653)(45,0.83911360475653)(45,0.83911360475653)(45,0.83911360475653)(45,0.83911360475653)(45,0.83911360475653)(45,0.83911360475653)(45,0.83911360475653)(46,0.840165489288512)(46,0.840165489288512)(46,0.840165489288512)(46,0.840165489288512)(46,0.840165489288512)(46,0.840165489288512)(46,0.840165489288512)(46,0.840165489288512)(46,0.840165489288512)(46,0.840165489288512)(46,0.840165489288512)(46,0.840165489288512)(46,0.840165489288512)(46,0.840165489288512)(46,0.840165489288512)(46,0.840165489288512)(46,0.840165489288512)(46,0.840165489288512)(46,0.840165489288512)(46,0.840165489288512)(47,0.841519534640496)(47,0.841519534640496)(47,0.841519534640496)(47,0.841519534640496)(47,0.841519534640496)(47,0.841519534640496)(47,0.841519534640496)(47,0.841519534640496)(47,0.841519534640496)(47,0.841519534640496)(47,0.841519534640496)(47,0.841519534640496)(47,0.841519534640496)(47,0.841519534640496)(47,0.841519534640496)(47,0.841519534640496)(47,0.841519534640496)(47,0.841519534640496)(47,0.841519534640496)(47,0.841519534640496)(47,0.841519534640496)(47,0.841519534640496)(48,0.842937547168656)(48,0.842937547168656)(48,0.842937547168656)(48,0.842937547168656)(48,0.842937547168656)(48,0.842937547168656)(48,0.842937547168656)(48,0.842937547168656)(48,0.842937547168656)(49,0.844463742087862)(49,0.844463742087862)(49,0.844463742087862)(49,0.844463742087862)(49,0.844463742087862)(49,0.844463742087862)(49,0.844463742087862)(49,0.844463742087862)(49,0.844463742087862)(49,0.844463742087862)(49,0.844463742087862)(49,0.844463742087862)(49,0.844463742087862)(49,0.844463742087862)(49,0.844463742087862)(49,0.844463742087862)(49,0.844463742087862)(49,0.844463742087862)(49,0.844463742087862)(49,0.844463742087862)(50,0.845997992207294)(50,0.845997992207294)(50,0.845997992207294)(50,0.845997992207294)(50,0.845997992207294)(50,0.845997992207294)(50,0.845997992207294)(50,0.845997992207294)(50,0.845997992207294)(50,0.845997992207294)(50,0.845997992207294)(50,0.845997992207294)(50,0.845997992207294)(50,0.845997992207294)(50,0.845997992207294)(50,0.845997992207294)(50,0.845997992207294)(50,0.845997992207294)(50,0.845997992207294)(51,0.847649750433022)(51,0.847649750433022)(51,0.847649750433022)(51,0.847649750433022)(51,0.847649750433022)(51,0.847649750433022)(51,0.847649750433022)(51,0.847649750433022)(51,0.847649750433022)(51,0.847649750433022)(51,0.847649750433022)(51,0.847649750433022)(51,0.847649750433022)(51,0.847649750433022)(52,0.850004427720027)(52,0.850004427720027)(52,0.850004427720027)(52,0.850004427720027)(52,0.850004427720027)(52,0.850004427720027)(52,0.850004427720027)(52,0.850004427720027)(52,0.850004427720027)(52,0.850004427720027)(52,0.850004427720027)(52,0.850004427720027)(52,0.850004427720027)(52,0.850004427720027)(52,0.850004427720027)(52,0.850004427720027)(53,0.852261417456051)(53,0.852261417456051)(53,0.852261417456051)(53,0.852261417456051)(53,0.852261417456051)(53,0.852261417456051)(53,0.852261417456051)(54,0.854751441376187)(54,0.854751441376187)(54,0.854751441376187)(54,0.854751441376187)(54,0.854751441376187)(54,0.854751441376187)(54,0.854751441376187)(54,0.854751441376187)(54,0.854751441376187)(54,0.854751441376187)(54,0.854751441376187)(54,0.854751441376187)(54,0.854751441376187)(54,0.854751441376187)(54,0.854751441376187)(55,0.857406982603522)(55,0.857406982603522)(55,0.857406982603522)(55,0.857406982603522)(55,0.857406982603522)(55,0.857406982603522)(55,0.857406982603522)(55,0.857406982603522)(55,0.857406982603522)(55,0.857406982603522)(55,0.857406982603522)(55,0.857406982603522)(55,0.857406982603522)(55,0.857406982603522)(55,0.857406982603522)(55,0.857406982603522)(55,0.857406982603522)(55,0.857406982603522)(56,0.859328278945926)(56,0.859328278945926)(56,0.859328278945926)(56,0.859328278945926)(56,0.859328278945926)(56,0.859328278945926)(56,0.859328278945926)(56,0.859328278945926)(56,0.859328278945926)(56,0.859328278945926)(56,0.859328278945926)(56,0.859328278945926)(56,0.859328278945926)(57,0.861063335311418)(57,0.861063335311418)(57,0.861063335311418)(57,0.861063335311418)(57,0.861063335311418)(57,0.861063335311418)(57,0.861063335311418)(57,0.861063335311418)(57,0.861063335311418)(57,0.861063335311418)(57,0.861063335311418)(57,0.861063335311418)(57,0.861063335311418)(57,0.861063335311418)(58,0.862397066154262)(58,0.862397066154262)(58,0.862397066154262)(58,0.862397066154262)(58,0.862397066154262)(58,0.862397066154262)(58,0.862397066154262)(58,0.862397066154262)(58,0.862397066154262)(58,0.862397066154262)(58,0.862397066154262)(58,0.862397066154262)(58,0.862397066154262)(58,0.862397066154262)(58,0.862397066154262)(59,0.863641311211628)(59,0.863641311211628)(59,0.863641311211628)(59,0.863641311211628)(59,0.863641311211628)(59,0.863641311211628)(59,0.863641311211628)(59,0.863641311211628)(59,0.863641311211628)(59,0.863641311211628)(60,0.864923564406168)(60,0.864923564406168)(60,0.864923564406168)(60,0.864923564406168)(60,0.864923564406168)(60,0.864923564406168)(60,0.864923564406168)(60,0.864923564406168)(60,0.864923564406168)(60,0.864923564406168)(60,0.864923564406168)(60,0.864923564406168)(61,0.866069771229191)(61,0.866069771229191)(61,0.866069771229191)(61,0.866069771229191)(61,0.866069771229191)(61,0.866069771229191)(61,0.866069771229191)(61,0.866069771229191)(61,0.866069771229191)(61,0.866069771229191)(61,0.866069771229191)(61,0.866069771229191)(61,0.866069771229191)(61,0.866069771229191)(61,0.866069771229191)(61,0.866069771229191)(61,0.866069771229191)(61,0.866069771229191)(62,0.86714028889457)(62,0.86714028889457)(62,0.86714028889457)(62,0.86714028889457)(62,0.86714028889457)(62,0.86714028889457)(62,0.86714028889457)(62,0.86714028889457)(62,0.86714028889457)(62,0.86714028889457)(62,0.86714028889457)(63,0.868379210237588)(63,0.868379210237588)(63,0.868379210237588)(63,0.868379210237588)(63,0.868379210237588)(63,0.868379210237588)(64,0.86945236496554)(64,0.86945236496554)(64,0.86945236496554)(64,0.86945236496554)(64,0.86945236496554)(64,0.86945236496554)(64,0.86945236496554)(64,0.86945236496554)(64,0.86945236496554)(64,0.86945236496554)(64,0.86945236496554)(64,0.86945236496554)(64,0.86945236496554)(64,0.86945236496554)(65,0.870630399980386)(65,0.870630399980386)(65,0.870630399980386)(65,0.870630399980386)(65,0.870630399980386)(65,0.870630399980386)(65,0.870630399980386)(65,0.870630399980386)(65,0.870630399980386)(65,0.870630399980386)(65,0.870630399980386)(65,0.870630399980386)(66,0.871661181225682)(66,0.871661181225682)(66,0.871661181225682)(66,0.871661181225682)(66,0.871661181225682)(66,0.871661181225682)(66,0.871661181225682)(66,0.871661181225682)(66,0.871661181225682)(66,0.871661181225682)(67,0.872777656661999)(67,0.872777656661999)(67,0.872777656661999)(67,0.872777656661999)(67,0.872777656661999)(67,0.872777656661999)(67,0.872777656661999)(67,0.872777656661999)(67,0.872777656661999)(67,0.872777656661999)(67,0.872777656661999)(67,0.872777656661999)(67,0.872777656661999)(67,0.872777656661999)(67,0.872777656661999)(68,0.874094440732809)(68,0.874094440732809)(68,0.874094440732809)(68,0.874094440732809)(68,0.874094440732809)(68,0.874094440732809)(68,0.874094440732809)(68,0.874094440732809)(69,0.875274710403873)(69,0.875274710403873)(69,0.875274710403873)(69,0.875274710403873)(69,0.875274710403873)(69,0.875274710403873)(69,0.875274710403873)(69,0.875274710403873)(69,0.875274710403873)(69,0.875274710403873)(70,0.876213199750608)(70,0.876213199750608)(70,0.876213199750608)(70,0.876213199750608)(70,0.876213199750608)(70,0.876213199750608)(70,0.876213199750608)(70,0.876213199750608)(70,0.876213199750608)(70,0.876213199750608)(70,0.876213199750608)(70,0.876213199750608)(70,0.876213199750608)(70,0.876213199750608)(70,0.876213199750608)(71,0.877107762438385)(71,0.877107762438385)(71,0.877107762438385)(71,0.877107762438385)(71,0.877107762438385)(71,0.877107762438385)(71,0.877107762438385)(71,0.877107762438385)(71,0.877107762438385)(71,0.877107762438385)(71,0.877107762438385)(71,0.877107762438385)(71,0.877107762438385)(71,0.877107762438385)(71,0.877107762438385)(71,0.877107762438385)(71,0.877107762438385)(71,0.877107762438385)(72,0.877849086914781)(72,0.877849086914781)(72,0.877849086914781)(72,0.877849086914781)(72,0.877849086914781)(72,0.877849086914781)(72,0.877849086914781)(72,0.877849086914781)(72,0.877849086914781)(72,0.877849086914781)(72,0.877849086914781)(73,0.878339661501157)(73,0.878339661501157)(73,0.878339661501157)(73,0.878339661501157)(74,0.87895365306575)(74,0.87895365306575)(74,0.87895365306575)(74,0.87895365306575)(74,0.87895365306575)(74,0.87895365306575)(74,0.87895365306575)(75,0.879497079213324)(75,0.879497079213324)(75,0.879497079213324)(75,0.879497079213324)(75,0.879497079213324)(75,0.879497079213324)(75,0.879497079213324)(75,0.879497079213324)(75,0.879497079213324)(76,0.879983979577299)(76,0.879983979577299)(76,0.879983979577299)(76,0.879983979577299)(76,0.879983979577299)(76,0.879983979577299)(76,0.879983979577299)(76,0.879983979577299)(77,0.880428644612614)(77,0.880428644612614)(77,0.880428644612614)(77,0.880428644612614)(77,0.880428644612614)(77,0.880428644612614)(77,0.880428644612614)(77,0.880428644612614)(77,0.880428644612614)(77,0.880428644612614)(77,0.880428644612614)(78,0.880884441235036)(78,0.880884441235036)(78,0.880884441235036)(78,0.880884441235036)(78,0.880884441235036)(78,0.880884441235036)(78,0.880884441235036)(78,0.880884441235036)(78,0.880884441235036)(78,0.880884441235036)(79,0.881525322533769)(79,0.881525322533769)(79,0.881525322533769)(79,0.881525322533769)(79,0.881525322533769)(79,0.881525322533769)(79,0.881525322533769)(79,0.881525322533769)(80,0.882069292779447)(80,0.882069292779447)(80,0.882069292779447)(80,0.882069292779447)(80,0.882069292779447)(80,0.882069292779447)(80,0.882069292779447)(80,0.882069292779447)(80,0.882069292779447)(80,0.882069292779447)(80,0.882069292779447)(81,0.882661011436284)(81,0.882661011436284)(81,0.882661011436284)(81,0.882661011436284)(81,0.882661011436284)(81,0.882661011436284)(82,0.883283510096945)(82,0.883283510096945)(82,0.883283510096945)(82,0.883283510096945)(82,0.883283510096945)(82,0.883283510096945)(83,0.883868112059667)(83,0.883868112059667)(83,0.883868112059667)(83,0.883868112059667)(83,0.883868112059667)(83,0.883868112059667)(84,0.884270575794794)(84,0.884270575794794)(84,0.884270575794794)(84,0.884270575794794)(84,0.884270575794794)(84,0.884270575794794)(84,0.884270575794794)(84,0.884270575794794)(84,0.884270575794794)(84,0.884270575794794)(85,0.884624618568871)(85,0.884624618568871)(85,0.884624618568871)(85,0.884624618568871)(85,0.884624618568871)(85,0.884624618568871)(85,0.884624618568871)(85,0.884624618568871)(85,0.884624618568871)(86,0.884937108886295)(86,0.884937108886295)(86,0.884937108886295)(86,0.884937108886295)(86,0.884937108886295)(86,0.884937108886295)(86,0.884937108886295)(86,0.884937108886295)(86,0.884937108886295)(87,0.885080252494528)(87,0.885080252494528)(87,0.885080252494528)(87,0.885080252494528)(87,0.885080252494528)(87,0.885080252494528)(87,0.885080252494528)(87,0.885080252494528)(87,0.885080252494528)(87,0.885080252494528)(87,0.885080252494528)(88,0.885296675031833)(88,0.885296675031833)(88,0.885296675031833)(88,0.885296675031833)(88,0.885296675031833)(88,0.885296675031833)(88,0.885296675031833)(89,0.885504136120846)(89,0.885504136120846)(89,0.885504136120846)(89,0.885504136120846)(89,0.885504136120846)(89,0.885504136120846)(89,0.885504136120846)(90,0.885617405637599)(90,0.885617405637599)(90,0.885617405637599)(90,0.885617405637599)(90,0.885617405637599)(90,0.885617405637599)(90,0.885617405637599)(91,0.885896246301623)(91,0.885896246301623)(91,0.885896246301623)(91,0.885896246301623)(91,0.885896246301623)(91,0.885896246301623)(91,0.885896246301623)(91,0.885896246301623)(91,0.885896246301623)(91,0.885896246301623)(92,0.88621120038927)(92,0.88621120038927)(92,0.88621120038927)(92,0.88621120038927)(92,0.88621120038927)(92,0.88621120038927)(92,0.88621120038927)(92,0.88621120038927)(92,0.88621120038927)(92,0.88621120038927)(92,0.88621120038927)(93,0.8866810990757)(93,0.8866810990757)(93,0.8866810990757)(93,0.8866810990757)(93,0.8866810990757)(93,0.8866810990757)(93,0.8866810990757)(93,0.8866810990757)(93,0.8866810990757)(93,0.8866810990757)(93,0.8866810990757)(94,0.887144966388669)(94,0.887144966388669)(94,0.887144966388669)(94,0.887144966388669)(94,0.887144966388669)(94,0.887144966388669)(94,0.887144966388669)(94,0.887144966388669)(94,0.887144966388669)(95,0.887618801414262)(95,0.887618801414262)(95,0.887618801414262)(95,0.887618801414262)(95,0.887618801414262)(95,0.887618801414262)(95,0.887618801414262)(95,0.887618801414262)(95,0.887618801414262)(95,0.887618801414262)(96,0.888085972647855)(96,0.888085972647855)(96,0.888085972647855)(96,0.888085972647855)(96,0.888085972647855)(96,0.888085972647855)(97,0.888555526147879)(97,0.888555526147879)(97,0.888555526147879)(97,0.888555526147879)(97,0.888555526147879)(97,0.888555526147879)(97,0.888555526147879)(97,0.888555526147879)(97,0.888555526147879)(97,0.888555526147879)(98,0.889203931927193)(98,0.889203931927193)(98,0.889203931927193)(98,0.889203931927193)(98,0.889203931927193)(98,0.889203931927193)(98,0.889203931927193)(99,0.889779126406695)(99,0.889779126406695)(99,0.889779126406695)(99,0.889779126406695)(99,0.889779126406695)(100,0.890410517944253)(100,0.890410517944253)(100,0.890410517944253)(100,0.890410517944253)(100,0.890410517944253)(100,0.890410517944253)(100,0.890410517944253)(100,0.890410517944253)(100,0.890410517944253)(100,0.890410517944253)(100,0.890410517944253)(100,0.890410517944253)(100,0.890410517944253)(100,0.890410517944253)(101,0.891098887988349)(101,0.891098887988349)(101,0.891098887988349)(101,0.891098887988349)(102,0.891861261990652)(102,0.891861261990652)(102,0.891861261990652)(102,0.891861261990652)(102,0.891861261990652)(103,0.89262418989616)(103,0.89262418989616)(103,0.89262418989616)(103,0.89262418989616)(103,0.89262418989616)(103,0.89262418989616)(103,0.89262418989616)(104,0.893483787450481)(104,0.893483787450481)(104,0.893483787450481)(104,0.893483787450481)(104,0.893483787450481)(104,0.893483787450481)(104,0.893483787450481)(104,0.893483787450481)(104,0.893483787450481)(105,0.894412274850794)(105,0.894412274850794)(105,0.894412274850794)(105,0.894412274850794)(105,0.894412274850794)(105,0.894412274850794)(105,0.894412274850794)(106,0.895370698323643)(106,0.895370698323643)(106,0.895370698323643)(107,0.896344780150986)(107,0.896344780150986)(107,0.896344780150986)(107,0.896344780150986)(107,0.896344780150986)(107,0.896344780150986)(107,0.896344780150986)(108,0.897336442355504)(108,0.897336442355504)(108,0.897336442355504)(108,0.897336442355504)(108,0.897336442355504)(109,0.89832614256793)(109,0.89832614256793)(109,0.89832614256793)(109,0.89832614256793)(109,0.89832614256793)(110,0.899215723414439)(110,0.899215723414439)(111,0.900091671229955)(111,0.900091671229955)(111,0.900091671229955)(111,0.900091671229955)(111,0.900091671229955)(111,0.900091671229955)(111,0.900091671229955)(111,0.900091671229955)(112,0.900895387171251)(112,0.900895387171251)(112,0.900895387171251)(112,0.900895387171251)(112,0.900895387171251)(112,0.900895387171251)(113,0.901621042139023)(113,0.901621042139023)(113,0.901621042139023)(113,0.901621042139023)(113,0.901621042139023)(113,0.901621042139023)(114,0.902265862459695)(114,0.902265862459695)(114,0.902265862459695)(114,0.902265862459695)(114,0.902265862459695)(114,0.902265862459695)(114,0.902265862459695)(114,0.902265862459695)(115,0.90282856860764)(115,0.90282856860764)(115,0.90282856860764)(115,0.90282856860764)(115,0.90282856860764)(115,0.90282856860764)(115,0.90282856860764)(115,0.90282856860764)(115,0.90282856860764)(116,0.903302394529468)(116,0.903302394529468)(116,0.903302394529468)(116,0.903302394529468)(116,0.903302394529468)(117,0.90366445589929)(117,0.90366445589929)(117,0.90366445589929)(117,0.90366445589929)(117,0.90366445589929)(117,0.90366445589929)(117,0.90366445589929)(118,0.904052072986014)(118,0.904052072986014)(118,0.904052072986014)(118,0.904052072986014)(118,0.904052072986014)(119,0.90442940929921)(119,0.90442940929921)(119,0.90442940929921)(119,0.90442940929921)(119,0.90442940929921)(119,0.90442940929921)(120,0.904820989336109)(120,0.904820989336109)(120,0.904820989336109)(120,0.904820989336109)(121,0.905290476626895)(121,0.905290476626895)(121,0.905290476626895)(121,0.905290476626895)(121,0.905290476626895)(122,0.905762975609968)(122,0.905762975609968)(122,0.905762975609968)(122,0.905762975609968)(122,0.905762975609968)(122,0.905762975609968)(123,0.906288210386602)(123,0.906288210386602)(123,0.906288210386602)(123,0.906288210386602)(123,0.906288210386602)(123,0.906288210386602)(124,0.906857921242519)(124,0.906857921242519)(124,0.906857921242519)(124,0.906857921242519)(124,0.906857921242519)(124,0.906857921242519)(124,0.906857921242519)(125,0.907458971897663)(125,0.907458971897663)(125,0.907458971897663)(125,0.907458971897663)(125,0.907458971897663)(125,0.907458971897663)(125,0.907458971897663)(125,0.907458971897663)(125,0.907458971897663)(126,0.908073697422993)(126,0.908073697422993)(126,0.908073697422993)(126,0.908073697422993)(126,0.908073697422993)(127,0.908682117119428)(127,0.908682117119428)(127,0.908682117119428)(127,0.908682117119428)(127,0.908682117119428)(127,0.908682117119428)(127,0.908682117119428)(127,0.908682117119428)(128,0.909266995551112)(128,0.909266995551112)(128,0.909266995551112)(128,0.909266995551112)(128,0.909266995551112)(128,0.909266995551112)(128,0.909266995551112)(129,0.909804698477688)(129,0.909804698477688)(129,0.909804698477688)(129,0.909804698477688)(129,0.909804698477688)(129,0.909804698477688)(130,0.910271424406664)(130,0.910271424406664)(130,0.910271424406664)(130,0.910271424406664)(130,0.910271424406664)(130,0.910271424406664)(130,0.910271424406664)(131,0.910666087030618)(131,0.910666087030618)(131,0.910666087030618)(132,0.910994912722979)(132,0.910994912722979)(132,0.910994912722979)(132,0.910994912722979)(132,0.910994912722979)(132,0.910994912722979)(132,0.910994912722979)(133,0.911265224144584)(133,0.911265224144584)(133,0.911265224144584)(133,0.911265224144584)(133,0.911265224144584)(133,0.911265224144584)(134,0.911479562692085)(134,0.911479562692085)(134,0.911479562692085)(134,0.911479562692085)(134,0.911479562692085)(135,0.911647070464452)(135,0.911647070464452)(135,0.911647070464452)(135,0.911647070464452)(135,0.911647070464452)(135,0.911647070464452)(135,0.911647070464452)(135,0.911647070464452)(136,0.911782068522749)(136,0.911782068522749)(136,0.911782068522749)(136,0.911782068522749)(136,0.911782068522749)(137,0.911902289743262)(137,0.911902289743262)(137,0.911902289743262)(137,0.911902289743262)(138,0.912025538806745)(138,0.912025538806745)(138,0.912025538806745)(138,0.912025538806745)(138,0.912025538806745)(138,0.912025538806745)(138,0.912025538806745)(138,0.912025538806745)(138,0.912025538806745)(139,0.912157017168855)(139,0.912157017168855)(139,0.912157017168855)(139,0.912157017168855)(139,0.912157017168855)(139,0.912157017168855)(140,0.912288966028396)(140,0.912288966028396)(140,0.912288966028396)(140,0.912288966028396)(141,0.912411892748327)(141,0.912411892748327)(141,0.912411892748327)(141,0.912411892748327)(141,0.912411892748327)(141,0.912411892748327)(141,0.912411892748327)(141,0.912411892748327)(142,0.912531494087763)(142,0.912531494087763)(142,0.912531494087763)(142,0.912531494087763)(142,0.912531494087763)(142,0.912531494087763)(143,0.912659214864361)(143,0.912659214864361)(143,0.912659214864361)(143,0.912659214864361)(143,0.912659214864361)(143,0.912659214864361)(143,0.912659214864361)(143,0.912659214864361)(143,0.912659214864361)(144,0.912797002438885)(144,0.912797002438885)(144,0.912797002438885)(144,0.912797002438885)(145,0.912970339199763)(145,0.912970339199763)(145,0.912970339199763)(145,0.912970339199763)(145,0.912970339199763)(145,0.912970339199763)(145,0.912970339199763)(145,0.912970339199763)(146,0.913049215410552)(146,0.913049215410552)(146,0.913049215410552)(146,0.913049215410552)(147,0.913085221356563)(147,0.913085221356563)(147,0.913085221356563)(147,0.913085221356563)(147,0.913085221356563)(148,0.913100331670334)(148,0.913100331670334)(148,0.913100331670334)(148,0.913100331670334)(148,0.913100331670334)(148,0.913100331670334)(148,0.913100331670334)(148,0.913100331670334)(149,0.91312152334848)(149,0.91312152334848)(149,0.91312152334848)(149,0.91312152334848)(149,0.91312152334848)(149,0.91312152334848)(150,0.913170983171252)(150,0.913170983171252)(150,0.913170983171252)(150,0.913170983171252)(150,0.913170983171252)(150,0.913170983171252)(150,0.913170983171252)(150,0.913170983171252)(150,0.913170983171252)(151,0.913453143866604)(151,0.913453143866604)(151,0.913453143866604)(152,0.913646758974046)(152,0.913646758974046)(152,0.913646758974046)(152,0.913646758974046)(152,0.913646758974046)(152,0.913646758974046)(152,0.913646758974046)(152,0.913646758974046)(152,0.913646758974046)(152,0.913646758974046)(153,0.913917254304116)(153,0.913917254304116)(153,0.913917254304116)(153,0.913917254304116)(153,0.913917254304116)(154,0.914265043301814)(154,0.914265043301814)(154,0.914265043301814)(154,0.914265043301814)(154,0.914265043301814)(154,0.914265043301814)(154,0.914265043301814)(154,0.914265043301814)(155,0.914675968349094)(155,0.914675968349094)(155,0.914675968349094)(155,0.914675968349094)(155,0.914675968349094)(155,0.914675968349094)(155,0.914675968349094)(156,0.915138446186076)(156,0.915138446186076)(156,0.915138446186076)(156,0.915138446186076)(156,0.915138446186076)(156,0.915138446186076)(157,0.915640033422324)(157,0.915640033422324)(157,0.915640033422324)(157,0.915640033422324)(157,0.915640033422324)(157,0.915640033422324)(158,0.916175342762926)(158,0.916175342762926)(158,0.916175342762926)(158,0.916175342762926)(158,0.916175342762926)(158,0.916175342762926)(158,0.916175342762926)(159,0.916696768984411)(159,0.916696768984411)(159,0.916696768984411)(159,0.916696768984411)(159,0.916696768984411)(159,0.916696768984411)(159,0.916696768984411)(159,0.916696768984411)(160,0.917326693682256)(160,0.917326693682256)(160,0.917326693682256)(160,0.917326693682256)(160,0.917326693682256)(161,0.917867490864265)(161,0.917867490864265)(161,0.917867490864265)(161,0.917867490864265)(161,0.917867490864265)(161,0.917867490864265)(161,0.917867490864265)(161,0.917867490864265)(161,0.917867490864265)(162,0.918341190066385)(162,0.918341190066385)(162,0.918341190066385)(162,0.918341190066385)(162,0.918341190066385)(162,0.918341190066385)(162,0.918341190066385)(162,0.918341190066385)(163,0.918717931029981)(163,0.918717931029981)(163,0.918717931029981)(164,0.918974578453551)(164,0.918974578453551)(164,0.918974578453551)(164,0.918974578453551)(164,0.918974578453551)(164,0.918974578453551)(165,0.918892313269438)(165,0.918892313269438)(165,0.918892313269438)(165,0.918892313269438)(165,0.918892313269438)(166,0.919004286565191)(166,0.919004286565191)(166,0.919004286565191)(166,0.919004286565191)(166,0.919004286565191)(166,0.919004286565191)(166,0.919004286565191)(167,0.919208466633211)(167,0.919208466633211)(167,0.919208466633211)(167,0.919208466633211)(167,0.919208466633211)(167,0.919208466633211)(167,0.919208466633211)(167,0.919208466633211)(168,0.919209525369934)(168,0.919209525369934)(168,0.919209525369934)(168,0.919209525369934)(168,0.919209525369934)(169,0.919227589419664)(169,0.919227589419664)(169,0.919227589419664)(169,0.919227589419664)(169,0.919227589419664)(169,0.919227589419664)(169,0.919227589419664)(169,0.919227589419664)(169,0.919227589419664)(170,0.919228938879919)(170,0.919228938879919)(170,0.919228938879919)(170,0.919228938879919)(170,0.919228938879919)(170,0.919228938879919)(170,0.919228938879919)(170,0.919228938879919)(170,0.919228938879919)(171,0.919302490922463)(171,0.919302490922463)(171,0.919302490922463)(171,0.919302490922463)(171,0.919302490922463)(171,0.919302490922463)(171,0.919302490922463)(172,0.919447206153786)(172,0.919447206153786)(172,0.919447206153786)(172,0.919447206153786)(172,0.919447206153786)(172,0.919447206153786)(172,0.919447206153786)(172,0.919447206153786)(172,0.919447206153786)(173,0.919678970076311)(173,0.919678970076311)(173,0.919678970076311)(173,0.919678970076311)(173,0.919678970076311)(174,0.920003216764254)(174,0.920003216764254)(174,0.920003216764254)(175,0.920147093836222)(175,0.920147093836222)(175,0.920147093836222)(175,0.920147093836222)(175,0.920147093836222)(176,0.920589996613363)(176,0.920589996613363)(176,0.920589996613363)(176,0.920589996613363)(176,0.920589996613363)(176,0.920589996613363)(176,0.920589996613363)(177,0.92106950351799)(177,0.92106950351799)(177,0.92106950351799)(178,0.921693998148347)(178,0.921693998148347)(178,0.921693998148347)(178,0.921693998148347)(178,0.921693998148347)(178,0.921693998148347)(178,0.921693998148347)(179,0.922058867817332)(179,0.922058867817332)(179,0.922058867817332)(180,0.922369993301596)(180,0.922369993301596)(180,0.922369993301596)(180,0.922369993301596)(180,0.922369993301596)(180,0.922369993301596)(181,0.9226598000701)(181,0.9226598000701)(181,0.9226598000701)(181,0.9226598000701)(181,0.9226598000701)(181,0.9226598000701)(181,0.9226598000701)(182,0.922879363294031)(182,0.922879363294031)(182,0.922879363294031)(182,0.922879363294031)(183,0.923010624099443)(183,0.923010624099443)(183,0.923010624099443)(183,0.923010624099443)(183,0.923010624099443)(183,0.923010624099443)(183,0.923010624099443)(183,0.923010624099443)(183,0.923010624099443)(183,0.923010624099443)(183,0.923010624099443)(184,0.923081888901675)(184,0.923081888901675)(184,0.923081888901675)(184,0.923081888901675)(184,0.923081888901675)(184,0.923081888901675)(184,0.923081888901675)(184,0.923081888901675)(185,0.923142748599112)(185,0.923142748599112)(185,0.923142748599112)(185,0.923142748599112)(185,0.923142748599112)(185,0.923142748599112)(186,0.923233115335294)(186,0.923233115335294)(186,0.923233115335294)(186,0.923233115335294)(186,0.923233115335294)(186,0.923233115335294)(186,0.923233115335294)(187,0.923366134857737)(187,0.923366134857737)(187,0.923366134857737)(187,0.923366134857737)(187,0.923366134857737)(188,0.923539277155194)(188,0.923539277155194)(188,0.923539277155194)(188,0.923539277155194)(188,0.923539277155194)(188,0.923539277155194)(188,0.923539277155194)(188,0.923539277155194)(189,0.92387069681485)(189,0.92387069681485)(189,0.92387069681485)(189,0.92387069681485)(189,0.92387069681485)(189,0.92387069681485)(189,0.92387069681485)(189,0.92387069681485)(189,0.92387069681485)(190,0.924032343083996)(190,0.924032343083996)(190,0.924032343083996)(190,0.924032343083996)(191,0.924164300915139)(191,0.924164300915139)(191,0.924164300915139)(191,0.924164300915139)(192,0.924230284390243)(192,0.924230284390243)(192,0.924230284390243)(192,0.924230284390243)(192,0.924230284390243)(192,0.924230284390243)(193,0.924206051309703)(193,0.924206051309703)(193,0.924206051309703)(193,0.924206051309703)(193,0.924206051309703)(193,0.924206051309703)(193,0.924206051309703)(193,0.924206051309703)(193,0.924206051309703)(193,0.924206051309703)(194,0.92408993528439)(194,0.92408993528439)(194,0.92408993528439)(195,0.923921577475897)(195,0.923921577475897)(195,0.923921577475897)(195,0.923921577475897)(195,0.923921577475897)(196,0.923686836089653)(196,0.923686836089653)(196,0.923686836089653)(196,0.923686836089653)(196,0.923686836089653)(196,0.923686836089653)(196,0.923686836089653)(196,0.923686836089653)(196,0.923686836089653)(196,0.923686836089653)(197,0.92340494913761)(197,0.92340494913761)(197,0.92340494913761)(197,0.92340494913761)(197,0.92340494913761)(197,0.92340494913761)(197,0.92340494913761)(198,0.923096399506605)(198,0.923096399506605)(198,0.923096399506605)(198,0.923096399506605)(198,0.923096399506605)(198,0.923096399506605)(198,0.923096399506605)(198,0.923096399506605)(198,0.923096399506605)(198,0.923096399506605)(198,0.923096399506605)(198,0.923096399506605)(199,0.922788035664489)(199,0.922788035664489)(199,0.922788035664489)(199,0.922788035664489)(199,0.922788035664489)(199,0.922788035664489)(199,0.922788035664489)(199,0.922788035664489)(199,0.922788035664489)(200,0.922501962114206)(200,0.922501962114206)(200,0.922501962114206)(200,0.922501962114206)(200,0.922501962114206)(200,0.922501962114206)(200,0.922501962114206)(201,0.922252395594164)(201,0.922252395594164)(201,0.922252395594164)(201,0.922252395594164)(201,0.922252395594164)(201,0.922252395594164)(201,0.922252395594164)(201,0.922252395594164)(201,0.922252395594164)(201,0.922252395594164)(202,0.921987535368166)(202,0.921987535368166)(202,0.921987535368166)(202,0.921987535368166)(202,0.921987535368166)(203,0.921866979232156)(203,0.921866979232156)(203,0.921866979232156)(203,0.921866979232156)(203,0.921866979232156)(203,0.921866979232156)(204,0.921862204062474)(204,0.921862204062474)(204,0.921862204062474)(204,0.921862204062474)(204,0.921862204062474)(205,0.922018885850143)(205,0.922018885850143)(205,0.922018885850143)(205,0.922018885850143)(205,0.922018885850143)(205,0.922018885850143)(205,0.922018885850143)(206,0.922404613114075)(206,0.922404613114075)(206,0.922404613114075)(206,0.922404613114075)(206,0.922404613114075)(206,0.922404613114075)(206,0.922404613114075)(206,0.922404613114075)(206,0.922404613114075)(206,0.922404613114075)(207,0.92287352706863)(207,0.92287352706863)(207,0.92287352706863)(207,0.92287352706863)(207,0.92287352706863)(207,0.92287352706863)(207,0.92287352706863)(207,0.92287352706863)(208,0.923395728059796)(208,0.923395728059796)(208,0.923395728059796)(208,0.923395728059796)(208,0.923395728059796)(208,0.923395728059796)(209,0.923946230294363)(209,0.923946230294363)(209,0.923946230294363)(209,0.923946230294363)(209,0.923946230294363)(210,0.924521795051165)(210,0.924521795051165)(210,0.924521795051165)(210,0.924521795051165)(210,0.924521795051165)(210,0.924521795051165)(210,0.924521795051165)(210,0.924521795051165)(210,0.924521795051165)(211,0.92507066762613)(211,0.92507066762613)(211,0.92507066762613)(211,0.92507066762613)(211,0.92507066762613)(211,0.92507066762613)(211,0.92507066762613)(211,0.92507066762613)(211,0.92507066762613)(212,0.925593336605208)(212,0.925593336605208)(212,0.925593336605208)(212,0.925593336605208)(212,0.925593336605208)(212,0.925593336605208)(212,0.925593336605208)(212,0.925593336605208)(212,0.925593336605208)(212,0.925593336605208)(212,0.925593336605208)(212,0.925593336605208)(213,0.92609027066118)(213,0.92609027066118)(213,0.92609027066118)(213,0.92609027066118)(213,0.92609027066118)(213,0.92609027066118)(213,0.92609027066118)(213,0.92609027066118)(214,0.926557249575485)(214,0.926557249575485)(214,0.926557249575485)(214,0.926557249575485)(214,0.926557249575485)(215,0.926909574017588)(215,0.926909574017588)(215,0.926909574017588)(215,0.926909574017588)(215,0.926909574017588)(215,0.926909574017588)(215,0.926909574017588)(216,0.9273384048181)(216,0.9273384048181)(216,0.9273384048181)(216,0.9273384048181)(216,0.9273384048181)(216,0.9273384048181)(216,0.9273384048181)(216,0.9273384048181)(216,0.9273384048181)(216,0.9273384048181)(216,0.9273384048181)(216,0.9273384048181)(217,0.927795265164317)(217,0.927795265164317)(217,0.927795265164317)(217,0.927795265164317)(217,0.927795265164317)(217,0.927795265164317)(218,0.928300989943222)(218,0.928300989943222)(218,0.928300989943222)(218,0.928300989943222)(218,0.928300989943222)(218,0.928300989943222)(219,0.928858880520682)(219,0.928858880520682)(219,0.928858880520682)(220,0.929448392740186)(220,0.929448392740186)(220,0.929448392740186)(220,0.929448392740186)(220,0.929448392740186)(220,0.929448392740186)(220,0.929448392740186)(221,0.930009018963276)(221,0.930009018963276)(221,0.930009018963276)(222,0.930486980244315)(222,0.930486980244315)(222,0.930486980244315)(223,0.930590901576675)(223,0.930590901576675)(223,0.930590901576675)(223,0.930590901576675)(223,0.930590901576675)(223,0.930590901576675)(223,0.930590901576675)(223,0.930590901576675)(223,0.930590901576675)(223,0.930590901576675)(224,0.93114736483215)(224,0.93114736483215)(224,0.93114736483215)(224,0.93114736483215)(225,0.931330264124905)(225,0.931330264124905)(225,0.931330264124905)(225,0.931330264124905)(225,0.931330264124905)(225,0.931330264124905)(225,0.931330264124905)(225,0.931330264124905)(225,0.931330264124905)(225,0.931330264124905)(226,0.931425315510514)(226,0.931425315510514)(226,0.931425315510514)(226,0.931425315510514)(226,0.931425315510514)(226,0.931425315510514)(226,0.931425315510514)(226,0.931425315510514)(226,0.931425315510514)(227,0.931485377316195)(227,0.931485377316195)(227,0.931485377316195)(227,0.931485377316195)(227,0.931485377316195)(227,0.931485377316195)(227,0.931485377316195)(227,0.931485377316195)(227,0.931485377316195)(227,0.931485377316195)(227,0.931485377316195)(228,0.93135741700128)(228,0.93135741700128)(228,0.93135741700128)(228,0.93135741700128)(228,0.93135741700128)(228,0.93135741700128)(228,0.93135741700128)(228,0.93135741700128)(229,0.931229067391955)(229,0.931229067391955)(229,0.931229067391955)(229,0.931229067391955)(229,0.931229067391955)(229,0.931229067391955)(229,0.931229067391955)(229,0.931229067391955)(229,0.931229067391955)(229,0.931229067391955)(230,0.931123285836686)(230,0.931123285836686)(230,0.931123285836686)(230,0.931123285836686)(230,0.931123285836686)(230,0.931123285836686)(230,0.931123285836686)(230,0.931123285836686)(231,0.931040292263274)(231,0.931040292263274)(231,0.931040292263274)(231,0.931040292263274)(231,0.931040292263274)(231,0.931040292263274)(232,0.930990176843429)(232,0.930990176843429)(232,0.930990176843429)(232,0.930990176843429)(233,0.930969949307365)(233,0.930969949307365)(233,0.930969949307365)(233,0.930969949307365)(233,0.930969949307365)(233,0.930969949307365)(233,0.930969949307365)(234,0.93095587475416)(234,0.93095587475416)(234,0.93095587475416)(234,0.93095587475416)(234,0.93095587475416)(235,0.930932180950164)(235,0.930932180950164)(235,0.930932180950164)(235,0.930932180950164)(235,0.930932180950164)(235,0.930932180950164)(235,0.930932180950164)(236,0.930864021882447)(236,0.930864021882447)(236,0.930864021882447)(236,0.930864021882447)(236,0.930864021882447)(236,0.930864021882447)(237,0.930785214149691)(237,0.930785214149691)(237,0.930785214149691)(237,0.930785214149691)(237,0.930785214149691)(237,0.930785214149691)(237,0.930785214149691)(237,0.930785214149691)(237,0.930785214149691)(238,0.930545805270038)(238,0.930545805270038)(238,0.930545805270038)(238,0.930545805270038)(238,0.930545805270038)(238,0.930545805270038)(238,0.930545805270038)(238,0.930545805270038)(239,0.930439951570301)(239,0.930439951570301)(239,0.930439951570301)(239,0.930439951570301)(239,0.930439951570301)(240,0.930370388221364)(240,0.930370388221364)(240,0.930370388221364)(240,0.930370388221364)(240,0.930370388221364)(240,0.930370388221364)(240,0.930370388221364)(240,0.930370388221364)(240,0.930370388221364)(241,0.930344219139873)(241,0.930344219139873)(241,0.930344219139873)(241,0.930344219139873)(241,0.930344219139873)(241,0.930344219139873)(241,0.930344219139873)(241,0.930344219139873)(241,0.930344219139873)(241,0.930344219139873)(241,0.930344219139873)(241,0.930344219139873)(242,0.930359789430444)(242,0.930359789430444)(242,0.930359789430444)(242,0.930359789430444)(242,0.930359789430444)(242,0.930359789430444)(242,0.930359789430444)(242,0.930359789430444)(243,0.930414236149959)(243,0.930414236149959)(243,0.930414236149959)(243,0.930414236149959)(243,0.930414236149959)(243,0.930414236149959)(243,0.930414236149959)(243,0.930414236149959)(243,0.930414236149959)(243,0.930414236149959)(244,0.930556630074303)(244,0.930556630074303)(244,0.930556630074303)(244,0.930556630074303)(244,0.930556630074303)(244,0.930556630074303)(244,0.930556630074303)(244,0.930556630074303)(245,0.930647427813468)(245,0.930647427813468)(245,0.930647427813468)(245,0.930647427813468)(246,0.930785229644968)(246,0.930785229644968)(247,0.930972597381396)(247,0.930972597381396)(247,0.930972597381396)(247,0.930972597381396)(247,0.930972597381396)(247,0.930972597381396)(247,0.930972597381396)(247,0.930972597381396)(247,0.930972597381396)(247,0.930972597381396)(247,0.930972597381396)(247,0.930972597381396)(247,0.930972597381396)(247,0.930972597381396)(247,0.930972597381396)(248,0.931226570515829)(248,0.931226570515829)(248,0.931226570515829)(248,0.931226570515829)(248,0.931226570515829)(248,0.931226570515829)(248,0.931226570515829)(248,0.931226570515829)(248,0.931226570515829)(248,0.931226570515829)(248,0.931226570515829)(249,0.931564331868745)(249,0.931564331868745)(249,0.931564331868745)(249,0.931564331868745)(249,0.931564331868745)(249,0.931564331868745)(249,0.931564331868745)(249,0.931564331868745)(249,0.931564331868745)(249,0.931564331868745)(249,0.931564331868745)(250,0.931971380568781)(250,0.931971380568781)(250,0.931971380568781)(250,0.931971380568781)(250,0.931971380568781)(250,0.931971380568781)(250,0.931971380568781)(250,0.931971380568781)(251,0.932419799014061)(251,0.932419799014061)(251,0.932419799014061)(251,0.932419799014061)(251,0.932419799014061)(251,0.932419799014061)(251,0.932419799014061)(252,0.932887537332004)(252,0.932887537332004)(252,0.932887537332004)(252,0.932887537332004)(252,0.932887537332004)(252,0.932887537332004)(252,0.932887537332004)(252,0.932887537332004)(252,0.932887537332004)(253,0.933346342165302)(253,0.933346342165302)(253,0.933346342165302)(253,0.933346342165302)(254,0.933785813449489)(254,0.933785813449489)(254,0.933785813449489)(254,0.933785813449489)(254,0.933785813449489)(254,0.933785813449489)(254,0.933785813449489)(254,0.933785813449489)(254,0.933785813449489)(255,0.934212384311136)(255,0.934212384311136)(255,0.934212384311136)(255,0.934212384311136)(255,0.934212384311136)(255,0.934212384311136)(256,0.934631342762802)(256,0.934631342762802)(256,0.934631342762802)(256,0.934631342762802)(256,0.934631342762802)(256,0.934631342762802)(256,0.934631342762802)(256,0.934631342762802)(256,0.934631342762802)(257,0.935052139100331)(257,0.935052139100331)(257,0.935052139100331)(257,0.935052139100331)(257,0.935052139100331)(257,0.935052139100331)(257,0.935052139100331)(257,0.935052139100331)(257,0.935052139100331)(257,0.935052139100331)(257,0.935052139100331)(257,0.935052139100331)(257,0.935052139100331)(257,0.935052139100331)(258,0.935508809929941)(258,0.935508809929941)(258,0.935508809929941)(259,0.936016701477708)(259,0.936016701477708)(259,0.936016701477708)(259,0.936016701477708)(259,0.936016701477708)(259,0.936016701477708)(259,0.936016701477708)(260,0.93633319614183)(260,0.93633319614183)(260,0.93633319614183)(260,0.93633319614183)(261,0.937102954477552)(261,0.937102954477552)(261,0.937102954477552)(261,0.937102954477552)(262,0.937606462772912)(262,0.937606462772912)(262,0.937606462772912)(262,0.937606462772912)(262,0.937606462772912)(262,0.937606462772912)(262,0.937606462772912)(263,0.937912104505906)(263,0.937912104505906)(263,0.937912104505906)(263,0.937912104505906)(263,0.937912104505906)(263,0.937912104505906)(263,0.937912104505906)(263,0.937912104505906)(263,0.937912104505906)(263,0.937912104505906)(263,0.937912104505906)(263,0.937912104505906)(264,0.938342346329133)(264,0.938342346329133)(264,0.938342346329133)(265,0.938622825548547)(265,0.938622825548547)(265,0.938622825548547)(265,0.938622825548547)(265,0.938622825548547)(265,0.938622825548547)(265,0.938622825548547)(266,0.93891634941725)(266,0.93891634941725)(266,0.93891634941725)(266,0.93891634941725)(266,0.93891634941725)(266,0.93891634941725)(266,0.93891634941725)(266,0.93891634941725)(266,0.93891634941725)(267,0.939141462624407)(267,0.939141462624407)(267,0.939141462624407)(267,0.939141462624407)(267,0.939141462624407)(268,0.939370326124655)(268,0.939370326124655)(268,0.939370326124655)(268,0.939370326124655)(268,0.939370326124655)(268,0.939370326124655)(269,0.939452165348286)(269,0.939452165348286)(269,0.939452165348286)(269,0.939452165348286)(269,0.939452165348286)(269,0.939452165348286)(269,0.939452165348286)(269,0.939452165348286)(269,0.939452165348286)(269,0.939452165348286)(269,0.939452165348286)(270,0.939579828203442)(270,0.939579828203442)(270,0.939579828203442)(270,0.939579828203442)(270,0.939579828203442)(270,0.939579828203442)(270,0.939579828203442)(270,0.939579828203442)(271,0.939735728780331)(271,0.939735728780331)(271,0.939735728780331)(271,0.939735728780331)(271,0.939735728780331)(271,0.939735728780331)(271,0.939735728780331)(271,0.939735728780331)(271,0.939735728780331)(271,0.939735728780331)(271,0.939735728780331)(271,0.939735728780331)(272,0.940097573071384)(272,0.940097573071384)(272,0.940097573071384)(272,0.940097573071384)(272,0.940097573071384)(272,0.940097573071384)(272,0.940097573071384)(272,0.940097573071384)(273,0.940360968649988)(273,0.940360968649988)(273,0.940360968649988)(273,0.940360968649988)(273,0.940360968649988)(273,0.940360968649988)(273,0.940360968649988)(274,0.940679299029566)(274,0.940679299029566)(274,0.940679299029566)(274,0.940679299029566)(274,0.940679299029566)(274,0.940679299029566)(274,0.940679299029566)(275,0.941044841474603)(275,0.941044841474603)(275,0.941044841474603)(275,0.941044841474603)(275,0.941044841474603)(275,0.941044841474603)(275,0.941044841474603)(275,0.941044841474603)(276,0.941449229898842)(276,0.941449229898842)(276,0.941449229898842)(276,0.941449229898842)(276,0.941449229898842)(276,0.941449229898842)(276,0.941449229898842)(276,0.941449229898842)(276,0.941449229898842)(277,0.941873046097793)(277,0.941873046097793)(277,0.941873046097793)(277,0.941873046097793)(277,0.941873046097793)(277,0.941873046097793)(277,0.941873046097793)(278,0.942285337021387)(278,0.942285337021387)(279,0.942664781164883)(279,0.942664781164883)(279,0.942664781164883)(279,0.942664781164883)(279,0.942664781164883)(279,0.942664781164883)(280,0.943000623995104)(280,0.943000623995104)(280,0.943000623995104)(280,0.943000623995104)(280,0.943000623995104)(280,0.943000623995104)(281,0.943291953907347)(281,0.943291953907347)(281,0.943291953907347)(281,0.943291953907347)(281,0.943291953907347)(281,0.943291953907347)(282,0.943552364856462)(282,0.943552364856462)(282,0.943552364856462)(282,0.943552364856462)(282,0.943552364856462)(283,0.943784962514867)(283,0.943784962514867)(283,0.943784962514867)(283,0.943784962514867)(283,0.943784962514867)(283,0.943784962514867)(283,0.943784962514867)(283,0.943784962514867)(283,0.943784962514867)(283,0.943784962514867)(284,0.943984390365138)(284,0.943984390365138)(284,0.943984390365138)(284,0.943984390365138)(284,0.943984390365138)(284,0.943984390365138)(285,0.944144980860743)(285,0.944144980860743)(285,0.944144980860743)(285,0.944144980860743)(285,0.944144980860743)(286,0.944286228332822)(286,0.944286228332822)(286,0.944286228332822)(287,0.944394054141991)(287,0.944394054141991)(287,0.944394054141991)(287,0.944394054141991)(287,0.944394054141991)(287,0.944394054141991)(287,0.944394054141991)(288,0.94452553944118)(288,0.94452553944118)(288,0.94452553944118)(288,0.94452553944118)(289,0.944692690010736)(289,0.944692690010736)(289,0.944692690010736)(290,0.944917019878289)(290,0.944917019878289)(290,0.944917019878289)(290,0.944917019878289)(291,0.945207576196414)(291,0.945207576196414)(291,0.945207576196414)(291,0.945207576196414)(291,0.945207576196414)(291,0.945207576196414)(291,0.945207576196414)(291,0.945207576196414)(292,0.94557134831913)(292,0.94557134831913)(292,0.94557134831913)(293,0.94589533748509)(293,0.94589533748509)(293,0.94589533748509)(293,0.94589533748509)(293,0.94589533748509)(293,0.94589533748509)(293,0.94589533748509)(293,0.94589533748509)(293,0.94589533748509)(293,0.94589533748509)(293,0.94589533748509)(294,0.946519065796333)(294,0.946519065796333)(294,0.946519065796333)(295,0.94706717789734)(295,0.94706717789734)(295,0.94706717789734)(295,0.94706717789734)(295,0.94706717789734)(296,0.947644294913962)(296,0.947644294913962)(296,0.947644294913962)(296,0.947644294913962)(296,0.947644294913962)(296,0.947644294913962)(296,0.947644294913962)(296,0.947644294913962)(297,0.948322053791683)(297,0.948322053791683)(297,0.948322053791683)(298,0.948937412765218)(298,0.948937412765218)(298,0.948937412765218)(298,0.948937412765218)(298,0.948937412765218)(298,0.948937412765218)(298,0.948937412765218)(298,0.948937412765218)(299,0.949572832647687)(299,0.949572832647687)(299,0.949572832647687)(299,0.949572832647687)(299,0.949572832647687)(299,0.949572832647687)(299,0.949572832647687)(300,0.950228249802406)(300,0.950228249802406)(300,0.950228249802406)(300,0.950228249802406)(300,0.950228249802406)(300,0.950228249802406)(300,0.950228249802406)(300,0.950228249802406)(300,0.950228249802406)(300,0.950228249802406)(300,0.950228249802406)(301,0.95090035956721)(301,0.95090035956721)(301,0.95090035956721)(301,0.95090035956721)(301,0.95090035956721)(301,0.95090035956721)(301,0.95090035956721)(301,0.95090035956721)(302,0.951583610289534)(302,0.951583610289534)(302,0.951583610289534)(302,0.951583610289534)(302,0.951583610289534)(302,0.951583610289534)(302,0.951583610289534)(302,0.951583610289534)(302,0.951583610289534)(302,0.951583610289534)(303,0.952278762226028)(303,0.952278762226028)(303,0.952278762226028)(303,0.952278762226028)(303,0.952278762226028)(303,0.952278762226028)(303,0.952278762226028)(303,0.952278762226028)(303,0.952278762226028)(304,0.952978886687073)(304,0.952978886687073)(304,0.952978886687073)(304,0.952978886687073)(304,0.952978886687073)(305,0.953669302637451)(305,0.953669302637451)(305,0.953669302637451)(305,0.953669302637451)(305,0.953669302637451)(305,0.953669302637451)(305,0.953669302637451)(305,0.953669302637451)(305,0.953669302637451)(305,0.953669302637451)(305,0.953669302637451)(306,0.954336361550128)(306,0.954336361550128)(306,0.954336361550128)(306,0.954336361550128)(306,0.954336361550128)(306,0.954336361550128)(306,0.954336361550128)(307,0.954961117376979)(307,0.954961117376979)(307,0.954961117376979)(307,0.954961117376979)(307,0.954961117376979)(307,0.954961117376979)(307,0.954961117376979)(307,0.954961117376979)(307,0.954961117376979)(307,0.954961117376979)(308,0.955524529771646)(308,0.955524529771646)(308,0.955524529771646)(308,0.955524529771646)(308,0.955524529771646)(308,0.955524529771646)(308,0.955524529771646)(309,0.956009312602022)(309,0.956009312602022)(309,0.956009312602022)(309,0.956009312602022)(309,0.956009312602022)(310,0.956406909840839)(310,0.956406909840839)(310,0.956406909840839)(310,0.956406909840839)(310,0.956406909840839)(310,0.956406909840839)(310,0.956406909840839)(310,0.956406909840839)(310,0.956406909840839)(311,0.956712479644112)(311,0.956712479644112)(311,0.956712479644112)(312,0.956933905690378)(312,0.956933905690378)(312,0.956933905690378)(312,0.956933905690378)(312,0.956933905690378)(312,0.956933905690378)(312,0.956933905690378)(312,0.956933905690378)(313,0.957087205412894)(313,0.957087205412894)(313,0.957087205412894)(313,0.957087205412894)(313,0.957087205412894)(314,0.957153415696524)(314,0.957153415696524)(314,0.957153415696524)(314,0.957153415696524)(314,0.957153415696524)(315,0.957257115405326)(315,0.957257115405326)(315,0.957257115405326)(315,0.957257115405326)(315,0.957257115405326)(315,0.957257115405326)(315,0.957257115405326)(315,0.957257115405326)(315,0.957257115405326)(315,0.957257115405326)(315,0.957257115405326)(316,0.957299632416401)(316,0.957299632416401)(316,0.957299632416401)(316,0.957299632416401)(316,0.957299632416401)(316,0.957299632416401)(317,0.957332779610485)(317,0.957332779610485)(317,0.957332779610485)(317,0.957332779610485)(317,0.957332779610485)(317,0.957332779610485)(317,0.957332779610485)(317,0.957332779610485)(317,0.957332779610485)(317,0.957332779610485)(318,0.957369601128973)(318,0.957369601128973)(318,0.957369601128973)(318,0.957369601128973)(319,0.957419501599356)(319,0.957419501599356)(319,0.957419501599356)(319,0.957419501599356)(319,0.957419501599356)(319,0.957419501599356)(319,0.957419501599356)(320,0.957554578418521)(320,0.957554578418521)(320,0.957554578418521)(321,0.95757200454874)(321,0.95757200454874)(321,0.95757200454874)(321,0.95757200454874)(321,0.95757200454874)(321,0.95757200454874)(321,0.95757200454874)(322,0.957744224569694)(322,0.957744224569694)(322,0.957744224569694)(322,0.957744224569694)(322,0.957744224569694)(322,0.957744224569694)(322,0.957744224569694)(323,0.957865192539705)(323,0.957865192539705)(323,0.957865192539705)(323,0.957865192539705)(324,0.957997338079995)(324,0.957997338079995)(324,0.957997338079995)(324,0.957997338079995)(324,0.957997338079995)(325,0.958144410462858)(325,0.958144410462858)(325,0.958144410462858)(325,0.958144410462858)(326,0.958309372346608)(326,0.958309372346608)(326,0.958309372346608)(326,0.958309372346608)(326,0.958309372346608)(326,0.958309372346608)(326,0.958309372346608)(326,0.958309372346608)(327,0.958565900893205)(327,0.958565900893205)(327,0.958565900893205)(327,0.958565900893205)(327,0.958565900893205)(327,0.958565900893205)(327,0.958565900893205)(327,0.958565900893205)(328,0.958757008128417)(328,0.958757008128417)(328,0.958757008128417)(328,0.958757008128417)(328,0.958757008128417)(328,0.958757008128417)(328,0.958757008128417)(328,0.958757008128417)(328,0.958757008128417)(328,0.958757008128417)(329,0.958965442289426)(329,0.958965442289426)(329,0.958965442289426)(329,0.958965442289426)(330,0.959192098622117)(330,0.959192098622117)(330,0.959192098622117)(330,0.959192098622117)(330,0.959192098622117)(330,0.959192098622117)(330,0.959192098622117)(330,0.959192098622117)(330,0.959192098622117)(330,0.959192098622117)(330,0.959192098622117)(331,0.959461168890595)(331,0.959461168890595)(331,0.959461168890595)(331,0.959461168890595)(331,0.959461168890595)(331,0.959461168890595)(332,0.959712129129329)(332,0.959712129129329)(332,0.959712129129329)(332,0.959712129129329)(333,0.959975800674616)(333,0.959975800674616)(333,0.959975800674616)(333,0.959975800674616)(333,0.959975800674616)(333,0.959975800674616)(334,0.960239914606844)(334,0.960239914606844)(334,0.960239914606844)(334,0.960239914606844)(335,0.960511767772069)(335,0.960511767772069)(335,0.960511767772069)(335,0.960511767772069)(335,0.960511767772069)(335,0.960511767772069)(335,0.960511767772069)(335,0.960511767772069)(336,0.960787048933944)(336,0.960787048933944)(336,0.960787048933944)(336,0.960787048933944)(337,0.961016494061257)(337,0.961016494061257)(337,0.961016494061257)(337,0.961016494061257)(337,0.961016494061257)(338,0.961277467829354)(338,0.961277467829354)(338,0.961277467829354)(338,0.961277467829354)(339,0.961460023501389)(339,0.961460023501389)(339,0.961460023501389)(339,0.961460023501389)(339,0.961460023501389)(340,0.961682983258959)(340,0.961682983258959)(340,0.961682983258959)(341,0.9618102486193)(341,0.9618102486193)(341,0.9618102486193)(341,0.9618102486193)(341,0.9618102486193)(341,0.9618102486193)(341,0.9618102486193)(341,0.9618102486193)(342,0.961996971644293)(342,0.961996971644293)(342,0.961996971644293)(343,0.962094787604586)(343,0.962094787604586)(343,0.962094787604586)(343,0.962094787604586)(344,0.962252116891891)(344,0.962252116891891)(344,0.962252116891891)(344,0.962252116891891)(344,0.962252116891891)(344,0.962252116891891)(344,0.962252116891891)(344,0.962252116891891)(345,0.962333738978964)(345,0.962333738978964)(345,0.962333738978964)(346,0.962415743467908)(346,0.962415743467908)(346,0.962415743467908)(346,0.962415743467908)(347,0.962533489062679)(347,0.962533489062679)(347,0.962533489062679)(347,0.962533489062679)(347,0.962533489062679)(347,0.962533489062679)(347,0.962533489062679)(347,0.962533489062679)(347,0.962533489062679)(347,0.962533489062679)(348,0.962628978097089)(349,0.962678182208823)(349,0.962678182208823)(349,0.962678182208823)(349,0.962678182208823)(349,0.962678182208823)(350,0.962713468270638)(351,0.962756014630323)(351,0.962756014630323)(351,0.962756014630323)(351,0.962756014630323)(351,0.962756014630323)(352,0.962771307621729)(352,0.962771307621729)(352,0.962771307621729)(352,0.962771307621729)(352,0.962771307621729)(353,0.962781379915339)(353,0.962781379915339)(353,0.962781379915339)(353,0.962781379915339)(353,0.962781379915339)(353,0.962781379915339)(354,0.962791240004302)(354,0.962791240004302)(354,0.962791240004302)(354,0.962791240004302)(354,0.962791240004302)(355,0.962795724513849)(355,0.962795724513849)(355,0.962795724513849)(355,0.962795724513849)(355,0.962795724513849)(356,0.962803454805453)(356,0.962803454805453)(356,0.962803454805453)(356,0.962803454805453)(357,0.962814412623668)(357,0.962814412623668)(357,0.962814412623668)(357,0.962814412623668)(357,0.962814412623668)(357,0.962814412623668)(357,0.962814412623668)(357,0.962814412623668)(358,0.96283229362331)(358,0.96283229362331)(358,0.96283229362331)(359,0.962859207368946)(359,0.962859207368946)(359,0.962859207368946)(359,0.962859207368946)(360,0.96289508040129)(360,0.96289508040129)(360,0.96289508040129)(360,0.96289508040129)(360,0.96289508040129)(360,0.96289508040129)(360,0.96289508040129)(361,0.962938760176549)(361,0.962938760176549)(361,0.962938760176549)(361,0.962938760176549)(362,0.962988834705437)(362,0.962988834705437)(363,0.963043960579464)(363,0.963043960579464)(363,0.963043960579464)(363,0.963043960579464)(363,0.963043960579464)(364,0.96310301669742)(364,0.96310301669742)(364,0.96310301669742)(364,0.96310301669742)(365,0.963165115333559)(365,0.963165115333559)(365,0.963165115333559)(365,0.963165115333559)(366,0.963229596987339)(366,0.963229596987339)(366,0.963229596987339)(366,0.963229596987339)(366,0.963229596987339)(367,0.963295992720769)(367,0.963295992720769)(368,0.96336398418692)(368,0.96336398418692)(368,0.96336398418692)(368,0.96336398418692)(368,0.96336398418692)(369,0.963433325115759)(369,0.963433325115759)(370,0.963503798971485)(370,0.963503798971485)(370,0.963503798971485)(370,0.963503798971485)(372,0.963647623799517)(372,0.963647623799517)(372,0.963647623799517)(372,0.963647623799517)(373,0.963720855700045)(374,0.963794965505029)(374,0.963794965505029)(374,0.963794965505029)(374,0.963794965505029)(375,0.963869999387688)(375,0.963869999387688)(376,0.963946037131751)(376,0.963946037131751)(377,0.964023179324354)(377,0.964023179324354)(378,0.964101532858321)(379,0.964181216008743)(379,0.964181216008743)(379,0.964181216008743)(380,0.964262373287412)(380,0.964262373287412)(381,0.964345173191453)(381,0.964345173191453)(382,0.964429775192126)(382,0.964429775192126)(382,0.964429775192126)(384,0.96460503106239)(384,0.96460503106239)(384,0.96460503106239)(384,0.96460503106239)(386,0.964789293406043)(386,0.964789293406043)(386,0.964789293406043)(388,0.96498346986617)(388,0.96498346986617)(388,0.96498346986617)(389,0.965084580603826)(390,0.965188508045022)(394,0.965633587009639)(394,0.965633587009639)(395,0.965752382541782)(395,0.965752382541782)(399,0.966256640961473) 
};
\addlegendentry{\acl};

\addplot [
color=orange,
densely dotted,
line width=1.0pt,
]
coordinates{
 %(1,0.224860681110937)(2,0.271657574537976)(3,0.300526725391003)(4,0.336698550774987)(5,0.399993575392944)(6,0.461334623266916)
 (7,0.521011954842947)(8,0.562358428008174)(9,0.607658122888917)(10,0.66054827318862)(11,0.686505563462091)(12,0.737145896938609)(13,0.75044076165349)(14,0.754566989131763)(15,0.759609535471249)(16,0.763370393469323)(17,0.769519617423755)(18,0.773177204779801)(19,0.777443076958797)(20,0.782968566319223)(21,0.786483100767601)(22,0.785074491810638)(23,0.789226026155552)(24,0.793135169048584)(25,0.795151944725089)(26,0.796172687356919)(27,0.798211143846043)(28,0.799994297833531)(29,0.805666958670374)(30,0.8070602883488)(31,0.808168619170309)(32,0.808960949448589)(33,0.810372929753187)(34,0.811784074463852)(35,0.814254952425789)(36,0.812496930769639)(37,0.812946125851982)(38,0.815614291432219)(39,0.816491124846746)(40,0.816744636601799)(41,0.817363179803974)(42,0.820367231272113)(43,0.820922473517271)(44,0.823008917182164)(45,0.824021010614181)(46,0.824863284959863)(47,0.824725558453891)(48,0.824294919736229)(49,0.824679546025999)(50,0.824311578655163)(51,0.824205301487044)(52,0.823833808446607)(53,0.824087083088491)(54,0.823292521878205)(55,0.824030546405838)(56,0.825411334352975)(57,0.826700143514461)(58,0.828193218735384)(59,0.827948816823345)(60,0.827779948642111)(61,0.827986235291611)(62,0.829177169405015)(63,0.828625329394402)(64,0.828870815930447)(65,0.829646316228933)(66,0.829390734369972)(67,0.829335977848882)(68,0.828927372290689)(69,0.829637104383602)(70,0.829952732990551)(71,0.82998198581965)(72,0.831956435813732)(73,0.833370645582724)(74,0.833463558400175)(75,0.833682884027741)(76,0.833895921208776)(77,0.833966192526803)(78,0.834758200324826)(79,0.83502984355942)(80,0.83484146307135)(81,0.835275285328352)(82,0.835731851173602)(83,0.835243388487006)(84,0.835159098785831)(85,0.835025415703491)(86,0.835222645405416)(87,0.835259933004103)(88,0.835592794928088)(89,0.835725337599699)(90,0.835735576105857)(91,0.835393830805628)(92,0.836165319774548)(93,0.835506596050083)(94,0.836017637023527)(95,0.836063151848645)(96,0.836298064698853)(97,0.836643825620152)(98,0.836245457127367)(99,0.835959389441024)(100,0.835060643413199)(101,0.83555398907344)(102,0.835719288113516)(103,0.83546296340582)(104,0.835084317137969)(105,0.834715558325055)(106,0.834755139454)(107,0.834179408650832)(108,0.834192640916434)(109,0.833993438146871)(110,0.833972007081018)(111,0.834117906788762)(112,0.834055741325407)(113,0.834330933346347)(114,0.834339474384642)(115,0.833390358891719)(116,0.83179147871771)(117,0.832172351669302)(118,0.832394683917233)(119,0.832571924775869)(120,0.832501884520966)(121,0.832798372911935)(122,0.832046530206314)(123,0.83285347621058)(124,0.832779605366282)(125,0.832060756998194)(126,0.832790314204287)(127,0.833048633330265)(128,0.833118384720763)(129,0.834197277220518)(130,0.836168649214958)(131,0.835961670104255)(132,0.835749597018871)(133,0.835832927790962)(134,0.836160322740637)(135,0.837098503566625)(136,0.837150483482235)(137,0.837317007155232)(138,0.837188259280032)(139,0.83690258812139)(140,0.837282631269581)(141,0.837819083780669)(142,0.838186814963269)(143,0.837880799157636)(144,0.837863602186617)(145,0.837927274997628)(146,0.837963697664746)(147,0.837105621465609)(148,0.837324512920695)(149,0.837585909570754)(150,0.837620918561657)(151,0.837004825930956)(152,0.838479684370174)(153,0.838255090036944)(154,0.838280188403978)(155,0.838014832032923)(156,0.83868084437339)(157,0.839202104047557)(158,0.838451734035649)(159,0.838954173156235)(160,0.839108068226942)(161,0.839319076102629)(162,0.840090978880918)(163,0.840099246791148)(164,0.840712187801424)(165,0.840181337161989)(166,0.840730095720347)(167,0.840703782448069)(168,0.840655677989356)(169,0.840600990384795)(170,0.840493236798727)(171,0.840631857900639)(172,0.841206865124004)(173,0.840972311662485)(174,0.840946062390863)(175,0.841005758170465)(176,0.843306644322325)(177,0.844648762251038)(178,0.845345280613819)(179,0.845053323643291)(180,0.84663029800284)(181,0.847644058535551)(182,0.846497249263622)(183,0.845529967289168)(184,0.845493487314694)(185,0.845412996465238)(186,0.844750101852994)(187,0.845811883682118)(188,0.845397161342969)(189,0.845980808596278)(190,0.846129003031199)(191,0.846336492425972)(192,0.845913156668414)(193,0.846516957491294)(194,0.847111586505774)(195,0.847371139664417)(196,0.847582860390966)(197,0.84767174001193)(198,0.848219225587195)(199,0.849046381179625)(200,0.848137983357976)(201,0.84828741274325)(202,0.849542394981176)(203,0.848903207569113)(204,0.849386382696889)(205,0.849605428764755)(206,0.849967776286865)(207,0.849715061990358)(208,0.848671056363338)(209,0.848432836780735)(210,0.849606208255471)(211,0.849456199092059)(212,0.848776798396715)(213,0.848891129002995)(214,0.848995632669929)(215,0.849564056790582)(216,0.849500186569316)(217,0.849308973801306)(218,0.849402098498904)(219,0.849898045228539)(220,0.849902425286075)(221,0.850547264253541)(222,0.851501628002389)(223,0.851172629671942)(224,0.851129109440701)(225,0.850775126416633)(226,0.850951252197102)(227,0.851754212418794)(228,0.851605740284866)(229,0.851649434272881)(230,0.851558630765602)(231,0.851598218131407)(232,0.851962790320182)(233,0.852366807760718)(234,0.852089445996467)(235,0.852537124604484)(236,0.852904441833355)(237,0.853083294628219)(238,0.854263033887182)(239,0.854339323788847)(240,0.853795702150466)(241,0.854800805067732)(242,0.854528907077196)(243,0.854778644002598)(244,0.855049676571196)(245,0.855213623448217)(246,0.855738489571889)(247,0.855586778362248)(248,0.855770937393432)(249,0.855898458098024)(250,0.857441820649875)(251,0.858472379039627)(252,0.858162745423022)(253,0.857536005858049)(254,0.858279130961187)(255,0.858867751510376)(256,0.859196368233877)(257,0.859344512412849)(258,0.859568941641158)(259,0.859666662943741)(260,0.859813106088838)(261,0.860423511528802)(262,0.860431881002055)(263,0.861419789700762)(264,0.862312462030995)(265,0.862755566926718)(266,0.862196224266963)(267,0.862338644555277)(268,0.861567885431912)(269,0.861475335579324)(270,0.86158137756972)(271,0.861509136299235)(272,0.861832144600986)(273,0.861565757651831)(274,0.861995172229548)(275,0.863068393666117)(276,0.863024543820004)(277,0.863301038531269)(278,0.863766563519668)(279,0.863316134662935)(280,0.863697715224162)(281,0.863749908257971)(282,0.864252215703064)(283,0.864472976187412)(284,0.864775195551738)(285,0.86519386042378)(286,0.865636012543916)(287,0.86646747103479)(288,0.866441901534585)(289,0.866741760198148)(290,0.867201028040239)(291,0.867101471260481)(292,0.867152495025552)(293,0.866816220568863)(294,0.865893591169799)(295,0.866288746717615)(296,0.866409069905442)(297,0.866825652236795)(298,0.866953722851941)(299,0.866250306484104)(300,0.865843385147491)(301,0.865644512216642)(302,0.866124480178253)(303,0.866074992732495)(304,0.866493043589884)(305,0.86791670686168)(306,0.868356804137647)(307,0.86878195867829)(308,0.868197984242929)(309,0.868635995235281)(310,0.868712638634917)(311,0.868990840217332)(312,0.869360433036578)(313,0.86955594989676)(314,0.869785915907376)(315,0.869803058735923)(316,0.869208203401292)(317,0.869210823358746)(318,0.869376099994361)(319,0.870063211758464)(320,0.870053580073986)(321,0.868941797785078)(322,0.868753099831308)(323,0.868832543607943)(324,0.868582402414483)(325,0.86815798686017)(326,0.867355608467076)(327,0.867587551531779)(328,0.867021454791122)(329,0.86698689217776)(330,0.866457548575319)(331,0.866223677256025)(332,0.867056935115353)(333,0.866365812878413)(334,0.86651851312302)(335,0.865842917020892)(336,0.866615606202456)(337,0.866637105835841)(338,0.866887960938208)(339,0.867045720751775)(340,0.867132926845874)(341,0.867081336082402)(342,0.867481230784531)(343,0.867437241612164)(344,0.868133552382396)(345,0.868333630237264)(346,0.868055907819735)(347,0.868239570033844)(348,0.867779475345867)(349,0.868121244344315)(350,0.86848529728583)(351,0.869148230961578)(352,0.869178345302263)(353,0.869346519408282)(354,0.869391424842273)(355,0.869302086058643)(356,0.869371719792832)(357,0.869769054789172)(358,0.869817836022075)(359,0.870048478492933)(360,0.870119694436893)(361,0.870127661319849)(362,0.870144070769588)(363,0.870277403203908)(364,0.869777802599742)(365,0.869871745818336)(366,0.869969010383461)(367,0.868890906644128)(368,0.869472921166453)(369,0.869100786557179)(370,0.869740240159517)(371,0.869270885076448)(372,0.869829561844128)(373,0.870099946142159)(374,0.870242818538054)(375,0.870027300007806)(376,0.870201631400002)(377,0.870270127986535)(378,0.870097503907793)(379,0.870444875150442)(380,0.869870477244065)(381,0.870063483709145)(382,0.86977432795887)(383,0.869737595487331)(384,0.869849178481679)(385,0.869465405481991)(386,0.868993508934025)(387,0.868739438273858)(388,0.868684276807169)(389,0.869158900732787)(390,0.869701724901112)(391,0.86964835211747)(392,0.869631366340443)(393,0.870266447832913)(394,0.87045464022884)(395,0.870058449071775)(396,0.869882452435122)(397,0.869270722426079)(398,0.87009107640224)(399,0.870805489355955)(400,0.870830393752248)
 %(401,0.870767608134393)(402,0.870569604934946)(403,0.871128126482302)(404,0.871553805436769)(405,0.871184313937322)(406,0.870719103996045)(407,0.870637326883559)(408,0.87121864299276)(409,0.871733669291836)(410,0.871675117487857)(411,0.872140383057057)(412,0.871412490193481)(413,0.872083068461845)(414,0.87158452265157)(415,0.870950190424399)(416,0.870988566093079)(417,0.871039467446057)(418,0.871655609074423)(419,0.871881602444002)(420,0.871555951367955) 
};
\addlegendentry{\istr};

\addplot [
color=blue,
solid,
line width=1.3pt,
]
coordinates{
 %(1,0.220938455085396)(1,0.220938455085396)(1,0.220938455085396)(1,0.220938455085396)(1,0.220938455085396)(1,0.220938455085396)(1,0.220938455085396)(1,0.220938455085396)(1,0.220938455085396)(1,0.220938455085396)(1,0.220938455085396)(1,0.220938455085396)(3,0.231323276824903)(3,0.231323276824903)(4,0.241042827260135)(4,0.241042827260135)(4,0.241042827260135)(4,0.241042827260135)(4,0.241042827260135)(4,0.241042827260135)(4,0.241042827260135)(4,0.241042827260135)(4,0.241042827260135)(4,0.241042827260135)(4,0.241042827260135)(4,0.241042827260135)(4,0.241042827260135)(4,0.241042827260135)(4,0.241042827260135)(4,0.241042827260135)(4,0.241042827260135)(4,0.241042827260135)(4,0.241042827260135)(4,0.241042827260135)(5,0.253567782444175)(5,0.253567782444175)(5,0.253567782444175)(5,0.253567782444175)(5,0.253567782444175)(5,0.253567782444175)(5,0.253567782444175)(5,0.253567782444175)(5,0.253567782444175)(5,0.253567782444175)(5,0.253567782444175)(6,0.268818159879306)(7,0.286790674235947)(7,0.286790674235947)(7,0.286790674235947)(8,0.307570522312801)(8,0.307570522312801)(9,0.331397655920556)(9,0.331397655920556)(10,0.358709621680403)(12,0.425737638690259)(13,0.463811067921611)(14,0.495409633762597)
 (15,0.523304234694612)(15,0.523304234694612)(15,0.523304234694612)(15,0.523304234694612)(16,0.547863201323318)(16,0.547863201323318)(16,0.547863201323318)(17,0.570029897439578)(17,0.570029897439578)(17,0.570029897439578)(18,0.590037904089374)(18,0.590037904089374)(18,0.590037904089374)(18,0.590037904089374)(18,0.590037904089374)(18,0.590037904089374)(18,0.590037904089374)(18,0.590037904089374)(18,0.590037904089374)(18,0.590037904089374)(18,0.590037904089374)(18,0.590037904089374)(19,0.609791016038758)(19,0.609791016038758)(19,0.609791016038758)(19,0.609791016038758)(19,0.609791016038758)(19,0.609791016038758)(19,0.609791016038758)(19,0.609791016038758)(20,0.625888037364671)(20,0.625888037364671)(20,0.625888037364671)(20,0.625888037364671)(20,0.625888037364671)(20,0.625888037364671)(20,0.625888037364671)(20,0.625888037364671)(20,0.625888037364671)(20,0.625888037364671)(20,0.625888037364671)(20,0.625888037364671)(21,0.640023560947027)(21,0.640023560947027)(21,0.640023560947027)(21,0.640023560947027)(21,0.640023560947027)(21,0.640023560947027)(21,0.640023560947027)(21,0.640023560947027)(21,0.640023560947027)(22,0.651148379898084)(22,0.651148379898084)(22,0.651148379898084)(22,0.651148379898084)(22,0.651148379898084)(22,0.651148379898084)(22,0.651148379898084)(22,0.651148379898084)(22,0.651148379898084)(22,0.651148379898084)(22,0.651148379898084)(23,0.660589322091515)(23,0.660589322091515)(23,0.660589322091515)(23,0.660589322091515)(23,0.660589322091515)(23,0.660589322091515)(23,0.660589322091515)(23,0.660589322091515)(23,0.660589322091515)(23,0.660589322091515)(24,0.66866986633411)(24,0.66866986633411)(24,0.66866986633411)(24,0.66866986633411)(24,0.66866986633411)(24,0.66866986633411)(24,0.66866986633411)(24,0.66866986633411)(25,0.677571278685022)(25,0.677571278685022)(25,0.677571278685022)(25,0.677571278685022)(25,0.677571278685022)(25,0.677571278685022)(25,0.677571278685022)(25,0.677571278685022)(25,0.677571278685022)(26,0.689285755668829)(26,0.689285755668829)(26,0.689285755668829)(26,0.689285755668829)(26,0.689285755668829)(26,0.689285755668829)(26,0.689285755668829)(26,0.689285755668829)(26,0.689285755668829)(26,0.689285755668829)(26,0.689285755668829)(26,0.689285755668829)(26,0.689285755668829)(26,0.689285755668829)(26,0.689285755668829)(26,0.689285755668829)(26,0.689285755668829)(27,0.699252662871777)(27,0.699252662871777)(27,0.699252662871777)(27,0.699252662871777)(27,0.699252662871777)(27,0.699252662871777)(27,0.699252662871777)(27,0.699252662871777)(27,0.699252662871777)(27,0.699252662871777)(27,0.699252662871777)(27,0.699252662871777)(27,0.699252662871777)(27,0.699252662871777)(28,0.708298547658073)(28,0.708298547658073)(28,0.708298547658073)(28,0.708298547658073)(28,0.708298547658073)(28,0.708298547658073)(28,0.708298547658073)(28,0.708298547658073)(28,0.708298547658073)(28,0.708298547658073)(28,0.708298547658073)(28,0.708298547658073)(28,0.708298547658073)(28,0.708298547658073)(28,0.708298547658073)(29,0.713428303444608)(29,0.713428303444608)(29,0.713428303444608)(29,0.713428303444608)(29,0.713428303444608)(29,0.713428303444608)(29,0.713428303444608)(29,0.713428303444608)(29,0.713428303444608)(29,0.713428303444608)(29,0.713428303444608)(29,0.713428303444608)(29,0.713428303444608)(29,0.713428303444608)(29,0.713428303444608)(29,0.713428303444608)(29,0.713428303444608)(29,0.713428303444608)(29,0.713428303444608)(30,0.716607861127105)(30,0.716607861127105)(30,0.716607861127105)(30,0.716607861127105)(30,0.716607861127105)(30,0.716607861127105)(30,0.716607861127105)(30,0.716607861127105)(30,0.716607861127105)(30,0.716607861127105)(30,0.716607861127105)(30,0.716607861127105)(30,0.716607861127105)(30,0.716607861127105)(30,0.716607861127105)(31,0.719876745433144)(31,0.719876745433144)(31,0.719876745433144)(31,0.719876745433144)(31,0.719876745433144)(31,0.719876745433144)(31,0.719876745433144)(32,0.727917549979346)(32,0.727917549979346)(32,0.727917549979346)(32,0.727917549979346)(32,0.727917549979346)(32,0.727917549979346)(32,0.727917549979346)(32,0.727917549979346)(32,0.727917549979346)(32,0.727917549979346)(32,0.727917549979346)(32,0.727917549979346)(32,0.727917549979346)(32,0.727917549979346)(32,0.727917549979346)(32,0.727917549979346)(32,0.727917549979346)(32,0.727917549979346)(32,0.727917549979346)(32,0.727917549979346)(32,0.727917549979346)(32,0.727917549979346)(33,0.737170208977593)(33,0.737170208977593)(33,0.737170208977593)(33,0.737170208977593)(33,0.737170208977593)(33,0.737170208977593)(33,0.737170208977593)(33,0.737170208977593)(33,0.737170208977593)(33,0.737170208977593)(33,0.737170208977593)(33,0.737170208977593)(33,0.737170208977593)(33,0.737170208977593)(33,0.737170208977593)(33,0.737170208977593)(33,0.737170208977593)(33,0.737170208977593)(33,0.737170208977593)(33,0.737170208977593)(34,0.74619873576212)(34,0.74619873576212)(34,0.74619873576212)(34,0.74619873576212)(34,0.74619873576212)(34,0.74619873576212)(34,0.74619873576212)(34,0.74619873576212)(34,0.74619873576212)(34,0.74619873576212)(34,0.74619873576212)(34,0.74619873576212)(34,0.74619873576212)(34,0.74619873576212)(34,0.74619873576212)(34,0.74619873576212)(34,0.74619873576212)(35,0.7553133514884)(35,0.7553133514884)(35,0.7553133514884)(35,0.7553133514884)(35,0.7553133514884)(35,0.7553133514884)(35,0.7553133514884)(35,0.7553133514884)(35,0.7553133514884)(35,0.7553133514884)(35,0.7553133514884)(35,0.7553133514884)(35,0.7553133514884)(35,0.7553133514884)(35,0.7553133514884)(35,0.7553133514884)(35,0.7553133514884)(35,0.7553133514884)(35,0.7553133514884)(35,0.7553133514884)(35,0.7553133514884)(35,0.7553133514884)(36,0.761697286534032)(36,0.761697286534032)(36,0.761697286534032)(36,0.761697286534032)(36,0.761697286534032)(36,0.761697286534032)(36,0.761697286534032)(36,0.761697286534032)(36,0.761697286534032)(36,0.761697286534032)(37,0.765890938498729)(37,0.765890938498729)(37,0.765890938498729)(37,0.765890938498729)(37,0.765890938498729)(37,0.765890938498729)(37,0.765890938498729)(37,0.765890938498729)(37,0.765890938498729)(37,0.765890938498729)(37,0.765890938498729)(38,0.769772874549799)(38,0.769772874549799)(38,0.769772874549799)(38,0.769772874549799)(38,0.769772874549799)(38,0.769772874549799)(38,0.769772874549799)(38,0.769772874549799)(38,0.769772874549799)(38,0.769772874549799)(38,0.769772874549799)(38,0.769772874549799)(38,0.769772874549799)(38,0.769772874549799)(38,0.769772874549799)(38,0.769772874549799)(38,0.769772874549799)(38,0.769772874549799)(39,0.775939483912812)(39,0.775939483912812)(39,0.775939483912812)(39,0.775939483912812)(39,0.775939483912812)(39,0.775939483912812)(39,0.775939483912812)(39,0.775939483912812)(39,0.775939483912812)(39,0.775939483912812)(39,0.775939483912812)(40,0.781930965165482)(40,0.781930965165482)(40,0.781930965165482)(40,0.781930965165482)(40,0.781930965165482)(40,0.781930965165482)(40,0.781930965165482)(40,0.781930965165482)(40,0.781930965165482)(40,0.781930965165482)(40,0.781930965165482)(41,0.78796319471534)(41,0.78796319471534)(41,0.78796319471534)(41,0.78796319471534)(41,0.78796319471534)(41,0.78796319471534)(41,0.78796319471534)(41,0.78796319471534)(42,0.792801916134067)(42,0.792801916134067)(42,0.792801916134067)(42,0.792801916134067)(42,0.792801916134067)(42,0.792801916134067)(42,0.792801916134067)(42,0.792801916134067)(42,0.792801916134067)(42,0.792801916134067)(42,0.792801916134067)(43,0.796420740026223)(43,0.796420740026223)(43,0.796420740026223)(43,0.796420740026223)(43,0.796420740026223)(43,0.796420740026223)(43,0.796420740026223)(43,0.796420740026223)(43,0.796420740026223)(43,0.796420740026223)(43,0.796420740026223)(43,0.796420740026223)(43,0.796420740026223)(43,0.796420740026223)(44,0.799439521176078)(44,0.799439521176078)(44,0.799439521176078)(44,0.799439521176078)(44,0.799439521176078)(44,0.799439521176078)(44,0.799439521176078)(44,0.799439521176078)(44,0.799439521176078)(45,0.802019323790032)(45,0.802019323790032)(45,0.802019323790032)(45,0.802019323790032)(45,0.802019323790032)(45,0.802019323790032)(45,0.802019323790032)(45,0.802019323790032)(45,0.802019323790032)(45,0.802019323790032)(45,0.802019323790032)(45,0.802019323790032)(46,0.804203124651504)(46,0.804203124651504)(46,0.804203124651504)(46,0.804203124651504)(46,0.804203124651504)(46,0.804203124651504)(46,0.804203124651504)(46,0.804203124651504)(46,0.804203124651504)(46,0.804203124651504)(46,0.804203124651504)(46,0.804203124651504)(47,0.806188288653638)(47,0.806188288653638)(47,0.806188288653638)(47,0.806188288653638)(47,0.806188288653638)(47,0.806188288653638)(47,0.806188288653638)(47,0.806188288653638)(47,0.806188288653638)(48,0.80864791157466)(48,0.80864791157466)(48,0.80864791157466)(48,0.80864791157466)(48,0.80864791157466)(48,0.80864791157466)(48,0.80864791157466)(48,0.80864791157466)(48,0.80864791157466)(48,0.80864791157466)(48,0.80864791157466)(49,0.811158449221028)(49,0.811158449221028)(49,0.811158449221028)(49,0.811158449221028)(49,0.811158449221028)(49,0.811158449221028)(49,0.811158449221028)(49,0.811158449221028)(50,0.814378182626064)(50,0.814378182626064)(50,0.814378182626064)(50,0.814378182626064)(50,0.814378182626064)(51,0.817241934411336)(51,0.817241934411336)(51,0.817241934411336)(51,0.817241934411336)(51,0.817241934411336)(51,0.817241934411336)(51,0.817241934411336)(51,0.817241934411336)(52,0.820168062117717)(52,0.820168062117717)(52,0.820168062117717)(52,0.820168062117717)(53,0.823058092251003)(53,0.823058092251003)(53,0.823058092251003)(53,0.823058092251003)(54,0.82587057369876)(54,0.82587057369876)(54,0.82587057369876)(54,0.82587057369876)(54,0.82587057369876)(54,0.82587057369876)(54,0.82587057369876)(55,0.828131712676084)(55,0.828131712676084)(55,0.828131712676084)(55,0.828131712676084)(56,0.830323390944062)(56,0.830323390944062)(56,0.830323390944062)(56,0.830323390944062)(56,0.830323390944062)(56,0.830323390944062)(56,0.830323390944062)(57,0.832538015632764)(58,0.834257056896573)(58,0.834257056896573)(59,0.835947010194516)(59,0.835947010194516)(59,0.835947010194516)(59,0.835947010194516)(59,0.835947010194516)(59,0.835947010194516)(59,0.835947010194516)(60,0.837248820932508)(60,0.837248820932508)(60,0.837248820932508)(60,0.837248820932508)(60,0.837248820932508)(60,0.837248820932508)(60,0.837248820932508)(60,0.837248820932508)(61,0.838563601895987)(61,0.838563601895987)(61,0.838563601895987)(62,0.83983334702344)(62,0.83983334702344)(62,0.83983334702344)(62,0.83983334702344)(62,0.83983334702344)(63,0.840857377877403)(63,0.840857377877403)(63,0.840857377877403)(63,0.840857377877403)(64,0.84159504432822)(64,0.84159504432822)(64,0.84159504432822)(64,0.84159504432822)(64,0.84159504432822)(64,0.84159504432822)(65,0.842275595932688)(66,0.84287328493556)(66,0.84287328493556)(66,0.84287328493556)(67,0.843799928950204)(67,0.843799928950204)(67,0.843799928950204)(67,0.843799928950204)(67,0.843799928950204)(68,0.845003947967706)(68,0.845003947967706)(68,0.845003947967706)(69,0.845955157328271)(70,0.847541252717718)(70,0.847541252717718)(71,0.849120688125106)(71,0.849120688125106)(71,0.849120688125106)(71,0.849120688125106)(71,0.849120688125106)(72,0.850559673935257)(72,0.850559673935257)(72,0.850559673935257)(73,0.85171136201106)(73,0.85171136201106)(73,0.85171136201106)(74,0.852955979932608)(74,0.852955979932608)(75,0.853997718963786)(76,0.854983748793227)(76,0.854983748793227)(76,0.854983748793227)(76,0.854983748793227)(76,0.854983748793227)(76,0.854983748793227)(77,0.855880150221621)(78,0.856830458228946)(78,0.856830458228946)(79,0.857862835045513)(79,0.857862835045513)(79,0.857862835045513)(80,0.858909681627862)(80,0.858909681627862)(80,0.858909681627862)(80,0.858909681627862)(80,0.858909681627862)(80,0.858909681627862)(80,0.858909681627862)(81,0.860015128128857)(81,0.860015128128857)(81,0.860015128128857)(81,0.860015128128857)(82,0.861143463722743)(82,0.861143463722743)(83,0.862357161053115)(83,0.862357161053115)(84,0.863603323549631)(84,0.863603323549631)(84,0.863603323549631)(84,0.863603323549631)(85,0.864818673187409)(86,0.865983942328819)(86,0.865983942328819)(86,0.865983942328819)(86,0.865983942328819)(87,0.867103086428041)(88,0.868058697726766)(88,0.868058697726766)(89,0.86890106557005)(89,0.86890106557005)(89,0.86890106557005)(90,0.869677826616982)(91,0.870531067213418)(91,0.870531067213418)(91,0.870531067213418)(91,0.870531067213418)(92,0.871242730446496)(93,0.871946908384776)(93,0.871946908384776)(94,0.872712640092262)(96,0.874131122641525)(96,0.874131122641525)(98,0.875492418743013)(98,0.875492418743013)(99,0.876141649954396)(99,0.876141649954396)(99,0.876141649954396)(99,0.876141649954396)(99,0.876141649954396)(101,0.877396608162428)(101,0.877396608162428)(101,0.877396608162428)(102,0.878018217891635)(102,0.878018217891635)(102,0.878018217891635)(103,0.878604109889722)(103,0.878604109889722)(105,0.879779040377465)(105,0.879779040377465)(105,0.879779040377465)(106,0.880378604186497)(107,0.880935891924375)(108,0.881554318657585)(108,0.881554318657585)(110,0.882779859829523)(110,0.882779859829523)(111,0.883369738560668)(111,0.883369738560668)(116,0.886682444455636)(116,0.886682444455636)(117,0.887286211607411)(118,0.887929163009689)(118,0.887929163009689)(118,0.887929163009689)(119,0.888535617602415)(119,0.888535617602415)(120,0.889108852205416)(120,0.889108852205416)(120,0.889108852205416)(121,0.889648746159544)(122,0.890185129085898)(122,0.890185129085898)(127,0.892695537181966)(127,0.892695537181966)(127,0.892695537181966)(127,0.892695537181966)(129,0.893518953506654)(131,0.894310562753156)(132,0.894683040304722)(132,0.894683040304722)(132,0.894683040304722)(133,0.895093212279909)(133,0.895093212279909)(134,0.895480363115795)(134,0.895480363115795)(134,0.895480363115795)(134,0.895480363115795)(135,0.89588625955135)(135,0.89588625955135)(135,0.89588625955135)(136,0.896309993866957)(137,0.896736567510451)(138,0.897164804016648)(140,0.898079341597708)(141,0.898514913754424)(141,0.898514913754424)(142,0.898945827191928)(143,0.899402175675402)(144,0.899847270162602)(145,0.900259612107747)(146,0.900683187001542)(146,0.900683187001542)(147,0.90109260831503)(148,0.901487836939616)(148,0.901487836939616)(148,0.901487836939616)(149,0.901869334025776)(149,0.901869334025776)(149,0.901869334025776)(150,0.902236777063175)(152,0.902945168678068)(152,0.902945168678068)(152,0.902945168678068)(152,0.902945168678068)(153,0.90328683608458)(154,0.903610796072314)(154,0.903610796072314)(155,0.903919938903654)(156,0.904231920490538)(157,0.90451920033125)(157,0.90451920033125)(157,0.90451920033125)(157,0.90451920033125)(158,0.904798606497403)(159,0.905073360889353)(163,0.906165485654471)(164,0.906426896599335)(165,0.906688528875504)(165,0.906688528875504)(166,0.90695227044934)(167,0.907213159121576)(170,0.907986532163287)(170,0.907986532163287)(171,0.908244865022407)(172,0.908493202245174)(173,0.908738081033581)(175,0.909235155761773)(178,0.909960018851784)(179,0.910220461662794)(179,0.910220461662794)(180,0.910450899622425)(181,0.91071945216619)(182,0.910953218769165)(183,0.911187778674887)(184,0.911373064618509)(186,0.911850975195081)(189,0.912735998070836)(189,0.912735998070836)(192,0.913628024455555)(192,0.913628024455555)(193,0.913976313048435)(194,0.914326314130956)(197,0.915312185697541)(197,0.915312185697541)(199,0.915984306886101)(199,0.915984306886101)(201,0.916537211329734)(202,0.916866605165222)(203,0.917165231148646)(204,0.917466010458705)(204,0.917466010458705)(205,0.917790825398063)(206,0.918092178266955)(207,0.918377539950053)(210,0.919311200351219)(214,0.920478066767764)(217,0.921333999466729)(219,0.921886314953248)(222,0.922688171679882)(222,0.922688171679882)(222,0.922688171679882)(223,0.92295431155814)(223,0.92295431155814)(224,0.923218055874184)(225,0.923492528215362)(226,0.923767741031281)(228,0.924263360258458)(230,0.924719053952084)(230,0.924719053952084)(236,0.92598246994033)(236,0.92598246994033)(237,0.92622010179304)(237,0.92622010179304)(237,0.92622010179304)(239,0.926597670551612)(246,0.927695479567424)(247,0.927809349068473)(248,0.927915232302799)(249,0.928056102462041)(252,0.928398939493177)(253,0.928470491270794)(257,0.928895884106329)(257,0.928895884106329)(258,0.928947042897936)(259,0.928998015287202)(260,0.929090057008664)(261,0.929141759646305)(261,0.929141759646305)(263,0.929286653835183)(263,0.929286653835183)(266,0.929524367750849)(267,0.929590311050968)(268,0.929680175255784)(271,0.929911124073943)(272,0.9299967493436)(272,0.9299967493436)(272,0.9299967493436)(274,0.930182343617982)(274,0.930182343617982)(277,0.930473400829693)(278,0.930575844161153)(279,0.930680727316938)(280,0.930787438823234)(280,0.930787438823234)(280,0.930787438823234)(283,0.931104726246044)(284,0.931203924705687)(290,0.931855262259801)(291,0.931964991630378)(292,0.932074233739184)(294,0.932291382624454)(296,0.932507081795058)(296,0.932507081795058)(297,0.932614522844889)(297,0.932614522844889)(299,0.932828847181936)(299,0.932828847181936)(301,0.933042704103362)(304,0.933363197811702)(304,0.933363197811702)(304,0.933363197811702)(305,0.933470046284895)(307,0.933683864238729)(307,0.933683864238729)(309,0.933897914030922)(313,0.934326993609083)(314,0.934434515904725)(314,0.934434515904725)(316,0.934649933013727)(318,0.934865934395507)(319,0.934974185431162)(319,0.934974185431162)(320,0.935082620050945)(322,0.935300094558085)(323,0.935409158607033)(325,0.935627990203764)(325,0.935627990203764)(326,0.935737775869307)(326,0.935737775869307)(327,0.935847821924555)(330,0.936179633098762)(340,0.937306685104144)(343,0.937651861711398)(343,0.937651861711398)(343,0.937651861711398)(348,0.938234969046891)(349,0.938352777876526)(352,0.938708606418845)(352,0.938708606418845)(352,0.938708606418845)(356,0.939188733766141)(356,0.939188733766141)(357,0.939309795895579)(357,0.939309795895579)(359,0.939553165244424)(360,0.939675472050157)(362,0.939921328248393)(362,0.939921328248393)(366,0.940417991718077)(368,0.940668780235067)(368,0.940668780235067)(369,0.940794785819324)(369,0.940794785819324)(377,0.941817299666279)(379,0.942076881945994)(384,0.942732593963712)(385,0.942864877440697)(388,0.943263986912833)(389,0.943397775704156)(397,0.944481584829304)(400,0.944894180810377)
 %(402,0.945171130922129)(404,0.945449616367898)(406,0.945729662235803)(408,0.946011286213955)(412,0.946579317525129)(417,0.947298526319659)(420,0.947735122042499) 
};
\addlegendentry{\iacl};

\addplot [
color=green!50!black,
solid,
line width=1.3pt,
]
coordinates{
 (30,0.651577102920854)(60,0.780385479638578)(90,0.84346197917087)(120,0.873962102022871)(150,0.891449244338554)(180,0.908090774483905)(210,0.921811959124149)(240,0.918294963399954)(270,0.929751620480508)(300,0.927711294191924)(330,0.928942362430249)(360,0.942385153643279)(390,0.938825312056569)
 %(420,0.942190766898956) 
};
\addlegendentry{\ibacl};

\end{axis}
\end{tikzpicture}%

%% This file was created by matlab2tikz v0.2.3.
% Copyright (c) 2008--2012, Nico Schlömer <nico.schloemer@gmail.com>
% All rights reserved.
% 
% 
% 
\begin{tikzpicture}

\begin{axis}[%
tick label style={font=\tiny},
label style={font=\tiny},
label shift={-4pt},
xlabel shift={-6pt},
legend style={font=\tiny},
view={0}{90},
width=\figurewidth,
height=\figureheight,
scale only axis,
xmin=0, xmax=400,
xlabel={Samples},
ymin=0.5, ymax=1,
ylabel={$F_1$-score},
axis lines*=left,
legend cell align=left,
legend style={at={(1.03,0)},anchor=south east,fill=none,draw=none,align=left,row sep=-0.2em},
clip=false]

\addplot [
color=orange,
mark size=0.1pt,
only marks,
mark=*,
mark options={solid,fill=black},
forget plot
]
coordinates{
 (7,0.626907073509015)(8,0.69079754601227)(9,0.683612040133779)(10,0.701578192252511)(11,0.704845814977973)(12,0.717638152914459)(13,0.719391634980988)(14,0.72234762979684)(15,0.722097378277153)(16,0.732745961820852)(17,0.741433021806853)(18,0.73953488372093)(19,0.746153846153846)(20,0.74884437596302)(21,0.785947712418301)(22,0.7893864013267)(23,0.7893864013267)(24,0.794019933554817)(25,0.79734219269103)(26,0.796019900497512)(27,0.794701986754967)(28,0.810457516339869)(29,0.811594202898551)(30,0.810679611650485)(31,0.812955465587044)(32,0.811594202898551)(33,0.811594202898551)(34,0.811735941320293)(35,0.809795918367347)(36,0.810766721044045)(37,0.809795918367347)(38,0.818401937046005)(39,0.819063004846527)(40,0.825061025223759)(41,0.824104234527687)(42,0.861610633307271)(43,0.860483242400623)(44,0.862313139260425)(45,0.862313139260425)(46,0.878576952822893)(47,0.879629629629629)(48,0.88135593220339)(49,0.879629629629629)(50,0.878388845855926)(51,0.877708978328173)(52,0.872274143302181)(53,0.872755659640905)(54,0.865035516969219)(55,0.865035516969219)(56,0.875389408099688)(57,0.874120406567631)(58,0.873040752351097)(59,0.873923257635082)(60,0.876755070202808)(61,0.878315132605304)(62,0.878315132605304)(63,0.878315132605304)(64,0.878125)(65,0.884585592563904)(66,0.883720930232558)(67,0.879000780640125)(68,0.870866141732283)(69,0.874411302982732)(70,0.875294117647059)(71,0.876175548589342)(72,0.887509697439876)(73,0.883177570093458)(74,0.878125)(75,0.878125)(76,0.878125)(77,0.879000780640125)(78,0.878125)(79,0.877247849882721)(80,0.878125)(81,0.881619937694704)(82,0.886821705426357)(83,0.885093167701863)(84,0.888544891640867)(85,0.88074824629774)(86,0.8798751950078)(87,0.878125)(88,0.881619937694704)(89,0.878125)(90,0.879000780640125)(91,0.879000780640125)(92,0.876369327073552)(93,0.878125)(94,0.879000780640125)(95,0.882490272373541)(96,0.88074824629774)(97,0.879000780640125)(98,0.8798751950078)(99,0.878125)(100,0.87548942834769)(101,0.87548942834769)(102,0.876369327073552)(103,0.879189399844115)(104,0.8765625)(105,0.8765625)(106,0.8765625)(107,0.879000780640125)(108,0.879000780640125)(109,0.877247849882721)(110,0.876369327073552)(111,0.87548942834769)(112,0.877247849882721)(113,0.876369327073552)(114,0.877247849882721)(115,0.874608150470219)(116,0.874608150470219)(117,0.874608150470219)(118,0.877247849882721)(119,0.879000780640125)(120,0.88074824629774)(121,0.882490272373541)(122,0.881619937694704)(123,0.881619937694704)(124,0.881804043545879)(125,0.882672882672883)(126,0.882672882672883)(127,0.882672882672883)(128,0.882854926299457)(129,0.882854926299457)(130,0.882854926299457)(131,0.884585592563904)(132,0.884406516679596)(133,0.880933852140078)(134,0.8798751950078)(135,0.874608150470219)(136,0.87548942834769)(137,0.877247849882721)(138,0.877247849882721)(139,0.875684128225176)(140,0.8765625)(141,0.878504672897196)(142,0.879189399844115)(143,0.879189399844115)(144,0.87743950039032)(145,0.878315132605304)(146,0.87743950039032)(147,0.873040752351097)(148,0.8712715855573)(149,0.872156862745098)(150,0.8712715855573)(151,0.870384917517675)(152,0.870384917517675)(153,0.873040752351097)(154,0.866614048934491)(155,0.865718799368088)(156,0.864822134387352)(157,0.864822134387352)(158,0.871069182389937)(159,0.871956009426551)(160,0.871956009426551)(161,0.871956009426551)(162,0.871069182389937)(163,0.869291338582677)(164,0.867507886435331)(165,0.867507886435331)(166,0.865718799368088)(167,0.864822134387352)(168,0.865718799368088)(169,0.863924050632911)(170,0.863924050632911)(171,0.859412231930103)(172,0.858505564387917)(173,0.858505564387917)(174,0.856687898089172)(175,0.856687898089172)(176,0.857597454256165)(177,0.849117174959871)(178,0.85828025477707)(179,0.85828025477707)(180,0.879000780640125)(181,0.886821705426357)(182,0.886996904024768)(183,0.885093167701863)(184,0.887857695282289)(185,0.88597842835131)(186,0.885448916408669)(187,0.887683965917893)(188,0.886821705426357)(189,0.885958107059736)(190,0.883359253499222)(191,0.878811571540266)(192,0.877934272300469)(193,0.8796875)(194,0.877934272300469)(195,0.878811571540266)(196,0.877934272300469)(197,0.877055599060298)(198,0.877055599060298)(199,0.877934272300469)(200,0.877055599060298)(201,0.876175548589342)(202,0.875294117647059)(203,0.875294117647059)(204,0.877055599060298)(205,0.877055599060298)(206,0.876175548589342)(207,0.876175548589342)(208,0.876175548589342)(209,0.876175548589342)(210,0.876175548589342)(211,0.877055599060298)(212,0.877055599060298)(213,0.877055599060298)(214,0.877055599060298)(215,0.88056206088993)(216,0.8796875)(217,0.8796875)(218,0.878811571540266)(219,0.878811571540266)(220,0.881435257410296)(221,0.881435257410296)(222,0.883177570093458)(223,0.883177570093458)(224,0.883177570093458)(225,0.883177570093458)(226,0.882307092751364)(227,0.882307092751364)(228,0.882307092751364)(229,0.883177570093458)(230,0.882307092751364)(231,0.882307092751364)(232,0.890951276102088)(233,0.891808346213292)(234,0.892664092664093)(235,0.892664092664093)(236,0.894371626831149)(237,0.894371626831149)(238,0.898617511520737)(239,0.894371626831149)(240,0.894371626831149)(241,0.891808346213292)(242,0.891808346213292)(243,0.889233152594888)(244,0.887509697439876)(245,0.887509697439876)(246,0.887509697439876)(247,0.889233152594888)(248,0.888372093023256)(249,0.888372093023256)(250,0.885780885780886)(251,0.886645962732919)(252,0.887509697439876)(253,0.884046692607004)(254,0.883177570093458)(255,0.883177570093458)(256,0.883177570093458)(257,0.886645962732919)(258,0.886645962732919)(259,0.886645962732919)(260,0.888372093023256)(261,0.898773006134969)(262,0.897770945426595)(263,0.899462778204144)(264,0.899462778204144)(265,0.899462778204144)(266,0.896923076923077)(267,0.896923076923077)(268,0.897770945426595)(269,0.898617511520737)(270,0.898617511520737)(271,0.898617511520737)(272,0.897770945426595)(273,0.898617511520737)(274,0.898617511520737)(275,0.897770945426595)(276,0.896073903002309)(277,0.894371626831149)(278,0.895223420647149)(279,0.895223420647149)(280,0.895223420647149)(281,0.896073903002309)(282,0.897770945426595)(283,0.896923076923077)(284,0.900306748466258)(285,0.900306748466258)(286,0.904507257448434)(287,0.906178489702517)(288,0.907012195121951)(289,0.907012195121951)(290,0.907012195121951)(291,0.907012195121951)(292,0.907844630616908)(293,0.907844630616908)(294,0.907844630616908)(295,0.908124525436598)(296,0.908814589665653)(297,0.90978013646702)(298,0.909643128321944)(299,0.907844630616908)(300,0.905343511450382)(301,0.905343511450382)(302,0.905343511450382)(303,0.90283091048202)(304,0.90283091048202)(305,0.906178489702517)(306,0.907012195121951)(307,0.907012195121951)(308,0.906178489702517)(309,0.905343511450382)(310,0.904507257448434)(311,0.904507257448434)(312,0.904507257448434)(313,0.901990811638591)(314,0.901990811638591)(315,0.901990811638591)(316,0.905343511450382)(317,0.903669724770642)(318,0.903669724770642)(319,0.903669724770642)(320,0.903669724770642)(321,0.903669724770642)(322,0.903669724770642)(323,0.903669724770642)(324,0.903669724770642)(325,0.90283091048202)(326,0.903669724770642)(327,0.903669724770642)(328,0.903669724770642)(329,0.903669724770642)(330,0.904507257448434)(331,0.906178489702517)(332,0.906463878326996)(333,0.908124525436598)(334,0.907435508345979)(335,0.908263836239575)(336,0.908263836239575)(337,0.909090909090909)(338,0.909090909090909)(339,0.909090909090909)(340,0.91238670694864)(341,0.91238670694864)(342,0.91238670694864)(343,0.913207547169811)(344,0.914027149321267)(345,0.91566265060241)(346,0.91566265060241)(347,0.91566265060241)(348,0.91566265060241)(349,0.91566265060241)(350,0.91647855530474)(351,0.91904047976012)(352,0.91904047976012)(353,0.919161676646707)(354,0.922388059701492)(355,0.923994038748137)(356,0.924107142857143)(357,0.923994038748137)(358,0.923191648023863)(359,0.923191648023863)(360,0.924795234549516)(361,0.925595238095238)(362,0.925595238095238)(363,0.924795234549516)(364,0.923994038748137)(365,0.923191648023863)(366,0.923191648023863)(367,0.923994038748137)(368,0.925595238095238)(369,0.922961854899027)(370,0.922961854899027)(371,0.921348314606741)(372,0.918918918918919)(373,0.918918918918919)(374,0.921348314606741)(375,0.92065868263473)(376,0.921348314606741)(377,0.920539730134932)(378,0.919729932483121)(379,0.919729932483121)(380,0.913636363636364)(381,0.913636363636364)(382,0.911987860394537)(383,0.904507257448434)(384,0.90283091048202)(385,0.90283091048202)(386,0.90283091048202)(387,0.903669724770642)(388,0.903669724770642)(389,0.90283091048202)(390,0.90283091048202)(391,0.90283091048202)(392,0.90283091048202)(393,0.899462778204144)(394,0.901990811638591)(395,0.90283091048202)(396,0.897770945426595)(397,0.894371626831149)(398,0.895223420647149)(399,0.899462778204144)(400,0.899462778204144)(10,0.507892930679478)(11,0.514742014742015)(12,0.658400495970242)(13,0.687100893997446)(14,0.696286472148541)(15,0.729957805907173)(16,0.746054519368723)(17,0.766939687267312)(18,0.770909090909091)(19,0.773158278628738)(20,0.785923753665689)(21,0.783941605839416)(22,0.775800711743772)(23,0.815592203898051)(24,0.807462686567164)(25,0.801781737193764)(26,0.802952029520295)(27,0.803545051698671)(28,0.804733727810651)(29,0.804138950480414)(30,0.802650957290132)(31,0.806835066864784)(32,0.80206033848418)(33,0.804428044280443)(34,0.805617147080562)(35,0.806213017751479)(36,0.792727272727273)(37,0.79100145137881)(38,0.795040116703136)(39,0.793304221251819)(40,0.797366495976591)(41,0.796201607012418)(42,0.796201607012418)(43,0.784737221022318)(44,0.788712011577424)(45,0.784737221022318)(46,0.787003610108303)(47,0.788141720896601)(48,0.788141720896601)(49,0.789283128167994)(50,0.795040116703136)(51,0.792151162790698)(52,0.79388201019665)(53,0.796783625730994)(54,0.797366495976591)(55,0.796783625730994)(56,0.796783625730994)(57,0.797366495976591)(58,0.796201607012418)(59,0.795040116703136)(60,0.796201607012418)(61,0.796201607012418)(62,0.80206033848418)(63,0.80206033848418)(64,0.807407407407407)(65,0.807407407407407)(66,0.80680977054034)(67,0.80680977054034)(68,0.806213017751479)(69,0.806213017751479)(70,0.808005930318755)(71,0.802650957290132)(72,0.805617147080562)(73,0.800293685756241)(74,0.805617147080562)(75,0.808005930318755)(76,0.805617147080562)(77,0.80680977054034)(78,0.80680977054034)(79,0.808005930318755)(80,0.807407407407407)(81,0.80920564216778)(82,0.805617147080562)(83,0.805617147080562)(84,0.805617147080562)(85,0.804428044280443)(86,0.807407407407407)(87,0.807407407407407)(88,0.807407407407407)(89,0.809806835066865)(90,0.808005930318755)(91,0.809806835066865)(92,0.811615785554728)(93,0.811011904761905)(94,0.814040328603435)(95,0.814648729446936)(96,0.813432835820895)(97,0.81282624906786)(98,0.81282624906786)(99,0.814040328603435)(100,0.814040328603435)(101,0.816479400749064)(102,0.818318318318318)(103,0.815868263473054)(104,0.814648729446936)(105,0.814040328603435)(106,0.814040328603435)(107,0.81282624906786)(108,0.809806835066865)(109,0.809806835066865)(110,0.809806835066865)(111,0.80680977054034)(112,0.801470588235294)(113,0.80206033848418)(114,0.799706529713866)(115,0.800881704628949)(116,0.796783625730994)(117,0.796783625730994)(118,0.795620437956204)(119,0.796783625730994)(120,0.796783625730994)(121,0.799120234604106)(122,0.799120234604106)(123,0.805022156573117)(124,0.805022156573117)(125,0.803242446573323)(126,0.805022156573117)(127,0.805022156573117)(128,0.808605341246291)(129,0.807407407407407)(130,0.809806835066865)(131,0.810408921933085)(132,0.809806835066865)(133,0.81282624906786)(134,0.811615785554728)(135,0.814040328603435)(136,0.807407407407407)(137,0.807407407407407)(138,0.808005930318755)(139,0.808605341246291)(140,0.808605341246291)(141,0.805022156573117)(142,0.803834808259587)(143,0.803834808259587)(144,0.803834808259587)(145,0.803834808259587)(146,0.803834808259587)(147,0.804428044280443)(148,0.804428044280443)(149,0.807407407407407)(150,0.80680977054034)(151,0.80680977054034)(152,0.80680977054034)(153,0.804428044280443)(154,0.80680977054034)(155,0.805022156573117)(156,0.806213017751479)(157,0.80680977054034)(158,0.80920564216778)(159,0.817704426106526)(160,0.816479400749064)(161,0.81525804038893)(162,0.812220566318927)(163,0.812220566318927)(164,0.815868263473054)(165,0.814040328603435)(166,0.810408921933085)(167,0.809806835066865)(168,0.809806835066865)(169,0.809806835066865)(170,0.808005930318755)(171,0.808005930318755)(172,0.80920564216778)(173,0.80920564216778)(174,0.810408921933085)(175,0.810408921933085)(176,0.811615785554728)(177,0.812220566318927)(178,0.81282624906786)(179,0.812220566318927)(180,0.812220566318927)(181,0.811011904761905)(182,0.811615785554728)(183,0.813432835820895)(184,0.812220566318927)(185,0.812220566318927)(186,0.812220566318927)(187,0.812220566318927)(188,0.817091454272864)(189,0.819548872180451)(190,0.82078313253012)(191,0.819548872180451)(192,0.821401657874906)(193,0.822641509433962)(194,0.822021116138763)(195,0.825132475397426)(196,0.823262839879154)(197,0.822641509433962)(198,0.822021116138763)(199,0.820165537998495)(200,0.817091454272864)(201,0.821401657874906)(202,0.822641509433962)(203,0.822641509433962)(204,0.83015993907083)(205,0.831426392067124)(206,0.831426392067124)(207,0.81893313298272)(208,0.814648729446936)(209,0.808605341246291)(210,0.81525804038893)(211,0.816479400749064)(212,0.816479400749064)(213,0.817704426106526)(214,0.81893313298272)(215,0.82078313253012)(216,0.821401657874906)(217,0.822021116138763)(218,0.822021116138763)(219,0.819548872180451)(220,0.81893313298272)(221,0.81893313298272)(222,0.819548872180451)(223,0.817704426106526)(224,0.817704426106526)(225,0.817704426106526)(226,0.81893313298272)(227,0.822641509433962)(228,0.813432835820895)(229,0.814040328603435)(230,0.822021116138763)(231,0.825132475397426)(232,0.826383623957544)(233,0.827638572513288)(234,0.828897338403042)(235,0.83015993907083)(236,0.83015993907083)(237,0.829528158295281)(238,0.846930846930847)(239,0.846930846930847)(240,0.841698841698842)(241,0.843000773395205)(242,0.841698841698842)(243,0.83910700538876)(244,0.842349304482226)(245,0.834609494640122)(246,0.844306738962045)(247,0.849571317225253)(248,0.851996867658575)(249,0.851996867658575)(250,0.8515625)(251,0.852664576802508)(252,0.852664576802508)(253,0.853333333333333)(254,0.851996867658575)(255,0.852664576802508)(256,0.852664576802508)(257,0.851968503937008)(258,0.85377358490566)(259,0.85377358490566)(260,0.854673998428908)(261,0.854673998428908)(262,0.854673998428908)(263,0.85173501577287)(264,0.85173501577287)(265,0.85173501577287)(266,0.85377358490566)(267,0.854003139717425)(268,0.853333333333333)(269,0.854445318646735)(270,0.854445318646735)(271,0.854445318646735)(272,0.85511811023622)(273,0.85377358490566)(274,0.854003139717425)(275,0.855791962174941)(276,0.854673998428908)(277,0.855791962174941)(278,0.852614896988906)(279,0.850793650793651)(280,0.850793650793651)(281,0.850793650793651)(282,0.849880857823669)(283,0.847808764940239)(284,0.848484848484848)(285,0.848484848484848)(286,0.849162011173184)(287,0.849162011173184)(288,0.845476381104884)(289,0.845476381104884)(290,0.845906902086677)(291,0.845659163987138)(292,0.845659163987138)(293,0.846586345381526)(294,0.845659163987138)(295,0.846586345381526)(296,0.847512038523274)(297,0.848436246992783)(298,0.8512)(299,0.848436246992783)(300,0.84775641025641)(301,0.84775641025641)(302,0.84775641025641)(303,0.848436246992783)(304,0.848436246992783)(305,0.848436246992783)(306,0.848436246992783)(307,0.852118305355715)(308,0.850280224179343)(309,0.853035143769968)(310,0.851674641148325)(311,0.851030110935024)(312,0.851030110935024)(313,0.85193982581156)(314,0.851704996034893)(315,0.851030110935024)(316,0.85214626391097)(317,0.851910828025478)(318,0.853102906520031)(319,0.851097178683385)(320,0.847352024922118)(321,0.848673946957878)(322,0.848012470771629)(323,0.849765258215962)(324,0.849765258215962)(325,0.849571317225253)(326,0.848249027237354)(327,0.850863422291994)(328,0.848864526233359)(329,0.848864526233359)(330,0.851531814611155)(331,0.851531814611155)(332,0.851531814611155)(333,0.851233094669849)(334,0.850556438791733)(335,0.850079744816587)(336,0.850079744816587)(337,0.8496)(338,0.850996015936255)(339,0.850996015936255)(340,0.850556438791733)(341,0.851233094669849)(342,0.851233094669849)(343,0.851233094669849)(344,0.849402390438247)(345,0.849402390438247)(346,0.850079744816587)(347,0.850996015936255)(348,0.850079744816587)(349,0.850079744816587)(350,0.850079744816587)(351,0.850079744816587)(352,0.849642004773269)(353,0.850556438791733)(354,0.849880857823669)(355,0.849683544303797)(356,0.850356294536817)(357,0.851030110935024)(358,0.848966613672496)(359,0.850079744816587)(360,0.850079744816587)(361,0.850079744816587)(362,0.849358974358974)(363,0.849358974358974)(364,0.849358974358974)(365,0.849358974358974)(366,0.847512038523274)(367,0.839805825242718)(368,0.838866396761134)(369,0.838866396761134)(370,0.839805825242718)(371,0.83698296836983)(372,0.835093419983753)(373,0.83319772172498)(374,0.832247557003257)(375,0.832247557003257)(376,0.82843137254902)(377,0.829387755102041)(378,0.829387755102041)(379,0.824590163934426)(380,0.823625922887613)(381,0.825552825552825)(382,0.826513911620295)(383,0.826513911620295)(384,0.826513911620295)(385,0.825552825552825)(386,0.825552825552825)(387,0.824590163934426)(388,0.824590163934426)(389,0.824590163934426)(390,0.823625922887613)(391,0.825552825552825)(392,0.822660098522167)(393,0.822660098522167)(394,0.822660098522167)(395,0.811930405965203)(396,0.812913907284768)(397,0.808970099667774)(398,0.812913907284768)(399,0.813895781637717)(400,0.813895781637717)(4,0.694767441860465)(5,0.701277955271565)(6,0.682887266828873)(7,0.68595041322314)(8,0.741029641185647)(9,0.741049125728559)(10,0.731623931623932)(11,0.747855917667238)(12,0.743787489288774)(13,0.757627118644068)(14,0.759087066779374)(15,0.774354704412989)(16,0.778790389395195)(17,0.779436152570481)(18,0.791461412151067)(19,0.783828382838284)(20,0.78441127694859)(21,0.786776859504132)(22,0.786776859504132)(23,0.786776859504132)(24,0.787778695293146)(25,0.793729372937294)(26,0.802653399668325)(27,0.798663324979114)(28,0.785774767146486)(29,0.787162162162162)(30,0.789958158995816)(31,0.786912751677852)(32,0.792988313856427)(33,0.794338051623647)(34,0.794338051623647)(35,0.797353184449959)(36,0.803615447822514)(37,0.799339388934765)(38,0.804311774461028)(39,0.803319502074689)(40,0.804311774461028)(41,0.803319502074689)(42,0.802653399668325)(43,0.804979253112033)(44,0.804979253112033)(45,0.801996672212978)(46,0.802325581395349)(47,0.8)(48,0.793969849246231)(49,0.791946308724832)(50,0.793969849246231)(51,0.791946308724832)(52,0.789915966386555)(53,0.792958927074602)(54,0.791946308724832)(55,0.790931989924433)(56,0.79163179916318)(57,0.788590604026846)(58,0.788590604026846)(59,0.79465776293823)(60,0.793650793650794)(61,0.793650793650794)(62,0.792958927074602)(63,0.791946308724832)(64,0.792958927074602)(65,0.796992481203007)(66,0.795986622073578)(67,0.791946308724832)(68,0.793969849246231)(69,0.790931989924433)(70,0.790931989924433)(71,0.790931989924433)(72,0.793969849246231)(73,0.797996661101836)(74,0.796992481203007)(75,0.793969849246231)(76,0.797996661101836)(77,0.795986622073578)(78,0.792958927074602)(79,0.794979079497908)(80,0.789915966386555)(81,0.790931989924433)(82,0.792958927074602)(83,0.791946308724832)(84,0.785834738617201)(85,0.783783783783784)(86,0.78135593220339)(87,0.78135593220339)(88,0.780322307039864)(89,0.779286926994907)(90,0.777210884353741)(91,0.777210884353741)(92,0.777210884353741)(93,0.776170212765957)(94,0.776831345826235)(95,0.777872340425532)(96,0.772260273972603)(97,0.772260273972603)(98,0.768041237113402)(99,0.768041237113402)(100,0.766981943250215)(101,0.768041237113402)(102,0.768041237113402)(103,0.770154373927959)(104,0.770154373927959)(105,0.770154373927959)(106,0.77120822622108)(107,0.77120822622108)(108,0.774358974358974)(109,0.773310521813516)(110,0.772260273972603)(111,0.77120822622108)(112,0.77120822622108)(113,0.772260273972603)(114,0.770154373927959)(115,0.768041237113402)(116,0.768041237113402)(117,0.768041237113402)(118,0.77120822622108)(119,0.770154373927959)(120,0.766981943250215)(121,0.768041237113402)(122,0.768041237113402)(123,0.770154373927959)(124,0.77120822622108)(125,0.77120822622108)(126,0.77120822622108)(127,0.769098712446352)(128,0.769098712446352)(129,0.770154373927959)(130,0.775405636208369)(131,0.776450511945392)(132,0.774358974358974)(133,0.774358974358974)(134,0.777872340425532)(135,0.779949022939677)(136,0.779949022939677)(137,0.780984719864176)(138,0.782018659881255)(139,0.780984719864176)(140,0.780984719864176)(141,0.779949022939677)(142,0.780984719864176)(143,0.780984719864176)(144,0.780984719864176)(145,0.780984719864176)(146,0.780984719864176)(147,0.780984719864176)(148,0.782018659881255)(149,0.782018659881255)(150,0.782018659881255)(151,0.782018659881255)(152,0.782018659881255)(153,0.782018659881255)(154,0.779949022939677)(155,0.780984719864176)(156,0.784081287044877)(157,0.784081287044877)(158,0.779574468085106)(159,0.780612244897959)(160,0.780612244897959)(161,0.781648258283772)(162,0.777493606138107)(163,0.777493606138107)(164,0.777493606138107)(165,0.777493606138107)(166,0.777493606138107)(167,0.777493606138107)(168,0.777493606138107)(169,0.779574468085106)(170,0.778534923339012)(171,0.778534923339012)(172,0.779574468085106)(173,0.779574468085106)(174,0.780612244897959)(175,0.781648258283772)(176,0.780612244897959)(177,0.781648258283772)(178,0.791912384161752)(179,0.794290512174643)(180,0.792262405382674)(181,0.791912384161752)(182,0.789873417721519)(183,0.789873417721519)(184,0.789873417721519)(185,0.789873417721519)(186,0.783715012722646)(187,0.785774767146486)(188,0.784745762711864)(189,0.784745762711864)(190,0.784745762711864)(191,0.784745762711864)(192,0.780612244897959)(193,0.780612244897959)(194,0.778534923339012)(195,0.778534923339012)(196,0.779574468085106)(197,0.779574468085106)(198,0.781648258283772)(199,0.781648258283772)(200,0.781648258283772)(201,0.781648258283772)(202,0.781648258283772)(203,0.781648258283772)(204,0.781648258283772)(205,0.782682512733446)(206,0.790893760539629)(207,0.790893760539629)(208,0.797988264878458)(209,0.797988264878458)(210,0.796979865771812)(211,0.797988264878458)(212,0.798994974874372)(213,0.8)(214,0.8)(215,0.798994974874372)(216,0.797988264878458)(217,0.801003344481605)(218,0.798994974874372)(219,0.798994974874372)(220,0.8)(221,0.802005012531328)(222,0.805)(223,0.8)(224,0.8)(225,0.8)(226,0.802005012531328)(227,0.801003344481605)(228,0.798994974874372)(229,0.8)(230,0.804003336113428)(231,0.808970099667774)(232,0.816158285243199)(233,0.811930405965203)(234,0.811930405965203)(235,0.818106995884774)(236,0.821985233798195)(237,0.821018062397373)(238,0.821018062397373)(239,0.821018062397373)(240,0.81420313790256)(241,0.812241521918941)(242,0.813223140495868)(243,0.812913907284768)(244,0.806988352745424)(245,0.807980049875312)(246,0.812913907284768)(247,0.807980049875312)(248,0.809958506224066)(249,0.811930405965203)(250,0.815854665565648)(251,0.815854665565648)(252,0.815854665565648)(253,0.817807089859852)(254,0.817807089859852)(255,0.817807089859852)(256,0.817807089859852)(257,0.817807089859852)(258,0.816831683168317)(259,0.816831683168317)(260,0.816831683168317)(261,0.816831683168317)(262,0.816831683168317)(263,0.816831683168317)(264,0.815854665565648)(265,0.815854665565648)(266,0.816831683168317)(267,0.816831683168317)(268,0.813895781637717)(269,0.817807089859852)(270,0.816831683168317)(271,0.816831683168317)(272,0.816831683168317)(273,0.816831683168317)(274,0.817807089859852)(275,0.817807089859852)(276,0.817807089859852)(277,0.819753086419753)(278,0.818780889621087)(279,0.816831683168317)(280,0.818780889621087)(281,0.818780889621087)(282,0.817807089859852)(283,0.817807089859852)(284,0.820723684210526)(285,0.820723684210526)(286,0.820723684210526)(287,0.820723684210526)(288,0.824590163934426)(289,0.824590163934426)(290,0.826513911620295)(291,0.826513911620295)(292,0.826513911620295)(293,0.826513911620295)(294,0.820723684210526)(295,0.820723684210526)(296,0.817807089859852)(297,0.820723684210526)(298,0.824590163934426)(299,0.818780889621087)(300,0.810945273631841)(301,0.810945273631841)(302,0.809958506224066)(303,0.809958506224066)(304,0.809958506224066)(305,0.809958506224066)(306,0.811930405965203)(307,0.812913907284768)(308,0.812913907284768)(309,0.818780889621087)(310,0.818780889621087)(311,0.818780889621087)(312,0.818780889621087)(313,0.821692686935086)(314,0.817807089859852)(315,0.817807089859852)(316,0.812913907284768)(317,0.807980049875312)(318,0.807980049875312)(319,0.807980049875312)(320,0.807980049875312)(321,0.806988352745424)(322,0.807980049875312)(323,0.807980049875312)(324,0.801003344481605)(325,0.8)(326,0.8)(327,0.8)(328,0.8)(329,0.79394449116905)(330,0.792929292929293)(331,0.788851351351351)(332,0.789873417721519)(333,0.789873417721519)(334,0.786802030456853)(335,0.781648258283772)(336,0.782682512733446)(337,0.782682512733446)(338,0.784745762711864)(339,0.784745762711864)(340,0.784745762711864)(341,0.787827557058326)(342,0.787827557058326)(343,0.787827557058326)(344,0.787827557058326)(345,0.787827557058326)(346,0.784745762711864)(347,0.789873417721519)(348,0.789873417721519)(349,0.789873417721519)(350,0.789873417721519)(351,0.789873417721519)(352,0.789873417721519)(353,0.788851351351351)(354,0.788851351351351)(355,0.790893760539629)(356,0.790893760539629)(357,0.789873417721519)(358,0.789873417721519)(359,0.79394449116905)(360,0.796979865771812)(361,0.797988264878458)(362,0.796979865771812)(363,0.796979865771812)(364,0.794957983193277)(365,0.796979865771812)(366,0.796979865771812)(367,0.795969773299748)(368,0.795969773299748)(369,0.796979865771812)(370,0.796979865771812)(371,0.802005012531328)(372,0.802005012531328)(373,0.802005012531328)(374,0.8)(375,0.795969773299748)(376,0.802005012531328)(377,0.802005012531328)(378,0.802005012531328)(379,0.802005012531328)(380,0.802005012531328)(381,0.802005012531328)(382,0.802005012531328)(383,0.805995004163197)(384,0.805995004163197)(385,0.805)(386,0.808970099667774)(387,0.807980049875312)(388,0.805)(389,0.805)(390,0.809958506224066)(391,0.810945273631841)(392,0.809958506224066)(393,0.810945273631841)(394,0.810945273631841)(395,0.810945273631841)(396,0.809958506224066)(397,0.810945273631841)(398,0.810945273631841)(399,0.809958506224066)(400,0.809958506224066)(8,0.566939890710382)(9,0.639949109414758)(10,0.71195652173913)(11,0.712595685455811)(12,0.734828496042216)(13,0.741935483870968)(14,0.75707702435813)(15,0.750830564784053)(16,0.730307076101469)(17,0.776659959758551)(18,0.781624500665779)(19,0.788860103626943)(20,0.809587217043941)(21,0.809587217043941)(22,0.810740013097577)(23,0.808973487423521)(24,0.806607019958706)(25,0.799482535575679)(26,0.806068601583113)(27,0.819512195121951)(28,0.817610062893082)(29,0.817610062893082)(30,0.810408921933085)(31,0.81203007518797)(32,0.814528593508501)(33,0.812400635930048)(34,0.811526479750779)(35,0.814992025518341)(36,0.819698173153296)(37,0.820269200316706)(38,0.863602110022607)(39,0.876488095238095)(40,0.869760479041916)(41,0.873511904761905)(42,0.870605833956619)(43,0.855835240274599)(44,0.858231707317073)(45,0.859554873369148)(46,0.852484472049689)(47,0.852484472049689)(48,0.853146853146853)(49,0.858024691358025)(50,0.857582755966128)(51,0.861302681992337)(52,0.865648854961832)(53,0.865443425076453)(54,0.86412213740458)(55,0.870943396226415)(56,0.868181818181818)(57,0.875)(58,0.87874251497006)(59,0.87856071964018)(60,0.87874251497006)(61,0.878195488721804)(62,0.877245508982036)(63,0.876233864844343)(64,0.878603945371775)(65,0.878122634367903)(66,0.878012048192771)(67,0.877166541070083)(68,0.877166541070083)(69,0.881203007518797)(70,0.881203007518797)(71,0.890044576523031)(72,0.884586746090841)(73,0.887724550898204)(74,0.88689138576779)(75,0.887724550898204)(76,0.88689138576779)(77,0.888722927557879)(78,0.887892376681614)(79,0.882882882882883)(80,0.882706766917293)(81,0.881866064710308)(82,0.881024096385542)(83,0.886056971514243)(84,0.888722927557879)(85,0.88955223880597)(86,0.889393939393939)(87,0.898021308980213)(88,0.896499238964992)(89,0.893974065598779)(90,0.893974065598779)(91,0.893974065598779)(92,0.889739663093415)(93,0.889739663093415)(94,0.890421455938697)(95,0.891271056661562)(96,0.89739663093415)(97,0.900763358778626)(98,0.903274942878903)(99,0.904942965779468)(100,0.904109589041096)(101,0.899082568807339)(102,0.899082568807339)(103,0.898240244835501)(104,0.89739663093415)(105,0.89739663093415)(106,0.901451489686784)(107,0.901451489686784)(108,0.901451489686784)(109,0.906463878326996)(110,0.903963414634146)(111,0.903963414634146)(112,0.899770466717674)(113,0.898084291187739)(114,0.898928024502297)(115,0.898928024502297)(116,0.896392939370683)(117,0.898084291187739)(118,0.898928024502297)(119,0.898928024502297)(120,0.898928024502297)(121,0.898928024502297)(122,0.894696387394312)(123,0.894696387394312)(124,0.895545314900153)(125,0.897239263803681)(126,0.895545314900153)(127,0.904109589041096)(128,0.900151285930409)(129,0.904109589041096)(130,0.902050113895216)(131,0.90702947845805)(132,0.908132530120482)(133,0.908132530120482)(134,0.910876132930514)(135,0.910741301059001)(136,0.912518853695324)(137,0.914156626506024)(138,0.91578947368421)(139,0.917293233082707)(140,0.918106686701728)(141,0.919729932483121)(142,0.918918918918919)(143,0.919850187265918)(144,0.922734026745914)(145,0.925219941348973)(146,0.925329428989751)(147,0.921364985163205)(148,0.91904047976012)(149,0.919729932483121)(150,0.91904047976012)(151,0.922272047832586)(152,0.921465968586387)(153,0.921465968586387)(154,0.924570575056012)(155,0.924570575056012)(156,0.924570575056012)(157,0.925373134328358)(158,0.923880597014925)(159,0.926284437825763)(160,0.928783382789317)(161,0.92820133234641)(162,0.92820133234641)(163,0.927881040892193)(164,0.926284437825763)(165,0.922155688622754)(166,0.922961854899027)(167,0.922961854899027)(168,0.922846441947565)(169,0.922155688622754)(170,0.922846441947565)(171,0.924457741211668)(172,0.924457741211668)(173,0.926064227035101)(174,0.926064227035101)(175,0.926064227035101)(176,0.926064227035101)(177,0.926064227035101)(178,0.925261584454409)(179,0.927665920954512)(180,0.927665920954512)(181,0.930059523809524)(182,0.930059523809524)(183,0.930059523809524)(184,0.927665920954512)(185,0.929262844378257)(186,0.928464977645305)(187,0.928464977645305)(188,0.928464977645305)(189,0.928464977645305)(190,0.927665920954512)(191,0.928464977645305)(192,0.928464977645305)(193,0.928464977645305)(194,0.928464977645305)(195,0.928464977645305)(196,0.942124542124542)(197,0.943672275054865)(198,0.942982456140351)(199,0.942982456140351)(200,0.942209217264082)(201,0.942209217264082)(202,0.941605839416058)(203,0.940919037199125)(204,0.940158615717376)(205,0.94014598540146)(206,0.939283101682516)(207,0.938686131386861)(208,0.937134502923976)(209,0.939548434085943)(210,0.938864628820961)(211,0.934017595307918)(212,0.934017595307918)(213,0.934017595307918)(214,0.934798534798535)(215,0.935389133627019)(216,0.937728937728938)(217,0.937728937728938)(218,0.937820043891734)(219,0.937042459736457)(220,0.936857562408223)(221,0.936857562408223)(222,0.939970717423133)(223,0.936950146627566)(224,0.933823529411765)(225,0.93460690668626)(226,0.93460690668626)(227,0.935389133627019)(228,0.935389133627019)(229,0.931365313653136)(230,0.931365313653136)(231,0.920777279521674)(232,0.921583271097834)(233,0.921583271097834)(234,0.919970082273747)(235,0.919161676646707)(236,0.919970082273747)(237,0.919970082273747)(238,0.919970082273747)(239,0.919161676646707)(240,0.923306031273269)(241,0.924107142857143)(242,0.923191648023863)(243,0.922503725782414)(244,0.925705794947994)(245,0.927407407407407)(246,0.927407407407407)(247,0.92820133234641)(248,0.926503340757238)(249,0.932841328413284)(250,0.933628318584071)(251,0.933628318584071)(252,0.921583271097834)(253,0.921583271097834)(254,0.923191648023863)(255,0.923994038748137)(256,0.923994038748137)(257,0.923994038748137)(258,0.923994038748137)(259,0.923994038748137)(260,0.924795234549516)(261,0.922272047832586)(262,0.922272047832586)(263,0.919850187265918)(264,0.933628318584071)(265,0.932841328413284)(266,0.933726067746686)(267,0.926284437825763)(268,0.926284437825763)(269,0.926284437825763)(270,0.926284437825763)(271,0.926284437825763)(272,0.928783382789317)(273,0.928094885100074)(274,0.928783382789317)(275,0.928783382789317)(276,0.928783382789317)(277,0.926394052044609)(278,0.927191679049034)(279,0.927988121752041)(280,0.927988121752041)(281,0.927988121752041)(282,0.927299703264095)(283,0.928094885100074)(284,0.927407407407407)(285,0.92603550295858)(286,0.927407407407407)(287,0.927407407407407)(288,0.927407407407407)(289,0.927407407407407)(290,0.927407407407407)(291,0.926720947446336)(292,0.927514792899408)(293,0.926720947446336)(294,0.926720947446336)(295,0.927514792899408)(296,0.930780559646539)(297,0.930780559646539)(298,0.930780559646539)(299,0.930780559646539)(300,0.93235294117647)(301,0.93235294117647)(302,0.93235294117647)(303,0.93235294117647)(304,0.93235294117647)(305,0.93235294117647)(306,0.931567328918322)(307,0.930780559646539)(308,0.930780559646539)(309,0.930780559646539)(310,0.930780559646539)(311,0.930780559646539)(312,0.931567328918322)(313,0.929992630803242)(314,0.929992630803242)(315,0.929203539823009)(316,0.928413284132841)(317,0.928413284132841)(318,0.926612305411416)(319,0.928307464892831)(320,0.926612305411416)(321,0.925816023738872)(322,0.925816023738872)(323,0.925816023738872)(324,0.928307464892831)(325,0.924107142857143)(326,0.928307464892831)(327,0.928307464892831)(328,0.928307464892831)(329,0.928307464892831)(330,0.928307464892831)(331,0.928307464892831)(332,0.929098966026588)(333,0.922272047832586)(334,0.922961854899027)(335,0.922961854899027)(336,0.924570575056012)(337,0.925373134328358)(338,0.917293233082707)(339,0.917293233082707)(340,0.917293233082707)(341,0.917293233082707)(342,0.929577464788732)(343,0.924570575056012)(344,0.925373134328358)(345,0.930473372781065)(346,0.930473372781065)(347,0.930473372781065)(348,0.930473372781065)(349,0.930473372781065)(350,0.930473372781065)(351,0.930473372781065)(352,0.930473372781065)(353,0.930473372781065)(354,0.930473372781065)(355,0.930473372781065)(356,0.930473372781065)(357,0.930473372781065)(358,0.930473372781065)(359,0.930473372781065)(360,0.930473372781065)(361,0.930473372781065)(362,0.930473372781065)(363,0.930473372781065)(364,0.930473372781065)(365,0.930473372781065)(366,0.930473372781065)(367,0.930473372781065)(368,0.930473372781065)(369,0.930473372781065)(370,0.934414148857774)(371,0.936764705882353)(372,0.93598233995585)(373,0.934414148857774)(374,0.93519882179676)(375,0.932841328413284)(376,0.93205317577548)(377,0.933628318584071)(378,0.933628318584071)(379,0.93598233995585)(380,0.936764705882353)(381,0.936764705882353)(382,0.936764705882353)(383,0.939104915627293)(384,0.941434846266471)(385,0.936764705882353)(386,0.936764705882353)(387,0.936764705882353)(388,0.938325991189427)(389,0.942209217264082)(390,0.939104915627293)(391,0.939104915627293)(392,0.937545922116091)(393,0.937545922116091)(394,0.937545922116091)(395,0.939882697947214)(396,0.940659340659341)(397,0.942209217264082)(398,0.943754565376187)(399,0.946831755280408)(400,0.947598253275109)(3,0.528228423101882)(4,0.563133018230925)(5,0.601858470335954)(6,0.635568513119534)(7,0.624434389140271)(8,0.65623118603251)(9,0.686830497794581)(10,0.688131313131313)(11,0.688566013897663)(12,0.699166132135985)(13,0.68639798488665)(14,0.690310322989233)(15,0.692503176620076)(16,0.698717948717949)(17,0.700064226075787)(18,0.687263556116015)(19,0.680824484697064)(20,0.696485623003195)(21,0.695596681557116)(22,0.690747782002535)(23,0.707792207792208)(24,0.708712613784135)(25,0.714754098360656)(26,0.707332900713822)(27,0.705501618122977)(28,0.699166132135985)(29,0.71102413568167)(30,0.698029243483789)(31,0.710576314122862)(32,0.71190781049936)(33,0.693495038588754)(34,0.693200663349917)(35,0.741701244813278)(36,0.741316744427164)(37,0.734631147540983)(38,0.733982573039467)(39,0.7302396736359)(40,0.735112936344969)(41,0.7302396736359)(42,0.735112936344969)(43,0.733982573039467)(44,0.733128834355828)(45,0.739938080495356)(46,0.744161909704203)(47,0.74223602484472)(48,0.737275064267352)(49,0.739938080495356)(50,0.739938080495356)(51,0.738794435857805)(52,0.74300518134715)(53,0.757127771911299)(54,0.751966439433665)(55,0.75)(56,0.749216300940439)(57,0.749216300940439)(58,0.751178627553693)(59,0.741085271317829)(60,0.741085271317829)(61,0.739175257731959)(62,0.738414006179197)(63,0.736896197327852)(64,0.735761929194459)(65,0.735007688364941)(66,0.736139630390144)(67,0.735761929194459)(68,0.736896197327852)(69,0.737275064267352)(70,0.736896197327852)(71,0.733503836317135)(72,0.733879222108495)(73,0.733128834355828)(74,0.733128834355828)(75,0.733128834355828)(76,0.730886850152905)(77,0.728658536585366)(78,0.729770992366412)(79,0.730142566191446)(80,0.727180527383367)(81,0.7293997965412)(82,0.729770992366412)(83,0.730142566191446)(84,0.730514518593989)(85,0.729770992366412)(86,0.7293997965412)(87,0.72644376899696)(88,0.728658536585366)(89,0.730514518593989)(90,0.730142566191446)(91,0.727549467275495)(92,0.7293997965412)(93,0.730886850152905)(94,0.730142566191446)(95,0.730514518593989)(96,0.730514518593989)(97,0.731632653061224)(98,0.732006125574273)(99,0.730886850152905)(100,0.730886850152905)(101,0.730142566191446)(102,0.730514518593989)(103,0.730514518593989)(104,0.729028978139298)(105,0.729770992366412)(106,0.729770992366412)(107,0.729770992366412)(108,0.724608388074785)(109,0.724608388074785)(110,0.722418136020151)(111,0.721327967806841)(112,0.720603015075377)(113,0.720241084881969)(114,0.720241084881969)(115,0.720241084881969)(116,0.71735867933967)(117,0.718796992481203)(118,0.719157472417252)(119,0.72351160443996)(120,0.725341426403642)(121,0.724974721941355)(122,0.718436873747495)(123,0.721690991444388)(124,0.722418136020151)(125,0.720965309200603)(126,0.726075949367089)(127,0.72644376899696)(128,0.728288471305231)(129,0.7293997965412)(130,0.758328926493918)(131,0.758328926493918)(132,0.759131815775543)(133,0.760742705570292)(134,0.76033934252386)(135,0.765208110992529)(136,0.766844919786096)(137,0.770967741935484)(138,0.772213247172859)(139,0.770967741935484)(140,0.775554353704705)(141,0.782324058919803)(142,0.781897491821156)(143,0.781471389645776)(144,0.780620577027763)(145,0.780620577027763)(146,0.780620577027763)(147,0.780620577027763)(148,0.780620577027763)(149,0.778501628664495)(150,0.777235772357723)(151,0.775135135135135)(152,0.775135135135135)(153,0.775554353704705)(154,0.774298056155507)(155,0.788778877887789)(156,0.790082644628099)(157,0.787912087912088)(158,0.788778877887789)(159,0.787912087912088)(160,0.787047200878156)(161,0.789647577092511)(162,0.790082644628099)(163,0.792265193370166)(164,0.792265193370166)(165,0.791827719491993)(166,0.802911534154535)(167,0.801565120178871)(168,0.800670016750419)(169,0.797552836484983)(170,0.798440979955456)(171,0.800670016750419)(172,0.805165637282426)(173,0.80561797752809)(174,0.807887323943662)(175,0.809255079006772)(176,0.867513611615245)(177,0.868038740920097)(178,0.872262773722628)(179,0.872262773722628)(180,0.872262773722628)(181,0.871202916160389)(182,0.844025897586816)(183,0.843529411764706)(184,0.847517730496454)(185,0.848018923713779)(186,0.850533807829181)(187,0.873857404021938)(188,0.87599266951741)(189,0.876528117359413)(190,0.873325213154689)(191,0.870673952641166)(192,0.873325213154689)(193,0.871202916160389)(194,0.872793670115642)(195,0.872793670115642)(196,0.872793670115642)(197,0.873325213154689)(198,0.878138395590937)(199,0.87599266951741)(200,0.872262773722628)(201,0.875457875457875)(202,0.879754601226994)(203,0.870145631067961)(204,0.871202916160389)(205,0.870673952641166)(206,0.865942028985507)(207,0.865942028985507)(208,0.860227954409118)(209,0.861261261261261)(210,0.861612515042118)(211,0.861778846153846)(212,0.862815884476534)(213,0.862815884476534)(214,0.864897466827503)(215,0.863855421686747)(216,0.860227954409118)(217,0.859712230215827)(218,0.859712230215827)(219,0.859712230215827)(220,0.861261261261261)(221,0.860227954409118)(222,0.860227954409118)(223,0.859712230215827)(224,0.861261261261261)(225,0.861261261261261)(226,0.851543942992874)(227,0.85204991087344)(228,0.853571428571428)(229,0.855098389982111)(230,0.854079809410363)(231,0.853571428571428)(232,0.853571428571428)(233,0.854079809410363)(234,0.854079809410363)(235,0.853063652587745)(236,0.849526066350711)(237,0.850029638411381)(238,0.850533807829181)(239,0.858168761220826)(240,0.857142857142857)(241,0.85663082437276)(242,0.85663082437276)(243,0.857655502392344)(244,0.865942028985507)(245,0.865419432709716)(246,0.867513611615245)(247,0.871202916160389)(248,0.871732522796352)(249,0.872262773722628)(250,0.871732522796352)(251,0.870673952641166)(252,0.872262773722628)(253,0.871732522796352)(254,0.872262773722628)(255,0.873857404021938)(256,0.872262773722628)(257,0.873325213154689)(258,0.873325213154689)(259,0.873857404021938)(260,0.873857404021938)(261,0.874923733984136)(262,0.873857404021938)(263,0.87599266951741)(264,0.876528117359413)(265,0.875457875457875)(266,0.874923733984136)(267,0.875457875457875)(268,0.874390243902439)(269,0.874390243902439)(270,0.874390243902439)(271,0.877064220183486)(272,0.875688916105327)(273,0.875)(274,0.875688916105327)(275,0.877839165131983)(276,0.878378378378378)(277,0.880541871921182)(278,0.880541871921182)(279,0.877300613496932)(280,0.878378378378378)(281,0.878378378378378)(282,0.878378378378378)(283,0.883663366336633)(284,0.880937692782233)(285,0.883663366336633)(286,0.883663366336633)(287,0.887780548628429)(288,0.888888888888889)(289,0.894043887147335)(290,0.897856242118537)(291,0.897856242118537)(292,0.896161107614852)(293,0.893350062735257)(294,0.894472361809045)(295,0.895465994962216)(296,0.89685534591195)(297,0.89685534591195)(298,0.89572864321608)(299,0.896291640477687)(300,0.896291640477687)(301,0.894176581089543)(302,0.893617021276596)(303,0.896030245746692)(304,0.893777498428661)(305,0.894472361809045)(306,0.894604767879548)(307,0.89572864321608)(308,0.896291640477687)(309,0.896421845574388)(310,0.896421845574388)(311,0.896421845574388)(312,0.893058161350844)(313,0.892365456821026)(314,0.892365456821026)(315,0.89829437776374)(316,0.896030245746692)(317,0.897290485192186)(318,0.897290485192186)(319,0.903307888040712)(320,0.898862199747155)(321,0.899430740037951)(322,0.897727272727273)(323,0.895465994962216)(324,0.895465994962216)(325,0.898932831136221)(326,0.890683229813665)(327,0.886279357231149)(328,0.886279357231149)(329,0.886279357231149)(330,0.885731933292156)(331,0.885185185185185)(332,0.885185185185185)(333,0.883004926108374)(334,0.885185185185185)(335,0.871202916160389)(336,0.871732522796352)(337,0.871732522796352)(338,0.871202916160389)(339,0.872793670115642)(340,0.871202916160389)(341,0.868564506359782)(342,0.867513611615245)(343,0.869617950272892)(344,0.872262773722628)(345,0.887925696594427)(346,0.887925696594427)(347,0.887376237623762)(348,0.88682745825603)(349,0.887925696594427)(350,0.887376237623762)(351,0.889026658400496)(352,0.889026658400496)(353,0.889026658400496)(354,0.887925696594427)(355,0.887925696594427)(356,0.885731933292156)(357,0.885731933292156)(358,0.884639111659469)(359,0.880294659300184)(360,0.879754601226994)(361,0.879215205395463)(362,0.879215205395463)(363,0.876528117359413)(364,0.87599266951741)(365,0.875457875457875)(366,0.875457875457875)(367,0.874390243902439)(368,0.877064220183486)(369,0.87599266951741)(370,0.87599266951741)(371,0.877064220183486)(372,0.877064220183486)(373,0.876528117359413)(374,0.876528117359413)(375,0.87599266951741)(376,0.864897466827503)(377,0.864897466827503)(378,0.863855421686747)(379,0.863855421686747)(380,0.864376130198915)(381,0.863855421686747)(382,0.863855421686747)(383,0.863855421686747)(384,0.864376130198915)(385,0.864376130198915)(386,0.863855421686747)(387,0.864376130198915)(388,0.858168761220826)(389,0.858168761220826)(390,0.860227954409118)(391,0.859712230215827)(392,0.856119402985075)(393,0.855608591885441)(394,0.858682634730539)(395,0.858682634730539)(396,0.858168761220826)(397,0.853571428571428)(398,0.853571428571428)(399,0.853571428571428)(400,0.854588796185936)(10,0.648773006134969)(11,0.628661916072842)(12,0.673359073359073)(13,0.695024077046549)(14,0.701923076923077)(15,0.74400618716164)(16,0.761453396524487)(17,0.757976653696498)(18,0.779378316906747)(19,0.775478927203065)(20,0.78783151326053)(21,0.797488226059655)(22,0.791139240506329)(23,0.79140127388535)(24,0.7904)(25,0.806962025316456)(26,0.803174603174603)(27,0.803812549642573)(28,0.810126582278481)(29,0.875358166189112)(30,0.873744619799139)(31,0.875181422351234)(32,0.875542691751085)(33,0.873511904761905)(34,0.873051224944321)(35,0.880466472303207)(36,0.876811594202898)(37,0.87518573551263)(38,0.882005899705015)(39,0.879765395894428)(40,0.880586080586081)(41,0.892550143266475)(42,0.891583452211127)(43,0.899135446685879)(44,0.90028901734104)(45,0.889684813753582)(46,0.892241379310345)(47,0.89650249821556)(48,0.898860398860399)(49,0.892198581560284)(50,0.892573900504686)(51,0.892728581713463)(52,0.883652430044182)(53,0.881057268722467)(54,0.883345561261922)(55,0.885462555066079)(56,0.884303610906411)(57,0.890988372093023)(58,0.894391842680262)(59,0.895043731778425)(60,0.894083272461651)(61,0.898761835396941)(62,0.894508670520231)(63,0.89193083573487)(64,0.89193083573487)(65,0.891618497109826)(66,0.894015861571737)(67,0.890804597701149)(68,0.886201991465149)(69,0.890636168691923)(70,0.885593220338983)(71,0.886219081272085)(72,0.888415067519545)(73,0.896451846488052)(74,0.899055918663762)(75,0.897810218978102)(76,0.898466033601169)(77,0.897959183673469)(78,0.897155361050328)(79,0.896301667875272)(80,0.894049346879535)(81,0.892285298398836)(82,0.891352549889135)(83,0.895988112927191)(84,0.895988112927191)(85,0.896960711638251)(86,0.89792899408284)(87,0.901931649331352)(88,0.902548725637181)(89,0.901725431357839)(90,0.904334828101644)(91,0.902457185405808)(92,0.904089219330855)(93,0.902457185405808)(94,0.903273809523809)(95,0.906110283159463)(96,0.902840059790732)(97,0.899478778853313)(98,0.899777282850779)(99,0.899777282850779)(100,0.90067214339059)(101,0.901492537313433)(102,0.90096798212956)(103,0.903703703703704)(104,0.904515173945226)(105,0.904515173945226)(106,0.905325443786982)(107,0.905604719764012)(108,0.905604719764012)(109,0.907079646017699)(110,0.908689248895434)(111,0.910294117647059)(112,0.908554572271386)(113,0.911917098445596)(114,0.912979351032448)(115,0.911634756995582)(116,0.886503067484662)(117,0.889058913542463)(118,0.886153846153846)(119,0.892448512585812)(120,0.892448512585812)(121,0.894977168949772)(122,0.895817490494296)(123,0.895817490494296)(124,0.895499618611747)(125,0.894656488549618)(126,0.893812070282658)(127,0.891271056661562)(128,0.891271056661562)(129,0.891271056661562)(130,0.890421455938697)(131,0.892119357306809)(132,0.896341463414634)(133,0.895499618611747)(134,0.898021308980213)(135,0.898859315589353)(136,0.898859315589353)(137,0.897182025894897)(138,0.898859315589353)(139,0.90249433106576)(140,0.901365705614567)(141,0.903177004538578)(142,0.904295403165034)(143,0.903469079939668)(144,0.903323262839879)(145,0.90249433106576)(146,0.903323262839879)(147,0.90400604686319)(148,0.90400604686319)(149,0.90400604686319)(150,0.901515151515151)(151,0.90400604686319)(152,0.90400604686319)(153,0.90400604686319)(154,0.90400604686319)(155,0.904689863842663)(156,0.905263157894737)(157,0.907046476761619)(158,0.907726931732933)(159,0.907726931732933)(160,0.907726931732933)(161,0.907046476761619)(162,0.90622655663916)(163,0.90622655663916)(164,0.907046476761619)(165,0.907046476761619)(166,0.907185628742515)(167,0.907185628742515)(168,0.907865168539326)(169,0.907865168539326)(170,0.907865168539326)(171,0.907865168539326)(172,0.910313901345291)(173,0.910313901345291)(174,0.911260253542133)(175,0.912071535022355)(176,0.910846953937593)(177,0.927684441197955)(178,0.929248723559446)(179,0.928467153284671)(180,0.928467153284671)(181,0.929248723559446)(182,0.929248723559446)(183,0.929248723559446)(184,0.929248723559446)(185,0.928675400291121)(186,0.929351784413692)(187,0.92992700729927)(188,0.92992700729927)(189,0.92992700729927)(190,0.929145361577794)(191,0.927113702623907)(192,0.928467153284671)(193,0.926900584795322)(194,0.927578639356254)(195,0.926007326007326)(196,0.924431401320616)(197,0.926793557833089)(198,0.924320352681852)(199,0.925461254612546)(200,0.922962962962963)(201,0.922164566345441)(202,0.922962962962963)(203,0.922962962962963)(204,0.922164566345441)(205,0.922164566345441)(206,0.922962962962963)(207,0.922962962962963)(208,0.921364985163205)(209,0.922279792746114)(210,0.922962962962963)(211,0.922962962962963)(212,0.922962962962963)(213,0.922962962962963)(214,0.923647146034099)(215,0.922848664688427)(216,0.922962962962963)(217,0.922962962962963)(218,0.922962962962963)(219,0.921130952380952)(220,0.921013412816691)(221,0.921013412816691)(222,0.921013412816691)(223,0.921013412816691)(224,0.916167664670659)(225,0.913728432108027)(226,0.920089619118745)(227,0.912781954887218)(228,0.912781954887218)(229,0.912781954887218)(230,0.912781954887218)(231,0.913598797896318)(232,0.912781954887218)(233,0.91196388261851)(234,0.911144578313253)(235,0.910324039186134)(236,0.911144578313253)(237,0.910324039186134)(238,0.91196388261851)(239,0.910324039186134)(240,0.91196388261851)(241,0.913338357196684)(242,0.912518853695324)(243,0.913207547169811)(244,0.913207547169811)(245,0.91238670694864)(246,0.913207547169811)(247,0.913207547169811)(248,0.914027149321267)(249,0.914027149321267)(250,0.921465968586387)(251,0.921465968586387)(252,0.921465968586387)(253,0.921465968586387)(254,0.922272047832586)(255,0.923306031273269)(256,0.931466470154753)(257,0.932153392330383)(258,0.932452276064611)(259,0.933235509904622)(260,0.926720947446336)(261,0.926720947446336)(262,0.925816023738872)(263,0.926612305411416)(264,0.925816023738872)(265,0.925018559762435)(266,0.925018559762435)(267,0.924907063197026)(268,0.924107142857143)(269,0.925816023738872)(270,0.925925925925926)(271,0.925925925925926)(272,0.926829268292683)(273,0.926829268292683)(274,0.928413284132841)(275,0.929203539823009)(276,0.929203539823009)(277,0.927621861152142)(278,0.928413284132841)(279,0.928413284132841)(280,0.928413284132841)(281,0.929577464788732)(282,0.928518791451732)(283,0.927941176470588)(284,0.93040293040293)(285,0.93108504398827)(286,0.930300807043287)(287,0.93108504398827)(288,0.928833455612619)(289,0.928833455612619)(290,0.927835051546392)(291,0.921230307576894)(292,0.927083333333333)(293,0.927083333333333)(294,0.927083333333333)(295,0.927083333333333)(296,0.930983847283407)(297,0.93411420204978)(298,0.934893928310168)(299,0.931768158473954)(300,0.931667891256429)(301,0.931667891256429)(302,0.93235294117647)(303,0.931567328918322)(304,0.932253313696613)(305,0.932253313696613)(306,0.93460690668626)(307,0.937042459736457)(308,0.935389133627019)(309,0.931365313653136)(310,0.930678466076696)(311,0.931466470154753)(312,0.93470286133529)(313,0.93470286133529)(314,0.93470286133529)(315,0.93470286133529)(316,0.93470286133529)(317,0.935389133627019)(318,0.936076414401176)(319,0.935294117647059)(320,0.93460690668626)(321,0.93460690668626)(322,0.93460690668626)(323,0.93460690668626)(324,0.93460690668626)(325,0.93460690668626)(326,0.935389133627019)(327,0.935389133627019)(328,0.935389133627019)(329,0.936356986100951)(330,0.936950146627566)(331,0.936950146627566)(332,0.936356986100951)(333,0.936356986100951)(334,0.936356986100951)(335,0.933920704845815)(336,0.93460690668626)(337,0.93470286133529)(338,0.935483870967742)(339,0.936857562408223)(340,0.936076414401176)(341,0.936635105608157)(342,0.936635105608157)(343,0.935953420669578)(344,0.935953420669578)(345,0.935953420669578)(346,0.935272727272727)(347,0.93681917211329)(348,0.93681917211329)(349,0.936231884057971)(350,0.937590711175617)(351,0.93681917211329)(352,0.93681917211329)(353,0.935178441369264)(354,0.934402332361516)(355,0.934402332361516)(356,0.934989043097151)(357,0.934989043097151)(358,0.935178441369264)(359,0.935672514619883)(360,0.937042459736457)(361,0.937042459736457)(362,0.93841642228739)(363,0.935103244837758)(364,0.935103244837758)(365,0.936578171091445)(366,0.935103244837758)(367,0.935887988209285)(368,0.935007385524372)(369,0.932542624166049)(370,0.933333333333333)(371,0.934025203854707)(372,0.934025203854707)(373,0.934025203854707)(374,0.933234421364985)(375,0.934122871946706)(376,0.934122871946706)(377,0.934122871946706)(378,0.934122871946706)(379,0.934122871946706)(380,0.93491124260355)(381,0.934220251293422)(382,0.935793357933579)(383,0.934220251293422)(384,0.934220251293422)(385,0.933431952662722)(386,0.933431952662722)(387,0.93644996347699)(388,0.935860058309038)(389,0.935860058309038)(390,0.935178441369264)(391,0.935178441369264)(392,0.935178441369264)(393,0.936542669584245)(394,0.936542669584245)(395,0.933236574746009)(396,0.936356986100951)(397,0.935389133627019)(398,0.936170212765957)(399,0.936578171091445)(400,0.936076414401176)(9,0.66200215285253)(10,0.670103092783505)(11,0.670684931506849)(12,0.721778791334093)(13,0.74910394265233)(14,0.745486313337216)(15,0.745920745920746)(16,0.769895126465145)(17,0.772362739049969)(18,0.75745784695201)(19,0.764954682779456)(20,0.753846153846154)(21,0.767741935483871)(22,0.771374287523749)(23,0.772087067861716)(24,0.784695201037613)(25,0.784919653893696)(26,0.788387096774193)(27,0.790697674418605)(28,0.787330316742081)(29,0.8)(30,0.804583068109484)(31,0.7997432605905)(32,0.808900523560209)(33,0.812209688122097)(34,0.812869336835194)(35,0.809492419248517)(36,0.808929743926461)(37,0.815598149372108)(38,0.823293172690763)(39,0.83695652173913)(40,0.833333333333333)(41,0.835030549898167)(42,0.831081081081081)(43,0.831081081081081)(44,0.833333333333333)(45,0.841160220994475)(46,0.838487972508591)(47,0.853658536585366)(48,0.855371900826446)(49,0.869747899159664)(50,0.864902506963788)(51,0.866713189113747)(52,0.871398453970485)(53,0.871508379888268)(54,0.873867595818815)(55,0.873867595818815)(56,0.882434301521438)(57,0.888888888888889)(58,0.888888888888889)(59,0.887678692988427)(60,0.884224779959377)(61,0.883029073698445)(62,0.884381338742393)(63,0.883783783783784)(64,0.883029073698445)(65,0.88318703578663)(66,0.882907133243607)(67,0.885290148448043)(68,0.885003362474781)(69,0.897068847989093)(70,0.897068847989093)(71,0.892203389830508)(72,0.897068847989093)(73,0.898907103825137)(74,0.898293515358362)(75,0.897540983606557)(76,0.895095367847411)(77,0.903092783505155)(78,0.903092783505155)(79,0.908085694540428)(80,0.908713692946058)(81,0.90495867768595)(82,0.906832298136646)(83,0.906077348066298)(84,0.902959394356504)(85,0.903581267217631)(86,0.90420399724328)(87,0.903581267217631)(88,0.90420399724328)(89,0.902959394356504)(90,0.902959394356504)(91,0.901718213058419)(92,0.905479452054795)(93,0.904729266620973)(94,0.906593406593406)(95,0.908465244322092)(96,0.908591065292096)(97,0.911461908030199)(98,0.911724137931034)(99,0.911724137931034)(100,0.911724137931034)(101,0.913852515506547)(102,0.915113871635611)(103,0.914996544574983)(104,0.918282548476454)(105,0.917874396135266)(106,0.917874396135266)(107,0.917241379310345)(108,0.915347556779078)(109,0.910344827586207)(110,0.906593406593406)(111,0.908465244322092)(112,0.907840440165062)(113,0.909965635738831)(114,0.908465244322092)(115,0.907216494845361)(116,0.906593406593406)(117,0.904598490048044)(118,0.90335846470185)(119,0.905971173644475)(120,0.905971173644475)(121,0.906593406593406)(122,0.897765741367637)(123,0.898373983739837)(124,0.894701542588866)(125,0.891260840560373)(126,0.885941644562334)(127,0.885941644562334)(128,0.886529528865295)(129,0.886529528865295)(130,0.890501319261213)(131,0.890052356020942)(132,0.891218872870249)(133,0.892227302416721)(134,0.897135416666667)(135,0.90815671162492)(136,0.90920716112532)(137,0.907819453273999)(138,0.898678414096916)(139,0.898678414096916)(140,0.902208201892744)(141,0.906309751434034)(142,0.906309751434034)(143,0.906887755102041)(144,0.90978886756238)(145,0.90978886756238)(146,0.90978886756238)(147,0.90920716112532)(148,0.907692307692308)(149,0.906530089628681)(150,0.907584448693435)(151,0.903553299492386)(152,0.907124681933842)(153,0.908859145952836)(154,0.91234804862444)(155,0.916934964584675)(156,0.918117343649258)(157,0.917525773195876)(158,0.918117343649258)(159,0.918117343649258)(160,0.915755627009646)(161,0.917525773195876)(162,0.91661279896574)(163,0.91661279896574)(164,0.919406834300451)(165,0.918814432989691)(166,0.918117343649258)(167,0.918709677419355)(168,0.919302775984506)(169,0.919406834300451)(170,0.922380336351876)(171,0.922977346278317)(172,0.923575129533679)(173,0.927178153446034)(174,0.928385416666667)(175,0.926575698505523)(176,0.932026143790849)(177,0.933246073298429)(178,0.934469200524246)(179,0.933770491803279)(180,0.933770491803279)(181,0.934383202099737)(182,0.935947712418301)(183,0.935947712418301)(184,0.934116112198304)(185,0.934640522875817)(186,0.93403004572175)(187,0.932811480756686)(188,0.933420365535248)(189,0.933420365535248)(190,0.933507170795306)(191,0.933507170795306)(192,0.935336381450033)(193,0.934725848563969)(194,0.934116112198304)(195,0.934116112198304)(196,0.934116112198304)(197,0.934116112198304)(198,0.935336381450033)(199,0.935336381450033)(200,0.932291666666667)(201,0.933507170795306)(202,0.934725848563969)(203,0.934725848563969)(204,0.934725848563969)(205,0.934725848563969)(206,0.935947712418301)(207,0.935947712418301)(208,0.907020872865275)(209,0.919820397690827)(210,0.920410783055199)(211,0.919230769230769)(212,0.919820397690827)(213,0.918053777208707)(214,0.915708812260536)(215,0.91629392971246)(216,0.91395793499044)(217,0.90188679245283)(218,0.90188679245283)(219,0.90530303030303)(220,0.90530303030303)(221,0.907020872865275)(222,0.906447534766119)(223,0.905874921036008)(224,0.906447534766119)(225,0.904731861198738)(226,0.90530303030303)(227,0.902454373820012)(228,0.902454373820012)(229,0.90359168241966)(230,0.90188679245283)(231,0.902454373820012)(232,0.900753768844221)(233,0.900753768844221)(234,0.900753768844221)(235,0.897933625547902)(236,0.897933625547902)(237,0.897371714643304)(238,0.899059561128527)(239,0.896810506566604)(240,0.894572676232065)(241,0.893457943925234)(242,0.891236793039155)(243,0.893457943925234)(244,0.892345986309894)(245,0.894014962593516)(246,0.897933625547902)(247,0.892901618929016)(248,0.893457943925234)(249,0.891236793039155)(250,0.897933625547902)(251,0.897933625547902)(252,0.900188323917137)(253,0.900188323917137)(254,0.898496240601504)(255,0.898496240601504)(256,0.89625)(257,0.898496240601504)(258,0.901319924575739)(259,0.90188679245283)(260,0.90188679245283)(261,0.90188679245283)(262,0.90188679245283)(263,0.90188679245283)(264,0.901319924575739)(265,0.90188679245283)(266,0.90188679245283)(267,0.903022670025189)(268,0.90359168241966)(269,0.903022670025189)(270,0.90359168241966)(271,0.900753768844221)(272,0.900753768844221)(273,0.906447534766119)(274,0.91047619047619)(275,0.912213740458015)(276,0.912213740458015)(277,0.92337411461687)(278,0.924564796905222)(279,0.92159383033419)(280,0.92159383033419)(281,0.92159383033419)(282,0.924564796905222)(283,0.925161290322581)(284,0.923969072164948)(285,0.925758553905745)(286,0.923969072164948)(287,0.924564796905222)(288,0.923969072164948)(289,0.924564796905222)(290,0.92337411461687)(291,0.92337411461687)(292,0.923969072164948)(293,0.923969072164948)(294,0.923969072164948)(295,0.923969072164948)(296,0.923969072164948)(297,0.922779922779923)(298,0.921001926782273)(299,0.921001926782273)(300,0.919820397690827)(301,0.920410783055199)(302,0.921001926782273)(303,0.922186495176849)(304,0.922186495176849)(305,0.925662572721396)(306,0.928155339805825)(307,0.928756476683938)(308,0.927554980595084)(309,0.925758553905745)(310,0.927554980595084)(311,0.927554980595084)(312,0.927554980595084)(313,0.926955397543633)(314,0.928155339805825)(315,0.928756476683938)(316,0.929358392741413)(317,0.929358392741413)(318,0.928756476683938)(319,0.928155339805825)(320,0.928155339805825)(321,0.929358392741413)(322,0.928756476683938)(323,0.928756476683938)(324,0.929358392741413)(325,0.925758553905745)(326,0.925758553905745)(327,0.925758553905745)(328,0.924564796905222)(329,0.925758553905745)(330,0.925758553905745)(331,0.923969072164948)(332,0.925161290322581)(333,0.924564796905222)(334,0.925161290322581)(335,0.925161290322581)(336,0.924564796905222)(337,0.926356589147287)(338,0.925758553905745)(339,0.925758553905745)(340,0.925758553905745)(341,0.926955397543633)(342,0.928155339805825)(343,0.91746641074856)(344,0.918053777208707)(345,0.918053777208707)(346,0.915708812260536)(347,0.916879795396419)(348,0.895690193628982)(349,0.90530303030303)(350,0.893457943925234)(351,0.900753768844221)(352,0.900753768844221)(353,0.902454373820012)(354,0.902454373820012)(355,0.903022670025189)(356,0.907594936708861)(357,0.907594936708861)(358,0.907594936708861)(359,0.906447534766119)(360,0.908745247148289)(361,0.909321496512365)(362,0.909321496512365)(363,0.91163382072473)(364,0.903022670025189)(365,0.903022670025189)(366,0.902454373820012)(367,0.893457943925234)(368,0.89625)(369,0.898496240601504)(370,0.898496240601504)(371,0.899059561128527)(372,0.899623588456713)(373,0.899623588456713)(374,0.897933625547902)(375,0.896810506566604)(376,0.897371714643304)(377,0.897933625547902)(378,0.899059561128527)(379,0.891791044776119)(380,0.891236793039155)(381,0.891236793039155)(382,0.891236793039155)(383,0.895131086142322)(384,0.896810506566604)(385,0.897933625547902)(386,0.893457943925234)(387,0.894014962593516)(388,0.894572676232065)(389,0.89625)(390,0.89625)(391,0.89625)(392,0.895690193628982)(393,0.895690193628982)(394,0.895690193628982)(395,0.895131086142322)(396,0.895131086142322)(397,0.895131086142322)(398,0.908745247148289)(399,0.908745247148289)(400,0.909321496512365)(6,0.6590421729807)(7,0.675039246467818)(8,0.692675159235669)(9,0.694267515923567)(10,0.713487629688747)(11,0.788148148148148)(12,0.790909090909091)(13,0.792395437262357)(14,0.789992418498863)(15,0.786039453717754)(16,0.788199697428139)(17,0.787878787878788)(18,0.791824375473126)(19,0.802469135802469)(20,0.802155504234026)(21,0.801232665639445)(22,0.800929512006197)(23,0.801538461538461)(24,0.797536566589684)(25,0.796923076923077)(26,0.801850424055513)(27,0.809634809634809)(28,0.810264385692068)(29,0.812693498452012)(30,0.812693498452012)(31,0.815850815850816)(32,0.813899613899614)(33,0.817008352315869)(34,0.819018404907975)(35,0.820906994619523)(36,0.821538461538461)(37,0.822892498066512)(38,0.822892498066512)(39,0.82035466461064)(40,0.820987654320988)(41,0.821538461538461)(42,0.824345146379045)(43,0.824884792626728)(44,0.829754601226994)(45,0.830391404451266)(46,0.834232845026985)(47,0.835130970724191)(48,0.836027713625866)(49,0.835913312693498)(50,0.835266821345708)(51,0.835266821345708)(52,0.834621329211746)(53,0.834621329211746)(54,0.830482115085536)(55,0.830482115085536)(56,0.831652443754848)(57,0.834241245136187)(58,0.833592534992224)(59,0.831652443754848)(60,0.830745341614907)(61,0.831652443754848)(62,0.831390831390831)(63,0.832944832944833)(64,0.833592534992224)(65,0.830865159781761)(66,0.831775700934579)(67,0.830218068535825)(68,0.830218068535825)(69,0.830865159781761)(70,0.829953198127925)(71,0.830218068535825)(72,0.831775700934579)(73,0.830482115085536)(74,0.829192546583851)(75,0.830364058869094)(76,0.829457364341085)(77,0.830745341614907)(78,0.830745341614907)(79,0.83203732503888)(80,0.829306313328137)(81,0.829953198127925)(82,0.830601092896175)(83,0.829306313328137)(84,0.828660436137072)(85,0.828660436137072)(86,0.828015564202335)(87,0.828660436137072)(88,0.829192546583851)(89,0.828549262994569)(90,0.826625386996904)(91,0.826625386996904)(92,0.82753286929621)(93,0.827265685515104)(94,0.824074074074074)(95,0.824710424710425)(96,0.825347758887171)(97,0.82832948421863)(98,0.826585179526356)(99,0.826053639846743)(100,0.824789594491201)(101,0.824961948249619)(102,0.82442748091603)(103,0.825590251332825)(104,0.825590251332825)(105,0.824067022086824)(106,0.823082763857251)(107,0.823082763857251)(108,0.823082763857251)(109,0.824334600760456)(110,0.824334600760456)(111,0.826319816373374)(112,0.826952526799387)(113,0.827321565617805)(114,0.827321565617805)(115,0.83179012345679)(116,0.83179012345679)(117,0.832690824980725)(118,0.832049306625578)(119,0.831408775981524)(120,0.831408775981524)(121,0.832948421862971)(122,0.832690824980725)(123,0.833590138674884)(124,0.8328173374613)(125,0.8328173374613)(126,0.8328173374613)(127,0.834375)(128,0.837354085603113)(129,0.837354085603113)(130,0.837354085603113)(131,0.83579766536965)(132,0.835147744945568)(133,0.835147744945568)(134,0.836052836052836)(135,0.836560805577072)(136,0.836052836052836)(137,0.836052836052836)(138,0.835658914728682)(139,0.835266821345708)(140,0.834621329211746)(141,0.835011618900077)(142,0.835403726708074)(143,0.836702954898911)(144,0.836052836052836)(145,0.837245696400626)(146,0.838154808444097)(147,0.838557993730407)(148,0.839467501957713)(149,0.84037558685446)(150,0.84037558685446)(151,0.836052836052836)(152,0.837712519319938)(153,0.835774865073246)(154,0.834488067744419)(155,0.835774865073246)(156,0.835130970724191)(157,0.834488067744419)(158,0.833205226748655)(159,0.833205226748655)(160,0.833846153846154)(161,0.833846153846154)(162,0.835130970724191)(163,0.835774865073246)(164,0.835130970724191)(165,0.835130970724191)(166,0.835130970724191)(167,0.835130970724191)(168,0.831926323867997)(169,0.830651340996168)(170,0.82938026013772)(171,0.831926323867997)(172,0.837065637065637)(173,0.837065637065637)(174,0.834488067744419)(175,0.834488067744419)(176,0.837065637065637)(177,0.837065637065637)(178,0.828113063407181)(179,0.828113063407181)(180,0.82874617737003)(181,0.830015313935681)(182,0.829268292682927)(183,0.829528158295281)(184,0.828267477203647)(185,0.828267477203647)(186,0.828267477203647)(187,0.828267477203647)(188,0.828267477203647)(189,0.828267477203647)(190,0.82701062215478)(191,0.826383623957544)(192,0.827638572513288)(193,0.828897338403042)(194,0.83015993907083)(195,0.83015993907083)(196,0.828267477203647)(197,0.82701062215478)(198,0.827638572513288)(199,0.825132475397426)(200,0.825132475397426)(201,0.82701062215478)(202,0.82701062215478)(203,0.826383623957544)(204,0.82701062215478)(205,0.826383623957544)(206,0.82701062215478)(207,0.826383623957544)(208,0.826383623957544)(209,0.826383623957544)(210,0.823885109599395)(211,0.823885109599395)(212,0.823885109599395)(213,0.825757575757576)(214,0.823885109599395)(215,0.828267477203647)(216,0.82701062215478)(217,0.825132475397426)(218,0.825132475397426)(219,0.825132475397426)(220,0.825132475397426)(221,0.825757575757576)(222,0.825132475397426)(223,0.825132475397426)(224,0.825132475397426)(225,0.823262839879154)(226,0.823262839879154)(227,0.82701062215478)(228,0.826383623957544)(229,0.827638572513288)(230,0.827638572513288)(231,0.828267477203647)(232,0.828267477203647)(233,0.829528158295281)(234,0.828267477203647)(235,0.828267477203647)(236,0.828267477203647)(237,0.828267477203647)(238,0.827638572513288)(239,0.827638572513288)(240,0.829528158295281)(241,0.829528158295281)(242,0.828267477203647)(243,0.827638572513288)(244,0.82701062215478)(245,0.827638572513288)(246,0.827638572513288)(247,0.828267477203647)(248,0.828267477203647)(249,0.827638572513288)(250,0.825757575757576)(251,0.83015993907083)(252,0.830792682926829)(253,0.83015993907083)(254,0.831426392067124)(255,0.833970925784239)(256,0.831426392067124)(257,0.831426392067124)(258,0.831426392067124)(259,0.833970925784239)(260,0.837173579109063)(261,0.841698841698842)(262,0.841049382716049)(263,0.841049382716049)(264,0.835249042145594)(265,0.837173579109063)(266,0.837173579109063)(267,0.83910700538876)(268,0.83910700538876)(269,0.83910700538876)(270,0.837817063797079)(271,0.83910700538876)(272,0.839753466872111)(273,0.835249042145594)(274,0.835889570552147)(275,0.835249042145594)(276,0.833333333333333)(277,0.837817063797079)(278,0.837817063797079)(279,0.83910700538876)(280,0.839753466872111)(281,0.83910700538876)(282,0.843410852713178)(283,0.844065166795966)(284,0.841698841698842)(285,0.841698841698842)(286,0.844306738962045)(287,0.843410852713178)(288,0.84472049689441)(289,0.84472049689441)(290,0.843822843822844)(291,0.845616757176105)(292,0.850196078431372)(293,0.849529780564263)(294,0.849529780564263)(295,0.849250197316495)(296,0.849250197316495)(297,0.849250197316495)(298,0.849250197316495)(299,0.849250197316495)(300,0.847430830039526)(301,0.848341232227488)(302,0.848151062155783)(303,0.849056603773585)(304,0.849487785657998)(305,0.848389630793401)(306,0.8484375)(307,0.849100860046912)(308,0.849100860046912)(309,0.849100860046912)(310,0.849100860046912)(311,0.849100860046912)(312,0.849100860046912)(313,0.848864526233359)(314,0.849100860046912)(315,0.849100860046912)(316,0.849100860046912)(317,0.8484375)(318,0.849100860046912)(319,0.849529780564263)(320,0.849529780564263)(321,0.849100860046912)(322,0.847113884555382)(323,0.847113884555382)(324,0.847775175644028)(325,0.847775175644028)(326,0.847775175644028)(327,0.849100860046912)(328,0.845794392523364)(329,0.847775175644028)(330,0.849100860046912)(331,0.849100860046912)(332,0.849100860046912)(333,0.846453624318005)(334,0.846453624318005)(335,0.846453624318005)(336,0.846453624318005)(337,0.846453624318005)(338,0.847113884555382)(339,0.847113884555382)(340,0.849100860046912)(341,0.849100860046912)(342,0.851097178683385)(343,0.847775175644028)(344,0.847352024922118)(345,0.846034214618973)(346,0.846034214618973)(347,0.846692607003891)(348,0.846692607003891)(349,0.846034214618973)(350,0.845376845376845)(351,0.844065166795966)(352,0.845376845376845)(353,0.843410852713178)(354,0.84275755228505)(355,0.841453982985305)(356,0.84275755228505)(357,0.84275755228505)(358,0.84275755228505)(359,0.844065166795966)(360,0.841453982985305)(361,0.842349304482226)(362,0.842349304482226)(363,0.841698841698842)(364,0.842349304482226)(365,0.842349304482226)(366,0.841698841698842)(367,0.841698841698842)(368,0.841698841698842)(369,0.842349304482226)(370,0.842349304482226)(371,0.842349304482226)(372,0.842105263157895)(373,0.841453982985305)(374,0.842105263157895)(375,0.842105263157895)(376,0.842105263157895)(377,0.842105263157895)(378,0.842105263157895)(379,0.843000773395205)(380,0.842105263157895)(381,0.843000773395205)(382,0.843000773395205)(383,0.843000773395205)(384,0.843000773395205)(385,0.843000773395205)(386,0.843000773395205)(387,0.842349304482226)(388,0.842349304482226)(389,0.842349304482226)(390,0.842349304482226)(391,0.842349304482226)(392,0.842349304482226)(393,0.842349304482226)(394,0.842349304482226)(395,0.842349304482226)(396,0.843000773395205)(397,0.843653250773994)(398,0.843000773395205)(399,0.843000773395205)(400,0.842349304482226)(12,0.696048632218845)(13,0.718446601941747)(14,0.707333790267306)(15,0.703886925795053)(16,0.732075471698113)(17,0.755842062852538)(18,0.750597609561753)(19,0.785197103781174)(20,0.796178343949045)(21,0.795786061588331)(22,0.795786061588331)(23,0.800324675324675)(24,0.80161943319838)(25,0.80064829821718)(26,0.793159609120521)(27,0.8)(28,0.799352750809061)(29,0.801955990220049)(30,0.809369951534733)(31,0.813859790491539)(32,0.8176)(33,0.8144)(34,0.814696485623003)(35,0.815165876777251)(36,0.816455696202532)(37,0.815873015873016)(38,0.818834796488428)(39,0.818834796488428)(40,0.81897233201581)(41,0.822974036191975)(42,0.823529411764706)(43,0.822593476531424)(44,0.822593476531424)(45,0.817891373801917)(46,0.823248407643312)(47,0.82615629984051)(48,0.825498007968127)(49,0.822115384615385)(50,0.818986323411102)(51,0.818327974276527)(52,0.819855884707766)(53,0.815409309791332)(54,0.815409309791332)(55,0.815896188158962)(56,0.816855753646677)(57,0.817813765182186)(58,0.818108326596605)(59,0.818108326596605)(60,0.817813765182186)(61,0.816855753646677)(62,0.818770226537217)(63,0.818108326596605)(64,0.820388349514563)(65,0.817518248175182)(66,0.820967741935484)(67,0.820967741935484)(68,0.82030620467365)(69,0.82030620467365)(70,0.82598235765838)(71,0.824659727782226)(72,0.826923076923077)(73,0.825878594249201)(74,0.825219473264166)(75,0.826538768984812)(76,0.825320512820513)(77,0.828137490007994)(78,0.82615629984051)(79,0.825498007968127)(80,0.825119236883943)(81,0.827258320126783)(82,0.827258320126783)(83,0.827258320126783)(84,0.826984126984127)(85,0.82753164556962)(86,0.82818685669042)(87,0.82753164556962)(88,0.82753164556962)(89,0.828458498023715)(90,0.82780410742496)(91,0.828458498023715)(92,0.830307813733228)(93,0.832278481012658)(94,0.832278481012658)(95,0.832278481012658)(96,0.832937450514648)(97,0.832937450514648)(98,0.832543443917851)(99,0.832543443917851)(100,0.833201581027668)(101,0.833201581027668)(102,0.834258524980174)(103,0.833333333333333)(104,0.833333333333333)(105,0.833995234312947)(106,0.833995234312947)(107,0.834920634920635)(108,0.833995234312947)(109,0.834258524980174)(110,0.834920634920635)(111,0.833068362480127)(112,0.832140015910899)(113,0.834394904458599)(114,0.833068362480127)(115,0.833995234312947)(116,0.833333333333333)(117,0.833333333333333)(118,0.833333333333333)(119,0.833333333333333)(120,0.833333333333333)(121,0.83147853736089)(122,0.83147853736089)(123,0.83147853736089)(124,0.83147853736089)(125,0.832140015910899)(126,0.832140015910899)(127,0.832406671961874)(128,0.830427892234548)(129,0.831746031746032)(130,0.831746031746032)(131,0.829383886255924)(132,0.831086439333862)(133,0.831086439333862)(134,0.829500396510706)(135,0.83015873015873)(136,0.831086439333862)(137,0.831086439333862)(138,0.831086439333862)(139,0.831086439333862)(140,0.831086439333862)(141,0.831086439333862)(142,0.831086439333862)(143,0.831086439333862)(144,0.829770387965162)(145,0.830427892234548)(146,0.830427892234548)(147,0.830427892234548)(148,0.83147853736089)(149,0.83147853736089)(150,0.833466135458167)(151,0.832535885167464)(152,0.83054892601432)(153,0.831210191082802)(154,0.831210191082802)(155,0.829500396510706)(156,0.831353919239905)(157,0.834123222748815)(158,0.834123222748815)(159,0.831353919239905)(160,0.831353919239905)(161,0.831353919239905)(162,0.831353919239905)(163,0.830963665086888)(164,0.830963665086888)(165,0.830963665086888)(166,0.830696202531646)(167,0.830696202531646)(168,0.830039525691699)(169,0.830039525691699)(170,0.830575256107171)(171,0.830963665086888)(172,0.834905660377358)(173,0.834509803921568)(174,0.833070866141732)(175,0.833070866141732)(176,0.834905660377358)(177,0.834905660377358)(178,0.8318863456985)(179,0.834645669291339)(180,0.834645669291339)(181,0.834645669291339)(182,0.834123222748815)(183,0.836220472440945)(184,0.836078431372549)(185,0.836335160532498)(186,0.835423197492163)(187,0.837245696400626)(188,0.836591086786552)(189,0.837245696400626)(190,0.838154808444097)(191,0.8375)(192,0.839313572542902)(193,0.839313572542902)(194,0.839313572542902)(195,0.839313572542902)(196,0.838659392049883)(197,0.838006230529595)(198,0.839313572542902)(199,0.839313572542902)(200,0.836702954898911)(201,0.836702954898911)(202,0.836702954898911)(203,0.836052836052836)(204,0.836052836052836)(205,0.836052836052836)(206,0.836052836052836)(207,0.835521235521235)(208,0.835011618900077)(209,0.834755624515128)(210,0.836052836052836)(211,0.836052836052836)(212,0.836052836052836)(213,0.836052836052836)(214,0.836052836052836)(215,0.835403726708074)(216,0.836052836052836)(217,0.836052836052836)(218,0.836052836052836)(219,0.84037558685446)(220,0.84037558685446)(221,0.841033672670321)(222,0.841692789968652)(223,0.838407494145199)(224,0.836560805577072)(225,0.834621329211746)(226,0.834621329211746)(227,0.834876543209876)(228,0.834876543209876)(229,0.834876543209876)(230,0.834876543209876)(231,0.835521235521235)(232,0.835011618900077)(233,0.83695652173913)(234,0.83695652173913)(235,0.838910505836576)(236,0.840873634945398)(237,0.840873634945398)(238,0.839718530101642)(239,0.839718530101642)(240,0.838910505836576)(241,0.838910505836576)(242,0.841033672670321)(243,0.841692789968652)(244,0.841692789968652)(245,0.841033672670321)(246,0.841033672670321)(247,0.839313572542902)(248,0.839313572542902)(249,0.840625)(250,0.840625)(251,0.841121495327103)(252,0.842023346303502)(253,0.841368584758942)(254,0.840466926070039)(255,0.840466926070039)(256,0.840466926070039)(257,0.841121495327103)(258,0.841530054644809)(259,0.840625)(260,0.8421875)(261,0.841940532081377)(262,0.843091334894613)(263,0.843091334894613)(264,0.843091334894613)(265,0.842599843382929)(266,0.842599843382929)(267,0.845247446975648)(268,0.843676355066771)(269,0.843676355066771)(270,0.837858805275407)(271,0.837858805275407)(272,0.838109992254067)(273,0.838109992254067)(274,0.838109992254067)(275,0.840062111801242)(276,0.840062111801242)(277,0.843091334894613)(278,0.840961986035687)(279,0.839659178931061)(280,0.840961986035687)(281,0.840062111801242)(282,0.839410395655547)(283,0.839410395655547)(284,0.838759689922481)(285,0.840714840714841)(286,0.840714840714841)(287,0.841368584758942)(288,0.841368584758942)(289,0.837316885119506)(290,0.837712519319938)(291,0.837712519319938)(292,0.837712519319938)(293,0.837316885119506)(294,0.837712519319938)(295,0.837712519319938)(296,0.839659178931061)(297,0.838610038610038)(298,0.839258114374034)(299,0.834742505764796)(300,0.833716475095785)(301,0.833461243284727)(302,0.834742505764796)(303,0.835384615384615)(304,0.835384615384615)(305,0.835384615384615)(306,0.835384615384615)(307,0.835384615384615)(308,0.832183908045977)(309,0.83282208588957)(310,0.829640947288006)(311,0.830910482019893)(312,0.833461243284727)(313,0.829007633587786)(314,0.829268292682927)(315,0.826747720364742)(316,0.826119969627942)(317,0.826747720364742)(318,0.826747720364742)(319,0.826119969627942)(320,0.826119969627942)(321,0.826119969627942)(322,0.825493171471927)(323,0.822995461422088)(324,0.822995461422088)(325,0.822995461422088)(326,0.824242424242424)(327,0.824242424242424)(328,0.824242424242424)(329,0.824242424242424)(330,0.824242424242424)(331,0.824242424242424)(332,0.824242424242424)(333,0.824242424242424)(334,0.824242424242424)(335,0.824242424242424)(336,0.824867323730099)(337,0.824867323730099)(338,0.825132475397426)(339,0.825757575757576)(340,0.825757575757576)(341,0.825757575757576)(342,0.824867323730099)(343,0.826119969627942)(344,0.826119969627942)(345,0.826119969627942)(346,0.826119969627942)(347,0.826119969627942)(348,0.826119969627942)(349,0.826119969627942)(350,0.829640947288006)(351,0.829640947288006)(352,0.829640947288006)(353,0.829640947288006)(354,0.830275229357798)(355,0.828006088280061)(356,0.828006088280061)(357,0.827376425855513)(358,0.827376425855513)(359,0.827376425855513)(360,0.82837528604119)(361,0.828636709824829)(362,0.828636709824829)(363,0.832183908045977)(364,0.837461300309597)(365,0.838109992254067)(366,0.838360402165506)(367,0.837712519319938)(368,0.841368584758942)(369,0.840714840714841)(370,0.841368584758942)(371,0.841368584758942)(372,0.842023346303502)(373,0.842023346303502)(374,0.841368584758942)(375,0.841614906832298)(376,0.839009287925696)(377,0.840961986035687)(378,0.841368584758942)(379,0.841368584758942)(380,0.83641975308642)(381,0.83641975308642)(382,0.835774865073246)(383,0.835774865073246)(384,0.835774865073246)(385,0.83641975308642)(386,0.840062111801242)(387,0.840062111801242)(388,0.840961986035687)(389,0.837712519319938)(390,0.842433697347894)(391,0.840714840714841)(392,0.840961986035687)(393,0.838109992254067)(394,0.841368584758942)(395,0.84375)(396,0.84375)(397,0.844409695074277)(398,0.842679127725857)(399,0.842679127725857)(400,0.842679127725857)(12,0.559420289855072)(13,0.629503738953093)(14,0.677333333333333)(15,0.706365503080082)(16,0.686830497794581)(17,0.68639798488665)(18,0.714754098360656)(19,0.737982396750169)(20,0.760642009769714)(21,0.768136557610242)(22,0.770655270655271)(23,0.769940672379697)(24,0.772638436482085)(25,0.801624915368991)(26,0.802739726027397)(27,0.820694542877392)(28,0.817663817663818)(29,0.836734693877551)(30,0.799728445349626)(31,0.801906058543227)(32,0.820295983086681)(33,0.841399851079672)(34,0.841715976331361)(35,0.833457804331591)(36,0.84015444015444)(37,0.840400925212028)(38,0.846607669616519)(39,0.847232472324723)(40,0.844444444444444)(41,0.846210448859455)(42,0.851744186046512)(43,0.851744186046512)(44,0.868401486988847)(45,0.868401486988847)(46,0.867952522255193)(47,0.867756315007429)(48,0.862629246676514)(49,0.865185185185185)(50,0.86286137879911)(51,0.864544781643227)(52,0.848714069591528)(53,0.848714069591528)(54,0.844343204252088)(55,0.854166666666667)(56,0.855439642324888)(57,0.859281437125748)(58,0.857571214392803)(59,0.857573474001507)(60,0.856927710843373)(61,0.854531607006854)(62,0.865963855421687)(63,0.866415094339622)(64,0.861491628614916)(65,0.862566438876234)(66,0.859090909090909)(67,0.864661654135338)(68,0.861630516080778)(69,0.861630516080778)(70,0.863602110022607)(71,0.863396226415094)(72,0.864253393665158)(73,0.861468584405753)(74,0.861258529188779)(75,0.861258529188779)(76,0.861258529188779)(77,0.861911987860394)(78,0.863842662632375)(79,0.865312264860797)(80,0.865761689291101)(81,0.867269984917044)(82,0.866616428033157)(83,0.863568215892054)(84,0.868519909842224)(85,0.869172932330827)(86,0.869172932330827)(87,0.870022539444027)(88,0.871064467766117)(89,0.86721680420105)(90,0.867017280240421)(91,0.866817155756207)(92,0.869236583522298)(93,0.846703733121525)(94,0.843949044585987)(95,0.844868735083532)(96,0.845541401273885)(97,0.845541401273885)(98,0.845786963434022)(99,0.846703733121525)(100,0.846703733121525)(101,0.855791962174941)(102,0.853080568720379)(103,0.846703733121525)(104,0.846703733121525)(105,0.844868735083532)(106,0.841181165203511)(107,0.841181165203511)(108,0.842105263157895)(109,0.842105263157895)(110,0.843949044585987)(111,0.843949044585987)(112,0.841434262948207)(113,0.841434262948207)(114,0.838658146964856)(115,0.835868694955965)(116,0.835868694955965)(117,0.839584996009577)(118,0.838658146964856)(119,0.835868694955965)(120,0.834935897435897)(121,0.834935897435897)(122,0.834935897435897)(123,0.838658146964856)(124,0.838658146964856)(125,0.834935897435897)(126,0.836538461538461)(127,0.836538461538461)(128,0.836538461538461)(129,0.836538461538461)(130,0.836276083467095)(131,0.835341365461847)(132,0.835341365461847)(133,0.836276083467095)(134,0.838141025641025)(135,0.843277645186953)(136,0.844197138314785)(137,0.839584996009577)(138,0.839584996009577)(139,0.842356687898089)(140,0.841434262948207)(141,0.841434262948207)(142,0.840510366826156)(143,0.840510366826156)(144,0.845115170770453)(145,0.839328537170264)(146,0.840255591054313)(147,0.84185303514377)(148,0.839328537170264)(149,0.843027888446215)(150,0.849445324881141)(151,0.846703733121525)(152,0.846703733121525)(153,0.846703733121525)(154,0.858490566037736)(155,0.854889589905363)(156,0.856692913385827)(157,0.856692913385827)(158,0.855791962174941)(159,0.853985793212312)(160,0.853080568720379)(161,0.853985793212312)(162,0.853080568720379)(163,0.853080568720379)(164,0.853080568720379)(165,0.855791962174941)(166,0.855791962174941)(167,0.854889589905363)(168,0.856692913385827)(169,0.855791962174941)(170,0.857592446892211)(171,0.857592446892211)(172,0.857592446892211)(173,0.857592446892211)(174,0.857592446892211)(175,0.855791962174941)(176,0.853985793212312)(177,0.871042471042471)(178,0.871042471042471)(179,0.871042471042471)(180,0.873456790123457)(181,0.871913580246913)(182,0.871715610510046)(183,0.874325366229761)(184,0.875192604006163)(185,0.875192604006163)(186,0.872586872586873)(187,0.872586872586873)(188,0.870843000773395)(189,0.871517027863777)(190,0.871517027863777)(191,0.872389791183295)(192,0.872389791183295)(193,0.871517027863777)(194,0.88042650418888)(195,0.882442748091603)(196,0.882442748091603)(197,0.881769641495042)(198,0.88109756097561)(199,0.882442748091603)(200,0.878347360367253)(201,0.879019908116386)(202,0.875766871165644)(203,0.877409406322282)(204,0.876543209876543)(205,0.876543209876543)(206,0.877409406322282)(207,0.876543209876543)(208,0.875483372003093)(209,0.872372372372372)(210,0.873873873873874)(211,0.867112100965107)(212,0.859455481972038)(213,0.859237536656891)(214,0.858608058608059)(215,0.858608058608059)(216,0.857142857142857)(217,0.85986793837124)(218,0.860294117647059)(219,0.860294117647059)(220,0.860294117647059)(221,0.860294117647059)(222,0.861561119293078)(223,0.861561119293078)(224,0.861561119293078)(225,0.86283185840708)(226,0.864904552129222)(227,0.865934065934066)(228,0.866764275256222)(229,0.867399267399267)(230,0.868228404099561)(231,0.869056327724945)(232,0.867399267399267)(233,0.867399267399267)(234,0.867399267399267)(235,0.867399267399267)(236,0.866568914956012)(237,0.87039764359352)(238,0.875555555555555)(239,0.87750556792873)(240,0.87667161961367)(241,0.87750556792873)(242,0.877323420074349)(243,0.877323420074349)(244,0.877323420074349)(245,0.877323420074349)(246,0.880239520958084)(247,0.87815750371471)(248,0.87910447761194)(249,0.882043576258452)(250,0.918056562726613)(251,0.922743682310469)(252,0.923407301360057)(253,0.923734853884533)(254,0.935302390998593)(255,0.942415730337079)(256,0.941754385964912)(257,0.948611111111111)(258,0.949128919860627)(259,0.957241379310345)(260,0.955862068965517)(261,0.953296703296703)(262,0.953888506538197)(263,0.954419889502762)(264,0.955017301038062)(265,0.955678670360111)(266,0.955678670360111)(267,0.955678670360111)(268,0.955141476880607)(269,0.952973720608575)(270,0.953697304768486)(271,0.953038674033149)(272,0.951068228807719)(273,0.949759119064005)(274,0.951135581555402)(275,0.95264241592313)(276,0.953824948311509)(277,0.952446588559614)(278,0.951135581555402)(279,0.9525120440468)(280,0.953038674033149)(281,0.952182952182952)(282,0.952380952380952)(283,0.952380952380952)(284,0.952380952380952)(285,0.954230235783634)(286,0.954230235783634)(287,0.95363321799308)(288,0.953697304768486)(289,0.956641431520991)(290,0.956641431520991)(291,0.961218836565097)(292,0.961272475795297)(293,0.964656964656964)(294,0.95933838731909)(295,0.959394356503785)(296,0.95928226363009)(297,0.958677685950413)(298,0.957123098201936)(299,0.956461644782308)(300,0.956461644782308)(301,0.956401384083045)(302,0.959056210964608)(303,0.958391123439667)(304,0.958391123439667)(305,0.958391123439667)(306,0.959056210964608)(307,0.96038915913829)(308,0.959722222222222)(309,0.959056210964608)(310,0.957785467128028)(311,0.959056210964608)(312,0.959056210964608)(313,0.959056210964608)(314,0.964410327983252)(315,0.964410327983252)(316,0.963066202090592)(317,0.96373779637378)(318,0.962395543175487)(319,0.963066202090592)(320,0.963117606123869)(321,0.963636363636364)(322,0.962962962962963)(323,0.963066202090592)(324,0.961779013203613)(325,0.961779013203613)(326,0.962447844228094)(327,0.963117606123869)(328,0.96373779637378)(329,0.96373779637378)(330,0.963788300835654)(331,0.963788300835654)(332,0.963788300835654)(333,0.962447844228094)(334,0.963788300835654)(335,0.964459930313589)(336,0.964410327983252)(337,0.964410327983252)(338,0.964410327983252)(339,0.964410327983252)(340,0.962911126662001)(341,0.96)(342,0.961325966850829)(343,0.961325966850829)(344,0.962758620689655)(345,0.962758620689655)(346,0.9593837535014)(347,0.9593837535014)(348,0.9593837535014)(349,0.958538299367533)(350,0.958538299367533)(351,0.959326788218794)(352,0.963066202090592)(353,0.959269662921348)(354,0.961349262122277)(355,0.962078651685393)(356,0.961349262122277)(357,0.964260686755431)(358,0.964985994397759)(359,0.964985994397759)(360,0.964985994397759)(361,0.964985994397759)(362,0.965662228451296)(363,0.965662228451296)(364,0.965662228451296)(365,0.965662228451296)(366,0.965662228451296)(367,0.963957597173145)(368,0.964689265536723)(369,0.963328631875881)(370,0.962597035991531)(371,0.963328631875881)(372,0.964310706787963)(373,0.964310706787963)(374,0.963585434173669)(375,0.963585434173669)(376,0.964310706787963)(377,0.964310706787963)(378,0.964310706787963)(379,0.965034965034965)(380,0.965710286913926)(381,0.964985994397759)(382,0.964260686755431)(383,0.965662228451296)(384,0.964936886395512)(385,0.964936886395512)(386,0.962649753347428)(387,0.962277580071174)(388,0.962277580071174)(389,0.959198282032928)(390,0.959198282032928)(391,0.959198282032928)(392,0.961318051575931)(393,0.960629921259842)(394,0.961318051575931)(395,0.961318051575931)(396,0.95821325648415)(397,0.956709956709957)(398,0.961262553802009)(399,0.960629921259842)(400,0.959256611865618)(6,0.666227781435154)(7,0.741116751269035)(8,0.746530314097882)(9,0.744601638123604)(10,0.757530120481928)(11,0.783206106870229)(12,0.748822605965463)(13,0.764423076923077)(14,0.763811048839071)(15,0.765511684125705)(16,0.772213247172859)(17,0.771589991928975)(18,0.773569701853344)(19,0.773569701853344)(20,0.773569701853344)(21,0.770465489566613)(22,0.782051282051282)(23,0.783783783783784)(24,0.784099766173032)(25,0.790839694656488)(26,0.794272795779955)(27,0.79372197309417)(28,0.793939393939394)(29,0.801223241590214)(30,0.801532567049808)(31,0.806774441878368)(32,0.807424593967517)(33,0.810185185185185)(34,0.812596006144393)(35,0.816012317167051)(36,0.812883435582822)(37,0.815384615384615)(38,0.814186584425597)(39,0.813846153846154)(40,0.807017543859649)(41,0.810727969348659)(42,0.810727969348659)(43,0.811972371450499)(44,0.814984709480122)(45,0.817073170731707)(46,0.817696414950419)(47,0.817285822592873)(48,0.814814814814815)(49,0.814814814814815)(50,0.814814814814815)(51,0.819349962207105)(52,0.819076457229372)(53,0.820861678004535)(54,0.820861678004535)(55,0.821132075471698)(56,0.821132075471698)(57,0.819277108433735)(58,0.818660647103085)(59,0.82051282051282)(60,0.819894498869631)(61,0.82051282051282)(62,0.81969696969697)(63,0.817155756207675)(64,0.816541353383459)(65,0.821482602118003)(66,0.819622641509434)(67,0.817430503380917)(68,0.816204051012753)(69,0.81498127340824)(70,0.81437125748503)(71,0.81498127340824)(72,0.81498127340824)(73,0.817430503380917)(74,0.816204051012753)(75,0.816204051012753)(76,0.819894498869631)(77,0.8202416918429) 
};
\addplot [
color=orange,
mark size=0.1pt,
only marks,
mark=*,
mark options={solid,fill=black},
forget plot
]
coordinates{
 (77,0.8202416918429)(78,0.8202416918429)(79,0.822727272727273)(80,0.822104466313399)(81,0.822727272727273)(82,0.822995461422088)(83,0.82051282051282)(84,0.819894498869631)(85,0.819894498869631)(86,0.8202416918429)(87,0.819349962207105)(88,0.819969742813918)(89,0.819349962207105)(90,0.819349962207105)(91,0.821212121212121)(92,0.823082763857251)(93,0.823082763857251)(94,0.824242424242424)(95,0.822373393801965)(96,0.823618470855413)(97,0.822995461422088)(98,0.822995461422088)(99,0.825493171471927)(100,0.828006088280061)(101,0.828006088280061)(102,0.828006088280061)(103,0.828006088280061)(104,0.826747720364742)(105,0.826747720364742)(106,0.828006088280061)(107,0.821752265861027)(108,0.822373393801965)(109,0.823618470855413)(110,0.823618470855413)(111,0.822021116138763)(112,0.822021116138763)(113,0.822021116138763)(114,0.822021116138763)(115,0.819548872180451)(116,0.81893313298272)(117,0.817704426106526)(118,0.817704426106526)(119,0.814648729446936)(120,0.80920564216778)(121,0.811615785554728)(122,0.808005930318755)(123,0.809806835066865)(124,0.810408921933085)(125,0.811011904761905)(126,0.811011904761905)(127,0.810408921933085)(128,0.812220566318927)(129,0.81282624906786)(130,0.81282624906786)(131,0.81282624906786)(132,0.81282624906786)(133,0.810408921933085)(134,0.810408921933085)(135,0.80920564216778)(136,0.809806835066865)(137,0.809806835066865)(138,0.808005930318755)(139,0.805022156573117)(140,0.80920564216778)(141,0.810408921933085)(142,0.810408921933085)(143,0.810408921933085)(144,0.811011904761905)(145,0.813432835820895)(146,0.814648729446936)(147,0.811615785554728)(148,0.814040328603435)(149,0.814040328603435)(150,0.814040328603435)(151,0.81282624906786)(152,0.814040328603435)(153,0.811615785554728)(154,0.811615785554728)(155,0.811011904761905)(156,0.811011904761905)(157,0.810408921933085)(158,0.808605341246291)(159,0.80920564216778)(160,0.80920564216778)(161,0.80920564216778)(162,0.810408921933085)(163,0.809806835066865)(164,0.810408921933085)(165,0.810408921933085)(166,0.805617147080562)(167,0.807407407407407)(168,0.805022156573117)(169,0.805022156573117)(170,0.804428044280443)(171,0.809806835066865)(172,0.807407407407407)(173,0.805617147080562)(174,0.807407407407407)(175,0.804428044280443)(176,0.809806835066865)(177,0.809806835066865)(178,0.809806835066865)(179,0.811615785554728)(180,0.811615785554728)(181,0.814040328603435)(182,0.813432835820895)(183,0.816479400749064)(184,0.817091454272864)(185,0.813432835820895)(186,0.814648729446936)(187,0.811011904761905)(188,0.811011904761905)(189,0.810408921933085)(190,0.810408921933085)(191,0.810408921933085)(192,0.809806835066865)(193,0.809806835066865)(194,0.810408921933085)(195,0.812220566318927)(196,0.81282624906786)(197,0.815868263473054)(198,0.810408921933085)(199,0.80920564216778)(200,0.809806835066865)(201,0.80920564216778)(202,0.808005930318755)(203,0.810408921933085)(204,0.810408921933085)(205,0.816479400749064)(206,0.816479400749064)(207,0.815868263473054)(208,0.81893313298272)(209,0.818318318318318)(210,0.816479400749064)(211,0.817091454272864)(212,0.815868263473054)(213,0.817091454272864)(214,0.81282624906786)(215,0.822641509433962)(216,0.823262839879154)(217,0.82701062215478)(218,0.826383623957544)(219,0.82701062215478)(220,0.826383623957544)(221,0.826383623957544)(222,0.825132475397426)(223,0.823885109599395)(224,0.824508320726172)(225,0.824508320726172)(226,0.82701062215478)(227,0.826383623957544)(228,0.825757575757576)(229,0.826383623957544)(230,0.813432835820895)(231,0.81282624906786)(232,0.81282624906786)(233,0.814040328603435)(234,0.81525804038893)(235,0.816479400749064)(236,0.817091454272864)(237,0.817704426106526)(238,0.818318318318318)(239,0.817704426106526)(240,0.820165537998495)(241,0.821401657874906)(242,0.821401657874906)(243,0.821401657874906)(244,0.82078313253012)(245,0.827638572513288)(246,0.835249042145594)(247,0.835249042145594)(248,0.836531082118189)(249,0.838461538461538)(250,0.838461538461538)(251,0.83910700538876)(252,0.835249042145594)(253,0.825757575757576)(254,0.833970925784239)(255,0.835249042145594)(256,0.837173579109063)(257,0.837173579109063)(258,0.837817063797079)(259,0.837173579109063)(260,0.837173579109063)(261,0.836531082118189)(262,0.836531082118189)(263,0.833970925784239)(264,0.840400925212028)(265,0.840400925212028)(266,0.840400925212028)(267,0.840400925212028)(268,0.838461538461538)(269,0.837817063797079)(270,0.838461538461538)(271,0.837817063797079)(272,0.837817063797079)(273,0.837817063797079)(274,0.837817063797079)(275,0.838461538461538)(276,0.838461538461538)(277,0.838461538461538)(278,0.841049382716049)(279,0.841698841698842)(280,0.842349304482226)(281,0.837817063797079)(282,0.83910700538876)(283,0.83910700538876)(284,0.83910700538876)(285,0.83910700538876)(286,0.841698841698842)(287,0.842349304482226)(288,0.841698841698842)(289,0.841698841698842)(290,0.843000773395205)(291,0.842349304482226)(292,0.841049382716049)(293,0.840400925212028)(294,0.833333333333333)(295,0.837817063797079)(296,0.837173579109063)(297,0.837817063797079)(298,0.838461538461538)(299,0.840400925212028)(300,0.842349304482226)(301,0.843000773395205)(302,0.844961240310077)(303,0.844961240310077)(304,0.844306738962045)(305,0.844961240310077)(306,0.848909657320872)(307,0.846930846930847)(308,0.846930846930847)(309,0.846930846930847)(310,0.846930846930847)(311,0.846273291925466)(312,0.847589424572317)(313,0.857142857142857)(314,0.854889589905363)(315,0.857142857142857)(316,0.856240126382306)(317,0.856240126382306)(318,0.855106888361045)(319,0.855106888361045)(320,0.855106888361045)(321,0.853057982525814)(322,0.854199683042789)(323,0.853736089030207)(324,0.853736089030207)(325,0.853736089030207)(326,0.853736089030207)(327,0.852824184566428)(328,0.849402390438247)(329,0.848484848484848)(330,0.848484848484848)(331,0.85214626391097)(332,0.855106888361045)(333,0.856240126382306)(334,0.854430379746835)(335,0.854430379746835)(336,0.854430379746835)(337,0.854430379746835)(338,0.854430379746835)(339,0.854430379746835)(340,0.854430379746835)(341,0.855564325177585)(342,0.856466876971609)(343,0.857142857142857)(344,0.856466876971609)(345,0.852664576802508)(346,0.852664576802508)(347,0.853333333333333)(348,0.850664581704456)(349,0.850664581704456)(350,0.850664581704456)(351,0.850664581704456)(352,0.851330203442879)(353,0.850897736143638)(354,0.850664581704456)(355,0.850664581704456)(356,0.847589424572317)(357,0.848909657320872)(358,0.848909657320872)(359,0.848909657320872)(360,0.848909657320872)(361,0.848909657320872)(362,0.849571317225253)(363,0.851996867658575)(364,0.8515625)(365,0.8515625)(366,0.8515625)(367,0.8515625)(368,0.8515625)(369,0.850234009360374)(370,0.850897736143638)(371,0.842349304482226)(372,0.842349304482226)(373,0.842349304482226)(374,0.842349304482226)(375,0.842349304482226)(376,0.842349304482226)(377,0.842349304482226)(378,0.842349304482226)(379,0.846273291925466)(380,0.847589424572317)(381,0.847589424572317)(382,0.847589424572317)(383,0.847589424572317)(384,0.848249027237354)(385,0.847589424572317)(386,0.847589424572317)(387,0.847589424572317)(388,0.847589424572317)(389,0.847589424572317)(390,0.850234009360374)(391,0.850234009360374)(392,0.851764705882353)(393,0.851764705882353)(394,0.851764705882353)(395,0.85377358490566)(396,0.851764705882353)(397,0.852871754523997)(398,0.853543307086614)(399,0.854215918045705)(400,0.854215918045705)(5,0.63268156424581)(6,0.622886866059818)(7,0.708604483007954)(8,0.727941176470588)(9,0.68542199488491)(10,0.71619812583668)(11,0.723461798512508)(12,0.721024258760108)(13,0.711525649566955)(14,0.728629579375848)(15,0.731109598366235)(16,0.732605729877217)(17,0.74171270718232)(18,0.759547383309759)(19,0.758474576271186)(20,0.760623229461756)(21,0.7660485021398)(22,0.768240343347639)(23,0.767142857142857)(24,0.769886363636363)(25,0.782106782106782)(26,0.780979827089337)(27,0.780979827089337)(28,0.779856115107914)(29,0.779053084648494)(30,0.777380100214746)(31,0.782984859408796)(32,0.785249457700651)(33,0.786647314949201)(34,0.791240875912409)(35,0.791575889615105)(36,0.790697674418605)(37,0.792425345957756)(38,0.792425345957756)(39,0.793581327498176)(40,0.794160583941606)(41,0.794740686632578)(42,0.792425345957756)(43,0.7953216374269)(44,0.800588668138337)(45,0.797653958944281)(46,0.798239178283199)(47,0.801178203240059)(48,0.801768607221813)(49,0.800588668138337)(50,0.8)(51,0.801768607221813)(52,0.799412196914034)(53,0.801768607221813)(54,0.801768607221813)(55,0.801182557280118)(56,0.805040770941438)(57,0.809239940387481)(58,0.811659192825112)(59,0.81376215407629)(60,0.81437125748503)(61,0.813153961136024)(62,0.81437125748503)(63,0.813153961136024)(64,0.815315315315315)(65,0.814648729446936)(66,0.811615785554728)(67,0.81282624906786)(68,0.81525804038893)(69,0.81525804038893)(70,0.815868263473054)(71,0.813432835820895)(72,0.813432835820895)(73,0.814648729446936)(74,0.813432835820895)(75,0.811615785554728)(76,0.811615785554728)(77,0.811615785554728)(78,0.811615785554728)(79,0.812220566318927)(80,0.812220566318927)(81,0.814040328603435)(82,0.814648729446936)(83,0.816479400749064)(84,0.816479400749064)(85,0.817091454272864)(86,0.816479400749064)(87,0.814040328603435)(88,0.81282624906786)(89,0.816479400749064)(90,0.816479400749064)(91,0.816479400749064)(92,0.816479400749064)(93,0.817704426106526)(94,0.818318318318318)(95,0.815650865312265)(96,0.817430503380917)(97,0.815927873779113)(98,0.817430503380917)(99,0.818045112781955)(100,0.819894498869631)(101,0.82051282051282)(102,0.82051282051282)(103,0.818660647103085)(104,0.816816816816817)(105,0.818045112781955)(106,0.818660647103085)(107,0.819894498869631)(108,0.819894498869631)(109,0.813432835820895)(110,0.818318318318318)(111,0.817091454272864)(112,0.81893313298272)(113,0.817091454272864)(114,0.817704426106526)(115,0.817704426106526)(116,0.81893313298272)(117,0.81893313298272)(118,0.81893313298272)(119,0.81893313298272)(120,0.821752265861027)(121,0.821482602118003)(122,0.821482602118003)(123,0.822641509433962)(124,0.822727272727273)(125,0.822727272727273)(126,0.822104466313399)(127,0.822104466313399)(128,0.823975720789074)(129,0.824601366742597)(130,0.825227963525836)(131,0.824601366742597)(132,0.815868263473054)(133,0.815868263473054)(134,0.817704426106526)(135,0.818318318318318)(136,0.816479400749064)(137,0.821401657874906)(138,0.817091454272864)(139,0.817091454272864)(140,0.817704426106526)(141,0.815868263473054)(142,0.815868263473054)(143,0.815868263473054)(144,0.817091454272864)(145,0.817091454272864)(146,0.816479400749064)(147,0.816479400749064)(148,0.816479400749064)(149,0.816479400749064)(150,0.816479400749064)(151,0.815868263473054)(152,0.815868263473054)(153,0.813432835820895)(154,0.81282624906786)(155,0.80920564216778)(156,0.809806835066865)(157,0.808605341246291)(158,0.808605341246291)(159,0.809806835066865)(160,0.80680977054034)(161,0.804428044280443)(162,0.804428044280443)(163,0.80680977054034)(164,0.80680977054034)(165,0.808005930318755)(166,0.805617147080562)(167,0.805617147080562)(168,0.805022156573117)(169,0.808605341246291)(170,0.808605341246291)(171,0.809806835066865)(172,0.812220566318927)(173,0.812220566318927)(174,0.81282624906786)(175,0.81282624906786)(176,0.813432835820895)(177,0.81282624906786)(178,0.812220566318927)(179,0.81282624906786)(180,0.812220566318927)(181,0.82846975088968)(182,0.828225231646472)(183,0.798898071625344)(184,0.799448656099242)(185,0.801064537591484)(186,0.800263678312459)(187,0.804990151017728)(188,0.808426596445029)(189,0.810448760884126)(190,0.817218543046357)(191,0.837996096291477)(192,0.834625322997416)(193,0.854406130268199)(194,0.858942065491184)(195,0.860025220680958)(196,0.861675126903553)(197,0.861675126903553)(198,0.86848635235732)(199,0.861144945188794)(200,0.861313868613139)(201,0.855589123867069)(202,0.857142857142857)(203,0.858702243784111)(204,0.858181818181818)(205,0.858181818181818)(206,0.857662023016354)(207,0.857662023016354)(208,0.855937311633514)(209,0.854906682721252)(210,0.858524788391777)(211,0.860084797092671)(212,0.858524788391777)(213,0.85956416464891)(214,0.856626506024096)(215,0.856626506024096)(216,0.857142857142857)(217,0.859733978234583)(218,0.860084797092671)(219,0.86322188449848)(220,0.86322188449848)(221,0.870098039215686)(222,0.870098039215686)(223,0.871165644171779)(224,0.871165644171779)(225,0.870631514408338)(226,0.870631514408338)(227,0.879801734820322)(228,0.880893300248139)(229,0.880893300248139)(230,0.879256965944272)(231,0.879801734820322)(232,0.879801734820322)(233,0.879801734820322)(234,0.881440099317194)(235,0.884447220487195)(236,0.885)(237,0.884591391141609)(238,0.884591391141609)(239,0.884591391141609)(240,0.885839051777916)(241,0.88875077688005)(242,0.887507768800497)(243,0.889993784959602)(244,0.88985687616677)(245,0.88985687616677)(246,0.890410958904109)(247,0.890410958904109)(248,0.88985687616677)(249,0.88985687616677)(250,0.88985687616677)(251,0.890965732087227)(252,0.890410958904109)(253,0.891101431238332)(254,0.89221183800623)(255,0.892768079800499)(256,0.892768079800499)(257,0.893882646691635)(258,0.896681277395116)(259,0.90176322418136)(260,0.90107120352867)(261,0.906785034876347)(262,0.910133843212237)(263,0.91362763915547)(264,0.921885087153002)(265,0.922480620155039)(266,0.922480620155039)(267,0.924271844660194)(268,0.919510624597553)(269,0.915384615384615)(270,0.915384615384615)(271,0.915971776779987)(272,0.915275994865212)(273,0.915971776779987)(274,0.915971776779987)(275,0.915971776779987)(276,0.915971776779987)(277,0.914798206278027)(278,0.923076923076923)(279,0.918918918918919)(280,0.918918918918919)(281,0.918918918918919)(282,0.918918918918919)(283,0.920696324951644)(284,0.925469863901491)(285,0.925469863901491)(286,0.924870466321243)(287,0.924870466321243)(288,0.924870466321243)(289,0.925469863901491)(290,0.925469863901491)(291,0.925469863901491)(292,0.925469863901491)(293,0.925469863901491)(294,0.924870466321243)(295,0.923673997412678)(296,0.921885087153002)(297,0.921885087153002)(298,0.921885087153002)(299,0.923076923076923)(300,0.923076923076923)(301,0.923076923076923)(302,0.923076923076923)(303,0.921885087153002)(304,0.922480620155039)(305,0.923076923076923)(306,0.924870466321243)(307,0.921885087153002)(308,0.921885087153002)(309,0.921885087153002)(310,0.921885087153002)(311,0.919510624597553)(312,0.918327974276527)(313,0.918327974276527)(314,0.917148362235067)(315,0.917148362235067)(316,0.917737789203085)(317,0.917148362235067)(318,0.917737789203085)(319,0.917148362235067)(320,0.917737789203085)(321,0.918918918918919)(322,0.918327974276527)(323,0.918327974276527)(324,0.917737789203085)(325,0.917737789203085)(326,0.917148362235067)(327,0.917148362235067)(328,0.915971776779987)(329,0.917148362235067)(330,0.918327974276527)(331,0.917737789203085)(332,0.917148362235067)(333,0.917148362235067)(334,0.917737789203085)(335,0.917737789203085)(336,0.919127086007702)(337,0.919127086007702)(338,0.920308483290488)(339,0.920308483290488)(340,0.920900321543408)(341,0.919717405266538)(342,0.917948717948718)(343,0.919127086007702)(344,0.919127086007702)(345,0.918918918918919)(346,0.919717405266538)(347,0.919717405266538)(348,0.919717405266538)(349,0.919717405266538)(350,0.927178153446034)(351,0.931327665140615)(352,0.933158584534731)(353,0.933158584534731)(354,0.933158584534731)(355,0.933246073298429)(356,0.934640522875817)(357,0.933420365535248)(358,0.933420365535248)(359,0.934640522875817)(360,0.937090432503276)(361,0.935251798561151)(362,0.935251798561151)(363,0.935779816513761)(364,0.936393442622951)(365,0.937007874015748)(366,0.937007874015748)(367,0.937007874015748)(368,0.938856015779093)(369,0.936309914642154)(370,0.936393442622951)(371,0.936393442622951)(372,0.936393442622951)(373,0.936393442622951)(374,0.935779816513761)(375,0.936393442622951)(376,0.93717277486911)(377,0.93717277486911)(378,0.936559843034663)(379,0.942105263157895)(380,0.940867279894875)(381,0.942105263157895)(382,0.942105263157895)(383,0.938401048492792)(384,0.937786509495743)(385,0.939632545931758)(386,0.940249507550886)(387,0.939016393442623)(388,0.939016393442623)(389,0.939632545931758)(390,0.939632545931758)(391,0.940249507550886)(392,0.939632545931758)(393,0.939632545931758)(394,0.938320209973753)(395,0.938320209973753)(396,0.938320209973753)(397,0.938320209973753)(398,0.938320209973753)(399,0.938320209973753)(400,0.938320209973753)(2,0.655643421998562)(3,0.657332350773766)(4,0.695934959349593)(5,0.698983580922596)(6,0.752293577981651)(7,0.756508422664625)(8,0.761682242990654)(9,0.770685579196217)(10,0.768754833720031)(11,0.767925983037779)(12,0.774292272379495)(13,0.773109243697479)(14,0.780715396578538)(15,0.796674225245654)(16,0.785493827160494)(17,0.785493827160494)(18,0.786534047436878)(19,0.797814207650273)(20,0.795665634674923)(21,0.791698693312836)(22,0.795981452859351)(23,0.797564687975647)(24,0.805534204458109)(25,0.805534204458109)(26,0.804597701149425)(27,0.807311500380807)(28,0.807896735003796)(29,0.809415337889142)(30,0.808188021228203)(31,0.803936411809235)(32,0.804545454545454)(33,0.803328290468986)(34,0.803328290468986)(35,0.804528301886792)(36,0.803921568627451)(37,0.804528301886792)(38,0.803007518796992)(39,0.800599700149925)(40,0.80786686838124)(41,0.80786686838124)(42,0.80786686838124)(43,0.807256235827664)(44,0.80786686838124)(45,0.81060606060606)(46,0.811220621683093)(47,0.809704321455648)(48,0.805744520030234)(49,0.806037735849056)(50,0.808157099697885)(51,0.809056603773585)(52,0.808767951625094)(53,0.809667673716012)(54,0.809090909090909)(55,0.813404417364813)(56,0.813688212927757)(57,0.814307458143074)(58,0.813404417364813)(59,0.813404417364813)(60,0.812785388127854)(61,0.815151515151515)(62,0.81570996978852)(63,0.815151515151515)(64,0.815151515151515)(65,0.814253222137983)(66,0.816666666666667)(67,0.816666666666667)(68,0.816666666666667)(69,0.816666666666667)(70,0.815431164901664)(71,0.815151515151515)(72,0.814534443603331)(73,0.815209125475285)(74,0.813020439061317)(75,0.814253222137983)(76,0.814253222137983)(77,0.814253222137983)(78,0.814645308924485)(79,0.817974105102818)(80,0.816450875856816)(81,0.812785388127854)(82,0.814871016691957)(83,0.814253222137983)(84,0.814871016691957)(85,0.815431164901664)(86,0.813584905660377)(87,0.813584905660377)(88,0.812971342383107)(89,0.812971342383107)(90,0.812971342383107)(91,0.812971342383107)(92,0.814199395770393)(93,0.813584905660377)(94,0.813584905660377)(95,0.815431164901664)(96,0.814814814814815)(97,0.814199395770393)(98,0.814199395770393)(99,0.814814814814815)(100,0.814814814814815)(101,0.814814814814815)(102,0.814814814814815)(103,0.816048448145344)(104,0.816048448145344)(105,0.816048448145344)(106,0.816048448145344)(107,0.817905918057663)(108,0.817905918057663)(109,0.816730038022814)(110,0.816730038022814)(111,0.815151515151515)(112,0.815151515151515)(113,0.815431164901664)(114,0.815431164901664)(115,0.815431164901664)(116,0.815431164901664)(117,0.815431164901664)(118,0.815431164901664)(119,0.815431164901664)(120,0.815431164901664)(121,0.815431164901664)(122,0.814814814814815)(123,0.814814814814815)(124,0.814814814814815)(125,0.81570996978852)(126,0.81570996978852)(127,0.81570996978852)(128,0.81386586284853)(129,0.815094339622641)(130,0.816326530612245)(131,0.816326530612245)(132,0.816944024205749)(133,0.817562452687358)(134,0.816944024205749)(135,0.818181818181818)(136,0.818181818181818)(137,0.818181818181818)(138,0.817562452687358)(139,0.818456883509833)(140,0.818456883509833)(141,0.818731117824773)(142,0.818731117824773)(143,0.817838246409675)(144,0.817838246409675)(145,0.818802122820318)(146,0.818731117824773)(147,0.817496229260935)(148,0.819349962207105)(149,0.819969742813918)(150,0.819969742813918)(151,0.819969742813918)(152,0.819969742813918)(153,0.819969742813918)(154,0.817496229260935)(155,0.817496229260935)(156,0.819277108433735)(157,0.819277108433735)(158,0.819277108433735)(159,0.819277108433735)(160,0.819277108433735)(161,0.819277108433735)(162,0.817430503380917)(163,0.816816816816817)(164,0.817430503380917)(165,0.816816816816817)(166,0.819277108433735)(167,0.821132075471698)(168,0.821132075471698)(169,0.821132075471698)(170,0.821132075471698)(171,0.821132075471698)(172,0.821132075471698)(173,0.818660647103085)(174,0.817430503380917)(175,0.819277108433735)(176,0.82051282051282)(177,0.819894498869631)(178,0.819894498869631)(179,0.82051282051282)(180,0.82051282051282)(181,0.82051282051282)(182,0.82051282051282)(183,0.82051282051282)(184,0.819894498869631)(185,0.819894498869631)(186,0.819277108433735)(187,0.817430503380917)(188,0.818660647103085)(189,0.818660647103085)(190,0.814648729446936)(191,0.814648729446936)(192,0.81282624906786)(193,0.81282624906786)(194,0.813432835820895)(195,0.81525804038893)(196,0.816479400749064)(197,0.815868263473054)(198,0.815868263473054)(199,0.815868263473054)(200,0.817091454272864)(201,0.816479400749064)(202,0.817091454272864)(203,0.816816816816817)(204,0.816816816816817)(205,0.81498127340824)(206,0.816479400749064)(207,0.817091454272864)(208,0.817430503380917)(209,0.817430503380917)(210,0.817091454272864)(211,0.817091454272864)(212,0.815868263473054)(213,0.817704426106526)(214,0.817704426106526)(215,0.817704426106526)(216,0.817704426106526)(217,0.816479400749064)(218,0.814040328603435)(219,0.811615785554728)(220,0.811615785554728)(221,0.811615785554728)(222,0.814648729446936)(223,0.814040328603435)(224,0.813432835820895)(225,0.813432835820895)(226,0.813432835820895)(227,0.812220566318927)(228,0.812220566318927)(229,0.814040328603435)(230,0.81525804038893)(231,0.815868263473054)(232,0.815868263473054)(233,0.817091454272864)(234,0.817704426106526)(235,0.817091454272864)(236,0.817091454272864)(237,0.817704426106526)(238,0.817704426106526)(239,0.817704426106526)(240,0.818318318318318)(241,0.819548872180451)(242,0.819548872180451)(243,0.819548872180451)(244,0.822641509433962)(245,0.822641509433962)(246,0.814648729446936)(247,0.814648729446936)(248,0.81525804038893)(249,0.814648729446936)(250,0.814648729446936)(251,0.814648729446936)(252,0.814648729446936)(253,0.815868263473054)(254,0.816479400749064)(255,0.816479400749064)(256,0.81525804038893)(257,0.811615785554728)(258,0.811011904761905)(259,0.812220566318927)(260,0.812220566318927)(261,0.812220566318927)(262,0.812220566318927)(263,0.812220566318927)(264,0.814040328603435)(265,0.814040328603435)(266,0.814040328603435)(267,0.814040328603435)(268,0.814648729446936)(269,0.817704426106526)(270,0.81893313298272)(271,0.81893313298272)(272,0.81893313298272)(273,0.813432835820895)(274,0.81282624906786)(275,0.811615785554728)(276,0.81282624906786)(277,0.81282624906786)(278,0.814040328603435)(279,0.814040328603435)(280,0.81282624906786)(281,0.81282624906786)(282,0.814648729446936)(283,0.818318318318318)(284,0.820165537998495)(285,0.819548872180451)(286,0.82078313253012)(287,0.820165537998495)(288,0.81525804038893)(289,0.81525804038893)(290,0.820165537998495)(291,0.820165537998495)(292,0.82078313253012)(293,0.82078313253012)(294,0.820165537998495)(295,0.820165537998495)(296,0.820165537998495)(297,0.820165537998495)(298,0.822021116138763)(299,0.822021116138763)(300,0.822021116138763)(301,0.822021116138763)(302,0.822641509433962)(303,0.822021116138763)(304,0.822641509433962)(305,0.822641509433962)(306,0.822021116138763)(307,0.82078313253012)(308,0.821401657874906)(309,0.821401657874906)(310,0.822021116138763)(311,0.823262839879154)(312,0.822641509433962)(313,0.823262839879154)(314,0.823885109599395)(315,0.824508320726172)(316,0.825132475397426)(317,0.825132475397426)(318,0.824508320726172)(319,0.822021116138763)(320,0.823885109599395)(321,0.817091454272864)(322,0.820165537998495)(323,0.820165537998495)(324,0.819548872180451)(325,0.823262839879154)(326,0.818318318318318)(327,0.821401657874906)(328,0.823885109599395)(329,0.825757575757576)(330,0.825757575757576)(331,0.825757575757576)(332,0.825757575757576)(333,0.825757575757576)(334,0.825757575757576)(335,0.825757575757576)(336,0.825757575757576)(337,0.827638572513288)(338,0.829528158295281)(339,0.83015993907083)(340,0.83015993907083)(341,0.82701062215478)(342,0.828267477203647)(343,0.839753466872111)(344,0.83910700538876)(345,0.83910700538876)(346,0.841698841698842)(347,0.838461538461538)(348,0.842349304482226)(349,0.842349304482226)(350,0.843653250773994)(351,0.839753466872111)(352,0.841698841698842)(353,0.841698841698842)(354,0.841049382716049)(355,0.83910700538876)(356,0.837173579109063)(357,0.838461538461538)(358,0.841049382716049)(359,0.841049382716049)(360,0.838461538461538)(361,0.838461538461538)(362,0.839753466872111)(363,0.839753466872111)(364,0.839753466872111)(365,0.839753466872111)(366,0.842349304482226)(367,0.841049382716049)(368,0.843653250773994)(369,0.846273291925466)(370,0.846273291925466)(371,0.846273291925466)(372,0.844961240310077)(373,0.846273291925466)(374,0.845616757176105)(375,0.842349304482226)(376,0.842349304482226)(377,0.842349304482226)(378,0.843000773395205)(379,0.843000773395205)(380,0.843000773395205)(381,0.83910700538876)(382,0.83910700538876)(383,0.83910700538876)(384,0.83910700538876)(385,0.83910700538876)(386,0.83910700538876)(387,0.83910700538876)(388,0.841049382716049)(389,0.843000773395205)(390,0.843000773395205)(391,0.837173579109063)(392,0.837173579109063)(393,0.837173579109063)(394,0.837173579109063)(395,0.837817063797079)(396,0.837817063797079)(397,0.837817063797079)(398,0.837817063797079)(399,0.837817063797079)(400,0.837817063797079)(7,0.687186828919112)(8,0.674927113702624)(9,0.721131186174391)(10,0.738799661876585)(11,0.736408566721582)(12,0.746071133167907)(13,0.780844155844156)(14,0.769230769230769)(15,0.771241830065359)(16,0.778229082047116)(17,0.776508972267537)(18,0.786570743405276)(19,0.79672131147541)(20,0.791666666666667)(21,0.788429752066116)(22,0.794723825226711)(23,0.792733278282411)(24,0.792390405293631)(25,0.791735537190083)(26,0.799003322259136)(27,0.815359477124183)(28,0.829975825946817)(29,0.832797427652733)(30,0.832797427652733)(31,0.824476650563607)(32,0.827140549273021)(33,0.829701372074253)(34,0.833467417538214)(35,0.834138486312399)(36,0.82875605815832)(37,0.831587429492345)(38,0.830645161290322)(39,0.837359098228663)(40,0.840417000801925)(41,0.840417000801925)(42,0.840417000801925)(43,0.839486356340289)(44,0.839228295819936)(45,0.840160642570281)(46,0.840160642570281)(47,0.842777334397446)(48,0.842777334397446)(49,0.839071257005604)(50,0.837469975980785)(51,0.836276083467095)(52,0.836276083467095)(53,0.839071257005604)(54,0.837881219903692)(55,0.836276083467095)(56,0.834405144694534)(57,0.834405144694534)(58,0.833734939759036)(59,0.82875605815832)(60,0.826580226904376)(61,0.826860841423948)(62,0.829032258064516)(63,0.829032258064516)(64,0.828087167070218)(65,0.828087167070218)(66,0.828087167070218)(67,0.828087167070218)(68,0.826192400970089)(69,0.827140549273021)(70,0.826192400970089)(71,0.829975825946817)(72,0.828087167070218)(73,0.828087167070218)(74,0.826192400970089)(75,0.826192400970089)(76,0.826860841423948)(77,0.826860841423948)(78,0.826580226904376)(79,0.825910931174089)(80,0.826472962066182)(81,0.826472962066182)(82,0.824675324675325)(83,0.827140549273021)(84,0.826860841423948)(85,0.826860841423948)(86,0.825910931174089)(87,0.825910931174089)(88,0.826580226904376)(89,0.828200972447326)(90,0.832528180354267)(91,0.832528180354267)(92,0.832528180354267)(93,0.831315577078289)(94,0.832659660468876)(95,0.831315577078289)(96,0.831315577078289)(97,0.832258064516129)(98,0.832258064516129)(99,0.832258064516129)(100,0.832258064516129)(101,0.830769230769231)(102,0.832659660468876)(103,0.830769230769231)(104,0.830769230769231)(105,0.830769230769231)(106,0.830769230769231)(107,0.828872668288727)(108,0.829821717990275)(109,0.828872668288727)(110,0.828872668288727)(111,0.828872668288727)(112,0.827922077922078)(113,0.828872668288727)(114,0.827922077922078)(115,0.827922077922078)(116,0.827922077922078)(117,0.827922077922078)(118,0.827922077922078)(119,0.828872668288727)(120,0.828872668288727)(121,0.831442463533225)(122,0.831442463533225)(123,0.831715210355987)(124,0.832388663967611)(125,0.830494728304947)(126,0.830494728304947)(127,0.830494728304947)(128,0.830494728304947)(129,0.832388663967611)(130,0.832388663967611)(131,0.834276475343573)(132,0.834276475343573)(133,0.834276475343573)(134,0.834276475343573)(135,0.834276475343573)(136,0.834276475343573)(137,0.834276475343573)(138,0.834276475343573)(139,0.835218093699515)(140,0.835218093699515)(141,0.83454398708636)(142,0.83454398708636)(143,0.83454398708636)(144,0.83454398708636)(145,0.8348106365834)(146,0.8348106365834)(147,0.8348106365834)(148,0.835748792270531)(149,0.835748792270531)(150,0.835218093699515)(151,0.8348106365834)(152,0.835076427996782)(153,0.837881219903692)(154,0.837881219903692)(155,0.840672538030424)(156,0.838813151563753)(157,0.83855421686747)(158,0.839486356340289)(159,0.83668543845535)(160,0.837620578778135)(161,0.839486356340289)(162,0.839486356340289)(163,0.839486356340289)(164,0.839486356340289)(165,0.839486356340289)(166,0.845295055821372)(167,0.849880857823669)(168,0.849206349206349)(169,0.849206349206349)(170,0.850118953211737)(171,0.847671665351223)(172,0.847244094488189)(173,0.847244094488189)(174,0.847058823529412)(175,0.847244094488189)(176,0.846818538884525)(177,0.845911949685534)(178,0.847244094488189)(179,0.845911949685534)(180,0.845911949685534)(181,0.846577498033045)(182,0.848151062155783)(183,0.848151062155783)(184,0.847244094488189)(185,0.846703733121525)(186,0.845786963434022)(187,0.845786963434022)(188,0.847189231987332)(189,0.844621513944223)(190,0.8416)(191,0.8416)(192,0.840672538030424)(193,0.83974358974359)(194,0.83974358974359)(195,0.8416)(196,0.83454398708636)(197,0.834951456310679)(198,0.834008097165992)(199,0.834008097165992)(200,0.833333333333333)(201,0.832388663967611)(202,0.833333333333333)(203,0.834276475343573)(204,0.835218093699515)(205,0.837359098228663)(206,0.838813151563753)(207,0.837359098228663)(208,0.836422240128928)(209,0.83668543845535)(210,0.836422240128928)(211,0.836012861736334)(212,0.836012861736334)(213,0.836276083467095)(214,0.836538461538461)(215,0.837469975980785)(216,0.840255591054313)(217,0.840255591054313)(218,0.843027888446215)(219,0.843949044585987)(220,0.843027888446215)(221,0.842105263157895)(222,0.842105263157895)(223,0.84)(224,0.84)(225,0.84)(226,0.84185303514377)(227,0.840927258193445)(228,0.840927258193445)(229,0.84)(230,0.841346153846154)(231,0.837359098228663)(232,0.837359098228663)(233,0.838294448913918)(234,0.838294448913918)(235,0.840836012861736)(236,0.840836012861736)(237,0.838709677419355)(238,0.838709677419355)(239,0.83668543845535)(240,0.835218093699515)(241,0.833333333333333)(242,0.831442463533225)(243,0.831442463533225)(244,0.831442463533225)(245,0.829545454545454)(246,0.834008097165992)(247,0.833063209076175)(248,0.835748792270531)(249,0.837772397094431)(250,0.835627530364372)(251,0.835627530364372)(252,0.830894308943089)(253,0.832792207792208)(254,0.832792207792208)(255,0.832792207792208)(256,0.835627530364372)(257,0.835627530364372)(258,0.835627530364372)(259,0.835627530364372)(260,0.835627530364372)(261,0.835627530364372)(262,0.835627530364372)(263,0.836569579288026)(264,0.837510105092967)(265,0.841257050765512)(266,0.836569579288026)(267,0.839386602098466)(268,0.838449111470113)(269,0.839645447219984)(270,0.842443729903537)(271,0.840579710144927)(272,0.842525979216627)(273,0.842777334397446)(274,0.843700159489633)(275,0.842777334397446)(276,0.842777334397446)(277,0.842777334397446)(278,0.843700159489633)(279,0.845541401273885)(280,0.845541401273885)(281,0.846215139442231)(282,0.845541401273885)(283,0.845723421262989)(284,0.842190016103059)(285,0.837510105092967)(286,0.838449111470113)(287,0.839386602098466)(288,0.837510105092967)(289,0.841257050765512)(290,0.841257050765512)(291,0.840322580645161)(292,0.839386602098466)(293,0.838449111470113)(294,0.838449111470113)(295,0.838449111470113)(296,0.838449111470113)(297,0.838449111470113)(298,0.837510105092967)(299,0.838709677419355)(300,0.848726114649681)(301,0.843624699278268)(302,0.847808764940239)(303,0.839645447219984)(304,0.840579710144927)(305,0.840579710144927)(306,0.837772397094431)(307,0.839645447219984)(308,0.845476381104884)(309,0.843373493975903)(310,0.843624699278268)(311,0.843624699278268)(312,0.843624699278268)(313,0.839386602098466)(314,0.843121480289622)(315,0.843121480289622)(316,0.84688995215311)(317,0.847808764940239)(318,0.848726114649681)(319,0.851030110935024)(320,0.851704996034893)(321,0.851704996034893)(322,0.849642004773269)(323,0.851704996034893)(324,0.851704996034893)(325,0.848966613672496)(326,0.847133757961783)(327,0.84805091487669)(328,0.848966613672496)(329,0.851704996034893)(330,0.848966613672496)(331,0.850793650793651)(332,0.850793650793651)(333,0.845969672785315)(334,0.845969672785315)(335,0.845969672785315)(336,0.84688995215311)(337,0.84688995215311)(338,0.84688995215311)(339,0.847808764940239)(340,0.846215139442231)(341,0.84688995215311)(342,0.845969672785315)(343,0.845969672785315)(344,0.84688995215311)(345,0.845969672785315)(346,0.84664536741214)(347,0.845723421262989)(348,0.845723421262989)(349,0.850556438791733)(350,0.850556438791733)(351,0.84756584197925)(352,0.848242811501597)(353,0.849642004773269)(354,0.848726114649681)(355,0.848726114649681)(356,0.848726114649681)(357,0.848484848484848)(358,0.848)(359,0.84664536741214)(360,0.84430176565008)(361,0.840322580645161)(362,0.840322580645161)(363,0.840322580645161)(364,0.836569579288026)(365,0.836569579288026)(366,0.836569579288026)(367,0.832792207792208)(368,0.832792207792208)(369,0.831844029244517)(370,0.831844029244517)(371,0.833738848337388)(372,0.832792207792208)(373,0.830894308943089)(374,0.830894308943089)(375,0.833738848337388)(376,0.851233094669849)(377,0.851910828025478)(378,0.851910828025478)(379,0.851910828025478)(380,0.851910828025478)(381,0.851910828025478)(382,0.851910828025478)(383,0.850996015936255)(384,0.850996015936255)(385,0.850996015936255)(386,0.853057982525814)(387,0.85193982581156)(388,0.852848101265823)(389,0.852848101265823)(390,0.852848101265823)(391,0.852848101265823)(392,0.852848101265823)(393,0.852848101265823)(394,0.852848101265823)(395,0.851500789889415)(396,0.851500789889415)(397,0.851500789889415)(398,0.851500789889415)(399,0.851063829787234)(400,0.851063829787234)(12,0.715384615384615)(13,0.715170278637771)(14,0.712452830188679)(15,0.738023952095808)(16,0.733105802047782)(17,0.758909853249476)(18,0.778254649499284)(19,0.799398948159279)(20,0.801498127340824)(21,0.780487804878049)(22,0.776859504132231)(23,0.796338672768879)(24,0.805389221556886)(25,0.794419970631424)(26,0.796187683284457)(27,0.791272727272727)(28,0.799116997792494)(29,0.803254437869822)(30,0.797927461139896)(31,0.794701986754967)(32,0.794399410464259)(33,0.807778608825729)(34,0.816944024205749)(35,0.815431164901664)(36,0.814814814814815)(37,0.820395738203957)(38,0.822900763358779)(39,0.822543792840822)(40,0.825132475397426)(41,0.827113480578827)(42,0.826219512195122)(43,0.825590251332825)(44,0.82262996941896)(45,0.821374045801527)(46,0.821374045801527)(47,0.825153374233129)(48,0.824521072796935)(49,0.825153374233129)(50,0.825153374233129)(51,0.825786646201074)(52,0.823259372609028)(53,0.82262996941896)(54,0.824521072796935)(55,0.825153374233129)(56,0.824521072796935)(57,0.827113480578827)(58,0.827743902439024)(59,0.82648401826484)(60,0.82648401826484)(61,0.825855513307985)(62,0.824867323730099)(63,0.823439878234399)(64,0.822813688212928)(65,0.822813688212928)(66,0.822813688212928)(67,0.8202416918429)(68,0.824695121951219)(69,0.824067022086824)(70,0.824695121951219)(71,0.824695121951219)(72,0.823798627002288)(73,0.823170731707317)(74,0.823170731707317)(75,0.822543792840822)(76,0.822543792840822)(77,0.822543792840822)(78,0.821917808219178)(79,0.823798627002288)(80,0.825057295645531)(81,0.824159021406728)(82,0.82542113323124)(83,0.829268292682927)(84,0.827743902439024)(85,0.827743902439024)(86,0.827743902439024)(87,0.827743902439024)(88,0.82837528604119)(89,0.82837528604119)(90,0.82837528604119)(91,0.827743902439024)(92,0.83015993907083)(93,0.829528158295281)(94,0.829528158295281)(95,0.826747720364742)(96,0.827638572513288)(97,0.82701062215478)(98,0.828267477203647)(99,0.828267477203647)(100,0.829900839054157)(101,0.831288343558282)(102,0.833078101071975)(103,0.829268292682927)(104,0.831804281345566)(105,0.832440703902066)(106,0.832696715049656)(107,0.830792682926829)(108,0.831426392067124)(109,0.830792682926829)(110,0.831426392067124)(111,0.831426392067124)(112,0.831426392067124)(113,0.831926323867997)(114,0.831926323867997)(115,0.829230769230769)(116,0.82859338970023)(117,0.82859338970023)(118,0.830130668716372)(119,0.830015313935681)(120,0.831426392067124)(121,0.830130668716372)(122,0.830130668716372)(123,0.832183908045977)(124,0.832183908045977)(125,0.83282208588957)(126,0.834609494640122)(127,0.834355828220859)(128,0.834355828220859)(129,0.834355828220859)(130,0.834996162701458)(131,0.835637480798771)(132,0.831426392067124)(133,0.829528158295281)(134,0.829528158295281)(135,0.832696715049656)(136,0.833970925784239)(137,0.833970925784239)(138,0.835249042145594)(139,0.834609494640122)(140,0.834609494640122)(141,0.834609494640122)(142,0.834609494640122)(143,0.83206106870229)(144,0.833333333333333)(145,0.833333333333333)(146,0.835889570552147)(147,0.833970925784239)(148,0.835249042145594)(149,0.83282208588957)(150,0.833716475095785)(151,0.835130970724191)(152,0.835130970724191)(153,0.835774865073246)(154,0.833461243284727)(155,0.832696715049656)(156,0.83206106870229)(157,0.83206106870229)(158,0.83015993907083)(159,0.831426392067124)(160,0.831426392067124)(161,0.831426392067124)(162,0.831426392067124)(163,0.830792682926829)(164,0.830792682926829)(165,0.830792682926829)(166,0.830792682926829)(167,0.828267477203647)(168,0.827638572513288)(169,0.833970925784239)(170,0.82701062215478)(171,0.82701062215478)(172,0.82701062215478)(173,0.83015993907083)(174,0.827638572513288)(175,0.827638572513288)(176,0.827638572513288)(177,0.828267477203647)(178,0.825757575757576)(179,0.827638572513288)(180,0.825757575757576)(181,0.825757575757576)(182,0.825757575757576)(183,0.825757575757576)(184,0.825757575757576)(185,0.822641509433962)(186,0.81893313298272)(187,0.815868263473054)(188,0.816479400749064)(189,0.817091454272864)(190,0.81893313298272)(191,0.816479400749064)(192,0.818318318318318)(193,0.815868263473054)(194,0.815868263473054)(195,0.817091454272864)(196,0.81525804038893)(197,0.81525804038893)(198,0.81525804038893)(199,0.81525804038893)(200,0.81525804038893)(201,0.814040328603435)(202,0.81525804038893)(203,0.81525804038893)(204,0.815868263473054)(205,0.817091454272864)(206,0.817091454272864)(207,0.821401657874906)(208,0.821401657874906)(209,0.820165537998495)(210,0.82078313253012)(211,0.82701062215478)(212,0.826383623957544)(213,0.825132475397426)(214,0.83206106870229)(215,0.832696715049656)(216,0.832696715049656)(217,0.833970925784239)(218,0.833970925784239)(219,0.83206106870229)(220,0.831426392067124)(221,0.830792682926829)(222,0.835249042145594)(223,0.835249042145594)(224,0.838461538461538)(225,0.833970925784239)(226,0.833333333333333)(227,0.837173579109063)(228,0.834609494640122)(229,0.835249042145594)(230,0.833333333333333)(231,0.835889570552147)(232,0.835889570552147)(233,0.835889570552147)(234,0.835249042145594)(235,0.835249042145594)(236,0.834609494640122)(237,0.834609494640122)(238,0.835249042145594)(239,0.835249042145594)(240,0.835249042145594)(241,0.835889570552147)(242,0.836531082118189)(243,0.837173579109063)(244,0.835249042145594)(245,0.836531082118189)(246,0.835889570552147)(247,0.838461538461538)(248,0.837173579109063)(249,0.837173579109063)(250,0.835889570552147)(251,0.837173579109063)(252,0.837173579109063)(253,0.835249042145594)(254,0.833333333333333)(255,0.833333333333333)(256,0.833333333333333)(257,0.833333333333333)(258,0.832696715049656)(259,0.832696715049656)(260,0.835249042145594)(261,0.835249042145594)(262,0.837173579109063)(263,0.837173579109063)(264,0.836531082118189)(265,0.835889570552147)(266,0.836531082118189)(267,0.835249042145594)(268,0.835249042145594)(269,0.833970925784239)(270,0.833333333333333)(271,0.83206106870229)(272,0.830792682926829)(273,0.831426392067124)(274,0.829528158295281)(275,0.828897338403042)(276,0.829528158295281)(277,0.829528158295281)(278,0.828897338403042)(279,0.83015993907083)(280,0.83015993907083)(281,0.831426392067124)(282,0.831426392067124)(283,0.832696715049656)(284,0.833970925784239)(285,0.832696715049656)(286,0.83206106870229)(287,0.83206106870229)(288,0.83206106870229)(289,0.829528158295281)(290,0.830792682926829)(291,0.83015993907083)(292,0.829528158295281)(293,0.829528158295281)(294,0.829528158295281)(295,0.829528158295281)(296,0.828267477203647)(297,0.828897338403042)(298,0.829528158295281)(299,0.83015993907083)(300,0.829528158295281)(301,0.829528158295281)(302,0.831426392067124)(303,0.83015993907083)(304,0.829528158295281)(305,0.830792682926829)(306,0.832696715049656)(307,0.827638572513288)(308,0.823885109599395)(309,0.823885109599395)(310,0.823885109599395)(311,0.824508320726172)(312,0.824508320726172)(313,0.824508320726172)(314,0.824508320726172)(315,0.828267477203647)(316,0.823885109599395)(317,0.825132475397426)(318,0.825132475397426)(319,0.825132475397426)(320,0.825132475397426)(321,0.825132475397426)(322,0.825132475397426)(323,0.825132475397426)(324,0.821401657874906)(325,0.823885109599395)(326,0.821401657874906)(327,0.820165537998495)(328,0.816479400749064)(329,0.816479400749064)(330,0.816479400749064)(331,0.816479400749064)(332,0.816479400749064)(333,0.816479400749064)(334,0.820165537998495)(335,0.820165537998495)(336,0.821401657874906)(337,0.822021116138763)(338,0.822021116138763)(339,0.82078313253012)(340,0.820165537998495)(341,0.819548872180451)(342,0.819548872180451)(343,0.819548872180451)(344,0.819548872180451)(345,0.819548872180451)(346,0.819548872180451)(347,0.819548872180451)(348,0.819548872180451)(349,0.819548872180451)(350,0.82078313253012)(351,0.822021116138763)(352,0.822021116138763)(353,0.822021116138763)(354,0.822641509433962)(355,0.822641509433962)(356,0.819548872180451)(357,0.82078313253012)(358,0.822641509433962)(359,0.824508320726172)(360,0.825757575757576)(361,0.825757575757576)(362,0.825757575757576)(363,0.826383623957544)(364,0.82701062215478)(365,0.825757575757576)(366,0.825757575757576)(367,0.825757575757576)(368,0.825757575757576)(369,0.825132475397426)(370,0.821401657874906)(371,0.819548872180451)(372,0.818318318318318)(373,0.818318318318318)(374,0.818318318318318)(375,0.816479400749064)(376,0.816479400749064)(377,0.815868263473054)(378,0.815868263473054)(379,0.815868263473054)(380,0.815868263473054)(381,0.814040328603435)(382,0.815868263473054)(383,0.815868263473054)(384,0.81525804038893)(385,0.81525804038893)(386,0.814648729446936)(387,0.81525804038893)(388,0.814648729446936)(389,0.815868263473054)(390,0.814648729446936)(391,0.815868263473054)(392,0.815868263473054)(393,0.815868263473054)(394,0.815868263473054)(395,0.816479400749064)(396,0.816479400749064)(397,0.817704426106526)(398,0.817704426106526)(399,0.817091454272864)(400,0.817091454272864)(1,0.509358288770053)(2,0.6)(3,0.597579425113464)(4,0.62582056892779)(5,0.661514683153014)(6,0.70864819479429)(7,0.715596330275229)(8,0.727272727272727)(9,0.806451612903226)(10,0.805832693783576)(11,0.811481768813033)(12,0.811481768813033)(13,0.808641975308642)(14,0.807692307692308)(15,0.819032761310452)(16,0.819032761310452)(17,0.819519752130132)(18,0.823346303501945)(19,0.828729281767956)(20,0.824823252160251)(21,0.823712948517941)(22,0.821596244131455)(23,0.8203125)(24,0.825471698113207)(25,0.825471698113207)(26,0.816136539953452)(27,0.822517591868647)(28,0.820031298904538)(29,0.817472698907956)(30,0.819953234606391)(31,0.821681068342498)(32,0.828346456692913)(33,0.829652996845426)(34,0.839584996009577)(35,0.840510366826156)(36,0.839328537170264)(37,0.839584996009577)(38,0.840510366826156)(39,0.836220472440945)(40,0.836477987421384)(41,0.837282780410742)(42,0.843277645186953)(43,0.843949044585987)(44,0.844868735083532)(45,0.845541401273885)(46,0.843949044585987)(47,0.843027888446215)(48,0.843700159489633)(49,0.84)(50,0.837881219903692)(51,0.833199033037873)(52,0.832258064516129)(53,0.832929782082324)(54,0.832929782082324)(55,0.832929782082324)(56,0.839486356340289)(57,0.837620578778135)(58,0.838813151563753)(59,0.840672538030424)(60,0.840672538030424)(61,0.837881219903692)(62,0.837881219903692)(63,0.837209302325581)(64,0.837881219903692)(65,0.834138486312399)(66,0.836947791164658)(67,0.836947791164658)(68,0.836947791164658)(69,0.838813151563753)(70,0.836012861736334)(71,0.835076427996782)(72,0.835076427996782)(73,0.838141025641025)(74,0.838141025641025)(75,0.836276083467095)(76,0.836276083467095)(77,0.8368)(78,0.837881219903692)(79,0.837881219903692)(80,0.83974358974359)(81,0.83974358974359)(82,0.83974358974359)(83,0.83974358974359)(84,0.8432)(85,0.839486356340289)(86,0.839486356340289)(87,0.8416)(88,0.8416)(89,0.8416)(90,0.840417000801925)(91,0.840417000801925)(92,0.839486356340289)(93,0.839486356340289)(94,0.83974358974359)(95,0.840927258193445)(96,0.842105263157895)(97,0.84352660841938)(98,0.843277645186953)(99,0.843027888446215)(100,0.842607313195548)(101,0.842607313195548)(102,0.842607313195548)(103,0.842356687898089)(104,0.842356687898089)(105,0.840510366826156)(106,0.842356687898089)(107,0.843700159489633)(108,0.843277645186953)(109,0.842607313195548)(110,0.841686555290374)(111,0.841686555290374)(112,0.841686555290374)(113,0.841686555290374)(114,0.841686555290374)(115,0.842356687898089)(116,0.842687747035573)(117,0.840694006309148)(118,0.842022116903633)(119,0.842022116903633)(120,0.842687747035573)(121,0.841772151898734)(122,0.840189873417721)(123,0.842857142857143)(124,0.842857142857143)(125,0.842607313195548)(126,0.842607313195548)(127,0.842607313195548)(128,0.84310618066561)(129,0.842607313195548)(130,0.842607313195548)(131,0.842607313195548)(132,0.842607313195548)(133,0.840927258193445)(134,0.8416)(135,0.8416)(136,0.8416)(137,0.840510366826156)(138,0.840510366826156)(139,0.841181165203511)(140,0.841181165203511)(141,0.841181165203511)(142,0.842105263157895)(143,0.842105263157895)(144,0.840764331210191)(145,0.84185303514377)(146,0.840927258193445)(147,0.84185303514377)(148,0.84185303514377)(149,0.84185303514377)(150,0.842777334397446)(151,0.842607313195548)(152,0.842777334397446)(153,0.843700159489633)(154,0.843027888446215)(155,0.844197138314785)(156,0.844197138314785)(157,0.845541401273885)(158,0.846946867565424)(159,0.846946867565424)(160,0.846946867565424)(161,0.844759653270291)(162,0.844759653270291)(163,0.846518987341772)(164,0.847860538827258)(165,0.843921568627451)(166,0.843014128728414)(167,0.843014128728414)(168,0.843014128728414)(169,0.843260188087774)(170,0.84458398744113)(171,0.84458398744113)(172,0.839563862928349)(173,0.835658914728682)(174,0.829869130100077)(175,0.82859338970023)(176,0.833075734157651)(177,0.829230769230769)(178,0.82859338970023)(179,0.827956989247312)(180,0.834365325077399)(181,0.836052836052836)(182,0.835403726708074)(183,0.836052836052836)(184,0.837354085603113)(185,0.837354085603113)(186,0.837354085603113)(187,0.838659392049883)(188,0.838006230529595)(189,0.839313572542902)(190,0.836702954898911)(191,0.837354085603113)(192,0.837354085603113)(193,0.837354085603113)(194,0.837354085603113)(195,0.837354085603113)(196,0.836052836052836)(197,0.836052836052836)(198,0.832432432432432)(199,0.831530139103555)(200,0.832173240525909)(201,0.832173240525909)(202,0.841282251759187)(203,0.840625)(204,0.832173240525909)(205,0.831530139103555)(206,0.831530139103555)(207,0.8328173374613)(208,0.836702954898911)(209,0.836702954898911)(210,0.847911741528763)(211,0.849206349206349)(212,0.849206349206349)(213,0.850118953211737)(214,0.849880857823669)(215,0.849880857823669)(216,0.851030110935024)(217,0.850793650793651)(218,0.849880857823669)(219,0.849402390438247)(220,0.849642004773269)(221,0.848484848484848)(222,0.851704996034893)(223,0.849880857823669)(224,0.849642004773269)(225,0.849880857823669)(226,0.849921011058452)(227,0.849921011058452)(228,0.851704996034893)(229,0.849880857823669)(230,0.851469420174742)(231,0.851469420174742)(232,0.851469420174742)(233,0.851233094669849)(234,0.850556438791733)(235,0.851233094669849)(236,0.851233094669849)(237,0.851469420174742)(238,0.851469420174742)(239,0.851233094669849)(240,0.851469420174742)(241,0.851469420174742)(242,0.851469420174742)(243,0.851469420174742)(244,0.851233094669849)(245,0.850079744816587)(246,0.849162011173184)(247,0.849162011173184)(248,0.850996015936255)(249,0.851233094669849)(250,0.851233094669849)(251,0.85214626391097)(252,0.852380952380952)(253,0.852380952380952)(254,0.852380952380952)(255,0.852380952380952)(256,0.851469420174742)(257,0.852614896988906)(258,0.846577498033045)(259,0.847244094488189)(260,0.847244094488189)(261,0.847244094488189)(262,0.848580441640378)(263,0.850592885375494)(264,0.850556438791733)(265,0.850556438791733)(266,0.849880857823669)(267,0.851233094669849)(268,0.851233094669849)(269,0.851437699680511)(270,0.852118305355715)(271,0.852118305355715)(272,0.852118305355715)(273,0.850079744816587)(274,0.850519584332534)(275,0.851437699680511)(276,0.851437699680511)(277,0.851437699680511)(278,0.851437699680511)(279,0.851437699680511)(280,0.851437699680511)(281,0.851437699680511)(282,0.851437699680511)(283,0.851437699680511)(284,0.851437699680511)(285,0.851437699680511)(286,0.852354349561053)(287,0.852354349561053)(288,0.852354349561053)(289,0.852354349561053)(290,0.852354349561053)(291,0.852354349561053)(292,0.852354349561053)(293,0.852354349561053)(294,0.852354349561053)(295,0.851674641148325)(296,0.851674641148325)(297,0.85031847133758)(298,0.851233094669849)(299,0.85214626391097)(300,0.85214626391097)(301,0.852380952380952)(302,0.849724626278521)(303,0.845732184808144)(304,0.848389630793401)(305,0.84901185770751)(306,0.84901185770751)(307,0.850828729281768)(308,0.849487785657998)(309,0.849487785657998)(310,0.848580441640378)(311,0.849921011058452)(312,0.850592885375494)(313,0.848772763262074)(314,0.850592885375494)(315,0.849683544303797)(316,0.848772763262074)(317,0.849683544303797)(318,0.849683544303797)(319,0.848772763262074)(320,0.848772763262074)(321,0.850118953211737)(322,0.850118953211737)(323,0.849683544303797)(324,0.849683544303797)(325,0.85214626391097)(326,0.849683544303797)(327,0.849683544303797)(328,0.849683544303797)(329,0.849921011058452)(330,0.849250197316495)(331,0.848580441640378)(332,0.849487785657998)(333,0.849683544303797)(334,0.849250197316495)(335,0.848580441640378)(336,0.848341232227488)(337,0.848341232227488)(338,0.849683544303797)(339,0.849683544303797)(340,0.849250197316495)(341,0.848580441640378)(342,0.849250197316495)(343,0.849250197316495)(344,0.849250197316495)(345,0.84901185770751)(346,0.84901185770751)(347,0.850356294536817)(348,0.848151062155783)(349,0.847962382445141)(350,0.847058823529412)(351,0.848151062155783)(352,0.848151062155783)(353,0.848151062155783)(354,0.848151062155783)(355,0.848151062155783)(356,0.848151062155783)(357,0.848818897637795)(358,0.848151062155783)(359,0.849056603773585)(360,0.847723704866562)(361,0.847537138389367)(362,0.848200312989045)(363,0.848200312989045)(364,0.847537138389367)(365,0.847537138389367)(366,0.847537138389367)(367,0.849100860046912)(368,0.847537138389367)(369,0.848864526233359)(370,0.848864526233359)(371,0.848864526233359)(372,0.847962382445141)(373,0.847962382445141)(374,0.847962382445141)(375,0.847962382445141)(376,0.847962382445141)(377,0.843822843822844)(378,0.843822843822844)(379,0.843822843822844)(380,0.843822843822844)(381,0.842513576415826)(382,0.843822843822844)(383,0.843822843822844)(384,0.843822843822844)(385,0.84472049689441)(386,0.842105263157895)(387,0.842105263157895)(388,0.845136186770428)(389,0.845136186770428)(390,0.84447900466563)(391,0.845136186770428)(392,0.846692607003891)(393,0.846453624318005)(394,0.845794392523364)(395,0.846034214618973)(396,0.846034214618973)(397,0.846034214618973)(398,0.848864526233359)(399,0.848627450980392)(400,0.848627450980392)(1,0.604904632152588)(2,0.588081204977079)(3,0.62375)(4,0.653794940079893)(5,0.722141823444284)(6,0.73167044595616)(7,0.738700564971751)(8,0.77845220030349)(9,0.771407297096053)(10,0.775239498894621)(11,0.776141384388807)(12,0.764619883040936)(13,0.769117647058823)(14,0.777117384843982)(15,0.788519637462236)(16,0.789433962264151)(17,0.817270624518119)(18,0.814241486068111)(19,0.8140625)(20,0.81767955801105)(21,0.819542947202522)(22,0.819466248037676)(23,0.818823529411765)(24,0.819444444444444)(25,0.821705426356589)(26,0.825)(27,0.819107282693814)(28,0.820592823712948)(29,0.824726134585289)(30,0.8265625)(31,0.826356589147287)(32,0.824167312161115)(33,0.826833073322933)(34,0.831636648394675)(35,0.831899921813917)(36,0.828015564202335)(37,0.828015564202335)(38,0.82753164556962)(39,0.830963665086888)(40,0.830963665086888)(41,0.830963665086888)(42,0.831353919239905)(43,0.832278481012658)(44,0.834782608695652)(45,0.835443037974684)(46,0.833727344365642)(47,0.832807570977918)(48,0.835043409629045)(49,0.833464877663773)(50,0.832807570977918)(51,0.834645669291339)(52,0.833594976452119)(53,0.833594976452119)(54,0.830841856805665)(55,0.829536527886881)(56,0.832678711704635)(57,0.830188679245283)(58,0.832025117739403)(59,0.831899921813917)(60,0.83125)(61,0.83476898981989)(62,0.8328125)(63,0.8328125)(64,0.831899921813917)(65,0.838455476753349)(66,0.834509803921568)(67,0.832941176470588)(68,0.832941176470588)(69,0.833855799373041)(70,0.833594976452119)(71,0.833594976452119)(72,0.832941176470588)(73,0.834115805946792)(74,0.834115805946792)(75,0.8328125)(76,0.833463643471462)(77,0.83476898981989)(78,0.833850931677019)(79,0.832558139534884)(80,0.831913245546088)(81,0.831530139103555)(82,0.832173240525909)(83,0.832173240525909)(84,0.8328173374613)(85,0.8328173374613)(86,0.833462432223083)(87,0.833462432223083)(88,0.834108527131783)(89,0.834108527131783)(90,0.835403726708074)(91,0.836052836052836)(92,0.835403726708074)(93,0.835403726708074)(94,0.835658914728682)(95,0.835658914728682)(96,0.835913312693498)(97,0.836560805577072)(98,0.837209302325581)(99,0.837209302325581)(100,0.833333333333333)(101,0.833333333333333)(102,0.832690824980725)(103,0.832690824980725)(104,0.833333333333333)(105,0.825688073394495)(106,0.825688073394495)(107,0.825688073394495)(108,0.825057295645531)(109,0.826952526799387)(110,0.826952526799387)(111,0.826319816373374)(112,0.827743902439024)(113,0.826952526799387)(114,0.828220858895705)(115,0.828220858895705)(116,0.828220858895705)(117,0.827586206896552)(118,0.828220858895705)(119,0.82911877394636)(120,0.830910482019893)(121,0.830275229357798)(122,0.82911877394636)(123,0.829007633587786)(124,0.830910482019893)(125,0.830910482019893)(126,0.829640947288006)(127,0.830275229357798)(128,0.828483920367534)(129,0.828483920367534)(130,0.82911877394636)(131,0.829007633587786)(132,0.82837528604119)(133,0.827113480578827)(134,0.825227963525836)(135,0.825855513307985)(136,0.824242424242424)(137,0.821752265861027)(138,0.80920564216778)(139,0.816479400749064)(140,0.819894498869631)(141,0.816816816816817)(142,0.818045112781955)(143,0.818045112781955)(144,0.818660647103085)(145,0.818660647103085)(146,0.817430503380917)(147,0.818045112781955)(148,0.818045112781955)(149,0.814648729446936)(150,0.814040328603435)(151,0.812220566318927)(152,0.816479400749064)(153,0.814648729446936)(154,0.815868263473054)(155,0.816479400749064)(156,0.815592203898051)(157,0.817430503380917)(158,0.816204051012753)(159,0.816204051012753)(160,0.816204051012753)(161,0.818045112781955)(162,0.824961948249619)(163,0.825590251332825)(164,0.825057295645531)(165,0.823170731707317)(166,0.823798627002288)(167,0.822458270106221)(168,0.821752265861027)(169,0.822104466313399)(170,0.823082763857251)(171,0.821752265861027)(172,0.820861678004535)(173,0.81282624906786)(174,0.814040328603435)(175,0.814040328603435)(176,0.814040328603435)(177,0.814040328603435)(178,0.814040328603435)(179,0.814040328603435)(180,0.815592203898051)(181,0.82051282051282)(182,0.82051282051282)(183,0.821132075471698)(184,0.821132075471698)(185,0.821132075471698)(186,0.821132075471698)(187,0.821132075471698)(188,0.825590251332825)(189,0.824334600760456)(190,0.825590251332825)(191,0.824961948249619)(192,0.825590251332825)(193,0.827850038255547)(194,0.827850038255547)(195,0.827850038255547)(196,0.82874617737003)(197,0.827850038255547)(198,0.829007633587786)(199,0.830391404451266)(200,0.829268292682927)(201,0.829900839054157)(202,0.828006088280061)(203,0.822641509433962)(204,0.822641509433962)(205,0.823262839879154)(206,0.823262839879154)(207,0.823262839879154)(208,0.823262839879154)(209,0.823262839879154)(210,0.828006088280061)(211,0.831546707503828)(212,0.831168831168831)(213,0.831168831168831)(214,0.837712519319938)(215,0.837712519319938)(216,0.84031007751938)(217,0.84031007751938)(218,0.838759689922481)(219,0.845490196078431)(220,0.844827586206896)(221,0.850157728706624)(222,0.850157728706624)(223,0.849487785657998)(224,0.850157728706624)(225,0.852173913043478)(226,0.852173913043478)(227,0.852173913043478)(228,0.852173913043478)(229,0.852173913043478)(230,0.850828729281768)(231,0.851500789889415)(232,0.851500789889415)(233,0.851500789889415)(234,0.852173913043478)(235,0.851500789889415)(236,0.85126582278481)(237,0.852173913043478)(238,0.852173913043478)(239,0.852173913043478)(240,0.849293563579278)(241,0.847298355520752)(242,0.848151062155783)(243,0.851500789889415)(244,0.853523357086302)(245,0.850828729281768)(246,0.849293563579278)(247,0.849960722702278)(248,0.851298190401259)(249,0.851298190401259)(250,0.851298190401259)(251,0.851298190401259)(252,0.85062893081761)(253,0.85062893081761)(254,0.85263987391647)(255,0.851968503937008)(256,0.851531814611155)(257,0.853543307086614)(258,0.854215918045705)(259,0.854215918045705)(260,0.853543307086614)(261,0.851469420174742)(262,0.851469420174742)(263,0.851469420174742)(264,0.851469420174742)(265,0.851469420174742)(266,0.851469420174742)(267,0.851469420174742)(268,0.854215918045705)(269,0.854215918045705)(270,0.853543307086614)(271,0.853543307086614)(272,0.85377358490566)(273,0.85243328100471)(274,0.853563038371182)(275,0.854901960784314)(276,0.854901960784314)(277,0.85423197492163)(278,0.852895148669796)(279,0.85423197492163)(280,0.852895148669796)(281,0.85423197492163)(282,0.855791962174941)(283,0.85511811023622)(284,0.85511811023622)(285,0.85511811023622)(286,0.854199683042789)(287,0.854199683042789)(288,0.854199683042789)(289,0.852824184566428)(290,0.852354349561053)(291,0.852354349561053)(292,0.854183266932271)(293,0.853269537480064)(294,0.849840255591054)(295,0.853736089030207)(296,0.853736089030207)(297,0.853736089030207)(298,0.852824184566428)(299,0.851910828025478)(300,0.851910828025478)(301,0.844051446945338)(302,0.844979919678715)(303,0.844051446945338)(304,0.848)(305,0.848)(306,0.848)(307,0.851674641148325)(308,0.849840255591054)(309,0.849840255591054)(310,0.8496)(311,0.846832397754611)(312,0.84775641025641)(313,0.84775641025641)(314,0.846832397754611)(315,0.846832397754611)(316,0.845906902086677)(317,0.845906902086677)(318,0.844979919678715)(319,0.84473049074819)(320,0.84473049074819)(321,0.84473049074819)(322,0.84473049074819)(323,0.84473049074819)(324,0.84473049074819)(325,0.84473049074819)(326,0.841935483870968)(327,0.840064620355412)(328,0.838187702265372)(329,0.838187702265372)(330,0.838187702265372)(331,0.838187702265372)(332,0.838187702265372)(333,0.837246963562753)(334,0.837246963562753)(335,0.841000807102502)(336,0.848)(337,0.848)(338,0.848)(339,0.847077662129704)(340,0.847077662129704)(341,0.847077662129704)(342,0.847077662129704)(343,0.847077662129704)(344,0.848)(345,0.848)(346,0.846832397754611)(347,0.845659163987138)(348,0.843800322061191)(349,0.843800322061191)(350,0.842868654311039)(351,0.842868654311039)(352,0.842868654311039)(353,0.843800322061191)(354,0.843800322061191)(355,0.842868654311039)(356,0.84473049074819)(357,0.843800322061191)(358,0.843800322061191)(359,0.843800322061191)(360,0.843800322061191)(361,0.843800322061191)(362,0.843800322061191)(363,0.842868654311039)(364,0.842868654311039)(365,0.842868654311039)(366,0.843800322061191)(367,0.843800322061191)(368,0.845659163987138)(369,0.845659163987138)(370,0.849358974358974)(371,0.850280224179343)(372,0.8512)(373,0.849358974358974)(374,0.849358974358974)(375,0.849358974358974)(376,0.848436246992783)(377,0.848436246992783)(378,0.848436246992783)(379,0.849358974358974)(380,0.850280224179343)(381,0.8512)(382,0.850280224179343)(383,0.849358974358974)(384,0.854864433811802)(385,0.853950518754988)(386,0.853950518754988)(387,0.853035143769968)(388,0.8512)(389,0.8512)(390,0.8512)(391,0.8512)(392,0.848436246992783)(393,0.848436246992783)(394,0.850280224179343)(395,0.850280224179343)(396,0.850280224179343)(397,0.8512)(398,0.8512)(399,0.8512)(400,0.8512)(7,0.614705882352941)(8,0.655195234943746)(9,0.743954480796586)(10,0.711864406779661)(11,0.724878724878725)(12,0.707792207792208)(13,0.711952971913782)(14,0.743677375256322)(15,0.744695414099931)(16,0.771979985704074)(17,0.777618364418938)(18,0.774839170836311)(19,0.769886363636363)(20,0.781227436823105)(21,0.785868781542898)(22,0.786435786435786)(23,0.784172661870504)(24,0.784172661870504)(25,0.784737221022318)(26,0.779127948534667)(27,0.776906628652887)(28,0.78080229226361)(29,0.77524893314367)(30,0.774147727272727)(31,0.773598296664301)(32,0.774697938877043)(33,0.780243378668575)(34,0.78080229226361)(35,0.777460770328103)(36,0.776906628652887)(37,0.775800711743772)(38,0.776353276353276)(39,0.776906628652887)(40,0.779127948534667)(41,0.778015703069236)(42,0.774697938877043)(43,0.785302593659942)(44,0.785868781542898)(45,0.785302593659942)(46,0.788671023965142)(47,0.783861671469741)(48,0.784992784992785)(49,0.784426820475847)(50,0.78726483357453)(51,0.78726483357453)(52,0.785559566787004)(53,0.786127167630058)(54,0.78726483357453)(55,0.78726483357453)(56,0.786127167630058)(57,0.785868781542898)(58,0.787003610108303)(59,0.783608914450036)(60,0.781922525107604)(61,0.785302593659942)(62,0.785302593659942)(63,0.784737221022318)(64,0.785868781542898)(65,0.785302593659942)(66,0.785302593659942)(67,0.785249457700651)(68,0.787003610108303)(69,0.786435786435786)(70,0.785868781542898)(71,0.785868781542898)(72,0.785868781542898)(73,0.785868781542898)(74,0.785868781542898)(75,0.785868781542898)(76,0.787003610108303)(77,0.786435786435786)(78,0.78757225433526)(79,0.78757225433526)(80,0.788712011577424)(81,0.789283128167994)(82,0.789855072463768)(83,0.786435786435786)(84,0.787003610108303)(85,0.786435786435786)(86,0.786435786435786)(87,0.785302593659942)(88,0.785868781542898)(89,0.783608914450036)(90,0.784172661870504)(91,0.783608914450036)(92,0.78080229226361)(93,0.781922525107604)(94,0.784737221022318)(95,0.784172661870504)(96,0.784172661870504)(97,0.785868781542898)(98,0.785868781542898)(99,0.78080229226361)(100,0.779685264663805)(101,0.774697938877043)(102,0.774697938877043)(103,0.774697938877043)(104,0.774697938877043)(105,0.773598296664301)(106,0.773598296664301)(107,0.773598296664301)(108,0.774147727272727)(109,0.774147727272727)(110,0.774697938877043)(111,0.774147727272727)(112,0.774147727272727)(113,0.774147727272727)(114,0.774147727272727)(115,0.760111576011157)(116,0.758524704244955)(117,0.767605633802817)(118,0.76814658210007)(119,0.762771168649405)(120,0.762771168649405)(121,0.76814658210007)(122,0.768688293370945)(123,0.763305322128851)(124,0.761173184357542)(125,0.735311932162326)(126,0.740923076923077)(127,0.742368742368742)(128,0.737091988130564)(129,0.757248436611711)(130,0.762225969645868)(131,0.751101321585903)(132,0.760579064587973)(133,0.767847105115233)(134,0.767688022284122)(135,0.768299394606494)(136,0.767454645409565)(137,0.754045307443366)(138,0.76797385620915)(139,0.769230769230769)(140,0.762312633832976)(141,0.767491926803014)(142,0.76625470177324)(143,0.761146496815286)(144,0.755274261603375)(145,0.756471209720021)(146,0.763326226012793)(147,0.758474576271186)(148,0.757271285034373)(149,0.756329113924051)(150,0.759533898305085)(151,0.757127771911299)(152,0.757928118393235)(153,0.761146496815286)(154,0.761955366631243)(155,0.757928118393235)(156,0.760742705570292)(157,0.762765957446808)(158,0.761955366631243)(159,0.762360446570973)(160,0.75993640699523)(161,0.757527733755943)(162,0.757928118393235)(163,0.758328926493918)(164,0.757928118393235)(165,0.755532139093783)(166,0.753151260504202)(167,0.752360965372508)(168,0.753943217665615)(169,0.752360965372508)(170,0.753547031003678)(171,0.756329113924051)(172,0.757127771911299)(173,0.756728232189973)(174,0.755930416447022)(175,0.755134281200632)(176,0.754339821146765)(177,0.754339821146765)(178,0.753547031003678)(179,0.753943217665615)(180,0.753943217665615)(181,0.750392464678179)(182,0.749607945635128)(183,0.749607945635128)(184,0.749607945635128)(185,0.749607945635128)(186,0.75)(187,0.749216300940439)(188,0.747653806047967)(189,0.747264200104221)(190,0.748043818466354)(191,0.74648620510151)(192,0.746875)(193,0.74648620510151)(194,0.752755905511811)(195,0.753151260504202)(196,0.752360965372508)(197,0.752360965372508)(198,0.751966439433665)(199,0.753151260504202)(200,0.753151260504202)(201,0.751966439433665)(202,0.753547031003678)(203,0.752360965372508)(204,0.752360965372508)(205,0.751966439433665)(206,0.751966439433665)(207,0.751966439433665)(208,0.752360965372508)(209,0.752360965372508)(210,0.749216300940439)(211,0.751572327044025)(212,0.749216300940439)(213,0.749216300940439)(214,0.749607945635128)(215,0.746875)(216,0.74648620510151)(217,0.744548286604361)(218,0.745322245322245)(219,0.746097814776275)(220,0.746097814776275)(221,0.746097814776275)(222,0.750392464678179)(223,0.752755905511811)(224,0.751178627553693)(225,0.747653806047967)(226,0.747653806047967)(227,0.754736842105263)(228,0.756728232189973)(229,0.758730158730159)(230,0.75993640699523)(231,0.761550716941051)(232,0.75993640699523)(233,0.76033934252386)(234,0.758730158730159)(235,0.761146496815286)(236,0.761550716941051)(237,0.762360446570973)(238,0.762360446570973)(239,0.761955366631243)(240,0.762765957446808)(241,0.76317189994678)(242,0.763578274760383)(243,0.763578274760383)(244,0.764392324093817)(245,0.7648)(246,0.76317189994678)(247,0.762360446570973)(248,0.762765957446808)(249,0.761955366631243)(250,0.761955366631243)(251,0.762360446570973)(252,0.762765957446808)(253,0.761955366631243)(254,0.761955366631243)(255,0.76317189994678)(256,0.76317189994678)(257,0.763578274760383)(258,0.763578274760383)(259,0.765208110992529)(260,0.766844919786096)(261,0.765208110992529)(262,0.766435061464457)(263,0.766025641025641)(264,0.766025641025641)(265,0.765208110992529)(266,0.763985082578583)(267,0.762765957446808)(268,0.762765957446808)(269,0.762360446570973)(270,0.762360446570973)(271,0.768488745980707)(272,0.774298056155507)(273,0.773462783171521)(274,0.774716369529984)(275,0.773880194279546)(276,0.779347826086956)(277,0.780620577027763)(278,0.784463894967177)(279,0.783606557377049)(280,0.785753424657534)(281,0.784463894967177)(282,0.783606557377049)(283,0.783606557377049)(284,0.784034991798797)(285,0.784463894967177)(286,0.784463894967177)(287,0.784463894967177)(288,0.785323110624315)(289,0.785323110624315)(290,0.783178590933916)(291,0.783606557377049)(292,0.784893267651888)(293,0.784893267651888)(294,0.784893267651888)(295,0.784463894967177)(296,0.784463894967177)(297,0.797552836484983)(298,0.796666666666667)(299,0.796666666666667)(300,0.796666666666667)(301,0.796224319822321)(302,0.796666666666667)(303,0.796666666666667)(304,0.795782463928968)(305,0.795341098169717)(306,0.795782463928968)(307,0.802911534154535)(308,0.802013422818792)(309,0.802462227196418)(310,0.802462227196418)(311,0.806070826306914)(312,0.806070826306914)(313,0.805165637282426)(314,0.805165637282426)(315,0.805165637282426)(316,0.805165637282426)(317,0.805165637282426)(318,0.805165637282426)(319,0.816628701594533)(320,0.817094017094017)(321,0.817094017094017)(322,0.817094017094017)(323,0.817094017094017)(324,0.817094017094017)(325,0.817559863169897)(326,0.819428571428571)(327,0.820366132723112)(328,0.818960593946316)(329,0.818960593946316)(330,0.819428571428571)(331,0.819428571428571)(332,0.818960593946316)(333,0.818960593946316)(334,0.818960593946316)(335,0.818960593946316)(336,0.818960593946316)(337,0.817559863169897)(338,0.817559863169897)(339,0.818493150684931)(340,0.819428571428571)(341,0.819897084048027)(342,0.819428571428571)(343,0.819428571428571)(344,0.823191733639495)(345,0.823191733639495)(346,0.823664560597358)(347,0.823191733639495)(348,0.818026240730177)(349,0.818493150684931)(350,0.827944572748268)(351,0.827466820542412)(352,0.831786542923434)(353,0.832269297736506)(354,0.833720930232558)(355,0.833236490412551)(356,0.829381145170619)(357,0.829381145170619)(358,0.830341632889403)(359,0.833236490412551)(360,0.833720930232558)(361,0.8351776354106)(362,0.836151603498542)(363,0.835664335664336)(364,0.834691501746216)(365,0.834205933682373)(366,0.834691501746216)(367,0.834691501746216)(368,0.834691501746216)(369,0.834691501746216)(370,0.849526066350711)(371,0.849526066350711)(372,0.850029638411381)(373,0.851543942992874)(374,0.850533807829181)(375,0.850029638411381)(376,0.849526066350711)(377,0.848018923713779)(378,0.843529411764706)(379,0.842043452730476)(380,0.841055718475073)(381,0.84452296819788)(382,0.84452296819788)(383,0.84452296819788)(384,0.84452296819788)(385,0.84452296819788)(386,0.84452296819788)(387,0.845020624631703)(388,0.84452296819788)(389,0.84452296819788)(390,0.847517730496454)(391,0.847017129356172)(392,0.841549295774648)(393,0.847017129356172)(394,0.847517730496454)(395,0.846517119244392)(396,0.846517119244392)(397,0.847017129356172)(398,0.847017129356172)(399,0.847017129356172)(400,0.848018923713779)(9,0.768341998844598)(10,0.774703557312253)(11,0.778353483016695)(12,0.796048808832074)(13,0.804462712859659)(14,0.801162790697674)(15,0.796992481203007)(16,0.824249846907532)(17,0.827586206896552)(18,0.823457544288332)(19,0.828096118299445)(20,0.819001803968731)(21,0.820760410380205)(22,0.807737397420867)(23,0.811065332548558)(24,0.814946619217082)(25,0.79399884593191)(26,0.797219003476246)(27,0.798837209302325)(28,0.799767846778874)(29,0.815676668707899)(30,0.832189644416719)(31,0.833743842364532)(32,0.817422434367542)(33,0.82501503307276)(34,0.816617210682492)(35,0.817804154302671)(36,0.809384164222874)(37,0.809384164222874)(38,0.811221507890123)(39,0.800461361014994)(40,0.798619102416571)(41,0.799539170506912)(42,0.823250296559905)(43,0.82865671641791)(44,0.839220462850183)(45,0.842169408897014)(46,0.842682926829268)(47,0.842821782178218)(48,0.842821782178218)(49,0.838153846153846)(50,0.835070508890251)(51,0.829355608591885)(52,0.832335329341317)(53,0.833333333333333)(54,0.834834834834835)(55,0.837658418829209)(56,0.835748792270531)(57,0.836451418225709)(58,0.84)(59,0.842553191489362)(60,0.841530054644809)(61,0.841530054644809)(62,0.845121951219512)(63,0.841530054644809)(64,0.841530054644809)(65,0.838983050847458)(66,0.837968561064087)(67,0.841530054644809)(68,0.845597104945718)(69,0.84629294755877)(70,0.846987951807229)(71,0.843243243243243)(72,0.866831072749692)(73,0.870967741935484)(74,0.870967741935484)(75,0.868388683886839)(76,0.867321867321867)(77,0.863080684596577)(78,0.867692307692308)(79,0.868226600985222)(80,0.868226600985222)(81,0.868017188459177)(82,0.86589099816289)(83,0.849462365591398)(84,0.848955223880597)(85,0.86605504587156)(86,0.867647058823529)(87,0.862362971985384)(88,0.864996945632254)(89,0.864996945632254)(90,0.863414634146341)(91,0.863941427699817)(92,0.864634146341463)(93,0.864634146341463)(94,0.864634146341463)(95,0.863941427699817)(96,0.865689865689866)(97,0.862530413625304)(98,0.864634146341463)(99,0.864107251675807)(100,0.864996945632254)(101,0.865525672371638)(102,0.862864077669903)(103,0.864963503649635)(104,0.866017052375152)(105,0.86581663630844)(106,0.86319612590799)(107,0.862507571168988)(108,0.863030303030303)(109,0.864077669902913)(110,0.863553668890236)(111,0.859903381642512)(112,0.864602307225258)(113,0.863553668890236)(114,0.863553668890236)(115,0.864077669902913)(116,0.863553668890236)(117,0.863553668890236)(118,0.863553668890236)(119,0.866910866910867)(120,0.867278287461774)(121,0.866748166259169)(122,0.865525672371638)(123,0.869141813755326)(124,0.869300911854103)(125,0.870731707317073)(126,0.870731707317073)(127,0.870731707317073)(128,0.871419865935405)(129,0.871419865935405)(130,0.875230485556238)(131,0.876073619631902)(132,0.875536480686695)(133,0.878769230769231)(134,0.883116883116883)(135,0.884758364312268)(136,0.88695652173913)(137,0.890829694323144)(138,0.891942535915053)(139,0.890829694323144)(140,0.890274314214464)(141,0.893617021276596)(142,0.893058161350844)(143,0.894736842105263)(144,0.894736842105263)(145,0.894736842105263)(146,0.896421845574388)(147,0.896551724137931)(148,0.89711417816813)(149,0.896551724137931)(150,0.895989974937343)(151,0.895428929242329)(152,0.894308943089431)(153,0.894308943089431)(154,0.896551724137931)(155,0.904369854338189)(156,0.900821225521162)(157,0.900821225521162)(158,0.900821225521162)(159,0.901390644753477)(160,0.900252525252525)(161,0.899684542586751)(162,0.901390644753477)(163,0.901390644753477)(164,0.90253164556962)(165,0.903675538656527)(166,0.90482233502538)(167,0.905396825396825)(168,0.906547997457088)(169,0.905972045743329)(170,0.906091370558375)(171,0.904369854338189)(172,0.905516804058339)(173,0.904942965779468)(174,0.904942965779468)(175,0.904369854338189)(176,0.907819453273999)(177,0.916344916344916)(178,0.919896640826873)(179,0.919896640826873)(180,0.924075275794938)(181,0.921290322580645)(182,0.921885087153002)(183,0.921290322580645)(184,0.920103092783505)(185,0.923076923076923)(186,0.917148362235067)(187,0.915384615384615)(188,0.915971776779987)(189,0.916559691912708)(190,0.915384615384615)(191,0.917148362235067)(192,0.915384615384615)(193,0.914798206278027)(194,0.915971776779987)(195,0.915971776779987)(196,0.914798206278027)(197,0.912460063897764)(198,0.912571793235482)(199,0.912571793235482)(200,0.912571793235482)(201,0.910828025477707)(202,0.909669211195929)(203,0.910248249522597)(204,0.910248249522597)(205,0.90736040609137)(206,0.909669211195929)(207,0.910248249522597)(208,0.90736040609137)(209,0.90736040609137)(210,0.906210392902408)(211,0.90506329113924)(212,0.906785034876347)(213,0.90851334180432)(214,0.908396946564885)(215,0.907936507936508)(216,0.906210392902408)(217,0.906210392902408)(218,0.90736040609137)(219,0.90736040609137)(220,0.911408540471638)(221,0.912571793235482)(222,0.912460063897764)(223,0.915863840719332)(224,0.916452442159383)(225,0.917041800643087)(226,0.917631917631918)(227,0.918222794591114)(228,0.918222794591114)(229,0.918222794591114)(230,0.917041800643087)(231,0.918814432989691)(232,0.918222794591114)(233,0.919406834300451)(234,0.918222794591114)(235,0.917525773195876)(236,0.918222794591114)(237,0.918222794591114)(238,0.918814432989691)(239,0.923679060665362)(240,0.923076923076923)(241,0.924885695623775)(242,0.923579359895493)(243,0.925294888597641)(244,0.927403531720078)(245,0.927403531720078)(246,0.925490196078431)(247,0.924281984334204)(248,0.92217135382603)(249,0.921671018276762)(250,0.923479398299542)(251,0.923479398299542)(252,0.923479398299542)(253,0.923376623376623)(254,0.923376623376623)(255,0.924281984334204)(256,0.924281984334204)(257,0.921275211450878)(258,0.919093851132686)(259,0.91849935316947)(260,0.919093851132686)(261,0.919093851132686)(262,0.917312661498708)(263,0.91849935316947)(264,0.917312661498708)(265,0.920676202860858)(266,0.920676202860858)(267,0.921275211450878)(268,0.921275211450878)(269,0.919480519480519)(270,0.919480519480519)(271,0.914838709677419)(272,0.912959381044487)(273,0.912959381044487)(274,0.913548387096774)(275,0.913548387096774)(276,0.913548387096774)(277,0.913548387096774)(278,0.913548387096774)(279,0.913548387096774)(280,0.914138153647514)(281,0.914728682170542)(282,0.907584448693435)(283,0.908742820676452)(284,0.908163265306122)(285,0.908163265306122)(286,0.908163265306122)(287,0.909323116219668)(288,0.909904153354632)(289,0.909323116219668)(290,0.909323116219668)(291,0.907584448693435)(292,0.907006369426752)(293,0.907006369426752)(294,0.907006369426752)(295,0.907006369426752)(296,0.908163265306122)(297,0.908163265306122)(298,0.908163265306122)(299,0.908742820676452)(300,0.908163265306122)(301,0.907584448693435)(302,0.906429026098027)(303,0.907006369426752)(304,0.907006369426752)(305,0.909904153354632)(306,0.908742820676452)(307,0.910485933503836)(308,0.905396825396825)(309,0.90482233502538)(310,0.90482233502538)(311,0.908626198083067)(312,0.90920716112532)(313,0.908626198083067)(314,0.907466496490108)(315,0.908626198083067)(316,0.908626198083067)(317,0.907466496490108)(318,0.905732484076433)(319,0.906309751434034)(320,0.906309751434034)(321,0.90515595162317)(322,0.903675538656527)(323,0.906309751434034)(324,0.906429026098027)(325,0.908045977011494)(326,0.90676883780332)(327,0.907348242811502)(328,0.905852417302799)(329,0.903675538656527)(330,0.903675538656527)(331,0.90253164556962)(332,0.90253164556962)(333,0.903103229892337)(334,0.903675538656527)(335,0.903675538656527)(336,0.905396825396825)(337,0.905396825396825)(338,0.905396825396825)(339,0.905396825396825)(340,0.90482233502538)(341,0.90482233502538)(342,0.90424857324033)(343,0.90424857324033)(344,0.903675538656527)(345,0.901390644753477)(346,0.900821225521162)(347,0.901390644753477)(348,0.903103229892337)(349,0.901390644753477)(350,0.901390644753477)(351,0.901390644753477)(352,0.900821225521162)(353,0.901390644753477)(354,0.901390644753477)(355,0.901960784313725)(356,0.90424857324033)(357,0.905972045743329)(358,0.905972045743329)(359,0.900503778337531)(360,0.900503778337531)(361,0.900503778337531)(362,0.899937067337948)(363,0.899937067337948)(364,0.899937067337948)(365,0.899937067337948)(366,0.899937067337948)(367,0.900946372239747)(368,0.900946372239747)(369,0.901515151515151)(370,0.900946372239747)(371,0.901515151515151)(372,0.899810964083176)(373,0.900378310214376)(374,0.899244332493703)(375,0.899244332493703)(376,0.899244332493703)(377,0.900503778337531)(378,0.902084649399874)(379,0.90424857324033)(380,0.900946372239747)(381,0.901515151515151)(382,0.902084649399874)(383,0.902654867256637)(384,0.902654867256637)(385,0.903225806451613)(386,0.902654867256637)(387,0.902654867256637)(388,0.902084649399874)(389,0.902084649399874)(390,0.902084649399874)(391,0.902654867256637)(392,0.902654867256637)(393,0.902654867256637)(394,0.901515151515151)(395,0.902777777777778)(396,0.903348073278585)(397,0.902208201892744)(398,0.901639344262295)(399,0.904369854338189)(400,0.904369854338189)(3,0.554468362687541)(4,0.577914110429448)(5,0.653033401499659)(6,0.697376839411388)(7,0.719569603227976)(8,0.707509881422925)(9,0.705187130663165)(10,0.740179186767746)(11,0.753282653766413)(12,0.758524704244955)(13,0.762237762237762)(14,0.767605633802817)(15,0.744535519125683)(16,0.742001361470388)(17,0.735989196488859)(18,0.734006734006734)(19,0.734501347708895)(20,0.749656121045392)(21,0.75224292615597)(22,0.753282653766413)(23,0.758524704244955)(24,0.760642009769714)(25,0.767605633802817)(26,0.767065446868402)(27,0.767605633802817)(28,0.765987350667604)(29,0.761705101327743)(30,0.762771168649405)(31,0.762771168649405)(32,0.765449438202247)(33,0.767065446868402)(34,0.76814658210007)(35,0.767065446868402)(36,0.767065446868402)(37,0.767065446868402)(38,0.769774011299435)(39,0.774147727272727)(40,0.775800711743772)(41,0.773049645390071)(42,0.772501771793054)(43,0.773598296664301)(44,0.776353276353276)(45,0.773598296664301)(46,0.773598296664301)(47,0.771954674220963)(48,0.771408351026185)(49,0.769774011299435)(50,0.769230769230769)(51,0.770862800565771)(52,0.771954674220963)(53,0.767065446868402)(54,0.766526019690576)(55,0.767605633802817)(56,0.76814658210007)(57,0.770318021201413)(58,0.769230769230769)(59,0.772501771793054)(60,0.773049645390071)(61,0.775800711743772)(62,0.776353276353276)(63,0.776353276353276)(64,0.775800711743772)(65,0.775800711743772)(66,0.774147727272727)(67,0.77524893314367)(68,0.77524893314367)(69,0.774697938877043)(70,0.775800711743772)(71,0.775800711743772)(72,0.777460770328103)(73,0.795620437956204)(74,0.796783625730994)(75,0.796783625730994)(76,0.797366495976591)(77,0.795620437956204)(78,0.793304221251819)(79,0.79388201019665)(80,0.795040116703136)(81,0.795040116703136)(82,0.795040116703136)(83,0.795620437956204)(84,0.795620437956204)(85,0.795620437956204)(86,0.795620437956204)(87,0.795620437956204)(88,0.795620437956204)(89,0.795620437956204)(90,0.795620437956204)(91,0.795040116703136)(92,0.799120234604106)(93,0.799120234604106)(94,0.795620437956204)(95,0.795040116703136)(96,0.79388201019665)(97,0.79388201019665)(98,0.79388201019665)(99,0.792727272727273)(100,0.792151162790698)(101,0.792151162790698)(102,0.79042784626541)(103,0.79042784626541)(104,0.789855072463768)(105,0.79100145137881)(106,0.79100145137881)(107,0.79042784626541)(108,0.79042784626541)(109,0.79042784626541)(110,0.788712011577424)(111,0.79042784626541)(112,0.79100145137881)(113,0.792151162790698)(114,0.793304221251819)(115,0.79388201019665)(116,0.793304221251819)(117,0.793304221251819)(118,0.791575889615105)(119,0.79100145137881)(120,0.79100145137881)(121,0.79100145137881)(122,0.79100145137881)(123,0.79100145137881)(124,0.789855072463768)(125,0.788712011577424)(126,0.784737221022318)(127,0.783608914450036)(128,0.786435786435786)(129,0.78757225433526)(130,0.781922525107604)(131,0.783608914450036)(132,0.776353276353276)(133,0.774697938877043)(134,0.770862800565771)(135,0.773049645390071)(136,0.77524893314367)(137,0.779685264663805)(138,0.774697938877043)(139,0.769774011299435)(140,0.770318021201413)(141,0.771954674220963)(142,0.771954674220963)(143,0.772501771793054)(144,0.773598296664301)(145,0.773598296664301)(146,0.773598296664301)(147,0.773598296664301)(148,0.773598296664301)(149,0.773598296664301)(150,0.773598296664301)(151,0.773598296664301)(152,0.809556313993174)(153,0.810074880871341)(154,0.803324099722992)(155,0.782085561497326)(156,0.798913043478261)(157,0.804362644853442)(158,0.785009861932939)(159,0.785009861932939)(160,0.782950819672131)(161,0.778357235984354)(162,0.777850162866449)(163,0.771317829457364)(164,0.770819883795997)(165,0.770819883795997)(166,0.770322580645161)(167,0.767845659163987)(168,0.766859344894027)(169,0.765875561257216)(170,0.759541984732824)(171,0.758576874205845)(172,0.759541984732824)(173,0.762452107279693)(174,0.761479591836735)(175,0.761479591836735)(176,0.761479591836735)(177,0.761965539246969)(178,0.761965539246969)(179,0.761479591836735)(180,0.760509554140127)(181,0.761965539246969)(182,0.760509554140127)(183,0.760025461489497)(184,0.759059122695486)(185,0.760994263862333)(186,0.760025461489497)(187,0.761479591836735)(188,0.759059122695486)(189,0.760509554140127)(190,0.760509554140127)(191,0.761479591836735)(192,0.760025461489497)(193,0.760025461489497)(194,0.763427109974424)(195,0.762939297124601)(196,0.765384615384615)(197,0.765384615384615)(198,0.765384615384615)(199,0.764404609475032)(200,0.764404609475032)(201,0.764404609475032)(202,0.764894298526585)(203,0.765384615384615)(204,0.764894298526585)(205,0.764894298526585)(206,0.765875561257216)(207,0.764894298526585)(208,0.765875561257216)(209,0.765875561257216)(210,0.765875561257216)(211,0.765384615384615)(212,0.764894298526585)(213,0.763427109974424)(214,0.763427109974424)(215,0.763427109974424)(216,0.761965539246969)(217,0.761965539246969)(218,0.761479591836735)(219,0.761965539246969)(220,0.760994263862333)(221,0.761479591836735)(222,0.761965539246969)(223,0.762939297124601)(224,0.761965539246969)(225,0.761965539246969)(226,0.761965539246969)(227,0.763427109974424)(228,0.765384615384615)(229,0.764894298526585)(230,0.763915547024952)(231,0.763915547024952)(232,0.761965539246969)(233,0.762452107279693)(234,0.761965539246969)(235,0.760509554140127)(236,0.764894298526585)(237,0.765875561257216)(238,0.763915547024952)(239,0.763915547024952)(240,0.763915547024952)(241,0.762939297124601)(242,0.762939297124601)(243,0.763915547024952)(244,0.764894298526585)(245,0.765875561257216)(246,0.764404609475032)(247,0.762939297124601)(248,0.761479591836735)(249,0.760509554140127)(250,0.760509554140127)(251,0.760509554140127)(252,0.760509554140127)(253,0.759059122695486)(254,0.758576874205845)(255,0.758095238095238)(256,0.7571337983513)(257,0.751889168765743)(258,0.752362948960302)(259,0.747183979974969)(260,0.744853399875234)(261,0.745049504950495)(262,0.749542961608775)(263,0.775175644028103)(264,0.773041474654378)(265,0.773880194279546)(266,0.77807921866522)(267,0.781471389645776)(268,0.765616657768286)(269,0.765208110992529)(270,0.766844919786096)(271,0.76033934252386)(272,0.76033934252386)(273,0.760742705570292)(274,0.760742705570292)(275,0.772629310344828)(276,0.767665952890792)(277,0.766435061464457)(278,0.761550716941051)(279,0.761955366631243)(280,0.761955366631243)(281,0.762765957446808)(282,0.761955366631243)(283,0.761955366631243)(284,0.760742705570292)(285,0.76317189994678)(286,0.76317189994678)(287,0.77765726681128)(288,0.776814734561213)(289,0.77765726681128)(290,0.776394152680022)(291,0.776394152680022)(292,0.77765726681128)(293,0.776394152680022)(294,0.776394152680022)(295,0.776394152680022)(296,0.775554353704705)(297,0.775974025974026)(298,0.775974025974026)(299,0.776394152680022)(300,0.768900804289544)(301,0.76931330472103)(302,0.772213247172859)(303,0.785753424657534)(304,0.790954219525648)(305,0.792265193370166)(306,0.791390728476821)(307,0.790954219525648)(308,0.791390728476821)(309,0.791390728476821)(310,0.792265193370166)(311,0.796224319822321)(312,0.797552836484983)(313,0.797996661101836)(314,0.79933110367893)(315,0.797996661101836)(316,0.797552836484983)(317,0.797552836484983)(318,0.797552836484983)(319,0.798885793871866)(320,0.800670016750419)(321,0.799776910206358)(322,0.799776910206358)(323,0.800670016750419)(324,0.804713804713805)(325,0.803811659192825)(326,0.803811659192825)(327,0.803811659192825)(328,0.802911534154535)(329,0.802462227196418)(330,0.802013422818792)(331,0.800670016750419)(332,0.801565120178871)(333,0.801117318435754)(334,0.801565120178871)(335,0.801565120178871)(336,0.802013422818792)(337,0.802013422818792)(338,0.805165637282426)(339,0.806978052898143)(340,0.80652418447694)(341,0.806070826306914)(342,0.80652418447694)(343,0.80652418447694)(344,0.806978052898143)(345,0.806978052898143)(346,0.806978052898143)(347,0.80652418447694)(348,0.821776504297994)(349,0.822247706422018)(350,0.820366132723112)(351,0.819897084048027)(352,0.819428571428571)(353,0.823664560597358)(354,0.822719449225473)(355,0.822719449225473)(356,0.822719449225473)(357,0.821776504297994)(358,0.824137931034483)(359,0.824137931034483)(360,0.824611845888442)(361,0.824611845888442)(362,0.825561312607945)(363,0.826989619377163)(364,0.825561312607945)(365,0.825561312607945)(366,0.826989619377163)(367,0.826512968299712)(368,0.827944572748268)(369,0.827944572748268)(370,0.827944572748268)(371,0.836639439906651)(372,0.837616822429906)(373,0.83859649122807)(374,0.837127845884413)(375,0.836639439906651)(376,0.835664335664336)(377,0.836639439906651)(378,0.836151603498542)(379,0.836151603498542)(380,0.834691501746216)(381,0.834691501746216)(382,0.833720930232558)(383,0.833720930232558)(384,0.8351776354106)(385,0.8351776354106)(386,0.826989619377163)(387,0.826512968299712)(388,0.825561312607945)(389,0.82842287694974)(390,0.830822711471611)(391,0.831786542923434)(392,0.832269297736506)(393,0.832269297736506)(394,0.832269297736506)(395,0.826512968299712)(396,0.826036866359447)(397,0.827466820542412)(398,0.827466820542412)(399,0.826036866359447)(400,0.82842287694974)(3,0.619858156028369)(4,0.6289592760181)(5,0.679933665008292)(6,0.676616915422885)(7,0.775574940523394)(8,0.769938650306748)(9,0.769348659003831)(10,0.769465648854962)(11,0.777691711851278)(12,0.766355140186916)(13,0.808219178082192)(14,0.811306340718105)(15,0.814757878554958)(16,0.811349693251534)(17,0.805555555555555)(18,0.806501547987616)(19,0.810600155884645)(20,0.811526479750779)(21,0.816705336426914)(22,0.814814814814815)(23,0.814186584425597)(24,0.814814814814815)(25,0.812403100775194)(26,0.811773818745159)(27,0.813271604938271)(28,0.813899613899614)(29,0.818812644564379)(30,0.819165378670788)(31,0.819519752130132)(32,0.817337461300309)(33,0.82007722007722)(34,0.820710973724884)(35,0.827747466874513)(36,0.830865159781761)(37,0.830865159781761)(38,0.833204034134988)(39,0.833204034134988)(40,0.834365325077399)(41,0.835266821345708)(42,0.835266821345708)(43,0.835658914728682)(44,0.835658914728682)(45,0.835913312693498)(46,0.838109992254067)(47,0.834876543209876)(48,0.834876543209876)(49,0.833590138674884)(50,0.833590138674884)(51,0.834876543209876)(52,0.834876543209876)(53,0.835521235521235)(54,0.836166924265842)(55,0.827850038255547)(56,0.828113063407181)(57,0.826849733028223)(58,0.824961948249619)(59,0.824961948249619)(60,0.825227963525836)(61,0.82648401826484)(62,0.82837528604119)(63,0.82837528604119)(64,0.826585179526356)(65,0.827850038255547)(66,0.828483920367534)(67,0.826585179526356)(68,0.825324180015256)(69,0.825324180015256)(70,0.825324180015256)(71,0.823082763857251)(72,0.824961948249619)(73,0.824334600760456)(74,0.825324180015256)(75,0.827217125382263)(76,0.826585179526356)(77,0.827217125382263)(78,0.829754601226994)(79,0.829754601226994)(80,0.82911877394636)(81,0.831926323867997)(82,0.832948421862971)(83,0.831926323867997)(84,0.831926323867997)(85,0.832440703902066)(86,0.83282208588957)(87,0.830910482019893)(88,0.829007633587786)(89,0.829007633587786)(90,0.830275229357798)(91,0.828267477203647)(92,0.82701062215478)(93,0.822021116138763)(94,0.825132475397426)(95,0.820165537998495)(96,0.82078313253012)(97,0.81893313298272)(98,0.817091454272864)(99,0.81525804038893)(100,0.815868263473054)(101,0.81525804038893)(102,0.817091454272864)(103,0.817091454272864)(104,0.817704426106526)(105,0.817091454272864)(106,0.817091454272864)(107,0.818318318318318)(108,0.818318318318318)(109,0.81893313298272)(110,0.821401657874906)(111,0.82078313253012)(112,0.820165537998495)(113,0.820165537998495)(114,0.819548872180451)(115,0.818318318318318)(116,0.817704426106526)(117,0.82078313253012) 
};
\addplot [
color=orange,
mark size=0.1pt,
only marks,
mark=*,
mark options={solid,fill=black},
forget plot
]
coordinates{
 (117,0.82078313253012)(118,0.82078313253012)(119,0.82078313253012)(120,0.81893313298272)(121,0.817704426106526)(122,0.818318318318318)(123,0.818318318318318)(124,0.81893313298272)(125,0.82078313253012)(126,0.819548872180451)(127,0.819548872180451)(128,0.819548872180451)(129,0.820165537998495)(130,0.819548872180451)(131,0.82078313253012)(132,0.820165537998495)(133,0.81893313298272)(134,0.819548872180451)(135,0.819548872180451)(136,0.818318318318318)(137,0.82078313253012)(138,0.819548872180451)(139,0.817091454272864)(140,0.817704426106526)(141,0.817091454272864)(142,0.817704426106526)(143,0.818318318318318)(144,0.816479400749064)(145,0.816479400749064)(146,0.816479400749064)(147,0.81893313298272)(148,0.819548872180451)(149,0.821401657874906)(150,0.82078313253012)(151,0.821401657874906)(152,0.82078313253012)(153,0.822021116138763)(154,0.824508320726172)(155,0.824508320726172)(156,0.821401657874906)(157,0.82078313253012)(158,0.822021116138763)(159,0.822021116138763)(160,0.821401657874906)(161,0.821401657874906)(162,0.822021116138763)(163,0.821401657874906)(164,0.821401657874906)(165,0.821401657874906)(166,0.822021116138763)(167,0.822021116138763)(168,0.824508320726172)(169,0.825757575757576)(170,0.828267477203647)(171,0.827638572513288)(172,0.828267477203647)(173,0.827638572513288)(174,0.831426392067124)(175,0.833333333333333)(176,0.83206106870229)(177,0.832696715049656)(178,0.832696715049656)(179,0.831426392067124)(180,0.831426392067124)(181,0.83206106870229)(182,0.83206106870229)(183,0.833333333333333)(184,0.83910700538876)(185,0.840400925212028)(186,0.837173579109063)(187,0.834609494640122)(188,0.834609494640122)(189,0.836531082118189)(190,0.836531082118189)(191,0.836531082118189)(192,0.836531082118189)(193,0.835889570552147)(194,0.831426392067124)(195,0.82701062215478)(196,0.831426392067124)(197,0.828897338403042)(198,0.828897338403042)(199,0.828897338403042)(200,0.82701062215478)(201,0.821401657874906)(202,0.822641509433962)(203,0.822021116138763)(204,0.825757575757576)(205,0.823262839879154)(206,0.823262839879154)(207,0.825132475397426)(208,0.823262839879154)(209,0.823262839879154)(210,0.822021116138763)(211,0.821401657874906)(212,0.821401657874906)(213,0.821401657874906)(214,0.821401657874906)(215,0.821401657874906)(216,0.822021116138763)(217,0.822641509433962)(218,0.81893313298272)(219,0.817704426106526)(220,0.817704426106526)(221,0.817704426106526)(222,0.822021116138763)(223,0.822021116138763)(224,0.822641509433962)(225,0.823262839879154)(226,0.824508320726172)(227,0.823885109599395)(228,0.825757575757576)(229,0.828267477203647)(230,0.82701062215478)(231,0.82701062215478)(232,0.827638572513288)(233,0.828267477203647)(234,0.822021116138763)(235,0.821401657874906)(236,0.821401657874906)(237,0.821401657874906)(238,0.821401657874906)(239,0.82078313253012)(240,0.81893313298272)(241,0.819548872180451)(242,0.820165537998495)(243,0.82078313253012)(244,0.821401657874906)(245,0.821401657874906)(246,0.82078313253012)(247,0.82078313253012)(248,0.81893313298272)(249,0.82078313253012)(250,0.82078313253012)(251,0.819548872180451)(252,0.823262839879154)(253,0.823262839879154)(254,0.823262839879154)(255,0.822021116138763)(256,0.830792682926829)(257,0.828897338403042)(258,0.828897338403042)(259,0.825757575757576)(260,0.825757575757576)(261,0.826383623957544)(262,0.825757575757576)(263,0.82701062215478)(264,0.83015993907083)(265,0.830792682926829)(266,0.830792682926829)(267,0.828267477203647)(268,0.829528158295281)(269,0.828897338403042)(270,0.83015993907083)(271,0.832696715049656)(272,0.833333333333333)(273,0.83206106870229)(274,0.83206106870229)(275,0.830792682926829)(276,0.829528158295281)(277,0.829528158295281)(278,0.829528158295281)(279,0.824508320726172)(280,0.827638572513288)(281,0.829528158295281)(282,0.829528158295281)(283,0.829528158295281)(284,0.829528158295281)(285,0.834609494640122)(286,0.845616757176105)(287,0.845616757176105)(288,0.844961240310077)(289,0.848909657320872)(290,0.846930846930847)(291,0.847589424572317)(292,0.847589424572317)(293,0.846930846930847)(294,0.846930846930847)(295,0.848909657320872)(296,0.848909657320872)(297,0.848909657320872)(298,0.848909657320872)(299,0.847113884555382)(300,0.847298355520752)(301,0.847962382445141)(302,0.847962382445141)(303,0.847723704866562)(304,0.847723704866562)(305,0.847723704866562)(306,0.8484375)(307,0.851063829787234)(308,0.851298190401259)(309,0.849250197316495)(310,0.850157728706624)(311,0.849724626278521)(312,0.847723704866562)(313,0.849724626278521)(314,0.849724626278521)(315,0.849724626278521)(316,0.848818897637795)(317,0.849487785657998)(318,0.849724626278521)(319,0.849960722702278)(320,0.849293563579278)(321,0.848627450980392)(322,0.849765258215962)(323,0.849765258215962)(324,0.849100860046912)(325,0.849336455893833)(326,0.848673946957878)(327,0.848673946957878)(328,0.848909657320872)(329,0.849336455893833)(330,0.848673946957878)(331,0.848012470771629)(332,0.849336455893833)(333,0.849336455893833)(334,0.849336455893833)(335,0.851298190401259)(336,0.85062893081761)(337,0.850828729281768)(338,0.851030110935024)(339,0.850118953211737)(340,0.850556438791733)(341,0.849880857823669)(342,0.849880857823669)(343,0.850793650793651)(344,0.850793650793651)(345,0.850556438791733)(346,0.850556438791733)(347,0.851469420174742)(348,0.849445324881141)(349,0.84688995215311)(350,0.844979919678715)(351,0.844979919678715)(352,0.845476381104884)(353,0.845476381104884)(354,0.849642004773269)(355,0.848966613672496)(356,0.848292295472597)(357,0.848966613672496)(358,0.847808764940239)(359,0.847808764940239)(360,0.847808764940239)(361,0.847808764940239)(362,0.84664536741214)(363,0.84664536741214)(364,0.845906902086677)(365,0.843121480289622)(366,0.842190016103059)(367,0.843121480289622)(368,0.843121480289622)(369,0.838449111470113)(370,0.838449111470113)(371,0.838449111470113)(372,0.838449111470113)(373,0.844979919678715)(374,0.844051446945338)(375,0.844051446945338)(376,0.842190016103059)(377,0.844051446945338)(378,0.844051446945338)(379,0.843121480289622)(380,0.840064620355412)(381,0.839126919967664)(382,0.839126919967664)(383,0.841000807102502)(384,0.837246963562753)(385,0.836304700162074)(386,0.834415584415584)(387,0.832520325203252)(388,0.832520325203252)(389,0.831570382424735)(390,0.832520325203252)(391,0.832520325203252)(392,0.832520325203252)(393,0.833468724614135)(394,0.833468724614135)(395,0.832520325203252)(396,0.832520325203252)(397,0.832520325203252)(398,0.829665851670741)(399,0.838187702265372)(400,0.838187702265372)(5,0.602256699576869)(6,0.691185795814838)(7,0.703125)(8,0.760111576011157)(9,0.750172057811424)(10,0.775948460987831)(11,0.778682457438934)(12,0.776934749620637)(13,0.790446841294299)(14,0.79048349961627)(15,0.791730474732006)(16,0.784073506891271)(17,0.784012298232129)(18,0.786206896551724)(19,0.79577995478523)(20,0.797014925373134)(21,0.79940119760479)(22,0.798803290949888)(23,0.798798798798799)(24,0.801509433962264)(25,0.801520912547528)(26,0.807256235827664)(27,0.801492537313433)(28,0.803611738148984)(29,0.805135951661631)(30,0.807896735003796)(31,0.807896735003796)(32,0.808834729626809)(33,0.809090909090909)(34,0.808095952023988)(35,0.807518796992481)(36,0.809344385832705)(37,0.80806572068708)(38,0.809880239520958)(39,0.809274495138369)(40,0.806236080178174)(41,0.803849000740192)(42,0.802067946824224)(43,0.799116997792494)(44,0.799705449189985)(45,0.792727272727273)(46,0.792151162790698)(47,0.795040116703136)(48,0.795040116703136)(49,0.796201607012418)(50,0.796201607012418)(51,0.800881704628949)(52,0.800293685756241)(53,0.802650957290132)(54,0.803545051698671)(55,0.802660753880266)(56,0.800881704628949)(57,0.800881704628949)(58,0.799706529713866)(59,0.794460641399417)(60,0.798534798534798)(61,0.799120234604106)(62,0.797950219619326)(63,0.795620437956204)(64,0.795040116703136)(65,0.79388201019665)(66,0.79388201019665)(67,0.79042784626541)(68,0.789283128167994)(69,0.789855072463768)(70,0.792727272727273)(71,0.79388201019665)(72,0.792727272727273)(73,0.792727272727273)(74,0.792151162790698)(75,0.792727272727273)(76,0.792727272727273)(77,0.791575889615105)(78,0.794460641399417)(79,0.794460641399417)(80,0.795040116703136)(81,0.795620437956204)(82,0.796783625730994)(83,0.796783625730994)(84,0.795620437956204)(85,0.795620437956204)(86,0.795620437956204)(87,0.796201607012418)(88,0.796201607012418)(89,0.795620437956204)(90,0.79388201019665)(91,0.79100145137881)(92,0.79100145137881)(93,0.792727272727273)(94,0.792727272727273)(95,0.791575889615105)(96,0.789855072463768)(97,0.789855072463768)(98,0.785302593659942)(99,0.779685264663805)(100,0.77524893314367)(101,0.781362007168459)(102,0.782483847810481)(103,0.783608914450036)(104,0.781362007168459)(105,0.781362007168459)(106,0.780243378668575)(107,0.780243378668575)(108,0.779685264663805)(109,0.778015703069236)(110,0.778015703069236)(111,0.779685264663805)(112,0.774697938877043)(113,0.77524893314367)(114,0.781362007168459)(115,0.778571428571429)(116,0.779127948534667)(117,0.779127948534667)(118,0.779685264663805)(119,0.779685264663805)(120,0.779685264663805)(121,0.777460770328103)(122,0.779685264663805)(123,0.780243378668575)(124,0.781922525107604)(125,0.784172661870504)(126,0.784737221022318)(127,0.784737221022318)(128,0.785302593659942)(129,0.785868781542898)(130,0.784737221022318)(131,0.784737221022318)(132,0.783608914450036)(133,0.779685264663805)(134,0.776906628652887)(135,0.776353276353276)(136,0.776353276353276)(137,0.77524893314367)(138,0.77524893314367)(139,0.773598296664301)(140,0.77524893314367)(141,0.77524893314367)(142,0.776353276353276)(143,0.774697938877043)(144,0.77524893314367)(145,0.784737221022318)(146,0.784737221022318)(147,0.784737221022318)(148,0.78757225433526)(149,0.788712011577424)(150,0.788712011577424)(151,0.788712011577424)(152,0.788712011577424)(153,0.788712011577424)(154,0.788712011577424)(155,0.789855072463768)(156,0.79042784626541)(157,0.789855072463768)(158,0.791575889615105)(159,0.792151162790698)(160,0.796201607012418)(161,0.79388201019665)(162,0.792727272727273)(163,0.792727272727273)(164,0.793304221251819)(165,0.793304221251819)(166,0.793304221251819)(167,0.793304221251819)(168,0.793304221251819)(169,0.789855072463768)(170,0.791575889615105)(171,0.792151162790698)(172,0.795040116703136)(173,0.795620437956204)(174,0.795620437956204)(175,0.795040116703136)(176,0.795040116703136)(177,0.796201607012418)(178,0.796201607012418)(179,0.796201607012418)(180,0.796201607012418)(181,0.795620437956204)(182,0.798534798534798)(183,0.797366495976591)(184,0.795620437956204)(185,0.793304221251819)(186,0.795040116703136)(187,0.795040116703136)(188,0.795040116703136)(189,0.795620437956204)(190,0.795620437956204)(191,0.795620437956204)(192,0.795040116703136)(193,0.795040116703136)(194,0.797950219619326)(195,0.797366495976591)(196,0.797366495976591)(197,0.797950219619326)(198,0.797366495976591)(199,0.797366495976591)(200,0.796783625730994)(201,0.796783625730994)(202,0.796783625730994)(203,0.802650957290132)(204,0.802650957290132)(205,0.803242446573323)(206,0.803242446573323)(207,0.801470588235294)(208,0.800293685756241)(209,0.799706529713866)(210,0.799706529713866)(211,0.799706529713866)(212,0.800293685756241)(213,0.800293685756241)(214,0.800293685756241)(215,0.800293685756241)(216,0.800293685756241)(217,0.80206033848418)(218,0.803242446573323)(219,0.803242446573323)(220,0.803242446573323)(221,0.802650957290132)(222,0.802650957290132)(223,0.803242446573323)(224,0.803834808259587)(225,0.803242446573323)(226,0.800881704628949)(227,0.801470588235294)(228,0.801470588235294)(229,0.801470588235294)(230,0.802650957290132)(231,0.802650957290132)(232,0.803242446573323)(233,0.803242446573323)(234,0.804428044280443)(235,0.803834808259587)(236,0.803242446573323)(237,0.802650957290132)(238,0.803242446573323)(239,0.803242446573323)(240,0.803242446573323)(241,0.801470588235294)(242,0.799120234604106)(243,0.799120234604106)(244,0.799706529713866)(245,0.800881704628949)(246,0.799120234604106)(247,0.799120234604106)(248,0.799120234604106)(249,0.799120234604106)(250,0.799706529713866)(251,0.800881704628949)(252,0.801470588235294)(253,0.800881704628949)(254,0.800293685756241)(255,0.801470588235294)(256,0.801470588235294)(257,0.802650957290132)(258,0.802650957290132)(259,0.801470588235294)(260,0.80206033848418)(261,0.80680977054034)(262,0.806213017751479)(263,0.806213017751479)(264,0.80680977054034)(265,0.807407407407407)(266,0.80680977054034)(267,0.80680977054034)(268,0.808005930318755)(269,0.808005930318755)(270,0.808005930318755)(271,0.808605341246291)(272,0.80920564216778)(273,0.810408921933085)(274,0.810408921933085)(275,0.810408921933085)(276,0.810408921933085)(277,0.814648729446936)(278,0.813432835820895)(279,0.813432835820895)(280,0.814648729446936)(281,0.81525804038893)(282,0.81525804038893)(283,0.817704426106526)(284,0.817704426106526)(285,0.815868263473054)(286,0.817704426106526)(287,0.817704426106526)(288,0.820165537998495)(289,0.814040328603435)(290,0.814648729446936)(291,0.814648729446936)(292,0.814648729446936)(293,0.81282624906786)(294,0.811011904761905)(295,0.811011904761905)(296,0.811011904761905)(297,0.80920564216778)(298,0.80920564216778)(299,0.813432835820895)(300,0.80920564216778)(301,0.808605341246291)(302,0.808605341246291)(303,0.809806835066865)(304,0.809806835066865)(305,0.811615785554728)(306,0.811615785554728)(307,0.811615785554728)(308,0.812220566318927)(309,0.813432835820895)(310,0.812220566318927)(311,0.812220566318927)(312,0.812220566318927)(313,0.812220566318927)(314,0.812220566318927)(315,0.811011904761905)(316,0.80920564216778)(317,0.80920564216778)(318,0.811011904761905)(319,0.80920564216778)(320,0.80920564216778)(321,0.80920564216778)(322,0.809806835066865)(323,0.808005930318755)(324,0.808005930318755)(325,0.808005930318755)(326,0.808605341246291)(327,0.808605341246291)(328,0.808605341246291)(329,0.807407407407407)(330,0.808005930318755)(331,0.808005930318755)(332,0.808605341246291)(333,0.808605341246291)(334,0.808605341246291)(335,0.808605341246291)(336,0.808605341246291)(337,0.810408921933085)(338,0.811011904761905)(339,0.811011904761905)(340,0.811011904761905)(341,0.810408921933085)(342,0.810408921933085)(343,0.810408921933085)(344,0.810408921933085)(345,0.808005930318755)(346,0.808005930318755)(347,0.805022156573117)(348,0.803834808259587)(349,0.804428044280443)(350,0.803834808259587)(351,0.805022156573117)(352,0.805022156573117)(353,0.805022156573117)(354,0.805022156573117)(355,0.805022156573117)(356,0.806213017751479)(357,0.808005930318755)(358,0.808005930318755)(359,0.808005930318755)(360,0.808005930318755)(361,0.807407407407407)(362,0.807407407407407)(363,0.80680977054034)(364,0.807407407407407)(365,0.807407407407407)(366,0.807407407407407)(367,0.807407407407407)(368,0.808005930318755)(369,0.808005930318755)(370,0.808005930318755)(371,0.808005930318755)(372,0.810408921933085)(373,0.811615785554728)(374,0.811615785554728)(375,0.811615785554728)(376,0.813432835820895)(377,0.813432835820895)(378,0.81282624906786)(379,0.81282624906786)(380,0.81282624906786)(381,0.81893313298272)(382,0.81893313298272)(383,0.81893313298272)(384,0.818318318318318)(385,0.822641509433962)(386,0.81893313298272)(387,0.818318318318318)(388,0.818318318318318)(389,0.818318318318318)(390,0.818318318318318)(391,0.818318318318318)(392,0.818318318318318)(393,0.81893313298272)(394,0.818318318318318)(395,0.818318318318318)(396,0.822021116138763)(397,0.822021116138763)(398,0.823262839879154)(399,0.822641509433962)(400,0.822641509433962)(9,0.503210741389375)(10,0.697701826753094)(11,0.702764976958525)(12,0.712312312312312)(13,0.740740740740741)(14,0.731102850061958)(15,0.751622418879056)(16,0.751479289940828)(17,0.763095238095238)(18,0.752834467120181)(19,0.746136865342163)(20,0.758698092031425)(21,0.763896532746285)(22,0.745077168706759)(23,0.753093060785368)(24,0.769060773480663)(25,0.773259052924791)(26,0.76786703601108)(27,0.761750405186386)(28,0.773100054674686)(29,0.76473769605192)(30,0.762317271250677)(31,0.76123443421765)(32,0.759439050701186)(33,0.755913978494624)(34,0.759179265658747)(35,0.762472885032538)(36,0.762472885032538)(37,0.758176943699732)(38,0.749602964531498)(39,0.746835443037975)(40,0.748810153358012)(41,0.759548144163529)(42,0.793220338983051)(43,0.796590909090909)(44,0.796359499431172)(45,0.819025522041763)(46,0.821949795680093)(47,0.82051282051282)(48,0.824489795918367)(49,0.824214202561118)(50,0.823734729493892)(51,0.823255813953488)(52,0.813578826237054)(53,0.808)(54,0.808)(55,0.82332361516035)(56,0.829411764705882)(57,0.829899941141848)(58,0.830878020035356)(59,0.821678321678322)(60,0.826979472140762)(61,0.826979472140762)(62,0.84311377245509)(63,0.842925659472422)(64,0.843431313737252)(65,0.858903265557609)(66,0.86372121966397)(67,0.863354037267081)(68,0.864965774735532)(69,0.863890615288999)(70,0.866416978776529)(71,0.868505452212957)(72,0.873329089751751)(73,0.880568108457069)(74,0.885267275097783)(75,0.884690553745928)(76,0.885117493472585)(77,0.884665792922673)(78,0.884086444007858)(79,0.884967320261438)(80,0.882003955174687)(81,0.883751651254954)(82,0.885072655217966)(83,0.887425938117182)(84,0.886968085106383)(85,0.885486018641811)(86,0.885049833887043)(87,0.885049833887043)(88,0.885790172642762)(89,0.890945142101784)(90,0.891089108910891)(91,0.891089108910891)(92,0.896)(93,0.900066181336863)(94,0.901726427622842)(95,0.899798251513113)(96,0.904219691895512)(97,0.904441453566622)(98,0.90566037735849)(99,0.904441453566622)(100,0.899527983816588)(101,0.898589657488247)(102,0.902666666666667)(103,0.894910773298083)(104,0.894211576846307)(105,0.894211576846307)(106,0.895640686922061)(107,0.893758300132802)(108,0.894352159468438)(109,0.896735509660226)(110,0.896138482023968)(111,0.909214092140921)(112,0.912925170068027)(113,0.911945392491467)(114,0.912065439672801)(115,0.912806539509537)(116,0.91120218579235)(117,0.911825017088175)(118,0.912568306010929)(119,0.912065439672801)(120,0.91180461329715)(121,0.912423625254582)(122,0.91304347826087)(123,0.913131313131313)(124,0.914707857622565)(125,0.918809201623816)(126,0.921928038017651)(127,0.921195652173913)(128,0.917344173441734)(129,0.922448979591837)(130,0.922448979591837)(131,0.920245398773006)(132,0.91785471826205)(133,0.918478260869565)(134,0.920570264765784)(135,0.920245398773006)(136,0.921501706484641)(137,0.920980926430518)(138,0.919945725915875)(139,0.920785375761679)(140,0.919028340080971)(141,0.918408631153068)(142,0.918408631153068)(143,0.916275954454119)(144,0.916890080428954)(145,0.917505030181086)(146,0.918230563002681)(147,0.911745189117452)(148,0.914780292942743)(149,0.916611074049366)(150,0.916833000665336)(151,0.915502328675981)(152,0.91672218520986)(153,0.919387075283144)(154,0.918774966711052)(155,0.91816367265469)(156,0.915119363395225)(157,0.915726609157266)(158,0.915726609157266)(159,0.916334661354582)(160,0.916445623342175)(161,0.917662682602922)(162,0.930264048747461)(163,0.938356164383561)(164,0.941580756013746)(165,0.934326337169939)(166,0.941580756013746)(167,0.940934065934066)(168,0.937799043062201)(169,0.938271604938272)(170,0.938915579958819)(171,0.939560439560439)(172,0.937628512679918)(173,0.939560439560439)(174,0.940771349862259)(175,0.940852819807428)(176,0.942307692307692)(177,0.942307692307692)(178,0.942307692307692)(179,0.940934065934066)(180,0.94028826355525)(181,0.941257774706289)(182,0.941988950276243)(183,0.941988950276243)(184,0.941988950276243)(185,0.9426399447132)(186,0.9432918395574)(187,0.94271911663216)(188,0.94271911663216)(189,0.940360610263523)(190,0.941094941094941)(191,0.94150417827298)(192,0.942160278745644)(193,0.942160278745644)(194,0.941585535465925)(195,0.940931202223766)(196,0.942240779401531)(197,0.942976356050069)(198,0.941747572815534)(199,0.943213296398892)(200,0.944521497919556)(201,0.943944636678201)(202,0.943944636678201)(203,0.941666666666667)(204,0.942976356050069)(205,0.941666666666667)(206,0.943213296398892)(207,0.9426399447132)(208,0.943789035392089)(209,0.942401110340041)(210,0.943396226415094)(211,0.945479641131815)(212,0.943526170798898)(213,0.944903581267218)(214,0.943758573388203)(215,0.943758573388203)(216,0.941580756013746)(217,0.942798070296347)(218,0.941580756013746)(219,0.941660947151681)(220,0.942307692307692)(221,0.940370116518163)(222,0.941015089163237)(223,0.937883959044369)(224,0.937799043062201)(225,0.939082819986311)(226,0.941740918437286)(227,0.941095890410959)(228,0.942307692307692)(229,0.941660947151681)(230,0.943111720356408)(231,0.944558521560575)(232,0.944558521560575)(233,0.945704467353952)(234,0.945054945054945)(235,0.946354883081155)(236,0.946354883081155)(237,0.946280991735537)(238,0.945054945054945)(239,0.944406314344543)(240,0.944482522275531)(241,0.945704467353952)(242,0.945130315500686)(243,0.943989071038251)(244,0.943344709897611)(245,0.945130315500686)(246,0.945778997940974)(247,0.944558521560575)(248,0.94535519125683)(249,0.946938775510204)(250,0.946938775510204)(251,0.946938775510204)(252,0.947725729803123)(253,0.947725729803123)(254,0.947010869565217)(255,0.94708276797829)(256,0.947010869565217)(257,0.94708276797829)(258,0.948369565217391)(259,0.947725729803123)(260,0.948369565217391)(261,0.948369565217391)(262,0.948158253751705)(263,0.950953678474114)(264,0.951667801225323)(265,0.951667801225323)(266,0.951667801225323)(267,0.951153324287653)(268,0.951864406779661)(269,0.951864406779661)(270,0.951864406779661)(271,0.952445652173913)(272,0.954390742001361)(273,0.954390742001361)(274,0.953156822810591)(275,0.953156822810591)(276,0.951929587000677)(277,0.95)(278,0.95)(279,0.95)(280,0.95)(281,0.952574525745257)(282,0.952574525745257)(283,0.952510176390773)(284,0.952510176390773)(285,0.952510176390773)(286,0.953156822810591)(287,0.953867028493894)(288,0.954514596062457)(289,0.956462585034013)(290,0.95581237253569)(291,0.95581237253569)(292,0.95581237253569)(293,0.95581237253569)(294,0.955163043478261)(295,0.955163043478261)(296,0.955163043478261)(297,0.955163043478261)(298,0.95581237253569)(299,0.955163043478261)(300,0.955163043478261)(301,0.965421853388658)(302,0.966759002770083)(303,0.964851826326671)(304,0.966850828729282)(305,0.96551724137931)(306,0.96551724137931)(307,0.962861072902338)(308,0.960932145305003)(309,0.959561343385881)(310,0.961538461538461)(311,0.961538461538461)(312,0.96027397260274)(313,0.960932145305003)(314,0.960932145305003)(315,0.958960328317373)(316,0.959016393442623)(317,0.959671907040328)(318,0.959671907040328)(319,0.960985626283367)(320,0.963623884694578)(321,0.960328317373461)(322,0.960328317373461)(323,0.960328317373461)(324,0.959671907040328)(325,0.959726962457338)(326,0.958475153165419)(327,0.958475153165419)(328,0.959239130434782)(329,0.958587915818058)(330,0.958587915818058)(331,0.959183673469388)(332,0.959836623553438)(333,0.959183673469388)(334,0.959836623553438)(335,0.958531611148878)(336,0.960490463215259)(337,0.960490463215259)(338,0.958531611148878)(339,0.959183673469388)(340,0.959183673469388)(341,0.964432284541724)(342,0.964432284541724)(343,0.963114754098361)(344,0.955284552845528)(345,0.956521739130435)(346,0.956521739130435)(347,0.95593220338983)(348,0.953991880920162)(349,0.953991880920162)(350,0.952702702702703)(351,0.952702702702703)(352,0.950201884253028)(353,0.953346855983773)(354,0.954054054054054)(355,0.954054054054054)(356,0.955345060893099)(357,0.957937584803256)(358,0.958475153165419)(359,0.959128065395095)(360,0.959836623553438)(361,0.959836623553438)(362,0.959836623553438)(363,0.959183673469388)(364,0.958418541240627)(365,0.958418541240627)(366,0.958418541240627)(367,0.952830188679245)(368,0.951547779273217)(369,0.952830188679245)(370,0.954116059379217)(371,0.952188552188552)(372,0.952830188679245)(373,0.956639566395664)(374,0.956639566395664)(375,0.953409858203916)(376,0.950335570469799)(377,0.950335570469799)(378,0.953535353535353)(379,0.953535353535353)(380,0.954821308159137)(381,0.954821308159137)(382,0.953020134228188)(383,0.953020134228188)(384,0.953020134228188)(385,0.952380952380952)(386,0.951742627345844)(387,0.951105157401205)(388,0.951742627345844)(389,0.951742627345844)(390,0.950468540829987)(391,0.949832775919732)(392,0.949832775919732)(393,0.948563794255177)(394,0.949197860962567)(395,0.949197860962567)(396,0.949832775919732)(397,0.950468540829987)(398,0.951742627345844)(399,0.951742627345844)(400,0.951105157401205)(7,0.604370761115298)(8,0.679298245614035)(9,0.697938877043355)(10,0.752365930599369)(11,0.752922837100546)(12,0.759146341463415)(13,0.765564950038432)(14,0.781519185591229)(15,0.781499202551834)(16,0.779552715654952)(17,0.779552715654952)(18,0.801064537591484)(19,0.808912896691425)(20,0.812680115273775)(21,0.81917211328976)(22,0.823699421965318)(23,0.828510938602682)(24,0.831541218637993)(25,0.8327246165084)(26,0.83744557329463)(27,0.84241103848947)(28,0.851245551601423)(29,0.851457000710732)(30,0.855524079320113)(31,0.853521126760563)(32,0.855310049893086)(33,0.853067047075606)(34,0.85734664764622)(35,0.863157894736842)(36,0.856534090909091)(37,0.863537906137184)(38,0.869690424766019)(39,0.866618601297765)(40,0.865800865800866)(41,0.864091559370529)(42,0.863537906137184)(43,0.866666666666667)(44,0.868571428571428)(45,0.869192280200143)(46,0.874010079193664)(47,0.877801879971077)(48,0.877906976744186)(49,0.877090909090909)(50,0.877447425670776)(51,0.877476155539252)(52,0.881924198250729)(53,0.878013148283418)(54,0.882096069868996)(55,0.880174291938998)(56,0.880813953488372)(57,0.88029197080292)(58,0.87964989059081)(59,0.877655677655678)(60,0.881924198250729)(61,0.879297732260424)(62,0.877372262773722)(63,0.87819110138585)(64,0.878765613519471)(65,0.882005899705015)(66,0.880058866813834)(67,0.880058866813834)(68,0.879588839941263)(69,0.880235121234386)(70,0.880235121234386)(71,0.880235121234386)(72,0.880410858400587)(73,0.886131386861314)(74,0.887918486171761)(75,0.892263195950831)(76,0.895803183791606)(77,0.895003620564808)(78,0.897101449275362)(79,0.897101449275362)(80,0.897101449275362)(81,0.898909090909091)(82,0.90190336749634)(83,0.902706656912948)(84,0.903649635036496)(85,0.903225806451613)(86,0.903225806451613)(87,0.902421129860601)(88,0.902421129860601)(89,0.902706656912948)(90,0.902941176470588)(91,0.901033973412112)(92,0.901989683124539)(93,0.899122807017544)(94,0.899780541331382)(95,0.899338721528288)(96,0.901845018450184)(97,0.903605592347314)(98,0.904552129221733)(99,0.905077262693157)(100,0.904270986745214)(101,0.905355832721937)(102,0.904169714703731)(103,0.906158357771261)(104,0.905494505494505)(105,0.902277736958119)(106,0.902277736958119)(107,0.904270986745214)(108,0.903605592347314)(109,0.904089219330855)(110,0.906110283159463)(111,0.906110283159463)(112,0.908284023668639)(113,0.908956328645448)(114,0.906666666666667)(115,0.907338769458858)(116,0.90922619047619)(117,0.904334828101644)(118,0.909495548961424)(119,0.910436713545522)(120,0.907462686567164)(121,0.905011219147345)(122,0.900900900900901)(123,0.903370786516854)(124,0.903370786516854)(125,0.912881608339538)(126,0.920916481892091)(127,0.920916481892091)(128,0.920916481892091)(129,0.922509225092251)(130,0.922623434045689)(131,0.923416789396171)(132,0.924208977189109)(133,0.924208977189109)(134,0.924208977189109)(135,0.922623434045689)(136,0.92138133725202)(137,0.916363636363636)(138,0.920236336779911)(139,0.915430267062314)(140,0.912202380952381)(141,0.910581222056632)(142,0.910581222056632)(143,0.910581222056632)(144,0.910581222056632)(145,0.910581222056632)(146,0.907324364723468)(147,0.908140403286034)(148,0.905688622754491)(149,0.90375939849624)(150,0.902934537246049)(151,0.902934537246049)(152,0.901281085154484)(153,0.899622641509434)(154,0.901281085154484)(155,0.901281085154484)(156,0.90210843373494)(157,0.902934537246049)(158,0.902934537246049)(159,0.905405405405405)(160,0.912202380952381)(161,0.913011152416357)(162,0.912202380952381)(163,0.911940298507463)(164,0.914370811615785)(165,0.914370811615785)(166,0.915178571428571)(167,0.914370811615785)(168,0.915985130111524)(169,0.915985130111524)(170,0.915985130111524)(171,0.9150521609538)(172,0.909498878085265)(173,0.910313901345291)(174,0.909498878085265)(175,0.912621359223301)(176,0.909227306826707)(177,0.912751677852349)(178,0.913561847988077)(179,0.914370811615785)(180,0.915304606240713)(181,0.913561847988077)(182,0.905263157894737)(183,0.905263157894737)(184,0.906906906906907)(185,0.903469079939668)(186,0.905263157894737)(187,0.907185628742515)(188,0.902788244159759)(189,0.90936329588015)(190,0.919402985074627)(191,0.919402985074627)(192,0.919402985074627)(193,0.919402985074627)(194,0.919402985074627)(195,0.919402985074627)(196,0.926612305411416)(197,0.925816023738872)(198,0.933726067746686)(199,0.929785661492978)(200,0.930576070901034)(201,0.93841642228739)(202,0.931365313653136)(203,0.921013412816691)(204,0.92020879940343)(205,0.919402985074627)(206,0.919402985074627)(207,0.919402985074627)(208,0.919402985074627)(209,0.91685393258427)(210,0.916604057099925)(211,0.917664670658682)(212,0.918474195961107)(213,0.915915915915916)(214,0.915101427498122)(215,0.915915915915916)(216,0.915915915915916)(217,0.915101427498122)(218,0.915101427498122)(219,0.920777279521674)(220,0.921583271097834)(221,0.926394052044609)(222,0.923994038748137)(223,0.923191648023863)(224,0.925595238095238)(225,0.923994038748137)(226,0.921583271097834)(227,0.928783382789317)(228,0.926394052044609)(229,0.926394052044609)(230,0.926394052044609)(231,0.925595238095238)(232,0.927191679049034)(233,0.927191679049034)(234,0.927191679049034)(235,0.926394052044609)(236,0.925595238095238)(237,0.926394052044609)(238,0.930576070901034)(239,0.930576070901034)(240,0.932153392330383)(241,0.930576070901034)(242,0.930576070901034)(243,0.929785661492978)(244,0.930576070901034)(245,0.931365313653136)(246,0.931365313653136)(247,0.932940309506264)(248,0.932940309506264)(249,0.931365313653136)(250,0.929785661492978)(251,0.931365313653136)(252,0.933726067746686)(253,0.933726067746686)(254,0.932940309506264)(255,0.932940309506264)(256,0.932940309506264)(257,0.932153392330383)(258,0.932940309506264)(259,0.932940309506264)(260,0.936076414401176)(261,0.936170212765957)(262,0.935389133627019)(263,0.935389133627019)(264,0.936950146627566)(265,0.936950146627566)(266,0.936950146627566)(267,0.936076414401176)(268,0.936076414401176)(269,0.935389133627019)(270,0.936170212765957)(271,0.936170212765957)(272,0.936170212765957)(273,0.936170212765957)(274,0.936170212765957)(275,0.936857562408223)(276,0.936857562408223)(277,0.934510669610007)(278,0.936076414401176)(279,0.93519882179676)(280,0.932742054693274)(281,0.93598233995585)(282,0.93598233995585)(283,0.937545922116091)(284,0.938325991189427)(285,0.938325991189427)(286,0.934317343173432)(287,0.934220251293422)(288,0.934220251293422)(289,0.935793357933579)(290,0.939882697947214)(291,0.941520467836257)(292,0.934220251293422)(293,0.934220251293422)(294,0.930752047654505)(295,0.933927245731255)(296,0.931547619047619)(297,0.929955290611028)(298,0.928358208955224)(299,0.917043740573152)(300,0.917043740573152)(301,0.916226415094339)(302,0.917043740573152)(303,0.915407854984894)(304,0.915407854984894)(305,0.936390532544379)(306,0.933234421364985)(307,0.933234421364985)(308,0.933234421364985)(309,0.933927245731255)(310,0.934718100890208)(311,0.928358208955224)(312,0.93560325684678)(313,0.934814814814815)(314,0.934814814814815)(315,0.933234421364985)(316,0.933234421364985)(317,0.933234421364985)(318,0.934814814814815)(319,0.934025203854707)(320,0.934025203854707)(321,0.934025203854707)(322,0.933234421364985)(323,0.933234421364985)(324,0.936390532544379)(325,0.934814814814815)(326,0.934814814814815)(327,0.936578171091445)(328,0.934814814814815)(329,0.937637564196625)(330,0.937637564196625)(331,0.937637564196625)(332,0.940832724616508)(333,0.940832724616508)(334,0.942377826404085)(335,0.942377826404085)(336,0.941605839416058)(337,0.941690962099125)(338,0.941775836972343)(339,0.941775836972343)(340,0.941690962099125)(341,0.941775836972343)(342,0.942461762563729)(343,0.942628903413217)(344,0.944243301955105)(345,0.941176470588235)(346,0.941944847605225)(347,0.941944847605225)(348,0.941176470588235)(349,0.941944847605225)(350,0.943478260869565)(351,0.943478260869565)(352,0.936950146627566)(353,0.937728937728938)(354,0.937728937728938)(355,0.937728937728938)(356,0.93841642228739)(357,0.937454010301692)(358,0.936671575846833)(359,0.93841642228739)(360,0.936857562408223)(361,0.936578171091445)(362,0.936857562408223)(363,0.937269372693727)(364,0.932442464736451)(365,0.935887988209285)(366,0.935887988209285)(367,0.935887988209285)(368,0.935887988209285)(369,0.934317343173432)(370,0.931952662721893)(371,0.931952662721893)(372,0.931851851851852)(373,0.931060044477391)(374,0.937545922116091)(375,0.943231441048035)(376,0.941605839416058)(377,0.941605839416058)(378,0.941605839416058)(379,0.942377826404085)(380,0.942377826404085)(381,0.942377826404085)(382,0.941605839416058)(383,0.940832724616508)(384,0.940058479532164)(385,0.940058479532164)(386,0.940832724616508)(387,0.940832724616508)(388,0.940919037199125)(389,0.940919037199125)(390,0.940919037199125)(391,0.941775836972343)(392,0.937728937728938)(393,0.938686131386861)(394,0.939460247994165)(395,0.940233236151603)(396,0.940233236151603)(397,0.940233236151603)(398,0.941775836972343)(399,0.943313953488372)(400,0.943313953488372)(3,0.537634408602151)(4,0.601333333333333)(5,0.619266055045872)(6,0.71475843812045)(7,0.727645051194539)(8,0.752306600425834)(9,0.760268857356236)(10,0.763058289174867)(11,0.782098312545855)(12,0.782672540381791)(13,0.780911062906724)(14,0.779487179487179)(15,0.784660766961652)(16,0.80334092634776)(17,0.803639120545868)(18,0.800918836140888)(19,0.801223241590214)(20,0.803065134099617)(21,0.799389778794813)(22,0.798151001540832)(23,0.79783950617284)(24,0.801232665639445)(25,0.80061115355233)(26,0.802440884820747)(27,0.807098765432099)(28,0.809160305343511)(29,0.81832298136646)(30,0.819875776397516)(31,0.818604651162791)(32,0.819239720713732)(33,0.818812644564379)(34,0.820906994619523)(35,0.825273010920437)(36,0.825545171339564)(37,0.823161189358372)(38,0.82591725214676)(39,0.823082763857251)(40,0.824439288476411)(41,0.824884792626728)(42,0.826687116564417)(43,0.82859338970023)(44,0.829230769230769)(45,0.831029185867896)(46,0.831926323867997)(47,0.831926323867997)(48,0.819277108433735)(49,0.819894498869631)(50,0.819894498869631)(51,0.82051282051282)(52,0.819894498869631)(53,0.820165537998495)(54,0.81893313298272)(55,0.818045112781955)(56,0.818318318318318)(57,0.814648729446936)(58,0.822373393801965)(59,0.822373393801965)(60,0.81525804038893)(61,0.811615785554728)(62,0.821752265861027)(63,0.82078313253012)(64,0.822641509433962)(65,0.823885109599395)(66,0.822021116138763)(67,0.825757575757576)(68,0.825132475397426)(69,0.825132475397426)(70,0.825757575757576)(71,0.825757575757576)(72,0.826383623957544)(73,0.824508320726172)(74,0.826383623957544)(75,0.828897338403042)(76,0.828267477203647)(77,0.828897338403042)(78,0.828897338403042)(79,0.827638572513288)(80,0.82701062215478)(81,0.824508320726172)(82,0.822641509433962)(83,0.822021116138763)(84,0.821401657874906)(85,0.824508320726172)(86,0.826383623957544)(87,0.822641509433962)(88,0.822641509433962)(89,0.82078313253012)(90,0.821401657874906)(91,0.820165537998495)(92,0.820165537998495)(93,0.81893313298272)(94,0.817091454272864)(95,0.817091454272864)(96,0.820165537998495)(97,0.820165537998495)(98,0.813432835820895)(99,0.814040328603435)(100,0.813432835820895)(101,0.814040328603435)(102,0.81282624906786)(103,0.81282624906786)(104,0.812220566318927)(105,0.818318318318318)(106,0.816479400749064)(107,0.815868263473054)(108,0.816479400749064)(109,0.81525804038893)(110,0.813432835820895)(111,0.813432835820895)(112,0.814040328603435)(113,0.81525804038893)(114,0.813432835820895)(115,0.814648729446936)(116,0.813432835820895)(117,0.814040328603435)(118,0.814648729446936)(119,0.814040328603435)(120,0.814040328603435)(121,0.814648729446936)(122,0.817091454272864)(123,0.81525804038893)(124,0.81525804038893)(125,0.811011904761905)(126,0.81282624906786)(127,0.814040328603435)(128,0.814648729446936)(129,0.814648729446936)(130,0.817704426106526)(131,0.81893313298272)(132,0.823262839879154)(133,0.822641509433962)(134,0.817704426106526)(135,0.819548872180451)(136,0.822021116138763)(137,0.828267477203647)(138,0.828267477203647)(139,0.827638572513288)(140,0.829528158295281)(141,0.83015993907083)(142,0.829528158295281)(143,0.827376425855513)(144,0.827376425855513)(145,0.827376425855513)(146,0.828267477203647)(147,0.828267477203647)(148,0.829268292682927)(149,0.829900839054157)(150,0.831168831168831)(151,0.831168831168831)(152,0.829528158295281)(153,0.818318318318318)(154,0.815868263473054)(155,0.817091454272864)(156,0.817091454272864)(157,0.815868263473054)(158,0.811011904761905)(159,0.811011904761905)(160,0.811615785554728)(161,0.811615785554728)(162,0.811615785554728)(163,0.811615785554728)(164,0.811011904761905)(165,0.811011904761905)(166,0.811011904761905)(167,0.810408921933085)(168,0.80920564216778)(169,0.80920564216778)(170,0.810408921933085)(171,0.810408921933085)(172,0.812220566318927)(173,0.808005930318755)(174,0.807407407407407)(175,0.80920564216778)(176,0.80920564216778)(177,0.808605341246291)(178,0.807407407407407)(179,0.806213017751479)(180,0.808605341246291)(181,0.809806835066865)(182,0.809806835066865)(183,0.80680977054034)(184,0.80680977054034)(185,0.80680977054034)(186,0.80680977054034)(187,0.808005930318755)(188,0.795620437956204)(189,0.802650957290132)(190,0.80206033848418)(191,0.80206033848418)(192,0.803242446573323)(193,0.803834808259587)(194,0.802650957290132)(195,0.802650957290132)(196,0.795040116703136)(197,0.795620437956204)(198,0.799120234604106)(199,0.802650957290132)(200,0.80206033848418)(201,0.802650957290132)(202,0.803242446573323)(203,0.803242446573323)(204,0.80206033848418)(205,0.803242446573323)(206,0.796783625730994)(207,0.805617147080562)(208,0.803834808259587)(209,0.802650957290132)(210,0.804428044280443)(211,0.801470588235294)(212,0.796201607012418)(213,0.79388201019665)(214,0.791575889615105)(215,0.792151162790698)(216,0.792727272727273)(217,0.793304221251819)(218,0.793304221251819)(219,0.79388201019665)(220,0.791575889615105)(221,0.789855072463768)(222,0.789283128167994)(223,0.789283128167994)(224,0.789855072463768)(225,0.791575889615105)(226,0.791575889615105)(227,0.791575889615105)(228,0.791575889615105)(229,0.791575889615105)(230,0.791575889615105)(231,0.791575889615105)(232,0.791575889615105)(233,0.794460641399417)(234,0.79388201019665)(235,0.796201607012418)(236,0.797950219619326)(237,0.799120234604106)(238,0.801470588235294)(239,0.801470588235294)(240,0.800881704628949)(241,0.80206033848418)(242,0.799120234604106)(243,0.80206033848418)(244,0.797366495976591)(245,0.796783625730994)(246,0.796201607012418)(247,0.794460641399417)(248,0.799706529713866)(249,0.797950219619326)(250,0.798534798534798)(251,0.801470588235294)(252,0.801470588235294)(253,0.793304221251819)(254,0.793304221251819)(255,0.795040116703136)(256,0.794460641399417)(257,0.794460641399417)(258,0.795040116703136)(259,0.797950219619326)(260,0.79388201019665)(261,0.79388201019665)(262,0.79388201019665)(263,0.794460641399417)(264,0.794460641399417)(265,0.794460641399417)(266,0.792727272727273)(267,0.79388201019665)(268,0.79388201019665)(269,0.792151162790698)(270,0.79388201019665)(271,0.793304221251819)(272,0.795040116703136)(273,0.795620437956204)(274,0.795620437956204)(275,0.795620437956204)(276,0.795620437956204)(277,0.789855072463768)(278,0.793304221251819)(279,0.793304221251819)(280,0.796783625730994)(281,0.796201607012418)(282,0.796783625730994)(283,0.796783625730994)(284,0.796201607012418)(285,0.796201607012418)(286,0.796201607012418)(287,0.796783625730994)(288,0.797950219619326)(289,0.796201607012418)(290,0.796201607012418)(291,0.796201607012418)(292,0.796201607012418)(293,0.794460641399417)(294,0.79388201019665)(295,0.79388201019665)(296,0.79388201019665)(297,0.794460641399417)(298,0.796201607012418)(299,0.796201607012418)(300,0.795620437956204)(301,0.796201607012418)(302,0.795620437956204)(303,0.795620437956204)(304,0.795620437956204)(305,0.795620437956204)(306,0.799706529713866)(307,0.805022156573117)(308,0.804428044280443)(309,0.804428044280443)(310,0.80920564216778)(311,0.80920564216778)(312,0.810408921933085)(313,0.813432835820895)(314,0.813432835820895)(315,0.813432835820895)(316,0.810408921933085)(317,0.810408921933085)(318,0.808605341246291)(319,0.813432835820895)(320,0.813432835820895)(321,0.813432835820895)(322,0.80680977054034)(323,0.80920564216778)(324,0.808605341246291)(325,0.798534798534798)(326,0.797366495976591)(327,0.798534798534798)(328,0.798534798534798)(329,0.797366495976591)(330,0.799706529713866)(331,0.799706529713866)(332,0.80206033848418)(333,0.799706529713866)(334,0.797366495976591)(335,0.799706529713866)(336,0.795040116703136)(337,0.795040116703136)(338,0.795620437956204)(339,0.795040116703136)(340,0.796201607012418)(341,0.794460641399417)(342,0.79388201019665)(343,0.794460641399417)(344,0.79388201019665)(345,0.79388201019665)(346,0.79388201019665)(347,0.79388201019665)(348,0.79388201019665)(349,0.79388201019665)(350,0.79388201019665)(351,0.795620437956204)(352,0.793304221251819)(353,0.793304221251819)(354,0.792727272727273)(355,0.792727272727273)(356,0.792727272727273)(357,0.792727272727273)(358,0.792727272727273)(359,0.792727272727273)(360,0.792727272727273)(361,0.793304221251819)(362,0.793304221251819)(363,0.789855072463768)(364,0.792727272727273)(365,0.792727272727273)(366,0.793304221251819)(367,0.792727272727273)(368,0.793304221251819)(369,0.793304221251819)(370,0.793304221251819)(371,0.767425810904072)(372,0.779494382022472)(373,0.779094076655052)(374,0.783707865168539)(375,0.785034013605442)(376,0.792251169004676)(377,0.792528352234823)(378,0.793608521970706)(379,0.796791443850267)(380,0.795196797865243)(381,0.795727636849132)(382,0.794137241838774)(383,0.795196797865243)(384,0.796)(385,0.797595190380761)(386,0.792302587923026)(387,0.786561264822134)(388,0.791252485089463)(389,0.791777188328913)(390,0.792828685258964)(391,0.792828685258964)(392,0.790728476821192)(393,0.796531020680453)(394,0.791777188328913)(395,0.791777188328913)(396,0.78968253968254)(397,0.787079762689519)(398,0.787598944591029)(399,0.785526315789474)(400,0.788639365918098)(2,0.501722949689869)(3,0.564750473783954)(4,0.712788259958071)(5,0.71137409598948)(6,0.677860696517413)(7,0.64919594997022)(8,0.633153800119689)(9,0.673508659397049)(10,0.714189643577673)(11,0.751254480286738)(12,0.771826852531181)(13,0.778993435448578)(14,0.782991202346041)(15,0.784400294334069)(16,0.783247612049963)(17,0.780058651026393)(18,0.786135693215339)(19,0.789629629629629)(20,0.789085545722714)(21,0.789940828402367)(22,0.796130952380952)(23,0.792873051224944)(24,0.791111111111111)(25,0.799398948159279)(26,0.810185185185185)(27,0.807604562737643)(28,0.806990881458966)(29,0.808543096872616)(30,0.811926605504587)(31,0.81441717791411)(32,0.815325670498084)(33,0.81441717791411)(34,0.811550151975684)(35,0.810646387832699)(36,0.812167300380228)(37,0.815548780487805)(38,0.813971146545178)(39,0.814814814814815)(40,0.813584905660377)(41,0.809880239520958)(42,0.809880239520958)(43,0.810486891385768)(44,0.817905918057663)(45,0.819771863117871)(46,0.818181818181818)(47,0.818181818181818)(48,0.817905918057663)(49,0.817905918057663)(50,0.818112049117421)(51,0.818112049117421)(52,0.819571865443425)(53,0.817696414950419)(54,0.820198928844682)(55,0.819571865443425)(56,0.82262996941896)(57,0.825153374233129)(58,0.825153374233129)(59,0.82262996941896)(60,0.823529411764706)(61,0.824521072796935)(62,0.820198928844682)(63,0.821455938697318)(64,0.822085889570552)(65,0.822085889570552)(66,0.822085889570552)(67,0.821455938697318)(68,0.819571865443425)(69,0.821729150726855)(70,0.823348694316436)(71,0.822716807367613)(72,0.822358346094946)(73,0.821729150726855)(74,0.822358346094946)(75,0.822358346094946)(76,0.821455938697318)(77,0.817629179331307)(78,0.819497334348819)(79,0.817629179331307)(80,0.817629179331307)(81,0.818526955201215)(82,0.818526955201215)(83,0.818526955201215)(84,0.819497334348819)(85,0.814814814814815)(86,0.81570996978852)(87,0.817091454272864)(88,0.815868263473054)(89,0.817091454272864)(90,0.81525804038893)(91,0.81525804038893)(92,0.815868263473054)(93,0.81893313298272)(94,0.821401657874906)(95,0.821401657874906)(96,0.822021116138763)(97,0.82078313253012)(98,0.82078313253012)(99,0.822641509433962)(100,0.822641509433962)(101,0.822641509433962)(102,0.822641509433962)(103,0.822641509433962)(104,0.823262839879154)(105,0.823262839879154)(106,0.822021116138763)(107,0.82078313253012)(108,0.82078313253012)(109,0.822641509433962)(110,0.822641509433962)(111,0.822641509433962)(112,0.823885109599395)(113,0.823885109599395)(114,0.825132475397426)(115,0.823885109599395)(116,0.823885109599395)(117,0.822641509433962)(118,0.822641509433962)(119,0.825132475397426)(120,0.826383623957544)(121,0.823262839879154)(122,0.822641509433962)(123,0.825132475397426)(124,0.825132475397426)(125,0.822021116138763)(126,0.820165537998495)(127,0.822021116138763)(128,0.822641509433962)(129,0.823262839879154)(130,0.82701062215478)(131,0.82701062215478)(132,0.825757575757576)(133,0.825132475397426)(134,0.825132475397426)(135,0.824508320726172)(136,0.823885109599395)(137,0.82701062215478)(138,0.834742505764796)(139,0.834101382488479)(140,0.835130970724191)(141,0.834101382488479)(142,0.834101382488479)(143,0.834101382488479)(144,0.835384615384615)(145,0.833590138674884)(146,0.831168831168831)(147,0.823885109599395)(148,0.825132475397426)(149,0.827638572513288)(150,0.825132475397426)(151,0.825132475397426)(152,0.824508320726172)(153,0.823262839879154)(154,0.82701062215478)(155,0.826383623957544)(156,0.826383623957544)(157,0.82701062215478)(158,0.825132475397426)(159,0.825132475397426)(160,0.823885109599395)(161,0.829528158295281)(162,0.829528158295281)(163,0.83015993907083)(164,0.833970925784239)(165,0.835249042145594)(166,0.835249042145594)(167,0.835249042145594)(168,0.835249042145594)(169,0.835889570552147)(170,0.835889570552147)(171,0.835889570552147)(172,0.835889570552147)(173,0.835889570552147)(174,0.835889570552147)(175,0.835889570552147)(176,0.834996162701458)(177,0.835889570552147)(178,0.841368584758942)(179,0.841368584758942)(180,0.842679127725857)(181,0.842679127725857)(182,0.842679127725857)(183,0.842679127725857)(184,0.842679127725857)(185,0.840873634945398)(186,0.841860465116279)(187,0.841860465116279)(188,0.835249042145594)(189,0.834609494640122)(190,0.834609494640122)(191,0.833970925784239)(192,0.828267477203647)(193,0.830792682926829)(194,0.829528158295281)(195,0.828897338403042)(196,0.827638572513288)(197,0.833333333333333)(198,0.835249042145594)(199,0.836531082118189)(200,0.835249042145594)(201,0.835249042145594)(202,0.835249042145594)(203,0.835889570552147)(204,0.839907192575406)(205,0.839907192575406)(206,0.842679127725857)(207,0.841614906832298)(208,0.844961240310077)(209,0.837173579109063)(210,0.844306738962045)(211,0.842349304482226)(212,0.842349304482226)(213,0.842349304482226)(214,0.842349304482226)(215,0.842349304482226)(216,0.843653250773994)(217,0.839753466872111)(218,0.841208365608056)(219,0.837962962962963)(220,0.839907192575406)(221,0.841614906832298)(222,0.842679127725857)(223,0.846761453396524)(224,0.846761453396524)(225,0.846153846153846)(226,0.846761453396524)(227,0.846703733121525)(228,0.846703733121525)(229,0.844094488188976)(230,0.845425867507886)(231,0.844094488188976)(232,0.842845973416732)(233,0.843354430379747)(234,0.843354430379747)(235,0.844759653270291)(236,0.843183609141056)(237,0.849162011173184)(238,0.845360824742268)(239,0.844339622641509)(240,0.845360824742268)(241,0.847376788553259)(242,0.847376788553259)(243,0.84756584197925)(244,0.848242811501597)(245,0.84756584197925)(246,0.84664536741214)(247,0.848242811501597)(248,0.847322142286171)(249,0.8464)(250,0.8464)(251,0.844551282051282)(252,0.848242811501597)(253,0.849402390438247)(254,0.849402390438247)(255,0.849402390438247)(256,0.849402390438247)(257,0.849402390438247)(258,0.849402390438247)(259,0.848772763262074)(260,0.8464)(261,0.8464)(262,0.8464)(263,0.845476381104884)(264,0.846153846153846)(265,0.845476381104884)(266,0.842443729903537)(267,0.844051446945338)(268,0.843121480289622)(269,0.843121480289622)(270,0.843121480289622)(271,0.843121480289622)(272,0.844051446945338)(273,0.844051446945338)(274,0.848484848484848)(275,0.848484848484848)(276,0.848484848484848)(277,0.848484848484848)(278,0.852614896988906)(279,0.853291038858049)(280,0.853291038858049)(281,0.84756584197925)(282,0.848)(283,0.846153846153846)(284,0.84756584197925)(285,0.849402390438247)(286,0.843373493975903)(287,0.841512469831054)(288,0.841512469831054)(289,0.845723421262989)(290,0.851233094669849)(291,0.853291038858049)(292,0.853291038858049)(293,0.853291038858049)(294,0.85193982581156)(295,0.851030110935024)(296,0.85193982581156)(297,0.85193982581156)(298,0.85193982581156)(299,0.85126582278481)(300,0.85193982581156)(301,0.850592885375494)(302,0.852380952380952)(303,0.850793650793651)(304,0.850793650793651)(305,0.850793650793651)(306,0.850556438791733)(307,0.843875100080064)(308,0.8448)(309,0.843875100080064)(310,0.843875100080064)(311,0.8448)(312,0.8448)(313,0.850793650793651)(314,0.850793650793651)(315,0.84805091487669)(316,0.848484848484848)(317,0.847133757961783)(318,0.85126582278481)(319,0.852173913043478)(320,0.852173913043478)(321,0.852173913043478)(322,0.85240726124704)(323,0.85240726124704)(324,0.85240726124704)(325,0.851531814611155)(326,0.849529780564263)(327,0.850897736143638)(328,0.850897736143638)(329,0.852201257861635)(330,0.849250197316495)(331,0.849487785657998)(332,0.850196078431372)(333,0.852201257861635)(334,0.850828729281768)(335,0.851968503937008)(336,0.852173913043478)(337,0.850793650793651)(338,0.851500789889415)(339,0.852173913043478)(340,0.852173913043478)(341,0.851030110935024)(342,0.851030110935024)(343,0.850118953211737)(344,0.851739788199697)(345,0.849811320754717)(346,0.848530519969857)(347,0.850263355906697)(348,0.855842185128983)(349,0.853415195702226)(350,0.861702127659574)(351,0.870893812070283)(352,0.869631901840491)(353,0.868624420401855)(354,0.865280985373364)(355,0.865280985373364)(356,0.865073245952197)(357,0.865533230293663)(358,0.865900383141762)(359,0.867283950617284)(360,0.867749419953596)(361,0.86687306501548)(362,0.866409266409266)(363,0.872699386503067)(364,0.874425727411945)(365,0.873369148119724)(366,0.874809160305343)(367,0.872340425531915)(368,0.872810357958873)(369,0.874141876430206)(370,0.871559633027523)(371,0.874904067536454)(372,0.874039938556067)(373,0.876058506543495)(374,0.87673343605547)(375,0.878951426368543)(376,0.877598152424942)(377,0.877409406322282)(378,0.87673343605547)(379,0.876923076923077)(380,0.877409406322282)(381,0.876058506543495)(382,0.876923076923077)(383,0.876058506543495)(384,0.881226053639847)(385,0.882938026013772)(386,0.887366818873668)(387,0.886519421172887)(388,0.884820747520976)(389,0.886519421172887)(390,0.889057750759878)(391,0.888212927756654)(392,0.888382687927107)(393,0.889226100151745)(394,0.889393939393939)(395,0.88855193328279)(396,0.890068233510235)(397,0.889393939393939)(398,0.88855193328279)(399,0.88855193328279)(400,0.88855193328279)(7,0.528)(8,0.619896492236918)(9,0.611329220415031)(10,0.656479217603912)(11,0.700520833333333)(12,0.757894736842105)(13,0.793725490196078)(14,0.766847405112316)(15,0.78783151326053)(16,0.791967871485944)(17,0.790322580645161)(18,0.79967689822294)(19,0.80354267310789)(20,0.789902280130293)(21,0.820791311093871)(22,0.793180133432172)(23,0.808972503617945)(24,0.814869888475836)(25,0.826725403817915)(26,0.826470588235294)(27,0.831896551724138)(28,0.840348330914369)(29,0.840958605664488)(30,0.850290697674419)(31,0.851556842867487)(32,0.843615494978479)(33,0.849415204678362)(34,0.850036576444769)(35,0.852531181217902)(36,0.820405310971349)(37,0.829681978798587)(38,0.83026874115983)(39,0.829234012649332)(40,0.826666666666667)(41,0.81522491349481)(42,0.809621993127148)(43,0.812974465148378)(44,0.817487855655795)(45,0.827828531271961)(46,0.821478382147838)(47,0.811853893866299)(48,0.82033426183844)(49,0.828551434569629)(50,0.827972027972028)(51,0.825087108013937)(52,0.830294530154278)(53,0.830877192982456)(54,0.815934065934066)(55,0.812585499316006)(56,0.813141683778234)(57,0.821008984105045)(58,0.847092605886576)(59,0.851002865329513)(60,0.841281138790036)(61,0.840085287846482)(62,0.840085287846482)(63,0.841281138790036)(64,0.836158192090395)(65,0.832280701754386)(66,0.828212290502793)(67,0.827634333565946)(68,0.829951014695591)(69,0.827057182705718)(70,0.824183460736623)(71,0.827972027972028)(72,0.833684210526316)(73,0.832515767344078)(74,0.828451882845188)(75,0.82787456445993)(76,0.829030006978367)(77,0.829608938547486)(78,0.829370629370629)(79,0.827634333565946)(80,0.830532212885154)(81,0.839802399435427)(82,0.84375)(83,0.841956059532247)(84,0.84016973125884)(85,0.84016973125884)(86,0.84375)(87,0.840764331210191)(88,0.840764331210191)(89,0.838390966831334)(90,0.838390966831334)(91,0.838390966831334)(92,0.844729344729345)(93,0.845331432644333)(94,0.845551601423488)(95,0.84375)(96,0.840764331210191)(97,0.842553191489362)(98,0.842553191489362)(99,0.843373493975904)(100,0.842776203966006)(101,0.842179759377212)(102,0.83861874559549)(103,0.841359773371105)(104,0.836849507735584)(105,0.835674157303371)(106,0.835674157303371)(107,0.835087719298246)(108,0.835904628330996)(109,0.83298392732355)(110,0.831241283124128)(111,0.83182135380321)(112,0.835904628330996)(113,0.838255977496484)(114,0.838255977496484)(115,0.834733893557423)(116,0.836491228070175)(117,0.839210155148096)(118,0.839210155148096)(119,0.836261419536191)(120,0.836849507735584)(121,0.838390966831334)(122,0.841359773371105)(123,0.840764331210191)(124,0.839210155148096)(125,0.84016973125884)(126,0.842776203966006)(127,0.840989399293286)(128,0.840989399293286)(129,0.839802399435427)(130,0.845934379457917)(131,0.845551601423488)(132,0.845551601423488)(133,0.84796573875803)(134,0.849785407725322)(135,0.850393700787402)(136,0.849570200573066)(137,0.850179211469534)(138,0.849785407725322)(139,0.849177984274482)(140,0.849785407725322)(141,0.849177984274482)(142,0.845331432644333)(143,0.845331432644333)(144,0.84375)(145,0.844570617459191)(146,0.843373493975904)(147,0.843971631205674)(148,0.843971631205674)(149,0.845170454545454)(150,0.845170454545454)(151,0.845170454545454)(152,0.846372688477952)(153,0.844570617459191)(154,0.843373493975904)(155,0.842776203966006)(156,0.842179759377212)(157,0.843373493975904)(158,0.841584158415841)(159,0.843971631205674)(160,0.846975088967971)(161,0.845170454545454)(162,0.846372688477952)(163,0.846372688477952)(164,0.846372688477952)(165,0.846975088967971)(166,0.846975088967971)(167,0.845170454545454)(168,0.845170454545454)(169,0.846372688477952)(170,0.846372688477952)(171,0.845170454545454)(172,0.844570617459191)(173,0.845170454545454)(174,0.844570617459191)(175,0.846372688477952)(176,0.841584158415841)(177,0.843373493975904)(178,0.843151171043293)(179,0.83743842364532)(180,0.846153846153846)(181,0.848571428571429)(182,0.849785407725322)(183,0.849785407725322)(184,0.849785407725322)(185,0.849785407725322)(186,0.846153846153846)(187,0.84796573875803)(188,0.84796573875803)(189,0.84796573875803)(190,0.849177984274482)(191,0.850393700787402)(192,0.851002865329513)(193,0.849785407725322)(194,0.850393700787402)(195,0.851612903225806)(196,0.851002865329513)(197,0.853448275862069)(198,0.853448275862069)(199,0.862119013062409)(200,0.854676258992806)(201,0.854271356783919)(202,0.858381502890173)(203,0.857142857142857)(204,0.854271356783919)(205,0.856115107913669)(206,0.854885057471264)(207,0.85243553008596)(208,0.853658536585366)(209,0.85243553008596)(210,0.859002169197397)(211,0.859623733719247)(212,0.855907780979827)(213,0.854061826024443)(214,0.856731461483081)(215,0.856115107913669)(216,0.856115107913669)(217,0.855907780979827)(218,0.861493836113125)(219,0.861493836113125)(220,0.860869565217391)(221,0.860869565217391)(222,0.860869565217391)(223,0.861292665214234)(224,0.861918604651163)(225,0.863173216885007)(226,0.863173216885007)(227,0.861918604651163)(228,0.861918604651163)(229,0.861918604651163)(230,0.861918604651163)(231,0.861918604651163)(232,0.862545454545454)(233,0.861918604651163)(234,0.861292665214234)(235,0.863173216885007)(236,0.865693430656934)(237,0.865061998541211)(238,0.86380189366351)(239,0.865693430656934)(240,0.857761732851986)(241,0.872512896094326)(242,0.875739644970414)(243,0.874446085672083)(244,0.87380073800738)(245,0.874260355029586)(246,0.872968980797637)(247,0.872324723247232)(248,0.872324723247232)(249,0.86950146627566)(250,0.868864468864469)(251,0.868864468864469)(252,0.866959064327485)(253,0.867593269934162)(254,0.866959064327485)(255,0.866764275256222)(256,0.867593269934162)(257,0.86380189366351)(258,0.864431486880466)(259,0.860246198406951)(260,0.859623733719247)(261,0.859623733719247)(262,0.853658536585366)(263,0.8506075768406)(264,0.85)(265,0.853046594982079)(266,0.84180790960452)(267,0.840989399293286)(268,0.840989399293286)(269,0.840620592383639)(270,0.840620592383639)(271,0.840620592383639)(272,0.840620592383639)(273,0.840620592383639)(274,0.840620592383639)(275,0.846372688477952)(276,0.846372688477952)(277,0.846372688477952)(278,0.846372688477952)(279,0.846372688477952)(280,0.845771144278607)(281,0.845771144278607)(282,0.860246198406951)(283,0.851216022889843)(284,0.851825340014316)(285,0.853658536585366)(286,0.853658536585366)(287,0.854271356783919)(288,0.854271356783919)(289,0.854271356783919)(290,0.854271356783919)(291,0.853046594982079)(292,0.847578347578347)(293,0.846372688477952)(294,0.848787446504993)(295,0.848182466143977)(296,0.848182466143977)(297,0.848787446504993)(298,0.848787446504993)(299,0.848787446504993)(300,0.853658536585366)(301,0.853658536585366)(302,0.853658536585366)(303,0.856524873828407)(304,0.857348703170029)(305,0.857348703170029)(306,0.859205776173285)(307,0.859205776173285)(308,0.859205776173285)(309,0.872058823529412)(310,0.872058823529412)(311,0.873986735445836)(312,0.873343151693667)(313,0.873986735445836)(314,0.873986735445836)(315,0.873986735445836)(316,0.871418074944893)(317,0.872058823529412)(318,0.874631268436578)(319,0.874631268436578)(320,0.873986735445836)(321,0.851216022889843)(322,0.854271356783919)(323,0.854271356783919)(324,0.854271356783919)(325,0.854271356783919)(326,0.85243553008596)(327,0.854271356783919)(328,0.854885057471264)(329,0.854271356783919)(330,0.836491228070175)(331,0.833798882681564)(332,0.833798882681564)(333,0.833798882681564)(334,0.834381551362683)(335,0.833217027215632)(336,0.834965034965035)(337,0.83554933519944)(338,0.837307152875175)(339,0.837307152875175)(340,0.838483146067416)(341,0.838255977496484)(342,0.838255977496484)(343,0.838255977496484)(344,0.84121383203952)(345,0.840620592383639)(346,0.83943661971831)(347,0.840028188865398)(348,0.84121383203952)(349,0.84180790960452)(350,0.83943661971831)(351,0.840620592383639)(352,0.840620592383639)(353,0.83943661971831)(354,0.83943661971831)(355,0.83943661971831)(356,0.83943661971831)(357,0.84479092841956)(358,0.844192634560906)(359,0.844192634560906)(360,0.843595187544232)(361,0.847192608386638)(362,0.844192634560906)(363,0.844192634560906)(364,0.840253342716397)(365,0.840253342716397)(366,0.840253342716397)(367,0.837307152875175)(368,0.837307152875175)(369,0.837894736842105)(370,0.837894736842105)(371,0.842031029619182)(372,0.848614072494669)(373,0.849002849002849)(374,0.849002849002849)(375,0.849002849002849)(376,0.849002849002849)(377,0.849002849002849)(378,0.850213980028531)(379,0.855707106963388)(380,0.85632183908046)(381,0.857553956834532)(382,0.857553956834532)(383,0.857553956834532)(384,0.856937455068296)(385,0.842031029619182)(386,0.842625264643613)(387,0.843220338983051)(388,0.843816254416961)(389,0.842625264643613)(390,0.842625264643613)(391,0.842625264643613)(392,0.842625264643613)(393,0.852646638054363)(394,0.852646638054363)(395,0.853868194842407)(396,0.853256979241231)(397,0.850213980028531)(398,0.845609065155807)(399,0.847409510290986)(400,0.847409510290986)(11,0.711379879054425)(12,0.718426501035197)(13,0.732631578947368)(14,0.745708154506438)(15,0.747854077253219)(16,0.751757706868578)(17,0.739711384286478)(18,0.740106951871658)(19,0.707631318136769)(20,0.727087576374745)(21,0.729508196721311)(22,0.729911871435977)(23,0.744186046511628)(24,0.791157649796393)(25,0.793002915451895)(26,0.792276184903452)(27,0.784472769409038)(28,0.781017724413951)(29,0.788863109048724)(30,0.798780487804878)(31,0.820089001907183)(32,0.820447284345048)(33,0.823454429572976)(34,0.823379923761118)(35,0.822631913541004)(36,0.824716267339218)(37,0.826781326781327)(38,0.826273787599754)(39,0.847673677501593)(40,0.855670103092783)(41,0.855119124275595)(42,0.853720050441362)(43,0.858052196053469)(44,0.855154965211891)(45,0.857142857142857)(46,0.856055802155992)(47,0.85479797979798)(48,0.85479797979798)(49,0.857324032974001)(50,0.855696202531645)(51,0.85623812539582)(52,0.85623812539582)(53,0.860743541272842)(54,0.8585795097423)(55,0.859671302149178)(56,0.871794871794872)(57,0.873896595208071)(58,0.868636077938403)(59,0.868636077938403)(60,0.870277078085642)(61,0.876275510204082)(62,0.874602164226607)(63,0.874602164226607)(64,0.874602164226607)(65,0.872935196950445)(66,0.874121405750799)(67,0.880722114764668)(68,0.880204996796925)(69,0.882163554410818)(70,0.882011605415861)(71,0.882011605415861)(72,0.883720930232558)(73,0.884441575209813)(74,0.891361256544502)(75,0.892367906066536)(76,0.892508143322476)(77,0.892508143322476)(78,0.891786179921773)(79,0.890909090909091)(80,0.890909090909091)(81,0.888745148771022)(82,0.890038809831824)(83,0.889032258064516)(84,0.887741935483871)(85,0.888888888888889)(86,0.889896373056995)(87,0.890767230169051)(88,0.893840104849279)(89,0.896325459317585)(90,0.896325459317585)(91,0.895463510848126)(92,0.888599348534202)(93,0.885139519792343)(94,0.885714285714286)(95,0.889902280130293)(96,0.890767230169051)(97,0.890188434048083)(98,0.891786179921773)(99,0.890635232481991)(100,0.890635232481991)(101,0.894632206759443)(102,0.894806924101198)(103,0.896276595744681)(104,0.895542248835662)(105,0.896138482023968)(106,0.895542248835662)(107,0.897282968853545)(108,0.897418927862343)(109,0.897418927862343)(110,0.898473788984738)(111,0.900066181336863)(112,0.900262467191601)(113,0.900853578463559)(114,0.902038132807363)(115,0.900723208415516)(116,0.902166776099803)(117,0.900983606557377)(118,0.900783289817232)(119,0.903098220171391)(120,0.901909150757077)(121,0.901909150757077)(122,0.901909150757077)(123,0.904887714663144)(124,0.902503293807641)(125,0.900983606557377)(126,0.899934597776324)(127,0.898758981058132)(128,0.89960886571056)(129,0.901041666666667)(130,0.902470741222367)(131,0.901756668835394)(132,0.901041666666667)(133,0.902216427640156)(134,0.902216427640156)(135,0.90150032615786)(136,0.902088772845953)(137,0.902677988242978)(138,0.903267973856209)(139,0.901960784313725)(140,0.901628664495114)(141,0.900195694716243)(142,0.901960784313725)(143,0.903141361256544)(144,0.9043250327654)(145,0.901960784313725)(146,0.901960784313725)(147,0.901960784313725)(148,0.901960784313725)(149,0.901242642249836)(150,0.899672131147541)(151,0.892604501607717)(152,0.892604501607717)(153,0.895927601809955)(154,0.893178893178893)(155,0.892742453436095)(156,0.894193548387097)(157,0.899082568807339)(158,0.900262467191601)(159,0.900853578463559)(160,0.901445466491458)(161,0.905360688285903)(162,0.916387959866221)(163,0.913333333333333)(164,0.913448735019973)(165,0.912724850099933)(166,0.915163660654643)(167,0.918120805369127)(168,0.919678714859438)(169,0.919973100201748)(170,0.918846411804158)(171,0.918955123911587)(172,0.925675675675676)(173,0.926301555104801)(174,0.926301555104801)(175,0.926301555104801)(176,0.926301555104801)(177,0.926301555104801)(178,0.926301555104801)(179,0.926928281461434)(180,0.928378378378378)(181,0.927848954821308)(182,0.927848954821308)(183,0.927187708750835)(184,0.917709019091507)(185,0.917709019091507)(186,0.917105263157895)(187,0.917105263157895)(188,0.918918918918919)(189,0.918918918918919)(190,0.917709019091507)(191,0.913499344692005)(192,0.912303664921466)(193,0.911706998037933)(194,0.911706998037933)(195,0.912418300653595)(196,0.91324200913242)(197,0.914657980456026)(198,0.914657980456026)(199,0.914657980456026)(200,0.914657980456026)(201,0.9140625)(202,0.915254237288135)(203,0.916179337231969)(204,0.916179337231969)(205,0.916775032509753)(206,0.916775032509753)(207,0.916179337231969)(208,0.916179337231969)(209,0.914990266060999)(210,0.915584415584416)(211,0.913212435233161)(212,0.912031047865459)(213,0.912031047865459)(214,0.910264686894771)(215,0.911441499676794)(216,0.912621359223301)(217,0.913804277381724)(218,0.916179337231969)(219,0.916179337231969)(220,0.914396887159533)(221,0.91273432449903)(222,0.914990266060999)(223,0.915584415584416)(224,0.916179337231969)(225,0.916179337231969)(226,0.916179337231969)(227,0.916179337231969)(228,0.916179337231969)(229,0.913804277381724)(230,0.913804277381724)(231,0.912031047865459)(232,0.914396887159533)(233,0.914396887159533)(234,0.914396887159533)(235,0.914396887159533)(236,0.912621359223301)(237,0.909090909090909)(238,0.911441499676794)(239,0.911555842479019)(240,0.912258064516129)(241,0.925804333552199)(242,0.925804333552199)(243,0.925804333552199)(244,0.925804333552199)(245,0.929372937293729)(246,0.927021696252465)(247,0.927021696252465)(248,0.920469361147327)(249,0.916396629941672)(250,0.916396629941672)(251,0.92156862745098)(252,0.920469361147327)(253,0.920469361147327)(254,0.922774869109947)(255,0.920966688438929)(256,0.920365535248042)(257,0.92156862745098)(258,0.922774869109947)(259,0.920966688438929)(260,0.923379174852652)(261,0.923379174852652)(262,0.924083769633508)(263,0.922976501305483)(264,0.922976501305483)(265,0.924787442773054)(266,0.925998690242305)(267,0.925392670157068)(268,0.925392670157068)(269,0.925392670157068)(270,0.927213114754098)(271,0.92965154503616)(272,0.92904073587385)(273,0.930263157894737)(274,0.930263157894737)(275,0.930263157894737)(276,0.932102834541859)(277,0.932102834541859)(278,0.932102834541859)(279,0.932102834541859)(280,0.933862433862434)(281,0.934480476505625)(282,0.935719019218025)(283,0.937583001328021)(284,0.938829787234042)(285,0.939454424484365)(286,0.939454424484365)(287,0.939454424484365)(288,0.941255006675567)(289,0.940627084723149)(290,0.941176470588235)(291,0.941806020066889)(292,0.941806020066889)(293,0.942513368983957)(294,0.942513368983957)(295,0.94188376753507)(296,0.94188376753507)(297,0.94)(298,0.938748335552597)(299,0.939373750832778)(300,0.938748335552597)(301,0.93812375249501)(302,0.93812375249501)(303,0.93812375249501)(304,0.93812375249501)(305,0.937583001328021)(306,0.936085219707057)(307,0.936708860759493)(308,0.936708860759493)(309,0.940388479571333)(310,0.940388479571333)(311,0.940388479571333)(312,0.940388479571333)(313,0.940388479571333)(314,0.941018766756032)(315,0.940939597315436)(316,0.938502673796791)(317,0.939759036144578)(318,0.940388479571333)(319,0.939130434782609)(320,0.938041305796136)(321,0.937416777629827)(322,0.938041305796136)(323,0.938041305796136)(324,0.938666666666667)(325,0.938666666666667)(326,0.938666666666667)(327,0.941728064300067)(328,0.94243641231593)(329,0.941176470588235)(330,0.942513368983957)(331,0.942513368983957)(332,0.941806020066889)(333,0.940547762191049)(334,0.942513368983957)(335,0.942513368983957)(336,0.943775100401606)(337,0.944407233757535)(338,0.944407233757535)(339,0.944407233757535)(340,0.944407233757535)(341,0.945746818486269)(342,0.943219772879091)(343,0.94448160535117)(344,0.945040214477212)(345,0.944407233757535)(346,0.943775100401606)(347,0.945040214477212)(348,0.945040214477212)(349,0.945040214477212)(350,0.944407233757535)(351,0.941333333333333)(352,0.941333333333333)(353,0.942513368983957)(354,0.941961307538359)(355,0.941961307538359)(356,0.943850267379679)(357,0.94448160535117)(358,0.947796610169491)(359,0.947796610169491)(360,0.946367956551256)(361,0.946367956551256)(362,0.946440677966102)(363,0.946440677966102)(364,0.948579161028417)(365,0.948579161028417)(366,0.948787061994609)(367,0.95)(368,0.95)(369,0.947086403215003)(370,0.946452476572959)(371,0.946452476572959)(372,0.947086403215003)(373,0.946452476572959)(374,0.946452476572959)(375,0.945187165775401)(376,0.946524064171123)(377,0.945891783567134)(378,0.945891783567134)(379,0.945891783567134)(380,0.945891783567134)(381,0.944629753168779)(382,0.942743009320905)(383,0.942743009320905)(384,0.942743009320905)(385,0.942743009320905)(386,0.942666666666667)(387,0.94392523364486)(388,0.94392523364486)(389,0.94392523364486)(390,0.942038640906063)(391,0.942038640906063)(392,0.942038640906063)(393,0.942666666666667)(394,0.942038640906063)(395,0.942666666666667)(396,0.942666666666667)(397,0.941411451398136)(398,0.940785096473719)(399,0.942666666666667)(400,0.938992042440318)(1,0.601626016260162)(2,0.584461867426942)(3,0.578018995929444)(4,0.567024128686327)(5,0.675759454432734)(6,0.728253540121375)(7,0.717105263157895)(8,0.747252747252747)(9,0.756834532374101)(10,0.765772298767222)(11,0.773584905660377)(12,0.770788141720896)(13,0.770348837209302)(14,0.77542062911485)(15,0.713939393939394)(16,0.687981053878034)(17,0.703971119133574)(18,0.704531722054381)(19,0.727047146401985)(20,0.755220417633411)(21,0.757894736842105)(22,0.758356164383561)(23,0.74825550187869)(24,0.749865519096288)(25,0.741883980840873)(26,0.749462365591398)(27,0.749865519096288)(28,0.753921038399135)(29,0.771084337349397)(30,0.811661807580175)(31,0.806022003474233)(32,0.801381692573402)(33,0.8)(34,0.801381692573402)(35,0.803158488437676)(36,0.80225352112676)(37,0.796866256295467)(38,0.784615384615384)(39,0.782037239868565)(40,0.771474878444084)(41,0.777354382144801)(42,0.78003291278113)(43,0.780889621087315)(44,0.775354416575791)(45,0.774931880108992)(46,0.770731707317073)(47,0.759209823812066)(48,0.7584)(49,0.7584)(50,0.757591901971231)(51,0.752779248279513)(52,0.764420485175202)(53,0.756785524215008)(54,0.76048858204992)(55,0.760892667375133)(56,0.753943217665615)(57,0.756329113924051)(58,0.761297182349814)(59,0.773462783171521)(60,0.776572668112798)(61,0.77363587250135)(62,0.77363587250135)(63,0.77363587250135)(64,0.77995642701525)(65,0.781045751633987)(66,0.779347826086956)(67,0.783178590933916)(68,0.778501628664495)(69,0.775554353704705)(70,0.778501628664495)(71,0.782751091703057)(72,0.784034991798797)(73,0.785323110624315)(74,0.784893267651888)(75,0.787912087912088)(76,0.789212988442488)(77,0.787912087912088)(78,0.801565120178871)(79,0.802013422818792)(80,0.80561797752809)(81,0.803811659192825)(82,0.802013422818792)(83,0.800670016750419)(84,0.801117318435754)(85,0.790518191841235)(86,0.787912087912088)(87,0.795782463928968)(88,0.797109505280711)(89,0.804713804713805)(90,0.805165637282426)(91,0.804713804713805)(92,0.809255079006772)(93,0.814310051107325)(94,0.819897084048027)(95,0.821305841924398)(96,0.819897084048027)(97,0.824137931034483)(98,0.821305841924398)(99,0.822247706422018)(100,0.802911534154535)(101,0.803361344537815)(102,0.807432432432432)(103,0.80652418447694)(104,0.804262478968031)(105,0.802911534154535)(106,0.802911534154535)(107,0.790082644628099)(108,0.79490022172949)(109,0.79490022172949)(110,0.796224319822321)(111,0.793580520199225)(112,0.791390728476821)(113,0.787912087912088)(114,0.790082644628099)(115,0.790082644628099)(116,0.781471389645776)(117,0.778501628664495)(118,0.77765726681128)(119,0.772213247172859)(120,0.773462783171521)(121,0.774298056155507)(122,0.775135135135135)(123,0.76931330472103)(124,0.767665952890792)(125,0.77055346587856)(126,0.778924497555676)(127,0.780195865070729)(128,0.781471389645776)(129,0.773045822102426)(130,0.775554353704705)(131,0.775974025974026)(132,0.776394152680022)(133,0.775974025974026)(134,0.775135135135135)(135,0.775554353704705)(136,0.775974025974026)(137,0.776814734561213)(138,0.77765726681128)(139,0.778501628664495)(140,0.778501628664495)(141,0.778501628664495)(142,0.789647577092511)(143,0.788345244639912)(144,0.787047200878156)(145,0.786615469007131)(146,0.785323110624315)(147,0.785753424657534)(148,0.785323110624315)(149,0.787479406919275)(150,0.788778877887789)(151,0.793141592920354)(152,0.791827719491993)(153,0.791827719491993)(154,0.790518191841235)(155,0.787479406919275)(156,0.786184210526316)(157,0.786184210526316)(158,0.785753424657534)(159,0.786184210526316)(160,0.787479406919275)(161,0.790954219525648)(162,0.790518191841235)(163,0.790082644628099)(164,0.790518191841235)(165,0.792703150912106)(166,0.792703150912106)(167,0.792703150912106)(168,0.789212988442488)(169,0.785753424657534)(170,0.784893267651888)(171,0.784463894967177)(172,0.784463894967177)(173,0.787047200878156)(174,0.787479406919275)(175,0.785753424657534)(176,0.784463894967177)(177,0.783606557377049)(178,0.784034991798797)(179,0.774716369529984)(180,0.776814734561213)(181,0.775974025974026)(182,0.775554353704705)(183,0.771797631862217)(184,0.773045822102426)(185,0.775974025974026)(186,0.777235772357723)(187,0.781471389645776)(188,0.780195865070729)(189,0.776394152680022)(190,0.777235772357723)(191,0.777235772357723)(192,0.776394152680022)(193,0.77807921866522)(194,0.776814734561213)(195,0.776814734561213)(196,0.776394152680022)(197,0.770967741935484)(198,0.772629310344828)(199,0.772629310344828)(200,0.773462783171521)(201,0.779347826086956)(202,0.804713804713805)(203,0.802013422818792)(204,0.807432432432432)(205,0.808342728297632)(206,0.809712027103331)(207,0.809255079006772)(208,0.808798646362098)(209,0.809712027103331)(210,0.808798646362098)(211,0.808798646362098)(212,0.811544991511036)(213,0.810169491525424)(214,0.810169491525424)(215,0.809255079006772)(216,0.811085972850679)(217,0.808798646362098)(218,0.809255079006772)(219,0.809255079006772)(220,0.809712027103331)(221,0.808342728297632)(222,0.808342728297632)(223,0.808342728297632)(224,0.808342728297632)(225,0.807887323943662)(226,0.807432432432432)(227,0.807432432432432)(228,0.808342728297632)(229,0.808342728297632)(230,0.808342728297632)(231,0.808342728297632)(232,0.808798646362098)(233,0.807887323943662)(234,0.807887323943662)(235,0.806070826306914)(236,0.806978052898143)(237,0.80561797752809)(238,0.809712027103331)(239,0.808798646362098)(240,0.808342728297632)(241,0.806978052898143)(242,0.806978052898143)(243,0.806978052898143)(244,0.806978052898143)(245,0.806978052898143)(246,0.807432432432432)(247,0.807432432432432)(248,0.808798646362098)(249,0.808798646362098)(250,0.808342728297632)(251,0.816628701594533)(252,0.814772727272727)(253,0.815235929505401)(254,0.816163915765509)(255,0.817094017094017)(256,0.817094017094017)(257,0.818493150684931)(258,0.819897084048027)(259,0.818960593946316)(260,0.822247706422018)(261,0.822247706422018)(262,0.821305841924398)(263,0.821305841924398)(264,0.821776504297994)(265,0.821305841924398)(266,0.820835718374356)(267,0.821305841924398)(268,0.822247706422018)(269,0.822247706422018)(270,0.820835718374356)(271,0.820835718374356)(272,0.820366132723112)(273,0.820366132723112)(274,0.819897084048027)(275,0.826036866359447)(276,0.826036866359447)(277,0.825086306098964)(278,0.825086306098964)(279,0.824137931034483)(280,0.822719449225473)(281,0.823191733639495)(282,0.822719449225473)(283,0.823191733639495)(284,0.822719449225473)(285,0.823191733639495)(286,0.823191733639495)(287,0.824611845888442)(288,0.824611845888442)(289,0.825086306098964)(290,0.823664560597358)(291,0.822247706422018)(292,0.822719449225473)(293,0.820835718374356)(294,0.820835718374356)(295,0.820835718374356)(296,0.820835718374356)(297,0.820835718374356)(298,0.820366132723112)(299,0.820366132723112)(300,0.819428571428571)(301,0.818960593946316)(302,0.818026240730177)(303,0.818026240730177)(304,0.817094017094017)(305,0.825561312607945)(306,0.826036866359447)(307,0.826036866359447)(308,0.824137931034483)(309,0.822719449225473)(310,0.822719449225473)(311,0.822719449225473)(312,0.823664560597358)(313,0.823664560597358)(314,0.822719449225473)(315,0.822719449225473)(316,0.821305841924398)(317,0.822247706422018)(318,0.823191733639495)(319,0.824137931034483)(320,0.823664560597358)(321,0.822719449225473)(322,0.822719449225473)(323,0.822719449225473)(324,0.820366132723112)(325,0.818493150684931)(326,0.817094017094017)(327,0.815699658703072)(328,0.815235929505401)(329,0.815235929505401)(330,0.814772727272727)(331,0.814772727272727)(332,0.824611845888442)(333,0.822247706422018)(334,0.821305841924398)(335,0.816163915765509)(336,0.818493150684931)(337,0.815235929505401)(338,0.816628701594533)(339,0.816628701594533)(340,0.817094017094017)(341,0.817094017094017)(342,0.817094017094017)(343,0.816628701594533)(344,0.830341632889403)(345,0.830341632889403)(346,0.830341632889403)(347,0.828901734104046)(348,0.829861111111111)(349,0.829381145170619)(350,0.828901734104046)(351,0.830341632889403)(352,0.830341632889403)(353,0.830341632889403)(354,0.829861111111111)(355,0.829381145170619)(356,0.829381145170619)(357,0.826036866359447)(358,0.822247706422018)(359,0.823191733639495)(360,0.823191733639495)(361,0.823191733639495)(362,0.823664560597358)(363,0.823664560597358)(364,0.823664560597358)(365,0.823664560597358)(366,0.824137931034483)(367,0.824137931034483)(368,0.824137931034483)(369,0.824137931034483)(370,0.827944572748268)(371,0.826512968299712)(372,0.827466820542412)(373,0.827466820542412)(374,0.829381145170619)(375,0.82842287694974)(376,0.82842287694974)(377,0.828901734104046)(378,0.823191733639495)(379,0.823191733639495)(380,0.823664560597358)(381,0.825086306098964)(382,0.825086306098964)(383,0.825086306098964)(384,0.819897084048027)(385,0.819897084048027)(386,0.819897084048027)(387,0.820835718374356)(388,0.820835718374356)(389,0.820835718374356)(390,0.821305841924398)(391,0.821305841924398)(392,0.840562719812427)(393,0.841549295774648)(394,0.841549295774648)(395,0.840562719812427)(396,0.840562719812427)(397,0.840562719812427)(398,0.840070298769771)(399,0.842043452730476)(400,0.842043452730476)(8,0.662137121680049)(9,0.660110633066994)(10,0.686446392447741)(11,0.686274509803921)(12,0.758620689655172)(13,0.773228346456693)(14,0.772870662460568)(15,0.777690494893951)(16,0.782945736434108)(17,0.798479087452471)(18,0.798165137614679)(19,0.792626728110599)(20,0.803459119496855)(21,0.806299212598425)(22,0.804399057344855)(23,0.81578947368421)(24,0.815444015444015)(25,0.816705336426914)(26,0.816074188562597)(27,0.826086956521739)(28,0.826086956521739)(29,0.823248407643312)(30,0.829073482428115)(31,0.825320512820513)(32,0.829736211031175)(33,0.82570281124498)(34,0.8288)(35,0.826538768984812)(36,0.8268156424581)(37,0.826538768984812)(38,0.827476038338658)(39,0.829770387965162)(40,0.829770387965162)(41,0.830278884462151)(42,0.825775656324582)(43,0.825775656324582)(44,0.82670906200318)(45,0.82807570977918)(46,0.829268292682927)(47,0.825296442687747)(48,0.825949367088607)(49,0.831746031746032)(50,0.831746031746032)(51,0.829888712241653)(52,0.828458498023715)(53,0.828729281767956)(54,0.830427892234548)(55,0.830427892234548)(56,0.830427892234548)(57,0.831746031746032)(58,0.833464877663773)(59,0.834123222748815)(60,0.835043409629045)(61,0.834384858044164)(62,0.833070866141732)(63,0.832678711704635)(64,0.832678711704635)(65,0.832678711704635)(66,0.832678711704635)(67,0.831107619795758)(68,0.830070477682067)(69,0.829420970266041)(70,0.830721003134796)(71,0.830721003134796)(72,0.830070477682067)(73,0.830070477682067)(74,0.828125)(75,0.828772478498827)(76,0.830455259026687)(77,0.830721003134796)(78,0.834375)(79,0.835541699142634)(80,0.836052836052836)(81,0.834108527131783)(82,0.834108527131783)(83,0.834108527131783)(84,0.830888030888031)(85,0.833462432223083)(86,0.834755624515128)(87,0.834108527131783)(88,0.832173240525909)(89,0.830246913580247)(90,0.831530139103555)(91,0.8328173374613)(92,0.833462432223083)(93,0.833462432223083)(94,0.832173240525909)(95,0.833075734157651)(96,0.832432432432432)(97,0.835658914728682)(98,0.832432432432432)(99,0.83179012345679)(100,0.838258164852255)(101,0.83695652173913)(102,0.836307214895268)(103,0.836307214895268)(104,0.83695652173913)(105,0.836307214895268)(106,0.838258164852255)(107,0.838258164852255)(108,0.837606837606838)(109,0.837606837606838)(110,0.837606837606838)(111,0.838258164852255)(112,0.838258164852255)(113,0.839563862928349)(114,0.838910505836576)(115,0.838910505836576)(116,0.835658914728682)(117,0.836813611755607)(118,0.835011618900077)(119,0.836307214895268)(120,0.836307214895268)(121,0.836307214895268)(122,0.836307214895268)(123,0.837606837606838)(124,0.838910505836576)(125,0.838258164852255)(126,0.837606837606838)(127,0.838509316770186)(128,0.838509316770186)(129,0.839160839160839)(130,0.839410395655547)(131,0.834876543209876)(132,0.834876543209876)(133,0.835521235521235)(134,0.836813611755607)(135,0.834876543209876)(136,0.833590138674884)(137,0.834232845026985)(138,0.831288343558282)(139,0.827480916030534)(140,0.830015313935681)(141,0.833333333333333)(142,0.833976833976834)(143,0.835011618900077)(144,0.832307692307692)(145,0.823351023502653)(146,0.819622641509434)(147,0.818387339864356)(148,0.818387339864356)(149,0.818387339864356)(150,0.815315315315315)(151,0.815315315315315)(152,0.819004524886878)(153,0.818387339864356)(154,0.818387339864356)(155,0.817155756207675)(156,0.817771084337349)(157,0.816541353383459)(158,0.819004524886878)(159,0.818387339864356)(160,0.818318318318318)(161,0.818318318318318)(162,0.819548872180451)(163,0.818318318318318)(164,0.821212121212121)(165,0.819969742813918)(166,0.820590461771385)(167,0.822188449848024)(168,0.827056110684089)(169,0.826687116564417)(170,0.825954198473282)(171,0.826219512195122)(172,0.824961948249619)(173,0.821482602118003)(174,0.824334600760456)(175,0.82370820668693)(176,0.822813688212928)(177,0.824067022086824)(178,0.825057295645531)(179,0.82542113323124)(180,0.826053639846743)(181,0.825153374233129)(182,0.824521072796935)(183,0.824521072796935)(184,0.82642089093702)(185,0.827056110684089)(186,0.827956989247312)(187,0.82859338970023)(188,0.82859338970023)(189,0.832173240525909)(190,0.832173240525909)(191,0.832173240525909)(192,0.826687116564417)(193,0.825688073394495)(194,0.824789594491201)(195,0.824789594491201)(196,0.824789594491201)(197,0.824789594491201)(198,0.824789594491201)(199,0.849596478356566)(200,0.851063829787234)(201,0.850220264317181)(202,0.848351648351648)(203,0.85037037037037)(204,0.854612546125461)(205,0.855243722304284)(206,0.857142857142857)(207,0.857142857142857)(208,0.858823529411765)(209,0.861538461538461)(210,0.861741038771031)(211,0.857350800582241)(212,0.85610465116279)(213,0.85985401459854)(214,0.85985401459854)(215,0.860262008733624)(216,0.857764876632801)(217,0.857764876632801)(218,0.857764876632801)(219,0.860262008733624)(220,0.856521739130435)(221,0.861516034985423)(222,0.861516034985423)(223,0.862773722627737)(224,0.862144420131291)(225,0.861516034985423)(226,0.861516034985423)(227,0.862144420131291)(228,0.861943024105186)(229,0.861516034985423)(230,0.861516034985423)(231,0.861516034985423)(232,0.85838779956427)(233,0.860888565185725)(234,0.861516034985423)(235,0.861516034985423)(236,0.861516034985423)(237,0.862144420131291)(238,0.861516034985423)(239,0.862773722627737)(240,0.862773722627737)(241,0.862773722627737)(242,0.862773722627737)(243,0.862773722627737)(244,0.862773722627737)(245,0.862144420131291)(246,0.863403944485025)(247,0.862144420131291)(248,0.862144420131291)(249,0.862144420131291)(250,0.862144420131291)(251,0.862144420131291)(252,0.860888565185725)(253,0.860888565185725)(254,0.862144420131291)(255,0.862573099415205)(256,0.861741038771031)(257,0.861741038771031)(258,0.862573099415205)(259,0.861313868613139)(260,0.861313868613139)(261,0.861943024105186)(262,0.861943024105186)(263,0.861943024105186)(264,0.861943024105186)(265,0.863204096561814)(266,0.863204096561814)(267,0.863204096561814)(268,0.863204096561814)(269,0.863204096561814)(270,0.863204096561814)(271,0.863204096561814)(272,0.863204096561814)(273,0.863204096561814)(274,0.861313868613139)(275,0.862573099415205)(276,0.862573099415205)(277,0.862573099415205)(278,0.862573099415205)(279,0.862573099415205)(280,0.862573099415205)(281,0.862573099415205)(282,0.861313868613139)(283,0.861313868613139)(284,0.862573099415205)(285,0.863204096561814)(286,0.863204096561814)(287,0.863836017569546)(288,0.863836017569546)(289,0.863836017569546)(290,0.863836017569546)(291,0.863204096561814)(292,0.864468864468864)(293,0.865102639296188)(294,0.865102639296188)(295,0.865102639296188)(296,0.864468864468864)(297,0.862545454545454)(298,0.862545454545454)(299,0.863173216885007)(300,0.861918604651163)(301,0.861918604651163)(302,0.861292665214234)(303,0.861292665214234)(304,0.861292665214234)(305,0.861292665214234)(306,0.861918604651163)(307,0.860869565217391)(308,0.860869565217391)(309,0.861493836113125)(310,0.861493836113125)(311,0.86231884057971)(312,0.861694424330195)(313,0.859205776173285)(314,0.861493836113125)(315,0.861493836113125)(316,0.862745098039216)(317,0.862944162436548)(318,0.861918604651163)(319,0.861918604651163)(320,0.867009551800147)(321,0.868924889543446)(322,0.870206489675516)(323,0.870206489675516)(324,0.868479059515062)(325,0.869117647058823)(326,0.869117647058823)(327,0.866568914956012)(328,0.865934065934066)(329,0.864667154352597)(330,0.864667154352597)(331,0.863173216885007)(332,0.86380189366351)(333,0.864233576642336)(334,0.865300146412884)(335,0.864667154352597)(336,0.870139398385913)(337,0.86797957695113)(338,0.868613138686131)(339,0.867786705624543)(340,0.868421052631579)(341,0.868421052631579)(342,0.868421052631579)(343,0.869056327724945)(344,0.869692532942899)(345,0.869692532942899)(346,0.869692532942899)(347,0.870519385515728)(348,0.868613138686131)(349,0.869883040935672)(350,0.869883040935672)(351,0.869247626004383)(352,0.869247626004383)(353,0.869883040935672)(354,0.869247626004383)(355,0.869247626004383)(356,0.869247626004383)(357,0.869247626004383)(358,0.869247626004383)(359,0.869883040935672)(360,0.869883040935672)(361,0.869247626004383)(362,0.869247626004383)(363,0.869247626004383)(364,0.869247626004383)(365,0.870778267254038)(366,0.870778267254038)(367,0.87039764359352)(368,0.871681415929204)(369,0.871681415929204)(370,0.870778267254038)(371,0.870778267254038)(372,0.872058823529412)(373,0.872058823529412)(374,0.871418074944893)(375,0.871418074944893)(376,0.868613138686131)(377,0.86797957695113)(378,0.86797957695113)(379,0.866715222141296)(380,0.866715222141296)(381,0.866715222141296)(382,0.861894432393348)(383,0.861894432393348)(384,0.861694424330195)(385,0.862944162436548)(386,0.863570391872278)(387,0.863570391872278)(388,0.862944162436548)(389,0.872058823529412)(390,0.873343151693667)(391,0.873986735445836)(392,0.872700515084621)(393,0.872058823529412)(394,0.873343151693667)(395,0.873986735445836)(396,0.871606749816581)(397,0.866715222141296)(398,0.87032967032967)(399,0.870967741935484)(400,0.869692532942899)(2,0.541826554105909)(3,0.557536466774716)(4,0.527974783293932)(5,0.712816997943797)(6,0.712816997943797)(7,0.711141678129298)(8,0.706043956043956)(9,0.718023255813953)(10,0.724871606749817)(11,0.743838685586258)(12,0.77639751552795)(13,0.775956284153005)(14,0.788298691301001)(15,0.788207913110939)(16,0.790297339593114)(17,0.782012195121951)(18,0.782012195121951)(19,0.78234398782344)(20,0.781367392937641)(21,0.793478260869565)(22,0.793774319066148)(23,0.805471124620061)(24,0.80336648814078)(25,0.802741812642803)(26,0.802741812642803)(27,0.800608828006088)(28,0.833453496755587)(29,0.839683680805176)(30,0.835243553008596)(31,0.834408602150538)(32,0.839363241678726)(33,0.836838288614938)(34,0.833333333333333)(35,0.832735104091888)(36,0.833810888252149)(37,0.836286321757619)(38,0.839488636363636)(39,0.851093860268172)(40,0.855735397607319)(41,0.86950146627566)(42,0.869818181818182)(43,0.872278664731495)(44,0.876534296028881)(45,0.875912408759124)(46,0.875636363636364)(47,0.876093294460641)(48,0.878013148283418)(49,0.882697947214076)(50,0.88222384784199)(51,0.882738528769119)(52,0.87536231884058)(53,0.875985663082437)(54,0.879199428162973)(55,0.880458124552613)(56,0.884285714285714)(57,0.883321403006442)(58,0.882140274765004)(59,0.882778581765557)(60,0.886832740213523)(61,0.884831460674157)(62,0.885292047853624)(63,0.885292047853624)(64,0.891089108910891)(65,0.893617021276595)(66,0.888256227758007)(67,0.889518413597734)(68,0.889518413597734)(69,0.88936170212766)(70,0.888415067519545)(71,0.888573456352023)(72,0.890469416785206)(73,0.889684813753582)(74,0.888412017167382)(75,0.888412017167382)(76,0.89)(77,0.888096935138988)(78,0.887303851640513)(79,0.889836531627576)(80,0.88936170212766)(81,0.89484827099506)(82,0.89484827099506)(83,0.89484827099506)(84,0.895627644569817)(85,0.89484827099506)(86,0.89484827099506)(87,0.895921237693389)(88,0.894662921348314)(89,0.897130860741777)(90,0.896358543417367)(91,0.89727463312369)(92,0.896648044692737)(93,0.896648044692737)(94,0.896648044692737)(95,0.896648044692737)(96,0.895877009084556)(97,0.894957983193277)(98,0.896213183730715)(99,0.898102600140548)(100,0.899366643209008)(101,0.901685393258427)(102,0.901329601119664)(103,0.901329601119664)(104,0.900280898876404)(105,0.900913562895291)(106,0.900913562895291)(107,0.900913562895291)(108,0.90014064697609)(109,0.89950808151792)(110,0.901191310441485)(111,0.900837988826816)(112,0.902729181245626)(113,0.903631284916201)(114,0.904662491301322)(115,0.904033379694019)(116,0.903405142460042)(117,0.903405142460042)(118,0.907058001397624)(119,0.906293706293706)(120,0.906030855539972)(121,0.902404526166902)(122,0.903043170559094)(123,0.90148830616584)(124,0.902767920511001)(125,0.903043170559094)(126,0.903043170559094)(127,0.903954802259887)(128,0.901905434015526)(129,0.903453136011275)(130,0.903089887640449)(131,0.903859649122807)(132,0.903859649122807)(133,0.903859649122807)(134,0.903859649122807)(135,0.902319044272663)(136,0.902319044272663)(137,0.904225352112676)(138,0.904862579281184)(139,0.905367231638418)(140,0.904593639575972)(141,0.904593639575972)(142,0.904593639575972)(143,0.904593639575972)(144,0.904593639575972)(145,0.904593639575972)(146,0.903954802259887)(147,0.903954802259887)(148,0.903954802259887)(149,0.904593639575972)(150,0.904593639575972)(151,0.904593639575972)(152,0.903682719546742)(153,0.905606813342796)(154,0.904323175053154)(155,0.904323175053154)(156,0.904323175053154)(157,0.905606813342796)(158,0.908961593172119)(159,0.908961593172119)(160,0.910384068278805)(161,0.909608540925267)(162,0.913597733711048) 
};
\addplot [
color=orange,
mark size=0.1pt,
only marks,
mark=*,
mark options={solid,fill=black},
forget plot
]
coordinates{
 (162,0.913597733711048)(163,0.914487632508834)(164,0.914487632508834)(165,0.914487632508834)(166,0.913841807909604)(167,0.913841807909604)(168,0.913074204946996)(169,0.911785462244178)(170,0.912429378531073)(171,0.913074204946996)(172,0.913074204946996)(173,0.913074204946996)(174,0.913597733711048)(175,0.912552891396333)(176,0.910112359550562)(177,0.910112359550562)(178,0.907563025210084)(179,0.909473684210526)(180,0.909473684210526)(181,0.907563025210084)(182,0.905528341497551)(183,0.908835904628331)(184,0.908707865168539)(185,0.909473684210526)(186,0.909473684210526)(187,0.910112359550562)(188,0.912033779028853)(189,0.911392405063291)(190,0.910751932536894)(191,0.910751932536894)(192,0.910751932536894)(193,0.910112359550562)(194,0.910751932536894)(195,0.911764705882353)(196,0.912648497554158)(197,0.913165266106442)(198,0.913805185704275)(199,0.913043478260869)(200,0.912403644008409)(201,0.912403644008409)(202,0.913043478260869)(203,0.912133891213389)(204,0.912133891213389)(205,0.912133891213389)(206,0.911250873515024)(207,0.912011173184358)(208,0.912011173184358)(209,0.910739191073919)(210,0.910739191073919)(211,0.909217877094972)(212,0.911764705882353)(213,0.911764705882353)(214,0.912403644008409)(215,0.913043478260869)(216,0.913684210526316)(217,0.914772727272727)(218,0.915611814345991)(219,0.914968376669009)(220,0.914968376669009)(221,0.917134831460674)(222,0.915492957746479)(223,0.91484869809993)(224,0.915492957746479)(225,0.915492957746479)(226,0.915373765867419)(227,0.923186344238976)(228,0.923186344238976)(229,0.924059616749468)(230,0.924715909090909)(231,0.923512747875354)(232,0.922859164897381)(233,0.924274593064402)(234,0.922859164897381)(235,0.925501432664756)(236,0.924929178470255)(237,0.928774928774929)(238,0.926829268292683)(239,0.927038626609442)(240,0.927598566308244)(241,0.927598566308244)(242,0.928263988522238)(243,0.928263988522238)(244,0.926270579813887)(245,0.926270579813887)(246,0.92494639027877)(247,0.924285714285714)(248,0.923843416370107)(249,0.924608819345661)(250,0.924393723252496)(251,0.925714285714286)(252,0.929936305732484)(253,0.929378531073446)(254,0.928722653493296)(255,0.930035335689046)(256,0.930693069306931)(257,0.928722653493296)(258,0.929478138222849)(259,0.929478138222849)(260,0.930035335689046)(261,0.929378531073446)(262,0.928419560595322)(263,0.927762039660057)(264,0.929178470254957)(265,0.929278642149929)(266,0.928621908127208)(267,0.929278642149929)(268,0.932480454868515)(269,0.930496453900709)(270,0.932480454868515)(271,0.931914893617021)(272,0.931818181818182)(273,0.931818181818182)(274,0.931818181818182)(275,0.930496453900709)(276,0.930496453900709)(277,0.929836995038979)(278,0.929836995038979)(279,0.929178470254957)(280,0.928520877565463)(281,0.928520877565463)(282,0.928621908127208)(283,0.927966101694915)(284,0.927966101694915)(285,0.926657263751763)(286,0.928067700987306)(287,0.930134086097389)(288,0.930134086097389)(289,0.928169014084507)(290,0.929478138222849)(291,0.928823114869626)(292,0.926863572433193)(293,0.927515833919775)(294,0.927617709065355)(295,0.932287954383464)(296,0.932287954383464)(297,0.933618843683083)(298,0.933618843683083)(299,0.931526390870185)(300,0.931526390870185)(301,0.933618843683083)(302,0.931623931623932)(303,0.931721194879089)(304,0.932287954383464)(305,0.934285714285714)(306,0.933523945675482)(307,0.934285714285714)(308,0.932761087267525)(309,0.933428775948461)(310,0.931232091690544)(311,0.930465949820788)(312,0.929799426934097)(313,0.930465949820788)(314,0.931330472103004)(315,0.929799426934097)(316,0.929799426934097)(317,0.930366116295764)(318,0.931997136721546)(319,0.931997136721546)(320,0.929799426934097)(321,0.929799426934097)(322,0.930265995686556)(323,0.93103448275862)(324,0.930366116295764)(325,0.93103448275862)(326,0.93103448275862)(327,0.931133428981348)(328,0.931133428981348)(329,0.926864590876176)(330,0.926864590876176)(331,0.927745664739884)(332,0.926193921852388)(333,0.926193921852388)(334,0.926300578034682)(335,0.929394812680115)(336,0.92882818116463)(337,0.928622927180966)(338,0.928622927180966)(339,0.926618705035971)(340,0.927953890489913)(341,0.929799426934097)(342,0.931330472103004)(343,0.92816091954023)(344,0.92816091954023)(345,0.929698708751793)(346,0.92882818116463)(347,0.92882818116463)(348,0.92882818116463)(349,0.925524222704266)(350,0.924746743849494)(351,0.925631768953068)(352,0.931997136721546)(353,0.931997136721546)(354,0.934379457917261)(355,0.934285714285714)(356,0.932761087267525)(357,0.931997136721546)(358,0.93016558675306)(359,0.930935251798561)(360,0.930935251798561)(361,0.930935251798561)(362,0.930935251798561)(363,0.930535455861071)(364,0.928779069767442)(365,0.927431059506531)(366,0.93304535637149)(367,0.932276657060519)(368,0.932276657060519)(369,0.932276657060519)(370,0.932276657060519)(371,0.932276657060519)(372,0.932276657060519)(373,0.932276657060519)(374,0.932276657060519)(375,0.93294881038212)(376,0.934579439252336)(377,0.934579439252336)(378,0.934579439252336)(379,0.935437589670014)(380,0.935622317596566)(381,0.935046395431834)(382,0.935622317596566)(383,0.935622317596566)(384,0.935622317596566)(385,0.935622317596566)(386,0.935622317596566)(387,0.934953538241601)(388,0.934953538241601)(389,0.933618843683083)(390,0.933618843683083)(391,0.934285714285714)(392,0.934285714285714)(393,0.932287954383464)(394,0.932287954383464)(395,0.932287954383464)(396,0.930960854092526)(397,0.931623931623932)(398,0.931623931623932)(399,0.930960854092526)(400,0.93019943019943)(10,0.508796956728483)(11,0.624413145539906)(12,0.661764705882353)(13,0.667763157894737)(14,0.77845220030349)(15,0.760709010339734)(16,0.798425196850394)(17,0.810980392156863)(18,0.813639968279144)(19,0.80358598207009)(20,0.820634920634921)(21,0.818471337579618)(22,0.825545171339564)(23,0.823529411764706)(24,0.824547600314713)(25,0.825471698113207)(26,0.822517591868647)(27,0.823993685872139)(28,0.823993685872139)(29,0.823993685872139)(30,0.824644549763033)(31,0.82484076433121)(32,0.825949367088607)(33,0.8256735340729)(34,0.8224)(35,0.821371610845295)(36,0.821371610845295)(37,0.826086956521739)(38,0.829073482428115)(39,0.832535885167464)(40,0.832666132906325)(41,0.8336)(42,0.833200319233839)(43,0.832268370607029)(44,0.834796488427773)(45,0.834405144694534)(46,0.831587429492345)(47,0.830645161290322)(48,0.834267413931145)(49,0.834001603849238)(50,0.833333333333333)(51,0.833333333333333)(52,0.8330658105939)(53,0.827079934747145)(54,0.829943043124491)(55,0.827079934747145)(56,0.826688364524003)(57,0.828315703824247)(58,0.825448613376835)(59,0.824489795918367)(60,0.824489795918367)(61,0.824489795918367)(62,0.824489795918367)(63,0.823240589198036)(64,0.822276822276822)(65,0.820638820638821)(66,0.824489795918367)(67,0.824489795918367)(68,0.820638820638821)(69,0.821311475409836)(70,0.821311475409836)(71,0.821311475409836)(72,0.820344544708778)(73,0.820344544708778)(74,0.822276822276822)(75,0.822276822276822)(76,0.826797385620915)(77,0.827473426001635)(78,0.825552825552825)(79,0.825552825552825)(80,0.825552825552825)(81,0.824590163934426)(82,0.823625922887613)(83,0.818780889621087)(84,0.822660098522167)(85,0.825552825552825)(86,0.820723684210526)(87,0.820723684210526)(88,0.819753086419753)(89,0.823625922887613)(90,0.823625922887613)(91,0.829665851670741)(92,0.827755102040816)(93,0.831570382424735)(94,0.832520325203252)(95,0.83319772172498)(96,0.826513911620295)(97,0.836304700162074)(98,0.837510105092967)(99,0.840836012861736)(100,0.845723421262989)(101,0.843875100080064)(102,0.8448)(103,0.8448)(104,0.842948717948718)(105,0.843624699278268)(106,0.843624699278268)(107,0.845476381104884)(108,0.845476381104884)(109,0.848242811501597)(110,0.847808764940239)(111,0.850556438791733)(112,0.851469420174742)(113,0.849642004773269)(114,0.849642004773269)(115,0.849642004773269)(116,0.850079744816587)(117,0.850079744816587)(118,0.850079744816587)(119,0.850079744816587)(120,0.850079744816587)(121,0.850079744816587)(122,0.850079744816587)(123,0.850079744816587)(124,0.850079744816587)(125,0.850079744816587)(126,0.850079744816587)(127,0.850079744816587)(128,0.850079744816587)(129,0.850079744816587)(130,0.850079744816587)(131,0.850079744816587)(132,0.850079744816587)(133,0.854646544876886)(134,0.850079744816587)(135,0.854646544876886)(136,0.857369255150555)(137,0.857369255150555)(138,0.859177215189873)(139,0.865460267505901)(140,0.866352201257861)(141,0.869905956112852)(142,0.868131868131868)(143,0.871674491392801)(144,0.871674491392801)(145,0.871674491392801)(146,0.871674491392801)(147,0.871674491392801)(148,0.876071706936867)(149,0.875195007800312)(150,0.872556684910086)(151,0.872556684910086)(152,0.871674491392801)(153,0.871674491392801)(154,0.869905956112852)(155,0.868131868131868)(156,0.865460267505901)(157,0.866352201257861)(158,0.869905956112852)(159,0.869905956112852)(160,0.869019607843137)(161,0.870790916209867)(162,0.876947040498442)(163,0.87431693989071)(164,0.876947040498442)(165,0.876648564778898)(166,0.876264591439689)(167,0.876071706936867)(168,0.876071706936867)(169,0.8734375)(170,0.87431693989071)(171,0.875195007800312)(172,0.875195007800312)(173,0.87431693989071)(174,0.8734375)(175,0.87431693989071)(176,0.87431693989071)(177,0.87431693989071)(178,0.87431693989071)(179,0.875195007800312)(180,0.875195007800312)(181,0.874120406567631)(182,0.87956487956488)(183,0.87956487956488)(184,0.880434782608696)(185,0.880434782608696)(186,0.878693623639191)(187,0.880434782608696)(188,0.880434782608696)(189,0.878693623639191)(190,0.883397683397683)(191,0.883397683397683)(192,0.885119506553585)(193,0.885119506553585)(194,0.884259259259259)(195,0.884259259259259)(196,0.881987577639751)(197,0.881118881118881)(198,0.881118881118881)(199,0.881118881118881)(200,0.880248833592535)(201,0.876264591439689)(202,0.876264591439689)(203,0.875389408099688)(204,0.874512860483242)(205,0.875583203732504)(206,0.874512860483242)(207,0.874512860483242)(208,0.874512860483242)(209,0.875195007800312)(210,0.875195007800312)(211,0.871956009426551)(212,0.876755070202808)(213,0.877630553390491)(214,0.877630553390491)(215,0.868400315208826)(216,0.868400315208826)(217,0.866614048934491)(218,0.858505564387917)(219,0.862123613312203)(220,0.86122125297383)(221,0.86122125297383)(222,0.863024544734759)(223,0.862123613312203)(224,0.862123613312203)(225,0.862123613312203)(226,0.862123613312203)(227,0.862123613312203)(228,0.86122125297383)(229,0.863024544734759)(230,0.865718799368088)(231,0.866614048934491)(232,0.866614048934491)(233,0.869291338582677)(234,0.871069182389937)(235,0.871069182389937)(236,0.870180959874115)(237,0.872841444270016)(238,0.872841444270016)(239,0.873725490196078)(240,0.873725490196078)(241,0.877630553390491)(242,0.87956487956488)(243,0.877630553390491)(244,0.876755070202808)(245,0.872841444270016)(246,0.87548942834769)(247,0.878125)(248,0.878125)(249,0.87743950039032)(250,0.878125)(251,0.878125)(252,0.8798751950078)(253,0.8798751950078)(254,0.8798751950078)(255,0.879000780640125)(256,0.879000780640125)(257,0.879000780640125)(258,0.879000780640125)(259,0.877247849882721)(260,0.878125)(261,0.878125)(262,0.876369327073552)(263,0.8798751950078)(264,0.879000780640125)(265,0.88074824629774)(266,0.88074824629774)(267,0.88074824629774)(268,0.879000780640125)(269,0.878125)(270,0.879000780640125)(271,0.87548942834769)(272,0.87548942834769)(273,0.87548942834769)(274,0.873725490196078)(275,0.87548942834769)(276,0.869291338582677)(277,0.870180959874115)(278,0.865506329113924)(279,0.865506329113924)(280,0.861904761904762)(281,0.860095389507154)(282,0.864608076009501)(283,0.864608076009501)(284,0.864608076009501)(285,0.868192580899763)(286,0.864608076009501)(287,0.864608076009501)(288,0.864608076009501)(289,0.864608076009501)(290,0.869976359338061)(291,0.871754523996853)(292,0.870866141732283)(293,0.869976359338061)(294,0.869085173501577)(295,0.869085173501577)(296,0.869085173501577)(297,0.869085173501577)(298,0.869085173501577)(299,0.872841444270016)(300,0.878315132605304)(301,0.884406516679596)(302,0.883540372670807)(303,0.873923257635082)(304,0.88006230529595)(305,0.878315132605304)(306,0.88006230529595)(307,0.873923257635082)(308,0.873923257635082)(309,0.873923257635082)(310,0.873923257635082)(311,0.873923257635082)(312,0.865506329113924)(313,0.865506329113924)(314,0.866403162055336)(315,0.862807295796986)(316,0.862807295796986)(317,0.861904761904762)(318,0.861904761904762)(319,0.869085173501577)(320,0.873527101335428)(321,0.874608150470219)(322,0.875294117647059)(323,0.874608150470219)(324,0.874608150470219)(325,0.874608150470219)(326,0.873725490196078)(327,0.874608150470219)(328,0.874608150470219)(329,0.873725490196078)(330,0.873725490196078)(331,0.872641509433962)(332,0.872641509433962)(333,0.870866141732283)(334,0.870866141732283)(335,0.869976359338061)(336,0.870866141732283)(337,0.869085173501577)(338,0.867298578199052)(339,0.867298578199052)(340,0.868192580899763)(341,0.868192580899763)(342,0.867298578199052)(343,0.871754523996853)(344,0.871754523996853)(345,0.88056206088993)(346,0.881619937694704)(347,0.873527101335428)(348,0.870866141732283)(349,0.878125)(350,0.873725490196078)(351,0.872841444270016)(352,0.873725490196078)(353,0.872841444270016)(354,0.872841444270016)(355,0.872841444270016)(356,0.871754523996853)(357,0.872841444270016)(358,0.871956009426551)(359,0.871956009426551)(360,0.871956009426551)(361,0.871956009426551)(362,0.871956009426551)(363,0.872841444270016)(364,0.872841444270016)(365,0.872841444270016)(366,0.872841444270016)(367,0.873725490196078)(368,0.873725490196078)(369,0.873725490196078)(370,0.874608150470219)(371,0.873725490196078)(372,0.878125)(373,0.879000780640125)(374,0.879000780640125)(375,0.879000780640125)(376,0.879000780640125)(377,0.879000780640125)(378,0.878125)(379,0.879000780640125)(380,0.879000780640125)(381,0.878125)(382,0.878125)(383,0.878125)(384,0.877247849882721)(385,0.877247849882721)(386,0.877247849882721)(387,0.881619937694704)(388,0.881619937694704)(389,0.8798751950078)(390,0.8798751950078)(391,0.8798751950078)(392,0.8798751950078)(393,0.8798751950078)(394,0.87548942834769)(395,0.871754523996853)(396,0.871754523996853)(397,0.885271317829457)(398,0.885271317829457)(399,0.882672882672883)(400,0.882672882672883)(2,0.513235294117647)(4,0.577215189873418)(5,0.657552973342447)(6,0.681277395115842)(7,0.720265780730897)(8,0.727150537634408)(9,0.768033946251768)(10,0.770682148040639)(11,0.767375886524823)(12,0.770550393137956)(13,0.798219584569733)(14,0.7889634601044)(15,0.797583081570997)(16,0.798185941043084)(17,0.804841149773071)(18,0.804232804232804)(19,0.815650865312265)(20,0.817155756207675)(21,0.818387339864356)(22,0.81588999236058)(23,0.816858237547893)(24,0.81651376146789)(25,0.811746987951807)(26,0.817629179331307)(27,0.815151515151515)(28,0.807201800450112)(29,0.8095952023988)(30,0.812075471698113)(31,0.813303099017385)(32,0.803254437869822)(33,0.806547619047619)(34,0.805947955390334)(35,0.799703264094955)(36,0.800891530460624)(37,0.802680565897245)(38,0.802680565897245)(39,0.802680565897245)(40,0.811506434519303)(41,0.818740399385561)(42,0.825885978428351)(43,0.826319816373374)(44,0.826687116564417)(45,0.826219512195122)(46,0.826219512195122)(47,0.825590251332825)(48,0.824601366742597)(49,0.825855513307985)(50,0.825855513307985)(51,0.825227963525836)(52,0.823082763857251)(53,0.82370820668693)(54,0.819349962207105)(55,0.819349962207105)(56,0.819969742813918)(57,0.819969742813918)(58,0.819969742813918)(59,0.8202416918429)(60,0.820861678004535)(61,0.819894498869631)(62,0.819894498869631)(63,0.819894498869631)(64,0.821132075471698)(65,0.822373393801965)(66,0.822104466313399)(67,0.822104466313399)(68,0.820861678004535)(69,0.821482602118003)(70,0.821482602118003)(71,0.821482602118003)(72,0.820861678004535)(73,0.820861678004535)(74,0.822727272727273)(75,0.822995461422088)(76,0.822373393801965)(77,0.822373393801965)(78,0.822373393801965)(79,0.822373393801965)(80,0.823618470855413)(81,0.823351023502653)(82,0.823351023502653)(83,0.823351023502653)(84,0.824867323730099)(85,0.827376425855513)(86,0.828636709824829)(87,0.827376425855513)(88,0.827376425855513)(89,0.828006088280061)(90,0.830275229357798)(91,0.830275229357798)(92,0.830275229357798)(93,0.832183908045977)(94,0.831546707503828)(95,0.831546707503828)(96,0.831546707503828)(97,0.831546707503828)(98,0.830910482019893)(99,0.831546707503828)(100,0.831546707503828)(101,0.831546707503828)(102,0.831546707503828)(103,0.831546707503828)(104,0.828530259365994)(105,0.827485380116959)(106,0.827485380116959)(107,0.825674690007294)(108,0.833333333333333)(109,0.833333333333333)(110,0.831187410586552)(111,0.832378223495702)(112,0.837242359630419)(113,0.846098783106657)(114,0.845934379457917)(115,0.85383502170767)(116,0.85383502170767)(117,0.85383502170767)(118,0.849532037437005)(119,0.847041847041847)(120,0.854030501089325)(121,0.854030501089325)(122,0.854030501089325)(123,0.852173913043478)(124,0.85589519650655)(125,0.85589519650655)(126,0.858797972483707)(127,0.857556037599421)(128,0.857556037599421)(129,0.857556037599421)(130,0.860869565217391)(131,0.862119013062409)(132,0.862119013062409)(133,0.863372093023256)(134,0.865889212827988)(135,0.865889212827988)(136,0.865889212827988)(137,0.86608442503639)(138,0.864197530864197)(139,0.864197530864197)(140,0.865454545454545)(141,0.864)(142,0.864)(143,0.864)(144,0.864)(145,0.864)(146,0.860869565217391)(147,0.862745098039216)(148,0.862119013062409)(149,0.860246198406951)(150,0.860869565217391)(151,0.857761732851986)(152,0.858176555716353)(153,0.859420289855072)(154,0.860246198406951)(155,0.861271676300578)(156,0.861271676300578)(157,0.862518089725036)(158,0.862518089725036)(159,0.862518089725036)(160,0.86002886002886)(161,0.859408795962509)(162,0.861070911722142)(163,0.861694424330195)(164,0.859623733719247)(165,0.860649819494585)(166,0.859623733719247)(167,0.862545454545454)(168,0.86066763425254)(169,0.862545454545454)(170,0.865693430656934)(171,0.865693430656934)(172,0.86380189366351)(173,0.866325785244704)(174,0.865693430656934)(175,0.86380189366351)(176,0.865693430656934)(177,0.861918604651163)(178,0.861918604651163)(179,0.861918604651163)(180,0.862545454545454)(181,0.86380189366351)(182,0.86002886002886)(183,0.853466761972838)(184,0.853466761972838)(185,0.853466761972838)(186,0.851640513552068)(187,0.849217638691323)(188,0.848614072494669)(189,0.849217638691323)(190,0.849822064056939)(191,0.849822064056939)(192,0.849217638691323)(193,0.845010615711252)(194,0.847409510290986)(195,0.842625264643613)(196,0.842625264643613)(197,0.841437632135306)(198,0.841437632135306)(199,0.841437632135306)(200,0.842031029619182)(201,0.844413012729844)(202,0.845010615711252)(203,0.844413012729844)(204,0.845010615711252)(205,0.845010615711252)(206,0.848011363636364)(207,0.845010615711252)(208,0.844413012729844)(209,0.845010615711252)(210,0.845010615711252)(211,0.844413012729844)(212,0.843220338983051)(213,0.843220338983051)(214,0.843220338983051)(215,0.843816254416961)(216,0.839662447257384)(217,0.839662447257384)(218,0.840253342716397)(219,0.840253342716397)(220,0.840253342716397)(221,0.842031029619182)(222,0.842031029619182)(223,0.842031029619182)(224,0.839072382290935)(225,0.837307152875175)(226,0.837307152875175)(227,0.841437632135306)(228,0.841437632135306)(229,0.841437632135306)(230,0.841437632135306)(231,0.840845070422535)(232,0.837894736842105)(233,0.837894736842105)(234,0.837894736842105)(235,0.851033499643621)(236,0.849822064056939)(237,0.85042735042735)(238,0.852248394004283)(239,0.849822064056939)(240,0.849822064056939)(241,0.85042735042735)(242,0.85042735042735)(243,0.85042735042735)(244,0.85042735042735)(245,0.849822064056939)(246,0.848614072494669)(247,0.849217638691323)(248,0.86085075702956)(249,0.858992805755396)(250,0.858992805755396)(251,0.86271676300578)(252,0.861471861471861)(253,0.836720392431675)(254,0.836134453781513)(255,0.834965034965035)(256,0.833798882681564)(257,0.834381551362683)(258,0.834381551362683)(259,0.83554933519944)(260,0.832635983263598)(261,0.834381551362683)(262,0.833217027215632)(263,0.833217027215632)(264,0.83205574912892)(265,0.831476323119777)(266,0.831476323119777)(267,0.830897703549061)(268,0.830897703549061)(269,0.830897703549061)(270,0.830897703549061)(271,0.829742876997915)(272,0.830319888734353)(273,0.830319888734353)(274,0.830319888734353)(275,0.832635983263598)(276,0.832635983263598)(277,0.830897703549061)(278,0.832635983263598)(279,0.830319888734353)(280,0.831476323119777)(281,0.830897703549061)(282,0.830319888734353)(283,0.830897703549061)(284,0.830897703549061)(285,0.830319888734353)(286,0.829166666666667)(287,0.826869806094183)(288,0.846103470857891)(289,0.845033112582781)(290,0.873811033608117)(291,0.864491844416562)(292,0.861965039180229)(293,0.857655502392344)(294,0.857998801677651)(295,0.85288862418106)(296,0.848341232227488)(297,0.850356294536817)(298,0.865256797583081)(299,0.863335340156532)(300,0.862297053517739)(301,0.868038740920097)(302,0.872262773722628)(303,0.889026658400496)(304,0.889578163771712)(305,0.889578163771712)(306,0.907477820025348)(307,0.910954516335682)(308,0.909441233140655)(309,0.914728682170542)(310,0.916129032258064)(311,0.914948453608247)(312,0.913881748071979)(313,0.913881748071979)(314,0.912235746316464)(315,0.911295469049138)(316,0.910714285714286)(317,0.911182108626198)(318,0.911182108626198)(319,0.911182108626198)(320,0.910600255427842)(321,0.910600255427842)(322,0.911182108626198)(323,0.910600255427842)(324,0.910600255427842)(325,0.910600255427842)(326,0.910600255427842)(327,0.911182108626198)(328,0.911764705882353)(329,0.910600255427842)(330,0.911408540471638)(331,0.912101910828025)(332,0.913848117421825)(333,0.915015974440895)(334,0.915015974440895)(335,0.915015974440895)(336,0.916773367477593)(337,0.91618682021753)(338,0.920900321543408)(339,0.921391752577319)(340,0.921492921492922)(341,0.923870967741935)(342,0.920900321543408)(343,0.920900321543408)(344,0.923275306254029)(345,0.924467398321498)(346,0.922186495176849)(347,0.920410783055199)(348,0.922779922779923)(349,0.922779922779923)(350,0.92159383033419)(351,0.92159383033419)(352,0.923969072164948)(353,0.92337411461687)(354,0.92337411461687)(355,0.92159383033419)(356,0.920410783055199)(357,0.924564796905222)(358,0.924564796905222)(359,0.92337411461687)(360,0.92337411461687)(361,0.92337411461687)(362,0.925161290322581)(363,0.925161290322581)(364,0.925161290322581)(365,0.925161290322581)(366,0.924564796905222)(367,0.924564796905222)(368,0.924564796905222)(369,0.924564796905222)(370,0.922779922779923)(371,0.92337411461687)(372,0.922779922779923)(373,0.922779922779923)(374,0.922779922779923)(375,0.920410783055199)(376,0.921001926782273)(377,0.920410783055199)(378,0.906447534766119)(379,0.907020872865275)(380,0.907594936708861)(381,0.907594936708861)(382,0.90530303030303)(383,0.90530303030303)(384,0.905874921036008)(385,0.905874921036008)(386,0.905874921036008)(387,0.900188323917137)(388,0.901319924575739)(389,0.904731861198738)(390,0.901319924575739)(391,0.90188679245283)(392,0.903022670025189)(393,0.904161412358134)(394,0.903022670025189)(395,0.90359168241966)(396,0.90188679245283)(397,0.899623588456713)(398,0.903022670025189)(399,0.91279439847231)(400,0.91279439847231)(4,0.501718213058419)(5,0.572018585441404)(6,0.549927641099855)(7,0.618142313959804)(8,0.605900948366702)(9,0.66324200913242)(10,0.690997566909976)(11,0.68667719852554)(12,0.679206914082359)(13,0.674513817809621)(14,0.687631027253669)(15,0.701378254211332)(16,0.700662927078021)(17,0.708629441624365)(18,0.698721227621483)(19,0.712923728813559)(20,0.715654952076677)(21,0.715654952076677)(22,0.716188524590164)(23,0.737292669876939)(24,0.743257820927724)(25,0.742857142857143)(26,0.761957119296316)(27,0.763012181616833)(28,0.760348583877996)(29,0.757871878393051)(30,0.758845944474687)(31,0.762165117550574)(32,0.785109983079526)(33,0.790617848970252)(34,0.788362806617228)(35,0.798382437897169)(36,0.80442374854482)(37,0.813898704358068)(38,0.811032863849765)(39,0.814378314672952)(40,0.813898704358068)(41,0.816085156712005)(42,0.817535545023697)(43,0.817535545023697)(44,0.818505338078292)(45,0.81867145421903)(46,0.818399044205496)(47,0.817857142857143)(48,0.822262118491921)(49,0.818505338078292)(50,0.823880597014925)(51,0.828828828828829)(52,0.852357320099255)(53,0.869349845201238)(54,0.868811881188119)(55,0.873222016079159)(56,0.87880671224363)(57,0.876474239602731)(58,0.874922215308027)(59,0.868974042027194)(60,0.870807453416149)(61,0.871508379888268)(62,0.872433105164903)(63,0.871188550093342)(64,0.871890547263682)(65,0.865966646077826)(66,0.867574257425742)(67,0.867947923124612)(68,0.868862647607209)(69,0.869240348692403)(70,0.869402985074627)(71,0.867245657568238)(72,0.867245657568238)(73,0.86832298136646)(74,0.871571072319202)(75,0.88)(76,0.879746835443038)(77,0.904575163398693)(78,0.90741956664478)(79,0.906824146981627)(80,0.907782864617397)(81,0.908496732026144)(82,0.909685863874345)(83,0.913725490196078)(84,0.913128674069236)(85,0.912532637075718)(86,0.916230366492147)(87,0.91743119266055)(88,0.914435009797518)(89,0.910865322055953)(90,0.91358024691358)(91,0.921377517868746)(92,0.920181700194679)(93,0.918272425249169)(94,0.919387075283144)(95,0.92)(96,0.92)(97,0.923592493297587)(98,0.923592493297587)(99,0.91970802919708)(100,0.921244209133024)(101,0.923790589794566)(102,0.924817032601464)(103,0.925115970841617)(104,0.92686170212766)(105,0.928286852589642)(106,0.927152317880795)(107,0.926023778071334)(108,0.924603174603175)(109,0.924603174603175)(110,0.922772277227723)(111,0.920342330480579)(112,0.91913214990138)(113,0.918635170603675)(114,0.92005242463958)(115,0.923076923076923)(116,0.90897517504774)(117,0.905516804058339)(118,0.905516804058339)(119,0.905516804058339)(120,0.905972045743329)(121,0.908742820676452)(122,0.90482233502538)(123,0.906429026098027)(124,0.91106845809341)(125,0.907006369426752)(126,0.908163265306122)(127,0.906429026098027)(128,0.904126984126984)(129,0.900695762175838)(130,0.895428929242329)(131,0.896681277395116)(132,0.897243107769424)(133,0.897243107769424)(134,0.897243107769424)(135,0.894440974391006)(136,0.893325015595758)(137,0.893325015595758)(138,0.893325015595758)(139,0.893325015595758)(140,0.89221183800623)(141,0.886687306501548)(142,0.886138613861386)(143,0.886687306501548)(144,0.88833746898263)(145,0.890547263681592)(146,0.891236793039155)(147,0.892077354959451)(148,0.887376237623762)(149,0.887376237623762)(150,0.890130353817504)(151,0.889578163771712)(152,0.889026658400496)(153,0.889303482587065)(154,0.890410958904109)(155,0.888888888888889)(156,0.888888888888889)(157,0.870673952641166)(158,0.871202916160389)(159,0.870673952641166)(160,0.872262773722628)(161,0.871732522796352)(162,0.872793670115642)(163,0.873325213154689)(164,0.874390243902439)(165,0.873857404021938)(166,0.869090909090909)(167,0.865942028985507)(168,0.865942028985507)(169,0.866465256797583)(170,0.866465256797583)(171,0.859197124026363)(172,0.85663082437276)(173,0.862297053517739)(174,0.862297053517739)(175,0.862297053517739)(176,0.862297053517739)(177,0.861261261261261)(178,0.862297053517739)(179,0.865419432709716)(180,0.865419432709716)(181,0.863855421686747)(182,0.863855421686747)(183,0.860744297719088)(184,0.862297053517739)(185,0.861778846153846)(186,0.861261261261261)(187,0.861778846153846)(188,0.861778846153846)(189,0.858168761220826)(190,0.857655502392344)(191,0.858682634730539)(192,0.858682634730539)(193,0.860227954409118)(194,0.861261261261261)(195,0.859712230215827)(196,0.854079809410363)(197,0.849023090586146)(198,0.848018923713779)(199,0.847517730496454)(200,0.848018923713779)(201,0.848018923713779)(202,0.845020624631703)(203,0.847017129356172)(204,0.848018923713779)(205,0.859197124026363)(206,0.859712230215827)(207,0.860227954409118)(208,0.861778846153846)(209,0.859712230215827)(210,0.859197124026363)(211,0.859197124026363)(212,0.859197124026363)(213,0.85663082437276)(214,0.858168761220826)(215,0.858682634730539)(216,0.859197124026363)(217,0.861778846153846)(218,0.861261261261261)(219,0.861261261261261)(220,0.861778846153846)(221,0.861778846153846)(222,0.863335340156532)(223,0.861778846153846)(224,0.861261261261261)(225,0.862297053517739)(226,0.862297053517739)(227,0.862297053517739)(228,0.863335340156532)(229,0.862297053517739)(230,0.865419432709716)(231,0.865942028985507)(232,0.866465256797583)(233,0.862815884476534)(234,0.862297053517739)(235,0.863855421686747)(236,0.860227954409118)(237,0.859197124026363)(238,0.858682634730539)(239,0.858168761220826)(240,0.859712230215827)(241,0.859712230215827)(242,0.859712230215827)(243,0.861778846153846)(244,0.862297053517739)(245,0.862297053517739)(246,0.862297053517739)(247,0.859712230215827)(248,0.859712230215827)(249,0.858682634730539)(250,0.858168761220826)(251,0.859197124026363)(252,0.857142857142857)(253,0.855608591885441)(254,0.855608591885441)(255,0.855098389982111)(256,0.855608591885441)(257,0.855098389982111)(258,0.851038575667656)(259,0.850029638411381)(260,0.849023090586146)(261,0.852556480380499)(262,0.851038575667656)(263,0.85204991087344)(264,0.847517730496454)(265,0.848018923713779)(266,0.849023090586146)(267,0.848018923713779)(268,0.848018923713779)(269,0.848520710059172)(270,0.848520710059172)(271,0.841055718475073)(272,0.841055718475073)(273,0.841055718475073)(274,0.83859649122807)(275,0.83908718548859)(276,0.839578454332553)(277,0.840070298769771)(278,0.840070298769771)(279,0.838106370543542)(280,0.837127845884413)(281,0.836639439906651)(282,0.8351776354106)(283,0.834691501746216)(284,0.835664335664336)(285,0.834691501746216)(286,0.8351776354106)(287,0.834691501746216)(288,0.834691501746216)(289,0.834691501746216)(290,0.833720930232558)(291,0.836151603498542)(292,0.842043452730476)(293,0.842043452730476)(294,0.842043452730476)(295,0.842043452730476)(296,0.839578454332553)(297,0.838106370543542)(298,0.84452296819788)(299,0.84452296819788)(300,0.843529411764706)(301,0.845518867924528)(302,0.830341632889403)(303,0.830822711471611)(304,0.830822711471611)(305,0.830341632889403)(306,0.831786542923434)(307,0.831786542923434)(308,0.831786542923434)(309,0.832269297736506)(310,0.831786542923434)(311,0.832269297736506)(312,0.832752613240418)(313,0.829861111111111)(314,0.829861111111111)(315,0.826989619377163)(316,0.829861111111111)(317,0.829861111111111)(318,0.829861111111111)(319,0.829381145170619)(320,0.829381145170619)(321,0.829381145170619)(322,0.829861111111111)(323,0.829861111111111)(324,0.829861111111111)(325,0.829861111111111)(326,0.829381145170619)(327,0.834205933682373)(328,0.830341632889403)(329,0.834205933682373)(330,0.834691501746216)(331,0.834691501746216)(332,0.8351776354106)(333,0.835664335664336)(334,0.837127845884413)(335,0.838106370543542)(336,0.837616822429906)(337,0.837127845884413)(338,0.837127845884413)(339,0.837616822429906)(340,0.838106370543542)(341,0.83859649122807)(342,0.83859649122807)(343,0.837616822429906)(344,0.837616822429906)(345,0.847517730496454)(346,0.846517119244392)(347,0.846517119244392)(348,0.846517119244392)(349,0.847017129356172)(350,0.848520710059172)(351,0.85204991087344)(352,0.849526066350711)(353,0.851038575667656)(354,0.849526066350711)(355,0.848520710059172)(356,0.853063652587745)(357,0.853063652587745)(358,0.853063652587745)(359,0.85204991087344)(360,0.853063652587745)(361,0.854588796185936)(362,0.864376130198915)(363,0.864376130198915)(364,0.864897466827503)(365,0.864897466827503)(366,0.862297053517739)(367,0.863335340156532)(368,0.863855421686747)(369,0.863855421686747)(370,0.863335340156532)(371,0.868038740920097)(372,0.871732522796352)(373,0.871732522796352)(374,0.871202916160389)(375,0.870145631067961)(376,0.872793670115642)(377,0.873857404021938)(378,0.874923733984136)(379,0.874390243902439)(380,0.874923733984136)(381,0.874390243902439)(382,0.874390243902439)(383,0.874923733984136)(384,0.866989117291415)(385,0.870145631067961)(386,0.870145631067961)(387,0.870145631067961)(388,0.869617950272892)(389,0.869617950272892)(390,0.870145631067961)(391,0.870673952641166)(392,0.872793670115642)(393,0.873857404021938)(394,0.874390243902439)(395,0.873857404021938)(396,0.874923733984136)(397,0.874923733984136)(398,0.874390243902439)(399,0.874390243902439)(400,0.874390243902439)(3,0.537461300309597)(4,0.527102803738318)(5,0.560570071258907)(6,0.665257223396758)(7,0.665726375176305)(8,0.722811671087533)(9,0.720423000660939)(10,0.73007367716008)(11,0.727882037533512)(12,0.747237569060773)(13,0.755400696864111)(14,0.756290438533429)(15,0.757379409647228)(16,0.773775216138328)(17,0.764331210191083)(18,0.768136557610242)(19,0.76103714085494)(20,0.764788732394366)(21,0.763713080168776)(22,0.764788732394366)(23,0.765327695560254)(24,0.771050800278358)(25,0.761356365962892)(26,0.775749674054759)(27,0.781456953642384)(28,0.786950732356857)(29,0.78019801980198)(30,0.787234042553191)(31,0.789368770764119)(32,0.772815533980583)(33,0.763915547024952)(34,0.761479591836735)(35,0.797858099062918)(36,0.796791443850267)(37,0.800537273337811)(38,0.802153432032301)(39,0.793608521970706)(40,0.789403973509934)(41,0.790450928381963)(42,0.788881535407015)(43,0.784726793943384)(44,0.786279683377309)(45,0.81203007518797)(46,0.813096862210095)(47,0.81143635125936)(48,0.798392498325519)(49,0.794137241838774)(50,0.798927613941019)(51,0.789927104042412)(52,0.790450928381963)(53,0.790975447909754)(54,0.789927104042412)(55,0.788881535407015)(56,0.814356435643564)(57,0.816936488169365)(58,0.83780487804878)(59,0.839853300733496)(60,0.848300970873786)(61,0.839135654261705)(62,0.843693421846711)(63,0.847800237812128)(64,0.845878136200717)(65,0.834117647058823)(66,0.845878136200717)(67,0.846107423053711)(68,0.833530106257379)(69,0.835997631734754)(70,0.83402244536326)(71,0.83402244536326)(72,0.831077104178929)(73,0.836492890995261)(74,0.837982195845697)(75,0.837982195845697)(76,0.838978015448604)(77,0.835203780271707)(78,0.832744405182568)(79,0.83372641509434)(80,0.828755113968439)(81,0.836084905660377)(82,0.833922261484099)(83,0.836151603498542)(84,0.826989619377163)(85,0.828901734104046)(86,0.827466820542412)(87,0.826512968299712)(88,0.826512968299712)(89,0.825561312607945)(90,0.825561312607945)(91,0.825561312607945)(92,0.824137931034483)(93,0.824137931034483)(94,0.820366132723112)(95,0.814310051107325)(96,0.812925170068027)(97,0.812004530011325)(98,0.812004530011325)(99,0.813847900113507)(100,0.812464589235127)(101,0.812464589235127)(102,0.812925170068027)(103,0.813847900113507)(104,0.814310051107325)(105,0.823664560597358)(106,0.837127845884413)(107,0.836151603498542)(108,0.841055718475073)(109,0.841055718475073)(110,0.844025897586816)(111,0.840562719812427)(112,0.85086155674391)(113,0.856971873129862)(114,0.868038740920097)(115,0.868038740920097)(116,0.863335340156532)(117,0.869987849331713)(118,0.869987849331713)(119,0.894871794871795)(120,0.899285250162443)(121,0.896778435239974)(122,0.903684550743374)(123,0.908015768725361)(124,0.905735003295979)(125,0.902975420439845)(126,0.904887714663144)(127,0.907053394858273)(128,0.902564102564102)(129,0.902564102564102)(130,0.902564102564102)(131,0.904056664520283)(132,0.901734104046243)(133,0.901282051282051)(134,0.902439024390244)(135,0.901282051282051)(136,0.906391220142027)(137,0.906148867313916)(138,0.907563025210084)(139,0.905806451612903)(140,0.895597484276729)(141,0.887227414330218)(142,0.890274314214464)(143,0.889165628891656)(144,0.881627620221948)(145,0.88833746898263)(146,0.881773399014778)(147,0.894176581089543)(148,0.893617021276596)(149,0.8975487115022)(150,0.896984924623115)(151,0.902084649399874)(152,0.899810964083176)(153,0.898678414096916)(154,0.906091370558375)(155,0.903225806451613)(156,0.901515151515151)(157,0.902654867256637)(158,0.90379746835443)(159,0.908974358974359)(160,0.907348242811502)(161,0.909673286354901)(162,0.912123155869147)(163,0.911538461538461)(164,0.913294797687861)(165,0.914948453608247)(166,0.914948453608247)(167,0.911424903722721)(168,0.911424903722721)(169,0.913183279742765)(170,0.912483912483912)(171,0.912959381044487)(172,0.914212548015365)(173,0.918327974276527)(174,0.917737789203085)(175,0.922877511341542)(176,0.883950617283951)(177,0.879066912216083)(178,0.880688806888069)(179,0.880688806888069)(180,0.881230769230769)(181,0.880688806888069)(182,0.880147510755993)(183,0.87960687960688)(184,0.880688806888069)(185,0.883405305367057)(186,0.882860665844636)(187,0.883950617283951)(188,0.881230769230769)(189,0.882860665844636)(190,0.883950617283951)(191,0.881773399014778)(192,0.882860665844636)(193,0.882860665844636)(194,0.882860665844636)(195,0.882860665844636)(196,0.881773399014778)(197,0.880147510755993)(198,0.876913655848132)(199,0.878527607361963)(200,0.877450980392157)(201,0.879066912216083)(202,0.880688806888069)(203,0.886138613861386)(204,0.88504326328801)(205,0.88504326328801)(206,0.886138613861386)(207,0.885590599876314)(208,0.885590599876314)(209,0.883405305367057)(210,0.878527607361963)(211,0.877450980392157)(212,0.879066912216083)(213,0.879066912216083)(214,0.880147510755993)(215,0.879066912216083)(216,0.879066912216083)(217,0.882461538461538)(218,0.881376767055931)(219,0.882461538461538)(220,0.879754601226994)(221,0.879754601226994)(222,0.880294659300184)(223,0.880294659300184)(224,0.885185185185185)(225,0.886279357231149)(226,0.88682745825603)(227,0.889026658400496)(228,0.885731933292156)(229,0.871732522796352)(230,0.871732522796352)(231,0.865942028985507)(232,0.865942028985507)(233,0.854588796185936)(234,0.851038575667656)(235,0.845518867924528)(236,0.840070298769771)(237,0.83908718548859)(238,0.838106370543542)(239,0.841055718475073)(240,0.841549295774648)(241,0.842043452730476)(242,0.842043452730476)(243,0.840070298769771)(244,0.841055718475073)(245,0.835664335664336)(246,0.833720930232558)(247,0.831786542923434)(248,0.833720930232558)(249,0.828901734104046)(250,0.829861111111111)(251,0.829861111111111)(252,0.831304347826087)(253,0.831786542923434)(254,0.830822711471611)(255,0.830822711471611)(256,0.831786542923434)(257,0.831304347826087)(258,0.831786542923434)(259,0.831304347826087)(260,0.832269297736506)(261,0.830341632889403)(262,0.829861111111111)(263,0.831304347826087)(264,0.832269297736506)(265,0.826989619377163)(266,0.831786542923434)(267,0.830341632889403)(268,0.830341632889403)(269,0.830341632889403)(270,0.821776504297994)(271,0.830822711471611)(272,0.831304347826087)(273,0.831786542923434)(274,0.834205933682373)(275,0.834691501746216)(276,0.834205933682373)(277,0.835664335664336)(278,0.835664335664336)(279,0.835664335664336)(280,0.833720930232558)(281,0.832269297736506)(282,0.831786542923434)(283,0.831786542923434)(284,0.831786542923434)(285,0.831304347826087)(286,0.831304347826087)(287,0.829381145170619)(288,0.829381145170619)(289,0.829381145170619)(290,0.827944572748268)(291,0.823191733639495)(292,0.818493150684931)(293,0.818026240730177)(294,0.818493150684931)(295,0.816628701594533)(296,0.817094017094017)(297,0.815699658703072)(298,0.815699658703072)(299,0.815699658703072)(300,0.815699658703072)(301,0.816628701594533)(302,0.815699658703072)(303,0.821776504297994)(304,0.821776504297994)(305,0.824137931034483)(306,0.825561312607945)(307,0.825086306098964)(308,0.826036866359447)(309,0.826036866359447)(310,0.817559863169897)(311,0.815699658703072)(312,0.815235929505401)(313,0.818493150684931)(314,0.818493150684931)(315,0.817094017094017)(316,0.818026240730177)(317,0.817094017094017)(318,0.817559863169897)(319,0.818960593946316)(320,0.820835718374356)(321,0.821776504297994)(322,0.820835718374356)(323,0.819428571428571)(324,0.821305841924398)(325,0.825561312607945)(326,0.824137931034483)(327,0.829381145170619)(328,0.830341632889403)(329,0.830341632889403)(330,0.828901734104046)(331,0.828901734104046)(332,0.827466820542412)(333,0.82842287694974)(334,0.82842287694974)(335,0.82842287694974)(336,0.826512968299712)(337,0.825086306098964)(338,0.825086306098964)(339,0.820366132723112)(340,0.820366132723112)(341,0.818493150684931)(342,0.818493150684931)(343,0.818026240730177)(344,0.818960593946316)(345,0.824137931034483)(346,0.824137931034483)(347,0.825086306098964)(348,0.822247706422018)(349,0.816163915765509)(350,0.820366132723112)(351,0.824611845888442)(352,0.826512968299712)(353,0.827466820542412)(354,0.827466820542412)(355,0.820366132723112)(356,0.825561312607945)(357,0.825086306098964)(358,0.824137931034483)(359,0.824137931034483)(360,0.820366132723112)(361,0.823664560597358)(362,0.818026240730177)(363,0.819897084048027)(364,0.820835718374356)(365,0.821305841924398)(366,0.821305841924398)(367,0.820835718374356)(368,0.821305841924398)(369,0.818493150684931)(370,0.818493150684931)(371,0.818493150684931)(372,0.818493150684931)(373,0.821305841924398)(374,0.827466820542412)(375,0.802911534154535)(376,0.80561797752809)(377,0.791390728476821)(378,0.781897491821156)(379,0.788345244639912)(380,0.79490022172949)(381,0.794459833795014)(382,0.79933110367893)(383,0.799776910206358)(384,0.796666666666667)(385,0.797109505280711)(386,0.800670016750419)(387,0.800670016750419)(388,0.792703150912106)(389,0.792703150912106)(390,0.79490022172949)(391,0.79490022172949)(392,0.794019933554817)(393,0.786615469007131)(394,0.786184210526316)(395,0.787047200878156)(396,0.787047200878156)(397,0.785753424657534)(398,0.784893267651888)(399,0.784893267651888)(400,0.784893267651888)(3,0.670940170940171)(4,0.6602658788774)(5,0.701318851823119)(6,0.706530291109363)(7,0.742534301856336)(8,0.778898370830101)(9,0.782742681047766)(10,0.756155679110405)(11,0.778747026169706)(12,0.795256916996047)(13,0.79715864246251)(14,0.79715864246251)(15,0.803843074459568)(16,0.797417271993543)(17,0.801932367149758)(18,0.807692307692308)(19,0.807692307692308)(20,0.820634920634921)(21,0.805825242718447)(22,0.805825242718447)(23,0.815642458100559)(24,0.819277108433735)(25,0.818619582664526)(26,0.817963111467522)(27,0.817307692307692)(28,0.818910256410256)(29,0.815705128205128)(30,0.81635926222935)(31,0.81635926222935)(32,0.814814814814815)(33,0.814814814814815)(34,0.823341326938449)(35,0.818910256410256)(36,0.815705128205128)(37,0.817014446227929)(38,0.817014446227929)(39,0.815705128205128)(40,0.813804173354735)(41,0.813804173354735)(42,0.813804173354735)(43,0.815052041633307)(44,0.815052041633307)(45,0.813504823151125)(46,0.814457831325301)(47,0.816720257234727)(48,0.815112540192926)(49,0.819277108433735)(50,0.818619582664526)(51,0.819566960705694)(52,0.817670682730924)(53,0.816129032258064)(54,0.817082997582595)(55,0.82051282051282)(56,0.82051282051282)(57,0.82051282051282)(58,0.82051282051282)(59,0.824940047961631)(60,0.823341326938449)(61,0.825219473264166)(62,0.82615629984051)(63,0.827366746221161)(64,0.82818685669042)(65,0.82753164556962)(66,0.82753164556962)(67,0.82895783611774)(68,0.829888712241653)(69,0.829073482428115)(70,0.832802547770701)(71,0.832802547770701)(72,0.831872509960159)(73,0.831872509960159)(74,0.832140015910899)(75,0.832140015910899)(76,0.832535885167464)(77,0.833466135458167)(78,0.832535885167464)(79,0.833466135458167)(80,0.833466135458167)(81,0.833200319233839)(82,0.835322195704057)(83,0.831872509960159)(84,0.83147853736089)(85,0.831872509960159)(86,0.832802547770701)(87,0.832802547770701)(88,0.832802547770701)(89,0.837172359015091)(90,0.836913285600636)(91,0.836248012718601)(92,0.835322195704057)(93,0.835322195704057)(94,0.835583796664019)(95,0.835583796664019)(96,0.839016653449643)(97,0.840855106888361)(98,0.841772151898734)(99,0.848151062155783)(100,0.845425867507886)(101,0.849056603773585)(102,0.846335697399527)(103,0.848151062155783)(104,0.847244094488189)(105,0.851097178683385)(106,0.848864526233359)(107,0.847537138389367)(108,0.850430696945967)(109,0.849765258215962)(110,0.849765258215962)(111,0.850430696945967)(112,0.851097178683385)(113,0.850863422291994)(114,0.850863422291994)(115,0.850863422291994)(116,0.850430696945967)(117,0.850430696945967)(118,0.849100860046912)(119,0.849765258215962)(120,0.851097178683385)(121,0.850430696945967)(122,0.851298190401259)(123,0.85243328100471)(124,0.85062893081761)(125,0.85062893081761)(126,0.85062893081761)(127,0.851298190401259)(128,0.85062893081761)(129,0.850863422291994)(130,0.850863422291994)(131,0.85243328100471)(132,0.851298190401259)(133,0.851298190401259)(134,0.851298190401259)(135,0.851298190401259)(136,0.852201257861635)(137,0.851298190401259)(138,0.852201257861635)(139,0.852201257861635)(140,0.853102906520031)(141,0.853102906520031)(142,0.853102906520031)(143,0.85243328100471)(144,0.85243328100471)(145,0.85243328100471)(146,0.85243328100471)(147,0.85243328100471)(148,0.85243328100471)(149,0.851764705882353)(150,0.852664576802508)(151,0.851996867658575)(152,0.851330203442879)(153,0.851330203442879)(154,0.851330203442879)(155,0.851330203442879)(156,0.851330203442879)(157,0.851330203442879)(158,0.851330203442879)(159,0.851330203442879)(160,0.8515625)(161,0.850234009360374)(162,0.849571317225253)(163,0.848249027237354)(164,0.846930846930847)(165,0.844306738962045)(166,0.850897736143638)(167,0.852228303362001)(168,0.852228303362001)(169,0.852895148669796)(170,0.852895148669796)(171,0.853563038371182)(172,0.853333333333333)(173,0.85423197492163)(174,0.853333333333333)(175,0.853333333333333)(176,0.846273291925466)(177,0.844306738962045)(178,0.842349304482226)(179,0.843000773395205)(180,0.844961240310077)(181,0.844961240310077)(182,0.845616757176105)(183,0.843653250773994)(184,0.843000773395205)(185,0.842349304482226)(186,0.843000773395205)(187,0.842349304482226)(188,0.843000773395205)(189,0.843000773395205)(190,0.842349304482226)(191,0.841049382716049)(192,0.839753466872111)(193,0.841049382716049)(194,0.837817063797079)(195,0.837173579109063)(196,0.837817063797079)(197,0.835249042145594)(198,0.835249042145594)(199,0.835889570552147)(200,0.835889570552147)(201,0.833333333333333)(202,0.837173579109063)(203,0.837817063797079)(204,0.837173579109063)(205,0.835249042145594)(206,0.835249042145594)(207,0.836531082118189)(208,0.837173579109063)(209,0.837173579109063)(210,0.837173579109063)(211,0.837817063797079)(212,0.836531082118189)(213,0.843000773395205)(214,0.843000773395205)(215,0.843000773395205)(216,0.841698841698842)(217,0.840400925212028)(218,0.841049382716049)(219,0.845616757176105)(220,0.850234009360374)(221,0.850234009360374)(222,0.849571317225253)(223,0.846930846930847)(224,0.846273291925466)(225,0.843653250773994)(226,0.848249027237354)(227,0.846273291925466)(228,0.848909657320872)(229,0.849571317225253)(230,0.847589424572317)(231,0.844961240310077)(232,0.844961240310077)(233,0.841698841698842)(234,0.845616757176105)(235,0.844961240310077)(236,0.844961240310077)(237,0.845616757176105)(238,0.844961240310077)(239,0.844961240310077)(240,0.847589424572317)(241,0.847589424572317)(242,0.848249027237354)(243,0.848249027237354)(244,0.846930846930847)(245,0.846930846930847)(246,0.848909657320872)(247,0.848909657320872)(248,0.848909657320872)(249,0.841049382716049)(250,0.841698841698842)(251,0.840400925212028)(252,0.843000773395205)(253,0.844961240310077)(254,0.844961240310077)(255,0.844961240310077)(256,0.848249027237354)(257,0.848909657320872)(258,0.848249027237354)(259,0.848249027237354)(260,0.8515625)(261,0.8515625)(262,0.8515625)(263,0.8515625)(264,0.849571317225253)(265,0.849571317225253)(266,0.8515625)(267,0.83910700538876)(268,0.835889570552147)(269,0.835249042145594)(270,0.835249042145594)(271,0.833333333333333)(272,0.833970925784239)(273,0.835249042145594)(274,0.835249042145594)(275,0.834609494640122)(276,0.833333333333333)(277,0.833333333333333)(278,0.833970925784239)(279,0.833970925784239)(280,0.828897338403042)(281,0.829528158295281)(282,0.829528158295281)(283,0.83015993907083)(284,0.83015993907083)(285,0.83015993907083)(286,0.83015993907083)(287,0.83015993907083)(288,0.830792682926829)(289,0.828267477203647)(290,0.828897338403042)(291,0.830792682926829)(292,0.830792682926829)(293,0.830792682926829)(294,0.830792682926829)(295,0.830792682926829)(296,0.831426392067124)(297,0.830792682926829)(298,0.830792682926829)(299,0.830792682926829)(300,0.831426392067124)(301,0.83015993907083)(302,0.828267477203647)(303,0.830792682926829)(304,0.830792682926829)(305,0.83015993907083)(306,0.830792682926829)(307,0.83015993907083)(308,0.831426392067124)(309,0.831426392067124)(310,0.83206106870229)(311,0.831426392067124)(312,0.828897338403042)(313,0.828897338403042)(314,0.828897338403042)(315,0.83206106870229)(316,0.832696715049656)(317,0.83206106870229)(318,0.83206106870229)(319,0.83206106870229)(320,0.833333333333333)(321,0.833333333333333)(322,0.833333333333333)(323,0.833333333333333)(324,0.833333333333333)(325,0.832696715049656)(326,0.832696715049656)(327,0.83206106870229)(328,0.828897338403042)(329,0.831426392067124)(330,0.831426392067124)(331,0.831426392067124)(332,0.831426392067124)(333,0.831426392067124)(334,0.83206106870229)(335,0.832696715049656)(336,0.837173579109063)(337,0.837173579109063)(338,0.841698841698842)(339,0.840400925212028)(340,0.838461538461538)(341,0.838461538461538)(342,0.837173579109063)(343,0.837173579109063)(344,0.836531082118189)(345,0.837817063797079)(346,0.837817063797079)(347,0.835249042145594)(348,0.833970925784239)(349,0.833333333333333)(350,0.833333333333333)(351,0.833333333333333)(352,0.834609494640122)(353,0.835249042145594)(354,0.835249042145594)(355,0.835249042145594)(356,0.835249042145594)(357,0.835889570552147)(358,0.833333333333333)(359,0.834609494640122)(360,0.838461538461538)(361,0.838461538461538)(362,0.838461538461538)(363,0.843000773395205)(364,0.844306738962045)(365,0.847589424572317)(366,0.847113884555382)(367,0.847113884555382)(368,0.846453624318005)(369,0.846453624318005)(370,0.846453624318005)(371,0.846453624318005)(372,0.846453624318005)(373,0.846930846930847)(374,0.846453624318005)(375,0.846453624318005)(376,0.847298355520752)(377,0.845136186770428)(378,0.846930846930847)(379,0.847113884555382)(380,0.847113884555382)(381,0.847775175644028)(382,0.847775175644028)(383,0.846453624318005)(384,0.847113884555382)(385,0.849960722702278)(386,0.846453624318005)(387,0.845136186770428)(388,0.845136186770428)(389,0.845136186770428)(390,0.845136186770428)(391,0.845136186770428)(392,0.845136186770428)(393,0.844961240310077)(394,0.84447900466563)(395,0.84447900466563)(396,0.85263987391647)(397,0.85263987391647)(398,0.852173913043478)(399,0.852173913043478)(400,0.85193982581156)(4,0.518995492594977)(5,0.636879432624113)(6,0.654662379421222)(7,0.693766937669377)(8,0.729121278140886)(9,0.726875455207574)(10,0.728332119446467)(11,0.744186046511628)(12,0.756028368794326)(13,0.772826880934989)(14,0.771699489423778)(15,0.782417582417582)(16,0.801801801801802)(17,0.790286975717439)(18,0.799702159344751)(19,0.805097451274363)(20,0.802098950524737)(21,0.805369127516778)(22,0.804494382022472)(23,0.798206278026906)(24,0.795234549516009)(25,0.793462109955423)(26,0.794642857142857)(27,0.795304475421863)(28,0.798232695139911)(29,0.806572068707991)(30,0.807174887892377)(31,0.807174887892377)(32,0.804138950480414)(33,0.802962962962963)(34,0.80355819125278)(35,0.807749627421758)(36,0.812922614575507)(37,0.801768607221813)(38,0.804444444444444)(39,0.810240963855422)(40,0.808955223880597)(41,0.808955223880597)(42,0.808955223880597)(43,0.808955223880597)(44,0.811094452773613)(45,0.811463046757164)(46,0.812971342383107)(47,0.812075471698113)(48,0.808414725770098)(49,0.803584764749813)(50,0.805369127516778)(51,0.805369127516778)(52,0.804769001490313)(53,0.807778608825729)(54,0.806835066864784)(55,0.806835066864784)(56,0.808035714285714)(57,0.807434944237918)(58,0.805637982195846)(59,0.806835066864784)(60,0.808035714285714)(61,0.808035714285714)(62,0.807434944237918)(63,0.808035714285714)(64,0.804733727810651)(65,0.803545051698671)(66,0.804138950480414)(67,0.802650957290132)(68,0.802650957290132)(69,0.80206033848418)(70,0.807407407407407)(71,0.808005930318755)(72,0.808605341246291)(73,0.80680977054034)(74,0.808005930318755)(75,0.807407407407407)(76,0.807407407407407)(77,0.811011904761905)(78,0.811011904761905)(79,0.811615785554728)(80,0.812220566318927)(81,0.811615785554728)(82,0.807407407407407)(83,0.808005930318755)(84,0.809806835066865)(85,0.810408921933085)(86,0.810408921933085)(87,0.80920564216778)(88,0.811615785554728)(89,0.811615785554728)(90,0.808605341246291)(91,0.808605341246291)(92,0.808605341246291)(93,0.808005930318755)(94,0.811011904761905)(95,0.808605341246291)(96,0.810408921933085)(97,0.80920564216778)(98,0.808605341246291)(99,0.810408921933085)(100,0.808005930318755)(101,0.805022156573117)(102,0.805022156573117)(103,0.799706529713866)(104,0.802650957290132)(105,0.804428044280443)(106,0.804428044280443)(107,0.804428044280443)(108,0.805022156573117)(109,0.807407407407407)(110,0.808005930318755)(111,0.808005930318755)(112,0.807407407407407)(113,0.808605341246291)(114,0.80920564216778)(115,0.80680977054034)(116,0.80680977054034)(117,0.805022156573117)(118,0.803834808259587)(119,0.80206033848418)(120,0.802650957290132)(121,0.803242446573323)(122,0.802650957290132)(123,0.803242446573323)(124,0.803242446573323)(125,0.804428044280443)(126,0.80206033848418)(127,0.802650957290132)(128,0.80206033848418)(129,0.804428044280443)(130,0.80206033848418)(131,0.801470588235294)(132,0.796783625730994)(133,0.796783625730994)(134,0.796783625730994)(135,0.79100145137881)(136,0.79042784626541)(137,0.789283128167994)(138,0.788141720896601)(139,0.787003610108303)(140,0.786435786435786)(141,0.791575889615105)(142,0.791575889615105)(143,0.795040116703136)(144,0.795040116703136)(145,0.795040116703136)(146,0.796783625730994)(147,0.793304221251819)(148,0.793304221251819)(149,0.79388201019665)(150,0.79388201019665)(151,0.796201607012418)(152,0.796201607012418)(153,0.796201607012418)(154,0.796201607012418)(155,0.796201607012418)(156,0.796201607012418)(157,0.799120234604106)(158,0.801470588235294)(159,0.801470588235294)(160,0.801470588235294)(161,0.799706529713866)(162,0.799706529713866)(163,0.799120234604106)(164,0.800293685756241)(165,0.800293685756241)(166,0.799706529713866)(167,0.803834808259587)(168,0.802650957290132)(169,0.802650957290132)(170,0.805022156573117)(171,0.800293685756241)(172,0.795620437956204)(173,0.796783625730994)(174,0.799706529713866)(175,0.799706529713866)(176,0.798534798534798)(177,0.798534798534798)(178,0.798534798534798)(179,0.799706529713866)(180,0.785302593659942)(181,0.785302593659942)(182,0.785302593659942)(183,0.785302593659942)(184,0.785302593659942)(185,0.787003610108303)(186,0.78757225433526)(187,0.788712011577424)(188,0.788141720896601)(189,0.789283128167994)(190,0.789283128167994)(191,0.789283128167994)(192,0.79042784626541)(193,0.79042784626541)(194,0.810921501706485)(195,0.818745692625775)(196,0.801880456682337)(197,0.796)(198,0.797062750333778)(199,0.797595190380761)(200,0.801342281879194)(201,0.808395396073121)(202,0.808395396073121)(203,0.806212018906144)(204,0.806756756756757)(205,0.806756756756757)(206,0.800804828973843)(207,0.800804828973843)(208,0.794940079893475)(209,0.795469686875416)(210,0.800268096514745)(211,0.799732083054253)(212,0.799732083054253)(213,0.799732083054253)(214,0.797062750333778)(215,0.798128342245989)(216,0.798128342245989)(217,0.798128342245989)(218,0.799196787148594)(219,0.800268096514745)(220,0.799732083054253)(221,0.798128342245989)(222,0.798662207357859)(223,0.798128342245989)(224,0.797595190380761)(225,0.799732083054253)(226,0.799732083054253)(227,0.798662207357859)(228,0.798662207357859)(229,0.798662207357859)(230,0.798662207357859)(231,0.798662207357859)(232,0.797595190380761)(233,0.797595190380761)(234,0.797595190380761)(235,0.797595190380761)(236,0.797062750333778)(237,0.797595190380761)(238,0.799196787148594)(239,0.802958977807666)(240,0.799732083054253)(241,0.799196787148594)(242,0.798128342245989)(243,0.796)(244,0.793882978723404)(245,0.793882978723404)(246,0.793882978723404)(247,0.793882978723404)(248,0.79441117764471)(249,0.792828685258964)(250,0.791252485089463)(251,0.791777188328913)(252,0.791252485089463)(253,0.791252485089463)(254,0.791252485089463)(255,0.791252485089463)(256,0.791777188328913)(257,0.791777188328913)(258,0.791777188328913)(259,0.790205162144275)(260,0.791252485089463)(261,0.790728476821192)(262,0.791252485089463)(263,0.790728476821192)(264,0.791252485089463)(265,0.791252485089463)(266,0.791252485089463)(267,0.791252485089463)(268,0.791777188328913)(269,0.792302587923026)(270,0.790728476821192)(271,0.792302587923026)(272,0.792828685258964)(273,0.792828685258964)(274,0.792828685258964)(275,0.793355481727575)(276,0.793355481727575)(277,0.793882978723404)(278,0.793882978723404)(279,0.793355481727575)(280,0.793355481727575)(281,0.792828685258964)(282,0.793355481727575)(283,0.797062750333778)(284,0.788639365918098)(285,0.78968253968254)(286,0.788118811881188)(287,0.788118811881188)(288,0.788118811881188)(289,0.788118811881188)(290,0.787079762689519)(291,0.787079762689519)(292,0.788639365918098)(293,0.78916060806345)(294,0.78968253968254)(295,0.78916060806345)(296,0.792302587923026)(297,0.791252485089463)(298,0.797062750333778)(299,0.797062750333778)(300,0.790728476821192)(301,0.790728476821192)(302,0.79441117764471)(303,0.79441117764471)(304,0.822314049586777)(305,0.821748107364074)(306,0.822314049586777)(307,0.822314049586777)(308,0.821182943603851)(309,0.81169272603671)(310,0.808943089430894)(311,0.808943089430894)(312,0.808943089430894)(313,0.815017064846416)(314,0.828591256072172)(315,0.828591256072172)(316,0.828591256072172)(317,0.826297577854671)(318,0.826869806094183)(319,0.826869806094183)(320,0.826869806094183)(321,0.826869806094183)(322,0.827442827442827)(323,0.827442827442827)(324,0.827442827442827)(325,0.827442827442827)(326,0.828016643550624)(327,0.828016643550624)(328,0.828016643550624)(329,0.828016643550624)(330,0.829166666666667)(331,0.828016643550624)(332,0.828591256072172)(333,0.829742876997915)(334,0.829742876997915)(335,0.829742876997915)(336,0.829742876997915)(337,0.829742876997915)(338,0.829742876997915)(339,0.830897703549061)(340,0.834381551362683)(341,0.83554933519944)(342,0.836720392431675)(343,0.836134453781513)(344,0.836134453781513)(345,0.836134453781513)(346,0.836134453781513)(347,0.836134453781513)(348,0.832635983263598)(349,0.832635983263598)(350,0.83205574912892)(351,0.83205574912892)(352,0.829166666666667)(353,0.829166666666667)(354,0.831476323119777)(355,0.83205574912892)(356,0.83205574912892)(357,0.83205574912892)(358,0.83205574912892)(359,0.83205574912892)(360,0.83205574912892)(361,0.829166666666667)(362,0.830897703549061)(363,0.824016563146998)(364,0.824016563146998)(365,0.824016563146998)(366,0.824585635359116)(367,0.833798882681564)(368,0.833798882681564)(369,0.833217027215632)(370,0.833217027215632)(371,0.833217027215632)(372,0.83205574912892)(373,0.833217027215632)(374,0.837894736842105)(375,0.839072382290935)(376,0.839662447257384)(377,0.839662447257384)(378,0.838483146067416)(379,0.836134453781513)(380,0.83554933519944)(381,0.834965034965035)(382,0.833798882681564)(383,0.834965034965035)(384,0.83554933519944)(385,0.834965034965035)(386,0.836134453781513)(387,0.836134453781513)(388,0.836134453781513)(389,0.836134453781513)(390,0.836134453781513)(391,0.836134453781513)(392,0.847409510290986)(393,0.849822064056939)(394,0.846208362863218)(395,0.846208362863218)(396,0.853466761972838)(397,0.852857142857143)(398,0.852857142857143)(399,0.852857142857143)(400,0.852248394004283)(8,0.566775244299674)(9,0.737160120845921)(10,0.755966127790608)(11,0.763580719204285)(12,0.758928571428571)(13,0.764215314632297)(14,0.762121212121212)(15,0.766997708174179)(16,0.773060029282577)(17,0.777032065622669)(18,0.777364110201042)(19,0.786737000753579)(20,0.779686333084391)(21,0.786415094339623)(22,0.785822021116139)(23,0.785822021116139)(24,0.784048156508653)(25,0.788199697428139)(26,0.811102544333076)(27,0.814296814296814)(28,0.81578947368421)(29,0.816358024691358)(30,0.811437403400309)(31,0.816923076923077)(32,0.822706065318818)(33,0.82051282051282)(34,0.82015503875969)(35,0.81959564541213)(36,0.820791311093871)(37,0.816485225505443)(38,0.818604651162791)(39,0.814988290398126)(40,0.815047021943574)(41,0.815625)(42,0.813479623824451)(43,0.815686274509804)(44,0.815047021943574)(45,0.815686274509804)(46,0.815047021943574)(47,0.815748031496063)(48,0.815748031496063)(49,0.814229249011858)(50,0.816037735849056)(51,0.815748031496063)(52,0.815748031496063)(53,0.816967792615868)(54,0.817966903073286)(55,0.816101026045777)(56,0.817966903073286)(57,0.818253343823761)(58,0.819107282693814)(59,0.820392156862745)(60,0.819466248037676)(61,0.818823529411765)(62,0.818823529411765)(63,0.81974921630094)(64,0.81974921630094)(65,0.818823529411765)(66,0.817254901960784)(67,0.819107282693814)(68,0.817896389324961)(69,0.817896389324961)(70,0.818823529411765)(71,0.817254901960784)(72,0.816679779701023)(73,0.815748031496063)(74,0.814173228346457)(75,0.814173228346457)(76,0.814814814814815)(77,0.814814814814815)(78,0.815748031496063)(79,0.812943962115233)(80,0.812943962115233)(81,0.812943962115233)(82,0.812943962115233)(83,0.812943962115233)(84,0.813238770685579)(85,0.812943962115233)(86,0.813880126182965)(87,0.812006319115324)(88,0.813880126182965)(89,0.813880126182965)(90,0.814173228346457)(91,0.817610062893082)(92,0.815748031496063)(93,0.815748031496063)(94,0.81974921630094)(95,0.816967792615868)(96,0.817896389324961)(97,0.818823529411765)(98,0.817688130333592)(99,0.818677042801556)(100,0.818958818958819)(101,0.818958818958819)(102,0.818040435458787)(103,0.81591263650546)(104,0.81591263650546)(105,0.817757009345794)(106,0.818110850897736)(107,0.817472698907956)(108,0.816549570647931)(109,0.817254901960784)(110,0.815974941268598)(111,0.815336463223787)(112,0.815336463223787)(113,0.818466353677621)(114,0.821705426356589)(115,0.822618125484121)(116,0.822618125484121)(117,0.822618125484121)(118,0.822618125484121)(119,0.822618125484121)(120,0.822618125484121)(121,0.822618125484121)(122,0.822618125484121)(123,0.822618125484121)(124,0.823798627002288)(125,0.832579185520362)(126,0.836309523809524)(127,0.845986984815618)(128,0.856309263311451)(129,0.863930885529158)(130,0.878014184397163)(131,0.883590462833099)(132,0.877291960507757)(133,0.881307746979389)(134,0.878980891719745)(135,0.878504672897196)(136,0.882310469314079)(137,0.880521361332368)(138,0.880969351389879)(139,0.8810888252149)(140,0.881550610193826)(141,0.88034188034188)(142,0.881597717546362)(143,0.888888888888889)(144,0.890459363957597)(145,0.894366197183099)(146,0.896697118763176)(147,0.894035087719298)(148,0.894035087719298)(149,0.895585143658024)(150,0.895921237693389)(151,0.89703808180536)(152,0.898939929328622)(153,0.900494001411433)(154,0.902319044272663)(155,0.903589021815623)(156,0.906894100923952)(157,0.904964539007092)(158,0.904964539007092)(159,0.904964539007092)(160,0.905099150141643)(161,0.905099150141643)(162,0.907681465821001)(163,0.907681465821001)(164,0.909476661951909)(165,0.908833922261484)(166,0.908704883227176)(167,0.90295358649789)(168,0.904225352112676)(169,0.904225352112676)(170,0.901685393258427)(171,0.900420757363254)(172,0.903453136011275)(173,0.90436005625879)(174,0.905500705218618)(175,0.906139731827805)(176,0.906139731827805)(177,0.903818953323904)(178,0.902404526166902)(179,0.903043170559094)(180,0.903043170559094)(181,0.904323175053154)(182,0.906894100923952)(183,0.908701854493581)(184,0.908054169636493)(185,0.906116642958748)(186,0.906116642958748)(187,0.907407407407407)(188,0.907922912205567)(189,0.911785462244178)(190,0.91114245416079)(191,0.91114245416079)(192,0.902203856749311)(193,0.901582931865107)(194,0.902959394356504)(195,0.902825637491385)(196,0.905321354526607)(197,0.912403644008409)(198,0.913043478260869)(199,0.911888111888112)(200,0.900752908966461)(201,0.899521531100478)(202,0.898907103825137)(203,0.90530557421088)(204,0.907619689817937)(205,0.908115358819584)(206,0.901986754966887)(207,0.908727514990006)(208,0.908483633934536)(209,0.914851485148515)(210,0.914512922465209)(211,0.914134742404227)(212,0.914248021108179)(213,0.914248021108179)(214,0.915409836065574)(215,0.915143603133159)(216,0.912280701754386)(217,0.910264686894771)(218,0.911424903722721)(219,0.906190172303765)(220,0.901015228426396)(221,0.913183279742765)(222,0.911424903722721)(223,0.90676883780332)(224,0.907928388746803)(225,0.914948453608247)(226,0.913770913770914)(227,0.914948453608247)(228,0.914948453608247)(229,0.912596401028278)(230,0.915538362346873)(231,0.917312661498708)(232,0.917905623787977)(233,0.919093851132686)(234,0.919093851132686)(235,0.91849935316947)(236,0.917312661498708)(237,0.91672046481601)(238,0.914948453608247)(239,0.914948453608247)(240,0.915538362346873)(241,0.912708600770218)(242,0.915538362346873)(243,0.917905623787977)(244,0.91849935316947)(245,0.91849935316947)(246,0.919689119170984)(247,0.91849935316947)(248,0.91849935316947)(249,0.91849935316947)(250,0.915647134578236)(251,0.915647134578236)(252,0.912708600770218)(253,0.912123155869147)(254,0.913294797687861)(255,0.913183279742765)(256,0.913183279742765)(257,0.915057915057915)(258,0.916129032258064)(259,0.917631917631918)(260,0.923376623376623)(261,0.919198448610213)(262,0.921581335061568)(263,0.922977346278317)(264,0.923575129533679)(265,0.922977346278317)(266,0.925373134328358)(267,0.925373134328358)(268,0.930809399477807)(269,0.930809399477807)(270,0.931417374265186)(271,0.931417374265186)(272,0.928990228013029)(273,0.930202217873451)(274,0.929595827900912)(275,0.933508887425938)(276,0.934123847167325)(277,0.935973597359736)(278,0.938451356717406)(279,0.939695162359178)(280,0.940318302387268)(281,0.940318302387268)(282,0.941567065073041)(283,0.94281914893617)(284,0.94281914893617)(285,0.952830188679245)(286,0.950907868190988)(287,0.950907868190988)(288,0.935611038107753)(289,0.934996717005909)(290,0.933770491803279)(291,0.936842105263158)(292,0.936058009228741)(293,0.935441370223979)(294,0.939774983454666)(295,0.939153439153439)(296,0.939774983454666)(297,0.939774983454666)(298,0.94047619047619)(299,0.94047619047619)(300,0.939854593522802)(301,0.939854593522802)(302,0.939854593522802)(303,0.942895086321381)(304,0.942895086321381)(305,0.944148936170213)(306,0.943521594684385)(307,0.944148936170213)(308,0.945406125166445)(309,0.946035976015989)(310,0.949698189134809)(311,0.949698189134809)(312,0.948494983277592)(313,0.946595460614152)(314,0.933683519369665)(315,0.944703530979347)(316,0.945406125166445)(317,0.945406125166445)(318,0.944777112441783)(319,0.944703530979347)(320,0.944074567243675)(321,0.944074567243675)(322,0.944074567243675)(323,0.948494983277592)(324,0.949765572672472)(325,0.949765572672472)(326,0.949129852744311)(327,0.950335570469799)(328,0.950973807924782)(329,0.950973807924782)(330,0.950973807924782)(331,0.950973807924782)(332,0.950973807924782)(333,0.950973807924782)(334,0.950973807924782)(335,0.950973807924782)(336,0.953472690492245)(337,0.953472690492245)(338,0.953472690492245)(339,0.953472690492245)(340,0.953472690492245)(341,0.952830188679245)(342,0.952188552188552)(343,0.952188552188552)(344,0.952188552188552)(345,0.952188552188552)(346,0.952188552188552)(347,0.947721179624665)(348,0.943521594684385)(349,0.943371085942705)(350,0.943371085942705)(351,0.949698189134809)(352,0.944703530979347)(353,0.944703530979347)(354,0.945333333333333)(355,0.945333333333333)(356,0.941567065073041)(357,0.941567065073041)(358,0.940318302387268)(359,0.940318302387268)(360,0.941567065073041)(361,0.949698189134809)(362,0.949129852744311)(363,0.949765572672472)(364,0.947227788911156)(365,0.945963975983989)(366,0.947227788911156)(367,0.945963975983989)(368,0.945333333333333)(369,0.936058009228741)(370,0.936058009228741)(371,0.934739617666447)(372,0.931147540983607)(373,0.930628272251309)(374,0.929927963326784)(375,0.930537352555701)(376,0.931937172774869)(377,0.932026143790849)(378,0.932723709993468)(379,0.947157190635451)(380,0.949061662198391)(381,0.949061662198391)(382,0.947791164658634)(383,0.947791164658634)(384,0.944629753168779)(385,0.945891783567134)(386,0.947157190635451)(387,0.946524064171123)(388,0.950335570469799)(389,0.950335570469799)(390,0.932114882506527)(391,0.932723709993468)(392,0.932114882506527)(393,0.932723709993468)(394,0.932723709993468)(395,0.93089960886571)(396,0.9296875)(397,0.9296875)(398,0.9296875)(399,0.93029315960912)(400,0.93029315960912)(9,0.675305975521958)(10,0.687124749833222)(11,0.760603882099209)(12,0.772760917838638)(13,0.774145616641902)(14,0.769119769119769)(15,0.7893536121673)(16,0.788432267884323)(17,0.788432267884323)(18,0.789992418498863)(19,0.79631053036126)(20,0.79631053036126)(21,0.811865729898517)(22,0.808811959087333)(23,0.809148264984227)(24,0.815974941268598)(25,0.80824088748019)(26,0.815686274509804)(27,0.814756671899529)(28,0.816037735849056)(29,0.820392156862745)(30,0.820392156862745)(31,0.824271079590228)(32,0.822503961965135)(33,0.818759936406995)(34,0.818471337579618)(35,0.819488817891374)(36,0.851393188854489)(37,0.84904214559387)(38,0.847742922723795)(39,0.845740598618572)(40,0.847560975609756)(41,0.852160727824109)(42,0.84297520661157)(43,0.839192221391174)(44,0.840210052513128)(45,0.845283018867924)(46,0.845283018867924)(47,0.838514025777104)(48,0.840272520817562)(49,0.837386018237082)(50,0.837244511733535)(51,0.837490551776266)(52,0.836445108289768)(53,0.837313432835821)(54,0.83668903803132)(55,0.833087149187592)(56,0.832472324723247)(57,0.833087149187592)(58,0.826979472140762)(59,0.830022075055188)(60,0.832962138084632)(61,0.831727205337287)(62,0.831727205337287)(63,0.835798816568047)(64,0.834937083641747)(65,0.835051546391753)(66,0.841874084919473)(67,0.842721287490856)(68,0.843956043956044)(69,0.835387962291515)(70,0.837818181818182)(71,0.838427947598253)(72,0.845588235294118)(73,0.844574780058651)(74,0.843338213762811)(75,0.841874084919473)(76,0.838427947598253)(77,0.846041055718475)(78,0.846886446886447)(79,0.843408594319009)(80,0.845872899926954)(81,0.847110460863204)(82,0.849557522123894)(83,0.850184501845018)(84,0.848931466470155)(85,0.848931466470155)(86,0.848931466470155)(87,0.848931466470155)(88,0.848931466470155)(89,0.85820895522388)(90,0.859925093632959)(91,0.859925093632959)(92,0.86373790022338)(93,0.864824495892457)(94,0.863298662704309)(95,0.86094674556213)(96,0.863298662704309)(97,0.863298662704309)(98,0.863298662704309)(99,0.863501483679525)(100,0.859040590405904)(101,0.859040590405904)(102,0.860310421286031)(103,0.861584011843079)(104,0.859040590405904)(105,0.857142857142857)(106,0.855882352941176)(107,0.856512141280353)(108,0.856512141280353)(109,0.856512141280353)(110,0.853998532648569)(111,0.854838709677419)(112,0.855051244509517)(113,0.855051244509517)(114,0.853801169590643)(115,0.855051244509517)(116,0.8558888076079)(117,0.85463842220599)(118,0.852769679300291)(119,0.8558888076079)(120,0.85463842220599)(121,0.854425749817118)(122,0.852148579752367)(123,0.852983988355167)(124,0.852148579752367)(125,0.852148579752367)(126,0.852769679300291)(127,0.852148579752367)(128,0.852769679300291)(129,0.852769679300291)(130,0.854651162790698)(131,0.854030501089325)(132,0.854030501089325)(133,0.854030501089325)(134,0.854030501089325)(135,0.854030501089325)(136,0.855272727272727)(137,0.85610465116279)(138,0.85610465116279)(139,0.857350800582241)(140,0.855692530819434)(141,0.856313497822932)(142,0.856313497822932)(143,0.857975236707939)(144,0.858600583090379)(145,0.857975236707939)(146,0.857975236707939)(147,0.85610465116279)(148,0.856727272727273)(149,0.854862119013062)(150,0.854862119013062)(151,0.854862119013062)(152,0.854651162790698)(153,0.854651162790698)(154,0.854651162790698)(155,0.855272727272727)(156,0.854030501089325)(157,0.85589519650655)(158,0.854030501089325)(159,0.854651162790698)(160,0.854030501089325)(161,0.854030501089325)(162,0.852173913043478)(163,0.852173913043478)(164,0.852173913043478)(165,0.852791878172589)(166,0.860907759882869)(167,0.862170087976539)(168,0.863436123348018)(169,0.864070536370316)(170,0.864070536370316)(171,0.863436123348018)(172,0.863235294117647)(173,0.863235294117647)(174,0.865781710914454)(175,0.871946706143597)(176,0.871301775147929)(177,0.86950146627566)(178,0.869692532942899)(179,0.869692532942899)(180,0.868035190615836)(181,0.865693430656934)(182,0.865889212827988)(183,0.867153284671533)(184,0.866520787746171)(185,0.867786705624543)(186,0.867153284671533)(187,0.867786705624543)(188,0.866520787746171)(189,0.865889212827988)(190,0.867153284671533)(191,0.866520787746171)(192,0.866520787746171)(193,0.867786705624543)(194,0.867786705624543)(195,0.867786705624543)(196,0.867786705624543)(197,0.867786705624543)(198,0.867786705624543)(199,0.867786705624543)(200,0.868421052631579)(201,0.869692532942899)(202,0.869692532942899)(203,0.869692532942899)(204,0.868864468864469)(205,0.870139398385913)(206,0.871606749816581)(207,0.872246696035242)(208,0.872246696035242)(209,0.872887582659809)(210,0.872887582659809)(211,0.872887582659809)(212,0.873529411764706)(213,0.875460574797347)(214,0.874815905743741)(215,0.876752767527675)(216,0.876752767527675)(217,0.876752767527675)(218,0.876752767527675)(219,0.877400295420975)(220,0.878048780487805)(221,0.877400295420975)(222,0.876106194690265)(223,0.875460574797347)(224,0.875460574797347)(225,0.876106194690265)(226,0.876752767527675)(227,0.876106194690265)(228,0.876106194690265)(229,0.879169755374351)(230,0.879169755374351)(231,0.879821958456973)(232,0.87964338781575)(233,0.87964338781575)(234,0.880952380952381)(235,0.880297397769517)(236,0.87964338781575)(237,0.880297397769517)(238,0.880297397769517)(239,0.875739644970414)(240,0.875739644970414)(241,0.875739644970414)(242,0.875092387287509)(243,0.876387860843819)(244,0.876387860843819)(245,0.876387860843819)(246,0.875092387287509)(247,0.875092387287509)(248,0.875092387287509)(249,0.875092387287509)(250,0.876387860843819)(251,0.87750556792873)(252,0.876204595997035)(253,0.876204595997035)(254,0.874907475943745)(255,0.872968980797637)(256,0.872968980797637)(257,0.874260355029586)(258,0.874260355029586)(259,0.874260355029586)(260,0.874907475943745)(261,0.874260355029586)(262,0.875555555555555)(263,0.87815750371471)(264,0.87667161961367)(265,0.877323420074349)(266,0.87815750371471)(267,0.87750556792873)(268,0.876854599406528)(269,0.876204595997035)(270,0.879464285714286)(271,0.879464285714286)(272,0.879464285714286)(273,0.87815750371471)(274,0.87797619047619)(275,0.87797619047619)(276,0.87797619047619)(277,0.87797619047619)(278,0.878629932985852)(279,0.878629932985852)(280,0.878629932985852)(281,0.879940343027591)(282,0.879940343027591)(283,0.878629932985852)(284,0.878629932985852)(285,0.879284649776453)(286,0.87667161961367)(287,0.878629932985852)(288,0.877323420074349)(289,0.877323420074349)(290,0.87815750371471)(291,0.87815750371471)(292,0.878810408921933)(293,0.8801191362621)(294,0.882089552238806)(295,0.882748319641523)(296,0.881431767337807)(297,0.882748319641523)(298,0.882089552238806)(299,0.882089552238806)(300,0.876204595997035)(301,0.876854599406528)(302,0.875555555555555)(303,0.875555555555555)(304,0.879169755374351)(305,0.87964338781575)(306,0.881431767337807)(307,0.87964338781575)(308,0.880952380952381)(309,0.881608339538347)(310,0.883720930232558)(311,0.884384384384384)(312,0.887218045112782)(313,0.887885628291949)(314,0.887885628291949)(315,0.885885885885886)(316,0.887377173091459)(317,0.887377173091459)(318,0.887377173091459)(319,0.890589135424636)(320,0.890589135424636)(321,0.888888888888889)(322,0.889565879664889)(323,0.888888888888889)(324,0.888549618320611)(325,0.890922959572845)(326,0.890243902439024)(327,0.888888888888889)(328,0.889397406559878)(329,0.890243902439024)(330,0.890243902439024)(331,0.887878787878788)(332,0.888720666161998)(333,0.88855193328279)(334,0.892424242424242)(335,0.893778452200303)(336,0.893778452200303)(337,0.893778452200303)(338,0.893100833965125)(339,0.890577507598784)(340,0.890577507598784)(341,0.890577507598784)(342,0.891254752851711)(343,0.887048192771084)(344,0.88622754491018)(345,0.885735623599701)(346,0.885564697083022)(347,0.885564697083022)(348,0.885221305326331)(349,0.88455772113943)(350,0.885221305326331)(351,0.885221305326331)(352,0.885221305326331)(353,0.885221305326331)(354,0.886380737396539)(355,0.885714285714286)(356,0.888386123680241)(357,0.891419893697798)(358,0.892261001517451)(359,0.89158453373768)(360,0.890909090909091)(361,0.891419893697798)(362,0.893292682926829)(363,0.893455098934551)(364,0.892424242424242)(365,0.885048835462058)(366,0.887716654107008)(367,0.887716654107008)(368,0.887048192771084)(369,0.888386123680241)(370,0.889056603773585)(371,0.889056603773585)(372,0.889056603773585)(373,0.892424242424242)(374,0.893778452200303)(375,0.894457099468489)(376,0.895817490494296)(377,0.895817490494296)(378,0.895817490494296)(379,0.895817490494296)(380,0.895817490494296)(381,0.895817490494296)(382,0.895136778115501)(383,0.895136778115501)(384,0.895136778115501)(385,0.8910741301059)(386,0.892424242424242)(387,0.892424242424242)(388,0.891748675246026)(389,0.891748675246026)(390,0.895136778115501)(391,0.886380737396539)(392,0.886380737396539)(393,0.889728096676737)(394,0.890400604686319)(395,0.891748675246026)(396,0.879348630643967)(397,0.879348630643967)(398,0.879348630643967)(399,0.879348630643967)(400,0.88112927191679)(4,0.582602832097101)(5,0.667630057803468)(6,0.663232453316162)(7,0.710754843019372)(8,0.722925457102672)(9,0.720682302771855)(10,0.711864406779661)(11,0.737936154417223)(12,0.733823529411765)(13,0.767164179104477)(14,0.777364110201042)(15,0.780923994038748)(16,0.781319495922906)(17,0.78125)(18,0.780487804878049)(19,0.776588151320485)(20,0.781294964028777)(21,0.779612347451543)(22,0.779464931308749)(23,0.780841799709724)(24,0.793205317577548)(25,0.788546255506608)(26,0.788546255506608)(27,0.775539568345324)(28,0.78027556200145)(29,0.78027556200145)(30,0.776334776334776)(31,0.780028943560058)(32,0.78134110787172)(33,0.78134110787172)(34,0.773775216138328)(35,0.773775216138328)(36,0.78939617083947)(37,0.785923753665689)(38,0.78765613519471)(39,0.785347985347985)(40,0.785923753665689)(41,0.783505154639175)(42,0.789125642909625)(43,0.789667896678967)(44,0.789667896678967)(45,0.789667896678967)(46,0.788191881918819)(47,0.789356984478936)(48,0.787610619469026)(49,0.787878787878788)(50,0.789085545722714)(51,0.787296898079764)(52,0.790214974054855)(53,0.797003745318352)(54,0.796407185628742)(55,0.795216741405082)(56,0.794336810730253)(57,0.789629629629629)(58,0.792035398230089)(59,0.790595150624541)(60,0.794378698224852)(61,0.794966691339748)(62,0.797927461139896)(63,0.798518518518519)(64,0.797627872498147)(65,0.799405646359584)(66,0.798219584569733)(67,0.799405646359584)(68,0.798812175204157)(69,0.797627872498147)(70,0.797627872498147)(71,0.797627872498147)(72,0.798226164079823)(73,0.797935103244838)(74,0.800588668138337)(75,0.8)(76,0.8)(77,0.800588668138337)(78,0.798825256975037)(79,0.8)(80,0.79388201019665)(81,0.793304221251819)(82,0.793304221251819)(83,0.792727272727273)(84,0.792727272727273)(85,0.79388201019665)(86,0.788712011577424)(87,0.788712011577424)(88,0.788141720896601)(89,0.78757225433526)(90,0.789283128167994)(91,0.792151162790698)(92,0.79042784626541)(93,0.792727272727273)(94,0.793304221251819)(95,0.792151162790698)(96,0.794460641399417)(97,0.800293685756241)(98,0.804428044280443)(99,0.803834808259587)(100,0.803242446573323)(101,0.80206033848418)(102,0.802650957290132)(103,0.802650957290132)(104,0.803834808259587)(105,0.805022156573117)(106,0.804428044280443)(107,0.803242446573323)(108,0.78757225433526)(109,0.785302593659942)(110,0.785302593659942)(111,0.785302593659942)(112,0.785302593659942)(113,0.785302593659942)(114,0.785302593659942)(115,0.785302593659942)(116,0.783045977011494)(117,0.779685264663805)(118,0.781362007168459)(119,0.782483847810481)(120,0.781362007168459)(121,0.781362007168459)(122,0.788712011577424)(123,0.78757225433526)(124,0.788712011577424)(125,0.785868781542898)(126,0.785868781542898)(127,0.785868781542898)(128,0.785868781542898)(129,0.786435786435786)(130,0.786435786435786)(131,0.786435786435786)(132,0.785868781542898)(133,0.784172661870504)(134,0.787003610108303)(135,0.78757225433526)(136,0.788141720896601)(137,0.785868781542898)(138,0.787003610108303)(139,0.78757225433526)(140,0.787003610108303)(141,0.797950219619326)(142,0.796783625730994)(143,0.796783625730994)(144,0.796783625730994)(145,0.798534798534798)(146,0.799120234604106)(147,0.798534798534798)(148,0.805617147080562)(149,0.809806835066865)(150,0.808605341246291)(151,0.805022156573117)(152,0.811011904761905)(153,0.811011904761905)(154,0.811011904761905)(155,0.811011904761905)(156,0.811011904761905)(157,0.81282624906786)(158,0.810408921933085)(159,0.81282624906786)(160,0.814040328603435)(161,0.814040328603435)(162,0.814040328603435)(163,0.817704426106526)(164,0.819894498869631)(165,0.819894498869631)(166,0.819894498869631)(167,0.819894498869631)(168,0.82051282051282)(169,0.819894498869631)(170,0.822995461422088)(171,0.821752265861027)(172,0.821752265861027)(173,0.821752265861027)(174,0.821752265861027)(175,0.823618470855413)(176,0.822995461422088)(177,0.825493171471927)(178,0.826119969627942)(179,0.826747720364742)(180,0.825493171471927)(181,0.825493171471927)(182,0.825493171471927)(183,0.81525804038893)(184,0.820165537998495)(185,0.820165537998495)(186,0.820165537998495)(187,0.82078313253012)(188,0.819548872180451)(189,0.821132075471698)(190,0.824242424242424)(191,0.823618470855413)(192,0.822995461422088)(193,0.823618470855413)(194,0.821752265861027)(195,0.821132075471698)(196,0.822021116138763)(197,0.829900839054157)(198,0.829268292682927)(199,0.829900839054157)(200,0.829268292682927)(201,0.830910482019893)(202,0.835521235521235)(203,0.835521235521235)(204,0.84012539184953)(205,0.838963079340141)(206,0.839467501957713)(207,0.83695652173913)(208,0.836560805577072)(209,0.838910505836576)(210,0.838910505836576)(211,0.838258164852255)(212,0.837461300309597)(213,0.836560805577072)(214,0.838509316770186)(215,0.836813611755607)(216,0.835774865073246)(217,0.837065637065637)(218,0.837065637065637)(219,0.839160839160839)(220,0.839160839160839)(221,0.839813374805599)(222,0.838509316770186)(223,0.835266821345708)(224,0.835913312693498)(225,0.836166924265842)(226,0.835521235521235)(227,0.835521235521235)(228,0.837209302325581)(229,0.837209302325581)(230,0.836307214895268)(231,0.83641975308642)(232,0.835130970724191)(233,0.830910482019893)(234,0.826747720364742)(235,0.825493171471927)(236,0.825493171471927)(237,0.824867323730099)(238,0.824242424242424)(239,0.824867323730099)(240,0.828006088280061)(241,0.828006088280061)(242,0.823262839879154)(243,0.823262839879154)(244,0.823262839879154)(245,0.824242424242424)(246,0.824242424242424)(247,0.828006088280061)(248,0.828636709824829)(249,0.828006088280061)(250,0.828006088280061)(251,0.822995461422088)(252,0.823885109599395)(253,0.823262839879154)(254,0.823262839879154)(255,0.823885109599395)(256,0.823885109599395)(257,0.826383623957544)(258,0.824508320726172)(259,0.824508320726172)(260,0.824508320726172)(261,0.823885109599395)(262,0.823885109599395)(263,0.821401657874906)(264,0.82078313253012)(265,0.819548872180451)(266,0.817704426106526)(267,0.818318318318318)(268,0.815868263473054)(269,0.816479400749064)(270,0.81525804038893)(271,0.818318318318318)(272,0.820165537998495)(273,0.820165537998495)(274,0.816479400749064)(275,0.816479400749064)(276,0.818318318318318)(277,0.818318318318318)(278,0.817704426106526)(279,0.81525804038893)(280,0.813432835820895)(281,0.813432835820895)(282,0.813432835820895)(283,0.818318318318318)(284,0.816479400749064)(285,0.81282624906786)(286,0.81282624906786)(287,0.812220566318927)(288,0.809806835066865)(289,0.809806835066865)(290,0.809806835066865)(291,0.809806835066865)(292,0.811011904761905)(293,0.809806835066865)(294,0.809806835066865)(295,0.809806835066865)(296,0.809806835066865)(297,0.809806835066865)(298,0.809806835066865)(299,0.809806835066865)(300,0.809806835066865)(301,0.805022156573117)(302,0.803834808259587)(303,0.803834808259587)(304,0.803834808259587)(305,0.803834808259587)(306,0.803834808259587)(307,0.80206033848418)(308,0.802650957290132)(309,0.802650957290132)(310,0.802650957290132)(311,0.802650957290132)(312,0.801470588235294)(313,0.800293685756241)(314,0.802650957290132)(315,0.803242446573323)(316,0.803242446573323)(317,0.803242446573323)(318,0.802650957290132)(319,0.796201607012418)(320,0.799706529713866)(321,0.800293685756241)(322,0.800293685756241)(323,0.789855072463768)(324,0.792151162790698)(325,0.792151162790698)(326,0.792151162790698)(327,0.796201607012418)(328,0.797950219619326)(329,0.798534798534798)(330,0.799120234604106)(331,0.799120234604106)(332,0.80206033848418)(333,0.802650957290132)(334,0.805022156573117)(335,0.808605341246291)(336,0.799706529713866)(337,0.80206033848418)(338,0.797950219619326)(339,0.796201607012418)(340,0.797950219619326)(341,0.797366495976591)(342,0.797366495976591)(343,0.800881704628949)(344,0.800293685756241)(345,0.800293685756241)(346,0.800293685756241)(347,0.800293685756241)(348,0.800293685756241)(349,0.800293685756241)(350,0.800293685756241)(351,0.800293685756241)(352,0.807407407407407)(353,0.80920564216778)(354,0.810408921933085)(355,0.81282624906786)(356,0.812220566318927)(357,0.814648729446936)(358,0.814648729446936)(359,0.814648729446936)(360,0.814648729446936)(361,0.81525804038893)(362,0.81525804038893)(363,0.81525804038893)(364,0.81525804038893)(365,0.815868263473054)(366,0.816479400749064)(367,0.815868263473054)(368,0.81525804038893)(369,0.814648729446936)(370,0.814648729446936)(371,0.814648729446936)(372,0.814648729446936)(373,0.81282624906786)(374,0.815868263473054)(375,0.815868263473054)(376,0.81525804038893)(377,0.812220566318927)(378,0.812220566318927)(379,0.812220566318927)(380,0.811615785554728)(381,0.815868263473054)(382,0.817704426106526)(383,0.817704426106526)(384,0.817704426106526)(385,0.817704426106526)(386,0.821401657874906)(387,0.821401657874906)(388,0.822021116138763)(389,0.822641509433962)(390,0.823262839879154)(391,0.817704426106526)(392,0.822021116138763)(393,0.822641509433962)(394,0.823262839879154)(395,0.823262839879154)(396,0.823262839879154)(397,0.823885109599395)(398,0.823885109599395)(399,0.823885109599395)(400,0.823885109599395)(5,0.533770491803279)(6,0.537564766839378)(7,0.588965517241379)(8,0.635205992509363)(9,0.680824484697064)(10,0.713350785340314)(11,0.726786907147628)(12,0.730201342281879)(13,0.744186046511628)(14,0.752598752598753)(15,0.75103734439834)(16,0.75103734439834)(17,0.739484396200814)(18,0.738482384823848)(19,0.754325259515571)(20,0.751206064782908)(21,0.738983050847457)(22,0.734006734006734)(23,0.740992522093814)(24,0.744535519125683)(25,0.743519781718963)(26,0.737982396750169)(27,0.742506811989101)(28,0.753803596127247)(29,0.753803596127247)(30,0.759052924791086)(31,0.754325259515571)(32,0.754325259515571)(33,0.753803596127247)(34,0.76437587657784)(35,0.763305322128851)(36,0.775800711743772)(37,0.781922525107604)(38,0.787003610108303)(39,0.789855072463768)(40,0.78757225433526)(41,0.787003610108303)(42,0.786435786435786)(43,0.789855072463768)(44,0.786435786435786)(45,0.784172661870504)(46,0.785302593659942)(47,0.781362007168459)(48,0.781362007168459)(49,0.78080229226361)(50,0.782483847810481)(51,0.784172661870504)(52,0.784172661870504)(53,0.789283128167994)(54,0.78757225433526)(55,0.788141720896601)(56,0.789283128167994)(57,0.789283128167994)(58,0.79042784626541)(59,0.75)(60,0.762452107279693)(61,0.785526315789474)(62,0.790205162144275)(63,0.771317829457364)(64,0.773316062176166)(65,0.779373368146214)(66,0.785009861932939)(67,0.779882429784455)(68,0.782437745740498)(69,0.782437745740498)(70,0.77734375)(71,0.779373368146214)(72,0.781413612565445)(73,0.776837996096291)(74,0.773572803078897)(75,0.802011313639221)(76,0.828641370869033)(77,0.817535545023697)(78,0.817804154302671)(79,0.830528846153846)(80,0.8459250446163)(81,0.846611177170036)(82,0.846108140225787)(83,0.843102427471877)(84,0.841176470588235)(85,0.832847990681421)(86,0.830429732868757)(87,0.835472578763127)(88,0.830144927536232)(89,0.827067669172932)(90,0.829466357308585)(91,0.829466357308585)(92,0.826112073945696)(93,0.832363213038417)(94,0.829947765525247)(95,0.828505214368482)(96,0.828505214368482)(97,0.824682814302191)(98,0.825158684362377)(99,0.831786542923434)(100,0.829861111111111)(101,0.827466820542412)(102,0.827466820542412)(103,0.830822711471611)(104,0.829861111111111)(105,0.832752613240418)(106,0.835664335664336)(107,0.835664335664336)(108,0.829381145170619)(109,0.827466820542412)(110,0.826989619377163)(111,0.840681951793063)(112,0.841176470588235)(113,0.849851632047478)(114,0.853396901072706)(115,0.853731343283582)(116,0.853905784138342)(117,0.854578096947935)(118,0.859903381642512)(119,0.858695652173913)(120,0.859214501510574)(121,0.860254083484573)(122,0.863912515188335)(123,0.861818181818182)(124,0.861818181818182)(125,0.864437689969605)(126,0.863387978142076)(127,0.863912515188335)(128,0.863912515188335)(129,0.862864077669903)(130,0.863387978142076)(131,0.863912515188335)(132,0.861818181818182)(133,0.862340812613705)(134,0.856453558504222)(135,0.855937311633514)(136,0.855421686746988)(137,0.856970428485214)(138,0.856970428485214)(139,0.85956416464891)(140,0.858524788391777)(141,0.857487922705314)(142,0.864634146341463)(143,0.864634146341463)(144,0.866748166259169)(145,0.867115737905695)(146,0.866585067319461)(147,0.867647058823529)(148,0.86605504587156)(149,0.867115737905695)(150,0.86817903126916)(151,0.86817903126916)(152,0.865853658536585)(153,0.874689826302729)(154,0.874147551146931)(155,0.889170360987967)(156,0.89171974522293)(157,0.88748419721871)(158,0.889029803424223)(159,0.888324873096447)(160,0.889453621346887)(161,0.888888888888889)(162,0.90272373540856)(163,0.899806076276664)(164,0.906332453825857)(165,0.907407407407407)(166,0.912840984697272)(167,0.912234042553191)(168,0.91583830351226)(169,0.91583830351226)(170,0.91583830351226)(171,0.916445623342175)(172,0.916445623342175)(173,0.916445623342175)(174,0.914626075446724)(175,0.914626075446724)(176,0.91523178807947)(177,0.91523178807947)(178,0.91583830351226)(179,0.914626075446724)(180,0.91523178807947)(181,0.91523178807947)(182,0.916445623342175)(183,0.915455746367239)(184,0.915455746367239)(185,0.915455746367239)(186,0.915455746367239)(187,0.916556291390728)(188,0.918596955658504)(189,0.920844327176781)(190,0.921452145214521)(191,0.926343729263437)(192,0.929427430093209)(193,0.930666666666667)(194,0.926958831341301)(195,0.928095872170439)(196,0.92847317744154)(197,0.928032899246059)(198,0.927097661623109)(199,0.926530612244898)(200,0.930679478380233)(201,0.930679478380233)(202,0.928131416837782)(203,0.928131416837782)(204,0.928131416837782)(205,0.928767123287671) 
};
\addplot [
color=orange,
mark size=0.1pt,
only marks,
mark=*,
mark options={solid,fill=black},
forget plot
]
coordinates{
 (205,0.928767123287671)(206,0.928032899246059)(207,0.928032899246059)(208,0.927297668038409)(209,0.924541128484024)(210,0.924643584521385)(211,0.924541128484024)(212,0.920765027322404)(213,0.922237380627558)(214,0.927891156462585)(215,0.927891156462585)(216,0.928522804628999)(217,0.931693989071038)(218,0.931693989071038)(219,0.932330827067669)(220,0.931693989071038)(221,0.941988950276243)(222,0.940041350792557)(223,0.939393939393939)(224,0.940123881624226)(225,0.940123881624226)(226,0.939477303988996)(227,0.943370165745856)(228,0.94271911663216)(229,0.94271911663216)(230,0.943342776203966)(231,0.949128919860627)(232,0.94475138121547)(233,0.946058091286307)(234,0.947295423023578)(235,0.946786454733932)(236,0.944827586206897)(237,0.944827586206897)(238,0.946280991735537)(239,0.944827586206897)(240,0.940845070422535)(241,0.941342756183746)(242,0.940594059405941)(243,0.940594059405941)(244,0.942593905031892)(245,0.942593905031892)(246,0.942593905031892)(247,0.943422913719943)(248,0.943422913719943)(249,0.943342776203966)(250,0.942430703624733)(251,0.943100995732575)(252,0.943772241992882)(253,0.943692088382038)(254,0.943692088382038)(255,0.94361170592434)(256,0.944285714285714)(257,0.94511760513186)(258,0.94661921708185)(259,0.94661921708185)(260,0.948644793152639)(261,0.950679056468906)(262,0.950679056468906)(263,0.952040085898354)(264,0.953405017921147)(265,0.953405017921147)(266,0.952722063037249)(267,0.951578947368421)(268,0.953846153846154)(269,0.956036287508723)(270,0.956036287508723)(271,0.956036287508723)(272,0.956824512534819)(273,0.959610027855153)(274,0.957491289198606)(275,0.958217270194986)(276,0.956884561891516)(277,0.955244755244755)(278,0.955244755244755)(279,0.955974842767296)(280,0.955974842767296)(281,0.952247191011236)(282,0.954992967651195)(283,0.953716690042076)(284,0.954802259887006)(285,0.954063604240283)(286,0.955539872971066)(287,0.955414012738853)(288,0.956276445698166)(289,0.956951305575159)(290,0.956951305575159)(291,0.957865168539326)(292,0.955665024630542)(293,0.955665024630542)(294,0.957746478873239)(295,0.956214689265537)(296,0.957567185289957)(297,0.956829440905874)(298,0.955350815024805)(299,0.955350815024805)(300,0.957686882933709)(301,0.957627118644068)(302,0.956214689265537)(303,0.958362738179252)(304,0.957627118644068)(305,0.958362738179252)(306,0.96)(307,0.958362738179252)(308,0.957627118644068)(309,0.959097320169252)(310,0.96056338028169)(311,0.959943780744905)(312,0.96056338028169)(313,0.959097320169252)(314,0.959887403237157)(315,0.959717314487632)(316,0.96045197740113)(317,0.959097320169252)(318,0.959097320169252)(319,0.959887403237157)(320,0.959154929577465)(321,0.959212376933896)(322,0.959212376933896)(323,0.959943780744905)(324,0.960674157303371)(325,0.964109781843772)(326,0.964109781843772)(327,0.963431786216596)(328,0.963276836158192)(329,0.964059196617336)(330,0.964059196617336)(331,0.963328631875881)(332,0.962649753347428)(333,0.963328631875881)(334,0.964689265536723)(335,0.964689265536723)(336,0.963380281690141)(337,0.964059196617336)(338,0.966101694915254)(339,0.966101694915254)(340,0.966831333803811)(341,0.966101694915254)(342,0.964705882352941)(343,0.965373961218836)(344,0.965373961218836)(345,0.965373961218836)(346,0.965373961218836)(347,0.965421853388658)(348,0.964705882352941)(349,0.964705882352941)(350,0.964705882352941)(351,0.964038727524205)(352,0.964038727524205)(353,0.964038727524205)(354,0.963321799307958)(355,0.963321799307958)(356,0.963321799307958)(357,0.963321799307958)(358,0.963321799307958)(359,0.963321799307958)(360,0.963321799307958)(361,0.963321799307958)(362,0.96398891966759)(363,0.963321799307958)(364,0.963321799307958)(365,0.963321799307958)(366,0.96398891966759)(367,0.963938973647712)(368,0.964607911172797)(369,0.96398891966759)(370,0.963838664812239)(371,0.964459930313589)(372,0.963117606123869)(373,0.963788300835654)(374,0.963788300835654)(375,0.9625520110957)(376,0.9625520110957)(377,0.963168867268937)(378,0.965181058495822)(379,0.965181058495822)(380,0.96652719665272)(381,0.963483146067416)(382,0.963906581740976)(383,0.964589235127479)(384,0.963906581740976)(385,0.961349262122277)(386,0.963957597173145)(387,0.963380281690141)(388,0.9614576033637)(389,0.9614576033637)(390,0.962131837307153)(391,0.962131837307153)(392,0.963585434173669)(393,0.962859145059565)(394,0.962859145059565)(395,0.962859145059565)(396,0.962131837307153)(397,0.962237762237762)(398,0.962237762237762)(399,0.963585434173669)(400,0.962911126662001)(10,0.542446043165468)(11,0.506849315068493)(12,0.535101404056162)(13,0.638820638820639)(14,0.7390625)(15,0.740489130434783)(16,0.75224292615597)(17,0.770318021201413)(18,0.782106782106782)(19,0.792093704245974)(20,0.798523985239852)(21,0.800295857988166)(22,0.794701986754967)(23,0.802973977695167)(24,0.792064658339456)(25,0.794399410464259)(26,0.796420581655481)(27,0.796420581655481)(28,0.81089258698941)(29,0.813020439061317)(30,0.813353566009105)(31,0.821729150726855)(32,0.827799227799228)(33,0.821646341463415)(34,0.822900763358779)(35,0.822900763358779)(36,0.816048448145344)(37,0.816048448145344)(38,0.817838246409675)(39,0.820940819423369)(40,0.824067022086824)(41,0.823439878234399)(42,0.824159021406728)(43,0.824159021406728)(44,0.82542113323124)(45,0.827321565617805)(46,0.827056110684089)(47,0.82262996941896)(48,0.827056110684089)(49,0.82832948421863)(50,0.82832948421863)(51,0.829230769230769)(52,0.829230769230769)(53,0.825885978428351)(54,0.827799227799228)(55,0.82907965970611)(56,0.829721362229102)(57,0.832944832944833)(58,0.832298136645963)(59,0.832298136645963)(60,0.831652443754848)(61,0.831652443754848)(62,0.832944832944833)(63,0.833333333333333)(64,0.835403726708074)(65,0.834108527131783)(66,0.833462432223083)(67,0.834108527131783)(68,0.832690824980725)(69,0.833976833976834)(70,0.833976833976834)(71,0.833976833976834)(72,0.835384615384615)(73,0.833716475095785)(74,0.835249042145594)(75,0.835249042145594)(76,0.832183908045977)(77,0.83282208588957)(78,0.834355828220859)(79,0.826747720364742)(80,0.826119969627942)(81,0.823618470855413)(82,0.823618470855413)(83,0.827113480578827)(84,0.834742505764796)(85,0.834742505764796)(86,0.831408775981524)(87,0.832307692307692)(88,0.833205226748655)(89,0.833333333333333)(90,0.833205226748655)(91,0.832432432432432)(92,0.832565284178187)(93,0.831926323867997)(94,0.831926323867997)(95,0.831926323867997)(96,0.834355828220859)(97,0.834355828220859)(98,0.844062947067239)(99,0.826839826839827)(100,0.826783114992722)(101,0.846616541353383)(102,0.843445692883895)(103,0.845212383009359)(104,0.840974212034384)(105,0.840974212034384)(106,0.841577060931899)(107,0.840372226198998)(108,0.83737517831669)(109,0.838068181818182)(110,0.840682788051209)(111,0.841281138790036)(112,0.841281138790036)(113,0.847919655667145)(114,0.848962061560487)(115,0.8493543758967)(116,0.850574712643678)(117,0.850359712230216)(118,0.855282199710564)(119,0.853640951694304)(120,0.853640951694304)(121,0.854256854256854)(122,0.854873646209386)(123,0.855491329479769)(124,0.853623188405797)(125,0.854439592430859)(126,0.854651162790698)(127,0.85589519650655)(128,0.859021183345507)(129,0.85589519650655)(130,0.856313497822932)(131,0.856313497822932)(132,0.856935366739288)(133,0.855692530819434)(134,0.855692530819434)(135,0.856109906001446)(136,0.854873646209386)(137,0.854873646209386)(138,0.854873646209386)(139,0.855282199710564)(140,0.855282199710564)(141,0.856521739130435)(142,0.856521739130435)(143,0.856521739130435)(144,0.854453294713975)(145,0.861516034985423)(146,0.859431900946832)(147,0.860262008733624)(148,0.859431900946832)(149,0.856935366739288)(150,0.856521739130435)(151,0.856521739130435)(152,0.857764876632801)(153,0.856521739130435)(154,0.85859318346628)(155,0.85859318346628)(156,0.86046511627907)(157,0.86046511627907)(158,0.86046511627907)(159,0.86046511627907)(160,0.85859318346628)(161,0.85859318346628)(162,0.85921625544267)(163,0.859840232389252)(164,0.86046511627907)(165,0.86066763425254)(166,0.861717612809316)(167,0.854466858789625)(168,0.854466858789625)(169,0.855082912761355)(170,0.855082912761355)(171,0.856317689530686)(172,0.860043509789703)(173,0.860043509789703)(174,0.859420289855072)(175,0.860043509789703)(176,0.852835606604451)(177,0.852835606604451)(178,0.852835606604451)(179,0.852624011502516)(180,0.853851691864651)(181,0.853851691864651)(182,0.853851691864651)(183,0.855082912761355)(184,0.855699855699856)(185,0.854466858789625)(186,0.854466858789625)(187,0.853851691864651)(188,0.853237410071942)(189,0.853237410071942)(190,0.853237410071942)(191,0.853237410071942)(192,0.851399856424982)(193,0.853237410071942)(194,0.853851691864651)(195,0.853851691864651)(196,0.853851691864651)(197,0.853851691864651)(198,0.855082912761355)(199,0.855699855699856)(200,0.855699855699856)(201,0.856317689530686)(202,0.856936416184971)(203,0.860043509789703)(204,0.86066763425254)(205,0.862545454545454)(206,0.862545454545454)(207,0.862545454545454)(208,0.862545454545454)(209,0.862545454545454)(210,0.861292665214234)(211,0.861292665214234)(212,0.861493836113125)(213,0.861918604651163)(214,0.861918604651163)(215,0.862545454545454)(216,0.861918604651163)(217,0.862545454545454)(218,0.862545454545454)(219,0.862545454545454)(220,0.861292665214234)(221,0.861918604651163)(222,0.861717612809316)(223,0.861717612809316)(224,0.86046511627907)(225,0.861717612809316)(226,0.861090909090909)(227,0.859420289855072)(228,0.859420289855072)(229,0.85921625544267)(230,0.85859318346628)(231,0.857971014492753)(232,0.857761732851986)(233,0.857761732851986)(234,0.857761732851986)(235,0.858381502890173)(236,0.858381502890173)(237,0.858381502890173)(238,0.856524873828407)(239,0.856524873828407)(240,0.856731461483081)(241,0.856524873828407)(242,0.856524873828407)(243,0.859002169197397)(244,0.858381502890173)(245,0.856115107913669)(246,0.858797972483707)(247,0.857556037599421)(248,0.857556037599421)(249,0.863603209336251)(250,0.863603209336251)(251,0.862973760932945)(252,0.861717612809316)(253,0.861918604651163)(254,0.861292665214234)(255,0.857142857142857)(256,0.857142857142857)(257,0.857142857142857)(258,0.857142857142857)(259,0.855291576673866)(260,0.854676258992806)(261,0.860043509789703)(262,0.860043509789703)(263,0.859420289855072)(264,0.859420289855072)(265,0.859420289855072)(266,0.86066763425254)(267,0.861090909090909)(268,0.861090909090909)(269,0.864233576642336)(270,0.862345229424617)(271,0.862345229424617)(272,0.867399267399267)(273,0.871228844738778)(274,0.873156342182891)(275,0.872512896094326)(276,0.871870397643593)(277,0.873156342182891)(278,0.873156342182891)(279,0.869309838472834)(280,0.867399267399267)(281,0.866764275256222)(282,0.869309838472834)(283,0.868672046955246)(284,0.867399267399267)(285,0.871870397643593)(286,0.871870397643593)(287,0.875092387287509)(288,0.875739644970414)(289,0.875739644970414)(290,0.873156342182891)(291,0.869309838472834)(292,0.869309838472834)(293,0.869309838472834)(294,0.869309838472834)(295,0.869309838472834)(296,0.869309838472834)(297,0.869309838472834)(298,0.869309838472834)(299,0.875739644970414)(300,0.879284649776453)(301,0.87797619047619)(302,0.87797619047619)(303,0.878810408921933)(304,0.880597014925373)(305,0.8801191362621)(306,0.87815750371471)(307,0.876854599406528)(308,0.877037037037037)(309,0.880059970014992)(310,0.87964338781575)(311,0.878990348923534)(312,0.880952380952381)(313,0.87964338781575)(314,0.879464285714286)(315,0.879464285714286)(316,0.877037037037037)(317,0.881278538812785)(318,0.882758620689655)(319,0.882758620689655)(320,0.882758620689655)(321,0.883614088820827)(322,0.882578664620107)(323,0.883935434281322)(324,0.883935434281322)(325,0.883935434281322)(326,0.885296381832179)(327,0.884615384615385)(328,0.883935434281322)(329,0.883935434281322)(330,0.883935434281322)(331,0.884792626728111)(332,0.885648503453568)(333,0.885321100917431)(334,0.885321100917431)(335,0.886519421172887)(336,0.885496183206107)(337,0.886850152905199)(338,0.886172650878533)(339,0.887366818873668)(340,0.887366818873668)(341,0.88735632183908)(342,0.88735632183908)(343,0.888036809815951)(344,0.88871834228703)(345,0.88871834228703)(346,0.88735632183908)(347,0.88735632183908)(348,0.888382687927107)(349,0.886877828054299)(350,0.887366818873668)(351,0.887377173091459)(352,0.888212927756654)(353,0.887537993920972)(354,0.887537993920972)(355,0.886209495101733)(356,0.887547169811321)(357,0.887708649468892)(358,0.888720666161998)(359,0.888547271329746)(360,0.888547271329746)(361,0.888547271329746)(362,0.887864823348694)(363,0.888036809815951)(364,0.88871834228703)(365,0.887864823348694)(366,0.88871834228703)(367,0.88871834228703)(368,0.88871834228703)(369,0.88871834228703)(370,0.88871834228703)(371,0.88871834228703)(372,0.88871834228703)(373,0.88871834228703)(374,0.889230769230769)(375,0.889230769230769)(376,0.889230769230769)(377,0.889060092449923)(378,0.886661526599846)(379,0.888030888030888)(380,0.886661526599846)(381,0.885271317829457)(382,0.888720666161998)(383,0.888720666161998)(384,0.887708649468892)(385,0.888547271329746)(386,0.889908256880734)(387,0.891954022988506)(388,0.891271056661562)(389,0.889908256880734)(390,0.889739663093415)(391,0.886864085041761)(392,0.887708649468892)(393,0.887708649468892)(394,0.886864085041761)(395,0.886864085041761)(396,0.8855193328279)(397,0.8855193328279)(398,0.8855193328279)(399,0.8855193328279)(400,0.886191198786039)(9,0.724477244772448)(10,0.73200241984271)(11,0.714987714987715)(12,0.735074626865671)(13,0.729088639200999)(14,0.729411764705882)(15,0.747438215792646)(16,0.77897403419886)(17,0.77897403419886)(18,0.784876140808344)(19,0.794754098360656)(20,0.794484569927774)(21,0.800261096605744)(22,0.819506016466118)(23,0.818998716302952)(24,0.818181818181818)(25,0.82093316519546)(26,0.823081800887761)(27,0.823305889803673)(28,0.8248730964467)(29,0.826837060702875)(30,0.848243870112657)(31,0.857898215465961)(32,0.864181091877497)(33,0.859614105123087)(34,0.870445344129555)(35,0.871864406779661)(36,0.871690427698574)(37,0.866756393001346)(38,0.867244829886591)(39,0.857894736842105)(40,0.858645627876397)(41,0.86186384666226)(42,0.860557768924303)(43,0.861702127659574)(44,0.869565217391304)(45,0.867647058823529)(46,0.86479250334672)(47,0.865612648221344)(48,0.861237785016286)(49,0.856586632057106)(50,0.852903225806452)(51,0.871287128712871)(52,0.872031662269129)(53,0.872607260726073)(54,0.876494023904382)(55,0.881287726358149)(56,0.881605351170568)(57,0.884408602150538)(58,0.88231338264963)(59,0.88231338264963)(60,0.879356568364611)(61,0.878767582049565)(62,0.879946344735077)(63,0.880536912751678)(64,0.878767582049565)(65,0.875250166777852)(66,0.875250166777852)(67,0.874083944037308)(68,0.878179384203481)(69,0.875834445927904)(70,0.876419505678023)(71,0.878243512974052)(72,0.876657824933687)(73,0.880478087649402)(74,0.884)(75,0.884)(76,0.889187374076561)(77,0.889336016096579)(78,0.886363636363636)(79,0.885923949299533)(80,0.883720930232558)(81,0.883134130146082)(82,0.885486018641811)(83,0.887994634473508)(84,0.887994634473508)(85,0.888590604026845)(86,0.887700534759358)(87,0.887107548430194)(88,0.885333333333333)(89,0.879048248512888)(90,0.87962962962963)(91,0.878467635402906)(92,0.875081752779595)(93,0.878467635402906)(94,0.874424720578567)(95,0.872131147540984)(96,0.872703412073491)(97,0.865885416666667)(98,0.867579908675799)(99,0.865322055953155)(100,0.867014341590613)(101,0.867014341590613)(102,0.867579908675799)(103,0.870988867059594)(104,0.877146631439894)(105,0.878888153540701)(106,0.876567656765676)(107,0.880053015241882)(108,0.884154460719041)(109,0.884154460719041)(110,0.885180240320427)(111,0.88695652173913)(112,0.88993288590604)(113,0.891582491582491)(114,0.89054054054054)(115,0.891745602165088)(116,0.892349356804333)(117,0.892349356804333)(118,0.889939230249831)(119,0.891142663962136)(120,0.89054054054054)(121,0.890392422192151)(122,0.889790398918188)(123,0.890995260663507)(124,0.891142663962136)(125,0.892953929539295)(126,0.891745602165088)(127,0.891745602165088)(128,0.916138125440451)(129,0.916138125440451)(130,0.916784203102962)(131,0.919605077574048)(132,0.919605077574048)(133,0.922206506364922)(134,0.919914953933381)(135,0.921332388377037)(136,0.919572953736655)(137,0.919458303635068)(138,0.919572953736655)(139,0.919572953736655)(140,0.919572953736655)(141,0.920883820384889)(142,0.920996441281139)(143,0.920996441281139)(144,0.922309337134711)(145,0.919228020014296)(146,0.921203438395415)(147,0.920544022906228)(148,0.920315865039483)(149,0.921526277897768)(150,0.923076923076923)(151,0.923850574712644)(152,0.924623115577889)(153,0.923741007194245)(154,0.923959827833572)(155,0.923959827833572)(156,0.92451473759885)(157,0.924406047516199)(158,0.924406047516199)(159,0.924406047516199)(160,0.926406926406926)(161,0.927536231884058)(162,0.929190751445086)(163,0.929088277858177)(164,0.929862617498192)(165,0.930535455861071)(166,0.928985507246377)(167,0.929761042722665)(168,0.929761042722665)(169,0.929761042722665)(170,0.931286549707602)(171,0.931286549707602)(172,0.932067202337473)(173,0.932846715328467)(174,0.934497816593886)(175,0.933721777130371)(176,0.932944606413994)(177,0.932944606413994)(178,0.931386861313869)(179,0.931386861313869)(180,0.931486880466472)(181,0.930707512764405)(182,0.931486880466472)(183,0.92992700729927)(184,0.930707512764405)(185,0.929824561403509)(186,0.929824561403509)(187,0.929824561403509)(188,0.930707512764405)(189,0.930707512764405)(190,0.930707512764405)(191,0.929824561403509)(192,0.930606281957633)(193,0.930707512764405)(194,0.92992700729927)(195,0.929721815519766)(196,0.928937728937729)(197,0.929721815519766)(198,0.929721815519766)(199,0.929721815519766)(200,0.931286549707602)(201,0.931286549707602)(202,0.931386861313869)(203,0.932944606413994)(204,0.932166301969365)(205,0.932265112891478)(206,0.932265112891478)(207,0.932265112891478)(208,0.932265112891478)(209,0.932265112891478)(210,0.932265112891478)(211,0.932265112891478)(212,0.932265112891478)(213,0.932265112891478)(214,0.931586608442504)(215,0.932265112891478)(216,0.932265112891478)(217,0.933042212518195)(218,0.932944606413994)(219,0.932944606413994)(220,0.931386861313869)(221,0.932944606413994)(222,0.930606281957633)(223,0.929041697147037)(224,0.929824561403509)(225,0.929824561403509)(226,0.929824561403509)(227,0.931185944363104)(228,0.93108504398827)(229,0.931768158473954)(230,0.93108504398827)(231,0.933625091174325)(232,0.932363636363636)(233,0.935553946415641)(234,0.936231884057971)(235,0.935460478607687)(236,0.934687953555878)(237,0.935460478607687)(238,0.936046511627907)(239,0.935083880379285)(240,0.932551319648094)(241,0.931567328918322)(242,0.930780559646539)(243,0.931567328918322)(244,0.930780559646539)(245,0.931567328918322)(246,0.931567328918322)(247,0.935578330893118)(248,0.933920704845815)(249,0.933920704845815)(250,0.933823529411765)(251,0.933823529411765)(252,0.933137398971345)(253,0.932253313696613)(254,0.933038999264165)(255,0.933038999264165)(256,0.933038999264165)(257,0.935389133627019)(258,0.933333333333333)(259,0.933333333333333)(260,0.929411764705882)(261,0.928728875826598)(262,0.928728875826598)(263,0.928728875826598)(264,0.929515418502203)(265,0.93108504398827)(266,0.931868131868132)(267,0.931868131868132)(268,0.93108504398827)(269,0.931868131868132)(270,0.93108504398827)(271,0.930983847283407)(272,0.930983847283407)(273,0.930300807043287)(274,0.93040293040293)(275,0.930300807043287)(276,0.931185944363104)(277,0.93040293040293)(278,0.93040293040293)(279,0.93108504398827)(280,0.93108504398827)(281,0.93040293040293)(282,0.93040293040293)(283,0.930300807043287)(284,0.930300807043287)(285,0.93108504398827)(286,0.931286549707602)(287,0.931286549707602)(288,0.932265112891478)(289,0.934105720492397)(290,0.934201012292118)(291,0.935460478607687)(292,0.937184115523466)(293,0.944881889763779)(294,0.947368421052632)(295,0.952515946137491)(296,0.951841359773371)(297,0.954674220963173)(298,0.955539872971066)(299,0.954866008462623)(300,0.96)(301,0.96)(302,0.960056061667835)(303,0.958041958041958)(304,0.957372466806429)(305,0.956036287508723)(306,0.956703910614525)(307,0.958041958041958)(308,0.958712386284115)(309,0.958712386284115)(310,0.958041958041958)(311,0.955369595536959)(312,0.953375086986778)(313,0.953375086986778)(314,0.952712100139082)(315,0.952712100139082)(316,0.955369595536959)(317,0.954038997214484)(318,0.955369595536959)(319,0.955369595536959)(320,0.956703910614525)(321,0.956703910614525)(322,0.960056061667835)(323,0.958041958041958)(324,0.958041958041958)(325,0.958041958041958)(326,0.954703832752613)(327,0.956036287508723)(328,0.956036287508723)(329,0.957372466806429)(330,0.957372466806429)(331,0.956703910614525)(332,0.956703910614525)(333,0.956703910614525)(334,0.956703910614525)(335,0.956703910614525)(336,0.955369595536959)(337,0.955369595536959)(338,0.953375086986778)(339,0.954038997214484)(340,0.952712100139082)(341,0.954703832752613)(342,0.954703832752613)(343,0.955369595536959)(344,0.956036287508723)(345,0.953846153846154)(346,0.953846153846154)(347,0.953846153846154)(348,0.953846153846154)(349,0.953781512605042)(350,0.957446808510638)(351,0.957446808510638)(352,0.959660297239915)(353,0.961538461538461)(354,0.958393113342898)(355,0.957706093189964)(356,0.957706093189964)(357,0.956209619526202)(358,0.956209619526202)(359,0.955523672883788)(360,0.956272401433692)(361,0.95702005730659)(362,0.95702005730659)(363,0.957766642806013)(364,0.959256611865618)(365,0.959198282032928)(366,0.959942775393419)(367,0.959198282032928)(368,0.959198282032928)(369,0.962115796997855)(370,0.962910128388017)(371,0.956834532374101)(372,0.956834532374101)(373,0.956834532374101)(374,0.948830409356725)(375,0.948830409356725)(376,0.949598246895544)(377,0.95036496350365)(378,0.949598246895544)(379,0.95036496350365)(380,0.949598246895544)(381,0.949524506217995)(382,0.949524506217995)(383,0.952589350838804)(384,0.950292397660819)(385,0.951059167275383)(386,0.954115076474872)(387,0.947985347985348)(388,0.947985347985348)(389,0.947214076246334)(390,0.950219619326501)(391,0.947909024211299)(392,0.947909024211299)(393,0.947136563876652)(394,0.947136563876652)(395,0.947136563876652)(396,0.947136563876652)(397,0.947136563876652)(398,0.947136563876652)(399,0.947136563876652)(400,0.947136563876652)(4,0.526162790697674)(5,0.709210526315789)(6,0.756454989532449)(7,0.777217015140591)(8,0.782735918068764)(9,0.775451263537906)(10,0.777777777777778)(11,0.777697320782042)(12,0.790697674418605)(13,0.797069597069597)(14,0.806354009077156)(15,0.806060606060606)(16,0.805429864253393)(17,0.804545454545454)(18,0.805450416351249)(19,0.802710843373494)(20,0.795180722891566)(21,0.801195814648729)(22,0.803611738148984)(23,0.805097451274363)(24,0.809344385832705)(25,0.809954751131222)(26,0.803290949887808)(27,0.801788375558867)(28,0.805389221556886)(29,0.79940784603997)(30,0.802973977695167)(31,0.802973977695167)(32,0.802973977695167)(33,0.802377414561664)(34,0.802083333333333)(35,0.802083333333333)(36,0.802083333333333)(37,0.802083333333333)(38,0.803877703206562)(39,0.8)(40,0.803571428571428)(41,0.807778608825729)(42,0.808988764044944)(43,0.8095952023988)(44,0.81203007518797)(45,0.81203007518797)(46,0.815315315315315)(47,0.817704426106526)(48,0.81893313298272)(49,0.815868263473054)(50,0.81525804038893)(51,0.816479400749064)(52,0.81525804038893)(53,0.811615785554728)(54,0.813432835820895)(55,0.813432835820895)(56,0.81282624906786)(57,0.813432835820895)(58,0.813432835820895)(59,0.813483146067416)(60,0.813483146067416)(61,0.814092953523238)(62,0.81376215407629)(63,0.812546676624347)(64,0.81376215407629)(65,0.81282624906786)(66,0.812220566318927)(67,0.813432835820895)(68,0.814040328603435)(69,0.814040328603435)(70,0.813432835820895)(71,0.814648729446936)(72,0.814648729446936)(73,0.81282624906786)(74,0.81282624906786)(75,0.811615785554728)(76,0.810408921933085)(77,0.810408921933085)(78,0.811011904761905)(79,0.809806835066865)(80,0.809806835066865)(81,0.80920564216778)(82,0.808605341246291)(83,0.809806835066865)(84,0.810408921933085)(85,0.811615785554728)(86,0.81282624906786)(87,0.811011904761905)(88,0.814648729446936)(89,0.814648729446936)(90,0.814648729446936)(91,0.816479400749064)(92,0.81525804038893)(93,0.814040328603435)(94,0.814040328603435)(95,0.812220566318927)(96,0.813432835820895)(97,0.810408921933085)(98,0.810408921933085)(99,0.810408921933085)(100,0.809806835066865)(101,0.809806835066865)(102,0.809806835066865)(103,0.811011904761905)(104,0.80920564216778)(105,0.810408921933085)(106,0.812220566318927)(107,0.810408921933085)(108,0.810408921933085)(109,0.810408921933085)(110,0.809806835066865)(111,0.809806835066865)(112,0.80920564216778)(113,0.808005930318755)(114,0.808005930318755)(115,0.80920564216778)(116,0.80920564216778)(117,0.80920564216778)(118,0.80920564216778)(119,0.80920564216778)(120,0.80920564216778)(121,0.80920564216778)(122,0.809806835066865)(123,0.809806835066865)(124,0.809806835066865)(125,0.809806835066865)(126,0.809806835066865)(127,0.80920564216778)(128,0.810408921933085)(129,0.809806835066865)(130,0.811011904761905)(131,0.81282624906786)(132,0.812220566318927)(133,0.811615785554728)(134,0.81282624906786)(135,0.81282624906786)(136,0.832980972515856)(137,0.832378223495702)(138,0.81282624906786)(139,0.824224519940916)(140,0.834502103786816)(141,0.840085287846482)(142,0.84051724137931)(143,0.842255531763026)(144,0.842559309849029)(145,0.840892728581713)(146,0.844444444444444)(147,0.838297872340426)(148,0.839260312944523)(149,0.838203848895224)(150,0.838801711840228)(151,0.840663302090844)(152,0.843235504652827)(153,0.840057636887608)(154,0.840892728581713)(155,0.844667143879742)(156,0.844285714285714)(157,0.844285714285714)(158,0.843683083511777)(159,0.843971631205674)(160,0.844570617459191)(161,0.843373493975904)(162,0.842776203966006)(163,0.840989399293286)(164,0.840989399293286)(165,0.840989399293286)(166,0.839802399435427)(167,0.839802399435427)(168,0.839210155148096)(169,0.839210155148096)(170,0.840395480225989)(171,0.838028169014085)(172,0.838028169014085)(173,0.838028169014085)(174,0.836261419536191)(175,0.834502103786816)(176,0.836849507735584)(177,0.83743842364532)(178,0.836849507735584)(179,0.836849507735584)(180,0.836849507735584)(181,0.836849507735584)(182,0.837078651685393)(183,0.838483146067416)(184,0.840845070422535)(185,0.840845070422535)(186,0.840253342716397)(187,0.840253342716397)(188,0.841437632135306)(189,0.841437632135306)(190,0.841437632135306)(191,0.840845070422535)(192,0.839662447257384)(193,0.840253342716397)(194,0.840845070422535)(195,0.842031029619182)(196,0.842031029619182)(197,0.841437632135306)(198,0.840253342716397)(199,0.840253342716397)(200,0.840845070422535)(201,0.840845070422535)(202,0.841437632135306)(203,0.842031029619182)(204,0.842031029619182)(205,0.843220338983051)(206,0.842031029619182)(207,0.840253342716397)(208,0.841437632135306)(209,0.840845070422535)(210,0.840845070422535)(211,0.840845070422535)(212,0.840845070422535)(213,0.840253342716397)(214,0.842031029619182)(215,0.840253342716397)(216,0.842625264643613)(217,0.845010615711252)(218,0.843220338983051)(219,0.848011363636364)(220,0.85042735042735)(221,0.85042735042735)(222,0.847409510290986)(223,0.846208362863218)(224,0.847409510290986)(225,0.854077253218884)(226,0.854077253218884)(227,0.853466761972838)(228,0.846808510638298)(229,0.852857142857143)(230,0.852248394004283)(231,0.852857142857143)(232,0.853466761972838)(233,0.852857142857143)(234,0.852857142857143)(235,0.852248394004283)(236,0.851640513552068)(237,0.851033499643621)(238,0.851033499643621)(239,0.851033499643621)(240,0.851033499643621)(241,0.851033499643621)(242,0.851033499643621)(243,0.851033499643621)(244,0.853466761972838)(245,0.855913978494624)(246,0.855300859598854)(247,0.855300859598854)(248,0.851033499643621)(249,0.854077253218884)(250,0.853466761972838)(251,0.853466761972838)(252,0.856527977044476)(253,0.856527977044476)(254,0.856527977044476)(255,0.855707106963388)(256,0.856527977044476)(257,0.851033499643621)(258,0.85042735042735)(259,0.848011363636364)(260,0.849217638691323)(261,0.851640513552068)(262,0.852248394004283)(263,0.851640513552068)(264,0.851640513552068)(265,0.853466761972838)(266,0.853466761972838)(267,0.853466761972838)(268,0.852857142857143)(269,0.852248394004283)(270,0.852248394004283)(271,0.852248394004283)(272,0.852248394004283)(273,0.851640513552068)(274,0.852248394004283)(275,0.852248394004283)(276,0.852248394004283)(277,0.853466761972838)(278,0.853466761972838)(279,0.854077253218884)(280,0.854077253218884)(281,0.846208362863218)(282,0.845010615711252)(283,0.843220338983051)(284,0.840845070422535)(285,0.840253342716397)(286,0.845010615711252)(287,0.845010615711252)(288,0.845010615711252)(289,0.845609065155807)(290,0.848614072494669)(291,0.848614072494669)(292,0.848614072494669)(293,0.848614072494669)(294,0.849822064056939)(295,0.849822064056939)(296,0.849217638691323)(297,0.849217638691323)(298,0.854688618468146)(299,0.855300859598854)(300,0.854688618468146)(301,0.855913978494624)(302,0.855913978494624)(303,0.855913978494624)(304,0.855300859598854)(305,0.855300859598854)(306,0.845010615711252)(307,0.847409510290986)(308,0.846208362863218)(309,0.845609065155807)(310,0.846808510638298)(311,0.846808510638298)(312,0.845609065155807)(313,0.845010615711252)(314,0.845609065155807)(315,0.843220338983051)(316,0.847409510290986)(317,0.848614072494669)(318,0.849217638691323)(319,0.848614072494669)(320,0.848011363636364)(321,0.847409510290986)(322,0.848614072494669)(323,0.856527977044476)(324,0.855913978494624)(325,0.855300859598854)(326,0.855300859598854)(327,0.855300859598854)(328,0.855300859598854)(329,0.854688618468146)(330,0.854688618468146)(331,0.857142857142857)(332,0.865021770682148)(333,0.865454545454545)(334,0.863570391872278)(335,0.864394488759971)(336,0.865649963689179)(337,0.865021770682148)(338,0.869692532942899)(339,0.86950146627566)(340,0.86950146627566)(341,0.871156661786237)(342,0.871606749816581)(343,0.871156661786237)(344,0.877218934911243)(345,0.877218934911243)(346,0.877218934911243)(347,0.878877400295421)(348,0.878877400295421)(349,0.88)(350,0.88)(351,0.88)(352,0.88)(353,0.88)(354,0.88)(355,0.869247626004383)(356,0.869247626004383)(357,0.876288659793814)(358,0.872434017595308)(359,0.866909090909091)(360,0.871794871794872)(361,0.871156661786237)(362,0.870519385515728)(363,0.870519385515728)(364,0.858992805755396)(365,0.858992805755396)(366,0.858992805755396)(367,0.858992805755396)(368,0.858375269590223)(369,0.858375269590223)(370,0.859611231101512)(371,0.859611231101512)(372,0.859611231101512)(373,0.855913978494624)(374,0.855913978494624)(375,0.854077253218884)(376,0.853466761972838)(377,0.851033499643621)(378,0.852857142857143)(379,0.852248394004283)(380,0.853466761972838)(381,0.853466761972838)(382,0.85042735042735)(383,0.848614072494669)(384,0.848614072494669)(385,0.848614072494669)(386,0.857758620689655)(387,0.857758620689655)(388,0.858375269590223)(389,0.858375269590223)(390,0.858375269590223)(391,0.858375269590223)(392,0.857758620689655)(393,0.857142857142857)(394,0.857142857142857)(395,0.857142857142857)(396,0.855913978494624)(397,0.857142857142857)(398,0.857142857142857)(399,0.858375269590223)(400,0.858375269590223)(7,0.690349319123742)(8,0.692400482509047)(9,0.693317422434367)(10,0.724031960663798)(11,0.725765306122449)(12,0.731800766283525)(13,0.735330836454432)(14,0.740458015267175)(15,0.763341067285383)(16,0.756041426927503)(17,0.757417102966841)(18,0.740994854202401)(19,0.738444193912063)(20,0.736486486486486)(21,0.737197523916713)(22,0.730088495575221)(23,0.738617200674536)(24,0.746878547105562)(25,0.75242995997713)(26,0.749430523917995)(27,0.768867924528302)(28,0.768779342723004)(29,0.767700409596255)(30,0.768599882835384)(31,0.766279069767442)(32,0.772300469483568)(33,0.771395076201641)(34,0.770491803278688)(35,0.770128354725788)(36,0.774492753623188)(37,0.775842044134727)(38,0.777003484320558)(39,0.777906976744186)(40,0.784452296819788)(41,0.799313893653516)(42,0.791714614499424)(43,0.793453724604966)(44,0.792772444946358)(45,0.798879551820728)(46,0.800223838836038)(47,0.800223838836038)(48,0.801569506726457)(49,0.812961910176236)(50,0.809739524348811)(51,0.810011376564277)(52,0.808631459398069)(53,0.80955088118249)(54,0.808390022675737)(55,0.806779661016949)(56,0.805414551607445)(57,0.804960541149944)(58,0.806324110671937)(59,0.806324110671937)(60,0.812749003984064)(61,0.816466552315609)(62,0.828588030214991)(63,0.830518345952242)(64,0.829552065154159)(65,0.819747416762342)(66,0.821161587119034)(67,0.815068493150685)(68,0.815068493150685)(69,0.820011500862565)(70,0.8204833141542)(71,0.819690898683457)(72,0.819690898683457)(73,0.818753573470555)(74,0.821100917431193)(75,0.822515795519816)(76,0.83908718548859)(77,0.840070298769771)(78,0.83859649122807)(79,0.829861111111111)(80,0.830341632889403)(81,0.829381145170619)(82,0.848018923713779)(83,0.851038575667656)(84,0.850029638411381)(85,0.849526066350711)(86,0.848018923713779)(87,0.849023090586146)(88,0.849526066350711)(89,0.849526066350711)(90,0.849526066350711)(91,0.849526066350711)(92,0.847017129356172)(93,0.843529411764706)(94,0.843033509700176)(95,0.844025897586816)(96,0.843529411764706)(97,0.841549295774648)(98,0.837127845884413)(99,0.840070298769771)(100,0.83859649122807)(101,0.83859649122807)(102,0.836151603498542)(103,0.837616822429906)(104,0.836151603498542)(105,0.837127845884413)(106,0.838106370543542)(107,0.838106370543542)(108,0.83908718548859)(109,0.839578454332553)(110,0.844025897586816)(111,0.846017699115044)(112,0.846517119244392)(113,0.846517119244392)(114,0.846017699115044)(115,0.848520710059172)(116,0.849023090586146)(117,0.848018923713779)(118,0.847017129356172)(119,0.847517730496454)(120,0.847017129356172)(121,0.848520710059172)(122,0.849023090586146)(123,0.848520710059172)(124,0.850029638411381)(125,0.848018923713779)(126,0.846517119244392)(127,0.846517119244392)(128,0.844025897586816)(129,0.843529411764706)(130,0.842043452730476)(131,0.841549295774648)(132,0.841549295774648)(133,0.83908718548859)(134,0.841055718475073)(135,0.840562719812427)(136,0.83908718548859)(137,0.840070298769771)(138,0.840562719812427)(139,0.839578454332553)(140,0.837127845884413)(141,0.833720930232558)(142,0.833236490412551)(143,0.832752613240418)(144,0.833236490412551)(145,0.833236490412551)(146,0.834205933682373)(147,0.834205933682373)(148,0.840070298769771)(149,0.840562719812427)(150,0.841549295774648)(151,0.84452296819788)(152,0.846017699115044)(153,0.845518867924528)(154,0.845518867924528)(155,0.847017129356172)(156,0.849023090586146)(157,0.853571428571428)(158,0.85288862418106)(159,0.853396901072706)(160,0.85288862418106)(161,0.855436081242533)(162,0.854925373134328)(163,0.853905784138342)(164,0.853905784138342)(165,0.856971873129862)(166,0.859028194361128)(167,0.859028194361128)(168,0.860576923076923)(169,0.86006006006006)(170,0.86006006006006)(171,0.86006006006006)(172,0.86006006006006)(173,0.859543817527011)(174,0.858513189448441)(175,0.859543817527011)(176,0.86006006006006)(177,0.862131246237206)(178,0.862131246237206)(179,0.863691194209891)(180,0.863170584689572)(181,0.863170584689572)(182,0.862650602409638)(183,0.860576923076923)(184,0.860576923076923)(185,0.860576923076923)(186,0.86006006006006)(187,0.860576923076923)(188,0.857998801677651)(189,0.85748502994012)(190,0.854925373134328)(191,0.85645933014354)(192,0.85645933014354)(193,0.857998801677651)(194,0.858682634730539)(195,0.863855421686747)(196,0.863691194209891)(197,0.863691194209891)(198,0.865779927448609)(199,0.871202916160389)(200,0.87599266951741)(201,0.877600979192166)(202,0.871046228710462)(203,0.873703477730323)(204,0.873703477730323)(205,0.87599266951741)(206,0.876528117359413)(207,0.876528117359413)(208,0.878676470588235)(209,0.877600979192166)(210,0.878138395590937)(211,0.878138395590937)(212,0.877988963825874)(213,0.875840978593272)(214,0.875305623471883)(215,0.873170731707317)(216,0.873170731707317)(217,0.875840978593272)(218,0.876376988984088)(219,0.876376988984088)(220,0.87960687960688)(221,0.880147510755993)(222,0.880147510755993)(223,0.880688806888069)(224,0.881773399014778)(225,0.880688806888069)(226,0.877450980392157)(227,0.877450980392157)(228,0.877450980392157)(229,0.875840978593272)(230,0.876376988984088)(231,0.877450980392157)(232,0.883405305367057)(233,0.883405305367057)(234,0.883405305367057)(235,0.882860665844636)(236,0.882860665844636)(237,0.882860665844636)(238,0.882316697473814)(239,0.880147510755993)(240,0.881230769230769)(241,0.880147510755993)(242,0.879066912216083)(243,0.877450980392157)(244,0.879066912216083)(245,0.881773399014778)(246,0.880147510755993)(247,0.881773399014778)(248,0.880147510755993)(249,0.880147510755993)(250,0.887786732796032)(251,0.88833746898263)(252,0.883405305367057)(253,0.881773399014778)(254,0.882316697473814)(255,0.883405305367057)(256,0.883405305367057)(257,0.88504326328801)(258,0.88833746898263)(259,0.887647423960273)(260,0.880688806888069)(261,0.882316697473814)(262,0.882316697473814)(263,0.87960687960688)(264,0.87960687960688)(265,0.87960687960688)(266,0.880147510755993)(267,0.881773399014778)(268,0.883950617283951)(269,0.881773399014778)(270,0.880688806888069)(271,0.880688806888069)(272,0.880147510755993)(273,0.880147510755993)(274,0.880147510755993)(275,0.880147510755993)(276,0.880147510755993)(277,0.88833746898263)(278,0.888888888888889)(279,0.888888888888889)(280,0.88944099378882)(281,0.88833746898263)(282,0.890547263681592)(283,0.891101431238332)(284,0.888888888888889)(285,0.888888888888889)(286,0.891656288916563)(287,0.899937067337948)(288,0.89937106918239)(289,0.900503778337531)(290,0.900503778337531)(291,0.900503778337531)(292,0.900503778337531)(293,0.900503778337531)(294,0.899937067337948)(295,0.900503778337531)(296,0.900503778337531)(297,0.902208201892744)(298,0.901639344262295)(299,0.90107120352867)(300,0.901639344262295)(301,0.902777777777778)(302,0.903470031545741)(303,0.902900378310214)(304,0.902900378310214)(305,0.902900378310214)(306,0.902900378310214)(307,0.903470031545741)(308,0.904040404040404)(309,0.904611497157296)(310,0.905755850727388)(311,0.903470031545741)(312,0.904040404040404)(313,0.907477820025348)(314,0.908053265694356)(315,0.906903103229892)(316,0.908053265694356)(317,0.92258064516129)(318,0.932203389830508)(319,0.931596091205212)(320,0.932811480756686)(321,0.932811480756686)(322,0.934640522875817)(323,0.93403004572175)(324,0.935251798561151)(325,0.93586387434555)(326,0.935251798561151)(327,0.93586387434555)(328,0.94017094017094)(329,0.939553219448095)(330,0.94078947368421)(331,0.94017094017094)(332,0.939553219448095)(333,0.939553219448095)(334,0.944444444444444)(335,0.943196829590489)(336,0.942574257425743)(337,0.95130086724483)(338,0.951935914552737)(339,0.954423592493297)(340,0.954423592493297)(341,0.954423592493297)(342,0.954423592493297)(343,0.953784326858674)(344,0.953784326858674)(345,0.953784326858674)(346,0.955704697986577)(347,0.955704697986577)(348,0.955704697986577)(349,0.955704697986577)(350,0.955063715627096)(351,0.955704697986577)(352,0.955704697986577)(353,0.956169925826028)(354,0.955525606469003)(355,0.955525606469003)(356,0.950600801068091)(357,0.95123580494322)(358,0.946949602122016)(359,0.946949602122016)(360,0.949400798934754)(361,0.955704697986577)(362,0.955063715627096)(363,0.955063715627096)(364,0.958277254374159)(365,0.956815114709851)(366,0.951935914552737)(367,0.951935914552737)(368,0.951935914552737)(369,0.948207171314741)(370,0.948207171314741)(371,0.947019867549669)(372,0.929778933680104)(373,0.926860841423948)(374,0.926860841423948)(375,0.926860841423948)(376,0.926860841423948)(377,0.926261319534282)(378,0.926261319534282)(379,0.926261319534282)(380,0.926955397543633)(381,0.928155339805825)(382,0.924564796905222)(383,0.925758553905745)(384,0.925161290322581)(385,0.926356589147287)(386,0.925758553905745)(387,0.925758553905745)(388,0.922186495176849)(389,0.922186495176849)(390,0.922186495176849)(391,0.922186495176849)(392,0.92159383033419)(393,0.925758553905745)(394,0.925758553905745)(395,0.926356589147287)(396,0.925758553905745)(397,0.926356589147287)(398,0.926356589147287)(399,0.931079323797139)(400,0.931685100845803)(10,0.553745928338762)(11,0.55229662423907)(12,0.663910218546958)(13,0.704530950861519)(14,0.705248990578735)(15,0.709677419354839)(16,0.727272727272727)(17,0.715309446254072)(18,0.723097112860892)(19,0.721854304635761)(20,0.732595966167859)(21,0.735836177474403)(22,0.742581090407177)(23,0.750555144337528)(24,0.756400877834674)(25,0.758471521268926)(26,0.762177650429799)(27,0.769675090252707)(28,0.769898697539797)(29,0.775539568345324)(30,0.775481111903065)(31,0.774658027357811)(32,0.773121387283237)(33,0.778339350180505)(34,0.783053323593864)(35,0.783430232558139)(36,0.793225123500353)(37,0.788612099644128)(38,0.790299572039943)(39,0.790830945558739)(40,0.792235801581596)(41,0.800285306704707)(42,0.801987224982257)(43,0.803406671398155)(44,0.804564907275321)(45,0.807965860597439)(46,0.808813077469794)(47,0.820979020979021)(48,0.849154746423927)(49,0.854568854568854)(50,0.856584093872229)(51,0.857142857142857)(52,0.860858257477243)(53,0.858638743455497)(54,0.857329842931937)(55,0.859375)(56,0.861618798955613)(57,0.861237785016286)(58,0.862091084028223)(59,0.863549007046765)(60,0.862091084028223)(61,0.868571428571429)(62,0.859110832811522)(63,0.866540164452878)(64,0.866581956797967)(65,0.866031746031746)(66,0.86845466155811)(67,0.864417568427753)(68,0.864795918367347)(69,0.86569064290261)(70,0.87051282051282)(71,0.860567823343849)(72,0.860552763819095)(73,0.857680250783699)(74,0.8575)(75,0.856964397251718)(76,0.861059190031153)(77,0.859987554449284)(78,0.85785536159601)(79,0.858585858585858)(80,0.865384615384615)(81,0.865003198976327)(82,0.865900383141762)(83,0.867852604828462)(84,0.868020304568528)(85,0.868187579214195)(86,0.869176770899808)(87,0.870843989769821)(88,0.869731800766283)(89,0.869955156950672)(90,0.871595330739299)(91,0.877284595300261)(92,0.879423328964613)(93,0.878688524590164)(94,0.888297872340425)(95,0.88681757656458)(96,0.886378737541528)(97,0.880952380952381)(98,0.880952380952381)(99,0.881535407015221)(100,0.883289124668435)(101,0.883443708609271)(102,0.881921280853902)(103,0.878787878787879)(104,0.879574184963406)(105,0.88031914893617)(106,0.876567656765676)(107,0.877146631439894)(108,0.878369493754109)(109,0.876640419947507)(110,0.874918140144073)(111,0.87696335078534)(112,0.877537655533726)(113,0.878529218647406)(114,0.878529218647406)(115,0.879106438896189)(116,0.879684418145957)(117,0.883355176933158)(118,0.883355176933158)(119,0.884514435695538)(120,0.881621975147155)(121,0.883934426229508)(122,0.884514435695538)(123,0.886842105263158)(124,0.887425938117182)(125,0.886408404464872)(126,0.887868852459016)(127,0.887139107611548)(128,0.887139107611548)(129,0.887139107611548)(130,0.886408404464872)(131,0.885826771653543)(132,0.885826771653543)(133,0.885826771653543)(134,0.887573964497041)(135,0.885224274406332)(136,0.886393659180978)(137,0.888153540701522)(138,0.890501319261213)(139,0.896782841823056)(140,0.894385026737968)(141,0.893787575150301)(142,0.895123580494322)(143,0.89572192513369)(144,0.893645484949833)(145,0.894842598794374)(146,0.896182183523108)(147,0.897795591182365)(148,0.898861352980576)(149,0.90020093770931)(150,0.908108108108108)(151,0.908722109533468)(152,0.91156462585034)(153,0.909708078750849)(154,0.909708078750849)(155,0.909090909090909)(156,0.906630581867388)(157,0.906630581867388)(158,0.907244414353419)(159,0.907244414353419)(160,0.907244414353419)(161,0.909090909090909)(162,0.909090909090909)(163,0.909090909090909)(164,0.907608695652174)(165,0.906992532247115)(166,0.907118644067796)(167,0.910326086956522)(168,0.911444141689373)(169,0.908967391304348)(170,0.909462219196732)(171,0.909462219196732)(172,0.910944935418083)(173,0.912660798916723)(174,0.911546252532073)(175,0.908479138627187)(176,0.907868190988568)(177,0.909703504043127)(178,0.909703504043127)(179,0.909703504043127)(180,0.911665542818611)(181,0.912280701754386)(182,0.91339648173207)(183,0.913630229419703)(184,0.912896691424713)(185,0.910316925151719)(186,0.910316925151719)(187,0.910316925151719)(188,0.909090909090909)(189,0.909090909090909)(190,0.909090909090909)(191,0.909581646423752)(192,0.909090909090909)(193,0.910558170813719)(194,0.914209115281501)(195,0.917505030181086)(196,0.918230563002681)(197,0.918846411804158)(198,0.917615539182853)(199,0.924832214765101)(200,0.924832214765101)(201,0.924832214765101)(202,0.924932975871314)(203,0.924932975871314)(204,0.927090301003344)(205,0.927090301003344)(206,0.927090301003344)(207,0.925851703406814)(208,0.935135135135135)(209,0.933333333333333)(210,0.933333333333333)(211,0.933962264150943)(212,0.933333333333333)(213,0.900577293136626)(214,0.902688860435339)(215,0.904214559386973)(216,0.904214559386973)(217,0.903760356915232)(218,0.903760356915232)(219,0.902609802673456)(220,0.903184713375796)(221,0.903307888040712)(222,0.903882877148313)(223,0.903184713375796)(224,0.901015228426396)(225,0.901587301587301)(226,0.901015228426396)(227,0.903307888040712)(228,0.903225806451613)(229,0.892634207240949)(230,0.892077354959451)(231,0.88985687616677)(232,0.88875077688005)(233,0.88875077688005)(234,0.88875077688005)(235,0.88875077688005)(236,0.88985687616677)(237,0.89711417816813)(238,0.897677338355304)(239,0.897677338355304)(240,0.895989974937343)(241,0.896421845574388)(242,0.900378310214376)(243,0.902084649399874)(244,0.902084649399874)(245,0.90379746835443)(246,0.90379746835443)(247,0.90379746835443)(248,0.902777777777778)(249,0.903919089759798)(250,0.906210392902408)(251,0.90563647878404)(252,0.904731861198738)(253,0.904731861198738)(254,0.902454373820012)(255,0.902454373820012)(256,0.900753768844221)(257,0.901319924575739)(258,0.899623588456713)(259,0.903470031545741)(260,0.900753768844221)(261,0.900188323917137)(262,0.90188679245283)(263,0.90359168241966)(264,0.90359168241966)(265,0.90359168241966)(266,0.90359168241966)(267,0.902454373820012)(268,0.900753768844221)(269,0.899059561128527)(270,0.896681277395116)(271,0.898932831136221)(272,0.898932831136221)(273,0.899623588456713)(274,0.898496240601504)(275,0.900188323917137)(276,0.901319924575739)(277,0.904731861198738)(278,0.882461538461538)(279,0.882461538461538)(280,0.881918819188192)(281,0.881376767055931)(282,0.880835380835381)(283,0.879754601226994)(284,0.880294659300184)(285,0.880294659300184)(286,0.874390243902439)(287,0.875457875457875)(288,0.867513611615245)(289,0.863855421686747)(290,0.85663082437276)(291,0.858682634730539)(292,0.848520710059172)(293,0.847517730496454)(294,0.848018923713779)(295,0.848018923713779)(296,0.858168761220826)(297,0.857142857142857)(298,0.856119402985075)(299,0.855608591885441)(300,0.854079809410363)(301,0.850533807829181)(302,0.850533807829181)(303,0.857655502392344)(304,0.855608591885441)(305,0.855608591885441)(306,0.855098389982111)(307,0.855098389982111)(308,0.859712230215827)(309,0.861261261261261)(310,0.861778846153846)(311,0.866989117291415)(312,0.871732522796352)(313,0.874923733984136)(314,0.87599266951741)(315,0.874390243902439)(316,0.873325213154689)(317,0.874923733984136)(318,0.875457875457875)(319,0.87599266951741)(320,0.882860665844636)(321,0.882860665844636)(322,0.881918819188192)(323,0.883004926108374)(324,0.883405305367057)(325,0.883405305367057)(326,0.880688806888069)(327,0.878527607361963)(328,0.880688806888069)(329,0.872262773722628)(330,0.872262773722628)(331,0.863855421686747)(332,0.863335340156532)(333,0.860744297719088)(334,0.860227954409118)(335,0.860227954409118)(336,0.860744297719088)(337,0.861261261261261)(338,0.861261261261261)(339,0.868564506359782)(340,0.868564506359782)(341,0.850029638411381)(342,0.833720930232558)(343,0.839578454332553)(344,0.827466820542412)(345,0.826989619377163)(346,0.827466820542412)(347,0.827466820542412)(348,0.828901734104046)(349,0.832269297736506)(350,0.829861111111111)(351,0.829861111111111)(352,0.839578454332553)(353,0.841055718475073)(354,0.83908718548859)(355,0.83908718548859)(356,0.839578454332553)(357,0.842043452730476)(358,0.841549295774648)(359,0.841549295774648)(360,0.841549295774648)(361,0.841549295774648)(362,0.841549295774648)(363,0.843033509700176)(364,0.843033509700176)(365,0.847517730496454)(366,0.847517730496454)(367,0.847517730496454)(368,0.847517730496454)(369,0.847517730496454)(370,0.847017129356172)(371,0.847017129356172)(372,0.847017129356172)(373,0.847517730496454)(374,0.848520710059172)(375,0.848520710059172)(376,0.847517730496454)(377,0.847017129356172)(378,0.847517730496454)(379,0.847017129356172)(380,0.847517730496454)(381,0.845518867924528)(382,0.847017129356172)(383,0.847517730496454)(384,0.849023090586146)(385,0.850533807829181)(386,0.853571428571428)(387,0.854588796185936)(388,0.845020624631703)(389,0.847017129356172)(390,0.844025897586816)(391,0.850029638411381)(392,0.854588796185936)(393,0.851038575667656)(394,0.853063652587745)(395,0.853571428571428)(396,0.857142857142857)(397,0.85663082437276)(398,0.85663082437276)(399,0.85663082437276)(400,0.857655502392344)(8,0.553914327917282)(9,0.693026599568656)(10,0.708112206216831)(11,0.788022064617809)(12,0.788976377952756)(13,0.779874213836478)(14,0.789156626506024)(15,0.794048551292091)(16,0.793846153846154)(17,0.793261868300153)(18,0.798151001540832)(19,0.788154897494305)(20,0.785607196401799)(21,0.789156626506024)(22,0.777448071216617)(23,0.789156626506024)(24,0.801526717557252)(25,0.790085205267235)(26,0.795631825273011)(27,0.808346213292117)(28,0.813559322033898)(29,0.808049535603715)(30,0.804615384615384)(31,0.806826997672614)(32,0.81377151799687)(33,0.815396700706991)(34,0.816967792615868)(35,0.830841856805665)(36,0.831761006289308)(37,0.834384858044164)(38,0.835043409629045)(39,0.832672482157018)(40,0.836507936507936)(41,0.837172359015091)(42,0.839271575613618)(43,0.837430610626487)(44,0.838351822503962)(45,0.833995234312947)(46,0.837172359015091)(47,0.837837837837838)(48,0.83968253968254)(49,0.842438638163104)(50,0.840602696272799)(51,0.838760921366164)(52,0.838760921366164)(53,0.839427662957075)(54,0.841269841269841)(55,0.842188739095955)(56,0.841938046068308)(57,0.842607313195548)(58,0.844690966719493)(59,0.844444444444444)(60,0.845360824742268)(61,0.844444444444444)(62,0.84352660841938)(63,0.84352660841938)(64,0.84352660841938)(65,0.84352660841938)(66,0.84352660841938)(67,0.844690966719493)(68,0.845360824742268)(69,0.845360824742268)(70,0.845360824742268)(71,0.845360824742268)(72,0.84352660841938)(73,0.842857142857143)(74,0.842857142857143)(75,0.845849802371541)(76,0.845849802371541)(77,0.845849802371541)(78,0.846518987341772)(79,0.846518987341772)(80,0.845849802371541)(81,0.845181674565561)(82,0.845115170770453)(83,0.844197138314785)(84,0.844197138314785)(85,0.845115170770453)(86,0.847808764940239)(87,0.847808764940239)(88,0.847808764940239)(89,0.84688995215311)(90,0.847808764940239)(91,0.848726114649681)(92,0.847376788553259)(93,0.848726114649681)(94,0.847376788553259)(95,0.847808764940239)(96,0.847808764940239)(97,0.847808764940239)(98,0.847808764940239)(99,0.847133757961783)(100,0.847133757961783)(101,0.84688995215311)(102,0.844124700239808)(103,0.844124700239808)(104,0.844124700239808)(105,0.844124700239808)(106,0.843450479233227)(107,0.842525979216627)(108,0.842525979216627)(109,0.842525979216627)(110,0.842777334397446)(111,0.842105263157895)(112,0.842105263157895)(113,0.843277645186953)(114,0.843277645186953)(115,0.843700159489633)(116,0.843700159489633)(117,0.843700159489633)(118,0.844124700239808)(119,0.844124700239808)(120,0.843450479233227)(121,0.844621513944223)(122,0.844621513944223)(123,0.845295055821372)(124,0.845295055821372)(125,0.845786963434022)(126,0.845786963434022)(127,0.843027888446215)(128,0.843450479233227)(129,0.843700159489633)(130,0.843700159489633)(131,0.843700159489633)(132,0.8432)(133,0.845047923322684)(134,0.845047923322684)(135,0.845047923322684)(136,0.844124700239808)(137,0.842948717948718)(138,0.84664536741214)(139,0.84664536741214)(140,0.84664536741214)(141,0.84664536741214)(142,0.84664536741214)(143,0.84688995215311)(144,0.842948717948718)(145,0.839486356340289)(146,0.838969404186795)(147,0.838969404186795)(148,0.839645447219984)(149,0.836833602584814)(150,0.838709677419355)(151,0.837772397094431)(152,0.832116788321168)(153,0.828315703824247)(154,0.828315703824247)(155,0.828315703824247)(156,0.828315703824247)(157,0.826688364524003)(158,0.827642276422764)(159,0.831442463533225)(160,0.82640586797066)(161,0.82640586797066)(162,0.825732899022801)(163,0.82640586797066)(164,0.82640586797066)(165,0.831442463533225)(166,0.827361563517915)(167,0.821311475409836)(168,0.823240589198036)(169,0.821311475409836)(170,0.82420278004906)(171,0.82420278004906)(172,0.82420278004906)(173,0.82420278004906)(174,0.827079934747145)(175,0.833333333333333)(176,0.835218093699515)(177,0.836158192090396)(178,0.838033843674456)(179,0.837096774193548)(180,0.837096774193548)(181,0.837096774193548)(182,0.838033843674456)(183,0.838033843674456)(184,0.834276475343573)(185,0.834008097165992)(186,0.834951456310679)(187,0.829268292682927)(188,0.828315703824247)(189,0.828315703824247)(190,0.828315703824247)(191,0.828315703824247)(192,0.828315703824247)(193,0.832116788321168)(194,0.841767068273092)(195,0.843624699278268)(196,0.837096774193548)(197,0.839903459372486)(198,0.834276475343573)(199,0.834008097165992)(200,0.831168831168831)(201,0.830894308943089)(202,0.829943043124491)(203,0.829943043124491)(204,0.829943043124491)(205,0.829943043124491)(206,0.830219333874898)(207,0.834276475343573)(208,0.836422240128928)(209,0.836422240128928)(210,0.833602584814216)(211,0.841091492776886)(212,0.840160642570281)(213,0.83454398708636)(214,0.830219333874898)(215,0.837359098228663)(216,0.834276475343573)(217,0.826122448979592)(218,0.826122448979592)(219,0.83589329021827)(220,0.836569579288026)(221,0.834683954619125)(222,0.827079934747145)(223,0.828990228013029)(224,0.828990228013029)(225,0.829943043124491)(226,0.836569579288026)(227,0.836569579288026)(228,0.843624699278268)(229,0.844551282051282)(230,0.845476381104884)(231,0.845476381104884)(232,0.8464)(233,0.8464)(234,0.845476381104884)(235,0.845476381104884)(236,0.845476381104884)(237,0.845476381104884)(238,0.845476381104884)(239,0.845476381104884)(240,0.843624699278268)(241,0.842696629213483)(242,0.839903459372486)(243,0.839903459372486)(244,0.839903459372486)(245,0.840836012861736)(246,0.841767068273092)(247,0.837096774193548)(248,0.840836012861736)(249,0.840836012861736)(250,0.839903459372486)(251,0.837096774193548)(252,0.839645447219984)(253,0.841767068273092)(254,0.845476381104884)(255,0.844551282051282)(256,0.841767068273092)(257,0.841767068273092)(258,0.841257050765512)(259,0.841512469831054)(260,0.840579710144927)(261,0.842190016103059)(262,0.842190016103059)(263,0.842190016103059)(264,0.840322580645161)(265,0.838449111470113)(266,0.840322580645161)(267,0.845723421262989)(268,0.84664536741214)(269,0.84664536741214)(270,0.845723421262989)(271,0.845723421262989)(272,0.84664536741214)(273,0.847133757961783)(274,0.847133757961783)(275,0.84805091487669)(276,0.84805091487669)(277,0.848966613672496)(278,0.85193982581156)(279,0.857592446892211)(280,0.856918238993711)(281,0.856918238993711)(282,0.857592446892211)(283,0.856918238993711)(284,0.856918238993711)(285,0.856018882769473)(286,0.856018882769473)(287,0.856692913385827)(288,0.856018882769473)(289,0.856018882769473)(290,0.856018882769473)(291,0.856692913385827)(292,0.856692913385827)(293,0.856692913385827)(294,0.857368006304176)(295,0.857368006304176)(296,0.857368006304176)(297,0.856918238993711)(298,0.856245090337785)(299,0.855572998430141)(300,0.85423197492163)(301,0.85423197492163)(302,0.853563038371182)(303,0.850234009360374)(304,0.8515625)(305,0.853563038371182)(306,0.853563038371182)(307,0.853563038371182)(308,0.8515625)(309,0.8515625)(310,0.852895148669796)(311,0.852895148669796)(312,0.852895148669796)(313,0.853563038371182)(314,0.85423197492163)(315,0.852228303362001)(316,0.852228303362001)(317,0.852895148669796)(318,0.852895148669796)(319,0.852895148669796)(320,0.8515625)(321,0.8515625)(322,0.8515625)(323,0.850234009360374)(324,0.850234009360374)(325,0.852228303362001)(326,0.8515625)(327,0.850897736143638)(328,0.850897736143638)(329,0.85423197492163)(330,0.85423197492163)(331,0.853563038371182)(332,0.847589424572317)(333,0.85423197492163)(334,0.854901960784314)(335,0.852228303362001)(336,0.848249027237354)(337,0.850897736143638)(338,0.849571317225253)(339,0.85423197492163)(340,0.85423197492163)(341,0.853563038371182)(342,0.85423197492163)(343,0.85423197492163)(344,0.85423197492163)(345,0.853563038371182)(346,0.853563038371182)(347,0.853563038371182)(348,0.853563038371182)(349,0.853563038371182)(350,0.853563038371182)(351,0.856245090337785)(352,0.857142857142857)(353,0.857142857142857)(354,0.856240126382306)(355,0.856240126382306)(356,0.856240126382306)(357,0.855335968379446)(358,0.856240126382306)(359,0.855335968379446)(360,0.855335968379446)(361,0.855335968379446)(362,0.855335968379446)(363,0.856692913385827)(364,0.856466876971609)(365,0.856466876971609)(366,0.856466876971609)(367,0.856466876971609)(368,0.856466876971609)(369,0.856466876971609)(370,0.856466876971609)(371,0.856466876971609)(372,0.856466876971609)(373,0.856466876971609)(374,0.856466876971609)(375,0.856466876971609)(376,0.856466876971609)(377,0.857368006304176)(378,0.857368006304176)(379,0.857368006304176)(380,0.857368006304176)(381,0.857368006304176)(382,0.857368006304176)(383,0.857592446892211)(384,0.857368006304176)(385,0.857368006304176)(386,0.857368006304176)(387,0.856245090337785)(388,0.855572998430141)(389,0.855572998430141)(390,0.855572998430141)(391,0.855572998430141)(392,0.855572998430141)(393,0.854901960784314)(394,0.855572998430141)(395,0.854673998428908)(396,0.856692913385827)(397,0.856692913385827)(398,0.856692913385827)(399,0.856692913385827)(400,0.856692913385827)(1,0.524853801169591)(2,0.581721147431621)(3,0.584)(4,0.670995670995671)(5,0.67536231884058)(6,0.686140459912989)(7,0.698575498575498)(8,0.759314015375517)(9,0.760803408399269)(10,0.772755417956656)(11,0.784313725490196)(12,0.79625)(13,0.791082802547771)(14,0.797492163009404)(15,0.781424148606811)(16,0.7750449910018)(17,0.776442307692308)(18,0.778728606356968)(19,0.802288620470438)(20,0.800763844684914)(21,0.808263395739186)(22,0.820610687022901)(23,0.812895069532238)(24,0.81443951868271)(25,0.82051282051282)(26,0.819923371647509)(27,0.819462227912932)(28,0.835662009314704)(29,0.830608240680183)(30,0.838709677419355)(31,0.838709677419355)(32,0.865410497981157)(33,0.893081761006289)(34,0.892503536067892)(35,0.891872791519434)(36,0.891273247496423)(37,0.889526542324247)(38,0.900494001411433)(39,0.899433427762039)(40,0.899930020993702)(41,0.897222222222222)(42,0.894736842105263)(43,0.894736842105263)(44,0.891798759476223)(45,0.896599583622484)(46,0.896599583622484)(47,0.896599583622484)(48,0.900069881201957)(49,0.89950808151792)(50,0.897959183673469)(51,0.897959183673469)(52,0.898102600140548)(53,0.898102600140548)(54,0.894444444444444)(55,0.896265560165975)(56,0.896265560165975)(57,0.896265560165975)(58,0.896885813148789)(59,0.899375433726579)(60,0.895646164478231)(61,0.896265560165975)(62,0.895790200138026)(63,0.899721448467966)(64,0.899095337508699)(65,0.899095337508699)(66,0.894919972164231)(67,0.89367616400278)(68,0.895543175487465)(69,0.894297635605007)(70,0.894919972164231)(71,0.895688456189152)(72,0.898186889818689)(73,0.897364771151179)(74,0.899031811894882)(75,0.90125173852573)(76,0.903047091412742)(77,0.901174844505874)(78,0.901038062283737)(79,0.89910775566232)(80,0.89910775566232)(81,0.902825637491385)(82,0.902015288394718)(83,0.902015288394718)(84,0.902234636871508)(85,0.903899721448468)(86,0.904529616724738)(87,0.902015288394718)(88,0.900763358778626)(89,0.901388888888889)(90,0.901388888888889)(91,0.90187891440501)(92,0.90187891440501)(93,0.900976290097629)(94,0.904895104895105)(95,0.903496503496503)(96,0.903496503496503)(97,0.903765690376569)(98,0.902729181245626)(99,0.901823281907433)(100,0.901823281907433)(101,0.900280898876404)(102,0.901052631578947)(103,0.89950808151792)(104,0.90014064697609)(105,0.900634249471459)(106,0.902542372881356)(107,0.903043170559094)(108,0.903180212014134)(109,0.903180212014134)(110,0.901269393511989)(111,0.897471910112359)(112,0.897471910112359)(113,0.898102600140548)(114,0.898102600140548)(115,0.897471910112359)(116,0.898245614035088)(117,0.898245614035088)(118,0.898245614035088)(119,0.89978976874562)(120,0.898876404494382)(121,0.898245614035088)(122,0.898245614035088)(123,0.898245614035088)(124,0.900774102744546)(125,0.90014064697609)(126,0.902266288951841)(127,0.901988636363636)(128,0.903271692745377)(129,0.903777619387028)(130,0.904558404558405)(131,0.902212705210564)(132,0.902212705210564)(133,0.903640256959315)(134,0.902352102637206)(135,0.9)(136,0.899357601713062)(137,0.9)(138,0.902718168812589)(139,0.900643316654753)(140,0.900643316654753)(141,0.9)(142,0.899069434502505)(143,0.898426323319027)(144,0.89586305278174)(145,0.89586305278174)(146,0.896797153024911)(147,0.897289586305278)(148,0.896650035637919)(149,0.896650035637919)(150,0.896650035637919)(151,0.896354538956397)(152,0.896995708154506)(153,0.896848137535816)(154,0.896551724137931)(155,0.898134863701578)(156,0.90050107372942)(157,0.90050107372942)(158,0.900785153461813)(159,0.901651112706389)(160,0.901932712956335)(161,0.901287553648068)(162,0.901932712956335)(163,0.906204906204906)(164,0.904624277456647)(165,0.903043170559094)(166,0.904593639575972)(167,0.904829545454545)(168,0.904829545454545)(169,0.905338078291815)(170,0.904829545454545)(171,0.903496503496503)(172,0.903361344537815)(173,0.902912621359223)(174,0.902912621359223)(175,0.902912621359223)(176,0.902912621359223)(177,0.902912621359223)(178,0.902642559109875)(179,0.902642559109875)(180,0.902777777777778)(181,0.902912621359223)(182,0.902912621359223)(183,0.902912621359223)(184,0.902912621359223)(185,0.902912621359223)(186,0.904033379694019)(187,0.904166666666667)(188,0.904166666666667)(189,0.904299583911234)(190,0.905058905058905)(191,0.905190311418685)(192,0.904166666666667)(193,0.904166666666667)(194,0.904794996525365)(195,0.904794996525365)(196,0.903939184519696)(197,0.903939184519696)(198,0.902691511387163)(199,0.902691511387163)(200,0.900689655172414)(201,0.901311249137336)(202,0.901311249137336)(203,0.901933701657458)(204,0.900068917987595)(205,0.904299583911234)(206,0.904299583911234)(207,0.903047091412742)(208,0.90242214532872)(209,0.90179806362379)(210,0.903047091412742)(211,0.90179806362379)(212,0.903047091412742)(213,0.903047091412742)(214,0.90242214532872)(215,0.904033379694019)(216,0.904662491301322)(217,0.904996481351161)(218,0.904996481351161)(219,0.904996481351161)(220,0.905263157894737)(221,0.904628330995792)(222,0.905898876404494)(223,0.905898876404494)(224,0.905898876404494)(225,0.905898876404494)(226,0.904262753319357)(227,0.907042253521127)(228,0.907702984038862)(229,0.908460471567268)(230,0.908460471567268)(231,0.908333333333333)(232,0.908333333333333)(233,0.907317073170732)(234,0.908077994428969)(235,0.908077994428969)(236,0.909217877094972)(237,0.909344490934449)(238,0.909217877094972)(239,0.909217877094972)(240,0.908710801393728)(241,0.909217877094972)(242,0.909217877094972)(243,0.908583391486392)(244,0.908583391486392)(245,0.910104529616725)(246,0.909470752089136)(247,0.909470752089136)(248,0.909470752089136)(249,0.90883785664579)(250,0.90883785664579)(251,0.90883785664579)(252,0.910354412786657)(253,0.910354412786657)(254,0.909090909090909)(255,0.906574394463668)(256,0.908587257617728)(257,0.906574394463668)(258,0.906574394463668)(259,0.905321354526607)(260,0.905321354526607)(261,0.905321354526607)(262,0.905321354526607)(263,0.905321354526607)(264,0.904696132596685)(265,0.905321354526607)(266,0.906077348066298)(267,0.907330567081604)(268,0.906703524533517)(269,0.906703524533517)(270,0.905947441217151)(271,0.905947441217151)(272,0.906574394463668)(273,0.906444906444907)(274,0.906444906444907)(275,0.906444906444907)(276,0.906444906444907)(277,0.907073509015257)(278,0.903581267217631)(279,0.903581267217631)(280,0.902825637491385)(281,0.902825637491385)(282,0.901582931865107)(283,0.905555555555555)(284,0.905555555555555)(285,0.905555555555555)(286,0.905424200278164)(287,0.906184850590688)(288,0.902825637491385)(289,0.902825637491385)(290,0.904071773636991)(291,0.904071773636991)(292,0.904071773636991)(293,0.90771349862259)(294,0.90771349862259)(295,0.908465244322092)(296,0.90771349862259)(297,0.90771349862259)(298,0.90646492434663)(299,0.908339076498966)(300,0.90771349862259)(301,0.907088781830695)(302,0.907088781830695)(303,0.907088781830695)(304,0.907088781830695)(305,0.90633608815427)(306,0.906960716747071)(307,0.907830907830908)(308,0.908587257617728)(309,0.910104529616725)(310,0.910104529616725)(311,0.910104529616725)(312,0.910104529616725)(313,0.910104529616725)(314,0.910104529616725)(315,0.910104529616725)(316,0.910104529616725)(317,0.911126662001399)(318,0.911126662001399)(319,0.911126662001399)(320,0.911002102312544)(321,0.911250873515024)(322,0.910364145658263)(323,0.910364145658263)(324,0.910364145658263)(325,0.910364145658263)(326,0.91114245416079)(327,0.91114245416079)(328,0.91114245416079)(329,0.91114245416079)(330,0.91114245416079)(331,0.91114245416079)(332,0.912429378531073)(333,0.913841807909604)(334,0.914205344585091)(335,0.914968376669009)(336,0.914968376669009)(337,0.914205344585091)(338,0.915611814345991)(339,0.913441238564391)(340,0.914728682170542)(341,0.914728682170542)(342,0.915492957746479)(343,0.915492957746479)(344,0.915848527349229)(345,0.915848527349229)(346,0.916901408450704)(347,0.914245216158753)(348,0.914245216158753)(349,0.915014164305949)(350,0.915014164305949)(351,0.915782024062279)(352,0.915901060070671)(353,0.916373858046381)(354,0.916373858046381)(355,0.916373858046381)(356,0.915542938254081)(357,0.915542938254081)(358,0.915542938254081)(359,0.914893617021276)(360,0.915662650602409)(361,0.915662650602409)(362,0.914893617021276)(363,0.91643059490085)(364,0.918265813788202)(365,0.917663617171006)(366,0.917663617171006)(367,0.917663617171006)(368,0.917663617171006)(369,0.917018284106892)(370,0.917018284106892)(371,0.917018284106892)(372,0.917018284106892)(373,0.917018284106892)(374,0.917547568710359)(375,0.923188405797101)(376,0.923188405797101)(377,0.923188405797101)(378,0.923188405797101)(379,0.923188405797101)(380,0.920863309352518)(381,0.919458303635068)(382,0.918918918918919)(383,0.919914953933381)(384,0.918918918918919)(385,0.919687277896233)(386,0.919687277896233)(387,0.919687277896233)(388,0.919687277896233)(389,0.919687277896233)(390,0.919687277896233)(391,0.919034090909091)(392,0.91643059490085)(393,0.915901060070671)(394,0.915901060070671)(395,0.915901060070671)(396,0.917080085046067)(397,0.921975662133142)(398,0.920883820384889)(399,0.920883820384889)(400,0.920996441281139)(12,0.763005780346821)(13,0.782608695652174)(14,0.787794729542302)(15,0.788811188811189)(16,0.798289379900214)(17,0.785197103781174)(18,0.79428117553614)(19,0.778964667214462)(20,0.796969696969697)(21,0.798789712556732)(22,0.797337278106509)(23,0.813253012048193)(24,0.814145974416855)(25,0.824451410658307)(26,0.822134387351778)(27,0.82120253164557)(28,0.818037974683544)(29,0.820953870211102)(30,0.829343629343629)(31,0.83206106870229)(32,0.831804281345566)(33,0.844009042954032)(34,0.855430711610487)(35,0.854973424449506)(36,0.858654572940287)(37,0.859090909090909)(38,0.858886346300534)(39,0.859770114942529)(40,0.869300911854103)(41,0.870159453302961)(42,0.870820668693009)(43,0.867164179104477)(44,0.874349442379182)(45,0.875)(46,0.870991797166294)(47,0.870149253731343)(48,0.871449925261584)(49,0.883308714918759)(50,0.885439763488544)(51,0.883755588673621)(52,0.892561983471074)(53,0.890730972117558)(54,0.897744360902256)(55,0.897069872276484)(56,0.895723930982746)(57,0.892238131122833)(58,0.893905191873589)(59,0.89438202247191)(60,0.902133922001472)(61,0.903367496339678)(62,0.907501820830299)(63,0.911655530809206)(64,0.915680473372781)(65,0.915555555555555)(66,0.91475166790215)(67,0.914625092798812)(68,0.914625092798812)(69,0.91475166790215)(70,0.91475166790215)(71,0.913561847988077)(72,0.913173652694611)(73,0.915178571428571)(74,0.9150521609538)(75,0.917472118959108)(76,0.918959107806691)(77,0.918959107806691)(78,0.91778774289985)(79,0.919402985074627)(80,0.918717375093214)(81,0.917910447761194)(82,0.921130952380952)(83,0.920327624720774)(84,0.920327624720774)(85,0.919523099850969)(86,0.919523099850969)(87,0.914670658682635)(88,0.914670658682635)(89,0.914798206278027)(90,0.915608663181479)(91,0.9152288072018)(92,0.92020879940343)(93,0.919402985074627)(94,0.919402985074627)(95,0.920089619118745)(96,0.918474195961107)(97,0.917664670658682)(98,0.918229557389347)(99,0.918229557389347)(100,0.91497366440933)(101,0.91156462585034)(102,0.91904047976012)(103,0.91685393258427)(104,0.917102315160568)(105,0.918229557389347)(106,0.917417417417417)(107,0.918352059925094)(108,0.91578947368421)(109,0.917417417417417)(110,0.917417417417417)(111,0.921348314606741)(112,0.921465968586387)(113,0.922155688622754)(114,0.922038980509745)(115,0.922038980509745)(116,0.922038980509745)(117,0.92042042042042)(118,0.916542473919523)(119,0.916417910447761)(120,0.918154761904762)(121,0.91798344620015)(122,0.914027149321267)(123,0.914027149321267)(124,0.914156626506024)(125,0.91566265060241)(126,0.91566265060241)(127,0.910470409711684)(128,0.910470409711684)(129,0.910470409711684)(130,0.910470409711684)(131,0.910470409711684)(132,0.910470409711684)(133,0.907984790874525)(134,0.911296436694465)(135,0.910470409711684)(136,0.911296436694465)(137,0.912121212121212)(138,0.910470409711684)(139,0.911296436694465)(140,0.911296436694465)(141,0.911296436694465)(142,0.911296436694465)(143,0.913767019667171)(144,0.915407854984894)(145,0.913767019667171)(146,0.907984790874525)(147,0.907984790874525)(148,0.907984790874525)(149,0.907984790874525)(150,0.907984790874525)(151,0.907984790874525)(152,0.908814589665653)(153,0.908814589665653)(154,0.907153729071537)(155,0.907984790874525)(156,0.907984790874525)(157,0.912254160363086)(158,0.907153729071537)(159,0.907984790874525)(160,0.908814589665653)(161,0.908814589665653)(162,0.907984790874525)(163,0.907153729071537)(164,0.908814589665653)(165,0.908814589665653)(166,0.908814589665653)(167,0.908814589665653)(168,0.907984790874525)(169,0.907984790874525)(170,0.908814589665653)(171,0.907984790874525)(172,0.907153729071537)(173,0.906321401370906)(174,0.905487804878049)(175,0.908814589665653)(176,0.907153729071537)(177,0.910470409711684)(178,0.909643128321944)(179,0.906321401370906)(180,0.905487804878049)(181,0.905487804878049)(182,0.906321401370906)(183,0.908814589665653)(184,0.906321401370906)(185,0.907153729071537)(186,0.907153729071537)(187,0.907153729071537)(188,0.909643128321944)(189,0.910470409711684)(190,0.909643128321944)(191,0.909643128321944)(192,0.909643128321944)(193,0.908814589665653)(194,0.908814589665653)(195,0.907984790874525)(196,0.907984790874525)(197,0.907984790874525)(198,0.917043740573152)(199,0.912121212121212)(200,0.908814589665653)(201,0.909643128321944)(202,0.910470409711684)(203,0.911296436694465)(204,0.910470409711684)(205,0.909643128321944)(206,0.910470409711684)(207,0.905487804878049)(208,0.905487804878049)(209,0.905487804878049)(210,0.905487804878049)(211,0.905487804878049)(212,0.905487804878049)(213,0.905487804878049)(214,0.905487804878049)(215,0.90465293668955)(216,0.920993227990971)(217,0.918552036199095)(218,0.918552036199095)(219,0.921804511278195)(220,0.921804511278195)(221,0.921804511278195)(222,0.921804511278195)(223,0.921804511278195)(224,0.921804511278195)(225,0.921804511278195)(226,0.921804511278195)(227,0.920180722891566)(228,0.921804511278195)(229,0.921804511278195)(230,0.920180722891566)(231,0.920993227990971)(232,0.920180722891566)(233,0.917735849056604)(234,0.917735849056604)(235,0.91281273692191)(236,0.91281273692191)(237,0.911987860394537)(238,0.910334346504559)(239,0.910334346504559)(240,0.910334346504559)(241,0.910334346504559)(242,0.909505703422053)(243,0.909505703422053)(244,0.911161731207289)(245,0.910334346504559)(246,0.907844630616908)(247,0.910334346504559)(248,0.910334346504559)(249,0.908675799086758)(250,0.908675799086758)(251,0.908675799086758)(252,0.909505703422053)(253,0.909505703422053)(254,0.909505703422053)(255,0.910334346504559)(256,0.910334346504559)(257,0.910334346504559)(258,0.910334346504559)(259,0.926646706586826)(260,0.923423423423423)(261,0.931547619047619)(262,0.934814814814815)(263,0.93491124260355)(264,0.933135215453195)(265,0.927449513836948)(266,0.92503748125937)(267,0.927449513836948)(268,0.924231057764441)(269,0.923423423423423)(270,0.923423423423423)(271,0.923423423423423)(272,0.923423423423423)(273,0.923423423423423)(274,0.922614575507137)(275,0.922614575507137)(276,0.922614575507137)(277,0.922614575507137)(278,0.924231057764441)(279,0.924231057764441)(280,0.924231057764441)(281,0.924231057764441)(282,0.924231057764441)(283,0.924231057764441)(284,0.924231057764441)(285,0.924231057764441)(286,0.924231057764441)(287,0.926646706586826)(288,0.930648769574944)(289,0.930648769574944)(290,0.930648769574944)(291,0.931547619047619)(292,0.931445603576751)(293,0.929850746268657)(294,0.930648769574944)(295,0.927449513836948)(296,0.92503748125937)(297,0.924231057764441)(298,0.925842696629213)(299,0.924231057764441)(300,0.924231057764441)(301,0.926646706586826)(302,0.930648769574944)(303,0.931445603576751)(304,0.928251121076233)(305,0.933135215453195)(306,0.929051530993279)(307,0.929051530993279)(308,0.929051530993279)(309,0.927449513836948)(310,0.927449513836948)(311,0.927449513836948)(312,0.927449513836948)(313,0.931445603576751)(314,0.931445603576751)(315,0.933828996282528)(316,0.933828996282528)(317,0.933035714285714)(318,0.933035714285714)(319,0.933035714285714)(320,0.937777777777778)(321,0.939349112426035)(322,0.939349112426035)(323,0.938564026646928)(324,0.938564026646928)(325,0.940133037694013)(326,0.940133037694013)(327,0.940915805022156)(328,0.942477876106195)(329,0.940915805022156)(330,0.937777777777778)(331,0.94169741697417)(332,0.943257184966838)(333,0.943257184966838)(334,0.944035346097202)(335,0.943257184966838)(336,0.943257184966838)(337,0.943257184966838)(338,0.943257184966838)(339,0.944812362030905)(340,0.943257184966838)(341,0.942477876106195)(342,0.94169741697417)(343,0.945588235294118)(344,0.947136563876652)(345,0.948680351906158)(346,0.947136563876652)(347,0.947136563876652)(348,0.945588235294118)(349,0.943257184966838)(350,0.944035346097202)(351,0.944035346097202)(352,0.944035346097202)(353,0.945588235294118)(354,0.946362968405584)(355,0.945588235294118)(356,0.945588235294118)(357,0.944812362030905)(358,0.943257184966838)(359,0.945588235294118)(360,0.947136563876652)(361,0.945588235294118)(362,0.945588235294118)(363,0.945588235294118)(364,0.946362968405584)(365,0.944812362030905)(366,0.944812362030905)(367,0.944812362030905)(368,0.944812362030905)(369,0.944812362030905)(370,0.945588235294118)(371,0.944812362030905)(372,0.947136563876652)(373,0.947136563876652)(374,0.947136563876652)(375,0.947136563876652)(376,0.946362968405584)(377,0.942477876106195)(378,0.942477876106195)(379,0.940915805022156)(380,0.937777777777778)(381,0.936990363232024)(382,0.940133037694013)(383,0.940133037694013)(384,0.939349112426035)(385,0.940133037694013)(386,0.935412026726058)(387,0.935412026726058)(388,0.932241250930752)(389,0.929850746268657)(390,0.931445603576751)(391,0.933828996282528)(392,0.933828996282528)(393,0.937777777777778)(394,0.934621099554235)(395,0.933828996282528)(396,0.931445603576751)(397,0.931445603576751)(398,0.931445603576751)(399,0.931445603576751)(400,0.931445603576751)(3,0.587002096436059)(4,0.635383639822448)(5,0.695890410958904)(6,0.718706047819972)(7,0.717149220489978)(8,0.723172628304821)(9,0.717829457364341)(10,0.722306525037936)(11,0.742205323193916)(12,0.784660766961652)(13,0.789045151739452)(14,0.777858703568827)(15,0.789356984478936)(16,0.78877400295421)(17,0.789356984478936)(18,0.797337278106509)(19,0.797337278106509)(20,0.796747967479675)(21,0.798816568047337)(22,0.801781737193764)(23,0.798226164079823)(24,0.797636632200886)(25,0.801186943620178)(26,0.802973977695167)(27,0.805970149253731)(28,0.805369127516778)(29,0.79940784603997)(30,0.800593031875463)(31,0.802973977695167)(32,0.803571428571428)(33,0.803571428571428)(34,0.805022156573117)(35,0.804428044280443)(36,0.806213017751479)(37,0.80920564216778)(38,0.810408921933085)(39,0.807407407407407)(40,0.806213017751479)(41,0.80680977054034)(42,0.806213017751479)(43,0.80680977054034)(44,0.808605341246291)(45,0.80680977054034)(46,0.800293685756241)(47,0.806213017751479)(48,0.805617147080562)(49,0.805617147080562)(50,0.805022156573117)(51,0.80680977054034)(52,0.807407407407407)(53,0.805617147080562)(54,0.806213017751479)(55,0.800293685756241)(56,0.796201607012418)(57,0.780243378668575)(58,0.784737221022318)(59,0.778571428571429)(60,0.787003610108303)(61,0.785302593659942)(62,0.786435786435786)(63,0.788141720896601)(64,0.786435786435786)(65,0.788141720896601)(66,0.785868781542898)(67,0.79042784626541)(68,0.778571428571429)(69,0.781362007168459)(70,0.782483847810481)(71,0.78757225433526)(72,0.786435786435786)(73,0.785868781542898)(74,0.784737221022318)(75,0.783608914450036)(76,0.783608914450036)(77,0.79042784626541)(78,0.789855072463768)(79,0.78757225433526)(80,0.785868781542898)(81,0.785868781542898)(82,0.785868781542898)(83,0.789283128167994)(84,0.791575889615105)(85,0.792151162790698)(86,0.79388201019665)(87,0.794460641399417)(88,0.794460641399417)(89,0.796201607012418)(90,0.796201607012418)(91,0.796201607012418)(92,0.796783625730994)(93,0.797366495976591)(94,0.796783625730994)(95,0.796783625730994)(96,0.797366495976591)(97,0.797366495976591)(98,0.797366495976591)(99,0.796783625730994)(100,0.796201607012418)(101,0.797950219619326)(102,0.798534798534798)(103,0.797366495976591)(104,0.797366495976591)(105,0.798534798534798)(106,0.799706529713866)(107,0.799706529713866)(108,0.799706529713866)(109,0.799706529713866)(110,0.799120234604106)(111,0.796783625730994)(112,0.796783625730994)(113,0.796783625730994)(114,0.796201607012418)(115,0.796201607012418)(116,0.795620437956204)(117,0.799120234604106)(118,0.798534798534798)(119,0.798534798534798)(120,0.801470588235294)(121,0.803242446573323)(122,0.800881704628949)(123,0.803834808259587)(124,0.806213017751479)(125,0.806213017751479)(126,0.80680977054034)(127,0.805617147080562)(128,0.805022156573117)(129,0.805022156573117)(130,0.804428044280443)(131,0.804428044280443)(132,0.803834808259587)(133,0.804428044280443)(134,0.805022156573117)(135,0.804428044280443)(136,0.804428044280443)(137,0.806213017751479)(138,0.806213017751479)(139,0.806213017751479)(140,0.806213017751479)(141,0.80680977054034)(142,0.805617147080562)(143,0.805617147080562)(144,0.803834808259587)(145,0.804428044280443)(146,0.804428044280443)(147,0.804428044280443)(148,0.805022156573117)(149,0.803834808259587)(150,0.803834808259587)(151,0.80680977054034)(152,0.805022156573117)(153,0.80680977054034)(154,0.80680977054034)(155,0.80680977054034)(156,0.80680977054034)(157,0.80680977054034)(158,0.80680977054034)(159,0.80680977054034)(160,0.803834808259587)(161,0.804428044280443)(162,0.804428044280443)(163,0.800881704628949)(164,0.801470588235294)(165,0.80206033848418)(166,0.80206033848418)(167,0.802650957290132)(168,0.802650957290132)(169,0.802650957290132)(170,0.799706529713866)(171,0.799706529713866)(172,0.798534798534798)(173,0.798534798534798)(174,0.799706529713866)(175,0.800293685756241)(176,0.800293685756241)(177,0.799706529713866)(178,0.799706529713866)(179,0.800293685756241)(180,0.801470588235294)(181,0.80206033848418)(182,0.80206033848418)(183,0.801470588235294)(184,0.801470588235294)(185,0.802650957290132)(186,0.802650957290132)(187,0.803834808259587)(188,0.804428044280443)(189,0.803834808259587)(190,0.805617147080562)(191,0.805022156573117)(192,0.804428044280443)(193,0.804428044280443)(194,0.805617147080562)(195,0.80680977054034)(196,0.80680977054034)(197,0.806213017751479)(198,0.80680977054034)(199,0.807407407407407)(200,0.808005930318755)(201,0.808005930318755)(202,0.808605341246291)(203,0.808605341246291)(204,0.808605341246291)(205,0.808605341246291)(206,0.808005930318755)(207,0.797366495976591)(208,0.797366495976591)(209,0.803834808259587)(210,0.796201607012418)(211,0.805022156573117)(212,0.803834808259587)(213,0.803834808259587)(214,0.803242446573323)(215,0.804428044280443)(216,0.814040328603435)(217,0.814040328603435)(218,0.814040328603435)(219,0.81525804038893)(220,0.81525804038893)(221,0.81525804038893)(222,0.814040328603435)(223,0.814040328603435)(224,0.814040328603435)(225,0.814040328603435)(226,0.814648729446936)(227,0.814648729446936)(228,0.814648729446936)(229,0.81282624906786)(230,0.81282624906786)(231,0.81282624906786)(232,0.81282624906786)(233,0.811615785554728)(234,0.811615785554728)(235,0.811615785554728)(236,0.811615785554728)(237,0.810408921933085)(238,0.811011904761905)(239,0.811011904761905)(240,0.811011904761905)(241,0.811011904761905)(242,0.811011904761905)(243,0.811615785554728)(244,0.81525804038893)(245,0.81525804038893)(246,0.81525804038893)(247,0.814648729446936)(248,0.814648729446936)(249,0.814648729446936)(250,0.814040328603435)(251,0.81282624906786)(252,0.812220566318927)(253,0.81282624906786)(254,0.81282624906786)(255,0.81282624906786)(256,0.819047619047619)(257,0.818978102189781)(258,0.818978102189781)(259,0.826398852223816)(260,0.829650748396294)(261,0.83048433048433)(262,0.840663302090844)(263,0.83982683982684)(264,0.842867487328023)(265,0.846872753414809)(266,0.842776203966006)(267,0.842553191489362)(268,0.849785407725322)(269,0.84479092841956)(270,0.845609065155807)(271,0.848614072494669)(272,0.844192634560906)(273,0.847409510290986)(274,0.848787446504993)(275,0.85243553008596)(276,0.851216022889843)(277,0.848787446504993)(278,0.848787446504993)(279,0.849393290506781)(280,0.846975088967971)(281,0.847578347578347)(282,0.846975088967971)(283,0.846208362863218)(284,0.845609065155807)(285,0.841437632135306)(286,0.840253342716397)(287,0.838483146067416)(288,0.837307152875175)(289,0.83554933519944)(290,0.834965034965035)(291,0.83554933519944)(292,0.837894736842105)(293,0.83554933519944)(294,0.83554933519944)(295,0.838483146067416)(296,0.837307152875175)(297,0.836720392431675)(298,0.837894736842105)(299,0.837894736842105)(300,0.834965034965035)(301,0.834965034965035)(302,0.83205574912892)(303,0.83205574912892)(304,0.833798882681564)(305,0.834381551362683)(306,0.833798882681564)(307,0.834965034965035)(308,0.836720392431675)(309,0.836720392431675)(310,0.837307152875175)(311,0.837307152875175)(312,0.838483146067416)(313,0.837307152875175)(314,0.837307152875175)(315,0.837307152875175)(316,0.837307152875175)(317,0.837307152875175)(318,0.836134453781513)(319,0.836134453781513)(320,0.836134453781513)(321,0.83554933519944)(322,0.834965034965035)(323,0.834965034965035)(324,0.833798882681564)(325,0.833798882681564)(326,0.833798882681564)(327,0.834965034965035)(328,0.834965034965035)(329,0.834965034965035)(330,0.834965034965035)(331,0.834965034965035)(332,0.832635983263598)(333,0.833217027215632)(334,0.833217027215632)(335,0.833217027215632)(336,0.833798882681564)(337,0.833798882681564)(338,0.83554933519944)(339,0.83554933519944)(340,0.834965034965035)(341,0.834381551362683)(342,0.834381551362683)(343,0.834381551362683)(344,0.83554933519944)(345,0.83554933519944)(346,0.83554933519944)(347,0.836720392431675)(348,0.836720392431675)(349,0.836720392431675)(350,0.837307152875175)(351,0.837307152875175)(352,0.837307152875175)(353,0.83554933519944)(354,0.83554933519944)(355,0.83554933519944)(356,0.834965034965035)(357,0.83554933519944)(358,0.834381551362683)(359,0.834381551362683)(360,0.834381551362683)(361,0.834381551362683)(362,0.833798882681564)(363,0.834381551362683)(364,0.833798882681564)(365,0.833798882681564)(366,0.833217027215632)(367,0.831476323119777)(368,0.83205574912892)(369,0.831476323119777)(370,0.831476323119777)(371,0.83205574912892)(372,0.83205574912892)(373,0.83205574912892)(374,0.831476323119777)(375,0.831476323119777)(376,0.832635983263598)(377,0.83298392732355)(378,0.833217027215632)(379,0.833217027215632)(380,0.832635983263598)(381,0.83182135380321)(382,0.831241283124128)(383,0.831241283124128)(384,0.831241283124128)(385,0.830662020905923)(386,0.83008356545961)(387,0.829268292682927)(388,0.828690807799443)(389,0.865333333333333)(390,0.863874345549738)(391,0.86105675146771)(392,0.872286079182631)(393,0.884302689180738)(394,0.887367766023646)(395,0.885856079404466)(396,0.888612321095208)(397,0.889305816135084)(398,0.890977443609023)(399,0.89375)(400,0.89711417816813) 
};

\end{axis}
\end{tikzpicture}%

%% This file was created by matlab2tikz v0.2.3.
% Copyright (c) 2008--2012, Nico Schlömer <nico.schloemer@gmail.com>
% All rights reserved.
% 
% 
% 
\begin{tikzpicture}

\begin{axis}[%
tick label style={font=\tiny},
label style={font=\tiny},
label shift={-4pt},
xlabel shift={-6pt},
legend style={font=\tiny},
view={0}{90},
width=\figurewidth,
height=\figureheight,
scale only axis,
xmin=0, xmax=400,
xlabel={Samples},
ymin=0.5, ymax=1,
ylabel={$F_1$-score},
axis lines*=left,
legend cell align=left,
legend style={at={(1.03,0)},anchor=south east,fill=none,draw=none,align=left,row sep=-0.2em},
clip=false]

\addplot [
color=blue,
mark size=0.1pt,
only marks,
mark=*,
mark options={solid,fill=blue},
forget plot
]
coordinates{
 (1,0.526455026455026)(1,0.666666666666667)(1,0.500738552437223)(1,0.544726301735648)(7,0.517696044413602)(9,0.531114327062229)(11,0.583908045977011)(12,0.554929577464789)(13,0.594292803970223)(13,0.631229235880398)(13,0.576315789473684)(14,0.531458179126573)(14,0.527758738862234)(14,0.640748740100792)(14,0.624019312009656)(14,0.520292747837658)(14,0.670769230769231)(14,0.595103578154425)(15,0.608323133414933)(15,0.588313413014608)(15,0.614047791455467)(15,0.622503653190453)(15,0.558533145275035)(15,0.548137737174982)(16,0.630887185104052)(16,0.511502029769959)(16,0.625719769673704)(16,0.598455598455598)(16,0.568644818423383)(17,0.650406504065041)(17,0.503909026297086)(17,0.603386351975372)(17,0.646279306829765)(17,0.55531453362256)(17,0.521845651286239)(17,0.572244897959184)(18,0.69134328358209)(18,0.546213476446034)(18,0.625258799171843)(18,0.693472584856397)(18,0.565137614678899)(18,0.693284936479129)(18,0.623529411764706)(18,0.656670113753878)(18,0.662365591397849)(18,0.534219788815018)(18,0.697829716193656)(18,0.695593220338983)(18,0.592427616926503)(18,0.622988505747126)(18,0.672862453531598)(18,0.609287925696594)(18,0.645346062052506)(18,0.694025974025974)(19,0.541185527328714)(19,0.603295310519645)(19,0.696365767878077)(19,0.50635593220339)(19,0.619241982507288)(19,0.706937799043062)(19,0.680161943319838)(19,0.709711846318036)(19,0.597679814385151)(19,0.688995215311005)(19,0.646052631578947)(19,0.584349593495935)(19,0.62639405204461)(19,0.660779985283296)(19,0.530612244897959)(19,0.564866014461931)(19,0.685492801771871)(19,0.698049764626765)(19,0.588777219430486)(19,0.650764525993884)(19,0.637896156439649)(19,0.622528735632184)(19,0.649282920469361)(19,0.689845474613686)(19,0.586687306501548)(19,0.593589743589743)(19,0.560785918952108)(20,0.725891291642314)(20,0.593250444049733)(20,0.701630049610205)(20,0.553667953667954)(20,0.739960500329164)(20,0.691358024691358)(20,0.551418439716312)(20,0.568075117370892)(20,0.664953751284686)(20,0.645092838196286)(20,0.736466389054134)(20,0.619612068965517)(20,0.588382783117426)(20,0.549635875814488)(20,0.521075581395349)(20,0.681936491410723)(20,0.567080045095829)(20,0.580071174377224)(20,0.587464154035231)(20,0.589417989417989)(20,0.589514066496164)(20,0.633116883116883)(20,0.665650406504065)(20,0.563611491108071)(20,0.637920101458465)(20,0.522635259477365)(21,0.531740104555638)(21,0.701522170747849)(21,0.755939524838013)(21,0.559937524404529)(21,0.639915522703273)(21,0.646498599439776)(21,0.598179453836151)(21,0.696594427244582)(21,0.664406779661017)(21,0.627533193570929)(21,0.654250238777459)(21,0.623827392120075)(21,0.574315789473684)(21,0.54604462474645)(21,0.503071919045898)(21,0.557684786859601)(21,0.623441396508728)(21,0.645306859205776)(21,0.614699331848552)(21,0.698224852071006)(21,0.547537227949599)(21,0.513978494623656)(21,0.696058784235137)(22,0.718877849210988)(22,0.705196182396607)(22,0.513754358775668)(22,0.734870317002882)(22,0.668390433096315)(22,0.613751730502999)(22,0.598945855294681)(22,0.686945500633713)(22,0.632145258910558)(22,0.617381489841986)(22,0.654028436018957)(22,0.614834673815907)(22,0.51975353388909)(22,0.68359375)(22,0.721878862793572)(22,0.707438016528925)(22,0.558761435608726)(22,0.730878186968838)(22,0.599489795918367)(22,0.706593406593406)(22,0.626865671641791)(22,0.738019169329073)(22,0.671651090342679)(22,0.589821661591996)(22,0.682179341657208)(22,0.704576976421637)(23,0.551963048498845)(23,0.65024154589372)(23,0.715142428785607)(23,0.708674304418985)(23,0.677783227072094)(23,0.548374760994264)(23,0.690867838910947)(23,0.737852429514097)(23,0.645397489539749)(23,0.680697534576067)(23,0.590909090909091)(23,0.710765239948119)(23,0.553571428571428)(23,0.737960339943343)(23,0.511047754811119)(23,0.710703363914373)(23,0.653211009174312)(23,0.618241398143091)(23,0.70730304772858)(23,0.606811145510836)(23,0.674858223062382)(23,0.615996025832091)(23,0.528177704648293)(23,0.677377892030848)(23,0.630730478589421)(24,0.731790916880891)(24,0.731104651162791)(24,0.790438247011952)(24,0.741912293314162)(24,0.512508934953538)(24,0.652882205513784)(24,0.535225048923679)(24,0.57062825130052)(24,0.738872403560831)(24,0.749514563106796)(24,0.622770919067215)(24,0.745718050065876)(24,0.659353348729792)(24,0.583862194016319)(24,0.534003091190108)(24,0.585554600171969)(24,0.733142037302726)(24,0.504929577464789)(24,0.544401544401544)(24,0.693658536585366)(24,0.656626506024096)(24,0.674102812803104)(24,0.631062670299727)(24,0.665992925720061)(24,0.582852431986809)(24,0.593297476210178)(25,0.621814475025484)(25,0.762843068261787)(25,0.535616971584274)(25,0.639494026704146)(25,0.659609120521172)(25,0.768939393939394)(25,0.716599190283401)(25,0.704669496664645)(25,0.69047619047619)(25,0.740827023878858)(25,0.542565266742338)(25,0.619178082191781)(25,0.666032350142721)(25,0.624085811799122)(25,0.613321799307958)(25,0.691638795986622)(25,0.755641521598968)(25,0.723842195540309)(25,0.547537227949599)(25,0.567695961995249)(25,0.709227467811159)(25,0.606566499721758)(25,0.764525993883792)(25,0.747142857142857)(25,0.702453987730061)(25,0.721238938053097)(25,0.56288032454361)(25,0.705755782678859)(25,0.738998482549317)(25,0.670448684889118)(25,0.760266370699223)(25,0.697234352256186)(25,0.746666666666667)(25,0.620358235037134)(25,0.561912225705329)(26,0.728581713462923)(26,0.761570827489481)(26,0.682352941176471)(26,0.576335877862595)(26,0.742690058479532)(26,0.59870250231696)(26,0.600671140939597)(26,0.651654411764706)(26,0.70679012345679)(26,0.611905860636825)(26,0.759403832505323)(26,0.728421052631579)(26,0.733823529411765)(26,0.576954732510288)(26,0.734615384615385)(26,0.782553729456384)(26,0.669826224328594)(26,0.771215207060421)(26,0.744089012517385)(26,0.719754977029096)(26,0.705019740552735)(26,0.742813918305598)(26,0.736318407960199)(26,0.537481259370315)(26,0.797527047913447)(26,0.74914089347079)(26,0.737842370039128)(26,0.789873417721519)(26,0.708010335917313)(26,0.701879126460132)(26,0.597054886211513)(26,0.703135423615744)(26,0.731879787860931)(26,0.649490069779925)(27,0.721079691516709)(27,0.709180098107919)(27,0.640862322390985)(27,0.75669244497323)(27,0.636155606407323)(27,0.743742550655542)(27,0.771535580524344)(27,0.692360633172746)(27,0.615831517792302)(27,0.778907242693774)(27,0.705413105413105)(27,0.666299559471366)(27,0.764099454214676)(27,0.76043673731535)(27,0.638792102206736)(27,0.555813953488372)(27,0.728444802578565)(27,0.724962630792227)(27,0.762490392006149)(27,0.695652173913043)(27,0.749820788530466)(27,0.781954887218045)(27,0.761904761904762)(27,0.698854337152209)(27,0.587945879458795)(27,0.712994350282486)(27,0.703150912106136)(27,0.741140215716487)(27,0.763392857142857)(27,0.742574257425742)(27,0.782956058588548)(27,0.730793254216115)(27,0.640176600441501)(27,0.70204081632653)(27,0.659517426273458)(27,0.760339943342776)(27,0.691775557263643)(27,0.70812324929972)(27,0.758819294456443)(27,0.701692936368943)(27,0.703910614525139)(27,0.68144690781797)(27,0.739484396200814)(27,0.735670562047857)(27,0.696600384862091)(28,0.732540861812778)(28,0.705761316872428)(28,0.753488372093023)(28,0.73094867807154)(28,0.614193548387097)(28,0.65597667638484)(28,0.737975592246949)(28,0.650045330915684)(28,0.695780903034789)(28,0.73065735892961)(28,0.635467980295566)(28,0.71609006040637)(28,0.790508474576271)(28,0.610354223433243)(28,0.750192159877017)(28,0.716392020815264)(28,0.71608832807571)(28,0.666046511627907)(28,0.778513612950699)(28,0.564789287120914)(28,0.716627634660421)(28,0.797368421052632)(28,0.702064896755162)(28,0.630697674418605)(28,0.779031187790312)(28,0.779512804497189)(28,0.780811808118081)(28,0.63901549680948)(28,0.741176470588235)(28,0.726964386659129)(28,0.733379026730637)(28,0.678)(28,0.671595810227973)(28,0.752011704462326)(28,0.748070175438596)(28,0.591828312009905)(28,0.733766233766234)(28,0.676970633693972)(28,0.758147512864494)(28,0.753246753246753)(28,0.659698025551684)(28,0.56094674556213)(28,0.746537396121884)(28,0.704195432819968)(29,0.749271137026239)(29,0.733512786002692)(29,0.705506783719074)(29,0.721739130434782)(29,0.728588661037394)(29,0.754668383773342)(29,0.778412698412698)(29,0.772413793103448)(29,0.726121372031662)(29,0.73541842772612)(29,0.754926108374384)(29,0.665004985044865)(29,0.694037145650049)(29,0.673499267935578)(29,0.761087267525036)(29,0.511412268188302)(29,0.700625558534405)(29,0.716323296354992)(29,0.682036503362152)(29,0.701201201201201)(29,0.76402767102229)(29,0.726483357452967)(29,0.700067704807041)(29,0.788918205804749)(29,0.756972111553785)(29,0.737819025522042)(29,0.713881019830028)(29,0.619469026548673)(29,0.538692712246431)(29,0.622395833333333)(29,0.792822185970636)(29,0.529151291512915)(29,0.800918836140888)(29,0.740506329113924)(29,0.69910897875257)(29,0.744752877454299)(29,0.711078928312817)(29,0.808960796515246)(29,0.704533611255862)(29,0.693610049153468)(29,0.759592795614722)(29,0.80564263322884)(29,0.717948717948718)(29,0.728483606557377)(29,0.686759956942949)(30,0.770759403832505)(30,0.796420581655481)(30,0.617571059431524)(30,0.662109375)(30,0.783141762452107)(30,0.693247588424437)(30,0.74370709382151)(30,0.782884310618067)(30,0.741754822650902)(30,0.70964304741609)(30,0.687263556116015)(30,0.704142011834319)(30,0.693049213597159)(30,0.750136091453457)(30,0.773148148148148)(30,0.748557295960429)(30,0.676906779661017)(30,0.726591760299625)(30,0.736787564766839)(30,0.707738542449286)(30,0.725637181409295)(30,0.763195435092725)(30,0.762942779291553)(30,0.701492537313433)(30,0.754203362690152)(30,0.757910228108904)(30,0.581119738455251)(30,0.822716807367613)(30,0.708560118753092)(30,0.72497897392767)(30,0.723300970873786)(30,0.714727085478888)(30,0.709090909090909)(30,0.73972602739726)(30,0.712661106899166)(30,0.68698347107438)(30,0.71436086270384)(30,0.773060029282577)(30,0.780735107731305)(30,0.764845605700713)(30,0.703853955375254)(30,0.757234726688103)(30,0.743384615384615)(30,0.672361195851129)(30,0.736761320030698)(30,0.734111543450065)(30,0.735242548217417)(30,0.739562624254473)(30,0.705760249091853)(30,0.612936344969199)(30,0.743348982785602)(31,0.616352201257862)(31,0.738406658739596)(31,0.716707021791767)(31,0.779636363636364)(31,0.782329317269076)(31,0.667748917748918)(31,0.645219755323969)(31,0.760869565217391)(31,0.753945061367621)(31,0.735618115055079)(31,0.709459459459459)(31,0.713826366559486)(31,0.741304347826087)(31,0.71841704718417)(31,0.543181818181818)(31,0.537504711647192)(31,0.797180892717306)(31,0.700934579439252)(31,0.640987654320987)(31,0.729225551158847)(31,0.777421423989737)(31,0.762057877813505)(31,0.757985257985258)(31,0.719346049046321)(31,0.730482009504413)(31,0.72137657180675)(31,0.777245508982036)(31,0.670317634173056)(31,0.798171129980405)(31,0.74402332361516)(31,0.708711918760021)(31,0.756234915526951)(32,0.785016286644951)(32,0.760545905707196)(32,0.720848056537102)(32,0.765793528505393)(32,0.627571115973742)(32,0.631718061674009)(32,0.73900293255132)(32,0.801659751037344)(32,0.780525502318392)(32,0.746344564526383)(32,0.766787003610108)(32,0.770958083832335)(32,0.681818181818182)(32,0.726558113419427)(32,0.735476593344614)(32,0.747266881028939)(32,0.638565022421525)(32,0.710008554319932)(32,0.631067961165048)(32,0.655705996131528)(32,0.764572293716881)(32,0.688877498718606)(32,0.735395189003436)(32,0.809585492227979)(32,0.755399883245768)(32,0.726333907056798)(32,0.759800427655025)(32,0.634794156706507)(32,0.734082397003745)(32,0.737254901960784)(32,0.765749235474006)(32,0.713890426120642)(32,0.771794871794872)(32,0.661173048957828)(32,0.701554404145078)(32,0.60378947368421)(32,0.795713328868051)(32,0.732394366197183)(32,0.723577235772358)(32,0.772009029345372)(32,0.744063324538259)(32,0.721364744110479)(32,0.77663772691397)(32,0.752182672934855)(32,0.702010968921389)(32,0.7296)(32,0.742553191489362)(32,0.73974025974026)(33,0.772478498827209)(33,0.8)(33,0.723502304147465)(33,0.764959901295497)(33,0.685920577617329)(33,0.702675416456335)(33,0.754266211604095)(33,0.725467289719626)(33,0.773085182534001)(33,0.778362133734035)(33,0.701754385964912)(33,0.75594294770206)(33,0.713362068965517)(33,0.758276077451593)(33,0.718294051627385)(33,0.67345948568656)(33,0.7850622406639)(33,0.754748142031379)(33,0.756838905775076)(33,0.622936576889661)(33,0.80867544539117)(33,0.711445459430414)(33,0.768493150684931)(33,0.796934865900383)(33,0.549425287356322)(33,0.725167580743449)(33,0.729411764705882)(33,0.762928139691068)(33,0.700791556728232)(33,0.783191230207064)(33,0.744736842105263)(33,0.772752652149637)(33,0.729151291512915)(33,0.732984293193717)(33,0.831306990881459)(33,0.754887218045113)(33,0.795163584637269)(33,0.75798592311857)(33,0.765404183154324)(33,0.735945485519591)(33,0.698770491803279)(33,0.651075771749298)(33,0.658469945355191)(33,0.775030902348578)(33,0.730337078651685)(33,0.648851383028598)(33,0.74827369742624)(33,0.75107296137339)(33,0.694984646878199)(33,0.785149117468046)(33,0.72618254497002)(33,0.781920903954802)(33,0.716794731064764)(33,0.75619295958279)(33,0.766519823788546)(33,0.774679728711379)(33,0.788098693759071)(33,0.751932536893886)(33,0.759635599159075)(33,0.676190476190476)(33,0.752542372881356)(33,0.740506329113924)(33,0.651663405088062)(33,0.712676056338028)(34,0.709210526315789)(34,0.76632801161103)(34,0.781395348837209)(34,0.828680203045685)(34,0.742696629213483)(34,0.734780307040762)(34,0.791601866251944)(34,0.776649746192893)(34,0.646808510638298)(34,0.767428236672525)(34,0.803312629399586)(34,0.721044045676998)(34,0.744920993227991)(34,0.725053078556263)(34,0.773119605425401)(34,0.720770288858322)(34,0.788690476190476)(34,0.815415821501014)(34,0.836734693877551)(34,0.742331288343558)(34,0.780687397708674)(34,0.763346613545817)(34,0.758055067369654)(34,0.741500962155228)(34,0.820552147239264)(34,0.697297297297297)(34,0.836417468541821)(34,0.691056910569106)(34,0.708288482238966)(34,0.750272628135223)(34,0.784879725085911)(34,0.763557483731019)(34,0.730061349693251)(34,0.754433833560709)(34,0.757537688442211)(34,0.732533521524347)(34,0.70582226762002)(34,0.75103489059728)(34,0.827633378932969)(34,0.731543624161074)(34,0.605128205128205)(34,0.723618090452261)(34,0.718435091879075)(34,0.741339491916859)(34,0.795392953929539)(34,0.771902131018153)(34,0.708171206225681)(34,0.730121880441091)(34,0.814919071076706)(34,0.761523046092184)(34,0.757731958762886)(34,0.795786061588331)(35,0.793650793650794)(35,0.765957446808511)(35,0.728835978835979)(35,0.759358288770053)(35,0.746416758544653)(35,0.659354052407069)(35,0.776536312849162)(35,0.738813735691987)(35,0.665014866204162)(35,0.751807228915663)(35,0.746376811594203)(35,0.797460317460317)(35,0.725901992460958)(35,0.749680715197956)(35,0.740893470790378)(35,0.823312883435583)(35,0.782456140350877)(35,0.787830264211369)(35,0.774736842105263)(35,0.745205479452055)(35,0.768565248738284)(35,0.786717752234994)(35,0.577759871071716)(35,0.755360087960418)(35,0.755939524838013)(35,0.709278350515464)(35,0.725542041248017)(35,0.625140291806959)(35,0.733652312599681)(35,0.75462392108508)(35,0.765294771968854)(35,0.78266253869969)(35,0.648206990467544)(35,0.680034873583261)(35,0.754901960784314)(35,0.69364161849711)(35,0.738983050847458)(35,0.70874861572536)(35,0.714139344262295)(35,0.746335963923337)(35,0.757033248081841)(35,0.760147601476015)(35,0.614856161442679)(35,0.740447957839262)(35,0.752110514198005)(35,0.739965095986038)(35,0.683946488294314)(35,0.711437565582371)(35,0.767392832044975)(35,0.734463276836158)(36,0.834795321637427)(36,0.76056338028169)(36,0.765598650927487)(36,0.755290287574606)(36,0.754477180820335)(36,0.750700280112045)(36,0.779634049323787)(36,0.781550288276746)(36,0.761627906976744)(36,0.77326968973747)(36,0.793242156074014)(36,0.797089733225546)(36,0.787838730998017)(36,0.800618238021638)(36,0.776315789473684)(36,0.646125962845492)(36,0.775451263537906)(36,0.788405797101449)(36,0.74457429048414)(36,0.759479956663055)(36,0.80227416298168)(36,0.778404512489927)(36,0.792423046566693)(36,0.759349593495935)(36,0.750290360046458)(36,0.78328173374613)(36,0.790030211480362)(36,0.763921568627451)(36,0.748587570621469)(36,0.76420233463035)(36,0.792225201072386)(36,0.803738317757009)(36,0.799424874191229)(36,0.691431401684002)(36,0.588137702198258)(36,0.731216111541441)(36,0.773846153846154)(36,0.730868443680137)(36,0.712413261372398)(36,0.722911497105045)(37,0.804973821989529)(37,0.779933110367893)(37,0.76436303080766)(37,0.745438748913988)(37,0.790884718498659)(37,0.79080118694362)(37,0.509413854351687)(37,0.770161290322581)(37,0.739884393063584)(37,0.778365667254556)(37,0.774844720496894)(37,0.799254195152268)(37,0.762629757785467)(37,0.789625360230548)(37,0.788515406162465)(37,0.75956678700361)(37,0.71970802919708)(37,0.780487804878049)(37,0.777636594663278)(37,0.720265780730897)(37,0.795724465558195)(37,0.686741136474016)(37,0.782608695652174)(37,0.808378588052754)(37,0.773940345368917)(37,0.766642282370153)(37,0.776758409785933)(37,0.751669702489375)(37,0.782667569397427)(37,0.755744680851064)(37,0.814186584425597)(37,0.77445432497979)(37,0.759715380405036)(38,0.704545454545454)(38,0.762430939226519)(38,0.74221961244862)(38,0.732365145228216)(38,0.800847457627119)(38,0.704180064308682)(38,0.775645268034414)(38,0.749126484975541)(38,0.790986790986791)(38,0.789444132509826)(38,0.772226926333615)(38,0.829117828500925)(38,0.742252456538171)(38,0.846524432209222)(38,0.833333333333333)(38,0.797858099062918)(38,0.775646371976647)(38,0.716657739689341)(38,0.814526588845655)(38,0.731985484707102)(38,0.807174887892377)(38,0.798860398860399)(38,0.770238095238095)(38,0.805442176870748)(38,0.794171220400729)(38,0.76417419884963)(38,0.768091168091168)(38,0.746602717825739)(38,0.748789671866595)(38,0.802139037433155)(38,0.763005780346821)(38,0.703465982028241)(38,0.755580357142857)(38,0.71591526344378)(38,0.730812013348164)(38,0.762223710649699)(38,0.784388995521433)(38,0.828976848394324)(38,0.787172011661807)(38,0.731255265374894)(38,0.734550561797753)(38,0.710137544574631)(38,0.768242848803269)(38,0.765776699029126)(39,0.77888198757764)(39,0.839400428265524)(39,0.780048076923077)(39,0.762898550724638)(39,0.783927217589083)(39,0.776572668112798)(39,0.818537859007833)(39,0.729935559461043)(39,0.752333882482152)(39,0.811459027315123)(39,0.726796014682748)(39,0.744696415508412)(39,0.760969976905312)(39,0.7456)(39,0.801509433962264)(39,0.773333333333333)(39,0.798325191905094)(39,0.771672771672771)(39,0.811851851851852)(39,0.79136690647482)(39,0.710718002081165)(39,0.829733163913596)(39,0.80565371024735)(39,0.764455264759586)(39,0.618698441796517)(39,0.807843137254902)(39,0.781852082038533)(39,0.813084112149533)(39,0.795234549516009)(39,0.796190476190476)(39,0.755952380952381)(40,0.833333333333333)(40,0.78117998506348)(40,0.754473872584109)(40,0.825129533678756)(40,0.777960526315789)(40,0.790354989953114)(40,0.769130998702983)(40,0.798711755233494)(40,0.786476868327402)(40,0.759601706970128)(40,0.778045838359469)(40,0.764247150569886)(40,0.746987951807229)(40,0.717896865520728)(40,0.741468459152016)(40,0.751247327156094)(40,0.813370473537604)(40,0.808853118712274)(40,0.797573919636088)(40,0.837606837606838)(40,0.790262172284644)(40,0.768428890543559)(40,0.831920903954802)(40,0.791989664082687)(40,0.747747747747748)(40,0.762446657183499)(40,0.813888888888889)(40,0.818435754189944)(40,0.787057938299473)(40,0.759100642398287)(40,0.779393939393939)(40,0.799768518518518)(40,0.741433021806853)(40,0.788876276958002)(40,0.807833537331701)(40,0.764982373678026)(40,0.780516431924883)(40,0.731900452488688)(40,0.831428571428571)(40,0.786577181208054)(40,0.756291390728477)(40,0.785507246376811)(40,0.825327510917031)(41,0.78521351179095)(41,0.771890686001115)(41,0.77085020242915)(41,0.796787148594377)(41,0.775300171526586)(41,0.768392370572207)(41,0.806161745827985)(41,0.762852404643449)(41,0.778378378378378)(41,0.751740139211137)(41,0.775351014040561)(41,0.767171129220023)(41,0.806613946800863)(41,0.777860882572924)(41,0.783546864463924)(41,0.774193548387097)(41,0.785973397823458)(41,0.800776196636481)(41,0.71733561058924)(41,0.794906166219839)(41,0.769811320754717)(41,0.799685781618225)(41,0.827357237715803)(41,0.752027809965237)(41,0.797633136094674)(41,0.815899581589958)(41,0.756788665879575)(41,0.708920187793427)(41,0.772104607721046)(41,0.786816269284712)(41,0.72707182320442)(41,0.844505874222529)(41,0.797183098591549)(41,0.729312762973352)(42,0.80783722682743)(42,0.803174603174603)(42,0.802881466928618)(42,0.805877803557618)(42,0.807151979565772)(42,0.795256916996047)(42,0.786764705882353)(42,0.810166799046863)(42,0.795741849634065)(42,0.755168661588683)(42,0.804564907275321)(42,0.797791580400276)(42,0.78735275883447)(42,0.780776826859776)(42,0.738630136986301)(42,0.783850931677019)(42,0.770691994572592)(42,0.832831325301205)(42,0.778581765557164)(42,0.767741935483871)(42,0.814432989690721)(42,0.745)(42,0.767270288397049)(42,0.804140127388535)(42,0.762472885032538)(42,0.799261083743842)(42,0.779755283648498)(42,0.790372670807453)(42,0.8003300330033)(42,0.837874659400545)(43,0.781832927818329)(43,0.788075178224238)(43,0.797202797202797)(43,0.788115715402658)(43,0.803159973666886)(43,0.750266808964781)(43,0.831430490261921)(43,0.778666666666667)(43,0.813614262560778)(43,0.821952776005105)(43,0.8)(43,0.803955788248982)(43,0.816265060240964)(43,0.8126834997064)(43,0.788370520622042)(43,0.813471502590673)(43,0.785545023696682)(43,0.801677651288196)(43,0.773259820813232)(43,0.760987357013847)(43,0.750805585392052)(43,0.769653655854865)(43,0.781973203410475)(43,0.780871670702179)(43,0.796296296296296)(43,0.823604060913705)(43,0.787371134020618)(43,0.808654496281271)(43,0.793704328274311)(43,0.814249363867684)(43,0.785327720986169)(43,0.757660167130919)(44,0.795212765957447)(44,0.809041309431021)(44,0.804647785039942)(44,0.786244099797707)(44,0.80639213275968)(44,0.786967418546366)(44,0.843863471314452)(44,0.78778386844166)(44,0.70690537084399)(44,0.849162011173184)(44,0.835365853658537)(44,0.800810263335584)(44,0.789366053169734)(44,0.821736249171637)(44,0.788849347568209)(44,0.761421319796954)(44,0.781987133666905)(44,0.835462058602554)(44,0.792168674698795)(44,0.819407008086253)(44,0.719028340080971)(44,0.780637254901961)(44,0.790349417637271)(45,0.819444444444444)(45,0.798901853122855)(45,0.850574712643678)(45,0.800650935720097)(45,0.809097174362509)(45,0.779766536964981)(45,0.839262187088274)(45,0.775649794801642)(45,0.797437950360288)(45,0.744235924932976)(45,0.762632197414806)(45,0.820795107033639)(45,0.817760106030484)(45,0.8)(45,0.736497545008183)(45,0.816301703163017)(45,0.802267895109851)(45,0.824352694191742)(45,0.850590687977762)(45,0.787530762920426)(45,0.778551532033426)(45,0.735551663747811)(45,0.758327427356485)(45,0.776978417266187)(45,0.798336798336798)(45,0.869372693726937)(45,0.767895878524946)(45,0.805210183540557)(45,0.740823136818687)(45,0.779888268156424)(45,0.789473684210526)(46,0.794692737430168)(46,0.827913279132791)(46,0.853493613824192)(46,0.788461538461538)(46,0.811940298507463)(46,0.848056537102473)(46,0.804819277108434)(46,0.78290025146689)(46,0.82256020278834)(46,0.807778608825729)(46,0.766573816155989)(46,0.772005772005772)(46,0.808957312806158)(46,0.789356984478936)(46,0.780346820809248)(46,0.831186440677966)(46,0.818246614397719)(46,0.840993788819876)(46,0.786384976525822)(46,0.768243785084202)(46,0.779874213836478)(46,0.796971782518926)(46,0.798842257597684)(46,0.804012345679012)(46,0.798387096774194)(46,0.845016077170418)(46,0.822857142857143)(46,0.78866587957497)(46,0.836560805577072)(47,0.826330532212885)(47,0.870498084291187)(47,0.831050228310502)(47,0.790754257907543)(47,0.788225674570728)(47,0.743871513102282)(47,0.811688311688312)(47,0.807137954701441)(47,0.783114992721979)(47,0.747390396659708)(47,0.805466237942122)(47,0.787247087676272)(47,0.772451790633609)(47,0.826558265582656)(47,0.804863221884498)(47,0.810315186246418)(47,0.793180133432172)(47,0.804332129963899)(47,0.789928057553957)(47,0.806523350630096)(47,0.812893081761006)(47,0.784225352112676)(47,0.78005540166205)(47,0.849682427664079)(47,0.820343461030383)(47,0.781766381766382)(47,0.79646017699115)(47,0.848373235113567)(47,0.758620689655172)(47,0.760904684975767)(48,0.813660874775314)(48,0.778693722257451)(48,0.832550860719875)(48,0.827027027027027)(48,0.809917355371901)(48,0.82018927444795)(48,0.81353591160221)(48,0.796551724137931)(48,0.826743350107836)(48,0.820391227030231)(48,0.809001731102135)(48,0.889206576125804)(48,0.815987933634992)(48,0.782861292665214)(48,0.814229249011858)(48,0.827260458839406)(48,0.858029480217223)(48,0.801204819277108)(48,0.782172701949861)(48,0.82172373081464)(48,0.836486486486486)(48,0.808244846970643)(48,0.836633663366336)(49,0.818696883852691)(49,0.796925048046124)(49,0.80559646539028)(49,0.817528735632184)(49,0.829333333333333)(49,0.807339449541284)(49,0.723535457348407)(49,0.788339670468948)(49,0.83371298405467)(49,0.833590138674884)(49,0.785842831433713)(49,0.839802399435427)(49,0.854802680565897)(49,0.74850809889173)(49,0.821558630735615)(49,0.791270285394516)(49,0.829268292682927)(49,0.834261838440111)(49,0.795902285263987)(49,0.844500632111251)(49,0.832995267072346)(50,0.785953177257525)(50,0.781104457751164)(50,0.840425531914893)(50,0.819875776397516)(50,0.859226841721371)(50,0.802163833075734)(50,0.832177531206657)(50,0.8286951144094)(50,0.772331154684096)(50,0.752779248279513)(50,0.802435723951285)(50,0.829787234042553)(50,0.810344827586207)(50,0.834302325581395)(50,0.836896762370189)(50,0.820582261340555)(50,0.857142857142857)(50,0.854489164086687)(50,0.792870313460357)(50,0.855233853006681)(50,0.804270462633452)(51,0.790106007067138)(51,0.820645161290323)(51,0.830808080808081)(51,0.806924101198402)(51,0.846441947565543)(51,0.814710042432815)(51,0.845801526717557)(51,0.831168831168831)(51,0.831808585503167)(51,0.803846153846154)(51,0.807352941176471)(51,0.853046594982079)(51,0.78871473354232)(51,0.796039603960396)(51,0.811428571428571)(51,0.830967741935484)(51,0.819341840161182)(51,0.835266821345708)(52,0.765760555234239)(52,0.814414414414414)(52,0.802564102564102)(52,0.805)(52,0.844859813084112)(52,0.848062015503876)(52,0.787066246056782)(52,0.816262705238467)(52,0.822070675759454)(52,0.834473324213406)(52,0.813704496788008)(52,0.826684107259647)(52,0.805949008498584)(52,0.815286624203821)(53,0.831948881789137)(53,0.81447963800905)(53,0.812631578947368)(53,0.821492537313433)(53,0.806078147612156)(53,0.845969672785315)(53,0.819542947202522)(53,0.804655029093932)(53,0.818250950570342)(53,0.803760282021151)(53,0.842320819112628)(53,0.868537666174298)(53,0.778816199376947)(53,0.831872509960159)(54,0.844838505383154)(54,0.845477386934673)(54,0.825)(54,0.850447966919366)(54,0.829416224412434)(54,0.850798056904927)(54,0.823448275862069)(54,0.819505094614265)(54,0.82093991671624)(54,0.84631288766368)(54,0.786752827140549)(54,0.789562289562289)(54,0.821630347054076)(54,0.784182305630027)(54,0.832958801498127)(54,0.858400586940572)(54,0.857142857142857)(54,0.825194621372965)(54,0.799148332150461)(54,0.844507845934379)(55,0.836417468541821)(55,0.796048808832074)(55,0.86459802538787)(55,0.824484697064335)(55,0.794366197183098)(55,0.83617963314358)(55,0.807843137254902)(55,0.855735397607319)(55,0.851536952256377)(55,0.791985857395404)(55,0.846321922796795)(55,0.809002433090024)(55,0.794366197183099)(56,0.801204819277108)(56,0.837237977805179)(56,0.813483146067416)(56,0.871993793638479)(56,0.796982167352538)(56,0.847290640394089)(56,0.865979381443299)(56,0.790007806401249)(56,0.789692435577722)(56,0.818518518518519)(56,0.845637583892617)(56,0.783906554185594)(56,0.809349890430971)(56,0.819571865443425)(56,0.885857860732232)(56,0.824572514249525)(56,0.841121495327103)(56,0.840848806366048)(57,0.842105263157895)(57,0.847058823529412)(57,0.845272206303725)(57,0.806192660550459)(57,0.742105263157895)(57,0.858151023288638)(57,0.737275064267352)(57,0.812176165803109)(58,0.819559228650138)(58,0.821052631578947)(58,0.866415094339622)(58,0.804110468850353)(58,0.861428571428571)(58,0.808153477218225)(58,0.853808353808354)(58,0.842729970326409)(58,0.884038199181446)(58,0.816326530612245)(58,0.836173001310616)(58,0.75801382778127)(58,0.846006389776358)(58,0.840182648401826)(59,0.856182795698925)(59,0.858490566037736)(59,0.80785246876859)(59,0.833889259506337)(59,0.797797010228167)(59,0.850664581704456)(59,0.798013245033113)(59,0.86604774535809)(59,0.863636363636364)(59,0.774703557312253)(59,0.823035392921416)(59,0.81419624217119)(60,0.86993006993007)(60,0.870056497175141)(60,0.82549504950495)(60,0.853110968569596)(60,0.869498069498069)(60,0.868778280542986)(60,0.853519340519975)(60,0.788898233809924)(60,0.83047619047619)(60,0.837310195227766)(60,0.819883795997418)(60,0.84514435695538)(60,0.798325191905094)(60,0.832932692307692)(60,0.83386378103119)(60,0.844563042028019)(60,0.834136269786648)(61,0.809487375669472)(61,0.803231390652048)(61,0.832678711704635)(61,0.849523809523809)(61,0.865217391304348)(61,0.875089221984297)(61,0.723063223508459)(61,0.837912087912088)(61,0.850220264317181)(61,0.873578595317726)(61,0.827439886845827)(61,0.820152314001172)(62,0.856400259909032)(62,0.836335160532498)(62,0.84778156996587)(62,0.837966101694915)(62,0.847142857142857)(62,0.8032)(62,0.843205574912892)(62,0.776045939294504)(62,0.861517976031957)(62,0.745833333333333)(62,0.85968514715948)(62,0.805855161787365)(63,0.861918604651163)(63,0.834224598930481)(63,0.831944444444444)(63,0.806163828061638)(63,0.88317107093185)(63,0.833841463414634)(63,0.858867924528302)(64,0.822733423545331)(64,0.823452768729642)(64,0.856240126382306)(64,0.842028985507246)(64,0.87832973362131)(64,0.791765637371338)(64,0.832274459974587)(64,0.874711760184473)(64,0.82183908045977)(64,0.868123587038432)(64,0.859492803289925)(64,0.89432063263839)(65,0.858475894245723)(65,0.802172226532195)(65,0.880239520958084)(65,0.774030354131534)(65,0.853252647503782)(65,0.856619890176937)(65,0.837703756201276)(65,0.870944484498918)(65,0.848062015503876)(66,0.861675126903553)(66,0.883081155433287)(66,0.867811799850635)(66,0.837041491280818)(66,0.859431900946832)(66,0.850202429149797)(66,0.879662209711471)(67,0.83939774153074)(67,0.799512492382693)(67,0.858151023288638)(67,0.839786381842456)(67,0.818923327895595)(67,0.833870967741935)(67,0.864306784660767)(67,0.754716981132075)(67,0.874424720578567)(67,0.855394032134659)(67,0.879884225759768)(67,0.862121212121212)(67,0.802431610942249)(68,0.791223404255319)(68,0.883928571428571)(68,0.82640586797066)(68,0.842767295597484)(68,0.859934853420195)(68,0.868715083798883)(68,0.826747720364742)(69,0.864604810996564)(69,0.845295055821372)(69,0.861493836113125)(69,0.849569251159708)(69,0.883556810162315)(69,0.874172185430464)(69,0.875507442489851)(69,0.866952789699571)(69,0.841692789968652)(69,0.818313953488372)(69,0.868035190615836)(69,0.809468822170901)(69,0.885007278020378)(69,0.876560332871012)(69,0.821614583333333)(70,0.850092535471931)(70,0.847784200385356)(70,0.791643139469226)(70,0.830164765525982)(70,0.880636604774536)(70,0.863836017569546)(70,0.873280231716148)(70,0.844411979547115)(71,0.81447963800905)(71,0.82922954725973)(71,0.876651982378855)(71,0.855875831485588)(71,0.846768336964415)(71,0.859342197340798)(71,0.829900839054157)(71,0.870838881491345)(72,0.820603907637655)(72,0.878629932985852)(72,0.860046911649726)(72,0.845973416731822)(72,0.871085214857975)(72,0.879947229551451)(73,0.822641509433962)(73,0.848084544253633)(73,0.871994801819363)(73,0.873027798647633)(73,0.861006761833208)(73,0.886348352387357)(73,0.865017667844523)(73,0.852919438285292)(73,0.86643598615917)(73,0.855345911949685)(74,0.887966804979253)(74,0.846946867565424)(74,0.874432677760968)(74,0.839385474860335)(74,0.874906367041198)(74,0.815936254980079)(74,0.873661670235546)(74,0.87250172294969)(75,0.837423312883435)(75,0.874718679669917)(75,0.83690036900369)(76,0.795786061588331)(76,0.853146853146853)(76,0.881330309901738)(76,0.893928310168252)(76,0.84375)(76,0.787375415282392)(76,0.877168632893824)(76,0.869257950530035)(76,0.829957028852056)(76,0.896969696969697)(76,0.84472049689441)(76,0.871313672922252)(77,0.839779005524862)(77,0.846649484536082)(77,0.883174136664217)(77,0.88728323699422)(77,0.789005658852061)(77,0.882978723404255)(77,0.852963818321786)(78,0.828035859820701)(78,0.884753042233357)(78,0.874371859296482)(78,0.86449864498645)(78,0.878183069511356)(78,0.819524727039178)(79,0.870218579234972)(79,0.863843648208469)(79,0.887864823348694)(79,0.843283582089552)(79,0.849802371541502)(79,0.870105655686762)(79,0.855875831485588)(79,0.791979949874687)(79,0.884180790960452)(79,0.876373626373626)(80,0.86084142394822)(80,0.879120879120879)(80,0.898843930635838)(80,0.856018882769473)(80,0.855784469096672)(80,0.82598235765838)(80,0.894345238095238)(80,0.868713259307642)(80,0.899478778853313)(80,0.853840924541128)(80,0.827091633466135)(81,0.893886156008433)(81,0.855590062111801)(81,0.860613810741688)(81,0.850033624747814)(81,0.847168347556245)(82,0.877862595419847)(82,0.884902840059791)(82,0.868613138686131)(82,0.894814814814815)(82,0.859581070597362)(82,0.857573474001507)(82,0.869109947643979)(82,0.776550552251487)(82,0.875)(82,0.860014357501795)(82,0.886486486486486)(83,0.864446165762975)(83,0.866176470588235)(83,0.890225563909774)(84,0.844979919678715)(84,0.871473354231975)(84,0.863606121091151)(84,0.901615271659324)(84,0.887272727272727)(84,0.889196675900277)(84,0.878980891719745)(84,0.864285714285714)(85,0.848012470771629)(85,0.860479409956976)(85,0.845049130763416)(85,0.866197183098591)(85,0.854932301740812)(85,0.829975825946817)(85,0.807142857142857)(86,0.882232811436351)(86,0.816326530612245)(86,0.903994393833216)(86,0.830788804071247)(86,0.869565217391304)(86,0.88681757656458)(87,0.881019830028329)(87,0.878254750175932)(87,0.853523357086302)(87,0.868658790826963)(87,0.872964169381107)(88,0.806451612903226)(88,0.854570637119114)(88,0.878200155159038)(89,0.860062893081761)(89,0.882895670688431)(89,0.845132743362832)(89,0.888738127544098)(89,0.865914786967419)(89,0.882758620689655)(89,0.870903010033445)(89,0.888248017303533)(90,0.894736842105263)(90,0.872592592592592)(90,0.895500725689405)(90,0.860681114551084)(90,0.866407263294423)(91,0.848645076007931)(91,0.880587058038692)(91,0.823104693140794)(91,0.864385297845374)(91,0.820430965682362)(91,0.849872773536896)(91,0.876488095238095)(91,0.82843137254902)(91,0.87762490948588)(92,0.900680272108843)(92,0.873614190687361)(92,0.848806366047745)(92,0.904794058068872)(92,0.9002284843869)(93,0.881644223954642)(93,0.878936319104269)(94,0.855670103092783)(94,0.906841339155749)(94,0.86774628879892)(94,0.88339670468948)(94,0.903581267217631)(94,0.878556557945871)(94,0.879573170731707)(95,0.869135802469136)(95,0.868571428571429)(96,0.859554873369148)(96,0.873595505617978)(96,0.887632508833922)(96,0.84074373484236)(96,0.890532544378698)(96,0.864828513786147)(96,0.86578449905482)(97,0.880804953560371)(98,0.889397406559878)(98,0.885139519792343)(98,0.938775510204082)(98,0.900398406374502)(99,0.900808229243203)(99,0.906474820143885)(99,0.904486251808972)(99,0.886191198786039)(99,0.870473537604457)(99,0.911634756995582)(99,0.840425531914893)(99,0.843505477308294)(99,0.869025450031037)(99,0.901311249137336)(99,0.915523465703971)(100,0.8763197586727)(100,0.882534775888717)(100,0.865317515701326)(100,0.871661237785016)(100,0.821696480092325)(101,0.880758807588076)(101,0.869038607115821)(101,0.863988724453841)(101,0.935664335664336)(101,0.886408404464872)(101,0.811930405965203)(101,0.943475226796929)(101,0.875)(102,0.832672482157018)(102,0.923389830508474)(102,0.892978868438991)(102,0.907324364723468)(102,0.895903290799194)(103,0.879785090664876)(103,0.860282574568289)(103,0.888594164456233)(103,0.865361077111383)(103,0.924263674614306)(103,0.877566539923954)(103,0.842105263157895)(104,0.879194630872483)(104,0.886346300533943)(104,0.893499308437068)(104,0.905491698595147)(104,0.89650249821556)(104,0.91683038637852)(105,0.894308943089431)(105,0.880918220946915)(105,0.896276595744681)(105,0.889210716871832)(106,0.886864085041761)(106,0.892057026476578)(106,0.894451962110961)(106,0.891390728476821)(106,0.868789808917197)(106,0.856687898089172)(106,0.906555090655509)(107,0.842419716206124)(107,0.886908841672378)(107,0.90190336749634)(107,0.913653136531365)(108,0.900302114803625)(108,0.896694214876033)(108,0.841019417475728)(109,0.84430176565008)(109,0.934285714285714)(109,0.905740609496811)(109,0.860394537177542)(109,0.832748538011696)(109,0.906824146981627)(109,0.894660894660895)(110,0.907366885485047)(110,0.878627968337731)(110,0.869699306090979)(110,0.913852515506547)(111,0.816831683168317)(111,0.79327731092437)(111,0.906646751306945)(111,0.93006993006993)(112,0.856950067476383)(112,0.90383322125084)(114,0.891461649782923)(114,0.903405142460042)(114,0.902189781021898)(114,0.895209580838323)(115,0.884748102139406)(115,0.88671875)(115,0.881136950904393)(115,0.936986301369863)(115,0.903962390866353)(115,0.881422924901186)(116,0.883036405886909)(116,0.906876790830945)(116,0.909752547307132)(116,0.922852983988355)(117,0.8848660391021)(117,0.896503496503496)(117,0.845518867924528)(117,0.907749077490775)(118,0.908850726552179)(118,0.927007299270073)(118,0.889514426460239)(118,0.907738095238095)(119,0.893243243243243)(119,0.866566716641679)(119,0.894230769230769)(119,0.887724550898204)(120,0.884785819793205)(120,0.888726207906296)(120,0.906703524533517)(120,0.862284820031299)(120,0.902203856749311)(120,0.909090909090909)(121,0.851460534493474)(121,0.9221140472879)(121,0.88631090487239)(121,0.863287250384025)(122,0.89039039039039)(122,0.917134831460674)(122,0.903100775193798)(123,0.891207153502235)(123,0.913173652694611)(123,0.901751713632902)(123,0.853503184713376)(123,0.899850523168909)(124,0.889043963712491)(124,0.910979228486647)(125,0.867619639527657)(125,0.911208151382824)(125,0.903988183161004)(127,0.911516853932584)(127,0.887001959503592)(127,0.884444444444444)(127,0.890501319261213)(127,0.893100833965125)(127,0.888725128960943)(127,0.880211780277962)(128,0.934182590233546)(128,0.888275862068965)(128,0.932506887052341)(128,0.876146788990826)(128,0.883233532934132)(128,0.87708066581306)(128,0.935437589670014)(129,0.880530973451327)(129,0.86833855799373)(129,0.896032831737346)(129,0.915608663181479)(130,0.944793850454228)(130,0.84052757793765)(130,0.905982905982906)(130,0.892351274787535)(131,0.881401617250674)(131,0.883969465648855)(131,0.907338769458858)(132,0.891478514202476)(132,0.892617449664429)(132,0.890566037735849)(133,0.90646492434663)(133,0.882043576258452)(133,0.868544600938967)(133,0.917018284106892)(133,0.889985895627644)(133,0.9)(133,0.876447876447876)(134,0.845659163987138)(134,0.903272727272727)(134,0.917567567567567)(134,0.918687589158345)(134,0.877247849882721)(134,0.901849217638691)(134,0.894977168949772)(135,0.911917098445596)(135,0.883783783783784)(135,0.896969696969697)(135,0.909752547307132)(135,0.909830508474576)(135,0.892882818116463)(135,0.863064396743153)(135,0.879288437102922)(135,0.883183568677792)(136,0.900600400266844)(136,0.940416367552046)(136,0.902571041948579)(137,0.924581005586592)(138,0.907608695652174)(138,0.857369255150555)(138,0.906528189910979)(138,0.930679478380233)(139,0.892612338156893)(139,0.894582723279648)(140,0.91441111923921)(140,0.906411201179071)(140,0.920923722883135)(141,0.894361633182112)(141,0.899012908124525)(141,0.89859363434493)(141,0.91475166790215)(142,0.896206156048676)(143,0.913286713286713)(143,0.910948905109489)(144,0.901845018450184)(144,0.923747276688453)(144,0.895035460992908)(145,0.896290688872067)(145,0.892284186401833)(145,0.905263157894737)(145,0.914877868245744)(145,0.926585887384177)(145,0.897239263803681)(145,0.919893190921228)(146,0.931982633863965)(146,0.895052473763118)(146,0.913649025069638)(147,0.892388451443569)(148,0.901675163874727)(148,0.897069872276484)(148,0.922972972972973)(148,0.923734853884533)(148,0.878693623639191)(148,0.910798122065728)(149,0.923184357541899)(149,0.8528)(149,0.905424200278164)(149,0.910593538692712)(149,0.928823114869626)(149,0.917391304347826)(150,0.890165111270639)(150,0.924418604651163)(151,0.918478260869565)(151,0.897506925207756)(151,0.911894273127753)(152,0.9)(152,0.902548725637181)(152,0.876071706936867)(152,0.918732782369146)(152,0.937226277372263)(153,0.903485254691689)(153,0.923905723905724)(153,0.909361069836553)(153,0.922330097087379)(154,0.865073245952197)(154,0.931286549707602)(154,0.863024544734759)(155,0.896706586826347)(155,0.920162381596752)(155,0.880493446414803)(156,0.869085173501577)(156,0.900445765230312)(156,0.933779264214047)(156,0.908263836239575)(157,0.90400604686319)(157,0.889320388349514)(157,0.889196675900277)(157,0.929824561403509)(157,0.915347556779078)(157,0.92867332382311)(158,0.912790697674418)(158,0.913907284768212)(158,0.885998469778118)(159,0.894419306184012)(159,0.915662650602409)(159,0.953888506538197)(160,0.9150036954915)(161,0.945682451253482)(161,0.908958485069191)(161,0.917225950782998)(161,0.908132530120482)(161,0.932107496463932)(162,0.929597701149425)(163,0.922966162706983)(164,0.89556724267468)(164,0.913818722139673)(165,0.897400820793434)(165,0.938980617372577)(165,0.893100833965125)(165,0.923076923076923)(166,0.90190114068441)(166,0.928823114869626)(166,0.932650073206442)(166,0.897684839432412)(167,0.924715909090909)(167,0.910558170813719)(167,0.901129943502825)(167,0.912592592592593)(167,0.918322295805739)(169,0.939481268011527)(169,0.875684128225176)(169,0.90190114068441)(169,0.888048411497731)(169,0.939890710382514)(169,0.906779661016949)(170,0.894498381877023)(170,0.906403940886699)(170,0.929041697147037)(171,0.917827298050139)(171,0.853968253968254)(171,0.891288696904248)(172,0.885735623599701)(172,0.892994611239415)(173,0.883756735950731)(174,0.90190114068441)(175,0.922394678492239)(175,0.923959827833572)(175,0.91005291005291)(176,0.930134086097389)(176,0.930733019502354)(177,0.954866008462623)(178,0.941261783901378)(178,0.910324039186134)(179,0.917613636363636)(179,0.887864823348694)(179,0.924406047516199)(179,0.907738095238095)(180,0.946403385049365)(180,0.922431865828092)(180,0.919825072886297)(180,0.8898847631242)(181,0.955570745044429)(181,0.946132596685083)(181,0.915328467153285)(181,0.929411764705882)(182,0.927719298245614)(183,0.911284599006387)(184,0.896907216494845)(184,0.938040345821326)(184,0.949689869055824)(184,0.921938088829071)(185,0.942496493688639)(185,0.932685634975711)(185,0.937365010799136)(186,0.885780885780886)(186,0.924050632911392)(186,0.926164874551971)(187,0.922974767596282)(188,0.931276297335203)(189,0.9152288072018)(189,0.902088772845953)(189,0.894703254626675)(190,0.893854748603352)(190,0.898617511520737)(192,0.890817064352856)(192,0.926760563380282)(193,0.930612244897959)(193,0.935371785962474)(193,0.903274942878903)(194,0.853035143769968)(194,0.89153605015674)(194,0.928677563150074)(195,0.93943661971831)(195,0.904619970193741)(195,0.920080591000672)(195,0.926345609065156)(195,0.921595598349381)(196,0.944484498918529)(196,0.913728432108027)(197,0.874120406567631)(197,0.940340909090909)(198,0.934659090909091)(199,0.923420074349442)(199,0.917613636363636)(199,0.945165945165945)(200,0.950105411103303)(201,0.906040268456376)(201,0.94819020581973)(202,0.910179640718563)(202,0.93560325684678)(202,0.903057905009759)(202,0.908730158730159)(203,0.924050632911392)(203,0.946801346801347)(203,0.93085501858736)(204,0.934860415175376)(204,0.922407541696882)(204,0.939673527324343)(204,0.886486486486486)(204,0.954321855235418)(205,0.900306748466258)(205,0.956280360860513)(205,0.934017595307918)(206,0.926898509581263)(206,0.910934105720492)(206,0.910470409711684)(207,0.89807976366322)(207,0.901664145234493)(208,0.902587519025875)(208,0.917615539182853)(210,0.926362297496318)(211,0.92089552238806)(211,0.950314905528341)(211,0.899770466717674)(211,0.942545454545454)(212,0.948311509303928)(212,0.961748633879781)(213,0.948148148148148)(214,0.932542624166049)(215,0.942253521126761)(215,0.927835051546392)(215,0.949410949410949)(215,0.943603851444291)(216,0.942279942279942)(217,0.91072768192048)(217,0.94)(218,0.920454545454545)(218,0.911126662001399)(219,0.921052631578947)(219,0.92716133424098)(222,0.93)(222,0.925461254612546)(222,0.922388059701492)(222,0.916063675832127)(222,0.943342776203966)(222,0.931754874651811)(223,0.936727272727273)(223,0.91798344620015)(223,0.942562592047128)(224,0.919458303635068)(224,0.908122503328895)(225,0.92882818116463)(225,0.948771929824561)(226,0.923985239852398)(226,0.940425531914894)(226,0.920704845814978)(226,0.945135332845647)(227,0.924963924963925)(227,0.941761363636364)(228,0.907294832826748)(229,0.89247311827957)(229,0.942079553384508)(230,0.909578030810449)(230,0.945010183299389)(233,0.920539730134932)(234,0.950453593859037)(234,0.931618144888287)(234,0.934379457917261)(234,0.913235294117647)(234,0.956645344705046)(236,0.912359550561798)(236,0.941013185287994)(236,0.932094353109364)(237,0.908263836239575)(237,0.94057226705796)(237,0.925827814569536)(237,0.939457202505219)(237,0.937210839391936)(239,0.944598337950138)(239,0.917127071823204)(240,0.923601637107776)(240,0.926261319534282)(241,0.919762258543833)(242,0.92790863668808)(244,0.953604568165596)(244,0.913338357196684)(246,0.927366104181952)(246,0.940244780417567)(246,0.94759511844939)(246,0.947368421052631)(246,0.926301555104801)(247,0.953058321479374)(247,0.929692832764505)(248,0.945636623748212)(248,0.922962962962963)(249,0.913897280966767)(249,0.921348314606741)(249,0.904854368932039)(250,0.94982332155477)(252,0.910170749814402)(252,0.942798070296347)(252,0.945205479452055)(252,0.938307030129125)(253,0.94402211472011)(253,0.94273127753304)(255,0.94489247311828)(256,0.92433234421365)(256,0.938659058487874)(257,0.945848375451263)(257,0.944837340876945)(258,0.928619079386257)(259,0.909819639278557)(259,0.896392939370683)(260,0.934058463630183)(260,0.947887323943662)(260,0.936857562408223)(261,0.940111420612813)(261,0.966101694915254)(261,0.946266573621772)(262,0.940072202166065)(263,0.941176470588235)(263,0.892638036809816)(263,0.91307634164777)(263,0.946001367053999)(264,0.939914163090129)(264,0.957746478873239)(264,0.956104252400549)(266,0.929761042722665)(267,0.909495548961424)(267,0.921700223713646)(267,0.965468639887244)(268,0.935832732516222)(268,0.951289398280802)(268,0.95578947368421)(269,0.930808448652586)(270,0.93411420204978)(271,0.918687589158345)(271,0.940919037199125)(272,0.93408929836995)(272,0.902457185405808)(272,0.944837340876945)(273,0.909774436090226)(274,0.937728937728938)(274,0.910741301059001)(274,0.958512160228898)(275,0.952650176678445)(275,0.948126801152738)(277,0.933528122717312)(277,0.926756352765321)(277,0.937269372693727)(278,0.942363112391931)(279,0.921465968586387)(279,0.940695296523517)(280,0.935461485079806)(280,0.925129725722758)(280,0.931133428981348)(280,0.927350427350427)(280,0.964234620886981)(281,0.930800542740841)(282,0.948611111111111)(282,0.947521865889213)(283,0.928524590163934)(283,0.943672275054865)(284,0.957686882933709)(284,0.950344827586207)(284,0.946502057613169)(285,0.9104589917231)(285,0.938745387453874)(286,0.946831755280408)(287,0.954545454545454)(287,0.948116560056859)(287,0.933981931897151)(289,0.934316353887399)(289,0.967514124293785)(290,0.926865671641791)(290,0.920525224602626)(290,0.957132817990161)(291,0.910334346504559)(291,0.949176807444524)(291,0.949421965317919)(292,0.941852117731515)(292,0.93899931459904)(293,0.959430604982206)(294,0.925531914893617)(294,0.950796950796951)(295,0.953741496598639)(296,0.941860465116279)(296,0.882307092751364)(296,0.954992967651195)(297,0.947807933194154)(297,0.949620427881297)(297,0.951704545454545)(297,0.920454545454545)(298,0.92503748125937)(298,0.944081336238199)(298,0.943478260869565)(299,0.909361069836553)(299,0.926223520818115)(300,0.962962962962963)(301,0.950564971751412)(302,0.95572733661279)(302,0.945404284727021)(303,0.949394155381326)(303,0.947063089195069)(304,0.947445255474452)(304,0.921465968586387)(304,0.945295404814004)(304,0.928571428571429)(305,0.940845070422535)(305,0.962643678160919)(305,0.933601609657948)(306,0.955431754874652)(307,0.921713441654357)(307,0.942461762563729)(308,0.939794419970631)(309,0.924907063197026)(309,0.94273127753304)(310,0.951497860199715)(313,0.907153729071537)(313,0.91399416909621)(313,0.93002915451895)(313,0.944639103013315)(314,0.911698113207547)(314,0.928623988226637)(316,0.935672514619883)(316,0.938947368421053)(318,0.929051530993279)(318,0.91798344620015)(319,0.905087319665907)(319,0.938271604938271)(319,0.953375086986778)(319,0.911516853932584)(319,0.954257565095003)(320,0.951130561633844)(320,0.958999305072967)(320,0.958712386284115)(322,0.952045133991537)(323,0.938202247191011)(323,0.957252978276104)(323,0.928413284132841)(325,0.931586608442504)(325,0.90978013646702)(325,0.958421423537702)(325,0.960662525879917)(325,0.955617198335645)(325,0.961756373937677)(326,0.952173913043478)(326,0.925845932325414)(327,0.914542728635682)(327,0.961672473867596)(328,0.970464135021097)(328,0.896710022953328)(329,0.954866008462623)(329,0.95467032967033)(330,0.952513966480447)(331,0.913897280966767)(331,0.956272401433692)(331,0.928467153284671)(331,0.947368421052631)(331,0.958133150308854)(331,0.9152288072018)(331,0.925461254612546)(332,0.945070422535211)(334,0.955459770114942)(334,0.96)(334,0.916604057099925)(335,0.944992947813822)(336,0.939568345323741)(337,0.953890489913545)(337,0.948116560056859)(337,0.964509394572025)(337,0.955650929899857)(338,0.921348314606741)(340,0.946712802768166)(341,0.952449567723343)(341,0.97054698457223)(343,0.963788300835654)(343,0.955072463768116)(343,0.926253687315634)(343,0.943267259056733)(345,0.96011396011396)(346,0.944847605224964)(347,0.939042089985486)(348,0.936872309899569)(348,0.935689045936396)(349,0.919523099850969)(351,0.933333333333333)(352,0.93943661971831)(352,0.952924393723252)(352,0.918840579710145)(352,0.936842105263158)(353,0.942982456140351)(353,0.960339943342776)(353,0.933920704845815)(355,0.960629921259842)(355,0.958273381294964)(356,0.960056061667835)(356,0.928358208955224)(357,0.956582633053221)(357,0.940746159473299)(358,0.960729312762973)(359,0.938745387453874)(360,0.93731778425656)(360,0.96049896049896)(360,0.942279942279942)(360,0.961538461538461)(361,0.925524222704266)(362,0.953338119167265)(362,0.939521800281294)(363,0.966573816155989)(364,0.962025316455696)(364,0.9578270192995)(365,0.929577464788732)(366,0.92503748125937)(366,0.935550935550935)(367,0.936296296296296)(367,0.962169878658101)(368,0.939636363636364)(368,0.957264957264957)(369,0.933995741660752)(369,0.949201741654572)(369,0.937961595273264)(369,0.938925680647535)(369,0.959372772630078)(370,0.958620689655172)(370,0.945527908540686)(371,0.947148817802503)(372,0.946454413892909)(375,0.965083798882682)(377,0.938746438746439)(378,0.905487804878049)(378,0.943710910354413)(379,0.9484827099506)(379,0.959666203059805)(379,0.966053748231966)(379,0.971034482758621)(379,0.929785661492978)(380,0.955801104972376)(382,0.942877801879971)(383,0.954102920723227)(383,0.959039548022599)(384,0.924778761061947)(385,0.937961595273264)(386,0.933137398971345)(386,0.958333333333333)(387,0.963493199713672)(387,0.945273631840796)(388,0.940659340659341)(389,0.949671772428884)(390,0.959717314487632)(390,0.95373665480427)(391,0.950413223140496)(391,0.970527758738862)(392,0.966197183098591)(392,0.965564738292011)(393,0.940665701881331)(394,0.935294117647059)(394,0.943741209563994)(395,0.942209217264082)(395,0.941508104298802)(396,0.939349112426035)(397,0.928783382789317)(397,0.945215485756026)(397,0.950608446671439)(398,0.941836019621584)(399,0.952712100139082)(400,0.933628318584071)(400,0.96090973702914) 
};

\end{axis}
\end{tikzpicture}%

%% This file was created by matlab2tikz v0.2.3.
% Copyright (c) 2008--2012, Nico Schlömer <nico.schloemer@gmail.com>
% All rights reserved.
% 
% 
% 
\begin{tikzpicture}

\begin{axis}[%
tick label style={font=\tiny},
label style={font=\tiny},
label shift={-4pt},
xlabel shift={-6pt},
legend style={font=\tiny},
view={0}{90},
width=\figurewidth,
height=\figureheight,
scale only axis,
xmin=0, xmax=400,
xlabel={Samples},
ymin=3, ymax=9,
ytick={3, 6, 7, 9},
ylabel={$F_1$-score},
axis lines*=left,
legend cell align=left,
legend style={at={(1.03,0)},anchor=south east,fill=none,draw=none,align=left,row sep=-0.2em},
clip=false]

\addplot [
color=magenta,
solid,
line width=1.0pt,
]
coordinates{
 (1,3.11107800375095)(2,4.68535331467431)(3,4.71832395871032)(4,4.81408876620426)(5,5.77976119531478)(6,5.83116607331716)(7,6.11354134124088)(8,6.20930298692382)(9,6.25557680644656)(10,6.26386077972881)(11,6.27554202433931)(12,6.30989171068877)(13,6.32875879668475)(14,6.35564481520696)(15,6.3645679493329)(16,6.36454240055896)(17,6.36126013059943)(18,6.36129244543381)(19,6.3603352648338)(20,6.38967801411228)(21,6.41810754827449)(22,6.62233125309798)(23,6.61146587677335)(24,6.6109038260506)(25,6.62359319717757)(26,6.63928660711124)(27,6.63874424224279)(28,6.64064146276172)(29,6.65302695016459)(30,6.64432358700447)(31,6.65743611981381)(32,6.64565961439267)(33,6.65621532447185)(34,6.65617841684159)(35,6.65607128387988)(36,6.65631732918343)(37,6.64496017161674)(38,6.64497914266466)(39,6.65381418713583)(40,6.65388420164123)(41,6.65912241184582)(42,6.6591746500797)(43,6.66186336017589)(44,6.66394240033063)(45,6.63492144422698)(46,6.63019616170362)(47,6.64181926430687)(48,6.64136696215738)(49,6.64301787326739)(50,6.64299917108371)(51,6.64394208909223)(52,6.64458772559633)(53,6.6446820910748)(54,6.64472543718778)(55,6.64523974895257)(56,6.68399116393431)(57,6.68942101248755)(58,6.68684699692351)(59,6.68490723553944)(60,6.68707974160038)(61,6.69368321470288)(62,6.69406012262291)(63,6.69404892709908)(64,6.69466913695537)(65,6.69784239832511)(66,6.69772102017116)(67,6.69817163061821)(68,6.69835254950537)(69,6.69951310987449)(70,6.68039489078891)(71,6.65245786893124)(72,6.65848261909785)(73,6.66132414082075)(74,6.65874537551552)(75,6.65835280020539)(76,6.65839314833856)(77,6.65839949720556)(78,6.6467830888831)(79,6.64682180059248)(80,6.6429228726541)(81,6.63862241299849)(82,6.64437415628062)(83,6.64431840810706)(84,6.64440138859625)(85,6.64444609265641)(86,6.6444218911624)(87,6.64688076415734)(88,6.64718817613189)(89,6.65291029015572)(90,6.65259623377883)(91,6.64992140950789)(92,6.65040798102368)(93,6.65043056920749)(94,6.65048799981976)(95,6.65045239832789)(96,6.65044749618464)(97,6.65016137137163)(98,6.65018065443973)(99,6.65010908473107)(100,6.66360864600037)(101,6.66359546140966)(102,6.66649921978685)(103,6.66766668668447)(104,6.66989925920933)(105,6.65950030816001)(106,6.66013558508985)(107,6.66006275311901)(108,6.65935305667284)(109,6.65419143463242)(110,6.65450441129936)(111,6.6768033469851)(112,6.68012475836662)(113,6.67293386735258)(114,6.67182276953892)(115,6.66471778741251)(116,6.66184424661227)(117,6.66536455836333)(118,6.66499727831587)(119,6.66535338061546)(120,6.65567084154609)(121,6.65597657292825)(122,6.65606161136518)(123,6.65610284062585)(124,6.63621988446709)(125,6.63665584791321)(126,6.63764773383404)(127,6.63767494622253)(128,6.6377260001386)(129,6.65790464868446)(130,6.64940201231812)(131,6.64944430982523)(132,6.64596383654935)(133,6.64005464514274)(134,6.63563872292456)(135,6.62433037152995)(136,6.6298349175556)(137,6.64000337736127)(138,6.64000748039641)(139,6.64002123180669)(140,6.64036486695554)(141,6.63982758631698)(142,6.6397273655147)(143,6.63971406757535)(144,6.6390157631161)(145,6.63871623113112)(146,6.63863209803011)(147,6.63418449108599)(148,6.62636933149448)(149,6.61159902917889)(150,6.61159377271159)(151,6.60941526518887)(152,6.60941906137584)(153,6.61013220262517)(154,6.61076094594478)(155,6.610826596999)(156,6.61082665366191)(157,6.61091498218333)(158,6.60575748407116)(159,6.60643951215919)(160,6.60015776533677)(161,6.60017353608503)(162,6.59904250846663)(163,6.59910196942139)(164,6.59757900782608)(165,6.5975879050273)(166,6.59834723040083)(167,6.59835592013388)(168,6.59815083090642)(169,6.5981470565688)(170,6.59808496141425)(171,6.59807119092715)(172,6.59791358955124)(173,6.59794777109505)(174,6.60201086010728)(175,6.60473802737125)(176,6.59282608166848)(177,6.5842309751775)(178,6.58423945496954)(179,6.58423489049903)(180,6.58319944805127)(181,6.58298003381784)(182,6.58364112117374)(183,6.58362576978417)(184,6.58425220749878)(185,6.58454657532143)(186,6.5848563438134)(187,6.58485434326985)(188,6.58484996849062)(189,6.58485433639062)(190,6.58487461060133)(191,6.58398162239304)(192,6.57790047437368)(193,6.57522341741832)(194,6.57932997565811)(195,6.57893155987627)(196,6.57143142150128)(197,6.57763320094053)(198,6.5775820571927)(199,6.57760819686969)(200,6.57760408440824)(201,6.5783239873107)(202,6.5783761820105)(203,6.57812059292856)(204,6.57763474165568)(205,6.57871633818666)(206,6.57254411748999)(207,6.57298736970559)(208,6.57337455115017)(209,6.5733732462492)(210,6.57747236825562)(211,6.57753732351011)(212,6.57761117698087)(213,6.57759690430762)(214,6.57711531736778)(215,6.57712396536159)(216,6.57699825983984)(217,6.57699764472452)(218,6.57352594902528)(219,6.57420373130139)(220,6.57952594384374)(221,6.57948586252751)(222,6.57152425994447)(223,6.57529930589117)(224,6.57543429390458)(225,6.57549985942071)(226,6.57959381137728)(227,6.57861971808064)(228,6.57805417976326)(229,6.57831294872169)(230,6.57845561826816)(231,6.57844963474745)(232,6.58582595584783)(233,6.58604770830611)(234,6.58604587519845)(235,6.58611584155743)(236,6.58663757846549)(237,6.58276941327767)(238,6.58274665039264)(239,6.58274633162668)(240,6.58318596255435)(241,6.58079845109007)(242,6.58081091647044)(243,6.57881886392753)(244,6.59072870958466)(245,6.59103524797771)(246,6.59163124313972)(247,6.59175917722587)(248,6.59249191286349)(249,6.59236449753027)(250,6.59236630043861)(251,6.5925342656882)(252,6.59242209315555)(253,6.59241797444123)(254,6.59241918086079)(255,6.58945531288299)(256,6.58958931492929)(257,6.59002692686173)(258,6.59000778375554)(259,6.58997778874357)(260,6.58991571804779)(261,6.58995242466315)(262,6.59040019331885)(263,6.59039959249892)(264,6.59042303694565)(265,6.59042607585633)(266,6.59029754841761)(267,6.59059214252934)(268,6.59086964996425)(269,6.59102523912565)(270,6.59079583836366)(271,6.59069959770532)(272,6.59065271766602)(273,6.59062061621137)(274,6.59060471031149)(275,6.59072637313631)(276,6.59072444387478)(277,6.59071374688623)(278,6.59074997982346)(279,6.5906172619297)(280,6.59118882224369)(281,6.59117254767303)(282,6.5911268809406)(283,6.58223910005556)(284,6.55909740157628)(285,6.55331125821957)(286,6.55309991128468)(287,6.54111815433445)(288,6.53572944156023)(289,6.53264106820865)(290,6.52861032367928)(291,6.52626106809637)(292,6.52625859016443)(293,6.52711691191497)(294,6.52813176370594)(295,6.52896534985566)(296,6.52898250194908)(297,6.5253999892736)(298,6.5265442104436)(299,6.52869130693952)(300,6.52781993600729)(301,6.52920950073925)(302,6.52060655794837)(303,6.51846684330667)(304,6.51843727049828)(305,6.50985215581566)(306,6.51005878337891)(307,6.50614261197931)(308,6.50712117641081)(309,6.50752186459772)(310,6.50754337433502)(311,6.50515929619286)(312,6.50777639680803)(313,6.50644968232977)(314,6.51561326371773)(315,6.51670865051255)(316,6.52067287697774)(317,6.5206442382646)(318,6.52202503999202)(319,6.52241676136816)(320,6.52260808245931)(321,6.52348690929803)(322,6.52469655016172)(323,6.52652841258517)(324,6.52570628044383)(325,6.52564140999526)(326,6.52667122264878)(327,6.52677486058512)(328,6.53440485581001)(329,6.53438990933381)(330,6.53448844984065)(331,6.53447560716328)(332,6.53449172594808)(333,6.53445526828621)(334,6.5345036581628)(335,6.53550410605789)(336,6.53678514859248)(337,6.53722014258446)(338,6.53598746571255)(339,6.53594927440794)(340,6.53595854888359)(341,6.53623332105126)(342,6.5376724975643)(343,6.53594001424752)(344,6.54114339345626)(345,6.54344075385425)(346,6.53987529489418)(347,6.5389364686997)(348,6.54002805258848)(349,6.53990320540484)(350,6.54055070531315)(351,6.54062822200345)(352,6.54049659945405)(353,6.54048164035753)(354,6.54048441170132)(355,6.54048136480962)(356,6.5392680122757)(357,6.5323715157289)(358,6.52886862242012)(359,6.52577408012947)(360,6.52288800066215)(361,6.52517004773417)(362,6.52517882090898)(363,6.52578949049675)(364,6.52085931933042)(365,6.52071594551147)(366,6.51766486830301)(367,6.5176913414243)(368,6.51706490562203)(369,6.51986443770088)(370,6.53133557218084)(371,6.53127575391312)(372,6.53468851116122)(373,6.53469504233905)(374,6.53440807396025)(375,6.53461413086918)(376,6.53466166078993)(377,6.53363237340582)(378,6.53397521651686)(379,6.53346457320137)(380,6.53479497000668)(381,6.53496891465875)(382,6.534956016108)(383,6.53393310011274)(384,6.53394938244687)(385,6.53396621038596)(386,6.53395329951076)(387,6.53387852392769)(388,6.53390435988664)(389,6.5339285852209)(390,6.53393693360888)(391,6.53409332776458)(392,6.53411574493062)(393,6.53409623355479)(394,6.53411044214725)(395,6.53772488567078)(396,6.54282302629155)(397,6.54269665410177)(398,6.54296871793651)(399,6.54171808226612)(400,6.54172331324431) 
};
\addlegendentry{$\omega\max_{\*x\in D}\mu_{t-1}(\*x)$};

\addplot [
color=darkgray,
densely dotted,
line width=1.0pt,
]
coordinates{
 (1,7)(400,7)
};
\addlegendentry{$h$};

\end{axis}
\end{tikzpicture}%

%% This file was created by matlab2tikz v0.2.3.
% Copyright (c) 2008--2012, Nico Schlömer <nico.schloemer@gmail.com>
% All rights reserved.
% 
% 
% 
\begin{tikzpicture}

\begin{axis}[%
tick label style={font=\tiny},
label style={font=\tiny},
label shift={-4pt},
xlabel shift={-6pt},
legend style={font=\tiny},
view={0}{90},
width=\figurewidth,
height=\figureheight,
scale only axis,
xmin=0, xmax=400,
xlabel={Samples},
ymin=3, ymax=9,
ytick={3, 6, 7, 9},
ylabel={$F_1$-score},
axis lines*=left,
legend cell align=left,
legend style={at={(1.03,0)},anchor=south east,fill=none,draw=none,align=left,row sep=-0.2em},
clip=false]

\addplot [
color=magenta,
solid,
line width=1.0pt,
]
coordinates{
 %(1,8.30445443671755)
 %(2,8.30439836908009)(3,8.30411381713662)(4,8.29109471167835)
 (5,8.28090858446535)(6,8.28090858446535)(7,8.28090858446535)(8,8.28090858446535)(9,8.28006981273306)(10,8.28006981273306)(11,8.28006981273306)(12,8.28006981273306)(13,8.28006981273306)(14,8.28006981273306)(15,8.28006981273306)(16,8.28006981273306)(17,8.28006981273306)(18,8.28006981273306)(19,8.28006981273306)(20,8.28006981273306)(21,8.27811775914642)(22,8.27811775914642)(23,8.27811775914642)(24,8.27811775914642)(25,8.27811775914642)(26,8.27811775914642)(27,8.27811775914642)(28,8.27811775914642)(29,8.27811775914642)(30,8.27811775914642)(31,8.27811775914642)(32,8.27811775914642)(33,8.27811775914642)(34,8.27811775914642)(35,8.27811775914642)(36,8.27811775914642)(37,8.27811775914642)(38,8.27811775914642)(39,8.27811775914642)(40,8.27811775914642)(41,8.27811775914642)(42,8.27811775914642)(43,8.27811775914642)(44,8.27811775914642)(45,8.0866853206189)(46,8.08487415267328)(47,8.08303703991649)(48,8.08303703991649)(49,8.08303703991649)(50,8.08303703991649)(51,8.08303703991649)(52,8.08303703991649)(53,7.93425115732777)(54,7.93418678442933)(55,7.93392471233648)(56,7.93392471233648)(57,7.93392471233648)(58,7.93392471233648)(59,7.93392471233648)(60,7.93392471233648)(61,7.93392471233648)(62,7.93392471233648)(63,7.93392471233648)(64,7.93392471233648)(65,7.93392471233648)(66,7.65922558929011)(67,7.65917337530369)(68,7.65722392352335)(69,7.65722392352335)(70,7.65722392352335)(71,7.65722392352335)(72,7.65722392352335)(73,7.65722392352335)(74,7.65722392352335)(75,7.65722392352335)(76,7.65722392352335)(77,7.56147566005119)(78,7.56147566005119)(79,7.56147566005119)(80,7.56147566005119)(81,7.55214020267695)(82,7.55214020267695)(83,7.55110533389376)(84,7.55110533389376)(85,7.55110533389376)(86,7.55110533389376)(87,7.55110533389376)(88,7.55110533389376)(89,7.55110533389376)(90,7.55110533389376)(91,7.44654167193519)(92,7.43417203735168)(93,7.43417203735168)(94,7.43417203735168)(95,7.39022785389079)(96,7.39022785389079)(97,7.39022785389079)(98,7.39022785389079)(99,7.39022785389079)(100,7.39022785389079)(101,7.39022785389079)(102,7.39022785389079)(103,7.39022785389079)(104,7.39022785389079)(105,7.39022785389079)(106,7.38956336199327)(107,7.38956336199327)(108,7.38956336199327)(109,7.38956336199327)(110,7.38956336199327)(111,7.38956336199327)(112,7.38956336199327)(113,7.38956336199327)(114,7.38956336199327)(115,7.38956336199327)(116,7.38956336199327)(117,7.38956336199327)(118,7.38956336199327)(119,7.38956336199327)(120,7.38956336199327)(121,7.38956336199327)(122,7.38956336199327)(123,7.38956336199327)(124,7.38956336199327)(125,7.38956336199327)(126,7.38956336199327)(127,7.38956336199327)(128,7.38956336199327)(129,7.38956336199327)(130,7.38956336199327)(131,7.38956336199327)(132,7.38956336199327)(133,7.38956336199327)(134,7.38956336199327)(135,7.38956336199327)(136,7.38956336199327)(137,7.38956336199327)(138,7.38956336199327)(139,7.38956336199327)(140,7.38956336199327)(141,7.38956336199327)(142,7.38956336199327)(143,7.38956336199327)(144,7.38956336199327)(145,7.38956336199327)(146,7.38956336199327)(147,7.38956336199327)(148,7.38956336199327)(149,7.38956336199327)(150,7.38956336199327)(151,7.38956336199327)(152,7.38956336199327)(153,7.38956336199327)(154,7.38956336199327)(155,7.38956336199327)(156,7.38956336199327)(157,7.38956336199327)(158,7.38956336199327)(159,7.38956336199327)(160,7.38956336199327)(161,7.38956336199327)(162,7.38956336199327)(163,7.38956336199327)(164,7.38956336199327)(165,7.38956336199327)(166,7.38956336199327)(167,7.38956336199327)(168,7.38956336199327)(169,7.38956336199327)(170,7.38956336199327)(171,7.38956336199327)(172,7.38956336199327)(173,7.38956336199327)(174,7.38956336199327)(175,7.38956336199327)(176,7.38956336199327)(177,7.38956336199327)(178,7.38956336199327)(179,7.38956336199327)(180,7.38956336199327)(181,7.38944633947611)(182,7.38944633947611)(183,7.38876767280655)(184,7.38876767280655)(185,7.38876767280655)(186,7.38876767280655)(187,7.38876767280655)(188,7.38876767280655)(189,7.38876767280655)(190,7.38876767280655)(191,7.38876767280655)(192,7.38876767280655)(193,7.38876767280655)(194,7.38876767280655)(195,7.38876767280655)(196,7.38876767280655)(197,7.38876767280655)(198,7.38876767280655)(199,7.38876767280655)(200,7.38876767280655)(201,7.38876767280655)(202,7.38876767280655)(203,7.38876767280655)(204,7.38876767280655)(205,7.38876767280655)(206,7.38876767280655)(207,7.38876767280655)(208,7.38876767280655)(209,7.38876767280655)(210,7.38876767280655)(211,7.38876767280655)(212,7.38876767280655)(213,7.38876767280655)(214,7.38876767280655)(215,7.38876767280655)(216,7.38876767280655)(217,7.38876767280655)(218,7.38876767280655)(219,7.38876767280655)(220,7.38876767280655)(221,7.38876767280655)(222,7.38876767280655)(223,7.38876767280655)(224,7.38876767280655)(225,7.38876767280655)(226,7.38876767280655)(227,7.38876767280655)(228,7.38876767280655)(229,7.38876767280655)(230,7.38876767280655)(231,7.38876767280655)(232,7.38876767280655)(233,7.38876767280655)(234,7.38876767280655)(235,7.38876767280655)(236,7.38876767280655)(237,7.38876767280655)(238,7.38876767280655)(239,7.38876767280655)(240,7.38876767280655)(241,7.38876767280655)(242,7.38876767280655)(243,7.38876767280655)(244,7.38876767280655)(245,7.38876767280655)(246,7.38876767280655)(247,7.38876767280655)(248,7.38876767280655)(249,7.38876767280655)(250,7.38876767280655)(251,7.38876767280655)(252,7.38876767280655)(253,7.38876767280655)(254,7.38876767280655)(255,7.38876767280655)(256,7.38876767280655)(257,7.38876767280655)(258,7.38876767280655)(259,7.38876767280655)(260,7.38876767280655)(261,7.38876767280655)(262,7.38876767280655)(263,7.38876767280655)(264,7.38876767280655)(265,7.38876767280655)(266,7.38876767280655)(267,7.38876767280655)(268,7.38876767280655)(269,7.38876767280655)(270,7.38876767280655)(271,7.38876767280655)(272,7.38876767280655)(273,7.38876767280655)(274,7.38876767280655)(275,7.38876767280655)(276,7.38612342738956)(277,7.38612342738956)(278,7.38612342738956)(279,7.3804933709359)(280,7.3804933709359)(281,7.37984213330014)(282,7.37984213330014)(283,7.34853719980384)(284,7.34853334818234)(285,7.34686645540143)(286,7.34686645540143)(287,7.34650536882912)(288,7.34650536882912)(289,7.34650536882912)(290,7.34650536882912)(291,7.34650536882912)(292,7.34650536882912)(293,7.34650536882912)(294,7.34650536882912)(295,7.34650536882912)(296,7.34650536882912)(297,7.34650536882912)(298,7.34650536882912)(299,7.34650536882912)(300,7.34650536882912)(301,7.34650536882912)(302,7.34650536882912)(303,7.34650536882912)(304,7.34650536882912)(305,7.34650536882912)(306,7.34650536882912)(307,7.34650536882912)(308,7.34650536882912)(309,7.34650536882912)(310,7.34650536882912)(311,7.34650536882912)(312,7.34650536882912)(313,7.34650536882912)(314,7.34650536882912)(315,7.34650536882912)(316,7.34650536882912)(317,7.34650536882912)(318,7.34639553466514)(319,7.34639553466514)(320,7.34639553466514)(321,7.34639553466514)(322,7.34639553466514)(323,7.34639553466514)(324,7.34639553466514)(325,7.34639553466514)(326,7.34639553466514)(327,7.34580895009784)(328,7.30994142811539)(329,7.30994142811539)(330,7.30943237782337)(331,7.30943237782337)(332,7.30763329611856)(333,7.30763329611856)(334,7.30035761112879)(335,7.30035761112879)(336,7.30035761112879)(337,7.30035761112879)(338,7.30035761112879)(339,7.30035761112879)(340,7.30035761112879)(341,7.30035761112879)(342,7.30035761112879)(343,7.30035761112879)(344,7.30035761112879)(345,7.30035761112879)(346,7.30035761112879)(347,7.30035761112879)(348,7.30035761112879)(349,7.30035761112879)(350,7.30035761112879)(351,7.30035761112879)(352,7.30035761112879)(353,7.30035761112879)(354,7.30035761112879)(355,7.30035761112879)(356,7.30035761112879)(357,7.30035761112879)(358,7.30035761112879)(359,7.30035761112879)(360,7.30035761112879)(361,7.30035761112879)(362,7.30035761112879)(363,7.30035761112879)(364,7.30035761112879)(365,7.29716957657648)(366,7.29716957657648)(367,7.29716957657648)(368,7.29716957657648)(369,7.29716957657648)(370,7.29716957657648)(371,7.29716957657648)(372,7.29716957657648)(373,7.29716957657648)(374,7.29716957657648)(375,7.29716957657648)(376,7.29716957657648)(377,7.29716957657648)(378,7.29716957657648)(379,7.29716957657648)(380,7.29716957657648)(381,7.29716957657648)(382,7.29716957657648)(383,7.29716957657648) 
};
\addlegendentry{$h_{opt}$};

\addplot [
color=darkgray,
densely dotted,
line width=1.0pt,
]
coordinates{
 (1,7)(383,7)
};
\addlegendentry{$h$};

\addplot [
color=blue,
solid,
line width=1.0pt,
]
coordinates{
 %(1,-4.63667922025472)
 %(2,0.178294507861271)(3,0.680779498205593)(4,0.680779498205593)
 (5,3.2000058640654)(6,3.20128465379757)(7,3.20128465379757)(8,3.20137790676286)(9,3.20137790676286)(10,3.20137790676286)(11,3.21991185276437)(12,3.21991185276437)(13,3.23232278602911)(14,3.23264552784319)(15,3.23264552784319)(16,3.23264552784319)(17,3.23264552784319)(18,3.23264552784319)(19,3.23264552784319)(20,3.23264552784319)(21,3.27376613376668)(22,3.2943955677726)(23,3.2943955677726)(24,3.2943955677726)(25,3.91660752096021)(26,4.86530805450772)(27,4.86550836025148)(28,4.86565833883642)(29,4.86565833883642)(30,4.86565833883642)(31,4.87174693508099)(32,4.87632797493825)(33,4.87632797493825)(34,4.87632797493825)(35,4.87965804775542)(36,5.33792219169101)(37,5.33792219169101)(38,5.33792219169101)(39,5.33792219169101)(40,5.33792219169101)(41,5.3435460147102)(42,5.36389146766212)(43,5.36389146766212)(44,5.36389146766212)(45,5.50946450196729)(46,5.50946450196729)(47,5.51067620871419)(48,5.51067620871419)(49,5.51486751805929)(50,5.51486751805929)(51,5.51486751805929)(52,5.51756837586168)(53,5.51756837586168)(54,5.51756837586168)(55,5.51756837586168)(56,5.51756837586168)(57,5.51756837586168)(58,5.51756837586168)(59,5.51756837586168)(60,5.51756837586168)(61,5.51756837586168)(62,5.63425374885359)(63,5.63425374885359)(64,5.63425374885359)(65,5.64125228490949)(66,5.8473146905907)(67,5.8473146905907)(68,5.8623889408609)(69,5.86983796739614)(70,5.86983796739614)(71,5.87342575491512)(72,5.87343121467343)(73,5.87392360367461)(74,5.87696941650131)(75,5.87696941650131)(76,5.87810629753897)(77,5.93757192701823)(78,5.93960348843663)(79,5.94229590083515)(80,5.94229590083515)(81,5.94229590083515)(82,5.94229590083515)(83,5.94229590083515)(84,5.95369376016269)(85,5.95488791910334)(86,5.95488791910334)(87,5.95488791910334)(88,5.95488791910334)(89,5.95488791910334)(90,5.95488791910334)(91,5.95488791910334)(92,5.95488791910334)(93,5.95488791910334)(94,5.95488791910334)(95,5.97478310092679)(96,5.98063660672509)(97,5.98063683686128)(98,5.9807061151607)(99,5.98079764121391)(100,5.98587414220385)(101,5.98588516598665)(102,5.98588516598665)(103,5.98588516598665)(104,5.98588516598665)(105,5.98588516598665)(106,5.98588516598665)(107,5.98588516598665)(108,5.98588516598665)(109,5.98588516598665)(110,6.14872626959614)(111,6.14872626959614)(112,6.15900626472545)(113,6.15938840015678)(114,6.15938840015678)(115,6.15938840015678)(116,6.15938840015678)(117,6.15938840015678)(118,6.15938840015678)(119,6.15938840015678)(120,6.15938840015678)(121,6.16017157408113)(122,6.16463740870697)(123,6.16529547520198)(124,6.16544222915474)(125,6.16544222915474)(126,6.16544749965475)(127,6.16636091878674)(128,6.16907679902357)(129,6.16919153576619)(130,6.17315766242906)(131,6.30008824835084)(132,6.30009654923379)(133,6.30009654923379)(134,6.30009654923379)(135,6.30009654923379)(136,6.30009654923379)(137,6.30009654923379)(138,6.30009654923379)(139,6.30009654923379)(140,6.30009654923379)(141,6.30009654923379)(142,6.30009654923379)(143,6.30009654923379)(144,6.30009654923379)(145,6.30009654923379)(146,6.34247387907388)(147,6.34247387907388)(148,6.34247387907388)(149,6.34247387907388)(150,6.34260132835365)(151,6.34264681951091)(152,6.34616014327677)(153,6.34616014327677)(154,6.34616014327677)(155,6.34616014327677)(156,6.34616014327677)(157,6.34616014327677)(158,6.34616014327677)(159,6.34616014327677)(160,6.34616014327677)(161,6.34616014327677)(162,6.37036206681732)(163,6.37036206681732)(164,6.37036206681732)(165,6.37036206681732)(166,6.37036206681732)(167,6.38786013462908)(168,6.38786013462908)(169,6.38786013462908)(170,6.38786013462908)(171,6.38786013462908)(172,6.38786013462908)(173,6.38786013462908)(174,6.38786013462908)(175,6.38786013462908)(176,6.38940535603155)(177,6.39176433744831)(178,6.39176433744831)(179,6.39176433744831)(180,6.40118775394593)(181,6.45438649083134)(182,6.45449159180552)(183,6.45449159180552)(184,6.45449159180552)(185,6.45449159180552)(186,6.45449159180552)(187,6.45449159180552)(188,6.45449159180552)(189,6.45449159180552)(190,6.51498402891031)(191,6.51505677842199)(192,6.51617529349867)(193,6.51625644219012)(194,6.51678887928836)(195,6.51678887928836)(196,6.51887067118684)(197,6.51887529591094)(198,6.51888715722534)(199,6.52840301954725)(200,6.52855776920047)(201,6.52855776920047)(202,6.52855776920047)(203,6.52855776920047)(204,6.52855776920047)(205,6.52855776920047)(206,6.52863963351837)(207,6.52863963351837)(208,6.52863963351837)(209,6.52863963351837)(210,6.52863963351837)(211,6.52863963351837)(212,6.52863963351837)(213,6.52863963351837)(214,6.52863963351837)(215,6.52863963351837)(216,6.52863963351837)(217,6.52863963351837)(218,6.52863963351837)(219,6.52863963351837)(220,6.52863963351837)(221,6.52863963351837)(222,6.53691992258123)(223,6.53958062730174)(224,6.53976829572928)(225,6.53978810692249)(226,6.53979096115919)(227,6.53979410304873)(228,6.53979410304873)(229,6.53979410304873)(230,6.53979410304873)(231,6.55404706670502)(232,6.55404706670502)(233,6.56112937895876)(234,6.56113006791454)(235,6.56113006791454)(236,6.56113006791454)(237,6.56113006791454)(238,6.56113006791454)(239,6.56113006791454)(240,6.56113006791454)(241,6.57965294097825)(242,6.57966102919014)(243,6.57966102919014)(244,6.58503141356109)(245,6.58503141356109)(246,6.58503141356109)(247,6.58525482484515)(248,6.58779915919219)(249,6.58779915919219)(250,6.58878078270463)(251,6.60212569625203)(252,6.60212569625203)(253,6.60212569625203)(254,6.60212569625203)(255,6.60215713477472)(256,6.60413045725322)(257,6.60413045725322)(258,6.60413045725322)(259,6.60413045725322)(260,6.60413045725322)(261,6.60413045725322)(262,6.60581055648945)(263,6.60586496186798)(264,6.60592980366062)(265,6.60694388972501)(266,6.60694388972501)(267,6.60694388972501)(268,6.60694388972501)(269,6.60694388972501)(270,6.60694388972501)(271,6.61453314720329)(272,6.61453314720329)(273,6.61453314720329)(274,6.61453314720329)(275,6.61453314720329)(276,6.61453314720329)(277,6.61453314720329)(278,6.61453314720329)(279,6.61453314720329)(280,6.61453314720329)(281,6.61940130965601)(282,6.61969105242717)(283,6.61969105242717)(284,6.61969105242717)(285,6.61969105242717)(286,6.61969105242717)(287,6.61969105242717)(288,6.64127312772182)(289,6.64176483434516)(290,6.64179005802007)(291,6.64191677091611)(292,6.64191961665964)(293,6.64308534935576)(294,6.6432748862342)(295,6.64327649114841)(296,6.64327649114841)(297,6.64327649114841)(298,6.64327649114841)(299,6.64327649114841)(300,6.64327649114841)(301,6.64327649114841)(302,6.64327649114841)(303,6.64327649114841)(304,6.64327649114841)(305,6.64327649114841)(306,6.64327649114841)(307,6.64327649114841)(308,6.65344381301762)(309,6.65344381301762)(310,6.65344381301762)(311,6.65344381301762)(312,6.65344381301762)(313,6.65344381301762)(314,6.65344381301762)(315,6.65344381301762)(316,6.65344381301762)(317,6.65344381301762)(318,6.65344381301762)(319,6.65344381301762)(320,6.65344381301762)(321,6.65344381301762)(322,6.65344381301762)(323,6.65344381301762)(324,6.65344381301762)(325,6.65344381301762)(326,6.65344381301762)(327,6.65929103326546)(328,6.65929103326546)(329,6.65929103326546)(330,6.65929103326546)(331,6.65929103326546)(332,6.65929103326546)(333,6.65929103326546)(334,6.65929103326546)(335,6.65929103326546)(336,6.65929103326546)(337,6.65929103326546)(338,6.66498409227681)(339,6.66498409227681)(340,6.66498409227681)(341,6.66498409227681)(342,6.66498409227681)(343,6.66498409227681)(344,6.66498409227681)(345,6.66498409227681)(346,6.66498409227681)(347,6.66498409227681)(348,6.66498409227681)(349,6.66498409227681)(350,6.66498409227681)(351,6.66498409227681)(352,6.66498409227681)(353,6.66498409227681)(354,6.66498409227681)(355,6.66539310957791)(356,6.66539310957791)(357,6.66539310957791)(358,6.66539310957791)(359,6.66539310957791)(360,6.66539310957791)(361,6.66539310957791)(362,6.66618544834278)(363,6.66621308640892)(364,6.6663182843986)(365,6.6663182843986)(366,6.6663182843986)(367,6.6663182843986)(368,6.6663182843986)(369,6.6663182843986)(370,6.6663182843986)(371,6.6663182843986)(372,6.6663182843986)(373,6.6663182843986)(374,6.6663182843986)(375,6.6663182843986)(376,6.6663182843986)(377,6.6663182843986)(378,6.70514427580417)(379,6.70514427580417)(380,6.70630647652538)(381,6.70630647652538)(382,6.70630647652538)(383,6.70920543701038) 
};
\addlegendentry{$h_{pes}$};

\end{axis}
\end{tikzpicture}%

\renewcommand\trimlen{2pt}
\begin{figure}[tbp]
  \begin{subfigure}[b]{0.49\textwidth}
    \centering
    \adjincludegraphics[width=\linewidth,clip=true,trim=\trimlen{} \trimlen{} \trimlen{} \trimlen{}]{figures/ev_chl_imp}
    \caption{\textsf{[C]} Mean performance}
	  \label{fig:chl-imp}
  \end{subfigure}
  \hfill
  \begin{subfigure}[b]{0.49\textwidth}
    \centering
    \adjincludegraphics[width=\linewidth,clip=true,trim=\trimlen{} \trimlen{} \trimlen{} \trimlen{}]{figures/ev_bgape_imp}
    \caption{\textsf{[A]} Mean performance}
	\label{fig:bgape-imp}
  \end{subfigure}

  \begin{subfigure}[b]{0.49\textwidth}
    \centering
    \vspace{12pt} % space of this row from above captions
    \adjincludegraphics[width=\linewidth,clip=true,trim=\trimlen{} \trimlen{} \trimlen{} \trimlen{}]{figures/ev_bgape_imp_istr_sc}
    \caption{\textsf{[A]} \istr scatter}
	  \label{fig:bgape-imp-istr-sc}
  \end{subfigure}
  \hfill
  \begin{subfigure}[b]{0.49\textwidth}
    \centering
    \adjincludegraphics[width=\linewidth,clip=true,trim=\trimlen{} \trimlen{} \trimlen{} \trimlen{}]{figures/ev_bgape_imp_iacl_sc}
    \caption{\textsf{[A]} \iacl scatter}
	\label{fig:bgape-imp-iacl-sc}
  \end{subfigure}

  \begin{subfigure}[b]{0.49\textwidth}
    \centering
    \vspace{12pt} % space of this row from above captions
    \adjincludegraphics[width=\linewidth,clip=true,trim=\trimlen{} \trimlen{} \trimlen{} \trimlen{}]{figures/ev_bgape_imp_istr_h}
    \caption{\textsf{[A]} \istr threshold level estimate}
	  \label{fig:bgape-imp-istr-h}
  \end{subfigure}
  \hfill
  \begin{subfigure}[b]{0.49\textwidth}
    \centering
    \adjincludegraphics[width=\linewidth,clip=true,trim=\trimlen{} \trimlen{} \trimlen{} \trimlen{}]{figures/ev_bgape_imp_iacl_h}
    \caption{\textsf{[A]} \iacl threshold level estimates}
	\label{fig:bgape-imp-iacl-h}
  \end{subfigure}

  \caption{Performance of implicit threshold algorithms on the environmental
           monitoring datasets.
           \textbf{(a), (b)} \iacl and \ibacl perform somewhat worse than their
           explicit threshold counterparts, while \istr performs notably worse.
           \textbf{(c)} Most of \istr executions fail to achieve high
           $F_1$-scores and top off at about $0.8$.
           \textbf{(d)} \iacl always achieves an $F_1$-score of at least $0.9$
           after $400$ iterations.
           \textbf{(e)} \istr's implicit threshold level estimate is lower than
           the true level and becomes worse over time.
           \textbf{(f)} \iacl's implicit threshold level estimates correctly bound
           the true level and converge towards it over time.
           }
  \label{fig:exp-imp}
\end{figure}

\section{Results III:\hspace{0.33em}Path planning}
We now present the results from applying our batch-based path planning
algorithm on the two environmental monitoring datasets. Since path planning
executions are costly, we used a fixed small value of $\epsilon$ for each
dataset and executed the algorithm $30$ times for each of five different
batch sizes, while measuring total traveled path length and $F_1$-score
after each path planning iteration (similarly to what we did in
\figref{fig:exp-rule}).

\figref{fig:exp-pp} displays the dramatically reduced travel
lengths by using batches of samples for path planning.
For example, in \figref{fig:bgape-pp} we can see that
planning with $B=30$ samples at a time achieves an $F_1$-score
of $0.9$ after a travel length of about $4$ transect lengths,
while planning with $B=5$ (or even worse sequentially) does
not achieve that accuracy even after $15$ transect lengths.
Also note that the effect of the batch size on the travel
lengths seems to have a diminishing returns property
(e.g. increasing from $B=5$ to $B=15$ makes a much larger
difference than increasing from $B=15$ to $B=60$).

%\setlength\figureheight{1.3in}\setlength\figurewidth{2.1in}
%% This file was created by matlab2tikz v0.2.3.
% Copyright (c) 2008--2012, Nico Schlömer <nico.schloemer@gmail.com>
% All rights reserved.
% 
% 
% 
\begin{tikzpicture}

\begin{axis}[%
tick label style={font=\tiny},
label style={font=\tiny},
label shift={-4pt},
xlabel shift={-6pt},
legend style={font=\tiny},
view={0}{90},
width=\figurewidth,
height=\figureheight,
scale only axis,
xmin=0, xmax=15,
xlabel={Normalized travel length},
ymin=0.4, ymax=1,
ylabel={$F_1$-score},
axis lines*=left,
legend cell align=left,
legend style={at={(1.03,0)},anchor=south east,fill=none,draw=none,align=left,row sep=-0.2em},
clip=false]

\addplot [
color=blue,
solid,
line width=1.0pt,
]
coordinates{
 (3.62977761188785,0.499988384901358)(3.63251548608583,0.500062134930255)(3.63648404477314,0.500169028553008)(3.64300533825072,0.500344661618858)(3.65459488647268,0.500656736584086)(3.65879935799805,0.500769933357956)(3.66873432174284,0.501037373455535)(3.70059545314026,0.501894682492363)(3.70328940775669,0.501967145216712)(3.72098832492287,0.502443117560375)(3.72105913104727,0.502445021389891)(3.73097055955662,0.502711492326663)(3.73490708311425,0.502817311888192)(3.75122892037801,0.503255977703358)(3.77791364859128,0.503972847735195)(3.78847641970825,0.504256504999566)(3.79032204265889,0.50430606203294)(3.79272072535801,0.504370466614031)(3.79807449788593,0.50451420420148)(3.80188827711809,0.504616586852428)(3.82977893492615,0.50536508865295)(3.84195755468258,0.505691796639747)(3.87114847395959,0.506474562926597)(3.87356625172733,0.506539376434997)(3.88232012273585,0.50677401620226)(3.89505337837366,0.507115247892124)(3.90227716103415,0.507308796317965)(3.9117405637803,0.5075623099976)(3.96305893983105,0.508936256206159)(3.96527323433222,0.508995508667643)(3.98204200438965,0.509444142966649)(3.99023178876102,0.509663200922448)(3.99753899262633,0.509858622671342)(4.02374630249953,0.510559277480777)(4.02652362710421,0.510633508847622)(4.03506521212526,0.510861780610098)(4.05815516503287,0.511478667331723)(4.07319024846861,0.511880208713651)(4.08473905834684,0.512188564354237)(4.08719938557037,0.51225424686831)(4.09109833550274,0.512358329521994)(4.11531772887643,0.513004695254475)(4.1212732489414,0.513163590495259)(4.13160132121068,0.513439104561743)(4.17742654765628,0.514660904170553)(4.17914046997946,0.514706580702583)(4.18406314276361,0.514837763245202)(4.1985639582354,0.515224120288706)(4.20690403414035,0.515446284473898)(4.21223823442081,0.515588359635287)(4.26031468623023,0.516868231309709)(4.26753455991004,0.517060337780842)(4.26892666625549,0.517097376026848)(4.27085600166179,0.517148706170057)(4.27470843523076,0.517251195070268)(4.29668352132032,0.517835674723233)(4.30316863013285,0.518008116685228)(4.30723243432876,0.518116164657025)(4.33582218164695,0.518876079564531)(4.36373942380151,0.519617738850681)(4.38038551090368,0.520059786679976)(4.41192648878955,0.520897015466047)(4.4344602310355,0.521494865428892)(4.45061385062732,0.521923294672805)(4.4727767046545,0.522510902200228)(4.47740099523599,0.522633477762435)(4.48734137816572,0.522896932547584)(4.50713596915293,0.523421420902911)(4.50866240087836,0.523461858503063)(4.51328473855283,0.523584304966501)(4.51384938865972,0.523599261958129)(4.52749768448503,0.523960746091139)(4.52934292990268,0.524009612015391)(4.5395572625227,0.524280080250826)(4.55753768116636,0.524756071991621)(4.60617377865887,0.526042861923356)(4.60987967387075,0.526140866423737)(4.61270903634685,0.526215686317836)(4.63648077014355,0.526844164657319)(4.65670612390385,0.527378682606703)(4.66873432174284,0.527696477922218)(4.67091299812364,0.527754033478933)(4.68060215264229,0.528009972628495)(4.68306247986581,0.528074955519282)(4.6989544443528,0.528494633688357)(4.69945636642296,0.528507886712529)(4.70034988752156,0.528531479453558)(4.70662372194154,0.528697125248603)(4.76007974821897,0.530107798607816)(4.76167671431048,0.530149922242248)(4.76780219874862,0.530311485538447)(4.76969243479806,0.53036133831958)(4.78847641970825,0.530856659220954)(4.80065503946468,0.531177718924074)(4.80625132836185,0.531325229851582)(4.84201616316958,0.532267624378142)(4.87125882449377,0.53303775210929)(4.88902987732892,0.533505587695901)(4.90781799582092,0.534000052209933)(4.92122960780434,0.534352926722998)(4.94569107616879,0.534996340072361)(4.94978359254241,0.535103961548943)(4.95914924413716,0.535350225014488)(4.96335241253796,0.535460732559122)(4.96530880244529,0.535512166441958)(4.97182214902151,0.535683392124544)(4.97602013851925,0.535793741327578)(4.99025146906653,0.536167774443221)(4.99778760603249,0.536365807964868)(5.00205603910505,0.536477962804532)(5.00354477653887,0.536517078222529)(5.0066839429853,0.536599554377359)(5.01628775197944,0.536851852549115)(5.0281544909563,0.537163546857827)(5.03042987310772,0.537223305918814)(5.03280879654874,0.537285782043814)(5.03455011084398,0.53733151158103)(5.03513936876525,0.537346986100311)(5.03884410739445,0.537444273087681)(5.04305757426907,0.537554912520685)(5.07026205770055,0.538269087613203)(5.08719975632941,0.538713585844832)(5.10993181174422,0.539309963703581)(5.11319758647709,0.539395624531346)(5.12541915467703,0.539716156571389)(5.13764304175749,0.540036689521051)(5.14816741125086,0.540312610115458)(5.16175130171279,0.540668677835614)(5.1718658916762,0.540933758884184)(5.1791448282812,0.541124498630673)(5.19224838427155,0.541467815894779)(5.22251097624517,0.542260446922344)(5.26110340621845,0.543270731768653)(5.26895720008287,0.543476260036154)(5.27523831372067,0.543640615205898)(5.28500797191317,0.54389622335724)(5.28959203946257,0.544016145798534)(5.29185400336048,0.544075317392733)(5.32109744790585,0.544840131463963)(5.33288935751903,0.545148437025834)(5.34682555324209,0.545512737924369)(5.34707841373229,0.545519347176012)(5.36377759143573,0.545955775713985)(5.37595621119216,0.546273994356945)(5.41428329591074,0.547275090838521)(5.42033356812087,0.547433072766013)(5.43283185790009,0.547759379342268)(5.43390552593332,0.54778740807317)(5.47141171944407,0.548766262326726)(5.4952436867186,0.54938797216448)(5.52935396494963,0.550277455345254)(5.55761389638896,0.551014063007879)(5.57611999906774,0.551496277901308)(5.58940382734642,0.551842340448592)(5.61029649403547,0.552386497569363)(5.65715258706288,0.553606323725309)(5.66882008265716,0.553909949894576)(5.68314428760473,0.554282647795312)(5.68720382752354,0.554388259059266)(5.8006900264067,0.557338393367784)(5.80696386082668,0.557501358501293)(5.84052606426861,0.558372927239846)(5.84179070228642,0.558405761045113)(5.84265249227742,0.558428135402863)(5.8463874404647,0.558525101757498)(5.85216942766115,0.558675204096867)(5.86903578292332,0.559112997790983)(5.86997642940901,0.55913741106347)(5.87125882449377,0.559170693517281)(5.87876282252551,0.559365436691605)(5.88137868219089,0.559433318969757)(5.89568711996893,0.559804587549843)(5.89795101384625,0.559863323918478)(5.92295152616976,0.560511848271317)(5.9318987121694,0.5607438934063)(5.9517800311422,0.561259423483397)(5.96532623553827,0.561610609577307)(5.97602778265258,0.561888006656616)(5.99028749258398,0.562257579453614)(5.99397798341926,0.562353216488179)(6.00355408424747,0.56260135592506)(6.00675250963805,0.562684228296576)(6.02285729163321,0.563101460535568)(6.03260167230774,0.563353872404558)(6.03282190168847,0.563359576737215)(6.03517539228269,0.563420535390031)(6.04381723397354,0.563644356290733)(6.07304240120689,0.564401107928716)(6.07690642738493,0.564501142742562)(6.08722101043732,0.564768152057641)(6.13089259276758,0.565898301773895)(6.13507562759817,0.566006521691273)(6.15002890634935,0.566393337254894)(6.17921898492118,0.567148241943307)(6.20098381381541,0.567710952748145)(6.20890304781372,0.567915663060019)(6.21639350452124,0.568109272647555)(6.21845814252504,0.568162635491802)(6.23492963943635,0.56858831489815)(6.27651566357121,0.569662688712191)(6.27725185610956,0.569681703735616)(6.32221412819564,0.570842735206094)(6.34577036753027,0.57145078396971)(6.37611235386426,0.572233761650206)(6.4142972862946,0.57321876442357)(6.41679900964918,0.573283283847998)(6.42815067117817,0.573576021603921)(6.4387365121903,0.573848978645233)(6.46088550611562,0.574419994473078)(6.48120651840765,0.574943767369282)(6.48521127478523,0.575046976667076)(6.49071992988883,0.575188936958538)(6.51550973713928,0.575827680458114)(6.52936271179989,0.57618455054255)(6.53206145745321,0.576254067746403)(6.5422364719452,0.576516149550273)(6.55517113205863,0.576849273481264)(6.56828206253424,0.577186892623225)(6.57978567878514,0.577483085117766)(6.58063394671307,0.577504924771621)(6.59503471536925,0.577875661145052)(6.59879837969502,0.577972544901393)(6.62865282004978,0.578740925584781)(6.65722674370287,0.579476136836113)(6.66169482274955,0.579591082556986)(6.66664208098303,0.579718349788654)(6.66882008265716,0.579774376503093)(6.67712889506664,0.579988100731516)(6.69596157444128,0.580472462446119)(6.70065441882879,0.580593145011375)(6.733479286649,0.581437127127869)(6.74602552372026,0.581759642064228)(6.75843919454755,0.582078711542339)(6.77189557838791,0.58242453979814)(6.81259814981968,0.583470328737227)(6.81780027576426,0.583603960769247)(6.83280741166511,0.583989427691376)(6.84179703195334,0.584220305844426)(6.86903578292332,0.584919756310478)(6.88744052122866,0.585392264873459)(6.89807539036457,0.585665260381496)(6.91660914040788,0.586140956997453)(6.93976555824051,0.586735191408297)(6.94197985274168,0.586792007739506)(6.96556686310927,0.587397155707424)(6.96972294301576,0.587503771110857)(6.99026191467097,0.588030598352523)(6.99397798341926,0.58812590599553)(7.00374966054823,0.588376509932263)(7.02460190926508,0.588911215490566)(7.02632413832155,0.588955373642779)(7.03282826510529,0.589122134376455)(7.03473312956121,0.589170971871438)(7.03494503604774,0.589176404745723)(7.03549122780095,0.589190408005588)(7.07759362550715,0.59026963501429)(7.0809150672589,0.590354758619564)(7.08729709960147,0.590518314003842)(7.11570125514174,0.591246136086726)(7.1272197090711,0.591541234079101)(7.13093400528865,0.591636386877437)(7.13596981805192,0.591765389671138)(7.14403984856118,0.591972109232135)(7.16401993769212,0.592483855292346)(7.16565170378291,0.592525645752309)(7.17238505025209,0.592698084801766)(7.17262096594693,0.592704126365374)(7.19831097955064,0.59336195366307)(7.20543347882133,0.593544311034182)(7.2223323486809,0.593976931821686)(7.28319247329244,0.595534513367256)(7.32071472829001,0.596494450621326)(7.32957251416708,0.596721020900603)(7.34956322665466,0.597232301276001)(7.35208973793918,0.597296913641831)(7.36652911900157,0.597666159527282)(7.37613360797217,0.59791174506622)(7.38291358968089,0.59808509787948)(7.38850588712921,0.598228077107456)(7.39064904333862,0.59828287000019)(7.40665237177917,0.598691991223205)(7.43106506732143,0.599316004573856)(7.44963485264998,0.59979059398588)(7.45897669526456,0.600029320443077)(7.49713259060739,0.601004213449905)(7.51128042190032,0.601365628202073)(7.52599239420278,0.601741416799889)(7.54327041756475,0.602182701736164)(7.55210314074942,0.602408271569517)(7.57483845878995,0.60298882349425)(7.5860015420779,0.603273843111404)(7.5986701021837,0.60359727537452)(7.60130653196232,0.603664580823719)(7.61245241196242,0.603949111173133)(7.62663854701965,0.604311222512216)(7.63545700285608,0.604536303019931)(7.63647597352993,0.604562310190218)(7.64687088240826,0.604827609425969)(7.65762238041811,0.605101990759595)(7.66339380372388,0.605249271305679)(7.67477062917624,0.605539579743712)(7.67724598882462,0.60560274199958)(7.68158433909921,0.605713438656249)(7.68595728982,0.605825015079)(7.71758770586558,0.606631977517534)(7.73350295056609,0.607037950385008)(7.73777954175208,0.6071470326565)(7.76200762253151,0.607764959786015)(7.7635563706854,0.60780445683904)(7.77178736803964,0.608014362133341)(7.77406876084965,0.608072539906722)(7.81484372915884,0.609112206780133)(7.82455435629052,0.609359767792129)(7.83472325077126,0.609618996531738)(7.84179855390166,0.609799353276186)(7.85488442578486,0.610132905876816)(7.8553155188593,0.610143893798738)(7.8708699845903,0.61054033559928)(7.87953547392778,0.6107611803057)(7.91199233434952,0.611588263580169)(7.9155814319507,0.611679713535427)(7.93542244536036,0.612185227829286)(7.96649194176762,0.612976711902131)(7.97017538507593,0.613070537276016)(7.98731405106194,0.613507071813114)(7.98899235639144,0.613549817305206)(7.9922958258573,0.613633953676718)(7.9940104414929,0.613677622788653)(8.02323520625137,0.614421878966965)(8.0255971802531,0.614482025396457)(8.02992916468972,0.614592335147153)(8.03290242174527,0.614668044766161)(8.03482502023162,0.614717000275138)(8.04303732253342,0.614926106171316)(8.0564856894452,0.615268515957271)(8.05705222597726,0.615282940053208)(8.06108868934475,0.615385707716464)(8.07936480609797,0.615850987531377)(8.08110084659158,0.615895181973744)(8.08121005151561,0.615897961994019)(8.08761106110591,0.616060909210409)(8.09044043741296,0.6161329334836)(8.09400791188707,0.616223745209163)(8.09743960217356,0.616311098921216)(8.12875178667721,0.617108080586672)(8.13402505090281,0.617242287364789)(8.16565763591117,0.618047275955378)(8.17845425996071,0.618372889828948)(8.19242626127477,0.618728388334521)(8.19652772478223,0.618832739907796)(8.19807820874422,0.618872187591277)(8.20310756812673,0.619000143402884)(8.24513546770782,0.620069287608348)(8.25053246287345,0.620206565944611)(8.25492312567578,0.620318244588448)(8.26172591775692,0.620491272403318)(8.27347555932941,0.620790109608892)(8.28624914420343,0.621114971158719)(8.2893770541712,0.621194518182871)(8.29556260502206,0.621351821851022)(8.31348152335585,0.621807489833711)(8.32232605291717,0.622032387667832)(8.32541543913575,0.622110942179458)(8.34941591227132,0.622721170803775)(8.37677314947021,0.623416668939256)(8.37997332193617,0.623498020945566)(8.42533145875871,0.624650956327375)(8.43482770296675,0.624892308538797)(8.4471931153579,0.625206567906759)(8.45203150945306,0.625329528341786)(8.46138016996402,0.62556710340374)(8.47693104010109,0.625962273460307)(8.48532977905488,0.626175687156674)(8.48898086451042,0.62626845973705)(8.55829638036951,0.628029483642384)(8.582502922576,0.628644361336593)(8.60721621369946,0.629272053134301)(8.61002719041064,0.629343445343155)(8.62701572837674,0.629774898554697)(8.63941741832399,0.630089843898393)(8.64697905679143,0.630281867447363)(8.65734895671771,0.630545196703998)(8.66715301969035,0.630794148323286)(8.68020582433457,0.631125580809004)(8.68159161049135,0.631160767334541)(8.69499434022094,0.631501067504033)(8.70617119640859,0.631784839784721)(8.70742583433364,0.631816693447096)(8.73040494704162,0.632400079652114)(8.74833799807465,0.632855326461132)(8.76807983997862,0.633356459492526)(8.77075595843822,0.633424388391568)(8.78017808544808,0.633663548955397)(8.81269082683146,0.634488759252955)(8.84258095444303,0.6352473292053)(8.86920403699037,0.635922926669042)(8.9289190720963,0.637438080848187)(8.93333464999735,0.637550107084783)(8.9415788237862,0.637759263609924)(8.94206527812347,0.637771604910264)(8.94218237135149,0.637774575551775)(8.96772261172604,0.638422504861793)(8.97018354744599,0.638484933777299)(8.97277619791265,0.638550703563159)(8.97399705464245,0.638581673823188)(9.00314120863768,0.639320961399224)(9.0062533943811,0.639399903435462)(9.02060871804801,0.639764024314928)(9.02141090078307,0.639784371160249)(9.02559826983031,0.639890580357026)(9.03494576802855,0.640127667759532)(9.06116994775193,0.640792779712072)(9.07738392808263,0.641203985011807)(9.09054197533455,0.641537675883869)(9.13528945367456,0.642672400036615)(9.13729715868082,0.642723309317403)(9.14053099067183,0.642805308928357)(9.14370420737727,0.642885770915524)(9.14796323001642,0.642993764296224)(9.15817153843053,0.643252605583476)(9.16571262419528,0.6434438129462)(9.18218862522636,0.643861557168444)(9.2155593596881,0.644707614500952)(9.21629745785516,0.644726326991823)(9.22309318673851,0.644898612962517)(9.22705016492173,0.644998929469343)(9.24296345746357,0.645402351312161)(9.28372844265426,0.646435732260164)(9.33942654775011,0.647847524417296)(9.36799884816114,0.648571693317656)(9.37344539743324,0.648709732441093)(9.39471155650073,0.649248695667836)(9.39708551961547,0.649308859390917)(9.40452605280604,0.649497424339519)(9.41073635118052,0.649654809610462)(9.45204025630332,0.650701515682853)(9.46138452826576,0.650938304061772)(9.46704766460831,0.651081808892974)(9.47140108134026,0.651192124298816)(9.4857309734926,0.651555237670653)(9.48575497434138,0.651555845834903)(9.48924692010956,0.651644328981134)(9.49811247362638,0.651868972890959)(9.52938569113176,0.652661378885774)(9.53657018579421,0.652843415462428)(9.54787409590901,0.653129823459261)(9.55221150300425,0.653239719391681)(9.57486190632385,0.653813596475499)(9.58241797560796,0.654005035085068)(9.60648899256734,0.654614878748957)(9.61089052003069,0.654726390099318)(9.63295979917439,0.655285498711234)(9.68161567347146,0.65651810317487)(9.6910927784675,0.656758179177538)(9.69151921502932,0.656768981694476)(9.69363261647664,0.656822518428952)(9.69928984808546,0.656965826902243)(9.69962682066624,0.656974363031135)(9.70797774135184,0.657185905996747)(9.71147326225968,0.657274452862834)(9.72033500229215,0.657498932663359)(9.72529852910652,0.65762466450614)(9.72784210768408,0.657689096009302)(9.72785090060297,0.657689318742811)(9.72977436252897,0.657738041923024)(9.73091751671349,0.657766999092872)(9.73125972717019,0.65777566759831)(9.74010728564041,0.657999783267048)(9.77307351332047,0.658834825418926)(9.78079516850711,0.659030412628708)(9.78088850439303,0.659032776789427)(9.781954505066,0.659059778145178)(9.78841371537213,0.6592233866566)(9.78865096964742,0.659229396166813)(9.79920209530629,0.659496648517561)(9.84251710091916,0.660593758224829)(9.85946933925981,0.661023123508747)(9.89329181861984,0.661879758872096)(9.90711775131721,0.662229927030769)(9.93209897222901,0.662862614805241)(9.94315459130315,0.663142611631112)(9.95251344181596,0.663379634045941)(9.97421452173617,0.663929230159387)(9.980376632542,0.664085288797616)(9.9865157198383,0.664240763740688)(10.0068841417905,0.664756598257587)(10.014089722187,0.664939079561129)(10.0256002340475,0.665230581658389)(10.0270346088403,0.665266906864895)(10.0278232740964,0.665286879615779)(10.0384456330091,0.665555887355257)(10.0532341646169,0.665930399479108)(10.0738714045446,0.666453022444252)(10.0970131819583,0.667039064811866)(10.140555573969,0.66814171641171)(10.1420004523868,0.66817830565795)(10.145432650999,0.668265220546674)(10.1555346797301,0.668521037488654)(10.159773120794,0.668628368631851)(10.1809369445787,0.66916430336426)(10.1827265070734,0.669209620561779)(10.2034349288295,0.669734019409224)(10.2353418316563,0.670541991407936)(10.2533318909145,0.670997547372155)(10.255039002647,0.671040775876359)(10.2594118213386,0.671151506941075)(10.2920840740181,0.67197885009726)(10.2971884531642,0.67210810536869)(10.3244566949432,0.672798602086803)(10.3360625544573,0.673092489472201)(10.3374890340935,0.673128611238917)(10.3480945635276,0.673397167683915)(10.357669335779,0.673639622836708)(10.3602871247683,0.673705911227139)(10.3983569363357,0.674669925226112)(10.4116673548791,0.675006975216947)(10.4310751814248,0.675498425665011)(10.4471889867,0.675906464368493)(10.4652912925678,0.676364857100964)(10.4857414377482,0.676882703632947)(10.515973394092,0.677648251545313)(10.5193302536782,0.677733255753994)(10.5219084805042,0.677798543036984)(10.5292837147129,0.677985302952351)(10.5482843621625,0.678466449447339)(10.5486280594759,0.678475152786305)(10.5648273267518,0.678885362458898)(10.57699393254,0.679193455113748)(10.5907067906414,0.679540704306069)(10.6167406969539,0.680199961463706)(10.6495907378915,0.681031830815695)(10.6590803243151,0.681272139455767)(10.6799642375279,0.681800993959842)(10.680798866228,0.681822129788905)(10.7138246185393,0.682658466961166)(10.7200026354714,0.682814918928038)(10.7403952731105,0.683331344402259)(10.7584833865928,0.683789413522225)(10.7617545273143,0.683872253308141)(10.7744361600523,0.684193409606132)(10.7809785259687,0.6843590926163)(10.8014720837841,0.684878087664926)(10.8095155101541,0.685081787119988)(10.8223390795327,0.685406545147631)(10.8235809802353,0.68543799650111)(10.8389678222222,0.685827672593109)(10.8411283998467,0.685882390069731)(10.847628688808,0.686047012803647)(10.8530073599677,0.686183230479555)(10.8552389739775,0.686239747389386)(10.8712106679395,0.686644241664897)(10.8754784137611,0.686752325880094)(10.8794887380765,0.6868538909066)(10.8848929302015,0.686990757221181)(10.8933970622414,0.687206133281501)(10.9402415717996,0.688392537900005)(10.9626178017763,0.688959259395899)(10.9703490487066,0.689155069986764)(10.972730790131,0.689215392934591)(11.0092723840833,0.690140901752143)(11.0269948137457,0.690589774877355)(11.0317426354223,0.690710028398714)(11.0331339727667,0.690745268462612)(11.0343368094469,0.690775734169881)(11.0371382344906,0.690846689358509)(11.0574162818413,0.691360300548409)(11.0833833823022,0.692018016339497)(11.0884350428271,0.692145970188366)(11.0950432373357,0.692313350231193)(11.1152899022521,0.69282618587194)(11.1456418333022,0.693594994681027)(11.1506870070511,0.693722789520109)(11.1522362353419,0.693762031740672)(11.1545516353421,0.693820681299768)(11.159847277434,0.693954821350662)(11.1628674482422,0.694031323312155)(11.1784802345913,0.694426803043321)(11.206053981164,0.695125270200019)(11.211280221094,0.695257656922154)(11.2167852734428,0.695397106799848)(11.2169725098915,0.695401849743098)(11.2222622134221,0.695535845079146)(11.2540089811653,0.696340043831424)(11.2656780675848,0.696635645733521)(11.3107262042017,0.69777682972093)(11.3359231366917,0.698415147540479)(11.3446897915946,0.698637237150158)(11.3481238135571,0.69872423313343)(11.3562806347066,0.698930875221677)(11.3567355367255,0.698942399585877)(11.3603044089887,0.699032812549772)(11.400732693621,0.70005702829587)(11.4080815240814,0.700243207551905)(11.4351912043361,0.700930026689714)(11.4397490925238,0.701045501206618)(11.4486865184183,0.701271932704052)(11.4577517051398,0.701501602379425)(11.459068078391,0.701534953266979)(11.4762753269083,0.701970908943433)(11.4798319641411,0.702061019013217)(11.5091849438028,0.702804706365133)(11.5133939526689,0.702911346950516)(11.529763373902,0.703326089603075)(11.5489469223638,0.703812137318785)(11.5548076367122,0.703960629536143)(11.5657318060942,0.704237415241614)(11.568720455608,0.70431313896122)(11.6073174659537,0.705291086640572)(11.6775064070059,0.707069541672582)(11.7011271735748,0.707668060468464)(11.7059248116606,0.707789626982592)(11.7062689297639,0.7077983465406)(11.7079794836211,0.707841690031845)(11.7091232370512,0.707870671461505)(11.7131759046472,0.707973361625428)(11.7138757019216,0.70799109374023)(11.7300204581455,0.708400185988013)(11.7341110689241,0.708503838454229)(11.7414199647715,0.708689039823476)(11.7487335335727,0.708874360102954)(11.7503058587462,0.708914201692921)(11.758423621204,0.709119900312202)(11.7586677276456,0.709126085813617)(11.7731118431599,0.709492091385057)(11.7812727927531,0.709698885907052)(11.7935850841235,0.710010874399274)(11.7980589792211,0.710124241371588)(11.8112884284275,0.710459471904039)(11.8197107499042,0.710672891764907)(11.845153143532,0.711317599193143)(11.8505494168149,0.711454340590739)(11.8715289527605,0.71198596258047)(11.8731979156235,0.712028254215979)(11.8871323184361,0.712381353306224)(11.8881737465665,0.712407743216689)(11.8931518175737,0.712533888155801)(11.9441291146298,0.713825661326241)(11.9525601010607,0.714039303979736)(11.9572734319847,0.714158740567167)(11.9703736320037,0.714490701620131)(11.983855474417,0.714832333138898)(12.0093497917966,0.715478360515815)(12.0135981094767,0.715586012834887)(12.0209524051198,0.715772370372198)(12.0234199491311,0.715834897763707)(12.0372501195462,0.716185352608345)(12.0574316039119,0.71669674653738)(12.0699659642032,0.717014362491676)(12.1071415925043,0.717956369497885)(12.1180743040662,0.718233394420564)(12.1247147364386,0.718401656131832)(12.1339029346715,0.718634474577336)(12.1465258810459,0.718954323557454)(12.1530446507255,0.719119499683576)(12.1859640691718,0.719953618474079)(12.1896987779431,0.720048247983248)(12.1987132374007,0.720276653884066)(12.2013091433912,0.720342427924598)(12.2133807961565,0.72064829278233)(12.2274181899911,0.721003960362606)(12.2324045214491,0.721130298678209)(12.2540716047364,0.721679268732654)(12.2769903788268,0.722259938607757)(12.2856634787161,0.722479676232602)(12.2885590383786,0.722553036278593)(12.354410082038,0.724221320605319)(12.3647367733356,0.724482924767325)(12.3730496232811,0.724693509533442)(12.3883508995665,0.725081120620837)(12.394020729883,0.72522474588077)(12.4023109908657,0.725434747978971)(12.4061056397365,0.725530869873475)(12.4075948083846,0.725568591683555)(12.4344264556926,0.726248240264426)(12.4552966930706,0.726776860755037)(12.4849368993198,0.727527574609479)(12.4858108270377,0.727549708302314)(12.4886959988238,0.727622779824519)(12.4918171472203,0.727701827273947)(12.5019491350608,0.727958430023984)(12.5097993776701,0.728157240983366)(12.5462498028592,0.72908031279724)(12.5499682316233,0.729174473416955)(12.5587755197299,0.729397493792726)(12.609138307557,0.730672683128072)(12.6144373395058,0.730806843503379)(12.6431764558602,0.731534416819424)(12.6505316861709,0.731720613913444)(12.6644462526905,0.732072846806979)(12.669529696392,0.732201524508746)(12.677868473091,0.732412599523639)(12.6963074377985,0.732879311494974)(12.7086190734831,0.733190914782537)(12.7265318051589,0.733644252062185)(12.7266227006833,0.733646552369512)(12.7304453381032,0.733743291655743)(12.7311210215434,0.733760390978319)(12.7380551701527,0.733935868587013)(12.7441819875365,0.734090911258256)(12.7443428791593,0.734094982658326)(12.7604428655599,0.73450238187183)(12.7845908112984,0.735113372992927)(12.8101467179526,0.735759911210395)(12.8109265054555,0.735779637741217)(12.8122964772479,0.735814294161341)(12.8159663980217,0.735907131607313)(12.8485734502274,0.736731909058843)(12.8516664479989,0.736810137216547)(12.8552986144984,0.736902000330115)(12.8702359841822,0.737279769823002)(12.892111599696,0.737832950285328)(12.9103645627498,0.7382944671929)(12.9331727201218,0.7388710856932)(12.9344913048301,0.738904418556966)(12.9463210375895,0.739203452625372)(12.9607012771446,0.73956692680875)(12.989158183029,0.740286094615569)(12.9967129699365,0.740476995699772)(13.0078857476429,0.74075929974284)(13.0126933017948,0.740880765518293)(13.07497032852,0.742453813310424)(13.0837316996084,0.74267505019479)(13.0948529625973,0.742955853154765)(13.1002703837125,0.743092628603524)(13.1031076515458,0.743164259388704)(13.107825966995,0.743283375751825)(13.1119102343148,0.74338648107711)(13.1168595521807,0.743511418963771)(13.1314018400677,0.743878483015318)(13.1358268303972,0.743990164773164)(13.1523369281818,0.744406818563662)(13.1528315304513,0.744419299449174)(13.1663657717957,0.74476080095087)(13.1686526761066,0.74481850042381)(13.1723986044962,0.744913008707538)(13.2368396280764,0.74653823621253)(13.2393624220657,0.746601838232086)(13.2557609680432,0.747015215115463)(13.2658096891723,0.747268484673518)(13.2867725093457,0.747796734213945)(13.293891644482,0.747976100160599)(13.302290623187,0.748187690560774)(13.3272995418244,0.748817586761953)(13.3343048601324,0.748993990728509)(13.3375677128258,0.749076148244476)(13.3463014514184,0.749296042471896)(13.3704543117774,0.749904010719277)(13.3805539333184,0.750158171925392)(13.3845227059246,0.750258037308526)(13.4109802690701,0.750923629308841)(13.4385591035105,0.751617134408066)(13.4437467071268,0.751747548599174)(13.4454162436376,0.751789517665438)(13.4491972917216,0.75188456191587)(13.4672388229256,0.752337988494672)(13.4698575603395,0.752403791946981)(13.4741003736319,0.752510398712158)(13.4815808873293,0.752698338049728)(13.4950395354079,0.753036408305444)(13.4964986587729,0.753073055378039)(13.5028374439794,0.753232247841467)(13.5374789579157,0.75410190395028)(13.5423394033624,0.754223876708113)(13.568843013496,0.754888777502596)(13.5735599227509,0.75500707411451)(13.5773090804369,0.755101092029934)(13.5802761746941,0.755175492900849)(13.588726193232,0.755387354630753)(13.5966444887762,0.75558585049928)(13.59992856546,0.755668165902585)(13.610306082879,0.755928240058569)(13.6108151387697,0.755940996152902)(13.6144374076044,0.756031760073499)(13.6155415898179,0.75605942635093)(13.6206953812178,0.756188550285183)(13.6278356999769,0.756367420493953)(13.6374695017502,0.756608708629109)(13.6493887468193,0.756907164274936)(13.6513016815345,0.756955056113899)(13.6533252703769,0.757005715933563)(13.6692591747627,0.757404530806065)(13.6799623642338,0.757672338726102)(13.7343168701222,0.759031250660753)(13.7383468022591,0.759131926067366)(13.7543109344304,0.759530632650467)(13.7576722804182,0.759614560529863)(13.7585623014898,0.759636781741649)(13.7692503229447,0.759903587097448)(13.7924169592592,0.760481618000295)(13.8123453075263,0.760978536848944)(13.8230315872202,0.761244879392889)(13.8331393368355,0.761496722229848)(13.8351213560944,0.761546096567644)(13.8353396949168,0.7615515354468)(13.8392056260664,0.76164783065813)(13.8506137506471,0.761931923149099)(13.8637971271445,0.762260094333481)(13.8744354695612,0.762524809202693)(13.8991037567311,0.763138268658256)(13.9042103427949,0.763265196165502)(13.909779885601,0.763403604952274)(13.9527385792112,0.76447024221194)(13.9540348063073,0.764502400478002)(13.9624635669694,0.764711472207033)(13.9664861830386,0.764811227763212)(13.9910069955212,0.765418974398998)(13.9912162713464,0.76542415875741)(14.0217823171426,0.766180894421525)(14.0472228941544,0.766810001074969)(14.0514925243159,0.766915515039334)(14.1003633019448,0.768121806556662)(14.1031476624435,0.768190452002533)(14.1143585330244,0.768466752170953)(14.1355799716399,0.768989359980798)(14.1427412105861,0.769165592093744)(14.1466347100665,0.769261381307138)(14.1493178270543,0.769327381313329)(14.1659000879274,0.769735075390464)(14.1743099195203,0.769941707040224)(14.1765758834532,0.76999736673451)(14.1929066657301,0.7703983072373)(14.1998324649341,0.770568237470643)(14.2163926404618,0.770974292941103)(14.2223445304652,0.771120141671206)(14.2290623058221,0.771284699121191)(14.2375491276444,0.771492500469293)(14.2493491356643,0.771781255990491)(14.2633157741964,0.77212277125856)(14.2658210017534,0.772183999479051)(14.2671135652837,0.772215586352325)(14.2710264006262,0.772311190758942)(14.283824079422,0.772623723853022)(14.3171117720226,0.773435473982318)(14.3199781382439,0.773505291936622)(14.3217950403837,0.773549540643799)(14.322612468108,0.773569446504816)(14.328980990692,0.773724495053653)(14.3403061553626,0.774000056615907)(14.3578675700245,0.77442694098552)(14.3665354500698,0.774637450690989)(14.4009857613399,0.775472841395147)(14.428528184182,0.776139206756628)(14.4445734262513,0.776526764659557)(14.4514528180097,0.776692781616934)(14.4555440711997,0.776791471064677)(14.4615807592598,0.776937029735315)(14.4768852183906,0.777305738996026)(14.4918989305077,0.777666994527571)(14.5007009824697,0.777878576400117)(14.5082988681204,0.778061085871241)(14.5228446029657,0.778410157621378)(14.538552963156,0.778786631203298)(14.5405796554001,0.778835165522605)(14.5664787278814,0.779454599355116)(14.577337927177,0.779713880173907)(14.5901916165688,0.780020438453133)(14.6376941451985,0.781150025264269)(14.6513241391607,0.781473137373723)(14.6551291908994,0.781563257710563)(14.6576891998403,0.781623869633607)(14.6623171440859,0.781733401218727)(14.6983321583306,0.782583914950363)(14.7086547462402,0.782827064078328)(14.731621220387,0.783367013172044)(14.7520153723356,0.783845270585838)(14.7744956318311,0.784371086909716)(14.778940329606,0.784474876559798)(14.7803705380656,0.784508261595036)(14.7848154233734,0.784611979323614)(14.7911293179355,0.784759208887785)(14.7989801992137,0.784942113551725)(14.8149291619354,0.785313112189336)(14.8158578071983,0.785334690212295)(14.826096831004,0.785572429097521)(14.8276360396215,0.785608139892192)(14.8353578619484,0.785787180882971)(14.8428854523246,0.78596153881981)(14.8436115564736,0.785978347786177)(14.8453573945156,0.78601875628795)(14.85424406325,0.786224293727966)(14.8882693681773,0.787008896164764)(14.8969424536531,0.787208281161414)(14.9042196146061,0.787375380026224)(14.9286741576591,0.787935580932292)(14.9370821628438,0.78812771037218)(14.952696588005,0.788483848332901)(14.9806207997553,0.789118556248093)(15.0272397906451,0.790171696923104) 
};
\addlegendentry{$B = 1$};

\addplot [
color=magenta,
solid,
line width=1.0pt,
]
coordinates{
 (2.47010852067517,0.49908536152778)(2.51424471534462,0.50162274560472)(2.52083517369182,0.50200122053239)(2.78194313663416,0.516914017607029)(2.81388340783146,0.518727728066091)(2.86335435247593,0.521532622871038)(2.86760245529177,0.521773241714006)(2.89724120986952,0.523450986512687)(2.9111913388668,0.524240030585756)(2.91466314462933,0.524436340604733)(2.96117568755077,0.527064012065473)(2.98566411992226,0.528445732400661)(2.98635710849483,0.528484816072158)(3.02604478248885,0.530721598511571)(3.06785007894498,0.533074478673673)(3.08631831257808,0.53411286230869)(3.1184752948018,0.535919401349359)(3.13159076448954,0.536655671360718)(3.13855493788234,0.537046496706884)(3.1521303236888,0.537808089500107)(3.16491642507124,0.538525102323466)(3.16977271733204,0.538797355229331)(3.18931331721795,0.539892418568233)(3.19383099099058,0.540145495727285)(3.20448584247273,0.540742232953458)(3.20975894471383,0.541037486308186)(3.22290284278728,0.541773235205007)(3.23482907364747,0.542440566911341)(3.26618745173596,0.544194070460044)(3.29313344576611,0.545699522963573)(3.36053835385118,0.549460162536225)(3.36491588331636,0.549704139114426)(3.41701724712073,0.552605626107549)(3.4928576823584,0.556821645972661)(3.64226568336793,0.565102716507441)(3.68821284028606,0.567643159992166)(3.6929369266544,0.567904198383167)(3.7061732840845,0.568635442860851)(3.72326942568561,0.569579584658296)(3.83272397588671,0.575615523602903)(3.86030599837799,0.577134247730665)(3.88101646244141,0.57827402160827)(3.9051450830183,0.579601284175704)(3.92027159377075,0.58043302208288)(3.95283559168052,0.582222697948852)(3.9700736923598,0.583169608389864)(3.9907823398773,0.584306733969085)(3.9936537177029,0.584464366851623)(4.00035488497597,0.584832213272025)(4.01511331295033,0.585642177994787)(4.02537464362474,0.586205200062939)(4.02733782025468,0.586312903725575)(4.03616912262743,0.586797356387168)(4.03846973435597,0.586923546158684)(4.07459662946537,0.588904412035435)(4.08023124361737,0.589213242572109)(4.08286469982074,0.589357570088303)(4.12567086079994,0.591702607608907)(4.1509148193516,0.593084696761926)(4.16876044338707,0.594061361740046)(4.17045416536636,0.594154040881833)(4.21494315490241,0.596587476396302)(4.41752486255438,0.607645857683981)(4.45264877707284,0.609559685649711)(4.5821806549725,0.616609465047878)(4.64344340690031,0.619939491922883)(4.64350739550922,0.61994296877999)(4.65575066638336,0.620608163496812)(4.6878222646546,0.622350191417534)(4.69812922228912,0.622909890935059)(4.70444081769683,0.623252596381717)(4.79550077092798,0.628194188878003)(4.80694550267537,0.628814911445587)(4.82032279840814,0.629540352683503)(4.82165840561693,0.629612776021448)(4.8642490428154,0.631921721249473)(4.87856742405021,0.632697728092971)(4.8812590461547,0.632843592095358)(4.89711956695898,0.633703022773753)(4.92234317453373,0.635069529212668)(4.92641553912721,0.635290120566491)(4.98102293809979,0.638247251749748)(5.00069376789458,0.639312104588898)(5.01753128590888,0.640223426873014)(5.04469921038366,0.641693584375298)(5.08478808668159,0.643862296592355)(5.10631942696923,0.645026780003142)(5.12745200192831,0.646169490406578)(5.17766447513722,0.648883848572614)(5.2033902541878,0.65027409191602)(5.21581467690912,0.650945417187773)(5.22438791117215,0.651408614265825)(5.25071409677998,0.652830780727928)(5.43552055707481,0.662806435808621)(5.4612724782004,0.664195471276025)(5.49755509135498,0.666152118942149)(5.55168850738656,0.66907056483029)(5.5602722711509,0.669533240627238)(5.58229932566359,0.670720412149859)(5.58709599092478,0.670978911735316)(5.59350918900852,0.671324516612753)(5.6118285180413,0.672311662361422)(5.64357913572551,0.674022293859811)(5.72030405879448,0.67815463428696)(5.78676779202122,0.681732757000062)(5.82564490731805,0.683825067330658)(5.82876557060206,0.683992995595205)(5.84060889809152,0.684630276559749)(5.84246906483946,0.68473036664284)(5.84839669427034,0.685049307354873)(5.8726122472164,0.686352124610464)(5.91536519925335,0.688651798331496)(5.92547865734355,0.689195710751728)(6.04811420327393,0.695788412323873)(6.12300069328943,0.699811588045893)(6.13856382084949,0.700647436333134)(6.14315578898759,0.700894039500374)(6.14437772467547,0.700959659949419)(6.17295614783995,0.702494216301426)(6.18303009635293,0.703035074666468)(6.1839867681618,0.703086435166832)(6.2034677440742,0.704132224297217)(6.27811207051363,0.708137880547032)(6.31650492898075,0.710197228336772)(6.33597875980452,0.711241525293394)(6.35018032566801,0.712002981404685)(6.40169167960824,0.714764079645856)(6.40545601538228,0.714965802226602)(6.42898309349942,0.71622639949834)(6.46155687807236,0.717971233719813)(6.50412782902239,0.720250666506836)(6.55202687691075,0.722814079280904)(6.55319697203153,0.72287668102148)(6.56400236926271,0.723454742573205)(6.57958935542713,0.724288469995999)(6.61300660809202,0.726075349932043)(6.61616469859508,0.726244177560285)(6.74152795661758,0.73293961703513)(6.75940484057871,0.733893291145183)(6.76169725803135,0.734015562513972)(6.77997838163451,0.734990447395125)(6.82051212706772,0.737150813177771)(6.85025434258517,0.738734903572808)(6.8514898259026,0.738800684968704)(6.8523184739502,0.738844804079711)(6.91688970521357,0.742280215736895)(6.92555851537279,0.742741029911141)(6.99156776536617,0.746246547772479)(7.02406713694637,0.747970107948234)(7.02765312569621,0.748160183392098)(7.08717988198218,0.75131221262492)(7.10812051977773,0.752419531630649)(7.12307168608983,0.753209616074839)(7.14352309904563,0.754289629985029)(7.14943744248403,0.75460179773497)(7.16927570068337,0.755648343125553)(7.23714077728886,0.75922165624166)(7.24662419249892,0.759720083378004)(7.26577906426752,0.760726093887143)(7.27854311321007,0.761395902650446)(7.32030560698439,0.763584148163657)(7.38651441451018,0.767041979543656)(7.41323104742907,0.768432923332003)(7.42204336057777,0.768891122871461)(7.46310707177869,0.771022119065275)(7.48909899765423,0.772367241208575)(7.5064741623112,0.773264713187314)(7.50893486919585,0.773391699613042)(7.52942543417631,0.774447981251383)(7.55871146336998,0.775953959240182)(7.56349826776563,0.776199678696232)(7.61686749633714,0.778930439754981)(7.61792720744935,0.778984490563226)(7.62011324324216,0.779095967737251)(7.64830952924558,0.78053110758952)(7.76068359735759,0.786194221044145)(7.79533971035713,0.787919805422111)(7.84571997934065,0.790407851819641)(7.87068304150873,0.791630880922972)(7.8729600569198,0.791742097891525)(7.98731287564943,0.797245451814513)(8.01222742992587,0.798420774985822)(8.03208562509352,0.799350755287449)(8.03250802371555,0.799370468864727)(8.0370186573283,0.79958080340519)(8.08428437309398,0.801764387799898)(8.10059737771466,0.802508927500839)(8.165580618138,0.805424302043876)(8.19508880889847,0.806719478113025)(8.19779544963135,0.806837325581305)(8.19965676499654,0.806918272629739)(8.23030377000429,0.808239732228889)(8.26122404468754,0.809550455992299)(8.31095825027294,0.811608205570389)(8.31503559411748,0.811774016769322)(8.35144791342761,0.813234484915778)(8.36790543087571,0.813882288746797)(8.39792228907714,0.815043397604002)(8.40677038038801,0.815380514588091)(8.42125698225793,0.815927306735061)(8.42159839315419,0.81594011535796)(8.42908883910765,0.816220224111519)(8.47846320276166,0.818022406335791)(8.50577038320264,0.81898549421209)(8.50659472918594,0.819014189825798)(8.54728822271514,0.820402993182772)(8.63044687084559,0.822851082025942)(8.64001847946596,0.823147211938916)(8.66087588443783,0.823782475479752)(8.68423769720313,0.8241999438158)(8.68990174791854,0.82436852620093)(8.70516253825909,0.824752068290435)(8.78175981342456,0.826421116797644)(8.88108560290579,0.828231254044159)(8.91123809710711,0.828667972907062)(8.91232772902897,0.828696805053767)(8.95623604893744,0.829753192087308)(8.96360276536403,0.829867373851067)(8.96466465473631,0.829881531875446)(8.99768321252681,0.830381990759364)(9.03212543890173,0.830956131720001)(9.04850883466003,0.83121333515785)(9.05923331840849,0.831390954636761)(9.06408752197346,0.831472341655569)(9.07249717641255,0.8315819918205)(9.08692046361128,0.83189188813136)(9.16436299389386,0.833229651209754)(9.21946323829423,0.834300230858625)(9.23386940758931,0.834612605360257)(9.23419697469608,0.834616688483538)(9.24207410815615,0.834775748284567)(9.24608994608744,0.834865085432991)(9.2771604702577,0.835396867354619)(9.31690093076087,0.835937078656138)(9.32118871066247,0.836029583044428)(9.33440745682778,0.836285641948071)(9.34177100894893,0.836375234826819)(9.38883863803944,0.837340996102824)(9.46888198990813,0.838890872083835)(9.48232204737126,0.839048156079081)(9.51918710280358,0.839637576792222)(9.56379300505433,0.840401580792767)(9.57894506971866,0.840670917467714)(9.64863632083254,0.841563343413666)(9.7657149645011,0.843279406303291)(9.77560239119789,0.843388798133087)(9.775765627053,0.843390602859011)(9.78553954454967,0.843542334472998)(9.78988214879177,0.843591447902178)(9.79452176994594,0.843642601909921)(9.83237568401423,0.844156832337594)(9.85106446677099,0.844388986125992)(9.9087921902672,0.845135017787491)(9.91030683893299,0.84515389009042)(9.92145568988399,0.845281922559983)(9.94128090609883,0.845521544737282)(9.95151918274824,0.845661820057372)(10.0136994960996,0.846350475903028)(10.0235046691282,0.846467828929605)(10.0491918426706,0.846740735738103)(10.0936796814605,0.847332151532892)(10.1051522509594,0.847459345304033)(10.1054669826583,0.847464016514603)(10.1068336899215,0.84748429449765)(10.1667024076719,0.848361362345121)(10.1689787181193,0.848385054392735)(10.1698634606106,0.848394260817823)(10.2056950546993,0.848865491631668)(10.2361211292726,0.849264161969111)(10.2521923463688,0.84944492804307)(10.3006525306973,0.850122906563239)(10.3173926589516,0.850358187807381)(10.457746591914,0.852198047316243)(10.5083066679625,0.852757076256496)(10.5091665781778,0.852765577839091)(10.5168182364431,0.852841178194698)(10.5330294788672,0.853008768292505)(10.5578320796071,0.853252628389855)(10.5925700872606,0.853677683408251)(10.628441373519,0.854110384262924)(10.6300223955744,0.854125733903076)(10.6500013152122,0.854319385682772)(10.6586580972031,0.854403111814562)(10.7299563887542,0.855258930937492)(10.7317250651457,0.855282083032961)(10.7659861215392,0.855666864076039)(10.7875854558541,0.855915504353534)(10.8094714085242,0.856123588593711)(10.8099892051338,0.856130294297187)(10.810086687038,0.856131556665379)(10.8565998673142,0.856647371752413)(10.9034690104253,0.857144526568324)(10.9078005509529,0.857194540209364)(10.9548546600513,0.857714980081066)(10.9927078032563,0.85808627224006)(11.0073906578306,0.858267393084714)(11.0558804491799,0.858762957994589)(11.0830447959343,0.85907950250706)(11.1070790238327,0.859343195797464)(11.1447104334491,0.859807998501649)(11.1654466860777,0.86003747861496)(11.1916078946802,0.860309315645576)(11.2103060826261,0.86053797693267)(11.2198541745978,0.860654426757058)(11.2361406898084,0.860852566029576)(11.2566594098814,0.861095315822999)(11.3024466586874,0.861572601136054)(11.3255287824387,0.861781852897945)(11.3757639786826,0.862267865620483)(11.3761432325887,0.862272377751276)(11.4259408929703,0.862861948691432)(11.4700916889129,0.863329581106983)(11.4814693580532,0.863429795998915)(11.4964096307144,0.863592453763711)(11.5410180478705,0.863993497767587)(11.6088962540131,0.864677707398893)(11.6105898175912,0.864691820438077)(11.615896118784,0.864736015360738)(11.6391865876961,0.864929563192715)(11.6500144580175,0.865029793705402)(11.6788909289036,0.865291454717197)(11.6852558502027,0.865346591710911)(11.6906453599057,0.865404281921395)(11.7372014639284,0.865802865603949)(11.7653352421492,0.866049233452484)(11.7656994159316,0.866053327561173)(11.7731993349696,0.866137582383191)(11.785195987318,0.866253240713167)(11.8624253813026,0.866955294997681)(11.8776451160027,0.867115398092785)(11.9326557378982,0.867651108442014)(11.9616361366015,0.867967042652074)(11.9637449651202,0.867990026748181)(12.022535105862,0.868626907931593)(12.0761725937419,0.869178673796225)(12.0846620927958,0.869269095304872)(12.0991983972282,0.869423558286294)(12.1537556134162,0.869947101452022)(12.1677071916203,0.870093540102394)(12.2136441561937,0.870485965076751)(12.2252145997503,0.870606325361684)(12.233363721791,0.870671967758839)(12.2549599788746,0.87084976445699)(12.272185044186,0.870979711561681)(12.3231069315915,0.871462514312675)(12.324963103077,0.871481523377171)(12.3602982115198,0.871787537168641)(12.4828230060063,0.872850771333324)(12.5006243204726,0.872980748121135)(12.5290581818553,0.873193227685973)(12.5377082220016,0.87327621434236)(12.5591628008339,0.873484711333148)(12.6245656896102,0.873997711797037)(12.6348396495369,0.874077984975374)(12.651059364318,0.874229850417353)(12.7009095127402,0.874665894912835)(12.7254707092871,0.874867470860076)(12.7298698392601,0.874898571401881)(12.7564372583718,0.875142909185252)(12.7933609254833,0.875470706245473)(12.8158693527755,0.875661838487788)(12.8256192916266,0.875745828134951)(12.8404310342176,0.875886698952581)(12.8641225106341,0.876111180868136)(12.8702436128153,0.876169008885779)(12.8943445188411,0.876396006756676)(12.9282297407215,0.876700675242062)(12.9289138923829,0.876707061114602)(12.9528629675936,0.876917845316717)(12.9556324184948,0.876936782879386)(12.9607500317707,0.876971761962067)(12.9836383702035,0.877130456632914)(13.0125431717021,0.877390336679351)(13.0322279814937,0.877523779573006)(13.1234781754379,0.878188443033942)(13.1820905840097,0.878638736677326)(13.2159659429976,0.878928324380904)(13.2263791619465,0.879021256654234)(13.2338831011534,0.879071093013881)(13.2509514958988,0.879220834743071)(13.2796718739838,0.879448563801525)(13.2866125859964,0.879506557731327)(13.3370765948993,0.879871492748989)(13.341693928849,0.879911729741927)(13.3458087270773,0.879941651224637)(13.3621345434885,0.880049975001848)(13.3794274884727,0.880168918185597)(13.4585553873332,0.880764896743358)(13.459196339706,0.880769017897539)(13.508499710657,0.881187366025807)(13.517265348638,0.881261986307381)(13.5251036783263,0.881326162273158)(13.5610468715826,0.881629942325682)(13.6049353320679,0.881924168149821)(13.6539917865266,0.882273713492471)(13.6678399667693,0.88237447211053)(13.6946880124559,0.882571770321471)(13.7174864117888,0.882759810225382)(13.7226354715894,0.882802150847108)(13.7531484469981,0.883039293267064)(13.7940319444593,0.883341150902701)(13.8018508544956,0.883404515130412)(13.8205163059102,0.883533695369513)(13.8408631362619,0.883658818835747)(13.8504199081389,0.883717515568462)(13.8624921446209,0.883799955896973)(13.9080343082407,0.884123246133736)(13.9230900283565,0.88421506688524)(13.9435714165139,0.884372777845593)(13.9477999923836,0.884405007220541)(14.0490143543843,0.885085757280128)(14.052108290783,0.885108929591854)(14.0542799973198,0.885125876119596)(14.0555522827157,0.885135800392812)(14.0880072247615,0.885352546495295)(14.1149729900561,0.885534764759899)(14.1166166991278,0.885547472338915)(14.1555552949978,0.885847176468695)(14.1562134653336,0.885852220518094)(14.2246118317839,0.886340736534948)(14.2273467396415,0.88635719943691)(14.2380220648391,0.886437910937081)(14.3120428105709,0.886913675231343)(14.3775423193415,0.887361474968609)(14.4025100084846,0.887535484085865)(14.4499613210505,0.887869416531309)(14.4948893985252,0.888157229961182)(14.5085151705238,0.888237444947624)(14.5125953081666,0.888267036747086)(14.5129627112437,0.888269700179932)(14.5415365396229,0.888456877542085)(14.5466413793325,0.888489295447853)(14.5778862120409,0.888687485507701)(14.5979048894086,0.888830983909609)(14.620384122097,0.888967458993234)(14.6521611964597,0.889162186432278)(14.6670110891642,0.889246684851923)(14.6706777506504,0.889267537316596)(14.6839981612319,0.88934454131011)(14.7109851731915,0.889531067174936)(14.7262364764494,0.889634007153056)(14.7700303607943,0.889913446198061)(14.7794863906809,0.889969487893211)(14.7864144260036,0.890017814025117)(14.8223040057823,0.89023150700446)(14.850971124455,0.890419403223892)(14.8771026193025,0.89058671483752)(14.8935980755739,0.890687645203059)(14.8940306021562,0.890690063970434)(14.9620664302594,0.891084326443486)(14.9654205518772,0.891107203588166)(15.0466071551798,0.891608221648546) 
};
\addlegendentry{$B = 5$};

\addplot [
color=green!50!black,
solid,
line width=1.0pt,
]
coordinates{
 (1.89980336604506,0.492042159689038)(2.03982467383612,0.518556165487532)(2.05260512658551,0.52095966889694)(2.07361017814553,0.52490441949047)(2.0874707925148,0.527503813826932)(2.12582646921603,0.534682561279333)(2.1518033096882,0.539533001845554)(2.16266029671321,0.541557627139844)(2.21914957237188,0.552068479046307)(2.25384249192977,0.558505667090944)(2.26113671943966,0.559857473508547)(2.30601244712953,0.568162567090588)(2.30871816610276,0.568662707091249)(2.319577112662,0.570669279918687)(2.32468646720603,0.57161305668092)(2.33720167888682,0.573923871184462)(2.37739166392048,0.581336028038711)(2.40618700049223,0.586639307452975)(2.43050806675703,0.591114215566548)(2.43176671083894,0.59134569596924)(2.4320670374892,0.59140092837151)(2.43418619237118,0.591790641829909)(2.46061907158318,0.596649439104519)(2.47258166585961,0.598847071081512)(2.48268464975622,0.600702486182097)(2.48668053132439,0.60143618610505)(2.49270420921719,0.602542066708097)(2.52726488811676,0.608883690502622)(2.52980346179468,0.60934928731983)(2.59823924935453,0.621891469607953)(2.8428123779318,0.666632247065133)(2.8951549789253,0.676208629131098)(2.92038565281646,0.68082707357198)(2.9859475373729,0.69283837261446)(3.02174841102614,0.69940526631819)(3.04608609183557,0.703873395833889)(3.05835928612725,0.706127938685884)(3.08886232656195,0.711735376426454)(3.13478982987821,0.720190441861693)(3.14794871214974,0.72261582196728)(3.19588381762494,0.731462538866246)(3.19761461433623,0.731782315139449)(3.26383115845814,0.744034525059115)(3.26623464197607,0.744479900836646)(3.28143265805611,0.747297154435278)(3.29795653283975,0.750362044118719)(3.31190105288061,0.75294989433017)(3.31737072191076,0.753965281840717)(3.35003823866318,0.760032695495148)(3.35729366922405,0.761380796714209)(3.36809123410584,0.763387251446313)(3.37297805709399,0.764295391560495)(3.38204901171612,0.765981101146719)(3.40465132682752,0.770180876253019)(3.40669777854829,0.770561046987593)(3.43815877986875,0.776401928450896)(3.44156964170881,0.777034594216319)(3.46325776625043,0.781053443865862)(3.4918327094354,0.786333744087651)(3.50572983128755,0.788893311318489)(3.77647274515402,0.834317596481309)(3.83798580770984,0.84183681160932)(3.87492237910685,0.843383054145004)(3.88048767092901,0.843636873958689)(3.9217089832807,0.846034204302883)(3.9469443109712,0.847403917460818)(3.96774940454036,0.848675143661495)(3.99355295078366,0.850603113802683)(4.06352292587419,0.854339357920128)(4.08936886399495,0.855498735947265)(4.09133465347086,0.855558919749828)(4.09547523676931,0.855685495848609)(4.11580399435796,0.856569958430152)(4.11720476127688,0.856612021665901)(4.12830937600143,0.857042489843742)(4.17212765964191,0.858701175166688)(4.22123795997631,0.860454225491863)(4.22685013639795,0.860714893260184)(4.266008267887,0.862230259869466)(4.27440373400506,0.862496160149795)(4.34445471218721,0.864846519112476)(4.36462966218432,0.865436612292019)(4.36658326269394,0.865486570319564)(4.38200953898532,0.865892100226553)(4.38822175259238,0.866138455112869)(4.39467668052181,0.866393773677011)(4.40247628479141,0.866605587752878)(4.40502260588809,0.866669027849194)(4.41317768028049,0.866871408099133)(4.42700226808479,0.867211716191656)(4.60341994822517,0.871982935364571)(4.61595052406189,0.872437173898437)(4.63178152179812,0.873006984959993)(4.67903417546323,0.874675702076594)(4.79342583498751,0.878374723342226)(4.79996656019611,0.878576326976707)(4.80868365763092,0.878842416163107)(4.84662192916901,0.879794755819909)(4.89614314395091,0.880907618259098)(4.89716508089497,0.880936534025828)(4.96593510261115,0.882489330986689)(4.98986228657935,0.882962553386306)(5.00925060813198,0.88339286107758)(5.03587586340188,0.884053780827966)(5.03793873759957,0.884105937061296)(5.05699579326449,0.884532810198602)(5.12176611707272,0.885802347784217)(5.14466525044998,0.886239408188044)(5.20861920483889,0.887558804812606)(5.25360776206711,0.88841902392485)(5.26608669559033,0.888677924323482)(5.26698854313373,0.888698274889458)(5.2694301385146,0.888753344818906)(5.28616253971755,0.889019058997126)(5.30228141248475,0.889302177262734)(5.30424286513124,0.889332187567671)(5.31223848058127,0.889454186137978)(5.32387905568371,0.889644862205396)(5.32897285308438,0.889728917326481)(5.34487138671821,0.890034949856538)(5.34710155984685,0.890084050545999)(5.36409647792552,0.890389958432989)(5.44827632062637,0.891614238790592)(5.50584430933749,0.892416434338479)(5.65931788806777,0.895333566058127)(5.69521887378349,0.896019857061511)(5.71411262274377,0.896374742596593)(5.75627301667593,0.897049114873843)(5.79692609088558,0.897726302777715)(5.87902394116171,0.898903545367081)(5.88465816099315,0.898969135993533)(5.89631910084042,0.899104549712333)(5.90073278591505,0.89916132563196)(5.90666390234226,0.899257517368863)(5.9225829693161,0.899512292904142)(5.92877967916734,0.899608512606025)(5.97068009442833,0.900188472726687)(6.02196105849876,0.900893397823748)(6.03171535120239,0.901001096522759)(6.07261019789887,0.901486465996475)(6.1427262676469,0.902333701996013)(6.15862705165534,0.902538081602155)(6.17860488330997,0.902766821492012)(6.20890980687429,0.903106424436632)(6.21166856000521,0.903135564911165)(6.21513715363145,0.903183581215718)(6.22088176171626,0.903262967388881)(6.22356133628966,0.903297201963242)(6.23012655648247,0.903364550482027)(6.26664137582095,0.90379488229956)(6.31741340234192,0.904384864362869)(6.31940885846567,0.904411645796512)(6.32934916215943,0.904514411261782)(6.40508377055324,0.905314940112161)(6.44389380410852,0.905819438499703)(6.59960497627094,0.907526458968089)(6.62498671769322,0.907831206664807)(6.6467535206256,0.908080320629082)(6.72712909347385,0.908911332559591)(6.76949574672619,0.909297966162259)(6.77563698723647,0.90934797520899)(6.79694669129672,0.909580749147452)(6.84020725972752,0.909966499199195)(6.84897506051518,0.910045072371993)(6.86480171427423,0.910199068377106)(6.90086626686724,0.910515524727277)(6.91273825010344,0.910620401394566)(6.96165809637739,0.911062494688755)(6.99178477874306,0.911344486106625)(7.05315829942823,0.91185748747002)(7.06690549429046,0.911969257622911)(7.07360672967587,0.91201875018209)(7.10180199747841,0.912277364908278)(7.11087860103859,0.912364618127294)(7.11334330013563,0.912388246710932)(7.14870485186487,0.912671434126587)(7.15127662754937,0.912690037441129)(7.15587987523802,0.912724971435313)(7.17126385113498,0.912836515165486)(7.1804722423991,0.912902829783244)(7.18850480942477,0.912977488669645)(7.19771910072073,0.913051981260353)(7.22165845623893,0.91323678058207)(7.27000511477496,0.913620691541156)(7.28512928485621,0.91375839401614)(7.41685040929188,0.914807776479739)(7.47968388371521,0.915245265593769)(7.60526165955198,0.916223602771205)(7.61624167017498,0.916311981361307)(7.64899992188864,0.916544578711693)(7.6654505250208,0.916674928544892)(7.70967462472947,0.916963188930607)(7.74965494244072,0.917265916613002)(7.78075577758124,0.917488324162452)(7.78879567013815,0.917538788917347)(7.83487611694454,0.91785757681582)(7.84015621862917,0.917890435007533)(7.8406791215486,0.917893687986046)(7.8834873158379,0.918198310525302)(7.89450367667937,0.918277209759228)(7.95394357529854,0.918680142315516)(7.96172766577357,0.918727699307358)(7.97239940883658,0.918792848561909)(7.9745488458593,0.918805963537159)(8.0137309841504,0.919074478488785)(8.03064298096283,0.919176940584475)(8.03167412283563,0.919183178171551)(8.05003168562766,0.919294455910655)(8.05192985629338,0.91930588622955)(8.06144466167576,0.91936435904302)(8.06811017437628,0.919404349496101)(8.10928506260839,0.919667981800357)(8.14851067495395,0.919909761293117)(8.1692407870557,0.920033787084546)(8.18196293073748,0.920115484615886)(8.31996818700389,0.920940498477529)(8.51579829721367,0.922188432384252)(8.53754565638737,0.922325593727919)(8.53881268354253,0.922332168810551)(8.54534105756314,0.922366004117181)(8.57151592492176,0.922527975211458)(8.5867295721128,0.922612125693232)(8.59216411555606,0.92264708675402)(8.62006983601051,0.922825744695909)(8.64921248337284,0.922982057373659)(8.71020252249802,0.923308591590064)(8.71127070618144,0.92331414463269)(8.73866782878884,0.923479034866841)(8.75089120652279,0.923539197365858)(8.75777318133359,0.923573083049678)(8.84456249015454,0.92403414305698)(8.84471326129971,0.924034886095834)(8.86098030085545,0.924127520847901)(8.9392271136158,0.924556261411592)(8.94107419412211,0.924565268516829)(8.94863744504291,0.924602699656207)(8.95637240074091,0.924646431125843)(8.98418736966002,0.924794066150124)(9.02877340739248,0.925015109622008)(9.03437350500985,0.925042475535045)(9.04091454444974,0.925074692460788)(9.07328725579313,0.925232785327997)(9.07934905577683,0.92526347134937)(9.13656474185339,0.925563060108172)(9.1398266200702,0.925579219521072)(9.14971646289853,0.925626469310359)(9.21603155269211,0.925943090307104)(9.28006685756022,0.926269078462041)(9.41963860048872,0.926963471628362)(9.42497031232755,0.926987653936728)(9.4261149252083,0.92699284344086)(9.45574948134726,0.927132143397159)(9.45819663493836,0.927144294223289)(9.46218922146701,0.927164103383502)(9.4900343177692,0.927299450608399)(9.55866580840515,0.927622750025862)(9.62783264009141,0.927947080147802)(9.63126190287345,0.927962681769748)(9.66862722728927,0.928139629394637)(9.67034920333138,0.928147505627565)(9.67482465004733,0.928168003057882)(9.69853187243843,0.928278947027938)(9.73994161522337,0.928474109000001)(9.74483431332778,0.928497030887431)(9.7981089039264,0.928751541715813)(9.83073340056736,0.928907196194721)(9.84776145044961,0.928988652937227)(9.88533781108295,0.929168941989477)(9.90979867696819,0.929287303203376)(9.94564581968366,0.929460236122705)(9.99440520749812,0.929697105515229)(9.99895644478382,0.929719312521076)(10.0057050273755,0.929752380154167)(10.0159840932451,0.929802766426684)(10.0489904941007,0.929964168662944)(10.0507760458748,0.929972911286423)(10.0846960867796,0.930138316978549)(10.0985061974218,0.930205448092883)(10.1013912044503,0.930219360276285)(10.3147522921596,0.931248234517853)(10.3580951657554,0.931456394330028)(10.3720047593896,0.931523096974921)(10.3751708878639,0.931538306314295)(10.3918566867551,0.93161805211783)(10.3995163879679,0.931654751475726)(10.4375369792536,0.931835985596423)(10.4814206520108,0.932043932546837)(10.5184073470261,0.932218621779119)(10.5291773810522,0.932269342892113)(10.5437207787924,0.932338261919243)(10.5711446708509,0.932466461219576)(10.5793282163134,0.93250461215815)(10.5904099112349,0.932556453965644)(10.6326865389537,0.932753735891166)(10.6484896735758,0.932827899395489)(10.6964834121129,0.933053515142471)(10.7486457929437,0.933298472940973)(10.7514495988242,0.933311622958764)(10.8164691872666,0.933616102513228)(10.8402715358177,0.933727227088235)(10.8791449944165,0.933908098560159)(10.8851813265711,0.93393604022686)(10.8941656334989,0.933977581131411)(10.925738855936,0.934123126630886)(10.9282838686098,0.934134843252361)(11.0077934536943,0.934502161530838)(11.0207571075612,0.934562232648871)(11.022403216073,0.934569698109629)(11.2167651461134,0.935462021103616)(11.2241783867103,0.935495542190509)(11.2691450859293,0.935696087353173)(11.2748022859514,0.935721126107717)(11.2760800974429,0.935726778648772)(11.2780846004837,0.935735643692777)(11.3113849828062,0.935884253419366)(11.3428209762567,0.936024129328758)(11.3671782271686,0.936132376930464)(11.3881650083828,0.936226158090385)(11.4108863817713,0.936325759144634)(11.4226253632681,0.93637715576814)(11.4586786518427,0.936538547290963)(11.4843172677114,0.936651919624747)(11.4995510049625,0.936718665476088)(11.5013568251893,0.936726549694656)(11.5680996278807,0.937019322608753)(11.6414928489828,0.937343370329956)(11.6651587802003,0.937446739907486)(11.6711861644979,0.937473346066543)(11.6763986530602,0.937496335263343)(11.7372764154003,0.93776234106206)(11.7395788755952,0.937772356551929)(11.8034294735437,0.938052290428612)(11.827185507277,0.938155712194552)(11.829245678673,0.938164682569761)(11.834725014087,0.938188540541436)(11.8566059801427,0.938283799701303)(11.9241016354104,0.938578278385597)(11.9407756831212,0.938651312166429)(11.9585664634951,0.938729263736337)(12.0976385197708,0.939345801396014)(12.1064910287956,0.939386029195302)(12.1339255853488,0.939509159013783)(12.1647296934585,0.939644289883488)(12.1783453148473,0.939706851648266)(12.1852849366731,0.939737334835398)(12.2420165919479,0.939989258522875)(12.2974182794741,0.94024334478705)(12.3020459770114,0.940264912137383)(12.3167824512141,0.94033364594689)(12.3404736711781,0.940442355595784)(12.3583518346003,0.940522768361206)(12.3804061307722,0.940619107031866)(12.4196694480272,0.940793972398705)(12.4244988940042,0.94081634014225)(12.4674371183199,0.94100890901017)(12.5318374987173,0.941300961888079)(12.5602599371203,0.941427053932383)(12.5655699428353,0.941450120325401)(12.5945287330533,0.941580794765199)(12.5962254552048,0.94158861031534)(12.6097190507208,0.941647155065281)(12.7228144876628,0.942148294789604)(12.7384881069623,0.942219844234531)(12.7479511214217,0.94226227333267)(12.7543217552097,0.942291083206874)(12.8062318091926,0.942519370126023)(12.8101203473507,0.942537039423807)(12.8736482455059,0.942813599462375)(12.9316820032927,0.943071253113546)(12.9970105750485,0.943367717099822)(13.0142921777272,0.943444059777151)(13.0466815548889,0.943587404410682)(13.0495140875812,0.943599307755261)(13.0680528993481,0.943679457637874)(13.0741390636174,0.943704892000001)(13.1065533854908,0.943841325312994)(13.1147901102704,0.943875781880995)(13.1318633321775,0.943947313127818)(13.1370601868406,0.943969121555573)(13.1930913336436,0.944211655058809)(13.2360483190745,0.944395412107592)(13.2373067213658,0.944400827564863)(13.2592205877774,0.944493917814136)(13.3341666204521,0.944812314616365)(13.3380961473492,0.944828871081151)(13.4449279114538,0.945279430958646)(13.4515443181445,0.945307497224133)(13.4810838441437,0.94543166781192)(13.4955358364722,0.945491658882885)(13.5594849261847,0.945763338553462)(13.5776025297098,0.945838742400792)(13.5969724122172,0.945921006527785)(13.6220235799337,0.946027038875865)(13.6491815796265,0.946141953348324)(13.6534010576337,0.946159646571984)(13.6992861081512,0.94635161846658)(13.8042536192808,0.946787821673519)(13.807884251214,0.946802966093587)(13.8835582745232,0.947120049581674)(13.8867371560184,0.947133374733302)(13.905728718943,0.947212165266547)(13.9061453709021,0.947213907498991)(13.9115182769187,0.947236281235129)(13.9245410658223,0.947290588703878)(13.9254054826215,0.947294186578799)(13.933643149658,0.947328338275339)(14.0303893962906,0.94772071584535)(14.0411130498113,0.947763940976278)(14.0831355980061,0.947935768464527)(14.0862395059818,0.947948429532537)(14.0869207582612,0.947951206537698)(14.1259818908371,0.94810903978192)(14.1983929040719,0.948400443190463)(14.2003965261177,0.948408525863758)(14.2339673610865,0.948543214177061)(14.284631944596,0.948745795195541)(14.3249684322683,0.948905058567162)(14.3709474635978,0.949087332096611)(14.3715301256812,0.949089687068525)(14.3833641208959,0.949137454381746)(14.4592319573653,0.949439907981517)(14.5524998315031,0.949811083351586)(14.5584115485051,0.949835083392561)(14.5584443117856,0.949835210253086)(14.558820931347,0.949836668386898)(14.5633179460213,0.949854057686709)(14.5674664283755,0.94987006347945)(14.6952190375986,0.950366781194311)(14.7153450939376,0.950444160990142)(14.7176872687417,0.950452630407586)(14.7748862370189,0.950685326861429)(14.7961981308597,0.950775428681357)(14.7990829258626,0.950787649681298)(14.8107788473161,0.950836328337913)(14.8135689448072,0.95084766676126)(14.8274425246517,0.950896228127163)(14.8529694664993,0.95099497161803)(14.8608257356162,0.9510249485813)(14.9026795593511,0.951181632388067)(14.9406033522147,0.951311094231434)(14.9497482280975,0.95134134601439)(14.972260290592,0.951426815866981)(15.0165157525024,0.951602292035396) 
};
\addlegendentry{$B = 15$};

\addplot [
color=orange,
solid,
line width=1.0pt,
]
coordinates{
 (1.06375939298373,0.49572606959705)(1.11638230571792,0.508019214779541)(1.11791484708712,0.508376256351346)(1.14258877320586,0.514116585997756)(1.14319452851525,0.514257317925532)(1.15247062609978,0.516411173392694)(1.15485811588477,0.516965161971804)(1.17427973800941,0.521465879443395)(1.17634250006052,0.521943273356539)(1.18833461714531,0.524716201197942)(1.20250880859108,0.527988116804173)(1.20945997662836,0.529590415043793)(1.21572210481232,0.531032561642003)(1.22190445452521,0.532455086463869)(1.27393226888817,0.544375707832621)(1.28404827438888,0.546682691866054)(1.38693949776658,0.569937989069267)(1.40855367126253,0.574773195526196)(1.40963204599883,0.575013971199178)(1.4365771115355,0.581015886167189)(1.44761236866726,0.583465998536895)(1.48330789696862,0.591359413698504)(1.53745923360577,0.603240404754047)(1.6291521795105,0.623107577417069)(1.72311285181868,0.643157358989902)(1.72395698347176,0.643336148991334)(1.7336225665659,0.645381733722443)(1.81502163108587,0.662498739155686)(1.89203164663899,0.678542300922436)(1.97516990716099,0.695760526014893)(2.04276919773106,0.709744561534497)(2.06392188266658,0.714128891530995)(2.06890468915854,0.715162812912147)(2.09457744979611,0.720498722085006)(2.09907002083813,0.721434248339957)(2.11078796689622,0.72387730056927)(2.12818855552844,0.727513883132987)(2.14980800845665,0.732049343342524)(2.18571909877116,0.739635502623672)(2.19676099668838,0.741983981098097)(2.21219360288467,0.745281032162985)(2.21693165913739,0.746297003998071)(2.22062982121319,0.74709128224609)(2.22872020171335,0.748833003052121)(2.23305134841218,0.749767824165983)(2.28595238737658,0.76134236546543)(2.29455985815946,0.763257112964526)(2.34292718578576,0.774217832622573)(2.39651936807655,0.786820276085746)(2.41977620471627,0.792438377046045)(2.45873050061433,0.801964277231211)(2.49455132741362,0.810622438177044)(2.54496034189059,0.819507590292425)(2.5789424824946,0.825880585482527)(2.63970449994385,0.833668108372801)(2.73066292442983,0.841224467764895)(2.73533153904009,0.841491540302013)(2.82742580708895,0.849733625398418)(2.87648181839011,0.852671279720744)(2.89935649983313,0.853676541912621)(2.966952635083,0.857091550657192)(2.9874785136946,0.858217219297771)(3.01358963272253,0.859438113779456)(3.02016507901632,0.859714021878794)(3.05793871085095,0.861474411580593)(3.0719646792879,0.86228437615634)(3.07751016730075,0.862553080981384)(3.08844563927786,0.862999911217802)(3.1183415415743,0.864212512798324)(3.13464855015056,0.864955098398725)(3.15716551560858,0.866152100204394)(3.1631776227236,0.86646059841396)(3.16445297737483,0.866525459165574)(3.18390636343978,0.867403365470329)(3.18651648903024,0.867506342474029)(3.20782554777439,0.868342969481012)(3.25095695509064,0.870282187739868)(3.29332468117388,0.872108850511463)(3.35929054105823,0.875086035786216)(3.3684784145417,0.875544157584973)(3.38832785067458,0.876551278787735)(3.42039586269027,0.878164898300419)(3.50207949503476,0.882039092715215)(3.51854642151719,0.882808919912532)(3.58191484963931,0.885151514602447)(3.69165943486505,0.888472855559655)(3.71201142516013,0.889114008391766)(3.73905373298163,0.8899835649821)(3.85291436501311,0.893049757469165)(3.8805342659381,0.893733016145315)(3.93699562711854,0.895158111326879)(3.94191236476309,0.895279513422722)(4.00072860403491,0.896716293839122)(4.01025994441957,0.896938211218422)(4.04760212745353,0.897843904968283)(4.04999758868507,0.897903162071243)(4.05829872306238,0.898108428110631)(4.0649639439667,0.898263901481336)(4.07119901548451,0.898404935422103)(4.07174754220266,0.898417278697855)(4.076943352914,0.898533709197439)(4.09487483991616,0.898978816592026)(4.12842666300646,0.899810218852869)(4.13664411711852,0.90001342752772)(4.13969407922686,0.900088336751225)(4.1540135993961,0.900413255420058)(4.16488270758092,0.900657372595384)(4.20402053088445,0.901584556037434)(4.28052956714638,0.903404638926947)(4.28911930662435,0.903620789639152)(4.29053914994439,0.903656947778838)(4.29244105155606,0.903705578057744)(4.43952280842439,0.907730616657097)(4.44098997939334,0.907770099283791)(4.48641616179191,0.908975279978481)(4.50378352504645,0.909423819852598)(4.61180476271878,0.912000239056426)(4.72020388964701,0.914316013410329)(4.81528191199159,0.916191407823554)(4.83511587594419,0.916556554860017)(4.85306019814051,0.916875848867385)(4.85798827830454,0.916971458146878)(4.89265596247657,0.91758996952397)(4.89916717216332,0.917715033655292)(4.92384670199235,0.918164919579841)(4.9285493258495,0.918252426514657)(4.94155305219071,0.918491505792444)(4.94571685286331,0.918567202219972)(4.95243103096135,0.918688454687777)(4.98158770185837,0.919193046991735)(4.98252345899815,0.919209661662642)(4.99836634068117,0.919478908251975)(5.05163333464477,0.920383732864754)(5.0579453893233,0.9204911122493)(5.06768031750557,0.920663207319091)(5.08661033712854,0.920999101101169)(5.16632499020401,0.922394814000823)(5.17410286692223,0.922529432579516)(5.17963174296103,0.922625187084424)(5.18234124267083,0.922673245377331)(5.19127884484129,0.922833739589248)(5.19483233859382,0.922895429899192)(5.30258927559845,0.924934591132064)(5.37155341530987,0.92628807914131)(5.41734866250043,0.927184607240029)(5.4860800182616,0.928395791768747)(5.49535023234524,0.92853737801311)(5.57519685188792,0.929863488277773)(5.69993780061296,0.931660414644403)(5.73814823843411,0.932250342130738)(5.75352343034021,0.93243909261821)(5.76686273378866,0.932600579999786)(5.77298112482348,0.932679850931725)(5.79158124973022,0.93291074339563)(5.82395226650179,0.933322668880855)(5.86377353519412,0.933863976578279)(5.86493960980185,0.933879356061582)(5.8670215055635,0.933906759112299)(5.88314375170531,0.934116956907597)(5.92366121124215,0.934568734754845)(5.9333654205097,0.934674433025235)(5.96501826756375,0.935016030554839)(5.97105826764847,0.935080820711578)(5.97665073730436,0.935144434522084)(6.00019332124506,0.935435930251538)(6.00849082512099,0.935538470084479)(6.08750508267728,0.936477121834856)(6.09198977681977,0.936523989759026)(6.0962682573923,0.93656866383936)(6.10737222916674,0.936704325222677)(6.10762274734435,0.936707445786578)(6.18581030291037,0.937685265402747)(6.28247171433977,0.938807906389435)(6.30815319490765,0.939065326861657)(6.33916889816978,0.939366351721245)(6.38885435192329,0.939927070089799)(6.40673653086041,0.940104832803591)(6.44632778458184,0.940454208764521)(6.60197206331787,0.941943773265713)(6.62090456836313,0.942099192840011)(6.66917960969889,0.942479250254988)(6.68290982865196,0.942582667189531)(6.68428864984004,0.942592997683573)(6.68474836025719,0.942596439868474)(6.7553044235141,0.943234641487587)(6.76655929960245,0.943316152138559)(6.77913187680507,0.943421948085454)(6.78614926564766,0.943487114980405)(6.81487504426067,0.943735424300129)(6.84738181372271,0.943966533239586)(6.86694847933278,0.944106617106956)(6.87608548219687,0.944172397546264)(6.89857921395068,0.944335397643064)(6.92174294294197,0.944541428584079)(6.92216023466588,0.944545184710306)(6.94296088375043,0.94469762706999)(7.00763971114209,0.945231769285252)(7.00872034042799,0.9452411861799)(7.01206836855628,0.945270330566789)(7.01219886279739,0.945271465537327)(7.0186682483577,0.94532763743191)(7.16727624177022,0.946508154104122)(7.19498258010757,0.946703407735557)(7.26103030012233,0.947171856595253)(7.27992483200689,0.947304858099951)(7.28006964902958,0.947305859237315)(7.32842792677505,0.947637321085473)(7.3571601752355,0.947830267023575)(7.5085239029907,0.948801209349249)(7.51557040889746,0.948847388279081)(7.5599857976349,0.949131825508954)(7.5831866988417,0.949279091703331)(7.59450917170659,0.94935050258713)(7.59822066283694,0.949373858904104)(7.65189702203194,0.94969684478138)(7.67899327897987,0.949864954593597)(7.68025966267199,0.94987255541199)(7.75829558419052,0.950355456494336)(7.77968315026379,0.950490352491499)(7.78496218236785,0.950523747370776)(7.79702600913644,0.950600238733485)(7.79928258453228,0.95061457553983)(7.80240025797047,0.950634398733708)(7.8086663564566,0.950674295926038)(7.87454789185514,0.951096697917826)(7.91116670311572,0.951334080266918)(7.92544550691631,0.951427700123385)(7.92998341957165,0.951457510757592)(7.93428460019259,0.951486095081952)(7.93891868208777,0.95151700212033)(7.99578279289141,0.951904759865066)(8.068417026943,0.952398037844727)(8.11536746371102,0.952719690669503)(8.14765366226907,0.952929807387541)(8.19344774981557,0.953243965755295)(8.19993395565919,0.953286153492819)(8.20154026594099,0.95329729526452)(8.24109755507392,0.95355445012476)(8.42896298072418,0.954706685406654)(8.47879276675318,0.954947067226914)(8.48043306877064,0.954954572084101)(8.51500401997354,0.955110085285709)(8.52103731459019,0.955136950411571)(8.53427459853959,0.955195485038125)(8.58163149913414,0.955449785016439)(8.6126999129073,0.95562484767423)(8.64279316227892,0.955787378860092)(8.69024406501547,0.956035164809721)(8.69671305102329,0.956061536822023)(8.69770149831197,0.956065568621367)(8.71260064823256,0.95612642004222)(8.72018420273861,0.956157457034844)(8.72233350231406,0.956166262168205)(8.79545403633465,0.956481426162218)(8.8018569907556,0.956507978656605)(8.8256279803874,0.956611465984389)(8.83278299108823,0.956650725306555)(8.85288757304015,0.956761242488956)(8.85312469452583,0.956762547801462)(8.85510688751985,0.956773459238936)(8.89764185366144,0.956988951470667)(8.98777528057684,0.957428913841205)(8.99053417286433,0.957443116231474)(9.11590259367886,0.957993827963725)(9.11791686936894,0.958001360382483)(9.12294432093987,0.958020368612921)(9.16361347604525,0.958188494703506)(9.18046167941811,0.958252008978864)(9.31864499709441,0.958741968997399)(9.38235873294367,0.958936453645894)(9.44208855504453,0.959107737854643)(9.44830115494713,0.959126447999685)(9.45179280442335,0.959137153458701)(9.47483718893519,0.959205623542637)(9.53403395709632,0.959404609865257)(9.53607501576372,0.9594117213595)(9.58851180416052,0.959596330000124)(9.59929338856313,0.959629688736221)(9.62294834654173,0.959701215028647)(9.64222345610451,0.959761575838413)(9.64734439721952,0.959777917512552)(9.66176785304115,0.959824594468579)(9.70584001301816,0.959977118129679)(9.72802131985434,0.960052175025408)(9.74447795310461,0.960109469925829)(9.76494130422179,0.960182352184355)(9.76576533769007,0.960185299825744)(9.7768605674435,0.960224323472342)(9.77709898020711,0.960225174426688)(9.80510659631368,0.960324514477976)(9.80850747158029,0.960336468829026)(9.89390496354047,0.960629249135417)(9.93046453232346,0.960749570366976)(10.027102820604,0.961069102361088)(10.0689945648456,0.96120810127469)(10.080444400725,0.961240067879288)(10.0871570250193,0.961265131248384)(10.1032138283584,0.961316312203018)(10.2621917040697,0.961832773330735)(10.3345649750309,0.962044035560612)(10.3642906720162,0.962122426786583)(10.364423951343,0.962122774040465)(10.3893968823628,0.96219837293583)(10.4293244225573,0.962309726818052)(10.4482701132929,0.962369294174559)(10.4653844150186,0.962418934889071)(10.4689145335099,0.962427750826999)(10.4891875852172,0.962483537518615)(10.5046828737936,0.962533097858343)(10.5530455731412,0.962684525948322)(10.5646588603112,0.962713793078423)(10.5827418999867,0.962760791677971)(10.6089806534826,0.962831741047465)(10.6456001145698,0.962936472930904)(10.6650516475643,0.962994441627955)(10.6692768288175,0.963008047366018)(10.6791107302059,0.963039788306279)(10.6872923904711,0.963064312204447)(10.6896708039763,0.963071340463466)(10.7190731306738,0.963162936888101)(10.7864255767994,0.963379392620825)(10.8223802469306,0.963490699878359)(10.8278426506754,0.963508038530784)(10.9598053065204,0.963909051007191)(11.0082669681331,0.964050325967472)(11.0242699837208,0.964098668949073)(11.0310640000338,0.964118648289689)(11.0338068773148,0.964126618969114)(11.1844040276442,0.964554904277072)(11.2659586823852,0.96472855280393)(11.276297878111,0.964747632195547)(11.2956622038294,0.964790315185531)(11.3106431806829,0.964831854506876)(11.3820094088201,0.965022952247662)(11.3870848550137,0.965037064986725)(11.3895840071668,0.965040981265635)(11.3898738589439,0.965041435534323)(11.4020752481056,0.965060571269113)(11.4749844497292,0.965231976405905)(11.4915478539345,0.965256070184578)(11.5189470305834,0.965296216141745)(11.5440151810706,0.965356388244666)(11.5885960091858,0.965420339501778)(11.5976817221947,0.965433545125361)(11.6096323406708,0.965457975407063)(11.6122284086592,0.965463597261422)(11.6290449606371,0.965495556647262)(11.6416960971232,0.965517866518307)(11.6520616206667,0.965542177288039)(11.6561579007762,0.96555168847069)(11.7087240071448,0.965666653669234)(11.7404375411714,0.965713502893738)(11.7733647614503,0.965769982354116)(11.8799536109447,0.965911510295784)(11.8804224449014,0.965912027278004)(11.9262852753476,0.965965903575602)(11.9363275074287,0.965977089003509)(11.9736254149332,0.966016127401772)(12.1149736871875,0.966156920448894)(12.1657516061931,0.966194701769602)(12.1978322513186,0.966223925871955)(12.2277286152024,0.966248047235853)(12.2433653277021,0.966266495862337)(12.2989809992332,0.966335365619592)(12.3136688043854,0.96635208799211)(12.342527860478,0.966378915716205)(12.3429843260909,0.966379552695491)(12.3442810650627,0.966381365284669)(12.4040660447196,0.966469084829444)(12.4151721870248,0.966478900158084)(12.4225383627534,0.966485385340361)(12.535779338262,0.966604342667474)(12.5407584015943,0.966609121786253)(12.5420526940638,0.966610646358129)(12.5425638694795,0.96661136064911)(12.5450352841665,0.966614810396545)(12.5624727905336,0.966633365909983)(12.573009325414,0.966644843114802)(12.5852864898695,0.966657111598032)(12.6190204549191,0.966700370234918)(12.6491260615242,0.966736773456902)(12.8027640639588,0.966855184262217)(12.8093036632092,0.966857286417444)(12.8121512424122,0.966858122995916)(12.8281598134555,0.966866273707697)(12.834390514689,0.966869543944863)(12.8429293767084,0.966875268479544)(12.9066411483752,0.966899189181243)(13.016484268674,0.966937644508184)(13.1302510393248,0.966976989310487)(13.1359230312594,0.966979832309514)(13.1373022901613,0.966980547625445)(13.171269913572,0.966999714658294)(13.2294403277508,0.967040437386473)(13.252054918477,0.96705778052696)(13.2543608121121,0.967059459810259)(13.2655690452568,0.967067801632689)(13.3228398259997,0.967120944181017)(13.3260300100256,0.967123624977184)(13.3596848791097,0.967157735081718)(13.3964497838333,0.967210319743312)(13.4369530847751,0.96727566428168)(13.4416935412197,0.967283671277637)(13.4440904243174,0.967287742975833)(13.4737477204114,0.96732191030513)(13.4747496992055,0.967323671171612)(13.4749867302842,0.967324088066238)(13.5058434402101,0.967375263871803)(13.5471689776757,0.967435781440567)(13.5632444626009,0.967456125580483)(13.6714929639419,0.967584849796187)(13.7036059357516,0.96763197673858)(13.7060140901465,0.967636805557008)(13.7688350696408,0.967736459515542)(13.7838453974607,0.967752330241497)(13.7864811772335,0.967755165683742)(13.790496965093,0.967759521208371)(13.8494146946013,0.967855145015132)(13.9584794574494,0.968031192081868)(14.0745514750881,0.968207049276579)(14.1030279454391,0.968247340746514)(14.1191672407004,0.968270415896908)(14.1426625046607,0.968305457003586)(14.1773527118054,0.968360988779179)(14.1864029597123,0.96837630528098)(14.1978833497141,0.968396236913485)(14.2046874169181,0.968408311249456)(14.2579370558296,0.968509037704689)(14.2777990049765,0.968549104597512)(14.3024171673623,0.968600319740332)(14.3247354364099,0.968648041111209)(14.3364840877576,0.968673596532405)(14.3484207866982,0.968699838307518)(14.3730228848254,0.968754718662056)(14.3876483441323,0.968787712489606)(14.3889594169473,0.968790128176359)(14.3974026047726,0.968809377970499)(14.4489365845712,0.968927753138518)(14.4966850541111,0.969033454352511)(14.4981900596316,0.969036889347814)(14.5818017613273,0.969224194804031)(14.6455719373644,0.969368641232999)(14.6479671540464,0.969374113881144)(14.6573755808791,0.969394270890867)(14.6818540685393,0.969450619273842)(14.7474312894887,0.96960261780908)(14.7607523948836,0.96963353964853)(14.8130280860345,0.969754642212159)(14.9012652251779,0.969957386624737)(15.0197920746918,0.97022562101495) 
};
\addlegendentry{$B = 30$};

\addplot [
color=red,
solid,
line width=1.0pt,
]
coordinates{
 (1.10138062057991,0.727742587709612)(1.1381002519296,0.730728321386809)(1.18961014683136,0.735057084861097)(1.19897415482006,0.735862785196266)(1.20187820874146,0.736113876804859)(1.2233277114423,0.737986519149392)(1.22643175946567,0.738260170565647)(1.25678526037091,0.740972003610041)(1.25765047998217,0.741050270613059)(1.26196557214542,0.741441419761163)(1.28123651598799,0.743204748764536)(1.29152538985678,0.744157231038111)(1.30381701484107,0.745305118214485)(1.31972097866295,0.746806345339474)(1.34996969020933,0.749710686187322)(1.35845693924337,0.750536999155294)(1.37294999042091,0.751959411975208)(1.39920309357576,0.754571782696159)(1.40035375915221,0.754687323008759)(1.40525231540892,0.755180163250047)(1.50026475405387,0.765034896192851)(1.53171733421669,0.768409499728415)(1.54319686230709,0.769653048576304)(1.55550879376483,0.770993299093098)(1.59525710372046,0.775362481491794)(1.60777112835281,0.776749689763715)(1.62956918837131,0.77917620598195)(1.66855961914445,0.783533222933811)(1.82361589454995,0.800517899080031)(1.9512869472485,0.817357026385906)(2.06658575908746,0.832906554820087)(2.15322732866835,0.843642037193374)(2.19622290782805,0.848734501719312)(2.21177293010742,0.850561493549391)(2.21506675939376,0.85088413298027)(2.2191646445,0.851280619794124)(2.22196360903113,0.851684502808341)(2.23886830604258,0.853448991954846)(2.24683034831664,0.8544164271095)(2.26550575542267,0.856179572988819)(2.29035191841609,0.858394341626354)(2.29280668901138,0.858714308444753)(2.31569428087886,0.860935450361769)(2.31620266181734,0.860967898607751)(2.36723789713165,0.865038785171272)(2.38191289420887,0.865815703981953)(2.42371255834864,0.869006650749246)(2.43370312253676,0.869471940915276)(2.43520992371349,0.869556229686599)(2.45334504093588,0.870427096937427)(2.48359019967194,0.87164451209307)(2.50624914042266,0.872487002342498)(2.54284534902226,0.874099386270982)(2.58618133412855,0.876191028526318)(2.62406274896062,0.87837963833428)(2.71820898442271,0.882340150016806)(2.76686095320827,0.884063252481019)(2.84620856044412,0.88733997469806)(2.89372563323841,0.888595170562785)(3.02930027908001,0.89552455866756)(3.09206958241984,0.899039958709631)(3.12881353348485,0.90097052450607)(3.14731185180979,0.901963881019281)(3.18056091990728,0.903853221516844)(3.25036784906727,0.907499717571141)(3.27006069081609,0.908530960578565)(3.29074898067875,0.909676313990753)(3.30685739711539,0.910564685414816)(3.31353436733076,0.91090748801243)(3.33005985507677,0.911850814138959)(3.35603439878133,0.913166554050568)(3.36658134152756,0.91374517068182)(3.3731198448916,0.914129591134007)(3.38317097892353,0.914729680972935)(3.3937777921919,0.915256263312642)(3.40349620970448,0.915825262699272)(3.43791591409209,0.917740814968842)(3.48760879449497,0.919866774158633)(3.56540084830146,0.922853458418876)(3.58495352970912,0.923223627561638)(3.5992265419237,0.923582973852488)(3.63353159647553,0.924532234915675)(3.65495984113958,0.92476284653489)(3.73570335642854,0.926364183524087)(3.81532311567358,0.928088973406575)(3.90972809758201,0.929914138050761)(3.9452726022705,0.9305768003348)(4.00533277657282,0.93164657082575)(4.02304084306637,0.931905416771761)(4.13089244804144,0.933833308405932)(4.15163360871568,0.934217128285419)(4.19707846963794,0.935058360306273)(4.20785940960807,0.935247918098494)(4.212515829401,0.935324599264001)(4.21253049627962,0.935324868066995)(4.21558181342488,0.935380775288303)(4.25146747393376,0.936009946823492)(4.28657661352903,0.936627928940123)(4.28981224280588,0.936686957374917)(4.29223267806414,0.936731110433503)(4.29350303338515,0.936754285664141)(4.30517902974962,0.936962072474054)(4.30934663699628,0.937038357636665)(4.31408502070842,0.937122730798544)(4.31527528246443,0.93714332130899)(4.35819302920115,0.937903714739672)(4.37901843100615,0.938284309487424)(4.41287392719562,0.938923338758729)(4.50233121356281,0.940735918829259)(4.55679257504255,0.941947533693583)(4.60855876956602,0.943074293036248)(4.69340765956436,0.944636896846173)(4.73399625877463,0.945124336648737)(4.77274771333465,0.945706368516564)(4.81190340316009,0.946547309006092)(4.83114062006921,0.946952456124339)(4.86195194063557,0.947307862955278)(4.9511529458683,0.948464338783083)(4.95726764031867,0.948533735151108)(5.0944807512498,0.949921823026804)(5.10738761117922,0.950047994575659)(5.14142957445022,0.950355965816877)(5.14477396683807,0.950384802476798)(5.15589751660828,0.950479513377247)(5.1729655063792,0.950623010540315)(5.20071365280965,0.950857475805404)(5.22004769975785,0.951025608766574)(5.22680216930223,0.951085868304175)(5.24054283965105,0.951208327324892)(5.24477079850812,0.951246915269018)(5.24662579221453,0.951263891757068)(5.25252700695595,0.951318365424108)(5.26977075411127,0.951482667724152)(5.29598194392354,0.951736957333963)(5.2986279394011,0.95176309193729)(5.30972735425951,0.951872672421734)(5.32688625334284,0.952040610111209)(5.33002756801009,0.952071334776129)(5.50550390680294,0.95365006605696)(5.52985971194737,0.953911356392654)(5.62675487508388,0.954829775355755)(5.65187335205531,0.955009405058503)(5.70364650663273,0.955395010613215)(5.71386400130185,0.955446514845851)(5.76108301649281,0.955642159302318)(5.79907580893237,0.95587647966231)(5.84306549380753,0.956219522457964)(5.87626670956145,0.956355562230583)(5.8975154443391,0.956469250206595)(6.04054712941971,0.957082955929483)(6.07743806650568,0.957198668466339)(6.10064064710893,0.957277606415623)(6.12917300339295,0.95738636584607)(6.13388562823489,0.957405076293303)(6.16956154914165,0.957560765973353)(6.19237379629707,0.957669838451191)(6.19301178448757,0.957672918563046)(6.20821781839935,0.957751847128358)(6.21361141841267,0.957782358851203)(6.2281703861549,0.957863175560474)(6.24862773978939,0.957987647111905)(6.25160413220727,0.958005688027146)(6.25871165349587,0.958047267532131)(6.30693413790047,0.95830891905362)(6.3450820141161,0.958503145202319)(6.36723192414866,0.95862157919487)(6.37741706403013,0.958663196376751)(6.45932527454296,0.959223229712362)(6.56555252871885,0.960050886943242)(6.59078892398649,0.960172364612748)(6.65159827511719,0.960605867644977)(6.67850678288848,0.960781299296564)(6.70849618236637,0.960933071294506)(6.73926040614066,0.961071523643181)(6.74991153170631,0.961119472080211)(6.78201920560449,0.961264541587868)(6.80023895298,0.961359685366754)(6.84214853574535,0.96163876144232)(6.89095683255668,0.961996041061915)(6.99851312381078,0.962622183622012)(7.04953977748264,0.96277468351253)(7.06678286408997,0.96278841170806)(7.08162496200028,0.962841273933467)(7.09773981100356,0.962865454174349)(7.13153682799615,0.962922082765966)(7.14627569627994,0.96296730213252)(7.17357409685491,0.963020082839867)(7.17880011660371,0.963036261956469)(7.18041994364294,0.963041039934595)(7.18158949086866,0.963044426970334)(7.19218601046747,0.963066230098949)(7.24715993717321,0.963181218281593)(7.27646173306898,0.963255781198426)(7.28621128762034,0.963273381349717)(7.32938898214555,0.963330337124888)(7.3625791794982,0.963325159150714)(7.36445992478171,0.963330888399931)(7.3997665417789,0.963336938344963)(7.59106890659716,0.964372938472782)(7.64698252779793,0.964618722044927)(7.67379849147823,0.964774744258829)(7.68867153647342,0.964806844676296)(7.689976885813,0.964805723719075)(7.72188496016032,0.964876514668421)(7.81051258235928,0.965096347119192)(7.87988366635146,0.965478664367775)(7.89992297537686,0.965606915170728)(7.9559523194204,0.965971989282485)(7.97809371924492,0.966146543129022)(8.00264876848444,0.966255859038983)(8.03111559359174,0.966357099777087)(8.03899527178102,0.96638293056799)(8.06780405952149,0.966492929376611)(8.07017391762827,0.96649799341224)(8.11361593922987,0.966655082874629)(8.13202933820285,0.966691876945712)(8.13307534577039,0.966692089870282)(8.15141006833673,0.966775396585422)(8.15723123124674,0.966800444388185)(8.16165301777322,0.966818282372699)(8.17548971310202,0.966853837405693)(8.24663697465755,0.966867102554379)(8.25321687673652,0.966873827600809)(8.27517184939463,0.966873537671031)(8.28148978023876,0.96688130714086)(8.36351204946087,0.966940590036434)(8.36653926126643,0.966942320222984)(8.39672081673467,0.966955782626565)(8.5201456445738,0.966957796265144)(8.59949393999704,0.966905261432473)(8.6543030040922,0.966861639711187)(8.7268974141487,0.966750423499784)(8.745593064481,0.966720449686159)(8.75941786431149,0.966696322898958)(8.81089223632424,0.966616553029668)(8.94569326806147,0.966406374939616)(9.00863004055475,0.966324613398656)(9.01634773978523,0.966315750161666)(9.02223770224987,0.96630914896704)(9.03440951253387,0.966295943130175)(9.06276326941764,0.966267307110682)(9.07666408929262,0.966254239519283)(9.08312535938643,0.966248357091925)(9.08900981739524,0.966243097209772)(9.17391145671356,0.966174107516305)(9.22187165458539,0.96613783678081)(9.44251652649976,0.965998841446542)(9.53420565686739,0.965969200103321)(9.7544178559087,0.965912033760646)(9.88425495298035,0.965872476431294)(10.0446996997102,0.965814090531583)(10.1830509588341,0.965758616150901)(10.8728266063692,0.965483913274743) 
};
\addlegendentry{$B = 60$};

\end{axis}
\end{tikzpicture}%

%% This file was created by matlab2tikz v0.2.3.
% Copyright (c) 2008--2012, Nico Schlömer <nico.schloemer@gmail.com>
% All rights reserved.
% 
% 
% 
\begin{tikzpicture}

\begin{axis}[%
tick label style={font=\tiny},
label style={font=\tiny},
label shift={-4pt},
xlabel shift={-6pt},
legend style={font=\tiny},
view={0}{90},
width=\figurewidth,
height=\figureheight,
scale only axis,
xmin=0, xmax=15,
xlabel={Normalized travel length},
ymin=0.32, ymax=1,
ylabel={$F_1$-score},
axis lines*=left,
legend cell align=left,
legend style={at={(1.03,0)},anchor=south east,fill=none,draw=none,align=left,row sep=-0.2em},
clip=false]

\addplot [
color=blue,
solid,
line width=1.0pt,
]
coordinates{
 (2.30331409648935,0.398338109166751)(2.58599245959046,0.406635648928468)(2.59325042489984,0.406848134090519)(2.61551549216525,0.407499794902058)(2.62488417831667,0.407773921955803)(2.63005086548606,0.407925078948803)(2.63649277107084,0.408113524035164)(2.67625735949363,0.409276271960123)(2.69707676004474,0.409884714792838)(2.69953708726826,0.409956602275664)(2.75779413843179,0.41165786583125)(2.80775757110511,0.413115516269312)(2.80839630181133,0.413134142374548)(2.81184075992426,0.413234582974847)(2.91129174281826,0.416131916468839)(2.93190021307113,0.416731666915994)(2.96698943852184,0.417752333266924)(3.23201713477366,0.42544102807042)(3.23718382194305,0.425590564117821)(3.28776133187261,0.427053687130546)(3.3811311449889,0.429751355105108)(3.59339841402172,0.435868150272114)(3.63005767532418,0.436922297265329)(3.63649277107084,0.437107272443329)(3.69959606245894,0.438920091735582)(3.70734609321303,0.439142599738568)(3.75779886749095,0.440590418477558)(3.78286063638629,0.441309147322868)(3.78593604541569,0.441397323997916)(3.81184548898343,0.442140006980241)(3.91131220245892,0.444988194056347)(3.98805364242944,0.447182437792975)(4.00854720024488,0.447767932949322)(4.28310372153796,0.455593030340284)(4.47702355566059,0.461099049801412)(4.49339582554673,0.46156313468121)(4.55583723656665,0.463331985476621)(4.5934099226211,0.464395514511527)(4.59828460719996,0.46453345102271)(4.60468392019604,0.46471451321884)(4.63006860130862,0.465432569263486)(4.63651565950939,0.465614891815671)(4.65418665907158,0.466114533160847)(4.65686201857084,0.466190166048042)(4.70617230311162,0.467583614666113)(4.74700553676806,0.468736708705655)(4.75787302413093,0.469043474741755)(4.78823414146219,0.469900233371319)(4.81191964562341,0.470568336754612)(4.91682951628172,0.47352466219923)(5.00964574352475,0.476136288584297)(5.01284416891532,0.47622621980699)(5.09106392727551,0.478424216429134)(5.10482182962958,0.478810552867557)(5.10510199406074,0.478818419361951)(5.10879699837877,0.478922165131439)(5.1456374557237,0.479956235737278)(5.28516320835399,0.483867496280584)(5.36638313944136,0.486140645149649)(5.39815424023647,0.487029119056614)(5.51551686686672,0.490307665797243)(5.52540764034806,0.490583718027895)(5.57809622085288,0.492053615458123)(5.59367433773721,0.492488004068693)(5.59761086129484,0.492597757305492)(5.6305109094207,0.493514800812888)(5.69972382734377,0.495442655176257)(5.75787393967854,0.497060953477557)(5.77794260082498,0.497619160553059)(6.01287836124114,0.50414272033296)(6.06482346059623,0.505582376345874)(6.10274789320286,0.506632838271586)(6.28516731316706,0.511678530753213)(6.28705334378036,0.511730637335205)(6.36793932795321,0.513964174425244)(6.37818610686093,0.514246963021529)(6.39822572674912,0.514799909338747)(6.40560670841968,0.515003535889588)(6.41433363169815,0.515244270490905)(6.45492770277041,0.516363731481147)(6.49533493274147,0.517477490772624)(6.51962855014507,0.518146842564114)(6.5457343766978,0.518865907196579)(6.57853601039424,0.519769083797179)(6.60780852577756,0.520574789681595)(6.61081720243476,0.52065758556307)(6.61211409841872,0.520693273979418)(6.71951578202761,0.523646896960035)(7.02625668186149,0.532062414029932)(7.03266884202535,0.532238024451819)(7.14420627343455,0.535290743729579)(7.14621194999657,0.535345604253014)(7.15192789442145,0.535501943861569)(7.17331438507624,0.536086811246373)(7.18374050943466,0.536371891304323)(7.20810519908113,0.537037968019383)(7.23477532018514,0.537766872496347)(7.34823949291388,0.540865607485158)(7.38663953591977,0.541913494827501)(7.41442713351921,0.542671524578203)(7.47513570036634,0.544326873222371)(7.47957803302811,0.544447962956462)(7.5877899772988,0.547395957179068)(7.63235406183642,0.548609088660071)(7.68171565741073,0.549952202555964)(7.78023487338692,0.552630969759217)(7.78414986021352,0.552737367412926)(7.9414129842854,0.557008084035669)(8.04069006183206,0.55970092464493)(8.14422428245912,0.562506689493977)(8.15192789442145,0.562715354105078)(8.16803752277961,0.56315166370247)(8.28779934732656,0.566393361955039)(8.33361935476135,0.567632736066357)(8.47515695447425,0.571458157857209)(8.47724823261424,0.571514646646042)(8.50543383585951,0.572275891358685)(8.62259338116077,0.575438318158983)(8.67957896263906,0.576975437789035)(8.75798855042103,0.57908932959044)(8.78025304041856,0.579689338747024)(8.78345146580913,0.579775525209996)(8.8029556213474,0.580301049023367)(8.94144022335963,0.584030185872514)(9.0026419269648,0.585677016927522)(9.00504832176211,0.585741753787154)(9.01049159199827,0.585888184531872)(9.01052351089795,0.585889043172951)(9.03321326712253,0.586499363535835)(9.21504493196616,0.59138679371286)(9.22789652314067,0.591731993637182)(9.2589571245365,0.592566169576602)(9.28787984168808,0.593342769236878)(9.31588169361822,0.594094496393647)(9.38588466780407,0.595973144804666)(9.38677073997261,0.595996918369115)(9.43730110319872,0.597352429003882)(9.45674609550803,0.5978739323065)(9.47751475557526,0.59843086140243)(9.48533355358657,0.59864050924719)(9.56060988878918,0.600658365924318)(9.61173877065987,0.602028361182491)(9.65700014260739,0.603240758970675)(9.66204381341561,0.603375839854617)(9.68246681262978,0.603922769224291)(9.68986737227345,0.604120939099607)(9.73232706050636,0.605257730710565)(9.74801308226074,0.605677621790456)(9.85492231956084,0.608538314440399)(9.91185834670384,0.610061035951078)(9.97189766941129,0.611666170955997)(9.97706435658068,0.611804273135029)(9.98488494105514,0.612013303900622)(9.99288187670442,0.612227037827939)(10.0030476137307,0.612498722091779)(10.0157792036451,0.612838956088871)(10.0434354445969,0.613577938995682)(10.0456616775056,0.613637419144782)(10.0863290581748,0.614723821246718)(10.1569231167594,0.616609053063068)(10.2488718740315,0.61906334914889)(10.4400738912881,0.624162505034946)(10.447505810916,0.624360586168868)(10.4853897851832,0.625370158315532)(10.5013982052365,0.625796697139653)(10.6495745477974,0.629742812096815)(10.6521491299115,0.629811344440456)(10.6621018256963,0.630076263200997)(10.7836492099752,0.633310262408057)(10.7952666032507,0.633619236502187)(10.8034723223387,0.633837460719521)(10.8035494044405,0.633839510599929)(10.8717968040689,0.63565405402987)(10.9449808807533,0.637598977505737)(10.9569861523102,0.63791793977361)(10.966070724274,0.638159286981381)(11.056192230648,0.640552748310908)(11.1116902994893,0.642025970726406)(11.1774074105842,0.643769757204879)(11.1776347179455,0.643775787406281)(11.2280069598015,0.645111874486174)(11.3994380633007,0.649655451794401)(11.4216800999894,0.650244543167224)(11.4677619990933,0.651464739522522)(11.4906172227899,0.652069765293878)(11.505631249217,0.652467161741693)(11.5412785931597,0.653410507249375)(11.6136093845745,0.655323822434723)(11.6226367957172,0.655562542238819)(11.6523324205194,0.656347689154625)(11.7731759431761,0.659540812988913)(11.8138692781566,0.66061534940259)(11.9357735457067,0.663832011998521)(11.985766840482,0.665150135853536)(12.1015921148104,0.668201533666088)(12.1170986399649,0.668609780805599)(12.1580678045856,0.669688076544372)(12.1675435874922,0.669937409681089)(12.2082073508724,0.671007092320316)(12.2297714428188,0.671574154033006)(12.235440984452,0.671723220933431)(12.2382004215226,0.671795770235025)(12.2479502089007,0.672052087162168)(12.2490413911647,0.672080772029025)(12.4307978516867,0.676853591346929)(12.4418996462824,0.677144768553149)(12.4450328254061,0.677226937745399)(12.5314430449684,0.679491727509162)(12.5509937205041,0.680003770455839)(12.5717129820623,0.680546262947438)(12.6540488223908,0.682700417527674)(12.6722388593717,0.683175957258729)(12.6874781661325,0.683574250923115)(12.7541146828591,0.685314695967069)(13.0031194178548,0.691799802070376)(13.0161999861021,0.69213958108878)(13.056700770572,0.693191003625315)(13.0671632002418,0.693462458515029)(13.1974931276839,0.696838281018434)(13.2354532022626,0.69781943216463)(13.2784435029461,0.698929375449308)(13.2788389540507,0.698939579188086)(13.3106216816548,0.69975928302125)(13.3699317181015,0.701286874581851)(13.4147659165607,0.702439756519054)(13.4203373214795,0.70258290482236)(13.4321136998227,0.702885393222566)(13.4914982814058,0.704408904686215)(13.5520129491691,0.705958095594486)(13.6935238767641,0.709566491135651)(13.7128582861063,0.710057816033286)(13.7405123366957,0.710759807028983)(13.8648842093809,0.713905312112683)(13.8920539036301,0.714589761026494)(13.9051859467814,0.714920210247762)(13.9156523091321,0.715183406344126)(14.0567172776631,0.718714667246796)(14.2098949895353,0.722511308720269)(14.2354882690963,0.723141365654148)(14.3702929063551,0.726437373971193)(14.4147699192379,0.727515904666189)(14.4163691319331,0.727554596500948)(14.5213679111103,0.730080961191485)(14.528495688895,0.730251424357688)(14.5520186207928,0.730813006135829)(14.582425675469,0.731536673385907)(14.7002130368673,0.734314524425307)(14.7128582936728,0.734610227740956)(14.7167948172305,0.734702177954992)(14.7335723323483,0.735093514683314)(14.7641570968312,0.735804554742412)(14.8649306314654,0.738124998469641)(14.9355173824019,0.73972886223996)(14.9661796959084,0.740419789068127)(15.0567435481393,0.742439245078162) 
};
\addlegendentry{$B = 1$};

\addplot [
color=magenta,
solid,
line width=1.0pt,
]
coordinates{
 (1.01071130720986,0.422205180754055)(1.02444665090105,0.423114136749262)(1.1872934431047,0.433846816496851)(1.23250332110668,0.436812038589856)(1.27611878927235,0.439666741514382)(1.28303492440161,0.440118877160925)(1.31558531239563,0.442244852174904)(1.48587836506519,0.453313924684418)(2.02645740144039,0.487859954264289)(2.02802449729448,0.487958809455477)(2.23948216055479,0.501231267299079)(2.26137003054263,0.502597655727431)(2.28790473119636,0.504252292385898)(2.29079794272628,0.504432584213096)(2.4260949912022,0.51283714917377)(2.49599393610754,0.517159018203519)(2.9790155818445,0.546665396917298)(3.02885992500979,0.549676383797205)(3.03030616204858,0.549763657947556)(3.03347200863538,0.549954685500362)(3.20604191708145,0.560331660553359)(3.27365064246032,0.564378365481172)(3.30021041377436,0.565965296001223)(3.49613306754972,0.57762440868457)(3.83088323805407,0.597365152646345)(3.87238504702756,0.599797890741604)(3.95527297235397,0.604647492633955)(3.98260557409102,0.606244072769039)(3.99980055645596,0.607247837503864)(4.22963967959693,0.62061856355321)(4.24639985290517,0.621590327660857)(4.50026098909019,0.636259006714123)(4.71010712699698,0.648317950560005)(4.71205576469691,0.648429667966301)(4.78273083130257,0.652478416278133)(4.92918902014951,0.660849627743353)(4.96526414973994,0.66290778219696)(4.97772658699614,0.663618443802073)(5.04485054940203,0.667443125334875)(5.25198879706628,0.679214058347295)(5.58480682903451,0.698027893711214)(5.71266277752334,0.705222743991839)(5.72115998017838,0.705700250140792)(5.9410203842468,0.718025573623737)(5.9710789419758,0.719705955979912)(6.045610124749,0.723867277766509)(6.15963099013593,0.730217979428697)(6.22184696652025,0.733674639771426)(6.48450007974802,0.74818455499116)(6.55162410484249,0.751865877733035)(6.72147980148375,0.761113399137512)(6.73900230899034,0.762060675454814)(6.83023296600711,0.766967387910112)(6.884646417016,0.76987077604689)(7.16516758675935,0.784436905131503)(7.2219881907089,0.787273094382045)(7.38779788602712,0.795225290596525)(7.48661314488673,0.799681395942355)(7.57035517263405,0.803263707757261)(7.60227347496313,0.80458021691945)(7.61852008739808,0.805240035623136)(7.74595243626557,0.809110501707263)(8.11311980055337,0.817234136608784)(8.16605665120446,0.81802743177794)(8.20213546674745,0.819121203133506)(8.3618991684079,0.822596621177327)(8.38791027711785,0.823373829716596)(8.47987844985315,0.826009269494125)(8.48116094278507,0.826044732216484)(8.7461251184884,0.831930010606118)(8.95399066068279,0.834982265509621)(9.00216342496634,0.835475912388395)(9.03116615482801,0.835799563138789)(9.11539372502695,0.837297664903857)(9.48006144026467,0.842336753179184)(9.60448059812639,0.844068951622493)(9.67571115713866,0.844747581413754)(9.71355604089827,0.84522756533545)(9.74965357089672,0.845738646109526)(10.0009891203042,0.848025181869853)(10.0055663991989,0.848059960829192)(10.2276547056256,0.850329655192854)(10.2570540784071,0.850673388992192)(10.5825931434881,0.853691326727266)(10.605117136743,0.853840734054536)(10.7337784640375,0.854829775272152)(10.7499343674,0.854986509990047)(10.9854391545496,0.856675413671618)(11.0070582426538,0.856810050322569)(11.0122965491391,0.856848765216387)(11.3327970696088,0.859395376403159)(11.3883833097854,0.859724127051637)(11.3908312623224,0.859738481910555)(11.6490220016547,0.861492902134172)(11.7339685996266,0.862034096040663)(11.8782343131258,0.862859130565746)(11.884790389992,0.862906754879709)(11.9114654483148,0.863099601311295)(12.3157852499755,0.865706672670606)(12.3305160740469,0.865784091929927)(12.3884136216078,0.866092740131035)(12.5656637868929,0.867144340379815)(12.6056523120729,0.867366437167477)(12.7355755391531,0.868129230590885)(12.8865230628985,0.868918653097127)(13.0025109994513,0.869488086144181)(13.3324238113648,0.871327500366573)(13.3839515759047,0.871570797395362)(13.3885325417432,0.871592367561979)(13.4654230860231,0.87200252103518)(13.4776692306987,0.872061871554067)(13.6645739029305,0.873073050790381)(13.6794987131289,0.873151755939848)(13.7826803276516,0.873621910793668)(13.8411088789067,0.873886636454975)(14.1161292360614,0.875294614808963)(14.1352735884484,0.875379254579095)(14.3386252480515,0.876334078145893)(14.4048911387851,0.876670929706484)(14.4425306431434,0.876866131144766)(14.5647580996392,0.87746678326208)(14.6403838738341,0.877803272336678)(14.6694753668845,0.877922037962099)(15.0466505634992,0.879562304027506) 
};
\addlegendentry{$B = 5$};

\addplot [
color=green!50!black,
solid,
line width=1.0pt,
]
coordinates{
 (1.06799913999964,0.602709038195672)(1.09722697801927,0.605199562085167)(1.12856647988888,0.607867022570516)(1.20723282376708,0.614548464264622)(1.27948719989772,0.62066525774314)(1.28959084001615,0.621518855435219)(1.35647026147049,0.627156982096241)(1.50362747951349,0.639477015915647)(2.0863827212271,0.68704528277144)(2.10839449593838,0.688813634501726)(2.14738635784453,0.691943445698963)(2.22258130946649,0.697971789752039)(2.28787387163235,0.703200755663982)(2.30282012849205,0.704397240130088)(2.37142769995215,0.709887795315785)(2.51836541387944,0.721639773723517)(2.96500946292726,0.757642209586542)(3.11790247346327,0.770347541470361)(3.16145637176215,0.773945185478585)(3.17401739404854,0.774969669581306)(3.23872089982699,0.780074077231959)(3.31254702759792,0.784131483811)(3.37925001717601,0.787104873987749)(3.40654487178008,0.788294928068258)(4.09992916744448,0.820073509536883)(4.12324324823119,0.821203140583132)(4.16867096953124,0.82342531804706)(4.24104715844383,0.826991740930974)(4.29927190548326,0.829808512538816)(4.33024200092227,0.831084173392904)(4.3897136460847,0.83284820345326)(4.53799586967306,0.837012948972583)(5.03877805388183,0.851155821024684)(5.10706395835558,0.853000876375062)(5.1365596556612,0.85379088242996)(5.1754800572253,0.854821278374042)(5.30872879715283,0.85787782832007)(5.33587942788084,0.858348852948676)(5.39131263478015,0.859284399836248)(5.45277929925354,0.860300263601661)(6.09031824039678,0.871655449171508)(6.11622101590758,0.872152392898921)(6.18258235812309,0.873406788719569)(6.25068359431543,0.874642623543043)(6.32098943880624,0.875834886034367)(6.34475291099661,0.876167590255443)(6.39622164722174,0.876763319873205)(6.5525851383928,0.878592339038753)(7.05428497670062,0.884781137724149)(7.12021097064694,0.885540212023961)(7.17944228498922,0.886161116244054)(7.25785698089544,0.886896632557537)(7.3315612577268,0.887558116012504)(7.35539014405221,0.887770122493314)(7.39039167984443,0.888081724888473)(7.56001801168528,0.889604806810512)(8.02976902789537,0.893672267134659)(8.0358402635477,0.893724402559713)(8.16150345790775,0.894822551042998)(8.19886638338836,0.895162599549224)(8.33580917812698,0.896234125294652)(8.35945566403294,0.896368303426586)(8.40227494255649,0.896601771914566)(8.56661845026677,0.897526370256428)(9.00763827099145,0.90045873973038)(9.10646775587947,0.901170924585912)(9.12879032848755,0.901323292195408)(9.2720469125779,0.902159191652357)(9.34687229152578,0.902537072088309)(9.36972298369516,0.902663478355685)(9.37594657457543,0.902698650183477)(9.49236166041513,0.903397063071294)(10.0084995376655,0.906463892027045)(10.13485029499,0.907089116472621)(10.1516161803803,0.907168522315469)(10.2320823268501,0.907551066810119)(10.2760793356558,0.907832397231067)(10.3128932999745,0.908076396550446)(10.3739779501945,0.908487424214547)(10.4364982368791,0.908914552245087)(11.0007345719493,0.912600906054909)(11.1207936703978,0.913345140570777)(11.1362298441653,0.913442299856523)(11.1861286157393,0.913768268061262)(11.2785981829682,0.914475040229796)(11.2795238831282,0.91448252670753)(11.380120565246,0.915284810789707)(11.4310556794439,0.91568556490045)(12.0160236722172,0.920407316815386)(12.1379062065921,0.921385787501975)(12.1947539177048,0.921796705704779)(12.2056140907679,0.921864453216232)(12.2635618439919,0.922153321898254)(12.2827263105279,0.922243661366629)(12.3647148009413,0.922617148758294)(12.4434462755013,0.922973102870145)(12.9474949129654,0.925557486092796)(12.9894318008664,0.925786673974285)(13.0436724366926,0.926077113367535)(13.1292323025711,0.926514207094187)(13.2117878684956,0.92687719627914)(13.2836099256445,0.927154834250506)(13.3260598122578,0.92732632322463)(13.4499538081486,0.927854054190176)(13.9912050496885,0.930229243475835)(14.0601116350987,0.930512254956633)(14.1175038127071,0.930730658689891)(14.1825055166885,0.930953259348355)(14.2084513487097,0.931034856769693)(14.2663053602753,0.931221261648093)(14.3011968143463,0.931331249111327)(14.393481141401,0.931625250143318)(14.996928607842,0.933577340181967)(15.0848286012534,0.93388681956742) 
};
\addlegendentry{$B = 15$};

\addplot [
color=orange,
solid,
line width=1.0pt,
]
coordinates{
 (1.08315556955529,0.668084062873174)(1.15222102273396,0.677936376847547)(1.2031049747275,0.685197139397285)(1.21239049961454,0.686522695811531)(1.24819266484039,0.691635584820088)(1.2791481266906,0.696058422255026)(1.31753814493303,0.701545727770446)(1.53366967581589,0.73239242742936)(2.10285311947107,0.811048717463907)(2.17403541506915,0.820784555095412)(2.22673278987887,0.828050574935372)(2.23508695184893,0.829302089326579)(2.26612011961165,0.833037583676838)(2.2984435392437,0.835795586960118)(2.32896926559545,0.837414822057002)(2.5507304225807,0.848579453675319)(3.11784036291964,0.872222046069263)(3.18607510091114,0.875252788739806)(3.24177163372309,0.877768038048762)(3.24875003430429,0.878083851079077)(3.28585717947922,0.87924978989755)(3.31680021317205,0.880069889710963)(3.34727107599286,0.880713573568267)(3.55776419478956,0.883755889070584)(4.13485455192554,0.889790678187083)(4.20539339369873,0.890819064184186)(4.25900731580171,0.891913882267865)(4.25980807653104,0.891932433211888)(4.30372895702931,0.892771907585427)(4.32716218221225,0.893178725551059)(4.36718615189609,0.893822372971443)(4.57370193267126,0.896281568541841)(5.14254712193031,0.902144027637163)(5.21608101808304,0.903203336542037)(5.2681986358511,0.904059231159089)(5.28069238744788,0.90428210924923)(5.31508124664559,0.90470634757475)(5.33456966232229,0.904937189945413)(5.37547379319609,0.905354545445129)(5.58784510529113,0.907139944890103)(6.16250010328593,0.91175203326003)(6.1703273900802,0.91182376569332)(6.27659223051368,0.91290171391119)(6.29448925054608,0.913082541153195)(6.32917492634828,0.913425884094194)(6.35531477117419,0.913669310203095)(6.39381512974279,0.913872196497471)(6.6064345794306,0.914240202155135)(7.17800893720993,0.916092317501562)(7.23547655088735,0.916305365645722)(7.27517324258519,0.916558553748562)(7.3105192506746,0.916772204380601)(7.3404632077727,0.917063718546708)(7.36168821944024,0.917340394927566)(7.3916641599435,0.917640478794326)(7.61879119452537,0.918660501353004)(8.18893428823727,0.921696049568655)(8.24586565188693,0.922340274512539)(8.29460205458899,0.922847825088403)(8.31985166739094,0.922976380289242)(8.34413257726324,0.923124585175007)(8.35253549349031,0.923202692705977)(8.37317703900916,0.923394266169384)(8.62287840455345,0.924214398746498)(9.18913639495836,0.925810413357067)(9.23105103775596,0.926103328367743)(9.29110115771451,0.926690066744964)(9.33272095910105,0.927046752496034)(9.33864811685288,0.927076335752212)(9.35111452336157,0.927174757482038)(9.38068761869613,0.927632730171071)(9.63198159236236,0.928722432680855)(10.1856309332787,0.929909854130385)(10.2316904291743,0.930228333675844)(10.2507372219722,0.930403350090792)(10.3094791319187,0.930988814688618)(10.3748504156047,0.931472261643316)(10.3889738004737,0.931627910230036)(10.398297572285,0.931719281742888)(10.6522587147941,0.932142650385751)(11.1863409100782,0.931917681858403)(11.2269118193186,0.932115203272301)(11.2855351887863,0.932672967089362)(11.302356234079,0.932870078785133)(11.3804134863833,0.933614024959909)(11.3889295578462,0.933676884364259)(11.3990594264078,0.933742263740262)(11.6298987164905,0.934238574333333)(12.1837022288034,0.934435868605126)(12.2333127265147,0.934742814043649)(12.2915901010017,0.935254178977544)(12.298539779447,0.935333835928495)(12.3689498072657,0.935921294661203)(12.3874412605783,0.936054852923251)(12.4041729599631,0.93616155114989)(12.608188116168,0.936712512164894)(13.1450563493479,0.937112705451296)(13.1933636865809,0.937291876380009)(13.2825229409089,0.937968941089949)(13.3271108891479,0.938146672663008)(13.3284476723195,0.938152714197694)(13.3794377102856,0.938466474417298)(13.3948588465342,0.938573849805974)(13.5821295044445,0.939024750933006)(14.172433137022,0.93907822176958)(14.2330173828288,0.93930956605485)(14.2741182568974,0.93950975317021)(14.3175253902496,0.939730331527465)(14.3267937792393,0.939779825640572)(14.3326220903313,0.939807052779976)(14.3982813937849,0.940255106658587)(14.6102866957182,0.941460803072674)(15.1805679015249,0.943428316634734) 
};
\addlegendentry{$B = 30$};

\addplot [
color=red,
solid,
line width=1.0pt,
]
coordinates{
 (1.11149289167308,0.809662843509571)(1.12781351240186,0.809633910455327)(1.40335762628549,0.812553805162551)(1.40518261597728,0.812589484322408)(1.42197757464083,0.812924156373275)(1.74525632931462,0.826025814473595)(1.77126350174107,0.829308202972498)(1.86390829692007,0.841716147340677)(2.15163171722714,0.870883278816085)(2.269317483721,0.878112640220468)(2.43953634092311,0.892684419870028)(2.46004085563838,0.893929085387345)(2.66165343230016,0.895317045189569)(2.8230171625728,0.895935723377203)(2.98716682081622,0.900636065088147)(3.19683252625431,0.906556780621398)(3.26114170250946,0.907953807452631)(3.34239703502045,0.909929276492119)(3.46315039466313,0.912034723685888)(3.50306521572065,0.912535789980292)(3.71074609584112,0.914894368345758)(3.8569853199401,0.917068136534757)(4.11450637805023,0.923431490982326)(4.31487352779505,0.926274168552053)(4.40275150290074,0.927308290359748)(4.51625439787484,0.927345077314949)(4.54365869308746,0.927305619021813)(4.79387949903937,0.926253217172131)(4.84906642306763,0.926894531866861)(4.98650202681761,0.9255173277963)(5.19090072944019,0.932241077978564)(5.35505961856854,0.937378828003391)(5.43699226573918,0.939726387303061)(5.55004147163531,0.939404934403697)(5.56360673574246,0.939310195634086)(5.89101134639636,0.938115410531811)(6.0687730301649,0.936420542842528)(6.10898404254417,0.936821210294741)(6.2417985227643,0.938089771877021)(6.46327767754905,0.940000569894224)(6.47927731959705,0.940111938209238)(6.58379046378205,0.940555006472282)(6.62015646427086,0.940729354422192)(6.9415227333492,0.942409650781199)(7.17732717638826,0.941935047604525)(7.27347777550513,0.942373872506511)(7.32038339621963,0.942622060525622)(7.54232736323775,0.943436176642489)(7.62613328192513,0.943343868152989)(7.65362602154484,0.943581165812542)(7.82856149419802,0.945310275736207)(7.9711341172577,0.947534808375679)(8.2181446130085,0.94942232286411)(8.30811979668052,0.94958026419976)(8.39551002308015,0.949243102003962)(8.54996488562034,0.948798783733405)(8.68986086215503,0.948246782899658)(8.69740592647489,0.948334249367989)(8.98616008968897,0.950040020790515)(9.10924261110429,0.95145816042333)(9.28186142623613,0.953031469836335)(9.34912270235765,0.953942869963779)(9.46057007168314,0.954298776280023)(9.7203155050461,0.954609727570081)(9.7237129736957,0.954597867484224)(9.75587711515752,0.954490758962698)(10.0110615554117,0.955522155474343)(10.1070056908633,0.955997962678143)(10.3806216372598,0.956554067198923)(10.3964045212476,0.956505202798827)(10.4989018238386,0.956273734333403)(10.7544802178403,0.955835052125531)(10.7656806494118,0.955842587136711)(10.8938690442694,0.956039657686681)(10.9959576053303,0.956171244517135)(11.1427846539831,0.956903176591067)(11.4268030088859,0.958449335492206)(11.5234232694489,0.958925367674799)(11.6559984099122,0.958728448005193)(11.8925393446645,0.95860989508375)(11.9609681112911,0.958785706688163)(11.9735239961966,0.958805807840525)(12.0568011394164,0.958889023566573)(12.1617184880713,0.959104823114163)(12.3851047593255,0.959704927491914)(12.505842798169,0.960498095336127)(12.8686770991451,0.958853191406622)(12.8896694322895,0.958830912998484)(12.8900075333581,0.958831237757411)(13.0387597154088,0.959371437199149)(13.1488319716816,0.959800346463314)(13.5757830931432,0.961533919982895)(13.6050489359874,0.961682132946285)(13.7536490109914,0.963120761579556)(13.9463332286157,0.961104953793693)(13.9529645387615,0.961038382746472)(14.0626766523005,0.960863839764516)(14.0656652544696,0.960862098267915)(14.1635230472342,0.960479587030028)(14.4416797619046,0.962408294567738)(14.7023267088234,0.962317176404387)(14.9795715897188,0.96281162796925)(15.0633805431559,0.962732149549881) 
};
\addlegendentry{$B = 60$};

\end{axis}
\end{tikzpicture}%

\renewcommand\trimlen{2pt}
\begin{figure}[tb]
  \begin{subfigure}[b]{0.49\textwidth}
    \centering
    \adjincludegraphics[width=\linewidth,clip=true,trim=\trimlen{} \trimlen{} \trimlen{} \trimlen{}]{figures/ev_chl_pp}
    \caption{\textsf{[C]} Explicit threshold}
	  \label{fig:chl-pp}
  \end{subfigure}
  \hfill
  \begin{subfigure}[b]{0.49\textwidth}
    \centering
    \adjincludegraphics[width=\linewidth,clip=true,trim=\trimlen{} \trimlen{} \trimlen{} \trimlen{}]{figures/ev_bgape_pp}
    \caption{\textsf{[A]} Explicit threshold}
	\label{fig:bgape-pp}
  \end{subfigure}

  \caption{Performance of the path planning algorithm for different batch
           sizes on the two environmental monitoring datasets.
           The total lengths of the sampling paths are normalized by the
           lake transect length ($1478$ m).
           Note the dramatically reduced travel lengths as the batch size is
           increased ($B = 1$ corresponds to strictly sequential sampling).
           }
  \label{fig:exp-pp}
\end{figure}
