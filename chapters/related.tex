\chapter{Related work} \label{ch:related}
Previous work on level set~\cite{dantu07,srinivasan08} and
boundary~\cite{singh06} estimation and tracking in the context of mobile
sensor networks has primarily focused on controlling the movement and
communication of sensor nodes, without giving much attention to
individual sampling locations and the choice thereof.

In contrast, we consider the problem of level set estimation in the setting of
\emph{pool-based active learning}~\cite{settles09}, where we need to make
sequential decisions by choosing sampling locations from a given set.
For this problem, \citet{bryan05} have proposed the
\emph{straddle} heuristic, which selects where to sample by trading off
uncertainty and proximity to the desired threshold level, both estimated
using GPs.
However, no theoretical justification has been given for the use of straddle,
neither for its extension to composite functions~\cite{bryan08}.
\citet{garnett12} consider the problem of
\emph{active search}, which is also about sequential sampling from a domain of
two (or more) classes (in our case the super- and sublevel sets).
In contrast to our goal of detecting the class boundaries, however,
their goal is to sample as many points as possible from one of the classes.

In the setting of multi-armed bandit optimization, which is similar to ours
in terms of sequential sampling, but different in terms of objectives,
GPs have been used both for modeling, as well as for sample
selection~\mbox{\cite{brochu10}}. In particular, the \gpucb algorithm
makes use of GP-inferred upper confidence bounds for selecting samples and
has been shown to achieve sublinear regret~\cite{srinivas10}.
An extension of \gpucb to the multi-objective
optimization problem has been proposed by
\citet{zuluaga13}, who use a similar GP-based
classification scheme to ours to classify points as being Pareto-optimal
or not.

Existing approaches for performing multiple evaluations in
parallel in the context of GP optimization, include
\emph{simulation matching}~\cite{azimi10}, which combines GP modeling with
Monte-Carlo simulations, and the \gpbucb~\cite{desautels12} algorithm,
which obtains similar regret bounds to \gpucb, and from which we borrow
the main idea for performing batch sample selection.

To our knowledge, there has been no previous work on actively estimating
level sets with respect to implicitly defined threshold levels.