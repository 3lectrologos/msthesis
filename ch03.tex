\chapter{Extensions} \label{sect:ext} 
We now extend \acl to deal with the two problem variants introduced
in \sectref{sect:prelim}, namely implicitly defined threshold levels and
batch point selection. We highlight the key differences
in the extended versions of the algorithm and the resulting implications about
the convergence bound of \theoremref{thm:acl}.

\section{Implicit threshold level} \label{sect:iacl}
The substitution of the explicit threshold level by an implicit level
${h = \omega \max_{\*x\in D} f(\*x)}$ requires modifying the
classification rules as well as the selection rule of \acl, which results
in what we call the \iacl algorithm.

Since $h$ is now an estimated quantity that depends
on the function maximum, we have to take into account the
uncertainty associated with it when making classification decisions.
Concretely, we can obtain an \emph{optimistic estimate} of the function
maximum as ${f^{opt}_{t} = \max_{\*x\in U_t} \max(C_t(\*x))}$ and, analogously,
a \emph{pessimistic estimate} as
${f^{pes}_{t} = \max_{\*x\in U_t} \min(C_t(\*x))}$. The
corresponding estimates of the implicit level are defined as
${h^{opt}_t = \omega f^{opt}_t}$ and ${h^{pes}_t = \omega f^{pes}_t}$, and
can be used in a similar classification scheme to that of \acl.
However, for the above estimates to be correct, we have to ensure that $U_t$
always contains all points that could be maximizers of $f$,
i.e. all points that satisfy $\max(C_t(\*x)) \geq f^{pes}_{t}$.
For that purpose, points that should be classified, but are still possible
function maximizers according to the above inequality, are kept in two sets
$M^H_t$ and $M^L_t$ respectively, while a new set
$Z_t = U_t \cup M^H_t \cup M^L_t$ is used in place of $U_t$ to
obtain the optimistic and pessimistic estimates $h^{opt}_t$
and $h^{pes}_t$.
The resulting classification rules are shown in
\linesref{ln:iclass1}{ln:iclass2} of
\algoref{alg:iacl}, where the conditions are,
again, relaxed by an accuracy parameter $\epsilon$.

\begin{algorithm}[tb]
\caption{The \iacl extension}
\label{alg:iacl}
\small{
\begin{algorithmic}[1]
  \REQUIRE sample set $D$, GP prior ($\mu_0 = 0$, $k$, $\sigma_0$),\\
           \hspace{1.35em}threshold ratio $\omega$, accuracy parameter $\epsilon$
  \ENSURE predicted sets $\hat{H}$, $\hat{L}$
  \STATE $H_0 \gets \varnothing$,\enskip $L_0 \gets \varnothing$,\enskip $U_0 \gets D$\new{,\enskip $Z_0 \gets D$}
  %\LET{$H_0$}{$\varnothing$} \label{lin:init1}
  %\LET{$L_0$}{$\varnothing$}
  %\LET{$U_0$}{$D$}
  %\LET{$Z_0$}{$D$}
  \LET{$C_0(\*x)$}{$\mathbb{R}$, for all $\*x \in D$} \label{lin:init2}
  \LET{$t$}{1}
  \WHILE{$U_{t-1} \neq \varnothing$}
    \STATE $H_t \gets H_{t-1}$,\enskip $L_t \gets L_{t-1}$,\enskip $U_t \gets U_{t-1}$\new{,\enskip $Z_t \gets Z_{t-1}$}
    %\LET{$H_t$}{$H_{t-1}$}
    %\LET{$L_t$}{$L_{t-1}$}
    %\LET{$U_t$}{$U_{t-1}$}
    %\LET{$Z_t$}{$Z_{t-1}$}
    \FORALL{$\*x \in U_{t-1}$}
      \LET{$C_{t}(\*x)$}{$C_{t-1}(\*x) \cap Q_t(\*x)$}
      \new{
      \LET{$h^{opt}_{t}$}{$\omega\max_{\*x\in Z_{t-1}}\max(C_t(\*x))$}
      \LET{$f^{pes}_{t}$}{$\max_{\*x\in Z_{t-1}}\min(C_t(\*x))$}
      \LET{$h^{pes}_{t}$}{$\omega f_{pes}$}
      }
      \IF{$\min(C_t(\*x)) + \epsilon \geq h^{\new{opt}}_{\new{t}}$} \label{ln:iclass1}
        \LET{$U_t$}{$U_t \setminus \{\*x\}$}
        \new{
        \IF{$\max(C_t(\*x)) < f^{pes}_{t}$}
        \LET{$H_t$}{$H_t \cup \{\*x\}$}
        \ELSE
        \LET{$M^H_t$}{$M^H_t \cup \{\*x\}$}
        \ENDIF
        }
      \ELSIF{$\max(C_t(\*x)) - \epsilon \leq h^{\new{pes}}_{\new{t}}$}
        \LET{$U_t$}{$U_t \setminus \{\*x\}$}
        \new{
        \IF{$\max(C_t(\*x)) < f^{pes}_{t}$}
        \LET{$L_t$}{$L_t \cup \{\*x\}$}
        \ELSE
        \LET{$M^L_t$}{$M^L_t \cup \{\*x\}$}
        \ENDIF
        }
      \ENDIF \label{ln:iclass2}
    \ENDFOR
    \new{
    \LET{$Z_t$}{$U_t \cup M^H_t \cup M^L_t$}
    }
    \LET{$\*x_t$}{$\argmax_{\*x \in \new{Z_t}}(\new{w_t}(\*x))$}
    \LET{$y_t$}{$f(\*x_t) + \nu_t$} \label{lin:sel2}
    \STATE Compute $\mu_t(\*x)$ and $\sigma_t(\*x)$, for all $\*x \in U_t$ \label{lin:inf}
    \LET{$t$}{$t + 1$}
  \ENDWHILE
  \LET{$\hat{H}$}{$H_{t-1} \new{\ \cup\ M^H_{t-1}}$}
  \LET{$\hat{L}$}{$L_{t-1} \new{\ \cup\ M^L_{t-1}}$}
\end{algorithmic}
}
\end{algorithm}

In contrast to \acl, which solely focuses on sampling the most ambiguous
points, in \iacl it is also of importance to have a more exploratory
sampling policy in order to obtain more accurate estimates
$f^{opt}_{t}$ and $f^{pes}_{t}$. To this end, we select at each
iteration the point with the largest confidence region \emph{width},
defined as
\begin{align*}
w_t(\*x) = \max(C_t(\*x)) - \min(C_t(\*x)).
\end{align*}
Note that, if confidence intervals were not intersected, this would be
equivalent to maximum variance sampling (within $Z_t$).

%\setlength\figureheight{1.5in}\setlength\figurewidth{2.5in}
%% This file was created by matlab2tikz v0.2.3.
% Copyright (c) 2008--2012, Nico Schlömer <nico.schloemer@gmail.com>
% All rights reserved.
% 
% 
%

\definecolor{locol}{rgb}{0.26, 0.45, 0.65}

\begin{tikzpicture}

\begin{axis}[%
tick label style={font=\tiny},
label style={font=\tiny},
xlabel shift={-10pt},
ylabel shift={-17pt},
legend style={font=\tiny},
view={0}{90},
width=\figurewidth,
height=\figureheight,
scale only axis,
xmin=0, xmax=1478,
xtick={0, 400, 1000, 1400},
xlabel={Length (m)},
ymin=-18, ymax=0,
ytick={0, -4, -14, -18},
ylabel={Depth (m)},
name=plot1,
axis lines*=box,
tickwidth=0.1cm,
clip=false
]

\addplot [fill=locol,draw=none,forget plot] coordinates{ (1478.00014645918,0)(1478.00007323068,-0.0602006688963211)(1478,-0.120401337792642)(1478,-0.180602006688963)(1478,-0.240802675585284)(1478,-0.301003344481605)(1478,-0.361204013377926)(1478,-0.421404682274247)(1478,-0.481605351170569)(1478,-0.54180602006689)(1478,-0.602006688963211)(1478,-0.662207357859532)(1478,-0.722408026755853)(1478,-0.782608695652174)(1478,-0.842809364548495)(1478,-0.903010033444816)(1478,-0.963210702341137)(1478,-1.02341137123746)(1478,-1.08361204013378)(1478,-1.1438127090301)(1478,-1.20401337792642)(1478,-1.26421404682274)(1478,-1.32441471571906)(1478,-1.38461538461538)(1478,-1.44481605351171)(1478,-1.50501672240803)(1478,-1.56521739130435)(1478,-1.62541806020067)(1478,-1.68561872909699)(1478,-1.74581939799331)(1478,-1.80602006688963)(1478,-1.86622073578595)(1478,-1.92642140468227)(1478,-1.9866220735786)(1478,-2.04682274247492)(1478,-2.10702341137124)(1478,-2.16722408026756)(1478,-2.22742474916388)(1478,-2.2876254180602)(1478,-2.34782608695652)(1478,-2.40802675585284)(1478,-2.46822742474916)(1478,-2.52842809364549)(1478,-2.58862876254181)(1478,-2.64882943143813)(1478,-2.70903010033445)(1478,-2.76923076923077)(1478,-2.82943143812709)(1478,-2.88963210702341)(1478,-2.94983277591973)(1478,-3.01003344481605)(1478,-3.07023411371237)(1478,-3.1304347826087)(1478,-3.19063545150502)(1478,-3.25083612040134)(1478,-3.31103678929766)(1478,-3.37123745819398)(1478,-3.4314381270903)(1478,-3.49163879598662)(1478,-3.55183946488294)(1478,-3.61204013377926)(1478,-3.67224080267559)(1478,-3.73244147157191)(1478,-3.79264214046823)(1478,-3.85284280936455)(1478,-3.91304347826087)(1478,-3.97324414715719)(1478,-4.03344481605351)(1478,-4.09364548494983)(1478,-4.15384615384615)(1478,-4.21404682274247)(1478,-4.2742474916388)(1478,-4.33444816053512)(1478,-4.39464882943144)(1478,-4.45484949832776)(1478,-4.51505016722408)(1478,-4.5752508361204)(1478,-4.63545150501672)(1478,-4.69565217391304)(1478,-4.75585284280936)(1478,-4.81605351170569)(1478,-4.87625418060201)(1478,-4.93645484949833)(1478,-4.99665551839465)(1478,-5.05685618729097)(1478,-5.11705685618729)(1478,-5.17725752508361)(1478,-5.23745819397993)(1478,-5.29765886287625)(1478,-5.35785953177258)(1478,-5.4180602006689)(1478,-5.47826086956522)(1478,-5.53846153846154)(1478,-5.59866220735786)(1478,-5.65886287625418)(1478,-5.7190635451505)(1478,-5.77926421404682)(1478,-5.83946488294314)(1478,-5.89966555183946)(1478,-5.95986622073579)(1478,-6.02006688963211)(1478,-6.08026755852843)(1478,-6.14046822742475)(1478,-6.20066889632107)(1478,-6.26086956521739)(1478,-6.32107023411371)(1478,-6.38127090301003)(1478,-6.44147157190635)(1478,-6.50167224080268)(1478,-6.561872909699)(1478,-6.62207357859532)(1478,-6.68227424749164)(1478,-6.74247491638796)(1478,-6.80267558528428)(1478,-6.8628762541806)(1478,-6.92307692307692)(1478,-6.98327759197324)(1478,-7.04347826086957)(1478,-7.10367892976589)(1478,-7.16387959866221)(1478,-7.22408026755853)(1478,-7.28428093645485)(1478,-7.34448160535117)(1478,-7.40468227424749)(1478,-7.46488294314381)(1478,-7.52508361204013)(1478,-7.58528428093646)(1478,-7.64548494983278)(1478,-7.7056856187291)(1478.00001830787,-7.76588628762542)(1478.00007323068,-7.82608695652174)(1478.00012815226,-7.88628762541806)(1478.00016476597,-7.94648829431438)(1478.00016476597,-8.0066889632107)(1478.00016476597,-8.06688963210702)(1478.00016476597,-8.12709030100334)(1478.00016476597,-8.18729096989967)(1478.00016476597,-8.24749163879599)(1478.00016476597,-8.30769230769231)(1478.00016476597,-8.36789297658863)(1478.00016476597,-8.42809364548495)(1478.00016476597,-8.48829431438127)(1478.00016476597,-8.54849498327759)(1478.00016476597,-8.60869565217391)(1478.00016476597,-8.66889632107023)(1478.00016476597,-8.72909698996656)(1478.00016476597,-8.78929765886288)(1478.00016476597,-8.8494983277592)(1478.00016476597,-8.90969899665552)(1478.00016476597,-8.96989966555184)(1478.00016476597,-9.03010033444816)(1478.00016476597,-9.09030100334448)(1478.00016476597,-9.1505016722408)(1478.00020137914,-9.21070234113712)(1478.00020137914,-9.27090301003344)(1478.00020137914,-9.33110367892977)(1478.00016476597,-9.39130434782609)(1478.00016476597,-9.45150501672241)(1478.00016476597,-9.51170568561873)(1478.00016476597,-9.57190635451505)(1478.00016476597,-9.63210702341137)(1478.00016476597,-9.69230769230769)(1478.00016476597,-9.75250836120401)(1478.00016476597,-9.81270903010033)(1478.00016476597,-9.87290969899666)(1478.00016476597,-9.93311036789298)(1478.00016476597,-9.9933110367893)(1478.00016476597,-10.0535117056856)(1478.00016476597,-10.1137123745819)(1478.00016476597,-10.1739130434783)(1478.00016476597,-10.2341137123746)(1478.00016476597,-10.2943143812709)(1478.00016476597,-10.3545150501672)(1478.00016476597,-10.4147157190635)(1478.00016476597,-10.4749163879599)(1478.00016476597,-10.5351170568562)(1478.00012815226,-10.5953177257525)(1478.00012815226,-10.6555183946488)(1478.00007323068,-10.7157190635452)(1478.00005492321,-10.7759197324415)(1478,-10.8361204013378)(1478,-10.8963210702341)(1478,-10.9565217391304)(1478,-11.0167224080268)(1478,-11.0769230769231)(1478,-11.1371237458194)(1478,-11.1973244147157)(1478,-11.257525083612)(1478,-11.3177257525084)(1478,-11.3779264214047)(1478,-11.438127090301)(1478,-11.4983277591973)(1478,-11.5585284280936)(1478,-11.61872909699)(1478,-11.6789297658863)(1478,-11.7391304347826)(1478,-11.7993311036789)(1478,-11.8595317725753)(1478,-11.9197324414716)(1478,-11.9799331103679)(1478,-12.0401337792642)(1478,-12.1003344481605)(1478,-12.1605351170569)(1478,-12.2207357859532)(1478,-12.2809364548495)(1478,-12.3411371237458)(1478,-12.4013377926421)(1478,-12.4615384615385)(1478,-12.5217391304348)(1478,-12.5819397993311)(1478,-12.6421404682274)(1478,-12.7023411371237)(1478,-12.7625418060201)(1478,-12.8227424749164)(1478,-12.8829431438127)(1478,-12.943143812709)(1478,-13.0033444816054)(1478,-13.0635451505017)(1478,-13.123745819398)(1478,-13.1839464882943)(1478,-13.2441471571906)(1478,-13.304347826087)(1478,-13.3645484949833)(1478,-13.4247491638796)(1478,-13.4849498327759)(1478,-13.5451505016722)(1478,-13.6053511705686)(1478,-13.6655518394649)(1478,-13.7257525083612)(1478,-13.7859531772575)(1478,-13.8461538461538)(1478,-13.9063545150502)(1478,-13.9665551839465)(1478,-14.0267558528428)(1478,-14.0869565217391)(1478,-14.1471571906355)(1478,-14.2073578595318)(1478,-14.2675585284281)(1478,-14.3277591973244)(1478,-14.3879598662207)(1478,-14.4481605351171)(1478,-14.5083612040134)(1478,-14.5685618729097)(1478,-14.628762541806)(1478,-14.6889632107023)(1478,-14.7491638795987)(1478,-14.809364548495)(1478,-14.8695652173913)(1478,-14.9297658862876)(1478,-14.9899665551839)(1478,-15.0501672240803)(1478,-15.1103678929766)(1478,-15.1705685618729)(1478,-15.2307692307692)(1478,-15.2909698996656)(1478,-15.3511705685619)(1478,-15.4113712374582)(1478,-15.4715719063545)(1478,-15.5317725752508)(1478,-15.5919732441472)(1478,-15.6521739130435)(1478,-15.7123745819398)(1478,-15.7725752508361)(1478,-15.8327759197324)(1478,-15.8929765886288)(1478,-15.9531772575251)(1478,-16.0133779264214)(1478,-16.0735785953177)(1478,-16.133779264214)(1478,-16.1939799331104)(1478,-16.2541806020067)(1478,-16.314381270903)(1478,-16.3745819397993)(1478,-16.4347826086957)(1478,-16.494983277592)(1478,-16.5551839464883)(1478,-16.6153846153846)(1478,-16.6755852842809)(1478,-16.7357859531773)(1478,-16.7959866220736)(1478,-16.8561872909699)(1478,-16.9163879598662)(1478,-16.9765886287625)(1478,-17.0367892976589)(1478,-17.0969899665552)(1478,-17.1571906354515)(1478,-17.2173913043478)(1478,-17.2775919732441)(1478,-17.3377926421405)(1478,-17.3979933110368)(1478,-17.4581939799331)(1478,-17.5183946488294)(1478,-17.5785953177258)(1478,-17.6387959866221)(1478,-17.6989966555184)(1478,-17.7591973244147)(1478,-17.819397993311)(1478,-17.8795986622074)(1478,-17.9397993311037)(1478,-18)(1473.05685618729,-18)(1468.11371237458,-18)(1463.17056856187,-18)(1458.22742474916,-18)(1453.28428093645,-18)(1448.34113712375,-18)(1443.39799331104,-18)(1438.45484949833,-18)(1433.51170568562,-18)(1428.56856187291,-18)(1423.6254180602,-18)(1418.68227424749,-18)(1413.73913043478,-18)(1408.79598662207,-18)(1403.85284280936,-18)(1398.90969899666,-18)(1393.96655518395,-18)(1389.02341137124,-18)(1384.08026755853,-18)(1379.13712374582,-18)(1374.19397993311,-18)(1369.2508361204,-18)(1364.30769230769,-18)(1359.36454849498,-18)(1354.42140468227,-18)(1349.47826086957,-18)(1344.53511705686,-18)(1339.59197324415,-18)(1334.64882943144,-18)(1329.70568561873,-18)(1324.76254180602,-18)(1319.81939799331,-18)(1314.8762541806,-18)(1309.93311036789,-18)(1304.98996655518,-18)(1300.04682274247,-18)(1295.10367892977,-18)(1290.16053511706,-18)(1285.21739130435,-18)(1280.27424749164,-18)(1275.33110367893,-18)(1270.38795986622,-18)(1265.44481605351,-18)(1260.5016722408,-18)(1255.55852842809,-18)(1250.61538461538,-18)(1245.67224080268,-18)(1240.72909698997,-18)(1235.78595317726,-18)(1230.84280936455,-18)(1225.89966555184,-18)(1220.95652173913,-18)(1216.01337792642,-18)(1211.07023411371,-18)(1206.127090301,-18)(1201.18394648829,-18)(1196.24080267559,-18)(1191.29765886288,-18)(1186.35451505017,-18)(1181.41137123746,-18)(1176.46822742475,-18)(1171.52508361204,-18)(1166.58193979933,-18)(1161.63879598662,-18)(1156.69565217391,-18)(1151.7525083612,-18)(1146.8093645485,-18)(1141.86622073579,-18)(1136.92307692308,-18)(1131.97993311037,-18)(1127.03678929766,-18)(1122.09364548495,-18)(1117.15050167224,-18)(1112.20735785953,-18)(1107.26421404682,-18)(1102.32107023411,-18)(1097.3779264214,-18)(1092.4347826087,-18)(1087.49163879599,-18)(1082.54849498328,-18)(1077.60535117057,-18)(1072.66220735786,-18)(1067.71906354515,-18)(1062.77591973244,-18)(1057.83277591973,-18)(1052.88963210702,-18)(1047.94648829431,-18)(1043.00334448161,-18)(1038.0602006689,-18)(1033.11705685619,-18)(1028.17391304348,-18)(1023.23076923077,-18)(1018.28762541806,-18)(1013.34448160535,-18)(1008.40133779264,-18)(1003.45819397993,-18)(998.515050167224,-18)(993.571906354515,-18)(988.628762541806,-18)(983.685618729097,-18)(978.742474916388,-18)(973.799331103679,-18)(968.85618729097,-18)(963.913043478261,-18)(958.969899665552,-18)(954.026755852843,-18)(949.083612040134,-18)(944.140468227425,-18)(939.197324414716,-18)(934.254180602007,-18)(929.311036789298,-18)(924.367892976589,-18)(919.42474916388,-18)(914.481605351171,-18)(909.538461538462,-18)(904.595317725752,-18)(899.652173913044,-18)(894.709030100334,-18)(889.765886287625,-18)(884.822742474916,-18)(879.879598662207,-18)(874.936454849498,-18)(869.993311036789,-18)(865.05016722408,-18)(860.107023411371,-18)(855.163879598662,-18)(850.220735785953,-18)(845.277591973244,-18)(840.334448160535,-18)(835.391304347826,-18)(830.448160535117,-18)(825.505016722408,-18)(820.561872909699,-18)(815.61872909699,-18)(810.675585284281,-18)(805.732441471572,-18)(800.789297658863,-18)(795.846153846154,-18)(790.903010033445,-18)(785.959866220736,-18)(781.016722408027,-18)(776.073578595318,-18)(771.130434782609,-18)(766.1872909699,-18)(761.244147157191,-18)(756.301003344482,-18)(751.357859531773,-18)(746.414715719064,-18)(741.471571906354,-18)(736.528428093646,-18)(731.585284280936,-18)(726.642140468227,-18)(721.698996655518,-18)(716.755852842809,-18)(711.8127090301,-18)(706.869565217391,-18)(701.926421404682,-18)(696.983277591973,-18)(692.040133779264,-18)(687.096989966555,-18)(682.153846153846,-18)(677.210702341137,-18)(672.267558528428,-18)(667.324414715719,-18)(662.38127090301,-18)(657.438127090301,-18)(652.494983277592,-18)(647.551839464883,-18)(642.608695652174,-18)(637.665551839465,-18)(632.722408026756,-18)(627.779264214047,-18)(622.836120401338,-18)(617.892976588629,-18)(612.94983277592,-18)(608.006688963211,-18)(603.063545150502,-18)(598.120401337793,-18)(593.177257525084,-18)(588.234113712375,-18)(583.290969899666,-18)(578.347826086957,-18)(573.404682274248,-18)(568.461538461538,-18)(563.518394648829,-18)(558.57525083612,-18)(553.632107023411,-18)(548.688963210702,-18)(543.745819397993,-18)(538.802675585284,-18)(533.859531772575,-18)(528.916387959866,-18)(523.973244147157,-18)(519.030100334448,-18)(514.086956521739,-18)(509.14381270903,-18)(504.200668896321,-18)(499.257525083612,-18)(494.314381270903,-18)(489.371237458194,-18)(484.428093645485,-18)(479.484949832776,-18)(474.541806020067,-18)(469.598662207358,-18)(464.655518394649,-18)(459.71237458194,-18)(454.769230769231,-18)(449.826086956522,-18)(444.882943143813,-18)(439.939799331104,-18)(434.996655518395,-18)(430.053511705686,-18)(425.110367892977,-18)(420.167224080268,-18)(415.224080267559,-18)(410.28093645485,-18)(405.33779264214,-18)(400.394648829431,-18)(395.451505016722,-18)(390.508361204013,-18)(385.565217391304,-18)(380.622073578595,-18)(375.678929765886,-18)(370.735785953177,-18)(365.792642140468,-18)(360.849498327759,-18)(355.90635451505,-18)(350.963210702341,-18)(346.020066889632,-18)(341.076923076923,-18)(336.133779264214,-18)(331.190635451505,-18)(326.247491638796,-18)(321.304347826087,-18)(316.361204013378,-18)(311.418060200669,-18)(306.47491638796,-18)(301.531772575251,-18)(296.588628762542,-18)(291.645484949833,-18)(286.702341137124,-18)(281.759197324415,-18)(276.816053511706,-18)(271.872909698997,-18)(266.929765886288,-18)(261.986622073579,-18)(257.04347826087,-18)(252.100334448161,-18)(247.157190635452,-18)(242.214046822742,-18)(237.270903010033,-18)(232.327759197324,-18)(227.384615384615,-18)(222.441471571906,-18)(217.498327759197,-18)(212.555183946488,-18)(207.612040133779,-18)(202.66889632107,-18)(197.725752508361,-18)(192.782608695652,-18)(187.839464882943,-18)(182.896321070234,-18)(177.953177257525,-18)(173.010033444816,-18)(168.066889632107,-18)(163.123745819398,-18)(158.180602006689,-18)(153.23745819398,-18)(148.294314381271,-18)(143.351170568562,-18)(138.408026755853,-18)(133.464882943144,-18)(128.521739130435,-18)(123.578595317726,-18)(118.635451505017,-18)(113.692307692308,-18)(108.749163879599,-18)(103.80602006689,-18)(98.8628762541806,-18)(93.9197324414716,-18)(88.9765886287625,-18)(84.0334448160535,-18)(79.0903010033445,-18)(74.1471571906355,-18)(69.2040133779264,-18)(64.2608695652174,-18)(59.3177257525084,-18)(54.3745819397993,-18)(49.4314381270903,-18)(44.4882943143813,-18)(39.5451505016722,-18)(34.6020066889632,-18)(29.6588628762542,-18)(24.7157190635452,-18)(19.7725752508361,-18)(14.8294314381271,-18)(9.88628762541806,-18)(4.94314381270903,-18)(0,-18)(0,-17.9397993311037)(0,-17.8795986622074)(0,-17.819397993311)(0,-17.7591973244147)(0,-17.6989966555184)(0,-17.6387959866221)(0,-17.5785953177258)(0,-17.5183946488294)(0,-17.4581939799331)(0,-17.3979933110368)(0,-17.3377926421405)(0,-17.2775919732441)(0,-17.2173913043478)(0,-17.1571906354515)(0,-17.0969899665552)(0,-17.0367892976589)(0,-16.9765886287625)(0,-16.9163879598662)(0,-16.8561872909699)(0,-16.7959866220736)(0,-16.7357859531773)(0,-16.6755852842809)(0,-16.6153846153846)(0,-16.5551839464883)(0,-16.494983277592)(0,-16.4347826086957)(0,-16.3745819397993)(0,-16.314381270903)(0,-16.2541806020067)(0,-16.1939799331104)(0,-16.133779264214)(0,-16.0735785953177)(0,-16.0133779264214)(0,-15.9531772575251)(0,-15.8929765886288)(0,-15.8327759197324)(0,-15.7725752508361)(0,-15.7123745819398)(0,-15.6521739130435)(0,-15.5919732441472)(0,-15.5317725752508)(0,-15.4715719063545)(0,-15.4113712374582)(0,-15.3511705685619)(0,-15.2909698996656)(0,-15.2307692307692)(0,-15.1705685618729)(0,-15.1103678929766)(0,-15.0501672240803)(0,-14.9899665551839)(0,-14.9297658862876)(0,-14.8695652173913)(0,-14.809364548495)(0,-14.7491638795987)(0,-14.6889632107023)(0,-14.628762541806)(0,-14.5685618729097)(0,-14.5083612040134)(0,-14.4481605351171)(0,-14.3879598662207)(0,-14.3277591973244)(0,-14.2675585284281)(0,-14.2073578595318)(0,-14.1471571906355)(0,-14.0869565217391)(0,-14.0267558528428)(0,-13.9665551839465)(0,-13.9063545150502)(0,-13.8461538461538)(0,-13.7859531772575)(0,-13.7257525083612)(0,-13.6655518394649)(0,-13.6053511705686)(0,-13.5451505016722)(0,-13.4849498327759)(0,-13.4247491638796)(0,-13.3645484949833)(0,-13.304347826087)(0,-13.2441471571906)(0,-13.1839464882943)(0,-13.123745819398)(0,-13.0635451505017)(0,-13.0033444816054)(0,-12.943143812709)(0,-12.8829431438127)(0,-12.8227424749164)(0,-12.7625418060201)(0,-12.7023411371237)(0,-12.6421404682274)(0,-12.5819397993311)(0,-12.5217391304348)(0,-12.4615384615385)(0,-12.4013377926421)(0,-12.3411371237458)(0,-12.2809364548495)(0,-12.2207357859532)(0,-12.1605351170569)(0,-12.1003344481605)(0,-12.0401337792642)(0,-11.9799331103679)(0,-11.9197324414716)(0,-11.8595317725753)(0,-11.7993311036789)(0,-11.7391304347826)(0,-11.6789297658863)(0,-11.61872909699)(0,-11.5585284280936)(0,-11.4983277591973)(0,-11.438127090301)(0,-11.3779264214047)(0,-11.3177257525084)(0,-11.257525083612)(0,-11.1973244147157)(0,-11.1371237458194)(0,-11.0769230769231)(0,-11.0167224080268)(0,-10.9565217391304)(0,-10.8963210702341)(0,-10.8361204013378)(0,-10.7759197324415)(0,-10.7157190635452)(0,-10.6555183946488)(0,-10.5953177257525)(0,-10.5351170568562)(0,-10.4749163879599)(0,-10.4147157190635)(0,-10.3545150501672)(0,-10.2943143812709)(0,-10.2341137123746)(0,-10.1739130434783)(0,-10.1137123745819)(0,-10.0535117056856)(0,-9.9933110367893)(0,-9.93311036789298)(0,-9.87290969899666)(0,-9.81270903010033)(0,-9.75250836120401)(0,-9.69230769230769)(0,-9.63210702341137)(0,-9.57190635451505)(0,-9.51170568561873)(0,-9.45150501672241)(0,-9.39130434782609)(0,-9.33110367892977)(0,-9.27090301003344)(0,-9.21070234113712)(0,-9.1505016722408)(0,-9.09030100334448)(0,-9.03010033444816)(0,-8.96989966555184)(0,-8.90969899665552)(0,-8.8494983277592)(0,-8.78929765886288)(0,-8.72909698996656)(-1.83078722397555e-05,-8.66889632107023)(-7.32306752895604e-05,-8.60869565217391)(-0.00012815225786371,-8.54849498327759)(-0.000164765968224447,-8.48829431438127)(-0.000164765968224447,-8.42809364548495)(-0.000164765968224447,-8.36789297658863)(-0.000164765968224447,-8.30769230769231)(-0.000164765968224447,-8.24749163879599)(-0.000164765968224447,-8.18729096989967)(-0.000164765968224447,-8.12709030100334)(-0.000164765968224447,-8.06688963210702)(-0.000164765968224447,-8.0066889632107)(-0.000164765968224447,-7.94648829431438)(-0.000164765968224447,-7.88628762541806)(-0.000164765968224447,-7.82608695652174)(-0.00012815225786371,-7.76588628762542)(-0.00012815225786371,-7.7056856187291)(-0.00012815225786371,-7.64548494983278)(-0.000164765968224447,-7.58528428093646)(-0.000164765968224447,-7.52508361204013)(-0.000164765968224447,-7.46488294314381)(-0.000164765968224447,-7.40468227424749)(-0.000164765968224447,-7.34448160535117)(-0.000164765968224447,-7.28428093645485)(-0.000164765968224447,-7.22408026755853)(-0.000164765968224447,-7.16387959866221)(-0.000164765968224447,-7.10367892976589)(-0.000164765968224447,-7.04347826086957)(-0.000164765968224447,-6.98327759197324)(-0.000164765968224447,-6.92307692307692)(-0.000164765968224447,-6.8628762541806)(-0.000164765968224447,-6.80267558528428)(-0.000164765968224447,-6.74247491638796)(-0.000164765968224447,-6.68227424749164)(-0.00010984519927805,-6.62207357859532)(-5.49232098834381e-05,-6.561872909699)(0,-6.50167224080268)(-7.32306752895604e-05,-6.44147157190635)(-7.32306752895604e-05,-6.38127090301003)(-7.32306752895604e-05,-6.32107023411371)(0,-6.26086956521739)(0,-6.20066889632107)(0,-6.14046822742475)(0,-6.08026755852843)(0,-6.02006688963211)(0,-5.95986622073579)(0,-5.89966555183946)(0,-5.83946488294314)(0,-5.77926421404682)(0,-5.7190635451505)(0,-5.65886287625418)(0,-5.59866220735786)(0,-5.53846153846154)(0,-5.47826086956522)(0,-5.4180602006689)(0,-5.35785953177258)(0,-5.29765886287625)(0,-5.23745819397993)(0,-5.17725752508361)(0,-5.11705685618729)(0,-5.05685618729097)(0,-4.99665551839465)(0,-4.93645484949833)(0,-4.87625418060201)(0,-4.81605351170569)(0,-4.75585284280936)(0,-4.69565217391304)(0,-4.63545150501672)(0,-4.5752508361204)(0,-4.51505016722408)(0,-4.45484949832776)(0,-4.39464882943144)(0,-4.33444816053512)(0,-4.2742474916388)(0,-4.21404682274247)(0,-4.15384615384615)(0,-4.09364548494983)(0,-4.03344481605351)(0,-3.97324414715719)(0,-3.91304347826087)(0,-3.85284280936455)(0,-3.79264214046823)(0,-3.73244147157191)(0,-3.67224080267559)(0,-3.61204013377926)(0,-3.55183946488294)(0,-3.49163879598662)(0,-3.4314381270903)(0,-3.37123745819398)(0,-3.31103678929766)(0,-3.25083612040134)(0,-3.19063545150502)(0,-3.1304347826087)(0,-3.07023411371237)(0,-3.01003344481605)(0,-2.94983277591973)(0,-2.88963210702341)(0,-2.82943143812709)(0,-2.76923076923077)(0,-2.70903010033445)(0,-2.64882943143813)(0,-2.58862876254181)(0,-2.52842809364549)(0,-2.46822742474916)(0,-2.40802675585284)(0,-2.34782608695652)(0,-2.2876254180602)(0,-2.22742474916388)(0,-2.16722408026756)(0,-2.10702341137124)(0,-2.04682274247492)(0,-1.9866220735786)(0,-1.92642140468227)(0,-1.86622073578595)(0,-1.80602006688963)(0,-1.74581939799331)(0,-1.68561872909699)(0,-1.62541806020067)(0,-1.56521739130435)(0,-1.50501672240803)(0,-1.44481605351171)(0,-1.38461538461538)(0,-1.32441471571906)(0,-1.26421404682274)(0,-1.20401337792642)(0,-1.1438127090301)(0,-1.08361204013378)(0,-1.02341137123746)(0,-0.963210702341137)(0,-0.903010033444816)(0,-0.842809364548495)(0,-0.782608695652174)(0,-0.722408026755853)(0,-0.662207357859532)(0,-0.602006688963211)(0,-0.54180602006689)(0,-0.481605351170569)(0,-0.421404682274247)(0,-0.361204013377926)(0,-0.301003344481605)(0,-0.240802675585284)(0,-0.180602006688963)(0,-0.120401337792642)(0,-0.0602006688963211)(0,0)(4.94314381270903,0)(9.88628762541806,0)(14.8294314381271,0)(19.7725752508361,0)(24.7157190635452,0)(29.6588628762542,0)(34.6020066889632,0)(39.5451505016722,0)(44.4882943143813,0)(49.4314381270903,0)(54.3745819397993,0)(59.3177257525084,0)(64.2608695652174,0)(69.2040133779264,0)(74.1471571906355,0)(79.0903010033445,0)(84.0334448160535,0)(88.9765886287625,0)(93.9197324414716,0)(98.8628762541806,0)(103.80602006689,0)(108.749163879599,0)(113.692307692308,0)(118.635451505017,0)(123.578595317726,0)(128.521739130435,0)(133.464882943144,0)(138.408026755853,0)(143.351170568562,0)(148.294314381271,0)(153.23745819398,0)(158.180602006689,0)(163.123745819398,0)(168.066889632107,0)(173.010033444816,0)(177.953177257525,0)(182.896321070234,0)(187.839464882943,0)(192.782608695652,0)(197.725752508361,0)(202.66889632107,0)(207.612040133779,0)(212.555183946488,0)(217.498327759197,0)(222.441471571906,0)(227.384615384615,0)(232.327759197324,0)(237.270903010033,0)(242.214046822742,0)(247.157190635452,0)(252.100334448161,0)(257.04347826087,0)(261.986622073579,0)(266.929765886288,0)(271.872909698997,0)(276.816053511706,0)(281.759197324415,0)(286.702341137124,0)(291.645484949833,0)(296.588628762542,0)(301.531772575251,0)(306.47491638796,0)(311.418060200669,0)(316.361204013378,0)(321.304347826087,0)(326.247491638796,0)(331.190635451505,0)(336.133779264214,0)(341.076923076923,0)(346.020066889632,0)(350.963210702341,0)(355.90635451505,0)(360.849498327759,0)(365.792642140468,0)(370.735785953177,0)(375.678929765886,0)(380.622073578595,0)(385.565217391304,0)(390.508361204013,0)(395.451505016722,0)(400.394648829431,0)(405.33779264214,0)(410.28093645485,0)(415.224080267559,0)(420.167224080268,0)(425.110367892977,0)(430.053511705686,0)(434.996655518395,0)(439.939799331104,0)(444.882943143813,0)(449.826086956522,0)(454.769230769231,0)(459.71237458194,0)(464.655518394649,0)(469.598662207358,0)(474.541806020067,0)(479.484949832776,0)(484.428093645485,0)(489.371237458194,0)(494.314381270903,0)(499.257525083612,0)(504.200668896321,0)(509.14381270903,0)(514.086956521739,0)(519.030100334448,0)(523.973244147157,0)(528.916387959866,0)(533.859531772575,0)(538.802675585284,0)(543.745819397993,0)(548.688963210702,0)(553.632107023411,0)(558.57525083612,0)(563.518394648829,0)(568.461538461538,0)(573.404682274248,0)(578.347826086957,0)(583.290969899666,0)(588.234113712375,0)(593.177257525084,0)(598.120401337793,0)(603.063545150502,0)(608.006688963211,0)(612.94983277592,0)(617.892976588629,0)(622.836120401338,0)(627.779264214047,0)(632.722408026756,0)(637.665551839465,0)(642.608695652174,0)(647.551839464883,0)(652.494983277592,0)(657.438127090301,0)(662.38127090301,0)(667.324414715719,0)(672.267558528428,0)(677.210702341137,0)(682.153846153846,0)(687.096989966555,0)(692.040133779264,0)(696.983277591973,0)(701.926421404682,0)(706.869565217391,0)(711.8127090301,0)(716.755852842809,0)(721.698996655518,0)(726.642140468227,0)(731.585284280936,0)(736.528428093646,0)(741.471571906354,0)(746.414715719064,0)(751.357859531773,0)(756.301003344482,0)(761.244147157191,0)(766.1872909699,0)(771.130434782609,0)(776.073578595318,0)(781.016722408027,0)(785.959866220736,0)(790.903010033445,0)(795.846153846154,0)(800.789297658863,0)(805.732441471572,0)(810.675585284281,0)(815.61872909699,0)(820.561872909699,0)(825.505016722408,0)(830.448160535117,0)(835.391304347826,0)(840.334448160535,0)(845.277591973244,0)(850.220735785953,0)(855.163879598662,0)(860.107023411371,0)(865.05016722408,0)(869.993311036789,0)(874.936454849498,0)(879.879598662207,0)(884.822742474916,0)(889.765886287625,0)(894.709030100334,0)(899.652173913044,0)(904.595317725752,0)(909.538461538462,0)(914.481605351171,0)(919.42474916388,0)(924.367892976589,0)(929.311036789298,0)(934.254180602007,0)(939.197324414716,0)(944.140468227425,0)(949.083612040134,0)(954.026755852843,0)(958.969899665552,0)(963.913043478261,0)(968.85618729097,0)(973.799331103679,0)(978.742474916388,0)(983.685618729097,0)(988.628762541806,0)(993.571906354515,0)(998.515050167224,0)(1003.45819397993,0)(1008.40133779264,0)(1013.34448160535,0)(1018.28762541806,0)(1023.23076923077,0)(1028.17391304348,0)(1033.11705685619,0)(1038.0602006689,0)(1043.00334448161,0)(1047.94648829431,0)(1052.88963210702,0)(1057.83277591973,0)(1062.77591973244,0)(1067.71906354515,0)(1072.66220735786,0)(1077.60535117057,0)(1082.54849498328,0)(1087.49163879599,0)(1092.4347826087,0)(1097.3779264214,0)(1102.32107023411,0)(1107.26421404682,0)(1112.20735785953,0)(1117.15050167224,0)(1122.09364548495,0)(1127.03678929766,0)(1131.97993311037,0)(1136.92307692308,0)(1141.86622073579,0)(1146.8093645485,0)(1151.7525083612,0)(1156.69565217391,0)(1161.63879598662,0)(1166.58193979933,0)(1171.52508361204,0)(1176.46822742475,0)(1181.41137123746,0)(1186.35451505017,0)(1191.29765886288,0)(1196.24080267559,0)(1201.18394648829,0)(1206.127090301,0)(1211.07023411371,0)(1216.01337792642,0)(1220.95652173913,0)(1225.89966555184,0)(1230.84280936455,0)(1235.78595317726,0)(1240.72909698997,0)(1245.67224080268,0)(1250.61538461538,0)(1255.55852842809,0)(1260.5016722408,0)(1265.44481605351,0)(1270.38795986622,0)(1275.33110367893,0)(1280.27424749164,0)(1285.21739130435,0)(1290.16053511706,0)(1295.10367892977,0)(1300.04682274247,0)(1304.98996655518,0)(1309.93311036789,0)(1314.8762541806,0)(1319.81939799331,0)(1324.76254180602,0)(1329.70568561873,0)(1334.64882943144,0)(1339.59197324415,0)(1344.53511705686,0)(1349.47826086957,0)(1354.42140468227,0)(1359.36454849498,0)(1364.30769230769,0)(1369.2508361204,0)(1374.19397993311,0)(1379.13712374582,0)(1384.08026755853,0)(1389.02341137124,0)(1393.96655518395,0)(1398.90969899666,0)(1403.85284280936,0)(1408.79598662207,0)(1413.73913043478,0)(1418.68227424749,0)(1423.6254180602,0)(1428.56856187291,0)(1433.51170568562,0)(1438.45484949833,0)(1443.39799331104,0)(1448.34113712375,0)(1453.28428093645,0)(1458.22742474916,0)(1463.17056856187,0)(1468.11371237458,0)(1473.05685618729,8.91848548857975e-07)(1478,1.78367067334156e-06)(1478.00014645918,0)};

\addplot [fill=darkgray,draw=none,forget plot] coordinates{ (546.217391304348,-6.20066889632107)(548.688963210702,-6.21070234113712)(553.632107023411,-6.23076923076923)(558.57525083612,-6.23076923076923)(563.518394648829,-6.23076923076923)(568.461538461538,-6.23076923076923)(573.404682274248,-6.23076923076923)(578.347826086957,-6.23076923076923)(583.290969899666,-6.23076923076923)(588.234113712375,-6.23076923076923)(593.177257525084,-6.23076923076923)(598.120401337793,-6.23076923076923)(603.063545150502,-6.23076923076923)(608.006688963211,-6.23076923076923)(612.94983277592,-6.23076923076923)(617.892976588629,-6.23076923076923)(622.836120401338,-6.23076923076923)(627.779264214047,-6.25083612040134)(630.250836120401,-6.26086956521739)(632.722408026756,-6.27090301003345)(637.665551839465,-6.29096989966555)(642.608695652174,-6.29096989966555)(647.551839464883,-6.29096989966555)(652.494983277592,-6.29096989966555)(657.438127090301,-6.29096989966555)(662.38127090301,-6.29096989966555)(667.324414715719,-6.31103678929766)(669.795986622074,-6.32107023411371)(672.267558528428,-6.33110367892977)(677.210702341137,-6.35117056856187)(682.153846153846,-6.35117056856187)(687.096989966555,-6.35117056856187)(692.040133779264,-6.35117056856187)(696.983277591973,-6.35117056856187)(701.926421404682,-6.37123745819398)(704.397993311037,-6.38127090301003)(706.869565217391,-6.39130434782609)(711.8127090301,-6.41137123745819)(716.755852842809,-6.41137123745819)(721.698996655518,-6.41137123745819)(726.642140468227,-6.41137123745819)(731.585284280936,-6.4314381270903)(734.056856187291,-6.44147157190635)(736.528428093646,-6.45150501672241)(741.471571906354,-6.47157190635452)(746.414715719064,-6.47157190635452)(751.357859531773,-6.47157190635452)(756.301003344482,-6.47157190635452)(761.244147157191,-6.47157190635452)(766.1872909699,-6.49163879598662)(768.658862876254,-6.50167224080268)(771.130434782609,-6.51170568561873)(776.073578595318,-6.53177257525084)(781.016722408027,-6.53177257525084)(785.959866220736,-6.53177257525084)(790.903010033445,-6.55183946488294)(793.374581939799,-6.561872909699)(795.846153846154,-6.57190635451505)(800.789297658863,-6.59197324414716)(805.732441471572,-6.59197324414716)(810.675585284281,-6.59197324414716)(815.61872909699,-6.59197324414716)(820.561872909699,-6.59197324414716)(825.505016722408,-6.61204013377926)(827.976588628763,-6.62207357859532)(830.448160535117,-6.63210702341137)(835.391304347826,-6.65217391304348)(840.334448160535,-6.65217391304348)(845.277591973244,-6.65217391304348)(850.220735785953,-6.65217391304348)(855.163879598662,-6.65217391304348)(860.107023411371,-6.65217391304348)(865.05016722408,-6.65217391304348)(869.993311036789,-6.65217391304348)(874.936454849498,-6.65217391304348)(879.879598662207,-6.65217391304348)(884.822742474916,-6.65217391304348)(889.765886287625,-6.65217391304348)(894.709030100335,-6.67224080267559)(897.180602006689,-6.68227424749164)(899.652173913044,-6.69230769230769)(904.595317725752,-6.7123745819398)(909.538461538462,-6.7123745819398)(914.481605351171,-6.7123745819398)(919.42474916388,-6.7123745819398)(924.367892976589,-6.7123745819398)(929.311036789298,-6.7123745819398)(934.254180602007,-6.7123745819398)(939.197324414716,-6.7123745819398)(944.140468227425,-6.7123745819398)(949.083612040134,-6.7123745819398)(954.026755852843,-6.69230769230769)(956.498327759197,-6.68227424749164)(958.969899665552,-6.67224080267559)(963.913043478261,-6.65217391304348)(968.85618729097,-6.65217391304348)(973.799331103679,-6.67224080267559)(976.270903010033,-6.68227424749164)(978.742474916388,-6.69230769230769)(983.685618729097,-6.7123745819398)(988.628762541806,-6.7123745819398)(993.571906354515,-6.7123745819398)(998.515050167224,-6.7123745819398)(1003.45819397993,-6.7123745819398)(1008.40133779264,-6.7123745819398)(1013.34448160535,-6.7123745819398)(1018.28762541806,-6.7123745819398)(1023.23076923077,-6.7123745819398)(1028.17391304348,-6.7123745819398)(1033.11705685619,-6.7123745819398)(1038.0602006689,-6.7123745819398)(1043.00334448161,-6.7123745819398)(1047.94648829431,-6.7123745819398)(1052.88963210702,-6.7123745819398)(1057.83277591973,-6.7123745819398)(1062.77591973244,-6.7123745819398)(1067.71906354515,-6.7123745819398)(1072.66220735786,-6.7123745819398)(1077.60535117057,-6.7123745819398)(1082.54849498328,-6.7123745819398)(1087.49163879599,-6.7123745819398)(1092.4347826087,-6.7123745819398)(1097.3779264214,-6.7123745819398)(1102.32107023411,-6.7123745819398)(1107.26421404682,-6.7123745819398)(1112.20735785953,-6.7123745819398)(1117.15050167224,-6.7123745819398)(1122.09364548495,-6.7123745819398)(1127.03678929766,-6.7123745819398)(1131.97993311037,-6.7123745819398)(1136.92307692308,-6.7123745819398)(1141.86622073579,-6.7123745819398)(1146.8093645485,-6.73244147157191)(1149.28093645485,-6.74247491638796)(1151.7525083612,-6.75250836120401)(1156.69565217391,-6.77257525083612)(1161.63879598662,-6.77257525083612)(1166.58193979933,-6.77257525083612)(1171.52508361204,-6.77257525083612)(1176.46822742475,-6.77257525083612)(1181.41137123746,-6.77257525083612)(1186.35451505017,-6.79264214046823)(1188.82608695652,-6.80267558528428)(1191.29765886288,-6.81270903010034)(1196.24080267559,-6.83277591973244)(1201.18394648829,-6.83277591973244)(1206.127090301,-6.83277591973244)(1211.07023411371,-6.83277591973244)(1216.01337792642,-6.83277591973244)(1220.95652173913,-6.85284280936455)(1223.42809364549,-6.8628762541806)(1225.89966555184,-6.87290969899666)(1230.84280936455,-6.89297658862876)(1235.78595317726,-6.89297658862876)(1240.72909698997,-6.89297658862876)(1245.67224080268,-6.89297658862876)(1250.61538461538,-6.89297658862876)(1255.55852842809,-6.89297658862876)(1260.5016722408,-6.89297658862876)(1265.44481605351,-6.91304347826087)(1267.91638795987,-6.92307692307692)(1270.38795986622,-6.93311036789298)(1275.33110367893,-6.95317725752508)(1280.27424749164,-6.95317725752508)(1285.21739130435,-6.95317725752508)(1290.16053511706,-6.95317725752508)(1295.10367892977,-6.95317725752508)(1300.04682274247,-6.95317725752508)(1304.98996655518,-6.95317725752508)(1309.93311036789,-6.97324414715719)(1312.40468227425,-6.98327759197324)(1314.8762541806,-6.9933110367893)(1319.81939799331,-7.0133779264214)(1324.76254180602,-7.0133779264214)(1329.70568561873,-7.0133779264214)(1334.64882943144,-7.0133779264214)(1339.59197324415,-7.03344481605351)(1342.0635451505,-7.04347826086957)(1344.53511705686,-7.05351170568562)(1349.47826086957,-7.07357859531773)(1354.42140468227,-7.07357859531773)(1359.36454849498,-7.09364548494983)(1361.83612040134,-7.10367892976589)(1364.30769230769,-7.11371237458194)(1369.2508361204,-7.13377926421405)(1374.19397993311,-7.15384615384615)(1376.66555183947,-7.16387959866221)(1379.13712374582,-7.17391304347826)(1384.08026755853,-7.19397993311037)(1389.02341137124,-7.21404682274247)(1391.49498327759,-7.22408026755853)(1393.96655518395,-7.23411371237458)(1398.90969899666,-7.2742474916388)(1401.38127090301,-7.28428093645485)(1403.85284280936,-7.2943143812709)(1408.79598662207,-7.33444816053512)(1410.03177257525,-7.34448160535117)(1413.73913043478,-7.37458193979933)(1417.44648829431,-7.40468227424749)(1418.68227424749,-7.41471571906355)(1423.6254180602,-7.45484949832776)(1424.86120401338,-7.46488294314381)(1428.56856187291,-7.49498327759197)(1432.27591973244,-7.52508361204013)(1433.51170568562,-7.53511705685619)(1438.45484949833,-7.5752508361204)(1439.6906354515,-7.58528428093646)(1443.39799331104,-7.61538461538462)(1447.10535117057,-7.64548494983278)(1448.34113712375,-7.65551839464883)(1453.28428093645,-7.69565217391304)(1455.75585284281,-7.7056856187291)(1458.22742474916,-7.71571906354515)(1463.17056856187,-7.75585284280937)(1465.64214046823,-7.76588628762542)(1468.11371237458,-7.77591973244147)(1473.05685618729,-7.81605351170569)(1475.52842809365,-7.82608695652174)(1478,-7.83612040133779)(1478.00004576866,-7.88628762541806)(1478.00008238298,-7.94648829431438)(1478.00008238298,-8.0066889632107)(1478.00008238298,-8.06688963210702)(1478.00008238298,-8.12709030100334)(1478.00008238298,-8.18729096989967)(1478.00008238298,-8.24749163879599)(1478.00008238298,-8.30769230769231)(1478.00008238298,-8.36789297658863)(1478.00008238298,-8.42809364548495)(1478.00008238298,-8.48829431438127)(1478.00008238298,-8.54849498327759)(1478.00008238298,-8.60869565217391)(1478.00008238298,-8.66889632107023)(1478.00008238298,-8.72909698996656)(1478.00008238298,-8.78929765886288)(1478.00008238298,-8.8494983277592)(1478.00008238298,-8.90969899665552)(1478.00008238298,-8.96989966555184)(1478.00008238298,-9.03010033444816)(1478.00008238298,-9.09030100334448)(1478.00008238298,-9.1505016722408)(1478.00011899676,-9.21070234113712)(1478.00011899676,-9.27090301003344)(1478.00011899676,-9.33110367892977)(1478.00008238298,-9.39130434782609)(1478.00008238298,-9.45150501672241)(1478.00008238298,-9.51170568561873)(1478.00008238298,-9.57190635451505)(1478.00008238298,-9.63210702341137)(1478.00008238298,-9.69230769230769)(1478.00008238298,-9.75250836120401)(1478.00008238298,-9.81270903010033)(1478.00008238298,-9.87290969899666)(1478.00008238298,-9.93311036789298)(1478.00008238298,-9.9933110367893)(1478.00008238298,-10.0535117056856)(1478.00008238298,-10.1137123745819)(1478.00008238298,-10.1739130434783)(1478.00008238298,-10.2341137123746)(1478.00008238298,-10.2943143812709)(1478.00008238298,-10.3545150501672)(1478.00008238298,-10.4147157190635)(1478.00008238298,-10.4749163879599)(1478.00008238298,-10.5351170568562)(1478.00004576866,-10.5953177257525)(1478.00004576866,-10.6555183946488)(1478,-10.7056856187291)(1475.52842809365,-10.7157190635452)(1473.05685618729,-10.7307692307692)(1468.11371237458,-10.7458193979933)(1463.17056856187,-10.7458193979933)(1458.22742474916,-10.7658862876254)(1455.75585284281,-10.7759197324415)(1453.28428093645,-10.7859531772575)(1448.34113712375,-10.8060200668896)(1443.39799331104,-10.8260869565217)(1440.92642140468,-10.8361204013378)(1438.45484949833,-10.8461538461538)(1433.51170568562,-10.866220735786)(1428.56856187291,-10.866220735786)(1423.6254180602,-10.8862876254181)(1421.15384615385,-10.8963210702341)(1418.68227424749,-10.9063545150502)(1413.73913043478,-10.9264214046823)(1408.79598662207,-10.9264214046823)(1403.85284280936,-10.9464882943144)(1401.38127090301,-10.9565217391304)(1398.90969899666,-10.9665551839465)(1393.96655518395,-10.9866220735786)(1389.02341137124,-10.9866220735786)(1384.08026755853,-10.9866220735786)(1379.13712374582,-10.9866220735786)(1374.19397993311,-11.0066889632107)(1371.72240802676,-11.0167224080268)(1369.2508361204,-11.0267558528428)(1364.30769230769,-11.0468227424749)(1359.36454849498,-11.0468227424749)(1354.42140468227,-11.0468227424749)(1349.47826086957,-11.0468227424749)(1344.53511705686,-11.0468227424749)(1339.59197324415,-11.0468227424749)(1334.64882943144,-11.0468227424749)(1329.70568561873,-11.0468227424749)(1324.76254180602,-11.0468227424749)(1319.81939799331,-11.0468227424749)(1314.8762541806,-11.0468227424749)(1309.93311036789,-11.0468227424749)(1304.98996655518,-11.0468227424749)(1300.04682274247,-11.0468227424749)(1295.10367892977,-11.0468227424749)(1290.16053511706,-11.0468227424749)(1285.21739130435,-11.0468227424749)(1280.27424749164,-11.0468227424749)(1275.33110367893,-11.0468227424749)(1270.38795986622,-11.0468227424749)(1265.44481605351,-11.0468227424749)(1260.5016722408,-11.0468227424749)(1255.55852842809,-11.0468227424749)(1250.61538461538,-11.0468227424749)(1245.67224080268,-11.0468227424749)(1240.72909698997,-11.0468227424749)(1235.78595317726,-11.0468227424749)(1230.84280936455,-11.0267558528428)(1228.37123745819,-11.0167224080268)(1225.89966555184,-11.0066889632107)(1220.95652173913,-10.9866220735786)(1216.01337792642,-10.9866220735786)(1211.07023411371,-10.9866220735786)(1206.127090301,-10.9866220735786)(1201.18394648829,-10.9866220735786)(1196.24080267559,-10.9866220735786)(1191.29765886288,-10.9866220735786)(1186.35451505017,-10.9866220735786)(1181.41137123746,-10.9866220735786)(1176.46822742475,-10.9665551839465)(1173.99665551839,-10.9565217391304)(1171.52508361204,-10.9464882943144)(1166.58193979933,-10.9264214046823)(1161.63879598662,-10.9264214046823)(1156.69565217391,-10.9264214046823)(1151.7525083612,-10.9264214046823)(1146.8093645485,-10.9264214046823)(1141.86622073579,-10.9264214046823)(1136.92307692308,-10.9063545150502)(1134.45150501672,-10.8963210702341)(1131.97993311037,-10.8862876254181)(1127.03678929766,-10.866220735786)(1122.09364548495,-10.866220735786)(1117.15050167224,-10.866220735786)(1112.20735785953,-10.866220735786)(1107.26421404682,-10.8461538461538)(1104.79264214047,-10.8361204013378)(1102.32107023411,-10.8260869565217)(1097.3779264214,-10.8060200668896)(1092.4347826087,-10.8060200668896)(1087.49163879599,-10.8060200668896)(1082.54849498328,-10.8060200668896)(1077.60535117057,-10.8060200668896)(1072.66220735786,-10.7859531772575)(1070.1906354515,-10.7759197324415)(1067.71906354515,-10.7658862876254)(1062.77591973244,-10.7458193979933)(1057.83277591973,-10.7458193979933)(1052.88963210702,-10.7458193979933)(1047.94648829431,-10.7458193979933)(1043.00334448161,-10.7458193979933)(1038.0602006689,-10.7458193979933)(1033.11705685619,-10.7458193979933)(1028.17391304348,-10.7458193979933)(1023.23076923077,-10.7257525083612)(1020.75919732441,-10.7157190635452)(1018.28762541806,-10.7056856187291)(1013.34448160535,-10.685618729097)(1008.40133779264,-10.685618729097)(1003.45819397993,-10.685618729097)(998.515050167224,-10.685618729097)(993.571906354515,-10.685618729097)(988.628762541806,-10.685618729097)(983.685618729097,-10.685618729097)(978.742474916388,-10.685618729097)(973.799331103679,-10.6655518394649)(971.327759197324,-10.6555183946488)(968.85618729097,-10.6454849498328)(963.913043478261,-10.6254180602007)(958.969899665552,-10.6254180602007)(954.026755852843,-10.6254180602007)(949.083612040134,-10.6254180602007)(944.140468227425,-10.6254180602007)(939.197324414716,-10.6053511705686)(936.725752508361,-10.5953177257525)(934.254180602007,-10.5852842809365)(929.311036789298,-10.5652173913043)(924.367892976589,-10.5652173913043)(919.42474916388,-10.5652173913043)(914.481605351171,-10.5451505016722)(912.010033444816,-10.5351170568562)(909.538461538462,-10.5250836120401)(904.595317725752,-10.505016722408)(899.652173913044,-10.505016722408)(894.709030100334,-10.505016722408)(889.765886287625,-10.4849498327759)(887.294314381271,-10.4749163879599)(884.822742474916,-10.4648829431438)(879.879598662207,-10.4448160535117)(874.936454849498,-10.4448160535117)(869.993311036789,-10.4448160535117)(865.05016722408,-10.4247491638796)(862.578595317726,-10.4147157190635)(860.107023411371,-10.4046822742475)(855.163879598662,-10.3846153846154)(850.220735785953,-10.3846153846154)(845.277591973244,-10.3846153846154)(840.334448160535,-10.3645484949833)(837.862876254181,-10.3545150501672)(835.391304347826,-10.3444816053512)(830.448160535117,-10.3244147157191)(825.505016722408,-10.3244147157191)(820.561872909699,-10.3244147157191)(815.61872909699,-10.3244147157191)(810.675585284281,-10.304347826087)(808.204013377926,-10.2943143812709)(805.732441471572,-10.2842809364548)(800.789297658863,-10.2642140468227)(795.846153846154,-10.2642140468227)(790.903010033445,-10.2642140468227)(785.959866220736,-10.2642140468227)(781.016722408027,-10.2441471571906)(778.545150501672,-10.2341137123746)(776.073578595318,-10.2240802675585)(771.130434782609,-10.2040133779264)(766.1872909699,-10.2040133779264)(761.244147157191,-10.2040133779264)(756.301003344482,-10.2040133779264)(751.357859531772,-10.1839464882943)(748.886287625418,-10.1739130434783)(746.414715719064,-10.1638795986622)(741.471571906354,-10.1438127090301)(736.528428093646,-10.1438127090301)(731.585284280936,-10.1438127090301)(726.642140468227,-10.123745819398)(724.170568561873,-10.1137123745819)(721.698996655518,-10.1036789297659)(716.755852842809,-10.0836120401338)(711.8127090301,-10.0836120401338)(706.869565217391,-10.0836120401338)(701.926421404682,-10.0635451505017)(699.454849498328,-10.0535117056856)(696.983277591973,-10.0434782608696)(692.040133779264,-10.0234113712375)(687.096989966555,-10.0234113712375)(682.153846153846,-10.0234113712375)(677.210702341137,-10.0234113712375)(672.267558528428,-10.0033444816054)(669.795986622074,-9.9933110367893)(667.324414715719,-9.98327759197324)(662.38127090301,-9.96321070234114)(657.438127090301,-9.96321070234114)(652.494983277592,-9.96321070234114)(647.551839464883,-9.96321070234114)(642.608695652174,-9.96321070234114)(637.665551839465,-9.94314381270903)(635.19397993311,-9.93311036789298)(632.722408026756,-9.92307692307692)(627.779264214047,-9.90301003344482)(622.836120401338,-9.90301003344482)(617.892976588629,-9.90301003344482)(612.94983277592,-9.90301003344482)(608.006688963211,-9.90301003344482)(603.063545150502,-9.90301003344482)(598.120401337793,-9.90301003344482)(593.177257525084,-9.90301003344482)(588.234113712375,-9.90301003344482)(583.290969899666,-9.90301003344482)(578.347826086957,-9.90301003344482)(573.404682274248,-9.90301003344482)(568.461538461538,-9.90301003344482)(563.518394648829,-9.90301003344482)(558.57525083612,-9.90301003344482)(553.632107023411,-9.90301003344482)(548.688963210702,-9.90301003344482)(543.745819397993,-9.90301003344482)(538.802675585284,-9.90301003344482)(533.859531772575,-9.90301003344482)(528.916387959866,-9.90301003344482)(523.973244147157,-9.90301003344482)(519.030100334448,-9.90301003344482)(514.086956521739,-9.90301003344482)(509.14381270903,-9.90301003344482)(504.200668896321,-9.90301003344482)(499.257525083612,-9.90301003344482)(494.314381270903,-9.90301003344482)(489.371237458194,-9.90301003344482)(484.428093645485,-9.90301003344482)(479.484949832776,-9.90301003344482)(474.541806020067,-9.90301003344482)(469.598662207358,-9.90301003344482)(464.655518394649,-9.90301003344482)(459.71237458194,-9.90301003344482)(454.769230769231,-9.90301003344482)(449.826086956522,-9.88795986622074)(444.882943143813,-9.88795986622074)(439.939799331104,-9.88795986622074)(434.996655518395,-9.90301003344482)(430.053511705686,-9.90301003344482)(425.110367892977,-9.90301003344482)(420.167224080268,-9.90301003344482)(415.224080267559,-9.90301003344482)(410.28093645485,-9.90301003344482)(405.33779264214,-9.90301003344482)(400.394648829431,-9.90301003344482)(395.451505016722,-9.90301003344482)(390.508361204013,-9.90301003344482)(385.565217391304,-9.90301003344482)(380.622073578595,-9.90301003344482)(375.678929765886,-9.90301003344482)(370.735785953177,-9.90301003344482)(365.792642140468,-9.90301003344482)(360.849498327759,-9.90301003344482)(355.90635451505,-9.90301003344482)(350.963210702341,-9.90301003344482)(346.020066889632,-9.90301003344482)(341.076923076923,-9.90301003344482)(336.133779264214,-9.90301003344482)(331.190635451505,-9.90301003344482)(326.247491638796,-9.90301003344482)(321.304347826087,-9.90301003344482)(316.361204013378,-9.90301003344482)(311.418060200669,-9.90301003344482)(306.47491638796,-9.90301003344482)(301.531772575251,-9.88294314381271)(299.060200668896,-9.87290969899666)(296.588628762542,-9.8628762541806)(291.645484949833,-9.8428093645485)(286.702341137124,-9.8428093645485)(281.759197324415,-9.8428093645485)(276.816053511706,-9.8428093645485)(271.872909698997,-9.8428093645485)(266.929765886288,-9.8428093645485)(261.986622073579,-9.82274247491639)(259.515050167224,-9.81270903010033)(257.04347826087,-9.80267558528428)(252.100334448161,-9.78260869565217)(247.157190635452,-9.78260869565217)(242.214046822742,-9.78260869565217)(237.270903010033,-9.78260869565217)(232.327759197324,-9.78260869565217)(227.384615384615,-9.78260869565217)(222.441471571906,-9.78260869565217)(217.498327759197,-9.78260869565217)(212.555183946488,-9.76254180602007)(210.083612040134,-9.75250836120401)(207.612040133779,-9.74247491638796)(202.66889632107,-9.72240802675585)(197.725752508361,-9.72240802675585)(192.782608695652,-9.72240802675585)(187.839464882943,-9.72240802675585)(182.896321070234,-9.72240802675585)(177.953177257525,-9.72240802675585)(173.010033444816,-9.72240802675585)(168.066889632107,-9.72240802675585)(163.123745819398,-9.72240802675585)(158.180602006689,-9.72240802675585)(153.23745819398,-9.70234113712375)(150.765886287625,-9.69230769230769)(148.294314381271,-9.68227424749164)(143.351170568562,-9.66220735785953)(138.408026755853,-9.66220735785953)(133.464882943144,-9.64214046822742)(130.993311036789,-9.63210702341137)(128.521739130435,-9.62207357859532)(123.578595317726,-9.60200668896321)(118.635451505017,-9.5819397993311)(116.163879598662,-9.57190635451505)(113.692307692308,-9.561872909699)(108.749163879599,-9.52173913043478)(107.513377926421,-9.51170568561873)(103.80602006689,-9.48160535117057)(100.098662207358,-9.45150501672241)(98.8628762541806,-9.44147157190636)(93.9197324414716,-9.40133779264214)(92.6839464882943,-9.39130434782609)(88.9765886287625,-9.36120401337793)(86.505016722408,-9.33110367892977)(84.0334448160535,-9.30100334448161)(81.561872909699,-9.27090301003344)(79.0903010033445,-9.24080267558528)(76.61872909699,-9.21070234113712)(74.1471571906355,-9.18060200668896)(70.4397993311037,-9.1505016722408)(69.2040133779264,-9.14046822742475)(64.2608695652174,-9.10033444816053)(63.0250836120401,-9.09030100334448)(59.3177257525084,-9.06020066889632)(56.8461538461538,-9.03010033444816)(54.3745819397993,-9)(51.9030100334448,-8.96989966555184)(49.4314381270903,-8.93979933110368)(46.9598662207358,-8.90969899665552)(44.4882943143813,-8.87959866220736)(40.7809364548495,-8.8494983277592)(39.5451505016722,-8.83946488294314)(34.6020066889632,-8.79933110367893)(33.366220735786,-8.78929765886288)(29.6588628762542,-8.75919732441472)(25.9515050167224,-8.72909698996656)(24.7157190635452,-8.7190635451505)(19.7725752508361,-8.67892976588629)(17.3010033444816,-8.66889632107023)(14.8294314381271,-8.65886287625418)(9.88628762541806,-8.63879598662207)(4.94314381270903,-8.61872909698996)(2.47157190635452,-8.60869565217391)(0,-8.59866220735786)(-4.57686635229888e-05,-8.54849498327759)(-8.23829841122233e-05,-8.48829431438127)(-8.23829841122233e-05,-8.42809364548495)(-8.23829841122233e-05,-8.36789297658863)(-8.23829841122233e-05,-8.30769230769231)(-8.23829841122233e-05,-8.24749163879599)(-8.23829841122233e-05,-8.18729096989967)(-8.23829841122233e-05,-8.12709030100334)(-8.23829841122233e-05,-8.06688963210702)(-8.23829841122233e-05,-8.0066889632107)(-8.23829841122233e-05,-7.94648829431438)(-8.23829841122233e-05,-7.88628762541806)(-8.23829841122233e-05,-7.82608695652174)(-4.57686635229888e-05,-7.76588628762542)(-4.57686635229888e-05,-7.7056856187291)(-4.57686635229888e-05,-7.64548494983278)(-8.23829841122233e-05,-7.58528428093646)(-8.23829841122233e-05,-7.52508361204013)(-8.23829841122233e-05,-7.46488294314381)(-8.23829841122233e-05,-7.40468227424749)(-8.23829841122233e-05,-7.34448160535117)(-8.23829841122233e-05,-7.28428093645485)(-8.23829841122233e-05,-7.22408026755853)(-8.23829841122233e-05,-7.16387959866221)(-8.23829841122233e-05,-7.10367892976589)(-8.23829841122233e-05,-7.04347826086957)(-8.23829841122233e-05,-6.98327759197324)(-8.23829841122233e-05,-6.92307692307692)(-8.23829841122233e-05,-6.8628762541806)(-8.23829841122233e-05,-6.80267558528428)(-8.23829841122233e-05,-6.74247491638796)(-8.23829841122233e-05,-6.68227424749164)(-2.74612998192381e-05,-6.62207357859532)(0,-6.59197324414716)(4.94314381270903,-6.59197324414716)(9.88628762541806,-6.59197324414716)(14.8294314381271,-6.59197324414716)(19.7725752508361,-6.59197324414716)(24.7157190635452,-6.59197324414716)(29.6588628762542,-6.59197324414716)(34.6020066889632,-6.59197324414716)(39.5451505016722,-6.59197324414716)(44.4882943143813,-6.59197324414716)(49.4314381270903,-6.59197324414716)(54.3745819397993,-6.59197324414716)(59.3177257525084,-6.59197324414716)(64.2608695652174,-6.59197324414716)(69.2040133779264,-6.59197324414716)(74.1471571906355,-6.59197324414716)(79.0903010033445,-6.59197324414716)(84.0334448160535,-6.59197324414716)(88.9765886287625,-6.59197324414716)(93.9197324414716,-6.59197324414716)(98.8628762541806,-6.59197324414716)(103.80602006689,-6.59197324414716)(108.749163879599,-6.59197324414716)(113.692307692308,-6.59197324414716)(118.635451505017,-6.59197324414716)(123.578595317726,-6.59197324414716)(128.521739130435,-6.59197324414716)(133.464882943144,-6.59197324414716)(138.408026755853,-6.61204013377926)(140.879598662207,-6.62207357859532)(143.351170568562,-6.63210702341137)(148.294314381271,-6.65217391304348)(153.23745819398,-6.65217391304348)(158.180602006689,-6.65217391304348)(163.123745819398,-6.65217391304348)(168.066889632107,-6.65217391304348)(173.010033444816,-6.65217391304348)(177.953177257525,-6.65217391304348)(182.896321070234,-6.67224080267559)(185.367892976589,-6.68227424749164)(187.839464882943,-6.69230769230769)(192.782608695652,-6.7123745819398)(197.725752508361,-6.7123745819398)(202.66889632107,-6.7123745819398)(207.612040133779,-6.7123745819398)(212.555183946488,-6.7123745819398)(217.498327759197,-6.7123745819398)(222.441471571906,-6.7123745819398)(227.384615384615,-6.7123745819398)(232.327759197324,-6.7123745819398)(237.270903010033,-6.7123745819398)(242.214046822742,-6.7123745819398)(247.157190635452,-6.7123745819398)(252.100334448161,-6.7123745819398)(257.04347826087,-6.7123745819398)(261.986622073579,-6.69230769230769)(264.458193979933,-6.68227424749164)(266.929765886288,-6.67224080267559)(271.872909698997,-6.65217391304348)(276.816053511706,-6.65217391304348)(281.759197324415,-6.65217391304348)(286.702341137124,-6.65217391304348)(291.645484949833,-6.63210702341137)(294.117056856187,-6.62207357859532)(296.588628762542,-6.61204013377926)(301.531772575251,-6.59197324414716)(306.47491638796,-6.59197324414716)(311.418060200669,-6.57190635451505)(313.889632107023,-6.561872909699)(316.361204013378,-6.55183946488294)(321.304347826087,-6.53177257525084)(326.247491638796,-6.51170568561873)(328.71906354515,-6.50167224080268)(331.190635451505,-6.49163879598662)(336.133779264214,-6.47157190635452)(341.076923076923,-6.47157190635452)(346.020066889632,-6.45150501672241)(348.491638795987,-6.44147157190635)(350.963210702341,-6.4314381270903)(355.90635451505,-6.41137123745819)(360.849498327759,-6.39130434782609)(363.321070234114,-6.38127090301003)(365.792642140468,-6.37123745819398)(370.735785953177,-6.35117056856187)(375.678929765886,-6.35117056856187)(380.622073578595,-6.33110367892977)(383.09364548495,-6.32107023411371)(385.565217391304,-6.31103678929766)(390.508361204013,-6.29096989966555)(395.451505016722,-6.29096989966555)(400.394648829431,-6.29096989966555)(405.33779264214,-6.27090301003345)(407.809364548495,-6.26086956521739)(410.28093645485,-6.25083612040134)(415.224080267559,-6.23076923076923)(420.167224080268,-6.23076923076923)(425.110367892977,-6.23076923076923)(430.053511705686,-6.23076923076923)(434.996655518395,-6.23076923076923)(439.939799331104,-6.23076923076923)(444.882943143813,-6.23076923076923)(449.826086956522,-6.21070234113712)(452.297658862876,-6.20066889632107)(454.769230769231,-6.19063545150502)(459.71237458194,-6.17056856187291)(464.655518394649,-6.17056856187291)(469.598662207358,-6.17056856187291)(474.541806020067,-6.17056856187291)(479.484949832776,-6.17056856187291)(484.428093645485,-6.17056856187291)(489.371237458194,-6.17056856187291)(494.314381270903,-6.17056856187291)(499.257525083612,-6.17056856187291)(504.200668896321,-6.17056856187291)(509.14381270903,-6.17056856187291)(514.086956521739,-6.17056856187291)(519.030100334448,-6.17056856187291)(523.973244147157,-6.17056856187291)(528.916387959866,-6.17056856187291)(533.859531772575,-6.17056856187291)(538.802675585284,-6.17056856187291)(543.745819397993,-6.19063545150502)(546.217391304348,-6.20066889632107)};

\addplot [fill=green!60!black,draw=none,forget plot] coordinates{ (1241.96488294314,-9.57190635451505)(1240.72909698997,-9.57692307692308)(1235.78595317726,-9.59698996655519)(1230.84280936455,-9.61705685618729)(1225.89966555184,-9.61705685618729)(1220.95652173913,-9.61705685618729)(1216.01337792642,-9.61404682274248)(1211.07023411371,-9.61404682274248)(1206.127090301,-9.61404682274248)(1201.18394648829,-9.61705685618729)(1196.24080267559,-9.61705685618729)(1191.29765886288,-9.59698996655519)(1186.35451505017,-9.57692307692308)(1185.11872909699,-9.57190635451505)(1181.41137123746,-9.55685618729097)(1176.46822742475,-9.55685618729097)(1171.52508361204,-9.55685618729097)(1166.58193979933,-9.55685618729097)(1161.63879598662,-9.55685618729097)(1156.69565217391,-9.55685618729097)(1151.7525083612,-9.53678929765886)(1146.8093645485,-9.51672240802676)(1145.57357859532,-9.51170568561873)(1141.86622073579,-9.49665551839465)(1136.92307692308,-9.49665551839465)(1131.97993311037,-9.47658862876254)(1127.03678929766,-9.45652173913043)(1125.80100334448,-9.45150501672241)(1122.09364548495,-9.43645484949833)(1117.15050167224,-9.41638795986622)(1112.20735785953,-9.39632107023412)(1111.58946488294,-9.39130434782609)(1107.26421404682,-9.3561872909699)(1102.32107023411,-9.33612040133779)(1101.08528428094,-9.33110367892977)(1097.3779264214,-9.31605351170569)(1092.4347826087,-9.29598662207358)(1087.49163879599,-9.27591973244147)(1086.8737458194,-9.27090301003344)(1082.54849498328,-9.23578595317726)(1079.45903010033,-9.21070234113712)(1077.60535117057,-9.188127090301)(1074.51588628763,-9.1505016722408)(1072.66220735786,-9.12792642140468)(1070.60256410256,-9.09030100334448)(1070.60256410256,-9.03010033444816)(1072.2502787068,-8.96989966555184)(1072.66220735786,-8.9623745819398)(1075.54570791527,-8.90969899665552)(1077.60535117057,-8.88461538461539)(1081.93060200669,-8.8494983277592)(1082.54849498328,-8.84448160535117)(1087.49163879599,-8.811872909699)(1089.34531772575,-8.78929765886288)(1092.4347826087,-8.76421404682274)(1096.76003344482,-8.72909698996656)(1097.3779264214,-8.72408026755853)(1102.32107023411,-8.70401337792642)(1106.64632107023,-8.66889632107023)(1107.26421404682,-8.66387959866221)(1112.20735785953,-8.6438127090301)(1117.15050167224,-8.62374581939799)(1120.85785953177,-8.60869565217391)(1122.09364548495,-8.60367892976588)(1127.03678929766,-8.58361204013378)(1131.97993311037,-8.56354515050167)(1136.92307692308,-8.56354515050167)(1140.63043478261,-8.54849498327759)(1141.86622073579,-8.54347826086956)(1146.8093645485,-8.52341137123746)(1151.7525083612,-8.50334448160535)(1156.69565217391,-8.50334448160535)(1161.63879598662,-8.50334448160535)(1166.58193979933,-8.50334448160535)(1171.52508361204,-8.50334448160535)(1176.46822742475,-8.50334448160535)(1181.41137123746,-8.50334448160535)(1185.11872909699,-8.48829431438127)(1186.35451505017,-8.48327759197324)(1191.29765886288,-8.46321070234114)(1196.24080267559,-8.44314381270903)(1201.18394648829,-8.44314381270903)(1206.127090301,-8.44314381270903)(1211.07023411371,-8.44314381270903)(1216.01337792642,-8.44314381270903)(1220.95652173913,-8.46321070234114)(1225.89966555184,-8.48327759197324)(1227.13545150502,-8.48829431438127)(1230.84280936455,-8.50334448160535)(1235.78595317726,-8.52341137123746)(1240.72909698997,-8.54347826086956)(1241.34698996656,-8.54849498327759)(1245.67224080268,-8.58361204013378)(1250.61538461538,-8.60367892976588)(1251.23327759197,-8.60869565217391)(1255.55852842809,-8.6438127090301)(1260.5016722408,-8.66387959866221)(1261.11956521739,-8.66889632107023)(1265.44481605351,-8.70401337792642)(1268.53428093645,-8.72909698996656)(1270.38795986622,-8.75167224080268)(1275.33110367893,-8.78428093645485)(1275.94899665552,-8.78929765886288)(1280.27424749164,-8.82441471571907)(1283.36371237458,-8.8494983277592)(1285.21739130435,-8.87207357859532)(1288.30685618729,-8.90969899665552)(1290.16053511706,-8.93227424749164)(1292.22017837235,-8.96989966555184)(1293.86789297659,-9.03010033444816)(1295.10367892977,-9.0752508361204)(1295.51560758083,-9.09030100334448)(1295.51560758083,-9.1505016722408)(1295.51560758083,-9.21070234113712)(1295.10367892977,-9.21822742474916)(1292.22017837235,-9.27090301003344)(1290.16053511706,-9.30852842809364)(1288.30685618729,-9.33110367892977)(1285.21739130435,-9.3561872909699)(1280.89214046823,-9.39130434782609)(1280.27424749164,-9.39632107023412)(1275.33110367893,-9.41638795986622)(1271.00585284281,-9.45150501672241)(1270.38795986622,-9.45652173913043)(1265.44481605351,-9.47658862876254)(1261.11956521739,-9.51170568561873)(1260.5016722408,-9.51672240802676)(1255.55852842809,-9.53678929765886)(1250.61538461538,-9.55685618729097)(1245.67224080268,-9.55685618729097)(1241.96488294314,-9.57190635451505)};

\addplot [fill=darkgray,draw=none,forget plot] coordinates{ (1478.00006407589,0)(1478,-0.0526755852842809)(1473.67474916388,0)(1478,7.80355919588601e-07)(1478.00006407589,0)};

\addplot [
color=white,
draw=white,
only marks,
mark=x,
mark options={solid},
mark size=2.0pt,
line width=0.3pt,
forget plot
]
coordinates{
 (370.735785953177,-4.75585284280936)(1478,-18)(0,-15.8327759197324)(1478,0)(1478,-9.27090301003344)(0,0)(588.234113712375,-11.1973244147157)(696.983277591973,-18)(1156.69565217391,-4.63545150501672)(0,-9.03010033444816)(741.471571906354,0)(1250.61538461538,-13.7257525083612)(830.448160535117,-7.82608695652174)(0,-12.4615384615385)(608.006688963211,-14.628762541806)(0,-3.67224080267559)(1478,-5.95986622073579)(207.612040133779,-18)(1057.83277591973,-10.8361204013378)(1478,-2.82943143812709)(751.357859531773,-3.01003344481605)(1077.60535117057,-16.6153846153846)(385.565217391304,-7.88628762541806)(355.90635451505,-1.38461538461538)(0,-6.38127090301003)(1478,-12.0401337792642)(1112.20735785953,-1.02341137123746)(1478,-15.4113712374582)(286.702341137124,-13.7859531772575)(1186.35451505017,-7.64548494983278)(751.357859531773,-5.53846153846154)(884.822742474916,-13.1839464882943)(252.100334448161,-10.7157190635452)(652.494983277592,-9.39130434782609)(1211.07023411371,-9.51170568561873)(528.916387959866,-6.50167224080268)(0,-10.6555183946488)(1478,-7.64548494983278)(1018.28762541806,-6.14046822742475)(232.327759197324,-6.32107023411371)(1240.72909698997,-11.7391304347826)(914.481605351171,-9.51170568561873)(247.157190635452,-9.09030100334448)(825.505016722408,-10.9565217391304)(1478,-10.6555183946488)(1270.38795986622,-6.02006688963211)(0,-7.7056856187291)(444.882943143813,-9.81270903010033)(612.94983277592,-7.88628762541806)(1028.17391304348,-8.30769230769231) 
};

%\node at (axis cs:50, -2.5) [shape=circle,fill=white,draw=black,inner sep=0pt,anchor=south west] {\scriptsize\color{locol}$\*L_{\*t}$};
%\node at (axis cs:460, -5.9) [shape=circle,fill=white,draw=black,inner sep=0pt,anchor=south west] {\scriptsize\color{orange!50!yellow}$\*H_{\*t}$};
%\node at (axis cs:160, -5.2) [shape=circle,fill=white,draw=black,inner sep=0pt,anchor=south west] {\scriptsize\color{darkgray}$\*U_{\*t}$};

\node at (axis cs:805, -17) [shape=circle,fill=green!60!black,draw=black,inner sep=0.2pt,anchor=south west,minimum size=16pt]
  {\scriptsize\color{white}$\*M_{\*t}$};
\node at (axis cs:980, -17) [shape=circle,fill=red!40!yellow,draw=black,inner sep=0.2pt,anchor=south west,minimum size=16pt]
  {\scriptsize\color{white}$\*H_{\*t}$};
\node at (axis cs:1155, -17) [shape=circle,fill=locol,draw=black,inner sep=0.2pt,anchor=south west,minimum size=16pt]
  {\scriptsize\color{white}$\*L_{\*t}$};
\node at (axis cs:1330, -17) [shape=circle,fill=darkgray,draw=black,inner sep=0.2pt,anchor=south west,minimum size=16pt]
  {\scriptsize\color{white}$\*U_{\*t}$};

\end{axis}
\end{tikzpicture}%

%% This file was created by matlab2tikz v0.2.3.
% Copyright (c) 2008--2012, Nico Schlömer <nico.schloemer@gmail.com>
% All rights reserved.
% 
% 
%

\definecolor{locol}{rgb}{0.26, 0.45, 0.65}

\begin{tikzpicture}

\begin{axis}[%
tick label style={font=\tiny},
label style={font=\tiny},
xlabel shift={-10pt},
ylabel shift={-17pt},
legend style={font=\tiny},
view={0}{90},
width=\figurewidth,
height=\figureheight,
scale only axis,
xmin=0, xmax=1478,
xtick={0, 400, 1000, 1400},
xlabel={Length (m)},
ymin=-18, ymax=0,
ytick={0, -4, -14, -18},
ylabel={Depth (m)},
name=plot1,
axis lines*=box,
tickwidth=0.1cm,
clip=false
]

\addplot [fill=locol,draw=none,forget plot] coordinates{ (1478.00014645918,0)(1478.00007323068,-0.0602006688963211)(1478,-0.120401337792642)(1478,-0.180602006688963)(1478,-0.240802675585284)(1478,-0.301003344481605)(1478,-0.361204013377926)(1478,-0.421404682274247)(1478,-0.481605351170569)(1478,-0.54180602006689)(1478,-0.602006688963211)(1478,-0.662207357859532)(1478,-0.722408026755853)(1478,-0.782608695652174)(1478,-0.842809364548495)(1478,-0.903010033444816)(1478,-0.963210702341137)(1478,-1.02341137123746)(1478,-1.08361204013378)(1478,-1.1438127090301)(1478,-1.20401337792642)(1478,-1.26421404682274)(1478,-1.32441471571906)(1478,-1.38461538461538)(1478,-1.44481605351171)(1478,-1.50501672240803)(1478,-1.56521739130435)(1478,-1.62541806020067)(1478,-1.68561872909699)(1478,-1.74581939799331)(1478,-1.80602006688963)(1478,-1.86622073578595)(1478,-1.92642140468227)(1478,-1.9866220735786)(1478,-2.04682274247492)(1478,-2.10702341137124)(1478,-2.16722408026756)(1478,-2.22742474916388)(1478,-2.2876254180602)(1478,-2.34782608695652)(1478,-2.40802675585284)(1478,-2.46822742474916)(1478,-2.52842809364549)(1478,-2.58862876254181)(1478,-2.64882943143813)(1478,-2.70903010033445)(1478,-2.76923076923077)(1478,-2.82943143812709)(1478,-2.88963210702341)(1478,-2.94983277591973)(1478,-3.01003344481605)(1478,-3.07023411371237)(1478,-3.1304347826087)(1478,-3.19063545150502)(1478,-3.25083612040134)(1478,-3.31103678929766)(1478,-3.37123745819398)(1478,-3.4314381270903)(1478,-3.49163879598662)(1478,-3.55183946488294)(1478,-3.61204013377926)(1478,-3.67224080267559)(1478,-3.73244147157191)(1478,-3.79264214046823)(1478,-3.85284280936455)(1478,-3.91304347826087)(1478,-3.97324414715719)(1478,-4.03344481605351)(1478,-4.09364548494983)(1478,-4.15384615384615)(1478,-4.21404682274247)(1478,-4.2742474916388)(1478,-4.33444816053512)(1478,-4.39464882943144)(1478,-4.45484949832776)(1478,-4.51505016722408)(1478,-4.5752508361204)(1478,-4.63545150501672)(1478,-4.69565217391304)(1478,-4.75585284280936)(1478,-4.81605351170569)(1478,-4.87625418060201)(1478,-4.93645484949833)(1478,-4.99665551839465)(1478,-5.05685618729097)(1478,-5.11705685618729)(1478,-5.17725752508361)(1478,-5.23745819397993)(1478,-5.29765886287625)(1478,-5.35785953177258)(1478,-5.4180602006689)(1478,-5.47826086956522)(1478,-5.53846153846154)(1478,-5.59866220735786)(1478,-5.65886287625418)(1478,-5.7190635451505)(1478,-5.77926421404682)(1478,-5.83946488294314)(1478,-5.89966555183946)(1478,-5.95986622073579)(1478,-6.02006688963211)(1478,-6.08026755852843)(1478,-6.14046822742475)(1478,-6.20066889632107)(1478,-6.26086956521739)(1478,-6.32107023411371)(1478,-6.38127090301003)(1478,-6.44147157190635)(1478,-6.50167224080268)(1478,-6.561872909699)(1478,-6.62207357859532)(1478,-6.68227424749164)(1478,-6.74247491638796)(1478,-6.80267558528428)(1478,-6.8628762541806)(1478,-6.92307692307692)(1478,-6.98327759197324)(1478,-7.04347826086957)(1478,-7.10367892976589)(1478,-7.16387959866221)(1478,-7.22408026755853)(1478,-7.28428093645485)(1478,-7.34448160535117)(1478,-7.40468227424749)(1478,-7.46488294314381)(1478,-7.52508361204013)(1478,-7.58528428093646)(1478,-7.64548494983278)(1478,-7.7056856187291)(1478,-7.76588628762542)(1478,-7.82608695652174)(1478,-7.88628762541806)(1478,-7.94648829431438)(1478,-8.0066889632107)(1478,-8.06688963210702)(1478,-8.12709030100334)(1478,-8.18729096989967)(1478,-8.24749163879599)(1478,-8.30769230769231)(1478,-8.36789297658863)(1478,-8.42809364548495)(1478,-8.48829431438127)(1478,-8.54849498327759)(1478,-8.60869565217391)(1478,-8.66889632107023)(1478,-8.72909698996656)(1478,-8.78929765886288)(1478.00001830787,-8.8494983277592)(1478.00007323068,-8.90969899665552)(1478.00012815226,-8.96989966555184)(1478.00016476597,-9.03010033444816)(1478.00016476597,-9.09030100334448)(1478.00016476597,-9.1505016722408)(1478.00020137914,-9.21070234113712)(1478.00020137914,-9.27090301003344)(1478.00020137914,-9.33110367892977)(1478.00016476597,-9.39130434782609)(1478.00016476597,-9.45150501672241)(1478.00016476597,-9.51170568561873)(1478.00016476597,-9.57190635451505)(1478.00016476597,-9.63210702341137)(1478.00016476597,-9.69230769230769)(1478.00016476597,-9.75250836120401)(1478.00016476597,-9.81270903010033)(1478.00016476597,-9.87290969899666)(1478.00016476597,-9.93311036789298)(1478.00012815226,-9.9933110367893)(1478.00012815226,-10.0535117056856)(1478.00009153801,-10.1137123745819)(1478.00007323068,-10.1739130434783)(1478.00001830787,-10.2341137123746)(1478,-10.2943143812709)(1478,-10.3545150501672)(1478,-10.4147157190635)(1478,-10.4749163879599)(1478,-10.5351170568562)(1478,-10.5953177257525)(1478,-10.6555183946488)(1478,-10.7157190635452)(1478,-10.7759197324415)(1478,-10.8361204013378)(1478,-10.8963210702341)(1478,-10.9565217391304)(1478,-11.0167224080268)(1478,-11.0769230769231)(1478,-11.1371237458194)(1478,-11.1973244147157)(1478,-11.257525083612)(1478,-11.3177257525084)(1478,-11.3779264214047)(1478,-11.438127090301)(1478,-11.4983277591973)(1478,-11.5585284280936)(1478,-11.61872909699)(1478,-11.6789297658863)(1478,-11.7391304347826)(1478,-11.7993311036789)(1478,-11.8595317725753)(1478,-11.9197324414716)(1478,-11.9799331103679)(1478,-12.0401337792642)(1478,-12.1003344481605)(1478,-12.1605351170569)(1478,-12.2207357859532)(1478,-12.2809364548495)(1478,-12.3411371237458)(1478,-12.4013377926421)(1478,-12.4615384615385)(1478,-12.5217391304348)(1478,-12.5819397993311)(1478,-12.6421404682274)(1478,-12.7023411371237)(1478,-12.7625418060201)(1478,-12.8227424749164)(1478,-12.8829431438127)(1478,-12.943143812709)(1478,-13.0033444816054)(1478,-13.0635451505017)(1478,-13.123745819398)(1478,-13.1839464882943)(1478,-13.2441471571906)(1478,-13.304347826087)(1478,-13.3645484949833)(1478,-13.4247491638796)(1478,-13.4849498327759)(1478,-13.5451505016722)(1478,-13.6053511705686)(1478,-13.6655518394649)(1478,-13.7257525083612)(1478,-13.7859531772575)(1478,-13.8461538461538)(1478,-13.9063545150502)(1478,-13.9665551839465)(1478,-14.0267558528428)(1478,-14.0869565217391)(1478,-14.1471571906355)(1478,-14.2073578595318)(1478,-14.2675585284281)(1478,-14.3277591973244)(1478,-14.3879598662207)(1478,-14.4481605351171)(1478,-14.5083612040134)(1478,-14.5685618729097)(1478,-14.628762541806)(1478,-14.6889632107023)(1478,-14.7491638795987)(1478,-14.809364548495)(1478,-14.8695652173913)(1478,-14.9297658862876)(1478,-14.9899665551839)(1478,-15.0501672240803)(1478,-15.1103678929766)(1478,-15.1705685618729)(1478,-15.2307692307692)(1478,-15.2909698996656)(1478,-15.3511705685619)(1478,-15.4113712374582)(1478,-15.4715719063545)(1478,-15.5317725752508)(1478,-15.5919732441472)(1478,-15.6521739130435)(1478,-15.7123745819398)(1478,-15.7725752508361)(1478,-15.8327759197324)(1478,-15.8929765886288)(1478,-15.9531772575251)(1478,-16.0133779264214)(1478,-16.0735785953177)(1478,-16.133779264214)(1478,-16.1939799331104)(1478,-16.2541806020067)(1478,-16.314381270903)(1478,-16.3745819397993)(1478,-16.4347826086957)(1478,-16.494983277592)(1478,-16.5551839464883)(1478,-16.6153846153846)(1478,-16.6755852842809)(1478,-16.7357859531773)(1478,-16.7959866220736)(1478,-16.8561872909699)(1478,-16.9163879598662)(1478,-16.9765886287625)(1478,-17.0367892976589)(1478,-17.0969899665552)(1478,-17.1571906354515)(1478,-17.2173913043478)(1478,-17.2775919732441)(1478,-17.3377926421405)(1478,-17.3979933110368)(1478,-17.4581939799331)(1478,-17.5183946488294)(1478,-17.5785953177258)(1478,-17.6387959866221)(1478,-17.6989966555184)(1478,-17.7591973244147)(1478,-17.819397993311)(1478,-17.8795986622074)(1478,-17.9397993311037)(1478,-18)(1473.05685618729,-18)(1468.11371237458,-18)(1463.17056856187,-18)(1458.22742474916,-18)(1453.28428093645,-18)(1448.34113712375,-18)(1443.39799331104,-18)(1438.45484949833,-18)(1433.51170568562,-18)(1428.56856187291,-18)(1423.6254180602,-18)(1418.68227424749,-18)(1413.73913043478,-18)(1408.79598662207,-18)(1403.85284280936,-18)(1398.90969899666,-18)(1393.96655518395,-18)(1389.02341137124,-18)(1384.08026755853,-18)(1379.13712374582,-18)(1374.19397993311,-18)(1369.2508361204,-18)(1364.30769230769,-18)(1359.36454849498,-18)(1354.42140468227,-18)(1349.47826086957,-18)(1344.53511705686,-18)(1339.59197324415,-18)(1334.64882943144,-18)(1329.70568561873,-18)(1324.76254180602,-18)(1319.81939799331,-18)(1314.8762541806,-18)(1309.93311036789,-18)(1304.98996655518,-18)(1300.04682274247,-18)(1295.10367892977,-18)(1290.16053511706,-18)(1285.21739130435,-18)(1280.27424749164,-18)(1275.33110367893,-18)(1270.38795986622,-18)(1265.44481605351,-18)(1260.5016722408,-18)(1255.55852842809,-18)(1250.61538461538,-18)(1245.67224080268,-18)(1240.72909698997,-18)(1235.78595317726,-18)(1230.84280936455,-18)(1225.89966555184,-18)(1220.95652173913,-18)(1216.01337792642,-18)(1211.07023411371,-18)(1206.127090301,-18)(1201.18394648829,-18)(1196.24080267559,-18)(1191.29765886288,-18)(1186.35451505017,-18)(1181.41137123746,-18)(1176.46822742475,-18)(1171.52508361204,-18)(1166.58193979933,-18)(1161.63879598662,-18)(1156.69565217391,-18)(1151.7525083612,-18)(1146.8093645485,-18)(1141.86622073579,-18)(1136.92307692308,-18)(1131.97993311037,-18)(1127.03678929766,-18)(1122.09364548495,-18)(1117.15050167224,-18)(1112.20735785953,-18)(1107.26421404682,-18)(1102.32107023411,-18)(1097.3779264214,-18)(1092.4347826087,-18)(1087.49163879599,-18)(1082.54849498328,-18)(1077.60535117057,-18)(1072.66220735786,-18)(1067.71906354515,-18)(1062.77591973244,-18)(1057.83277591973,-18)(1052.88963210702,-18)(1047.94648829431,-18)(1043.00334448161,-18)(1038.0602006689,-18)(1033.11705685619,-18)(1028.17391304348,-18)(1023.23076923077,-18)(1018.28762541806,-18)(1013.34448160535,-18)(1008.40133779264,-18)(1003.45819397993,-18)(998.515050167224,-18)(993.571906354515,-18)(988.628762541806,-18)(983.685618729097,-18)(978.742474916388,-18)(973.799331103679,-18)(968.85618729097,-18)(963.913043478261,-18)(958.969899665552,-18)(954.026755852843,-18)(949.083612040134,-18)(944.140468227425,-18)(939.197324414716,-18)(934.254180602007,-18)(929.311036789298,-18)(924.367892976589,-18)(919.42474916388,-18)(914.481605351171,-18)(909.538461538462,-18)(904.595317725752,-18)(899.652173913044,-18)(894.709030100334,-18)(889.765886287625,-18)(884.822742474916,-18)(879.879598662207,-18)(874.936454849498,-18)(869.993311036789,-18)(865.05016722408,-18)(860.107023411371,-18)(855.163879598662,-18)(850.220735785953,-18)(845.277591973244,-18)(840.334448160535,-18)(835.391304347826,-18)(830.448160535117,-18)(825.505016722408,-18)(820.561872909699,-18)(815.61872909699,-18)(810.675585284281,-18)(805.732441471572,-18)(800.789297658863,-18)(795.846153846154,-18)(790.903010033445,-18)(785.959866220736,-18)(781.016722408027,-18)(776.073578595318,-18)(771.130434782609,-18)(766.1872909699,-18)(761.244147157191,-18)(756.301003344482,-18)(751.357859531773,-18)(746.414715719064,-18)(741.471571906354,-18)(736.528428093646,-18)(731.585284280936,-18)(726.642140468227,-18)(721.698996655518,-18)(716.755852842809,-18)(711.8127090301,-18)(706.869565217391,-18)(701.926421404682,-18)(696.983277591973,-18)(692.040133779264,-18)(687.096989966555,-18)(682.153846153846,-18)(677.210702341137,-18)(672.267558528428,-18)(667.324414715719,-18)(662.38127090301,-18)(657.438127090301,-18)(652.494983277592,-18)(647.551839464883,-18)(642.608695652174,-18)(637.665551839465,-18)(632.722408026756,-18)(627.779264214047,-18)(622.836120401338,-18)(617.892976588629,-18)(612.94983277592,-18)(608.006688963211,-18)(603.063545150502,-18)(598.120401337793,-18)(593.177257525084,-18)(588.234113712375,-18)(583.290969899666,-18)(578.347826086957,-18)(573.404682274248,-18)(568.461538461538,-18)(563.518394648829,-18)(558.57525083612,-18)(553.632107023411,-18)(548.688963210702,-18)(543.745819397993,-18)(538.802675585284,-18)(533.859531772575,-18)(528.916387959866,-18)(523.973244147157,-18)(519.030100334448,-18)(514.086956521739,-18)(509.14381270903,-18)(504.200668896321,-18)(499.257525083612,-18)(494.314381270903,-18)(489.371237458194,-18)(484.428093645485,-18)(479.484949832776,-18)(474.541806020067,-18)(469.598662207358,-18)(464.655518394649,-18)(459.71237458194,-18)(454.769230769231,-18)(449.826086956522,-18)(444.882943143813,-18)(439.939799331104,-18)(434.996655518395,-18)(430.053511705686,-18)(425.110367892977,-18)(420.167224080268,-18)(415.224080267559,-18)(410.28093645485,-18)(405.33779264214,-18)(400.394648829431,-18)(395.451505016722,-18)(390.508361204013,-18)(385.565217391304,-18)(380.622073578595,-18)(375.678929765886,-18)(370.735785953177,-18)(365.792642140468,-18)(360.849498327759,-18)(355.90635451505,-18)(350.963210702341,-18)(346.020066889632,-18)(341.076923076923,-18)(336.133779264214,-18)(331.190635451505,-18)(326.247491638796,-18)(321.304347826087,-18)(316.361204013378,-18)(311.418060200669,-18)(306.47491638796,-18)(301.531772575251,-18)(296.588628762542,-18)(291.645484949833,-18)(286.702341137124,-18)(281.759197324415,-18)(276.816053511706,-18)(271.872909698997,-18)(266.929765886288,-18)(261.986622073579,-18)(257.04347826087,-18)(252.100334448161,-18)(247.157190635452,-18)(242.214046822742,-18)(237.270903010033,-18)(232.327759197324,-18)(227.384615384615,-18)(222.441471571906,-18)(217.498327759197,-18)(212.555183946488,-18)(207.612040133779,-18)(202.66889632107,-18)(197.725752508361,-18)(192.782608695652,-18)(187.839464882943,-18)(182.896321070234,-18)(177.953177257525,-18)(173.010033444816,-18)(168.066889632107,-18)(163.123745819398,-18)(158.180602006689,-18)(153.23745819398,-18)(148.294314381271,-18)(143.351170568562,-18)(138.408026755853,-18)(133.464882943144,-18)(128.521739130435,-18)(123.578595317726,-18)(118.635451505017,-18)(113.692307692308,-18)(108.749163879599,-18)(103.80602006689,-18)(98.8628762541806,-18)(93.9197324414716,-18)(88.9765886287625,-18)(84.0334448160535,-18)(79.0903010033445,-18)(74.1471571906355,-18)(69.2040133779264,-18)(64.2608695652174,-18)(59.3177257525084,-18)(54.3745819397993,-18)(49.4314381270903,-18)(44.4882943143813,-18)(39.5451505016722,-18)(34.6020066889632,-18)(29.6588628762542,-18)(24.7157190635452,-18)(19.7725752508361,-18)(14.8294314381271,-18)(9.88628762541806,-18)(4.94314381270903,-18)(0,-18)(0,-17.9397993311037)(0,-17.8795986622074)(0,-17.819397993311)(0,-17.7591973244147)(0,-17.6989966555184)(0,-17.6387959866221)(0,-17.5785953177258)(0,-17.5183946488294)(0,-17.4581939799331)(0,-17.3979933110368)(0,-17.3377926421405)(0,-17.2775919732441)(0,-17.2173913043478)(0,-17.1571906354515)(0,-17.0969899665552)(0,-17.0367892976589)(0,-16.9765886287625)(0,-16.9163879598662)(0,-16.8561872909699)(0,-16.7959866220736)(0,-16.7357859531773)(0,-16.6755852842809)(0,-16.6153846153846)(0,-16.5551839464883)(0,-16.494983277592)(0,-16.4347826086957)(0,-16.3745819397993)(0,-16.314381270903)(0,-16.2541806020067)(0,-16.1939799331104)(0,-16.133779264214)(0,-16.0735785953177)(0,-16.0133779264214)(0,-15.9531772575251)(0,-15.8929765886288)(0,-15.8327759197324)(0,-15.7725752508361)(0,-15.7123745819398)(0,-15.6521739130435)(0,-15.5919732441472)(0,-15.5317725752508)(0,-15.4715719063545)(0,-15.4113712374582)(0,-15.3511705685619)(0,-15.2909698996656)(0,-15.2307692307692)(0,-15.1705685618729)(0,-15.1103678929766)(0,-15.0501672240803)(0,-14.9899665551839)(0,-14.9297658862876)(0,-14.8695652173913)(0,-14.809364548495)(0,-14.7491638795987)(0,-14.6889632107023)(0,-14.628762541806)(0,-14.5685618729097)(0,-14.5083612040134)(0,-14.4481605351171)(0,-14.3879598662207)(0,-14.3277591973244)(0,-14.2675585284281)(0,-14.2073578595318)(0,-14.1471571906355)(0,-14.0869565217391)(0,-14.0267558528428)(0,-13.9665551839465)(0,-13.9063545150502)(0,-13.8461538461538)(0,-13.7859531772575)(0,-13.7257525083612)(0,-13.6655518394649)(0,-13.6053511705686)(0,-13.5451505016722)(0,-13.4849498327759)(0,-13.4247491638796)(0,-13.3645484949833)(0,-13.304347826087)(0,-13.2441471571906)(0,-13.1839464882943)(0,-13.123745819398)(0,-13.0635451505017)(0,-13.0033444816054)(0,-12.943143812709)(0,-12.8829431438127)(0,-12.8227424749164)(0,-12.7625418060201)(0,-12.7023411371237)(0,-12.6421404682274)(0,-12.5819397993311)(0,-12.5217391304348)(0,-12.4615384615385)(0,-12.4013377926421)(0,-12.3411371237458)(0,-12.2809364548495)(0,-12.2207357859532)(0,-12.1605351170569)(0,-12.1003344481605)(0,-12.0401337792642)(0,-11.9799331103679)(0,-11.9197324414716)(0,-11.8595317725753)(0,-11.7993311036789)(0,-11.7391304347826)(0,-11.6789297658863)(0,-11.61872909699)(0,-11.5585284280936)(0,-11.4983277591973)(0,-11.438127090301)(0,-11.3779264214047)(0,-11.3177257525084)(0,-11.257525083612)(0,-11.1973244147157)(0,-11.1371237458194)(0,-11.0769230769231)(0,-11.0167224080268)(0,-10.9565217391304)(0,-10.8963210702341)(0,-10.8361204013378)(0,-10.7759197324415)(0,-10.7157190635452)(0,-10.6555183946488)(0,-10.5953177257525)(0,-10.5351170568562)(0,-10.4749163879599)(0,-10.4147157190635)(0,-10.3545150501672)(0,-10.2943143812709)(0,-10.2341137123746)(0,-10.1739130434783)(0,-10.1137123745819)(0,-10.0535117056856)(0,-9.9933110367893)(0,-9.93311036789298)(0,-9.87290969899666)(0,-9.81270903010033)(0,-9.75250836120401)(0,-9.69230769230769)(0,-9.63210702341137)(0,-9.57190635451505)(0,-9.51170568561873)(0,-9.45150501672241)(0,-9.39130434782609)(0,-9.33110367892977)(0,-9.27090301003344)(0,-9.21070234113712)(0,-9.1505016722408)(0,-9.09030100334448)(0,-9.03010033444816)(0,-8.96989966555184)(0,-8.90969899665552)(0,-8.8494983277592)(0,-8.78929765886288)(0,-8.72909698996656)(0,-8.66889632107023)(0,-8.60869565217391)(0,-8.54849498327759)(0,-8.48829431438127)(0,-8.42809364548495)(0,-8.36789297658863)(0,-8.30769230769231)(0,-8.24749163879599)(0,-8.18729096989967)(0,-8.12709030100334)(0,-8.06688963210702)(0,-8.0066889632107)(0,-7.94648829431438)(0,-7.88628762541806)(0,-7.82608695652174)(0,-7.76588628762542)(0,-7.7056856187291)(0,-7.64548494983278)(0,-7.58528428093646)(0,-7.52508361204013)(0,-7.46488294314381)(0,-7.40468227424749)(0,-7.34448160535117)(0,-7.28428093645485)(0,-7.22408026755853)(0,-7.16387959866221)(0,-7.10367892976589)(0,-7.04347826086957)(0,-6.98327759197324)(0,-6.92307692307692)(0,-6.8628762541806)(0,-6.80267558528428)(0,-6.74247491638796)(0,-6.68227424749164)(0,-6.62207357859532)(0,-6.561872909699)(0,-6.50167224080268)(-7.32306752895604e-05,-6.44147157190635)(-7.32306752895604e-05,-6.38127090301003)(-7.32306752895604e-05,-6.32107023411371)(0,-6.26086956521739)(0,-6.20066889632107)(0,-6.14046822742475)(0,-6.08026755852843)(0,-6.02006688963211)(0,-5.95986622073579)(0,-5.89966555183946)(0,-5.83946488294314)(0,-5.77926421404682)(0,-5.7190635451505)(0,-5.65886287625418)(0,-5.59866220735786)(0,-5.53846153846154)(0,-5.47826086956522)(0,-5.4180602006689)(0,-5.35785953177258)(0,-5.29765886287625)(0,-5.23745819397993)(0,-5.17725752508361)(0,-5.11705685618729)(0,-5.05685618729097)(0,-4.99665551839465)(0,-4.93645484949833)(0,-4.87625418060201)(0,-4.81605351170569)(0,-4.75585284280936)(0,-4.69565217391304)(0,-4.63545150501672)(0,-4.5752508361204)(0,-4.51505016722408)(0,-4.45484949832776)(0,-4.39464882943144)(0,-4.33444816053512)(0,-4.2742474916388)(0,-4.21404682274247)(0,-4.15384615384615)(0,-4.09364548494983)(0,-4.03344481605351)(0,-3.97324414715719)(0,-3.91304347826087)(0,-3.85284280936455)(0,-3.79264214046823)(0,-3.73244147157191)(0,-3.67224080267559)(0,-3.61204013377926)(0,-3.55183946488294)(0,-3.49163879598662)(0,-3.4314381270903)(0,-3.37123745819398)(0,-3.31103678929766)(0,-3.25083612040134)(0,-3.19063545150502)(0,-3.1304347826087)(0,-3.07023411371237)(0,-3.01003344481605)(0,-2.94983277591973)(0,-2.88963210702341)(0,-2.82943143812709)(0,-2.76923076923077)(0,-2.70903010033445)(0,-2.64882943143813)(0,-2.58862876254181)(0,-2.52842809364549)(0,-2.46822742474916)(0,-2.40802675585284)(0,-2.34782608695652)(0,-2.2876254180602)(0,-2.22742474916388)(0,-2.16722408026756)(0,-2.10702341137124)(0,-2.04682274247492)(0,-1.9866220735786)(0,-1.92642140468227)(0,-1.86622073578595)(0,-1.80602006688963)(0,-1.74581939799331)(0,-1.68561872909699)(0,-1.62541806020067)(0,-1.56521739130435)(0,-1.50501672240803)(0,-1.44481605351171)(0,-1.38461538461538)(0,-1.32441471571906)(0,-1.26421404682274)(0,-1.20401337792642)(0,-1.1438127090301)(0,-1.08361204013378)(0,-1.02341137123746)(0,-0.963210702341137)(0,-0.903010033444816)(0,-0.842809364548495)(0,-0.782608695652174)(0,-0.722408026755853)(0,-0.662207357859532)(0,-0.602006688963211)(0,-0.54180602006689)(0,-0.481605351170569)(0,-0.421404682274247)(0,-0.361204013377926)(0,-0.301003344481605)(0,-0.240802675585284)(0,-0.180602006688963)(0,-0.120401337792642)(0,-0.0602006688963211)(0,0)(4.94314381270903,0)(9.88628762541806,0)(14.8294314381271,0)(19.7725752508361,0)(24.7157190635452,0)(29.6588628762542,0)(34.6020066889632,0)(39.5451505016722,0)(44.4882943143813,0)(49.4314381270903,0)(54.3745819397993,0)(59.3177257525084,0)(64.2608695652174,0)(69.2040133779264,0)(74.1471571906355,0)(79.0903010033445,0)(84.0334448160535,0)(88.9765886287625,0)(93.9197324414716,0)(98.8628762541806,0)(103.80602006689,0)(108.749163879599,0)(113.692307692308,0)(118.635451505017,0)(123.578595317726,0)(128.521739130435,0)(133.464882943144,0)(138.408026755853,0)(143.351170568562,0)(148.294314381271,0)(153.23745819398,0)(158.180602006689,0)(163.123745819398,0)(168.066889632107,0)(173.010033444816,0)(177.953177257525,0)(182.896321070234,0)(187.839464882943,0)(192.782608695652,0)(197.725752508361,0)(202.66889632107,0)(207.612040133779,0)(212.555183946488,0)(217.498327759197,0)(222.441471571906,0)(227.384615384615,0)(232.327759197324,0)(237.270903010033,0)(242.214046822742,0)(247.157190635452,0)(252.100334448161,0)(257.04347826087,0)(261.986622073579,0)(266.929765886288,0)(271.872909698997,0)(276.816053511706,0)(281.759197324415,0)(286.702341137124,0)(291.645484949833,0)(296.588628762542,0)(301.531772575251,0)(306.47491638796,0)(311.418060200669,0)(316.361204013378,0)(321.304347826087,0)(326.247491638796,0)(331.190635451505,0)(336.133779264214,0)(341.076923076923,0)(346.020066889632,0)(350.963210702341,0)(355.90635451505,0)(360.849498327759,0)(365.792642140468,0)(370.735785953177,0)(375.678929765886,0)(380.622073578595,0)(385.565217391304,0)(390.508361204013,0)(395.451505016722,0)(400.394648829431,0)(405.33779264214,0)(410.28093645485,0)(415.224080267559,0)(420.167224080268,0)(425.110367892977,0)(430.053511705686,0)(434.996655518395,0)(439.939799331104,0)(444.882943143813,0)(449.826086956522,0)(454.769230769231,0)(459.71237458194,0)(464.655518394649,0)(469.598662207358,0)(474.541806020067,0)(479.484949832776,0)(484.428093645485,0)(489.371237458194,0)(494.314381270903,0)(499.257525083612,0)(504.200668896321,0)(509.14381270903,0)(514.086956521739,0)(519.030100334448,0)(523.973244147157,0)(528.916387959866,0)(533.859531772575,0)(538.802675585284,0)(543.745819397993,0)(548.688963210702,0)(553.632107023411,0)(558.57525083612,0)(563.518394648829,0)(568.461538461538,0)(573.404682274248,0)(578.347826086957,0)(583.290969899666,0)(588.234113712375,0)(593.177257525084,0)(598.120401337793,0)(603.063545150502,0)(608.006688963211,0)(612.94983277592,0)(617.892976588629,0)(622.836120401338,0)(627.779264214047,0)(632.722408026756,0)(637.665551839465,0)(642.608695652174,0)(647.551839464883,0)(652.494983277592,0)(657.438127090301,0)(662.38127090301,0)(667.324414715719,0)(672.267558528428,0)(677.210702341137,0)(682.153846153846,0)(687.096989966555,0)(692.040133779264,0)(696.983277591973,0)(701.926421404682,0)(706.869565217391,0)(711.8127090301,0)(716.755852842809,0)(721.698996655518,0)(726.642140468227,0)(731.585284280936,0)(736.528428093646,0)(741.471571906354,0)(746.414715719064,0)(751.357859531773,0)(756.301003344482,0)(761.244147157191,0)(766.1872909699,0)(771.130434782609,0)(776.073578595318,0)(781.016722408027,0)(785.959866220736,0)(790.903010033445,0)(795.846153846154,0)(800.789297658863,0)(805.732441471572,0)(810.675585284281,0)(815.61872909699,0)(820.561872909699,0)(825.505016722408,0)(830.448160535117,0)(835.391304347826,0)(840.334448160535,0)(845.277591973244,0)(850.220735785953,0)(855.163879598662,0)(860.107023411371,0)(865.05016722408,0)(869.993311036789,0)(874.936454849498,0)(879.879598662207,0)(884.822742474916,0)(889.765886287625,0)(894.709030100334,0)(899.652173913044,0)(904.595317725752,0)(909.538461538462,0)(914.481605351171,0)(919.42474916388,0)(924.367892976589,0)(929.311036789298,0)(934.254180602007,0)(939.197324414716,0)(944.140468227425,0)(949.083612040134,0)(954.026755852843,0)(958.969899665552,0)(963.913043478261,0)(968.85618729097,0)(973.799331103679,0)(978.742474916388,0)(983.685618729097,0)(988.628762541806,0)(993.571906354515,0)(998.515050167224,0)(1003.45819397993,0)(1008.40133779264,0)(1013.34448160535,0)(1018.28762541806,0)(1023.23076923077,0)(1028.17391304348,0)(1033.11705685619,0)(1038.0602006689,0)(1043.00334448161,0)(1047.94648829431,0)(1052.88963210702,0)(1057.83277591973,0)(1062.77591973244,0)(1067.71906354515,0)(1072.66220735786,0)(1077.60535117057,0)(1082.54849498328,0)(1087.49163879599,0)(1092.4347826087,0)(1097.3779264214,0)(1102.32107023411,0)(1107.26421404682,0)(1112.20735785953,0)(1117.15050167224,0)(1122.09364548495,0)(1127.03678929766,0)(1131.97993311037,0)(1136.92307692308,0)(1141.86622073579,0)(1146.8093645485,0)(1151.7525083612,0)(1156.69565217391,0)(1161.63879598662,0)(1166.58193979933,0)(1171.52508361204,0)(1176.46822742475,0)(1181.41137123746,0)(1186.35451505017,0)(1191.29765886288,0)(1196.24080267559,0)(1201.18394648829,0)(1206.127090301,0)(1211.07023411371,0)(1216.01337792642,0)(1220.95652173913,0)(1225.89966555184,0)(1230.84280936455,0)(1235.78595317726,0)(1240.72909698997,0)(1245.67224080268,0)(1250.61538461538,0)(1255.55852842809,0)(1260.5016722408,0)(1265.44481605351,0)(1270.38795986622,0)(1275.33110367893,0)(1280.27424749164,0)(1285.21739130435,0)(1290.16053511706,0)(1295.10367892977,0)(1300.04682274247,0)(1304.98996655518,0)(1309.93311036789,0)(1314.8762541806,0)(1319.81939799331,0)(1324.76254180602,0)(1329.70568561873,0)(1334.64882943144,0)(1339.59197324415,0)(1344.53511705686,0)(1349.47826086957,0)(1354.42140468227,0)(1359.36454849498,0)(1364.30769230769,0)(1369.2508361204,0)(1374.19397993311,0)(1379.13712374582,0)(1384.08026755853,0)(1389.02341137124,0)(1393.96655518395,0)(1398.90969899666,0)(1403.85284280936,0)(1408.79598662207,0)(1413.73913043478,0)(1418.68227424749,0)(1423.6254180602,0)(1428.56856187291,0)(1433.51170568562,0)(1438.45484949833,0)(1443.39799331104,0)(1448.34113712375,0)(1453.28428093645,0)(1458.22742474916,0)(1463.17056856187,0)(1468.11371237458,0)(1473.05685618729,8.91848548857975e-07)(1478,1.78367067334156e-06)(1478.00014645918,0)};

\addplot [fill=darkgray,draw=none,forget plot] coordinates{ (556.103678929766,-6.8628762541806)(558.57525083612,-6.87290969899666)(563.518394648829,-6.89297658862876)(568.461538461538,-6.89297658862876)(573.404682274248,-6.91304347826087)(575.876254180602,-6.92307692307692)(578.347826086957,-6.93311036789298)(583.290969899666,-6.95317725752508)(588.234113712375,-6.97324414715719)(590.705685618729,-6.98327759197324)(593.177257525084,-6.9933110367893)(598.120401337793,-7.0133779264214)(603.063545150502,-7.03344481605351)(605.535117056856,-7.04347826086957)(608.006688963211,-7.05852842809364)(612.94983277592,-7.09364548494983)(615.421404682274,-7.10367892976589)(617.892976588629,-7.11371237458194)(622.836120401338,-7.15384615384615)(625.307692307692,-7.16387959866221)(627.779264214047,-7.17391304347826)(632.722408026756,-7.21404682274247)(633.958193979933,-7.22408026755853)(637.665551839465,-7.25418060200669)(641.372909698997,-7.28428093645485)(642.608695652174,-7.2943143812709)(647.551839464883,-7.33444816053512)(648.78762541806,-7.34448160535117)(652.494983277592,-7.37458193979933)(654.966555183946,-7.40468227424749)(657.438127090301,-7.43478260869565)(661.145484949833,-7.46488294314381)(662.38127090301,-7.47491638795987)(667.324414715719,-7.51505016722408)(669.795986622074,-7.52508361204013)(672.267558528428,-7.53511705685619)(677.210702341137,-7.5752508361204)(679.682274247492,-7.58528428093646)(682.153846153846,-7.59531772575251)(687.096989966555,-7.63545150501672)(689.56856187291,-7.64548494983278)(692.040133779264,-7.65551839464883)(696.983277591973,-7.67558528428094)(701.926421404682,-7.69565217391304)(704.397993311037,-7.7056856187291)(706.869565217391,-7.71571906354515)(711.8127090301,-7.73578595317726)(716.755852842809,-7.73578595317726)(721.698996655518,-7.73578595317726)(726.642140468227,-7.73578595317726)(731.585284280936,-7.73578595317726)(736.528428093646,-7.73578595317726)(741.471571906354,-7.73578595317726)(746.414715719064,-7.73578595317726)(751.357859531773,-7.73578595317726)(756.301003344482,-7.71571906354515)(758.772575250836,-7.7056856187291)(761.244147157191,-7.69565217391304)(766.1872909699,-7.67558528428094)(771.130434782609,-7.67558528428094)(776.073578595318,-7.67558528428094)(781.016722408027,-7.67558528428094)(785.959866220736,-7.67558528428094)(790.903010033445,-7.65551839464883)(793.374581939799,-7.64548494983278)(795.846153846154,-7.63545150501672)(800.789297658863,-7.61538461538462)(805.732441471572,-7.59531772575251)(808.204013377926,-7.58528428093646)(810.675585284281,-7.5752508361204)(815.61872909699,-7.55518394648829)(820.561872909699,-7.53511705685619)(823.033444816054,-7.52508361204013)(825.505016722408,-7.51505016722408)(830.448160535117,-7.49498327759197)(835.391304347826,-7.47491638795987)(837.862876254181,-7.46488294314381)(840.334448160535,-7.45484949832776)(845.277591973244,-7.43478260869565)(850.220735785953,-7.41471571906355)(852.692307692308,-7.40468227424749)(855.163879598662,-7.39464882943144)(860.107023411371,-7.37458193979933)(865.05016722408,-7.35451505016722)(867.521739130435,-7.34448160535117)(869.993311036789,-7.33444816053512)(874.936454849498,-7.2943143812709)(877.408026755853,-7.28428093645485)(879.879598662207,-7.2742474916388)(884.822742474916,-7.25418060200669)(889.765886287625,-7.25418060200669)(894.709030100335,-7.23411371237458)(897.180602006689,-7.22408026755853)(899.652173913044,-7.21404682274247)(904.595317725752,-7.19397993311037)(909.538461538462,-7.19397993311037)(914.481605351171,-7.17391304347826)(916.953177257525,-7.16387959866221)(919.42474916388,-7.15384615384615)(924.367892976589,-7.13377926421405)(929.311036789298,-7.13377926421405)(934.254180602007,-7.11371237458194)(936.725752508361,-7.10367892976589)(939.197324414716,-7.09364548494983)(944.140468227425,-7.07357859531773)(949.083612040134,-7.07357859531773)(954.026755852843,-7.07357859531773)(958.969899665552,-7.07357859531773)(963.913043478261,-7.07357859531773)(968.85618729097,-7.07357859531773)(973.799331103679,-7.05351170568562)(976.270903010033,-7.04347826086957)(978.742474916388,-7.03344481605351)(983.685618729097,-7.0133779264214)(988.628762541806,-7.0133779264214)(993.571906354515,-7.0133779264214)(998.515050167224,-7.0133779264214)(1003.45819397993,-7.0133779264214)(1008.40133779264,-7.0133779264214)(1013.34448160535,-7.0133779264214)(1018.28762541806,-7.0133779264214)(1023.23076923077,-7.0133779264214)(1028.17391304348,-7.0133779264214)(1033.11705685619,-7.0133779264214)(1038.0602006689,-7.0133779264214)(1043.00334448161,-7.0133779264214)(1047.94648829431,-7.0133779264214)(1052.88963210702,-7.0133779264214)(1057.83277591973,-7.0133779264214)(1062.77591973244,-7.0133779264214)(1067.71906354515,-7.0133779264214)(1072.66220735786,-7.0133779264214)(1077.60535117057,-7.0133779264214)(1082.54849498328,-7.03344481605351)(1085.02006688963,-7.04347826086957)(1087.49163879599,-7.05351170568562)(1092.4347826087,-7.07357859531773)(1097.3779264214,-7.07357859531773)(1102.32107023411,-7.07357859531773)(1107.26421404682,-7.07357859531773)(1112.20735785953,-7.07357859531773)(1117.15050167224,-7.09364548494983)(1119.6220735786,-7.10367892976589)(1122.09364548495,-7.11371237458194)(1127.03678929766,-7.13377926421405)(1131.97993311037,-7.13377926421405)(1136.92307692308,-7.13377926421405)(1141.86622073579,-7.13377926421405)(1146.8093645485,-7.13377926421405)(1151.7525083612,-7.15384615384615)(1154.22408026756,-7.16387959866221)(1156.69565217391,-7.17391304347826)(1161.63879598662,-7.19397993311037)(1166.58193979933,-7.19397993311037)(1171.52508361204,-7.19397993311037)(1176.46822742475,-7.19397993311037)(1181.41137123746,-7.19397993311037)(1186.35451505017,-7.21404682274247)(1188.82608695652,-7.22408026755853)(1191.29765886288,-7.23411371237458)(1196.24080267559,-7.25418060200669)(1201.18394648829,-7.25418060200669)(1206.127090301,-7.25418060200669)(1211.07023411371,-7.25418060200669)(1216.01337792642,-7.25418060200669)(1220.95652173913,-7.2742474916388)(1223.42809364549,-7.28428093645485)(1225.89966555184,-7.2943143812709)(1230.84280936455,-7.31438127090301)(1235.78595317726,-7.31438127090301)(1240.72909698997,-7.31438127090301)(1245.67224080268,-7.31438127090301)(1250.61538461538,-7.31438127090301)(1255.55852842809,-7.33444816053512)(1258.03010033445,-7.34448160535117)(1260.5016722408,-7.35451505016722)(1265.44481605351,-7.37458193979933)(1270.38795986622,-7.37458193979933)(1275.33110367893,-7.37458193979933)(1280.27424749164,-7.37458193979933)(1285.21739130435,-7.39464882943144)(1287.6889632107,-7.40468227424749)(1290.16053511706,-7.41471571906355)(1295.10367892977,-7.43478260869565)(1300.04682274247,-7.43478260869565)(1304.98996655518,-7.45484949832776)(1307.46153846154,-7.46488294314381)(1309.93311036789,-7.47491638795987)(1314.8762541806,-7.49498327759197)(1319.81939799331,-7.51505016722408)(1322.29096989967,-7.52508361204013)(1324.76254180602,-7.53511705685619)(1329.70568561873,-7.5752508361204)(1332.17725752508,-7.58528428093646)(1334.64882943144,-7.59531772575251)(1339.59197324415,-7.61538461538462)(1344.53511705686,-7.63545150501672)(1347.00668896321,-7.64548494983278)(1349.47826086957,-7.65551839464883)(1354.42140468227,-7.69565217391304)(1356.89297658863,-7.7056856187291)(1359.36454849498,-7.71571906354515)(1364.30769230769,-7.75585284280937)(1366.77926421405,-7.76588628762542)(1369.2508361204,-7.77591973244147)(1374.19397993311,-7.81605351170569)(1375.42976588629,-7.82608695652174)(1379.13712374582,-7.8561872909699)(1382.84448160535,-7.88628762541806)(1384.08026755853,-7.89632107023411)(1389.02341137124,-7.93645484949833)(1390.25919732441,-7.94648829431438)(1393.96655518395,-7.97658862876254)(1396.4381270903,-8.0066889632107)(1398.90969899666,-8.03678929765886)(1402.61705685619,-8.06688963210702)(1403.85284280936,-8.07692307692308)(1408.79598662207,-8.11705685618729)(1410.03177257525,-8.12709030100334)(1413.73913043478,-8.1571906354515)(1416.21070234114,-8.18729096989967)(1418.68227424749,-8.21739130434783)(1421.15384615385,-8.24749163879599)(1423.6254180602,-8.27759197324415)(1426.09698996656,-8.30769230769231)(1428.56856187291,-8.33779264214047)(1431.04013377926,-8.36789297658863)(1433.51170568562,-8.39799331103679)(1437.21906354515,-8.42809364548495)(1438.45484949833,-8.438127090301)(1443.39799331104,-8.47826086956522)(1444.63377926421,-8.48829431438127)(1448.34113712375,-8.51839464882943)(1450.8127090301,-8.54849498327759)(1453.28428093645,-8.57859531772575)(1455.75585284281,-8.60869565217391)(1458.22742474916,-8.63879598662207)(1460.69899665552,-8.66889632107023)(1463.17056856187,-8.71404682274247)(1463.99442586399,-8.72909698996656)(1467.28985507246,-8.78929765886288)(1468.11371237458,-8.80434782608696)(1470.58528428094,-8.8494983277592)(1473.05685618729,-8.87959866220736)(1476.76421404682,-8.90969899665552)(1478,-8.91973244147157)(1478.00004576866,-8.96989966555184)(1478.00008238298,-9.03010033444816)(1478.00008238298,-9.09030100334448)(1478.00008238298,-9.1505016722408)(1478.00011899676,-9.21070234113712)(1478.00011899676,-9.27090301003344)(1478.00011899676,-9.33110367892977)(1478.00008238298,-9.39130434782609)(1478.00008238298,-9.45150501672241)(1478.00008238298,-9.51170568561873)(1478.00008238298,-9.57190635451505)(1478.00008238298,-9.63210702341137)(1478.00008238298,-9.69230769230769)(1478.00008238298,-9.75250836120401)(1478.00008238298,-9.81270903010033)(1478.00008238298,-9.87290969899666)(1478.00008238298,-9.93311036789298)(1478.00004576866,-9.9933110367893)(1478.00004576866,-10.0535117056856)(1478.0000091538,-10.1137123745819)(1478,-10.1438127090301)(1475.52842809365,-10.1739130434783)(1473.05685618729,-10.1839464882943)(1468.11371237458,-10.2040133779264)(1463.17056856187,-10.2240802675585)(1460.69899665552,-10.2341137123746)(1458.22742474916,-10.2441471571906)(1453.28428093645,-10.2642140468227)(1448.34113712375,-10.2842809364548)(1445.86956521739,-10.2943143812709)(1443.39799331104,-10.304347826087)(1438.45484949833,-10.3244147157191)(1433.51170568562,-10.3244147157191)(1428.56856187291,-10.3444816053512)(1426.09698996656,-10.3545150501672)(1423.6254180602,-10.3645484949833)(1418.68227424749,-10.3846153846154)(1413.73913043478,-10.3846153846154)(1408.79598662207,-10.4046822742475)(1406.32441471572,-10.4147157190635)(1403.85284280936,-10.4247491638796)(1398.90969899666,-10.4448160535117)(1393.96655518395,-10.4448160535117)(1389.02341137124,-10.4448160535117)(1384.08026755853,-10.4448160535117)(1379.13712374582,-10.4448160535117)(1374.19397993311,-10.4448160535117)(1369.2508361204,-10.4247491638796)(1364.30769230769,-10.4247491638796)(1359.36454849498,-10.4247491638796)(1354.42140468227,-10.4448160535117)(1349.47826086957,-10.4448160535117)(1344.53511705686,-10.4448160535117)(1339.59197324415,-10.4448160535117)(1334.64882943144,-10.4448160535117)(1329.70568561873,-10.4448160535117)(1324.76254180602,-10.4448160535117)(1319.81939799331,-10.4448160535117)(1314.8762541806,-10.4448160535117)(1309.93311036789,-10.4448160535117)(1304.98996655518,-10.4448160535117)(1300.04682274247,-10.4448160535117)(1295.10367892977,-10.4448160535117)(1290.16053511706,-10.4448160535117)(1285.21739130435,-10.4448160535117)(1280.27424749164,-10.4448160535117)(1275.33110367893,-10.4448160535117)(1270.38795986622,-10.4448160535117)(1265.44481605351,-10.4448160535117)(1260.5016722408,-10.4448160535117)(1255.55852842809,-10.4448160535117)(1250.61538461538,-10.4247491638796)(1248.14381270903,-10.4147157190635)(1245.67224080268,-10.4046822742475)(1240.72909698997,-10.3846153846154)(1235.78595317726,-10.3846153846154)(1230.84280936455,-10.3846153846154)(1225.89966555184,-10.3846153846154)(1220.95652173913,-10.3846153846154)(1216.01337792642,-10.3846153846154)(1211.07023411371,-10.3846153846154)(1206.127090301,-10.3846153846154)(1201.18394648829,-10.3846153846154)(1196.24080267559,-10.3846153846154)(1191.29765886288,-10.3846153846154)(1186.35451505017,-10.3846153846154)(1181.41137123746,-10.3846153846154)(1176.46822742475,-10.3846153846154)(1171.52508361204,-10.3846153846154)(1166.58193979933,-10.3846153846154)(1161.63879598662,-10.3846153846154)(1156.69565217391,-10.3846153846154)(1151.7525083612,-10.3846153846154)(1146.8093645485,-10.3846153846154)(1141.86622073579,-10.3645484949833)(1139.39464882943,-10.3545150501672)(1136.92307692308,-10.3444816053512)(1131.97993311037,-10.3244147157191)(1127.03678929766,-10.3244147157191)(1122.09364548495,-10.3244147157191)(1117.15050167224,-10.3244147157191)(1112.20735785953,-10.3244147157191)(1107.26421404682,-10.3244147157191)(1102.32107023411,-10.3244147157191)(1097.3779264214,-10.3244147157191)(1092.4347826087,-10.3244147157191)(1087.49163879599,-10.3244147157191)(1082.54849498328,-10.3244147157191)(1077.60535117057,-10.3444816053512)(1075.13377926421,-10.3545150501672)(1072.66220735786,-10.3645484949833)(1067.71906354515,-10.3846153846154)(1062.77591973244,-10.3846153846154)(1057.83277591973,-10.3645484949833)(1055.36120401338,-10.3545150501672)(1052.88963210702,-10.3444816053512)(1047.94648829431,-10.3244147157191)(1043.00334448161,-10.3244147157191)(1038.0602006689,-10.3244147157191)(1033.11705685619,-10.3244147157191)(1028.17391304348,-10.3244147157191)(1023.23076923077,-10.3244147157191)(1018.28762541806,-10.3244147157191)(1013.34448160535,-10.3244147157191)(1008.40133779264,-10.3244147157191)(1003.45819397993,-10.3244147157191)(998.515050167224,-10.3244147157191)(993.571906354515,-10.3244147157191)(988.628762541806,-10.3244147157191)(983.685618729097,-10.304347826087)(981.214046822742,-10.2943143812709)(978.742474916388,-10.2842809364548)(973.799331103679,-10.2642140468227)(968.85618729097,-10.2642140468227)(963.913043478261,-10.2642140468227)(958.969899665552,-10.2642140468227)(954.026755852843,-10.2642140468227)(949.083612040134,-10.2441471571906)(946.612040133779,-10.2341137123746)(944.140468227425,-10.2240802675585)(939.197324414716,-10.2040133779264)(934.254180602007,-10.2040133779264)(929.311036789298,-10.2040133779264)(924.367892976589,-10.1839464882943)(921.896321070234,-10.1739130434783)(919.42474916388,-10.1638795986622)(914.481605351171,-10.1438127090301)(909.538461538462,-10.1438127090301)(904.595317725752,-10.1438127090301)(899.652173913044,-10.123745819398)(897.180602006689,-10.1137123745819)(894.709030100334,-10.1036789297659)(889.765886287625,-10.0836120401338)(884.822742474916,-10.0836120401338)(879.879598662207,-10.0635451505017)(877.408026755853,-10.0535117056856)(874.936454849498,-10.0434782608696)(869.993311036789,-10.0234113712375)(865.05016722408,-10.0033444816054)(862.578595317726,-9.9933110367893)(860.107023411371,-9.98327759197324)(855.163879598662,-9.96321070234114)(850.220735785953,-9.96321070234114)(845.277591973244,-9.94314381270903)(842.80602006689,-9.93311036789298)(840.334448160535,-9.92307692307692)(835.391304347826,-9.90301003344482)(830.448160535117,-9.88294314381271)(827.976588628763,-9.87290969899666)(825.505016722408,-9.8628762541806)(820.561872909699,-9.82274247491639)(818.090301003345,-9.81270903010033)(815.61872909699,-9.79765886287625)(810.675585284281,-9.76755852842809)(805.732441471572,-9.76254180602007)(803.260869565217,-9.75250836120401)(800.789297658863,-9.74247491638796)(795.846153846154,-9.72240802675585)(790.903010033445,-9.70234113712375)(788.43143812709,-9.69230769230769)(785.959866220736,-9.68227424749164)(781.016722408027,-9.66220735785953)(776.073578595318,-9.64214046822742)(773.602006688963,-9.63210702341137)(771.130434782609,-9.62207357859532)(766.1872909699,-9.60200668896321)(761.244147157191,-9.5819397993311)(758.772575250836,-9.57190635451505)(756.301003344482,-9.561872909699)(751.357859531773,-9.54180602006689)(746.414715719064,-9.52173913043478)(743.943143812709,-9.51170568561873)(741.471571906354,-9.50167224080268)(736.528428093646,-9.48160535117057)(731.585284280936,-9.46153846153846)(729.113712374582,-9.45150501672241)(726.642140468227,-9.44147157190636)(721.698996655518,-9.42140468227425)(716.755852842809,-9.42140468227425)(711.8127090301,-9.40133779264214)(709.341137123746,-9.39130434782609)(706.869565217391,-9.38127090301004)(701.926421404682,-9.36120401337793)(696.983277591973,-9.34113712374582)(694.511705685619,-9.33110367892977)(692.040133779264,-9.32107023411371)(687.096989966555,-9.30100334448161)(682.153846153846,-9.30100334448161)(677.210702341137,-9.2809364548495)(674.739130434783,-9.27090301003344)(672.267558528428,-9.26086956521739)(667.324414715719,-9.24080267558528)(662.38127090301,-9.24080267558528)(657.438127090301,-9.24080267558528)(652.494983277592,-9.24080267558528)(647.551839464883,-9.24080267558528)(642.608695652174,-9.24080267558528)(637.665551839465,-9.22073578595318)(635.19397993311,-9.21070234113712)(632.722408026756,-9.20066889632107)(627.779264214047,-9.18060200668896)(622.836120401338,-9.18060200668896)(617.892976588629,-9.18060200668896)(612.94983277592,-9.18060200668896)(608.006688963211,-9.18060200668896)(603.063545150502,-9.20066889632107)(600.591973244147,-9.21070234113712)(598.120401337793,-9.22073578595318)(593.177257525084,-9.24080267558528)(588.234113712375,-9.24080267558528)(583.290969899666,-9.24080267558528)(578.347826086957,-9.24080267558528)(573.404682274248,-9.24080267558528)(568.461538461538,-9.24080267558528)(563.518394648829,-9.24080267558528)(558.57525083612,-9.24080267558528)(553.632107023411,-9.24080267558528)(548.688963210702,-9.24080267558528)(543.745819397993,-9.24080267558528)(538.802675585284,-9.24080267558528)(533.859531772575,-9.24080267558528)(528.916387959866,-9.24080267558528)(523.973244147157,-9.24080267558528)(519.030100334448,-9.24080267558528)(514.086956521739,-9.24080267558528)(509.14381270903,-9.24080267558528)(504.200668896321,-9.24080267558528)(499.257525083612,-9.24080267558528)(494.314381270903,-9.24080267558528)(489.371237458194,-9.24080267558528)(484.428093645485,-9.24080267558528)(479.484949832776,-9.24080267558528)(474.541806020067,-9.24080267558528)(469.598662207358,-9.24080267558528)(464.655518394649,-9.24080267558528)(459.71237458194,-9.24080267558528)(454.769230769231,-9.24080267558528)(449.826086956522,-9.24080267558528)(444.882943143813,-9.22073578595318)(442.411371237458,-9.21070234113712)(439.939799331104,-9.20066889632107)(434.996655518395,-9.18060200668896)(430.053511705686,-9.18060200668896)(425.110367892977,-9.18060200668896)(420.167224080268,-9.18060200668896)(415.224080267559,-9.16053511705686)(412.752508361204,-9.1505016722408)(410.28093645485,-9.14046822742475)(405.33779264214,-9.12040133779264)(400.394648829431,-9.12040133779264)(395.451505016722,-9.12040133779264)(390.508361204013,-9.10033444816053)(388.036789297659,-9.09030100334448)(385.565217391304,-9.08026755852843)(380.622073578595,-9.06020066889632)(375.678929765886,-9.06020066889632)(370.735785953177,-9.04013377926421)(368.264214046823,-9.03010033444816)(365.792642140468,-9.02006688963211)(360.849498327759,-9)(355.90635451505,-9)(350.963210702341,-8.97993311036789)(348.491638795987,-8.96989966555184)(346.020066889632,-8.95986622073579)(341.076923076923,-8.93979933110368)(336.133779264214,-8.93979933110368)(331.190635451505,-8.91973244147157)(328.71906354515,-8.90969899665552)(326.247491638796,-8.89966555183947)(321.304347826087,-8.87959866220736)(316.361204013378,-8.87959866220736)(311.418060200669,-8.87959866220736)(306.47491638796,-8.87959866220736)(301.531772575251,-8.87959866220736)(296.588628762542,-8.87959866220736)(291.645484949833,-8.87959866220736)(286.702341137124,-8.87959866220736)(281.759197324415,-8.87959866220736)(276.816053511706,-8.87959866220736)(271.872909698997,-8.87959866220736)(266.929765886288,-8.87959866220736)(261.986622073579,-8.89966555183947)(259.515050167224,-8.90969899665552)(257.04347826087,-8.91973244147157)(252.100334448161,-8.93979933110368)(247.157190635452,-8.93979933110368)(242.214046822742,-8.93979933110368)(237.270903010033,-8.93979933110368)(232.327759197324,-8.95986622073579)(229.85618729097,-8.96989966555184)(227.384615384615,-8.97993311036789)(222.441471571906,-9)(217.498327759197,-9)(212.555183946488,-9)(207.612040133779,-9)(202.66889632107,-9)(197.725752508361,-9)(192.782608695652,-9)(187.839464882943,-9.02006688963211)(185.367892976589,-9.03010033444816)(182.896321070234,-9.04013377926421)(177.953177257525,-9.04013377926421)(175.481605351171,-9.03010033444816)(173.010033444816,-9.02006688963211)(168.066889632107,-9)(163.123745819398,-9)(158.180602006689,-9)(153.23745819398,-9)(148.294314381271,-9)(143.351170568562,-9)(138.408026755853,-8.97993311036789)(135.936454849498,-8.96989966555184)(133.464882943144,-8.95986622073579)(128.521739130435,-8.93979933110368)(123.578595317726,-8.91973244147157)(121.107023411371,-8.90969899665552)(118.635451505017,-8.89966555183947)(113.692307692308,-8.85953177257525)(111.220735785953,-8.8494983277592)(108.749163879599,-8.83946488294314)(103.80602006689,-8.79933110367893)(102.570234113712,-8.78929765886288)(98.8628762541806,-8.75919732441472)(96.3913043478261,-8.72909698996656)(93.9197324414716,-8.69899665551839)(91.4481605351171,-8.66889632107023)(88.9765886287625,-8.63879598662207)(86.505016722408,-8.60869565217391)(84.0334448160535,-8.57859531772575)(81.561872909699,-8.54849498327759)(79.0903010033445,-8.51839464882943)(76.61872909699,-8.48829431438127)(74.1471571906355,-8.45819397993311)(71.6755852842809,-8.42809364548495)(69.2040133779264,-8.39799331103679)(66.7324414715719,-8.36789297658863)(64.2608695652174,-8.33779264214047)(61.7892976588629,-8.30769230769231)(59.3177257525084,-8.26254180602007)(58.4938684503902,-8.24749163879599)(55.1984392419175,-8.18729096989967)(54.3745819397993,-8.17224080267559)(51.9030100334448,-8.12709030100334)(49.4314381270903,-8.09698996655519)(46.9598662207358,-8.06688963210702)(44.4882943143813,-8.02173913043478)(43.6644370122631,-8.0066889632107)(40.3690078037904,-7.94648829431438)(39.5451505016722,-7.9314381270903)(37.0735785953177,-7.88628762541806)(34.6020066889632,-7.84113712374582)(33.778149386845,-7.82608695652174)(32.1304347826087,-7.76588628762542)(30.4827201783723,-7.7056856187291)(29.6588628762542,-7.67558528428094)(28.835005574136,-7.64548494983278)(27.1872909698997,-7.58528428093646)(28.835005574136,-7.52508361204013)(29.6588628762542,-7.49498327759197)(30.4827201783723,-7.46488294314381)(32.1304347826087,-7.40468227424749)(33.778149386845,-7.34448160535117)(34.6020066889632,-7.32943143812709)(37.0735785953177,-7.28428093645485)(39.5451505016722,-7.25418060200669)(42.0167224080268,-7.22408026755853)(44.4882943143813,-7.19397993311037)(46.9598662207358,-7.16387959866221)(49.4314381270903,-7.13377926421405)(53.1387959866221,-7.10367892976589)(54.3745819397993,-7.09364548494983)(59.3177257525084,-7.07357859531773)(64.2608695652174,-7.05351170568562)(66.7324414715719,-7.04347826086957)(69.2040133779264,-7.03344481605351)(74.1471571906355,-7.0133779264214)(79.0903010033445,-6.9933110367893)(81.561872909699,-6.98327759197324)(84.0334448160535,-6.97324414715719)(88.9765886287625,-6.95317725752508)(93.9197324414716,-6.95317725752508)(98.8628762541806,-6.95317725752508)(103.80602006689,-6.95317725752508)(108.749163879599,-6.95317725752508)(113.692307692308,-6.95317725752508)(118.635451505017,-6.95317725752508)(123.578595317726,-6.95317725752508)(128.521739130435,-6.95317725752508)(133.464882943144,-6.95317725752508)(138.408026755853,-6.95317725752508)(143.351170568562,-6.95317725752508)(148.294314381271,-6.97324414715719)(150.765886287625,-6.98327759197324)(153.23745819398,-6.9933110367893)(158.180602006689,-7.0133779264214)(163.123745819398,-7.0133779264214)(168.066889632107,-7.0133779264214)(173.010033444816,-7.0133779264214)(177.953177257525,-7.0133779264214)(182.896321070234,-7.03344481605351)(185.367892976589,-7.04347826086957)(187.839464882943,-7.05351170568562)(192.782608695652,-7.07357859531773)(197.725752508361,-7.07357859531773)(202.66889632107,-7.07357859531773)(207.612040133779,-7.07357859531773)(212.555183946488,-7.07357859531773)(217.498327759197,-7.07357859531773)(222.441471571906,-7.07357859531773)(227.384615384615,-7.07357859531773)(232.327759197324,-7.09364548494983)(234.799331103679,-7.10367892976589)(237.270903010033,-7.11371237458194)(242.214046822742,-7.13377926421405)(247.157190635452,-7.13377926421405)(252.100334448161,-7.13377926421405)(257.04347826087,-7.13377926421405)(261.986622073579,-7.13377926421405)(266.929765886288,-7.13377926421405)(271.872909698997,-7.13377926421405)(276.816053511706,-7.13377926421405)(281.759197324415,-7.13377926421405)(286.702341137124,-7.13377926421405)(291.645484949833,-7.13377926421405)(296.588628762542,-7.13377926421405)(301.531772575251,-7.13377926421405)(306.47491638796,-7.13377926421405)(311.418060200669,-7.13377926421405)(316.361204013378,-7.13377926421405)(321.304347826087,-7.13377926421405)(326.247491638796,-7.13377926421405)(331.190635451505,-7.13377926421405)(336.133779264214,-7.13377926421405)(341.076923076923,-7.13377926421405)(346.020066889632,-7.13377926421405)(350.963210702341,-7.11371237458194)(353.434782608696,-7.10367892976589)(355.90635451505,-7.09364548494983)(360.849498327759,-7.07357859531773)(365.792642140468,-7.07357859531773)(370.735785953177,-7.07357859531773)(375.678929765886,-7.07357859531773)(380.622073578595,-7.07357859531773)(385.565217391304,-7.05351170568562)(388.036789297659,-7.04347826086957)(390.508361204013,-7.03344481605351)(395.451505016722,-7.0133779264214)(400.394648829431,-7.0133779264214)(405.33779264214,-7.0133779264214)(410.28093645485,-7.0133779264214)(415.224080267559,-6.9933110367893)(417.695652173913,-6.98327759197324)(420.167224080268,-6.97324414715719)(425.110367892977,-6.95317725752508)(430.053511705686,-6.95317725752508)(434.996655518395,-6.95317725752508)(439.939799331104,-6.93311036789298)(442.411371237458,-6.92307692307692)(444.882943143813,-6.91304347826087)(449.826086956522,-6.89297658862876)(454.769230769231,-6.89297658862876)(459.71237458194,-6.89297658862876)(464.655518394649,-6.87290969899666)(467.127090301003,-6.8628762541806)(469.598662207358,-6.85284280936455)(474.541806020067,-6.83277591973244)(479.484949832776,-6.83277591973244)(484.428093645485,-6.83277591973244)(489.371237458194,-6.83277591973244)(494.314381270903,-6.83277591973244)(499.257525083612,-6.83277591973244)(504.200668896321,-6.83277591973244)(509.14381270903,-6.83277591973244)(514.086956521739,-6.83277591973244)(519.030100334448,-6.83277591973244)(523.973244147157,-6.83277591973244)(528.916387959866,-6.83277591973244)(533.859531772575,-6.83277591973244)(538.802675585284,-6.83277591973244)(543.745819397993,-6.83277591973244)(548.688963210702,-6.83277591973244)(553.632107023411,-6.85284280936455)(556.103678929766,-6.8628762541806)};

\addplot [fill=red!40!yellow,draw=none,forget plot] coordinates{ (1085.02006688963,-7.58528428093646)(1087.49163879599,-7.59030100334448)(1092.4347826087,-7.60033444816053)(1097.3779264214,-7.60033444816053)(1102.32107023411,-7.60033444816053)(1107.26421404682,-7.60033444816053)(1112.20735785953,-7.61036789297659)(1117.15050167224,-7.62040133779264)(1122.09364548495,-7.6304347826087)(1127.03678929766,-7.6304347826087)(1131.97993311037,-7.6304347826087)(1136.92307692308,-7.6304347826087)(1141.86622073579,-7.6304347826087)(1146.8093645485,-7.6304347826087)(1151.7525083612,-7.6304347826087)(1156.69565217391,-7.6304347826087)(1161.63879598662,-7.6304347826087)(1166.58193979933,-7.63946488294314)(1167.81772575251,-7.64548494983278)(1171.52508361204,-7.66053511705686)(1176.46822742475,-7.68060200668896)(1181.41137123746,-7.68060200668896)(1186.35451505017,-7.68060200668896)(1191.29765886288,-7.68060200668896)(1196.24080267559,-7.69966555183947)(1198.71237458194,-7.7056856187291)(1201.18394648829,-7.71070234113712)(1206.127090301,-7.72073578595318)(1211.07023411371,-7.73076923076923)(1216.01337792642,-7.74080267558528)(1220.95652173913,-7.75986622073579)(1223.42809364549,-7.76588628762542)(1225.89966555184,-7.77090301003344)(1230.84280936455,-7.7809364548495)(1235.78595317726,-7.79096989966555)(1240.72909698997,-7.80802675585284)(1244.4364548495,-7.82608695652174)(1245.67224080268,-7.83110367892977)(1250.61538461538,-7.84113712374582)(1255.55852842809,-7.84113712374582)(1260.5016722408,-7.8561872909699)(1265.44481605351,-7.87876254180602)(1266.68060200669,-7.88628762541806)(1270.38795986622,-7.90133779264214)(1275.33110367893,-7.91638795986622)(1280.27424749164,-7.93896321070234)(1281.51003344482,-7.94648829431438)(1285.21739130435,-7.9690635451505)(1290.16053511706,-7.98411371237458)(1293.86789297659,-8.0066889632107)(1295.10367892977,-8.01421404682274)(1300.04682274247,-8.0443143812709)(1303.75418060201,-8.06688963210702)(1304.98996655518,-8.07692307692308)(1309.93311036789,-8.11705685618729)(1312.40468227425,-8.12709030100334)(1314.8762541806,-8.13461538461539)(1319.81939799331,-8.16471571906354)(1323.52675585284,-8.18729096989967)(1324.76254180602,-8.19732441471572)(1329.70568561873,-8.23745819397993)(1330.94147157191,-8.24749163879599)(1334.64882943144,-8.27006688963211)(1338.76811594203,-8.30769230769231)(1339.59197324415,-8.31772575250836)(1344.53511705686,-8.35785953177258)(1345.77090301003,-8.36789297658863)(1349.47826086957,-8.39799331103679)(1351.94983277592,-8.42809364548495)(1354.42140468227,-8.45819397993311)(1356.89297658863,-8.48829431438127)(1359.36454849498,-8.51086956521739)(1362.45401337793,-8.54849498327759)(1364.30769230769,-8.57859531772575)(1366.77926421405,-8.60869565217391)(1369.2508361204,-8.63879598662207)(1371.72240802676,-8.66889632107023)(1374.19397993311,-8.69899665551839)(1376.66555183947,-8.72909698996656)(1379.13712374582,-8.7742474916388)(1379.96098104794,-8.78929765886288)(1383.25641025641,-8.8494983277592)(1384.08026755853,-8.86454849498328)(1386.55183946488,-8.90969899665552)(1389.02341137124,-8.95484949832776)(1389.84726867336,-8.96989966555184)(1393.14269788183,-9.03010033444816)(1393.96655518395,-9.06020066889632)(1394.79041248606,-9.09030100334448)(1396.4381270903,-9.1505016722408)(1396.4381270903,-9.21070234113712)(1396.4381270903,-9.27090301003344)(1396.4381270903,-9.33110367892977)(1396.4381270903,-9.39130434782609)(1396.4381270903,-9.45150501672241)(1394.79041248606,-9.51170568561873)(1393.96655518395,-9.52675585284281)(1391.49498327759,-9.57190635451505)(1389.02341137124,-9.60200668896321)(1386.55183946488,-9.63210702341137)(1384.08026755853,-9.66220735785953)(1381.60869565217,-9.69230769230769)(1379.13712374582,-9.72240802675585)(1375.42976588629,-9.75250836120401)(1374.19397993311,-9.76254180602007)(1369.2508361204,-9.78260869565217)(1364.30769230769,-9.80267558528428)(1361.83612040134,-9.81270903010033)(1359.36454849498,-9.82274247491639)(1354.42140468227,-9.8428093645485)(1349.47826086957,-9.8628762541806)(1347.00668896321,-9.87290969899666)(1344.53511705686,-9.88294314381271)(1339.59197324415,-9.90301003344482)(1334.64882943144,-9.90301003344482)(1329.70568561873,-9.90301003344482)(1324.76254180602,-9.90301003344482)(1319.81939799331,-9.91053511705686)(1314.8762541806,-9.91505016722408)(1309.93311036789,-9.91505016722408)(1304.98996655518,-9.91053511705686)(1300.04682274247,-9.90301003344482)(1295.10367892977,-9.91053511705686)(1290.16053511706,-9.91505016722408)(1285.21739130435,-9.9180602006689)(1280.27424749164,-9.9180602006689)(1275.33110367893,-9.9180602006689)(1270.38795986622,-9.92809364548495)(1267.91638795987,-9.93311036789298)(1265.44481605351,-9.93913043478261)(1260.5016722408,-9.94816053511706)(1255.55852842809,-9.94816053511706)(1250.61538461538,-9.94816053511706)(1245.67224080268,-9.94816053511706)(1240.72909698997,-9.94816053511706)(1235.78595317726,-9.94816053511706)(1230.84280936455,-9.94816053511706)(1225.89966555184,-9.94816053511706)(1220.95652173913,-9.94816053511706)(1216.01337792642,-9.94816053511706)(1211.07023411371,-9.94816053511706)(1206.127090301,-9.94816053511706)(1201.18394648829,-9.93913043478261)(1196.24080267559,-9.93913043478261)(1191.29765886288,-9.93913043478261)(1186.35451505017,-9.94816053511706)(1181.41137123746,-9.94816053511706)(1176.46822742475,-9.94816053511706)(1171.52508361204,-9.94816053511706)(1166.58193979933,-9.94816053511706)(1161.63879598662,-9.94816053511706)(1156.69565217391,-9.94816053511706)(1151.7525083612,-9.95117056856187)(1146.8093645485,-9.95117056856187)(1141.86622073579,-9.95117056856187)(1136.92307692308,-9.94816053511706)(1131.97993311037,-9.94816053511706)(1127.03678929766,-9.94816053511706)(1122.09364548495,-9.94816053511706)(1117.15050167224,-9.94816053511706)(1112.20735785953,-9.94816053511706)(1107.26421404682,-9.93913043478261)(1104.79264214047,-9.93311036789298)(1102.32107023411,-9.92809364548495)(1097.3779264214,-9.9180602006689)(1092.4347826087,-9.9180602006689)(1087.49163879599,-9.90802675585284)(1082.54849498328,-9.89799331103679)(1077.60535117057,-9.88795986622074)(1072.66220735786,-9.88795986622074)(1067.71906354515,-9.88795986622074)(1062.77591973244,-9.88795986622074)(1057.83277591973,-9.88795986622074)(1052.88963210702,-9.87892976588629)(1050.41806020067,-9.87290969899666)(1047.94648829431,-9.86789297658863)(1043.00334448161,-9.85785953177258)(1038.0602006689,-9.85785953177258)(1033.11705685619,-9.85785953177258)(1028.17391304348,-9.85785953177258)(1023.23076923077,-9.85785953177258)(1018.28762541806,-9.84782608695652)(1013.34448160535,-9.83779264214047)(1008.40133779264,-9.81872909698997)(1005.92976588629,-9.81270903010033)(1003.45819397993,-9.80769230769231)(998.515050167224,-9.79765886287625)(993.571906354515,-9.79765886287625)(988.628762541806,-9.79765886287625)(983.685618729097,-9.79765886287625)(978.742474916388,-9.7876254180602)(973.799331103679,-9.77056856187291)(970.091973244147,-9.75250836120401)(968.85618729097,-9.74749163879599)(963.913043478261,-9.73745819397993)(958.969899665552,-9.73745819397993)(954.026755852843,-9.73745819397993)(949.083612040134,-9.72240802675585)(944.140468227425,-9.69983277591973)(942.904682274248,-9.69230769230769)(939.197324414716,-9.67725752508361)(934.254180602007,-9.67725752508361)(929.311036789298,-9.66220735785953)(924.367892976589,-9.63963210702341)(923.132107023411,-9.63210702341137)(919.42474916388,-9.61404682274248)(914.481605351171,-9.60953177257525)(909.538461538462,-9.5819397993311)(907.066889632107,-9.57190635451505)(904.595317725752,-9.56588628762542)(899.652173913044,-9.54933110367893)(894.709030100334,-9.52173913043478)(892.23745819398,-9.51170568561873)(889.765886287625,-9.5056856187291)(884.822742474916,-9.47408026755853)(881.115384615385,-9.45150501672241)(879.879598662207,-9.44548494983278)(874.936454849498,-9.42892976588629)(869.993311036789,-9.40133779264214)(868.757525083612,-9.39130434782609)(865.05016722408,-9.36872909698997)(860.930880713489,-9.33110367892977)(860.107023411371,-9.32508361204013)(855.163879598662,-9.29347826086957)(852.692307692308,-9.27090301003344)(850.220735785953,-9.24832775919732)(846.101449275362,-9.21070234113712)(845.277591973244,-9.20317725752508)(840.334448160535,-9.16053511705686)(839.098662207358,-9.1505016722408)(835.391304347826,-9.10535117056856)(834.567447045708,-9.09030100334448)(831.066053511706,-9.03010033444816)(830.448160535117,-9)(829.624303232999,-8.96989966555184)(826.122909698997,-8.90969899665552)(825.505016722408,-8.87959866220736)(824.68115942029,-8.8494983277592)(823.033444816054,-8.78929765886288)(823.033444816054,-8.72909698996656)(824.68115942029,-8.66889632107023)(825.505016722408,-8.63879598662207)(826.328874024526,-8.60869565217391)(829.624303232999,-8.54849498327759)(830.448160535117,-8.53344481605351)(832.919732441472,-8.48829431438127)(835.391304347826,-8.45819397993311)(837.862876254181,-8.42809364548495)(840.334448160535,-8.39799331103679)(842.80602006689,-8.36789297658863)(845.277591973244,-8.33779264214047)(847.749163879599,-8.30769230769231)(850.220735785953,-8.27759197324415)(853.928093645485,-8.24749163879599)(855.163879598662,-8.23745819397993)(860.107023411371,-8.19481605351171)(861.342809364548,-8.18729096989967)(865.05016722408,-8.1571906354515)(868.757525083612,-8.12709030100334)(869.993311036789,-8.11705685618729)(874.936454849498,-8.07441471571907)(877.408026755853,-8.06688963210702)(879.879598662207,-8.05685618729097)(884.822742474916,-8.01421404682274)(887.294314381271,-8.0066889632107)(889.765886287625,-7.99665551839465)(894.709030100334,-7.95401337792642)(897.180602006689,-7.94648829431438)(899.652173913044,-7.93645484949833)(904.595317725752,-7.8938127090301)(907.066889632107,-7.88628762541806)(909.538461538462,-7.87625418060201)(914.481605351171,-7.84866220735786)(919.42474916388,-7.83210702341137)(920.660535117057,-7.82608695652174)(924.367892976589,-7.80351170568562)(929.311036789298,-7.78394648829432)(934.254180602007,-7.77090301003344)(935.489966555184,-7.76588628762542)(939.197324414716,-7.7433110367893)(944.140468227425,-7.72374581939799)(949.083612040134,-7.72073578595318)(952.790969899665,-7.7056856187291)(954.026755852843,-7.69816053511706)(958.969899665552,-7.67558528428094)(963.913043478261,-7.66053511705686)(968.85618729097,-7.66053511705686)(973.799331103679,-7.66053511705686)(978.742474916388,-7.6505016722408)(981.214046822742,-7.64548494983278)(983.685618729097,-7.63946488294314)(988.628762541806,-7.62040133779264)(993.571906354515,-7.61036789297659)(998.515050167224,-7.60033444816053)(1003.45819397993,-7.60033444816053)(1008.40133779264,-7.60033444816053)(1013.34448160535,-7.60033444816053)(1018.28762541806,-7.60033444816053)(1023.23076923077,-7.60033444816053)(1028.17391304348,-7.59030100334448)(1030.64548494983,-7.58528428093646)(1033.11705685619,-7.57926421404682)(1038.0602006689,-7.57023411371237)(1043.00334448161,-7.57023411371237)(1047.94648829431,-7.57023411371237)(1052.88963210702,-7.57023411371237)(1057.83277591973,-7.57023411371237)(1062.77591973244,-7.57023411371237)(1067.71906354515,-7.57023411371237)(1072.66220735786,-7.57023411371237)(1077.60535117057,-7.57023411371237)(1082.54849498328,-7.57926421404682)(1085.02006688963,-7.58528428093646)};

\addplot [fill=green!60!black,draw=none,forget plot] coordinates{ (1114.67892976589,-7.7056856187291)(1117.15050167224,-7.71170568561873)(1122.09364548495,-7.72073578595318)(1127.03678929766,-7.72073578595318)(1131.97993311037,-7.72073578595318)(1136.92307692308,-7.72073578595318)(1141.86622073579,-7.72073578595318)(1146.8093645485,-7.72073578595318)(1151.7525083612,-7.72073578595318)(1156.69565217391,-7.72073578595318)(1161.63879598662,-7.72073578595318)(1166.58193979933,-7.73076923076923)(1171.52508361204,-7.75083612040134)(1175.23244147157,-7.76588628762542)(1176.46822742475,-7.77190635451505)(1181.41137123746,-7.7809364548495)(1186.35451505017,-7.7809364548495)(1191.29765886288,-7.7809364548495)(1196.24080267559,-7.79096989966555)(1201.18394648829,-7.80100334448161)(1206.127090301,-7.81103678929766)(1211.07023411371,-7.82107023411371)(1213.54180602007,-7.82608695652174)(1216.01337792642,-7.83210702341137)(1220.95652173913,-7.85117056856187)(1225.89966555184,-7.86120401337793)(1230.84280936455,-7.87123745819398)(1235.78595317726,-7.88127090301003)(1237.02173913044,-7.88628762541806)(1240.72909698997,-7.90434782608696)(1245.67224080268,-7.92140468227425)(1250.61538461538,-7.9314381270903)(1255.55852842809,-7.9314381270903)(1259.26588628763,-7.94648829431438)(1260.5016722408,-7.95401337792642)(1265.44481605351,-7.97658862876254)(1270.38795986622,-7.99163879598662)(1272.85953177258,-8.0066889632107)(1275.33110367893,-8.02926421404682)(1280.27424749164,-8.04882943143813)(1282.74581939799,-8.06688963210702)(1285.21739130435,-8.09698996655519)(1290.16053511706,-8.10451505016722)(1292.63210702341,-8.12709030100334)(1295.10367892977,-8.1571906354515)(1300.04682274247,-8.17976588628763)(1300.87068004459,-8.18729096989967)(1304.98996655518,-8.23745819397993)(1309.93311036789,-8.23996655518395)(1311.16889632107,-8.24749163879599)(1314.8762541806,-8.27759197324415)(1318.58361204013,-8.30769230769231)(1319.81939799331,-8.32274247491639)(1323.52675585284,-8.36789297658863)(1324.76254180602,-8.37792642140468)(1329.70568561873,-8.4180602006689)(1330.94147157191,-8.42809364548495)(1334.64882943144,-8.47324414715719)(1335.88461538462,-8.48829431438127)(1339.59197324415,-8.51839464882943)(1342.0635451505,-8.54849498327759)(1344.53511705686,-8.57859531772575)(1348.24247491639,-8.60869565217391)(1349.47826086957,-8.61872909698996)(1353.59754738016,-8.66889632107023)(1354.42140468227,-8.69899665551839)(1355.65719063545,-8.72909698996656)(1359.36454849498,-8.7742474916388)(1360.60033444816,-8.78929765886288)(1364.30769230769,-8.83444816053512)(1365.54347826087,-8.8494983277592)(1368.01505016722,-8.90969899665552)(1368.01505016722,-8.96989966555184)(1369.2508361204,-9)(1370.48662207358,-9.03010033444816)(1371.72240802676,-9.09030100334448)(1374.19397993311,-9.12040133779264)(1375.01783723523,-9.1505016722408)(1374.19397993311,-9.16555183946488)(1370.48662207358,-9.21070234113712)(1369.2508361204,-9.24080267558528)(1368.42697881828,-9.27090301003344)(1365.13154960981,-9.33110367892977)(1364.30769230769,-9.36120401337793)(1363.48383500557,-9.39130434782609)(1359.36454849498,-9.44147157190636)(1358.54069119287,-9.45150501672241)(1355.65719063545,-9.51170568561873)(1354.42140468227,-9.52173913043478)(1350.30211817168,-9.57190635451505)(1349.47826086957,-9.58695652173913)(1344.53511705686,-9.60200668896321)(1342.0635451505,-9.63210702341137)(1339.59197324415,-9.66220735785953)(1334.64882943144,-9.68227424749164)(1333.41304347826,-9.69230769230769)(1329.70568561873,-9.72240802675585)(1324.76254180602,-9.74247491638796)(1323.52675585284,-9.75250836120401)(1319.81939799331,-9.78260869565217)(1316.11204013378,-9.81270903010033)(1314.8762541806,-9.81772575250836)(1309.93311036789,-9.81772575250836)(1307.46153846154,-9.81270903010033)(1304.98996655518,-9.80267558528428)(1300.04682274247,-9.78260869565217)(1295.10367892977,-9.80267558528428)(1292.63210702341,-9.81270903010033)(1290.16053511706,-9.81772575250836)(1285.21739130435,-9.82775919732441)(1280.27424749164,-9.82775919732441)(1275.33110367893,-9.82775919732441)(1270.38795986622,-9.83779264214047)(1265.44481605351,-9.84782608695652)(1260.5016722408,-9.85785953177258)(1255.55852842809,-9.85785953177258)(1250.61538461538,-9.85785953177258)(1245.67224080268,-9.85785953177258)(1240.72909698997,-9.85785953177258)(1235.78595317726,-9.85785953177258)(1230.84280936455,-9.85785953177258)(1225.89966555184,-9.85785953177258)(1220.95652173913,-9.85785953177258)(1216.01337792642,-9.85785953177258)(1211.07023411371,-9.85785953177258)(1206.127090301,-9.85785953177258)(1201.18394648829,-9.84782608695652)(1196.24080267559,-9.84782608695652)(1191.29765886288,-9.84782608695652)(1186.35451505017,-9.85785953177258)(1181.41137123746,-9.85785953177258)(1176.46822742475,-9.85785953177258)(1171.52508361204,-9.85785953177258)(1166.58193979933,-9.85785953177258)(1161.63879598662,-9.85785953177258)(1156.69565217391,-9.85785953177258)(1151.7525083612,-9.85785953177258)(1146.8093645485,-9.85785953177258)(1141.86622073579,-9.85785953177258)(1136.92307692308,-9.85785953177258)(1131.97993311037,-9.85785953177258)(1127.03678929766,-9.85785953177258)(1122.09364548495,-9.85785953177258)(1117.15050167224,-9.85785953177258)(1112.20735785953,-9.85785953177258)(1107.26421404682,-9.84782608695652)(1102.32107023411,-9.83779264214047)(1097.3779264214,-9.82775919732441)(1092.4347826087,-9.82775919732441)(1087.49163879599,-9.81772575250836)(1085.02006688963,-9.81270903010033)(1082.54849498328,-9.8066889632107)(1077.60535117057,-9.79765886287625)(1072.66220735786,-9.79765886287625)(1067.71906354515,-9.79765886287625)(1062.77591973244,-9.79765886287625)(1057.83277591973,-9.79765886287625)(1052.88963210702,-9.7876254180602)(1047.94648829431,-9.77759197324415)(1043.00334448161,-9.76755852842809)(1038.0602006689,-9.76755852842809)(1033.11705685619,-9.76755852842809)(1028.17391304348,-9.76755852842809)(1023.23076923077,-9.76755852842809)(1018.28762541806,-9.75752508361204)(1015.81605351171,-9.75250836120401)(1013.34448160535,-9.74648829431438)(1008.40133779264,-9.72742474916388)(1003.45819397993,-9.71739130434782)(998.515050167224,-9.70735785953177)(993.571906354515,-9.70735785953177)(988.628762541806,-9.70735785953177)(983.685618729097,-9.70735785953177)(978.742474916388,-9.69732441471572)(977.506688963211,-9.69230769230769)(973.799331103679,-9.67424749163879)(968.85618729097,-9.6571906354515)(963.913043478261,-9.64715719063545)(958.969899665552,-9.64715719063545)(954.026755852843,-9.64715719063545)(950.319397993311,-9.63210702341137)(949.083612040134,-9.62458193979933)(944.140468227425,-9.60200668896321)(939.197324414716,-9.58695652173913)(934.254180602007,-9.58695652173913)(930.546822742475,-9.57190635451505)(929.311036789298,-9.56438127090301)(924.367892976589,-9.54180602006689)(919.42474916388,-9.51672240802676)(916.953177257525,-9.51170568561873)(914.481605351171,-9.50167224080268)(909.538461538462,-9.47408026755853)(904.595317725752,-9.45752508361204)(902.123745819398,-9.45150501672241)(899.652173913044,-9.44147157190636)(894.709030100334,-9.41387959866221)(889.765886287625,-9.39732441471572)(888.530100334448,-9.39130434782609)(884.822742474916,-9.36872909698997)(879.879598662207,-9.3371237458194)(877.408026755853,-9.33110367892977)(874.936454849498,-9.32107023411371)(869.993311036789,-9.27842809364548)(868.757525083612,-9.27090301003344)(865.05016722408,-9.24832775919732)(860.107023411371,-9.21672240802675)(859.283166109253,-9.21070234113712)(855.163879598662,-9.17307692307692)(853.310200668896,-9.1505016722408)(850.220735785953,-9.1128762541806)(847.749163879599,-9.09030100334448)(845.277591973244,-9.06020066889632)(842.80602006689,-9.03010033444816)(840.334448160535,-8.98494983277592)(839.716555183946,-8.96989966555184)(836.215161649944,-8.90969899665552)(835.391304347826,-8.87959866220736)(834.773411371237,-8.8494983277592)(832.919732441472,-8.78929765886288)(834.567447045708,-8.72909698996656)(835.391304347826,-8.69899665551839)(836.215161649944,-8.66889632107023)(839.510590858417,-8.60869565217391)(840.334448160535,-8.59364548494983)(842.80602006689,-8.54849498327759)(845.277591973244,-8.51839464882943)(847.749163879599,-8.48829431438127)(850.220735785953,-8.45819397993311)(852.692307692308,-8.42809364548495)(855.163879598662,-8.39799331103679)(857.635451505017,-8.36789297658863)(860.107023411371,-8.33779264214047)(863.814381270903,-8.30769230769231)(865.05016722408,-8.30016722408027)(869.993311036789,-8.25752508361204)(871.229096989967,-8.24749163879599)(874.936454849498,-8.21739130434783)(878.64381270903,-8.18729096989967)(879.879598662207,-8.17976588628763)(884.822742474916,-8.1371237458194)(887.294314381271,-8.12709030100334)(889.765886287625,-8.1195652173913)(894.709030100334,-8.07692307692308)(897.180602006689,-8.06688963210702)(899.652173913044,-8.05936454849499)(904.595317725752,-8.01672240802676)(907.066889632107,-8.0066889632107)(909.538461538462,-7.99916387959866)(914.481605351171,-7.95652173913043)(916.953177257525,-7.94648829431438)(919.42474916388,-7.94046822742475)(924.367892976589,-7.90886287625418)(928.07525083612,-7.88628762541806)(929.311036789298,-7.88127090301003)(934.254180602007,-7.86822742474916)(939.197324414716,-7.84866220735786)(942.904682274248,-7.82608695652174)(944.140468227425,-7.82107023411371)(949.083612040134,-7.81103678929766)(954.026755852843,-7.79598662207358)(958.969899665552,-7.77341137123746)(960.205685618729,-7.76588628762542)(963.913043478261,-7.75083612040134)(968.85618729097,-7.75083612040134)(973.799331103679,-7.75083612040134)(978.742474916388,-7.74080267558528)(983.685618729097,-7.73076923076923)(988.628762541806,-7.71170568561873)(991.100334448161,-7.7056856187291)(993.571906354515,-7.70066889632107)(998.515050167224,-7.69063545150502)(1003.45819397993,-7.69063545150502)(1008.40133779264,-7.69063545150502)(1013.34448160535,-7.69063545150502)(1018.28762541806,-7.69063545150502)(1023.23076923077,-7.69063545150502)(1028.17391304348,-7.68060200668896)(1033.11705685619,-7.67056856187291)(1038.0602006689,-7.66053511705686)(1043.00334448161,-7.66053511705686)(1047.94648829431,-7.66053511705686)(1052.88963210702,-7.66053511705686)(1057.83277591973,-7.66053511705686)(1062.77591973244,-7.66053511705686)(1067.71906354515,-7.67056856187291)(1072.66220735786,-7.67056856187291)(1077.60535117057,-7.67056856187291)(1082.54849498328,-7.67056856187291)(1087.49163879599,-7.68060200668896)(1092.4347826087,-7.69063545150502)(1097.3779264214,-7.69063545150502)(1102.32107023411,-7.69063545150502)(1107.26421404682,-7.69063545150502)(1112.20735785953,-7.70066889632107)(1114.67892976589,-7.7056856187291)};

\addplot [fill=darkgray,draw=none,forget plot] coordinates{ (1478.00006407589,0)(1478,-0.0526755852842809)(1473.67474916388,0)(1478,7.80355919588601e-07)(1478.00006407589,0)};

\addplot [
color=white,
draw=white,
only marks,
mark=x,
mark options={solid},
mark size=2.0pt,
line width=0.3pt,
forget plot
]
coordinates{
 (370.735785953177,-4.75585284280936)(1478,-18)(0,-15.8327759197324)(1478,0)(1478,-9.27090301003344)(0,0)(588.234113712375,-11.1973244147157)(696.983277591973,-18)(1156.69565217391,-4.63545150501672)(0,-9.03010033444816)(741.471571906354,0)(1250.61538461538,-13.7257525083612)(830.448160535117,-7.82608695652174)(0,-12.4615384615385)(608.006688963211,-14.628762541806)(0,-3.67224080267559)(1478,-5.95986622073579)(207.612040133779,-18)(1057.83277591973,-10.8361204013378)(1478,-2.82943143812709)(751.357859531773,-3.01003344481605)(1077.60535117057,-16.6153846153846)(385.565217391304,-7.88628762541806)(355.90635451505,-1.38461538461538)(0,-6.38127090301003)(1478,-12.0401337792642)(1112.20735785953,-1.02341137123746)(1478,-15.4113712374582)(286.702341137124,-13.7859531772575)(1186.35451505017,-7.64548494983278)(751.357859531773,-5.53846153846154)(884.822742474916,-13.1839464882943)(252.100334448161,-10.7157190635452)(652.494983277592,-9.39130434782609)(1211.07023411371,-9.51170568561873)(528.916387959866,-6.50167224080268)(0,-10.6555183946488)(1478,-7.64548494983278)(1018.28762541806,-6.14046822742475)(232.327759197324,-6.32107023411371)(1240.72909698997,-11.7391304347826)(914.481605351171,-9.51170568561873)(247.157190635452,-9.09030100334448)(825.505016722408,-10.9565217391304)(1478,-10.6555183946488)(1270.38795986622,-6.02006688963211)(0,-7.7056856187291)(444.882943143813,-9.81270903010033)(612.94983277592,-7.88628762541806)(1028.17391304348,-8.30769230769231)(177.953177257525,-7.7056856187291)(790.903010033445,-6.68227424749164)(1329.70568561873,-8.42809364548495)(1304.98996655518,-10.5351170568562)(385.565217391304,-6.44147157190635)(1344.53511705686,-7.16387959866221)(1003.45819397993,-7.16387959866221)(484.428093645485,-8.72909698996656)(133.464882943144,-9.45150501672241)(741.471571906354,-10.0535117056856)(1067.71906354515,-9.63210702341137)(781.016722408027,-8.72909698996656)(647.551839464883,-6.62207357859532)(1478,-8.48829431438127)(1146.8093645485,-6.8628762541806)(118.635451505017,-6.80267558528428)(1359.36454849498,-9.63210702341137)(1166.58193979933,-8.60869565217391)(123.578595317726,-8.42809364548495)(1176.46822742475,-10.5953177257525)(341.076923076923,-9.51170568561873)(954.026755852843,-10.3545150501672)(484.428093645485,-7.40468227424749)(291.645484949833,-7.28428093645485)(1478,-10.0535117056856)(924.367892976589,-8.60869565217391)(711.8127090301,-7.58528428093646)(558.57525083612,-9.39130434782609)(914.481605351171,-7.16387959866221)(311.418060200669,-8.42809364548495)(14.8294314381271,-7.34448160535117)(647.551839464883,-8.54849498327759)(1319.81939799331,-7.76588628762542)(815.61872909699,-9.69230769230769)(1314.8762541806,-9.03010033444816)(1364.30769230769,-10.4147157190635)(1057.83277591973,-9.03010033444816)(1072.66220735786,-7.7056856187291)(49.4314381270903,-8.24749163879599)(499.257525083612,-8.18729096989967)(1478,-8.8494983277592)(395.451505016722,-6.98327759197324)(1146.8093645485,-10.0535117056856)(603.063545150502,-7.16387959866221)(182.896321070234,-6.98327759197324)(1216.01337792642,-7.22408026755853)(395.451505016722,-8.96989966555184)(909.538461538462,-7.94648829431438)(227.384615384615,-8.24749163879599)(919.42474916388,-10.1739130434783) 
};

%\node at (axis cs:50, -2.5) [shape=circle,fill=white,draw=black,inner sep=0pt,anchor=south west] {\scriptsize\color{locol}$\*L_{\*t}$};
%\node at (axis cs:460, -5.9) [shape=circle,fill=white,draw=black,inner sep=0pt,anchor=south west] {\scriptsize\color{orange!50!yellow}$\*H_{\*t}$};
%\node at (axis cs:160, -5.2) [shape=circle,fill=white,draw=black,inner sep=0pt,anchor=south west] {\scriptsize\color{darkgray}$\*U_{\*t}$};

\node at (axis cs:805, -17) [shape=circle,fill=green!60!black,draw=black,inner sep=0.2pt,anchor=south west,minimum size=16pt]
  {\scriptsize\color{white}$\*M_{\*t}$};
\node at (axis cs:980, -17) [shape=circle,fill=red!40!yellow,draw=black,inner sep=0.2pt,anchor=south west,minimum size=16pt]
  {\scriptsize\color{white}$\*H_{\*t}$};
\node at (axis cs:1155, -17) [shape=circle,fill=locol,draw=black,inner sep=0.2pt,anchor=south west,minimum size=16pt]
  {\scriptsize\color{white}$\*L_{\*t}$};
\node at (axis cs:1330, -17) [shape=circle,fill=darkgray,draw=black,inner sep=0.2pt,anchor=south west,minimum size=16pt]
  {\scriptsize\color{white}$\*U_{\*t}$};

\end{axis}
\end{tikzpicture}%

%% This file was created by matlab2tikz v0.2.3.
% Copyright (c) 2008--2012, Nico Schlömer <nico.schloemer@gmail.com>
% All rights reserved.
% 
% 
%

\definecolor{locol}{rgb}{0.26, 0.45, 0.65}

\begin{tikzpicture}

\begin{axis}[%
tick label style={font=\tiny},
label style={font=\tiny},
xlabel shift={-10pt},
ylabel shift={-17pt},
legend style={font=\tiny},
view={0}{90},
width=\figurewidth,
height=\figureheight,
scale only axis,
xmin=0, xmax=1478,
xtick={0, 400, 1000, 1400},
xlabel={Length (m)},
ymin=-18, ymax=0,
ytick={0, -4, -14, -18},
ylabel={Depth (m)},
name=plot1,
axis lines*=box,
tickwidth=0.1cm,
clip=false
]

\addplot [fill=locol,draw=none,forget plot] coordinates{ (1478.00014645918,0)(1478.00007323068,-0.0602006688963211)(1478,-0.120401337792642)(1478,-0.180602006688963)(1478,-0.240802675585284)(1478,-0.301003344481605)(1478,-0.361204013377926)(1478,-0.421404682274247)(1478,-0.481605351170569)(1478,-0.54180602006689)(1478,-0.602006688963211)(1478,-0.662207357859532)(1478,-0.722408026755853)(1478,-0.782608695652174)(1478,-0.842809364548495)(1478,-0.903010033444816)(1478,-0.963210702341137)(1478,-1.02341137123746)(1478,-1.08361204013378)(1478,-1.1438127090301)(1478,-1.20401337792642)(1478,-1.26421404682274)(1478,-1.32441471571906)(1478,-1.38461538461538)(1478,-1.44481605351171)(1478,-1.50501672240803)(1478,-1.56521739130435)(1478,-1.62541806020067)(1478,-1.68561872909699)(1478,-1.74581939799331)(1478,-1.80602006688963)(1478,-1.86622073578595)(1478,-1.92642140468227)(1478,-1.9866220735786)(1478,-2.04682274247492)(1478,-2.10702341137124)(1478,-2.16722408026756)(1478,-2.22742474916388)(1478,-2.2876254180602)(1478,-2.34782608695652)(1478,-2.40802675585284)(1478,-2.46822742474916)(1478,-2.52842809364549)(1478,-2.58862876254181)(1478,-2.64882943143813)(1478,-2.70903010033445)(1478,-2.76923076923077)(1478,-2.82943143812709)(1478,-2.88963210702341)(1478,-2.94983277591973)(1478,-3.01003344481605)(1478,-3.07023411371237)(1478,-3.1304347826087)(1478,-3.19063545150502)(1478,-3.25083612040134)(1478,-3.31103678929766)(1478,-3.37123745819398)(1478,-3.4314381270903)(1478,-3.49163879598662)(1478,-3.55183946488294)(1478,-3.61204013377926)(1478,-3.67224080267559)(1478,-3.73244147157191)(1478,-3.79264214046823)(1478,-3.85284280936455)(1478,-3.91304347826087)(1478,-3.97324414715719)(1478,-4.03344481605351)(1478,-4.09364548494983)(1478,-4.15384615384615)(1478,-4.21404682274247)(1478,-4.2742474916388)(1478,-4.33444816053512)(1478,-4.39464882943144)(1478,-4.45484949832776)(1478,-4.51505016722408)(1478,-4.5752508361204)(1478,-4.63545150501672)(1478,-4.69565217391304)(1478,-4.75585284280936)(1478,-4.81605351170569)(1478,-4.87625418060201)(1478,-4.93645484949833)(1478,-4.99665551839465)(1478,-5.05685618729097)(1478,-5.11705685618729)(1478,-5.17725752508361)(1478,-5.23745819397993)(1478,-5.29765886287625)(1478,-5.35785953177258)(1478,-5.4180602006689)(1478,-5.47826086956522)(1478,-5.53846153846154)(1478,-5.59866220735786)(1478,-5.65886287625418)(1478,-5.7190635451505)(1478,-5.77926421404682)(1478,-5.83946488294314)(1478,-5.89966555183946)(1478,-5.95986622073579)(1478,-6.02006688963211)(1478,-6.08026755852843)(1478,-6.14046822742475)(1478,-6.20066889632107)(1478,-6.26086956521739)(1478,-6.32107023411371)(1478,-6.38127090301003)(1478,-6.44147157190635)(1478,-6.50167224080268)(1478,-6.561872909699)(1478,-6.62207357859532)(1478,-6.68227424749164)(1478,-6.74247491638796)(1478,-6.80267558528428)(1478,-6.8628762541806)(1478,-6.92307692307692)(1478,-6.98327759197324)(1478,-7.04347826086957)(1478,-7.10367892976589)(1478,-7.16387959866221)(1478,-7.22408026755853)(1478,-7.28428093645485)(1478,-7.34448160535117)(1478,-7.40468227424749)(1478,-7.46488294314381)(1478,-7.52508361204013)(1478,-7.58528428093646)(1478,-7.64548494983278)(1478,-7.7056856187291)(1478,-7.76588628762542)(1478,-7.82608695652174)(1478,-7.88628762541806)(1478,-7.94648829431438)(1478,-8.0066889632107)(1478,-8.06688963210702)(1478,-8.12709030100334)(1478,-8.18729096989967)(1478,-8.24749163879599)(1478,-8.30769230769231)(1478,-8.36789297658863)(1478,-8.42809364548495)(1478,-8.48829431438127)(1478,-8.54849498327759)(1478,-8.60869565217391)(1478,-8.66889632107023)(1478,-8.72909698996656)(1478,-8.78929765886288)(1478,-8.8494983277592)(1478,-8.90969899665552)(1478,-8.96989966555184)(1478,-9.03010033444816)(1478,-9.09030100334448)(1478,-9.1505016722408)(1478.00007323068,-9.21070234113712)(1478.00007323068,-9.27090301003344)(1478.00007323068,-9.33110367892977)(1478,-9.39130434782609)(1478,-9.45150501672241)(1478,-9.51170568561873)(1478,-9.57190635451505)(1478,-9.63210702341137)(1478,-9.69230769230769)(1478,-9.75250836120401)(1478,-9.81270903010033)(1478,-9.87290969899666)(1478,-9.93311036789298)(1478,-9.9933110367893)(1478,-10.0535117056856)(1478,-10.1137123745819)(1478,-10.1739130434783)(1478,-10.2341137123746)(1478,-10.2943143812709)(1478,-10.3545150501672)(1478,-10.4147157190635)(1478,-10.4749163879599)(1478,-10.5351170568562)(1478,-10.5953177257525)(1478,-10.6555183946488)(1478,-10.7157190635452)(1478,-10.7759197324415)(1478,-10.8361204013378)(1478,-10.8963210702341)(1478,-10.9565217391304)(1478,-11.0167224080268)(1478,-11.0769230769231)(1478,-11.1371237458194)(1478,-11.1973244147157)(1478,-11.257525083612)(1478,-11.3177257525084)(1478,-11.3779264214047)(1478,-11.438127090301)(1478,-11.4983277591973)(1478,-11.5585284280936)(1478,-11.61872909699)(1478,-11.6789297658863)(1478,-11.7391304347826)(1478,-11.7993311036789)(1478,-11.8595317725753)(1478,-11.9197324414716)(1478,-11.9799331103679)(1478,-12.0401337792642)(1478,-12.1003344481605)(1478,-12.1605351170569)(1478,-12.2207357859532)(1478,-12.2809364548495)(1478,-12.3411371237458)(1478,-12.4013377926421)(1478,-12.4615384615385)(1478,-12.5217391304348)(1478,-12.5819397993311)(1478,-12.6421404682274)(1478,-12.7023411371237)(1478,-12.7625418060201)(1478,-12.8227424749164)(1478,-12.8829431438127)(1478,-12.943143812709)(1478,-13.0033444816054)(1478,-13.0635451505017)(1478,-13.123745819398)(1478,-13.1839464882943)(1478,-13.2441471571906)(1478,-13.304347826087)(1478,-13.3645484949833)(1478,-13.4247491638796)(1478,-13.4849498327759)(1478,-13.5451505016722)(1478,-13.6053511705686)(1478,-13.6655518394649)(1478,-13.7257525083612)(1478,-13.7859531772575)(1478,-13.8461538461538)(1478,-13.9063545150502)(1478,-13.9665551839465)(1478,-14.0267558528428)(1478,-14.0869565217391)(1478,-14.1471571906355)(1478,-14.2073578595318)(1478,-14.2675585284281)(1478,-14.3277591973244)(1478,-14.3879598662207)(1478,-14.4481605351171)(1478,-14.5083612040134)(1478,-14.5685618729097)(1478,-14.628762541806)(1478,-14.6889632107023)(1478,-14.7491638795987)(1478,-14.809364548495)(1478,-14.8695652173913)(1478,-14.9297658862876)(1478,-14.9899665551839)(1478,-15.0501672240803)(1478,-15.1103678929766)(1478,-15.1705685618729)(1478,-15.2307692307692)(1478,-15.2909698996656)(1478,-15.3511705685619)(1478,-15.4113712374582)(1478,-15.4715719063545)(1478,-15.5317725752508)(1478,-15.5919732441472)(1478,-15.6521739130435)(1478,-15.7123745819398)(1478,-15.7725752508361)(1478,-15.8327759197324)(1478,-15.8929765886288)(1478,-15.9531772575251)(1478,-16.0133779264214)(1478,-16.0735785953177)(1478,-16.133779264214)(1478,-16.1939799331104)(1478,-16.2541806020067)(1478,-16.314381270903)(1478,-16.3745819397993)(1478,-16.4347826086957)(1478,-16.494983277592)(1478,-16.5551839464883)(1478,-16.6153846153846)(1478,-16.6755852842809)(1478,-16.7357859531773)(1478,-16.7959866220736)(1478,-16.8561872909699)(1478,-16.9163879598662)(1478,-16.9765886287625)(1478,-17.0367892976589)(1478,-17.0969899665552)(1478,-17.1571906354515)(1478,-17.2173913043478)(1478,-17.2775919732441)(1478,-17.3377926421405)(1478,-17.3979933110368)(1478,-17.4581939799331)(1478,-17.5183946488294)(1478,-17.5785953177258)(1478,-17.6387959866221)(1478,-17.6989966555184)(1478,-17.7591973244147)(1478,-17.819397993311)(1478,-17.8795986622074)(1478,-17.9397993311037)(1478,-18)(1473.05685618729,-18)(1468.11371237458,-18)(1463.17056856187,-18)(1458.22742474916,-18)(1453.28428093645,-18)(1448.34113712375,-18)(1443.39799331104,-18)(1438.45484949833,-18)(1433.51170568562,-18)(1428.56856187291,-18)(1423.6254180602,-18)(1418.68227424749,-18)(1413.73913043478,-18)(1408.79598662207,-18)(1403.85284280936,-18)(1398.90969899666,-18)(1393.96655518395,-18)(1389.02341137124,-18)(1384.08026755853,-18)(1379.13712374582,-18)(1374.19397993311,-18)(1369.2508361204,-18)(1364.30769230769,-18)(1359.36454849498,-18)(1354.42140468227,-18)(1349.47826086957,-18)(1344.53511705686,-18)(1339.59197324415,-18)(1334.64882943144,-18)(1329.70568561873,-18)(1324.76254180602,-18)(1319.81939799331,-18)(1314.8762541806,-18)(1309.93311036789,-18)(1304.98996655518,-18)(1300.04682274247,-18)(1295.10367892977,-18)(1290.16053511706,-18)(1285.21739130435,-18)(1280.27424749164,-18)(1275.33110367893,-18)(1270.38795986622,-18)(1265.44481605351,-18)(1260.5016722408,-18)(1255.55852842809,-18)(1250.61538461538,-18)(1245.67224080268,-18)(1240.72909698997,-18)(1235.78595317726,-18)(1230.84280936455,-18)(1225.89966555184,-18)(1220.95652173913,-18)(1216.01337792642,-18)(1211.07023411371,-18)(1206.127090301,-18)(1201.18394648829,-18)(1196.24080267559,-18)(1191.29765886288,-18)(1186.35451505017,-18)(1181.41137123746,-18)(1176.46822742475,-18)(1171.52508361204,-18)(1166.58193979933,-18)(1161.63879598662,-18)(1156.69565217391,-18)(1151.7525083612,-18)(1146.8093645485,-18)(1141.86622073579,-18)(1136.92307692308,-18)(1131.97993311037,-18)(1127.03678929766,-18)(1122.09364548495,-18)(1117.15050167224,-18)(1112.20735785953,-18)(1107.26421404682,-18)(1102.32107023411,-18)(1097.3779264214,-18)(1092.4347826087,-18)(1087.49163879599,-18)(1082.54849498328,-18)(1077.60535117057,-18)(1072.66220735786,-18)(1067.71906354515,-18)(1062.77591973244,-18)(1057.83277591973,-18)(1052.88963210702,-18)(1047.94648829431,-18)(1043.00334448161,-18)(1038.0602006689,-18)(1033.11705685619,-18)(1028.17391304348,-18)(1023.23076923077,-18)(1018.28762541806,-18)(1013.34448160535,-18)(1008.40133779264,-18)(1003.45819397993,-18)(998.515050167224,-18)(993.571906354515,-18)(988.628762541806,-18)(983.685618729097,-18)(978.742474916388,-18)(973.799331103679,-18)(968.85618729097,-18)(963.913043478261,-18)(958.969899665552,-18)(954.026755852843,-18)(949.083612040134,-18)(944.140468227425,-18)(939.197324414716,-18)(934.254180602007,-18)(929.311036789298,-18)(924.367892976589,-18)(919.42474916388,-18)(914.481605351171,-18)(909.538461538462,-18)(904.595317725752,-18)(899.652173913044,-18)(894.709030100334,-18)(889.765886287625,-18)(884.822742474916,-18)(879.879598662207,-18)(874.936454849498,-18)(869.993311036789,-18)(865.05016722408,-18)(860.107023411371,-18)(855.163879598662,-18)(850.220735785953,-18)(845.277591973244,-18)(840.334448160535,-18)(835.391304347826,-18)(830.448160535117,-18)(825.505016722408,-18)(820.561872909699,-18)(815.61872909699,-18)(810.675585284281,-18)(805.732441471572,-18)(800.789297658863,-18)(795.846153846154,-18)(790.903010033445,-18)(785.959866220736,-18)(781.016722408027,-18)(776.073578595318,-18)(771.130434782609,-18)(766.1872909699,-18)(761.244147157191,-18)(756.301003344482,-18)(751.357859531773,-18)(746.414715719064,-18)(741.471571906354,-18)(736.528428093646,-18)(731.585284280936,-18)(726.642140468227,-18)(721.698996655518,-18)(716.755852842809,-18)(711.8127090301,-18)(706.869565217391,-18)(701.926421404682,-18)(696.983277591973,-18)(692.040133779264,-18)(687.096989966555,-18)(682.153846153846,-18)(677.210702341137,-18)(672.267558528428,-18)(667.324414715719,-18)(662.38127090301,-18)(657.438127090301,-18)(652.494983277592,-18)(647.551839464883,-18)(642.608695652174,-18)(637.665551839465,-18)(632.722408026756,-18)(627.779264214047,-18)(622.836120401338,-18)(617.892976588629,-18)(612.94983277592,-18)(608.006688963211,-18)(603.063545150502,-18)(598.120401337793,-18)(593.177257525084,-18)(588.234113712375,-18)(583.290969899666,-18)(578.347826086957,-18)(573.404682274248,-18)(568.461538461538,-18)(563.518394648829,-18)(558.57525083612,-18)(553.632107023411,-18)(548.688963210702,-18)(543.745819397993,-18)(538.802675585284,-18)(533.859531772575,-18)(528.916387959866,-18)(523.973244147157,-18)(519.030100334448,-18)(514.086956521739,-18)(509.14381270903,-18)(504.200668896321,-18)(499.257525083612,-18)(494.314381270903,-18)(489.371237458194,-18)(484.428093645485,-18)(479.484949832776,-18)(474.541806020067,-18)(469.598662207358,-18)(464.655518394649,-18)(459.71237458194,-18)(454.769230769231,-18)(449.826086956522,-18)(444.882943143813,-18)(439.939799331104,-18)(434.996655518395,-18)(430.053511705686,-18)(425.110367892977,-18)(420.167224080268,-18)(415.224080267559,-18)(410.28093645485,-18)(405.33779264214,-18)(400.394648829431,-18)(395.451505016722,-18)(390.508361204013,-18)(385.565217391304,-18)(380.622073578595,-18)(375.678929765886,-18)(370.735785953177,-18)(365.792642140468,-18)(360.849498327759,-18)(355.90635451505,-18)(350.963210702341,-18)(346.020066889632,-18)(341.076923076923,-18)(336.133779264214,-18)(331.190635451505,-18)(326.247491638796,-18)(321.304347826087,-18)(316.361204013378,-18)(311.418060200669,-18)(306.47491638796,-18)(301.531772575251,-18)(296.588628762542,-18)(291.645484949833,-18)(286.702341137124,-18)(281.759197324415,-18)(276.816053511706,-18)(271.872909698997,-18)(266.929765886288,-18)(261.986622073579,-18)(257.04347826087,-18)(252.100334448161,-18)(247.157190635452,-18)(242.214046822742,-18)(237.270903010033,-18)(232.327759197324,-18)(227.384615384615,-18)(222.441471571906,-18)(217.498327759197,-18)(212.555183946488,-18)(207.612040133779,-18)(202.66889632107,-18)(197.725752508361,-18)(192.782608695652,-18)(187.839464882943,-18)(182.896321070234,-18)(177.953177257525,-18)(173.010033444816,-18)(168.066889632107,-18)(163.123745819398,-18)(158.180602006689,-18)(153.23745819398,-18)(148.294314381271,-18)(143.351170568562,-18)(138.408026755853,-18)(133.464882943144,-18)(128.521739130435,-18)(123.578595317726,-18)(118.635451505017,-18)(113.692307692308,-18)(108.749163879599,-18)(103.80602006689,-18)(98.8628762541806,-18)(93.9197324414716,-18)(88.9765886287625,-18)(84.0334448160535,-18)(79.0903010033445,-18)(74.1471571906355,-18)(69.2040133779264,-18)(64.2608695652174,-18)(59.3177257525084,-18)(54.3745819397993,-18)(49.4314381270903,-18)(44.4882943143813,-18)(39.5451505016722,-18)(34.6020066889632,-18)(29.6588628762542,-18)(24.7157190635452,-18)(19.7725752508361,-18)(14.8294314381271,-18)(9.88628762541806,-18)(4.94314381270903,-18)(0,-18)(0,-17.9397993311037)(0,-17.8795986622074)(0,-17.819397993311)(0,-17.7591973244147)(0,-17.6989966555184)(0,-17.6387959866221)(0,-17.5785953177258)(0,-17.5183946488294)(0,-17.4581939799331)(0,-17.3979933110368)(0,-17.3377926421405)(0,-17.2775919732441)(0,-17.2173913043478)(0,-17.1571906354515)(0,-17.0969899665552)(0,-17.0367892976589)(0,-16.9765886287625)(0,-16.9163879598662)(0,-16.8561872909699)(0,-16.7959866220736)(0,-16.7357859531773)(0,-16.6755852842809)(0,-16.6153846153846)(0,-16.5551839464883)(0,-16.494983277592)(0,-16.4347826086957)(0,-16.3745819397993)(0,-16.314381270903)(0,-16.2541806020067)(0,-16.1939799331104)(0,-16.133779264214)(0,-16.0735785953177)(0,-16.0133779264214)(0,-15.9531772575251)(0,-15.8929765886288)(0,-15.8327759197324)(0,-15.7725752508361)(0,-15.7123745819398)(0,-15.6521739130435)(0,-15.5919732441472)(0,-15.5317725752508)(0,-15.4715719063545)(0,-15.4113712374582)(0,-15.3511705685619)(0,-15.2909698996656)(0,-15.2307692307692)(0,-15.1705685618729)(0,-15.1103678929766)(0,-15.0501672240803)(0,-14.9899665551839)(0,-14.9297658862876)(0,-14.8695652173913)(0,-14.809364548495)(0,-14.7491638795987)(0,-14.6889632107023)(0,-14.628762541806)(0,-14.5685618729097)(0,-14.5083612040134)(0,-14.4481605351171)(0,-14.3879598662207)(0,-14.3277591973244)(0,-14.2675585284281)(0,-14.2073578595318)(0,-14.1471571906355)(0,-14.0869565217391)(0,-14.0267558528428)(0,-13.9665551839465)(0,-13.9063545150502)(0,-13.8461538461538)(0,-13.7859531772575)(0,-13.7257525083612)(0,-13.6655518394649)(0,-13.6053511705686)(0,-13.5451505016722)(0,-13.4849498327759)(0,-13.4247491638796)(0,-13.3645484949833)(0,-13.304347826087)(0,-13.2441471571906)(0,-13.1839464882943)(0,-13.123745819398)(0,-13.0635451505017)(0,-13.0033444816054)(0,-12.943143812709)(0,-12.8829431438127)(0,-12.8227424749164)(0,-12.7625418060201)(0,-12.7023411371237)(0,-12.6421404682274)(0,-12.5819397993311)(0,-12.5217391304348)(0,-12.4615384615385)(0,-12.4013377926421)(0,-12.3411371237458)(0,-12.2809364548495)(0,-12.2207357859532)(0,-12.1605351170569)(0,-12.1003344481605)(0,-12.0401337792642)(0,-11.9799331103679)(0,-11.9197324414716)(0,-11.8595317725753)(0,-11.7993311036789)(0,-11.7391304347826)(0,-11.6789297658863)(0,-11.61872909699)(0,-11.5585284280936)(0,-11.4983277591973)(0,-11.438127090301)(0,-11.3779264214047)(0,-11.3177257525084)(0,-11.257525083612)(0,-11.1973244147157)(0,-11.1371237458194)(0,-11.0769230769231)(0,-11.0167224080268)(0,-10.9565217391304)(0,-10.8963210702341)(0,-10.8361204013378)(0,-10.7759197324415)(0,-10.7157190635452)(0,-10.6555183946488)(0,-10.5953177257525)(0,-10.5351170568562)(0,-10.4749163879599)(0,-10.4147157190635)(0,-10.3545150501672)(0,-10.2943143812709)(0,-10.2341137123746)(0,-10.1739130434783)(0,-10.1137123745819)(0,-10.0535117056856)(0,-9.9933110367893)(0,-9.93311036789298)(0,-9.87290969899666)(0,-9.81270903010033)(0,-9.75250836120401)(0,-9.69230769230769)(0,-9.63210702341137)(0,-9.57190635451505)(0,-9.51170568561873)(0,-9.45150501672241)(0,-9.39130434782609)(0,-9.33110367892977)(0,-9.27090301003344)(0,-9.21070234113712)(0,-9.1505016722408)(0,-9.09030100334448)(0,-9.03010033444816)(0,-8.96989966555184)(0,-8.90969899665552)(0,-8.8494983277592)(0,-8.78929765886288)(0,-8.72909698996656)(0,-8.66889632107023)(0,-8.60869565217391)(0,-8.54849498327759)(0,-8.48829431438127)(0,-8.42809364548495)(0,-8.36789297658863)(0,-8.30769230769231)(0,-8.24749163879599)(0,-8.18729096989967)(0,-8.12709030100334)(0,-8.06688963210702)(0,-8.0066889632107)(0,-7.94648829431438)(0,-7.88628762541806)(0,-7.82608695652174)(0,-7.76588628762542)(0,-7.7056856187291)(0,-7.64548494983278)(0,-7.58528428093646)(0,-7.52508361204013)(0,-7.46488294314381)(0,-7.40468227424749)(0,-7.34448160535117)(0,-7.28428093645485)(0,-7.22408026755853)(0,-7.16387959866221)(0,-7.10367892976589)(0,-7.04347826086957)(0,-6.98327759197324)(0,-6.92307692307692)(0,-6.8628762541806)(0,-6.80267558528428)(0,-6.74247491638796)(0,-6.68227424749164)(0,-6.62207357859532)(0,-6.561872909699)(0,-6.50167224080268)(-7.32306752895604e-05,-6.44147157190635)(-7.32306752895604e-05,-6.38127090301003)(-7.32306752895604e-05,-6.32107023411371)(0,-6.26086956521739)(0,-6.20066889632107)(0,-6.14046822742475)(0,-6.08026755852843)(0,-6.02006688963211)(0,-5.95986622073579)(0,-5.89966555183946)(0,-5.83946488294314)(0,-5.77926421404682)(0,-5.7190635451505)(0,-5.65886287625418)(0,-5.59866220735786)(0,-5.53846153846154)(0,-5.47826086956522)(0,-5.4180602006689)(0,-5.35785953177258)(0,-5.29765886287625)(0,-5.23745819397993)(0,-5.17725752508361)(0,-5.11705685618729)(0,-5.05685618729097)(0,-4.99665551839465)(0,-4.93645484949833)(0,-4.87625418060201)(0,-4.81605351170569)(0,-4.75585284280936)(0,-4.69565217391304)(0,-4.63545150501672)(0,-4.5752508361204)(0,-4.51505016722408)(0,-4.45484949832776)(0,-4.39464882943144)(0,-4.33444816053512)(0,-4.2742474916388)(0,-4.21404682274247)(0,-4.15384615384615)(0,-4.09364548494983)(0,-4.03344481605351)(0,-3.97324414715719)(0,-3.91304347826087)(0,-3.85284280936455)(0,-3.79264214046823)(0,-3.73244147157191)(0,-3.67224080267559)(0,-3.61204013377926)(0,-3.55183946488294)(0,-3.49163879598662)(0,-3.4314381270903)(0,-3.37123745819398)(0,-3.31103678929766)(0,-3.25083612040134)(0,-3.19063545150502)(0,-3.1304347826087)(0,-3.07023411371237)(0,-3.01003344481605)(0,-2.94983277591973)(0,-2.88963210702341)(0,-2.82943143812709)(0,-2.76923076923077)(0,-2.70903010033445)(0,-2.64882943143813)(0,-2.58862876254181)(0,-2.52842809364549)(0,-2.46822742474916)(0,-2.40802675585284)(0,-2.34782608695652)(0,-2.2876254180602)(0,-2.22742474916388)(0,-2.16722408026756)(0,-2.10702341137124)(0,-2.04682274247492)(0,-1.9866220735786)(0,-1.92642140468227)(0,-1.86622073578595)(0,-1.80602006688963)(0,-1.74581939799331)(0,-1.68561872909699)(0,-1.62541806020067)(0,-1.56521739130435)(0,-1.50501672240803)(0,-1.44481605351171)(0,-1.38461538461538)(0,-1.32441471571906)(0,-1.26421404682274)(0,-1.20401337792642)(0,-1.1438127090301)(0,-1.08361204013378)(0,-1.02341137123746)(0,-0.963210702341137)(0,-0.903010033444816)(0,-0.842809364548495)(0,-0.782608695652174)(0,-0.722408026755853)(0,-0.662207357859532)(0,-0.602006688963211)(0,-0.54180602006689)(0,-0.481605351170569)(0,-0.421404682274247)(0,-0.361204013377926)(0,-0.301003344481605)(0,-0.240802675585284)(0,-0.180602006688963)(0,-0.120401337792642)(0,-0.0602006688963211)(0,0)(4.94314381270903,0)(9.88628762541806,0)(14.8294314381271,0)(19.7725752508361,0)(24.7157190635452,0)(29.6588628762542,0)(34.6020066889632,0)(39.5451505016722,0)(44.4882943143813,0)(49.4314381270903,0)(54.3745819397993,0)(59.3177257525084,0)(64.2608695652174,0)(69.2040133779264,0)(74.1471571906355,0)(79.0903010033445,0)(84.0334448160535,0)(88.9765886287625,0)(93.9197324414716,0)(98.8628762541806,0)(103.80602006689,0)(108.749163879599,0)(113.692307692308,0)(118.635451505017,0)(123.578595317726,0)(128.521739130435,0)(133.464882943144,0)(138.408026755853,0)(143.351170568562,0)(148.294314381271,0)(153.23745819398,0)(158.180602006689,0)(163.123745819398,0)(168.066889632107,0)(173.010033444816,0)(177.953177257525,0)(182.896321070234,0)(187.839464882943,0)(192.782608695652,0)(197.725752508361,0)(202.66889632107,0)(207.612040133779,0)(212.555183946488,0)(217.498327759197,0)(222.441471571906,0)(227.384615384615,0)(232.327759197324,0)(237.270903010033,0)(242.214046822742,0)(247.157190635452,0)(252.100334448161,0)(257.04347826087,0)(261.986622073579,0)(266.929765886288,0)(271.872909698997,0)(276.816053511706,0)(281.759197324415,0)(286.702341137124,0)(291.645484949833,0)(296.588628762542,0)(301.531772575251,0)(306.47491638796,0)(311.418060200669,0)(316.361204013378,0)(321.304347826087,0)(326.247491638796,0)(331.190635451505,0)(336.133779264214,0)(341.076923076923,0)(346.020066889632,0)(350.963210702341,0)(355.90635451505,0)(360.849498327759,0)(365.792642140468,0)(370.735785953177,0)(375.678929765886,0)(380.622073578595,0)(385.565217391304,0)(390.508361204013,0)(395.451505016722,0)(400.394648829431,0)(405.33779264214,0)(410.28093645485,0)(415.224080267559,0)(420.167224080268,0)(425.110367892977,0)(430.053511705686,0)(434.996655518395,0)(439.939799331104,0)(444.882943143813,0)(449.826086956522,0)(454.769230769231,0)(459.71237458194,0)(464.655518394649,0)(469.598662207358,0)(474.541806020067,0)(479.484949832776,0)(484.428093645485,0)(489.371237458194,0)(494.314381270903,0)(499.257525083612,0)(504.200668896321,0)(509.14381270903,0)(514.086956521739,0)(519.030100334448,0)(523.973244147157,0)(528.916387959866,0)(533.859531772575,0)(538.802675585284,0)(543.745819397993,0)(548.688963210702,0)(553.632107023411,0)(558.57525083612,0)(563.518394648829,0)(568.461538461538,0)(573.404682274248,0)(578.347826086957,0)(583.290969899666,0)(588.234113712375,0)(593.177257525084,0)(598.120401337793,0)(603.063545150502,0)(608.006688963211,0)(612.94983277592,0)(617.892976588629,0)(622.836120401338,0)(627.779264214047,0)(632.722408026756,0)(637.665551839465,0)(642.608695652174,0)(647.551839464883,0)(652.494983277592,0)(657.438127090301,0)(662.38127090301,0)(667.324414715719,0)(672.267558528428,0)(677.210702341137,0)(682.153846153846,0)(687.096989966555,0)(692.040133779264,0)(696.983277591973,0)(701.926421404682,0)(706.869565217391,0)(711.8127090301,0)(716.755852842809,0)(721.698996655518,0)(726.642140468227,0)(731.585284280936,0)(736.528428093646,0)(741.471571906354,0)(746.414715719064,0)(751.357859531773,0)(756.301003344482,0)(761.244147157191,0)(766.1872909699,0)(771.130434782609,0)(776.073578595318,0)(781.016722408027,0)(785.959866220736,0)(790.903010033445,0)(795.846153846154,0)(800.789297658863,0)(805.732441471572,0)(810.675585284281,0)(815.61872909699,0)(820.561872909699,0)(825.505016722408,0)(830.448160535117,0)(835.391304347826,0)(840.334448160535,0)(845.277591973244,0)(850.220735785953,0)(855.163879598662,0)(860.107023411371,0)(865.05016722408,0)(869.993311036789,0)(874.936454849498,0)(879.879598662207,0)(884.822742474916,0)(889.765886287625,0)(894.709030100334,0)(899.652173913044,0)(904.595317725752,0)(909.538461538462,0)(914.481605351171,0)(919.42474916388,0)(924.367892976589,0)(929.311036789298,0)(934.254180602007,0)(939.197324414716,0)(944.140468227425,0)(949.083612040134,0)(954.026755852843,0)(958.969899665552,0)(963.913043478261,0)(968.85618729097,0)(973.799331103679,0)(978.742474916388,0)(983.685618729097,0)(988.628762541806,0)(993.571906354515,0)(998.515050167224,0)(1003.45819397993,0)(1008.40133779264,0)(1013.34448160535,0)(1018.28762541806,0)(1023.23076923077,0)(1028.17391304348,0)(1033.11705685619,0)(1038.0602006689,0)(1043.00334448161,0)(1047.94648829431,0)(1052.88963210702,0)(1057.83277591973,0)(1062.77591973244,0)(1067.71906354515,0)(1072.66220735786,0)(1077.60535117057,0)(1082.54849498328,0)(1087.49163879599,0)(1092.4347826087,0)(1097.3779264214,0)(1102.32107023411,0)(1107.26421404682,0)(1112.20735785953,0)(1117.15050167224,0)(1122.09364548495,0)(1127.03678929766,0)(1131.97993311037,0)(1136.92307692308,0)(1141.86622073579,0)(1146.8093645485,0)(1151.7525083612,0)(1156.69565217391,0)(1161.63879598662,0)(1166.58193979933,0)(1171.52508361204,0)(1176.46822742475,0)(1181.41137123746,0)(1186.35451505017,0)(1191.29765886288,0)(1196.24080267559,0)(1201.18394648829,0)(1206.127090301,0)(1211.07023411371,0)(1216.01337792642,0)(1220.95652173913,0)(1225.89966555184,0)(1230.84280936455,0)(1235.78595317726,0)(1240.72909698997,0)(1245.67224080268,0)(1250.61538461538,0)(1255.55852842809,0)(1260.5016722408,0)(1265.44481605351,0)(1270.38795986622,0)(1275.33110367893,0)(1280.27424749164,0)(1285.21739130435,0)(1290.16053511706,0)(1295.10367892977,0)(1300.04682274247,0)(1304.98996655518,0)(1309.93311036789,0)(1314.8762541806,0)(1319.81939799331,0)(1324.76254180602,0)(1329.70568561873,0)(1334.64882943144,0)(1339.59197324415,0)(1344.53511705686,0)(1349.47826086957,0)(1354.42140468227,0)(1359.36454849498,0)(1364.30769230769,0)(1369.2508361204,0)(1374.19397993311,0)(1379.13712374582,0)(1384.08026755853,0)(1389.02341137124,0)(1393.96655518395,0)(1398.90969899666,0)(1403.85284280936,0)(1408.79598662207,0)(1413.73913043478,0)(1418.68227424749,0)(1423.6254180602,0)(1428.56856187291,0)(1433.51170568562,0)(1438.45484949833,0)(1443.39799331104,0)(1448.34113712375,0)(1453.28428093645,0)(1458.22742474916,0)(1463.17056856187,0)(1468.11371237458,0)(1473.05685618729,8.91848548857975e-07)(1478,1.78367067334156e-06)(1478.00014645918,0)};

\addplot [fill=darkgray,draw=none,forget plot] coordinates{ (1000.98662207358,-7.16387959866221)(1003.45819397993,-7.17391304347826)(1008.40133779264,-7.21404682274247)(1009.63712374582,-7.22408026755853)(1013.34448160535,-7.25418060200669)(1018.28762541806,-7.25418060200669)(1023.23076923077,-7.25418060200669)(1028.17391304348,-7.25418060200669)(1033.11705685619,-7.25418060200669)(1038.0602006689,-7.25418060200669)(1043.00334448161,-7.25418060200669)(1047.94648829431,-7.2742474916388)(1050.41806020067,-7.28428093645485)(1052.88963210702,-7.2943143812709)(1057.83277591973,-7.31438127090301)(1062.77591973244,-7.31438127090301)(1067.71906354515,-7.31438127090301)(1072.66220735786,-7.31438127090301)(1077.60535117057,-7.33444816053512)(1080.07692307692,-7.34448160535117)(1082.54849498328,-7.35451505016722)(1087.49163879599,-7.37458193979933)(1092.4347826087,-7.37458193979933)(1097.3779264214,-7.37458193979933)(1102.32107023411,-7.37458193979933)(1107.26421404682,-7.37458193979933)(1112.20735785953,-7.39464882943144)(1114.67892976589,-7.40468227424749)(1117.15050167224,-7.41471571906355)(1122.09364548495,-7.43478260869565)(1127.03678929766,-7.43478260869565)(1131.97993311037,-7.43478260869565)(1136.92307692308,-7.43478260869565)(1141.86622073579,-7.43478260869565)(1146.8093645485,-7.45484949832776)(1149.28093645485,-7.46488294314381)(1151.7525083612,-7.47491638795987)(1156.69565217391,-7.49498327759197)(1161.63879598662,-7.49498327759197)(1166.58193979933,-7.49498327759197)(1171.52508361204,-7.49498327759197)(1176.46822742475,-7.49498327759197)(1181.41137123746,-7.49498327759197)(1186.35451505017,-7.49498327759197)(1191.29765886288,-7.51505016722408)(1193.76923076923,-7.52508361204013)(1196.24080267559,-7.53511705685619)(1201.18394648829,-7.55518394648829)(1206.127090301,-7.55518394648829)(1211.07023411371,-7.55518394648829)(1216.01337792642,-7.55518394648829)(1220.95652173913,-7.55518394648829)(1225.89966555184,-7.55518394648829)(1230.84280936455,-7.5752508361204)(1233.3143812709,-7.58528428093646)(1235.78595317726,-7.59531772575251)(1240.72909698997,-7.61538461538462)(1245.67224080268,-7.61538461538462)(1250.61538461538,-7.61538461538462)(1255.55852842809,-7.61538461538462)(1260.5016722408,-7.63545150501672)(1262.97324414716,-7.64548494983278)(1265.44481605351,-7.65551839464883)(1270.38795986622,-7.67558528428094)(1275.33110367893,-7.67558528428094)(1280.27424749164,-7.67558528428094)(1285.21739130435,-7.69565217391304)(1287.6889632107,-7.7056856187291)(1290.16053511706,-7.71571906354515)(1295.10367892977,-7.73578595317726)(1300.04682274247,-7.75585284280937)(1302.51839464883,-7.76588628762542)(1304.98996655518,-7.77591973244147)(1309.93311036789,-7.79598662207358)(1314.8762541806,-7.77591973244147)(1319.81939799331,-7.79598662207358)(1324.76254180602,-7.81605351170569)(1325.58639910814,-7.82608695652174)(1329.70568561873,-7.87625418060201)(1332.17725752508,-7.88628762541806)(1334.64882943144,-7.89632107023411)(1339.59197324415,-7.93645484949833)(1342.0635451505,-7.94648829431438)(1344.53511705686,-7.95652173913043)(1349.47826086957,-7.99665551839465)(1351.94983277592,-8.0066889632107)(1354.42140468227,-8.01672240802676)(1359.36454849498,-8.05685618729097)(1361.83612040134,-8.06688963210702)(1364.30769230769,-8.07692307692308)(1369.2508361204,-8.11705685618729)(1370.48662207358,-8.12709030100334)(1374.19397993311,-8.1571906354515)(1377.90133779264,-8.18729096989967)(1379.13712374582,-8.19732441471572)(1384.08026755853,-8.23745819397993)(1385.31605351171,-8.24749163879599)(1389.02341137124,-8.27759197324415)(1391.49498327759,-8.30769230769231)(1393.96655518395,-8.33779264214047)(1397.67391304348,-8.36789297658863)(1398.90969899666,-8.37792642140468)(1403.02898550725,-8.42809364548495)(1403.85284280936,-8.438127090301)(1408.79598662207,-8.47826086956522)(1409.61984392419,-8.48829431438127)(1413.73913043478,-8.53846153846154)(1414.97491638796,-8.54849498327759)(1418.68227424749,-8.59364548494983)(1419.50613154961,-8.60869565217391)(1422.80156075808,-8.66889632107023)(1423.6254180602,-8.68394648829431)(1426.09698996656,-8.72909698996656)(1428.56856187291,-8.75919732441472)(1431.04013377926,-8.78929765886288)(1433.51170568562,-8.83444816053512)(1434.33556298774,-8.8494983277592)(1437.63099219621,-8.90969899665552)(1438.45484949833,-8.93979933110368)(1439.27870680045,-8.96989966555184)(1442.57413600892,-9.03010033444816)(1443.39799331104,-9.06020066889632)(1444.22185061316,-9.09030100334448)(1447.51727982163,-9.1505016722408)(1448.34113712375,-9.18060200668896)(1449.16499442586,-9.21070234113712)(1452.46042363434,-9.27090301003344)(1453.28428093645,-9.30100334448161)(1454.10813823857,-9.33110367892977)(1455.75585284281,-9.39130434782609)(1455.75585284281,-9.45150501672241)(1455.75585284281,-9.51170568561873)(1455.75585284281,-9.57190635451505)(1455.75585284281,-9.63210702341137)(1454.10813823857,-9.69230769230769)(1453.28428093645,-9.70735785953177)(1450.8127090301,-9.75250836120401)(1448.34113712375,-9.78260869565217)(1445.86956521739,-9.81270903010033)(1443.39799331104,-9.8428093645485)(1439.6906354515,-9.87290969899666)(1438.45484949833,-9.88294314381271)(1433.51170568562,-9.92307692307692)(1431.04013377926,-9.93311036789298)(1428.56856187291,-9.94314381270903)(1423.6254180602,-9.98327759197324)(1421.15384615385,-9.9933110367893)(1418.68227424749,-10.0033444816054)(1413.73913043478,-10.0434782608696)(1411.26755852843,-10.0535117056856)(1408.79598662207,-10.0635451505017)(1403.85284280936,-10.0836120401338)(1398.90969899666,-10.0836120401338)(1393.96655518395,-10.1036789297659)(1391.49498327759,-10.1137123745819)(1389.02341137124,-10.123745819398)(1384.08026755853,-10.1438127090301)(1379.13712374582,-10.1438127090301)(1374.19397993311,-10.1438127090301)(1369.2508361204,-10.1438127090301)(1364.30769230769,-10.1438127090301)(1359.36454849498,-10.1638795986622)(1356.89297658863,-10.1739130434783)(1354.42140468227,-10.1839464882943)(1349.47826086957,-10.2040133779264)(1344.53511705686,-10.2040133779264)(1339.59197324415,-10.2040133779264)(1334.64882943144,-10.2040133779264)(1329.70568561873,-10.2040133779264)(1324.76254180602,-10.2040133779264)(1319.81939799331,-10.2040133779264)(1314.8762541806,-10.2040133779264)(1309.93311036789,-10.2040133779264)(1304.98996655518,-10.2040133779264)(1300.04682274247,-10.2240802675585)(1297.57525083612,-10.2341137123746)(1295.10367892977,-10.2441471571906)(1290.16053511706,-10.2642140468227)(1285.21739130435,-10.2642140468227)(1280.27424749164,-10.2642140468227)(1275.33110367893,-10.2642140468227)(1270.38795986622,-10.2642140468227)(1265.44481605351,-10.2642140468227)(1260.5016722408,-10.2642140468227)(1255.55852842809,-10.2642140468227)(1250.61538461538,-10.2642140468227)(1245.67224080268,-10.2642140468227)(1240.72909698997,-10.2642140468227)(1235.78595317726,-10.2642140468227)(1230.84280936455,-10.2642140468227)(1225.89966555184,-10.2642140468227)(1222.19230769231,-10.2943143812709)(1220.95652173913,-10.304347826087)(1216.01337792642,-10.304347826087)(1211.07023411371,-10.304347826087)(1209.83444816054,-10.2943143812709)(1206.127090301,-10.2642140468227)(1201.18394648829,-10.2642140468227)(1196.24080267559,-10.2642140468227)(1191.29765886288,-10.2642140468227)(1186.35451505017,-10.2642140468227)(1181.41137123746,-10.2642140468227)(1176.46822742475,-10.2642140468227)(1171.52508361204,-10.2642140468227)(1166.58193979933,-10.2642140468227)(1161.63879598662,-10.2642140468227)(1156.69565217391,-10.2642140468227)(1151.7525083612,-10.2642140468227)(1146.8093645485,-10.2642140468227)(1141.86622073579,-10.2642140468227)(1136.92307692308,-10.2642140468227)(1131.97993311037,-10.2642140468227)(1127.03678929766,-10.2642140468227)(1122.09364548495,-10.2642140468227)(1117.15050167224,-10.2642140468227)(1112.20735785953,-10.2441471571906)(1109.73578595318,-10.2341137123746)(1107.26421404682,-10.2240802675585)(1102.32107023411,-10.2040133779264)(1098.61371237458,-10.2341137123746)(1097.3779264214,-10.2441471571906)(1092.4347826087,-10.2441471571906)(1087.49163879599,-10.2441471571906)(1086.25585284281,-10.2341137123746)(1082.54849498328,-10.2040133779264)(1077.60535117057,-10.2040133779264)(1072.66220735786,-10.2040133779264)(1067.71906354515,-10.2040133779264)(1062.77591973244,-10.2040133779264)(1057.83277591973,-10.2040133779264)(1052.88963210702,-10.2040133779264)(1047.94648829431,-10.2040133779264)(1043.00334448161,-10.2040133779264)(1038.0602006689,-10.2040133779264)(1033.11705685619,-10.2040133779264)(1028.17391304348,-10.1839464882943)(1025.70234113712,-10.1739130434783)(1023.23076923077,-10.1638795986622)(1018.28762541806,-10.1438127090301)(1013.34448160535,-10.1438127090301)(1008.40133779264,-10.1438127090301)(1003.45819397993,-10.1438127090301)(998.515050167224,-10.1438127090301)(993.571906354515,-10.1438127090301)(988.628762541806,-10.1438127090301)(983.685618729097,-10.1438127090301)(978.742474916388,-10.123745819398)(976.270903010033,-10.1137123745819)(973.799331103679,-10.1036789297659)(968.85618729097,-10.0836120401338)(963.913043478261,-10.0836120401338)(958.969899665552,-10.0836120401338)(954.026755852843,-10.0836120401338)(949.083612040134,-10.0836120401338)(944.140468227425,-10.0635451505017)(941.66889632107,-10.0535117056856)(939.197324414716,-10.0434782608696)(934.254180602007,-10.0234113712375)(929.311036789298,-10.0234113712375)(924.367892976589,-10.0234113712375)(919.42474916388,-10.0033444816054)(916.953177257525,-9.9933110367893)(914.481605351171,-9.98327759197324)(909.538461538462,-9.96321070234114)(904.595317725752,-9.96321070234114)(899.652173913044,-9.94314381270903)(897.180602006689,-9.93311036789298)(894.709030100334,-9.92307692307692)(889.765886287625,-9.90301003344482)(884.822742474916,-9.88294314381271)(882.351170568562,-9.87290969899666)(879.879598662207,-9.8628762541806)(874.936454849498,-9.8428093645485)(869.993311036789,-9.82274247491639)(867.521739130435,-9.81270903010033)(865.05016722408,-9.80267558528428)(860.107023411371,-9.78260869565217)(855.163879598662,-9.76254180602007)(852.692307692308,-9.75250836120401)(850.220735785953,-9.74247491638796)(845.277591973244,-9.70234113712375)(842.80602006689,-9.69230769230769)(840.334448160535,-9.68227424749164)(835.391304347826,-9.64214046822742)(832.919732441472,-9.63210702341137)(830.448160535117,-9.62207357859532)(825.505016722408,-9.5819397993311)(823.033444816054,-9.57190635451505)(820.561872909699,-9.561872909699)(815.61872909699,-9.52173913043478)(814.382943143813,-9.51170568561873)(810.675585284281,-9.48160535117057)(808.204013377926,-9.45150501672241)(805.732441471572,-9.42140468227425)(802.02508361204,-9.39130434782609)(800.789297658863,-9.38127090301004)(795.846153846154,-9.34113712374582)(794.610367892977,-9.33110367892977)(790.903010033445,-9.30100334448161)(788.43143812709,-9.27090301003344)(785.959866220736,-9.24080267558528)(783.488294314381,-9.21070234113712)(781.016722408027,-9.18060200668896)(778.545150501672,-9.1505016722408)(776.073578595318,-9.10535117056856)(775.2497212932,-9.09030100334448)(771.954292084727,-9.03010033444816)(771.130434782609,-9.01505016722408)(768.658862876254,-8.96989966555184)(766.1872909699,-8.9247491638796)(765.363433667781,-8.90969899665552)(763.715719063545,-8.8494983277592)(762.068004459309,-8.78929765886288)(761.244147157191,-8.75919732441472)(760.420289855073,-8.72909698996656)(758.772575250836,-8.66889632107023)(758.772575250836,-8.60869565217391)(760.420289855073,-8.54849498327759)(761.244147157191,-8.51839464882943)(762.068004459309,-8.48829431438127)(765.363433667781,-8.42809364548495)(766.1872909699,-8.41304347826087)(768.658862876254,-8.36789297658863)(771.130434782609,-8.33779264214047)(773.602006688963,-8.30769230769231)(776.073578595318,-8.26254180602007)(776.897435897436,-8.24749163879599)(781.016722408027,-8.19732441471572)(781.840579710145,-8.18729096989967)(785.959866220736,-8.1371237458194)(787.195652173913,-8.12709030100334)(790.903010033445,-8.09698996655519)(793.374581939799,-8.06688963210702)(795.846153846154,-8.03678929765886)(798.317725752508,-8.0066889632107)(800.789297658863,-7.97658862876254)(804.496655518395,-7.94648829431438)(805.732441471572,-7.93645484949833)(810.675585284281,-7.89632107023411)(813.147157190635,-7.88628762541806)(815.61872909699,-7.87625418060201)(820.561872909699,-7.83612040133779)(821.385730211817,-7.82608695652174)(825.505016722408,-7.77591973244147)(826.740802675585,-7.76588628762542)(830.448160535117,-7.7433110367893)(835.391304347826,-7.71321070234114)(837.862876254181,-7.7056856187291)(840.334448160535,-7.69565217391304)(845.277591973244,-7.65551839464883)(847.749163879599,-7.64548494983278)(850.220735785953,-7.63545150501672)(854.340022296544,-7.58528428093645)(855.163879598662,-7.5752508361204)(860.107023411371,-7.55518394648829)(863.814381270903,-7.52508361204013)(865.05016722408,-7.51755852842809)(869.993311036789,-7.51505016722408)(874.936454849498,-7.49498327759197)(879.879598662207,-7.47491638795987)(882.351170568562,-7.46488294314381)(884.822742474916,-7.45484949832776)(889.765886287625,-7.43478260869565)(894.709030100335,-7.41471571906355)(897.180602006689,-7.40468227424749)(899.652173913044,-7.39464882943144)(904.595317725752,-7.37458193979933)(909.538461538462,-7.35451505016722)(912.010033444816,-7.34448160535117)(914.481605351171,-7.33444816053512)(919.42474916388,-7.31438127090301)(924.367892976589,-7.31438127090301)(929.311036789298,-7.2943143812709)(931.782608695652,-7.28428093645485)(934.254180602007,-7.2742474916388)(939.197324414716,-7.25418060200669)(944.140468227425,-7.25418060200669)(949.083612040134,-7.25418060200669)(954.026755852843,-7.25418060200669)(958.969899665552,-7.25418060200669)(963.913043478261,-7.25418060200669)(968.85618729097,-7.25418060200669)(973.799331103679,-7.23411371237458)(975.035117056856,-7.22408026755853)(978.742474916388,-7.19397993311037)(983.685618729097,-7.17391304347826)(988.628762541806,-7.17391304347826)(993.571906354515,-7.17391304347826)(996.04347826087,-7.16387959866221)(998.515050167224,-7.15384615384615)(1000.98662207358,-7.16387959866221)};

\addplot [fill=red!40!yellow,draw=none,forget plot] coordinates{ (1025.70234113712,-7.46488294314381)(1028.17391304348,-7.47491638795987)(1033.11705685619,-7.49498327759197)(1038.0602006689,-7.49498327759197)(1043.00334448161,-7.49498327759197)(1047.94648829431,-7.49498327759197)(1052.88963210702,-7.49498327759197)(1057.83277591973,-7.49498327759197)(1062.77591973244,-7.51505016722408)(1065.2474916388,-7.52508361204013)(1067.71906354515,-7.53010033444816)(1072.66220735786,-7.54013377926421)(1077.60535117057,-7.54013377926421)(1082.54849498328,-7.54314381270903)(1087.49163879599,-7.54765886287625)(1092.4347826087,-7.55518394648829)(1097.3779264214,-7.55518394648829)(1102.32107023411,-7.55518394648829)(1107.26421404682,-7.55518394648829)(1112.20735785953,-7.55518394648829)(1117.15050167224,-7.55518394648829)(1122.09364548495,-7.55518394648829)(1127.03678929766,-7.55518394648829)(1131.97993311037,-7.55518394648829)(1136.92307692308,-7.55518394648829)(1141.86622073579,-7.5752508361204)(1144.33779264214,-7.58528428093646)(1146.8093645485,-7.59030100334448)(1151.7525083612,-7.60033444816053)(1156.69565217391,-7.60033444816053)(1161.63879598662,-7.60033444816053)(1166.58193979933,-7.60334448160535)(1171.52508361204,-7.60785953177257)(1176.46822742475,-7.61538461538462)(1181.41137123746,-7.61538461538462)(1186.35451505017,-7.63545150501672)(1188.82608695652,-7.64548494983278)(1191.29765886288,-7.6505016722408)(1196.24080267559,-7.66354515050167)(1201.18394648829,-7.6680602006689)(1206.127090301,-7.67558528428094)(1211.07023411371,-7.67558528428094)(1216.01337792642,-7.67558528428094)(1220.95652173913,-7.69565217391304)(1223.42809364549,-7.7056856187291)(1225.89966555184,-7.71321070234114)(1230.84280936455,-7.73578595317726)(1235.78595317726,-7.73578595317726)(1240.72909698997,-7.73578595317726)(1245.67224080268,-7.75585284280937)(1248.14381270903,-7.76588628762542)(1250.61538461538,-7.77591973244147)(1255.55852842809,-7.79598662207358)(1260.5016722408,-7.79598662207358)(1265.44481605351,-7.79598662207358)(1270.38795986622,-7.81605351170569)(1272.85953177258,-7.82608695652174)(1275.33110367893,-7.83612040133779)(1280.27424749164,-7.8561872909699)(1285.21739130435,-7.87625418060201)(1287.6889632107,-7.88628762541806)(1290.16053511706,-7.89632107023411)(1295.10367892977,-7.91638795986622)(1300.04682274247,-7.93645484949833)(1302.51839464883,-7.94648829431438)(1304.98996655518,-7.95652173913043)(1309.93311036789,-7.99665551839465)(1312.40468227425,-8.0066889632107)(1314.8762541806,-8.01672240802676)(1319.81939799331,-8.05685618729097)(1322.29096989967,-8.06688963210702)(1324.76254180602,-8.07692307692308)(1329.70568561873,-8.11705685618729)(1332.17725752508,-8.12709030100334)(1334.64882943144,-8.1371237458194)(1339.59197324415,-8.17725752508361)(1340.82775919732,-8.18729096989967)(1344.53511705686,-8.21739130434783)(1348.24247491639,-8.24749163879599)(1349.47826086957,-8.25752508361204)(1354.42140468227,-8.29765886287626)(1355.65719063545,-8.30769230769231)(1359.36454849498,-8.33779264214047)(1363.07190635452,-8.36789297658863)(1364.30769230769,-8.37792642140468)(1369.2508361204,-8.4180602006689)(1370.48662207358,-8.42809364548495)(1374.19397993311,-8.45819397993311)(1376.66555183947,-8.48829431438127)(1379.13712374582,-8.51839464882943)(1381.60869565217,-8.54849498327759)(1384.08026755853,-8.57859531772575)(1386.55183946488,-8.60869565217391)(1389.02341137124,-8.63879598662207)(1391.49498327759,-8.66889632107023)(1393.96655518395,-8.69899665551839)(1396.4381270903,-8.72909698996656)(1398.90969899666,-8.75919732441472)(1401.38127090301,-8.78929765886288)(1403.85284280936,-8.81939799331104)(1406.32441471572,-8.8494983277592)(1408.79598662207,-8.89464882943144)(1409.61984392419,-8.90969899665552)(1412.91527313266,-8.96989966555184)(1413.73913043478,-9)(1414.5629877369,-9.03010033444816)(1417.85841694537,-9.09030100334448)(1418.68227424749,-9.12040133779264)(1419.50613154961,-9.1505016722408)(1421.15384615385,-9.21070234113712)(1421.15384615385,-9.27090301003344)(1421.15384615385,-9.33110367892977)(1421.15384615385,-9.39130434782609)(1421.15384615385,-9.45150501672241)(1419.50613154961,-9.51170568561873)(1418.68227424749,-9.52675585284281)(1416.21070234114,-9.57190635451505)(1413.73913043478,-9.60200668896321)(1411.26755852843,-9.63210702341137)(1408.79598662207,-9.66220735785953)(1406.32441471572,-9.69230769230769)(1403.85284280936,-9.72240802675585)(1401.38127090301,-9.75250836120401)(1398.90969899666,-9.78260869565217)(1395.20234113712,-9.81270903010033)(1393.96655518395,-9.82274247491639)(1389.02341137124,-9.8428093645485)(1384.08026755853,-9.8428093645485)(1379.13712374582,-9.8628762541806)(1376.66555183947,-9.87290969899666)(1374.19397993311,-9.88294314381271)(1369.2508361204,-9.90301003344482)(1364.30769230769,-9.90301003344482)(1359.36454849498,-9.92307692307692)(1356.89297658863,-9.93311036789298)(1354.42140468227,-9.94314381270903)(1349.47826086957,-9.96321070234114)(1344.53511705686,-9.96321070234114)(1339.59197324415,-9.96321070234114)(1334.64882943144,-9.96321070234114)(1329.70568561873,-9.96321070234114)(1324.76254180602,-9.98327759197324)(1322.29096989967,-9.9933110367893)(1319.81939799331,-10.0033444816054)(1314.8762541806,-10.0234113712375)(1309.93311036789,-10.0234113712375)(1304.98996655518,-10.0234113712375)(1300.04682274247,-10.0234113712375)(1295.10367892977,-10.0234113712375)(1290.16053511706,-10.0234113712375)(1285.21739130435,-10.0234113712375)(1280.27424749164,-10.0234113712375)(1275.33110367893,-10.0234113712375)(1270.38795986622,-10.0234113712375)(1265.44481605351,-10.0234113712375)(1260.5016722408,-10.0234113712375)(1255.55852842809,-10.0234113712375)(1250.61538461538,-10.0234113712375)(1245.67224080268,-10.0234113712375)(1240.72909698997,-10.0434782608696)(1238.25752508361,-10.0535117056856)(1235.78595317726,-10.0635451505017)(1230.84280936455,-10.0836120401338)(1225.89966555184,-10.0836120401338)(1220.95652173913,-10.0836120401338)(1216.01337792642,-10.0836120401338)(1211.07023411371,-10.0836120401338)(1206.127090301,-10.0836120401338)(1201.18394648829,-10.0836120401338)(1196.24080267559,-10.0836120401338)(1191.29765886288,-10.0836120401338)(1186.35451505017,-10.0836120401338)(1181.41137123746,-10.0836120401338)(1176.46822742475,-10.0836120401338)(1171.52508361204,-10.0836120401338)(1166.58193979933,-10.0836120401338)(1161.63879598662,-10.0836120401338)(1156.69565217391,-10.0836120401338)(1151.7525083612,-10.0836120401338)(1146.8093645485,-10.0836120401338)(1141.86622073579,-10.0836120401338)(1136.92307692308,-10.0836120401338)(1131.97993311037,-10.0836120401338)(1127.03678929766,-10.0836120401338)(1122.09364548495,-10.0836120401338)(1117.15050167224,-10.0836120401338)(1112.20735785953,-10.0836120401338)(1107.26421404682,-10.0836120401338)(1102.32107023411,-10.0836120401338)(1097.3779264214,-10.0836120401338)(1092.4347826087,-10.0836120401338)(1087.49163879599,-10.0836120401338)(1082.54849498328,-10.0836120401338)(1077.60535117057,-10.0635451505017)(1075.13377926421,-10.0535117056856)(1072.66220735786,-10.0434782608696)(1067.71906354515,-10.0234113712375)(1062.77591973244,-10.0234113712375)(1057.83277591973,-10.0234113712375)(1052.88963210702,-10.0234113712375)(1047.94648829431,-10.0234113712375)(1043.00334448161,-10.0234113712375)(1038.0602006689,-10.0234113712375)(1033.11705685619,-10.0234113712375)(1028.17391304348,-10.0234113712375)(1023.23076923077,-10.0234113712375)(1018.28762541806,-10.0234113712375)(1013.34448160535,-10.0234113712375)(1008.40133779264,-10.0234113712375)(1003.45819397993,-10.0033444816054)(1000.98662207358,-9.9933110367893)(998.515050167224,-9.98327759197324)(993.571906354515,-9.96321070234114)(988.628762541806,-9.96321070234114)(983.685618729097,-9.96321070234114)(978.742474916388,-9.96321070234114)(973.799331103679,-9.96321070234114)(968.85618729097,-9.96321070234114)(963.913043478261,-9.94314381270903)(961.441471571906,-9.93311036789298)(958.969899665552,-9.92307692307692)(954.026755852843,-9.90301003344482)(949.083612040134,-9.90301003344482)(944.140468227425,-9.90301003344482)(939.197324414716,-9.88294314381271)(936.725752508361,-9.87290969899666)(934.254180602007,-9.8628762541806)(929.311036789298,-9.8428093645485)(924.367892976589,-9.8428093645485)(919.42474916388,-9.82274247491639)(916.953177257525,-9.81270903010033)(914.481605351171,-9.80267558528428)(909.538461538462,-9.78260869565217)(904.595317725752,-9.76254180602007)(902.123745819398,-9.75250836120401)(899.652173913044,-9.74247491638796)(894.709030100334,-9.72240802675585)(889.765886287625,-9.70234113712375)(887.294314381271,-9.69230769230769)(884.822742474916,-9.68227424749164)(879.879598662207,-9.64214046822742)(877.408026755853,-9.63210702341137)(874.936454849498,-9.62207357859532)(869.993311036789,-9.5819397993311)(867.521739130435,-9.57190635451505)(865.05016722408,-9.561872909699)(860.107023411371,-9.52173913043478)(857.635451505017,-9.51170568561873)(855.163879598662,-9.50167224080268)(850.220735785953,-9.46153846153846)(848.984949832776,-9.45150501672241)(845.277591973244,-9.42140468227425)(842.80602006689,-9.39130434782609)(840.334448160535,-9.36120401337793)(836.627090301003,-9.33110367892977)(835.391304347826,-9.32107023411371)(830.448160535117,-9.2809364548495)(829.21237458194,-9.27090301003344)(825.505016722408,-9.24080267558528)(823.033444816054,-9.21070234113712)(820.561872909699,-9.16555183946488)(819.738015607581,-9.1505016722408)(816.442586399108,-9.09030100334448)(815.61872909699,-9.0752508361204)(813.147157190635,-9.03010033444816)(810.675585284281,-8.98494983277592)(809.851727982163,-8.96989966555184)(808.204013377926,-8.90969899665552)(806.55629877369,-8.8494983277592)(805.732441471572,-8.81939799331104)(804.908584169454,-8.78929765886288)(803.260869565217,-8.72909698996656)(804.908584169454,-8.66889632107023)(805.732441471572,-8.63879598662207)(806.55629877369,-8.60869565217391)(808.204013377926,-8.54849498327759)(809.851727982163,-8.48829431438127)(810.675585284281,-8.45819397993311)(811.499442586399,-8.42809364548495)(814.794871794872,-8.36789297658863)(815.61872909699,-8.35284280936455)(818.090301003345,-8.30769230769231)(820.561872909699,-8.27759197324415)(823.033444816054,-8.24749163879599)(825.505016722408,-8.21739130434783)(827.976588628763,-8.18729096989967)(830.448160535117,-8.1571906354515)(834.155518394649,-8.12709030100334)(835.391304347826,-8.11705685618729)(840.334448160535,-8.07692307692308)(841.570234113712,-8.06688963210702)(845.277591973244,-8.03678929765886)(848.984949832776,-8.0066889632107)(850.220735785953,-7.99665551839465)(855.163879598662,-7.95652173913043)(856.399665551839,-7.94648829431438)(860.107023411371,-7.91638795986622)(863.814381270903,-7.88628762541806)(865.05016722408,-7.87625418060201)(869.993311036789,-7.83612040133779)(872.464882943144,-7.82608695652174)(874.936454849498,-7.81605351170569)(879.879598662207,-7.77591973244147)(882.351170568562,-7.76588628762542)(884.822742474916,-7.75585284280937)(889.765886287625,-7.71571906354515)(892.23745819398,-7.7056856187291)(894.709030100334,-7.69565217391304)(899.652173913044,-7.65551839464883)(902.123745819398,-7.64548494983278)(904.595317725752,-7.63545150501672)(909.538461538462,-7.61538461538462)(914.481605351171,-7.59531772575251)(916.953177257525,-7.58528428093646)(919.42474916388,-7.5752508361204)(924.367892976589,-7.55518394648829)(929.311036789298,-7.55518394648829)(934.254180602007,-7.53511705685619)(936.725752508361,-7.52508361204013)(939.197324414716,-7.51505016722408)(944.140468227425,-7.49498327759197)(949.083612040134,-7.49498327759197)(954.026755852843,-7.49498327759197)(958.969899665552,-7.47491638795987)(961.441471571906,-7.46488294314381)(963.913043478261,-7.45484949832776)(968.85618729097,-7.43478260869565)(973.799331103679,-7.43478260869565)(978.742474916388,-7.43478260869565)(983.685618729097,-7.43478260869565)(988.628762541806,-7.43478260869565)(993.571906354515,-7.43478260869565)(998.515050167224,-7.43478260869565)(1003.45819397993,-7.43478260869565)(1008.40133779264,-7.43478260869565)(1013.34448160535,-7.43478260869565)(1018.28762541806,-7.43478260869565)(1023.23076923077,-7.45484949832776)(1025.70234113712,-7.46488294314381)};

\addplot [fill=green!60!black,draw=none,forget plot] coordinates{ (1085.02006688963,-7.64548494983278)(1087.49163879599,-7.65551839464883)(1092.4347826087,-7.67558528428094)(1097.3779264214,-7.67558528428094)(1102.32107023411,-7.67558528428094)(1107.26421404682,-7.67558528428094)(1112.20735785953,-7.67558528428094)(1117.15050167224,-7.67558528428094)(1122.09364548495,-7.67558528428094)(1127.03678929766,-7.67558528428094)(1131.97993311037,-7.67558528428094)(1136.92307692308,-7.67558528428094)(1141.86622073579,-7.68311036789298)(1146.8093645485,-7.6876254180602)(1151.7525083612,-7.69063545150502)(1156.69565217391,-7.69063545150502)(1161.63879598662,-7.69063545150502)(1166.58193979933,-7.70066889632107)(1169.05351170569,-7.7056856187291)(1171.52508361204,-7.71571906354515)(1176.46822742475,-7.73578595317726)(1181.41137123746,-7.73578595317726)(1186.35451505017,-7.7433110367893)(1191.29765886288,-7.74782608695652)(1196.24080267559,-7.76086956521739)(1198.71237458194,-7.76588628762542)(1201.18394648829,-7.77591973244147)(1206.127090301,-7.79598662207358)(1211.07023411371,-7.79598662207358)(1216.01337792642,-7.79598662207358)(1220.95652173913,-7.8185618729097)(1223.42809364549,-7.82608695652174)(1225.89966555184,-7.83612040133779)(1230.84280936455,-7.8561872909699)(1235.78595317726,-7.8561872909699)(1240.72909698997,-7.87625418060201)(1243.20066889632,-7.88628762541806)(1245.67224080268,-7.89632107023411)(1250.61538461538,-7.9314381270903)(1255.55852842809,-7.9314381270903)(1258.03010033445,-7.94648829431438)(1260.5016722408,-7.95652173913043)(1265.44481605351,-7.95652173913043)(1270.38795986622,-7.97658862876254)(1274.09531772575,-8.0066889632107)(1275.33110367893,-8.01672240802676)(1280.27424749164,-8.03678929765886)(1282.74581939799,-8.06688963210702)(1285.21739130435,-8.09698996655519)(1290.16053511706,-8.09698996655519)(1292.63210702341,-8.12709030100334)(1295.10367892977,-8.1571906354515)(1300.04682274247,-8.17725752508361)(1300.87068004459,-8.18729096989967)(1304.98996655518,-8.23745819397993)(1309.93311036789,-8.23745819397993)(1311.16889632107,-8.24749163879599)(1314.8762541806,-8.27759197324415)(1318.58361204013,-8.30769230769231)(1319.81939799331,-8.32274247491639)(1323.52675585284,-8.36789297658863)(1324.76254180602,-8.37792642140468)(1329.70568561873,-8.4180602006689)(1330.94147157191,-8.42809364548495)(1334.64882943144,-8.47324414715719)(1335.88461538462,-8.48829431438127)(1339.59197324415,-8.51839464882943)(1342.0635451505,-8.54849498327759)(1344.53511705686,-8.57859531772575)(1347.00668896321,-8.60869565217391)(1349.47826086957,-8.63879598662207)(1351.94983277592,-8.66889632107023)(1354.42140468227,-8.71404682274247)(1355.65719063545,-8.72909698996656)(1359.36454849498,-8.7742474916388)(1360.60033444816,-8.78929765886288)(1364.30769230769,-8.83444816053512)(1365.54347826087,-8.8494983277592)(1365.54347826087,-8.90969899665552)(1365.54347826087,-8.96989966555184)(1368.01505016722,-9.03010033444816)(1369.2508361204,-9.06020066889632)(1371.72240802676,-9.09030100334448)(1374.19397993311,-9.12040133779264)(1375.01783723523,-9.1505016722408)(1374.19397993311,-9.16555183946488)(1370.48662207358,-9.21070234113712)(1369.2508361204,-9.24080267558528)(1368.42697881828,-9.27090301003344)(1365.13154960981,-9.33110367892977)(1364.30769230769,-9.36120401337793)(1363.48383500557,-9.39130434782609)(1359.36454849498,-9.44147157190636)(1358.54069119287,-9.45150501672241)(1354.42140468227,-9.50167224080268)(1353.1856187291,-9.51170568561873)(1349.47826086957,-9.55685618729097)(1348.24247491639,-9.57190635451505)(1344.53511705686,-9.60200668896321)(1342.0635451505,-9.63210702341137)(1339.59197324415,-9.66220735785953)(1334.64882943144,-9.68227424749164)(1333.41304347826,-9.69230769230769)(1329.70568561873,-9.72240802675585)(1324.76254180602,-9.74247491638796)(1323.52675585284,-9.75250836120401)(1319.81939799331,-9.78260869565217)(1316.11204013378,-9.81270903010033)(1314.8762541806,-9.82274247491639)(1309.93311036789,-9.82274247491639)(1307.46153846154,-9.81270903010033)(1304.98996655518,-9.80267558528428)(1300.04682274247,-9.78260869565217)(1295.10367892977,-9.80267558528428)(1292.63210702341,-9.81270903010033)(1290.16053511706,-9.82274247491639)(1285.21739130435,-9.8428093645485)(1280.27424749164,-9.8428093645485)(1275.33110367893,-9.8428093645485)(1270.38795986622,-9.8428093645485)(1265.44481605351,-9.8628762541806)(1262.97324414716,-9.87290969899666)(1260.5016722408,-9.88294314381271)(1255.55852842809,-9.90301003344482)(1250.61538461538,-9.90301003344482)(1245.67224080268,-9.90301003344482)(1240.72909698997,-9.90301003344482)(1235.78595317726,-9.90301003344482)(1230.84280936455,-9.90301003344482)(1225.89966555184,-9.90301003344482)(1220.95652173913,-9.90301003344482)(1216.01337792642,-9.90301003344482)(1211.07023411371,-9.90301003344482)(1206.127090301,-9.90301003344482)(1201.18394648829,-9.88294314381271)(1196.24080267559,-9.88294314381271)(1191.29765886288,-9.88294314381271)(1186.35451505017,-9.90301003344482)(1181.41137123746,-9.90301003344482)(1176.46822742475,-9.90301003344482)(1171.52508361204,-9.90301003344482)(1166.58193979933,-9.90301003344482)(1161.63879598662,-9.90301003344482)(1156.69565217391,-9.90301003344482)(1151.7525083612,-9.90301003344482)(1146.8093645485,-9.90301003344482)(1141.86622073579,-9.90301003344482)(1136.92307692308,-9.90301003344482)(1131.97993311037,-9.90301003344482)(1127.03678929766,-9.90301003344482)(1122.09364548495,-9.90301003344482)(1117.15050167224,-9.90301003344482)(1112.20735785953,-9.90301003344482)(1107.26421404682,-9.88294314381271)(1104.79264214047,-9.87290969899666)(1102.32107023411,-9.8628762541806)(1097.3779264214,-9.8428093645485)(1092.4347826087,-9.8428093645485)(1087.49163879599,-9.8428093645485)(1082.54849498328,-9.8428093645485)(1077.60535117057,-9.8428093645485)(1072.66220735786,-9.8428093645485)(1067.71906354515,-9.8428093645485)(1062.77591973244,-9.8428093645485)(1057.83277591973,-9.8428093645485)(1052.88963210702,-9.82274247491639)(1050.41806020067,-9.81270903010033)(1047.94648829431,-9.80267558528428)(1043.00334448161,-9.78260869565217)(1038.0602006689,-9.78260869565217)(1033.11705685619,-9.78260869565217)(1028.17391304348,-9.78260869565217)(1023.23076923077,-9.78260869565217)(1018.28762541806,-9.78260869565217)(1013.34448160535,-9.78260869565217)(1008.40133779264,-9.76254180602007)(1005.92976588629,-9.75250836120401)(1003.45819397993,-9.74247491638796)(998.515050167224,-9.72240802675585)(993.571906354515,-9.72240802675585)(988.628762541806,-9.72240802675585)(983.685618729097,-9.72240802675585)(978.742474916388,-9.72240802675585)(973.799331103679,-9.70234113712375)(971.327759197324,-9.69230769230769)(968.85618729097,-9.68227424749164)(963.913043478261,-9.66220735785953)(958.969899665552,-9.66220735785953)(954.026755852843,-9.66220735785953)(949.083612040134,-9.64214046822742)(946.612040133779,-9.63210702341137)(944.140468227425,-9.62207357859532)(939.197324414716,-9.60200668896321)(934.254180602007,-9.60200668896321)(929.311036789298,-9.5819397993311)(926.839464882943,-9.57190635451505)(924.367892976589,-9.561872909699)(919.42474916388,-9.52173913043478)(916.953177257525,-9.51170568561873)(914.481605351171,-9.50167224080268)(909.538461538462,-9.48160535117057)(904.595317725752,-9.46153846153846)(902.123745819398,-9.45150501672241)(899.652173913044,-9.44147157190636)(894.709030100334,-9.42140468227425)(889.765886287625,-9.40133779264214)(887.294314381271,-9.39130434782609)(884.822742474916,-9.38127090301004)(879.879598662207,-9.34113712374582)(877.408026755853,-9.33110367892977)(874.936454849498,-9.32107023411371)(869.993311036789,-9.2809364548495)(867.521739130435,-9.27090301003344)(865.05016722408,-9.25585284280936)(861.342809364548,-9.21070234113712)(860.107023411371,-9.20066889632107)(855.163879598662,-9.16053511705686)(853.928093645485,-9.1505016722408)(850.220735785953,-9.12040133779264)(847.749163879599,-9.09030100334448)(845.277591973244,-9.06020066889632)(842.80602006689,-9.03010033444816)(840.334448160535,-8.98494983277592)(839.510590858417,-8.96989966555184)(836.215161649944,-8.90969899665552)(835.391304347826,-8.87959866220736)(834.567447045708,-8.8494983277592)(832.919732441472,-8.78929765886288)(834.567447045708,-8.72909698996656)(835.391304347826,-8.69899665551839)(836.215161649944,-8.66889632107023)(839.510590858417,-8.60869565217391)(840.334448160535,-8.59364548494983)(844.041806020067,-8.54849498327759)(845.277591973244,-8.53344481605351)(847.749163879599,-8.48829431438127)(850.220735785953,-8.45819397993311)(852.692307692308,-8.42809364548495)(855.163879598662,-8.39799331103679)(857.635451505017,-8.36789297658863)(860.107023411371,-8.33779264214047)(863.814381270903,-8.30769230769231)(865.05016722408,-8.29765886287626)(869.993311036789,-8.25752508361204)(871.229096989967,-8.24749163879599)(874.936454849498,-8.21739130434783)(878.64381270903,-8.18729096989967)(879.879598662207,-8.17725752508361)(884.822742474916,-8.1371237458194)(887.294314381271,-8.12709030100334)(889.765886287625,-8.11705685618729)(894.709030100334,-8.07692307692308)(897.180602006689,-8.06688963210702)(899.652173913044,-8.05685618729097)(904.595317725752,-8.01672240802676)(907.066889632107,-8.0066889632107)(909.538461538462,-7.99665551839465)(914.481605351171,-7.95652173913043)(916.953177257525,-7.94648829431438)(919.42474916388,-7.93645484949833)(924.367892976589,-7.89632107023411)(926.839464882943,-7.88628762541806)(929.311036789298,-7.87625418060201)(934.254180602007,-7.8561872909699)(939.197324414716,-7.83612040133779)(941.66889632107,-7.82608695652174)(944.140468227425,-7.81605351170569)(949.083612040134,-7.79598662207358)(954.026755852843,-7.77591973244147)(956.498327759197,-7.76588628762542)(958.969899665552,-7.75585284280937)(963.913043478261,-7.73578595317726)(968.85618729097,-7.73578595317726)(973.799331103679,-7.73578595317726)(978.742474916388,-7.71571906354515)(981.214046822742,-7.7056856187291)(983.685618729097,-7.69565217391304)(988.628762541806,-7.67558528428094)(993.571906354515,-7.67558528428094)(998.515050167224,-7.67558528428094)(1003.45819397993,-7.67558528428094)(1008.40133779264,-7.67558528428094)(1013.34448160535,-7.67558528428094)(1018.28762541806,-7.67558528428094)(1023.23076923077,-7.67558528428094)(1028.17391304348,-7.65551839464883)(1030.64548494983,-7.64548494983278)(1033.11705685619,-7.63545150501672)(1038.0602006689,-7.61538461538462)(1043.00334448161,-7.61538461538462)(1047.94648829431,-7.61538461538462)(1052.88963210702,-7.61538461538462)(1057.83277591973,-7.61538461538462)(1062.77591973244,-7.62290969899666)(1067.71906354515,-7.63795986622074)(1072.66220735786,-7.63946488294314)(1077.60535117057,-7.63946488294314)(1082.54849498328,-7.64046822742475)(1085.02006688963,-7.64548494983278)};

\addplot [fill=darkgray,draw=none,forget plot] coordinates{ (556.103678929766,-6.98327759197324)(558.57525083612,-6.9933110367893)(563.518394648829,-7.0133779264214)(568.461538461538,-7.0133779264214)(573.404682274248,-7.0133779264214)(578.347826086957,-7.03344481605351)(580.819397993311,-7.04347826086957)(583.290969899666,-7.05351170568562)(588.234113712375,-7.09364548494983)(590.705685618729,-7.10367892976589)(593.177257525084,-7.11371237458194)(598.120401337793,-7.15384615384615)(600.591973244147,-7.16387959866221)(603.063545150502,-7.17391304347826)(608.006688963211,-7.21404682274247)(609.242474916388,-7.22408026755853)(612.94983277592,-7.25418060200669)(615.421404682274,-7.28428093645485)(617.892976588629,-7.31438127090301)(620.364548494983,-7.34448160535117)(622.836120401338,-7.37458193979933)(625.307692307692,-7.40468227424749)(627.779264214047,-7.43478260869565)(630.250836120401,-7.46488294314381)(632.722408026756,-7.49498327759197)(635.19397993311,-7.52508361204013)(637.665551839465,-7.55518394648829)(640.137123745819,-7.58528428093646)(642.608695652174,-7.61538461538462)(645.080267558528,-7.64548494983278)(647.551839464883,-7.69063545150502)(648.375696767001,-7.7056856187291)(651.671125975474,-7.76588628762542)(652.494983277592,-7.7809364548495)(654.966555183946,-7.82608695652174)(657.438127090301,-7.87123745819398)(658.261984392419,-7.88628762541806)(661.557413600892,-7.94648829431438)(662.38127090301,-7.97658862876254)(663.205128205128,-8.0066889632107)(664.852842809365,-8.06688963210702)(664.852842809365,-8.12709030100334)(664.852842809365,-8.18729096989967)(664.852842809365,-8.24749163879599)(664.852842809365,-8.30769230769231)(664.852842809365,-8.36789297658863)(664.852842809365,-8.42809364548495)(663.205128205128,-8.48829431438127)(662.38127090301,-8.50334448160535)(659.909698996655,-8.54849498327759)(657.438127090301,-8.57859531772575)(653.730769230769,-8.60869565217391)(652.494983277592,-8.62374581939799)(650.023411371237,-8.66889632107023)(647.551839464883,-8.69899665551839)(643.844481605351,-8.72909698996656)(642.608695652174,-8.73913043478261)(637.665551839465,-8.77926421404682)(636.841694537347,-8.78929765886288)(632.722408026756,-8.83946488294314)(631.486622073579,-8.8494983277592)(627.779264214047,-8.87959866220736)(624.071906354515,-8.90969899665552)(622.836120401338,-8.91722408026756)(617.892976588629,-8.91973244147157)(612.94983277592,-8.93979933110368)(608.006688963211,-8.95986622073579)(605.535117056856,-8.96989966555184)(603.063545150502,-8.97993311036789)(598.120401337793,-9.02006688963211)(595.648829431438,-9.03010033444816)(593.177257525084,-9.04013377926421)(588.234113712375,-9.06020066889632)(583.290969899666,-9.06020066889632)(578.347826086957,-9.08026755852843)(575.876254180602,-9.09030100334448)(573.404682274248,-9.10033444816053)(568.461538461538,-9.12040133779264)(563.518394648829,-9.12040133779264)(558.57525083612,-9.14046822742475)(556.103678929766,-9.1505016722408)(553.632107023411,-9.16053511705686)(548.688963210702,-9.18060200668896)(543.745819397993,-9.18060200668896)(538.802675585284,-9.18060200668896)(533.859531772575,-9.18060200668896)(528.916387959866,-9.20066889632107)(527.680602006689,-9.21070234113712)(523.973244147157,-9.24080267558528)(519.030100334448,-9.26086956521739)(514.086956521739,-9.26086956521739)(509.14381270903,-9.24080267558528)(504.200668896321,-9.24080267558528)(499.257525083612,-9.24080267558528)(494.314381270903,-9.22073578595318)(491.842809364549,-9.21070234113712)(489.371237458194,-9.20066889632107)(484.428093645485,-9.18060200668896)(479.484949832776,-9.18060200668896)(474.541806020067,-9.18060200668896)(469.598662207358,-9.16053511705686)(464.655518394649,-9.18060200668896)(459.71237458194,-9.16053511705686)(454.769230769231,-9.16053511705686)(453.945373467113,-9.1505016722408)(449.826086956522,-9.10033444816053)(447.354515050167,-9.09030100334448)(444.882943143813,-9.08026755852843)(439.939799331104,-9.04013377926421)(437.468227424749,-9.03010033444816)(434.996655518395,-9.02006688963211)(430.053511705686,-8.97993311036789)(427.581939799331,-8.96989966555184)(425.110367892977,-8.95986622073579)(420.167224080268,-8.91973244147157)(418.93143812709,-8.90969899665552)(415.224080267559,-8.87959866220736)(412.752508361204,-8.8494983277592)(410.28093645485,-8.81939799331104)(406.573578595318,-8.78929765886288)(405.33779264214,-8.77926421404682)(400.394648829431,-8.73913043478261)(399.158862876254,-8.72909698996656)(395.451505016722,-8.69899665551839)(392.979933110368,-8.66889632107023)(390.508361204013,-8.63879598662207)(388.036789297659,-8.60869565217391)(385.565217391304,-8.57859531772575)(383.09364548495,-8.54849498327759)(380.622073578595,-8.51839464882943)(376.914715719064,-8.48829431438127)(375.678929765886,-8.47826086956522)(370.735785953177,-8.438127090301)(369.5,-8.42809364548495)(365.792642140468,-8.39799331103679)(362.085284280936,-8.36789297658863)(360.849498327759,-8.35785953177258)(355.90635451505,-8.31772575250836)(353.434782608696,-8.30769230769231)(350.963210702341,-8.29765886287626)(346.020066889632,-8.25752508361204)(343.548494983278,-8.24749163879599)(341.076923076923,-8.23745819397993)(336.133779264214,-8.21739130434783)(331.190635451505,-8.21739130434783)(326.247491638796,-8.23745819397993)(323.775919732441,-8.24749163879599)(321.304347826087,-8.25752508361204)(316.361204013378,-8.27759197324415)(311.418060200669,-8.29765886287626)(308.946488294314,-8.30769230769231)(306.47491638796,-8.31772575250836)(301.531772575251,-8.35785953177258)(299.060200668896,-8.36789297658863)(296.588628762542,-8.37792642140468)(291.645484949833,-8.4180602006689)(289.173913043478,-8.42809364548495)(286.702341137124,-8.438127090301)(281.759197324415,-8.47826086956522)(279.28762541806,-8.48829431438127)(276.816053511706,-8.49832775919732)(271.872909698997,-8.53846153846154)(269.401337792642,-8.54849498327759)(266.929765886288,-8.55852842809364)(261.986622073579,-8.57859531772575)(257.04347826087,-8.59866220735786)(254.571906354515,-8.60869565217391)(252.100334448161,-8.61872909698996)(247.98104793757,-8.66889632107023)(247.157190635452,-8.67892976588629)(242.214046822742,-8.69899665551839)(237.270903010033,-8.7190635451505)(232.327759197324,-8.69899665551839)(227.384615384615,-8.7190635451505)(224.913043478261,-8.72909698996656)(222.441471571906,-8.73913043478261)(217.498327759197,-8.75919732441472)(212.555183946488,-8.75919732441472)(207.612040133779,-8.75919732441472)(202.66889632107,-8.75919732441472)(197.725752508361,-8.75919732441472)(192.782608695652,-8.75919732441472)(187.839464882943,-8.75919732441472)(182.896321070234,-8.75919732441472)(177.953177257525,-8.73913043478261)(175.481605351171,-8.72909698996656)(173.010033444816,-8.7190635451505)(168.066889632107,-8.69899665551839)(163.123745819398,-8.69899665551839)(158.180602006689,-8.67892976588629)(155.709030100334,-8.66889632107023)(153.23745819398,-8.65886287625418)(148.294314381271,-8.61872909698996)(147.058528428094,-8.60869565217391)(143.351170568562,-8.57859531772575)(140.879598662207,-8.54849498327759)(138.408026755853,-8.51839464882943)(135.936454849498,-8.48829431438127)(133.464882943144,-8.45819397993311)(130.993311036789,-8.42809364548495)(128.521739130435,-8.39799331103679)(126.05016722408,-8.36789297658863)(123.578595317726,-8.33779264214047)(121.107023411371,-8.30769230769231)(118.635451505017,-8.26254180602007)(117.811594202899,-8.24749163879599)(114.516164994426,-8.18729096989967)(113.692307692308,-8.1571906354515)(112.86845039019,-8.12709030100334)(109.573021181717,-8.06688963210702)(108.749163879599,-8.03678929765886)(107.92530657748,-8.0066889632107)(106.277591973244,-7.94648829431438)(106.277591973244,-7.88628762541806)(106.277591973244,-7.82608695652174)(107.92530657748,-7.76588628762542)(108.749163879599,-7.73578595317726)(109.573021181717,-7.7056856187291)(113.692307692308,-7.65551839464883)(114.516164994426,-7.64548494983278)(118.635451505017,-7.59531772575251)(121.107023411371,-7.58528428093646)(123.578595317726,-7.5752508361204)(128.521739130435,-7.53511705685619)(130.993311036789,-7.52508361204013)(133.464882943144,-7.51505016722408)(138.408026755853,-7.49498327759197)(143.351170568562,-7.49498327759197)(148.294314381271,-7.47491638795987)(150.765886287625,-7.46488294314381)(153.23745819398,-7.45484949832776)(158.180602006689,-7.43478260869565)(163.123745819398,-7.43478260869565)(168.066889632107,-7.43478260869565)(173.010033444816,-7.43478260869565)(177.953177257525,-7.43478260869565)(182.896321070234,-7.43478260869565)(187.839464882943,-7.45484949832776)(190.311036789298,-7.46488294314381)(192.782608695652,-7.47491638795987)(197.725752508361,-7.49498327759197)(202.66889632107,-7.49498327759197)(207.612040133779,-7.49498327759197)(212.555183946488,-7.49498327759197)(217.498327759197,-7.49498327759197)(222.441471571906,-7.51505016722408)(224.913043478261,-7.52508361204013)(227.384615384615,-7.53511705685619)(232.327759197324,-7.55518394648829)(237.270903010033,-7.55518394648829)(242.214046822742,-7.55518394648829)(247.157190635452,-7.5752508361204)(249.628762541806,-7.58528428093646)(252.100334448161,-7.59531772575251)(257.04347826087,-7.61538461538462)(261.986622073579,-7.61538461538462)(266.929765886288,-7.63545150501672)(269.401337792642,-7.64548494983278)(271.872909698997,-7.65551839464883)(276.816053511706,-7.67558528428094)(281.759197324415,-7.69565217391304)(284.230769230769,-7.7056856187291)(286.702341137124,-7.71571906354515)(291.645484949833,-7.73578595317726)(296.588628762542,-7.75585284280937)(299.060200668896,-7.76588628762542)(301.531772575251,-7.77591973244147)(306.47491638796,-7.79598662207358)(311.418060200669,-7.79598662207358)(316.361204013378,-7.81605351170569)(318.832775919732,-7.82608695652174)(321.304347826087,-7.83612040133779)(326.247491638796,-7.83612040133779)(331.190635451505,-7.83612040133779)(333.66220735786,-7.82608695652174)(336.133779264214,-7.8185618729097)(341.076923076923,-7.81605351170569)(346.020066889632,-7.79598662207358)(350.963210702341,-7.79598662207358)(355.90635451505,-7.77591973244147)(358.377926421405,-7.76588628762542)(360.849498327759,-7.75585284280937)(365.792642140468,-7.71571906354515)(368.264214046823,-7.7056856187291)(370.735785953177,-7.69565217391304)(375.678929765886,-7.65551839464883)(378.150501672241,-7.64548494983278)(380.622073578595,-7.63545150501672)(385.565217391304,-7.59531772575251)(388.036789297659,-7.58528428093646)(390.508361204013,-7.5752508361204)(395.451505016722,-7.53511705685619)(397.923076923077,-7.52508361204013)(400.394648829431,-7.51505016722408)(405.33779264214,-7.47491638795987)(407.809364548495,-7.46488294314381)(410.28093645485,-7.45484949832776)(415.224080267559,-7.41471571906355)(417.695652173913,-7.40468227424749)(420.167224080268,-7.39464882943144)(425.110367892977,-7.35451505016722)(427.581939799331,-7.34448160535117)(430.053511705686,-7.33444816053512)(434.996655518395,-7.2943143812709)(437.468227424749,-7.28428093645485)(439.939799331104,-7.2742474916388)(444.882943143813,-7.23411371237458)(447.354515050167,-7.22408026755853)(449.826086956522,-7.21404682274247)(454.769230769231,-7.19397993311037)(459.71237458194,-7.17391304347826)(462.183946488294,-7.16387959866221)(464.655518394649,-7.15384615384615)(469.598662207358,-7.13377926421405)(474.541806020067,-7.11371237458194)(477.013377926421,-7.10367892976589)(479.484949832776,-7.09364548494983)(484.428093645485,-7.07357859531773)(489.371237458194,-7.05351170568562)(491.842809364549,-7.04347826086957)(494.314381270903,-7.03344481605351)(499.257525083612,-7.0133779264214)(504.200668896321,-7.0133779264214)(509.14381270903,-7.0133779264214)(514.086956521739,-7.0133779264214)(519.030100334448,-6.9933110367893)(521.501672240803,-6.98327759197324)(523.973244147157,-6.97324414715719)(528.916387959866,-6.95317725752508)(533.859531772575,-6.95317725752508)(538.802675585284,-6.95317725752508)(543.745819397993,-6.95317725752508)(548.688963210702,-6.95317725752508)(553.632107023411,-6.97324414715719)(556.103678929766,-6.98327759197324)};

\addplot [fill=red!40!yellow,draw=none,forget plot] coordinates{ (551.160535117057,-7.28428093645485)(553.632107023411,-7.2943143812709)(558.57525083612,-7.31438127090301)(563.518394648829,-7.33444816053512)(565.989966555184,-7.34448160535117)(568.461538461538,-7.35451505016722)(573.404682274248,-7.37458193979933)(578.347826086957,-7.39464882943144)(579.583612040134,-7.40468227424749)(583.290969899666,-7.43478260869565)(585.76254180602,-7.46488294314381)(588.234113712375,-7.49498327759197)(590.705685618729,-7.52508361204013)(593.177257525084,-7.55518394648829)(595.648829431438,-7.58528428093646)(598.120401337793,-7.61538461538462)(600.591973244147,-7.64548494983278)(603.063545150502,-7.69063545150502)(603.88740245262,-7.7056856187291)(607.182831661092,-7.76588628762542)(608.006688963211,-7.7809364548495)(610.478260869565,-7.82608695652174)(612.94983277592,-7.87123745819398)(613.773690078038,-7.88628762541806)(615.421404682274,-7.94648829431438)(615.421404682274,-8.0066889632107)(615.421404682274,-8.06688963210702)(615.421404682274,-8.12709030100334)(615.421404682274,-8.18729096989967)(613.773690078038,-8.24749163879599)(612.94983277592,-8.27759197324415)(612.125975473802,-8.30769230769231)(608.830546265329,-8.36789297658863)(608.006688963211,-8.38294314381271)(605.535117056856,-8.42809364548495)(603.063545150502,-8.45819397993311)(600.591973244147,-8.48829431438127)(598.120401337793,-8.51839464882943)(594.413043478261,-8.54849498327759)(593.177257525084,-8.55852842809364)(588.234113712375,-8.59866220735786)(586.998327759197,-8.60869565217391)(583.290969899666,-8.63879598662207)(579.583612040134,-8.66889632107023)(578.347826086957,-8.67892976588629)(573.404682274248,-8.7190635451505)(570.933110367893,-8.72909698996656)(568.461538461538,-8.73913043478261)(563.518394648829,-8.77926421404682)(561.046822742475,-8.78929765886288)(558.57525083612,-8.79933110367893)(553.632107023411,-8.81939799331104)(548.688963210702,-8.83946488294314)(546.217391304348,-8.8494983277592)(543.745819397993,-8.85953177257525)(538.802675585284,-8.87959866220736)(533.859531772575,-8.87959866220736)(528.916387959866,-8.87959866220736)(523.973244147157,-8.87959866220736)(519.030100334448,-8.87959866220736)(514.086956521739,-8.87959866220736)(509.14381270903,-8.87959866220736)(504.200668896321,-8.85953177257525)(501.729096989967,-8.8494983277592)(499.257525083612,-8.83946488294314)(494.314381270903,-8.81939799331104)(489.371237458194,-8.79933110367893)(486.899665551839,-8.78929765886288)(484.428093645485,-8.77926421404682)(479.484949832776,-8.73913043478261)(477.013377926421,-8.72909698996656)(474.541806020067,-8.7190635451505)(469.598662207358,-8.67892976588629)(468.362876254181,-8.66889632107023)(464.655518394649,-8.63879598662207)(462.183946488294,-8.60869565217391)(459.71237458194,-8.57859531772575)(457.240802675585,-8.54849498327759)(454.769230769231,-8.51839464882943)(452.297658862876,-8.48829431438127)(449.826086956522,-8.45819397993311)(447.354515050167,-8.42809364548495)(444.882943143813,-8.39799331103679)(442.411371237458,-8.36789297658863)(439.939799331104,-8.32274247491639)(439.115942028986,-8.30769230769231)(437.468227424749,-8.24749163879599)(437.468227424749,-8.18729096989967)(437.468227424749,-8.12709030100334)(437.468227424749,-8.06688963210702)(437.468227424749,-8.0066889632107)(437.468227424749,-7.94648829431438)(439.115942028986,-7.88628762541806)(439.939799331104,-7.8561872909699)(440.763656633222,-7.82608695652174)(444.059085841695,-7.76588628762542)(444.882943143813,-7.75083612040134)(447.354515050167,-7.7056856187291)(449.826086956522,-7.68311036789298)(453.945373467113,-7.64548494983278)(454.769230769231,-7.63545150501672)(458.888517279822,-7.58528428093646)(459.71237458194,-7.5752508361204)(463.831661092531,-7.52508361204013)(464.655518394649,-7.51505016722408)(469.598662207358,-7.47491638795987)(472.070234113712,-7.46488294314381)(474.541806020067,-7.45484949832776)(479.484949832776,-7.41471571906355)(481.95652173913,-7.40468227424749)(484.428093645485,-7.39464882943144)(489.371237458194,-7.35451505016722)(491.842809364549,-7.34448160535117)(494.314381270903,-7.33444816053512)(499.257525083612,-7.31438127090301)(504.200668896321,-7.31438127090301)(509.14381270903,-7.2943143812709)(511.615384615385,-7.28428093645485)(514.086956521739,-7.2742474916388)(519.030100334448,-7.25418060200669)(523.973244147157,-7.25418060200669)(528.916387959866,-7.25418060200669)(533.859531772575,-7.25418060200669)(538.802675585284,-7.25418060200669)(543.745819397993,-7.25418060200669)(548.688963210702,-7.2742474916388)(551.160535117057,-7.28428093645485)};

\addplot [fill=darkgray,draw=none,forget plot] coordinates{ (1478.00006407589,0)(1478,-0.0526755852842809)(1473.67474916388,0)(1478,7.80355919588601e-07)(1478.00006407589,0)};

\addplot [
color=white,
draw=white,
only marks,
mark=x,
mark options={solid},
mark size=2.0pt,
line width=0.3pt,
forget plot
]
coordinates{
 (370.735785953177,-4.75585284280936)(1478,-18)(0,-15.8327759197324)(1478,0)(1478,-9.27090301003344)(0,0)(588.234113712375,-11.1973244147157)(696.983277591973,-18)(1156.69565217391,-4.63545150501672)(0,-9.03010033444816)(741.471571906354,0)(1250.61538461538,-13.7257525083612)(830.448160535117,-7.82608695652174)(0,-12.4615384615385)(608.006688963211,-14.628762541806)(0,-3.67224080267559)(1478,-5.95986622073579)(207.612040133779,-18)(1057.83277591973,-10.8361204013378)(1478,-2.82943143812709)(751.357859531773,-3.01003344481605)(1077.60535117057,-16.6153846153846)(385.565217391304,-7.88628762541806)(355.90635451505,-1.38461538461538)(0,-6.38127090301003)(1478,-12.0401337792642)(1112.20735785953,-1.02341137123746)(1478,-15.4113712374582)(286.702341137124,-13.7859531772575)(1186.35451505017,-7.64548494983278)(751.357859531773,-5.53846153846154)(884.822742474916,-13.1839464882943)(252.100334448161,-10.7157190635452)(652.494983277592,-9.39130434782609)(1211.07023411371,-9.51170568561873)(528.916387959866,-6.50167224080268)(0,-10.6555183946488)(1478,-7.64548494983278)(1018.28762541806,-6.14046822742475)(232.327759197324,-6.32107023411371)(1240.72909698997,-11.7391304347826)(914.481605351171,-9.51170568561873)(247.157190635452,-9.09030100334448)(825.505016722408,-10.9565217391304)(1478,-10.6555183946488)(1270.38795986622,-6.02006688963211)(0,-7.7056856187291)(444.882943143813,-9.81270903010033)(612.94983277592,-7.88628762541806)(1028.17391304348,-8.30769230769231)(177.953177257525,-7.7056856187291)(790.903010033445,-6.68227424749164)(1329.70568561873,-8.42809364548495)(1304.98996655518,-10.5351170568562)(385.565217391304,-6.44147157190635)(1344.53511705686,-7.16387959866221)(1003.45819397993,-7.16387959866221)(484.428093645485,-8.72909698996656)(133.464882943144,-9.45150501672241)(741.471571906354,-10.0535117056856)(1067.71906354515,-9.63210702341137)(781.016722408027,-8.72909698996656)(647.551839464883,-6.62207357859532)(1478,-8.48829431438127)(1146.8093645485,-6.8628762541806)(118.635451505017,-6.80267558528428)(1359.36454849498,-9.63210702341137)(1166.58193979933,-8.60869565217391)(123.578595317726,-8.42809364548495)(1176.46822742475,-10.5953177257525)(341.076923076923,-9.51170568561873)(954.026755852843,-10.3545150501672)(484.428093645485,-7.40468227424749)(291.645484949833,-7.28428093645485)(1478,-10.0535117056856)(924.367892976589,-8.60869565217391)(711.8127090301,-7.58528428093646)(558.57525083612,-9.39130434782609)(914.481605351171,-7.16387959866221)(311.418060200669,-8.42809364548495)(14.8294314381271,-7.34448160535117)(647.551839464883,-8.54849498327759)(1319.81939799331,-7.76588628762542)(815.61872909699,-9.69230769230769)(1314.8762541806,-9.03010033444816)(1364.30769230769,-10.4147157190635)(1057.83277591973,-9.03010033444816)(1072.66220735786,-7.7056856187291)(49.4314381270903,-8.24749163879599)(499.257525083612,-8.18729096989967)(1478,-8.8494983277592)(395.451505016722,-6.98327759197324)(1146.8093645485,-10.0535117056856)(603.063545150502,-7.16387959866221)(182.896321070234,-6.98327759197324)(1216.01337792642,-7.22408026755853)(395.451505016722,-8.96989966555184)(909.538461538462,-7.94648829431438)(227.384615384615,-8.24749163879599)(919.42474916388,-10.1739130434783)(1478,-10.1137123745819)(761.244147157191,-8.96989966555184)(1216.01337792642,-8.42809364548495)(138.408026755853,-8.96989966555184)(1389.02341137124,-7.94648829431438)(1067.71906354515,-7.04347826086957)(761.244147157191,-7.7056856187291)(1265.44481605351,-9.87290969899666)(954.026755852843,-9.1505016722408)(1047.94648829431,-10.2943143812709)(69.2040133779264,-7.04347826086957)(553.632107023411,-9.21070234113712)(499.257525083612,-6.8628762541806)(919.42474916388,-7.16387959866221)(306.47491638796,-7.58528428093646)(1478,-9.45150501672241)(1146.8093645485,-9.1505016722408)(603.063545150502,-8.36789297658863)(1364.30769230769,-8.96989966555184)(1092.4347826087,-8.12709030100334)(123.578595317726,-7.76588628762542)(1374.19397993311,-10.4147157190635)(850.220735785953,-8.60869565217391)(395.451505016722,-8.42809364548495)(840.334448160535,-9.87290969899666)(222.441471571906,-8.96989966555184)(1260.5016722408,-7.40468227424749)(632.722408026756,-7.22408026755853)(711.8127090301,-9.21070234113712)(1216.01337792642,-10.2943143812709)(479.484949832776,-7.64548494983278)(444.882943143813,-9.21070234113712)(726.642140468227,-8.24749163879599)(973.799331103679,-7.94648829431438)(1013.34448160535,-9.63210702341137)(69.2040133779264,-8.30769230769231)(1413.73913043478,-8.30769230769231)(227.384615384615,-7.28428093645485)(1245.67224080268,-8.24749163879599)(860.107023411371,-7.46488294314381)(1250.61538461538,-9.21070234113712)(1102.32107023411,-7.16387959866221)(380.622073578595,-7.22408026755853)(1478,-9.9933110367893)(1092.4347826087,-10.2943143812709)(1033.11705685619,-8.72909698996656)(301.531772575251,-8.72909698996656)(553.632107023411,-6.8628762541806)(1354.42140468227,-9.57190635451505)(860.107023411371,-9.1505016722408)(1478,-9.09030100334448)(588.234113712375,-9.21070234113712)(968.85618729097,-10.2341137123746)(237.270903010033,-8.06688963210702)(84.0334448160535,-7.28428093645485)(1280.27424749164,-10.3545150501672)(603.063545150502,-7.76588628762542)(153.23745819398,-8.90969899665552)(1329.70568561873,-7.7056856187291)(1151.7525083612,-8.0066889632107)(439.939799331104,-8.18729096989967)(1136.92307692308,-9.45150501672241)(983.685618729097,-7.22408026755853)(805.732441471572,-8.18729096989967)(573.404682274248,-8.60869565217391)(459.71237458194,-9.21070234113712)(790.903010033445,-9.45150501672241)(484.428093645485,-6.8628762541806)(944.140468227425,-8.42809364548495)(1354.42140468227,-8.60869565217391)(652.494983277592,-7.40468227424749)(682.153846153846,-8.54849498327759)(879.879598662207,-9.93311036789298)(331.190635451505,-7.88628762541806)(1117.15050167224,-8.72909698996656)(1393.96655518395,-9.81270903010033)(1186.35451505017,-10.2943143812709)(1166.58193979933,-7.40468227424749)(988.628762541806,-9.39130434782609)(123.578595317726,-7.82608695652174)(1240.72909698997,-8.90969899665552)(1018.28762541806,-7.82608695652174)(860.107023411371,-7.58528428093646)(375.678929765886,-8.72909698996656)(454.769230769231,-7.58528428093646)(227.384615384615,-7.52508361204013)(1057.83277591973,-10.2341137123746)(1255.55852842809,-8.0066889632107)(128.521739130435,-8.48829431438127)(1408.79598662207,-8.48829431438127)(874.936454849498,-8.8494983277592)(1379.13712374582,-10.2341137123746)(1255.55852842809,-9.63210702341137)(519.030100334448,-9.21070234113712)(242.214046822742,-8.66889632107023)(998.515050167224,-7.22408026755853)(766.1872909699,-8.30769230769231)(1379.13712374582,-9.33110367892977)(1112.20735785953,-9.57190635451505)(627.779264214047,-8.8494983277592) 
};

%\node at (axis cs:50, -2.5) [shape=circle,fill=white,draw=black,inner sep=0pt,anchor=south west] {\scriptsize\color{locol}$\*L_{\*t}$};
%\node at (axis cs:460, -5.9) [shape=circle,fill=white,draw=black,inner sep=0pt,anchor=south west] {\scriptsize\color{orange!50!yellow}$\*H_{\*t}$};
%\node at (axis cs:160, -5.2) [shape=circle,fill=white,draw=black,inner sep=0pt,anchor=south west] {\scriptsize\color{darkgray}$\*U_{\*t}$};

\node at (axis cs:805, -17) [shape=circle,fill=green!60!black,draw=black,inner sep=0.2pt,anchor=south west,minimum size=16pt]
  {\scriptsize\color{white}$\*M_{\*t}$};
\node at (axis cs:980, -17) [shape=circle,fill=red!40!yellow,draw=black,inner sep=0.2pt,anchor=south west,minimum size=16pt]
  {\scriptsize\color{white}$\*H_{\*t}$};
\node at (axis cs:1155, -17) [shape=circle,fill=locol,draw=black,inner sep=0.2pt,anchor=south west,minimum size=16pt]
  {\scriptsize\color{white}$\*L_{\*t}$};
\node at (axis cs:1330, -17) [shape=circle,fill=darkgray,draw=black,inner sep=0.2pt,anchor=south west,minimum size=16pt]
  {\scriptsize\color{white}$\*U_{\*t}$};

\end{axis}
\end{tikzpicture}%

%% This file was created by matlab2tikz v0.2.3.
% Copyright (c) 2008--2012, Nico Schlömer <nico.schloemer@gmail.com>
% All rights reserved.
% 
% 
%

\definecolor{locol}{rgb}{0.26, 0.45, 0.65}

\begin{tikzpicture}

\begin{axis}[%
tick label style={font=\tiny},
label style={font=\tiny},
xlabel shift={-10pt},
ylabel shift={-17pt},
legend style={font=\tiny},
view={0}{90},
width=\figurewidth,
height=\figureheight,
scale only axis,
xmin=0, xmax=1478,
xtick={0, 400, 1000, 1400},
xlabel={Length (m)},
ymin=-18, ymax=0,
ytick={0, -4, -14, -18},
ylabel={Depth (m)},
name=plot1,
axis lines*=box,
tickwidth=0.1cm,
clip=false
]

\addplot [fill=locol,draw=none,forget plot] coordinates{ (1478.00014645918,0)(1478.00007323068,-0.0602006688963211)(1478,-0.120401337792642)(1478,-0.180602006688963)(1478,-0.240802675585284)(1478,-0.301003344481605)(1478,-0.361204013377926)(1478,-0.421404682274247)(1478,-0.481605351170569)(1478,-0.54180602006689)(1478,-0.602006688963211)(1478,-0.662207357859532)(1478,-0.722408026755853)(1478,-0.782608695652174)(1478,-0.842809364548495)(1478,-0.903010033444816)(1478,-0.963210702341137)(1478,-1.02341137123746)(1478,-1.08361204013378)(1478,-1.1438127090301)(1478,-1.20401337792642)(1478,-1.26421404682274)(1478,-1.32441471571906)(1478,-1.38461538461538)(1478,-1.44481605351171)(1478,-1.50501672240803)(1478,-1.56521739130435)(1478,-1.62541806020067)(1478,-1.68561872909699)(1478,-1.74581939799331)(1478,-1.80602006688963)(1478,-1.86622073578595)(1478,-1.92642140468227)(1478,-1.9866220735786)(1478,-2.04682274247492)(1478,-2.10702341137124)(1478,-2.16722408026756)(1478,-2.22742474916388)(1478,-2.2876254180602)(1478,-2.34782608695652)(1478,-2.40802675585284)(1478,-2.46822742474916)(1478,-2.52842809364549)(1478,-2.58862876254181)(1478,-2.64882943143813)(1478,-2.70903010033445)(1478,-2.76923076923077)(1478,-2.82943143812709)(1478,-2.88963210702341)(1478,-2.94983277591973)(1478,-3.01003344481605)(1478,-3.07023411371237)(1478,-3.1304347826087)(1478,-3.19063545150502)(1478,-3.25083612040134)(1478,-3.31103678929766)(1478,-3.37123745819398)(1478,-3.4314381270903)(1478,-3.49163879598662)(1478,-3.55183946488294)(1478,-3.61204013377926)(1478,-3.67224080267559)(1478,-3.73244147157191)(1478,-3.79264214046823)(1478,-3.85284280936455)(1478,-3.91304347826087)(1478,-3.97324414715719)(1478,-4.03344481605351)(1478,-4.09364548494983)(1478,-4.15384615384615)(1478,-4.21404682274247)(1478,-4.2742474916388)(1478,-4.33444816053512)(1478,-4.39464882943144)(1478,-4.45484949832776)(1478,-4.51505016722408)(1478,-4.5752508361204)(1478,-4.63545150501672)(1478,-4.69565217391304)(1478,-4.75585284280936)(1478,-4.81605351170569)(1478,-4.87625418060201)(1478,-4.93645484949833)(1478,-4.99665551839465)(1478,-5.05685618729097)(1478,-5.11705685618729)(1478,-5.17725752508361)(1478,-5.23745819397993)(1478,-5.29765886287625)(1478,-5.35785953177258)(1478,-5.4180602006689)(1478,-5.47826086956522)(1478,-5.53846153846154)(1478,-5.59866220735786)(1478,-5.65886287625418)(1478,-5.7190635451505)(1478,-5.77926421404682)(1478,-5.83946488294314)(1478,-5.89966555183946)(1478,-5.95986622073579)(1478,-6.02006688963211)(1478,-6.08026755852843)(1478,-6.14046822742475)(1478,-6.20066889632107)(1478,-6.26086956521739)(1478,-6.32107023411371)(1478,-6.38127090301003)(1478,-6.44147157190635)(1478,-6.50167224080268)(1478,-6.561872909699)(1478,-6.62207357859532)(1478,-6.68227424749164)(1478,-6.74247491638796)(1478,-6.80267558528428)(1478,-6.8628762541806)(1478,-6.92307692307692)(1478,-6.98327759197324)(1478,-7.04347826086957)(1478,-7.10367892976589)(1478,-7.16387959866221)(1478,-7.22408026755853)(1478,-7.28428093645485)(1478,-7.34448160535117)(1478,-7.40468227424749)(1478,-7.46488294314381)(1478,-7.52508361204013)(1478,-7.58528428093646)(1478,-7.64548494983278)(1478,-7.7056856187291)(1478,-7.76588628762542)(1478,-7.82608695652174)(1478,-7.88628762541806)(1478,-7.94648829431438)(1478,-8.0066889632107)(1478,-8.06688963210702)(1478,-8.12709030100334)(1478,-8.18729096989967)(1478,-8.24749163879599)(1478,-8.30769230769231)(1478,-8.36789297658863)(1478,-8.42809364548495)(1478,-8.48829431438127)(1478,-8.54849498327759)(1478,-8.60869565217391)(1478,-8.66889632107023)(1478,-8.72909698996656)(1478,-8.78929765886288)(1478,-8.8494983277592)(1478,-8.90969899665552)(1478,-8.96989966555184)(1478,-9.03010033444816)(1478,-9.09030100334448)(1478,-9.1505016722408)(1478.00007323068,-9.21070234113712)(1478.00007323068,-9.27090301003344)(1478.00007323068,-9.33110367892977)(1478,-9.39130434782609)(1478,-9.45150501672241)(1478,-9.51170568561873)(1478,-9.57190635451505)(1478,-9.63210702341137)(1478,-9.69230769230769)(1478,-9.75250836120401)(1478,-9.81270903010033)(1478,-9.87290969899666)(1478,-9.93311036789298)(1478,-9.9933110367893)(1478,-10.0535117056856)(1478,-10.1137123745819)(1478,-10.1739130434783)(1478,-10.2341137123746)(1478,-10.2943143812709)(1478,-10.3545150501672)(1478,-10.4147157190635)(1478,-10.4749163879599)(1478,-10.5351170568562)(1478,-10.5953177257525)(1478,-10.6555183946488)(1478,-10.7157190635452)(1478,-10.7759197324415)(1478,-10.8361204013378)(1478,-10.8963210702341)(1478,-10.9565217391304)(1478,-11.0167224080268)(1478,-11.0769230769231)(1478,-11.1371237458194)(1478,-11.1973244147157)(1478,-11.257525083612)(1478,-11.3177257525084)(1478,-11.3779264214047)(1478,-11.438127090301)(1478,-11.4983277591973)(1478,-11.5585284280936)(1478,-11.61872909699)(1478,-11.6789297658863)(1478,-11.7391304347826)(1478,-11.7993311036789)(1478,-11.8595317725753)(1478,-11.9197324414716)(1478,-11.9799331103679)(1478,-12.0401337792642)(1478,-12.1003344481605)(1478,-12.1605351170569)(1478,-12.2207357859532)(1478,-12.2809364548495)(1478,-12.3411371237458)(1478,-12.4013377926421)(1478,-12.4615384615385)(1478,-12.5217391304348)(1478,-12.5819397993311)(1478,-12.6421404682274)(1478,-12.7023411371237)(1478,-12.7625418060201)(1478,-12.8227424749164)(1478,-12.8829431438127)(1478,-12.943143812709)(1478,-13.0033444816054)(1478,-13.0635451505017)(1478,-13.123745819398)(1478,-13.1839464882943)(1478,-13.2441471571906)(1478,-13.304347826087)(1478,-13.3645484949833)(1478,-13.4247491638796)(1478,-13.4849498327759)(1478,-13.5451505016722)(1478,-13.6053511705686)(1478,-13.6655518394649)(1478,-13.7257525083612)(1478,-13.7859531772575)(1478,-13.8461538461538)(1478,-13.9063545150502)(1478,-13.9665551839465)(1478,-14.0267558528428)(1478,-14.0869565217391)(1478,-14.1471571906355)(1478,-14.2073578595318)(1478,-14.2675585284281)(1478,-14.3277591973244)(1478,-14.3879598662207)(1478,-14.4481605351171)(1478,-14.5083612040134)(1478,-14.5685618729097)(1478,-14.628762541806)(1478,-14.6889632107023)(1478,-14.7491638795987)(1478,-14.809364548495)(1478,-14.8695652173913)(1478,-14.9297658862876)(1478,-14.9899665551839)(1478,-15.0501672240803)(1478,-15.1103678929766)(1478,-15.1705685618729)(1478,-15.2307692307692)(1478,-15.2909698996656)(1478,-15.3511705685619)(1478,-15.4113712374582)(1478,-15.4715719063545)(1478,-15.5317725752508)(1478,-15.5919732441472)(1478,-15.6521739130435)(1478,-15.7123745819398)(1478,-15.7725752508361)(1478,-15.8327759197324)(1478,-15.8929765886288)(1478,-15.9531772575251)(1478,-16.0133779264214)(1478,-16.0735785953177)(1478,-16.133779264214)(1478,-16.1939799331104)(1478,-16.2541806020067)(1478,-16.314381270903)(1478,-16.3745819397993)(1478,-16.4347826086957)(1478,-16.494983277592)(1478,-16.5551839464883)(1478,-16.6153846153846)(1478,-16.6755852842809)(1478,-16.7357859531773)(1478,-16.7959866220736)(1478,-16.8561872909699)(1478,-16.9163879598662)(1478,-16.9765886287625)(1478,-17.0367892976589)(1478,-17.0969899665552)(1478,-17.1571906354515)(1478,-17.2173913043478)(1478,-17.2775919732441)(1478,-17.3377926421405)(1478,-17.3979933110368)(1478,-17.4581939799331)(1478,-17.5183946488294)(1478,-17.5785953177258)(1478,-17.6387959866221)(1478,-17.6989966555184)(1478,-17.7591973244147)(1478,-17.819397993311)(1478,-17.8795986622074)(1478,-17.9397993311037)(1478,-18)(1473.05685618729,-18)(1468.11371237458,-18)(1463.17056856187,-18)(1458.22742474916,-18)(1453.28428093645,-18)(1448.34113712375,-18)(1443.39799331104,-18)(1438.45484949833,-18)(1433.51170568562,-18)(1428.56856187291,-18)(1423.6254180602,-18)(1418.68227424749,-18)(1413.73913043478,-18)(1408.79598662207,-18)(1403.85284280936,-18)(1398.90969899666,-18)(1393.96655518395,-18)(1389.02341137124,-18)(1384.08026755853,-18)(1379.13712374582,-18)(1374.19397993311,-18)(1369.2508361204,-18)(1364.30769230769,-18)(1359.36454849498,-18)(1354.42140468227,-18)(1349.47826086957,-18)(1344.53511705686,-18)(1339.59197324415,-18)(1334.64882943144,-18)(1329.70568561873,-18)(1324.76254180602,-18)(1319.81939799331,-18)(1314.8762541806,-18)(1309.93311036789,-18)(1304.98996655518,-18)(1300.04682274247,-18)(1295.10367892977,-18)(1290.16053511706,-18)(1285.21739130435,-18)(1280.27424749164,-18)(1275.33110367893,-18)(1270.38795986622,-18)(1265.44481605351,-18)(1260.5016722408,-18)(1255.55852842809,-18)(1250.61538461538,-18)(1245.67224080268,-18)(1240.72909698997,-18)(1235.78595317726,-18)(1230.84280936455,-18)(1225.89966555184,-18)(1220.95652173913,-18)(1216.01337792642,-18)(1211.07023411371,-18)(1206.127090301,-18)(1201.18394648829,-18)(1196.24080267559,-18)(1191.29765886288,-18)(1186.35451505017,-18)(1181.41137123746,-18)(1176.46822742475,-18)(1171.52508361204,-18)(1166.58193979933,-18)(1161.63879598662,-18)(1156.69565217391,-18)(1151.7525083612,-18)(1146.8093645485,-18)(1141.86622073579,-18)(1136.92307692308,-18)(1131.97993311037,-18)(1127.03678929766,-18)(1122.09364548495,-18)(1117.15050167224,-18)(1112.20735785953,-18)(1107.26421404682,-18)(1102.32107023411,-18)(1097.3779264214,-18)(1092.4347826087,-18)(1087.49163879599,-18)(1082.54849498328,-18)(1077.60535117057,-18)(1072.66220735786,-18)(1067.71906354515,-18)(1062.77591973244,-18)(1057.83277591973,-18)(1052.88963210702,-18)(1047.94648829431,-18)(1043.00334448161,-18)(1038.0602006689,-18)(1033.11705685619,-18)(1028.17391304348,-18)(1023.23076923077,-18)(1018.28762541806,-18)(1013.34448160535,-18)(1008.40133779264,-18)(1003.45819397993,-18)(998.515050167224,-18)(993.571906354515,-18)(988.628762541806,-18)(983.685618729097,-18)(978.742474916388,-18)(973.799331103679,-18)(968.85618729097,-18)(963.913043478261,-18)(958.969899665552,-18)(954.026755852843,-18)(949.083612040134,-18)(944.140468227425,-18)(939.197324414716,-18)(934.254180602007,-18)(929.311036789298,-18)(924.367892976589,-18)(919.42474916388,-18)(914.481605351171,-18)(909.538461538462,-18)(904.595317725752,-18)(899.652173913044,-18)(894.709030100334,-18)(889.765886287625,-18)(884.822742474916,-18)(879.879598662207,-18)(874.936454849498,-18)(869.993311036789,-18)(865.05016722408,-18)(860.107023411371,-18)(855.163879598662,-18)(850.220735785953,-18)(845.277591973244,-18)(840.334448160535,-18)(835.391304347826,-18)(830.448160535117,-18)(825.505016722408,-18)(820.561872909699,-18)(815.61872909699,-18)(810.675585284281,-18)(805.732441471572,-18)(800.789297658863,-18)(795.846153846154,-18)(790.903010033445,-18)(785.959866220736,-18)(781.016722408027,-18)(776.073578595318,-18)(771.130434782609,-18)(766.1872909699,-18)(761.244147157191,-18)(756.301003344482,-18)(751.357859531773,-18)(746.414715719064,-18)(741.471571906354,-18)(736.528428093646,-18)(731.585284280936,-18)(726.642140468227,-18)(721.698996655518,-18)(716.755852842809,-18)(711.8127090301,-18)(706.869565217391,-18)(701.926421404682,-18)(696.983277591973,-18)(692.040133779264,-18)(687.096989966555,-18)(682.153846153846,-18)(677.210702341137,-18)(672.267558528428,-18)(667.324414715719,-18)(662.38127090301,-18)(657.438127090301,-18)(652.494983277592,-18)(647.551839464883,-18)(642.608695652174,-18)(637.665551839465,-18)(632.722408026756,-18)(627.779264214047,-18)(622.836120401338,-18)(617.892976588629,-18)(612.94983277592,-18)(608.006688963211,-18)(603.063545150502,-18)(598.120401337793,-18)(593.177257525084,-18)(588.234113712375,-18)(583.290969899666,-18)(578.347826086957,-18)(573.404682274248,-18)(568.461538461538,-18)(563.518394648829,-18)(558.57525083612,-18)(553.632107023411,-18)(548.688963210702,-18)(543.745819397993,-18)(538.802675585284,-18)(533.859531772575,-18)(528.916387959866,-18)(523.973244147157,-18)(519.030100334448,-18)(514.086956521739,-18)(509.14381270903,-18)(504.200668896321,-18)(499.257525083612,-18)(494.314381270903,-18)(489.371237458194,-18)(484.428093645485,-18)(479.484949832776,-18)(474.541806020067,-18)(469.598662207358,-18)(464.655518394649,-18)(459.71237458194,-18)(454.769230769231,-18)(449.826086956522,-18)(444.882943143813,-18)(439.939799331104,-18)(434.996655518395,-18)(430.053511705686,-18)(425.110367892977,-18)(420.167224080268,-18)(415.224080267559,-18)(410.28093645485,-18)(405.33779264214,-18)(400.394648829431,-18)(395.451505016722,-18)(390.508361204013,-18)(385.565217391304,-18)(380.622073578595,-18)(375.678929765886,-18)(370.735785953177,-18)(365.792642140468,-18)(360.849498327759,-18)(355.90635451505,-18)(350.963210702341,-18)(346.020066889632,-18)(341.076923076923,-18)(336.133779264214,-18)(331.190635451505,-18)(326.247491638796,-18)(321.304347826087,-18)(316.361204013378,-18)(311.418060200669,-18)(306.47491638796,-18)(301.531772575251,-18)(296.588628762542,-18)(291.645484949833,-18)(286.702341137124,-18)(281.759197324415,-18)(276.816053511706,-18)(271.872909698997,-18)(266.929765886288,-18)(261.986622073579,-18)(257.04347826087,-18)(252.100334448161,-18)(247.157190635452,-18)(242.214046822742,-18)(237.270903010033,-18)(232.327759197324,-18)(227.384615384615,-18)(222.441471571906,-18)(217.498327759197,-18)(212.555183946488,-18)(207.612040133779,-18)(202.66889632107,-18)(197.725752508361,-18)(192.782608695652,-18)(187.839464882943,-18)(182.896321070234,-18)(177.953177257525,-18)(173.010033444816,-18)(168.066889632107,-18)(163.123745819398,-18)(158.180602006689,-18)(153.23745819398,-18)(148.294314381271,-18)(143.351170568562,-18)(138.408026755853,-18)(133.464882943144,-18)(128.521739130435,-18)(123.578595317726,-18)(118.635451505017,-18)(113.692307692308,-18)(108.749163879599,-18)(103.80602006689,-18)(98.8628762541806,-18)(93.9197324414716,-18)(88.9765886287625,-18)(84.0334448160535,-18)(79.0903010033445,-18)(74.1471571906355,-18)(69.2040133779264,-18)(64.2608695652174,-18)(59.3177257525084,-18)(54.3745819397993,-18)(49.4314381270903,-18)(44.4882943143813,-18)(39.5451505016722,-18)(34.6020066889632,-18)(29.6588628762542,-18)(24.7157190635452,-18)(19.7725752508361,-18)(14.8294314381271,-18)(9.88628762541806,-18)(4.94314381270903,-18)(0,-18)(0,-17.9397993311037)(0,-17.8795986622074)(0,-17.819397993311)(0,-17.7591973244147)(0,-17.6989966555184)(0,-17.6387959866221)(0,-17.5785953177258)(0,-17.5183946488294)(0,-17.4581939799331)(0,-17.3979933110368)(0,-17.3377926421405)(0,-17.2775919732441)(0,-17.2173913043478)(0,-17.1571906354515)(0,-17.0969899665552)(0,-17.0367892976589)(0,-16.9765886287625)(0,-16.9163879598662)(0,-16.8561872909699)(0,-16.7959866220736)(0,-16.7357859531773)(0,-16.6755852842809)(0,-16.6153846153846)(0,-16.5551839464883)(0,-16.494983277592)(0,-16.4347826086957)(0,-16.3745819397993)(0,-16.314381270903)(0,-16.2541806020067)(0,-16.1939799331104)(0,-16.133779264214)(0,-16.0735785953177)(0,-16.0133779264214)(0,-15.9531772575251)(0,-15.8929765886288)(0,-15.8327759197324)(0,-15.7725752508361)(0,-15.7123745819398)(0,-15.6521739130435)(0,-15.5919732441472)(0,-15.5317725752508)(0,-15.4715719063545)(0,-15.4113712374582)(0,-15.3511705685619)(0,-15.2909698996656)(0,-15.2307692307692)(0,-15.1705685618729)(0,-15.1103678929766)(0,-15.0501672240803)(0,-14.9899665551839)(0,-14.9297658862876)(0,-14.8695652173913)(0,-14.809364548495)(0,-14.7491638795987)(0,-14.6889632107023)(0,-14.628762541806)(0,-14.5685618729097)(0,-14.5083612040134)(0,-14.4481605351171)(0,-14.3879598662207)(0,-14.3277591973244)(0,-14.2675585284281)(0,-14.2073578595318)(0,-14.1471571906355)(0,-14.0869565217391)(0,-14.0267558528428)(0,-13.9665551839465)(0,-13.9063545150502)(0,-13.8461538461538)(0,-13.7859531772575)(0,-13.7257525083612)(0,-13.6655518394649)(0,-13.6053511705686)(0,-13.5451505016722)(0,-13.4849498327759)(0,-13.4247491638796)(0,-13.3645484949833)(0,-13.304347826087)(0,-13.2441471571906)(0,-13.1839464882943)(0,-13.123745819398)(0,-13.0635451505017)(0,-13.0033444816054)(0,-12.943143812709)(0,-12.8829431438127)(0,-12.8227424749164)(0,-12.7625418060201)(0,-12.7023411371237)(0,-12.6421404682274)(0,-12.5819397993311)(0,-12.5217391304348)(0,-12.4615384615385)(0,-12.4013377926421)(0,-12.3411371237458)(0,-12.2809364548495)(0,-12.2207357859532)(0,-12.1605351170569)(0,-12.1003344481605)(0,-12.0401337792642)(0,-11.9799331103679)(0,-11.9197324414716)(0,-11.8595317725753)(0,-11.7993311036789)(0,-11.7391304347826)(0,-11.6789297658863)(0,-11.61872909699)(0,-11.5585284280936)(0,-11.4983277591973)(0,-11.438127090301)(0,-11.3779264214047)(0,-11.3177257525084)(0,-11.257525083612)(0,-11.1973244147157)(0,-11.1371237458194)(0,-11.0769230769231)(0,-11.0167224080268)(0,-10.9565217391304)(0,-10.8963210702341)(0,-10.8361204013378)(0,-10.7759197324415)(0,-10.7157190635452)(0,-10.6555183946488)(0,-10.5953177257525)(0,-10.5351170568562)(0,-10.4749163879599)(0,-10.4147157190635)(0,-10.3545150501672)(0,-10.2943143812709)(0,-10.2341137123746)(0,-10.1739130434783)(0,-10.1137123745819)(0,-10.0535117056856)(0,-9.9933110367893)(0,-9.93311036789298)(0,-9.87290969899666)(0,-9.81270903010033)(0,-9.75250836120401)(0,-9.69230769230769)(0,-9.63210702341137)(0,-9.57190635451505)(0,-9.51170568561873)(0,-9.45150501672241)(0,-9.39130434782609)(0,-9.33110367892977)(0,-9.27090301003344)(0,-9.21070234113712)(0,-9.1505016722408)(0,-9.09030100334448)(0,-9.03010033444816)(0,-8.96989966555184)(0,-8.90969899665552)(0,-8.8494983277592)(0,-8.78929765886288)(0,-8.72909698996656)(0,-8.66889632107023)(0,-8.60869565217391)(0,-8.54849498327759)(0,-8.48829431438127)(0,-8.42809364548495)(0,-8.36789297658863)(0,-8.30769230769231)(0,-8.24749163879599)(0,-8.18729096989967)(0,-8.12709030100334)(0,-8.06688963210702)(0,-8.0066889632107)(0,-7.94648829431438)(0,-7.88628762541806)(0,-7.82608695652174)(0,-7.76588628762542)(0,-7.7056856187291)(0,-7.64548494983278)(0,-7.58528428093646)(0,-7.52508361204013)(0,-7.46488294314381)(0,-7.40468227424749)(0,-7.34448160535117)(0,-7.28428093645485)(0,-7.22408026755853)(0,-7.16387959866221)(0,-7.10367892976589)(0,-7.04347826086957)(0,-6.98327759197324)(0,-6.92307692307692)(0,-6.8628762541806)(0,-6.80267558528428)(0,-6.74247491638796)(0,-6.68227424749164)(0,-6.62207357859532)(0,-6.561872909699)(0,-6.50167224080268)(-7.32306752895604e-05,-6.44147157190635)(-7.32306752895604e-05,-6.38127090301003)(-7.32306752895604e-05,-6.32107023411371)(0,-6.26086956521739)(0,-6.20066889632107)(0,-6.14046822742475)(0,-6.08026755852843)(0,-6.02006688963211)(0,-5.95986622073579)(0,-5.89966555183946)(0,-5.83946488294314)(0,-5.77926421404682)(0,-5.7190635451505)(0,-5.65886287625418)(0,-5.59866220735786)(0,-5.53846153846154)(0,-5.47826086956522)(0,-5.4180602006689)(0,-5.35785953177258)(0,-5.29765886287625)(0,-5.23745819397993)(0,-5.17725752508361)(0,-5.11705685618729)(0,-5.05685618729097)(0,-4.99665551839465)(0,-4.93645484949833)(0,-4.87625418060201)(0,-4.81605351170569)(0,-4.75585284280936)(0,-4.69565217391304)(0,-4.63545150501672)(0,-4.5752508361204)(0,-4.51505016722408)(0,-4.45484949832776)(0,-4.39464882943144)(0,-4.33444816053512)(0,-4.2742474916388)(0,-4.21404682274247)(0,-4.15384615384615)(0,-4.09364548494983)(0,-4.03344481605351)(0,-3.97324414715719)(0,-3.91304347826087)(0,-3.85284280936455)(0,-3.79264214046823)(0,-3.73244147157191)(0,-3.67224080267559)(0,-3.61204013377926)(0,-3.55183946488294)(0,-3.49163879598662)(0,-3.4314381270903)(0,-3.37123745819398)(0,-3.31103678929766)(0,-3.25083612040134)(0,-3.19063545150502)(0,-3.1304347826087)(0,-3.07023411371237)(0,-3.01003344481605)(0,-2.94983277591973)(0,-2.88963210702341)(0,-2.82943143812709)(0,-2.76923076923077)(0,-2.70903010033445)(0,-2.64882943143813)(0,-2.58862876254181)(0,-2.52842809364549)(0,-2.46822742474916)(0,-2.40802675585284)(0,-2.34782608695652)(0,-2.2876254180602)(0,-2.22742474916388)(0,-2.16722408026756)(0,-2.10702341137124)(0,-2.04682274247492)(0,-1.9866220735786)(0,-1.92642140468227)(0,-1.86622073578595)(0,-1.80602006688963)(0,-1.74581939799331)(0,-1.68561872909699)(0,-1.62541806020067)(0,-1.56521739130435)(0,-1.50501672240803)(0,-1.44481605351171)(0,-1.38461538461538)(0,-1.32441471571906)(0,-1.26421404682274)(0,-1.20401337792642)(0,-1.1438127090301)(0,-1.08361204013378)(0,-1.02341137123746)(0,-0.963210702341137)(0,-0.903010033444816)(0,-0.842809364548495)(0,-0.782608695652174)(0,-0.722408026755853)(0,-0.662207357859532)(0,-0.602006688963211)(0,-0.54180602006689)(0,-0.481605351170569)(0,-0.421404682274247)(0,-0.361204013377926)(0,-0.301003344481605)(0,-0.240802675585284)(0,-0.180602006688963)(0,-0.120401337792642)(0,-0.0602006688963211)(0,0)(4.94314381270903,0)(9.88628762541806,0)(14.8294314381271,0)(19.7725752508361,0)(24.7157190635452,0)(29.6588628762542,0)(34.6020066889632,0)(39.5451505016722,0)(44.4882943143813,0)(49.4314381270903,0)(54.3745819397993,0)(59.3177257525084,0)(64.2608695652174,0)(69.2040133779264,0)(74.1471571906355,0)(79.0903010033445,0)(84.0334448160535,0)(88.9765886287625,0)(93.9197324414716,0)(98.8628762541806,0)(103.80602006689,0)(108.749163879599,0)(113.692307692308,0)(118.635451505017,0)(123.578595317726,0)(128.521739130435,0)(133.464882943144,0)(138.408026755853,0)(143.351170568562,0)(148.294314381271,0)(153.23745819398,0)(158.180602006689,0)(163.123745819398,0)(168.066889632107,0)(173.010033444816,0)(177.953177257525,0)(182.896321070234,0)(187.839464882943,0)(192.782608695652,0)(197.725752508361,0)(202.66889632107,0)(207.612040133779,0)(212.555183946488,0)(217.498327759197,0)(222.441471571906,0)(227.384615384615,0)(232.327759197324,0)(237.270903010033,0)(242.214046822742,0)(247.157190635452,0)(252.100334448161,0)(257.04347826087,0)(261.986622073579,0)(266.929765886288,0)(271.872909698997,0)(276.816053511706,0)(281.759197324415,0)(286.702341137124,0)(291.645484949833,0)(296.588628762542,0)(301.531772575251,0)(306.47491638796,0)(311.418060200669,0)(316.361204013378,0)(321.304347826087,0)(326.247491638796,0)(331.190635451505,0)(336.133779264214,0)(341.076923076923,0)(346.020066889632,0)(350.963210702341,0)(355.90635451505,0)(360.849498327759,0)(365.792642140468,0)(370.735785953177,0)(375.678929765886,0)(380.622073578595,0)(385.565217391304,0)(390.508361204013,0)(395.451505016722,0)(400.394648829431,0)(405.33779264214,0)(410.28093645485,0)(415.224080267559,0)(420.167224080268,0)(425.110367892977,0)(430.053511705686,0)(434.996655518395,0)(439.939799331104,0)(444.882943143813,0)(449.826086956522,0)(454.769230769231,0)(459.71237458194,0)(464.655518394649,0)(469.598662207358,0)(474.541806020067,0)(479.484949832776,0)(484.428093645485,0)(489.371237458194,0)(494.314381270903,0)(499.257525083612,0)(504.200668896321,0)(509.14381270903,0)(514.086956521739,0)(519.030100334448,0)(523.973244147157,0)(528.916387959866,0)(533.859531772575,0)(538.802675585284,0)(543.745819397993,0)(548.688963210702,0)(553.632107023411,0)(558.57525083612,0)(563.518394648829,0)(568.461538461538,0)(573.404682274248,0)(578.347826086957,0)(583.290969899666,0)(588.234113712375,0)(593.177257525084,0)(598.120401337793,0)(603.063545150502,0)(608.006688963211,0)(612.94983277592,0)(617.892976588629,0)(622.836120401338,0)(627.779264214047,0)(632.722408026756,0)(637.665551839465,0)(642.608695652174,0)(647.551839464883,0)(652.494983277592,0)(657.438127090301,0)(662.38127090301,0)(667.324414715719,0)(672.267558528428,0)(677.210702341137,0)(682.153846153846,0)(687.096989966555,0)(692.040133779264,0)(696.983277591973,0)(701.926421404682,0)(706.869565217391,0)(711.8127090301,0)(716.755852842809,0)(721.698996655518,0)(726.642140468227,0)(731.585284280936,0)(736.528428093646,0)(741.471571906354,0)(746.414715719064,0)(751.357859531773,0)(756.301003344482,0)(761.244147157191,0)(766.1872909699,0)(771.130434782609,0)(776.073578595318,0)(781.016722408027,0)(785.959866220736,0)(790.903010033445,0)(795.846153846154,0)(800.789297658863,0)(805.732441471572,0)(810.675585284281,0)(815.61872909699,0)(820.561872909699,0)(825.505016722408,0)(830.448160535117,0)(835.391304347826,0)(840.334448160535,0)(845.277591973244,0)(850.220735785953,0)(855.163879598662,0)(860.107023411371,0)(865.05016722408,0)(869.993311036789,0)(874.936454849498,0)(879.879598662207,0)(884.822742474916,0)(889.765886287625,0)(894.709030100334,0)(899.652173913044,0)(904.595317725752,0)(909.538461538462,0)(914.481605351171,0)(919.42474916388,0)(924.367892976589,0)(929.311036789298,0)(934.254180602007,0)(939.197324414716,0)(944.140468227425,0)(949.083612040134,0)(954.026755852843,0)(958.969899665552,0)(963.913043478261,0)(968.85618729097,0)(973.799331103679,0)(978.742474916388,0)(983.685618729097,0)(988.628762541806,0)(993.571906354515,0)(998.515050167224,0)(1003.45819397993,0)(1008.40133779264,0)(1013.34448160535,0)(1018.28762541806,0)(1023.23076923077,0)(1028.17391304348,0)(1033.11705685619,0)(1038.0602006689,0)(1043.00334448161,0)(1047.94648829431,0)(1052.88963210702,0)(1057.83277591973,0)(1062.77591973244,0)(1067.71906354515,0)(1072.66220735786,0)(1077.60535117057,0)(1082.54849498328,0)(1087.49163879599,0)(1092.4347826087,0)(1097.3779264214,0)(1102.32107023411,0)(1107.26421404682,0)(1112.20735785953,0)(1117.15050167224,0)(1122.09364548495,0)(1127.03678929766,0)(1131.97993311037,0)(1136.92307692308,0)(1141.86622073579,0)(1146.8093645485,0)(1151.7525083612,0)(1156.69565217391,0)(1161.63879598662,0)(1166.58193979933,0)(1171.52508361204,0)(1176.46822742475,0)(1181.41137123746,0)(1186.35451505017,0)(1191.29765886288,0)(1196.24080267559,0)(1201.18394648829,0)(1206.127090301,0)(1211.07023411371,0)(1216.01337792642,0)(1220.95652173913,0)(1225.89966555184,0)(1230.84280936455,0)(1235.78595317726,0)(1240.72909698997,0)(1245.67224080268,0)(1250.61538461538,0)(1255.55852842809,0)(1260.5016722408,0)(1265.44481605351,0)(1270.38795986622,0)(1275.33110367893,0)(1280.27424749164,0)(1285.21739130435,0)(1290.16053511706,0)(1295.10367892977,0)(1300.04682274247,0)(1304.98996655518,0)(1309.93311036789,0)(1314.8762541806,0)(1319.81939799331,0)(1324.76254180602,0)(1329.70568561873,0)(1334.64882943144,0)(1339.59197324415,0)(1344.53511705686,0)(1349.47826086957,0)(1354.42140468227,0)(1359.36454849498,0)(1364.30769230769,0)(1369.2508361204,0)(1374.19397993311,0)(1379.13712374582,0)(1384.08026755853,0)(1389.02341137124,0)(1393.96655518395,0)(1398.90969899666,0)(1403.85284280936,0)(1408.79598662207,0)(1413.73913043478,0)(1418.68227424749,0)(1423.6254180602,0)(1428.56856187291,0)(1433.51170568562,0)(1438.45484949833,0)(1443.39799331104,0)(1448.34113712375,0)(1453.28428093645,0)(1458.22742474916,0)(1463.17056856187,0)(1468.11371237458,0)(1473.05685618729,8.91848548857975e-07)(1478,1.78367067334156e-06)(1478.00014645918,0)};

\addplot [fill=red!40!yellow,draw=none,forget plot] coordinates{ (1005.92976588629,-7.22408026755853)(1008.40133779264,-7.23913043478261)(1013.34448160535,-7.2742474916388)(1015.81605351171,-7.28428093645485)(1018.28762541806,-7.2943143812709)(1023.23076923077,-7.31438127090301)(1028.17391304348,-7.31438127090301)(1033.11705685619,-7.33444816053512)(1035.58862876254,-7.34448160535117)(1038.0602006689,-7.35451505016722)(1043.00334448161,-7.37458193979933)(1047.94648829431,-7.37458193979933)(1052.88963210702,-7.39464882943144)(1055.36120401338,-7.40468227424749)(1057.83277591973,-7.41471571906355)(1062.77591973244,-7.43478260869565)(1067.71906354515,-7.43478260869565)(1072.66220735786,-7.43478260869565)(1077.60535117057,-7.45484949832776)(1080.07692307692,-7.46488294314381)(1082.54849498328,-7.47491638795987)(1087.49163879599,-7.49498327759197)(1092.4347826087,-7.49498327759197)(1097.3779264214,-7.51505016722408)(1102.32107023411,-7.51505016722408)(1104.79264214047,-7.52508361204013)(1107.26421404682,-7.53511705685619)(1112.20735785953,-7.53511705685619)(1117.15050167224,-7.54013377926421)(1122.09364548495,-7.54013377926421)(1127.03678929766,-7.54013377926421)(1131.97993311037,-7.55518394648829)(1136.92307692308,-7.55518394648829)(1141.86622073579,-7.5752508361204)(1144.33779264214,-7.58528428093646)(1146.8093645485,-7.59197324414716)(1151.7525083612,-7.60535117056856)(1156.69565217391,-7.60535117056856)(1161.63879598662,-7.60535117056856)(1166.58193979933,-7.60785953177257)(1171.52508361204,-7.61108456760631)(1176.46822742475,-7.61538461538462)(1181.41137123746,-7.61538461538462)(1186.35451505017,-7.61538461538462)(1191.29765886288,-7.61538461538462)(1196.24080267559,-7.61538461538462)(1201.18394648829,-7.61538461538462)(1206.127090301,-7.61538461538462)(1211.07023411371,-7.61538461538462)(1216.01337792642,-7.61538461538462)(1220.95652173913,-7.59531772575251)(1225.89966555184,-7.59531772575251)(1230.84280936455,-7.61538461538462)(1234.55016722408,-7.64548494983278)(1235.78595317726,-7.65551839464883)(1240.72909698997,-7.67558528428094)(1245.67224080268,-7.67558528428094)(1250.61538461538,-7.65551839464883)(1253.08695652174,-7.64548494983278)(1255.55852842809,-7.63545150501672)(1260.5016722408,-7.63545150501672)(1262.97324414716,-7.64548494983278)(1265.44481605351,-7.65551839464883)(1270.38795986622,-7.69565217391304)(1272.85953177258,-7.7056856187291)(1275.33110367893,-7.71571906354515)(1280.27424749164,-7.71571906354515)(1285.21739130435,-7.71571906354515)(1290.16053511706,-7.73578595317726)(1293.86789297659,-7.76588628762542)(1295.10367892977,-7.77591973244147)(1300.04682274247,-7.79598662207358)(1304.98996655518,-7.81605351170569)(1307.46153846154,-7.82608695652174)(1309.93311036789,-7.83612040133779)(1314.8762541806,-7.84113712374582)(1319.81939799331,-7.87123745819398)(1324.76254180602,-7.87123745819398)(1325.9983277592,-7.88628762541806)(1329.70568561873,-7.91638795986622)(1334.64882943144,-7.93645484949833)(1335.47268673356,-7.94648829431438)(1339.59197324415,-7.99665551839465)(1342.0635451505,-8.0066889632107)(1344.53511705686,-8.01672240802676)(1349.47826086957,-8.05685618729097)(1350.71404682274,-8.06688963210702)(1354.42140468227,-8.11204013377926)(1355.65719063545,-8.12709030100334)(1359.36454849498,-8.1571906354515)(1363.07190635452,-8.18729096989967)(1364.30769230769,-8.19732441471572)(1369.2508361204,-8.23745819397993)(1371.72240802676,-8.24749163879599)(1374.19397993311,-8.25752508361204)(1378.3132664437,-8.30769230769231)(1379.13712374582,-8.32274247491639)(1381.60869565217,-8.36789297658863)(1384.08026755853,-8.39799331103679)(1386.55183946488,-8.42809364548495)(1389.02341137124,-8.45819397993311)(1391.49498327759,-8.48829431438127)(1393.96655518395,-8.51839464882943)(1396.4381270903,-8.54849498327759)(1398.90969899666,-8.57859531772575)(1402.61705685619,-8.60869565217391)(1403.85284280936,-8.62374581939799)(1406.32441471572,-8.66889632107023)(1408.79598662207,-8.69899665551839)(1411.26755852843,-8.72909698996656)(1413.73913043478,-8.75919732441472)(1417.44648829431,-8.78929765886288)(1418.68227424749,-8.81939799331104)(1419.50613154961,-8.8494983277592)(1422.80156075808,-8.90969899665552)(1423.6254180602,-8.9247491638796)(1426.09698996656,-8.96989966555184)(1428.56856187291,-9)(1431.04013377926,-9.03010033444816)(1433.51170568562,-9.06020066889632)(1435.98327759197,-9.09030100334448)(1438.45484949833,-9.13545150501672)(1439.27870680045,-9.1505016722408)(1442.57413600892,-9.21070234113712)(1443.39799331104,-9.2257525083612)(1447.10535117057,-9.27090301003344)(1448.34113712375,-9.28595317725753)(1450.8127090301,-9.33110367892977)(1453.28428093645,-9.37625418060201)(1454.10813823857,-9.39130434782609)(1453.28428093645,-9.42140468227425)(1452.46042363434,-9.45150501672241)(1453.28428093645,-9.48160535117057)(1454.10813823857,-9.51170568561873)(1454.10813823857,-9.57190635451505)(1455.75585284281,-9.63210702341137)(1453.28428093645,-9.67725752508361)(1452.46042363434,-9.69230769230769)(1448.34113712375,-9.74247491638796)(1445.86956521739,-9.75250836120401)(1443.39799331104,-9.76254180602007)(1438.45484949833,-9.80267558528428)(1437.21906354515,-9.81270903010033)(1433.51170568562,-9.8428093645485)(1429.80434782609,-9.87290969899666)(1428.56856187291,-9.88294314381271)(1423.6254180602,-9.90301003344482)(1418.68227424749,-9.92307692307692)(1416.21070234114,-9.93311036789298)(1413.73913043478,-9.94314381270903)(1408.79598662207,-9.97826086956522)(1403.85284280936,-9.97826086956522)(1401.38127090301,-9.9933110367893)(1398.90969899666,-10.0033444816054)(1393.96655518395,-10.0083612040134)(1389.02341137124,-10.0384615384615)(1384.08026755853,-10.0384615384615)(1379.13712374582,-10.0234113712375)(1374.19397993311,-10.0234113712375)(1369.2508361204,-10.0234113712375)(1364.30769230769,-10.0234113712375)(1359.36454849498,-10.0434782608696)(1356.89297658863,-10.0535117056856)(1354.42140468227,-10.0635451505017)(1349.47826086957,-10.0836120401338)(1344.53511705686,-10.0836120401338)(1339.59197324415,-10.0836120401338)(1334.64882943144,-10.0836120401338)(1329.70568561873,-10.1036789297659)(1324.76254180602,-10.1036789297659)(1319.81939799331,-10.1036789297659)(1314.8762541806,-10.0836120401338)(1309.93311036789,-10.0836120401338)(1304.98996655518,-10.0836120401338)(1300.04682274247,-10.0836120401338)(1295.10367892977,-10.0836120401338)(1290.16053511706,-10.0836120401338)(1285.21739130435,-10.1036789297659)(1280.27424749164,-10.1036789297659)(1275.33110367893,-10.1036789297659)(1270.38795986622,-10.0836120401338)(1265.44481605351,-10.0836120401338)(1260.5016722408,-10.0836120401338)(1255.55852842809,-10.0836120401338)(1250.61538461538,-10.0836120401338)(1245.67224080268,-10.0836120401338)(1240.72909698997,-10.0836120401338)(1235.78595317726,-10.1036789297659)(1233.3143812709,-10.1137123745819)(1230.84280936455,-10.123745819398)(1225.89966555184,-10.1638795986622)(1220.95652173913,-10.1638795986622)(1216.01337792642,-10.1638795986622)(1211.07023411371,-10.1438127090301)(1206.127090301,-10.1438127090301)(1201.18394648829,-10.1438127090301)(1196.24080267559,-10.1438127090301)(1191.29765886288,-10.1438127090301)(1186.35451505017,-10.1438127090301)(1181.41137123746,-10.1438127090301)(1176.46822742475,-10.1438127090301)(1171.52508361204,-10.1438127090301)(1166.58193979933,-10.1638795986622)(1161.63879598662,-10.1638795986622)(1156.69565217391,-10.1638795986622)(1151.7525083612,-10.1438127090301)(1146.8093645485,-10.1638795986622)(1144.33779264214,-10.1739130434783)(1141.86622073579,-10.1839464882943)(1136.92307692308,-10.2040133779264)(1131.97993311037,-10.2240802675585)(1127.03678929766,-10.2240802675585)(1122.09364548495,-10.2240802675585)(1117.15050167224,-10.1839464882943)(1112.20735785953,-10.1839464882943)(1107.26421404682,-10.1839464882943)(1102.32107023411,-10.2040133779264)(1097.3779264214,-10.1889632107023)(1094.90635451505,-10.1739130434783)(1092.4347826087,-10.1638795986622)(1087.49163879599,-10.1438127090301)(1082.54849498328,-10.1438127090301)(1077.60535117057,-10.1438127090301)(1072.66220735786,-10.1438127090301)(1067.71906354515,-10.1438127090301)(1062.77591973244,-10.1438127090301)(1057.83277591973,-10.1438127090301)(1052.88963210702,-10.1438127090301)(1047.94648829431,-10.1438127090301)(1043.00334448161,-10.1438127090301)(1038.0602006689,-10.1638795986622)(1033.11705685619,-10.1638795986622)(1028.17391304348,-10.1438127090301)(1024.46655518395,-10.1137123745819)(1023.23076923077,-10.0986622073579)(1018.28762541806,-10.0836120401338)(1013.34448160535,-10.0836120401338)(1008.40133779264,-10.0836120401338)(1003.45819397993,-10.0836120401338)(998.515050167224,-10.0836120401338)(993.571906354515,-10.0635451505017)(991.100334448161,-10.0535117056856)(988.628762541806,-10.0434782608696)(983.685618729097,-10.0234113712375)(978.742474916388,-10.0234113712375)(973.799331103679,-10.0234113712375)(968.85618729097,-10.0234113712375)(963.913043478261,-10.0234113712375)(958.969899665552,-10.0033444816054)(956.498327759197,-9.9933110367893)(954.026755852843,-9.98327759197324)(949.083612040134,-9.96321070234114)(944.140468227425,-9.96321070234114)(939.197324414716,-9.96321070234114)(934.254180602007,-9.94314381270903)(931.782608695652,-9.93311036789298)(929.311036789298,-9.92307692307692)(924.367892976589,-9.90301003344482)(919.42474916388,-9.90301003344482)(914.481605351171,-9.90301003344482)(909.538461538462,-9.88294314381271)(907.066889632107,-9.87290969899666)(904.595317725752,-9.8628762541806)(899.652173913044,-9.8428093645485)(894.709030100334,-9.82274247491639)(892.23745819398,-9.81270903010033)(889.765886287625,-9.80267558528428)(884.822742474916,-9.78260869565217)(879.879598662207,-9.76254180602007)(877.408026755853,-9.75250836120401)(874.936454849498,-9.74247491638796)(869.993311036789,-9.72240802675585)(865.05016722408,-9.72240802675585)(860.107023411371,-9.72240802675585)(855.163879598662,-9.70234113712375)(853.928093645485,-9.69230769230769)(850.220735785953,-9.66220735785953)(846.513377926422,-9.63210702341137)(845.277591973244,-9.62207357859532)(840.334448160535,-9.5819397993311)(837.862876254181,-9.57190635451505)(835.391304347826,-9.561872909699)(830.448160535117,-9.52173913043478)(827.976588628763,-9.51170568561873)(825.505016722408,-9.50167224080268)(820.561872909699,-9.46153846153846)(818.090301003345,-9.45150501672241)(815.61872909699,-9.44147157190636)(810.675585284281,-9.40133779264214)(808.204013377926,-9.39130434782609)(805.732441471572,-9.38127090301004)(800.789297658863,-9.36120401337793)(797.081939799331,-9.33110367892977)(795.846153846154,-9.31605351170569)(793.374581939799,-9.27090301003344)(793.374581939799,-9.21070234113712)(791.726867335563,-9.1505016722408)(790.903010033445,-9.12040133779264)(790.079152731327,-9.09030100334448)(786.783723522854,-9.03010033444816)(785.959866220736,-9)(785.136008918618,-8.96989966555184)(781.840579710145,-8.90969899665552)(781.016722408027,-8.87959866220736)(780.192865105909,-8.8494983277592)(778.545150501672,-8.78929765886288)(778.545150501672,-8.72909698996656)(778.545150501672,-8.66889632107023)(780.192865105909,-8.60869565217391)(781.016722408027,-8.57859531772575)(781.840579710145,-8.54849498327759)(785.136008918618,-8.48829431438127)(785.959866220736,-8.45819397993311)(786.783723522854,-8.42809364548495)(790.079152731327,-8.36789297658863)(790.903010033445,-8.35284280936455)(793.374581939799,-8.30769230769231)(795.846153846154,-8.26254180602007)(796.670011148272,-8.24749163879599)(799.965440356745,-8.18729096989967)(800.789297658863,-8.17224080267559)(803.260869565217,-8.12709030100334)(805.732441471572,-8.09698996655519)(808.204013377926,-8.06688963210702)(810.675585284281,-8.02173913043478)(811.499442586399,-8.0066889632107)(815.61872909699,-7.95652173913043)(818.090301003345,-7.94648829431438)(820.561872909699,-7.93645484949833)(824.68115942029,-7.88628762541806)(825.505016722408,-7.87625418060201)(830.448160535117,-7.83612040133779)(831.272017837235,-7.82608695652174)(834.567447045708,-7.76588628762542)(835.391304347826,-7.75585284280937)(840.334448160535,-7.73578595317726)(844.041806020067,-7.7056856187291)(845.277591973244,-7.69565217391304)(850.220735785953,-7.69565217391304)(854.340022296544,-7.64548494983278)(855.163879598662,-7.63545150501672)(860.107023411371,-7.59531772575251)(861.342809364548,-7.58528428093646)(865.05016722408,-7.55518394648829)(869.993311036789,-7.53511705685619)(872.464882943144,-7.52508361204013)(874.936454849498,-7.51505016722408)(879.879598662207,-7.47491638795987)(882.351170568562,-7.46488294314381)(884.822742474916,-7.45484949832776)(889.765886287625,-7.43478260869565)(894.709030100335,-7.41471571906355)(897.180602006689,-7.40468227424749)(899.652173913044,-7.39464882943144)(904.595317725752,-7.37458193979933)(909.538461538462,-7.35451505016722)(912.010033444816,-7.34448160535117)(914.481605351171,-7.33444816053512)(919.42474916388,-7.31438127090301)(924.367892976589,-7.31438127090301)(929.311036789298,-7.2943143812709)(931.782608695652,-7.28428093645485)(934.254180602007,-7.2742474916388)(939.197324414716,-7.25418060200669)(944.140468227425,-7.25418060200669)(949.083612040134,-7.25418060200669)(954.026755852843,-7.25418060200669)(958.969899665552,-7.25418060200669)(963.913043478261,-7.25418060200669)(968.85618729097,-7.25418060200669)(973.799331103679,-7.23411371237458)(976.270903010033,-7.22408026755853)(978.742474916388,-7.21404682274247)(983.685618729097,-7.21404682274247)(986.157190635452,-7.22408026755853)(988.628762541806,-7.23411371237458)(993.571906354515,-7.23411371237458)(996.04347826087,-7.22408026755853)(998.515050167224,-7.21404682274247)(1003.45819397993,-7.21404682274247)(1005.92976588629,-7.22408026755853)};

\addplot [fill=red!40!yellow,draw=none,forget plot] coordinates{ (556.103678929766,-7.22408026755853)(558.57525083612,-7.23411371237458)(563.518394648829,-7.26923076923077)(568.461538461538,-7.26923076923077)(573.404682274248,-7.26923076923077)(578.347826086957,-7.2742474916388)(583.290969899666,-7.25418060200669)(586.998327759197,-7.28428093645485)(588.234113712375,-7.29933110367893)(593.177257525084,-7.32943143812709)(594.001114827202,-7.34448160535117)(598.120401337793,-7.39464882943144)(600.591973244147,-7.40468227424749)(603.063545150502,-7.41471571906355)(608.006688963211,-7.45484949832776)(609.242474916388,-7.46488294314381)(612.94983277592,-7.49498327759197)(616.657190635452,-7.52508361204013)(617.892976588629,-7.53511705685619)(622.836120401338,-7.57023411371237)(624.071906354515,-7.58528428093646)(625.307692307692,-7.64548494983278)(627.779264214047,-7.69063545150502)(628.603121516165,-7.7056856187291)(630.250836120401,-7.76588628762542)(632.722408026756,-7.81103678929766)(633.546265328874,-7.82608695652174)(633.546265328874,-7.88628762541806)(636.841694537347,-7.94648829431438)(637.665551839465,-7.97658862876254)(638.489409141583,-8.0066889632107)(640.137123745819,-8.06688963210702)(640.137123745819,-8.12709030100334)(640.137123745819,-8.18729096989967)(638.489409141583,-8.24749163879599)(637.665551839465,-8.27759197324415)(636.841694537347,-8.30769230769231)(635.19397993311,-8.36789297658863)(635.19397993311,-8.42809364548495)(633.546265328874,-8.48829431438127)(632.722408026756,-8.50334448160535)(630.250836120401,-8.54849498327759)(627.779264214047,-8.59364548494983)(626.955406911929,-8.60869565217391)(622.836120401338,-8.65886287625418)(621.600334448161,-8.66889632107023)(617.892976588629,-8.69899665551839)(614.185618729097,-8.72909698996656)(612.94983277592,-8.73913043478261)(608.006688963211,-8.77926421404682)(607.182831661092,-8.78929765886288)(603.063545150502,-8.83946488294314)(600.591973244147,-8.8494983277592)(598.120401337793,-8.85953177257525)(593.177257525084,-8.89966555183947)(590.705685618729,-8.90969899665552)(588.234113712375,-8.91973244147157)(583.290969899666,-8.95986622073579)(580.819397993311,-8.96989966555184)(578.347826086957,-8.97993311036789)(573.404682274248,-9.01505016722408)(570.933110367893,-9.03010033444816)(568.461538461538,-9.04013377926421)(563.518394648829,-9.06020066889632)(558.57525083612,-9.06020066889632)(553.632107023411,-9.06020066889632)(548.688963210702,-9.06020066889632)(543.745819397993,-9.06020066889632)(538.802675585284,-9.08026755852843)(536.33110367893,-9.09030100334448)(533.859531772575,-9.10033444816053)(528.916387959866,-9.12040133779264)(523.973244147157,-9.13545150501672)(519.030100334448,-9.13545150501672)(514.086956521739,-9.13545150501672)(509.14381270903,-9.12040133779264)(504.200668896321,-9.10033444816053)(501.729096989967,-9.09030100334448)(499.257525083612,-9.08026755852843)(494.314381270903,-9.06020066889632)(489.371237458194,-9.0752508361204)(484.428093645485,-9.0752508361204)(479.484949832776,-9.04515050167224)(477.013377926421,-9.03010033444816)(474.541806020067,-9.02006688963211)(469.598662207358,-8.97993311036789)(467.127090301003,-8.96989966555184)(464.655518394649,-8.95986622073579)(459.71237458194,-8.91973244147157)(458.476588628763,-8.90969899665552)(454.769230769231,-8.87959866220736)(451.061872909699,-8.8494983277592)(449.826086956522,-8.83946488294314)(444.882943143813,-8.79933110367893)(443.647157190635,-8.78929765886288)(439.939799331104,-8.74414715719064)(439.115942028986,-8.72909698996656)(435.820512820513,-8.66889632107023)(434.996655518395,-8.65384615384615)(432.52508361204,-8.60869565217391)(430.053511705686,-8.56354515050167)(429.229654403567,-8.54849498327759)(427.581939799331,-8.48829431438127)(425.934225195095,-8.42809364548495)(425.110367892977,-8.41304347826087)(422.638795986622,-8.36789297658863)(420.167224080268,-8.32274247491639)(419.343366778149,-8.30769230769231)(417.695652173913,-8.24749163879599)(417.695652173913,-8.18729096989967)(417.695652173913,-8.12709030100334)(417.695652173913,-8.06688963210702)(417.695652173913,-8.0066889632107)(417.695652173913,-7.94648829431438)(419.343366778149,-7.88628762541806)(420.167224080268,-7.87123745819398)(422.638795986622,-7.82608695652174)(425.110367892977,-7.7809364548495)(425.934225195095,-7.76588628762542)(430.053511705686,-7.71571906354515)(430.877369007804,-7.7056856187291)(434.996655518395,-7.65551839464883)(436.232441471572,-7.64548494983278)(439.939799331104,-7.61538461538462)(442.411371237458,-7.58528428093646)(444.882943143813,-7.55518394648829)(449.826086956522,-7.53511705685619)(451.061872909699,-7.52508361204013)(454.769230769231,-7.47993311036789)(457.240802675585,-7.46488294314381)(459.71237458194,-7.45484949832776)(464.655518394649,-7.41471571906355)(467.127090301003,-7.40468227424749)(469.598662207358,-7.39464882943144)(474.541806020067,-7.35451505016722)(477.013377926421,-7.34448160535117)(479.484949832776,-7.33444816053512)(484.428093645485,-7.31438127090301)(489.371237458194,-7.31438127090301)(494.314381270903,-7.2943143812709)(496.785953177258,-7.28428093645485)(499.257525083612,-7.2742474916388)(504.200668896321,-7.25418060200669)(509.14381270903,-7.25418060200669)(514.086956521739,-7.23411371237458)(516.558528428094,-7.22408026755853)(519.030100334448,-7.21404682274247)(523.973244147157,-7.19397993311037)(528.916387959866,-7.19397993311037)(533.859531772575,-7.19397993311037)(538.802675585284,-7.19397993311037)(543.745819397993,-7.19397993311037)(548.688963210702,-7.19397993311037)(553.632107023411,-7.20903010033445)(556.103678929766,-7.22408026755853)};

\addplot [fill=red!40!yellow,draw=none,forget plot] coordinates{ (210.083612040134,-7.76588628762542)(212.555183946488,-7.77591973244147)(217.498327759197,-7.79598662207358)(222.441471571906,-7.79598662207358)(227.384615384615,-7.81605351170569)(229.85618729097,-7.82608695652174)(232.327759197324,-7.83612040133779)(237.270903010033,-7.8561872909699)(242.214046822742,-7.8561872909699)(245.921404682274,-7.88628762541806)(247.157190635452,-7.89632107023411)(252.100334448161,-7.93645484949833)(252.924191750279,-7.94648829431438)(256.219620958751,-8.0066889632107)(257.04347826087,-8.01672240802676)(261.16276477146,-8.06688963210702)(261.986622073579,-8.09698996655519)(262.810479375697,-8.12709030100334)(264.458193979933,-8.18729096989967)(264.458193979933,-8.24749163879599)(262.810479375697,-8.30769230769231)(261.986622073579,-8.32274247491639)(259.515050167224,-8.36789297658863)(257.04347826087,-8.39799331103679)(254.571906354515,-8.42809364548495)(252.100334448161,-8.45819397993311)(249.628762541806,-8.48829431438127)(247.157190635452,-8.51839464882943)(243.44983277592,-8.54849498327759)(242.214046822742,-8.56354515050167)(238.506688963211,-8.60869565217391)(237.270903010033,-8.61872909698996)(232.327759197324,-8.61872909698996)(227.384615384615,-8.65384615384615)(222.441471571906,-8.65384615384615)(219.969899665552,-8.66889632107023)(217.498327759197,-8.67892976588629)(212.555183946488,-8.67892976588629)(207.612040133779,-8.69899665551839)(202.66889632107,-8.67892976588629)(200.197324414716,-8.66889632107023)(197.725752508361,-8.65886287625418)(192.782608695652,-8.63879598662207)(187.839464882943,-8.63879598662207)(182.896321070234,-8.65384615384615)(177.953177257525,-8.62374581939799)(175.481605351171,-8.60869565217391)(173.010033444816,-8.59866220735786)(168.066889632107,-8.55852842809364)(166.83110367893,-8.54849498327759)(163.123745819398,-8.51839464882943)(160.652173913043,-8.48829431438127)(158.180602006689,-8.45819397993311)(155.709030100334,-8.42809364548495)(153.23745819398,-8.39799331103679)(150.765886287625,-8.36789297658863)(148.294314381271,-8.32274247491639)(147.470457079153,-8.30769230769231)(143.351170568562,-8.25752508361204)(142.527313266444,-8.24749163879599)(140.879598662207,-8.18729096989967)(139.231884057971,-8.12709030100334)(140.879598662207,-8.06688963210702)(139.231884057971,-8.0066889632107)(140.879598662207,-7.94648829431438)(143.351170568562,-7.90133779264214)(144.586956521739,-7.88628762541806)(148.294314381271,-7.8561872909699)(152.001672240803,-7.82608695652174)(153.23745819398,-7.81605351170569)(158.180602006689,-7.79598662207358)(163.123745819398,-7.77591973244147)(165.595317725753,-7.76588628762542)(168.066889632107,-7.75585284280937)(173.010033444816,-7.73578595317726)(177.953177257525,-7.73578595317726)(182.896321070234,-7.73578595317726)(187.839464882943,-7.73578595317726)(192.782608695652,-7.73578595317726)(197.725752508361,-7.73578595317726)(202.66889632107,-7.73578595317726)(207.612040133779,-7.75585284280937)(210.083612040134,-7.76588628762542)};

\addplot [
color=white,
draw=white,
only marks,
mark=x,
mark options={solid},
mark size=2.0pt,
line width=0.3pt,
forget plot
]
coordinates{
 (370.735785953177,-4.75585284280936)(1478,-18)(0,-15.8327759197324)(1478,0)(1478,-9.27090301003344)(0,0)(588.234113712375,-11.1973244147157)(696.983277591973,-18)(1156.69565217391,-4.63545150501672)(0,-9.03010033444816)(741.471571906354,0)(1250.61538461538,-13.7257525083612)(830.448160535117,-7.82608695652174)(0,-12.4615384615385)(608.006688963211,-14.628762541806)(0,-3.67224080267559)(1478,-5.95986622073579)(207.612040133779,-18)(1057.83277591973,-10.8361204013378)(1478,-2.82943143812709)(751.357859531773,-3.01003344481605)(1077.60535117057,-16.6153846153846)(385.565217391304,-7.88628762541806)(355.90635451505,-1.38461538461538)(0,-6.38127090301003)(1478,-12.0401337792642)(1112.20735785953,-1.02341137123746)(1478,-15.4113712374582)(286.702341137124,-13.7859531772575)(1186.35451505017,-7.64548494983278)(751.357859531773,-5.53846153846154)(884.822742474916,-13.1839464882943)(252.100334448161,-10.7157190635452)(652.494983277592,-9.39130434782609)(1211.07023411371,-9.51170568561873)(528.916387959866,-6.50167224080268)(0,-10.6555183946488)(1478,-7.64548494983278)(1018.28762541806,-6.14046822742475)(232.327759197324,-6.32107023411371)(1240.72909698997,-11.7391304347826)(914.481605351171,-9.51170568561873)(247.157190635452,-9.09030100334448)(825.505016722408,-10.9565217391304)(1478,-10.6555183946488)(1270.38795986622,-6.02006688963211)(0,-7.7056856187291)(444.882943143813,-9.81270903010033)(612.94983277592,-7.88628762541806)(1028.17391304348,-8.30769230769231)(177.953177257525,-7.7056856187291)(790.903010033445,-6.68227424749164)(1329.70568561873,-8.42809364548495)(1304.98996655518,-10.5351170568562)(385.565217391304,-6.44147157190635)(1344.53511705686,-7.16387959866221)(1003.45819397993,-7.16387959866221)(484.428093645485,-8.72909698996656)(133.464882943144,-9.45150501672241)(741.471571906354,-10.0535117056856)(1067.71906354515,-9.63210702341137)(781.016722408027,-8.72909698996656)(647.551839464883,-6.62207357859532)(1478,-8.48829431438127)(1146.8093645485,-6.8628762541806)(118.635451505017,-6.80267558528428)(1359.36454849498,-9.63210702341137)(1166.58193979933,-8.60869565217391)(123.578595317726,-8.42809364548495)(1176.46822742475,-10.5953177257525)(341.076923076923,-9.51170568561873)(954.026755852843,-10.3545150501672)(484.428093645485,-7.40468227424749)(291.645484949833,-7.28428093645485)(1478,-10.0535117056856)(924.367892976589,-8.60869565217391)(711.8127090301,-7.58528428093646)(558.57525083612,-9.39130434782609)(914.481605351171,-7.16387959866221)(311.418060200669,-8.42809364548495)(14.8294314381271,-7.34448160535117)(647.551839464883,-8.54849498327759)(1319.81939799331,-7.76588628762542)(815.61872909699,-9.69230769230769)(1314.8762541806,-9.03010033444816)(1364.30769230769,-10.4147157190635)(1057.83277591973,-9.03010033444816)(1072.66220735786,-7.7056856187291)(49.4314381270903,-8.24749163879599)(499.257525083612,-8.18729096989967)(1478,-8.8494983277592)(395.451505016722,-6.98327759197324)(1146.8093645485,-10.0535117056856)(603.063545150502,-7.16387959866221)(182.896321070234,-6.98327759197324)(1216.01337792642,-7.22408026755853)(395.451505016722,-8.96989966555184)(909.538461538462,-7.94648829431438)(227.384615384615,-8.24749163879599)(919.42474916388,-10.1739130434783)(1478,-10.1137123745819)(761.244147157191,-8.96989966555184)(1216.01337792642,-8.42809364548495)(138.408026755853,-8.96989966555184)(1389.02341137124,-7.94648829431438)(1067.71906354515,-7.04347826086957)(761.244147157191,-7.7056856187291)(1265.44481605351,-9.87290969899666)(954.026755852843,-9.1505016722408)(1047.94648829431,-10.2943143812709)(69.2040133779264,-7.04347826086957)(553.632107023411,-9.21070234113712)(499.257525083612,-6.8628762541806)(919.42474916388,-7.16387959866221)(306.47491638796,-7.58528428093646)(1478,-9.45150501672241)(1146.8093645485,-9.1505016722408)(603.063545150502,-8.36789297658863)(1364.30769230769,-8.96989966555184)(1092.4347826087,-8.12709030100334)(123.578595317726,-7.76588628762542)(1374.19397993311,-10.4147157190635)(850.220735785953,-8.60869565217391)(395.451505016722,-8.42809364548495)(840.334448160535,-9.87290969899666)(222.441471571906,-8.96989966555184)(1260.5016722408,-7.40468227424749)(632.722408026756,-7.22408026755853)(711.8127090301,-9.21070234113712)(1216.01337792642,-10.2943143812709)(479.484949832776,-7.64548494983278)(444.882943143813,-9.21070234113712)(726.642140468227,-8.24749163879599)(973.799331103679,-7.94648829431438)(1013.34448160535,-9.63210702341137)(69.2040133779264,-8.30769230769231)(1413.73913043478,-8.30769230769231)(227.384615384615,-7.28428093645485)(1245.67224080268,-8.24749163879599)(860.107023411371,-7.46488294314381)(1250.61538461538,-9.21070234113712)(1102.32107023411,-7.16387959866221)(380.622073578595,-7.22408026755853)(1478,-9.9933110367893)(1092.4347826087,-10.2943143812709)(1033.11705685619,-8.72909698996656)(301.531772575251,-8.72909698996656)(553.632107023411,-6.8628762541806)(1354.42140468227,-9.57190635451505)(860.107023411371,-9.1505016722408)(1478,-9.09030100334448)(588.234113712375,-9.21070234113712)(968.85618729097,-10.2341137123746)(237.270903010033,-8.06688963210702)(84.0334448160535,-7.28428093645485)(1280.27424749164,-10.3545150501672)(603.063545150502,-7.76588628762542)(153.23745819398,-8.90969899665552)(1329.70568561873,-7.7056856187291)(1151.7525083612,-8.0066889632107)(439.939799331104,-8.18729096989967)(1136.92307692308,-9.45150501672241)(983.685618729097,-7.22408026755853)(805.732441471572,-8.18729096989967)(573.404682274248,-8.60869565217391)(459.71237458194,-9.21070234113712)(790.903010033445,-9.45150501672241)(484.428093645485,-6.8628762541806)(944.140468227425,-8.42809364548495)(1354.42140468227,-8.60869565217391)(652.494983277592,-7.40468227424749)(682.153846153846,-8.54849498327759)(879.879598662207,-9.93311036789298)(331.190635451505,-7.88628762541806)(1117.15050167224,-8.72909698996656)(1393.96655518395,-9.81270903010033)(1186.35451505017,-10.2943143812709)(1166.58193979933,-7.40468227424749)(988.628762541806,-9.39130434782609)(123.578595317726,-7.82608695652174)(1240.72909698997,-8.90969899665552)(1018.28762541806,-7.82608695652174)(860.107023411371,-7.58528428093646)(375.678929765886,-8.72909698996656)(454.769230769231,-7.58528428093646)(227.384615384615,-7.52508361204013)(1057.83277591973,-10.2341137123746)(1255.55852842809,-8.0066889632107)(128.521739130435,-8.48829431438127)(1408.79598662207,-8.48829431438127)(874.936454849498,-8.8494983277592)(1379.13712374582,-10.2341137123746)(1255.55852842809,-9.63210702341137)(519.030100334448,-9.21070234113712)(242.214046822742,-8.66889632107023)(998.515050167224,-7.22408026755853)(766.1872909699,-8.30769230769231)(1379.13712374582,-9.33110367892977)(1112.20735785953,-9.57190635451505)(627.779264214047,-8.8494983277592)(568.461538461538,-7.34448160535117)(1003.45819397993,-8.78929765886288)(1122.09364548495,-8.06688963210702)(454.769230769231,-8.54849498327759)(662.38127090301,-8.0066889632107)(795.846153846154,-9.33110367892977)(390.508361204013,-7.58528428093646)(153.23745819398,-7.46488294314381)(919.42474916388,-9.9933110367893)(1349.47826086957,-8.0066889632107)(909.538461538462,-8.06688963210702)(1230.84280936455,-7.58528428093646)(1255.55852842809,-10.2341137123746)(291.645484949833,-8.12709030100334)(1245.67224080268,-8.72909698996656)(1077.60535117057,-7.34448160535117)(914.481605351171,-7.34448160535117)(1127.03678929766,-10.2341137123746)(1453.28428093645,-9.45150501672241)(988.628762541806,-9.57190635451505)(1112.20735785953,-8.78929765886288)(108.749163879599,-8.0066889632107)(464.655518394649,-7.16387959866221)(494.314381270903,-9.21070234113712)(815.61872909699,-7.88628762541806)(177.953177257525,-8.72909698996656)(1423.6254180602,-8.72909698996656)(899.652173913044,-9.09030100334448)(563.518394648829,-7.04347826086957)(766.1872909699,-8.90969899665552)(1196.24080267559,-9.57190635451505)(1008.40133779264,-8.06688963210702)(617.892976588629,-8.0066889632107)(247.157190635452,-7.58528428093646)(1028.17391304348,-10.1739130434783)(1389.02341137124,-10.1137123745819)(617.892976588629,-8.90969899665552)(400.394648829431,-8.72909698996656)(855.163879598662,-9.75250836120401)(1285.21739130435,-7.7056856187291)(1304.98996655518,-9.39130434782609)(420.167224080268,-8.0066889632107)(1122.09364548495,-7.46488294314381)(1290.16053511706,-8.54849498327759)(978.742474916388,-7.22408026755853)(869.993311036789,-8.36789297658863)(266.929765886288,-8.54849498327759)(1062.77591973244,-9.27090301003344)(1161.63879598662,-8.30769230769231)(1384.08026755853,-8.24749163879599)(1453.28428093645,-9.69230769230769)(1181.41137123746,-10.2341137123746)(182.896321070234,-8.18729096989967)(869.993311036789,-7.52508361204013)(978.742474916388,-10.1137123745819)(983.685618729097,-8.66889632107023)(622.836120401338,-7.52508361204013)(523.973244147157,-9.21070234113712)(761.244147157191,-8.54849498327759)(434.996655518395,-7.34448160535117)(1181.41137123746,-9.09030100334448)(1309.93311036789,-10.1137123745819)(1379.13712374582,-9.09030100334448)(133.464882943144,-7.64548494983278)(360.849498327759,-8.12709030100334)(1225.89966555184,-7.58528428093646)(800.789297658863,-9.33110367892977)(148.294314381271,-8.60869565217391)(1062.77591973244,-8.12709030100334)(642.608695652174,-8.72909698996656)(934.254180602007,-9.33110367892977)(1102.32107023411,-9.69230769230769)(430.053511705686,-8.90969899665552)(958.969899665552,-7.88628762541806)(558.57525083612,-7.10367892976589)(1433.51170568562,-8.8494983277592)(1339.59197324415,-7.94648829431438)(657.438127090301,-8.12709030100334)(271.872909698997,-7.82608695652174)(810.675585284281,-7.94648829431438)(1285.21739130435,-9.63210702341137)(1211.07023411371,-8.24749163879599)(1077.60535117057,-7.40468227424749)(222.441471571906,-8.72909698996656)(1403.85284280936,-10.0535117056856)(934.254180602007,-7.28428093645485)(899.652173913044,-9.87290969899666)(1077.60535117057,-8.66889632107023)(874.936454849498,-8.66889632107023)(593.177257525084,-9.03010033444816)(1033.11705685619,-10.1739130434783)(444.882943143813,-7.64548494983278)(1290.16053511706,-8.78929765886288)(118.635451505017,-8.12709030100334)(1453.28428093645,-9.33110367892977)(1181.41137123746,-9.69230769230769)(1166.58193979933,-7.52508361204013)(192.782608695652,-7.58528428093646)(425.110367892977,-8.42809364548495)(627.779264214047,-7.58528428093646)(1013.34448160535,-9.21070234113712)(776.073578595318,-9.03010033444816)(484.428093645485,-7.16387959866221)(1220.95652173913,-10.1739130434783)(1181.41137123746,-8.90969899665552)(840.334448160535,-7.7056856187291)(484.428093645485,-9.1505016722408)(1329.70568561873,-7.88628762541806)(954.026755852843,-8.24749163879599)(855.163879598662,-9.69230769230769)(1112.20735785953,-10.1739130434783)(286.702341137124,-8.24749163879599)(1408.79598662207,-8.54849498327759)(1013.34448160535,-7.28428093645485)(1324.76254180602,-10.1137123745819)(776.073578595318,-8.36789297658863)(1453.28428093645,-9.63210702341137)(1107.26421404682,-8.06688963210702)(637.665551839465,-8.30769230769231)(143.351170568562,-8.54849498327759)(1260.5016722408,-7.64548494983278)(1334.64882943144,-9.27090301003344)(924.367892976589,-9.03010033444816)(588.234113712375,-7.22408026755853)(968.85618729097,-10.0535117056856)(1072.66220735786,-9.45150501672241)(400.394648829431,-7.82608695652174)(1270.38795986622,-8.42809364548495)(568.461538461538,-9.09030100334448)(899.652173913044,-7.40468227424749)(148.294314381271,-7.64548494983278)(1013.34448160535,-7.82608695652174)(795.846153846154,-9.27090301003344)(430.053511705686,-8.8494983277592)(1062.77591973244,-8.60869565217391)(1225.89966555184,-9.27090301003344)(874.936454849498,-8.36789297658863)(1379.13712374582,-8.24749163879599)(1413.73913043478,-9.9933110367893)(1181.41137123746,-8.06688963210702)(232.327759197324,-8.60869565217391)(494.314381270903,-7.28428093645485)(642.608695652174,-7.82608695652174)(1127.03678929766,-10.1739130434783)(247.157190635452,-7.88628762541806)(1280.27424749164,-10.1137123745819)(973.799331103679,-9.57190635451505)(612.94983277592,-8.72909698996656)(1379.13712374582,-9.1505016722408)(1107.26421404682,-7.52508361204013)(1131.97993311037,-9.03010033444816)(415.224080267559,-8.18729096989967)(790.903010033445,-8.18729096989967)(973.799331103679,-8.72909698996656)(869.993311036789,-9.75250836120401)(939.197324414716,-7.28428093645485)(1255.55852842809,-7.64548494983278)(489.371237458194,-9.1505016722408)(850.220735785953,-7.64548494983278)(1052.88963210702,-10.1739130434783)(1453.28428093645,-9.57190635451505)(860.107023411371,-8.96989966555184)(1300.04682274247,-8.72909698996656)(1295.10367892977,-9.63210702341137)(954.026755852843,-7.94648829431438)(1201.18394648829,-10.1739130434783)(1433.51170568562,-8.96989966555184)(588.234113712375,-7.28428093645485)(1161.63879598662,-8.48829431438127)(138.408026755853,-8.18729096989967)(1033.11705685619,-7.34448160535117)(1151.7525083612,-9.57190635451505)(776.073578595318,-8.90969899665552)(1364.30769230769,-8.12709030100334)(637.665551839465,-8.36789297658863)(434.996655518395,-7.64548494983278)(266.929765886288,-8.30769230769231)(1033.11705685619,-9.57190635451505)(1161.63879598662,-7.58528428093646)(434.996655518395,-8.66889632107023)(1047.94648829431,-8.36789297658863)(815.61872909699,-9.45150501672241)(1354.42140468227,-10.0535117056856)(934.254180602007,-9.93311036789298)(222.441471571906,-7.76588628762542)(212.555183946488,-8.66889632107023)(558.57525083612,-9.09030100334448)(1235.78595317726,-9.03010033444816)(1300.04682274247,-7.76588628762542)(894.709030100334,-8.24749163879599)(973.799331103679,-8.96989966555184)(632.722408026756,-7.76588628762542)(1240.72909698997,-8.24749163879599)(558.57525083612,-7.22408026755853)(1087.49163879599,-9.09030100334448)(781.016722408027,-8.48829431438127)(1379.13712374582,-9.39130434782609)(1052.88963210702,-7.82608695652174)(1062.77591973244,-10.1739130434783)(884.822742474916,-7.46488294314381)(1220.95652173913,-9.69230769230769)(415.224080267559,-8.12709030100334)(1413.73913043478,-8.72909698996656)(904.595317725752,-9.27090301003344)(138.408026755853,-7.94648829431438)(608.006688963211,-8.78929765886288)(1453.28428093645,-9.51170568561873)(1161.63879598662,-10.1739130434783)(1127.03678929766,-8.42809364548495)(449.826086956522,-8.90969899665552)(1319.81939799331,-8.72909698996656)(835.391304347826,-7.76588628762542)(192.782608695652,-8.66889632107023)(1151.7525083612,-7.58528428093646)(958.969899665552,-9.9933110367893)(968.85618729097,-7.82608695652174)(1349.47826086957,-10.0535117056856)(637.665551839465,-8.06688963210702)(257.04347826087,-8.0066889632107)(810.675585284281,-9.39130434782609)(1082.54849498328,-9.63210702341137)(988.628762541806,-8.54849498327759)(1191.29765886288,-8.90969899665552)(855.163879598662,-8.60869565217391)(1290.16053511706,-9.39130434782609)(1240.72909698997,-7.64548494983278)(1240.72909698997,-10.1137123745819)(612.94983277592,-7.46488294314381)(1349.47826086957,-8.06688963210702)(1077.60535117057,-7.46488294314381)(143.351170568562,-8.30769230769231)(874.936454849498,-9.75250836120401)(543.745819397993,-9.09030100334448)(1448.34113712375,-9.69230769230769)(1028.17391304348,-9.09030100334448)(1260.5016722408,-8.30769230769231)(622.836120401338,-8.66889632107023)(1156.69565217391,-9.57190635451505)(1151.7525083612,-8.12709030100334)(1097.3779264214,-10.1739130434783) 
};

%\node at (axis cs:50, -2.5) [shape=circle,fill=white,draw=black,inner sep=0pt,anchor=south west] {\scriptsize\color{locol}$\*L_{\*t}$};
%\node at (axis cs:460, -5.9) [shape=circle,fill=white,draw=black,inner sep=0pt,anchor=south west] {\scriptsize\color{orange!50!yellow}$\*H_{\*t}$};
%\node at (axis cs:160, -5.2) [shape=circle,fill=white,draw=black,inner sep=0pt,anchor=south west] {\scriptsize\color{darkgray}$\*U_{\*t}$};

\node at (axis cs:1155, -17) [shape=circle,fill=red!40!yellow,draw=black,inner sep=0.2pt,anchor=south west,minimum size=16pt]
  {\scriptsize\color{white}$\hat{\*H}$};
\node at (axis cs:1330, -17) [shape=circle,fill=locol,draw=black,inner sep=0.2pt,anchor=south west,minimum size=16pt]
  {\scriptsize\color{white}$\hat{\*L}$};

\end{axis}
\end{tikzpicture}%

\renewcommand\trimlen{3pt}
\begin{figure}[tb]
  \begin{subfigure}[b]{0.49\linewidth}
    \centering
    \adjincludegraphics[width=\linewidth,clip=true,trim=\trimlen{} \trimlen{} \trimlen{} \trimlen{}]{figures/ch03/limno_bgape_imp_class50}
    \caption{$t = 50$}
    \label{fig:limno_bgape_imp_class1}
  \end{subfigure}
  \hfill
  \begin{subfigure}[b]{0.49\linewidth}
    \centering
    \adjincludegraphics[width=\linewidth,clip=true,trim=\trimlen{} \trimlen{} \trimlen{} \trimlen{}]{figures/ch03/limno_bgape_imp_class100}
    \caption{$t = 100$}
    \label{fig:limno_bgape_imp_class2}
  \end{subfigure}
  \begin{subfigure}[b]{0.49\linewidth}
    \centering
    \vspace{12pt} % space of this row from above captions
    \adjincludegraphics[width=\linewidth,clip=true,trim=\trimlen{} \trimlen{} \trimlen{} \trimlen{}]{figures/ch03/limno_bgape_imp_class200}
    \caption{$t = 200$}
    \label{fig:limno_bgape_imp_class3}
  \end{subfigure}
  \hfill
  \begin{subfigure}[b]{0.49\linewidth}
    \centering
    \adjincludegraphics[width=\linewidth,clip=true,trim=\trimlen{} \trimlen{} \trimlen{} \trimlen{}]{figures/ch03/limno_bgape_imp_class440}
    \caption{$t = 440$ (termination)}
    \label{fig:limno_bgape_imp_class4}
  \end{subfigure}
  \caption{
      Running \iacl with $\epsilon = 0.7$ on a regular grid of $100\times 100$ points
      sampled from the inferred algae concentration GP of \figref{fig:limno_bgape}.
      Regions of already classified points are shown in orange ($H_t$) and blue ($L_t$),
      regions of yet unclassified points ($U_t$) in black, regions of points that could
      be maximizers ($M_t = M_t^L\cup M_t^H$) in green, and observed points
      ($\{\*x_i\}_{1\leq i\leq t}$) as white marks.
  }
\end{figure}

\figsref{fig:limno_bgape_imp_class1} to \ref{fig:limno_bgape_imp_class4}
present an example of running the \iacl algorithm on a fine grid of points
sampled from the inferred GP of the algae concentration dataset shown in
\figref{fig:limno_bgape}. Note how, in contrast to the explicit threshold case,
the sampling process now also focuses on the regions of near-maximal function
value in addition to the ambiguous regions around the implied level set.

\paragraph{Theoretical analysis}
The idea of using the maximum information gain can again be used to provide
a convergence bound for the \iacl algorithm. Some modifications to the analysis
are required due to the difference in the classification rules, i.e. the use
of estimates $h_t^{opt}$ and $h_t^{pes}$ instead of $h$ and the use of the
extended set $Z_t$ instead of $U_t$.
The following theorem expresses for the \iacl algorithm a similar in form bound
to that of \theoremref{thm:acl}.

\begin{theorem}
\label{thm:iacl}
For any $\omega \in (0, 1)$, $\delta \in (0, 1)$, and $\epsilon > 0$,
if $\beta_t = 2\log(|D|\pi^2 t^2/(6\delta))$, \iacl terminates after
at most $T$ iterations, where $T$ is the smallest positive integer
satisfying
\begin{align*}
\frac{T}{\beta_T \gamma_T} \geq \frac{C_1(1+\omega)^2}{4\epsilon^2},
%T/(\beta_T \gamma_T) \geq C_1(1+\omega)^2/(4\epsilon^2),
\end{align*}
where $C_1 = 8/\log(1 + \sigma^{-2})$.

Furthermore, with probability at least $1-\delta$, the algorithm returns
an $\epsilon$-accurate solution with respect to the implicit level
$h = \omega \max_{\*x\in D} f(\*x)$, that is
\begin{align*}
\Pr\left\{\max_{\*x\in D}\ell_h(\*x) \leq \epsilon\right\} \geq 1 - \delta.
\end{align*}
\end{theorem}

The detailed proof of \theoremref{thm:iacl} can be found in
\sectref{sect:app_iacl}. Here we outline the main similarities and differences
compared to the proof of \theoremref{thm:acl}.
As in the explicit threshold case, using the maximum information gain $\gamma_t$,
we show that $w_t(\*x_t)$ decreases as
$\mathcal{O}((\frac{\beta_t \gamma_t}{t})^\frac{1}{2})$\footnote{In fact, we could
have used \iacl's selection rule in \acl without any change in the convergence bound
of \theoremref{thm:acl}. However, as we will see in the following chapter, the
ambiguity-based selection rule performs slightly better in practice.}. Furthermore,
as before, the ``validity'' of the confidence regions is guaranteed by choosing
$\beta_t$ appropriately. However, the steps of proving
termination and $\epsilon$-accuracy need to be modified as follows.
\begin{description}
\item[Termination.] Due to \iacl's classification rules, in order to prove that
      the algorithm terminates (\lemmaref{lem:isterm}), we need to show that
      the gap between the
      optimistic $h_t^{opt}$ and pessimistic $h_t^{pes}$ threshold level
      estimates gets
      small enough. This is accomplished by bounding $h_t^{opt} - h_t^{pes}$
      by a constant multiple of $w_t(\*x_t)$ (\lemmaref{lem:hdif}) and using
      the known decreasing bound on the latter (\lemmaref{lem:iwbound}).
\item[Solution accuracy.] To prove $\epsilon$-accuracy of the returned
      solution with respect to the implicit threshold level, we prove 
      that our optimistic and pessimistic estimates are valid upper and
      lower bounds of the true implicit level at each iteration, i.e.
      that $h_t^{opt} \geq h$ and $h_t^{pes} \leq h$, for all $t\geq 1$
      (\lemmasref{lem:hopt} and~\ref{lem:hpes}). This is the point
      where keeping all possible maximizers as sampling candidates in \iacl
      is formally required
      (cf. \lemmasref{lem:fpes_inc} and \lemmaref{lem:mmeq}).
\end{description}

Note that the sample complexity bound of \theoremref{thm:iacl} is a factor
$(1+\omega)^2\leq 4$ larger than that of \theoremref{thm:acl}, and that
$\omega=0$ actually reduces to an explicit threshold of $0$.

\section{Batch sample selection} \label{sect:bacl}
In the batch setting, the algorithms are only allowed to use the
observed values of previous batches when selecting samples for the
current batch.
A naive way of extending \acl (resp. \iacl) to this setting would be to
modify the selection rule so that, instead of picking the point with
the largest ambiguity (resp.~width), it chooses the $B$ highest ranked
points. However, this approach tends to select ``clusters'' of closely
located samples with high ambiguities (resp.~widths), ignoring
the decrease in the estimated variance of a point resulting from sampling
another point nearby.

Fortunately, we can handle the above issue by exploiting a key property of GPs,
namely that
the predictive variances \eqtref{eq:var} depend only on the selected points
$\*x_t$ and not on the observed values $y_t$ at those points. Therefore,
even if we do not
have available feedback for each selected point up to iteration $t$, we
can still obtain the following useful confidence intervals
\begin{align*}
Q_t^b(\*x) = \left[\mu_{\fb[t]}(\*x) \pm \eta_t^{1/2}\sigma_{t-1}(\*x)\right],
\end{align*}
which combine the most recent available mean estimate
with the always up-to-date variance estimate.
Here $\fb[t]$ is the index of the last available observation expressed
using the feedback function $\fb$ introduced in \sectref{sect:prelim}.
Confidence regions $C_t^b(\*x)$ are
defined as before by intersecting successive confidence intervals and are
used without any further changes in the algorithms.

The pseudocode of \algoref{alg:bacl} highlights the way in which
evaluation feedback is obtained in \bacl. Variable $t_{\fb}$ holds the
latest step for which there is available feedback at each iteration and
the inferred mean is updated whenever new feedback is available, as
dictated by $\fb[t+1]$. However, note that the inferred variance is
updated at each iteration, irrespectively of available feedback.
The batch extension of \iacl works in a completely analogous way.

\begin{algorithm}[tb]
  \caption{The \bacl extension}
  \label{alg:bacl}
\small{
\begin{algorithmic}[1]
  \REQUIRE sample set $D$, GP prior ($\mu_0 = 0$, $k$, $\sigma_0$),\\
           \hspace{1.35em}threshold value $h$, accuracy parameter $\epsilon$
  \ENSURE predicted sets $\hat{H}$, $\hat{L}$
  \STATE $H_0 \gets \varnothing$,\enskip $L_0 \gets \varnothing$,\enskip $U_0 \gets D$ \label{lin:binit1}
  %\LET{$H_0$}{$\varnothing$} \label{lin:init1}
  %\LET{$L_0$}{$\varnothing$}
  %\LET{$U_0$}{$D$}
  \LET{$C^{\new{b}}_0(\*x)$}{$\mathbb{R}$, for all $\*x \in D$} \label{lin:binit2}
  \LET{$t$}{1}
  \new{\LET{$t_{\fb}$}{0}}
  \WHILE{$U_{t-1} \neq \varnothing$}
    \STATE $H_t \gets H_{t-1}$,\enskip $L_t \gets L_{t-1}$,\enskip $U_t \gets U_{t-1}$
    %\LET{$H_t$}{$H_{t-1}$}
    %\LET{$L_t$}{$L_{t-1}$}
    %\LET{$U_t$}{$U_{t-1}$}
    \FORALL{$\*x \in U_{t-1}$}
      \LET{$C^{\new{b}}_{t}(\*x)$}{$C^{\new{b}}_{t-1}(\*x) \cap Q^{\new{b}}_t(\*x)$} \label{lin:bupd}
      \IF{$\min(C^{\new{b}}_t(\*x)) + \epsilon > h$} \label{lin:bclass1}
        \LET{$U_t$}{$U_t \setminus \{\*x\}$}
        \LET{$H_t$}{$H_t \cup \{\*x\}$} 
      \ELSIF{$\max(C^{\new{b}}_t(\*x)) - \epsilon \leq h$} \label{lin:bclassr2}
        \LET{$U_t$}{$U_t \setminus \{\*x\}$}
        \LET{$L_t$}{$L_t \cup \{\*x\}$}
      \ENDIF \label{lin:bclass2}
    \ENDFOR
    \LET{$\*x_t$}{$\argmax_{\*x \in U_t}(a^{\new{b}}_t(\*x))$} \label{lin:sel1}
    \new{
    \IF{$\fb[t+1] > t_{\fb}$}
      \FOR{$i = t_{\fb}+1,\ldots,\fb[t+1]$} \label{lin:bsample1}
        \LET{$y_i$}{$f(\*x_i) + \nu_i$}
      \ENDFOR \label{lin:bsample2}
      \STATE Compute $\mu_t(\*x)$ for all $\*x \in U_t$
      \LET{$t_{\fb}$}{$\fb[t+1]$}
    \ENDIF
    }
    \STATE Compute $\sigma_t(\*x)$ for all $\*x \in U_t$
    \LET{$t$}{$t + 1$}
  \ENDWHILE
  \LET{$\hat{H}$}{$H_{t-1}$} \label{lin:bret1}
  \LET{$\hat{L}$}{$L_{t-1}$} \label{lin:bret2}
\end{algorithmic}
}
\end{algorithm}

\paragraph{Theoretical analysis}
Intuitively, to extend the convergence guarantees of the sequential algorithms
we have to compensate for using outdated mean estimates
by employing a more conservative
(i.e. larger) scaling parameter $\eta_t$ as compared to $\beta_t$.
This ensures that the resulting confidence regions $C_t^b(\*x)$
still contain $f(\*x)$ with high probability.

To appropriately adjust the confidence interval scaling parameter,
in their analysis for extending the \gpucb algorithm to the batch
setting \citet{desautels12} utilized the
\emph{conditional information gain}
\begin{align*}
I(\*y_A; f \mid \*y_{1:\fb[t]}) = H(\*y_A\mid\*y_{1:\fb[t]}) - H(\*y_A\mid f),
\end{align*}
which quantifies the reduction in uncertainty about $f$ by obtaining
a number of observations $\*y_A$, given that we already have observations
$\*y_{1:\fb[t]}$ available.
Following a similar treatment, we extend the convergence bound of
\theoremref{thm:acl} to the batch selection setting of
\bacl via bounding the maximum conditional information gain, resulting
in the following theorem.

\begin{theorem}
\label{thm:bacl}
Assume that the feedback delay $t - \fb[t]$ is at most $B$ for all $t \geq 1$,
where $B$ is a known constant.
Also, assume that for all $t \geq 1$ the maximum conditional mutual information
acquired by any set of measurements since the last feedback is bounded
by a constant $C \geq 0$, i.e.
\begin{align*}
\max_{A\subseteq D, |A|\leq B-1} I(f; \*y_A \mid \*y_{1:\fb[t]}) \leq C
\end{align*}
Then, for any $h\in\mathbb{R}$, $\delta \in (0, 1)$, and $\epsilon \geq 0$,
if $\eta_t = e^C\beta_{fb[t]+1}$, \bacl terminates after
at most $T$ iterations, where $T$ is the smallest positive integer
satisfying
\begin{align*}
\frac{T}{\eta_T \gamma_T} \geq \frac{C_1}{4\epsilon^2},
\end{align*}
where $C_1 = 8/\log(1 + \sigma^{-2})$.

Furthermore, with probability at least $1-\delta$, the algorithm returns
an $\epsilon$-accurate solution, that is
\begin{align*}
\Pr\left\{\max_{\*x\in D}\ell_h(\*x) \leq \epsilon\right\} \geq 1 - \delta.
\end{align*}
\end{theorem}

The detailed proof of \theoremref{thm:bacl} can be found in
\sectref{sect:app_bacl} and a completely analogous theorem can be
formulated for extending \iacl to the batch setting.

Note that, as intuitively described above, the scaling parameter
$\eta_t$ has to increase by a factor of $e^C$ to compensate for the outdated
mean estimates used in the confidence regions $C_t^b(\*x)$. Normally, $C$
depends on the batch size $B$. However, \citet{desautels12} have shown that,
initializing their \gpbucb algorithm with a number of sequentially selected
maximum variance samples, results in a constant factor increase of $\eta_t$
compared to $\beta_t$, independently of $B$.
In the following chapter, we also use maximum variance initialization in our
experiments, which allows us to select a constant value of $\eta_t$
(larger than $\beta_t$)
that works well across different batch sizes.

\section{Path planning} \label{sect:pp} Up to this point, we have assumed that obtaining a sample incurs a fixed
cost independent of its location in the input space $D$. However, in practical
applications like the environmental monitoring example of \chapref{ch:intro},
obtaining a sample also involves moving a mobile sensor to the desired
location, which implies an additional traveling cost that depends on the
measurement location. In what follows, we present two practical ways for
computing appropriate sampling paths: the first is based on the batch sampling
extension of the previous section; the second is tailored to the task of
environmental monitoring in Lake Zurich.

\paragraph{Batch-based path planning}
Our first method is a straightforward application of batch point selection for
creating sampling paths. In particular, \bacl is used to select a batch of
$B$ points, but, instead of sampling at those points in the order in which
they were selected (\linesref{lin:bsample1}{lin:bsample2}), we first use a
Euclidean TSP solver to create a path connecting the last sampled point of
the previous batch and all points of the current batch.
\figref{fig:limno_bgape_ppbatch} shows an example of the above procedure
on the algae concentration dataset for a batch size of $B=30$.

%\setlength\figureheight{1.5in}\setlength\figurewidth{2.5in}
%% This file was created by matlab2tikz v0.2.3.
% Copyright (c) 2008--2012, Nico Schlömer <nico.schloemer@gmail.com>
% All rights reserved.
% 
% 
%

\definecolor{locol}{rgb}{0.26, 0.45, 0.65}

\begin{tikzpicture}

\begin{axis}[%
tick label style={font=\tiny},
label style={font=\tiny},
xlabel shift={-10pt},
ylabel shift={-17pt},
legend style={font=\tiny},
view={0}{90},
width=\figurewidth,
height=\figureheight,
scale only axis,
xmin=0, xmax=1478,
xtick={0, 400, 1000, 1400},
xlabel={Length (m)},
ymin=-18, ymax=0,
ytick={0, -4, -14, -18},
ylabel={Depth (m)},
name=plot1,
axis lines*=box,
tickwidth=0.1cm,
clip=false
]

\addplot [fill=locol,draw=none,forget plot] coordinates{ (1478,0)(1478,-0.181818181818182)(1478,-0.363636363636364)(1478,-0.545454545454545)(1478,-0.727272727272727)(1478,-0.909090909090909)(1478,-1.09090909090909)(1478,-1.27272727272727)(1478,-1.45454545454545)(1478,-1.63636363636364)(1478,-1.81818181818182)(1478,-2)(1478,-2.18181818181818)(1478,-2.36363636363636)(1478,-2.54545454545455)(1478,-2.72727272727273)(1478,-2.90909090909091)(1478,-3.09090909090909)(1478,-3.27272727272727)(1478,-3.45454545454545)(1478,-3.63636363636364)(1478,-3.81818181818182)(1478,-4)(1478,-4.18181818181818)(1478,-4.36363636363636)(1478,-4.54545454545455)(1478,-4.72727272727273)(1478,-4.90909090909091)(1478,-5.09090909090909)(1478,-5.27272727272727)(1478,-5.45454545454545)(1478,-5.63636363636364)(1478,-5.81818181818182)(1478.00037322299,-6)(1478.00074642733,-6.18181818181818)(1478.001119613,-6.36363636363636)(1478.001119613,-6.54545454545455)(1478.001119613,-6.72727272727273)(1478.001119613,-6.90909090909091)(1478.001119613,-7.09090909090909)(1478.00087082462,-7.27272727272727)(1478.00087082462,-7.45454545454545)(1478.00087082462,-7.63636363636364)(1478.001119613,-7.81818181818182)(1478.001119613,-8)(1478.001119613,-8.18181818181818)(1478.001119613,-8.36363636363636)(1478.001119613,-8.54545454545455)(1478.00087082462,-8.72727272727273)(1478.00087082462,-8.90909090909091)(1478.00087082462,-9.09090909090909)(1478.001119613,-9.27272727272727)(1478.001119613,-9.45454545454546)(1478.001119613,-9.63636363636364)(1478.00087082462,-9.81818181818182)(1478.00049762651,-10)(1478.00012440974,-10.1818181818182)(1478,-10.3636363636364)(1478,-10.5454545454545)(1478,-10.7272727272727)(1478,-10.9090909090909)(1478,-11.0909090909091)(1478,-11.2727272727273)(1478,-11.4545454545455)(1478,-11.6363636363636)(1478,-11.8181818181818)(1478,-12)(1478,-12.1818181818182)(1478,-12.3636363636364)(1478,-12.5454545454545)(1478,-12.7272727272727)(1478,-12.9090909090909)(1478,-13.0909090909091)(1478,-13.2727272727273)(1478,-13.4545454545455)(1478,-13.6363636363636)(1478,-13.8181818181818)(1478,-14)(1478,-14.1818181818182)(1478,-14.3636363636364)(1478,-14.5454545454545)(1478,-14.7272727272727)(1478,-14.9090909090909)(1478,-15.0909090909091)(1478,-15.2727272727273)(1478,-15.4545454545455)(1478,-15.6363636363636)(1478,-15.8181818181818)(1478,-16)(1478,-16.1818181818182)(1478,-16.3636363636364)(1478,-16.5454545454545)(1478,-16.7272727272727)(1478,-16.9090909090909)(1478,-17.0909090909091)(1478,-17.2727272727273)(1478,-17.4545454545455)(1478,-17.6363636363636)(1478,-17.8181818181818)(1478,-18)(1463.07070707071,-18)(1448.14141414141,-18)(1433.21212121212,-18)(1418.28282828283,-18)(1403.35353535354,-18)(1388.42424242424,-18)(1373.49494949495,-18)(1358.56565656566,-18)(1343.63636363636,-18)(1328.70707070707,-18)(1313.77777777778,-18)(1298.84848484848,-18)(1283.91919191919,-18)(1268.9898989899,-18)(1254.06060606061,-18)(1239.13131313131,-18)(1224.20202020202,-18)(1209.27272727273,-18)(1194.34343434343,-18)(1179.41414141414,-18)(1164.48484848485,-18)(1149.55555555556,-18)(1134.62626262626,-18)(1119.69696969697,-18)(1104.76767676768,-18)(1089.83838383838,-18)(1074.90909090909,-18)(1059.9797979798,-18)(1045.05050505051,-18)(1030.12121212121,-18)(1015.19191919192,-18)(1000.26262626263,-18)(985.333333333333,-18)(970.40404040404,-18)(955.474747474747,-18)(940.545454545455,-18)(925.616161616162,-18)(910.686868686869,-18)(895.757575757576,-18)(880.828282828283,-18)(865.89898989899,-18)(850.969696969697,-18)(836.040404040404,-18)(821.111111111111,-18)(806.181818181818,-18)(791.252525252525,-18)(776.323232323232,-18)(761.393939393939,-18)(746.464646464646,-18)(731.535353535354,-18)(716.606060606061,-18)(701.676767676768,-18)(686.747474747475,-18)(671.818181818182,-18)(656.888888888889,-18)(641.959595959596,-18)(627.030303030303,-18)(612.10101010101,-18)(597.171717171717,-18)(582.242424242424,-18)(567.313131313131,-18)(552.383838383838,-18)(537.454545454546,-18)(522.525252525253,-18)(507.59595959596,-18)(492.666666666667,-18)(477.737373737374,-18)(462.808080808081,-18)(447.878787878788,-18)(432.949494949495,-18)(418.020202020202,-18)(403.090909090909,-18)(388.161616161616,-18)(373.232323232323,-18)(358.30303030303,-18)(343.373737373737,-18)(328.444444444444,-18)(313.515151515152,-18)(298.585858585859,-18)(283.656565656566,-18)(268.727272727273,-18)(253.79797979798,-18)(238.868686868687,-18)(223.939393939394,-18)(209.010101010101,-18)(194.080808080808,-18)(179.151515151515,-18)(164.222222222222,-18)(149.292929292929,-18)(134.363636363636,-18)(119.434343434343,-18)(104.505050505051,-18)(89.5757575757576,-18)(74.6464646464647,-18)(59.7171717171717,-18)(44.7878787878788,-18)(29.8585858585859,-18)(14.9292929292929,-18)(0,-18)(0,-17.8181818181818)(0,-17.6363636363636)(0,-17.4545454545455)(0,-17.2727272727273)(0,-17.0909090909091)(0,-16.9090909090909)(0,-16.7272727272727)(0,-16.5454545454545)(0,-16.3636363636364)(0,-16.1818181818182)(0,-16)(0,-15.8181818181818)(0,-15.6363636363636)(0,-15.4545454545455)(0,-15.2727272727273)(0,-15.0909090909091)(0,-14.9090909090909)(0,-14.7272727272727)(0,-14.5454545454545)(0,-14.3636363636364)(0,-14.1818181818182)(0,-14)(0,-13.8181818181818)(0,-13.6363636363636)(0,-13.4545454545455)(0,-13.2727272727273)(0,-13.0909090909091)(0,-12.9090909090909)(0,-12.7272727272727)(0,-12.5454545454545)(0,-12.3636363636364)(0,-12.1818181818182)(0,-12)(0,-11.8181818181818)(0,-11.6363636363636)(0,-11.4545454545455)(0,-11.2727272727273)(0,-11.0909090909091)(0,-10.9090909090909)(-0.000373222992657963,-10.7272727272727)(-0.000746427325097138,-10.5454545454545)(-0.00111961299872307,-10.3636363636364)(-0.00111961299872307,-10.1818181818182)(-0.00111961299872307,-10)(-0.00111961299872307,-9.81818181818182)(-0.00111961299872307,-9.63636363636364)(-0.00111961299872307,-9.45454545454546)(-0.000746427325097138,-9.27272727272727)(-0.000373222992657963,-9.09090909090909)(0,-8.90909090909091)(0,-8.72727272727273)(0,-8.54545454545455)(0,-8.36363636363636)(0,-8.18181818181818)(0,-8)(-0.00012440973766168,-7.81818181818182)(-0.000248817401863321,-7.63636363636364)(-0.000248817401863321,-7.45454545454545)(-0.00012440973766168,-7.27272727272727)(0,-7.09090909090909)(0,-6.90909090909091)(0,-6.72727272727273)(0,-6.54545454545455)(0,-6.36363636363636)(0,-6.18181818181818)(0,-6)(0,-5.81818181818182)(0,-5.63636363636364)(0,-5.45454545454545)(0,-5.27272727272727)(0,-5.09090909090909)(0,-4.90909090909091)(0,-4.72727272727273)(0,-4.54545454545455)(0,-4.36363636363636)(0,-4.18181818181818)(0,-4)(0,-3.81818181818182)(0,-3.63636363636364)(0,-3.45454545454545)(0,-3.27272727272727)(0,-3.09090909090909)(0,-2.90909090909091)(0,-2.72727272727273)(0,-2.54545454545455)(0,-2.36363636363636)(0,-2.18181818181818)(0,-2)(0,-1.81818181818182)(0,-1.63636363636364)(0,-1.45454545454545)(0,-1.27272727272727)(0,-1.09090909090909)(0,-0.909090909090909)(0,-0.727272727272727)(0,-0.545454545454545)(0,-0.363636363636364)(0,-0.181818181818182)(0,0)(14.9292929292929,0)(29.8585858585859,0)(44.7878787878788,0)(59.7171717171717,0)(74.6464646464647,0)(89.5757575757576,0)(104.505050505051,0)(119.434343434343,0)(134.363636363636,0)(149.292929292929,0)(164.222222222222,0)(179.151515151515,0)(194.080808080808,0)(209.010101010101,0)(223.939393939394,0)(238.868686868687,0)(253.79797979798,0)(268.727272727273,0)(283.656565656566,0)(298.585858585859,0)(313.515151515152,0)(328.444444444444,0)(343.373737373737,0)(358.30303030303,0)(373.232323232323,0)(388.161616161616,0)(403.090909090909,0)(418.020202020202,0)(432.949494949495,0)(447.878787878788,0)(462.808080808081,0)(477.737373737374,0)(492.666666666667,0)(507.59595959596,0)(522.525252525253,0)(537.454545454546,0)(552.383838383838,0)(567.313131313131,0)(582.242424242424,0)(597.171717171717,0)(612.10101010101,0)(627.030303030303,0)(641.959595959596,0)(656.888888888889,0)(671.818181818182,0)(686.747474747475,0)(701.676767676768,0)(716.606060606061,0)(731.535353535354,0)(746.464646464646,0)(761.393939393939,0)(776.323232323232,0)(791.252525252525,0)(806.181818181818,0)(821.111111111111,0)(836.040404040404,0)(850.969696969697,0)(865.89898989899,0)(880.828282828283,0)(895.757575757576,0)(910.686868686869,0)(925.616161616162,0)(940.545454545455,0)(955.474747474747,0)(970.40404040404,0)(985.333333333333,0)(1000.26262626263,0)(1015.19191919192,0)(1030.12121212121,0)(1045.05050505051,0)(1059.9797979798,0)(1074.90909090909,0)(1089.83838383838,0)(1104.76767676768,0)(1119.69696969697,0)(1134.62626262626,0)(1149.55555555556,0)(1164.48484848485,0)(1179.41414141414,0)(1194.34343434343,0)(1209.27272727273,0)(1224.20202020202,0)(1239.13131313131,0)(1254.06060606061,0)(1268.9898989899,0)(1283.91919191919,0)(1298.84848484848,0)(1313.77777777778,0)(1328.70707070707,0)(1343.63636363636,0)(1358.56565656566,0)(1373.49494949495,0)(1388.42424242424,0)(1403.35353535354,0)(1418.28282828283,0)(1433.21212121212,0)(1448.14141414141,0)(1463.07070707071,0)(1478,0)};

\addplot [fill=darkgray,draw=none,forget plot] coordinates{ (694.212121212121,-5.27272727272727)(701.676767676768,-5.3030303030303)(716.606060606061,-5.42424242424242)(724.070707070707,-5.45454545454545)(731.535353535354,-5.48484848484848)(746.464646464646,-5.60606060606061)(753.929292929293,-5.63636363636364)(761.393939393939,-5.66666666666667)(776.323232323232,-5.78787878787879)(783.787878787879,-5.81818181818182)(791.252525252525,-5.84848484848485)(806.181818181818,-5.90909090909091)(821.111111111111,-5.96969696969697)(828.575757575758,-6)(836.040404040404,-6.03030303030303)(850.969696969697,-6.09090909090909)(865.89898989899,-6.09090909090909)(880.828282828283,-6.09090909090909)(895.757575757576,-6.09090909090909)(910.686868686869,-6.09090909090909)(925.616161616162,-6.15151515151515)(933.080808080808,-6.18181818181818)(940.545454545455,-6.21212121212121)(955.474747474747,-6.27272727272727)(970.40404040404,-6.27272727272727)(985.333333333333,-6.27272727272727)(1000.26262626263,-6.33333333333333)(1007.72727272727,-6.36363636363636)(1015.19191919192,-6.39393939393939)(1030.12121212121,-6.45454545454546)(1045.05050505051,-6.51515151515152)(1052.51515151515,-6.54545454545455)(1059.9797979798,-6.57575757575758)(1074.90909090909,-6.6969696969697)(1082.37373737374,-6.72727272727273)(1089.83838383838,-6.75757575757576)(1104.76767676768,-6.87878787878788)(1112.23232323232,-6.90909090909091)(1119.69696969697,-6.93939393939394)(1134.62626262626,-7.06060606060606)(1142.09090909091,-7.09090909090909)(1149.55555555556,-7.12121212121212)(1164.48484848485,-7.18181818181818)(1179.41414141414,-7.18181818181818)(1194.34343434343,-7.24242424242424)(1201.80808080808,-7.27272727272727)(1209.27272727273,-7.3030303030303)(1224.20202020202,-7.36363636363636)(1239.13131313131,-7.3030303030303)(1246.59595959596,-7.27272727272727)(1254.06060606061,-7.24242424242424)(1268.9898989899,-7.18181818181818)(1283.91919191919,-7.18181818181818)(1298.84848484848,-7.12121212121212)(1302.58080808081,-7.09090909090909)(1313.77777777778,-7)(1324.97474747475,-6.90909090909091)(1328.70707070707,-6.87878787878788)(1343.63636363636,-6.75757575757576)(1347.36868686869,-6.72727272727273)(1358.56565656566,-6.63636363636364)(1369.76262626263,-6.54545454545455)(1373.49494949495,-6.51515151515152)(1388.42424242424,-6.39393939393939)(1395.88888888889,-6.36363636363636)(1403.35353535354,-6.33333333333333)(1418.28282828283,-6.27272727272727)(1433.21212121212,-6.21212121212121)(1440.67676767677,-6.18181818181818)(1448.14141414141,-6.15151515151515)(1463.07070707071,-6.09090909090909)(1478,-6.09090909090909)(1478.00018660683,-6.18181818181818)(1478.0005598065,-6.36363636363636)(1478.0005598065,-6.54545454545455)(1478.0005598065,-6.72727272727273)(1478.0005598065,-6.90909090909091)(1478.0005598065,-7.09090909090909)(1478.00031100879,-7.27272727272727)(1478.00031100879,-7.45454545454545)(1478.00031100879,-7.63636363636364)(1478.0005598065,-7.81818181818182)(1478.0005598065,-8)(1478.0005598065,-8.18181818181818)(1478.0005598065,-8.36363636363636)(1478.0005598065,-8.54545454545455)(1478.00031100879,-8.72727272727273)(1478.00031100879,-8.90909090909091)(1478.00031100879,-9.09090909090909)(1478.0005598065,-9.27272727272727)(1478.0005598065,-9.45454545454546)(1478.0005598065,-9.63636363636364)(1478.00031100879,-9.81818181818182)(1478,-9.96969696969697)(1470.53535353535,-10)(1463.07070707071,-10.030303030303)(1448.14141414141,-10.1515151515152)(1444.40909090909,-10.1818181818182)(1433.21212121212,-10.2727272727273)(1422.01515151515,-10.3636363636364)(1418.28282828283,-10.3939393939394)(1403.35353535354,-10.5151515151515)(1395.88888888889,-10.5454545454545)(1388.42424242424,-10.5757575757576)(1373.49494949495,-10.6969696969697)(1369.76262626263,-10.7272727272727)(1358.56565656566,-10.8181818181818)(1351.10101010101,-10.9090909090909)(1343.63636363636,-11)(1336.17171717172,-11.0909090909091)(1328.70707070707,-11.1818181818182)(1317.5101010101,-11.2727272727273)(1313.77777777778,-11.3030303030303)(1298.84848484848,-11.4242424242424)(1295.11616161616,-11.4545454545455)(1283.91919191919,-11.5454545454545)(1276.45454545455,-11.6363636363636)(1268.9898989899,-11.7272727272727)(1257.79292929293,-11.8181818181818)(1254.06060606061,-11.8484848484849)(1239.13131313131,-11.969696969697)(1231.66666666667,-12)(1224.20202020202,-12.030303030303)(1209.27272727273,-12.0909090909091)(1194.34343434343,-12.0909090909091)(1179.41414141414,-12.0909090909091)(1164.48484848485,-12.0909090909091)(1149.55555555556,-12.0909090909091)(1134.62626262626,-12.0909090909091)(1119.69696969697,-12.0909090909091)(1104.76767676768,-12.0909090909091)(1089.83838383838,-12.0909090909091)(1074.90909090909,-12.030303030303)(1067.44444444444,-12)(1059.9797979798,-11.969696969697)(1045.05050505051,-11.9090909090909)(1030.12121212121,-11.9090909090909)(1015.19191919192,-11.8484848484849)(1007.72727272727,-11.8181818181818)(1000.26262626263,-11.7878787878788)(985.333333333333,-11.7272727272727)(970.40404040404,-11.7272727272727)(955.474747474747,-11.6666666666667)(948.010101010101,-11.6363636363636)(940.545454545455,-11.6060606060606)(925.616161616162,-11.5454545454545)(910.686868686869,-11.5454545454545)(895.757575757576,-11.5454545454545)(880.828282828283,-11.5454545454545)(865.89898989899,-11.4848484848485)(858.434343434343,-11.4545454545455)(850.969696969697,-11.4242424242424)(836.040404040404,-11.3636363636364)(821.111111111111,-11.3636363636364)(806.181818181818,-11.3636363636364)(791.252525252525,-11.3030303030303)(783.787878787879,-11.2727272727273)(776.323232323232,-11.2424242424242)(761.393939393939,-11.1818181818182)(746.464646464646,-11.1212121212121)(739,-11.0909090909091)(731.535353535354,-11.0606060606061)(716.606060606061,-10.9393939393939)(709.141414141414,-10.9090909090909)(701.676767676768,-10.8787878787879)(686.747474747475,-10.7575757575758)(679.282828282828,-10.7272727272727)(671.818181818182,-10.6969696969697)(656.888888888889,-10.6363636363636)(641.959595959596,-10.5757575757576)(634.494949494949,-10.5454545454545)(627.030303030303,-10.5151515151515)(612.10101010101,-10.4545454545455)(597.171717171717,-10.3939393939394)(589.707070707071,-10.3636363636364)(582.242424242424,-10.3333333333333)(567.313131313131,-10.2727272727273)(552.383838383838,-10.2727272727273)(537.454545454546,-10.2727272727273)(522.525252525253,-10.2727272727273)(507.59595959596,-10.2121212121212)(500.131313131313,-10.1818181818182)(492.666666666667,-10.1515151515152)(477.737373737374,-10.0909090909091)(462.808080808081,-10.0909090909091)(447.878787878788,-10.0909090909091)(432.949494949495,-10.0909090909091)(418.020202020202,-10.0909090909091)(403.090909090909,-10.0909090909091)(388.161616161616,-10.0909090909091)(373.232323232323,-10.0909090909091)(358.30303030303,-10.030303030303)(354.570707070707,-10)(343.373737373737,-9.90909090909091)(335.909090909091,-9.81818181818182)(328.444444444444,-9.68181818181818)(325.956228956229,-9.63636363636364)(313.515151515152,-9.48484848484848)(311.026936026936,-9.45454545454546)(298.585858585859,-9.3030303030303)(291.121212121212,-9.27272727272727)(283.656565656566,-9.24242424242424)(268.727272727273,-9.18181818181818)(253.79797979798,-9.18181818181818)(238.868686868687,-9.18181818181818)(227.671717171717,-9.27272727272727)(223.939393939394,-9.36363636363636)(221.451178451178,-9.45454545454546)(223.939393939394,-9.5)(231.40404040404,-9.63636363636364)(238.868686868687,-9.72727272727273)(250.065656565657,-9.81818181818182)(253.79797979798,-9.84848484848485)(268.727272727273,-9.96969696969697)(272.459595959596,-10)(283.656565656566,-10.1363636363636)(286.144781144781,-10.1818181818182)(286.144781144781,-10.3636363636364)(283.656565656566,-10.4090909090909)(272.459595959596,-10.5454545454545)(268.727272727273,-10.5757575757576)(253.79797979798,-10.6969696969697)(246.333333333333,-10.7272727272727)(238.868686868687,-10.7575757575758)(223.939393939394,-10.8181818181818)(209.010101010101,-10.8181818181818)(194.080808080808,-10.8181818181818)(179.151515151515,-10.8181818181818)(164.222222222222,-10.8181818181818)(149.292929292929,-10.8181818181818)(134.363636363636,-10.8181818181818)(119.434343434343,-10.8181818181818)(104.505050505051,-10.7575757575758)(97.040404040404,-10.7272727272727)(89.5757575757576,-10.6969696969697)(74.6464646464647,-10.6363636363636)(59.7171717171717,-10.6363636363636)(44.7878787878788,-10.6363636363636)(29.8585858585859,-10.6363636363636)(14.9292929292929,-10.6363636363636)(0,-10.6363636363636)(-0.000186606831274284,-10.5454545454545)(-0.000559806499359878,-10.3636363636364)(-0.000559806499359878,-10.1818181818182)(-0.000559806499359878,-10)(-0.000559806499359878,-9.81818181818182)(-0.000559806499359878,-9.63636363636364)(-0.000559806499359878,-9.45454545454546)(-0.000186606831274284,-9.27272727272727)(0,-9.18181818181818)(14.9292929292929,-9.18181818181818)(29.8585858585859,-9.18181818181818)(44.7878787878788,-9.12121212121212)(52.2525252525253,-9.09090909090909)(59.7171717171717,-9.06060606060606)(72.1582491582492,-8.90909090909091)(74.6464646464647,-8.81818181818182)(77.1346801346801,-8.72727272727273)(74.6464646464647,-8.63636363636364)(72.1582491582492,-8.54545454545455)(62.2053872053872,-8.36363636363636)(59.7171717171717,-8.31818181818182)(52.2525252525253,-8.18181818181818)(44.7878787878788,-8.09090909090909)(37.3232323232323,-8)(29.8585858585859,-7.90909090909091)(22.3939393939394,-7.81818181818182)(14.9292929292929,-7.68181818181818)(12.4410774410774,-7.63636363636364)(12.4410774410774,-7.45454545454545)(14.9292929292929,-7.40909090909091)(26.1262626262626,-7.27272727272727)(29.8585858585859,-7.24242424242424)(44.7878787878788,-7.12121212121212)(52.2525252525253,-7.09090909090909)(59.7171717171717,-7.06060606060606)(74.6464646464647,-6.93939393939394)(82.1111111111111,-6.90909090909091)(89.5757575757576,-6.87878787878788)(104.505050505051,-6.81818181818182)(119.434343434343,-6.75757575757576)(126.89898989899,-6.72727272727273)(134.363636363636,-6.6969696969697)(149.292929292929,-6.63636363636364)(164.222222222222,-6.63636363636364)(179.151515151515,-6.63636363636364)(194.080808080808,-6.63636363636364)(209.010101010101,-6.63636363636364)(223.939393939394,-6.63636363636364)(238.868686868687,-6.63636363636364)(253.79797979798,-6.63636363636364)(268.727272727273,-6.6969696969697)(276.191919191919,-6.72727272727273)(283.656565656566,-6.75757575757576)(298.585858585859,-6.81818181818182)(313.515151515152,-6.81818181818182)(328.444444444444,-6.81818181818182)(343.373737373737,-6.81818181818182)(358.30303030303,-6.81818181818182)(373.232323232323,-6.81818181818182)(388.161616161616,-6.81818181818182)(403.090909090909,-6.75757575757576)(406.823232323232,-6.72727272727273)(418.020202020202,-6.63636363636364)(425.484848484849,-6.54545454545455)(432.949494949495,-6.45454545454546)(440.414141414141,-6.36363636363636)(447.878787878788,-6.27272727272727)(455.343434343434,-6.18181818181818)(462.808080808081,-6.04545454545455)(465.296296296296,-6)(477.737373737374,-5.84848484848485)(480.225589225589,-5.81818181818182)(492.666666666667,-5.66666666666667)(500.131313131313,-5.63636363636364)(507.59595959596,-5.60606060606061)(522.525252525253,-5.48484848484848)(529.989898989899,-5.45454545454545)(537.454545454546,-5.42424242424242)(552.383838383838,-5.36363636363636)(567.313131313131,-5.3030303030303)(574.777777777778,-5.27272727272727)(582.242424242424,-5.24242424242424)(597.171717171717,-5.18181818181818)(612.10101010101,-5.18181818181818)(627.030303030303,-5.18181818181818)(641.959595959596,-5.18181818181818)(656.888888888889,-5.18181818181818)(671.818181818182,-5.18181818181818)(686.747474747475,-5.24242424242424)(694.212121212121,-5.27272727272727)};

\addplot [fill=locol,draw=none,forget plot] coordinates{ (757.661616161616,-6.90909090909091)(746.464646464646,-6.81818181818182)(731.535353535354,-6.81818181818182)(716.606060606061,-6.87878787878788)(712.873737373737,-6.90909090909091)(704.164983164983,-7.09090909090909)(712.873737373737,-7.27272727272727)(716.606060606061,-7.3030303030303)(731.535353535354,-7.36363636363636)(746.464646464646,-7.3030303030303)(750.19696969697,-7.27272727272727)(761.393939393939,-7.13636363636364)(763.882154882155,-7.09090909090909)(761.393939393939,-7)(757.661616161616,-6.90909090909091)};

\addplot [
color=white,
draw=white,
only marks,
mark=x,
mark options={solid},
mark size=2.0pt,
line width=0.3pt,
forget plot
]
coordinates{
 (731.535353535354,-1.63636363636364)(1478,-18)(0,-12.3636363636364)(1478,-8.90909090909091)(432.949494949495,-18)(0,-5.63636363636364)(1478,0)(0,0)(880.828282828283,-12.7272727272727)(731.535353535354,-7.27272727272727)(1478,-4.36363636363636)(1478,-13.6363636363636)(0,-16.5454545454545)(955.474747474747,-18)(268.727272727273,-9.27272727272727)(447.878787878788,-14.3636363636364)(388.161616161616,-3.81818181818182)(1000.26262626263,-4.36363636363636)(1030.12121212121,-9.81818181818182)(1045.05050505051,0)(432.949494949495,0)(1104.76767676768,-15.4545454545455)(0,-2.72727272727273)(552.383838383838,-11.0909090909091)(1209.27272727273,-6.90909090909091)(0,-8.54545454545455)(1478,-11.2727272727273)(1268.9898989899,-2.18181818181818)(358.30303030303,-6.54545454545455)
};

\addplot [
color=white,
draw=black,
only marks,
mark=*,
mark options={solid},
mark size=2pt,
line width=0.2pt,
forget plot
]
coordinates{
 (716.606060606061,-16)
};

\node at (axis cs:980, -17) [shape=circle,fill=red!40!yellow,draw=black,inner sep=0.2pt,anchor=south west,minimum size=16pt]
  {\scriptsize\color{white}$\*H_{\*t}$};
\node at (axis cs:1155, -17) [shape=circle,fill=locol,draw=black,inner sep=0.2pt,anchor=south west,minimum size=16pt]
  {\scriptsize\color{white}$\*L_{\*t}$};
\node at (axis cs:1330, -17) [shape=circle,fill=darkgray,draw=black,inner sep=0.2pt,anchor=south west,minimum size=16pt]
  {\scriptsize\color{white}$\*U_{\*t}$};

\end{axis}
\end{tikzpicture}%

%% This file was created by matlab2tikz v0.2.3.
% Copyright (c) 2008--2012, Nico Schlömer <nico.schloemer@gmail.com>
% All rights reserved.
% 
% 
%

\definecolor{locol}{rgb}{0.26, 0.45, 0.65}

\begin{tikzpicture}

\begin{axis}[%
tick label style={font=\tiny},
label style={font=\tiny},
xlabel shift={-10pt},
ylabel shift={-17pt},
legend style={font=\tiny},
view={0}{90},
width=\figurewidth,
height=\figureheight,
scale only axis,
xmin=0, xmax=1478,
xtick={0, 400, 1000, 1400},
xlabel={Length (m)},
ymin=-18, ymax=0,
ytick={0, -4, -14, -18},
ylabel={Depth (m)},
name=plot1,
axis lines*=box,
tickwidth=0.1cm,
clip=false
]

\addplot [fill=locol,draw=none,forget plot] coordinates{ (1478,0)(1478,-0.181818181818182)(1478,-0.363636363636364)(1478,-0.545454545454545)(1478,-0.727272727272727)(1478,-0.909090909090909)(1478,-1.09090909090909)(1478,-1.27272727272727)(1478,-1.45454545454545)(1478,-1.63636363636364)(1478,-1.81818181818182)(1478,-2)(1478,-2.18181818181818)(1478,-2.36363636363636)(1478,-2.54545454545455)(1478,-2.72727272727273)(1478,-2.90909090909091)(1478,-3.09090909090909)(1478,-3.27272727272727)(1478,-3.45454545454545)(1478,-3.63636363636364)(1478,-3.81818181818182)(1478,-4)(1478,-4.18181818181818)(1478,-4.36363636363636)(1478,-4.54545454545455)(1478,-4.72727272727273)(1478,-4.90909090909091)(1478,-5.09090909090909)(1478,-5.27272727272727)(1478,-5.45454545454545)(1478,-5.63636363636364)(1478,-5.81818181818182)(1478.00037322299,-6)(1478.00074642733,-6.18181818181818)(1478.001119613,-6.36363636363636)(1478.001119613,-6.54545454545455)(1478.001119613,-6.72727272727273)(1478.001119613,-6.90909090909091)(1478.001119613,-7.09090909090909)(1478.00087082462,-7.27272727272727)(1478.00087082462,-7.45454545454545)(1478.00087082462,-7.63636363636364)(1478.001119613,-7.81818181818182)(1478.001119613,-8)(1478.001119613,-8.18181818181818)(1478.001119613,-8.36363636363636)(1478.001119613,-8.54545454545455)(1478.00087082462,-8.72727272727273)(1478.00087082462,-8.90909090909091)(1478.00087082462,-9.09090909090909)(1478.001119613,-9.27272727272727)(1478.001119613,-9.45454545454546)(1478.001119613,-9.63636363636364)(1478.00087082462,-9.81818181818182)(1478.00049762651,-10)(1478.00012440974,-10.1818181818182)(1478,-10.3636363636364)(1478,-10.5454545454545)(1478,-10.7272727272727)(1478,-10.9090909090909)(1478,-11.0909090909091)(1478,-11.2727272727273)(1478,-11.4545454545455)(1478,-11.6363636363636)(1478,-11.8181818181818)(1478,-12)(1478,-12.1818181818182)(1478,-12.3636363636364)(1478,-12.5454545454545)(1478,-12.7272727272727)(1478,-12.9090909090909)(1478,-13.0909090909091)(1478,-13.2727272727273)(1478,-13.4545454545455)(1478,-13.6363636363636)(1478,-13.8181818181818)(1478,-14)(1478,-14.1818181818182)(1478,-14.3636363636364)(1478,-14.5454545454545)(1478,-14.7272727272727)(1478,-14.9090909090909)(1478,-15.0909090909091)(1478,-15.2727272727273)(1478,-15.4545454545455)(1478,-15.6363636363636)(1478,-15.8181818181818)(1478,-16)(1478,-16.1818181818182)(1478,-16.3636363636364)(1478,-16.5454545454545)(1478,-16.7272727272727)(1478,-16.9090909090909)(1478,-17.0909090909091)(1478,-17.2727272727273)(1478,-17.4545454545455)(1478,-17.6363636363636)(1478,-17.8181818181818)(1478,-18)(1463.07070707071,-18)(1448.14141414141,-18)(1433.21212121212,-18)(1418.28282828283,-18)(1403.35353535354,-18)(1388.42424242424,-18)(1373.49494949495,-18)(1358.56565656566,-18)(1343.63636363636,-18)(1328.70707070707,-18)(1313.77777777778,-18)(1298.84848484848,-18)(1283.91919191919,-18)(1268.9898989899,-18)(1254.06060606061,-18)(1239.13131313131,-18)(1224.20202020202,-18)(1209.27272727273,-18)(1194.34343434343,-18)(1179.41414141414,-18)(1164.48484848485,-18)(1149.55555555556,-18)(1134.62626262626,-18)(1119.69696969697,-18)(1104.76767676768,-18)(1089.83838383838,-18)(1074.90909090909,-18)(1059.9797979798,-18)(1045.05050505051,-18)(1030.12121212121,-18)(1015.19191919192,-18)(1000.26262626263,-18)(985.333333333333,-18)(970.40404040404,-18)(955.474747474747,-18)(940.545454545455,-18)(925.616161616162,-18)(910.686868686869,-18)(895.757575757576,-18)(880.828282828283,-18)(865.89898989899,-18)(850.969696969697,-18)(836.040404040404,-18)(821.111111111111,-18)(806.181818181818,-18)(791.252525252525,-18)(776.323232323232,-18)(761.393939393939,-18)(746.464646464646,-18)(731.535353535354,-18)(716.606060606061,-18)(701.676767676768,-18)(686.747474747475,-18)(671.818181818182,-18)(656.888888888889,-18)(641.959595959596,-18)(627.030303030303,-18)(612.10101010101,-18)(597.171717171717,-18)(582.242424242424,-18)(567.313131313131,-18)(552.383838383838,-18)(537.454545454546,-18)(522.525252525253,-18)(507.59595959596,-18)(492.666666666667,-18)(477.737373737374,-18)(462.808080808081,-18)(447.878787878788,-18)(432.949494949495,-18)(418.020202020202,-18)(403.090909090909,-18)(388.161616161616,-18)(373.232323232323,-18)(358.30303030303,-18)(343.373737373737,-18)(328.444444444444,-18)(313.515151515152,-18)(298.585858585859,-18)(283.656565656566,-18)(268.727272727273,-18)(253.79797979798,-18)(238.868686868687,-18)(223.939393939394,-18)(209.010101010101,-18)(194.080808080808,-18)(179.151515151515,-18)(164.222222222222,-18)(149.292929292929,-18)(134.363636363636,-18)(119.434343434343,-18)(104.505050505051,-18)(89.5757575757576,-18)(74.6464646464647,-18)(59.7171717171717,-18)(44.7878787878788,-18)(29.8585858585859,-18)(14.9292929292929,-18)(0,-18)(0,-17.8181818181818)(0,-17.6363636363636)(0,-17.4545454545455)(0,-17.2727272727273)(0,-17.0909090909091)(0,-16.9090909090909)(0,-16.7272727272727)(0,-16.5454545454545)(0,-16.3636363636364)(0,-16.1818181818182)(0,-16)(0,-15.8181818181818)(0,-15.6363636363636)(0,-15.4545454545455)(0,-15.2727272727273)(0,-15.0909090909091)(0,-14.9090909090909)(0,-14.7272727272727)(0,-14.5454545454545)(0,-14.3636363636364)(0,-14.1818181818182)(0,-14)(0,-13.8181818181818)(0,-13.6363636363636)(0,-13.4545454545455)(0,-13.2727272727273)(0,-13.0909090909091)(0,-12.9090909090909)(0,-12.7272727272727)(0,-12.5454545454545)(0,-12.3636363636364)(0,-12.1818181818182)(0,-12)(0,-11.8181818181818)(0,-11.6363636363636)(0,-11.4545454545455)(0,-11.2727272727273)(0,-11.0909090909091)(0,-10.9090909090909)(-0.000373222992657963,-10.7272727272727)(-0.000746427325097138,-10.5454545454545)(-0.00111961299872307,-10.3636363636364)(-0.00111961299872307,-10.1818181818182)(-0.00111961299872307,-10)(-0.00111961299872307,-9.81818181818182)(-0.00111961299872307,-9.63636363636364)(-0.00111961299872307,-9.45454545454546)(-0.000746427325097138,-9.27272727272727)(-0.000373222992657963,-9.09090909090909)(0,-8.90909090909091)(0,-8.72727272727273)(0,-8.54545454545455)(0,-8.36363636363636)(0,-8.18181818181818)(0,-8)(-0.00012440973766168,-7.81818181818182)(-0.000248817401863321,-7.63636363636364)(-0.000248817401863321,-7.45454545454545)(-0.00012440973766168,-7.27272727272727)(0,-7.09090909090909)(0,-6.90909090909091)(0,-6.72727272727273)(0,-6.54545454545455)(0,-6.36363636363636)(0,-6.18181818181818)(0,-6)(0,-5.81818181818182)(0,-5.63636363636364)(0,-5.45454545454545)(0,-5.27272727272727)(0,-5.09090909090909)(0,-4.90909090909091)(0,-4.72727272727273)(0,-4.54545454545455)(0,-4.36363636363636)(0,-4.18181818181818)(0,-4)(0,-3.81818181818182)(0,-3.63636363636364)(0,-3.45454545454545)(0,-3.27272727272727)(0,-3.09090909090909)(0,-2.90909090909091)(0,-2.72727272727273)(0,-2.54545454545455)(0,-2.36363636363636)(0,-2.18181818181818)(0,-2)(0,-1.81818181818182)(0,-1.63636363636364)(0,-1.45454545454545)(0,-1.27272727272727)(0,-1.09090909090909)(0,-0.909090909090909)(0,-0.727272727272727)(0,-0.545454545454545)(0,-0.363636363636364)(0,-0.181818181818182)(0,0)(14.9292929292929,0)(29.8585858585859,0)(44.7878787878788,0)(59.7171717171717,0)(74.6464646464647,0)(89.5757575757576,0)(104.505050505051,0)(119.434343434343,0)(134.363636363636,0)(149.292929292929,0)(164.222222222222,0)(179.151515151515,0)(194.080808080808,0)(209.010101010101,0)(223.939393939394,0)(238.868686868687,0)(253.79797979798,0)(268.727272727273,0)(283.656565656566,0)(298.585858585859,0)(313.515151515152,0)(328.444444444444,0)(343.373737373737,0)(358.30303030303,0)(373.232323232323,0)(388.161616161616,0)(403.090909090909,0)(418.020202020202,0)(432.949494949495,0)(447.878787878788,0)(462.808080808081,0)(477.737373737374,0)(492.666666666667,0)(507.59595959596,0)(522.525252525253,0)(537.454545454546,0)(552.383838383838,0)(567.313131313131,0)(582.242424242424,0)(597.171717171717,0)(612.10101010101,0)(627.030303030303,0)(641.959595959596,0)(656.888888888889,0)(671.818181818182,0)(686.747474747475,0)(701.676767676768,0)(716.606060606061,0)(731.535353535354,0)(746.464646464646,0)(761.393939393939,0)(776.323232323232,0)(791.252525252525,0)(806.181818181818,0)(821.111111111111,0)(836.040404040404,0)(850.969696969697,0)(865.89898989899,0)(880.828282828283,0)(895.757575757576,0)(910.686868686869,0)(925.616161616162,0)(940.545454545455,0)(955.474747474747,0)(970.40404040404,0)(985.333333333333,0)(1000.26262626263,0)(1015.19191919192,0)(1030.12121212121,0)(1045.05050505051,0)(1059.9797979798,0)(1074.90909090909,0)(1089.83838383838,0)(1104.76767676768,0)(1119.69696969697,0)(1134.62626262626,0)(1149.55555555556,0)(1164.48484848485,0)(1179.41414141414,0)(1194.34343434343,0)(1209.27272727273,0)(1224.20202020202,0)(1239.13131313131,0)(1254.06060606061,0)(1268.9898989899,0)(1283.91919191919,0)(1298.84848484848,0)(1313.77777777778,0)(1328.70707070707,0)(1343.63636363636,0)(1358.56565656566,0)(1373.49494949495,0)(1388.42424242424,0)(1403.35353535354,0)(1418.28282828283,0)(1433.21212121212,0)(1448.14141414141,0)(1463.07070707071,0)(1478,0)};

\addplot [fill=darkgray,draw=none,forget plot] coordinates{ (694.212121212121,-5.27272727272727)(701.676767676768,-5.3030303030303)(716.606060606061,-5.42424242424242)(724.070707070707,-5.45454545454545)(731.535353535354,-5.48484848484848)(746.464646464646,-5.60606060606061)(753.929292929293,-5.63636363636364)(761.393939393939,-5.66666666666667)(776.323232323232,-5.78787878787879)(783.787878787879,-5.81818181818182)(791.252525252525,-5.84848484848485)(806.181818181818,-5.90909090909091)(821.111111111111,-5.96969696969697)(828.575757575758,-6)(836.040404040404,-6.03030303030303)(850.969696969697,-6.09090909090909)(865.89898989899,-6.09090909090909)(880.828282828283,-6.09090909090909)(895.757575757576,-6.09090909090909)(910.686868686869,-6.09090909090909)(925.616161616162,-6.15151515151515)(933.080808080808,-6.18181818181818)(940.545454545455,-6.21212121212121)(955.474747474747,-6.27272727272727)(970.40404040404,-6.27272727272727)(985.333333333333,-6.27272727272727)(1000.26262626263,-6.33333333333333)(1007.72727272727,-6.36363636363636)(1015.19191919192,-6.39393939393939)(1030.12121212121,-6.45454545454546)(1045.05050505051,-6.51515151515152)(1052.51515151515,-6.54545454545455)(1059.9797979798,-6.57575757575758)(1074.90909090909,-6.6969696969697)(1082.37373737374,-6.72727272727273)(1089.83838383838,-6.75757575757576)(1104.76767676768,-6.87878787878788)(1112.23232323232,-6.90909090909091)(1119.69696969697,-6.93939393939394)(1134.62626262626,-7.06060606060606)(1142.09090909091,-7.09090909090909)(1149.55555555556,-7.12121212121212)(1164.48484848485,-7.18181818181818)(1179.41414141414,-7.18181818181818)(1194.34343434343,-7.24242424242424)(1201.80808080808,-7.27272727272727)(1209.27272727273,-7.3030303030303)(1224.20202020202,-7.36363636363636)(1239.13131313131,-7.3030303030303)(1246.59595959596,-7.27272727272727)(1254.06060606061,-7.24242424242424)(1268.9898989899,-7.18181818181818)(1283.91919191919,-7.18181818181818)(1298.84848484848,-7.12121212121212)(1302.58080808081,-7.09090909090909)(1313.77777777778,-7)(1324.97474747475,-6.90909090909091)(1328.70707070707,-6.87878787878788)(1343.63636363636,-6.75757575757576)(1347.36868686869,-6.72727272727273)(1358.56565656566,-6.63636363636364)(1369.76262626263,-6.54545454545455)(1373.49494949495,-6.51515151515152)(1388.42424242424,-6.39393939393939)(1395.88888888889,-6.36363636363636)(1403.35353535354,-6.33333333333333)(1418.28282828283,-6.27272727272727)(1433.21212121212,-6.21212121212121)(1440.67676767677,-6.18181818181818)(1448.14141414141,-6.15151515151515)(1463.07070707071,-6.09090909090909)(1478,-6.09090909090909)(1478.00018660683,-6.18181818181818)(1478.0005598065,-6.36363636363636)(1478.0005598065,-6.54545454545455)(1478.0005598065,-6.72727272727273)(1478.0005598065,-6.90909090909091)(1478.0005598065,-7.09090909090909)(1478.00031100879,-7.27272727272727)(1478.00031100879,-7.45454545454545)(1478.00031100879,-7.63636363636364)(1478.0005598065,-7.81818181818182)(1478.0005598065,-8)(1478.0005598065,-8.18181818181818)(1478.0005598065,-8.36363636363636)(1478.0005598065,-8.54545454545455)(1478.00031100879,-8.72727272727273)(1478.00031100879,-8.90909090909091)(1478.00031100879,-9.09090909090909)(1478.0005598065,-9.27272727272727)(1478.0005598065,-9.45454545454546)(1478.0005598065,-9.63636363636364)(1478.00031100879,-9.81818181818182)(1478,-9.96969696969697)(1470.53535353535,-10)(1463.07070707071,-10.030303030303)(1448.14141414141,-10.1515151515152)(1444.40909090909,-10.1818181818182)(1433.21212121212,-10.2727272727273)(1422.01515151515,-10.3636363636364)(1418.28282828283,-10.3939393939394)(1403.35353535354,-10.5151515151515)(1395.88888888889,-10.5454545454545)(1388.42424242424,-10.5757575757576)(1373.49494949495,-10.6969696969697)(1369.76262626263,-10.7272727272727)(1358.56565656566,-10.8181818181818)(1351.10101010101,-10.9090909090909)(1343.63636363636,-11)(1336.17171717172,-11.0909090909091)(1328.70707070707,-11.1818181818182)(1317.5101010101,-11.2727272727273)(1313.77777777778,-11.3030303030303)(1298.84848484848,-11.4242424242424)(1295.11616161616,-11.4545454545455)(1283.91919191919,-11.5454545454545)(1276.45454545455,-11.6363636363636)(1268.9898989899,-11.7272727272727)(1257.79292929293,-11.8181818181818)(1254.06060606061,-11.8484848484849)(1239.13131313131,-11.969696969697)(1231.66666666667,-12)(1224.20202020202,-12.030303030303)(1209.27272727273,-12.0909090909091)(1194.34343434343,-12.0909090909091)(1179.41414141414,-12.0909090909091)(1164.48484848485,-12.0909090909091)(1149.55555555556,-12.0909090909091)(1134.62626262626,-12.0909090909091)(1119.69696969697,-12.0909090909091)(1104.76767676768,-12.0909090909091)(1089.83838383838,-12.0909090909091)(1074.90909090909,-12.030303030303)(1067.44444444444,-12)(1059.9797979798,-11.969696969697)(1045.05050505051,-11.9090909090909)(1030.12121212121,-11.9090909090909)(1015.19191919192,-11.8484848484849)(1007.72727272727,-11.8181818181818)(1000.26262626263,-11.7878787878788)(985.333333333333,-11.7272727272727)(970.40404040404,-11.7272727272727)(955.474747474747,-11.6666666666667)(948.010101010101,-11.6363636363636)(940.545454545455,-11.6060606060606)(925.616161616162,-11.5454545454545)(910.686868686869,-11.5454545454545)(895.757575757576,-11.5454545454545)(880.828282828283,-11.5454545454545)(865.89898989899,-11.4848484848485)(858.434343434343,-11.4545454545455)(850.969696969697,-11.4242424242424)(836.040404040404,-11.3636363636364)(821.111111111111,-11.3636363636364)(806.181818181818,-11.3636363636364)(791.252525252525,-11.3030303030303)(783.787878787879,-11.2727272727273)(776.323232323232,-11.2424242424242)(761.393939393939,-11.1818181818182)(746.464646464646,-11.1212121212121)(739,-11.0909090909091)(731.535353535354,-11.0606060606061)(716.606060606061,-10.9393939393939)(709.141414141414,-10.9090909090909)(701.676767676768,-10.8787878787879)(686.747474747475,-10.7575757575758)(679.282828282828,-10.7272727272727)(671.818181818182,-10.6969696969697)(656.888888888889,-10.6363636363636)(641.959595959596,-10.5757575757576)(634.494949494949,-10.5454545454545)(627.030303030303,-10.5151515151515)(612.10101010101,-10.4545454545455)(597.171717171717,-10.3939393939394)(589.707070707071,-10.3636363636364)(582.242424242424,-10.3333333333333)(567.313131313131,-10.2727272727273)(552.383838383838,-10.2727272727273)(537.454545454546,-10.2727272727273)(522.525252525253,-10.2727272727273)(507.59595959596,-10.2121212121212)(500.131313131313,-10.1818181818182)(492.666666666667,-10.1515151515152)(477.737373737374,-10.0909090909091)(462.808080808081,-10.0909090909091)(447.878787878788,-10.0909090909091)(432.949494949495,-10.0909090909091)(418.020202020202,-10.0909090909091)(403.090909090909,-10.0909090909091)(388.161616161616,-10.0909090909091)(373.232323232323,-10.0909090909091)(358.30303030303,-10.030303030303)(354.570707070707,-10)(343.373737373737,-9.90909090909091)(335.909090909091,-9.81818181818182)(328.444444444444,-9.68181818181818)(325.956228956229,-9.63636363636364)(313.515151515152,-9.48484848484848)(311.026936026936,-9.45454545454546)(298.585858585859,-9.3030303030303)(291.121212121212,-9.27272727272727)(283.656565656566,-9.24242424242424)(268.727272727273,-9.18181818181818)(253.79797979798,-9.18181818181818)(238.868686868687,-9.18181818181818)(227.671717171717,-9.27272727272727)(223.939393939394,-9.36363636363636)(221.451178451178,-9.45454545454546)(223.939393939394,-9.5)(231.40404040404,-9.63636363636364)(238.868686868687,-9.72727272727273)(250.065656565657,-9.81818181818182)(253.79797979798,-9.84848484848485)(268.727272727273,-9.96969696969697)(272.459595959596,-10)(283.656565656566,-10.1363636363636)(286.144781144781,-10.1818181818182)(286.144781144781,-10.3636363636364)(283.656565656566,-10.4090909090909)(272.459595959596,-10.5454545454545)(268.727272727273,-10.5757575757576)(253.79797979798,-10.6969696969697)(246.333333333333,-10.7272727272727)(238.868686868687,-10.7575757575758)(223.939393939394,-10.8181818181818)(209.010101010101,-10.8181818181818)(194.080808080808,-10.8181818181818)(179.151515151515,-10.8181818181818)(164.222222222222,-10.8181818181818)(149.292929292929,-10.8181818181818)(134.363636363636,-10.8181818181818)(119.434343434343,-10.8181818181818)(104.505050505051,-10.7575757575758)(97.040404040404,-10.7272727272727)(89.5757575757576,-10.6969696969697)(74.6464646464647,-10.6363636363636)(59.7171717171717,-10.6363636363636)(44.7878787878788,-10.6363636363636)(29.8585858585859,-10.6363636363636)(14.9292929292929,-10.6363636363636)(0,-10.6363636363636)(-0.000186606831274284,-10.5454545454545)(-0.000559806499359878,-10.3636363636364)(-0.000559806499359878,-10.1818181818182)(-0.000559806499359878,-10)(-0.000559806499359878,-9.81818181818182)(-0.000559806499359878,-9.63636363636364)(-0.000559806499359878,-9.45454545454546)(-0.000186606831274284,-9.27272727272727)(0,-9.18181818181818)(14.9292929292929,-9.18181818181818)(29.8585858585859,-9.18181818181818)(44.7878787878788,-9.12121212121212)(52.2525252525253,-9.09090909090909)(59.7171717171717,-9.06060606060606)(72.1582491582492,-8.90909090909091)(74.6464646464647,-8.81818181818182)(77.1346801346801,-8.72727272727273)(74.6464646464647,-8.63636363636364)(72.1582491582492,-8.54545454545455)(62.2053872053872,-8.36363636363636)(59.7171717171717,-8.31818181818182)(52.2525252525253,-8.18181818181818)(44.7878787878788,-8.09090909090909)(37.3232323232323,-8)(29.8585858585859,-7.90909090909091)(22.3939393939394,-7.81818181818182)(14.9292929292929,-7.68181818181818)(12.4410774410774,-7.63636363636364)(12.4410774410774,-7.45454545454545)(14.9292929292929,-7.40909090909091)(26.1262626262626,-7.27272727272727)(29.8585858585859,-7.24242424242424)(44.7878787878788,-7.12121212121212)(52.2525252525253,-7.09090909090909)(59.7171717171717,-7.06060606060606)(74.6464646464647,-6.93939393939394)(82.1111111111111,-6.90909090909091)(89.5757575757576,-6.87878787878788)(104.505050505051,-6.81818181818182)(119.434343434343,-6.75757575757576)(126.89898989899,-6.72727272727273)(134.363636363636,-6.6969696969697)(149.292929292929,-6.63636363636364)(164.222222222222,-6.63636363636364)(179.151515151515,-6.63636363636364)(194.080808080808,-6.63636363636364)(209.010101010101,-6.63636363636364)(223.939393939394,-6.63636363636364)(238.868686868687,-6.63636363636364)(253.79797979798,-6.63636363636364)(268.727272727273,-6.6969696969697)(276.191919191919,-6.72727272727273)(283.656565656566,-6.75757575757576)(298.585858585859,-6.81818181818182)(313.515151515152,-6.81818181818182)(328.444444444444,-6.81818181818182)(343.373737373737,-6.81818181818182)(358.30303030303,-6.81818181818182)(373.232323232323,-6.81818181818182)(388.161616161616,-6.81818181818182)(403.090909090909,-6.75757575757576)(406.823232323232,-6.72727272727273)(418.020202020202,-6.63636363636364)(425.484848484849,-6.54545454545455)(432.949494949495,-6.45454545454546)(440.414141414141,-6.36363636363636)(447.878787878788,-6.27272727272727)(455.343434343434,-6.18181818181818)(462.808080808081,-6.04545454545455)(465.296296296296,-6)(477.737373737374,-5.84848484848485)(480.225589225589,-5.81818181818182)(492.666666666667,-5.66666666666667)(500.131313131313,-5.63636363636364)(507.59595959596,-5.60606060606061)(522.525252525253,-5.48484848484848)(529.989898989899,-5.45454545454545)(537.454545454546,-5.42424242424242)(552.383838383838,-5.36363636363636)(567.313131313131,-5.3030303030303)(574.777777777778,-5.27272727272727)(582.242424242424,-5.24242424242424)(597.171717171717,-5.18181818181818)(612.10101010101,-5.18181818181818)(627.030303030303,-5.18181818181818)(641.959595959596,-5.18181818181818)(656.888888888889,-5.18181818181818)(671.818181818182,-5.18181818181818)(686.747474747475,-5.24242424242424)(694.212121212121,-5.27272727272727)};

\addplot [fill=locol,draw=none,forget plot] coordinates{ (757.661616161616,-6.90909090909091)(746.464646464646,-6.81818181818182)(731.535353535354,-6.81818181818182)(716.606060606061,-6.87878787878788)(712.873737373737,-6.90909090909091)(704.164983164983,-7.09090909090909)(712.873737373737,-7.27272727272727)(716.606060606061,-7.3030303030303)(731.535353535354,-7.36363636363636)(746.464646464646,-7.3030303030303)(750.19696969697,-7.27272727272727)(761.393939393939,-7.13636363636364)(763.882154882155,-7.09090909090909)(761.393939393939,-7)(757.661616161616,-6.90909090909091)};

\addplot [
color=white,
draw=white,
only marks,
mark=x,
mark options={solid},
mark size=2.0pt,
line width=0.3pt,
forget plot
]
coordinates{
 (731.535353535354,-1.63636363636364)(1478,-18)(0,-12.3636363636364)(1478,-8.90909090909091)(432.949494949495,-18)(0,-5.63636363636364)(1478,0)(0,0)(880.828282828283,-12.7272727272727)(731.535353535354,-7.27272727272727)(1478,-4.36363636363636)(1478,-13.6363636363636)(0,-16.5454545454545)(955.474747474747,-18)(268.727272727273,-9.27272727272727)(447.878787878788,-14.3636363636364)(388.161616161616,-3.81818181818182)(1000.26262626263,-4.36363636363636)(1030.12121212121,-9.81818181818182)(1045.05050505051,0)(432.949494949495,0)(1104.76767676768,-15.4545454545455)(0,-2.72727272727273)(552.383838383838,-11.0909090909091)(1209.27272727273,-6.90909090909091)(0,-8.54545454545455)(1478,-11.2727272727273)(1268.9898989899,-2.18181818181818)(358.30303030303,-6.54545454545455)
};

\addplot [
color=white,
draw=black,
only marks,
mark=*,
mark options={solid},
mark size=2pt,
line width=0.2pt,
forget plot
]
coordinates{
 (716.606060606061,-16)
};

\addplot [
color=red,
draw=black,
only marks,
mark=*,
mark options={solid},
mark size=1.6pt,
line width=0.2pt,
forget plot
]
coordinates{
(507.59595959596,-8.36363636363636)
(238.868686868687,-7.81818181818182)
(507.59595959596,-7.45454545454545)
(686.747474747475,-8.90909090909091)
(821.111111111111,-9.09090909090909)
(836.040404040404,-8.90909090909091)
(850.969696969697,-9.45454545454546)
(865.89898989899,-8.54545454545455)
(865.89898989899,-9.63636363636364)
(880.828282828283,-9.81818181818182)
(895.757575757576,-8.36363636363636)
(910.686868686869,-8.36363636363636)
(940.545454545455,-10.1818181818182)
(955.474747474747,-10.1818181818182)
(1015.19191919192,-8.18181818181818)
(1030.12121212121,-10.3636363636364)
(1045.05050505051,-8.18181818181818)
(1089.83838383838,-8.18181818181818)
(1119.69696969697,-10.3636363636364)
(1134.62626262626,-10.1818181818182)
(1179.41414141414,-10)
(1179.41414141414,-8.36363636363636)
(1194.34343434343,-8.36363636363636)
(1224.20202020202,-9.81818181818182)
(1239.13131313131,-9.63636363636364)
(1239.13131313131,-8.54545454545455)
(1268.9898989899,-8.72727272727273)
(1268.9898989899,-9.45454545454546)
(1283.91919191919,-9.09090909090909)
(1478,-7.45454545454545) 
};

\node at (axis cs:980, -17) [shape=circle,fill=red!40!yellow,draw=black,inner sep=0.2pt,anchor=south west,minimum size=16pt]
  {\scriptsize\color{white}$\*H_{\*t}$};
\node at (axis cs:1155, -17) [shape=circle,fill=locol,draw=black,inner sep=0.2pt,anchor=south west,minimum size=16pt]
  {\scriptsize\color{white}$\*L_{\*t}$};
\node at (axis cs:1330, -17) [shape=circle,fill=darkgray,draw=black,inner sep=0.2pt,anchor=south west,minimum size=16pt]
  {\scriptsize\color{white}$\*U_{\*t}$};

\end{axis}
\end{tikzpicture}%
%% This file was created by matlab2tikz v0.2.3.
% Copyright (c) 2008--2012, Nico Schlömer <nico.schloemer@gmail.com>
% All rights reserved.
% 
% 
%

\definecolor{locol}{rgb}{0.26, 0.45, 0.65}

\begin{tikzpicture}

\begin{axis}[%
tick label style={font=\tiny},
label style={font=\tiny},
xlabel shift={-10pt},
ylabel shift={-17pt},
legend style={font=\tiny},
view={0}{90},
width=\figurewidth,
height=\figureheight,
scale only axis,
xmin=0, xmax=1478,
xtick={0, 400, 1000, 1400},
xlabel={Length (m)},
ymin=-18, ymax=0,
ytick={0, -4, -14, -18},
ylabel={Depth (m)},
name=plot1,
axis lines*=box,
tickwidth=0.1cm,
clip=false
]

\addplot [fill=locol,draw=none,forget plot] coordinates{ (1478,0)(1478,-0.181818181818182)(1478,-0.363636363636364)(1478,-0.545454545454545)(1478,-0.727272727272727)(1478,-0.909090909090909)(1478,-1.09090909090909)(1478,-1.27272727272727)(1478,-1.45454545454545)(1478,-1.63636363636364)(1478,-1.81818181818182)(1478,-2)(1478,-2.18181818181818)(1478,-2.36363636363636)(1478,-2.54545454545455)(1478,-2.72727272727273)(1478,-2.90909090909091)(1478,-3.09090909090909)(1478,-3.27272727272727)(1478,-3.45454545454545)(1478,-3.63636363636364)(1478,-3.81818181818182)(1478,-4)(1478,-4.18181818181818)(1478,-4.36363636363636)(1478,-4.54545454545455)(1478,-4.72727272727273)(1478,-4.90909090909091)(1478,-5.09090909090909)(1478,-5.27272727272727)(1478,-5.45454545454545)(1478,-5.63636363636364)(1478,-5.81818181818182)(1478.00037322299,-6)(1478.00074642733,-6.18181818181818)(1478.001119613,-6.36363636363636)(1478.001119613,-6.54545454545455)(1478.001119613,-6.72727272727273)(1478.001119613,-6.90909090909091)(1478.001119613,-7.09090909090909)(1478.00087082462,-7.27272727272727)(1478.00087082462,-7.45454545454545)(1478.00087082462,-7.63636363636364)(1478.001119613,-7.81818181818182)(1478.001119613,-8)(1478.001119613,-8.18181818181818)(1478.001119613,-8.36363636363636)(1478.001119613,-8.54545454545455)(1478.00087082462,-8.72727272727273)(1478.00087082462,-8.90909090909091)(1478.00087082462,-9.09090909090909)(1478.001119613,-9.27272727272727)(1478.001119613,-9.45454545454546)(1478.001119613,-9.63636363636364)(1478.00087082462,-9.81818181818182)(1478.00049762651,-10)(1478.00012440974,-10.1818181818182)(1478,-10.3636363636364)(1478,-10.5454545454545)(1478,-10.7272727272727)(1478,-10.9090909090909)(1478,-11.0909090909091)(1478,-11.2727272727273)(1478,-11.4545454545455)(1478,-11.6363636363636)(1478,-11.8181818181818)(1478,-12)(1478,-12.1818181818182)(1478,-12.3636363636364)(1478,-12.5454545454545)(1478,-12.7272727272727)(1478,-12.9090909090909)(1478,-13.0909090909091)(1478,-13.2727272727273)(1478,-13.4545454545455)(1478,-13.6363636363636)(1478,-13.8181818181818)(1478,-14)(1478,-14.1818181818182)(1478,-14.3636363636364)(1478,-14.5454545454545)(1478,-14.7272727272727)(1478,-14.9090909090909)(1478,-15.0909090909091)(1478,-15.2727272727273)(1478,-15.4545454545455)(1478,-15.6363636363636)(1478,-15.8181818181818)(1478,-16)(1478,-16.1818181818182)(1478,-16.3636363636364)(1478,-16.5454545454545)(1478,-16.7272727272727)(1478,-16.9090909090909)(1478,-17.0909090909091)(1478,-17.2727272727273)(1478,-17.4545454545455)(1478,-17.6363636363636)(1478,-17.8181818181818)(1478,-18)(1463.07070707071,-18)(1448.14141414141,-18)(1433.21212121212,-18)(1418.28282828283,-18)(1403.35353535354,-18)(1388.42424242424,-18)(1373.49494949495,-18)(1358.56565656566,-18)(1343.63636363636,-18)(1328.70707070707,-18)(1313.77777777778,-18)(1298.84848484848,-18)(1283.91919191919,-18)(1268.9898989899,-18)(1254.06060606061,-18)(1239.13131313131,-18)(1224.20202020202,-18)(1209.27272727273,-18)(1194.34343434343,-18)(1179.41414141414,-18)(1164.48484848485,-18)(1149.55555555556,-18)(1134.62626262626,-18)(1119.69696969697,-18)(1104.76767676768,-18)(1089.83838383838,-18)(1074.90909090909,-18)(1059.9797979798,-18)(1045.05050505051,-18)(1030.12121212121,-18)(1015.19191919192,-18)(1000.26262626263,-18)(985.333333333333,-18)(970.40404040404,-18)(955.474747474747,-18)(940.545454545455,-18)(925.616161616162,-18)(910.686868686869,-18)(895.757575757576,-18)(880.828282828283,-18)(865.89898989899,-18)(850.969696969697,-18)(836.040404040404,-18)(821.111111111111,-18)(806.181818181818,-18)(791.252525252525,-18)(776.323232323232,-18)(761.393939393939,-18)(746.464646464646,-18)(731.535353535354,-18)(716.606060606061,-18)(701.676767676768,-18)(686.747474747475,-18)(671.818181818182,-18)(656.888888888889,-18)(641.959595959596,-18)(627.030303030303,-18)(612.10101010101,-18)(597.171717171717,-18)(582.242424242424,-18)(567.313131313131,-18)(552.383838383838,-18)(537.454545454546,-18)(522.525252525253,-18)(507.59595959596,-18)(492.666666666667,-18)(477.737373737374,-18)(462.808080808081,-18)(447.878787878788,-18)(432.949494949495,-18)(418.020202020202,-18)(403.090909090909,-18)(388.161616161616,-18)(373.232323232323,-18)(358.30303030303,-18)(343.373737373737,-18)(328.444444444444,-18)(313.515151515152,-18)(298.585858585859,-18)(283.656565656566,-18)(268.727272727273,-18)(253.79797979798,-18)(238.868686868687,-18)(223.939393939394,-18)(209.010101010101,-18)(194.080808080808,-18)(179.151515151515,-18)(164.222222222222,-18)(149.292929292929,-18)(134.363636363636,-18)(119.434343434343,-18)(104.505050505051,-18)(89.5757575757576,-18)(74.6464646464647,-18)(59.7171717171717,-18)(44.7878787878788,-18)(29.8585858585859,-18)(14.9292929292929,-18)(0,-18)(0,-17.8181818181818)(0,-17.6363636363636)(0,-17.4545454545455)(0,-17.2727272727273)(0,-17.0909090909091)(0,-16.9090909090909)(0,-16.7272727272727)(0,-16.5454545454545)(0,-16.3636363636364)(0,-16.1818181818182)(0,-16)(0,-15.8181818181818)(0,-15.6363636363636)(0,-15.4545454545455)(0,-15.2727272727273)(0,-15.0909090909091)(0,-14.9090909090909)(0,-14.7272727272727)(0,-14.5454545454545)(0,-14.3636363636364)(0,-14.1818181818182)(0,-14)(0,-13.8181818181818)(0,-13.6363636363636)(0,-13.4545454545455)(0,-13.2727272727273)(0,-13.0909090909091)(0,-12.9090909090909)(0,-12.7272727272727)(0,-12.5454545454545)(0,-12.3636363636364)(0,-12.1818181818182)(0,-12)(0,-11.8181818181818)(0,-11.6363636363636)(0,-11.4545454545455)(0,-11.2727272727273)(0,-11.0909090909091)(0,-10.9090909090909)(-0.000373222992657963,-10.7272727272727)(-0.000746427325097138,-10.5454545454545)(-0.00111961299872307,-10.3636363636364)(-0.00111961299872307,-10.1818181818182)(-0.00111961299872307,-10)(-0.00111961299872307,-9.81818181818182)(-0.00111961299872307,-9.63636363636364)(-0.00111961299872307,-9.45454545454546)(-0.000746427325097138,-9.27272727272727)(-0.000373222992657963,-9.09090909090909)(0,-8.90909090909091)(0,-8.72727272727273)(0,-8.54545454545455)(0,-8.36363636363636)(0,-8.18181818181818)(0,-8)(-0.00012440973766168,-7.81818181818182)(-0.000248817401863321,-7.63636363636364)(-0.000248817401863321,-7.45454545454545)(-0.00012440973766168,-7.27272727272727)(0,-7.09090909090909)(0,-6.90909090909091)(0,-6.72727272727273)(0,-6.54545454545455)(0,-6.36363636363636)(0,-6.18181818181818)(0,-6)(0,-5.81818181818182)(0,-5.63636363636364)(0,-5.45454545454545)(0,-5.27272727272727)(0,-5.09090909090909)(0,-4.90909090909091)(0,-4.72727272727273)(0,-4.54545454545455)(0,-4.36363636363636)(0,-4.18181818181818)(0,-4)(0,-3.81818181818182)(0,-3.63636363636364)(0,-3.45454545454545)(0,-3.27272727272727)(0,-3.09090909090909)(0,-2.90909090909091)(0,-2.72727272727273)(0,-2.54545454545455)(0,-2.36363636363636)(0,-2.18181818181818)(0,-2)(0,-1.81818181818182)(0,-1.63636363636364)(0,-1.45454545454545)(0,-1.27272727272727)(0,-1.09090909090909)(0,-0.909090909090909)(0,-0.727272727272727)(0,-0.545454545454545)(0,-0.363636363636364)(0,-0.181818181818182)(0,0)(14.9292929292929,0)(29.8585858585859,0)(44.7878787878788,0)(59.7171717171717,0)(74.6464646464647,0)(89.5757575757576,0)(104.505050505051,0)(119.434343434343,0)(134.363636363636,0)(149.292929292929,0)(164.222222222222,0)(179.151515151515,0)(194.080808080808,0)(209.010101010101,0)(223.939393939394,0)(238.868686868687,0)(253.79797979798,0)(268.727272727273,0)(283.656565656566,0)(298.585858585859,0)(313.515151515152,0)(328.444444444444,0)(343.373737373737,0)(358.30303030303,0)(373.232323232323,0)(388.161616161616,0)(403.090909090909,0)(418.020202020202,0)(432.949494949495,0)(447.878787878788,0)(462.808080808081,0)(477.737373737374,0)(492.666666666667,0)(507.59595959596,0)(522.525252525253,0)(537.454545454546,0)(552.383838383838,0)(567.313131313131,0)(582.242424242424,0)(597.171717171717,0)(612.10101010101,0)(627.030303030303,0)(641.959595959596,0)(656.888888888889,0)(671.818181818182,0)(686.747474747475,0)(701.676767676768,0)(716.606060606061,0)(731.535353535354,0)(746.464646464646,0)(761.393939393939,0)(776.323232323232,0)(791.252525252525,0)(806.181818181818,0)(821.111111111111,0)(836.040404040404,0)(850.969696969697,0)(865.89898989899,0)(880.828282828283,0)(895.757575757576,0)(910.686868686869,0)(925.616161616162,0)(940.545454545455,0)(955.474747474747,0)(970.40404040404,0)(985.333333333333,0)(1000.26262626263,0)(1015.19191919192,0)(1030.12121212121,0)(1045.05050505051,0)(1059.9797979798,0)(1074.90909090909,0)(1089.83838383838,0)(1104.76767676768,0)(1119.69696969697,0)(1134.62626262626,0)(1149.55555555556,0)(1164.48484848485,0)(1179.41414141414,0)(1194.34343434343,0)(1209.27272727273,0)(1224.20202020202,0)(1239.13131313131,0)(1254.06060606061,0)(1268.9898989899,0)(1283.91919191919,0)(1298.84848484848,0)(1313.77777777778,0)(1328.70707070707,0)(1343.63636363636,0)(1358.56565656566,0)(1373.49494949495,0)(1388.42424242424,0)(1403.35353535354,0)(1418.28282828283,0)(1433.21212121212,0)(1448.14141414141,0)(1463.07070707071,0)(1478,0)};

\addplot [fill=darkgray,draw=none,forget plot] coordinates{ (694.212121212121,-5.27272727272727)(701.676767676768,-5.3030303030303)(716.606060606061,-5.42424242424242)(724.070707070707,-5.45454545454545)(731.535353535354,-5.48484848484848)(746.464646464646,-5.60606060606061)(753.929292929293,-5.63636363636364)(761.393939393939,-5.66666666666667)(776.323232323232,-5.78787878787879)(783.787878787879,-5.81818181818182)(791.252525252525,-5.84848484848485)(806.181818181818,-5.90909090909091)(821.111111111111,-5.96969696969697)(828.575757575758,-6)(836.040404040404,-6.03030303030303)(850.969696969697,-6.09090909090909)(865.89898989899,-6.09090909090909)(880.828282828283,-6.09090909090909)(895.757575757576,-6.09090909090909)(910.686868686869,-6.09090909090909)(925.616161616162,-6.15151515151515)(933.080808080808,-6.18181818181818)(940.545454545455,-6.21212121212121)(955.474747474747,-6.27272727272727)(970.40404040404,-6.27272727272727)(985.333333333333,-6.27272727272727)(1000.26262626263,-6.33333333333333)(1007.72727272727,-6.36363636363636)(1015.19191919192,-6.39393939393939)(1030.12121212121,-6.45454545454546)(1045.05050505051,-6.51515151515152)(1052.51515151515,-6.54545454545455)(1059.9797979798,-6.57575757575758)(1074.90909090909,-6.6969696969697)(1082.37373737374,-6.72727272727273)(1089.83838383838,-6.75757575757576)(1104.76767676768,-6.87878787878788)(1112.23232323232,-6.90909090909091)(1119.69696969697,-6.93939393939394)(1134.62626262626,-7.06060606060606)(1142.09090909091,-7.09090909090909)(1149.55555555556,-7.12121212121212)(1164.48484848485,-7.18181818181818)(1179.41414141414,-7.18181818181818)(1194.34343434343,-7.24242424242424)(1201.80808080808,-7.27272727272727)(1209.27272727273,-7.3030303030303)(1224.20202020202,-7.36363636363636)(1239.13131313131,-7.3030303030303)(1246.59595959596,-7.27272727272727)(1254.06060606061,-7.24242424242424)(1268.9898989899,-7.18181818181818)(1283.91919191919,-7.18181818181818)(1298.84848484848,-7.12121212121212)(1302.58080808081,-7.09090909090909)(1313.77777777778,-7)(1324.97474747475,-6.90909090909091)(1328.70707070707,-6.87878787878788)(1343.63636363636,-6.75757575757576)(1347.36868686869,-6.72727272727273)(1358.56565656566,-6.63636363636364)(1369.76262626263,-6.54545454545455)(1373.49494949495,-6.51515151515152)(1388.42424242424,-6.39393939393939)(1395.88888888889,-6.36363636363636)(1403.35353535354,-6.33333333333333)(1418.28282828283,-6.27272727272727)(1433.21212121212,-6.21212121212121)(1440.67676767677,-6.18181818181818)(1448.14141414141,-6.15151515151515)(1463.07070707071,-6.09090909090909)(1478,-6.09090909090909)(1478.00018660683,-6.18181818181818)(1478.0005598065,-6.36363636363636)(1478.0005598065,-6.54545454545455)(1478.0005598065,-6.72727272727273)(1478.0005598065,-6.90909090909091)(1478.0005598065,-7.09090909090909)(1478.00031100879,-7.27272727272727)(1478.00031100879,-7.45454545454545)(1478.00031100879,-7.63636363636364)(1478.0005598065,-7.81818181818182)(1478.0005598065,-8)(1478.0005598065,-8.18181818181818)(1478.0005598065,-8.36363636363636)(1478.0005598065,-8.54545454545455)(1478.00031100879,-8.72727272727273)(1478.00031100879,-8.90909090909091)(1478.00031100879,-9.09090909090909)(1478.0005598065,-9.27272727272727)(1478.0005598065,-9.45454545454546)(1478.0005598065,-9.63636363636364)(1478.00031100879,-9.81818181818182)(1478,-9.96969696969697)(1470.53535353535,-10)(1463.07070707071,-10.030303030303)(1448.14141414141,-10.1515151515152)(1444.40909090909,-10.1818181818182)(1433.21212121212,-10.2727272727273)(1422.01515151515,-10.3636363636364)(1418.28282828283,-10.3939393939394)(1403.35353535354,-10.5151515151515)(1395.88888888889,-10.5454545454545)(1388.42424242424,-10.5757575757576)(1373.49494949495,-10.6969696969697)(1369.76262626263,-10.7272727272727)(1358.56565656566,-10.8181818181818)(1351.10101010101,-10.9090909090909)(1343.63636363636,-11)(1336.17171717172,-11.0909090909091)(1328.70707070707,-11.1818181818182)(1317.5101010101,-11.2727272727273)(1313.77777777778,-11.3030303030303)(1298.84848484848,-11.4242424242424)(1295.11616161616,-11.4545454545455)(1283.91919191919,-11.5454545454545)(1276.45454545455,-11.6363636363636)(1268.9898989899,-11.7272727272727)(1257.79292929293,-11.8181818181818)(1254.06060606061,-11.8484848484849)(1239.13131313131,-11.969696969697)(1231.66666666667,-12)(1224.20202020202,-12.030303030303)(1209.27272727273,-12.0909090909091)(1194.34343434343,-12.0909090909091)(1179.41414141414,-12.0909090909091)(1164.48484848485,-12.0909090909091)(1149.55555555556,-12.0909090909091)(1134.62626262626,-12.0909090909091)(1119.69696969697,-12.0909090909091)(1104.76767676768,-12.0909090909091)(1089.83838383838,-12.0909090909091)(1074.90909090909,-12.030303030303)(1067.44444444444,-12)(1059.9797979798,-11.969696969697)(1045.05050505051,-11.9090909090909)(1030.12121212121,-11.9090909090909)(1015.19191919192,-11.8484848484849)(1007.72727272727,-11.8181818181818)(1000.26262626263,-11.7878787878788)(985.333333333333,-11.7272727272727)(970.40404040404,-11.7272727272727)(955.474747474747,-11.6666666666667)(948.010101010101,-11.6363636363636)(940.545454545455,-11.6060606060606)(925.616161616162,-11.5454545454545)(910.686868686869,-11.5454545454545)(895.757575757576,-11.5454545454545)(880.828282828283,-11.5454545454545)(865.89898989899,-11.4848484848485)(858.434343434343,-11.4545454545455)(850.969696969697,-11.4242424242424)(836.040404040404,-11.3636363636364)(821.111111111111,-11.3636363636364)(806.181818181818,-11.3636363636364)(791.252525252525,-11.3030303030303)(783.787878787879,-11.2727272727273)(776.323232323232,-11.2424242424242)(761.393939393939,-11.1818181818182)(746.464646464646,-11.1212121212121)(739,-11.0909090909091)(731.535353535354,-11.0606060606061)(716.606060606061,-10.9393939393939)(709.141414141414,-10.9090909090909)(701.676767676768,-10.8787878787879)(686.747474747475,-10.7575757575758)(679.282828282828,-10.7272727272727)(671.818181818182,-10.6969696969697)(656.888888888889,-10.6363636363636)(641.959595959596,-10.5757575757576)(634.494949494949,-10.5454545454545)(627.030303030303,-10.5151515151515)(612.10101010101,-10.4545454545455)(597.171717171717,-10.3939393939394)(589.707070707071,-10.3636363636364)(582.242424242424,-10.3333333333333)(567.313131313131,-10.2727272727273)(552.383838383838,-10.2727272727273)(537.454545454546,-10.2727272727273)(522.525252525253,-10.2727272727273)(507.59595959596,-10.2121212121212)(500.131313131313,-10.1818181818182)(492.666666666667,-10.1515151515152)(477.737373737374,-10.0909090909091)(462.808080808081,-10.0909090909091)(447.878787878788,-10.0909090909091)(432.949494949495,-10.0909090909091)(418.020202020202,-10.0909090909091)(403.090909090909,-10.0909090909091)(388.161616161616,-10.0909090909091)(373.232323232323,-10.0909090909091)(358.30303030303,-10.030303030303)(354.570707070707,-10)(343.373737373737,-9.90909090909091)(335.909090909091,-9.81818181818182)(328.444444444444,-9.68181818181818)(325.956228956229,-9.63636363636364)(313.515151515152,-9.48484848484848)(311.026936026936,-9.45454545454546)(298.585858585859,-9.3030303030303)(291.121212121212,-9.27272727272727)(283.656565656566,-9.24242424242424)(268.727272727273,-9.18181818181818)(253.79797979798,-9.18181818181818)(238.868686868687,-9.18181818181818)(227.671717171717,-9.27272727272727)(223.939393939394,-9.36363636363636)(221.451178451178,-9.45454545454546)(223.939393939394,-9.5)(231.40404040404,-9.63636363636364)(238.868686868687,-9.72727272727273)(250.065656565657,-9.81818181818182)(253.79797979798,-9.84848484848485)(268.727272727273,-9.96969696969697)(272.459595959596,-10)(283.656565656566,-10.1363636363636)(286.144781144781,-10.1818181818182)(286.144781144781,-10.3636363636364)(283.656565656566,-10.4090909090909)(272.459595959596,-10.5454545454545)(268.727272727273,-10.5757575757576)(253.79797979798,-10.6969696969697)(246.333333333333,-10.7272727272727)(238.868686868687,-10.7575757575758)(223.939393939394,-10.8181818181818)(209.010101010101,-10.8181818181818)(194.080808080808,-10.8181818181818)(179.151515151515,-10.8181818181818)(164.222222222222,-10.8181818181818)(149.292929292929,-10.8181818181818)(134.363636363636,-10.8181818181818)(119.434343434343,-10.8181818181818)(104.505050505051,-10.7575757575758)(97.040404040404,-10.7272727272727)(89.5757575757576,-10.6969696969697)(74.6464646464647,-10.6363636363636)(59.7171717171717,-10.6363636363636)(44.7878787878788,-10.6363636363636)(29.8585858585859,-10.6363636363636)(14.9292929292929,-10.6363636363636)(0,-10.6363636363636)(-0.000186606831274284,-10.5454545454545)(-0.000559806499359878,-10.3636363636364)(-0.000559806499359878,-10.1818181818182)(-0.000559806499359878,-10)(-0.000559806499359878,-9.81818181818182)(-0.000559806499359878,-9.63636363636364)(-0.000559806499359878,-9.45454545454546)(-0.000186606831274284,-9.27272727272727)(0,-9.18181818181818)(14.9292929292929,-9.18181818181818)(29.8585858585859,-9.18181818181818)(44.7878787878788,-9.12121212121212)(52.2525252525253,-9.09090909090909)(59.7171717171717,-9.06060606060606)(72.1582491582492,-8.90909090909091)(74.6464646464647,-8.81818181818182)(77.1346801346801,-8.72727272727273)(74.6464646464647,-8.63636363636364)(72.1582491582492,-8.54545454545455)(62.2053872053872,-8.36363636363636)(59.7171717171717,-8.31818181818182)(52.2525252525253,-8.18181818181818)(44.7878787878788,-8.09090909090909)(37.3232323232323,-8)(29.8585858585859,-7.90909090909091)(22.3939393939394,-7.81818181818182)(14.9292929292929,-7.68181818181818)(12.4410774410774,-7.63636363636364)(12.4410774410774,-7.45454545454545)(14.9292929292929,-7.40909090909091)(26.1262626262626,-7.27272727272727)(29.8585858585859,-7.24242424242424)(44.7878787878788,-7.12121212121212)(52.2525252525253,-7.09090909090909)(59.7171717171717,-7.06060606060606)(74.6464646464647,-6.93939393939394)(82.1111111111111,-6.90909090909091)(89.5757575757576,-6.87878787878788)(104.505050505051,-6.81818181818182)(119.434343434343,-6.75757575757576)(126.89898989899,-6.72727272727273)(134.363636363636,-6.6969696969697)(149.292929292929,-6.63636363636364)(164.222222222222,-6.63636363636364)(179.151515151515,-6.63636363636364)(194.080808080808,-6.63636363636364)(209.010101010101,-6.63636363636364)(223.939393939394,-6.63636363636364)(238.868686868687,-6.63636363636364)(253.79797979798,-6.63636363636364)(268.727272727273,-6.6969696969697)(276.191919191919,-6.72727272727273)(283.656565656566,-6.75757575757576)(298.585858585859,-6.81818181818182)(313.515151515152,-6.81818181818182)(328.444444444444,-6.81818181818182)(343.373737373737,-6.81818181818182)(358.30303030303,-6.81818181818182)(373.232323232323,-6.81818181818182)(388.161616161616,-6.81818181818182)(403.090909090909,-6.75757575757576)(406.823232323232,-6.72727272727273)(418.020202020202,-6.63636363636364)(425.484848484849,-6.54545454545455)(432.949494949495,-6.45454545454546)(440.414141414141,-6.36363636363636)(447.878787878788,-6.27272727272727)(455.343434343434,-6.18181818181818)(462.808080808081,-6.04545454545455)(465.296296296296,-6)(477.737373737374,-5.84848484848485)(480.225589225589,-5.81818181818182)(492.666666666667,-5.66666666666667)(500.131313131313,-5.63636363636364)(507.59595959596,-5.60606060606061)(522.525252525253,-5.48484848484848)(529.989898989899,-5.45454545454545)(537.454545454546,-5.42424242424242)(552.383838383838,-5.36363636363636)(567.313131313131,-5.3030303030303)(574.777777777778,-5.27272727272727)(582.242424242424,-5.24242424242424)(597.171717171717,-5.18181818181818)(612.10101010101,-5.18181818181818)(627.030303030303,-5.18181818181818)(641.959595959596,-5.18181818181818)(656.888888888889,-5.18181818181818)(671.818181818182,-5.18181818181818)(686.747474747475,-5.24242424242424)(694.212121212121,-5.27272727272727)};

\addplot [fill=locol,draw=none,forget plot] coordinates{ (757.661616161616,-6.90909090909091)(746.464646464646,-6.81818181818182)(731.535353535354,-6.81818181818182)(716.606060606061,-6.87878787878788)(712.873737373737,-6.90909090909091)(704.164983164983,-7.09090909090909)(712.873737373737,-7.27272727272727)(716.606060606061,-7.3030303030303)(731.535353535354,-7.36363636363636)(746.464646464646,-7.3030303030303)(750.19696969697,-7.27272727272727)(761.393939393939,-7.13636363636364)(763.882154882155,-7.09090909090909)(761.393939393939,-7)(757.661616161616,-6.90909090909091)};

\addplot [
color=white,
draw=white,
only marks,
mark=x,
mark options={solid},
mark size=2.0pt,
line width=0.3pt,
forget plot
]
coordinates{
 (731.535353535354,-1.63636363636364)(1478,-18)(0,-12.3636363636364)(1478,-8.90909090909091)(432.949494949495,-18)(0,-5.63636363636364)(1478,0)(0,0)(880.828282828283,-12.7272727272727)(731.535353535354,-7.27272727272727)(1478,-4.36363636363636)(1478,-13.6363636363636)(0,-16.5454545454545)(955.474747474747,-18)(268.727272727273,-9.27272727272727)(447.878787878788,-14.3636363636364)(388.161616161616,-3.81818181818182)(1000.26262626263,-4.36363636363636)(1030.12121212121,-9.81818181818182)(1045.05050505051,0)(432.949494949495,0)(1104.76767676768,-15.4545454545455)(0,-2.72727272727273)(552.383838383838,-11.0909090909091)(1209.27272727273,-6.90909090909091)(0,-8.54545454545455)(1478,-11.2727272727273)(1268.9898989899,-2.18181818181818)(358.30303030303,-6.54545454545455)
};

\addplot [
draw=red!50!white,
line width=0.6pt,
forget plot
]
coordinates{
(716.606060606061,-16)(507.59595959596,-8.36363636363636)(238.868686868687,-7.81818181818182)(507.59595959596,-7.45454545454545)(686.747474747475,-8.90909090909091)(821.111111111111,-9.09090909090909)(836.040404040404,-8.90909090909091)(850.969696969697,-9.45454545454546)(865.89898989899,-8.54545454545455)(865.89898989899,-9.63636363636364)(880.828282828283,-9.81818181818182)(895.757575757576,-8.36363636363636)(910.686868686869,-8.36363636363636)(940.545454545455,-10.1818181818182)(955.474747474747,-10.1818181818182)(1015.19191919192,-8.18181818181818)(1030.12121212121,-10.3636363636364)(1045.05050505051,-8.18181818181818)(1089.83838383838,-8.18181818181818)(1119.69696969697,-10.3636363636364)(1134.62626262626,-10.1818181818182)(1179.41414141414,-10)(1179.41414141414,-8.36363636363636)(1194.34343434343,-8.36363636363636)(1224.20202020202,-9.81818181818182)(1239.13131313131,-9.63636363636364)(1239.13131313131,-8.54545454545455)(1268.9898989899,-8.72727272727273)(1268.9898989899,-9.45454545454546)(1283.91919191919,-9.09090909090909)(1478,-7.45454545454545) 
};

\addplot [
color=white,
draw=black,
only marks,
mark=*,
mark options={solid},
mark size=2pt,
line width=0.2pt,
forget plot
]
coordinates{
 (716.606060606061,-16)
};

\addplot [
color=red,
draw=black,
only marks,
mark=*,
mark options={solid},
mark size=1.6pt,
line width=0.2pt,
forget plot
]
coordinates{
(507.59595959596,-8.36363636363636)
(238.868686868687,-7.81818181818182)
(507.59595959596,-7.45454545454545)
(686.747474747475,-8.90909090909091)
(821.111111111111,-9.09090909090909)
(836.040404040404,-8.90909090909091)
(850.969696969697,-9.45454545454546)
(865.89898989899,-8.54545454545455)
(865.89898989899,-9.63636363636364)
(880.828282828283,-9.81818181818182)
(895.757575757576,-8.36363636363636)
(910.686868686869,-8.36363636363636)
(940.545454545455,-10.1818181818182)
(955.474747474747,-10.1818181818182)
(1015.19191919192,-8.18181818181818)
(1030.12121212121,-10.3636363636364)
(1045.05050505051,-8.18181818181818)
(1089.83838383838,-8.18181818181818)
(1119.69696969697,-10.3636363636364)
(1134.62626262626,-10.1818181818182)
(1179.41414141414,-10)
(1179.41414141414,-8.36363636363636)
(1194.34343434343,-8.36363636363636)
(1224.20202020202,-9.81818181818182)
(1239.13131313131,-9.63636363636364)
(1239.13131313131,-8.54545454545455)
(1268.9898989899,-8.72727272727273)
(1268.9898989899,-9.45454545454546)
(1283.91919191919,-9.09090909090909)
(1478,-7.45454545454545) 
};

\node at (axis cs:980, -17) [shape=circle,fill=red!40!yellow,draw=black,inner sep=0.2pt,anchor=south west,minimum size=16pt]
  {\scriptsize\color{white}$\*H_{\*t}$};
\node at (axis cs:1155, -17) [shape=circle,fill=locol,draw=black,inner sep=0.2pt,anchor=south west,minimum size=16pt]
  {\scriptsize\color{white}$\*L_{\*t}$};
\node at (axis cs:1330, -17) [shape=circle,fill=darkgray,draw=black,inner sep=0.2pt,anchor=south west,minimum size=16pt]
  {\scriptsize\color{white}$\*U_{\*t}$};

\end{axis}
\end{tikzpicture}%

%% This file was created by matlab2tikz v0.2.3.
% Copyright (c) 2008--2012, Nico Schlömer <nico.schloemer@gmail.com>
% All rights reserved.
% 
% 
%

\definecolor{locol}{rgb}{0.26, 0.45, 0.65}

\begin{tikzpicture}

\begin{axis}[%
tick label style={font=\tiny},
label style={font=\tiny},
xlabel shift={-10pt},
ylabel shift={-17pt},
legend style={font=\tiny},
view={0}{90},
width=\figurewidth,
height=\figureheight,
scale only axis,
xmin=0, xmax=1478,
xtick={0, 400, 1000, 1400},
xlabel={Length (m)},
ymin=-18, ymax=0,
ytick={0, -4, -14, -18},
ylabel={Depth (m)},
name=plot1,
axis lines*=box,
tickwidth=0.1cm,
clip=false
]

\addplot [fill=locol,draw=none,forget plot] coordinates{ (1478,0)(1478,-0.181818181818182)(1478,-0.363636363636364)(1478,-0.545454545454545)(1478,-0.727272727272727)(1478,-0.909090909090909)(1478,-1.09090909090909)(1478,-1.27272727272727)(1478,-1.45454545454545)(1478,-1.63636363636364)(1478,-1.81818181818182)(1478,-2)(1478,-2.18181818181818)(1478,-2.36363636363636)(1478,-2.54545454545455)(1478,-2.72727272727273)(1478,-2.90909090909091)(1478,-3.09090909090909)(1478,-3.27272727272727)(1478,-3.45454545454545)(1478,-3.63636363636364)(1478,-3.81818181818182)(1478,-4)(1478,-4.18181818181818)(1478,-4.36363636363636)(1478,-4.54545454545455)(1478,-4.72727272727273)(1478,-4.90909090909091)(1478,-5.09090909090909)(1478,-5.27272727272727)(1478,-5.45454545454545)(1478,-5.63636363636364)(1478,-5.81818181818182)(1478,-6)(1478,-6.18181818181818)(1478,-6.36363636363636)(1478,-6.54545454545455)(1478,-6.72727272727273)(1478,-6.90909090909091)(1478,-7.09090909090909)(1478,-7.27272727272727)(1478,-7.45454545454545)(1478,-7.63636363636364)(1478,-7.81818181818182)(1478.0002488174,-8)(1478.00049762651,-8.18181818181818)(1478.00074642733,-8.36363636363636)(1478.00074642733,-8.54545454545455)(1478.00058056104,-8.72727272727273)(1478.00058056104,-8.90909090909091)(1478.00058056104,-9.09090909090909)(1478.00074642733,-9.27272727272727)(1478.00074642733,-9.45454545454546)(1478.00074642733,-9.63636363636364)(1478.00049762651,-9.81818181818182)(1478.0002488174,-10)(1478,-10.1818181818182)(1478,-10.3636363636364)(1478,-10.5454545454545)(1478,-10.7272727272727)(1478,-10.9090909090909)(1478,-11.0909090909091)(1478,-11.2727272727273)(1478,-11.4545454545455)(1478,-11.6363636363636)(1478,-11.8181818181818)(1478,-12)(1478,-12.1818181818182)(1478,-12.3636363636364)(1478,-12.5454545454545)(1478,-12.7272727272727)(1478,-12.9090909090909)(1478,-13.0909090909091)(1478,-13.2727272727273)(1478,-13.4545454545455)(1478,-13.6363636363636)(1478,-13.8181818181818)(1478,-14)(1478,-14.1818181818182)(1478,-14.3636363636364)(1478,-14.5454545454545)(1478,-14.7272727272727)(1478,-14.9090909090909)(1478,-15.0909090909091)(1478,-15.2727272727273)(1478,-15.4545454545455)(1478,-15.6363636363636)(1478,-15.8181818181818)(1478,-16)(1478,-16.1818181818182)(1478,-16.3636363636364)(1478,-16.5454545454545)(1478,-16.7272727272727)(1478,-16.9090909090909)(1478,-17.0909090909091)(1478,-17.2727272727273)(1478,-17.4545454545455)(1478,-17.6363636363636)(1478,-17.8181818181818)(1478,-18)(1463.07070707071,-18)(1448.14141414141,-18)(1433.21212121212,-18)(1418.28282828283,-18)(1403.35353535354,-18)(1388.42424242424,-18)(1373.49494949495,-18)(1358.56565656566,-18)(1343.63636363636,-18)(1328.70707070707,-18)(1313.77777777778,-18)(1298.84848484848,-18)(1283.91919191919,-18)(1268.9898989899,-18)(1254.06060606061,-18)(1239.13131313131,-18)(1224.20202020202,-18)(1209.27272727273,-18)(1194.34343434343,-18)(1179.41414141414,-18)(1164.48484848485,-18)(1149.55555555556,-18)(1134.62626262626,-18)(1119.69696969697,-18)(1104.76767676768,-18)(1089.83838383838,-18)(1074.90909090909,-18)(1059.9797979798,-18)(1045.05050505051,-18)(1030.12121212121,-18)(1015.19191919192,-18)(1000.26262626263,-18)(985.333333333333,-18)(970.40404040404,-18)(955.474747474747,-18)(940.545454545455,-18)(925.616161616162,-18)(910.686868686869,-18)(895.757575757576,-18)(880.828282828283,-18)(865.89898989899,-18)(850.969696969697,-18)(836.040404040404,-18)(821.111111111111,-18)(806.181818181818,-18)(791.252525252525,-18)(776.323232323232,-18)(761.393939393939,-18)(746.464646464646,-18)(731.535353535354,-18)(716.606060606061,-18)(701.676767676768,-18)(686.747474747475,-18)(671.818181818182,-18)(656.888888888889,-18)(641.959595959596,-18)(627.030303030303,-18)(612.10101010101,-18)(597.171717171717,-18)(582.242424242424,-18)(567.313131313131,-18)(552.383838383838,-18)(537.454545454546,-18)(522.525252525253,-18)(507.59595959596,-18)(492.666666666667,-18)(477.737373737374,-18)(462.808080808081,-18)(447.878787878788,-18)(432.949494949495,-18)(418.020202020202,-18)(403.090909090909,-18)(388.161616161616,-18)(373.232323232323,-18)(358.30303030303,-18)(343.373737373737,-18)(328.444444444444,-18)(313.515151515152,-18)(298.585858585859,-18)(283.656565656566,-18)(268.727272727273,-18)(253.79797979798,-18)(238.868686868687,-18)(223.939393939394,-18)(209.010101010101,-18)(194.080808080808,-18)(179.151515151515,-18)(164.222222222222,-18)(149.292929292929,-18)(134.363636363636,-18)(119.434343434343,-18)(104.505050505051,-18)(89.5757575757576,-18)(74.6464646464647,-18)(59.7171717171717,-18)(44.7878787878788,-18)(29.8585858585859,-18)(14.9292929292929,-18)(0,-18)(0,-17.8181818181818)(0,-17.6363636363636)(0,-17.4545454545455)(0,-17.2727272727273)(0,-17.0909090909091)(0,-16.9090909090909)(0,-16.7272727272727)(0,-16.5454545454545)(0,-16.3636363636364)(0,-16.1818181818182)(0,-16)(0,-15.8181818181818)(0,-15.6363636363636)(0,-15.4545454545455)(0,-15.2727272727273)(0,-15.0909090909091)(0,-14.9090909090909)(0,-14.7272727272727)(0,-14.5454545454545)(0,-14.3636363636364)(0,-14.1818181818182)(0,-14)(0,-13.8181818181818)(0,-13.6363636363636)(0,-13.4545454545455)(0,-13.2727272727273)(0,-13.0909090909091)(0,-12.9090909090909)(0,-12.7272727272727)(0,-12.5454545454545)(0,-12.3636363636364)(0,-12.1818181818182)(0,-12)(0,-11.8181818181818)(0,-11.6363636363636)(0,-11.4545454545455)(0,-11.2727272727273)(0,-11.0909090909091)(0,-10.9090909090909)(0,-10.7272727272727)(0,-10.5454545454545)(0,-10.3636363636364)(-0.000248817401863321,-10.1818181818182)(-0.000497626510092015,-10)(-0.000746427325097138,-9.81818181818182)(-0.000497626510092015,-9.63636363636364)(-0.000248817401863321,-9.45454545454546)(0,-9.27272727272727)(0,-9.09090909090909)(0,-8.90909090909091)(0,-8.72727272727273)(0,-8.54545454545455)(0,-8.36363636363636)(0,-8.18181818181818)(0,-8)(-8.29400554970238e-05,-7.81818181818182)(-0.000165879189445939,-7.63636363636364)(-0.000165879189445939,-7.45454545454545)(-8.29400554970238e-05,-7.27272727272727)(0,-7.09090909090909)(0,-6.90909090909091)(0,-6.72727272727273)(0,-6.54545454545455)(0,-6.36363636363636)(0,-6.18181818181818)(0,-6)(0,-5.81818181818182)(0,-5.63636363636364)(0,-5.45454545454545)(0,-5.27272727272727)(0,-5.09090909090909)(0,-4.90909090909091)(0,-4.72727272727273)(0,-4.54545454545455)(0,-4.36363636363636)(0,-4.18181818181818)(0,-4)(0,-3.81818181818182)(0,-3.63636363636364)(0,-3.45454545454545)(0,-3.27272727272727)(0,-3.09090909090909)(0,-2.90909090909091)(0,-2.72727272727273)(0,-2.54545454545455)(0,-2.36363636363636)(0,-2.18181818181818)(0,-2)(0,-1.81818181818182)(0,-1.63636363636364)(0,-1.45454545454545)(0,-1.27272727272727)(0,-1.09090909090909)(0,-0.909090909090909)(0,-0.727272727272727)(0,-0.545454545454545)(0,-0.363636363636364)(0,-0.181818181818182)(0,0)(14.9292929292929,0)(29.8585858585859,0)(44.7878787878788,0)(59.7171717171717,0)(74.6464646464647,0)(89.5757575757576,0)(104.505050505051,0)(119.434343434343,0)(134.363636363636,0)(149.292929292929,0)(164.222222222222,0)(179.151515151515,0)(194.080808080808,0)(209.010101010101,0)(223.939393939394,0)(238.868686868687,0)(253.79797979798,0)(268.727272727273,0)(283.656565656566,0)(298.585858585859,0)(313.515151515152,0)(328.444444444444,0)(343.373737373737,0)(358.30303030303,0)(373.232323232323,0)(388.161616161616,0)(403.090909090909,0)(418.020202020202,0)(432.949494949495,0)(447.878787878788,0)(462.808080808081,0)(477.737373737374,0)(492.666666666667,0)(507.59595959596,0)(522.525252525253,0)(537.454545454546,0)(552.383838383838,0)(567.313131313131,0)(582.242424242424,0)(597.171717171717,0)(612.10101010101,0)(627.030303030303,0)(641.959595959596,0)(656.888888888889,0)(671.818181818182,0)(686.747474747475,0)(701.676767676768,0)(716.606060606061,0)(731.535353535354,0)(746.464646464646,0)(761.393939393939,0)(776.323232323232,0)(791.252525252525,0)(806.181818181818,0)(821.111111111111,0)(836.040404040404,0)(850.969696969697,0)(865.89898989899,0)(880.828282828283,0)(895.757575757576,0)(910.686868686869,0)(925.616161616162,0)(940.545454545455,0)(955.474747474747,0)(970.40404040404,0)(985.333333333333,0)(1000.26262626263,0)(1015.19191919192,0)(1030.12121212121,0)(1045.05050505051,0)(1059.9797979798,0)(1074.90909090909,0)(1089.83838383838,0)(1104.76767676768,0)(1119.69696969697,0)(1134.62626262626,0)(1149.55555555556,0)(1164.48484848485,0)(1179.41414141414,0)(1194.34343434343,0)(1209.27272727273,0)(1224.20202020202,0)(1239.13131313131,0)(1254.06060606061,0)(1268.9898989899,0)(1283.91919191919,0)(1298.84848484848,0)(1313.77777777778,0)(1328.70707070707,0)(1343.63636363636,0)(1358.56565656566,0)(1373.49494949495,0)(1388.42424242424,0)(1403.35353535354,0)(1418.28282828283,0)(1433.21212121212,0)(1448.14141414141,0)(1463.07070707071,0)(1478,0)};

\addplot [fill=darkgray,draw=none,forget plot] coordinates{ (615.833333333333,-6.18181818181818)(627.030303030303,-6.27272727272727)(634.494949494949,-6.36363636363636)(641.959595959596,-6.45454545454546)(649.424242424242,-6.54545454545455)(656.888888888889,-6.63636363636364)(664.353535353535,-6.72727272727273)(671.818181818182,-6.81818181818182)(679.282828282828,-6.90909090909091)(686.747474747475,-7)(694.212121212121,-7.09090909090909)(701.676767676768,-7.18181818181818)(712.873737373737,-7.27272727272727)(716.606060606061,-7.3030303030303)(731.535353535354,-7.36363636363636)(746.464646464646,-7.3030303030303)(750.19696969697,-7.27272727272727)(761.393939393939,-7.13636363636364)(763.882154882155,-7.09090909090909)(773.835016835017,-6.90909090909091)(776.323232323232,-6.86363636363636)(787.520202020202,-6.72727272727273)(791.252525252525,-6.6969696969697)(806.181818181818,-6.57575757575758)(813.646464646465,-6.54545454545455)(821.111111111111,-6.51515151515152)(836.040404040404,-6.45454545454546)(850.969696969697,-6.39393939393939)(858.434343434343,-6.36363636363636)(865.89898989899,-6.33333333333333)(880.828282828283,-6.27272727272727)(895.757575757576,-6.27272727272727)(910.686868686869,-6.27272727272727)(925.616161616162,-6.27272727272727)(940.545454545455,-6.27272727272727)(955.474747474747,-6.27272727272727)(970.40404040404,-6.27272727272727)(985.333333333333,-6.33333333333333)(992.79797979798,-6.36363636363636)(1000.26262626263,-6.39393939393939)(1015.19191919192,-6.45454545454546)(1030.12121212121,-6.45454545454546)(1045.05050505051,-6.51515151515152)(1052.51515151515,-6.54545454545455)(1059.9797979798,-6.57575757575758)(1074.90909090909,-6.6969696969697)(1082.37373737374,-6.72727272727273)(1089.83838383838,-6.75757575757576)(1104.76767676768,-6.87878787878788)(1112.23232323232,-6.90909090909091)(1119.69696969697,-6.93939393939394)(1134.62626262626,-7.06060606060606)(1142.09090909091,-7.09090909090909)(1149.55555555556,-7.12121212121212)(1164.48484848485,-7.18181818181818)(1179.41414141414,-7.18181818181818)(1194.34343434343,-7.24242424242424)(1201.80808080808,-7.27272727272727)(1209.27272727273,-7.3030303030303)(1224.20202020202,-7.36363636363636)(1239.13131313131,-7.36363636363636)(1254.06060606061,-7.36363636363636)(1268.9898989899,-7.36363636363636)(1283.91919191919,-7.36363636363636)(1298.84848484848,-7.36363636363636)(1313.77777777778,-7.36363636363636)(1328.70707070707,-7.36363636363636)(1343.63636363636,-7.36363636363636)(1358.56565656566,-7.36363636363636)(1373.49494949495,-7.42424242424242)(1380.9595959596,-7.45454545454545)(1388.42424242424,-7.48484848484848)(1403.35353535354,-7.60606060606061)(1410.81818181818,-7.63636363636364)(1418.28282828283,-7.66666666666667)(1433.21212121212,-7.78787878787879)(1436.94444444444,-7.81818181818182)(1448.14141414141,-7.90909090909091)(1459.33838383838,-8)(1463.07070707071,-8.03030303030303)(1478,-8.09090909090909)(1478.00012440663,-8.18181818181818)(1478.00037321366,-8.36363636363636)(1478.00037321366,-8.54545454545455)(1478.00020734323,-8.72727272727273)(1478.00020734323,-8.90909090909091)(1478.00020734323,-9.09090909090909)(1478.00037321366,-9.27272727272727)(1478.00037321366,-9.45454545454546)(1478.00037321366,-9.63636363636364)(1478.00012440663,-9.81818181818182)(1478,-9.90909090909091)(1463.07070707071,-9.96969696969697)(1455.60606060606,-10)(1448.14141414141,-10.030303030303)(1433.21212121212,-10.1515151515152)(1425.74747474747,-10.1818181818182)(1418.28282828283,-10.2121212121212)(1403.35353535354,-10.2727272727273)(1388.42424242424,-10.3333333333333)(1380.9595959596,-10.3636363636364)(1373.49494949495,-10.3939393939394)(1358.56565656566,-10.4545454545455)(1343.63636363636,-10.5151515151515)(1336.17171717172,-10.5454545454545)(1328.70707070707,-10.5757575757576)(1313.77777777778,-10.6363636363636)(1298.84848484848,-10.6969696969697)(1291.38383838384,-10.7272727272727)(1283.91919191919,-10.7575757575758)(1268.9898989899,-10.8787878787879)(1261.52525252525,-10.9090909090909)(1254.06060606061,-10.9393939393939)(1239.13131313131,-11)(1224.20202020202,-11)(1209.27272727273,-11)(1194.34343434343,-11)(1179.41414141414,-11)(1164.48484848485,-11)(1149.55555555556,-10.9393939393939)(1142.09090909091,-10.9090909090909)(1134.62626262626,-10.8787878787879)(1119.69696969697,-10.7575757575758)(1112.23232323232,-10.7272727272727)(1104.76767676768,-10.7045454545455)(1089.83838383838,-10.6363636363636)(1074.90909090909,-10.5757575757576)(1067.44444444444,-10.5454545454545)(1059.9797979798,-10.5151515151515)(1045.05050505051,-10.3939393939394)(1030.12121212121,-10.3939393939394)(1015.19191919192,-10.3939393939394)(1000.26262626263,-10.3939393939394)(992.79797979798,-10.3636363636364)(985.333333333333,-10.3333333333333)(970.40404040404,-10.2121212121212)(962.939393939394,-10.1818181818182)(955.474747474747,-10.1666666666667)(940.545454545455,-10.1636363636364)(933.080808080808,-10.1818181818182)(925.616161616162,-10.2121212121212)(910.686868686869,-10.2727272727273)(895.757575757576,-10.2727272727273)(880.828282828283,-10.2727272727273)(865.89898989899,-10.2727272727273)(850.969696969697,-10.2727272727273)(836.040404040404,-10.2727272727273)(821.111111111111,-10.2727272727273)(806.181818181818,-10.2727272727273)(791.252525252525,-10.2727272727273)(776.323232323232,-10.2727272727273)(761.393939393939,-10.2121212121212)(753.929292929293,-10.1818181818182)(746.464646464646,-10.1515151515152)(731.535353535354,-10.0909090909091)(716.606060606061,-10.030303030303)(709.141414141414,-10)(701.676767676768,-9.96969696969697)(686.747474747475,-9.90909090909091)(671.818181818182,-9.90909090909091)(656.888888888889,-9.90909090909091)(641.959595959596,-9.90909090909091)(627.030303030303,-9.90909090909091)(612.10101010101,-9.90909090909091)(597.171717171717,-9.90909090909091)(582.242424242424,-9.90909090909091)(567.313131313131,-9.90909090909091)(552.383838383838,-9.90909090909091)(537.454545454546,-9.90909090909091)(522.525252525253,-9.90909090909091)(507.59595959596,-9.96969696969697)(500.131313131313,-10)(492.666666666667,-10.030303030303)(477.737373737374,-10.0909090909091)(462.808080808081,-10.0909090909091)(447.878787878788,-10.0909090909091)(432.949494949495,-10.0909090909091)(418.020202020202,-10.0909090909091)(403.090909090909,-10.0909090909091)(388.161616161616,-10.030303030303)(380.69696969697,-10)(373.232323232323,-9.96969696969697)(358.30303030303,-9.90909090909091)(343.373737373737,-9.84848484848485)(339.641414141414,-9.81818181818182)(328.444444444444,-9.68181818181818)(325.956228956229,-9.63636363636364)(313.515151515152,-9.48484848484848)(311.026936026936,-9.45454545454546)(298.585858585859,-9.3030303030303)(291.121212121212,-9.27272727272727)(283.656565656566,-9.24242424242424)(268.727272727273,-9.18181818181818)(253.79797979798,-9.18181818181818)(238.868686868687,-9.18181818181818)(223.939393939394,-9.18181818181818)(209.010101010101,-9.18181818181818)(194.080808080808,-9.24242424242424)(186.616161616162,-9.27272727272727)(179.151515151515,-9.3030303030303)(164.222222222222,-9.36363636363636)(149.292929292929,-9.36363636363636)(134.363636363636,-9.42424242424243)(126.89898989899,-9.45454545454546)(119.434343434343,-9.48484848484848)(104.505050505051,-9.59090909090909)(97.040404040404,-9.63636363636364)(89.5757575757576,-9.68181818181818)(78.3787878787879,-9.81818181818182)(74.6464646464647,-9.84848484848485)(59.7171717171717,-9.96969696969697)(52.2525252525253,-10)(44.7878787878788,-10.030303030303)(29.8585858585859,-10.0909090909091)(14.9292929292929,-10.0909090909091)(0,-10.0909090909091)(-0.000124406627524661,-10)(-0.000373213662548569,-9.81818181818182)(-0.000124406627524661,-9.63636363636364)(0,-9.54545454545455)(14.9292929292929,-9.54545454545455)(29.8585858585859,-9.54545454545455)(44.7878787878788,-9.54545454545455)(59.7171717171717,-9.54545454545455)(74.6464646464647,-9.5)(82.1111111111111,-9.45454545454546)(89.5757575757576,-9.36363636363636)(92.0639730639731,-9.27272727272727)(97.040404040404,-9.09090909090909)(97.040404040404,-8.90909090909091)(97.040404040404,-8.72727272727273)(92.0639730639731,-8.54545454545455)(89.5757575757576,-8.5)(82.1111111111111,-8.36363636363636)(74.6464646464647,-8.27272727272727)(67.1818181818182,-8.18181818181818)(59.7171717171717,-8.09090909090909)(48.520202020202,-8)(44.7878787878788,-7.96969696969697)(29.8585858585859,-7.84848484848485)(26.1262626262626,-7.81818181818182)(14.9292929292929,-7.68181818181818)(12.4410774410774,-7.63636363636364)(14.9292929292929,-7.54545454545455)(17.4175084175084,-7.45454545454545)(29.8585858585859,-7.3030303030303)(37.3232323232323,-7.27272727272727)(44.7878787878788,-7.24242424242424)(59.7171717171717,-7.18181818181818)(74.6464646464647,-7.12121212121212)(82.1111111111111,-7.09090909090909)(89.5757575757576,-7.06060606060606)(104.505050505051,-7)(119.434343434343,-7)(134.363636363636,-6.93939393939394)(141.828282828283,-6.90909090909091)(149.292929292929,-6.87878787878788)(164.222222222222,-6.81818181818182)(179.151515151515,-6.81818181818182)(194.080808080808,-6.81818181818182)(209.010101010101,-6.81818181818182)(223.939393939394,-6.81818181818182)(238.868686868687,-6.81818181818182)(253.79797979798,-6.81818181818182)(268.727272727273,-6.81818181818182)(283.656565656566,-6.81818181818182)(298.585858585859,-6.81818181818182)(313.515151515152,-6.81818181818182)(328.444444444444,-6.81818181818182)(343.373737373737,-6.81818181818182)(358.30303030303,-6.81818181818182)(373.232323232323,-6.81818181818182)(388.161616161616,-6.81818181818182)(403.090909090909,-6.75757575757576)(410.555555555556,-6.72727272727273)(418.020202020202,-6.6969696969697)(432.949494949495,-6.57575757575758)(440.414141414141,-6.54545454545455)(447.878787878788,-6.51515151515152)(462.808080808081,-6.39393939393939)(470.272727272727,-6.36363636363636)(477.737373737374,-6.33333333333333)(492.666666666667,-6.27272727272727)(507.59595959596,-6.21212121212121)(515.060606060606,-6.18181818181818)(522.525252525253,-6.15151515151515)(537.454545454546,-6.09090909090909)(552.383838383838,-6.09090909090909)(567.313131313131,-6.09090909090909)(582.242424242424,-6.09090909090909)(597.171717171717,-6.09090909090909)(612.10101010101,-6.15151515151515)(615.833333333333,-6.18181818181818)};

\addplot [fill=red!40!yellow,draw=none,forget plot] coordinates{ (1127.16161616162,-7.81818181818182)(1134.62626262626,-7.84848484848485)(1149.55555555556,-7.90909090909091)(1164.48484848485,-7.90909090909091)(1179.41414141414,-7.90909090909091)(1194.34343434343,-7.90909090909091)(1209.27272727273,-7.96969696969697)(1216.73737373737,-8)(1224.20202020202,-8.03030303030303)(1239.13131313131,-8.09090909090909)(1254.06060606061,-8.09090909090909)(1268.9898989899,-8.15151515151515)(1276.45454545455,-8.18181818181818)(1283.91919191919,-8.21212121212121)(1298.84848484848,-8.33333333333333)(1306.31313131313,-8.36363636363636)(1313.77777777778,-8.39393939393939)(1328.70707070707,-8.51515151515152)(1332.43939393939,-8.54545454545455)(1343.63636363636,-8.68181818181818)(1346.12457912458,-8.72727272727273)(1351.10101010101,-8.90909090909091)(1346.12457912458,-9.09090909090909)(1343.63636363636,-9.13636363636364)(1336.17171717172,-9.27272727272727)(1328.70707070707,-9.36363636363636)(1317.5101010101,-9.45454545454546)(1313.77777777778,-9.48484848484848)(1298.84848484848,-9.60606060606061)(1291.38383838384,-9.63636363636364)(1283.91919191919,-9.66666666666667)(1268.9898989899,-9.72727272727273)(1254.06060606061,-9.72727272727273)(1239.13131313131,-9.66666666666667)(1224.20202020202,-9.72727272727273)(1209.27272727273,-9.78787878787879)(1201.80808080808,-9.81818181818182)(1194.34343434343,-9.84848484848485)(1179.41414141414,-9.84848484848485)(1164.48484848485,-9.84848484848485)(1149.55555555556,-9.95454545454546)(1134.62626262626,-9.95454545454546)(1119.69696969697,-9.95454545454546)(1104.76767676768,-9.90909090909091)(1089.83838383838,-9.90909090909091)(1074.90909090909,-9.90909090909091)(1059.9797979798,-9.90909090909091)(1045.05050505051,-9.90909090909091)(1030.12121212121,-9.90909090909091)(1015.19191919192,-9.90909090909091)(1000.26262626263,-9.90909090909091)(985.333333333333,-9.90909090909091)(970.40404040404,-9.88636363636364)(955.474747474747,-9.87272727272727)(940.545454545455,-9.83636363636364)(933.080808080808,-9.81818181818182)(925.616161616162,-9.78787878787879)(910.686868686869,-9.72727272727273)(895.757575757576,-9.78787878787879)(880.828282828283,-9.78787878787879)(865.89898989899,-9.72727272727273)(858.434343434343,-9.63636363636364)(850.969696969697,-9.54545454545455)(839.772727272727,-9.45454545454546)(836.040404040404,-9.42424242424243)(821.111111111111,-9.3030303030303)(817.378787878788,-9.27272727272727)(806.181818181818,-9.18181818181818)(798.717171717172,-9.09090909090909)(791.252525252525,-8.95454545454546)(788.76430976431,-8.90909090909091)(783.787878787879,-8.72727272727273)(788.76430976431,-8.54545454545455)(791.252525252525,-8.45454545454545)(793.740740740741,-8.36363636363636)(806.181818181818,-8.21212121212121)(808.670033670034,-8.18181818181818)(821.111111111111,-8.03030303030303)(828.575757575758,-8)(836.040404040404,-7.96969696969697)(850.969696969697,-7.90909090909091)(865.89898989899,-7.84848484848485)(873.363636363636,-7.81818181818182)(880.828282828283,-7.78787878787879)(895.757575757576,-7.72727272727273)(910.686868686869,-7.72727272727273)(925.616161616162,-7.72727272727273)(940.545454545455,-7.72727272727273)(955.474747474747,-7.72727272727273)(970.40404040404,-7.72727272727273)(985.333333333333,-7.72727272727273)(1000.26262626263,-7.72727272727273)(1015.19191919192,-7.72727272727273)(1030.12121212121,-7.72727272727273)(1045.05050505051,-7.72727272727273)(1059.9797979798,-7.72727272727273)(1074.90909090909,-7.72727272727273)(1089.83838383838,-7.72727272727273)(1104.76767676768,-7.72727272727273)(1119.69696969697,-7.78787878787879)(1127.16161616162,-7.81818181818182)};

\addplot [
color=white,
draw=white,
only marks,
mark=x,
mark options={solid},
mark size=2.0pt,
line width=0.3pt,
forget plot
]
coordinates{
 (731.535353535354,-1.63636363636364)(1478,-18)(0,-12.3636363636364)(1478,-8.90909090909091)(432.949494949495,-18)(0,-5.63636363636364)(1478,0)(0,0)(880.828282828283,-12.7272727272727)(731.535353535354,-7.27272727272727)(1478,-4.36363636363636)(1478,-13.6363636363636)(0,-16.5454545454545)(955.474747474747,-18)(268.727272727273,-9.27272727272727)(447.878787878788,-14.3636363636364)(388.161616161616,-3.81818181818182)(1000.26262626263,-4.36363636363636)(1030.12121212121,-9.81818181818182)(1045.05050505051,0)(432.949494949495,0)(1104.76767676768,-15.4545454545455)(0,-2.72727272727273)(552.383838383838,-11.0909090909091)(1209.27272727273,-6.90909090909091)(0,-8.54545454545455)(1478,-11.2727272727273)(1268.9898989899,-2.18181818181818)(358.30303030303,-6.54545454545455)(716.606060606061,-16)(507.59595959596,-8.36363636363636)(238.868686868687,-7.81818181818182)(507.59595959596,-7.45454545454545)(686.747474747475,-8.90909090909091)(821.111111111111,-9.09090909090909)(836.040404040404,-8.90909090909091)(850.969696969697,-9.45454545454546)(865.89898989899,-8.54545454545455)(865.89898989899,-9.63636363636364)(880.828282828283,-9.81818181818182)(895.757575757576,-8.36363636363636)(910.686868686869,-8.36363636363636)(940.545454545455,-10.1818181818182)(955.474747474747,-10.1818181818182)(1015.19191919192,-8.18181818181818)(1030.12121212121,-10.3636363636364)(1045.05050505051,-8.18181818181818)(1089.83838383838,-8.18181818181818)(1119.69696969697,-10.3636363636364)(1134.62626262626,-10.1818181818182)(1179.41414141414,-10)(1179.41414141414,-8.36363636363636)(1194.34343434343,-8.36363636363636)(1224.20202020202,-9.81818181818182)(1239.13131313131,-9.63636363636364)(1239.13131313131,-8.54545454545455)(1268.9898989899,-8.72727272727273)(1268.9898989899,-9.45454545454546)(1283.91919191919,-9.09090909090909)
};

\addplot [
color=white,
draw=black,
only marks,
mark=*,
mark options={solid},
mark size=2pt,
line width=0.2pt,
forget plot
]
coordinates{
 (1478,-7.45454545454545) 
};

\node at (axis cs:980, -17) [shape=circle,fill=red!40!yellow,draw=black,inner sep=0.2pt,anchor=south west,minimum size=16pt]
  {\scriptsize\color{white}$\*H_{\*t}$};
\node at (axis cs:1155, -17) [shape=circle,fill=locol,draw=black,inner sep=0.2pt,anchor=south west,minimum size=16pt]
  {\scriptsize\color{white}$\*L_{\*t}$};
\node at (axis cs:1330, -17) [shape=circle,fill=darkgray,draw=black,inner sep=0.2pt,anchor=south west,minimum size=16pt]
  {\scriptsize\color{white}$\*U_{\*t}$};

\end{axis}
\end{tikzpicture}%

\renewcommand\trimlen{0pt}
\begin{figure}[tb]
  \begin{subfigure}[b]{0.49\linewidth}
    \centering
    \adjincludegraphics[width=\linewidth,clip=true,trim=\trimlen{} \trimlen{} \trimlen{} \trimlen{}]{figures/ch03/limno_bgape_ppbatch60_0}
    \caption{After previous batch}
    \label{fig:limno_bgape_ppbatch1}
  \end{subfigure}
  \hfill
  \begin{subfigure}[b]{0.49\linewidth}
    \centering
    \adjincludegraphics[width=\linewidth,clip=true,trim=\trimlen{} \trimlen{} \trimlen{} \trimlen{}]{figures/ch03/limno_bgape_ppbatch60_1}
    \caption{Current batch}
    \label{fig:limno_bgape_ppbatch2}
  \end{subfigure}
  \begin{subfigure}[b]{0.49\linewidth}
    \centering
    \vspace{12pt} % space of this row from above captions
    \adjincludegraphics[width=\linewidth,clip=true,trim=\trimlen{} \trimlen{} \trimlen{} \trimlen{}]{figures/ch03/limno_bgape_ppbatch60_2}
    \caption{TSP path}
    \label{fig:limno_bgape_ppbatch3}
  \end{subfigure}
  \hfill
  \begin{subfigure}[b]{0.49\linewidth}
    \centering
    \adjincludegraphics[width=\linewidth,clip=true,trim=\trimlen{} \trimlen{} \trimlen{} \trimlen{}]{figures/ch03/limno_bgape_ppbatch60_3}
    \caption{After current batch}
    \label{fig:limno_bgape_ppbatch4}
  \end{subfigure}
  \caption{
      Path planning with \bacl using $\epsilon = 0.7$ on a regular grid of
      $100\times 100$ points sampled from the inferred algae concentration
      GP of \figref{fig:limno_bgape}.
      \textbf{(a)} The current position, i.e. the last point of the previous
                   batch shown as a white circle.
      \textbf{(b)} The current batch of $B=30$ points shown as red circles.
      \textbf{(c)} TSP path connecting the points of the current batch with
                   the current position using a TSP path.
      \textbf{(d)} The situation after sampling along the TSP path.
  }
  \label{fig:limno_bgape_ppbatch}
\end{figure}

\paragraph{Graph-based path planning}
There are a number of shortcomings when applying the above batch-based method
in practice. First, since selection of the batch is decoupled from its
evaluation, there is no way to adapt the path during sampling of the points
in the TSP path.\footnote{An alternative would be to each time sample the first
point in the TSP path and then replan using a new batch. However, this fails
because it does not give enough incentive to explore areas of the input space
that lie far away from the current position.} Second, additional samples can be obtained while traveling from one location
to another, but there is no easy way to incorporate this when planning using
batches.
Finally, as can be seen in the example of \figref{fig:limno_bgape_ppbatch}
the resulting TSP path may contain several abrupt changes of direction that
are not achievable in practice due to kinematic constraints of the mobile
sensor equipment.

To deal with the above issues, we propose a graph-based algorithm that works
on a more constrained setting, which is suitable for tasks such as the
environmental lake monitoring at hand. In particular, we define a graph with
nodes on a regular grid over the input space and create edges between
adjacent columns of the grid that encode the domain constraints. For example,
in the lake sensing application the robotic vehicle travels at about constant
horizontal speed of $0.7$m/s and the sensor actuator can achieve a maximum
vertical speed of
$0.1$m/s, therefore, if $dx$ and $dy$ are the horizontal and vertical
distances between two nodes, we place an edge between the nodes only if
$dx/dy \geq 7$ (see \figref{fig:graph}).
For simplicity we assume that the sensor can travel across the length
of the transect an indefinite amount of times, but is not allowed to
change its horizontal travel direction unless it reaches the end of the
transect.
Furthermore, we assume that during the traversal of an edge the sensor
obtains $n_e$ measurements equally spaced in time.

%\setlength\figureheight{1.5in}\setlength\figurewidth{5.2in}
%% This file was created by matlab2tikz v0.2.3.
% Copyright (c) 2008--2012, Nico Schlömer <nico.schloemer@gmail.com>
% All rights reserved.
% 
% 
% 
\begin{tikzpicture}

\begin{axis}[%
tick label style={font=\tiny},
label style={font=\tiny},
xlabel shift={-10pt},
ylabel shift={-17pt},
legend style={font=\tiny},
view={0}{90},
width=\figurewidth,
height=\figureheight,
scale only axis,
xmin=0, xmax=1478,
xtick={0, 400, 1000, 1400},
xlabel={Length (m)},
ymin=-18, ymax=0,
ytick={0, -4, -14, -18},
ylabel={Depth (m)},
name=plot1,
axis lines*=box,
axis line style={draw=none},
tickwidth=0.0cm,
clip=false
]

\addplot [
color=darkgray,
solid,
forget plot
]
coordinates{
 (77.7894736842105,0)(0,-2) 
};
\addplot [
color=darkgray,
solid,
forget plot
]
coordinates{
 (77.7894736842105,-2)(0,-2) 
};
\addplot [
color=darkgray,
solid,
forget plot
]
coordinates{
 (77.7894736842105,-4)(0,-2) 
};
\addplot [
color=darkgray,
solid,
forget plot
]
coordinates{
 (77.7894736842105,-6)(0,-2) 
};
\addplot [
color=darkgray,
solid,
forget plot
]
coordinates{
 (77.7894736842105,-8)(0,-2) 
};
\addplot [
color=darkgray,
solid,
forget plot
]
coordinates{
 (77.7894736842105,-10)(0,-2) 
};
\addplot [
color=darkgray,
solid,
forget plot
]
coordinates{
 (77.7894736842105,-12)(0,-2) 
};
\addplot [
color=darkgray,
solid,
forget plot
]
coordinates{
 (77.7894736842105,0)(0,-4) 
};
\addplot [
color=darkgray,
solid,
forget plot
]
coordinates{
 (77.7894736842105,-2)(0,-4) 
};
\addplot [
color=darkgray,
solid,
forget plot
]
coordinates{
 (77.7894736842105,-4)(0,-4) 
};
\addplot [
color=darkgray,
solid,
forget plot
]
coordinates{
 (77.7894736842105,-6)(0,-4) 
};
\addplot [
color=darkgray,
solid,
forget plot
]
coordinates{
 (77.7894736842105,-8)(0,-4) 
};
\addplot [
color=darkgray,
solid,
forget plot
]
coordinates{
 (77.7894736842105,-10)(0,-4) 
};
\addplot [
color=darkgray,
solid,
forget plot
]
coordinates{
 (77.7894736842105,-12)(0,-4) 
};
\addplot [
color=darkgray,
solid,
forget plot
]
coordinates{
 (77.7894736842105,-14)(0,-4) 
};
\addplot [
color=darkgray,
solid,
forget plot
]
coordinates{
 (77.7894736842105,0)(0,-6) 
};
\addplot [
color=darkgray,
solid,
forget plot
]
coordinates{
 (77.7894736842105,-2)(0,-6) 
};
\addplot [
color=darkgray,
solid,
forget plot
]
coordinates{
 (77.7894736842105,-4)(0,-6) 
};
\addplot [
color=darkgray,
solid,
forget plot
]
coordinates{
 (77.7894736842105,-6)(0,-6) 
};
\addplot [
color=darkgray,
solid,
forget plot
]
coordinates{
 (77.7894736842105,-8)(0,-6) 
};
\addplot [
color=darkgray,
solid,
forget plot
]
coordinates{
 (77.7894736842105,-10)(0,-6) 
};
\addplot [
color=darkgray,
solid,
forget plot
]
coordinates{
 (77.7894736842105,-12)(0,-6) 
};
\addplot [
color=darkgray,
solid,
forget plot
]
coordinates{
 (77.7894736842105,-14)(0,-6) 
};
\addplot [
color=darkgray,
solid,
forget plot
]
coordinates{
 (77.7894736842105,-16)(0,-6) 
};
\addplot [
color=darkgray,
solid,
forget plot
]
coordinates{
 (77.7894736842105,0)(0,-8) 
};
\addplot [
color=darkgray,
solid,
forget plot
]
coordinates{
 (77.7894736842105,-2)(0,-8) 
};
\addplot [
color=darkgray,
solid,
forget plot
]
coordinates{
 (77.7894736842105,-4)(0,-8) 
};
\addplot [
color=darkgray,
solid,
forget plot
]
coordinates{
 (77.7894736842105,-6)(0,-8) 
};
\addplot [
color=darkgray,
solid,
forget plot
]
coordinates{
 (77.7894736842105,-8)(0,-8) 
};
\addplot [
color=darkgray,
solid,
forget plot
]
coordinates{
 (77.7894736842105,-10)(0,-8) 
};
\addplot [
color=darkgray,
solid,
forget plot
]
coordinates{
 (77.7894736842105,-12)(0,-8) 
};
\addplot [
color=darkgray,
solid,
forget plot
]
coordinates{
 (77.7894736842105,-14)(0,-8) 
};
\addplot [
color=darkgray,
solid,
forget plot
]
coordinates{
 (77.7894736842105,-16)(0,-8) 
};
\addplot [
color=darkgray,
solid,
forget plot
]
coordinates{
 (77.7894736842105,-18)(0,-8) 
};
\addplot [
color=darkgray,
solid,
forget plot
]
coordinates{
 (77.7894736842105,0)(0,-10) 
};
\addplot [
color=darkgray,
solid,
forget plot
]
coordinates{
 (77.7894736842105,-2)(0,-10) 
};
\addplot [
color=darkgray,
solid,
forget plot
]
coordinates{
 (77.7894736842105,-4)(0,-10) 
};
\addplot [
color=darkgray,
solid,
forget plot
]
coordinates{
 (77.7894736842105,-6)(0,-10) 
};
\addplot [
color=darkgray,
solid,
forget plot
]
coordinates{
 (77.7894736842105,-8)(0,-10) 
};
\addplot [
color=darkgray,
solid,
forget plot
]
coordinates{
 (77.7894736842105,-10)(0,-10) 
};
\addplot [
color=darkgray,
solid,
forget plot
]
coordinates{
 (77.7894736842105,-12)(0,-10) 
};
\addplot [
color=darkgray,
solid,
forget plot
]
coordinates{
 (77.7894736842105,-14)(0,-10) 
};
\addplot [
color=darkgray,
solid,
forget plot
]
coordinates{
 (77.7894736842105,-16)(0,-10) 
};
\addplot [
color=darkgray,
solid,
forget plot
]
coordinates{
 (77.7894736842105,-18)(0,-10) 
};
\addplot [
color=darkgray,
solid,
forget plot
]
coordinates{
 (77.7894736842105,-2)(0,-12) 
};
\addplot [
color=darkgray,
solid,
forget plot
]
coordinates{
 (77.7894736842105,-4)(0,-12) 
};
\addplot [
color=darkgray,
solid,
forget plot
]
coordinates{
 (77.7894736842105,-6)(0,-12) 
};
\addplot [
color=darkgray,
solid,
forget plot
]
coordinates{
 (77.7894736842105,-8)(0,-12) 
};
\addplot [
color=darkgray,
solid,
forget plot
]
coordinates{
 (77.7894736842105,-10)(0,-12) 
};
\addplot [
color=darkgray,
solid,
forget plot
]
coordinates{
 (77.7894736842105,-12)(0,-12) 
};
\addplot [
color=darkgray,
solid,
forget plot
]
coordinates{
 (77.7894736842105,-14)(0,-12) 
};
\addplot [
color=darkgray,
solid,
forget plot
]
coordinates{
 (77.7894736842105,-16)(0,-12) 
};
\addplot [
color=darkgray,
solid,
forget plot
]
coordinates{
 (77.7894736842105,-18)(0,-12) 
};
\addplot [
color=darkgray,
solid,
forget plot
]
coordinates{
 (77.7894736842105,-4)(0,-14) 
};
\addplot [
color=darkgray,
solid,
forget plot
]
coordinates{
 (77.7894736842105,-6)(0,-14) 
};
\addplot [
color=darkgray,
solid,
forget plot
]
coordinates{
 (77.7894736842105,-8)(0,-14) 
};
\addplot [
color=darkgray,
solid,
forget plot
]
coordinates{
 (77.7894736842105,-10)(0,-14) 
};
\addplot [
color=darkgray,
solid,
forget plot
]
coordinates{
 (77.7894736842105,-12)(0,-14) 
};
\addplot [
color=darkgray,
solid,
forget plot
]
coordinates{
 (77.7894736842105,-14)(0,-14) 
};
\addplot [
color=darkgray,
solid,
forget plot
]
coordinates{
 (77.7894736842105,-16)(0,-14) 
};
\addplot [
color=darkgray,
solid,
forget plot
]
coordinates{
 (77.7894736842105,-18)(0,-14) 
};
\addplot [
color=darkgray,
solid,
forget plot
]
coordinates{
 (77.7894736842105,-6)(0,-16) 
};
\addplot [
color=darkgray,
solid,
forget plot
]
coordinates{
 (77.7894736842105,-8)(0,-16) 
};
\addplot [
color=darkgray,
solid,
forget plot
]
coordinates{
 (77.7894736842105,-10)(0,-16) 
};
\addplot [
color=darkgray,
solid,
forget plot
]
coordinates{
 (77.7894736842105,-12)(0,-16) 
};
\addplot [
color=darkgray,
solid,
forget plot
]
coordinates{
 (77.7894736842105,-14)(0,-16) 
};
\addplot [
color=darkgray,
solid,
forget plot
]
coordinates{
 (77.7894736842105,-16)(0,-16) 
};
\addplot [
color=darkgray,
solid,
forget plot
]
coordinates{
 (77.7894736842105,-18)(0,-16) 
};
\addplot [
color=darkgray,
solid,
forget plot
]
coordinates{
 (77.7894736842105,-8)(0,-18) 
};
\addplot [
color=darkgray,
solid,
forget plot
]
coordinates{
 (77.7894736842105,-10)(0,-18) 
};
\addplot [
color=darkgray,
solid,
forget plot
]
coordinates{
 (77.7894736842105,-12)(0,-18) 
};
\addplot [
color=darkgray,
solid,
forget plot
]
coordinates{
 (77.7894736842105,-14)(0,-18) 
};
\addplot [
color=darkgray,
solid,
forget plot
]
coordinates{
 (77.7894736842105,-16)(0,-18) 
};
\addplot [
color=darkgray,
solid,
forget plot
]
coordinates{
 (77.7894736842105,-18)(0,-18) 
};
\addplot [
color=darkgray,
solid,
forget plot
]
coordinates{
 (0,0)(77.7894736842105,0) 
};
\addplot [
color=darkgray,
solid,
forget plot
]
coordinates{
 (0,-2)(77.7894736842105,0) 
};
\addplot [
color=darkgray,
solid,
forget plot
]
coordinates{
 (0,-4)(77.7894736842105,0) 
};
\addplot [
color=darkgray,
solid,
forget plot
]
coordinates{
 (0,-6)(77.7894736842105,0) 
};
\addplot [
color=darkgray,
solid,
forget plot
]
coordinates{
 (0,-8)(77.7894736842105,0) 
};
\addplot [
color=darkgray,
solid,
forget plot
]
coordinates{
 (0,-10)(77.7894736842105,0) 
};
\addplot [
color=darkgray,
solid,
forget plot
]
coordinates{
 (0,0)(77.7894736842105,-2) 
};
\addplot [
color=darkgray,
solid,
forget plot
]
coordinates{
 (0,-2)(77.7894736842105,-2) 
};
\addplot [
color=darkgray,
solid,
forget plot
]
coordinates{
 (0,-4)(77.7894736842105,-2) 
};
\addplot [
color=darkgray,
solid,
forget plot
]
coordinates{
 (0,-6)(77.7894736842105,-2) 
};
\addplot [
color=darkgray,
solid,
forget plot
]
coordinates{
 (0,-8)(77.7894736842105,-2) 
};
\addplot [
color=darkgray,
solid,
forget plot
]
coordinates{
 (0,-10)(77.7894736842105,-2) 
};
\addplot [
color=darkgray,
solid,
forget plot
]
coordinates{
 (0,-12)(77.7894736842105,-2) 
};
\addplot [
color=darkgray,
solid,
forget plot
]
coordinates{
 (0,0)(77.7894736842105,-4) 
};
\addplot [
color=darkgray,
solid,
forget plot
]
coordinates{
 (0,-2)(77.7894736842105,-4) 
};
\addplot [
color=darkgray,
solid,
forget plot
]
coordinates{
 (0,-4)(77.7894736842105,-4) 
};
\addplot [
color=darkgray,
solid,
forget plot
]
coordinates{
 (0,-6)(77.7894736842105,-4) 
};
\addplot [
color=darkgray,
solid,
forget plot
]
coordinates{
 (0,-8)(77.7894736842105,-4) 
};
\addplot [
color=darkgray,
solid,
forget plot
]
coordinates{
 (0,-10)(77.7894736842105,-4) 
};
\addplot [
color=darkgray,
solid,
forget plot
]
coordinates{
 (0,-12)(77.7894736842105,-4) 
};
\addplot [
color=darkgray,
solid,
forget plot
]
coordinates{
 (0,-14)(77.7894736842105,-4) 
};
\addplot [
color=darkgray,
solid,
forget plot
]
coordinates{
 (0,0)(77.7894736842105,-6) 
};
\addplot [
color=darkgray,
solid,
forget plot
]
coordinates{
 (0,-2)(77.7894736842105,-6) 
};
\addplot [
color=darkgray,
solid,
forget plot
]
coordinates{
 (0,-4)(77.7894736842105,-6) 
};
\addplot [
color=darkgray,
solid,
forget plot
]
coordinates{
 (0,-6)(77.7894736842105,-6) 
};
\addplot [
color=darkgray,
solid,
forget plot
]
coordinates{
 (0,-8)(77.7894736842105,-6) 
};
\addplot [
color=darkgray,
solid,
forget plot
]
coordinates{
 (0,-10)(77.7894736842105,-6) 
};
\addplot [
color=darkgray,
solid,
forget plot
]
coordinates{
 (0,-12)(77.7894736842105,-6) 
};
\addplot [
color=darkgray,
solid,
forget plot
]
coordinates{
 (0,-14)(77.7894736842105,-6) 
};
\addplot [
color=darkgray,
solid,
forget plot
]
coordinates{
 (0,-16)(77.7894736842105,-6) 
};
\addplot [
color=darkgray,
solid,
forget plot
]
coordinates{
 (0,0)(77.7894736842105,-8) 
};
\addplot [
color=darkgray,
solid,
forget plot
]
coordinates{
 (0,-2)(77.7894736842105,-8) 
};
\addplot [
color=darkgray,
solid,
forget plot
]
coordinates{
 (0,-4)(77.7894736842105,-8) 
};
\addplot [
color=darkgray,
solid,
forget plot
]
coordinates{
 (0,-6)(77.7894736842105,-8) 
};
\addplot [
color=darkgray,
solid,
forget plot
]
coordinates{
 (0,-8)(77.7894736842105,-8) 
};
\addplot [
color=darkgray,
solid,
forget plot
]
coordinates{
 (0,-10)(77.7894736842105,-8) 
};
\addplot [
color=darkgray,
solid,
forget plot
]
coordinates{
 (0,-12)(77.7894736842105,-8) 
};
\addplot [
color=darkgray,
solid,
forget plot
]
coordinates{
 (0,-14)(77.7894736842105,-8) 
};
\addplot [
color=darkgray,
solid,
forget plot
]
coordinates{
 (0,-16)(77.7894736842105,-8) 
};
\addplot [
color=darkgray,
solid,
forget plot
]
coordinates{
 (0,-18)(77.7894736842105,-8) 
};
\addplot [
color=darkgray,
solid,
forget plot
]
coordinates{
 (0,0)(77.7894736842105,-10) 
};
\addplot [
color=darkgray,
solid,
forget plot
]
coordinates{
 (0,-2)(77.7894736842105,-10) 
};
\addplot [
color=darkgray,
solid,
forget plot
]
coordinates{
 (0,-4)(77.7894736842105,-10) 
};
\addplot [
color=darkgray,
solid,
forget plot
]
coordinates{
 (0,-6)(77.7894736842105,-10) 
};
\addplot [
color=darkgray,
solid,
forget plot
]
coordinates{
 (0,-8)(77.7894736842105,-10) 
};
\addplot [
color=darkgray,
solid,
forget plot
]
coordinates{
 (0,-10)(77.7894736842105,-10) 
};
\addplot [
color=darkgray,
solid,
forget plot
]
coordinates{
 (0,-12)(77.7894736842105,-10) 
};
\addplot [
color=darkgray,
solid,
forget plot
]
coordinates{
 (0,-14)(77.7894736842105,-10) 
};
\addplot [
color=darkgray,
solid,
forget plot
]
coordinates{
 (0,-16)(77.7894736842105,-10) 
};
\addplot [
color=darkgray,
solid,
forget plot
]
coordinates{
 (0,-18)(77.7894736842105,-10) 
};
\addplot [
color=darkgray,
solid,
forget plot
]
coordinates{
 (0,-2)(77.7894736842105,-12) 
};
\addplot [
color=darkgray,
solid,
forget plot
]
coordinates{
 (0,-4)(77.7894736842105,-12) 
};
\addplot [
color=darkgray,
solid,
forget plot
]
coordinates{
 (0,-6)(77.7894736842105,-12) 
};
\addplot [
color=darkgray,
solid,
forget plot
]
coordinates{
 (0,-8)(77.7894736842105,-12) 
};
\addplot [
color=darkgray,
solid,
forget plot
]
coordinates{
 (0,-10)(77.7894736842105,-12) 
};
\addplot [
color=darkgray,
solid,
forget plot
]
coordinates{
 (0,-12)(77.7894736842105,-12) 
};
\addplot [
color=darkgray,
solid,
forget plot
]
coordinates{
 (0,-14)(77.7894736842105,-12) 
};
\addplot [
color=darkgray,
solid,
forget plot
]
coordinates{
 (0,-16)(77.7894736842105,-12) 
};
\addplot [
color=darkgray,
solid,
forget plot
]
coordinates{
 (0,-18)(77.7894736842105,-12) 
};
\addplot [
color=darkgray,
solid,
forget plot
]
coordinates{
 (0,-4)(77.7894736842105,-14) 
};
\addplot [
color=darkgray,
solid,
forget plot
]
coordinates{
 (0,-6)(77.7894736842105,-14) 
};
\addplot [
color=darkgray,
solid,
forget plot
]
coordinates{
 (0,-8)(77.7894736842105,-14) 
};
\addplot [
color=darkgray,
solid,
forget plot
]
coordinates{
 (0,-10)(77.7894736842105,-14) 
};
\addplot [
color=darkgray,
solid,
forget plot
]
coordinates{
 (0,-12)(77.7894736842105,-14) 
};
\addplot [
color=darkgray,
solid,
forget plot
]
coordinates{
 (0,-14)(77.7894736842105,-14) 
};
\addplot [
color=darkgray,
solid,
forget plot
]
coordinates{
 (0,-16)(77.7894736842105,-14) 
};
\addplot [
color=darkgray,
solid,
forget plot
]
coordinates{
 (0,-18)(77.7894736842105,-14) 
};
\addplot [
color=darkgray,
solid,
forget plot
]
coordinates{
 (0,-6)(77.7894736842105,-16) 
};
\addplot [
color=darkgray,
solid,
forget plot
]
coordinates{
 (0,-8)(77.7894736842105,-16) 
};
\addplot [
color=darkgray,
solid,
forget plot
]
coordinates{
 (0,-10)(77.7894736842105,-16) 
};
\addplot [
color=darkgray,
solid,
forget plot
]
coordinates{
 (0,-12)(77.7894736842105,-16) 
};
\addplot [
color=darkgray,
solid,
forget plot
]
coordinates{
 (0,-14)(77.7894736842105,-16) 
};
\addplot [
color=darkgray,
solid,
forget plot
]
coordinates{
 (0,-16)(77.7894736842105,-16) 
};
\addplot [
color=darkgray,
solid,
forget plot
]
coordinates{
 (0,-18)(77.7894736842105,-16) 
};
\addplot [
color=darkgray,
solid,
forget plot
]
coordinates{
 (0,-8)(77.7894736842105,-18) 
};
\addplot [
color=darkgray,
solid,
forget plot
]
coordinates{
 (0,-10)(77.7894736842105,-18) 
};
\addplot [
color=darkgray,
solid,
forget plot
]
coordinates{
 (0,-12)(77.7894736842105,-18) 
};
\addplot [
color=darkgray,
solid,
forget plot
]
coordinates{
 (0,-14)(77.7894736842105,-18) 
};
\addplot [
color=darkgray,
solid,
forget plot
]
coordinates{
 (0,-16)(77.7894736842105,-18) 
};
\addplot [
color=darkgray,
solid,
forget plot
]
coordinates{
 (0,-18)(77.7894736842105,-18) 
};
\addplot [
color=darkgray,
solid,
forget plot
]
coordinates{
 (77.7894736842105,0)(155.578947368421,0) 
};
\addplot [
color=darkgray,
solid,
forget plot
]
coordinates{
 (77.7894736842105,-2)(155.578947368421,0) 
};
\addplot [
color=darkgray,
solid,
forget plot
]
coordinates{
 (77.7894736842105,-4)(155.578947368421,0) 
};
\addplot [
color=darkgray,
solid,
forget plot
]
coordinates{
 (77.7894736842105,-6)(155.578947368421,0) 
};
\addplot [
color=darkgray,
solid,
forget plot
]
coordinates{
 (77.7894736842105,-8)(155.578947368421,0) 
};
\addplot [
color=darkgray,
solid,
forget plot
]
coordinates{
 (77.7894736842105,-10)(155.578947368421,0) 
};
\addplot [
color=darkgray,
solid,
forget plot
]
coordinates{
 (77.7894736842105,0)(155.578947368421,-2) 
};
\addplot [
color=darkgray,
solid,
forget plot
]
coordinates{
 (77.7894736842105,-2)(155.578947368421,-2) 
};
\addplot [
color=darkgray,
solid,
forget plot
]
coordinates{
 (77.7894736842105,-4)(155.578947368421,-2) 
};
\addplot [
color=darkgray,
solid,
forget plot
]
coordinates{
 (77.7894736842105,-6)(155.578947368421,-2) 
};
\addplot [
color=darkgray,
solid,
forget plot
]
coordinates{
 (77.7894736842105,-8)(155.578947368421,-2) 
};
\addplot [
color=darkgray,
solid,
forget plot
]
coordinates{
 (77.7894736842105,-10)(155.578947368421,-2) 
};
\addplot [
color=darkgray,
solid,
forget plot
]
coordinates{
 (77.7894736842105,-12)(155.578947368421,-2) 
};
\addplot [
color=darkgray,
solid,
forget plot
]
coordinates{
 (77.7894736842105,0)(155.578947368421,-4) 
};
\addplot [
color=darkgray,
solid,
forget plot
]
coordinates{
 (77.7894736842105,-2)(155.578947368421,-4) 
};
\addplot [
color=darkgray,
solid,
forget plot
]
coordinates{
 (77.7894736842105,-4)(155.578947368421,-4) 
};
\addplot [
color=darkgray,
solid,
forget plot
]
coordinates{
 (77.7894736842105,-6)(155.578947368421,-4) 
};
\addplot [
color=darkgray,
solid,
forget plot
]
coordinates{
 (77.7894736842105,-8)(155.578947368421,-4) 
};
\addplot [
color=darkgray,
solid,
forget plot
]
coordinates{
 (77.7894736842105,-10)(155.578947368421,-4) 
};
\addplot [
color=darkgray,
solid,
forget plot
]
coordinates{
 (77.7894736842105,-12)(155.578947368421,-4) 
};
\addplot [
color=darkgray,
solid,
forget plot
]
coordinates{
 (77.7894736842105,-14)(155.578947368421,-4) 
};
\addplot [
color=darkgray,
solid,
forget plot
]
coordinates{
 (77.7894736842105,0)(155.578947368421,-6) 
};
\addplot [
color=darkgray,
solid,
forget plot
]
coordinates{
 (77.7894736842105,-2)(155.578947368421,-6) 
};
\addplot [
color=darkgray,
solid,
forget plot
]
coordinates{
 (77.7894736842105,-4)(155.578947368421,-6) 
};
\addplot [
color=darkgray,
solid,
forget plot
]
coordinates{
 (77.7894736842105,-6)(155.578947368421,-6) 
};
\addplot [
color=darkgray,
solid,
forget plot
]
coordinates{
 (77.7894736842105,-8)(155.578947368421,-6) 
};
\addplot [
color=darkgray,
solid,
forget plot
]
coordinates{
 (77.7894736842105,-10)(155.578947368421,-6) 
};
\addplot [
color=darkgray,
solid,
forget plot
]
coordinates{
 (77.7894736842105,-12)(155.578947368421,-6) 
};
\addplot [
color=darkgray,
solid,
forget plot
]
coordinates{
 (77.7894736842105,-14)(155.578947368421,-6) 
};
\addplot [
color=darkgray,
solid,
forget plot
]
coordinates{
 (77.7894736842105,-16)(155.578947368421,-6) 
};
\addplot [
color=darkgray,
solid,
forget plot
]
coordinates{
 (77.7894736842105,0)(155.578947368421,-8) 
};
\addplot [
color=darkgray,
solid,
forget plot
]
coordinates{
 (77.7894736842105,-2)(155.578947368421,-8) 
};
\addplot [
color=darkgray,
solid,
forget plot
]
coordinates{
 (77.7894736842105,-4)(155.578947368421,-8) 
};
\addplot [
color=darkgray,
solid,
forget plot
]
coordinates{
 (77.7894736842105,-6)(155.578947368421,-8) 
};
\addplot [
color=darkgray,
solid,
forget plot
]
coordinates{
 (77.7894736842105,-8)(155.578947368421,-8) 
};
\addplot [
color=darkgray,
solid,
forget plot
]
coordinates{
 (77.7894736842105,-10)(155.578947368421,-8) 
};
\addplot [
color=darkgray,
solid,
forget plot
]
coordinates{
 (77.7894736842105,-12)(155.578947368421,-8) 
};
\addplot [
color=darkgray,
solid,
forget plot
]
coordinates{
 (77.7894736842105,-14)(155.578947368421,-8) 
};
\addplot [
color=darkgray,
solid,
forget plot
]
coordinates{
 (77.7894736842105,-16)(155.578947368421,-8) 
};
\addplot [
color=darkgray,
solid,
forget plot
]
coordinates{
 (77.7894736842105,-18)(155.578947368421,-8) 
};
\addplot [
color=darkgray,
solid,
forget plot
]
coordinates{
 (77.7894736842105,0)(155.578947368421,-10) 
};
\addplot [
color=darkgray,
solid,
forget plot
]
coordinates{
 (77.7894736842105,-2)(155.578947368421,-10) 
};
\addplot [
color=darkgray,
solid,
forget plot
]
coordinates{
 (77.7894736842105,-4)(155.578947368421,-10) 
};
\addplot [
color=darkgray,
solid,
forget plot
]
coordinates{
 (77.7894736842105,-6)(155.578947368421,-10) 
};
\addplot [
color=darkgray,
solid,
forget plot
]
coordinates{
 (77.7894736842105,-8)(155.578947368421,-10) 
};
\addplot [
color=darkgray,
solid,
forget plot
]
coordinates{
 (77.7894736842105,-10)(155.578947368421,-10) 
};
\addplot [
color=darkgray,
solid,
forget plot
]
coordinates{
 (77.7894736842105,-12)(155.578947368421,-10) 
};
\addplot [
color=darkgray,
solid,
forget plot
]
coordinates{
 (77.7894736842105,-14)(155.578947368421,-10) 
};
\addplot [
color=darkgray,
solid,
forget plot
]
coordinates{
 (77.7894736842105,-16)(155.578947368421,-10) 
};
\addplot [
color=darkgray,
solid,
forget plot
]
coordinates{
 (77.7894736842105,-18)(155.578947368421,-10) 
};
\addplot [
color=darkgray,
solid,
forget plot
]
coordinates{
 (77.7894736842105,-2)(155.578947368421,-12) 
};
\addplot [
color=darkgray,
solid,
forget plot
]
coordinates{
 (77.7894736842105,-4)(155.578947368421,-12) 
};
\addplot [
color=darkgray,
solid,
forget plot
]
coordinates{
 (77.7894736842105,-6)(155.578947368421,-12) 
};
\addplot [
color=darkgray,
solid,
forget plot
]
coordinates{
 (77.7894736842105,-8)(155.578947368421,-12) 
};
\addplot [
color=darkgray,
solid,
forget plot
]
coordinates{
 (77.7894736842105,-10)(155.578947368421,-12) 
};
\addplot [
color=darkgray,
solid,
forget plot
]
coordinates{
 (77.7894736842105,-12)(155.578947368421,-12) 
};
\addplot [
color=darkgray,
solid,
forget plot
]
coordinates{
 (77.7894736842105,-14)(155.578947368421,-12) 
};
\addplot [
color=darkgray,
solid,
forget plot
]
coordinates{
 (77.7894736842105,-16)(155.578947368421,-12) 
};
\addplot [
color=darkgray,
solid,
forget plot
]
coordinates{
 (77.7894736842105,-18)(155.578947368421,-12) 
};
\addplot [
color=darkgray,
solid,
forget plot
]
coordinates{
 (77.7894736842105,-4)(155.578947368421,-14) 
};
\addplot [
color=darkgray,
solid,
forget plot
]
coordinates{
 (77.7894736842105,-6)(155.578947368421,-14) 
};
\addplot [
color=darkgray,
solid,
forget plot
]
coordinates{
 (77.7894736842105,-8)(155.578947368421,-14) 
};
\addplot [
color=darkgray,
solid,
forget plot
]
coordinates{
 (77.7894736842105,-10)(155.578947368421,-14) 
};
\addplot [
color=darkgray,
solid,
forget plot
]
coordinates{
 (77.7894736842105,-12)(155.578947368421,-14) 
};
\addplot [
color=darkgray,
solid,
forget plot
]
coordinates{
 (77.7894736842105,-14)(155.578947368421,-14) 
};
\addplot [
color=darkgray,
solid,
forget plot
]
coordinates{
 (77.7894736842105,-16)(155.578947368421,-14) 
};
\addplot [
color=darkgray,
solid,
forget plot
]
coordinates{
 (77.7894736842105,-18)(155.578947368421,-14) 
};
\addplot [
color=darkgray,
solid,
forget plot
]
coordinates{
 (77.7894736842105,-6)(155.578947368421,-16) 
};
\addplot [
color=darkgray,
solid,
forget plot
]
coordinates{
 (77.7894736842105,-8)(155.578947368421,-16) 
};
\addplot [
color=darkgray,
solid,
forget plot
]
coordinates{
 (77.7894736842105,-10)(155.578947368421,-16) 
};
\addplot [
color=darkgray,
solid,
forget plot
]
coordinates{
 (77.7894736842105,-12)(155.578947368421,-16) 
};
\addplot [
color=darkgray,
solid,
forget plot
]
coordinates{
 (77.7894736842105,-14)(155.578947368421,-16) 
};
\addplot [
color=darkgray,
solid,
forget plot
]
coordinates{
 (77.7894736842105,-16)(155.578947368421,-16) 
};
\addplot [
color=darkgray,
solid,
forget plot
]
coordinates{
 (77.7894736842105,-18)(155.578947368421,-16) 
};
\addplot [
color=darkgray,
solid,
forget plot
]
coordinates{
 (77.7894736842105,-8)(155.578947368421,-18) 
};
\addplot [
color=darkgray,
solid,
forget plot
]
coordinates{
 (77.7894736842105,-10)(155.578947368421,-18) 
};
\addplot [
color=darkgray,
solid,
forget plot
]
coordinates{
 (77.7894736842105,-12)(155.578947368421,-18) 
};
\addplot [
color=darkgray,
solid,
forget plot
]
coordinates{
 (77.7894736842105,-14)(155.578947368421,-18) 
};
\addplot [
color=darkgray,
solid,
forget plot
]
coordinates{
 (77.7894736842105,-16)(155.578947368421,-18) 
};
\addplot [
color=darkgray,
solid,
forget plot
]
coordinates{
 (77.7894736842105,-18)(155.578947368421,-18) 
};
\addplot [
color=darkgray,
solid,
forget plot
]
coordinates{
 (155.578947368421,0)(233.368421052632,0) 
};
\addplot [
color=darkgray,
solid,
forget plot
]
coordinates{
 (155.578947368421,-2)(233.368421052632,0) 
};
\addplot [
color=darkgray,
solid,
forget plot
]
coordinates{
 (155.578947368421,-4)(233.368421052632,0) 
};
\addplot [
color=darkgray,
solid,
forget plot
]
coordinates{
 (155.578947368421,-6)(233.368421052632,0) 
};
\addplot [
color=darkgray,
solid,
forget plot
]
coordinates{
 (155.578947368421,-8)(233.368421052632,0) 
};
\addplot [
color=darkgray,
solid,
forget plot
]
coordinates{
 (155.578947368421,-10)(233.368421052632,0) 
};
\addplot [
color=darkgray,
solid,
forget plot
]
coordinates{
 (155.578947368421,0)(233.368421052632,-2) 
};
\addplot [
color=darkgray,
solid,
forget plot
]
coordinates{
 (155.578947368421,-2)(233.368421052632,-2) 
};
\addplot [
color=darkgray,
solid,
forget plot
]
coordinates{
 (155.578947368421,-4)(233.368421052632,-2) 
};
\addplot [
color=darkgray,
solid,
forget plot
]
coordinates{
 (155.578947368421,-6)(233.368421052632,-2) 
};
\addplot [
color=darkgray,
solid,
forget plot
]
coordinates{
 (155.578947368421,-8)(233.368421052632,-2) 
};
\addplot [
color=darkgray,
solid,
forget plot
]
coordinates{
 (155.578947368421,-10)(233.368421052632,-2) 
};
\addplot [
color=darkgray,
solid,
forget plot
]
coordinates{
 (155.578947368421,-12)(233.368421052632,-2) 
};
\addplot [
color=darkgray,
solid,
forget plot
]
coordinates{
 (155.578947368421,0)(233.368421052632,-4) 
};
\addplot [
color=darkgray,
solid,
forget plot
]
coordinates{
 (155.578947368421,-2)(233.368421052632,-4) 
};
\addplot [
color=darkgray,
solid,
forget plot
]
coordinates{
 (155.578947368421,-4)(233.368421052632,-4) 
};
\addplot [
color=darkgray,
solid,
forget plot
]
coordinates{
 (155.578947368421,-6)(233.368421052632,-4) 
};
\addplot [
color=darkgray,
solid,
forget plot
]
coordinates{
 (155.578947368421,-8)(233.368421052632,-4) 
};
\addplot [
color=darkgray,
solid,
forget plot
]
coordinates{
 (155.578947368421,-10)(233.368421052632,-4) 
};
\addplot [
color=darkgray,
solid,
forget plot
]
coordinates{
 (155.578947368421,-12)(233.368421052632,-4) 
};
\addplot [
color=darkgray,
solid,
forget plot
]
coordinates{
 (155.578947368421,-14)(233.368421052632,-4) 
};
\addplot [
color=darkgray,
solid,
forget plot
]
coordinates{
 (155.578947368421,0)(233.368421052632,-6) 
};
\addplot [
color=darkgray,
solid,
forget plot
]
coordinates{
 (155.578947368421,-2)(233.368421052632,-6) 
};
\addplot [
color=darkgray,
solid,
forget plot
]
coordinates{
 (155.578947368421,-4)(233.368421052632,-6) 
};
\addplot [
color=darkgray,
solid,
forget plot
]
coordinates{
 (155.578947368421,-6)(233.368421052632,-6) 
};
\addplot [
color=darkgray,
solid,
forget plot
]
coordinates{
 (155.578947368421,-8)(233.368421052632,-6) 
};
\addplot [
color=darkgray,
solid,
forget plot
]
coordinates{
 (155.578947368421,-10)(233.368421052632,-6) 
};
\addplot [
color=darkgray,
solid,
forget plot
]
coordinates{
 (155.578947368421,-12)(233.368421052632,-6) 
};
\addplot [
color=darkgray,
solid,
forget plot
]
coordinates{
 (155.578947368421,-14)(233.368421052632,-6) 
};
\addplot [
color=darkgray,
solid,
forget plot
]
coordinates{
 (155.578947368421,-16)(233.368421052632,-6) 
};
\addplot [
color=darkgray,
solid,
forget plot
]
coordinates{
 (155.578947368421,0)(233.368421052632,-8) 
};
\addplot [
color=darkgray,
solid,
forget plot
]
coordinates{
 (155.578947368421,-2)(233.368421052632,-8) 
};
\addplot [
color=darkgray,
solid,
forget plot
]
coordinates{
 (155.578947368421,-4)(233.368421052632,-8) 
};
\addplot [
color=darkgray,
solid,
forget plot
]
coordinates{
 (155.578947368421,-6)(233.368421052632,-8) 
};
\addplot [
color=darkgray,
solid,
forget plot
]
coordinates{
 (155.578947368421,-8)(233.368421052632,-8) 
};
\addplot [
color=darkgray,
solid,
forget plot
]
coordinates{
 (155.578947368421,-10)(233.368421052632,-8) 
};
\addplot [
color=darkgray,
solid,
forget plot
]
coordinates{
 (155.578947368421,-12)(233.368421052632,-8) 
};
\addplot [
color=darkgray,
solid,
forget plot
]
coordinates{
 (155.578947368421,-14)(233.368421052632,-8) 
};
\addplot [
color=darkgray,
solid,
forget plot
]
coordinates{
 (155.578947368421,-16)(233.368421052632,-8) 
};
\addplot [
color=darkgray,
solid,
forget plot
]
coordinates{
 (155.578947368421,-18)(233.368421052632,-8) 
};
\addplot [
color=darkgray,
solid,
forget plot
]
coordinates{
 (155.578947368421,0)(233.368421052632,-10) 
};
\addplot [
color=darkgray,
solid,
forget plot
]
coordinates{
 (155.578947368421,-2)(233.368421052632,-10) 
};
\addplot [
color=darkgray,
solid,
forget plot
]
coordinates{
 (155.578947368421,-4)(233.368421052632,-10) 
};
\addplot [
color=darkgray,
solid,
forget plot
]
coordinates{
 (155.578947368421,-6)(233.368421052632,-10) 
};
\addplot [
color=darkgray,
solid,
forget plot
]
coordinates{
 (155.578947368421,-8)(233.368421052632,-10) 
};
\addplot [
color=darkgray,
solid,
forget plot
]
coordinates{
 (155.578947368421,-10)(233.368421052632,-10) 
};
\addplot [
color=darkgray,
solid,
forget plot
]
coordinates{
 (155.578947368421,-12)(233.368421052632,-10) 
};
\addplot [
color=darkgray,
solid,
forget plot
]
coordinates{
 (155.578947368421,-14)(233.368421052632,-10) 
};
\addplot [
color=darkgray,
solid,
forget plot
]
coordinates{
 (155.578947368421,-16)(233.368421052632,-10) 
};
\addplot [
color=darkgray,
solid,
forget plot
]
coordinates{
 (155.578947368421,-18)(233.368421052632,-10) 
};
\addplot [
color=darkgray,
solid,
forget plot
]
coordinates{
 (155.578947368421,-2)(233.368421052632,-12) 
};
\addplot [
color=darkgray,
solid,
forget plot
]
coordinates{
 (155.578947368421,-4)(233.368421052632,-12) 
};
\addplot [
color=darkgray,
solid,
forget plot
]
coordinates{
 (155.578947368421,-6)(233.368421052632,-12) 
};
\addplot [
color=darkgray,
solid,
forget plot
]
coordinates{
 (155.578947368421,-8)(233.368421052632,-12) 
};
\addplot [
color=darkgray,
solid,
forget plot
]
coordinates{
 (155.578947368421,-10)(233.368421052632,-12) 
};
\addplot [
color=darkgray,
solid,
forget plot
]
coordinates{
 (155.578947368421,-12)(233.368421052632,-12) 
};
\addplot [
color=darkgray,
solid,
forget plot
]
coordinates{
 (155.578947368421,-14)(233.368421052632,-12) 
};
\addplot [
color=darkgray,
solid,
forget plot
]
coordinates{
 (155.578947368421,-16)(233.368421052632,-12) 
};
\addplot [
color=darkgray,
solid,
forget plot
]
coordinates{
 (155.578947368421,-18)(233.368421052632,-12) 
};
\addplot [
color=darkgray,
solid,
forget plot
]
coordinates{
 (155.578947368421,-4)(233.368421052632,-14) 
};
\addplot [
color=darkgray,
solid,
forget plot
]
coordinates{
 (155.578947368421,-6)(233.368421052632,-14) 
};
\addplot [
color=darkgray,
solid,
forget plot
]
coordinates{
 (155.578947368421,-8)(233.368421052632,-14) 
};
\addplot [
color=darkgray,
solid,
forget plot
]
coordinates{
 (155.578947368421,-10)(233.368421052632,-14) 
};
\addplot [
color=darkgray,
solid,
forget plot
]
coordinates{
 (155.578947368421,-12)(233.368421052632,-14) 
};
\addplot [
color=darkgray,
solid,
forget plot
]
coordinates{
 (155.578947368421,-14)(233.368421052632,-14) 
};
\addplot [
color=darkgray,
solid,
forget plot
]
coordinates{
 (155.578947368421,-16)(233.368421052632,-14) 
};
\addplot [
color=darkgray,
solid,
forget plot
]
coordinates{
 (155.578947368421,-18)(233.368421052632,-14) 
};
\addplot [
color=darkgray,
solid,
forget plot
]
coordinates{
 (155.578947368421,-6)(233.368421052632,-16) 
};
\addplot [
color=darkgray,
solid,
forget plot
]
coordinates{
 (155.578947368421,-8)(233.368421052632,-16) 
};
\addplot [
color=darkgray,
solid,
forget plot
]
coordinates{
 (155.578947368421,-10)(233.368421052632,-16) 
};
\addplot [
color=darkgray,
solid,
forget plot
]
coordinates{
 (155.578947368421,-12)(233.368421052632,-16) 
};
\addplot [
color=darkgray,
solid,
forget plot
]
coordinates{
 (155.578947368421,-14)(233.368421052632,-16) 
};
\addplot [
color=darkgray,
solid,
forget plot
]
coordinates{
 (155.578947368421,-16)(233.368421052632,-16) 
};
\addplot [
color=darkgray,
solid,
forget plot
]
coordinates{
 (155.578947368421,-18)(233.368421052632,-16) 
};
\addplot [
color=darkgray,
solid,
forget plot
]
coordinates{
 (155.578947368421,-8)(233.368421052632,-18) 
};
\addplot [
color=darkgray,
solid,
forget plot
]
coordinates{
 (155.578947368421,-10)(233.368421052632,-18) 
};
\addplot [
color=darkgray,
solid,
forget plot
]
coordinates{
 (155.578947368421,-12)(233.368421052632,-18) 
};
\addplot [
color=darkgray,
solid,
forget plot
]
coordinates{
 (155.578947368421,-14)(233.368421052632,-18) 
};
\addplot [
color=darkgray,
solid,
forget plot
]
coordinates{
 (155.578947368421,-16)(233.368421052632,-18) 
};
\addplot [
color=darkgray,
solid,
forget plot
]
coordinates{
 (155.578947368421,-18)(233.368421052632,-18) 
};
\addplot [
color=darkgray,
solid,
forget plot
]
coordinates{
 (233.368421052632,0)(311.157894736842,0) 
};
\addplot [
color=darkgray,
solid,
forget plot
]
coordinates{
 (233.368421052632,-2)(311.157894736842,0) 
};
\addplot [
color=darkgray,
solid,
forget plot
]
coordinates{
 (233.368421052632,-4)(311.157894736842,0) 
};
\addplot [
color=darkgray,
solid,
forget plot
]
coordinates{
 (233.368421052632,-6)(311.157894736842,0) 
};
\addplot [
color=darkgray,
solid,
forget plot
]
coordinates{
 (233.368421052632,-8)(311.157894736842,0) 
};
\addplot [
color=darkgray,
solid,
forget plot
]
coordinates{
 (233.368421052632,-10)(311.157894736842,0) 
};
\addplot [
color=darkgray,
solid,
forget plot
]
coordinates{
 (233.368421052632,0)(311.157894736842,-2) 
};
\addplot [
color=darkgray,
solid,
forget plot
]
coordinates{
 (233.368421052632,-2)(311.157894736842,-2) 
};
\addplot [
color=darkgray,
solid,
forget plot
]
coordinates{
 (233.368421052632,-4)(311.157894736842,-2) 
};
\addplot [
color=darkgray,
solid,
forget plot
]
coordinates{
 (233.368421052632,-6)(311.157894736842,-2) 
};
\addplot [
color=darkgray,
solid,
forget plot
]
coordinates{
 (233.368421052632,-8)(311.157894736842,-2) 
};
\addplot [
color=darkgray,
solid,
forget plot
]
coordinates{
 (233.368421052632,-10)(311.157894736842,-2) 
};
\addplot [
color=darkgray,
solid,
forget plot
]
coordinates{
 (233.368421052632,-12)(311.157894736842,-2) 
};
\addplot [
color=darkgray,
solid,
forget plot
]
coordinates{
 (233.368421052632,0)(311.157894736842,-4) 
};
\addplot [
color=darkgray,
solid,
forget plot
]
coordinates{
 (233.368421052632,-2)(311.157894736842,-4) 
};
\addplot [
color=darkgray,
solid,
forget plot
]
coordinates{
 (233.368421052632,-4)(311.157894736842,-4) 
};
\addplot [
color=darkgray,
solid,
forget plot
]
coordinates{
 (233.368421052632,-6)(311.157894736842,-4) 
};
\addplot [
color=darkgray,
solid,
forget plot
]
coordinates{
 (233.368421052632,-8)(311.157894736842,-4) 
};
\addplot [
color=darkgray,
solid,
forget plot
]
coordinates{
 (233.368421052632,-10)(311.157894736842,-4) 
};
\addplot [
color=darkgray,
solid,
forget plot
]
coordinates{
 (233.368421052632,-12)(311.157894736842,-4) 
};
\addplot [
color=darkgray,
solid,
forget plot
]
coordinates{
 (233.368421052632,-14)(311.157894736842,-4) 
};
\addplot [
color=darkgray,
solid,
forget plot
]
coordinates{
 (233.368421052632,0)(311.157894736842,-6) 
};
\addplot [
color=darkgray,
solid,
forget plot
]
coordinates{
 (233.368421052632,-2)(311.157894736842,-6) 
};
\addplot [
color=darkgray,
solid,
forget plot
]
coordinates{
 (233.368421052632,-4)(311.157894736842,-6) 
};
\addplot [
color=darkgray,
solid,
forget plot
]
coordinates{
 (233.368421052632,-6)(311.157894736842,-6) 
};
\addplot [
color=darkgray,
solid,
forget plot
]
coordinates{
 (233.368421052632,-8)(311.157894736842,-6) 
};
\addplot [
color=darkgray,
solid,
forget plot
]
coordinates{
 (233.368421052632,-10)(311.157894736842,-6) 
};
\addplot [
color=darkgray,
solid,
forget plot
]
coordinates{
 (233.368421052632,-12)(311.157894736842,-6) 
};
\addplot [
color=darkgray,
solid,
forget plot
]
coordinates{
 (233.368421052632,-14)(311.157894736842,-6) 
};
\addplot [
color=darkgray,
solid,
forget plot
]
coordinates{
 (233.368421052632,-16)(311.157894736842,-6) 
};
\addplot [
color=darkgray,
solid,
forget plot
]
coordinates{
 (233.368421052632,0)(311.157894736842,-8) 
};
\addplot [
color=darkgray,
solid,
forget plot
]
coordinates{
 (233.368421052632,-2)(311.157894736842,-8) 
};
\addplot [
color=darkgray,
solid,
forget plot
]
coordinates{
 (233.368421052632,-4)(311.157894736842,-8) 
};
\addplot [
color=darkgray,
solid,
forget plot
]
coordinates{
 (233.368421052632,-6)(311.157894736842,-8) 
};
\addplot [
color=darkgray,
solid,
forget plot
]
coordinates{
 (233.368421052632,-8)(311.157894736842,-8) 
};
\addplot [
color=darkgray,
solid,
forget plot
]
coordinates{
 (233.368421052632,-10)(311.157894736842,-8) 
};
\addplot [
color=darkgray,
solid,
forget plot
]
coordinates{
 (233.368421052632,-12)(311.157894736842,-8) 
};
\addplot [
color=darkgray,
solid,
forget plot
]
coordinates{
 (233.368421052632,-14)(311.157894736842,-8) 
};
\addplot [
color=darkgray,
solid,
forget plot
]
coordinates{
 (233.368421052632,-16)(311.157894736842,-8) 
};
\addplot [
color=darkgray,
solid,
forget plot
]
coordinates{
 (233.368421052632,-18)(311.157894736842,-8) 
};
\addplot [
color=darkgray,
solid,
forget plot
]
coordinates{
 (233.368421052632,0)(311.157894736842,-10) 
};
\addplot [
color=darkgray,
solid,
forget plot
]
coordinates{
 (233.368421052632,-2)(311.157894736842,-10) 
};
\addplot [
color=darkgray,
solid,
forget plot
]
coordinates{
 (233.368421052632,-4)(311.157894736842,-10) 
};
\addplot [
color=darkgray,
solid,
forget plot
]
coordinates{
 (233.368421052632,-6)(311.157894736842,-10) 
};
\addplot [
color=darkgray,
solid,
forget plot
]
coordinates{
 (233.368421052632,-8)(311.157894736842,-10) 
};
\addplot [
color=darkgray,
solid,
forget plot
]
coordinates{
 (233.368421052632,-10)(311.157894736842,-10) 
};
\addplot [
color=darkgray,
solid,
forget plot
]
coordinates{
 (233.368421052632,-12)(311.157894736842,-10) 
};
\addplot [
color=darkgray,
solid,
forget plot
]
coordinates{
 (233.368421052632,-14)(311.157894736842,-10) 
};
\addplot [
color=darkgray,
solid,
forget plot
]
coordinates{
 (233.368421052632,-16)(311.157894736842,-10) 
};
\addplot [
color=darkgray,
solid,
forget plot
]
coordinates{
 (233.368421052632,-18)(311.157894736842,-10) 
};
\addplot [
color=darkgray,
solid,
forget plot
]
coordinates{
 (233.368421052632,-2)(311.157894736842,-12) 
};
\addplot [
color=darkgray,
solid,
forget plot
]
coordinates{
 (233.368421052632,-4)(311.157894736842,-12) 
};
\addplot [
color=darkgray,
solid,
forget plot
]
coordinates{
 (233.368421052632,-6)(311.157894736842,-12) 
};
\addplot [
color=darkgray,
solid,
forget plot
]
coordinates{
 (233.368421052632,-8)(311.157894736842,-12) 
};
\addplot [
color=darkgray,
solid,
forget plot
]
coordinates{
 (233.368421052632,-10)(311.157894736842,-12) 
};
\addplot [
color=darkgray,
solid,
forget plot
]
coordinates{
 (233.368421052632,-12)(311.157894736842,-12) 
};
\addplot [
color=darkgray,
solid,
forget plot
]
coordinates{
 (233.368421052632,-14)(311.157894736842,-12) 
};
\addplot [
color=darkgray,
solid,
forget plot
]
coordinates{
 (233.368421052632,-16)(311.157894736842,-12) 
};
\addplot [
color=darkgray,
solid,
forget plot
]
coordinates{
 (233.368421052632,-18)(311.157894736842,-12) 
};
\addplot [
color=darkgray,
solid,
forget plot
]
coordinates{
 (233.368421052632,-4)(311.157894736842,-14) 
};
\addplot [
color=darkgray,
solid,
forget plot
]
coordinates{
 (233.368421052632,-6)(311.157894736842,-14) 
};
\addplot [
color=darkgray,
solid,
forget plot
]
coordinates{
 (233.368421052632,-8)(311.157894736842,-14) 
};
\addplot [
color=darkgray,
solid,
forget plot
]
coordinates{
 (233.368421052632,-10)(311.157894736842,-14) 
};
\addplot [
color=darkgray,
solid,
forget plot
]
coordinates{
 (233.368421052632,-12)(311.157894736842,-14) 
};
\addplot [
color=darkgray,
solid,
forget plot
]
coordinates{
 (233.368421052632,-14)(311.157894736842,-14) 
};
\addplot [
color=darkgray,
solid,
forget plot
]
coordinates{
 (233.368421052632,-16)(311.157894736842,-14) 
};
\addplot [
color=darkgray,
solid,
forget plot
]
coordinates{
 (233.368421052632,-18)(311.157894736842,-14) 
};
\addplot [
color=darkgray,
solid,
forget plot
]
coordinates{
 (233.368421052632,-6)(311.157894736842,-16) 
};
\addplot [
color=darkgray,
solid,
forget plot
]
coordinates{
 (233.368421052632,-8)(311.157894736842,-16) 
};
\addplot [
color=darkgray,
solid,
forget plot
]
coordinates{
 (233.368421052632,-10)(311.157894736842,-16) 
};
\addplot [
color=darkgray,
solid,
forget plot
]
coordinates{
 (233.368421052632,-12)(311.157894736842,-16) 
};
\addplot [
color=darkgray,
solid,
forget plot
]
coordinates{
 (233.368421052632,-14)(311.157894736842,-16) 
};
\addplot [
color=darkgray,
solid,
forget plot
]
coordinates{
 (233.368421052632,-16)(311.157894736842,-16) 
};
\addplot [
color=darkgray,
solid,
forget plot
]
coordinates{
 (233.368421052632,-18)(311.157894736842,-16) 
};
\addplot [
color=darkgray,
solid,
forget plot
]
coordinates{
 (233.368421052632,-8)(311.157894736842,-18) 
};
\addplot [
color=darkgray,
solid,
forget plot
]
coordinates{
 (233.368421052632,-10)(311.157894736842,-18) 
};
\addplot [
color=darkgray,
solid,
forget plot
]
coordinates{
 (233.368421052632,-12)(311.157894736842,-18) 
};
\addplot [
color=darkgray,
solid,
forget plot
]
coordinates{
 (233.368421052632,-14)(311.157894736842,-18) 
};
\addplot [
color=darkgray,
solid,
forget plot
]
coordinates{
 (233.368421052632,-16)(311.157894736842,-18) 
};
\addplot [
color=darkgray,
solid,
forget plot
]
coordinates{
 (233.368421052632,-18)(311.157894736842,-18) 
};
\addplot [
color=darkgray,
solid,
forget plot
]
coordinates{
 (311.157894736842,0)(388.947368421053,0) 
};
\addplot [
color=darkgray,
solid,
forget plot
]
coordinates{
 (311.157894736842,-2)(388.947368421053,0) 
};
\addplot [
color=darkgray,
solid,
forget plot
]
coordinates{
 (311.157894736842,-4)(388.947368421053,0) 
};
\addplot [
color=darkgray,
solid,
forget plot
]
coordinates{
 (311.157894736842,-6)(388.947368421053,0) 
};
\addplot [
color=darkgray,
solid,
forget plot
]
coordinates{
 (311.157894736842,-8)(388.947368421053,0) 
};
\addplot [
color=darkgray,
solid,
forget plot
]
coordinates{
 (311.157894736842,-10)(388.947368421053,0) 
};
\addplot [
color=darkgray,
solid,
forget plot
]
coordinates{
 (311.157894736842,0)(388.947368421053,-2) 
};
\addplot [
color=darkgray,
solid,
forget plot
]
coordinates{
 (311.157894736842,-2)(388.947368421053,-2) 
};
\addplot [
color=darkgray,
solid,
forget plot
]
coordinates{
 (311.157894736842,-4)(388.947368421053,-2) 
};
\addplot [
color=darkgray,
solid,
forget plot
]
coordinates{
 (311.157894736842,-6)(388.947368421053,-2) 
};
\addplot [
color=darkgray,
solid,
forget plot
]
coordinates{
 (311.157894736842,-8)(388.947368421053,-2) 
};
\addplot [
color=darkgray,
solid,
forget plot
]
coordinates{
 (311.157894736842,-10)(388.947368421053,-2) 
};
\addplot [
color=darkgray,
solid,
forget plot
]
coordinates{
 (311.157894736842,-12)(388.947368421053,-2) 
};
\addplot [
color=darkgray,
solid,
forget plot
]
coordinates{
 (311.157894736842,0)(388.947368421053,-4) 
};
\addplot [
color=darkgray,
solid,
forget plot
]
coordinates{
 (311.157894736842,-2)(388.947368421053,-4) 
};
\addplot [
color=darkgray,
solid,
forget plot
]
coordinates{
 (311.157894736842,-4)(388.947368421053,-4) 
};
\addplot [
color=darkgray,
solid,
forget plot
]
coordinates{
 (311.157894736842,-6)(388.947368421053,-4) 
};
\addplot [
color=darkgray,
solid,
forget plot
]
coordinates{
 (311.157894736842,-8)(388.947368421053,-4) 
};
\addplot [
color=darkgray,
solid,
forget plot
]
coordinates{
 (311.157894736842,-10)(388.947368421053,-4) 
};
\addplot [
color=darkgray,
solid,
forget plot
]
coordinates{
 (311.157894736842,-12)(388.947368421053,-4) 
};
\addplot [
color=darkgray,
solid,
forget plot
]
coordinates{
 (311.157894736842,-14)(388.947368421053,-4) 
};
\addplot [
color=darkgray,
solid,
forget plot
]
coordinates{
 (311.157894736842,0)(388.947368421053,-6) 
};
\addplot [
color=darkgray,
solid,
forget plot
]
coordinates{
 (311.157894736842,-2)(388.947368421053,-6) 
};
\addplot [
color=darkgray,
solid,
forget plot
]
coordinates{
 (311.157894736842,-4)(388.947368421053,-6) 
};
\addplot [
color=darkgray,
solid,
forget plot
]
coordinates{
 (311.157894736842,-6)(388.947368421053,-6) 
};
\addplot [
color=darkgray,
solid,
forget plot
]
coordinates{
 (311.157894736842,-8)(388.947368421053,-6) 
};
\addplot [
color=darkgray,
solid,
forget plot
]
coordinates{
 (311.157894736842,-10)(388.947368421053,-6) 
};
\addplot [
color=darkgray,
solid,
forget plot
]
coordinates{
 (311.157894736842,-12)(388.947368421053,-6) 
};
\addplot [
color=darkgray,
solid,
forget plot
]
coordinates{
 (311.157894736842,-14)(388.947368421053,-6) 
};
\addplot [
color=darkgray,
solid,
forget plot
]
coordinates{
 (311.157894736842,-16)(388.947368421053,-6) 
};
\addplot [
color=darkgray,
solid,
forget plot
]
coordinates{
 (311.157894736842,0)(388.947368421053,-8) 
};
\addplot [
color=darkgray,
solid,
forget plot
]
coordinates{
 (311.157894736842,-2)(388.947368421053,-8) 
};
\addplot [
color=darkgray,
solid,
forget plot
]
coordinates{
 (311.157894736842,-4)(388.947368421053,-8) 
};
\addplot [
color=darkgray,
solid,
forget plot
]
coordinates{
 (311.157894736842,-6)(388.947368421053,-8) 
};
\addplot [
color=darkgray,
solid,
forget plot
]
coordinates{
 (311.157894736842,-8)(388.947368421053,-8) 
};
\addplot [
color=darkgray,
solid,
forget plot
]
coordinates{
 (311.157894736842,-10)(388.947368421053,-8) 
};
\addplot [
color=darkgray,
solid,
forget plot
]
coordinates{
 (311.157894736842,-12)(388.947368421053,-8) 
};
\addplot [
color=darkgray,
solid,
forget plot
]
coordinates{
 (311.157894736842,-14)(388.947368421053,-8) 
};
\addplot [
color=darkgray,
solid,
forget plot
]
coordinates{
 (311.157894736842,-16)(388.947368421053,-8) 
};
\addplot [
color=darkgray,
solid,
forget plot
]
coordinates{
 (311.157894736842,-18)(388.947368421053,-8) 
};
\addplot [
color=darkgray,
solid,
forget plot
]
coordinates{
 (311.157894736842,0)(388.947368421053,-10) 
};
\addplot [
color=darkgray,
solid,
forget plot
]
coordinates{
 (311.157894736842,-2)(388.947368421053,-10) 
};
\addplot [
color=darkgray,
solid,
forget plot
]
coordinates{
 (311.157894736842,-4)(388.947368421053,-10) 
};
\addplot [
color=darkgray,
solid,
forget plot
]
coordinates{
 (311.157894736842,-6)(388.947368421053,-10) 
};
\addplot [
color=darkgray,
solid,
forget plot
]
coordinates{
 (311.157894736842,-8)(388.947368421053,-10) 
};
\addplot [
color=darkgray,
solid,
forget plot
]
coordinates{
 (311.157894736842,-10)(388.947368421053,-10) 
};
\addplot [
color=darkgray,
solid,
forget plot
]
coordinates{
 (311.157894736842,-12)(388.947368421053,-10) 
};
\addplot [
color=darkgray,
solid,
forget plot
]
coordinates{
 (311.157894736842,-14)(388.947368421053,-10) 
};
\addplot [
color=darkgray,
solid,
forget plot
]
coordinates{
 (311.157894736842,-16)(388.947368421053,-10) 
};
\addplot [
color=darkgray,
solid,
forget plot
]
coordinates{
 (311.157894736842,-18)(388.947368421053,-10) 
};
\addplot [
color=darkgray,
solid,
forget plot
]
coordinates{
 (311.157894736842,-2)(388.947368421053,-12) 
};
\addplot [
color=darkgray,
solid,
forget plot
]
coordinates{
 (311.157894736842,-4)(388.947368421053,-12) 
};
\addplot [
color=darkgray,
solid,
forget plot
]
coordinates{
 (311.157894736842,-6)(388.947368421053,-12) 
};
\addplot [
color=darkgray,
solid,
forget plot
]
coordinates{
 (311.157894736842,-8)(388.947368421053,-12) 
};
\addplot [
color=darkgray,
solid,
forget plot
]
coordinates{
 (311.157894736842,-10)(388.947368421053,-12) 
};
\addplot [
color=darkgray,
solid,
forget plot
]
coordinates{
 (311.157894736842,-12)(388.947368421053,-12) 
};
\addplot [
color=darkgray,
solid,
forget plot
]
coordinates{
 (311.157894736842,-14)(388.947368421053,-12) 
};
\addplot [
color=darkgray,
solid,
forget plot
]
coordinates{
 (311.157894736842,-16)(388.947368421053,-12) 
};
\addplot [
color=darkgray,
solid,
forget plot
]
coordinates{
 (311.157894736842,-18)(388.947368421053,-12) 
};
\addplot [
color=darkgray,
solid,
forget plot
]
coordinates{
 (311.157894736842,-4)(388.947368421053,-14) 
};
\addplot [
color=darkgray,
solid,
forget plot
]
coordinates{
 (311.157894736842,-6)(388.947368421053,-14) 
};
\addplot [
color=darkgray,
solid,
forget plot
]
coordinates{
 (311.157894736842,-8)(388.947368421053,-14) 
};
\addplot [
color=darkgray,
solid,
forget plot
]
coordinates{
 (311.157894736842,-10)(388.947368421053,-14) 
};
\addplot [
color=darkgray,
solid,
forget plot
]
coordinates{
 (311.157894736842,-12)(388.947368421053,-14) 
};
\addplot [
color=darkgray,
solid,
forget plot
]
coordinates{
 (311.157894736842,-14)(388.947368421053,-14) 
};
\addplot [
color=darkgray,
solid,
forget plot
]
coordinates{
 (311.157894736842,-16)(388.947368421053,-14) 
};
\addplot [
color=darkgray,
solid,
forget plot
]
coordinates{
 (311.157894736842,-18)(388.947368421053,-14) 
};
\addplot [
color=darkgray,
solid,
forget plot
]
coordinates{
 (311.157894736842,-6)(388.947368421053,-16) 
};
\addplot [
color=darkgray,
solid,
forget plot
]
coordinates{
 (311.157894736842,-8)(388.947368421053,-16) 
};
\addplot [
color=darkgray,
solid,
forget plot
]
coordinates{
 (311.157894736842,-10)(388.947368421053,-16) 
};
\addplot [
color=darkgray,
solid,
forget plot
]
coordinates{
 (311.157894736842,-12)(388.947368421053,-16) 
};
\addplot [
color=darkgray,
solid,
forget plot
]
coordinates{
 (311.157894736842,-14)(388.947368421053,-16) 
};
\addplot [
color=darkgray,
solid,
forget plot
]
coordinates{
 (311.157894736842,-16)(388.947368421053,-16) 
};
\addplot [
color=darkgray,
solid,
forget plot
]
coordinates{
 (311.157894736842,-18)(388.947368421053,-16) 
};
\addplot [
color=darkgray,
solid,
forget plot
]
coordinates{
 (311.157894736842,-8)(388.947368421053,-18) 
};
\addplot [
color=darkgray,
solid,
forget plot
]
coordinates{
 (311.157894736842,-10)(388.947368421053,-18) 
};
\addplot [
color=darkgray,
solid,
forget plot
]
coordinates{
 (311.157894736842,-12)(388.947368421053,-18) 
};
\addplot [
color=darkgray,
solid,
forget plot
]
coordinates{
 (311.157894736842,-14)(388.947368421053,-18) 
};
\addplot [
color=darkgray,
solid,
forget plot
]
coordinates{
 (311.157894736842,-16)(388.947368421053,-18) 
};
\addplot [
color=darkgray,
solid,
forget plot
]
coordinates{
 (311.157894736842,-18)(388.947368421053,-18) 
};
\addplot [
color=darkgray,
solid,
forget plot
]
coordinates{
 (388.947368421053,0)(466.736842105263,0) 
};
\addplot [
color=darkgray,
solid,
forget plot
]
coordinates{
 (388.947368421053,-2)(466.736842105263,0) 
};
\addplot [
color=darkgray,
solid,
forget plot
]
coordinates{
 (388.947368421053,-4)(466.736842105263,0) 
};
\addplot [
color=darkgray,
solid,
forget plot
]
coordinates{
 (388.947368421053,-6)(466.736842105263,0) 
};
\addplot [
color=darkgray,
solid,
forget plot
]
coordinates{
 (388.947368421053,-8)(466.736842105263,0) 
};
\addplot [
color=darkgray,
solid,
forget plot
]
coordinates{
 (388.947368421053,-10)(466.736842105263,0) 
};
\addplot [
color=darkgray,
solid,
forget plot
]
coordinates{
 (388.947368421053,0)(466.736842105263,-2) 
};
\addplot [
color=darkgray,
solid,
forget plot
]
coordinates{
 (388.947368421053,-2)(466.736842105263,-2) 
};
\addplot [
color=darkgray,
solid,
forget plot
]
coordinates{
 (388.947368421053,-4)(466.736842105263,-2) 
};
\addplot [
color=darkgray,
solid,
forget plot
]
coordinates{
 (388.947368421053,-6)(466.736842105263,-2) 
};
\addplot [
color=darkgray,
solid,
forget plot
]
coordinates{
 (388.947368421053,-8)(466.736842105263,-2) 
};
\addplot [
color=darkgray,
solid,
forget plot
]
coordinates{
 (388.947368421053,-10)(466.736842105263,-2) 
};
\addplot [
color=darkgray,
solid,
forget plot
]
coordinates{
 (388.947368421053,-12)(466.736842105263,-2) 
};
\addplot [
color=darkgray,
solid,
forget plot
]
coordinates{
 (388.947368421053,0)(466.736842105263,-4) 
};
\addplot [
color=darkgray,
solid,
forget plot
]
coordinates{
 (388.947368421053,-2)(466.736842105263,-4) 
};
\addplot [
color=darkgray,
solid,
forget plot
]
coordinates{
 (388.947368421053,-4)(466.736842105263,-4) 
};
\addplot [
color=darkgray,
solid,
forget plot
]
coordinates{
 (388.947368421053,-6)(466.736842105263,-4) 
};
\addplot [
color=darkgray,
solid,
forget plot
]
coordinates{
 (388.947368421053,-8)(466.736842105263,-4) 
};
\addplot [
color=darkgray,
solid,
forget plot
]
coordinates{
 (388.947368421053,-10)(466.736842105263,-4) 
};
\addplot [
color=darkgray,
solid,
forget plot
]
coordinates{
 (388.947368421053,-12)(466.736842105263,-4) 
};
\addplot [
color=darkgray,
solid,
forget plot
]
coordinates{
 (388.947368421053,-14)(466.736842105263,-4) 
};
\addplot [
color=darkgray,
solid,
forget plot
]
coordinates{
 (388.947368421053,0)(466.736842105263,-6) 
};
\addplot [
color=darkgray,
solid,
forget plot
]
coordinates{
 (388.947368421053,-2)(466.736842105263,-6) 
};
\addplot [
color=darkgray,
solid,
forget plot
]
coordinates{
 (388.947368421053,-4)(466.736842105263,-6) 
};
\addplot [
color=darkgray,
solid,
forget plot
]
coordinates{
 (388.947368421053,-6)(466.736842105263,-6) 
};
\addplot [
color=darkgray,
solid,
forget plot
]
coordinates{
 (388.947368421053,-8)(466.736842105263,-6) 
};
\addplot [
color=darkgray,
solid,
forget plot
]
coordinates{
 (388.947368421053,-10)(466.736842105263,-6) 
};
\addplot [
color=darkgray,
solid,
forget plot
]
coordinates{
 (388.947368421053,-12)(466.736842105263,-6) 
};
\addplot [
color=darkgray,
solid,
forget plot
]
coordinates{
 (388.947368421053,-14)(466.736842105263,-6) 
};
\addplot [
color=darkgray,
solid,
forget plot
]
coordinates{
 (388.947368421053,-16)(466.736842105263,-6) 
};
\addplot [
color=darkgray,
solid,
forget plot
]
coordinates{
 (388.947368421053,0)(466.736842105263,-8) 
};
\addplot [
color=darkgray,
solid,
forget plot
]
coordinates{
 (388.947368421053,-2)(466.736842105263,-8) 
};
\addplot [
color=darkgray,
solid,
forget plot
]
coordinates{
 (388.947368421053,-4)(466.736842105263,-8) 
};
\addplot [
color=darkgray,
solid,
forget plot
]
coordinates{
 (388.947368421053,-6)(466.736842105263,-8) 
};
\addplot [
color=darkgray,
solid,
forget plot
]
coordinates{
 (388.947368421053,-8)(466.736842105263,-8) 
};
\addplot [
color=darkgray,
solid,
forget plot
]
coordinates{
 (388.947368421053,-10)(466.736842105263,-8) 
};
\addplot [
color=darkgray,
solid,
forget plot
]
coordinates{
 (388.947368421053,-12)(466.736842105263,-8) 
};
\addplot [
color=darkgray,
solid,
forget plot
]
coordinates{
 (388.947368421053,-14)(466.736842105263,-8) 
};
\addplot [
color=darkgray,
solid,
forget plot
]
coordinates{
 (388.947368421053,-16)(466.736842105263,-8) 
};
\addplot [
color=darkgray,
solid,
forget plot
]
coordinates{
 (388.947368421053,-18)(466.736842105263,-8) 
};
\addplot [
color=darkgray,
solid,
forget plot
]
coordinates{
 (388.947368421053,0)(466.736842105263,-10) 
};
\addplot [
color=darkgray,
solid,
forget plot
]
coordinates{
 (388.947368421053,-2)(466.736842105263,-10) 
};
\addplot [
color=darkgray,
solid,
forget plot
]
coordinates{
 (388.947368421053,-4)(466.736842105263,-10) 
};
\addplot [
color=darkgray,
solid,
forget plot
]
coordinates{
 (388.947368421053,-6)(466.736842105263,-10) 
};
\addplot [
color=darkgray,
solid,
forget plot
]
coordinates{
 (388.947368421053,-8)(466.736842105263,-10) 
};
\addplot [
color=darkgray,
solid,
forget plot
]
coordinates{
 (388.947368421053,-10)(466.736842105263,-10) 
};
\addplot [
color=darkgray,
solid,
forget plot
]
coordinates{
 (388.947368421053,-12)(466.736842105263,-10) 
};
\addplot [
color=darkgray,
solid,
forget plot
]
coordinates{
 (388.947368421053,-14)(466.736842105263,-10) 
};
\addplot [
color=darkgray,
solid,
forget plot
]
coordinates{
 (388.947368421053,-16)(466.736842105263,-10) 
};
\addplot [
color=darkgray,
solid,
forget plot
]
coordinates{
 (388.947368421053,-18)(466.736842105263,-10) 
};
\addplot [
color=darkgray,
solid,
forget plot
]
coordinates{
 (388.947368421053,-2)(466.736842105263,-12) 
};
\addplot [
color=darkgray,
solid,
forget plot
]
coordinates{
 (388.947368421053,-4)(466.736842105263,-12) 
};
\addplot [
color=darkgray,
solid,
forget plot
]
coordinates{
 (388.947368421053,-6)(466.736842105263,-12) 
};
\addplot [
color=darkgray,
solid,
forget plot
]
coordinates{
 (388.947368421053,-8)(466.736842105263,-12) 
};
\addplot [
color=darkgray,
solid,
forget plot
]
coordinates{
 (388.947368421053,-10)(466.736842105263,-12) 
};
\addplot [
color=darkgray,
solid,
forget plot
]
coordinates{
 (388.947368421053,-12)(466.736842105263,-12) 
};
\addplot [
color=darkgray,
solid,
forget plot
]
coordinates{
 (388.947368421053,-14)(466.736842105263,-12) 
};
\addplot [
color=darkgray,
solid,
forget plot
]
coordinates{
 (388.947368421053,-16)(466.736842105263,-12) 
};
\addplot [
color=darkgray,
solid,
forget plot
]
coordinates{
 (388.947368421053,-18)(466.736842105263,-12) 
};
\addplot [
color=darkgray,
solid,
forget plot
]
coordinates{
 (388.947368421053,-4)(466.736842105263,-14) 
};
\addplot [
color=darkgray,
solid,
forget plot
]
coordinates{
 (388.947368421053,-6)(466.736842105263,-14) 
};
\addplot [
color=darkgray,
solid,
forget plot
]
coordinates{
 (388.947368421053,-8)(466.736842105263,-14) 
};
\addplot [
color=darkgray,
solid,
forget plot
]
coordinates{
 (388.947368421053,-10)(466.736842105263,-14) 
};
\addplot [
color=darkgray,
solid,
forget plot
]
coordinates{
 (388.947368421053,-12)(466.736842105263,-14) 
};
\addplot [
color=darkgray,
solid,
forget plot
]
coordinates{
 (388.947368421053,-14)(466.736842105263,-14) 
};
\addplot [
color=darkgray,
solid,
forget plot
]
coordinates{
 (388.947368421053,-16)(466.736842105263,-14) 
};
\addplot [
color=darkgray,
solid,
forget plot
]
coordinates{
 (388.947368421053,-18)(466.736842105263,-14) 
};
\addplot [
color=darkgray,
solid,
forget plot
]
coordinates{
 (388.947368421053,-6)(466.736842105263,-16) 
};
\addplot [
color=darkgray,
solid,
forget plot
]
coordinates{
 (388.947368421053,-8)(466.736842105263,-16) 
};
\addplot [
color=darkgray,
solid,
forget plot
]
coordinates{
 (388.947368421053,-10)(466.736842105263,-16) 
};
\addplot [
color=darkgray,
solid,
forget plot
]
coordinates{
 (388.947368421053,-12)(466.736842105263,-16) 
};
\addplot [
color=darkgray,
solid,
forget plot
]
coordinates{
 (388.947368421053,-14)(466.736842105263,-16) 
};
\addplot [
color=darkgray,
solid,
forget plot
]
coordinates{
 (388.947368421053,-16)(466.736842105263,-16) 
};
\addplot [
color=darkgray,
solid,
forget plot
]
coordinates{
 (388.947368421053,-18)(466.736842105263,-16) 
};
\addplot [
color=darkgray,
solid,
forget plot
]
coordinates{
 (388.947368421053,-8)(466.736842105263,-18) 
};
\addplot [
color=darkgray,
solid,
forget plot
]
coordinates{
 (388.947368421053,-10)(466.736842105263,-18) 
};
\addplot [
color=darkgray,
solid,
forget plot
]
coordinates{
 (388.947368421053,-12)(466.736842105263,-18) 
};
\addplot [
color=darkgray,
solid,
forget plot
]
coordinates{
 (388.947368421053,-14)(466.736842105263,-18) 
};
\addplot [
color=darkgray,
solid,
forget plot
]
coordinates{
 (388.947368421053,-16)(466.736842105263,-18) 
};
\addplot [
color=darkgray,
solid,
forget plot
]
coordinates{
 (388.947368421053,-18)(466.736842105263,-18) 
};
\addplot [
color=darkgray,
solid,
forget plot
]
coordinates{
 (466.736842105263,0)(544.526315789474,0) 
};
\addplot [
color=darkgray,
solid,
forget plot
]
coordinates{
 (466.736842105263,-2)(544.526315789474,0) 
};
\addplot [
color=darkgray,
solid,
forget plot
]
coordinates{
 (466.736842105263,-4)(544.526315789474,0) 
};
\addplot [
color=darkgray,
solid,
forget plot
]
coordinates{
 (466.736842105263,-6)(544.526315789474,0) 
};
\addplot [
color=darkgray,
solid,
forget plot
]
coordinates{
 (466.736842105263,-8)(544.526315789474,0) 
};
\addplot [
color=darkgray,
solid,
forget plot
]
coordinates{
 (466.736842105263,-10)(544.526315789474,0) 
};
\addplot [
color=darkgray,
solid,
forget plot
]
coordinates{
 (466.736842105263,0)(544.526315789474,-2) 
};
\addplot [
color=darkgray,
solid,
forget plot
]
coordinates{
 (466.736842105263,-2)(544.526315789474,-2) 
};
\addplot [
color=darkgray,
solid,
forget plot
]
coordinates{
 (466.736842105263,-4)(544.526315789474,-2) 
};
\addplot [
color=darkgray,
solid,
forget plot
]
coordinates{
 (466.736842105263,-6)(544.526315789474,-2) 
};
\addplot [
color=darkgray,
solid,
forget plot
]
coordinates{
 (466.736842105263,-8)(544.526315789474,-2) 
};
\addplot [
color=darkgray,
solid,
forget plot
]
coordinates{
 (466.736842105263,-10)(544.526315789474,-2) 
};
\addplot [
color=darkgray,
solid,
forget plot
]
coordinates{
 (466.736842105263,-12)(544.526315789474,-2) 
};
\addplot [
color=darkgray,
solid,
forget plot
]
coordinates{
 (466.736842105263,0)(544.526315789474,-4) 
};
\addplot [
color=darkgray,
solid,
forget plot
]
coordinates{
 (466.736842105263,-2)(544.526315789474,-4) 
};
\addplot [
color=darkgray,
solid,
forget plot
]
coordinates{
 (466.736842105263,-4)(544.526315789474,-4) 
};
\addplot [
color=darkgray,
solid,
forget plot
]
coordinates{
 (466.736842105263,-6)(544.526315789474,-4) 
};
\addplot [
color=darkgray,
solid,
forget plot
]
coordinates{
 (466.736842105263,-8)(544.526315789474,-4) 
};
\addplot [
color=darkgray,
solid,
forget plot
]
coordinates{
 (466.736842105263,-10)(544.526315789474,-4) 
};
\addplot [
color=darkgray,
solid,
forget plot
]
coordinates{
 (466.736842105263,-12)(544.526315789474,-4) 
};
\addplot [
color=darkgray,
solid,
forget plot
]
coordinates{
 (466.736842105263,-14)(544.526315789474,-4) 
};
\addplot [
color=darkgray,
solid,
forget plot
]
coordinates{
 (466.736842105263,0)(544.526315789474,-6) 
};
\addplot [
color=darkgray,
solid,
forget plot
]
coordinates{
 (466.736842105263,-2)(544.526315789474,-6) 
};
\addplot [
color=darkgray,
solid,
forget plot
]
coordinates{
 (466.736842105263,-4)(544.526315789474,-6) 
};
\addplot [
color=darkgray,
solid,
forget plot
]
coordinates{
 (466.736842105263,-6)(544.526315789474,-6) 
};
\addplot [
color=darkgray,
solid,
forget plot
]
coordinates{
 (466.736842105263,-8)(544.526315789474,-6) 
};
\addplot [
color=darkgray,
solid,
forget plot
]
coordinates{
 (466.736842105263,-10)(544.526315789474,-6) 
};
\addplot [
color=darkgray,
solid,
forget plot
]
coordinates{
 (466.736842105263,-12)(544.526315789474,-6) 
};
\addplot [
color=darkgray,
solid,
forget plot
]
coordinates{
 (466.736842105263,-14)(544.526315789474,-6) 
};
\addplot [
color=darkgray,
solid,
forget plot
]
coordinates{
 (466.736842105263,-16)(544.526315789474,-6) 
};
\addplot [
color=darkgray,
solid,
forget plot
]
coordinates{
 (466.736842105263,0)(544.526315789474,-8) 
};
\addplot [
color=darkgray,
solid,
forget plot
]
coordinates{
 (466.736842105263,-2)(544.526315789474,-8) 
};
\addplot [
color=darkgray,
solid,
forget plot
]
coordinates{
 (466.736842105263,-4)(544.526315789474,-8) 
};
\addplot [
color=darkgray,
solid,
forget plot
]
coordinates{
 (466.736842105263,-6)(544.526315789474,-8) 
};
\addplot [
color=darkgray,
solid,
forget plot
]
coordinates{
 (466.736842105263,-8)(544.526315789474,-8) 
};
\addplot [
color=darkgray,
solid,
forget plot
]
coordinates{
 (466.736842105263,-10)(544.526315789474,-8) 
};
\addplot [
color=darkgray,
solid,
forget plot
]
coordinates{
 (466.736842105263,-12)(544.526315789474,-8) 
};
\addplot [
color=darkgray,
solid,
forget plot
]
coordinates{
 (466.736842105263,-14)(544.526315789474,-8) 
};
\addplot [
color=darkgray,
solid,
forget plot
]
coordinates{
 (466.736842105263,-16)(544.526315789474,-8) 
};
\addplot [
color=darkgray,
solid,
forget plot
]
coordinates{
 (466.736842105263,-18)(544.526315789474,-8) 
};
\addplot [
color=darkgray,
solid,
forget plot
]
coordinates{
 (466.736842105263,0)(544.526315789474,-10) 
};
\addplot [
color=darkgray,
solid,
forget plot
]
coordinates{
 (466.736842105263,-2)(544.526315789474,-10) 
};
\addplot [
color=darkgray,
solid,
forget plot
]
coordinates{
 (466.736842105263,-4)(544.526315789474,-10) 
};
\addplot [
color=darkgray,
solid,
forget plot
]
coordinates{
 (466.736842105263,-6)(544.526315789474,-10) 
};
\addplot [
color=darkgray,
solid,
forget plot
]
coordinates{
 (466.736842105263,-8)(544.526315789474,-10) 
};
\addplot [
color=darkgray,
solid,
forget plot
]
coordinates{
 (466.736842105263,-10)(544.526315789474,-10) 
};
\addplot [
color=darkgray,
solid,
forget plot
]
coordinates{
 (466.736842105263,-12)(544.526315789474,-10) 
};
\addplot [
color=darkgray,
solid,
forget plot
]
coordinates{
 (466.736842105263,-14)(544.526315789474,-10) 
};
\addplot [
color=darkgray,
solid,
forget plot
]
coordinates{
 (466.736842105263,-16)(544.526315789474,-10) 
};
\addplot [
color=darkgray,
solid,
forget plot
]
coordinates{
 (466.736842105263,-18)(544.526315789474,-10) 
};
\addplot [
color=darkgray,
solid,
forget plot
]
coordinates{
 (466.736842105263,-2)(544.526315789474,-12) 
};
\addplot [
color=darkgray,
solid,
forget plot
]
coordinates{
 (466.736842105263,-4)(544.526315789474,-12) 
};
\addplot [
color=darkgray,
solid,
forget plot
]
coordinates{
 (466.736842105263,-6)(544.526315789474,-12) 
};
\addplot [
color=darkgray,
solid,
forget plot
]
coordinates{
 (466.736842105263,-8)(544.526315789474,-12) 
};
\addplot [
color=darkgray,
solid,
forget plot
]
coordinates{
 (466.736842105263,-10)(544.526315789474,-12) 
};
\addplot [
color=darkgray,
solid,
forget plot
]
coordinates{
 (466.736842105263,-12)(544.526315789474,-12) 
};
\addplot [
color=darkgray,
solid,
forget plot
]
coordinates{
 (466.736842105263,-14)(544.526315789474,-12) 
};
\addplot [
color=darkgray,
solid,
forget plot
]
coordinates{
 (466.736842105263,-16)(544.526315789474,-12) 
};
\addplot [
color=darkgray,
solid,
forget plot
]
coordinates{
 (466.736842105263,-18)(544.526315789474,-12) 
};
\addplot [
color=darkgray,
solid,
forget plot
]
coordinates{
 (466.736842105263,-4)(544.526315789474,-14) 
};
\addplot [
color=darkgray,
solid,
forget plot
]
coordinates{
 (466.736842105263,-6)(544.526315789474,-14) 
};
\addplot [
color=darkgray,
solid,
forget plot
]
coordinates{
 (466.736842105263,-8)(544.526315789474,-14) 
};
\addplot [
color=darkgray,
solid,
forget plot
]
coordinates{
 (466.736842105263,-10)(544.526315789474,-14) 
};
\addplot [
color=darkgray,
solid,
forget plot
]
coordinates{
 (466.736842105263,-12)(544.526315789474,-14) 
};
\addplot [
color=darkgray,
solid,
forget plot
]
coordinates{
 (466.736842105263,-14)(544.526315789474,-14) 
};
\addplot [
color=darkgray,
solid,
forget plot
]
coordinates{
 (466.736842105263,-16)(544.526315789474,-14) 
};
\addplot [
color=darkgray,
solid,
forget plot
]
coordinates{
 (466.736842105263,-18)(544.526315789474,-14) 
};
\addplot [
color=darkgray,
solid,
forget plot
]
coordinates{
 (466.736842105263,-6)(544.526315789474,-16) 
};
\addplot [
color=darkgray,
solid,
forget plot
]
coordinates{
 (466.736842105263,-8)(544.526315789474,-16) 
};
\addplot [
color=darkgray,
solid,
forget plot
]
coordinates{
 (466.736842105263,-10)(544.526315789474,-16) 
};
\addplot [
color=darkgray,
solid,
forget plot
]
coordinates{
 (466.736842105263,-12)(544.526315789474,-16) 
};
\addplot [
color=darkgray,
solid,
forget plot
]
coordinates{
 (466.736842105263,-14)(544.526315789474,-16) 
};
\addplot [
color=darkgray,
solid,
forget plot
]
coordinates{
 (466.736842105263,-16)(544.526315789474,-16) 
};
\addplot [
color=darkgray,
solid,
forget plot
]
coordinates{
 (466.736842105263,-18)(544.526315789474,-16) 
};
\addplot [
color=darkgray,
solid,
forget plot
]
coordinates{
 (466.736842105263,-8)(544.526315789474,-18) 
};
\addplot [
color=darkgray,
solid,
forget plot
]
coordinates{
 (466.736842105263,-10)(544.526315789474,-18) 
};
\addplot [
color=darkgray,
solid,
forget plot
]
coordinates{
 (466.736842105263,-12)(544.526315789474,-18) 
};
\addplot [
color=darkgray,
solid,
forget plot
]
coordinates{
 (466.736842105263,-14)(544.526315789474,-18) 
};
\addplot [
color=darkgray,
solid,
forget plot
]
coordinates{
 (466.736842105263,-16)(544.526315789474,-18) 
};
\addplot [
color=darkgray,
solid,
forget plot
]
coordinates{
 (466.736842105263,-18)(544.526315789474,-18) 
};
\addplot [
color=darkgray,
solid,
forget plot
]
coordinates{
 (544.526315789474,0)(622.315789473684,0) 
};
\addplot [
color=darkgray,
solid,
forget plot
]
coordinates{
 (544.526315789474,-2)(622.315789473684,0) 
};
\addplot [
color=darkgray,
solid,
forget plot
]
coordinates{
 (544.526315789474,-4)(622.315789473684,0) 
};
\addplot [
color=darkgray,
solid,
forget plot
]
coordinates{
 (544.526315789474,-6)(622.315789473684,0) 
};
\addplot [
color=darkgray,
solid,
forget plot
]
coordinates{
 (544.526315789474,-8)(622.315789473684,0) 
};
\addplot [
color=darkgray,
solid,
forget plot
]
coordinates{
 (544.526315789474,-10)(622.315789473684,0) 
};
\addplot [
color=darkgray,
solid,
forget plot
]
coordinates{
 (544.526315789474,0)(622.315789473684,-2) 
};
\addplot [
color=darkgray,
solid,
forget plot
]
coordinates{
 (544.526315789474,-2)(622.315789473684,-2) 
};
\addplot [
color=darkgray,
solid,
forget plot
]
coordinates{
 (544.526315789474,-4)(622.315789473684,-2) 
};
\addplot [
color=darkgray,
solid,
forget plot
]
coordinates{
 (544.526315789474,-6)(622.315789473684,-2) 
};
\addplot [
color=darkgray,
solid,
forget plot
]
coordinates{
 (544.526315789474,-8)(622.315789473684,-2) 
};
\addplot [
color=darkgray,
solid,
forget plot
]
coordinates{
 (544.526315789474,-10)(622.315789473684,-2) 
};
\addplot [
color=darkgray,
solid,
forget plot
]
coordinates{
 (544.526315789474,-12)(622.315789473684,-2) 
};
\addplot [
color=darkgray,
solid,
forget plot
]
coordinates{
 (544.526315789474,0)(622.315789473684,-4) 
};
\addplot [
color=darkgray,
solid,
forget plot
]
coordinates{
 (544.526315789474,-2)(622.315789473684,-4) 
};
\addplot [
color=darkgray,
solid,
forget plot
]
coordinates{
 (544.526315789474,-4)(622.315789473684,-4) 
};
\addplot [
color=darkgray,
solid,
forget plot
]
coordinates{
 (544.526315789474,-6)(622.315789473684,-4) 
};
\addplot [
color=darkgray,
solid,
forget plot
]
coordinates{
 (544.526315789474,-8)(622.315789473684,-4) 
};
\addplot [
color=darkgray,
solid,
forget plot
]
coordinates{
 (544.526315789474,-10)(622.315789473684,-4) 
};
\addplot [
color=darkgray,
solid,
forget plot
]
coordinates{
 (544.526315789474,-12)(622.315789473684,-4) 
};
\addplot [
color=darkgray,
solid,
forget plot
]
coordinates{
 (544.526315789474,-14)(622.315789473684,-4) 
};
\addplot [
color=darkgray,
solid,
forget plot
]
coordinates{
 (544.526315789474,0)(622.315789473684,-6) 
};
\addplot [
color=darkgray,
solid,
forget plot
]
coordinates{
 (544.526315789474,-2)(622.315789473684,-6) 
};
\addplot [
color=darkgray,
solid,
forget plot
]
coordinates{
 (544.526315789474,-4)(622.315789473684,-6) 
};
\addplot [
color=darkgray,
solid,
forget plot
]
coordinates{
 (544.526315789474,-6)(622.315789473684,-6) 
};
\addplot [
color=darkgray,
solid,
forget plot
]
coordinates{
 (544.526315789474,-8)(622.315789473684,-6) 
};
\addplot [
color=darkgray,
solid,
forget plot
]
coordinates{
 (544.526315789474,-10)(622.315789473684,-6) 
};
\addplot [
color=darkgray,
solid,
forget plot
]
coordinates{
 (544.526315789474,-12)(622.315789473684,-6) 
};
\addplot [
color=darkgray,
solid,
forget plot
]
coordinates{
 (544.526315789474,-14)(622.315789473684,-6) 
};
\addplot [
color=darkgray,
solid,
forget plot
]
coordinates{
 (544.526315789474,-16)(622.315789473684,-6) 
};
\addplot [
color=darkgray,
solid,
forget plot
]
coordinates{
 (544.526315789474,0)(622.315789473684,-8) 
};
\addplot [
color=darkgray,
solid,
forget plot
]
coordinates{
 (544.526315789474,-2)(622.315789473684,-8) 
};
\addplot [
color=darkgray,
solid,
forget plot
]
coordinates{
 (544.526315789474,-4)(622.315789473684,-8) 
};
\addplot [
color=darkgray,
solid,
forget plot
]
coordinates{
 (544.526315789474,-6)(622.315789473684,-8) 
};
\addplot [
color=darkgray,
solid,
forget plot
]
coordinates{
 (544.526315789474,-8)(622.315789473684,-8) 
};
\addplot [
color=darkgray,
solid,
forget plot
]
coordinates{
 (544.526315789474,-10)(622.315789473684,-8) 
};
\addplot [
color=darkgray,
solid,
forget plot
]
coordinates{
 (544.526315789474,-12)(622.315789473684,-8) 
};
\addplot [
color=darkgray,
solid,
forget plot
]
coordinates{
 (544.526315789474,-14)(622.315789473684,-8) 
};
\addplot [
color=darkgray,
solid,
forget plot
]
coordinates{
 (544.526315789474,-16)(622.315789473684,-8) 
};
\addplot [
color=darkgray,
solid,
forget plot
]
coordinates{
 (544.526315789474,-18)(622.315789473684,-8) 
};
\addplot [
color=darkgray,
solid,
forget plot
]
coordinates{
 (544.526315789474,0)(622.315789473684,-10) 
};
\addplot [
color=darkgray,
solid,
forget plot
]
coordinates{
 (544.526315789474,-2)(622.315789473684,-10) 
};
\addplot [
color=darkgray,
solid,
forget plot
]
coordinates{
 (544.526315789474,-4)(622.315789473684,-10) 
};
\addplot [
color=darkgray,
solid,
forget plot
]
coordinates{
 (544.526315789474,-6)(622.315789473684,-10) 
};
\addplot [
color=darkgray,
solid,
forget plot
]
coordinates{
 (544.526315789474,-8)(622.315789473684,-10) 
};
\addplot [
color=darkgray,
solid,
forget plot
]
coordinates{
 (544.526315789474,-10)(622.315789473684,-10) 
};
\addplot [
color=darkgray,
solid,
forget plot
]
coordinates{
 (544.526315789474,-12)(622.315789473684,-10) 
};
\addplot [
color=darkgray,
solid,
forget plot
]
coordinates{
 (544.526315789474,-14)(622.315789473684,-10) 
};
\addplot [
color=darkgray,
solid,
forget plot
]
coordinates{
 (544.526315789474,-16)(622.315789473684,-10) 
};
\addplot [
color=darkgray,
solid,
forget plot
]
coordinates{
 (544.526315789474,-18)(622.315789473684,-10) 
};
\addplot [
color=darkgray,
solid,
forget plot
]
coordinates{
 (544.526315789474,-2)(622.315789473684,-12) 
};
\addplot [
color=darkgray,
solid,
forget plot
]
coordinates{
 (544.526315789474,-4)(622.315789473684,-12) 
};
\addplot [
color=darkgray,
solid,
forget plot
]
coordinates{
 (544.526315789474,-6)(622.315789473684,-12) 
};
\addplot [
color=darkgray,
solid,
forget plot
]
coordinates{
 (544.526315789474,-8)(622.315789473684,-12) 
};
\addplot [
color=darkgray,
solid,
forget plot
]
coordinates{
 (544.526315789474,-10)(622.315789473684,-12) 
};
\addplot [
color=darkgray,
solid,
forget plot
]
coordinates{
 (544.526315789474,-12)(622.315789473684,-12) 
};
\addplot [
color=darkgray,
solid,
forget plot
]
coordinates{
 (544.526315789474,-14)(622.315789473684,-12) 
};
\addplot [
color=darkgray,
solid,
forget plot
]
coordinates{
 (544.526315789474,-16)(622.315789473684,-12) 
};
\addplot [
color=darkgray,
solid,
forget plot
]
coordinates{
 (544.526315789474,-18)(622.315789473684,-12) 
};
\addplot [
color=darkgray,
solid,
forget plot
]
coordinates{
 (544.526315789474,-4)(622.315789473684,-14) 
};
\addplot [
color=darkgray,
solid,
forget plot
]
coordinates{
 (544.526315789474,-6)(622.315789473684,-14) 
};
\addplot [
color=darkgray,
solid,
forget plot
]
coordinates{
 (544.526315789474,-8)(622.315789473684,-14) 
};
\addplot [
color=darkgray,
solid,
forget plot
]
coordinates{
 (544.526315789474,-10)(622.315789473684,-14) 
};
\addplot [
color=darkgray,
solid,
forget plot
]
coordinates{
 (544.526315789474,-12)(622.315789473684,-14) 
};
\addplot [
color=darkgray,
solid,
forget plot
]
coordinates{
 (544.526315789474,-14)(622.315789473684,-14) 
};
\addplot [
color=darkgray,
solid,
forget plot
]
coordinates{
 (544.526315789474,-16)(622.315789473684,-14) 
};
\addplot [
color=darkgray,
solid,
forget plot
]
coordinates{
 (544.526315789474,-18)(622.315789473684,-14) 
};
\addplot [
color=darkgray,
solid,
forget plot
]
coordinates{
 (544.526315789474,-6)(622.315789473684,-16) 
};
\addplot [
color=darkgray,
solid,
forget plot
]
coordinates{
 (544.526315789474,-8)(622.315789473684,-16) 
};
\addplot [
color=darkgray,
solid,
forget plot
]
coordinates{
 (544.526315789474,-10)(622.315789473684,-16) 
};
\addplot [
color=darkgray,
solid,
forget plot
]
coordinates{
 (544.526315789474,-12)(622.315789473684,-16) 
};
\addplot [
color=darkgray,
solid,
forget plot
]
coordinates{
 (544.526315789474,-14)(622.315789473684,-16) 
};
\addplot [
color=darkgray,
solid,
forget plot
]
coordinates{
 (544.526315789474,-16)(622.315789473684,-16) 
};
\addplot [
color=darkgray,
solid,
forget plot
]
coordinates{
 (544.526315789474,-18)(622.315789473684,-16) 
};
\addplot [
color=darkgray,
solid,
forget plot
]
coordinates{
 (544.526315789474,-8)(622.315789473684,-18) 
};
\addplot [
color=darkgray,
solid,
forget plot
]
coordinates{
 (544.526315789474,-10)(622.315789473684,-18) 
};
\addplot [
color=darkgray,
solid,
forget plot
]
coordinates{
 (544.526315789474,-12)(622.315789473684,-18) 
};
\addplot [
color=darkgray,
solid,
forget plot
]
coordinates{
 (544.526315789474,-14)(622.315789473684,-18) 
};
\addplot [
color=darkgray,
solid,
forget plot
]
coordinates{
 (544.526315789474,-16)(622.315789473684,-18) 
};
\addplot [
color=darkgray,
solid,
forget plot
]
coordinates{
 (544.526315789474,-18)(622.315789473684,-18) 
};
\addplot [
color=darkgray,
solid,
forget plot
]
coordinates{
 (622.315789473684,0)(700.105263157895,0) 
};
\addplot [
color=darkgray,
solid,
forget plot
]
coordinates{
 (622.315789473684,-2)(700.105263157895,0) 
};
\addplot [
color=darkgray,
solid,
forget plot
]
coordinates{
 (622.315789473684,-4)(700.105263157895,0) 
};
\addplot [
color=darkgray,
solid,
forget plot
]
coordinates{
 (622.315789473684,-6)(700.105263157895,0) 
};
\addplot [
color=darkgray,
solid,
forget plot
]
coordinates{
 (622.315789473684,-8)(700.105263157895,0) 
};
\addplot [
color=darkgray,
solid,
forget plot
]
coordinates{
 (622.315789473684,-10)(700.105263157895,0) 
};
\addplot [
color=darkgray,
solid,
forget plot
]
coordinates{
 (622.315789473684,0)(700.105263157895,-2) 
};
\addplot [
color=darkgray,
solid,
forget plot
]
coordinates{
 (622.315789473684,-2)(700.105263157895,-2) 
};
\addplot [
color=darkgray,
solid,
forget plot
]
coordinates{
 (622.315789473684,-4)(700.105263157895,-2) 
};
\addplot [
color=darkgray,
solid,
forget plot
]
coordinates{
 (622.315789473684,-6)(700.105263157895,-2) 
};
\addplot [
color=darkgray,
solid,
forget plot
]
coordinates{
 (622.315789473684,-8)(700.105263157895,-2) 
};
\addplot [
color=darkgray,
solid,
forget plot
]
coordinates{
 (622.315789473684,-10)(700.105263157895,-2) 
};
\addplot [
color=darkgray,
solid,
forget plot
]
coordinates{
 (622.315789473684,-12)(700.105263157895,-2) 
};
\addplot [
color=darkgray,
solid,
forget plot
]
coordinates{
 (622.315789473684,0)(700.105263157895,-4) 
};
\addplot [
color=darkgray,
solid,
forget plot
]
coordinates{
 (622.315789473684,-2)(700.105263157895,-4) 
};
\addplot [
color=darkgray,
solid,
forget plot
]
coordinates{
 (622.315789473684,-4)(700.105263157895,-4) 
};
\addplot [
color=darkgray,
solid,
forget plot
]
coordinates{
 (622.315789473684,-6)(700.105263157895,-4) 
};
\addplot [
color=darkgray,
solid,
forget plot
]
coordinates{
 (622.315789473684,-8)(700.105263157895,-4) 
};
\addplot [
color=darkgray,
solid,
forget plot
]
coordinates{
 (622.315789473684,-10)(700.105263157895,-4) 
};
\addplot [
color=darkgray,
solid,
forget plot
]
coordinates{
 (622.315789473684,-12)(700.105263157895,-4) 
};
\addplot [
color=darkgray,
solid,
forget plot
]
coordinates{
 (622.315789473684,-14)(700.105263157895,-4) 
};
\addplot [
color=darkgray,
solid,
forget plot
]
coordinates{
 (622.315789473684,0)(700.105263157895,-6) 
};
\addplot [
color=darkgray,
solid,
forget plot
]
coordinates{
 (622.315789473684,-2)(700.105263157895,-6) 
};
\addplot [
color=darkgray,
solid,
forget plot
]
coordinates{
 (622.315789473684,-4)(700.105263157895,-6) 
};
\addplot [
color=darkgray,
solid,
forget plot
]
coordinates{
 (622.315789473684,-6)(700.105263157895,-6) 
};
\addplot [
color=darkgray,
solid,
forget plot
]
coordinates{
 (622.315789473684,-8)(700.105263157895,-6) 
};
\addplot [
color=darkgray,
solid,
forget plot
]
coordinates{
 (622.315789473684,-10)(700.105263157895,-6) 
};
\addplot [
color=darkgray,
solid,
forget plot
]
coordinates{
 (622.315789473684,-12)(700.105263157895,-6) 
};
\addplot [
color=darkgray,
solid,
forget plot
]
coordinates{
 (622.315789473684,-14)(700.105263157895,-6) 
};
\addplot [
color=darkgray,
solid,
forget plot
]
coordinates{
 (622.315789473684,-16)(700.105263157895,-6) 
};
\addplot [
color=darkgray,
solid,
forget plot
]
coordinates{
 (622.315789473684,0)(700.105263157895,-8) 
};
\addplot [
color=darkgray,
solid,
forget plot
]
coordinates{
 (622.315789473684,-2)(700.105263157895,-8) 
};
\addplot [
color=darkgray,
solid,
forget plot
]
coordinates{
 (622.315789473684,-4)(700.105263157895,-8) 
};
\addplot [
color=darkgray,
solid,
forget plot
]
coordinates{
 (622.315789473684,-6)(700.105263157895,-8) 
};
\addplot [
color=darkgray,
solid,
forget plot
]
coordinates{
 (622.315789473684,-8)(700.105263157895,-8) 
};
\addplot [
color=darkgray,
solid,
forget plot
]
coordinates{
 (622.315789473684,-10)(700.105263157895,-8) 
};
\addplot [
color=darkgray,
solid,
forget plot
]
coordinates{
 (622.315789473684,-12)(700.105263157895,-8) 
};
\addplot [
color=darkgray,
solid,
forget plot
]
coordinates{
 (622.315789473684,-14)(700.105263157895,-8) 
};
\addplot [
color=darkgray,
solid,
forget plot
]
coordinates{
 (622.315789473684,-16)(700.105263157895,-8) 
};
\addplot [
color=darkgray,
solid,
forget plot
]
coordinates{
 (622.315789473684,-18)(700.105263157895,-8) 
};
\addplot [
color=darkgray,
solid,
forget plot
]
coordinates{
 (622.315789473684,0)(700.105263157895,-10) 
};
\addplot [
color=darkgray,
solid,
forget plot
]
coordinates{
 (622.315789473684,-2)(700.105263157895,-10) 
};
\addplot [
color=darkgray,
solid,
forget plot
]
coordinates{
 (622.315789473684,-4)(700.105263157895,-10) 
};
\addplot [
color=darkgray,
solid,
forget plot
]
coordinates{
 (622.315789473684,-6)(700.105263157895,-10) 
};
\addplot [
color=darkgray,
solid,
forget plot
]
coordinates{
 (622.315789473684,-8)(700.105263157895,-10) 
};
\addplot [
color=darkgray,
solid,
forget plot
]
coordinates{
 (622.315789473684,-10)(700.105263157895,-10) 
};
\addplot [
color=darkgray,
solid,
forget plot
]
coordinates{
 (622.315789473684,-12)(700.105263157895,-10) 
};
\addplot [
color=darkgray,
solid,
forget plot
]
coordinates{
 (622.315789473684,-14)(700.105263157895,-10) 
};
\addplot [
color=darkgray,
solid,
forget plot
]
coordinates{
 (622.315789473684,-16)(700.105263157895,-10) 
};
\addplot [
color=darkgray,
solid,
forget plot
]
coordinates{
 (622.315789473684,-18)(700.105263157895,-10) 
};
\addplot [
color=darkgray,
solid,
forget plot
]
coordinates{
 (622.315789473684,-2)(700.105263157895,-12) 
};
\addplot [
color=darkgray,
solid,
forget plot
]
coordinates{
 (622.315789473684,-4)(700.105263157895,-12) 
};
\addplot [
color=darkgray,
solid,
forget plot
]
coordinates{
 (622.315789473684,-6)(700.105263157895,-12) 
};
\addplot [
color=darkgray,
solid,
forget plot
]
coordinates{
 (622.315789473684,-8)(700.105263157895,-12) 
};
\addplot [
color=darkgray,
solid,
forget plot
]
coordinates{
 (622.315789473684,-10)(700.105263157895,-12) 
};
\addplot [
color=darkgray,
solid,
forget plot
]
coordinates{
 (622.315789473684,-12)(700.105263157895,-12) 
};
\addplot [
color=darkgray,
solid,
forget plot
]
coordinates{
 (622.315789473684,-14)(700.105263157895,-12) 
};
\addplot [
color=darkgray,
solid,
forget plot
]
coordinates{
 (622.315789473684,-16)(700.105263157895,-12) 
};
\addplot [
color=darkgray,
solid,
forget plot
]
coordinates{
 (622.315789473684,-18)(700.105263157895,-12) 
};
\addplot [
color=darkgray,
solid,
forget plot
]
coordinates{
 (622.315789473684,-4)(700.105263157895,-14) 
};
\addplot [
color=darkgray,
solid,
forget plot
]
coordinates{
 (622.315789473684,-6)(700.105263157895,-14) 
};
\addplot [
color=darkgray,
solid,
forget plot
]
coordinates{
 (622.315789473684,-8)(700.105263157895,-14) 
};
\addplot [
color=darkgray,
solid,
forget plot
]
coordinates{
 (622.315789473684,-10)(700.105263157895,-14) 
};
\addplot [
color=darkgray,
solid,
forget plot
]
coordinates{
 (622.315789473684,-12)(700.105263157895,-14) 
};
\addplot [
color=darkgray,
solid,
forget plot
]
coordinates{
 (622.315789473684,-14)(700.105263157895,-14) 
};
\addplot [
color=darkgray,
solid,
forget plot
]
coordinates{
 (622.315789473684,-16)(700.105263157895,-14) 
};
\addplot [
color=darkgray,
solid,
forget plot
]
coordinates{
 (622.315789473684,-18)(700.105263157895,-14) 
};
\addplot [
color=darkgray,
solid,
forget plot
]
coordinates{
 (622.315789473684,-6)(700.105263157895,-16) 
};
\addplot [
color=darkgray,
solid,
forget plot
]
coordinates{
 (622.315789473684,-8)(700.105263157895,-16) 
};
\addplot [
color=darkgray,
solid,
forget plot
]
coordinates{
 (622.315789473684,-10)(700.105263157895,-16) 
};
\addplot [
color=darkgray,
solid,
forget plot
]
coordinates{
 (622.315789473684,-12)(700.105263157895,-16) 
};
\addplot [
color=darkgray,
solid,
forget plot
]
coordinates{
 (622.315789473684,-14)(700.105263157895,-16) 
};
\addplot [
color=darkgray,
solid,
forget plot
]
coordinates{
 (622.315789473684,-16)(700.105263157895,-16) 
};
\addplot [
color=darkgray,
solid,
forget plot
]
coordinates{
 (622.315789473684,-18)(700.105263157895,-16) 
};
\addplot [
color=darkgray,
solid,
forget plot
]
coordinates{
 (622.315789473684,-8)(700.105263157895,-18) 
};
\addplot [
color=darkgray,
solid,
forget plot
]
coordinates{
 (622.315789473684,-10)(700.105263157895,-18) 
};
\addplot [
color=darkgray,
solid,
forget plot
]
coordinates{
 (622.315789473684,-12)(700.105263157895,-18) 
};
\addplot [
color=darkgray,
solid,
forget plot
]
coordinates{
 (622.315789473684,-14)(700.105263157895,-18) 
};
\addplot [
color=darkgray,
solid,
forget plot
]
coordinates{
 (622.315789473684,-16)(700.105263157895,-18) 
};
\addplot [
color=darkgray,
solid,
forget plot
]
coordinates{
 (622.315789473684,-18)(700.105263157895,-18) 
};
\addplot [
color=darkgray,
solid,
forget plot
]
coordinates{
 (700.105263157895,0)(777.894736842105,0) 
};
\addplot [
color=darkgray,
solid,
forget plot
]
coordinates{
 (700.105263157895,-2)(777.894736842105,0) 
};
\addplot [
color=darkgray,
solid,
forget plot
]
coordinates{
 (700.105263157895,-4)(777.894736842105,0) 
};
\addplot [
color=darkgray,
solid,
forget plot
]
coordinates{
 (700.105263157895,-6)(777.894736842105,0) 
};
\addplot [
color=darkgray,
solid,
forget plot
]
coordinates{
 (700.105263157895,-8)(777.894736842105,0) 
};
\addplot [
color=darkgray,
solid,
forget plot
]
coordinates{
 (700.105263157895,-10)(777.894736842105,0) 
};
\addplot [
color=darkgray,
solid,
forget plot
]
coordinates{
 (700.105263157895,0)(777.894736842105,-2) 
};
\addplot [
color=darkgray,
solid,
forget plot
]
coordinates{
 (700.105263157895,-2)(777.894736842105,-2) 
};
\addplot [
color=darkgray,
solid,
forget plot
]
coordinates{
 (700.105263157895,-4)(777.894736842105,-2) 
};
\addplot [
color=darkgray,
solid,
forget plot
]
coordinates{
 (700.105263157895,-6)(777.894736842105,-2) 
};
\addplot [
color=darkgray,
solid,
forget plot
]
coordinates{
 (700.105263157895,-8)(777.894736842105,-2) 
};
\addplot [
color=darkgray,
solid,
forget plot
]
coordinates{
 (700.105263157895,-10)(777.894736842105,-2) 
};
\addplot [
color=darkgray,
solid,
forget plot
]
coordinates{
 (700.105263157895,-12)(777.894736842105,-2) 
};
\addplot [
color=darkgray,
solid,
forget plot
]
coordinates{
 (700.105263157895,0)(777.894736842105,-4) 
};
\addplot [
color=darkgray,
solid,
forget plot
]
coordinates{
 (700.105263157895,-2)(777.894736842105,-4) 
};
\addplot [
color=darkgray,
solid,
forget plot
]
coordinates{
 (700.105263157895,-4)(777.894736842105,-4) 
};
\addplot [
color=darkgray,
solid,
forget plot
]
coordinates{
 (700.105263157895,-6)(777.894736842105,-4) 
};
\addplot [
color=darkgray,
solid,
forget plot
]
coordinates{
 (700.105263157895,-8)(777.894736842105,-4) 
};
\addplot [
color=darkgray,
solid,
forget plot
]
coordinates{
 (700.105263157895,-10)(777.894736842105,-4) 
};
\addplot [
color=darkgray,
solid,
forget plot
]
coordinates{
 (700.105263157895,-12)(777.894736842105,-4) 
};
\addplot [
color=darkgray,
solid,
forget plot
]
coordinates{
 (700.105263157895,-14)(777.894736842105,-4) 
};
\addplot [
color=darkgray,
solid,
forget plot
]
coordinates{
 (700.105263157895,0)(777.894736842105,-6) 
};
\addplot [
color=darkgray,
solid,
forget plot
]
coordinates{
 (700.105263157895,-2)(777.894736842105,-6) 
};
\addplot [
color=darkgray,
solid,
forget plot
]
coordinates{
 (700.105263157895,-4)(777.894736842105,-6) 
};
\addplot [
color=darkgray,
solid,
forget plot
]
coordinates{
 (700.105263157895,-6)(777.894736842105,-6) 
};
\addplot [
color=darkgray,
solid,
forget plot
]
coordinates{
 (700.105263157895,-8)(777.894736842105,-6) 
};
\addplot [
color=darkgray,
solid,
forget plot
]
coordinates{
 (700.105263157895,-10)(777.894736842105,-6) 
};
\addplot [
color=darkgray,
solid,
forget plot
]
coordinates{
 (700.105263157895,-12)(777.894736842105,-6) 
};
\addplot [
color=darkgray,
solid,
forget plot
]
coordinates{
 (700.105263157895,-14)(777.894736842105,-6) 
};
\addplot [
color=darkgray,
solid,
forget plot
]
coordinates{
 (700.105263157895,-16)(777.894736842105,-6) 
};
\addplot [
color=darkgray,
solid,
forget plot
]
coordinates{
 (700.105263157895,0)(777.894736842105,-8) 
};
\addplot [
color=darkgray,
solid,
forget plot
]
coordinates{
 (700.105263157895,-2)(777.894736842105,-8) 
};
\addplot [
color=darkgray,
solid,
forget plot
]
coordinates{
 (700.105263157895,-4)(777.894736842105,-8) 
};
\addplot [
color=darkgray,
solid,
forget plot
]
coordinates{
 (700.105263157895,-6)(777.894736842105,-8) 
};
\addplot [
color=darkgray,
solid,
forget plot
]
coordinates{
 (700.105263157895,-8)(777.894736842105,-8) 
};
\addplot [
color=darkgray,
solid,
forget plot
]
coordinates{
 (700.105263157895,-10)(777.894736842105,-8) 
};
\addplot [
color=darkgray,
solid,
forget plot
]
coordinates{
 (700.105263157895,-12)(777.894736842105,-8) 
};
\addplot [
color=darkgray,
solid,
forget plot
]
coordinates{
 (700.105263157895,-14)(777.894736842105,-8) 
};
\addplot [
color=darkgray,
solid,
forget plot
]
coordinates{
 (700.105263157895,-16)(777.894736842105,-8) 
};
\addplot [
color=darkgray,
solid,
forget plot
]
coordinates{
 (700.105263157895,-18)(777.894736842105,-8) 
};
\addplot [
color=darkgray,
solid,
forget plot
]
coordinates{
 (700.105263157895,0)(777.894736842105,-10) 
};
\addplot [
color=darkgray,
solid,
forget plot
]
coordinates{
 (700.105263157895,-2)(777.894736842105,-10) 
};
\addplot [
color=darkgray,
solid,
forget plot
]
coordinates{
 (700.105263157895,-4)(777.894736842105,-10) 
};
\addplot [
color=darkgray,
solid,
forget plot
]
coordinates{
 (700.105263157895,-6)(777.894736842105,-10) 
};
\addplot [
color=darkgray,
solid,
forget plot
]
coordinates{
 (700.105263157895,-8)(777.894736842105,-10) 
};
\addplot [
color=darkgray,
solid,
forget plot
]
coordinates{
 (700.105263157895,-10)(777.894736842105,-10) 
};
\addplot [
color=darkgray,
solid,
forget plot
]
coordinates{
 (700.105263157895,-12)(777.894736842105,-10) 
};
\addplot [
color=darkgray,
solid,
forget plot
]
coordinates{
 (700.105263157895,-14)(777.894736842105,-10) 
};
\addplot [
color=darkgray,
solid,
forget plot
]
coordinates{
 (700.105263157895,-16)(777.894736842105,-10) 
};
\addplot [
color=darkgray,
solid,
forget plot
]
coordinates{
 (700.105263157895,-18)(777.894736842105,-10) 
};
\addplot [
color=darkgray,
solid,
forget plot
]
coordinates{
 (700.105263157895,-2)(777.894736842105,-12) 
};
\addplot [
color=darkgray,
solid,
forget plot
]
coordinates{
 (700.105263157895,-4)(777.894736842105,-12) 
};
\addplot [
color=darkgray,
solid,
forget plot
]
coordinates{
 (700.105263157895,-6)(777.894736842105,-12) 
};
\addplot [
color=darkgray,
solid,
forget plot
]
coordinates{
 (700.105263157895,-8)(777.894736842105,-12) 
};
\addplot [
color=darkgray,
solid,
forget plot
]
coordinates{
 (700.105263157895,-10)(777.894736842105,-12) 
};
\addplot [
color=darkgray,
solid,
forget plot
]
coordinates{
 (700.105263157895,-12)(777.894736842105,-12) 
};
\addplot [
color=darkgray,
solid,
forget plot
]
coordinates{
 (700.105263157895,-14)(777.894736842105,-12) 
};
\addplot [
color=darkgray,
solid,
forget plot
]
coordinates{
 (700.105263157895,-16)(777.894736842105,-12) 
};
\addplot [
color=darkgray,
solid,
forget plot
]
coordinates{
 (700.105263157895,-18)(777.894736842105,-12) 
};
\addplot [
color=darkgray,
solid,
forget plot
]
coordinates{
 (700.105263157895,-4)(777.894736842105,-14) 
};
\addplot [
color=darkgray,
solid,
forget plot
]
coordinates{
 (700.105263157895,-6)(777.894736842105,-14) 
};
\addplot [
color=darkgray,
solid,
forget plot
]
coordinates{
 (700.105263157895,-8)(777.894736842105,-14) 
};
\addplot [
color=darkgray,
solid,
forget plot
]
coordinates{
 (700.105263157895,-10)(777.894736842105,-14) 
};
\addplot [
color=darkgray,
solid,
forget plot
]
coordinates{
 (700.105263157895,-12)(777.894736842105,-14) 
};
\addplot [
color=darkgray,
solid,
forget plot
]
coordinates{
 (700.105263157895,-14)(777.894736842105,-14) 
};
\addplot [
color=darkgray,
solid,
forget plot
]
coordinates{
 (700.105263157895,-16)(777.894736842105,-14) 
};
\addplot [
color=darkgray,
solid,
forget plot
]
coordinates{
 (700.105263157895,-18)(777.894736842105,-14) 
};
\addplot [
color=darkgray,
solid,
forget plot
]
coordinates{
 (700.105263157895,-6)(777.894736842105,-16) 
};
\addplot [
color=darkgray,
solid,
forget plot
]
coordinates{
 (700.105263157895,-8)(777.894736842105,-16) 
};
\addplot [
color=darkgray,
solid,
forget plot
]
coordinates{
 (700.105263157895,-10)(777.894736842105,-16) 
};
\addplot [
color=darkgray,
solid,
forget plot
]
coordinates{
 (700.105263157895,-12)(777.894736842105,-16) 
};
\addplot [
color=darkgray,
solid,
forget plot
]
coordinates{
 (700.105263157895,-14)(777.894736842105,-16) 
};
\addplot [
color=darkgray,
solid,
forget plot
]
coordinates{
 (700.105263157895,-16)(777.894736842105,-16) 
};
\addplot [
color=darkgray,
solid,
forget plot
]
coordinates{
 (700.105263157895,-18)(777.894736842105,-16) 
};
\addplot [
color=darkgray,
solid,
forget plot
]
coordinates{
 (700.105263157895,-8)(777.894736842105,-18) 
};
\addplot [
color=darkgray,
solid,
forget plot
]
coordinates{
 (700.105263157895,-10)(777.894736842105,-18) 
};
\addplot [
color=darkgray,
solid,
forget plot
]
coordinates{
 (700.105263157895,-12)(777.894736842105,-18) 
};
\addplot [
color=darkgray,
solid,
forget plot
]
coordinates{
 (700.105263157895,-14)(777.894736842105,-18) 
};
\addplot [
color=darkgray,
solid,
forget plot
]
coordinates{
 (700.105263157895,-16)(777.894736842105,-18) 
};
\addplot [
color=darkgray,
solid,
forget plot
]
coordinates{
 (700.105263157895,-18)(777.894736842105,-18) 
};
\addplot [
color=darkgray,
solid,
forget plot
]
coordinates{
 (777.894736842105,0)(855.684210526316,0) 
};
\addplot [
color=darkgray,
solid,
forget plot
]
coordinates{
 (777.894736842105,-2)(855.684210526316,0) 
};
\addplot [
color=darkgray,
solid,
forget plot
]
coordinates{
 (777.894736842105,-4)(855.684210526316,0) 
};
\addplot [
color=darkgray,
solid,
forget plot
]
coordinates{
 (777.894736842105,-6)(855.684210526316,0) 
};
\addplot [
color=darkgray,
solid,
forget plot
]
coordinates{
 (777.894736842105,-8)(855.684210526316,0) 
};
\addplot [
color=darkgray,
solid,
forget plot
]
coordinates{
 (777.894736842105,-10)(855.684210526316,0) 
};
\addplot [
color=darkgray,
solid,
forget plot
]
coordinates{
 (777.894736842105,0)(855.684210526316,-2) 
};
\addplot [
color=darkgray,
solid,
forget plot
]
coordinates{
 (777.894736842105,-2)(855.684210526316,-2) 
};
\addplot [
color=darkgray,
solid,
forget plot
]
coordinates{
 (777.894736842105,-4)(855.684210526316,-2) 
};
\addplot [
color=darkgray,
solid,
forget plot
]
coordinates{
 (777.894736842105,-6)(855.684210526316,-2) 
};
\addplot [
color=darkgray,
solid,
forget plot
]
coordinates{
 (777.894736842105,-8)(855.684210526316,-2) 
};
\addplot [
color=darkgray,
solid,
forget plot
]
coordinates{
 (777.894736842105,-10)(855.684210526316,-2) 
};
\addplot [
color=darkgray,
solid,
forget plot
]
coordinates{
 (777.894736842105,-12)(855.684210526316,-2) 
};
\addplot [
color=darkgray,
solid,
forget plot
]
coordinates{
 (777.894736842105,0)(855.684210526316,-4) 
};
\addplot [
color=darkgray,
solid,
forget plot
]
coordinates{
 (777.894736842105,-2)(855.684210526316,-4) 
};
\addplot [
color=darkgray,
solid,
forget plot
]
coordinates{
 (777.894736842105,-4)(855.684210526316,-4) 
};
\addplot [
color=darkgray,
solid,
forget plot
]
coordinates{
 (777.894736842105,-6)(855.684210526316,-4) 
};
\addplot [
color=darkgray,
solid,
forget plot
]
coordinates{
 (777.894736842105,-8)(855.684210526316,-4) 
};
\addplot [
color=darkgray,
solid,
forget plot
]
coordinates{
 (777.894736842105,-10)(855.684210526316,-4) 
};
\addplot [
color=darkgray,
solid,
forget plot
]
coordinates{
 (777.894736842105,-12)(855.684210526316,-4) 
};
\addplot [
color=darkgray,
solid,
forget plot
]
coordinates{
 (777.894736842105,-14)(855.684210526316,-4) 
};
\addplot [
color=darkgray,
solid,
forget plot
]
coordinates{
 (777.894736842105,0)(855.684210526316,-6) 
};
\addplot [
color=darkgray,
solid,
forget plot
]
coordinates{
 (777.894736842105,-2)(855.684210526316,-6) 
};
\addplot [
color=darkgray,
solid,
forget plot
]
coordinates{
 (777.894736842105,-4)(855.684210526316,-6) 
};
\addplot [
color=darkgray,
solid,
forget plot
]
coordinates{
 (777.894736842105,-6)(855.684210526316,-6) 
};
\addplot [
color=darkgray,
solid,
forget plot
]
coordinates{
 (777.894736842105,-8)(855.684210526316,-6) 
};
\addplot [
color=darkgray,
solid,
forget plot
]
coordinates{
 (777.894736842105,-10)(855.684210526316,-6) 
};
\addplot [
color=darkgray,
solid,
forget plot
]
coordinates{
 (777.894736842105,-12)(855.684210526316,-6) 
};
\addplot [
color=darkgray,
solid,
forget plot
]
coordinates{
 (777.894736842105,-14)(855.684210526316,-6) 
};
\addplot [
color=darkgray,
solid,
forget plot
]
coordinates{
 (777.894736842105,-16)(855.684210526316,-6) 
};
\addplot [
color=darkgray,
solid,
forget plot
]
coordinates{
 (777.894736842105,0)(855.684210526316,-8) 
};
\addplot [
color=darkgray,
solid,
forget plot
]
coordinates{
 (777.894736842105,-2)(855.684210526316,-8) 
};
\addplot [
color=darkgray,
solid,
forget plot
]
coordinates{
 (777.894736842105,-4)(855.684210526316,-8) 
};
\addplot [
color=darkgray,
solid,
forget plot
]
coordinates{
 (777.894736842105,-6)(855.684210526316,-8) 
};
\addplot [
color=darkgray,
solid,
forget plot
]
coordinates{
 (777.894736842105,-8)(855.684210526316,-8) 
};
\addplot [
color=darkgray,
solid,
forget plot
]
coordinates{
 (777.894736842105,-10)(855.684210526316,-8) 
};
\addplot [
color=darkgray,
solid,
forget plot
]
coordinates{
 (777.894736842105,-12)(855.684210526316,-8) 
};
\addplot [
color=darkgray,
solid,
forget plot
]
coordinates{
 (777.894736842105,-14)(855.684210526316,-8) 
};
\addplot [
color=darkgray,
solid,
forget plot
]
coordinates{
 (777.894736842105,-16)(855.684210526316,-8) 
};
\addplot [
color=darkgray,
solid,
forget plot
]
coordinates{
 (777.894736842105,-18)(855.684210526316,-8) 
};
\addplot [
color=darkgray,
solid,
forget plot
]
coordinates{
 (777.894736842105,0)(855.684210526316,-10) 
};
\addplot [
color=darkgray,
solid,
forget plot
]
coordinates{
 (777.894736842105,-2)(855.684210526316,-10) 
};
\addplot [
color=darkgray,
solid,
forget plot
]
coordinates{
 (777.894736842105,-4)(855.684210526316,-10) 
};
\addplot [
color=darkgray,
solid,
forget plot
]
coordinates{
 (777.894736842105,-6)(855.684210526316,-10) 
};
\addplot [
color=darkgray,
solid,
forget plot
]
coordinates{
 (777.894736842105,-8)(855.684210526316,-10) 
};
\addplot [
color=darkgray,
solid,
forget plot
]
coordinates{
 (777.894736842105,-10)(855.684210526316,-10) 
};
\addplot [
color=darkgray,
solid,
forget plot
]
coordinates{
 (777.894736842105,-12)(855.684210526316,-10) 
};
\addplot [
color=darkgray,
solid,
forget plot
]
coordinates{
 (777.894736842105,-14)(855.684210526316,-10) 
};
\addplot [
color=darkgray,
solid,
forget plot
]
coordinates{
 (777.894736842105,-16)(855.684210526316,-10) 
};
\addplot [
color=darkgray,
solid,
forget plot
]
coordinates{
 (777.894736842105,-18)(855.684210526316,-10) 
};
\addplot [
color=darkgray,
solid,
forget plot
]
coordinates{
 (777.894736842105,-2)(855.684210526316,-12) 
};
\addplot [
color=darkgray,
solid,
forget plot
]
coordinates{
 (777.894736842105,-4)(855.684210526316,-12) 
};
\addplot [
color=darkgray,
solid,
forget plot
]
coordinates{
 (777.894736842105,-6)(855.684210526316,-12) 
};
\addplot [
color=darkgray,
solid,
forget plot
]
coordinates{
 (777.894736842105,-8)(855.684210526316,-12) 
};
\addplot [
color=darkgray,
solid,
forget plot
]
coordinates{
 (777.894736842105,-10)(855.684210526316,-12) 
};
\addplot [
color=darkgray,
solid,
forget plot
]
coordinates{
 (777.894736842105,-12)(855.684210526316,-12) 
};
\addplot [
color=darkgray,
solid,
forget plot
]
coordinates{
 (777.894736842105,-14)(855.684210526316,-12) 
};
\addplot [
color=darkgray,
solid,
forget plot
]
coordinates{
 (777.894736842105,-16)(855.684210526316,-12) 
};
\addplot [
color=darkgray,
solid,
forget plot
]
coordinates{
 (777.894736842105,-18)(855.684210526316,-12) 
};
\addplot [
color=darkgray,
solid,
forget plot
]
coordinates{
 (777.894736842105,-4)(855.684210526316,-14) 
};
\addplot [
color=darkgray,
solid,
forget plot
]
coordinates{
 (777.894736842105,-6)(855.684210526316,-14) 
};
\addplot [
color=darkgray,
solid,
forget plot
]
coordinates{
 (777.894736842105,-8)(855.684210526316,-14) 
};
\addplot [
color=darkgray,
solid,
forget plot
]
coordinates{
 (777.894736842105,-10)(855.684210526316,-14) 
};
\addplot [
color=darkgray,
solid,
forget plot
]
coordinates{
 (777.894736842105,-12)(855.684210526316,-14) 
};
\addplot [
color=darkgray,
solid,
forget plot
]
coordinates{
 (777.894736842105,-14)(855.684210526316,-14) 
};
\addplot [
color=darkgray,
solid,
forget plot
]
coordinates{
 (777.894736842105,-16)(855.684210526316,-14) 
};
\addplot [
color=darkgray,
solid,
forget plot
]
coordinates{
 (777.894736842105,-18)(855.684210526316,-14) 
};
\addplot [
color=darkgray,
solid,
forget plot
]
coordinates{
 (777.894736842105,-6)(855.684210526316,-16) 
};
\addplot [
color=darkgray,
solid,
forget plot
]
coordinates{
 (777.894736842105,-8)(855.684210526316,-16) 
};
\addplot [
color=darkgray,
solid,
forget plot
]
coordinates{
 (777.894736842105,-10)(855.684210526316,-16) 
};
\addplot [
color=darkgray,
solid,
forget plot
]
coordinates{
 (777.894736842105,-12)(855.684210526316,-16) 
};
\addplot [
color=darkgray,
solid,
forget plot
]
coordinates{
 (777.894736842105,-14)(855.684210526316,-16) 
};
\addplot [
color=darkgray,
solid,
forget plot
]
coordinates{
 (777.894736842105,-16)(855.684210526316,-16) 
};
\addplot [
color=darkgray,
solid,
forget plot
]
coordinates{
 (777.894736842105,-18)(855.684210526316,-16) 
};
\addplot [
color=darkgray,
solid,
forget plot
]
coordinates{
 (777.894736842105,-8)(855.684210526316,-18) 
};
\addplot [
color=darkgray,
solid,
forget plot
]
coordinates{
 (777.894736842105,-10)(855.684210526316,-18) 
};
\addplot [
color=darkgray,
solid,
forget plot
]
coordinates{
 (777.894736842105,-12)(855.684210526316,-18) 
};
\addplot [
color=darkgray,
solid,
forget plot
]
coordinates{
 (777.894736842105,-14)(855.684210526316,-18) 
};
\addplot [
color=darkgray,
solid,
forget plot
]
coordinates{
 (777.894736842105,-16)(855.684210526316,-18) 
};
\addplot [
color=darkgray,
solid,
forget plot
]
coordinates{
 (777.894736842105,-18)(855.684210526316,-18) 
};
\addplot [
color=darkgray,
solid,
forget plot
]
coordinates{
 (855.684210526316,0)(933.473684210526,0) 
};
\addplot [
color=darkgray,
solid,
forget plot
]
coordinates{
 (855.684210526316,-2)(933.473684210526,0) 
};
\addplot [
color=darkgray,
solid,
forget plot
]
coordinates{
 (855.684210526316,-4)(933.473684210526,0) 
};
\addplot [
color=darkgray,
solid,
forget plot
]
coordinates{
 (855.684210526316,-6)(933.473684210526,0) 
};
\addplot [
color=darkgray,
solid,
forget plot
]
coordinates{
 (855.684210526316,-8)(933.473684210526,0) 
};
\addplot [
color=darkgray,
solid,
forget plot
]
coordinates{
 (855.684210526316,-10)(933.473684210526,0) 
};
\addplot [
color=darkgray,
solid,
forget plot
]
coordinates{
 (855.684210526316,0)(933.473684210526,-2) 
};
\addplot [
color=darkgray,
solid,
forget plot
]
coordinates{
 (855.684210526316,-2)(933.473684210526,-2) 
};
\addplot [
color=darkgray,
solid,
forget plot
]
coordinates{
 (855.684210526316,-4)(933.473684210526,-2) 
};
\addplot [
color=darkgray,
solid,
forget plot
]
coordinates{
 (855.684210526316,-6)(933.473684210526,-2) 
};
\addplot [
color=darkgray,
solid,
forget plot
]
coordinates{
 (855.684210526316,-8)(933.473684210526,-2) 
};
\addplot [
color=darkgray,
solid,
forget plot
]
coordinates{
 (855.684210526316,-10)(933.473684210526,-2) 
};
\addplot [
color=darkgray,
solid,
forget plot
]
coordinates{
 (855.684210526316,-12)(933.473684210526,-2) 
};
\addplot [
color=darkgray,
solid,
forget plot
]
coordinates{
 (855.684210526316,0)(933.473684210526,-4) 
};
\addplot [
color=darkgray,
solid,
forget plot
]
coordinates{
 (855.684210526316,-2)(933.473684210526,-4) 
};
\addplot [
color=darkgray,
solid,
forget plot
]
coordinates{
 (855.684210526316,-4)(933.473684210526,-4) 
};
\addplot [
color=darkgray,
solid,
forget plot
]
coordinates{
 (855.684210526316,-6)(933.473684210526,-4) 
};
\addplot [
color=darkgray,
solid,
forget plot
]
coordinates{
 (855.684210526316,-8)(933.473684210526,-4) 
};
\addplot [
color=darkgray,
solid,
forget plot
]
coordinates{
 (855.684210526316,-10)(933.473684210526,-4) 
};
\addplot [
color=darkgray,
solid,
forget plot
]
coordinates{
 (855.684210526316,-12)(933.473684210526,-4) 
};
\addplot [
color=darkgray,
solid,
forget plot
]
coordinates{
 (855.684210526316,-14)(933.473684210526,-4) 
};
\addplot [
color=darkgray,
solid,
forget plot
]
coordinates{
 (855.684210526316,0)(933.473684210526,-6) 
};
\addplot [
color=darkgray,
solid,
forget plot
]
coordinates{
 (855.684210526316,-2)(933.473684210526,-6) 
};
\addplot [
color=darkgray,
solid,
forget plot
]
coordinates{
 (855.684210526316,-4)(933.473684210526,-6) 
};
\addplot [
color=darkgray,
solid,
forget plot
]
coordinates{
 (855.684210526316,-6)(933.473684210526,-6) 
};
\addplot [
color=darkgray,
solid,
forget plot
]
coordinates{
 (855.684210526316,-8)(933.473684210526,-6) 
};
\addplot [
color=darkgray,
solid,
forget plot
]
coordinates{
 (855.684210526316,-10)(933.473684210526,-6) 
};
\addplot [
color=darkgray,
solid,
forget plot
]
coordinates{
 (855.684210526316,-12)(933.473684210526,-6) 
};
\addplot [
color=darkgray,
solid,
forget plot
]
coordinates{
 (855.684210526316,-14)(933.473684210526,-6) 
};
\addplot [
color=darkgray,
solid,
forget plot
]
coordinates{
 (855.684210526316,-16)(933.473684210526,-6) 
};
\addplot [
color=darkgray,
solid,
forget plot
]
coordinates{
 (855.684210526316,0)(933.473684210526,-8) 
};
\addplot [
color=darkgray,
solid,
forget plot
]
coordinates{
 (855.684210526316,-2)(933.473684210526,-8) 
};
\addplot [
color=darkgray,
solid,
forget plot
]
coordinates{
 (855.684210526316,-4)(933.473684210526,-8) 
};
\addplot [
color=darkgray,
solid,
forget plot
]
coordinates{
 (855.684210526316,-6)(933.473684210526,-8) 
};
\addplot [
color=darkgray,
solid,
forget plot
]
coordinates{
 (855.684210526316,-8)(933.473684210526,-8) 
};
\addplot [
color=darkgray,
solid,
forget plot
]
coordinates{
 (855.684210526316,-10)(933.473684210526,-8) 
};
\addplot [
color=darkgray,
solid,
forget plot
]
coordinates{
 (855.684210526316,-12)(933.473684210526,-8) 
};
\addplot [
color=darkgray,
solid,
forget plot
]
coordinates{
 (855.684210526316,-14)(933.473684210526,-8) 
};
\addplot [
color=darkgray,
solid,
forget plot
]
coordinates{
 (855.684210526316,-16)(933.473684210526,-8) 
};
\addplot [
color=darkgray,
solid,
forget plot
]
coordinates{
 (855.684210526316,-18)(933.473684210526,-8) 
};
\addplot [
color=darkgray,
solid,
forget plot
]
coordinates{
 (855.684210526316,0)(933.473684210526,-10) 
};
\addplot [
color=darkgray,
solid,
forget plot
]
coordinates{
 (855.684210526316,-2)(933.473684210526,-10) 
};
\addplot [
color=darkgray,
solid,
forget plot
]
coordinates{
 (855.684210526316,-4)(933.473684210526,-10) 
};
\addplot [
color=darkgray,
solid,
forget plot
]
coordinates{
 (855.684210526316,-6)(933.473684210526,-10) 
};
\addplot [
color=darkgray,
solid,
forget plot
]
coordinates{
 (855.684210526316,-8)(933.473684210526,-10) 
};
\addplot [
color=darkgray,
solid,
forget plot
]
coordinates{
 (855.684210526316,-10)(933.473684210526,-10) 
};
\addplot [
color=darkgray,
solid,
forget plot
]
coordinates{
 (855.684210526316,-12)(933.473684210526,-10) 
};
\addplot [
color=darkgray,
solid,
forget plot
]
coordinates{
 (855.684210526316,-14)(933.473684210526,-10) 
};
\addplot [
color=darkgray,
solid,
forget plot
]
coordinates{
 (855.684210526316,-16)(933.473684210526,-10) 
};
\addplot [
color=darkgray,
solid,
forget plot
]
coordinates{
 (855.684210526316,-18)(933.473684210526,-10) 
};
\addplot [
color=darkgray,
solid,
forget plot
]
coordinates{
 (855.684210526316,-2)(933.473684210526,-12) 
};
\addplot [
color=darkgray,
solid,
forget plot
]
coordinates{
 (855.684210526316,-4)(933.473684210526,-12) 
};
\addplot [
color=darkgray,
solid,
forget plot
]
coordinates{
 (855.684210526316,-6)(933.473684210526,-12) 
};
\addplot [
color=darkgray,
solid,
forget plot
]
coordinates{
 (855.684210526316,-8)(933.473684210526,-12) 
};
\addplot [
color=darkgray,
solid,
forget plot
]
coordinates{
 (855.684210526316,-10)(933.473684210526,-12) 
};
\addplot [
color=darkgray,
solid,
forget plot
]
coordinates{
 (855.684210526316,-12)(933.473684210526,-12) 
};
\addplot [
color=darkgray,
solid,
forget plot
]
coordinates{
 (855.684210526316,-14)(933.473684210526,-12) 
};
\addplot [
color=darkgray,
solid,
forget plot
]
coordinates{
 (855.684210526316,-16)(933.473684210526,-12) 
};
\addplot [
color=darkgray,
solid,
forget plot
]
coordinates{
 (855.684210526316,-18)(933.473684210526,-12) 
};
\addplot [
color=darkgray,
solid,
forget plot
]
coordinates{
 (855.684210526316,-4)(933.473684210526,-14) 
};
\addplot [
color=darkgray,
solid,
forget plot
]
coordinates{
 (855.684210526316,-6)(933.473684210526,-14) 
};
\addplot [
color=darkgray,
solid,
forget plot
]
coordinates{
 (855.684210526316,-8)(933.473684210526,-14) 
};
\addplot [
color=darkgray,
solid,
forget plot
]
coordinates{
 (855.684210526316,-10)(933.473684210526,-14) 
};
\addplot [
color=darkgray,
solid,
forget plot
]
coordinates{
 (855.684210526316,-12)(933.473684210526,-14) 
};
\addplot [
color=darkgray,
solid,
forget plot
]
coordinates{
 (855.684210526316,-14)(933.473684210526,-14) 
};
\addplot [
color=darkgray,
solid,
forget plot
]
coordinates{
 (855.684210526316,-16)(933.473684210526,-14) 
};
\addplot [
color=darkgray,
solid,
forget plot
]
coordinates{
 (855.684210526316,-18)(933.473684210526,-14) 
};
\addplot [
color=darkgray,
solid,
forget plot
]
coordinates{
 (855.684210526316,-6)(933.473684210526,-16) 
};
\addplot [
color=darkgray,
solid,
forget plot
]
coordinates{
 (855.684210526316,-8)(933.473684210526,-16) 
};
\addplot [
color=darkgray,
solid,
forget plot
]
coordinates{
 (855.684210526316,-10)(933.473684210526,-16) 
};
\addplot [
color=darkgray,
solid,
forget plot
]
coordinates{
 (855.684210526316,-12)(933.473684210526,-16) 
};
\addplot [
color=darkgray,
solid,
forget plot
]
coordinates{
 (855.684210526316,-14)(933.473684210526,-16) 
};
\addplot [
color=darkgray,
solid,
forget plot
]
coordinates{
 (855.684210526316,-16)(933.473684210526,-16) 
};
\addplot [
color=darkgray,
solid,
forget plot
]
coordinates{
 (855.684210526316,-18)(933.473684210526,-16) 
};
\addplot [
color=darkgray,
solid,
forget plot
]
coordinates{
 (855.684210526316,-8)(933.473684210526,-18) 
};
\addplot [
color=darkgray,
solid,
forget plot
]
coordinates{
 (855.684210526316,-10)(933.473684210526,-18) 
};
\addplot [
color=darkgray,
solid,
forget plot
]
coordinates{
 (855.684210526316,-12)(933.473684210526,-18) 
};
\addplot [
color=darkgray,
solid,
forget plot
]
coordinates{
 (855.684210526316,-14)(933.473684210526,-18) 
};
\addplot [
color=darkgray,
solid,
forget plot
]
coordinates{
 (855.684210526316,-16)(933.473684210526,-18) 
};
\addplot [
color=darkgray,
solid,
forget plot
]
coordinates{
 (855.684210526316,-18)(933.473684210526,-18) 
};
\addplot [
color=darkgray,
solid,
forget plot
]
coordinates{
 (933.473684210526,0)(1011.26315789474,0) 
};
\addplot [
color=darkgray,
solid,
forget plot
]
coordinates{
 (933.473684210526,-2)(1011.26315789474,0) 
};
\addplot [
color=darkgray,
solid,
forget plot
]
coordinates{
 (933.473684210526,-4)(1011.26315789474,0) 
};
\addplot [
color=darkgray,
solid,
forget plot
]
coordinates{
 (933.473684210526,-6)(1011.26315789474,0) 
};
\addplot [
color=darkgray,
solid,
forget plot
]
coordinates{
 (933.473684210526,-8)(1011.26315789474,0) 
};
\addplot [
color=darkgray,
solid,
forget plot
]
coordinates{
 (933.473684210526,-10)(1011.26315789474,0) 
};
\addplot [
color=darkgray,
solid,
forget plot
]
coordinates{
 (933.473684210526,0)(1011.26315789474,-2) 
};
\addplot [
color=darkgray,
solid,
forget plot
]
coordinates{
 (933.473684210526,-2)(1011.26315789474,-2) 
};
\addplot [
color=darkgray,
solid,
forget plot
]
coordinates{
 (933.473684210526,-4)(1011.26315789474,-2) 
};
\addplot [
color=darkgray,
solid,
forget plot
]
coordinates{
 (933.473684210526,-6)(1011.26315789474,-2) 
};
\addplot [
color=darkgray,
solid,
forget plot
]
coordinates{
 (933.473684210526,-8)(1011.26315789474,-2) 
};
\addplot [
color=darkgray,
solid,
forget plot
]
coordinates{
 (933.473684210526,-10)(1011.26315789474,-2) 
};
\addplot [
color=darkgray,
solid,
forget plot
]
coordinates{
 (933.473684210526,-12)(1011.26315789474,-2) 
};
\addplot [
color=darkgray,
solid,
forget plot
]
coordinates{
 (933.473684210526,0)(1011.26315789474,-4) 
};
\addplot [
color=darkgray,
solid,
forget plot
]
coordinates{
 (933.473684210526,-2)(1011.26315789474,-4) 
};
\addplot [
color=darkgray,
solid,
forget plot
]
coordinates{
 (933.473684210526,-4)(1011.26315789474,-4) 
};
\addplot [
color=darkgray,
solid,
forget plot
]
coordinates{
 (933.473684210526,-6)(1011.26315789474,-4) 
};
\addplot [
color=darkgray,
solid,
forget plot
]
coordinates{
 (933.473684210526,-8)(1011.26315789474,-4) 
};
\addplot [
color=darkgray,
solid,
forget plot
]
coordinates{
 (933.473684210526,-10)(1011.26315789474,-4) 
};
\addplot [
color=darkgray,
solid,
forget plot
]
coordinates{
 (933.473684210526,-12)(1011.26315789474,-4) 
};
\addplot [
color=darkgray,
solid,
forget plot
]
coordinates{
 (933.473684210526,-14)(1011.26315789474,-4) 
};
\addplot [
color=darkgray,
solid,
forget plot
]
coordinates{
 (933.473684210526,0)(1011.26315789474,-6) 
};
\addplot [
color=darkgray,
solid,
forget plot
]
coordinates{
 (933.473684210526,-2)(1011.26315789474,-6) 
};
\addplot [
color=darkgray,
solid,
forget plot
]
coordinates{
 (933.473684210526,-4)(1011.26315789474,-6) 
};
\addplot [
color=darkgray,
solid,
forget plot
]
coordinates{
 (933.473684210526,-6)(1011.26315789474,-6) 
};
\addplot [
color=darkgray,
solid,
forget plot
]
coordinates{
 (933.473684210526,-8)(1011.26315789474,-6) 
};
\addplot [
color=darkgray,
solid,
forget plot
]
coordinates{
 (933.473684210526,-10)(1011.26315789474,-6) 
};
\addplot [
color=darkgray,
solid,
forget plot
]
coordinates{
 (933.473684210526,-12)(1011.26315789474,-6) 
};
\addplot [
color=darkgray,
solid,
forget plot
]
coordinates{
 (933.473684210526,-14)(1011.26315789474,-6) 
};
\addplot [
color=darkgray,
solid,
forget plot
]
coordinates{
 (933.473684210526,-16)(1011.26315789474,-6) 
};
\addplot [
color=darkgray,
solid,
forget plot
]
coordinates{
 (933.473684210526,0)(1011.26315789474,-8) 
};
\addplot [
color=darkgray,
solid,
forget plot
]
coordinates{
 (933.473684210526,-2)(1011.26315789474,-8) 
};
\addplot [
color=darkgray,
solid,
forget plot
]
coordinates{
 (933.473684210526,-4)(1011.26315789474,-8) 
};
\addplot [
color=darkgray,
solid,
forget plot
]
coordinates{
 (933.473684210526,-6)(1011.26315789474,-8) 
};
\addplot [
color=darkgray,
solid,
forget plot
]
coordinates{
 (933.473684210526,-8)(1011.26315789474,-8) 
};
\addplot [
color=darkgray,
solid,
forget plot
]
coordinates{
 (933.473684210526,-10)(1011.26315789474,-8) 
};
\addplot [
color=darkgray,
solid,
forget plot
]
coordinates{
 (933.473684210526,-12)(1011.26315789474,-8) 
};
\addplot [
color=darkgray,
solid,
forget plot
]
coordinates{
 (933.473684210526,-14)(1011.26315789474,-8) 
};
\addplot [
color=darkgray,
solid,
forget plot
]
coordinates{
 (933.473684210526,-16)(1011.26315789474,-8) 
};
\addplot [
color=darkgray,
solid,
forget plot
]
coordinates{
 (933.473684210526,-18)(1011.26315789474,-8) 
};
\addplot [
color=darkgray,
solid,
forget plot
]
coordinates{
 (933.473684210526,0)(1011.26315789474,-10) 
};
\addplot [
color=darkgray,
solid,
forget plot
]
coordinates{
 (933.473684210526,-2)(1011.26315789474,-10) 
};
\addplot [
color=darkgray,
solid,
forget plot
]
coordinates{
 (933.473684210526,-4)(1011.26315789474,-10) 
};
\addplot [
color=darkgray,
solid,
forget plot
]
coordinates{
 (933.473684210526,-6)(1011.26315789474,-10) 
};
\addplot [
color=darkgray,
solid,
forget plot
]
coordinates{
 (933.473684210526,-8)(1011.26315789474,-10) 
};
\addplot [
color=darkgray,
solid,
forget plot
]
coordinates{
 (933.473684210526,-10)(1011.26315789474,-10) 
};
\addplot [
color=darkgray,
solid,
forget plot
]
coordinates{
 (933.473684210526,-12)(1011.26315789474,-10) 
};
\addplot [
color=darkgray,
solid,
forget plot
]
coordinates{
 (933.473684210526,-14)(1011.26315789474,-10) 
};
\addplot [
color=darkgray,
solid,
forget plot
]
coordinates{
 (933.473684210526,-16)(1011.26315789474,-10) 
};
\addplot [
color=darkgray,
solid,
forget plot
]
coordinates{
 (933.473684210526,-18)(1011.26315789474,-10) 
};
\addplot [
color=darkgray,
solid,
forget plot
]
coordinates{
 (933.473684210526,-2)(1011.26315789474,-12) 
};
\addplot [
color=darkgray,
solid,
forget plot
]
coordinates{
 (933.473684210526,-4)(1011.26315789474,-12) 
};
\addplot [
color=darkgray,
solid,
forget plot
]
coordinates{
 (933.473684210526,-6)(1011.26315789474,-12) 
};
\addplot [
color=darkgray,
solid,
forget plot
]
coordinates{
 (933.473684210526,-8)(1011.26315789474,-12) 
};
\addplot [
color=darkgray,
solid,
forget plot
]
coordinates{
 (933.473684210526,-10)(1011.26315789474,-12) 
};
\addplot [
color=darkgray,
solid,
forget plot
]
coordinates{
 (933.473684210526,-12)(1011.26315789474,-12) 
};
\addplot [
color=darkgray,
solid,
forget plot
]
coordinates{
 (933.473684210526,-14)(1011.26315789474,-12) 
};
\addplot [
color=darkgray,
solid,
forget plot
]
coordinates{
 (933.473684210526,-16)(1011.26315789474,-12) 
};
\addplot [
color=darkgray,
solid,
forget plot
]
coordinates{
 (933.473684210526,-18)(1011.26315789474,-12) 
};
\addplot [
color=darkgray,
solid,
forget plot
]
coordinates{
 (933.473684210526,-4)(1011.26315789474,-14) 
};
\addplot [
color=darkgray,
solid,
forget plot
]
coordinates{
 (933.473684210526,-6)(1011.26315789474,-14) 
};
\addplot [
color=darkgray,
solid,
forget plot
]
coordinates{
 (933.473684210526,-8)(1011.26315789474,-14) 
};
\addplot [
color=darkgray,
solid,
forget plot
]
coordinates{
 (933.473684210526,-10)(1011.26315789474,-14) 
};
\addplot [
color=darkgray,
solid,
forget plot
]
coordinates{
 (933.473684210526,-12)(1011.26315789474,-14) 
};
\addplot [
color=darkgray,
solid,
forget plot
]
coordinates{
 (933.473684210526,-14)(1011.26315789474,-14) 
};
\addplot [
color=darkgray,
solid,
forget plot
]
coordinates{
 (933.473684210526,-16)(1011.26315789474,-14) 
};
\addplot [
color=darkgray,
solid,
forget plot
]
coordinates{
 (933.473684210526,-18)(1011.26315789474,-14) 
};
\addplot [
color=darkgray,
solid,
forget plot
]
coordinates{
 (933.473684210526,-6)(1011.26315789474,-16) 
};
\addplot [
color=darkgray,
solid,
forget plot
]
coordinates{
 (933.473684210526,-8)(1011.26315789474,-16) 
};
\addplot [
color=darkgray,
solid,
forget plot
]
coordinates{
 (933.473684210526,-10)(1011.26315789474,-16) 
};
\addplot [
color=darkgray,
solid,
forget plot
]
coordinates{
 (933.473684210526,-12)(1011.26315789474,-16) 
};
\addplot [
color=darkgray,
solid,
forget plot
]
coordinates{
 (933.473684210526,-14)(1011.26315789474,-16) 
};
\addplot [
color=darkgray,
solid,
forget plot
]
coordinates{
 (933.473684210526,-16)(1011.26315789474,-16) 
};
\addplot [
color=darkgray,
solid,
forget plot
]
coordinates{
 (933.473684210526,-18)(1011.26315789474,-16) 
};
\addplot [
color=darkgray,
solid,
forget plot
]
coordinates{
 (933.473684210526,-8)(1011.26315789474,-18) 
};
\addplot [
color=darkgray,
solid,
forget plot
]
coordinates{
 (933.473684210526,-10)(1011.26315789474,-18) 
};
\addplot [
color=darkgray,
solid,
forget plot
]
coordinates{
 (933.473684210526,-12)(1011.26315789474,-18) 
};
\addplot [
color=darkgray,
solid,
forget plot
]
coordinates{
 (933.473684210526,-14)(1011.26315789474,-18) 
};
\addplot [
color=darkgray,
solid,
forget plot
]
coordinates{
 (933.473684210526,-16)(1011.26315789474,-18) 
};
\addplot [
color=darkgray,
solid,
forget plot
]
coordinates{
 (933.473684210526,-18)(1011.26315789474,-18) 
};
\addplot [
color=darkgray,
solid,
forget plot
]
coordinates{
 (1011.26315789474,0)(1089.05263157895,0) 
};
\addplot [
color=darkgray,
solid,
forget plot
]
coordinates{
 (1011.26315789474,-2)(1089.05263157895,0) 
};
\addplot [
color=darkgray,
solid,
forget plot
]
coordinates{
 (1011.26315789474,-4)(1089.05263157895,0) 
};
\addplot [
color=darkgray,
solid,
forget plot
]
coordinates{
 (1011.26315789474,-6)(1089.05263157895,0) 
};
\addplot [
color=darkgray,
solid,
forget plot
]
coordinates{
 (1011.26315789474,-8)(1089.05263157895,0) 
};
\addplot [
color=darkgray,
solid,
forget plot
]
coordinates{
 (1011.26315789474,-10)(1089.05263157895,0) 
};
\addplot [
color=darkgray,
solid,
forget plot
]
coordinates{
 (1011.26315789474,0)(1089.05263157895,-2) 
};
\addplot [
color=darkgray,
solid,
forget plot
]
coordinates{
 (1011.26315789474,-2)(1089.05263157895,-2) 
};
\addplot [
color=darkgray,
solid,
forget plot
]
coordinates{
 (1011.26315789474,-4)(1089.05263157895,-2) 
};
\addplot [
color=darkgray,
solid,
forget plot
]
coordinates{
 (1011.26315789474,-6)(1089.05263157895,-2) 
};
\addplot [
color=darkgray,
solid,
forget plot
]
coordinates{
 (1011.26315789474,-8)(1089.05263157895,-2) 
};
\addplot [
color=darkgray,
solid,
forget plot
]
coordinates{
 (1011.26315789474,-10)(1089.05263157895,-2) 
};
\addplot [
color=darkgray,
solid,
forget plot
]
coordinates{
 (1011.26315789474,-12)(1089.05263157895,-2) 
};
\addplot [
color=darkgray,
solid,
forget plot
]
coordinates{
 (1011.26315789474,0)(1089.05263157895,-4) 
};
\addplot [
color=darkgray,
solid,
forget plot
]
coordinates{
 (1011.26315789474,-2)(1089.05263157895,-4) 
};
\addplot [
color=darkgray,
solid,
forget plot
]
coordinates{
 (1011.26315789474,-4)(1089.05263157895,-4) 
};
\addplot [
color=darkgray,
solid,
forget plot
]
coordinates{
 (1011.26315789474,-6)(1089.05263157895,-4) 
};
\addplot [
color=darkgray,
solid,
forget plot
]
coordinates{
 (1011.26315789474,-8)(1089.05263157895,-4) 
};
\addplot [
color=darkgray,
solid,
forget plot
]
coordinates{
 (1011.26315789474,-10)(1089.05263157895,-4) 
};
\addplot [
color=darkgray,
solid,
forget plot
]
coordinates{
 (1011.26315789474,-12)(1089.05263157895,-4) 
};
\addplot [
color=darkgray,
solid,
forget plot
]
coordinates{
 (1011.26315789474,-14)(1089.05263157895,-4) 
};
\addplot [
color=darkgray,
solid,
forget plot
]
coordinates{
 (1011.26315789474,0)(1089.05263157895,-6) 
};
\addplot [
color=darkgray,
solid,
forget plot
]
coordinates{
 (1011.26315789474,-2)(1089.05263157895,-6) 
};
\addplot [
color=darkgray,
solid,
forget plot
]
coordinates{
 (1011.26315789474,-4)(1089.05263157895,-6) 
};
\addplot [
color=darkgray,
solid,
forget plot
]
coordinates{
 (1011.26315789474,-6)(1089.05263157895,-6) 
};
\addplot [
color=darkgray,
solid,
forget plot
]
coordinates{
 (1011.26315789474,-8)(1089.05263157895,-6) 
};
\addplot [
color=darkgray,
solid,
forget plot
]
coordinates{
 (1011.26315789474,-10)(1089.05263157895,-6) 
};
\addplot [
color=darkgray,
solid,
forget plot
]
coordinates{
 (1011.26315789474,-12)(1089.05263157895,-6) 
};
\addplot [
color=darkgray,
solid,
forget plot
]
coordinates{
 (1011.26315789474,-14)(1089.05263157895,-6) 
};
\addplot [
color=darkgray,
solid,
forget plot
]
coordinates{
 (1011.26315789474,-16)(1089.05263157895,-6) 
};
\addplot [
color=darkgray,
solid,
forget plot
]
coordinates{
 (1011.26315789474,0)(1089.05263157895,-8) 
};
\addplot [
color=darkgray,
solid,
forget plot
]
coordinates{
 (1011.26315789474,-2)(1089.05263157895,-8) 
};
\addplot [
color=darkgray,
solid,
forget plot
]
coordinates{
 (1011.26315789474,-4)(1089.05263157895,-8) 
};
\addplot [
color=darkgray,
solid,
forget plot
]
coordinates{
 (1011.26315789474,-6)(1089.05263157895,-8) 
};
\addplot [
color=darkgray,
solid,
forget plot
]
coordinates{
 (1011.26315789474,-8)(1089.05263157895,-8) 
};
\addplot [
color=darkgray,
solid,
forget plot
]
coordinates{
 (1011.26315789474,-10)(1089.05263157895,-8) 
};
\addplot [
color=darkgray,
solid,
forget plot
]
coordinates{
 (1011.26315789474,-12)(1089.05263157895,-8) 
};
\addplot [
color=darkgray,
solid,
forget plot
]
coordinates{
 (1011.26315789474,-14)(1089.05263157895,-8) 
};
\addplot [
color=darkgray,
solid,
forget plot
]
coordinates{
 (1011.26315789474,-16)(1089.05263157895,-8) 
};
\addplot [
color=darkgray,
solid,
forget plot
]
coordinates{
 (1011.26315789474,-18)(1089.05263157895,-8) 
};
\addplot [
color=darkgray,
solid,
forget plot
]
coordinates{
 (1011.26315789474,0)(1089.05263157895,-10) 
};
\addplot [
color=darkgray,
solid,
forget plot
]
coordinates{
 (1011.26315789474,-2)(1089.05263157895,-10) 
};
\addplot [
color=darkgray,
solid,
forget plot
]
coordinates{
 (1011.26315789474,-4)(1089.05263157895,-10) 
};
\addplot [
color=darkgray,
solid,
forget plot
]
coordinates{
 (1011.26315789474,-6)(1089.05263157895,-10) 
};
\addplot [
color=darkgray,
solid,
forget plot
]
coordinates{
 (1011.26315789474,-8)(1089.05263157895,-10) 
};
\addplot [
color=darkgray,
solid,
forget plot
]
coordinates{
 (1011.26315789474,-10)(1089.05263157895,-10) 
};
\addplot [
color=darkgray,
solid,
forget plot
]
coordinates{
 (1011.26315789474,-12)(1089.05263157895,-10) 
};
\addplot [
color=darkgray,
solid,
forget plot
]
coordinates{
 (1011.26315789474,-14)(1089.05263157895,-10) 
};
\addplot [
color=darkgray,
solid,
forget plot
]
coordinates{
 (1011.26315789474,-16)(1089.05263157895,-10) 
};
\addplot [
color=darkgray,
solid,
forget plot
]
coordinates{
 (1011.26315789474,-18)(1089.05263157895,-10) 
};
\addplot [
color=darkgray,
solid,
forget plot
]
coordinates{
 (1011.26315789474,-2)(1089.05263157895,-12) 
};
\addplot [
color=darkgray,
solid,
forget plot
]
coordinates{
 (1011.26315789474,-4)(1089.05263157895,-12) 
};
\addplot [
color=darkgray,
solid,
forget plot
]
coordinates{
 (1011.26315789474,-6)(1089.05263157895,-12) 
};
\addplot [
color=darkgray,
solid,
forget plot
]
coordinates{
 (1011.26315789474,-8)(1089.05263157895,-12) 
};
\addplot [
color=darkgray,
solid,
forget plot
]
coordinates{
 (1011.26315789474,-10)(1089.05263157895,-12) 
};
\addplot [
color=darkgray,
solid,
forget plot
]
coordinates{
 (1011.26315789474,-12)(1089.05263157895,-12) 
};
\addplot [
color=darkgray,
solid,
forget plot
]
coordinates{
 (1011.26315789474,-14)(1089.05263157895,-12) 
};
\addplot [
color=darkgray,
solid,
forget plot
]
coordinates{
 (1011.26315789474,-16)(1089.05263157895,-12) 
};
\addplot [
color=darkgray,
solid,
forget plot
]
coordinates{
 (1011.26315789474,-18)(1089.05263157895,-12) 
};
\addplot [
color=darkgray,
solid,
forget plot
]
coordinates{
 (1011.26315789474,-4)(1089.05263157895,-14) 
};
\addplot [
color=darkgray,
solid,
forget plot
]
coordinates{
 (1011.26315789474,-6)(1089.05263157895,-14) 
};
\addplot [
color=darkgray,
solid,
forget plot
]
coordinates{
 (1011.26315789474,-8)(1089.05263157895,-14) 
};
\addplot [
color=darkgray,
solid,
forget plot
]
coordinates{
 (1011.26315789474,-10)(1089.05263157895,-14) 
};
\addplot [
color=darkgray,
solid,
forget plot
]
coordinates{
 (1011.26315789474,-12)(1089.05263157895,-14) 
};
\addplot [
color=darkgray,
solid,
forget plot
]
coordinates{
 (1011.26315789474,-14)(1089.05263157895,-14) 
};
\addplot [
color=darkgray,
solid,
forget plot
]
coordinates{
 (1011.26315789474,-16)(1089.05263157895,-14) 
};
\addplot [
color=darkgray,
solid,
forget plot
]
coordinates{
 (1011.26315789474,-18)(1089.05263157895,-14) 
};
\addplot [
color=darkgray,
solid,
forget plot
]
coordinates{
 (1011.26315789474,-6)(1089.05263157895,-16) 
};
\addplot [
color=darkgray,
solid,
forget plot
]
coordinates{
 (1011.26315789474,-8)(1089.05263157895,-16) 
};
\addplot [
color=darkgray,
solid,
forget plot
]
coordinates{
 (1011.26315789474,-10)(1089.05263157895,-16) 
};
\addplot [
color=darkgray,
solid,
forget plot
]
coordinates{
 (1011.26315789474,-12)(1089.05263157895,-16) 
};
\addplot [
color=darkgray,
solid,
forget plot
]
coordinates{
 (1011.26315789474,-14)(1089.05263157895,-16) 
};
\addplot [
color=darkgray,
solid,
forget plot
]
coordinates{
 (1011.26315789474,-16)(1089.05263157895,-16) 
};
\addplot [
color=darkgray,
solid,
forget plot
]
coordinates{
 (1011.26315789474,-18)(1089.05263157895,-16) 
};
\addplot [
color=darkgray,
solid,
forget plot
]
coordinates{
 (1011.26315789474,-8)(1089.05263157895,-18) 
};
\addplot [
color=darkgray,
solid,
forget plot
]
coordinates{
 (1011.26315789474,-10)(1089.05263157895,-18) 
};
\addplot [
color=darkgray,
solid,
forget plot
]
coordinates{
 (1011.26315789474,-12)(1089.05263157895,-18) 
};
\addplot [
color=darkgray,
solid,
forget plot
]
coordinates{
 (1011.26315789474,-14)(1089.05263157895,-18) 
};
\addplot [
color=darkgray,
solid,
forget plot
]
coordinates{
 (1011.26315789474,-16)(1089.05263157895,-18) 
};
\addplot [
color=darkgray,
solid,
forget plot
]
coordinates{
 (1011.26315789474,-18)(1089.05263157895,-18) 
};
\addplot [
color=darkgray,
solid,
forget plot
]
coordinates{
 (1089.05263157895,0)(1166.84210526316,0) 
};
\addplot [
color=darkgray,
solid,
forget plot
]
coordinates{
 (1089.05263157895,-2)(1166.84210526316,0) 
};
\addplot [
color=darkgray,
solid,
forget plot
]
coordinates{
 (1089.05263157895,-4)(1166.84210526316,0) 
};
\addplot [
color=darkgray,
solid,
forget plot
]
coordinates{
 (1089.05263157895,-6)(1166.84210526316,0) 
};
\addplot [
color=darkgray,
solid,
forget plot
]
coordinates{
 (1089.05263157895,-8)(1166.84210526316,0) 
};
\addplot [
color=darkgray,
solid,
forget plot
]
coordinates{
 (1089.05263157895,-10)(1166.84210526316,0) 
};
\addplot [
color=darkgray,
solid,
forget plot
]
coordinates{
 (1089.05263157895,0)(1166.84210526316,-2) 
};
\addplot [
color=darkgray,
solid,
forget plot
]
coordinates{
 (1089.05263157895,-2)(1166.84210526316,-2) 
};
\addplot [
color=darkgray,
solid,
forget plot
]
coordinates{
 (1089.05263157895,-4)(1166.84210526316,-2) 
};
\addplot [
color=darkgray,
solid,
forget plot
]
coordinates{
 (1089.05263157895,-6)(1166.84210526316,-2) 
};
\addplot [
color=darkgray,
solid,
forget plot
]
coordinates{
 (1089.05263157895,-8)(1166.84210526316,-2) 
};
\addplot [
color=darkgray,
solid,
forget plot
]
coordinates{
 (1089.05263157895,-10)(1166.84210526316,-2) 
};
\addplot [
color=darkgray,
solid,
forget plot
]
coordinates{
 (1089.05263157895,-12)(1166.84210526316,-2) 
};
\addplot [
color=darkgray,
solid,
forget plot
]
coordinates{
 (1089.05263157895,0)(1166.84210526316,-4) 
};
\addplot [
color=darkgray,
solid,
forget plot
]
coordinates{
 (1089.05263157895,-2)(1166.84210526316,-4) 
};
\addplot [
color=darkgray,
solid,
forget plot
]
coordinates{
 (1089.05263157895,-4)(1166.84210526316,-4) 
};
\addplot [
color=darkgray,
solid,
forget plot
]
coordinates{
 (1089.05263157895,-6)(1166.84210526316,-4) 
};
\addplot [
color=darkgray,
solid,
forget plot
]
coordinates{
 (1089.05263157895,-8)(1166.84210526316,-4) 
};
\addplot [
color=darkgray,
solid,
forget plot
]
coordinates{
 (1089.05263157895,-10)(1166.84210526316,-4) 
};
\addplot [
color=darkgray,
solid,
forget plot
]
coordinates{
 (1089.05263157895,-12)(1166.84210526316,-4) 
};
\addplot [
color=darkgray,
solid,
forget plot
]
coordinates{
 (1089.05263157895,-14)(1166.84210526316,-4) 
};
\addplot [
color=darkgray,
solid,
forget plot
]
coordinates{
 (1089.05263157895,0)(1166.84210526316,-6) 
};
\addplot [
color=darkgray,
solid,
forget plot
]
coordinates{
 (1089.05263157895,-2)(1166.84210526316,-6) 
};
\addplot [
color=darkgray,
solid,
forget plot
]
coordinates{
 (1089.05263157895,-4)(1166.84210526316,-6) 
};
\addplot [
color=darkgray,
solid,
forget plot
]
coordinates{
 (1089.05263157895,-6)(1166.84210526316,-6) 
};
\addplot [
color=darkgray,
solid,
forget plot
]
coordinates{
 (1089.05263157895,-8)(1166.84210526316,-6) 
};
\addplot [
color=darkgray,
solid,
forget plot
]
coordinates{
 (1089.05263157895,-10)(1166.84210526316,-6) 
};
\addplot [
color=darkgray,
solid,
forget plot
]
coordinates{
 (1089.05263157895,-12)(1166.84210526316,-6) 
};
\addplot [
color=darkgray,
solid,
forget plot
]
coordinates{
 (1089.05263157895,-14)(1166.84210526316,-6) 
};
\addplot [
color=darkgray,
solid,
forget plot
]
coordinates{
 (1089.05263157895,-16)(1166.84210526316,-6) 
};
\addplot [
color=darkgray,
solid,
forget plot
]
coordinates{
 (1089.05263157895,0)(1166.84210526316,-8) 
};
\addplot [
color=darkgray,
solid,
forget plot
]
coordinates{
 (1089.05263157895,-2)(1166.84210526316,-8) 
};
\addplot [
color=darkgray,
solid,
forget plot
]
coordinates{
 (1089.05263157895,-4)(1166.84210526316,-8) 
};
\addplot [
color=darkgray,
solid,
forget plot
]
coordinates{
 (1089.05263157895,-6)(1166.84210526316,-8) 
};
\addplot [
color=darkgray,
solid,
forget plot
]
coordinates{
 (1089.05263157895,-8)(1166.84210526316,-8) 
};
\addplot [
color=darkgray,
solid,
forget plot
]
coordinates{
 (1089.05263157895,-10)(1166.84210526316,-8) 
};
\addplot [
color=darkgray,
solid,
forget plot
]
coordinates{
 (1089.05263157895,-12)(1166.84210526316,-8) 
};
\addplot [
color=darkgray,
solid,
forget plot
]
coordinates{
 (1089.05263157895,-14)(1166.84210526316,-8) 
};
\addplot [
color=darkgray,
solid,
forget plot
]
coordinates{
 (1089.05263157895,-16)(1166.84210526316,-8) 
};
\addplot [
color=darkgray,
solid,
forget plot
]
coordinates{
 (1089.05263157895,-18)(1166.84210526316,-8) 
};
\addplot [
color=darkgray,
solid,
forget plot
]
coordinates{
 (1089.05263157895,0)(1166.84210526316,-10) 
};
\addplot [
color=darkgray,
solid,
forget plot
]
coordinates{
 (1089.05263157895,-2)(1166.84210526316,-10) 
};
\addplot [
color=darkgray,
solid,
forget plot
]
coordinates{
 (1089.05263157895,-4)(1166.84210526316,-10) 
};
\addplot [
color=darkgray,
solid,
forget plot
]
coordinates{
 (1089.05263157895,-6)(1166.84210526316,-10) 
};
\addplot [
color=darkgray,
solid,
forget plot
]
coordinates{
 (1089.05263157895,-8)(1166.84210526316,-10) 
};
\addplot [
color=darkgray,
solid,
forget plot
]
coordinates{
 (1089.05263157895,-10)(1166.84210526316,-10) 
};
\addplot [
color=darkgray,
solid,
forget plot
]
coordinates{
 (1089.05263157895,-12)(1166.84210526316,-10) 
};
\addplot [
color=darkgray,
solid,
forget plot
]
coordinates{
 (1089.05263157895,-14)(1166.84210526316,-10) 
};
\addplot [
color=darkgray,
solid,
forget plot
]
coordinates{
 (1089.05263157895,-16)(1166.84210526316,-10) 
};
\addplot [
color=darkgray,
solid,
forget plot
]
coordinates{
 (1089.05263157895,-18)(1166.84210526316,-10) 
};
\addplot [
color=darkgray,
solid,
forget plot
]
coordinates{
 (1089.05263157895,-2)(1166.84210526316,-12) 
};
\addplot [
color=darkgray,
solid,
forget plot
]
coordinates{
 (1089.05263157895,-4)(1166.84210526316,-12) 
};
\addplot [
color=darkgray,
solid,
forget plot
]
coordinates{
 (1089.05263157895,-6)(1166.84210526316,-12) 
};
\addplot [
color=darkgray,
solid,
forget plot
]
coordinates{
 (1089.05263157895,-8)(1166.84210526316,-12) 
};
\addplot [
color=darkgray,
solid,
forget plot
]
coordinates{
 (1089.05263157895,-10)(1166.84210526316,-12) 
};
\addplot [
color=darkgray,
solid,
forget plot
]
coordinates{
 (1089.05263157895,-12)(1166.84210526316,-12) 
};
\addplot [
color=darkgray,
solid,
forget plot
]
coordinates{
 (1089.05263157895,-14)(1166.84210526316,-12) 
};
\addplot [
color=darkgray,
solid,
forget plot
]
coordinates{
 (1089.05263157895,-16)(1166.84210526316,-12) 
};
\addplot [
color=darkgray,
solid,
forget plot
]
coordinates{
 (1089.05263157895,-18)(1166.84210526316,-12) 
};
\addplot [
color=darkgray,
solid,
forget plot
]
coordinates{
 (1089.05263157895,-4)(1166.84210526316,-14) 
};
\addplot [
color=darkgray,
solid,
forget plot
]
coordinates{
 (1089.05263157895,-6)(1166.84210526316,-14) 
};
\addplot [
color=darkgray,
solid,
forget plot
]
coordinates{
 (1089.05263157895,-8)(1166.84210526316,-14) 
};
\addplot [
color=darkgray,
solid,
forget plot
]
coordinates{
 (1089.05263157895,-10)(1166.84210526316,-14) 
};
\addplot [
color=darkgray,
solid,
forget plot
]
coordinates{
 (1089.05263157895,-12)(1166.84210526316,-14) 
};
\addplot [
color=darkgray,
solid,
forget plot
]
coordinates{
 (1089.05263157895,-14)(1166.84210526316,-14) 
};
\addplot [
color=darkgray,
solid,
forget plot
]
coordinates{
 (1089.05263157895,-16)(1166.84210526316,-14) 
};
\addplot [
color=darkgray,
solid,
forget plot
]
coordinates{
 (1089.05263157895,-18)(1166.84210526316,-14) 
};
\addplot [
color=darkgray,
solid,
forget plot
]
coordinates{
 (1089.05263157895,-6)(1166.84210526316,-16) 
};
\addplot [
color=darkgray,
solid,
forget plot
]
coordinates{
 (1089.05263157895,-8)(1166.84210526316,-16) 
};
\addplot [
color=darkgray,
solid,
forget plot
]
coordinates{
 (1089.05263157895,-10)(1166.84210526316,-16) 
};
\addplot [
color=darkgray,
solid,
forget plot
]
coordinates{
 (1089.05263157895,-12)(1166.84210526316,-16) 
};
\addplot [
color=darkgray,
solid,
forget plot
]
coordinates{
 (1089.05263157895,-14)(1166.84210526316,-16) 
};
\addplot [
color=darkgray,
solid,
forget plot
]
coordinates{
 (1089.05263157895,-16)(1166.84210526316,-16) 
};
\addplot [
color=darkgray,
solid,
forget plot
]
coordinates{
 (1089.05263157895,-18)(1166.84210526316,-16) 
};
\addplot [
color=darkgray,
solid,
forget plot
]
coordinates{
 (1089.05263157895,-8)(1166.84210526316,-18) 
};
\addplot [
color=darkgray,
solid,
forget plot
]
coordinates{
 (1089.05263157895,-10)(1166.84210526316,-18) 
};
\addplot [
color=darkgray,
solid,
forget plot
]
coordinates{
 (1089.05263157895,-12)(1166.84210526316,-18) 
};
\addplot [
color=darkgray,
solid,
forget plot
]
coordinates{
 (1089.05263157895,-14)(1166.84210526316,-18) 
};
\addplot [
color=darkgray,
solid,
forget plot
]
coordinates{
 (1089.05263157895,-16)(1166.84210526316,-18) 
};
\addplot [
color=darkgray,
solid,
forget plot
]
coordinates{
 (1089.05263157895,-18)(1166.84210526316,-18) 
};
\addplot [
color=darkgray,
solid,
forget plot
]
coordinates{
 (1166.84210526316,0)(1244.63157894737,0) 
};
\addplot [
color=darkgray,
solid,
forget plot
]
coordinates{
 (1166.84210526316,-2)(1244.63157894737,0) 
};
\addplot [
color=darkgray,
solid,
forget plot
]
coordinates{
 (1166.84210526316,-4)(1244.63157894737,0) 
};
\addplot [
color=darkgray,
solid,
forget plot
]
coordinates{
 (1166.84210526316,-6)(1244.63157894737,0) 
};
\addplot [
color=darkgray,
solid,
forget plot
]
coordinates{
 (1166.84210526316,-8)(1244.63157894737,0) 
};
\addplot [
color=darkgray,
solid,
forget plot
]
coordinates{
 (1166.84210526316,-10)(1244.63157894737,0) 
};
\addplot [
color=darkgray,
solid,
forget plot
]
coordinates{
 (1166.84210526316,0)(1244.63157894737,-2) 
};
\addplot [
color=darkgray,
solid,
forget plot
]
coordinates{
 (1166.84210526316,-2)(1244.63157894737,-2) 
};
\addplot [
color=darkgray,
solid,
forget plot
]
coordinates{
 (1166.84210526316,-4)(1244.63157894737,-2) 
};
\addplot [
color=darkgray,
solid,
forget plot
]
coordinates{
 (1166.84210526316,-6)(1244.63157894737,-2) 
};
\addplot [
color=darkgray,
solid,
forget plot
]
coordinates{
 (1166.84210526316,-8)(1244.63157894737,-2) 
};
\addplot [
color=darkgray,
solid,
forget plot
]
coordinates{
 (1166.84210526316,-10)(1244.63157894737,-2) 
};
\addplot [
color=darkgray,
solid,
forget plot
]
coordinates{
 (1166.84210526316,-12)(1244.63157894737,-2) 
};
\addplot [
color=darkgray,
solid,
forget plot
]
coordinates{
 (1166.84210526316,0)(1244.63157894737,-4) 
};
\addplot [
color=darkgray,
solid,
forget plot
]
coordinates{
 (1166.84210526316,-2)(1244.63157894737,-4) 
};
\addplot [
color=darkgray,
solid,
forget plot
]
coordinates{
 (1166.84210526316,-4)(1244.63157894737,-4) 
};
\addplot [
color=darkgray,
solid,
forget plot
]
coordinates{
 (1166.84210526316,-6)(1244.63157894737,-4) 
};
\addplot [
color=darkgray,
solid,
forget plot
]
coordinates{
 (1166.84210526316,-8)(1244.63157894737,-4) 
};
\addplot [
color=darkgray,
solid,
forget plot
]
coordinates{
 (1166.84210526316,-10)(1244.63157894737,-4) 
};
\addplot [
color=darkgray,
solid,
forget plot
]
coordinates{
 (1166.84210526316,-12)(1244.63157894737,-4) 
};
\addplot [
color=darkgray,
solid,
forget plot
]
coordinates{
 (1166.84210526316,-14)(1244.63157894737,-4) 
};
\addplot [
color=darkgray,
solid,
forget plot
]
coordinates{
 (1166.84210526316,0)(1244.63157894737,-6) 
};
\addplot [
color=darkgray,
solid,
forget plot
]
coordinates{
 (1166.84210526316,-2)(1244.63157894737,-6) 
};
\addplot [
color=darkgray,
solid,
forget plot
]
coordinates{
 (1166.84210526316,-4)(1244.63157894737,-6) 
};
\addplot [
color=darkgray,
solid,
forget plot
]
coordinates{
 (1166.84210526316,-6)(1244.63157894737,-6) 
};
\addplot [
color=darkgray,
solid,
forget plot
]
coordinates{
 (1166.84210526316,-8)(1244.63157894737,-6) 
};
\addplot [
color=darkgray,
solid,
forget plot
]
coordinates{
 (1166.84210526316,-10)(1244.63157894737,-6) 
};
\addplot [
color=darkgray,
solid,
forget plot
]
coordinates{
 (1166.84210526316,-12)(1244.63157894737,-6) 
};
\addplot [
color=darkgray,
solid,
forget plot
]
coordinates{
 (1166.84210526316,-14)(1244.63157894737,-6) 
};
\addplot [
color=darkgray,
solid,
forget plot
]
coordinates{
 (1166.84210526316,-16)(1244.63157894737,-6) 
};
\addplot [
color=darkgray,
solid,
forget plot
]
coordinates{
 (1166.84210526316,0)(1244.63157894737,-8) 
};
\addplot [
color=darkgray,
solid,
forget plot
]
coordinates{
 (1166.84210526316,-2)(1244.63157894737,-8) 
};
\addplot [
color=darkgray,
solid,
forget plot
]
coordinates{
 (1166.84210526316,-4)(1244.63157894737,-8) 
};
\addplot [
color=darkgray,
solid,
forget plot
]
coordinates{
 (1166.84210526316,-6)(1244.63157894737,-8) 
};
\addplot [
color=darkgray,
solid,
forget plot
]
coordinates{
 (1166.84210526316,-8)(1244.63157894737,-8) 
};
\addplot [
color=darkgray,
solid,
forget plot
]
coordinates{
 (1166.84210526316,-10)(1244.63157894737,-8) 
};
\addplot [
color=darkgray,
solid,
forget plot
]
coordinates{
 (1166.84210526316,-12)(1244.63157894737,-8) 
};
\addplot [
color=darkgray,
solid,
forget plot
]
coordinates{
 (1166.84210526316,-14)(1244.63157894737,-8) 
};
\addplot [
color=darkgray,
solid,
forget plot
]
coordinates{
 (1166.84210526316,-16)(1244.63157894737,-8) 
};
\addplot [
color=darkgray,
solid,
forget plot
]
coordinates{
 (1166.84210526316,-18)(1244.63157894737,-8) 
};
\addplot [
color=darkgray,
solid,
forget plot
]
coordinates{
 (1166.84210526316,0)(1244.63157894737,-10) 
};
\addplot [
color=darkgray,
solid,
forget plot
]
coordinates{
 (1166.84210526316,-2)(1244.63157894737,-10) 
};
\addplot [
color=darkgray,
solid,
forget plot
]
coordinates{
 (1166.84210526316,-4)(1244.63157894737,-10) 
};
\addplot [
color=darkgray,
solid,
forget plot
]
coordinates{
 (1166.84210526316,-6)(1244.63157894737,-10) 
};
\addplot [
color=darkgray,
solid,
forget plot
]
coordinates{
 (1166.84210526316,-8)(1244.63157894737,-10) 
};
\addplot [
color=darkgray,
solid,
forget plot
]
coordinates{
 (1166.84210526316,-10)(1244.63157894737,-10) 
};
\addplot [
color=darkgray,
solid,
forget plot
]
coordinates{
 (1166.84210526316,-12)(1244.63157894737,-10) 
};
\addplot [
color=darkgray,
solid,
forget plot
]
coordinates{
 (1166.84210526316,-14)(1244.63157894737,-10) 
};
\addplot [
color=darkgray,
solid,
forget plot
]
coordinates{
 (1166.84210526316,-16)(1244.63157894737,-10) 
};
\addplot [
color=darkgray,
solid,
forget plot
]
coordinates{
 (1166.84210526316,-18)(1244.63157894737,-10) 
};
\addplot [
color=darkgray,
solid,
forget plot
]
coordinates{
 (1166.84210526316,-2)(1244.63157894737,-12) 
};
\addplot [
color=darkgray,
solid,
forget plot
]
coordinates{
 (1166.84210526316,-4)(1244.63157894737,-12) 
};
\addplot [
color=darkgray,
solid,
forget plot
]
coordinates{
 (1166.84210526316,-6)(1244.63157894737,-12) 
};
\addplot [
color=darkgray,
solid,
forget plot
]
coordinates{
 (1166.84210526316,-8)(1244.63157894737,-12) 
};
\addplot [
color=darkgray,
solid,
forget plot
]
coordinates{
 (1166.84210526316,-10)(1244.63157894737,-12) 
};
\addplot [
color=darkgray,
solid,
forget plot
]
coordinates{
 (1166.84210526316,-12)(1244.63157894737,-12) 
};
\addplot [
color=darkgray,
solid,
forget plot
]
coordinates{
 (1166.84210526316,-14)(1244.63157894737,-12) 
};
\addplot [
color=darkgray,
solid,
forget plot
]
coordinates{
 (1166.84210526316,-16)(1244.63157894737,-12) 
};
\addplot [
color=darkgray,
solid,
forget plot
]
coordinates{
 (1166.84210526316,-18)(1244.63157894737,-12) 
};
\addplot [
color=darkgray,
solid,
forget plot
]
coordinates{
 (1166.84210526316,-4)(1244.63157894737,-14) 
};
\addplot [
color=darkgray,
solid,
forget plot
]
coordinates{
 (1166.84210526316,-6)(1244.63157894737,-14) 
};
\addplot [
color=darkgray,
solid,
forget plot
]
coordinates{
 (1166.84210526316,-8)(1244.63157894737,-14) 
};
\addplot [
color=darkgray,
solid,
forget plot
]
coordinates{
 (1166.84210526316,-10)(1244.63157894737,-14) 
};
\addplot [
color=darkgray,
solid,
forget plot
]
coordinates{
 (1166.84210526316,-12)(1244.63157894737,-14) 
};
\addplot [
color=darkgray,
solid,
forget plot
]
coordinates{
 (1166.84210526316,-14)(1244.63157894737,-14) 
};
\addplot [
color=darkgray,
solid,
forget plot
]
coordinates{
 (1166.84210526316,-16)(1244.63157894737,-14) 
};
\addplot [
color=darkgray,
solid,
forget plot
]
coordinates{
 (1166.84210526316,-18)(1244.63157894737,-14) 
};
\addplot [
color=darkgray,
solid,
forget plot
]
coordinates{
 (1166.84210526316,-6)(1244.63157894737,-16) 
};
\addplot [
color=darkgray,
solid,
forget plot
]
coordinates{
 (1166.84210526316,-8)(1244.63157894737,-16) 
};
\addplot [
color=darkgray,
solid,
forget plot
]
coordinates{
 (1166.84210526316,-10)(1244.63157894737,-16) 
};
\addplot [
color=darkgray,
solid,
forget plot
]
coordinates{
 (1166.84210526316,-12)(1244.63157894737,-16) 
};
\addplot [
color=darkgray,
solid,
forget plot
]
coordinates{
 (1166.84210526316,-14)(1244.63157894737,-16) 
};
\addplot [
color=darkgray,
solid,
forget plot
]
coordinates{
 (1166.84210526316,-16)(1244.63157894737,-16) 
};
\addplot [
color=darkgray,
solid,
forget plot
]
coordinates{
 (1166.84210526316,-18)(1244.63157894737,-16) 
};
\addplot [
color=darkgray,
solid,
forget plot
]
coordinates{
 (1166.84210526316,-8)(1244.63157894737,-18) 
};
\addplot [
color=darkgray,
solid,
forget plot
]
coordinates{
 (1166.84210526316,-10)(1244.63157894737,-18) 
};
\addplot [
color=darkgray,
solid,
forget plot
]
coordinates{
 (1166.84210526316,-12)(1244.63157894737,-18) 
};
\addplot [
color=darkgray,
solid,
forget plot
]
coordinates{
 (1166.84210526316,-14)(1244.63157894737,-18) 
};
\addplot [
color=darkgray,
solid,
forget plot
]
coordinates{
 (1166.84210526316,-16)(1244.63157894737,-18) 
};
\addplot [
color=darkgray,
solid,
forget plot
]
coordinates{
 (1166.84210526316,-18)(1244.63157894737,-18) 
};
\addplot [
color=darkgray,
solid,
forget plot
]
coordinates{
 (1244.63157894737,0)(1322.42105263158,0) 
};
\addplot [
color=darkgray,
solid,
forget plot
]
coordinates{
 (1244.63157894737,-2)(1322.42105263158,0) 
};
\addplot [
color=darkgray,
solid,
forget plot
]
coordinates{
 (1244.63157894737,-4)(1322.42105263158,0) 
};
\addplot [
color=darkgray,
solid,
forget plot
]
coordinates{
 (1244.63157894737,-6)(1322.42105263158,0) 
};
\addplot [
color=darkgray,
solid,
forget plot
]
coordinates{
 (1244.63157894737,-8)(1322.42105263158,0) 
};
\addplot [
color=darkgray,
solid,
forget plot
]
coordinates{
 (1244.63157894737,-10)(1322.42105263158,0) 
};
\addplot [
color=darkgray,
solid,
forget plot
]
coordinates{
 (1244.63157894737,0)(1322.42105263158,-2) 
};
\addplot [
color=darkgray,
solid,
forget plot
]
coordinates{
 (1244.63157894737,-2)(1322.42105263158,-2) 
};
\addplot [
color=darkgray,
solid,
forget plot
]
coordinates{
 (1244.63157894737,-4)(1322.42105263158,-2) 
};
\addplot [
color=darkgray,
solid,
forget plot
]
coordinates{
 (1244.63157894737,-6)(1322.42105263158,-2) 
};
\addplot [
color=darkgray,
solid,
forget plot
]
coordinates{
 (1244.63157894737,-8)(1322.42105263158,-2) 
};
\addplot [
color=darkgray,
solid,
forget plot
]
coordinates{
 (1244.63157894737,-10)(1322.42105263158,-2) 
};
\addplot [
color=darkgray,
solid,
forget plot
]
coordinates{
 (1244.63157894737,-12)(1322.42105263158,-2) 
};
\addplot [
color=darkgray,
solid,
forget plot
]
coordinates{
 (1244.63157894737,0)(1322.42105263158,-4) 
};
\addplot [
color=darkgray,
solid,
forget plot
]
coordinates{
 (1244.63157894737,-2)(1322.42105263158,-4) 
};
\addplot [
color=darkgray,
solid,
forget plot
]
coordinates{
 (1244.63157894737,-4)(1322.42105263158,-4) 
};
\addplot [
color=darkgray,
solid,
forget plot
]
coordinates{
 (1244.63157894737,-6)(1322.42105263158,-4) 
};
\addplot [
color=darkgray,
solid,
forget plot
]
coordinates{
 (1244.63157894737,-8)(1322.42105263158,-4) 
};
\addplot [
color=darkgray,
solid,
forget plot
]
coordinates{
 (1244.63157894737,-10)(1322.42105263158,-4) 
};
\addplot [
color=darkgray,
solid,
forget plot
]
coordinates{
 (1244.63157894737,-12)(1322.42105263158,-4) 
};
\addplot [
color=darkgray,
solid,
forget plot
]
coordinates{
 (1244.63157894737,-14)(1322.42105263158,-4) 
};
\addplot [
color=darkgray,
solid,
forget plot
]
coordinates{
 (1244.63157894737,0)(1322.42105263158,-6) 
};
\addplot [
color=darkgray,
solid,
forget plot
]
coordinates{
 (1244.63157894737,-2)(1322.42105263158,-6) 
};
\addplot [
color=darkgray,
solid,
forget plot
]
coordinates{
 (1244.63157894737,-4)(1322.42105263158,-6) 
};
\addplot [
color=darkgray,
solid,
forget plot
]
coordinates{
 (1244.63157894737,-6)(1322.42105263158,-6) 
};
\addplot [
color=darkgray,
solid,
forget plot
]
coordinates{
 (1244.63157894737,-8)(1322.42105263158,-6) 
};
\addplot [
color=darkgray,
solid,
forget plot
]
coordinates{
 (1244.63157894737,-10)(1322.42105263158,-6) 
};
\addplot [
color=darkgray,
solid,
forget plot
]
coordinates{
 (1244.63157894737,-12)(1322.42105263158,-6) 
};
\addplot [
color=darkgray,
solid,
forget plot
]
coordinates{
 (1244.63157894737,-14)(1322.42105263158,-6) 
};
\addplot [
color=darkgray,
solid,
forget plot
]
coordinates{
 (1244.63157894737,-16)(1322.42105263158,-6) 
};
\addplot [
color=darkgray,
solid,
forget plot
]
coordinates{
 (1244.63157894737,0)(1322.42105263158,-8) 
};
\addplot [
color=darkgray,
solid,
forget plot
]
coordinates{
 (1244.63157894737,-2)(1322.42105263158,-8) 
};
\addplot [
color=darkgray,
solid,
forget plot
]
coordinates{
 (1244.63157894737,-4)(1322.42105263158,-8) 
};
\addplot [
color=darkgray,
solid,
forget plot
]
coordinates{
 (1244.63157894737,-6)(1322.42105263158,-8) 
};
\addplot [
color=darkgray,
solid,
forget plot
]
coordinates{
 (1244.63157894737,-8)(1322.42105263158,-8) 
};
\addplot [
color=darkgray,
solid,
forget plot
]
coordinates{
 (1244.63157894737,-10)(1322.42105263158,-8) 
};
\addplot [
color=darkgray,
solid,
forget plot
]
coordinates{
 (1244.63157894737,-12)(1322.42105263158,-8) 
};
\addplot [
color=darkgray,
solid,
forget plot
]
coordinates{
 (1244.63157894737,-14)(1322.42105263158,-8) 
};
\addplot [
color=darkgray,
solid,
forget plot
]
coordinates{
 (1244.63157894737,-16)(1322.42105263158,-8) 
};
\addplot [
color=darkgray,
solid,
forget plot
]
coordinates{
 (1244.63157894737,-18)(1322.42105263158,-8) 
};
\addplot [
color=darkgray,
solid,
forget plot
]
coordinates{
 (1244.63157894737,0)(1322.42105263158,-10) 
};
\addplot [
color=darkgray,
solid,
forget plot
]
coordinates{
 (1244.63157894737,-2)(1322.42105263158,-10) 
};
\addplot [
color=darkgray,
solid,
forget plot
]
coordinates{
 (1244.63157894737,-4)(1322.42105263158,-10) 
};
\addplot [
color=darkgray,
solid,
forget plot
]
coordinates{
 (1244.63157894737,-6)(1322.42105263158,-10) 
};
\addplot [
color=darkgray,
solid,
forget plot
]
coordinates{
 (1244.63157894737,-8)(1322.42105263158,-10) 
};
\addplot [
color=darkgray,
solid,
forget plot
]
coordinates{
 (1244.63157894737,-10)(1322.42105263158,-10) 
};
\addplot [
color=darkgray,
solid,
forget plot
]
coordinates{
 (1244.63157894737,-12)(1322.42105263158,-10) 
};
\addplot [
color=darkgray,
solid,
forget plot
]
coordinates{
 (1244.63157894737,-14)(1322.42105263158,-10) 
};
\addplot [
color=darkgray,
solid,
forget plot
]
coordinates{
 (1244.63157894737,-16)(1322.42105263158,-10) 
};
\addplot [
color=darkgray,
solid,
forget plot
]
coordinates{
 (1244.63157894737,-18)(1322.42105263158,-10) 
};
\addplot [
color=darkgray,
solid,
forget plot
]
coordinates{
 (1244.63157894737,-2)(1322.42105263158,-12) 
};
\addplot [
color=darkgray,
solid,
forget plot
]
coordinates{
 (1244.63157894737,-4)(1322.42105263158,-12) 
};
\addplot [
color=darkgray,
solid,
forget plot
]
coordinates{
 (1244.63157894737,-6)(1322.42105263158,-12) 
};
\addplot [
color=darkgray,
solid,
forget plot
]
coordinates{
 (1244.63157894737,-8)(1322.42105263158,-12) 
};
\addplot [
color=darkgray,
solid,
forget plot
]
coordinates{
 (1244.63157894737,-10)(1322.42105263158,-12) 
};
\addplot [
color=darkgray,
solid,
forget plot
]
coordinates{
 (1244.63157894737,-12)(1322.42105263158,-12) 
};
\addplot [
color=darkgray,
solid,
forget plot
]
coordinates{
 (1244.63157894737,-14)(1322.42105263158,-12) 
};
\addplot [
color=darkgray,
solid,
forget plot
]
coordinates{
 (1244.63157894737,-16)(1322.42105263158,-12) 
};
\addplot [
color=darkgray,
solid,
forget plot
]
coordinates{
 (1244.63157894737,-18)(1322.42105263158,-12) 
};
\addplot [
color=darkgray,
solid,
forget plot
]
coordinates{
 (1244.63157894737,-4)(1322.42105263158,-14) 
};
\addplot [
color=darkgray,
solid,
forget plot
]
coordinates{
 (1244.63157894737,-6)(1322.42105263158,-14) 
};
\addplot [
color=darkgray,
solid,
forget plot
]
coordinates{
 (1244.63157894737,-8)(1322.42105263158,-14) 
};
\addplot [
color=darkgray,
solid,
forget plot
]
coordinates{
 (1244.63157894737,-10)(1322.42105263158,-14) 
};
\addplot [
color=darkgray,
solid,
forget plot
]
coordinates{
 (1244.63157894737,-12)(1322.42105263158,-14) 
};
\addplot [
color=darkgray,
solid,
forget plot
]
coordinates{
 (1244.63157894737,-14)(1322.42105263158,-14) 
};
\addplot [
color=darkgray,
solid,
forget plot
]
coordinates{
 (1244.63157894737,-16)(1322.42105263158,-14) 
};
\addplot [
color=darkgray,
solid,
forget plot
]
coordinates{
 (1244.63157894737,-18)(1322.42105263158,-14) 
};
\addplot [
color=darkgray,
solid,
forget plot
]
coordinates{
 (1244.63157894737,-6)(1322.42105263158,-16) 
};
\addplot [
color=darkgray,
solid,
forget plot
]
coordinates{
 (1244.63157894737,-8)(1322.42105263158,-16) 
};
\addplot [
color=darkgray,
solid,
forget plot
]
coordinates{
 (1244.63157894737,-10)(1322.42105263158,-16) 
};
\addplot [
color=darkgray,
solid,
forget plot
]
coordinates{
 (1244.63157894737,-12)(1322.42105263158,-16) 
};
\addplot [
color=darkgray,
solid,
forget plot
]
coordinates{
 (1244.63157894737,-14)(1322.42105263158,-16) 
};
\addplot [
color=darkgray,
solid,
forget plot
]
coordinates{
 (1244.63157894737,-16)(1322.42105263158,-16) 
};
\addplot [
color=darkgray,
solid,
forget plot
]
coordinates{
 (1244.63157894737,-18)(1322.42105263158,-16) 
};
\addplot [
color=darkgray,
solid,
forget plot
]
coordinates{
 (1244.63157894737,-8)(1322.42105263158,-18) 
};
\addplot [
color=darkgray,
solid,
forget plot
]
coordinates{
 (1244.63157894737,-10)(1322.42105263158,-18) 
};
\addplot [
color=darkgray,
solid,
forget plot
]
coordinates{
 (1244.63157894737,-12)(1322.42105263158,-18) 
};
\addplot [
color=darkgray,
solid,
forget plot
]
coordinates{
 (1244.63157894737,-14)(1322.42105263158,-18) 
};
\addplot [
color=darkgray,
solid,
forget plot
]
coordinates{
 (1244.63157894737,-16)(1322.42105263158,-18) 
};
\addplot [
color=darkgray,
solid,
forget plot
]
coordinates{
 (1244.63157894737,-18)(1322.42105263158,-18) 
};
\addplot [
color=darkgray,
solid,
forget plot
]
coordinates{
 (1322.42105263158,0)(1400.21052631579,0) 
};
\addplot [
color=darkgray,
solid,
forget plot
]
coordinates{
 (1322.42105263158,-2)(1400.21052631579,0) 
};
\addplot [
color=darkgray,
solid,
forget plot
]
coordinates{
 (1322.42105263158,-4)(1400.21052631579,0) 
};
\addplot [
color=darkgray,
solid,
forget plot
]
coordinates{
 (1322.42105263158,-6)(1400.21052631579,0) 
};
\addplot [
color=darkgray,
solid,
forget plot
]
coordinates{
 (1322.42105263158,-8)(1400.21052631579,0) 
};
\addplot [
color=darkgray,
solid,
forget plot
]
coordinates{
 (1322.42105263158,-10)(1400.21052631579,0) 
};
\addplot [
color=darkgray,
solid,
forget plot
]
coordinates{
 (1322.42105263158,0)(1400.21052631579,-2) 
};
\addplot [
color=darkgray,
solid,
forget plot
]
coordinates{
 (1322.42105263158,-2)(1400.21052631579,-2) 
};
\addplot [
color=darkgray,
solid,
forget plot
]
coordinates{
 (1322.42105263158,-4)(1400.21052631579,-2) 
};
\addplot [
color=darkgray,
solid,
forget plot
]
coordinates{
 (1322.42105263158,-6)(1400.21052631579,-2) 
};
\addplot [
color=darkgray,
solid,
forget plot
]
coordinates{
 (1322.42105263158,-8)(1400.21052631579,-2) 
};
\addplot [
color=darkgray,
solid,
forget plot
]
coordinates{
 (1322.42105263158,-10)(1400.21052631579,-2) 
};
\addplot [
color=darkgray,
solid,
forget plot
]
coordinates{
 (1322.42105263158,-12)(1400.21052631579,-2) 
};
\addplot [
color=darkgray,
solid,
forget plot
]
coordinates{
 (1322.42105263158,0)(1400.21052631579,-4) 
};
\addplot [
color=darkgray,
solid,
forget plot
]
coordinates{
 (1322.42105263158,-2)(1400.21052631579,-4) 
};
\addplot [
color=darkgray,
solid,
forget plot
]
coordinates{
 (1322.42105263158,-4)(1400.21052631579,-4) 
};
\addplot [
color=darkgray,
solid,
forget plot
]
coordinates{
 (1322.42105263158,-6)(1400.21052631579,-4) 
};
\addplot [
color=darkgray,
solid,
forget plot
]
coordinates{
 (1322.42105263158,-8)(1400.21052631579,-4) 
};
\addplot [
color=darkgray,
solid,
forget plot
]
coordinates{
 (1322.42105263158,-10)(1400.21052631579,-4) 
};
\addplot [
color=darkgray,
solid,
forget plot
]
coordinates{
 (1322.42105263158,-12)(1400.21052631579,-4) 
};
\addplot [
color=darkgray,
solid,
forget plot
]
coordinates{
 (1322.42105263158,-14)(1400.21052631579,-4) 
};
\addplot [
color=darkgray,
solid,
forget plot
]
coordinates{
 (1322.42105263158,0)(1400.21052631579,-6) 
};
\addplot [
color=darkgray,
solid,
forget plot
]
coordinates{
 (1322.42105263158,-2)(1400.21052631579,-6) 
};
\addplot [
color=darkgray,
solid,
forget plot
]
coordinates{
 (1322.42105263158,-4)(1400.21052631579,-6) 
};
\addplot [
color=darkgray,
solid,
forget plot
]
coordinates{
 (1322.42105263158,-6)(1400.21052631579,-6) 
};
\addplot [
color=darkgray,
solid,
forget plot
]
coordinates{
 (1322.42105263158,-8)(1400.21052631579,-6) 
};
\addplot [
color=darkgray,
solid,
forget plot
]
coordinates{
 (1322.42105263158,-10)(1400.21052631579,-6) 
};
\addplot [
color=darkgray,
solid,
forget plot
]
coordinates{
 (1322.42105263158,-12)(1400.21052631579,-6) 
};
\addplot [
color=darkgray,
solid,
forget plot
]
coordinates{
 (1322.42105263158,-14)(1400.21052631579,-6) 
};
\addplot [
color=darkgray,
solid,
forget plot
]
coordinates{
 (1322.42105263158,-16)(1400.21052631579,-6) 
};
\addplot [
color=darkgray,
solid,
forget plot
]
coordinates{
 (1322.42105263158,0)(1400.21052631579,-8) 
};
\addplot [
color=darkgray,
solid,
forget plot
]
coordinates{
 (1322.42105263158,-2)(1400.21052631579,-8) 
};
\addplot [
color=darkgray,
solid,
forget plot
]
coordinates{
 (1322.42105263158,-4)(1400.21052631579,-8) 
};
\addplot [
color=darkgray,
solid,
forget plot
]
coordinates{
 (1322.42105263158,-6)(1400.21052631579,-8) 
};
\addplot [
color=darkgray,
solid,
forget plot
]
coordinates{
 (1322.42105263158,-8)(1400.21052631579,-8) 
};
\addplot [
color=darkgray,
solid,
forget plot
]
coordinates{
 (1322.42105263158,-10)(1400.21052631579,-8) 
};
\addplot [
color=darkgray,
solid,
forget plot
]
coordinates{
 (1322.42105263158,-12)(1400.21052631579,-8) 
};
\addplot [
color=darkgray,
solid,
forget plot
]
coordinates{
 (1322.42105263158,-14)(1400.21052631579,-8) 
};
\addplot [
color=darkgray,
solid,
forget plot
]
coordinates{
 (1322.42105263158,-16)(1400.21052631579,-8) 
};
\addplot [
color=darkgray,
solid,
forget plot
]
coordinates{
 (1322.42105263158,-18)(1400.21052631579,-8) 
};
\addplot [
color=darkgray,
solid,
forget plot
]
coordinates{
 (1322.42105263158,0)(1400.21052631579,-10) 
};
\addplot [
color=darkgray,
solid,
forget plot
]
coordinates{
 (1322.42105263158,-2)(1400.21052631579,-10) 
};
\addplot [
color=darkgray,
solid,
forget plot
]
coordinates{
 (1322.42105263158,-4)(1400.21052631579,-10) 
};
\addplot [
color=darkgray,
solid,
forget plot
]
coordinates{
 (1322.42105263158,-6)(1400.21052631579,-10) 
};
\addplot [
color=darkgray,
solid,
forget plot
]
coordinates{
 (1322.42105263158,-8)(1400.21052631579,-10) 
};
\addplot [
color=darkgray,
solid,
forget plot
]
coordinates{
 (1322.42105263158,-10)(1400.21052631579,-10) 
};
\addplot [
color=darkgray,
solid,
forget plot
]
coordinates{
 (1322.42105263158,-12)(1400.21052631579,-10) 
};
\addplot [
color=darkgray,
solid,
forget plot
]
coordinates{
 (1322.42105263158,-14)(1400.21052631579,-10) 
};
\addplot [
color=darkgray,
solid,
forget plot
]
coordinates{
 (1322.42105263158,-16)(1400.21052631579,-10) 
};
\addplot [
color=darkgray,
solid,
forget plot
]
coordinates{
 (1322.42105263158,-18)(1400.21052631579,-10) 
};
\addplot [
color=darkgray,
solid,
forget plot
]
coordinates{
 (1322.42105263158,-2)(1400.21052631579,-12) 
};
\addplot [
color=darkgray,
solid,
forget plot
]
coordinates{
 (1322.42105263158,-4)(1400.21052631579,-12) 
};
\addplot [
color=darkgray,
solid,
forget plot
]
coordinates{
 (1322.42105263158,-6)(1400.21052631579,-12) 
};
\addplot [
color=darkgray,
solid,
forget plot
]
coordinates{
 (1322.42105263158,-8)(1400.21052631579,-12) 
};
\addplot [
color=darkgray,
solid,
forget plot
]
coordinates{
 (1322.42105263158,-10)(1400.21052631579,-12) 
};
\addplot [
color=darkgray,
solid,
forget plot
]
coordinates{
 (1322.42105263158,-12)(1400.21052631579,-12) 
};
\addplot [
color=darkgray,
solid,
forget plot
]
coordinates{
 (1322.42105263158,-14)(1400.21052631579,-12) 
};
\addplot [
color=darkgray,
solid,
forget plot
]
coordinates{
 (1322.42105263158,-16)(1400.21052631579,-12) 
};
\addplot [
color=darkgray,
solid,
forget plot
]
coordinates{
 (1322.42105263158,-18)(1400.21052631579,-12) 
};
\addplot [
color=darkgray,
solid,
forget plot
]
coordinates{
 (1322.42105263158,-4)(1400.21052631579,-14) 
};
\addplot [
color=darkgray,
solid,
forget plot
]
coordinates{
 (1322.42105263158,-6)(1400.21052631579,-14) 
};
\addplot [
color=darkgray,
solid,
forget plot
]
coordinates{
 (1322.42105263158,-8)(1400.21052631579,-14) 
};
\addplot [
color=darkgray,
solid,
forget plot
]
coordinates{
 (1322.42105263158,-10)(1400.21052631579,-14) 
};
\addplot [
color=darkgray,
solid,
forget plot
]
coordinates{
 (1322.42105263158,-12)(1400.21052631579,-14) 
};
\addplot [
color=darkgray,
solid,
forget plot
]
coordinates{
 (1322.42105263158,-14)(1400.21052631579,-14) 
};
\addplot [
color=darkgray,
solid,
forget plot
]
coordinates{
 (1322.42105263158,-16)(1400.21052631579,-14) 
};
\addplot [
color=darkgray,
solid,
forget plot
]
coordinates{
 (1322.42105263158,-18)(1400.21052631579,-14) 
};
\addplot [
color=darkgray,
solid,
forget plot
]
coordinates{
 (1322.42105263158,-6)(1400.21052631579,-16) 
};
\addplot [
color=darkgray,
solid,
forget plot
]
coordinates{
 (1322.42105263158,-8)(1400.21052631579,-16) 
};
\addplot [
color=darkgray,
solid,
forget plot
]
coordinates{
 (1322.42105263158,-10)(1400.21052631579,-16) 
};
\addplot [
color=darkgray,
solid,
forget plot
]
coordinates{
 (1322.42105263158,-12)(1400.21052631579,-16) 
};
\addplot [
color=darkgray,
solid,
forget plot
]
coordinates{
 (1322.42105263158,-14)(1400.21052631579,-16) 
};
\addplot [
color=darkgray,
solid,
forget plot
]
coordinates{
 (1322.42105263158,-16)(1400.21052631579,-16) 
};
\addplot [
color=darkgray,
solid,
forget plot
]
coordinates{
 (1322.42105263158,-18)(1400.21052631579,-16) 
};
\addplot [
color=darkgray,
solid,
forget plot
]
coordinates{
 (1322.42105263158,-8)(1400.21052631579,-18) 
};
\addplot [
color=darkgray,
solid,
forget plot
]
coordinates{
 (1322.42105263158,-10)(1400.21052631579,-18) 
};
\addplot [
color=darkgray,
solid,
forget plot
]
coordinates{
 (1322.42105263158,-12)(1400.21052631579,-18) 
};
\addplot [
color=darkgray,
solid,
forget plot
]
coordinates{
 (1322.42105263158,-14)(1400.21052631579,-18) 
};
\addplot [
color=darkgray,
solid,
forget plot
]
coordinates{
 (1322.42105263158,-16)(1400.21052631579,-18) 
};
\addplot [
color=darkgray,
solid,
forget plot
]
coordinates{
 (1322.42105263158,-18)(1400.21052631579,-18) 
};
\addplot [
color=darkgray,
solid,
forget plot
]
coordinates{
 (1400.21052631579,0)(1478,0) 
};
\addplot [
color=darkgray,
solid,
forget plot
]
coordinates{
 (1400.21052631579,-2)(1478,0) 
};
\addplot [
color=darkgray,
solid,
forget plot
]
coordinates{
 (1400.21052631579,-4)(1478,0) 
};
\addplot [
color=darkgray,
solid,
forget plot
]
coordinates{
 (1400.21052631579,-6)(1478,0) 
};
\addplot [
color=darkgray,
solid,
forget plot
]
coordinates{
 (1400.21052631579,-8)(1478,0) 
};
\addplot [
color=darkgray,
solid,
forget plot
]
coordinates{
 (1400.21052631579,-10)(1478,0) 
};
\addplot [
color=darkgray,
solid,
forget plot
]
coordinates{
 (1400.21052631579,0)(1478,-2) 
};
\addplot [
color=darkgray,
solid,
forget plot
]
coordinates{
 (1400.21052631579,-2)(1478,-2) 
};
\addplot [
color=darkgray,
solid,
forget plot
]
coordinates{
 (1400.21052631579,-4)(1478,-2) 
};
\addplot [
color=darkgray,
solid,
forget plot
]
coordinates{
 (1400.21052631579,-6)(1478,-2) 
};
\addplot [
color=darkgray,
solid,
forget plot
]
coordinates{
 (1400.21052631579,-8)(1478,-2) 
};
\addplot [
color=darkgray,
solid,
forget plot
]
coordinates{
 (1400.21052631579,-10)(1478,-2) 
};
\addplot [
color=darkgray,
solid,
forget plot
]
coordinates{
 (1400.21052631579,-12)(1478,-2) 
};
\addplot [
color=darkgray,
solid,
forget plot
]
coordinates{
 (1400.21052631579,0)(1478,-4) 
};
\addplot [
color=darkgray,
solid,
forget plot
]
coordinates{
 (1400.21052631579,-2)(1478,-4) 
};
\addplot [
color=darkgray,
solid,
forget plot
]
coordinates{
 (1400.21052631579,-4)(1478,-4) 
};
\addplot [
color=darkgray,
solid,
forget plot
]
coordinates{
 (1400.21052631579,-6)(1478,-4) 
};
\addplot [
color=darkgray,
solid,
forget plot
]
coordinates{
 (1400.21052631579,-8)(1478,-4) 
};
\addplot [
color=darkgray,
solid,
forget plot
]
coordinates{
 (1400.21052631579,-10)(1478,-4) 
};
\addplot [
color=darkgray,
solid,
forget plot
]
coordinates{
 (1400.21052631579,-12)(1478,-4) 
};
\addplot [
color=darkgray,
solid,
forget plot
]
coordinates{
 (1400.21052631579,-14)(1478,-4) 
};
\addplot [
color=darkgray,
solid,
forget plot
]
coordinates{
 (1400.21052631579,0)(1478,-6) 
};
\addplot [
color=darkgray,
solid,
forget plot
]
coordinates{
 (1400.21052631579,-2)(1478,-6) 
};
\addplot [
color=darkgray,
solid,
forget plot
]
coordinates{
 (1400.21052631579,-4)(1478,-6) 
};
\addplot [
color=darkgray,
solid,
forget plot
]
coordinates{
 (1400.21052631579,-6)(1478,-6) 
};
\addplot [
color=darkgray,
solid,
forget plot
]
coordinates{
 (1400.21052631579,-8)(1478,-6) 
};
\addplot [
color=darkgray,
solid,
forget plot
]
coordinates{
 (1400.21052631579,-10)(1478,-6) 
};
\addplot [
color=darkgray,
solid,
forget plot
]
coordinates{
 (1400.21052631579,-12)(1478,-6) 
};
\addplot [
color=darkgray,
solid,
forget plot
]
coordinates{
 (1400.21052631579,-14)(1478,-6) 
};
\addplot [
color=darkgray,
solid,
forget plot
]
coordinates{
 (1400.21052631579,-16)(1478,-6) 
};
\addplot [
color=darkgray,
solid,
forget plot
]
coordinates{
 (1400.21052631579,0)(1478,-8) 
};
\addplot [
color=darkgray,
solid,
forget plot
]
coordinates{
 (1400.21052631579,-2)(1478,-8) 
};
\addplot [
color=darkgray,
solid,
forget plot
]
coordinates{
 (1400.21052631579,-4)(1478,-8) 
};
\addplot [
color=darkgray,
solid,
forget plot
]
coordinates{
 (1400.21052631579,-6)(1478,-8) 
};
\addplot [
color=darkgray,
solid,
forget plot
]
coordinates{
 (1400.21052631579,-8)(1478,-8) 
};
\addplot [
color=darkgray,
solid,
forget plot
]
coordinates{
 (1400.21052631579,-10)(1478,-8) 
};
\addplot [
color=darkgray,
solid,
forget plot
]
coordinates{
 (1400.21052631579,-12)(1478,-8) 
};
\addplot [
color=darkgray,
solid,
forget plot
]
coordinates{
 (1400.21052631579,-14)(1478,-8) 
};
\addplot [
color=darkgray,
solid,
forget plot
]
coordinates{
 (1400.21052631579,-16)(1478,-8) 
};
\addplot [
color=darkgray,
solid,
forget plot
]
coordinates{
 (1400.21052631579,-18)(1478,-8) 
};
\addplot [
color=darkgray,
solid,
forget plot
]
coordinates{
 (1400.21052631579,0)(1478,-10) 
};
\addplot [
color=darkgray,
solid,
forget plot
]
coordinates{
 (1400.21052631579,-2)(1478,-10) 
};
\addplot [
color=darkgray,
solid,
forget plot
]
coordinates{
 (1400.21052631579,-4)(1478,-10) 
};
\addplot [
color=darkgray,
solid,
forget plot
]
coordinates{
 (1400.21052631579,-6)(1478,-10) 
};
\addplot [
color=darkgray,
solid,
forget plot
]
coordinates{
 (1400.21052631579,-8)(1478,-10) 
};
\addplot [
color=darkgray,
solid,
forget plot
]
coordinates{
 (1400.21052631579,-10)(1478,-10) 
};
\addplot [
color=darkgray,
solid,
forget plot
]
coordinates{
 (1400.21052631579,-12)(1478,-10) 
};
\addplot [
color=darkgray,
solid,
forget plot
]
coordinates{
 (1400.21052631579,-14)(1478,-10) 
};
\addplot [
color=darkgray,
solid,
forget plot
]
coordinates{
 (1400.21052631579,-16)(1478,-10) 
};
\addplot [
color=darkgray,
solid,
forget plot
]
coordinates{
 (1400.21052631579,-18)(1478,-10) 
};
\addplot [
color=darkgray,
solid,
forget plot
]
coordinates{
 (1400.21052631579,-2)(1478,-12) 
};
\addplot [
color=darkgray,
solid,
forget plot
]
coordinates{
 (1400.21052631579,-4)(1478,-12) 
};
\addplot [
color=darkgray,
solid,
forget plot
]
coordinates{
 (1400.21052631579,-6)(1478,-12) 
};
\addplot [
color=darkgray,
solid,
forget plot
]
coordinates{
 (1400.21052631579,-8)(1478,-12) 
};
\addplot [
color=darkgray,
solid,
forget plot
]
coordinates{
 (1400.21052631579,-10)(1478,-12) 
};
\addplot [
color=darkgray,
solid,
forget plot
]
coordinates{
 (1400.21052631579,-12)(1478,-12) 
};
\addplot [
color=darkgray,
solid,
forget plot
]
coordinates{
 (1400.21052631579,-14)(1478,-12) 
};
\addplot [
color=darkgray,
solid,
forget plot
]
coordinates{
 (1400.21052631579,-16)(1478,-12) 
};
\addplot [
color=darkgray,
solid,
forget plot
]
coordinates{
 (1400.21052631579,-18)(1478,-12) 
};
\addplot [
color=darkgray,
solid,
forget plot
]
coordinates{
 (1400.21052631579,-4)(1478,-14) 
};
\addplot [
color=darkgray,
solid,
forget plot
]
coordinates{
 (1400.21052631579,-6)(1478,-14) 
};
\addplot [
color=darkgray,
solid,
forget plot
]
coordinates{
 (1400.21052631579,-8)(1478,-14) 
};
\addplot [
color=darkgray,
solid,
forget plot
]
coordinates{
 (1400.21052631579,-10)(1478,-14) 
};
\addplot [
color=darkgray,
solid,
forget plot
]
coordinates{
 (1400.21052631579,-12)(1478,-14) 
};
\addplot [
color=darkgray,
solid,
forget plot
]
coordinates{
 (1400.21052631579,-14)(1478,-14) 
};
\addplot [
color=darkgray,
solid,
forget plot
]
coordinates{
 (1400.21052631579,-16)(1478,-14) 
};
\addplot [
color=darkgray,
solid,
forget plot
]
coordinates{
 (1400.21052631579,-18)(1478,-14) 
};
\addplot [
color=darkgray,
solid,
forget plot
]
coordinates{
 (1400.21052631579,-6)(1478,-16) 
};
\addplot [
color=darkgray,
solid,
forget plot
]
coordinates{
 (1400.21052631579,-8)(1478,-16) 
};
\addplot [
color=darkgray,
solid,
forget plot
]
coordinates{
 (1400.21052631579,-10)(1478,-16) 
};
\addplot [
color=darkgray,
solid,
forget plot
]
coordinates{
 (1400.21052631579,-12)(1478,-16) 
};
\addplot [
color=darkgray,
solid,
forget plot
]
coordinates{
 (1400.21052631579,-14)(1478,-16) 
};
\addplot [
color=darkgray,
solid,
forget plot
]
coordinates{
 (1400.21052631579,-16)(1478,-16) 
};
\addplot [
color=darkgray,
solid,
forget plot
]
coordinates{
 (1400.21052631579,-18)(1478,-16) 
};
\addplot [
color=darkgray,
solid,
forget plot
]
coordinates{
 (1400.21052631579,-8)(1478,-18) 
};
\addplot [
color=darkgray,
solid,
forget plot
]
coordinates{
 (1400.21052631579,-10)(1478,-18) 
};
\addplot [
color=darkgray,
solid,
forget plot
]
coordinates{
 (1400.21052631579,-12)(1478,-18) 
};
\addplot [
color=darkgray,
solid,
forget plot
]
coordinates{
 (1400.21052631579,-14)(1478,-18) 
};
\addplot [
color=darkgray,
solid,
forget plot
]
coordinates{
 (1400.21052631579,-16)(1478,-18) 
};
\addplot [
color=darkgray,
solid,
forget plot
]
coordinates{
 (1400.21052631579,-18)(1478,-18) 
};
\addplot [
color=darkgray,
solid,
forget plot
]
coordinates{
 (1478,0)(1400.21052631579,0) 
};
\addplot [
color=darkgray,
solid,
forget plot
]
coordinates{
 (1478,-2)(1400.21052631579,0) 
};
\addplot [
color=darkgray,
solid,
forget plot
]
coordinates{
 (1478,-4)(1400.21052631579,0) 
};
\addplot [
color=darkgray,
solid,
forget plot
]
coordinates{
 (1478,-6)(1400.21052631579,0) 
};
\addplot [
color=darkgray,
solid,
forget plot
]
coordinates{
 (1478,-8)(1400.21052631579,0) 
};
\addplot [
color=darkgray,
solid,
forget plot
]
coordinates{
 (1478,-10)(1400.21052631579,0) 
};
\addplot [
color=darkgray,
solid,
forget plot
]
coordinates{
 (1478,0)(1400.21052631579,-2) 
};
\addplot [
color=darkgray,
solid,
forget plot
]
coordinates{
 (1478,-2)(1400.21052631579,-2) 
};
\addplot [
color=darkgray,
solid,
forget plot
]
coordinates{
 (1478,-4)(1400.21052631579,-2) 
};
\addplot [
color=darkgray,
solid,
forget plot
]
coordinates{
 (1478,-6)(1400.21052631579,-2) 
};
\addplot [
color=darkgray,
solid,
forget plot
]
coordinates{
 (1478,-8)(1400.21052631579,-2) 
};
\addplot [
color=darkgray,
solid,
forget plot
]
coordinates{
 (1478,-10)(1400.21052631579,-2) 
};
\addplot [
color=darkgray,
solid,
forget plot
]
coordinates{
 (1478,-12)(1400.21052631579,-2) 
};
\addplot [
color=darkgray,
solid,
forget plot
]
coordinates{
 (1478,0)(1400.21052631579,-4) 
};
\addplot [
color=darkgray,
solid,
forget plot
]
coordinates{
 (1478,-2)(1400.21052631579,-4) 
};
\addplot [
color=darkgray,
solid,
forget plot
]
coordinates{
 (1478,-4)(1400.21052631579,-4) 
};
\addplot [
color=darkgray,
solid,
forget plot
]
coordinates{
 (1478,-6)(1400.21052631579,-4) 
};
\addplot [
color=darkgray,
solid,
forget plot
]
coordinates{
 (1478,-8)(1400.21052631579,-4) 
};
\addplot [
color=darkgray,
solid,
forget plot
]
coordinates{
 (1478,-10)(1400.21052631579,-4) 
};
\addplot [
color=darkgray,
solid,
forget plot
]
coordinates{
 (1478,-12)(1400.21052631579,-4) 
};
\addplot [
color=darkgray,
solid,
forget plot
]
coordinates{
 (1478,-14)(1400.21052631579,-4) 
};
\addplot [
color=darkgray,
solid,
forget plot
]
coordinates{
 (1478,0)(1400.21052631579,-6) 
};
\addplot [
color=darkgray,
solid,
forget plot
]
coordinates{
 (1478,-2)(1400.21052631579,-6) 
};
\addplot [
color=darkgray,
solid,
forget plot
]
coordinates{
 (1478,-4)(1400.21052631579,-6) 
};
\addplot [
color=darkgray,
solid,
forget plot
]
coordinates{
 (1478,-6)(1400.21052631579,-6) 
};
\addplot [
color=darkgray,
solid,
forget plot
]
coordinates{
 (1478,-8)(1400.21052631579,-6) 
};
\addplot [
color=darkgray,
solid,
forget plot
]
coordinates{
 (1478,-10)(1400.21052631579,-6) 
};
\addplot [
color=darkgray,
solid,
forget plot
]
coordinates{
 (1478,-12)(1400.21052631579,-6) 
};
\addplot [
color=darkgray,
solid,
forget plot
]
coordinates{
 (1478,-14)(1400.21052631579,-6) 
};
\addplot [
color=darkgray,
solid,
forget plot
]
coordinates{
 (1478,-16)(1400.21052631579,-6) 
};
\addplot [
color=darkgray,
solid,
forget plot
]
coordinates{
 (1478,0)(1400.21052631579,-8) 
};
\addplot [
color=darkgray,
solid,
forget plot
]
coordinates{
 (1478,-2)(1400.21052631579,-8) 
};
\addplot [
color=darkgray,
solid,
forget plot
]
coordinates{
 (1478,-4)(1400.21052631579,-8) 
};
\addplot [
color=darkgray,
solid,
forget plot
]
coordinates{
 (1478,-6)(1400.21052631579,-8) 
};
\addplot [
color=darkgray,
solid,
forget plot
]
coordinates{
 (1478,-8)(1400.21052631579,-8) 
};
\addplot [
color=darkgray,
solid,
forget plot
]
coordinates{
 (1478,-10)(1400.21052631579,-8) 
};
\addplot [
color=darkgray,
solid,
forget plot
]
coordinates{
 (1478,-12)(1400.21052631579,-8) 
};
\addplot [
color=darkgray,
solid,
forget plot
]
coordinates{
 (1478,-14)(1400.21052631579,-8) 
};
\addplot [
color=darkgray,
solid,
forget plot
]
coordinates{
 (1478,-16)(1400.21052631579,-8) 
};
\addplot [
color=darkgray,
solid,
forget plot
]
coordinates{
 (1478,-18)(1400.21052631579,-8) 
};
\addplot [
color=darkgray,
solid,
forget plot
]
coordinates{
 (1478,0)(1400.21052631579,-10) 
};
\addplot [
color=darkgray,
solid,
forget plot
]
coordinates{
 (1478,-2)(1400.21052631579,-10) 
};
\addplot [
color=darkgray,
solid,
forget plot
]
coordinates{
 (1478,-4)(1400.21052631579,-10) 
};
\addplot [
color=darkgray,
solid,
forget plot
]
coordinates{
 (1478,-6)(1400.21052631579,-10) 
};
\addplot [
color=darkgray,
solid,
forget plot
]
coordinates{
 (1478,-8)(1400.21052631579,-10) 
};
\addplot [
color=darkgray,
solid,
forget plot
]
coordinates{
 (1478,-10)(1400.21052631579,-10) 
};
\addplot [
color=darkgray,
solid,
forget plot
]
coordinates{
 (1478,-12)(1400.21052631579,-10) 
};
\addplot [
color=darkgray,
solid,
forget plot
]
coordinates{
 (1478,-14)(1400.21052631579,-10) 
};
\addplot [
color=darkgray,
solid,
forget plot
]
coordinates{
 (1478,-16)(1400.21052631579,-10) 
};
\addplot [
color=darkgray,
solid,
forget plot
]
coordinates{
 (1478,-18)(1400.21052631579,-10) 
};
\addplot [
color=darkgray,
solid,
forget plot
]
coordinates{
 (1478,-2)(1400.21052631579,-12) 
};
\addplot [
color=darkgray,
solid,
forget plot
]
coordinates{
 (1478,-4)(1400.21052631579,-12) 
};
\addplot [
color=darkgray,
solid,
forget plot
]
coordinates{
 (1478,-6)(1400.21052631579,-12) 
};
\addplot [
color=darkgray,
solid,
forget plot
]
coordinates{
 (1478,-8)(1400.21052631579,-12) 
};
\addplot [
color=darkgray,
solid,
forget plot
]
coordinates{
 (1478,-10)(1400.21052631579,-12) 
};
\addplot [
color=darkgray,
solid,
forget plot
]
coordinates{
 (1478,-12)(1400.21052631579,-12) 
};
\addplot [
color=darkgray,
solid,
forget plot
]
coordinates{
 (1478,-14)(1400.21052631579,-12) 
};
\addplot [
color=darkgray,
solid,
forget plot
]
coordinates{
 (1478,-16)(1400.21052631579,-12) 
};
\addplot [
color=darkgray,
solid,
forget plot
]
coordinates{
 (1478,-18)(1400.21052631579,-12) 
};
\addplot [
color=darkgray,
solid,
forget plot
]
coordinates{
 (1478,-4)(1400.21052631579,-14) 
};
\addplot [
color=darkgray,
solid,
forget plot
]
coordinates{
 (1478,-6)(1400.21052631579,-14) 
};
\addplot [
color=darkgray,
solid,
forget plot
]
coordinates{
 (1478,-8)(1400.21052631579,-14) 
};
\addplot [
color=darkgray,
solid,
forget plot
]
coordinates{
 (1478,-10)(1400.21052631579,-14) 
};
\addplot [
color=darkgray,
solid,
forget plot
]
coordinates{
 (1478,-12)(1400.21052631579,-14) 
};
\addplot [
color=darkgray,
solid,
forget plot
]
coordinates{
 (1478,-14)(1400.21052631579,-14) 
};
\addplot [
color=darkgray,
solid,
forget plot
]
coordinates{
 (1478,-16)(1400.21052631579,-14) 
};
\addplot [
color=darkgray,
solid,
forget plot
]
coordinates{
 (1478,-18)(1400.21052631579,-14) 
};
\addplot [
color=darkgray,
solid,
forget plot
]
coordinates{
 (1478,-6)(1400.21052631579,-16) 
};
\addplot [
color=darkgray,
solid,
forget plot
]
coordinates{
 (1478,-8)(1400.21052631579,-16) 
};
\addplot [
color=darkgray,
solid,
forget plot
]
coordinates{
 (1478,-10)(1400.21052631579,-16) 
};
\addplot [
color=darkgray,
solid,
forget plot
]
coordinates{
 (1478,-12)(1400.21052631579,-16) 
};
\addplot [
color=darkgray,
solid,
forget plot
]
coordinates{
 (1478,-14)(1400.21052631579,-16) 
};
\addplot [
color=darkgray,
solid,
forget plot
]
coordinates{
 (1478,-16)(1400.21052631579,-16) 
};
\addplot [
color=darkgray,
solid,
forget plot
]
coordinates{
 (1478,-18)(1400.21052631579,-16) 
};
\addplot [
color=darkgray,
solid,
forget plot
]
coordinates{
 (1478,-8)(1400.21052631579,-18) 
};
\addplot [
color=darkgray,
solid,
forget plot
]
coordinates{
 (1478,-10)(1400.21052631579,-18) 
};
\addplot [
color=darkgray,
solid,
forget plot
]
coordinates{
 (1478,-12)(1400.21052631579,-18) 
};
\addplot [
color=darkgray,
solid,
forget plot
]
coordinates{
 (1478,-14)(1400.21052631579,-18) 
};
\addplot [
color=darkgray,
solid,
forget plot
]
coordinates{
 (1478,-16)(1400.21052631579,-18) 
};
\addplot [
color=darkgray,
solid,
forget plot
]
coordinates{
 (1478,-18)(1400.21052631579,-18) 
};
\addplot [
color=darkgray,
solid,
forget plot
]
coordinates{
 (1400.21052631579,0)(1322.42105263158,0) 
};
\addplot [
color=darkgray,
solid,
forget plot
]
coordinates{
 (1400.21052631579,-2)(1322.42105263158,0) 
};
\addplot [
color=darkgray,
solid,
forget plot
]
coordinates{
 (1400.21052631579,-4)(1322.42105263158,0) 
};
\addplot [
color=darkgray,
solid,
forget plot
]
coordinates{
 (1400.21052631579,-6)(1322.42105263158,0) 
};
\addplot [
color=darkgray,
solid,
forget plot
]
coordinates{
 (1400.21052631579,-8)(1322.42105263158,0) 
};
\addplot [
color=darkgray,
solid,
forget plot
]
coordinates{
 (1400.21052631579,-10)(1322.42105263158,0) 
};
\addplot [
color=darkgray,
solid,
forget plot
]
coordinates{
 (1400.21052631579,0)(1322.42105263158,-2) 
};
\addplot [
color=darkgray,
solid,
forget plot
]
coordinates{
 (1400.21052631579,-2)(1322.42105263158,-2) 
};
\addplot [
color=darkgray,
solid,
forget plot
]
coordinates{
 (1400.21052631579,-4)(1322.42105263158,-2) 
};
\addplot [
color=darkgray,
solid,
forget plot
]
coordinates{
 (1400.21052631579,-6)(1322.42105263158,-2) 
};
\addplot [
color=darkgray,
solid,
forget plot
]
coordinates{
 (1400.21052631579,-8)(1322.42105263158,-2) 
};
\addplot [
color=darkgray,
solid,
forget plot
]
coordinates{
 (1400.21052631579,-10)(1322.42105263158,-2) 
};
\addplot [
color=darkgray,
solid,
forget plot
]
coordinates{
 (1400.21052631579,-12)(1322.42105263158,-2) 
};
\addplot [
color=darkgray,
solid,
forget plot
]
coordinates{
 (1400.21052631579,0)(1322.42105263158,-4) 
};
\addplot [
color=darkgray,
solid,
forget plot
]
coordinates{
 (1400.21052631579,-2)(1322.42105263158,-4) 
};
\addplot [
color=darkgray,
solid,
forget plot
]
coordinates{
 (1400.21052631579,-4)(1322.42105263158,-4) 
};
\addplot [
color=darkgray,
solid,
forget plot
]
coordinates{
 (1400.21052631579,-6)(1322.42105263158,-4) 
};
\addplot [
color=darkgray,
solid,
forget plot
]
coordinates{
 (1400.21052631579,-8)(1322.42105263158,-4) 
};
\addplot [
color=darkgray,
solid,
forget plot
]
coordinates{
 (1400.21052631579,-10)(1322.42105263158,-4) 
};
\addplot [
color=darkgray,
solid,
forget plot
]
coordinates{
 (1400.21052631579,-12)(1322.42105263158,-4) 
};
\addplot [
color=darkgray,
solid,
forget plot
]
coordinates{
 (1400.21052631579,-14)(1322.42105263158,-4) 
};
\addplot [
color=darkgray,
solid,
forget plot
]
coordinates{
 (1400.21052631579,0)(1322.42105263158,-6) 
};
\addplot [
color=darkgray,
solid,
forget plot
]
coordinates{
 (1400.21052631579,-2)(1322.42105263158,-6) 
};
\addplot [
color=darkgray,
solid,
forget plot
]
coordinates{
 (1400.21052631579,-4)(1322.42105263158,-6) 
};
\addplot [
color=darkgray,
solid,
forget plot
]
coordinates{
 (1400.21052631579,-6)(1322.42105263158,-6) 
};
\addplot [
color=darkgray,
solid,
forget plot
]
coordinates{
 (1400.21052631579,-8)(1322.42105263158,-6) 
};
\addplot [
color=darkgray,
solid,
forget plot
]
coordinates{
 (1400.21052631579,-10)(1322.42105263158,-6) 
};
\addplot [
color=darkgray,
solid,
forget plot
]
coordinates{
 (1400.21052631579,-12)(1322.42105263158,-6) 
};
\addplot [
color=darkgray,
solid,
forget plot
]
coordinates{
 (1400.21052631579,-14)(1322.42105263158,-6) 
};
\addplot [
color=darkgray,
solid,
forget plot
]
coordinates{
 (1400.21052631579,-16)(1322.42105263158,-6) 
};
\addplot [
color=darkgray,
solid,
forget plot
]
coordinates{
 (1400.21052631579,0)(1322.42105263158,-8) 
};
\addplot [
color=darkgray,
solid,
forget plot
]
coordinates{
 (1400.21052631579,-2)(1322.42105263158,-8) 
};
\addplot [
color=darkgray,
solid,
forget plot
]
coordinates{
 (1400.21052631579,-4)(1322.42105263158,-8) 
};
\addplot [
color=darkgray,
solid,
forget plot
]
coordinates{
 (1400.21052631579,-6)(1322.42105263158,-8) 
};
\addplot [
color=darkgray,
solid,
forget plot
]
coordinates{
 (1400.21052631579,-8)(1322.42105263158,-8) 
};
\addplot [
color=darkgray,
solid,
forget plot
]
coordinates{
 (1400.21052631579,-10)(1322.42105263158,-8) 
};
\addplot [
color=darkgray,
solid,
forget plot
]
coordinates{
 (1400.21052631579,-12)(1322.42105263158,-8) 
};
\addplot [
color=darkgray,
solid,
forget plot
]
coordinates{
 (1400.21052631579,-14)(1322.42105263158,-8) 
};
\addplot [
color=darkgray,
solid,
forget plot
]
coordinates{
 (1400.21052631579,-16)(1322.42105263158,-8) 
};
\addplot [
color=darkgray,
solid,
forget plot
]
coordinates{
 (1400.21052631579,-18)(1322.42105263158,-8) 
};
\addplot [
color=darkgray,
solid,
forget plot
]
coordinates{
 (1400.21052631579,0)(1322.42105263158,-10) 
};
\addplot [
color=darkgray,
solid,
forget plot
]
coordinates{
 (1400.21052631579,-2)(1322.42105263158,-10) 
};
\addplot [
color=darkgray,
solid,
forget plot
]
coordinates{
 (1400.21052631579,-4)(1322.42105263158,-10) 
};
\addplot [
color=darkgray,
solid,
forget plot
]
coordinates{
 (1400.21052631579,-6)(1322.42105263158,-10) 
};
\addplot [
color=darkgray,
solid,
forget plot
]
coordinates{
 (1400.21052631579,-8)(1322.42105263158,-10) 
};
\addplot [
color=darkgray,
solid,
forget plot
]
coordinates{
 (1400.21052631579,-10)(1322.42105263158,-10) 
};
\addplot [
color=darkgray,
solid,
forget plot
]
coordinates{
 (1400.21052631579,-12)(1322.42105263158,-10) 
};
\addplot [
color=darkgray,
solid,
forget plot
]
coordinates{
 (1400.21052631579,-14)(1322.42105263158,-10) 
};
\addplot [
color=darkgray,
solid,
forget plot
]
coordinates{
 (1400.21052631579,-16)(1322.42105263158,-10) 
};
\addplot [
color=darkgray,
solid,
forget plot
]
coordinates{
 (1400.21052631579,-18)(1322.42105263158,-10) 
};
\addplot [
color=darkgray,
solid,
forget plot
]
coordinates{
 (1400.21052631579,-2)(1322.42105263158,-12) 
};
\addplot [
color=darkgray,
solid,
forget plot
]
coordinates{
 (1400.21052631579,-4)(1322.42105263158,-12) 
};
\addplot [
color=darkgray,
solid,
forget plot
]
coordinates{
 (1400.21052631579,-6)(1322.42105263158,-12) 
};
\addplot [
color=darkgray,
solid,
forget plot
]
coordinates{
 (1400.21052631579,-8)(1322.42105263158,-12) 
};
\addplot [
color=darkgray,
solid,
forget plot
]
coordinates{
 (1400.21052631579,-10)(1322.42105263158,-12) 
};
\addplot [
color=darkgray,
solid,
forget plot
]
coordinates{
 (1400.21052631579,-12)(1322.42105263158,-12) 
};
\addplot [
color=darkgray,
solid,
forget plot
]
coordinates{
 (1400.21052631579,-14)(1322.42105263158,-12) 
};
\addplot [
color=darkgray,
solid,
forget plot
]
coordinates{
 (1400.21052631579,-16)(1322.42105263158,-12) 
};
\addplot [
color=darkgray,
solid,
forget plot
]
coordinates{
 (1400.21052631579,-18)(1322.42105263158,-12) 
};
\addplot [
color=darkgray,
solid,
forget plot
]
coordinates{
 (1400.21052631579,-4)(1322.42105263158,-14) 
};
\addplot [
color=darkgray,
solid,
forget plot
]
coordinates{
 (1400.21052631579,-6)(1322.42105263158,-14) 
};
\addplot [
color=darkgray,
solid,
forget plot
]
coordinates{
 (1400.21052631579,-8)(1322.42105263158,-14) 
};
\addplot [
color=darkgray,
solid,
forget plot
]
coordinates{
 (1400.21052631579,-10)(1322.42105263158,-14) 
};
\addplot [
color=darkgray,
solid,
forget plot
]
coordinates{
 (1400.21052631579,-12)(1322.42105263158,-14) 
};
\addplot [
color=darkgray,
solid,
forget plot
]
coordinates{
 (1400.21052631579,-14)(1322.42105263158,-14) 
};
\addplot [
color=darkgray,
solid,
forget plot
]
coordinates{
 (1400.21052631579,-16)(1322.42105263158,-14) 
};
\addplot [
color=darkgray,
solid,
forget plot
]
coordinates{
 (1400.21052631579,-18)(1322.42105263158,-14) 
};
\addplot [
color=darkgray,
solid,
forget plot
]
coordinates{
 (1400.21052631579,-6)(1322.42105263158,-16) 
};
\addplot [
color=darkgray,
solid,
forget plot
]
coordinates{
 (1400.21052631579,-8)(1322.42105263158,-16) 
};
\addplot [
color=darkgray,
solid,
forget plot
]
coordinates{
 (1400.21052631579,-10)(1322.42105263158,-16) 
};
\addplot [
color=darkgray,
solid,
forget plot
]
coordinates{
 (1400.21052631579,-12)(1322.42105263158,-16) 
};
\addplot [
color=darkgray,
solid,
forget plot
]
coordinates{
 (1400.21052631579,-14)(1322.42105263158,-16) 
};
\addplot [
color=darkgray,
solid,
forget plot
]
coordinates{
 (1400.21052631579,-16)(1322.42105263158,-16) 
};
\addplot [
color=darkgray,
solid,
forget plot
]
coordinates{
 (1400.21052631579,-18)(1322.42105263158,-16) 
};
\addplot [
color=darkgray,
solid,
forget plot
]
coordinates{
 (1400.21052631579,-8)(1322.42105263158,-18) 
};
\addplot [
color=darkgray,
solid,
forget plot
]
coordinates{
 (1400.21052631579,-10)(1322.42105263158,-18) 
};
\addplot [
color=darkgray,
solid,
forget plot
]
coordinates{
 (1400.21052631579,-12)(1322.42105263158,-18) 
};
\addplot [
color=darkgray,
solid,
forget plot
]
coordinates{
 (1400.21052631579,-14)(1322.42105263158,-18) 
};
\addplot [
color=darkgray,
solid,
forget plot
]
coordinates{
 (1400.21052631579,-16)(1322.42105263158,-18) 
};
\addplot [
color=darkgray,
solid,
forget plot
]
coordinates{
 (1400.21052631579,-18)(1322.42105263158,-18) 
};
\addplot [
color=darkgray,
solid,
forget plot
]
coordinates{
 (1322.42105263158,0)(1244.63157894737,0) 
};
\addplot [
color=darkgray,
solid,
forget plot
]
coordinates{
 (1322.42105263158,-2)(1244.63157894737,0) 
};
\addplot [
color=darkgray,
solid,
forget plot
]
coordinates{
 (1322.42105263158,-4)(1244.63157894737,0) 
};
\addplot [
color=darkgray,
solid,
forget plot
]
coordinates{
 (1322.42105263158,-6)(1244.63157894737,0) 
};
\addplot [
color=darkgray,
solid,
forget plot
]
coordinates{
 (1322.42105263158,-8)(1244.63157894737,0) 
};
\addplot [
color=darkgray,
solid,
forget plot
]
coordinates{
 (1322.42105263158,-10)(1244.63157894737,0) 
};
\addplot [
color=darkgray,
solid,
forget plot
]
coordinates{
 (1322.42105263158,0)(1244.63157894737,-2) 
};
\addplot [
color=darkgray,
solid,
forget plot
]
coordinates{
 (1322.42105263158,-2)(1244.63157894737,-2) 
};
\addplot [
color=darkgray,
solid,
forget plot
]
coordinates{
 (1322.42105263158,-4)(1244.63157894737,-2) 
};
\addplot [
color=darkgray,
solid,
forget plot
]
coordinates{
 (1322.42105263158,-6)(1244.63157894737,-2) 
};
\addplot [
color=darkgray,
solid,
forget plot
]
coordinates{
 (1322.42105263158,-8)(1244.63157894737,-2) 
};
\addplot [
color=darkgray,
solid,
forget plot
]
coordinates{
 (1322.42105263158,-10)(1244.63157894737,-2) 
};
\addplot [
color=darkgray,
solid,
forget plot
]
coordinates{
 (1322.42105263158,-12)(1244.63157894737,-2) 
};
\addplot [
color=darkgray,
solid,
forget plot
]
coordinates{
 (1322.42105263158,0)(1244.63157894737,-4) 
};
\addplot [
color=darkgray,
solid,
forget plot
]
coordinates{
 (1322.42105263158,-2)(1244.63157894737,-4) 
};
\addplot [
color=darkgray,
solid,
forget plot
]
coordinates{
 (1322.42105263158,-4)(1244.63157894737,-4) 
};
\addplot [
color=darkgray,
solid,
forget plot
]
coordinates{
 (1322.42105263158,-6)(1244.63157894737,-4) 
};
\addplot [
color=darkgray,
solid,
forget plot
]
coordinates{
 (1322.42105263158,-8)(1244.63157894737,-4) 
};
\addplot [
color=darkgray,
solid,
forget plot
]
coordinates{
 (1322.42105263158,-10)(1244.63157894737,-4) 
};
\addplot [
color=darkgray,
solid,
forget plot
]
coordinates{
 (1322.42105263158,-12)(1244.63157894737,-4) 
};
\addplot [
color=darkgray,
solid,
forget plot
]
coordinates{
 (1322.42105263158,-14)(1244.63157894737,-4) 
};
\addplot [
color=darkgray,
solid,
forget plot
]
coordinates{
 (1322.42105263158,0)(1244.63157894737,-6) 
};
\addplot [
color=darkgray,
solid,
forget plot
]
coordinates{
 (1322.42105263158,-2)(1244.63157894737,-6) 
};
\addplot [
color=darkgray,
solid,
forget plot
]
coordinates{
 (1322.42105263158,-4)(1244.63157894737,-6) 
};
\addplot [
color=darkgray,
solid,
forget plot
]
coordinates{
 (1322.42105263158,-6)(1244.63157894737,-6) 
};
\addplot [
color=darkgray,
solid,
forget plot
]
coordinates{
 (1322.42105263158,-8)(1244.63157894737,-6) 
};
\addplot [
color=darkgray,
solid,
forget plot
]
coordinates{
 (1322.42105263158,-10)(1244.63157894737,-6) 
};
\addplot [
color=darkgray,
solid,
forget plot
]
coordinates{
 (1322.42105263158,-12)(1244.63157894737,-6) 
};
\addplot [
color=darkgray,
solid,
forget plot
]
coordinates{
 (1322.42105263158,-14)(1244.63157894737,-6) 
};
\addplot [
color=darkgray,
solid,
forget plot
]
coordinates{
 (1322.42105263158,-16)(1244.63157894737,-6) 
};
\addplot [
color=darkgray,
solid,
forget plot
]
coordinates{
 (1322.42105263158,0)(1244.63157894737,-8) 
};
\addplot [
color=darkgray,
solid,
forget plot
]
coordinates{
 (1322.42105263158,-2)(1244.63157894737,-8) 
};
\addplot [
color=darkgray,
solid,
forget plot
]
coordinates{
 (1322.42105263158,-4)(1244.63157894737,-8) 
};
\addplot [
color=darkgray,
solid,
forget plot
]
coordinates{
 (1322.42105263158,-6)(1244.63157894737,-8) 
};
\addplot [
color=darkgray,
solid,
forget plot
]
coordinates{
 (1322.42105263158,-8)(1244.63157894737,-8) 
};
\addplot [
color=darkgray,
solid,
forget plot
]
coordinates{
 (1322.42105263158,-10)(1244.63157894737,-8) 
};
\addplot [
color=darkgray,
solid,
forget plot
]
coordinates{
 (1322.42105263158,-12)(1244.63157894737,-8) 
};
\addplot [
color=darkgray,
solid,
forget plot
]
coordinates{
 (1322.42105263158,-14)(1244.63157894737,-8) 
};
\addplot [
color=darkgray,
solid,
forget plot
]
coordinates{
 (1322.42105263158,-16)(1244.63157894737,-8) 
};
\addplot [
color=darkgray,
solid,
forget plot
]
coordinates{
 (1322.42105263158,-18)(1244.63157894737,-8) 
};
\addplot [
color=darkgray,
solid,
forget plot
]
coordinates{
 (1322.42105263158,0)(1244.63157894737,-10) 
};
\addplot [
color=darkgray,
solid,
forget plot
]
coordinates{
 (1322.42105263158,-2)(1244.63157894737,-10) 
};
\addplot [
color=darkgray,
solid,
forget plot
]
coordinates{
 (1322.42105263158,-4)(1244.63157894737,-10) 
};
\addplot [
color=darkgray,
solid,
forget plot
]
coordinates{
 (1322.42105263158,-6)(1244.63157894737,-10) 
};
\addplot [
color=darkgray,
solid,
forget plot
]
coordinates{
 (1322.42105263158,-8)(1244.63157894737,-10) 
};
\addplot [
color=darkgray,
solid,
forget plot
]
coordinates{
 (1322.42105263158,-10)(1244.63157894737,-10) 
};
\addplot [
color=darkgray,
solid,
forget plot
]
coordinates{
 (1322.42105263158,-12)(1244.63157894737,-10) 
};
\addplot [
color=darkgray,
solid,
forget plot
]
coordinates{
 (1322.42105263158,-14)(1244.63157894737,-10) 
};
\addplot [
color=darkgray,
solid,
forget plot
]
coordinates{
 (1322.42105263158,-16)(1244.63157894737,-10) 
};
\addplot [
color=darkgray,
solid,
forget plot
]
coordinates{
 (1322.42105263158,-18)(1244.63157894737,-10) 
};
\addplot [
color=darkgray,
solid,
forget plot
]
coordinates{
 (1322.42105263158,-2)(1244.63157894737,-12) 
};
\addplot [
color=darkgray,
solid,
forget plot
]
coordinates{
 (1322.42105263158,-4)(1244.63157894737,-12) 
};
\addplot [
color=darkgray,
solid,
forget plot
]
coordinates{
 (1322.42105263158,-6)(1244.63157894737,-12) 
};
\addplot [
color=darkgray,
solid,
forget plot
]
coordinates{
 (1322.42105263158,-8)(1244.63157894737,-12) 
};
\addplot [
color=darkgray,
solid,
forget plot
]
coordinates{
 (1322.42105263158,-10)(1244.63157894737,-12) 
};
\addplot [
color=darkgray,
solid,
forget plot
]
coordinates{
 (1322.42105263158,-12)(1244.63157894737,-12) 
};
\addplot [
color=darkgray,
solid,
forget plot
]
coordinates{
 (1322.42105263158,-14)(1244.63157894737,-12) 
};
\addplot [
color=darkgray,
solid,
forget plot
]
coordinates{
 (1322.42105263158,-16)(1244.63157894737,-12) 
};
\addplot [
color=darkgray,
solid,
forget plot
]
coordinates{
 (1322.42105263158,-18)(1244.63157894737,-12) 
};
\addplot [
color=darkgray,
solid,
forget plot
]
coordinates{
 (1322.42105263158,-4)(1244.63157894737,-14) 
};
\addplot [
color=darkgray,
solid,
forget plot
]
coordinates{
 (1322.42105263158,-6)(1244.63157894737,-14) 
};
\addplot [
color=darkgray,
solid,
forget plot
]
coordinates{
 (1322.42105263158,-8)(1244.63157894737,-14) 
};
\addplot [
color=darkgray,
solid,
forget plot
]
coordinates{
 (1322.42105263158,-10)(1244.63157894737,-14) 
};
\addplot [
color=darkgray,
solid,
forget plot
]
coordinates{
 (1322.42105263158,-12)(1244.63157894737,-14) 
};
\addplot [
color=darkgray,
solid,
forget plot
]
coordinates{
 (1322.42105263158,-14)(1244.63157894737,-14) 
};
\addplot [
color=darkgray,
solid,
forget plot
]
coordinates{
 (1322.42105263158,-16)(1244.63157894737,-14) 
};
\addplot [
color=darkgray,
solid,
forget plot
]
coordinates{
 (1322.42105263158,-18)(1244.63157894737,-14) 
};
\addplot [
color=darkgray,
solid,
forget plot
]
coordinates{
 (1322.42105263158,-6)(1244.63157894737,-16) 
};
\addplot [
color=darkgray,
solid,
forget plot
]
coordinates{
 (1322.42105263158,-8)(1244.63157894737,-16) 
};
\addplot [
color=darkgray,
solid,
forget plot
]
coordinates{
 (1322.42105263158,-10)(1244.63157894737,-16) 
};
\addplot [
color=darkgray,
solid,
forget plot
]
coordinates{
 (1322.42105263158,-12)(1244.63157894737,-16) 
};
\addplot [
color=darkgray,
solid,
forget plot
]
coordinates{
 (1322.42105263158,-14)(1244.63157894737,-16) 
};
\addplot [
color=darkgray,
solid,
forget plot
]
coordinates{
 (1322.42105263158,-16)(1244.63157894737,-16) 
};
\addplot [
color=darkgray,
solid,
forget plot
]
coordinates{
 (1322.42105263158,-18)(1244.63157894737,-16) 
};
\addplot [
color=darkgray,
solid,
forget plot
]
coordinates{
 (1322.42105263158,-8)(1244.63157894737,-18) 
};
\addplot [
color=darkgray,
solid,
forget plot
]
coordinates{
 (1322.42105263158,-10)(1244.63157894737,-18) 
};
\addplot [
color=darkgray,
solid,
forget plot
]
coordinates{
 (1322.42105263158,-12)(1244.63157894737,-18) 
};
\addplot [
color=darkgray,
solid,
forget plot
]
coordinates{
 (1322.42105263158,-14)(1244.63157894737,-18) 
};
\addplot [
color=darkgray,
solid,
forget plot
]
coordinates{
 (1322.42105263158,-16)(1244.63157894737,-18) 
};
\addplot [
color=darkgray,
solid,
forget plot
]
coordinates{
 (1322.42105263158,-18)(1244.63157894737,-18) 
};
\addplot [
color=darkgray,
solid,
forget plot
]
coordinates{
 (1244.63157894737,0)(1166.84210526316,0) 
};
\addplot [
color=darkgray,
solid,
forget plot
]
coordinates{
 (1244.63157894737,-2)(1166.84210526316,0) 
};
\addplot [
color=darkgray,
solid,
forget plot
]
coordinates{
 (1244.63157894737,-4)(1166.84210526316,0) 
};
\addplot [
color=darkgray,
solid,
forget plot
]
coordinates{
 (1244.63157894737,-6)(1166.84210526316,0) 
};
\addplot [
color=darkgray,
solid,
forget plot
]
coordinates{
 (1244.63157894737,-8)(1166.84210526316,0) 
};
\addplot [
color=darkgray,
solid,
forget plot
]
coordinates{
 (1244.63157894737,-10)(1166.84210526316,0) 
};
\addplot [
color=darkgray,
solid,
forget plot
]
coordinates{
 (1244.63157894737,0)(1166.84210526316,-2) 
};
\addplot [
color=darkgray,
solid,
forget plot
]
coordinates{
 (1244.63157894737,-2)(1166.84210526316,-2) 
};
\addplot [
color=darkgray,
solid,
forget plot
]
coordinates{
 (1244.63157894737,-4)(1166.84210526316,-2) 
};
\addplot [
color=darkgray,
solid,
forget plot
]
coordinates{
 (1244.63157894737,-6)(1166.84210526316,-2) 
};
\addplot [
color=darkgray,
solid,
forget plot
]
coordinates{
 (1244.63157894737,-8)(1166.84210526316,-2) 
};
\addplot [
color=darkgray,
solid,
forget plot
]
coordinates{
 (1244.63157894737,-10)(1166.84210526316,-2) 
};
\addplot [
color=darkgray,
solid,
forget plot
]
coordinates{
 (1244.63157894737,-12)(1166.84210526316,-2) 
};
\addplot [
color=darkgray,
solid,
forget plot
]
coordinates{
 (1244.63157894737,0)(1166.84210526316,-4) 
};
\addplot [
color=darkgray,
solid,
forget plot
]
coordinates{
 (1244.63157894737,-2)(1166.84210526316,-4) 
};
\addplot [
color=darkgray,
solid,
forget plot
]
coordinates{
 (1244.63157894737,-4)(1166.84210526316,-4) 
};
\addplot [
color=darkgray,
solid,
forget plot
]
coordinates{
 (1244.63157894737,-6)(1166.84210526316,-4) 
};
\addplot [
color=darkgray,
solid,
forget plot
]
coordinates{
 (1244.63157894737,-8)(1166.84210526316,-4) 
};
\addplot [
color=darkgray,
solid,
forget plot
]
coordinates{
 (1244.63157894737,-10)(1166.84210526316,-4) 
};
\addplot [
color=darkgray,
solid,
forget plot
]
coordinates{
 (1244.63157894737,-12)(1166.84210526316,-4) 
};
\addplot [
color=darkgray,
solid,
forget plot
]
coordinates{
 (1244.63157894737,-14)(1166.84210526316,-4) 
};
\addplot [
color=darkgray,
solid,
forget plot
]
coordinates{
 (1244.63157894737,0)(1166.84210526316,-6) 
};
\addplot [
color=darkgray,
solid,
forget plot
]
coordinates{
 (1244.63157894737,-2)(1166.84210526316,-6) 
};
\addplot [
color=darkgray,
solid,
forget plot
]
coordinates{
 (1244.63157894737,-4)(1166.84210526316,-6) 
};
\addplot [
color=darkgray,
solid,
forget plot
]
coordinates{
 (1244.63157894737,-6)(1166.84210526316,-6) 
};
\addplot [
color=darkgray,
solid,
forget plot
]
coordinates{
 (1244.63157894737,-8)(1166.84210526316,-6) 
};
\addplot [
color=darkgray,
solid,
forget plot
]
coordinates{
 (1244.63157894737,-10)(1166.84210526316,-6) 
};
\addplot [
color=darkgray,
solid,
forget plot
]
coordinates{
 (1244.63157894737,-12)(1166.84210526316,-6) 
};
\addplot [
color=darkgray,
solid,
forget plot
]
coordinates{
 (1244.63157894737,-14)(1166.84210526316,-6) 
};
\addplot [
color=darkgray,
solid,
forget plot
]
coordinates{
 (1244.63157894737,-16)(1166.84210526316,-6) 
};
\addplot [
color=darkgray,
solid,
forget plot
]
coordinates{
 (1244.63157894737,0)(1166.84210526316,-8) 
};
\addplot [
color=darkgray,
solid,
forget plot
]
coordinates{
 (1244.63157894737,-2)(1166.84210526316,-8) 
};
\addplot [
color=darkgray,
solid,
forget plot
]
coordinates{
 (1244.63157894737,-4)(1166.84210526316,-8) 
};
\addplot [
color=darkgray,
solid,
forget plot
]
coordinates{
 (1244.63157894737,-6)(1166.84210526316,-8) 
};
\addplot [
color=darkgray,
solid,
forget plot
]
coordinates{
 (1244.63157894737,-8)(1166.84210526316,-8) 
};
\addplot [
color=darkgray,
solid,
forget plot
]
coordinates{
 (1244.63157894737,-10)(1166.84210526316,-8) 
};
\addplot [
color=darkgray,
solid,
forget plot
]
coordinates{
 (1244.63157894737,-12)(1166.84210526316,-8) 
};
\addplot [
color=darkgray,
solid,
forget plot
]
coordinates{
 (1244.63157894737,-14)(1166.84210526316,-8) 
};
\addplot [
color=darkgray,
solid,
forget plot
]
coordinates{
 (1244.63157894737,-16)(1166.84210526316,-8) 
};
\addplot [
color=darkgray,
solid,
forget plot
]
coordinates{
 (1244.63157894737,-18)(1166.84210526316,-8) 
};
\addplot [
color=darkgray,
solid,
forget plot
]
coordinates{
 (1244.63157894737,0)(1166.84210526316,-10) 
};
\addplot [
color=darkgray,
solid,
forget plot
]
coordinates{
 (1244.63157894737,-2)(1166.84210526316,-10) 
};
\addplot [
color=darkgray,
solid,
forget plot
]
coordinates{
 (1244.63157894737,-4)(1166.84210526316,-10) 
};
\addplot [
color=darkgray,
solid,
forget plot
]
coordinates{
 (1244.63157894737,-6)(1166.84210526316,-10) 
};
\addplot [
color=darkgray,
solid,
forget plot
]
coordinates{
 (1244.63157894737,-8)(1166.84210526316,-10) 
};
\addplot [
color=darkgray,
solid,
forget plot
]
coordinates{
 (1244.63157894737,-10)(1166.84210526316,-10) 
};
\addplot [
color=darkgray,
solid,
forget plot
]
coordinates{
 (1244.63157894737,-12)(1166.84210526316,-10) 
};
\addplot [
color=darkgray,
solid,
forget plot
]
coordinates{
 (1244.63157894737,-14)(1166.84210526316,-10) 
};
\addplot [
color=darkgray,
solid,
forget plot
]
coordinates{
 (1244.63157894737,-16)(1166.84210526316,-10) 
};
\addplot [
color=darkgray,
solid,
forget plot
]
coordinates{
 (1244.63157894737,-18)(1166.84210526316,-10) 
};
\addplot [
color=darkgray,
solid,
forget plot
]
coordinates{
 (1244.63157894737,-2)(1166.84210526316,-12) 
};
\addplot [
color=darkgray,
solid,
forget plot
]
coordinates{
 (1244.63157894737,-4)(1166.84210526316,-12) 
};
\addplot [
color=darkgray,
solid,
forget plot
]
coordinates{
 (1244.63157894737,-6)(1166.84210526316,-12) 
};
\addplot [
color=darkgray,
solid,
forget plot
]
coordinates{
 (1244.63157894737,-8)(1166.84210526316,-12) 
};
\addplot [
color=darkgray,
solid,
forget plot
]
coordinates{
 (1244.63157894737,-10)(1166.84210526316,-12) 
};
\addplot [
color=darkgray,
solid,
forget plot
]
coordinates{
 (1244.63157894737,-12)(1166.84210526316,-12) 
};
\addplot [
color=darkgray,
solid,
forget plot
]
coordinates{
 (1244.63157894737,-14)(1166.84210526316,-12) 
};
\addplot [
color=darkgray,
solid,
forget plot
]
coordinates{
 (1244.63157894737,-16)(1166.84210526316,-12) 
};
\addplot [
color=darkgray,
solid,
forget plot
]
coordinates{
 (1244.63157894737,-18)(1166.84210526316,-12) 
};
\addplot [
color=darkgray,
solid,
forget plot
]
coordinates{
 (1244.63157894737,-4)(1166.84210526316,-14) 
};
\addplot [
color=darkgray,
solid,
forget plot
]
coordinates{
 (1244.63157894737,-6)(1166.84210526316,-14) 
};
\addplot [
color=darkgray,
solid,
forget plot
]
coordinates{
 (1244.63157894737,-8)(1166.84210526316,-14) 
};
\addplot [
color=darkgray,
solid,
forget plot
]
coordinates{
 (1244.63157894737,-10)(1166.84210526316,-14) 
};
\addplot [
color=darkgray,
solid,
forget plot
]
coordinates{
 (1244.63157894737,-12)(1166.84210526316,-14) 
};
\addplot [
color=darkgray,
solid,
forget plot
]
coordinates{
 (1244.63157894737,-14)(1166.84210526316,-14) 
};
\addplot [
color=darkgray,
solid,
forget plot
]
coordinates{
 (1244.63157894737,-16)(1166.84210526316,-14) 
};
\addplot [
color=darkgray,
solid,
forget plot
]
coordinates{
 (1244.63157894737,-18)(1166.84210526316,-14) 
};
\addplot [
color=darkgray,
solid,
forget plot
]
coordinates{
 (1244.63157894737,-6)(1166.84210526316,-16) 
};
\addplot [
color=darkgray,
solid,
forget plot
]
coordinates{
 (1244.63157894737,-8)(1166.84210526316,-16) 
};
\addplot [
color=darkgray,
solid,
forget plot
]
coordinates{
 (1244.63157894737,-10)(1166.84210526316,-16) 
};
\addplot [
color=darkgray,
solid,
forget plot
]
coordinates{
 (1244.63157894737,-12)(1166.84210526316,-16) 
};
\addplot [
color=darkgray,
solid,
forget plot
]
coordinates{
 (1244.63157894737,-14)(1166.84210526316,-16) 
};
\addplot [
color=darkgray,
solid,
forget plot
]
coordinates{
 (1244.63157894737,-16)(1166.84210526316,-16) 
};
\addplot [
color=darkgray,
solid,
forget plot
]
coordinates{
 (1244.63157894737,-18)(1166.84210526316,-16) 
};
\addplot [
color=darkgray,
solid,
forget plot
]
coordinates{
 (1244.63157894737,-8)(1166.84210526316,-18) 
};
\addplot [
color=darkgray,
solid,
forget plot
]
coordinates{
 (1244.63157894737,-10)(1166.84210526316,-18) 
};
\addplot [
color=darkgray,
solid,
forget plot
]
coordinates{
 (1244.63157894737,-12)(1166.84210526316,-18) 
};
\addplot [
color=darkgray,
solid,
forget plot
]
coordinates{
 (1244.63157894737,-14)(1166.84210526316,-18) 
};
\addplot [
color=darkgray,
solid,
forget plot
]
coordinates{
 (1244.63157894737,-16)(1166.84210526316,-18) 
};
\addplot [
color=darkgray,
solid,
forget plot
]
coordinates{
 (1244.63157894737,-18)(1166.84210526316,-18) 
};
\addplot [
color=darkgray,
solid,
forget plot
]
coordinates{
 (1166.84210526316,0)(1089.05263157895,0) 
};
\addplot [
color=darkgray,
solid,
forget plot
]
coordinates{
 (1166.84210526316,-2)(1089.05263157895,0) 
};
\addplot [
color=darkgray,
solid,
forget plot
]
coordinates{
 (1166.84210526316,-4)(1089.05263157895,0) 
};
\addplot [
color=darkgray,
solid,
forget plot
]
coordinates{
 (1166.84210526316,-6)(1089.05263157895,0) 
};
\addplot [
color=darkgray,
solid,
forget plot
]
coordinates{
 (1166.84210526316,-8)(1089.05263157895,0) 
};
\addplot [
color=darkgray,
solid,
forget plot
]
coordinates{
 (1166.84210526316,-10)(1089.05263157895,0) 
};
\addplot [
color=darkgray,
solid,
forget plot
]
coordinates{
 (1166.84210526316,0)(1089.05263157895,-2) 
};
\addplot [
color=darkgray,
solid,
forget plot
]
coordinates{
 (1166.84210526316,-2)(1089.05263157895,-2) 
};
\addplot [
color=darkgray,
solid,
forget plot
]
coordinates{
 (1166.84210526316,-4)(1089.05263157895,-2) 
};
\addplot [
color=darkgray,
solid,
forget plot
]
coordinates{
 (1166.84210526316,-6)(1089.05263157895,-2) 
};
\addplot [
color=darkgray,
solid,
forget plot
]
coordinates{
 (1166.84210526316,-8)(1089.05263157895,-2) 
};
\addplot [
color=darkgray,
solid,
forget plot
]
coordinates{
 (1166.84210526316,-10)(1089.05263157895,-2) 
};
\addplot [
color=darkgray,
solid,
forget plot
]
coordinates{
 (1166.84210526316,-12)(1089.05263157895,-2) 
};
\addplot [
color=darkgray,
solid,
forget plot
]
coordinates{
 (1166.84210526316,0)(1089.05263157895,-4) 
};
\addplot [
color=darkgray,
solid,
forget plot
]
coordinates{
 (1166.84210526316,-2)(1089.05263157895,-4) 
};
\addplot [
color=darkgray,
solid,
forget plot
]
coordinates{
 (1166.84210526316,-4)(1089.05263157895,-4) 
};
\addplot [
color=darkgray,
solid,
forget plot
]
coordinates{
 (1166.84210526316,-6)(1089.05263157895,-4) 
};
\addplot [
color=darkgray,
solid,
forget plot
]
coordinates{
 (1166.84210526316,-8)(1089.05263157895,-4) 
};
\addplot [
color=darkgray,
solid,
forget plot
]
coordinates{
 (1166.84210526316,-10)(1089.05263157895,-4) 
};
\addplot [
color=darkgray,
solid,
forget plot
]
coordinates{
 (1166.84210526316,-12)(1089.05263157895,-4) 
};
\addplot [
color=darkgray,
solid,
forget plot
]
coordinates{
 (1166.84210526316,-14)(1089.05263157895,-4) 
};
\addplot [
color=darkgray,
solid,
forget plot
]
coordinates{
 (1166.84210526316,0)(1089.05263157895,-6) 
};
\addplot [
color=darkgray,
solid,
forget plot
]
coordinates{
 (1166.84210526316,-2)(1089.05263157895,-6) 
};
\addplot [
color=darkgray,
solid,
forget plot
]
coordinates{
 (1166.84210526316,-4)(1089.05263157895,-6) 
};
\addplot [
color=darkgray,
solid,
forget plot
]
coordinates{
 (1166.84210526316,-6)(1089.05263157895,-6) 
};
\addplot [
color=darkgray,
solid,
forget plot
]
coordinates{
 (1166.84210526316,-8)(1089.05263157895,-6) 
};
\addplot [
color=darkgray,
solid,
forget plot
]
coordinates{
 (1166.84210526316,-10)(1089.05263157895,-6) 
};
\addplot [
color=darkgray,
solid,
forget plot
]
coordinates{
 (1166.84210526316,-12)(1089.05263157895,-6) 
};
\addplot [
color=darkgray,
solid,
forget plot
]
coordinates{
 (1166.84210526316,-14)(1089.05263157895,-6) 
};
\addplot [
color=darkgray,
solid,
forget plot
]
coordinates{
 (1166.84210526316,-16)(1089.05263157895,-6) 
};
\addplot [
color=darkgray,
solid,
forget plot
]
coordinates{
 (1166.84210526316,0)(1089.05263157895,-8) 
};
\addplot [
color=darkgray,
solid,
forget plot
]
coordinates{
 (1166.84210526316,-2)(1089.05263157895,-8) 
};
\addplot [
color=darkgray,
solid,
forget plot
]
coordinates{
 (1166.84210526316,-4)(1089.05263157895,-8) 
};
\addplot [
color=darkgray,
solid,
forget plot
]
coordinates{
 (1166.84210526316,-6)(1089.05263157895,-8) 
};
\addplot [
color=darkgray,
solid,
forget plot
]
coordinates{
 (1166.84210526316,-8)(1089.05263157895,-8) 
};
\addplot [
color=darkgray,
solid,
forget plot
]
coordinates{
 (1166.84210526316,-10)(1089.05263157895,-8) 
};
\addplot [
color=darkgray,
solid,
forget plot
]
coordinates{
 (1166.84210526316,-12)(1089.05263157895,-8) 
};
\addplot [
color=darkgray,
solid,
forget plot
]
coordinates{
 (1166.84210526316,-14)(1089.05263157895,-8) 
};
\addplot [
color=darkgray,
solid,
forget plot
]
coordinates{
 (1166.84210526316,-16)(1089.05263157895,-8) 
};
\addplot [
color=darkgray,
solid,
forget plot
]
coordinates{
 (1166.84210526316,-18)(1089.05263157895,-8) 
};
\addplot [
color=darkgray,
solid,
forget plot
]
coordinates{
 (1166.84210526316,0)(1089.05263157895,-10) 
};
\addplot [
color=darkgray,
solid,
forget plot
]
coordinates{
 (1166.84210526316,-2)(1089.05263157895,-10) 
};
\addplot [
color=darkgray,
solid,
forget plot
]
coordinates{
 (1166.84210526316,-4)(1089.05263157895,-10) 
};
\addplot [
color=darkgray,
solid,
forget plot
]
coordinates{
 (1166.84210526316,-6)(1089.05263157895,-10) 
};
\addplot [
color=darkgray,
solid,
forget plot
]
coordinates{
 (1166.84210526316,-8)(1089.05263157895,-10) 
};
\addplot [
color=darkgray,
solid,
forget plot
]
coordinates{
 (1166.84210526316,-10)(1089.05263157895,-10) 
};
\addplot [
color=darkgray,
solid,
forget plot
]
coordinates{
 (1166.84210526316,-12)(1089.05263157895,-10) 
};
\addplot [
color=darkgray,
solid,
forget plot
]
coordinates{
 (1166.84210526316,-14)(1089.05263157895,-10) 
};
\addplot [
color=darkgray,
solid,
forget plot
]
coordinates{
 (1166.84210526316,-16)(1089.05263157895,-10) 
};
\addplot [
color=darkgray,
solid,
forget plot
]
coordinates{
 (1166.84210526316,-18)(1089.05263157895,-10) 
};
\addplot [
color=darkgray,
solid,
forget plot
]
coordinates{
 (1166.84210526316,-2)(1089.05263157895,-12) 
};
\addplot [
color=darkgray,
solid,
forget plot
]
coordinates{
 (1166.84210526316,-4)(1089.05263157895,-12) 
};
\addplot [
color=darkgray,
solid,
forget plot
]
coordinates{
 (1166.84210526316,-6)(1089.05263157895,-12) 
};
\addplot [
color=darkgray,
solid,
forget plot
]
coordinates{
 (1166.84210526316,-8)(1089.05263157895,-12) 
};
\addplot [
color=darkgray,
solid,
forget plot
]
coordinates{
 (1166.84210526316,-10)(1089.05263157895,-12) 
};
\addplot [
color=darkgray,
solid,
forget plot
]
coordinates{
 (1166.84210526316,-12)(1089.05263157895,-12) 
};
\addplot [
color=darkgray,
solid,
forget plot
]
coordinates{
 (1166.84210526316,-14)(1089.05263157895,-12) 
};
\addplot [
color=darkgray,
solid,
forget plot
]
coordinates{
 (1166.84210526316,-16)(1089.05263157895,-12) 
};
\addplot [
color=darkgray,
solid,
forget plot
]
coordinates{
 (1166.84210526316,-18)(1089.05263157895,-12) 
};
\addplot [
color=darkgray,
solid,
forget plot
]
coordinates{
 (1166.84210526316,-4)(1089.05263157895,-14) 
};
\addplot [
color=darkgray,
solid,
forget plot
]
coordinates{
 (1166.84210526316,-6)(1089.05263157895,-14) 
};
\addplot [
color=darkgray,
solid,
forget plot
]
coordinates{
 (1166.84210526316,-8)(1089.05263157895,-14) 
};
\addplot [
color=darkgray,
solid,
forget plot
]
coordinates{
 (1166.84210526316,-10)(1089.05263157895,-14) 
};
\addplot [
color=darkgray,
solid,
forget plot
]
coordinates{
 (1166.84210526316,-12)(1089.05263157895,-14) 
};
\addplot [
color=darkgray,
solid,
forget plot
]
coordinates{
 (1166.84210526316,-14)(1089.05263157895,-14) 
};
\addplot [
color=darkgray,
solid,
forget plot
]
coordinates{
 (1166.84210526316,-16)(1089.05263157895,-14) 
};
\addplot [
color=darkgray,
solid,
forget plot
]
coordinates{
 (1166.84210526316,-18)(1089.05263157895,-14) 
};
\addplot [
color=darkgray,
solid,
forget plot
]
coordinates{
 (1166.84210526316,-6)(1089.05263157895,-16) 
};
\addplot [
color=darkgray,
solid,
forget plot
]
coordinates{
 (1166.84210526316,-8)(1089.05263157895,-16) 
};
\addplot [
color=darkgray,
solid,
forget plot
]
coordinates{
 (1166.84210526316,-10)(1089.05263157895,-16) 
};
\addplot [
color=darkgray,
solid,
forget plot
]
coordinates{
 (1166.84210526316,-12)(1089.05263157895,-16) 
};
\addplot [
color=darkgray,
solid,
forget plot
]
coordinates{
 (1166.84210526316,-14)(1089.05263157895,-16) 
};
\addplot [
color=darkgray,
solid,
forget plot
]
coordinates{
 (1166.84210526316,-16)(1089.05263157895,-16) 
};
\addplot [
color=darkgray,
solid,
forget plot
]
coordinates{
 (1166.84210526316,-18)(1089.05263157895,-16) 
};
\addplot [
color=darkgray,
solid,
forget plot
]
coordinates{
 (1166.84210526316,-8)(1089.05263157895,-18) 
};
\addplot [
color=darkgray,
solid,
forget plot
]
coordinates{
 (1166.84210526316,-10)(1089.05263157895,-18) 
};
\addplot [
color=darkgray,
solid,
forget plot
]
coordinates{
 (1166.84210526316,-12)(1089.05263157895,-18) 
};
\addplot [
color=darkgray,
solid,
forget plot
]
coordinates{
 (1166.84210526316,-14)(1089.05263157895,-18) 
};
\addplot [
color=darkgray,
solid,
forget plot
]
coordinates{
 (1166.84210526316,-16)(1089.05263157895,-18) 
};
\addplot [
color=darkgray,
solid,
forget plot
]
coordinates{
 (1166.84210526316,-18)(1089.05263157895,-18) 
};
\addplot [
color=darkgray,
solid,
forget plot
]
coordinates{
 (1089.05263157895,0)(1011.26315789474,0) 
};
\addplot [
color=darkgray,
solid,
forget plot
]
coordinates{
 (1089.05263157895,-2)(1011.26315789474,0) 
};
\addplot [
color=darkgray,
solid,
forget plot
]
coordinates{
 (1089.05263157895,-4)(1011.26315789474,0) 
};
\addplot [
color=darkgray,
solid,
forget plot
]
coordinates{
 (1089.05263157895,-6)(1011.26315789474,0) 
};
\addplot [
color=darkgray,
solid,
forget plot
]
coordinates{
 (1089.05263157895,-8)(1011.26315789474,0) 
};
\addplot [
color=darkgray,
solid,
forget plot
]
coordinates{
 (1089.05263157895,-10)(1011.26315789474,0) 
};
\addplot [
color=darkgray,
solid,
forget plot
]
coordinates{
 (1089.05263157895,0)(1011.26315789474,-2) 
};
\addplot [
color=darkgray,
solid,
forget plot
]
coordinates{
 (1089.05263157895,-2)(1011.26315789474,-2) 
};
\addplot [
color=darkgray,
solid,
forget plot
]
coordinates{
 (1089.05263157895,-4)(1011.26315789474,-2) 
};
\addplot [
color=darkgray,
solid,
forget plot
]
coordinates{
 (1089.05263157895,-6)(1011.26315789474,-2) 
};
\addplot [
color=darkgray,
solid,
forget plot
]
coordinates{
 (1089.05263157895,-8)(1011.26315789474,-2) 
};
\addplot [
color=darkgray,
solid,
forget plot
]
coordinates{
 (1089.05263157895,-10)(1011.26315789474,-2) 
};
\addplot [
color=darkgray,
solid,
forget plot
]
coordinates{
 (1089.05263157895,-12)(1011.26315789474,-2) 
};
\addplot [
color=darkgray,
solid,
forget plot
]
coordinates{
 (1089.05263157895,0)(1011.26315789474,-4) 
};
\addplot [
color=darkgray,
solid,
forget plot
]
coordinates{
 (1089.05263157895,-2)(1011.26315789474,-4) 
};
\addplot [
color=darkgray,
solid,
forget plot
]
coordinates{
 (1089.05263157895,-4)(1011.26315789474,-4) 
};
\addplot [
color=darkgray,
solid,
forget plot
]
coordinates{
 (1089.05263157895,-6)(1011.26315789474,-4) 
};
\addplot [
color=darkgray,
solid,
forget plot
]
coordinates{
 (1089.05263157895,-8)(1011.26315789474,-4) 
};
\addplot [
color=darkgray,
solid,
forget plot
]
coordinates{
 (1089.05263157895,-10)(1011.26315789474,-4) 
};
\addplot [
color=darkgray,
solid,
forget plot
]
coordinates{
 (1089.05263157895,-12)(1011.26315789474,-4) 
};
\addplot [
color=darkgray,
solid,
forget plot
]
coordinates{
 (1089.05263157895,-14)(1011.26315789474,-4) 
};
\addplot [
color=darkgray,
solid,
forget plot
]
coordinates{
 (1089.05263157895,0)(1011.26315789474,-6) 
};
\addplot [
color=darkgray,
solid,
forget plot
]
coordinates{
 (1089.05263157895,-2)(1011.26315789474,-6) 
};
\addplot [
color=darkgray,
solid,
forget plot
]
coordinates{
 (1089.05263157895,-4)(1011.26315789474,-6) 
};
\addplot [
color=darkgray,
solid,
forget plot
]
coordinates{
 (1089.05263157895,-6)(1011.26315789474,-6) 
};
\addplot [
color=darkgray,
solid,
forget plot
]
coordinates{
 (1089.05263157895,-8)(1011.26315789474,-6) 
};
\addplot [
color=darkgray,
solid,
forget plot
]
coordinates{
 (1089.05263157895,-10)(1011.26315789474,-6) 
};
\addplot [
color=darkgray,
solid,
forget plot
]
coordinates{
 (1089.05263157895,-12)(1011.26315789474,-6) 
};
\addplot [
color=darkgray,
solid,
forget plot
]
coordinates{
 (1089.05263157895,-14)(1011.26315789474,-6) 
};
\addplot [
color=darkgray,
solid,
forget plot
]
coordinates{
 (1089.05263157895,-16)(1011.26315789474,-6) 
};
\addplot [
color=darkgray,
solid,
forget plot
]
coordinates{
 (1089.05263157895,0)(1011.26315789474,-8) 
};
\addplot [
color=darkgray,
solid,
forget plot
]
coordinates{
 (1089.05263157895,-2)(1011.26315789474,-8) 
};
\addplot [
color=darkgray,
solid,
forget plot
]
coordinates{
 (1089.05263157895,-4)(1011.26315789474,-8) 
};
\addplot [
color=darkgray,
solid,
forget plot
]
coordinates{
 (1089.05263157895,-6)(1011.26315789474,-8) 
};
\addplot [
color=darkgray,
solid,
forget plot
]
coordinates{
 (1089.05263157895,-8)(1011.26315789474,-8) 
};
\addplot [
color=darkgray,
solid,
forget plot
]
coordinates{
 (1089.05263157895,-10)(1011.26315789474,-8) 
};
\addplot [
color=darkgray,
solid,
forget plot
]
coordinates{
 (1089.05263157895,-12)(1011.26315789474,-8) 
};
\addplot [
color=darkgray,
solid,
forget plot
]
coordinates{
 (1089.05263157895,-14)(1011.26315789474,-8) 
};
\addplot [
color=darkgray,
solid,
forget plot
]
coordinates{
 (1089.05263157895,-16)(1011.26315789474,-8) 
};
\addplot [
color=darkgray,
solid,
forget plot
]
coordinates{
 (1089.05263157895,-18)(1011.26315789474,-8) 
};
\addplot [
color=darkgray,
solid,
forget plot
]
coordinates{
 (1089.05263157895,0)(1011.26315789474,-10) 
};
\addplot [
color=darkgray,
solid,
forget plot
]
coordinates{
 (1089.05263157895,-2)(1011.26315789474,-10) 
};
\addplot [
color=darkgray,
solid,
forget plot
]
coordinates{
 (1089.05263157895,-4)(1011.26315789474,-10) 
};
\addplot [
color=darkgray,
solid,
forget plot
]
coordinates{
 (1089.05263157895,-6)(1011.26315789474,-10) 
};
\addplot [
color=darkgray,
solid,
forget plot
]
coordinates{
 (1089.05263157895,-8)(1011.26315789474,-10) 
};
\addplot [
color=darkgray,
solid,
forget plot
]
coordinates{
 (1089.05263157895,-10)(1011.26315789474,-10) 
};
\addplot [
color=darkgray,
solid,
forget plot
]
coordinates{
 (1089.05263157895,-12)(1011.26315789474,-10) 
};
\addplot [
color=darkgray,
solid,
forget plot
]
coordinates{
 (1089.05263157895,-14)(1011.26315789474,-10) 
};
\addplot [
color=darkgray,
solid,
forget plot
]
coordinates{
 (1089.05263157895,-16)(1011.26315789474,-10) 
};
\addplot [
color=darkgray,
solid,
forget plot
]
coordinates{
 (1089.05263157895,-18)(1011.26315789474,-10) 
};
\addplot [
color=darkgray,
solid,
forget plot
]
coordinates{
 (1089.05263157895,-2)(1011.26315789474,-12) 
};
\addplot [
color=darkgray,
solid,
forget plot
]
coordinates{
 (1089.05263157895,-4)(1011.26315789474,-12) 
};
\addplot [
color=darkgray,
solid,
forget plot
]
coordinates{
 (1089.05263157895,-6)(1011.26315789474,-12) 
};
\addplot [
color=darkgray,
solid,
forget plot
]
coordinates{
 (1089.05263157895,-8)(1011.26315789474,-12) 
};
\addplot [
color=darkgray,
solid,
forget plot
]
coordinates{
 (1089.05263157895,-10)(1011.26315789474,-12) 
};
\addplot [
color=darkgray,
solid,
forget plot
]
coordinates{
 (1089.05263157895,-12)(1011.26315789474,-12) 
};
\addplot [
color=darkgray,
solid,
forget plot
]
coordinates{
 (1089.05263157895,-14)(1011.26315789474,-12) 
};
\addplot [
color=darkgray,
solid,
forget plot
]
coordinates{
 (1089.05263157895,-16)(1011.26315789474,-12) 
};
\addplot [
color=darkgray,
solid,
forget plot
]
coordinates{
 (1089.05263157895,-18)(1011.26315789474,-12) 
};
\addplot [
color=darkgray,
solid,
forget plot
]
coordinates{
 (1089.05263157895,-4)(1011.26315789474,-14) 
};
\addplot [
color=darkgray,
solid,
forget plot
]
coordinates{
 (1089.05263157895,-6)(1011.26315789474,-14) 
};
\addplot [
color=darkgray,
solid,
forget plot
]
coordinates{
 (1089.05263157895,-8)(1011.26315789474,-14) 
};
\addplot [
color=darkgray,
solid,
forget plot
]
coordinates{
 (1089.05263157895,-10)(1011.26315789474,-14) 
};
\addplot [
color=darkgray,
solid,
forget plot
]
coordinates{
 (1089.05263157895,-12)(1011.26315789474,-14) 
};
\addplot [
color=darkgray,
solid,
forget plot
]
coordinates{
 (1089.05263157895,-14)(1011.26315789474,-14) 
};
\addplot [
color=darkgray,
solid,
forget plot
]
coordinates{
 (1089.05263157895,-16)(1011.26315789474,-14) 
};
\addplot [
color=darkgray,
solid,
forget plot
]
coordinates{
 (1089.05263157895,-18)(1011.26315789474,-14) 
};
\addplot [
color=darkgray,
solid,
forget plot
]
coordinates{
 (1089.05263157895,-6)(1011.26315789474,-16) 
};
\addplot [
color=darkgray,
solid,
forget plot
]
coordinates{
 (1089.05263157895,-8)(1011.26315789474,-16) 
};
\addplot [
color=darkgray,
solid,
forget plot
]
coordinates{
 (1089.05263157895,-10)(1011.26315789474,-16) 
};
\addplot [
color=darkgray,
solid,
forget plot
]
coordinates{
 (1089.05263157895,-12)(1011.26315789474,-16) 
};
\addplot [
color=darkgray,
solid,
forget plot
]
coordinates{
 (1089.05263157895,-14)(1011.26315789474,-16) 
};
\addplot [
color=darkgray,
solid,
forget plot
]
coordinates{
 (1089.05263157895,-16)(1011.26315789474,-16) 
};
\addplot [
color=darkgray,
solid,
forget plot
]
coordinates{
 (1089.05263157895,-18)(1011.26315789474,-16) 
};
\addplot [
color=darkgray,
solid,
forget plot
]
coordinates{
 (1089.05263157895,-8)(1011.26315789474,-18) 
};
\addplot [
color=darkgray,
solid,
forget plot
]
coordinates{
 (1089.05263157895,-10)(1011.26315789474,-18) 
};
\addplot [
color=darkgray,
solid,
forget plot
]
coordinates{
 (1089.05263157895,-12)(1011.26315789474,-18) 
};
\addplot [
color=darkgray,
solid,
forget plot
]
coordinates{
 (1089.05263157895,-14)(1011.26315789474,-18) 
};
\addplot [
color=darkgray,
solid,
forget plot
]
coordinates{
 (1089.05263157895,-16)(1011.26315789474,-18) 
};
\addplot [
color=darkgray,
solid,
forget plot
]
coordinates{
 (1089.05263157895,-18)(1011.26315789474,-18) 
};
\addplot [
color=darkgray,
solid,
forget plot
]
coordinates{
 (1011.26315789474,0)(933.473684210526,0) 
};
\addplot [
color=darkgray,
solid,
forget plot
]
coordinates{
 (1011.26315789474,-2)(933.473684210526,0) 
};
\addplot [
color=darkgray,
solid,
forget plot
]
coordinates{
 (1011.26315789474,-4)(933.473684210526,0) 
};
\addplot [
color=darkgray,
solid,
forget plot
]
coordinates{
 (1011.26315789474,-6)(933.473684210526,0) 
};
\addplot [
color=darkgray,
solid,
forget plot
]
coordinates{
 (1011.26315789474,-8)(933.473684210526,0) 
};
\addplot [
color=darkgray,
solid,
forget plot
]
coordinates{
 (1011.26315789474,-10)(933.473684210526,0) 
};
\addplot [
color=darkgray,
solid,
forget plot
]
coordinates{
 (1011.26315789474,0)(933.473684210526,-2) 
};
\addplot [
color=darkgray,
solid,
forget plot
]
coordinates{
 (1011.26315789474,-2)(933.473684210526,-2) 
};
\addplot [
color=darkgray,
solid,
forget plot
]
coordinates{
 (1011.26315789474,-4)(933.473684210526,-2) 
};
\addplot [
color=darkgray,
solid,
forget plot
]
coordinates{
 (1011.26315789474,-6)(933.473684210526,-2) 
};
\addplot [
color=darkgray,
solid,
forget plot
]
coordinates{
 (1011.26315789474,-8)(933.473684210526,-2) 
};
\addplot [
color=darkgray,
solid,
forget plot
]
coordinates{
 (1011.26315789474,-10)(933.473684210526,-2) 
};
\addplot [
color=darkgray,
solid,
forget plot
]
coordinates{
 (1011.26315789474,-12)(933.473684210526,-2) 
};
\addplot [
color=darkgray,
solid,
forget plot
]
coordinates{
 (1011.26315789474,0)(933.473684210526,-4) 
};
\addplot [
color=darkgray,
solid,
forget plot
]
coordinates{
 (1011.26315789474,-2)(933.473684210526,-4) 
};
\addplot [
color=darkgray,
solid,
forget plot
]
coordinates{
 (1011.26315789474,-4)(933.473684210526,-4) 
};
\addplot [
color=darkgray,
solid,
forget plot
]
coordinates{
 (1011.26315789474,-6)(933.473684210526,-4) 
};
\addplot [
color=darkgray,
solid,
forget plot
]
coordinates{
 (1011.26315789474,-8)(933.473684210526,-4) 
};
\addplot [
color=darkgray,
solid,
forget plot
]
coordinates{
 (1011.26315789474,-10)(933.473684210526,-4) 
};
\addplot [
color=darkgray,
solid,
forget plot
]
coordinates{
 (1011.26315789474,-12)(933.473684210526,-4) 
};
\addplot [
color=darkgray,
solid,
forget plot
]
coordinates{
 (1011.26315789474,-14)(933.473684210526,-4) 
};
\addplot [
color=darkgray,
solid,
forget plot
]
coordinates{
 (1011.26315789474,0)(933.473684210526,-6) 
};
\addplot [
color=darkgray,
solid,
forget plot
]
coordinates{
 (1011.26315789474,-2)(933.473684210526,-6) 
};
\addplot [
color=darkgray,
solid,
forget plot
]
coordinates{
 (1011.26315789474,-4)(933.473684210526,-6) 
};
\addplot [
color=darkgray,
solid,
forget plot
]
coordinates{
 (1011.26315789474,-6)(933.473684210526,-6) 
};
\addplot [
color=darkgray,
solid,
forget plot
]
coordinates{
 (1011.26315789474,-8)(933.473684210526,-6) 
};
\addplot [
color=darkgray,
solid,
forget plot
]
coordinates{
 (1011.26315789474,-10)(933.473684210526,-6) 
};
\addplot [
color=darkgray,
solid,
forget plot
]
coordinates{
 (1011.26315789474,-12)(933.473684210526,-6) 
};
\addplot [
color=darkgray,
solid,
forget plot
]
coordinates{
 (1011.26315789474,-14)(933.473684210526,-6) 
};
\addplot [
color=darkgray,
solid,
forget plot
]
coordinates{
 (1011.26315789474,-16)(933.473684210526,-6) 
};
\addplot [
color=darkgray,
solid,
forget plot
]
coordinates{
 (1011.26315789474,0)(933.473684210526,-8) 
};
\addplot [
color=darkgray,
solid,
forget plot
]
coordinates{
 (1011.26315789474,-2)(933.473684210526,-8) 
};
\addplot [
color=darkgray,
solid,
forget plot
]
coordinates{
 (1011.26315789474,-4)(933.473684210526,-8) 
};
\addplot [
color=darkgray,
solid,
forget plot
]
coordinates{
 (1011.26315789474,-6)(933.473684210526,-8) 
};
\addplot [
color=darkgray,
solid,
forget plot
]
coordinates{
 (1011.26315789474,-8)(933.473684210526,-8) 
};
\addplot [
color=darkgray,
solid,
forget plot
]
coordinates{
 (1011.26315789474,-10)(933.473684210526,-8) 
};
\addplot [
color=darkgray,
solid,
forget plot
]
coordinates{
 (1011.26315789474,-12)(933.473684210526,-8) 
};
\addplot [
color=darkgray,
solid,
forget plot
]
coordinates{
 (1011.26315789474,-14)(933.473684210526,-8) 
};
\addplot [
color=darkgray,
solid,
forget plot
]
coordinates{
 (1011.26315789474,-16)(933.473684210526,-8) 
};
\addplot [
color=darkgray,
solid,
forget plot
]
coordinates{
 (1011.26315789474,-18)(933.473684210526,-8) 
};
\addplot [
color=darkgray,
solid,
forget plot
]
coordinates{
 (1011.26315789474,0)(933.473684210526,-10) 
};
\addplot [
color=darkgray,
solid,
forget plot
]
coordinates{
 (1011.26315789474,-2)(933.473684210526,-10) 
};
\addplot [
color=darkgray,
solid,
forget plot
]
coordinates{
 (1011.26315789474,-4)(933.473684210526,-10) 
};
\addplot [
color=darkgray,
solid,
forget plot
]
coordinates{
 (1011.26315789474,-6)(933.473684210526,-10) 
};
\addplot [
color=darkgray,
solid,
forget plot
]
coordinates{
 (1011.26315789474,-8)(933.473684210526,-10) 
};
\addplot [
color=darkgray,
solid,
forget plot
]
coordinates{
 (1011.26315789474,-10)(933.473684210526,-10) 
};
\addplot [
color=darkgray,
solid,
forget plot
]
coordinates{
 (1011.26315789474,-12)(933.473684210526,-10) 
};
\addplot [
color=darkgray,
solid,
forget plot
]
coordinates{
 (1011.26315789474,-14)(933.473684210526,-10) 
};
\addplot [
color=darkgray,
solid,
forget plot
]
coordinates{
 (1011.26315789474,-16)(933.473684210526,-10) 
};
\addplot [
color=darkgray,
solid,
forget plot
]
coordinates{
 (1011.26315789474,-18)(933.473684210526,-10) 
};
\addplot [
color=darkgray,
solid,
forget plot
]
coordinates{
 (1011.26315789474,-2)(933.473684210526,-12) 
};
\addplot [
color=darkgray,
solid,
forget plot
]
coordinates{
 (1011.26315789474,-4)(933.473684210526,-12) 
};
\addplot [
color=darkgray,
solid,
forget plot
]
coordinates{
 (1011.26315789474,-6)(933.473684210526,-12) 
};
\addplot [
color=darkgray,
solid,
forget plot
]
coordinates{
 (1011.26315789474,-8)(933.473684210526,-12) 
};
\addplot [
color=darkgray,
solid,
forget plot
]
coordinates{
 (1011.26315789474,-10)(933.473684210526,-12) 
};
\addplot [
color=darkgray,
solid,
forget plot
]
coordinates{
 (1011.26315789474,-12)(933.473684210526,-12) 
};
\addplot [
color=darkgray,
solid,
forget plot
]
coordinates{
 (1011.26315789474,-14)(933.473684210526,-12) 
};
\addplot [
color=darkgray,
solid,
forget plot
]
coordinates{
 (1011.26315789474,-16)(933.473684210526,-12) 
};
\addplot [
color=darkgray,
solid,
forget plot
]
coordinates{
 (1011.26315789474,-18)(933.473684210526,-12) 
};
\addplot [
color=darkgray,
solid,
forget plot
]
coordinates{
 (1011.26315789474,-4)(933.473684210526,-14) 
};
\addplot [
color=darkgray,
solid,
forget plot
]
coordinates{
 (1011.26315789474,-6)(933.473684210526,-14) 
};
\addplot [
color=darkgray,
solid,
forget plot
]
coordinates{
 (1011.26315789474,-8)(933.473684210526,-14) 
};
\addplot [
color=darkgray,
solid,
forget plot
]
coordinates{
 (1011.26315789474,-10)(933.473684210526,-14) 
};
\addplot [
color=darkgray,
solid,
forget plot
]
coordinates{
 (1011.26315789474,-12)(933.473684210526,-14) 
};
\addplot [
color=darkgray,
solid,
forget plot
]
coordinates{
 (1011.26315789474,-14)(933.473684210526,-14) 
};
\addplot [
color=darkgray,
solid,
forget plot
]
coordinates{
 (1011.26315789474,-16)(933.473684210526,-14) 
};
\addplot [
color=darkgray,
solid,
forget plot
]
coordinates{
 (1011.26315789474,-18)(933.473684210526,-14) 
};
\addplot [
color=darkgray,
solid,
forget plot
]
coordinates{
 (1011.26315789474,-6)(933.473684210526,-16) 
};
\addplot [
color=darkgray,
solid,
forget plot
]
coordinates{
 (1011.26315789474,-8)(933.473684210526,-16) 
};
\addplot [
color=darkgray,
solid,
forget plot
]
coordinates{
 (1011.26315789474,-10)(933.473684210526,-16) 
};
\addplot [
color=darkgray,
solid,
forget plot
]
coordinates{
 (1011.26315789474,-12)(933.473684210526,-16) 
};
\addplot [
color=darkgray,
solid,
forget plot
]
coordinates{
 (1011.26315789474,-14)(933.473684210526,-16) 
};
\addplot [
color=darkgray,
solid,
forget plot
]
coordinates{
 (1011.26315789474,-16)(933.473684210526,-16) 
};
\addplot [
color=darkgray,
solid,
forget plot
]
coordinates{
 (1011.26315789474,-18)(933.473684210526,-16) 
};
\addplot [
color=darkgray,
solid,
forget plot
]
coordinates{
 (1011.26315789474,-8)(933.473684210526,-18) 
};
\addplot [
color=darkgray,
solid,
forget plot
]
coordinates{
 (1011.26315789474,-10)(933.473684210526,-18) 
};
\addplot [
color=darkgray,
solid,
forget plot
]
coordinates{
 (1011.26315789474,-12)(933.473684210526,-18) 
};
\addplot [
color=darkgray,
solid,
forget plot
]
coordinates{
 (1011.26315789474,-14)(933.473684210526,-18) 
};
\addplot [
color=darkgray,
solid,
forget plot
]
coordinates{
 (1011.26315789474,-16)(933.473684210526,-18) 
};
\addplot [
color=darkgray,
solid,
forget plot
]
coordinates{
 (1011.26315789474,-18)(933.473684210526,-18) 
};
\addplot [
color=darkgray,
solid,
forget plot
]
coordinates{
 (933.473684210526,0)(855.684210526316,0) 
};
\addplot [
color=darkgray,
solid,
forget plot
]
coordinates{
 (933.473684210526,-2)(855.684210526316,0) 
};
\addplot [
color=darkgray,
solid,
forget plot
]
coordinates{
 (933.473684210526,-4)(855.684210526316,0) 
};
\addplot [
color=darkgray,
solid,
forget plot
]
coordinates{
 (933.473684210526,-6)(855.684210526316,0) 
};
\addplot [
color=darkgray,
solid,
forget plot
]
coordinates{
 (933.473684210526,-8)(855.684210526316,0) 
};
\addplot [
color=darkgray,
solid,
forget plot
]
coordinates{
 (933.473684210526,-10)(855.684210526316,0) 
};
\addplot [
color=darkgray,
solid,
forget plot
]
coordinates{
 (933.473684210526,0)(855.684210526316,-2) 
};
\addplot [
color=darkgray,
solid,
forget plot
]
coordinates{
 (933.473684210526,-2)(855.684210526316,-2) 
};
\addplot [
color=darkgray,
solid,
forget plot
]
coordinates{
 (933.473684210526,-4)(855.684210526316,-2) 
};
\addplot [
color=darkgray,
solid,
forget plot
]
coordinates{
 (933.473684210526,-6)(855.684210526316,-2) 
};
\addplot [
color=darkgray,
solid,
forget plot
]
coordinates{
 (933.473684210526,-8)(855.684210526316,-2) 
};
\addplot [
color=darkgray,
solid,
forget plot
]
coordinates{
 (933.473684210526,-10)(855.684210526316,-2) 
};
\addplot [
color=darkgray,
solid,
forget plot
]
coordinates{
 (933.473684210526,-12)(855.684210526316,-2) 
};
\addplot [
color=darkgray,
solid,
forget plot
]
coordinates{
 (933.473684210526,0)(855.684210526316,-4) 
};
\addplot [
color=darkgray,
solid,
forget plot
]
coordinates{
 (933.473684210526,-2)(855.684210526316,-4) 
};
\addplot [
color=darkgray,
solid,
forget plot
]
coordinates{
 (933.473684210526,-4)(855.684210526316,-4) 
};
\addplot [
color=darkgray,
solid,
forget plot
]
coordinates{
 (933.473684210526,-6)(855.684210526316,-4) 
};
\addplot [
color=darkgray,
solid,
forget plot
]
coordinates{
 (933.473684210526,-8)(855.684210526316,-4) 
};
\addplot [
color=darkgray,
solid,
forget plot
]
coordinates{
 (933.473684210526,-10)(855.684210526316,-4) 
};
\addplot [
color=darkgray,
solid,
forget plot
]
coordinates{
 (933.473684210526,-12)(855.684210526316,-4) 
};
\addplot [
color=darkgray,
solid,
forget plot
]
coordinates{
 (933.473684210526,-14)(855.684210526316,-4) 
};
\addplot [
color=darkgray,
solid,
forget plot
]
coordinates{
 (933.473684210526,0)(855.684210526316,-6) 
};
\addplot [
color=darkgray,
solid,
forget plot
]
coordinates{
 (933.473684210526,-2)(855.684210526316,-6) 
};
\addplot [
color=darkgray,
solid,
forget plot
]
coordinates{
 (933.473684210526,-4)(855.684210526316,-6) 
};
\addplot [
color=darkgray,
solid,
forget plot
]
coordinates{
 (933.473684210526,-6)(855.684210526316,-6) 
};
\addplot [
color=darkgray,
solid,
forget plot
]
coordinates{
 (933.473684210526,-8)(855.684210526316,-6) 
};
\addplot [
color=darkgray,
solid,
forget plot
]
coordinates{
 (933.473684210526,-10)(855.684210526316,-6) 
};
\addplot [
color=darkgray,
solid,
forget plot
]
coordinates{
 (933.473684210526,-12)(855.684210526316,-6) 
};
\addplot [
color=darkgray,
solid,
forget plot
]
coordinates{
 (933.473684210526,-14)(855.684210526316,-6) 
};
\addplot [
color=darkgray,
solid,
forget plot
]
coordinates{
 (933.473684210526,-16)(855.684210526316,-6) 
};
\addplot [
color=darkgray,
solid,
forget plot
]
coordinates{
 (933.473684210526,0)(855.684210526316,-8) 
};
\addplot [
color=darkgray,
solid,
forget plot
]
coordinates{
 (933.473684210526,-2)(855.684210526316,-8) 
};
\addplot [
color=darkgray,
solid,
forget plot
]
coordinates{
 (933.473684210526,-4)(855.684210526316,-8) 
};
\addplot [
color=darkgray,
solid,
forget plot
]
coordinates{
 (933.473684210526,-6)(855.684210526316,-8) 
};
\addplot [
color=darkgray,
solid,
forget plot
]
coordinates{
 (933.473684210526,-8)(855.684210526316,-8) 
};
\addplot [
color=darkgray,
solid,
forget plot
]
coordinates{
 (933.473684210526,-10)(855.684210526316,-8) 
};
\addplot [
color=darkgray,
solid,
forget plot
]
coordinates{
 (933.473684210526,-12)(855.684210526316,-8) 
};
\addplot [
color=darkgray,
solid,
forget plot
]
coordinates{
 (933.473684210526,-14)(855.684210526316,-8) 
};
\addplot [
color=darkgray,
solid,
forget plot
]
coordinates{
 (933.473684210526,-16)(855.684210526316,-8) 
};
\addplot [
color=darkgray,
solid,
forget plot
]
coordinates{
 (933.473684210526,-18)(855.684210526316,-8) 
};
\addplot [
color=darkgray,
solid,
forget plot
]
coordinates{
 (933.473684210526,0)(855.684210526316,-10) 
};
\addplot [
color=darkgray,
solid,
forget plot
]
coordinates{
 (933.473684210526,-2)(855.684210526316,-10) 
};
\addplot [
color=darkgray,
solid,
forget plot
]
coordinates{
 (933.473684210526,-4)(855.684210526316,-10) 
};
\addplot [
color=darkgray,
solid,
forget plot
]
coordinates{
 (933.473684210526,-6)(855.684210526316,-10) 
};
\addplot [
color=darkgray,
solid,
forget plot
]
coordinates{
 (933.473684210526,-8)(855.684210526316,-10) 
};
\addplot [
color=darkgray,
solid,
forget plot
]
coordinates{
 (933.473684210526,-10)(855.684210526316,-10) 
};
\addplot [
color=darkgray,
solid,
forget plot
]
coordinates{
 (933.473684210526,-12)(855.684210526316,-10) 
};
\addplot [
color=darkgray,
solid,
forget plot
]
coordinates{
 (933.473684210526,-14)(855.684210526316,-10) 
};
\addplot [
color=darkgray,
solid,
forget plot
]
coordinates{
 (933.473684210526,-16)(855.684210526316,-10) 
};
\addplot [
color=darkgray,
solid,
forget plot
]
coordinates{
 (933.473684210526,-18)(855.684210526316,-10) 
};
\addplot [
color=darkgray,
solid,
forget plot
]
coordinates{
 (933.473684210526,-2)(855.684210526316,-12) 
};
\addplot [
color=darkgray,
solid,
forget plot
]
coordinates{
 (933.473684210526,-4)(855.684210526316,-12) 
};
\addplot [
color=darkgray,
solid,
forget plot
]
coordinates{
 (933.473684210526,-6)(855.684210526316,-12) 
};
\addplot [
color=darkgray,
solid,
forget plot
]
coordinates{
 (933.473684210526,-8)(855.684210526316,-12) 
};
\addplot [
color=darkgray,
solid,
forget plot
]
coordinates{
 (933.473684210526,-10)(855.684210526316,-12) 
};
\addplot [
color=darkgray,
solid,
forget plot
]
coordinates{
 (933.473684210526,-12)(855.684210526316,-12) 
};
\addplot [
color=darkgray,
solid,
forget plot
]
coordinates{
 (933.473684210526,-14)(855.684210526316,-12) 
};
\addplot [
color=darkgray,
solid,
forget plot
]
coordinates{
 (933.473684210526,-16)(855.684210526316,-12) 
};
\addplot [
color=darkgray,
solid,
forget plot
]
coordinates{
 (933.473684210526,-18)(855.684210526316,-12) 
};
\addplot [
color=darkgray,
solid,
forget plot
]
coordinates{
 (933.473684210526,-4)(855.684210526316,-14) 
};
\addplot [
color=darkgray,
solid,
forget plot
]
coordinates{
 (933.473684210526,-6)(855.684210526316,-14) 
};
\addplot [
color=darkgray,
solid,
forget plot
]
coordinates{
 (933.473684210526,-8)(855.684210526316,-14) 
};
\addplot [
color=darkgray,
solid,
forget plot
]
coordinates{
 (933.473684210526,-10)(855.684210526316,-14) 
};
\addplot [
color=darkgray,
solid,
forget plot
]
coordinates{
 (933.473684210526,-12)(855.684210526316,-14) 
};
\addplot [
color=darkgray,
solid,
forget plot
]
coordinates{
 (933.473684210526,-14)(855.684210526316,-14) 
};
\addplot [
color=darkgray,
solid,
forget plot
]
coordinates{
 (933.473684210526,-16)(855.684210526316,-14) 
};
\addplot [
color=darkgray,
solid,
forget plot
]
coordinates{
 (933.473684210526,-18)(855.684210526316,-14) 
};
\addplot [
color=darkgray,
solid,
forget plot
]
coordinates{
 (933.473684210526,-6)(855.684210526316,-16) 
};
\addplot [
color=darkgray,
solid,
forget plot
]
coordinates{
 (933.473684210526,-8)(855.684210526316,-16) 
};
\addplot [
color=darkgray,
solid,
forget plot
]
coordinates{
 (933.473684210526,-10)(855.684210526316,-16) 
};
\addplot [
color=darkgray,
solid,
forget plot
]
coordinates{
 (933.473684210526,-12)(855.684210526316,-16) 
};
\addplot [
color=darkgray,
solid,
forget plot
]
coordinates{
 (933.473684210526,-14)(855.684210526316,-16) 
};
\addplot [
color=darkgray,
solid,
forget plot
]
coordinates{
 (933.473684210526,-16)(855.684210526316,-16) 
};
\addplot [
color=darkgray,
solid,
forget plot
]
coordinates{
 (933.473684210526,-18)(855.684210526316,-16) 
};
\addplot [
color=darkgray,
solid,
forget plot
]
coordinates{
 (933.473684210526,-8)(855.684210526316,-18) 
};
\addplot [
color=darkgray,
solid,
forget plot
]
coordinates{
 (933.473684210526,-10)(855.684210526316,-18) 
};
\addplot [
color=darkgray,
solid,
forget plot
]
coordinates{
 (933.473684210526,-12)(855.684210526316,-18) 
};
\addplot [
color=darkgray,
solid,
forget plot
]
coordinates{
 (933.473684210526,-14)(855.684210526316,-18) 
};
\addplot [
color=darkgray,
solid,
forget plot
]
coordinates{
 (933.473684210526,-16)(855.684210526316,-18) 
};
\addplot [
color=darkgray,
solid,
forget plot
]
coordinates{
 (933.473684210526,-18)(855.684210526316,-18) 
};
\addplot [
color=darkgray,
solid,
forget plot
]
coordinates{
 (855.684210526316,0)(777.894736842105,0) 
};
\addplot [
color=darkgray,
solid,
forget plot
]
coordinates{
 (855.684210526316,-2)(777.894736842105,0) 
};
\addplot [
color=darkgray,
solid,
forget plot
]
coordinates{
 (855.684210526316,-4)(777.894736842105,0) 
};
\addplot [
color=darkgray,
solid,
forget plot
]
coordinates{
 (855.684210526316,-6)(777.894736842105,0) 
};
\addplot [
color=darkgray,
solid,
forget plot
]
coordinates{
 (855.684210526316,-8)(777.894736842105,0) 
};
\addplot [
color=darkgray,
solid,
forget plot
]
coordinates{
 (855.684210526316,-10)(777.894736842105,0) 
};
\addplot [
color=darkgray,
solid,
forget plot
]
coordinates{
 (855.684210526316,0)(777.894736842105,-2) 
};
\addplot [
color=darkgray,
solid,
forget plot
]
coordinates{
 (855.684210526316,-2)(777.894736842105,-2) 
};
\addplot [
color=darkgray,
solid,
forget plot
]
coordinates{
 (855.684210526316,-4)(777.894736842105,-2) 
};
\addplot [
color=darkgray,
solid,
forget plot
]
coordinates{
 (855.684210526316,-6)(777.894736842105,-2) 
};
\addplot [
color=darkgray,
solid,
forget plot
]
coordinates{
 (855.684210526316,-8)(777.894736842105,-2) 
};
\addplot [
color=darkgray,
solid,
forget plot
]
coordinates{
 (855.684210526316,-10)(777.894736842105,-2) 
};
\addplot [
color=darkgray,
solid,
forget plot
]
coordinates{
 (855.684210526316,-12)(777.894736842105,-2) 
};
\addplot [
color=darkgray,
solid,
forget plot
]
coordinates{
 (855.684210526316,0)(777.894736842105,-4) 
};
\addplot [
color=darkgray,
solid,
forget plot
]
coordinates{
 (855.684210526316,-2)(777.894736842105,-4) 
};
\addplot [
color=darkgray,
solid,
forget plot
]
coordinates{
 (855.684210526316,-4)(777.894736842105,-4) 
};
\addplot [
color=darkgray,
solid,
forget plot
]
coordinates{
 (855.684210526316,-6)(777.894736842105,-4) 
};
\addplot [
color=darkgray,
solid,
forget plot
]
coordinates{
 (855.684210526316,-8)(777.894736842105,-4) 
};
\addplot [
color=darkgray,
solid,
forget plot
]
coordinates{
 (855.684210526316,-10)(777.894736842105,-4) 
};
\addplot [
color=darkgray,
solid,
forget plot
]
coordinates{
 (855.684210526316,-12)(777.894736842105,-4) 
};
\addplot [
color=darkgray,
solid,
forget plot
]
coordinates{
 (855.684210526316,-14)(777.894736842105,-4) 
};
\addplot [
color=darkgray,
solid,
forget plot
]
coordinates{
 (855.684210526316,0)(777.894736842105,-6) 
};
\addplot [
color=darkgray,
solid,
forget plot
]
coordinates{
 (855.684210526316,-2)(777.894736842105,-6) 
};
\addplot [
color=darkgray,
solid,
forget plot
]
coordinates{
 (855.684210526316,-4)(777.894736842105,-6) 
};
\addplot [
color=darkgray,
solid,
forget plot
]
coordinates{
 (855.684210526316,-6)(777.894736842105,-6) 
};
\addplot [
color=darkgray,
solid,
forget plot
]
coordinates{
 (855.684210526316,-8)(777.894736842105,-6) 
};
\addplot [
color=darkgray,
solid,
forget plot
]
coordinates{
 (855.684210526316,-10)(777.894736842105,-6) 
};
\addplot [
color=darkgray,
solid,
forget plot
]
coordinates{
 (855.684210526316,-12)(777.894736842105,-6) 
};
\addplot [
color=darkgray,
solid,
forget plot
]
coordinates{
 (855.684210526316,-14)(777.894736842105,-6) 
};
\addplot [
color=darkgray,
solid,
forget plot
]
coordinates{
 (855.684210526316,-16)(777.894736842105,-6) 
};
\addplot [
color=darkgray,
solid,
forget plot
]
coordinates{
 (855.684210526316,0)(777.894736842105,-8) 
};
\addplot [
color=darkgray,
solid,
forget plot
]
coordinates{
 (855.684210526316,-2)(777.894736842105,-8) 
};
\addplot [
color=darkgray,
solid,
forget plot
]
coordinates{
 (855.684210526316,-4)(777.894736842105,-8) 
};
\addplot [
color=darkgray,
solid,
forget plot
]
coordinates{
 (855.684210526316,-6)(777.894736842105,-8) 
};
\addplot [
color=darkgray,
solid,
forget plot
]
coordinates{
 (855.684210526316,-8)(777.894736842105,-8) 
};
\addplot [
color=darkgray,
solid,
forget plot
]
coordinates{
 (855.684210526316,-10)(777.894736842105,-8) 
};
\addplot [
color=darkgray,
solid,
forget plot
]
coordinates{
 (855.684210526316,-12)(777.894736842105,-8) 
};
\addplot [
color=darkgray,
solid,
forget plot
]
coordinates{
 (855.684210526316,-14)(777.894736842105,-8) 
};
\addplot [
color=darkgray,
solid,
forget plot
]
coordinates{
 (855.684210526316,-16)(777.894736842105,-8) 
};
\addplot [
color=darkgray,
solid,
forget plot
]
coordinates{
 (855.684210526316,-18)(777.894736842105,-8) 
};
\addplot [
color=darkgray,
solid,
forget plot
]
coordinates{
 (855.684210526316,0)(777.894736842105,-10) 
};
\addplot [
color=darkgray,
solid,
forget plot
]
coordinates{
 (855.684210526316,-2)(777.894736842105,-10) 
};
\addplot [
color=darkgray,
solid,
forget plot
]
coordinates{
 (855.684210526316,-4)(777.894736842105,-10) 
};
\addplot [
color=darkgray,
solid,
forget plot
]
coordinates{
 (855.684210526316,-6)(777.894736842105,-10) 
};
\addplot [
color=darkgray,
solid,
forget plot
]
coordinates{
 (855.684210526316,-8)(777.894736842105,-10) 
};
\addplot [
color=darkgray,
solid,
forget plot
]
coordinates{
 (855.684210526316,-10)(777.894736842105,-10) 
};
\addplot [
color=darkgray,
solid,
forget plot
]
coordinates{
 (855.684210526316,-12)(777.894736842105,-10) 
};
\addplot [
color=darkgray,
solid,
forget plot
]
coordinates{
 (855.684210526316,-14)(777.894736842105,-10) 
};
\addplot [
color=darkgray,
solid,
forget plot
]
coordinates{
 (855.684210526316,-16)(777.894736842105,-10) 
};
\addplot [
color=darkgray,
solid,
forget plot
]
coordinates{
 (855.684210526316,-18)(777.894736842105,-10) 
};
\addplot [
color=darkgray,
solid,
forget plot
]
coordinates{
 (855.684210526316,-2)(777.894736842105,-12) 
};
\addplot [
color=darkgray,
solid,
forget plot
]
coordinates{
 (855.684210526316,-4)(777.894736842105,-12) 
};
\addplot [
color=darkgray,
solid,
forget plot
]
coordinates{
 (855.684210526316,-6)(777.894736842105,-12) 
};
\addplot [
color=darkgray,
solid,
forget plot
]
coordinates{
 (855.684210526316,-8)(777.894736842105,-12) 
};
\addplot [
color=darkgray,
solid,
forget plot
]
coordinates{
 (855.684210526316,-10)(777.894736842105,-12) 
};
\addplot [
color=darkgray,
solid,
forget plot
]
coordinates{
 (855.684210526316,-12)(777.894736842105,-12) 
};
\addplot [
color=darkgray,
solid,
forget plot
]
coordinates{
 (855.684210526316,-14)(777.894736842105,-12) 
};
\addplot [
color=darkgray,
solid,
forget plot
]
coordinates{
 (855.684210526316,-16)(777.894736842105,-12) 
};
\addplot [
color=darkgray,
solid,
forget plot
]
coordinates{
 (855.684210526316,-18)(777.894736842105,-12) 
};
\addplot [
color=darkgray,
solid,
forget plot
]
coordinates{
 (855.684210526316,-4)(777.894736842105,-14) 
};
\addplot [
color=darkgray,
solid,
forget plot
]
coordinates{
 (855.684210526316,-6)(777.894736842105,-14) 
};
\addplot [
color=darkgray,
solid,
forget plot
]
coordinates{
 (855.684210526316,-8)(777.894736842105,-14) 
};
\addplot [
color=darkgray,
solid,
forget plot
]
coordinates{
 (855.684210526316,-10)(777.894736842105,-14) 
};
\addplot [
color=darkgray,
solid,
forget plot
]
coordinates{
 (855.684210526316,-12)(777.894736842105,-14) 
};
\addplot [
color=darkgray,
solid,
forget plot
]
coordinates{
 (855.684210526316,-14)(777.894736842105,-14) 
};
\addplot [
color=darkgray,
solid,
forget plot
]
coordinates{
 (855.684210526316,-16)(777.894736842105,-14) 
};
\addplot [
color=darkgray,
solid,
forget plot
]
coordinates{
 (855.684210526316,-18)(777.894736842105,-14) 
};
\addplot [
color=darkgray,
solid,
forget plot
]
coordinates{
 (855.684210526316,-6)(777.894736842105,-16) 
};
\addplot [
color=darkgray,
solid,
forget plot
]
coordinates{
 (855.684210526316,-8)(777.894736842105,-16) 
};
\addplot [
color=darkgray,
solid,
forget plot
]
coordinates{
 (855.684210526316,-10)(777.894736842105,-16) 
};
\addplot [
color=darkgray,
solid,
forget plot
]
coordinates{
 (855.684210526316,-12)(777.894736842105,-16) 
};
\addplot [
color=darkgray,
solid,
forget plot
]
coordinates{
 (855.684210526316,-14)(777.894736842105,-16) 
};
\addplot [
color=darkgray,
solid,
forget plot
]
coordinates{
 (855.684210526316,-16)(777.894736842105,-16) 
};
\addplot [
color=darkgray,
solid,
forget plot
]
coordinates{
 (855.684210526316,-18)(777.894736842105,-16) 
};
\addplot [
color=darkgray,
solid,
forget plot
]
coordinates{
 (855.684210526316,-8)(777.894736842105,-18) 
};
\addplot [
color=darkgray,
solid,
forget plot
]
coordinates{
 (855.684210526316,-10)(777.894736842105,-18) 
};
\addplot [
color=darkgray,
solid,
forget plot
]
coordinates{
 (855.684210526316,-12)(777.894736842105,-18) 
};
\addplot [
color=darkgray,
solid,
forget plot
]
coordinates{
 (855.684210526316,-14)(777.894736842105,-18) 
};
\addplot [
color=darkgray,
solid,
forget plot
]
coordinates{
 (855.684210526316,-16)(777.894736842105,-18) 
};
\addplot [
color=darkgray,
solid,
forget plot
]
coordinates{
 (855.684210526316,-18)(777.894736842105,-18) 
};
\addplot [
color=darkgray,
solid,
forget plot
]
coordinates{
 (777.894736842105,0)(700.105263157895,0) 
};
\addplot [
color=darkgray,
solid,
forget plot
]
coordinates{
 (777.894736842105,-2)(700.105263157895,0) 
};
\addplot [
color=darkgray,
solid,
forget plot
]
coordinates{
 (777.894736842105,-4)(700.105263157895,0) 
};
\addplot [
color=darkgray,
solid,
forget plot
]
coordinates{
 (777.894736842105,-6)(700.105263157895,0) 
};
\addplot [
color=darkgray,
solid,
forget plot
]
coordinates{
 (777.894736842105,-8)(700.105263157895,0) 
};
\addplot [
color=darkgray,
solid,
forget plot
]
coordinates{
 (777.894736842105,-10)(700.105263157895,0) 
};
\addplot [
color=darkgray,
solid,
forget plot
]
coordinates{
 (777.894736842105,0)(700.105263157895,-2) 
};
\addplot [
color=darkgray,
solid,
forget plot
]
coordinates{
 (777.894736842105,-2)(700.105263157895,-2) 
};
\addplot [
color=darkgray,
solid,
forget plot
]
coordinates{
 (777.894736842105,-4)(700.105263157895,-2) 
};
\addplot [
color=darkgray,
solid,
forget plot
]
coordinates{
 (777.894736842105,-6)(700.105263157895,-2) 
};
\addplot [
color=darkgray,
solid,
forget plot
]
coordinates{
 (777.894736842105,-8)(700.105263157895,-2) 
};
\addplot [
color=darkgray,
solid,
forget plot
]
coordinates{
 (777.894736842105,-10)(700.105263157895,-2) 
};
\addplot [
color=darkgray,
solid,
forget plot
]
coordinates{
 (777.894736842105,-12)(700.105263157895,-2) 
};
\addplot [
color=darkgray,
solid,
forget plot
]
coordinates{
 (777.894736842105,0)(700.105263157895,-4) 
};
\addplot [
color=darkgray,
solid,
forget plot
]
coordinates{
 (777.894736842105,-2)(700.105263157895,-4) 
};
\addplot [
color=darkgray,
solid,
forget plot
]
coordinates{
 (777.894736842105,-4)(700.105263157895,-4) 
};
\addplot [
color=darkgray,
solid,
forget plot
]
coordinates{
 (777.894736842105,-6)(700.105263157895,-4) 
};
\addplot [
color=darkgray,
solid,
forget plot
]
coordinates{
 (777.894736842105,-8)(700.105263157895,-4) 
};
\addplot [
color=darkgray,
solid,
forget plot
]
coordinates{
 (777.894736842105,-10)(700.105263157895,-4) 
};
\addplot [
color=darkgray,
solid,
forget plot
]
coordinates{
 (777.894736842105,-12)(700.105263157895,-4) 
};
\addplot [
color=darkgray,
solid,
forget plot
]
coordinates{
 (777.894736842105,-14)(700.105263157895,-4) 
};
\addplot [
color=darkgray,
solid,
forget plot
]
coordinates{
 (777.894736842105,0)(700.105263157895,-6) 
};
\addplot [
color=darkgray,
solid,
forget plot
]
coordinates{
 (777.894736842105,-2)(700.105263157895,-6) 
};
\addplot [
color=darkgray,
solid,
forget plot
]
coordinates{
 (777.894736842105,-4)(700.105263157895,-6) 
};
\addplot [
color=darkgray,
solid,
forget plot
]
coordinates{
 (777.894736842105,-6)(700.105263157895,-6) 
};
\addplot [
color=darkgray,
solid,
forget plot
]
coordinates{
 (777.894736842105,-8)(700.105263157895,-6) 
};
\addplot [
color=darkgray,
solid,
forget plot
]
coordinates{
 (777.894736842105,-10)(700.105263157895,-6) 
};
\addplot [
color=darkgray,
solid,
forget plot
]
coordinates{
 (777.894736842105,-12)(700.105263157895,-6) 
};
\addplot [
color=darkgray,
solid,
forget plot
]
coordinates{
 (777.894736842105,-14)(700.105263157895,-6) 
};
\addplot [
color=darkgray,
solid,
forget plot
]
coordinates{
 (777.894736842105,-16)(700.105263157895,-6) 
};
\addplot [
color=darkgray,
solid,
forget plot
]
coordinates{
 (777.894736842105,0)(700.105263157895,-8) 
};
\addplot [
color=darkgray,
solid,
forget plot
]
coordinates{
 (777.894736842105,-2)(700.105263157895,-8) 
};
\addplot [
color=darkgray,
solid,
forget plot
]
coordinates{
 (777.894736842105,-4)(700.105263157895,-8) 
};
\addplot [
color=darkgray,
solid,
forget plot
]
coordinates{
 (777.894736842105,-6)(700.105263157895,-8) 
};
\addplot [
color=darkgray,
solid,
forget plot
]
coordinates{
 (777.894736842105,-8)(700.105263157895,-8) 
};
\addplot [
color=darkgray,
solid,
forget plot
]
coordinates{
 (777.894736842105,-10)(700.105263157895,-8) 
};
\addplot [
color=darkgray,
solid,
forget plot
]
coordinates{
 (777.894736842105,-12)(700.105263157895,-8) 
};
\addplot [
color=darkgray,
solid,
forget plot
]
coordinates{
 (777.894736842105,-14)(700.105263157895,-8) 
};
\addplot [
color=darkgray,
solid,
forget plot
]
coordinates{
 (777.894736842105,-16)(700.105263157895,-8) 
};
\addplot [
color=darkgray,
solid,
forget plot
]
coordinates{
 (777.894736842105,-18)(700.105263157895,-8) 
};
\addplot [
color=darkgray,
solid,
forget plot
]
coordinates{
 (777.894736842105,0)(700.105263157895,-10) 
};
\addplot [
color=darkgray,
solid,
forget plot
]
coordinates{
 (777.894736842105,-2)(700.105263157895,-10) 
};
\addplot [
color=darkgray,
solid,
forget plot
]
coordinates{
 (777.894736842105,-4)(700.105263157895,-10) 
};
\addplot [
color=darkgray,
solid,
forget plot
]
coordinates{
 (777.894736842105,-6)(700.105263157895,-10) 
};
\addplot [
color=darkgray,
solid,
forget plot
]
coordinates{
 (777.894736842105,-8)(700.105263157895,-10) 
};
\addplot [
color=darkgray,
solid,
forget plot
]
coordinates{
 (777.894736842105,-10)(700.105263157895,-10) 
};
\addplot [
color=darkgray,
solid,
forget plot
]
coordinates{
 (777.894736842105,-12)(700.105263157895,-10) 
};
\addplot [
color=darkgray,
solid,
forget plot
]
coordinates{
 (777.894736842105,-14)(700.105263157895,-10) 
};
\addplot [
color=darkgray,
solid,
forget plot
]
coordinates{
 (777.894736842105,-16)(700.105263157895,-10) 
};
\addplot [
color=darkgray,
solid,
forget plot
]
coordinates{
 (777.894736842105,-18)(700.105263157895,-10) 
};
\addplot [
color=darkgray,
solid,
forget plot
]
coordinates{
 (777.894736842105,-2)(700.105263157895,-12) 
};
\addplot [
color=darkgray,
solid,
forget plot
]
coordinates{
 (777.894736842105,-4)(700.105263157895,-12) 
};
\addplot [
color=darkgray,
solid,
forget plot
]
coordinates{
 (777.894736842105,-6)(700.105263157895,-12) 
};
\addplot [
color=darkgray,
solid,
forget plot
]
coordinates{
 (777.894736842105,-8)(700.105263157895,-12) 
};
\addplot [
color=darkgray,
solid,
forget plot
]
coordinates{
 (777.894736842105,-10)(700.105263157895,-12) 
};
\addplot [
color=darkgray,
solid,
forget plot
]
coordinates{
 (777.894736842105,-12)(700.105263157895,-12) 
};
\addplot [
color=darkgray,
solid,
forget plot
]
coordinates{
 (777.894736842105,-14)(700.105263157895,-12) 
};
\addplot [
color=darkgray,
solid,
forget plot
]
coordinates{
 (777.894736842105,-16)(700.105263157895,-12) 
};
\addplot [
color=darkgray,
solid,
forget plot
]
coordinates{
 (777.894736842105,-18)(700.105263157895,-12) 
};
\addplot [
color=darkgray,
solid,
forget plot
]
coordinates{
 (777.894736842105,-4)(700.105263157895,-14) 
};
\addplot [
color=darkgray,
solid,
forget plot
]
coordinates{
 (777.894736842105,-6)(700.105263157895,-14) 
};
\addplot [
color=darkgray,
solid,
forget plot
]
coordinates{
 (777.894736842105,-8)(700.105263157895,-14) 
};
\addplot [
color=darkgray,
solid,
forget plot
]
coordinates{
 (777.894736842105,-10)(700.105263157895,-14) 
};
\addplot [
color=darkgray,
solid,
forget plot
]
coordinates{
 (777.894736842105,-12)(700.105263157895,-14) 
};
\addplot [
color=darkgray,
solid,
forget plot
]
coordinates{
 (777.894736842105,-14)(700.105263157895,-14) 
};
\addplot [
color=darkgray,
solid,
forget plot
]
coordinates{
 (777.894736842105,-16)(700.105263157895,-14) 
};
\addplot [
color=darkgray,
solid,
forget plot
]
coordinates{
 (777.894736842105,-18)(700.105263157895,-14) 
};
\addplot [
color=darkgray,
solid,
forget plot
]
coordinates{
 (777.894736842105,-6)(700.105263157895,-16) 
};
\addplot [
color=darkgray,
solid,
forget plot
]
coordinates{
 (777.894736842105,-8)(700.105263157895,-16) 
};
\addplot [
color=darkgray,
solid,
forget plot
]
coordinates{
 (777.894736842105,-10)(700.105263157895,-16) 
};
\addplot [
color=darkgray,
solid,
forget plot
]
coordinates{
 (777.894736842105,-12)(700.105263157895,-16) 
};
\addplot [
color=darkgray,
solid,
forget plot
]
coordinates{
 (777.894736842105,-14)(700.105263157895,-16) 
};
\addplot [
color=darkgray,
solid,
forget plot
]
coordinates{
 (777.894736842105,-16)(700.105263157895,-16) 
};
\addplot [
color=darkgray,
solid,
forget plot
]
coordinates{
 (777.894736842105,-18)(700.105263157895,-16) 
};
\addplot [
color=darkgray,
solid,
forget plot
]
coordinates{
 (777.894736842105,-8)(700.105263157895,-18) 
};
\addplot [
color=darkgray,
solid,
forget plot
]
coordinates{
 (777.894736842105,-10)(700.105263157895,-18) 
};
\addplot [
color=darkgray,
solid,
forget plot
]
coordinates{
 (777.894736842105,-12)(700.105263157895,-18) 
};
\addplot [
color=darkgray,
solid,
forget plot
]
coordinates{
 (777.894736842105,-14)(700.105263157895,-18) 
};
\addplot [
color=darkgray,
solid,
forget plot
]
coordinates{
 (777.894736842105,-16)(700.105263157895,-18) 
};
\addplot [
color=darkgray,
solid,
forget plot
]
coordinates{
 (777.894736842105,-18)(700.105263157895,-18) 
};
\addplot [
color=darkgray,
solid,
forget plot
]
coordinates{
 (700.105263157895,0)(622.315789473684,0) 
};
\addplot [
color=darkgray,
solid,
forget plot
]
coordinates{
 (700.105263157895,-2)(622.315789473684,0) 
};
\addplot [
color=darkgray,
solid,
forget plot
]
coordinates{
 (700.105263157895,-4)(622.315789473684,0) 
};
\addplot [
color=darkgray,
solid,
forget plot
]
coordinates{
 (700.105263157895,-6)(622.315789473684,0) 
};
\addplot [
color=darkgray,
solid,
forget plot
]
coordinates{
 (700.105263157895,-8)(622.315789473684,0) 
};
\addplot [
color=darkgray,
solid,
forget plot
]
coordinates{
 (700.105263157895,-10)(622.315789473684,0) 
};
\addplot [
color=darkgray,
solid,
forget plot
]
coordinates{
 (700.105263157895,0)(622.315789473684,-2) 
};
\addplot [
color=darkgray,
solid,
forget plot
]
coordinates{
 (700.105263157895,-2)(622.315789473684,-2) 
};
\addplot [
color=darkgray,
solid,
forget plot
]
coordinates{
 (700.105263157895,-4)(622.315789473684,-2) 
};
\addplot [
color=darkgray,
solid,
forget plot
]
coordinates{
 (700.105263157895,-6)(622.315789473684,-2) 
};
\addplot [
color=darkgray,
solid,
forget plot
]
coordinates{
 (700.105263157895,-8)(622.315789473684,-2) 
};
\addplot [
color=darkgray,
solid,
forget plot
]
coordinates{
 (700.105263157895,-10)(622.315789473684,-2) 
};
\addplot [
color=darkgray,
solid,
forget plot
]
coordinates{
 (700.105263157895,-12)(622.315789473684,-2) 
};
\addplot [
color=darkgray,
solid,
forget plot
]
coordinates{
 (700.105263157895,0)(622.315789473684,-4) 
};
\addplot [
color=darkgray,
solid,
forget plot
]
coordinates{
 (700.105263157895,-2)(622.315789473684,-4) 
};
\addplot [
color=darkgray,
solid,
forget plot
]
coordinates{
 (700.105263157895,-4)(622.315789473684,-4) 
};
\addplot [
color=darkgray,
solid,
forget plot
]
coordinates{
 (700.105263157895,-6)(622.315789473684,-4) 
};
\addplot [
color=darkgray,
solid,
forget plot
]
coordinates{
 (700.105263157895,-8)(622.315789473684,-4) 
};
\addplot [
color=darkgray,
solid,
forget plot
]
coordinates{
 (700.105263157895,-10)(622.315789473684,-4) 
};
\addplot [
color=darkgray,
solid,
forget plot
]
coordinates{
 (700.105263157895,-12)(622.315789473684,-4) 
};
\addplot [
color=darkgray,
solid,
forget plot
]
coordinates{
 (700.105263157895,-14)(622.315789473684,-4) 
};
\addplot [
color=darkgray,
solid,
forget plot
]
coordinates{
 (700.105263157895,0)(622.315789473684,-6) 
};
\addplot [
color=darkgray,
solid,
forget plot
]
coordinates{
 (700.105263157895,-2)(622.315789473684,-6) 
};
\addplot [
color=darkgray,
solid,
forget plot
]
coordinates{
 (700.105263157895,-4)(622.315789473684,-6) 
};
\addplot [
color=darkgray,
solid,
forget plot
]
coordinates{
 (700.105263157895,-6)(622.315789473684,-6) 
};
\addplot [
color=darkgray,
solid,
forget plot
]
coordinates{
 (700.105263157895,-8)(622.315789473684,-6) 
};
\addplot [
color=darkgray,
solid,
forget plot
]
coordinates{
 (700.105263157895,-10)(622.315789473684,-6) 
};
\addplot [
color=darkgray,
solid,
forget plot
]
coordinates{
 (700.105263157895,-12)(622.315789473684,-6) 
};
\addplot [
color=darkgray,
solid,
forget plot
]
coordinates{
 (700.105263157895,-14)(622.315789473684,-6) 
};
\addplot [
color=darkgray,
solid,
forget plot
]
coordinates{
 (700.105263157895,-16)(622.315789473684,-6) 
};
\addplot [
color=darkgray,
solid,
forget plot
]
coordinates{
 (700.105263157895,0)(622.315789473684,-8) 
};
\addplot [
color=darkgray,
solid,
forget plot
]
coordinates{
 (700.105263157895,-2)(622.315789473684,-8) 
};
\addplot [
color=darkgray,
solid,
forget plot
]
coordinates{
 (700.105263157895,-4)(622.315789473684,-8) 
};
\addplot [
color=darkgray,
solid,
forget plot
]
coordinates{
 (700.105263157895,-6)(622.315789473684,-8) 
};
\addplot [
color=darkgray,
solid,
forget plot
]
coordinates{
 (700.105263157895,-8)(622.315789473684,-8) 
};
\addplot [
color=darkgray,
solid,
forget plot
]
coordinates{
 (700.105263157895,-10)(622.315789473684,-8) 
};
\addplot [
color=darkgray,
solid,
forget plot
]
coordinates{
 (700.105263157895,-12)(622.315789473684,-8) 
};
\addplot [
color=darkgray,
solid,
forget plot
]
coordinates{
 (700.105263157895,-14)(622.315789473684,-8) 
};
\addplot [
color=darkgray,
solid,
forget plot
]
coordinates{
 (700.105263157895,-16)(622.315789473684,-8) 
};
\addplot [
color=darkgray,
solid,
forget plot
]
coordinates{
 (700.105263157895,-18)(622.315789473684,-8) 
};
\addplot [
color=darkgray,
solid,
forget plot
]
coordinates{
 (700.105263157895,0)(622.315789473684,-10) 
};
\addplot [
color=darkgray,
solid,
forget plot
]
coordinates{
 (700.105263157895,-2)(622.315789473684,-10) 
};
\addplot [
color=darkgray,
solid,
forget plot
]
coordinates{
 (700.105263157895,-4)(622.315789473684,-10) 
};
\addplot [
color=darkgray,
solid,
forget plot
]
coordinates{
 (700.105263157895,-6)(622.315789473684,-10) 
};
\addplot [
color=darkgray,
solid,
forget plot
]
coordinates{
 (700.105263157895,-8)(622.315789473684,-10) 
};
\addplot [
color=darkgray,
solid,
forget plot
]
coordinates{
 (700.105263157895,-10)(622.315789473684,-10) 
};
\addplot [
color=darkgray,
solid,
forget plot
]
coordinates{
 (700.105263157895,-12)(622.315789473684,-10) 
};
\addplot [
color=darkgray,
solid,
forget plot
]
coordinates{
 (700.105263157895,-14)(622.315789473684,-10) 
};
\addplot [
color=darkgray,
solid,
forget plot
]
coordinates{
 (700.105263157895,-16)(622.315789473684,-10) 
};
\addplot [
color=darkgray,
solid,
forget plot
]
coordinates{
 (700.105263157895,-18)(622.315789473684,-10) 
};
\addplot [
color=darkgray,
solid,
forget plot
]
coordinates{
 (700.105263157895,-2)(622.315789473684,-12) 
};
\addplot [
color=darkgray,
solid,
forget plot
]
coordinates{
 (700.105263157895,-4)(622.315789473684,-12) 
};
\addplot [
color=darkgray,
solid,
forget plot
]
coordinates{
 (700.105263157895,-6)(622.315789473684,-12) 
};
\addplot [
color=darkgray,
solid,
forget plot
]
coordinates{
 (700.105263157895,-8)(622.315789473684,-12) 
};
\addplot [
color=darkgray,
solid,
forget plot
]
coordinates{
 (700.105263157895,-10)(622.315789473684,-12) 
};
\addplot [
color=darkgray,
solid,
forget plot
]
coordinates{
 (700.105263157895,-12)(622.315789473684,-12) 
};
\addplot [
color=darkgray,
solid,
forget plot
]
coordinates{
 (700.105263157895,-14)(622.315789473684,-12) 
};
\addplot [
color=darkgray,
solid,
forget plot
]
coordinates{
 (700.105263157895,-16)(622.315789473684,-12) 
};
\addplot [
color=darkgray,
solid,
forget plot
]
coordinates{
 (700.105263157895,-18)(622.315789473684,-12) 
};
\addplot [
color=darkgray,
solid,
forget plot
]
coordinates{
 (700.105263157895,-4)(622.315789473684,-14) 
};
\addplot [
color=darkgray,
solid,
forget plot
]
coordinates{
 (700.105263157895,-6)(622.315789473684,-14) 
};
\addplot [
color=darkgray,
solid,
forget plot
]
coordinates{
 (700.105263157895,-8)(622.315789473684,-14) 
};
\addplot [
color=darkgray,
solid,
forget plot
]
coordinates{
 (700.105263157895,-10)(622.315789473684,-14) 
};
\addplot [
color=darkgray,
solid,
forget plot
]
coordinates{
 (700.105263157895,-12)(622.315789473684,-14) 
};
\addplot [
color=darkgray,
solid,
forget plot
]
coordinates{
 (700.105263157895,-14)(622.315789473684,-14) 
};
\addplot [
color=darkgray,
solid,
forget plot
]
coordinates{
 (700.105263157895,-16)(622.315789473684,-14) 
};
\addplot [
color=darkgray,
solid,
forget plot
]
coordinates{
 (700.105263157895,-18)(622.315789473684,-14) 
};
\addplot [
color=darkgray,
solid,
forget plot
]
coordinates{
 (700.105263157895,-6)(622.315789473684,-16) 
};
\addplot [
color=darkgray,
solid,
forget plot
]
coordinates{
 (700.105263157895,-8)(622.315789473684,-16) 
};
\addplot [
color=darkgray,
solid,
forget plot
]
coordinates{
 (700.105263157895,-10)(622.315789473684,-16) 
};
\addplot [
color=darkgray,
solid,
forget plot
]
coordinates{
 (700.105263157895,-12)(622.315789473684,-16) 
};
\addplot [
color=darkgray,
solid,
forget plot
]
coordinates{
 (700.105263157895,-14)(622.315789473684,-16) 
};
\addplot [
color=darkgray,
solid,
forget plot
]
coordinates{
 (700.105263157895,-16)(622.315789473684,-16) 
};
\addplot [
color=darkgray,
solid,
forget plot
]
coordinates{
 (700.105263157895,-18)(622.315789473684,-16) 
};
\addplot [
color=darkgray,
solid,
forget plot
]
coordinates{
 (700.105263157895,-8)(622.315789473684,-18) 
};
\addplot [
color=darkgray,
solid,
forget plot
]
coordinates{
 (700.105263157895,-10)(622.315789473684,-18) 
};
\addplot [
color=darkgray,
solid,
forget plot
]
coordinates{
 (700.105263157895,-12)(622.315789473684,-18) 
};
\addplot [
color=darkgray,
solid,
forget plot
]
coordinates{
 (700.105263157895,-14)(622.315789473684,-18) 
};
\addplot [
color=darkgray,
solid,
forget plot
]
coordinates{
 (700.105263157895,-16)(622.315789473684,-18) 
};
\addplot [
color=darkgray,
solid,
forget plot
]
coordinates{
 (700.105263157895,-18)(622.315789473684,-18) 
};
\addplot [
color=darkgray,
solid,
forget plot
]
coordinates{
 (622.315789473684,0)(544.526315789474,0) 
};
\addplot [
color=darkgray,
solid,
forget plot
]
coordinates{
 (622.315789473684,-2)(544.526315789474,0) 
};
\addplot [
color=darkgray,
solid,
forget plot
]
coordinates{
 (622.315789473684,-4)(544.526315789474,0) 
};
\addplot [
color=darkgray,
solid,
forget plot
]
coordinates{
 (622.315789473684,-6)(544.526315789474,0) 
};
\addplot [
color=darkgray,
solid,
forget plot
]
coordinates{
 (622.315789473684,-8)(544.526315789474,0) 
};
\addplot [
color=darkgray,
solid,
forget plot
]
coordinates{
 (622.315789473684,-10)(544.526315789474,0) 
};
\addplot [
color=darkgray,
solid,
forget plot
]
coordinates{
 (622.315789473684,0)(544.526315789474,-2) 
};
\addplot [
color=darkgray,
solid,
forget plot
]
coordinates{
 (622.315789473684,-2)(544.526315789474,-2) 
};
\addplot [
color=darkgray,
solid,
forget plot
]
coordinates{
 (622.315789473684,-4)(544.526315789474,-2) 
};
\addplot [
color=darkgray,
solid,
forget plot
]
coordinates{
 (622.315789473684,-6)(544.526315789474,-2) 
};
\addplot [
color=darkgray,
solid,
forget plot
]
coordinates{
 (622.315789473684,-8)(544.526315789474,-2) 
};
\addplot [
color=darkgray,
solid,
forget plot
]
coordinates{
 (622.315789473684,-10)(544.526315789474,-2) 
};
\addplot [
color=darkgray,
solid,
forget plot
]
coordinates{
 (622.315789473684,-12)(544.526315789474,-2) 
};
\addplot [
color=darkgray,
solid,
forget plot
]
coordinates{
 (622.315789473684,0)(544.526315789474,-4) 
};
\addplot [
color=darkgray,
solid,
forget plot
]
coordinates{
 (622.315789473684,-2)(544.526315789474,-4) 
};
\addplot [
color=darkgray,
solid,
forget plot
]
coordinates{
 (622.315789473684,-4)(544.526315789474,-4) 
};
\addplot [
color=darkgray,
solid,
forget plot
]
coordinates{
 (622.315789473684,-6)(544.526315789474,-4) 
};
\addplot [
color=darkgray,
solid,
forget plot
]
coordinates{
 (622.315789473684,-8)(544.526315789474,-4) 
};
\addplot [
color=darkgray,
solid,
forget plot
]
coordinates{
 (622.315789473684,-10)(544.526315789474,-4) 
};
\addplot [
color=darkgray,
solid,
forget plot
]
coordinates{
 (622.315789473684,-12)(544.526315789474,-4) 
};
\addplot [
color=darkgray,
solid,
forget plot
]
coordinates{
 (622.315789473684,-14)(544.526315789474,-4) 
};
\addplot [
color=darkgray,
solid,
forget plot
]
coordinates{
 (622.315789473684,0)(544.526315789474,-6) 
};
\addplot [
color=darkgray,
solid,
forget plot
]
coordinates{
 (622.315789473684,-2)(544.526315789474,-6) 
};
\addplot [
color=darkgray,
solid,
forget plot
]
coordinates{
 (622.315789473684,-4)(544.526315789474,-6) 
};
\addplot [
color=darkgray,
solid,
forget plot
]
coordinates{
 (622.315789473684,-6)(544.526315789474,-6) 
};
\addplot [
color=darkgray,
solid,
forget plot
]
coordinates{
 (622.315789473684,-8)(544.526315789474,-6) 
};
\addplot [
color=darkgray,
solid,
forget plot
]
coordinates{
 (622.315789473684,-10)(544.526315789474,-6) 
};
\addplot [
color=darkgray,
solid,
forget plot
]
coordinates{
 (622.315789473684,-12)(544.526315789474,-6) 
};
\addplot [
color=darkgray,
solid,
forget plot
]
coordinates{
 (622.315789473684,-14)(544.526315789474,-6) 
};
\addplot [
color=darkgray,
solid,
forget plot
]
coordinates{
 (622.315789473684,-16)(544.526315789474,-6) 
};
\addplot [
color=darkgray,
solid,
forget plot
]
coordinates{
 (622.315789473684,0)(544.526315789474,-8) 
};
\addplot [
color=darkgray,
solid,
forget plot
]
coordinates{
 (622.315789473684,-2)(544.526315789474,-8) 
};
\addplot [
color=darkgray,
solid,
forget plot
]
coordinates{
 (622.315789473684,-4)(544.526315789474,-8) 
};
\addplot [
color=darkgray,
solid,
forget plot
]
coordinates{
 (622.315789473684,-6)(544.526315789474,-8) 
};
\addplot [
color=darkgray,
solid,
forget plot
]
coordinates{
 (622.315789473684,-8)(544.526315789474,-8) 
};
\addplot [
color=darkgray,
solid,
forget plot
]
coordinates{
 (622.315789473684,-10)(544.526315789474,-8) 
};
\addplot [
color=darkgray,
solid,
forget plot
]
coordinates{
 (622.315789473684,-12)(544.526315789474,-8) 
};
\addplot [
color=darkgray,
solid,
forget plot
]
coordinates{
 (622.315789473684,-14)(544.526315789474,-8) 
};
\addplot [
color=darkgray,
solid,
forget plot
]
coordinates{
 (622.315789473684,-16)(544.526315789474,-8) 
};
\addplot [
color=darkgray,
solid,
forget plot
]
coordinates{
 (622.315789473684,-18)(544.526315789474,-8) 
};
\addplot [
color=darkgray,
solid,
forget plot
]
coordinates{
 (622.315789473684,0)(544.526315789474,-10) 
};
\addplot [
color=darkgray,
solid,
forget plot
]
coordinates{
 (622.315789473684,-2)(544.526315789474,-10) 
};
\addplot [
color=darkgray,
solid,
forget plot
]
coordinates{
 (622.315789473684,-4)(544.526315789474,-10) 
};
\addplot [
color=darkgray,
solid,
forget plot
]
coordinates{
 (622.315789473684,-6)(544.526315789474,-10) 
};
\addplot [
color=darkgray,
solid,
forget plot
]
coordinates{
 (622.315789473684,-8)(544.526315789474,-10) 
};
\addplot [
color=darkgray,
solid,
forget plot
]
coordinates{
 (622.315789473684,-10)(544.526315789474,-10) 
};
\addplot [
color=darkgray,
solid,
forget plot
]
coordinates{
 (622.315789473684,-12)(544.526315789474,-10) 
};
\addplot [
color=darkgray,
solid,
forget plot
]
coordinates{
 (622.315789473684,-14)(544.526315789474,-10) 
};
\addplot [
color=darkgray,
solid,
forget plot
]
coordinates{
 (622.315789473684,-16)(544.526315789474,-10) 
};
\addplot [
color=darkgray,
solid,
forget plot
]
coordinates{
 (622.315789473684,-18)(544.526315789474,-10) 
};
\addplot [
color=darkgray,
solid,
forget plot
]
coordinates{
 (622.315789473684,-2)(544.526315789474,-12) 
};
\addplot [
color=darkgray,
solid,
forget plot
]
coordinates{
 (622.315789473684,-4)(544.526315789474,-12) 
};
\addplot [
color=darkgray,
solid,
forget plot
]
coordinates{
 (622.315789473684,-6)(544.526315789474,-12) 
};
\addplot [
color=darkgray,
solid,
forget plot
]
coordinates{
 (622.315789473684,-8)(544.526315789474,-12) 
};
\addplot [
color=darkgray,
solid,
forget plot
]
coordinates{
 (622.315789473684,-10)(544.526315789474,-12) 
};
\addplot [
color=darkgray,
solid,
forget plot
]
coordinates{
 (622.315789473684,-12)(544.526315789474,-12) 
};
\addplot [
color=darkgray,
solid,
forget plot
]
coordinates{
 (622.315789473684,-14)(544.526315789474,-12) 
};
\addplot [
color=darkgray,
solid,
forget plot
]
coordinates{
 (622.315789473684,-16)(544.526315789474,-12) 
};
\addplot [
color=darkgray,
solid,
forget plot
]
coordinates{
 (622.315789473684,-18)(544.526315789474,-12) 
};
\addplot [
color=darkgray,
solid,
forget plot
]
coordinates{
 (622.315789473684,-4)(544.526315789474,-14) 
};
\addplot [
color=darkgray,
solid,
forget plot
]
coordinates{
 (622.315789473684,-6)(544.526315789474,-14) 
};
\addplot [
color=darkgray,
solid,
forget plot
]
coordinates{
 (622.315789473684,-8)(544.526315789474,-14) 
};
\addplot [
color=darkgray,
solid,
forget plot
]
coordinates{
 (622.315789473684,-10)(544.526315789474,-14) 
};
\addplot [
color=darkgray,
solid,
forget plot
]
coordinates{
 (622.315789473684,-12)(544.526315789474,-14) 
};
\addplot [
color=darkgray,
solid,
forget plot
]
coordinates{
 (622.315789473684,-14)(544.526315789474,-14) 
};
\addplot [
color=darkgray,
solid,
forget plot
]
coordinates{
 (622.315789473684,-16)(544.526315789474,-14) 
};
\addplot [
color=darkgray,
solid,
forget plot
]
coordinates{
 (622.315789473684,-18)(544.526315789474,-14) 
};
\addplot [
color=darkgray,
solid,
forget plot
]
coordinates{
 (622.315789473684,-6)(544.526315789474,-16) 
};
\addplot [
color=darkgray,
solid,
forget plot
]
coordinates{
 (622.315789473684,-8)(544.526315789474,-16) 
};
\addplot [
color=darkgray,
solid,
forget plot
]
coordinates{
 (622.315789473684,-10)(544.526315789474,-16) 
};
\addplot [
color=darkgray,
solid,
forget plot
]
coordinates{
 (622.315789473684,-12)(544.526315789474,-16) 
};
\addplot [
color=darkgray,
solid,
forget plot
]
coordinates{
 (622.315789473684,-14)(544.526315789474,-16) 
};
\addplot [
color=darkgray,
solid,
forget plot
]
coordinates{
 (622.315789473684,-16)(544.526315789474,-16) 
};
\addplot [
color=darkgray,
solid,
forget plot
]
coordinates{
 (622.315789473684,-18)(544.526315789474,-16) 
};
\addplot [
color=darkgray,
solid,
forget plot
]
coordinates{
 (622.315789473684,-8)(544.526315789474,-18) 
};
\addplot [
color=darkgray,
solid,
forget plot
]
coordinates{
 (622.315789473684,-10)(544.526315789474,-18) 
};
\addplot [
color=darkgray,
solid,
forget plot
]
coordinates{
 (622.315789473684,-12)(544.526315789474,-18) 
};
\addplot [
color=darkgray,
solid,
forget plot
]
coordinates{
 (622.315789473684,-14)(544.526315789474,-18) 
};
\addplot [
color=darkgray,
solid,
forget plot
]
coordinates{
 (622.315789473684,-16)(544.526315789474,-18) 
};
\addplot [
color=darkgray,
solid,
forget plot
]
coordinates{
 (622.315789473684,-18)(544.526315789474,-18) 
};
\addplot [
color=darkgray,
solid,
forget plot
]
coordinates{
 (544.526315789474,0)(466.736842105263,0) 
};
\addplot [
color=darkgray,
solid,
forget plot
]
coordinates{
 (544.526315789474,-2)(466.736842105263,0) 
};
\addplot [
color=darkgray,
solid,
forget plot
]
coordinates{
 (544.526315789474,-4)(466.736842105263,0) 
};
\addplot [
color=darkgray,
solid,
forget plot
]
coordinates{
 (544.526315789474,-6)(466.736842105263,0) 
};
\addplot [
color=darkgray,
solid,
forget plot
]
coordinates{
 (544.526315789474,-8)(466.736842105263,0) 
};
\addplot [
color=darkgray,
solid,
forget plot
]
coordinates{
 (544.526315789474,-10)(466.736842105263,0) 
};
\addplot [
color=darkgray,
solid,
forget plot
]
coordinates{
 (544.526315789474,0)(466.736842105263,-2) 
};
\addplot [
color=darkgray,
solid,
forget plot
]
coordinates{
 (544.526315789474,-2)(466.736842105263,-2) 
};
\addplot [
color=darkgray,
solid,
forget plot
]
coordinates{
 (544.526315789474,-4)(466.736842105263,-2) 
};
\addplot [
color=darkgray,
solid,
forget plot
]
coordinates{
 (544.526315789474,-6)(466.736842105263,-2) 
};
\addplot [
color=darkgray,
solid,
forget plot
]
coordinates{
 (544.526315789474,-8)(466.736842105263,-2) 
};
\addplot [
color=darkgray,
solid,
forget plot
]
coordinates{
 (544.526315789474,-10)(466.736842105263,-2) 
};
\addplot [
color=darkgray,
solid,
forget plot
]
coordinates{
 (544.526315789474,-12)(466.736842105263,-2) 
};
\addplot [
color=darkgray,
solid,
forget plot
]
coordinates{
 (544.526315789474,0)(466.736842105263,-4) 
};
\addplot [
color=darkgray,
solid,
forget plot
]
coordinates{
 (544.526315789474,-2)(466.736842105263,-4) 
};
\addplot [
color=darkgray,
solid,
forget plot
]
coordinates{
 (544.526315789474,-4)(466.736842105263,-4) 
};
\addplot [
color=darkgray,
solid,
forget plot
]
coordinates{
 (544.526315789474,-6)(466.736842105263,-4) 
};
\addplot [
color=darkgray,
solid,
forget plot
]
coordinates{
 (544.526315789474,-8)(466.736842105263,-4) 
};
\addplot [
color=darkgray,
solid,
forget plot
]
coordinates{
 (544.526315789474,-10)(466.736842105263,-4) 
};
\addplot [
color=darkgray,
solid,
forget plot
]
coordinates{
 (544.526315789474,-12)(466.736842105263,-4) 
};
\addplot [
color=darkgray,
solid,
forget plot
]
coordinates{
 (544.526315789474,-14)(466.736842105263,-4) 
};
\addplot [
color=darkgray,
solid,
forget plot
]
coordinates{
 (544.526315789474,0)(466.736842105263,-6) 
};
\addplot [
color=darkgray,
solid,
forget plot
]
coordinates{
 (544.526315789474,-2)(466.736842105263,-6) 
};
\addplot [
color=darkgray,
solid,
forget plot
]
coordinates{
 (544.526315789474,-4)(466.736842105263,-6) 
};
\addplot [
color=darkgray,
solid,
forget plot
]
coordinates{
 (544.526315789474,-6)(466.736842105263,-6) 
};
\addplot [
color=darkgray,
solid,
forget plot
]
coordinates{
 (544.526315789474,-8)(466.736842105263,-6) 
};
\addplot [
color=darkgray,
solid,
forget plot
]
coordinates{
 (544.526315789474,-10)(466.736842105263,-6) 
};
\addplot [
color=darkgray,
solid,
forget plot
]
coordinates{
 (544.526315789474,-12)(466.736842105263,-6) 
};
\addplot [
color=darkgray,
solid,
forget plot
]
coordinates{
 (544.526315789474,-14)(466.736842105263,-6) 
};
\addplot [
color=darkgray,
solid,
forget plot
]
coordinates{
 (544.526315789474,-16)(466.736842105263,-6) 
};
\addplot [
color=darkgray,
solid,
forget plot
]
coordinates{
 (544.526315789474,0)(466.736842105263,-8) 
};
\addplot [
color=darkgray,
solid,
forget plot
]
coordinates{
 (544.526315789474,-2)(466.736842105263,-8) 
};
\addplot [
color=darkgray,
solid,
forget plot
]
coordinates{
 (544.526315789474,-4)(466.736842105263,-8) 
};
\addplot [
color=darkgray,
solid,
forget plot
]
coordinates{
 (544.526315789474,-6)(466.736842105263,-8) 
};
\addplot [
color=darkgray,
solid,
forget plot
]
coordinates{
 (544.526315789474,-8)(466.736842105263,-8) 
};
\addplot [
color=darkgray,
solid,
forget plot
]
coordinates{
 (544.526315789474,-10)(466.736842105263,-8) 
};
\addplot [
color=darkgray,
solid,
forget plot
]
coordinates{
 (544.526315789474,-12)(466.736842105263,-8) 
};
\addplot [
color=darkgray,
solid,
forget plot
]
coordinates{
 (544.526315789474,-14)(466.736842105263,-8) 
};
\addplot [
color=darkgray,
solid,
forget plot
]
coordinates{
 (544.526315789474,-16)(466.736842105263,-8) 
};
\addplot [
color=darkgray,
solid,
forget plot
]
coordinates{
 (544.526315789474,-18)(466.736842105263,-8) 
};
\addplot [
color=darkgray,
solid,
forget plot
]
coordinates{
 (544.526315789474,0)(466.736842105263,-10) 
};
\addplot [
color=darkgray,
solid,
forget plot
]
coordinates{
 (544.526315789474,-2)(466.736842105263,-10) 
};
\addplot [
color=darkgray,
solid,
forget plot
]
coordinates{
 (544.526315789474,-4)(466.736842105263,-10) 
};
\addplot [
color=darkgray,
solid,
forget plot
]
coordinates{
 (544.526315789474,-6)(466.736842105263,-10) 
};
\addplot [
color=darkgray,
solid,
forget plot
]
coordinates{
 (544.526315789474,-8)(466.736842105263,-10) 
};
\addplot [
color=darkgray,
solid,
forget plot
]
coordinates{
 (544.526315789474,-10)(466.736842105263,-10) 
};
\addplot [
color=darkgray,
solid,
forget plot
]
coordinates{
 (544.526315789474,-12)(466.736842105263,-10) 
};
\addplot [
color=darkgray,
solid,
forget plot
]
coordinates{
 (544.526315789474,-14)(466.736842105263,-10) 
};
\addplot [
color=darkgray,
solid,
forget plot
]
coordinates{
 (544.526315789474,-16)(466.736842105263,-10) 
};
\addplot [
color=darkgray,
solid,
forget plot
]
coordinates{
 (544.526315789474,-18)(466.736842105263,-10) 
};
\addplot [
color=darkgray,
solid,
forget plot
]
coordinates{
 (544.526315789474,-2)(466.736842105263,-12) 
};
\addplot [
color=darkgray,
solid,
forget plot
]
coordinates{
 (544.526315789474,-4)(466.736842105263,-12) 
};
\addplot [
color=darkgray,
solid,
forget plot
]
coordinates{
 (544.526315789474,-6)(466.736842105263,-12) 
};
\addplot [
color=darkgray,
solid,
forget plot
]
coordinates{
 (544.526315789474,-8)(466.736842105263,-12) 
};
\addplot [
color=darkgray,
solid,
forget plot
]
coordinates{
 (544.526315789474,-10)(466.736842105263,-12) 
};
\addplot [
color=darkgray,
solid,
forget plot
]
coordinates{
 (544.526315789474,-12)(466.736842105263,-12) 
};
\addplot [
color=darkgray,
solid,
forget plot
]
coordinates{
 (544.526315789474,-14)(466.736842105263,-12) 
};
\addplot [
color=darkgray,
solid,
forget plot
]
coordinates{
 (544.526315789474,-16)(466.736842105263,-12) 
};
\addplot [
color=darkgray,
solid,
forget plot
]
coordinates{
 (544.526315789474,-18)(466.736842105263,-12) 
};
\addplot [
color=darkgray,
solid,
forget plot
]
coordinates{
 (544.526315789474,-4)(466.736842105263,-14) 
};
\addplot [
color=darkgray,
solid,
forget plot
]
coordinates{
 (544.526315789474,-6)(466.736842105263,-14) 
};
\addplot [
color=darkgray,
solid,
forget plot
]
coordinates{
 (544.526315789474,-8)(466.736842105263,-14) 
};
\addplot [
color=darkgray,
solid,
forget plot
]
coordinates{
 (544.526315789474,-10)(466.736842105263,-14) 
};
\addplot [
color=darkgray,
solid,
forget plot
]
coordinates{
 (544.526315789474,-12)(466.736842105263,-14) 
};
\addplot [
color=darkgray,
solid,
forget plot
]
coordinates{
 (544.526315789474,-14)(466.736842105263,-14) 
};
\addplot [
color=darkgray,
solid,
forget plot
]
coordinates{
 (544.526315789474,-16)(466.736842105263,-14) 
};
\addplot [
color=darkgray,
solid,
forget plot
]
coordinates{
 (544.526315789474,-18)(466.736842105263,-14) 
};
\addplot [
color=darkgray,
solid,
forget plot
]
coordinates{
 (544.526315789474,-6)(466.736842105263,-16) 
};
\addplot [
color=darkgray,
solid,
forget plot
]
coordinates{
 (544.526315789474,-8)(466.736842105263,-16) 
};
\addplot [
color=darkgray,
solid,
forget plot
]
coordinates{
 (544.526315789474,-10)(466.736842105263,-16) 
};
\addplot [
color=darkgray,
solid,
forget plot
]
coordinates{
 (544.526315789474,-12)(466.736842105263,-16) 
};
\addplot [
color=darkgray,
solid,
forget plot
]
coordinates{
 (544.526315789474,-14)(466.736842105263,-16) 
};
\addplot [
color=darkgray,
solid,
forget plot
]
coordinates{
 (544.526315789474,-16)(466.736842105263,-16) 
};
\addplot [
color=darkgray,
solid,
forget plot
]
coordinates{
 (544.526315789474,-18)(466.736842105263,-16) 
};
\addplot [
color=darkgray,
solid,
forget plot
]
coordinates{
 (544.526315789474,-8)(466.736842105263,-18) 
};
\addplot [
color=darkgray,
solid,
forget plot
]
coordinates{
 (544.526315789474,-10)(466.736842105263,-18) 
};
\addplot [
color=darkgray,
solid,
forget plot
]
coordinates{
 (544.526315789474,-12)(466.736842105263,-18) 
};
\addplot [
color=darkgray,
solid,
forget plot
]
coordinates{
 (544.526315789474,-14)(466.736842105263,-18) 
};
\addplot [
color=darkgray,
solid,
forget plot
]
coordinates{
 (544.526315789474,-16)(466.736842105263,-18) 
};
\addplot [
color=darkgray,
solid,
forget plot
]
coordinates{
 (544.526315789474,-18)(466.736842105263,-18) 
};
\addplot [
color=darkgray,
solid,
forget plot
]
coordinates{
 (466.736842105263,0)(388.947368421053,0) 
};
\addplot [
color=darkgray,
solid,
forget plot
]
coordinates{
 (466.736842105263,-2)(388.947368421053,0) 
};
\addplot [
color=darkgray,
solid,
forget plot
]
coordinates{
 (466.736842105263,-4)(388.947368421053,0) 
};
\addplot [
color=darkgray,
solid,
forget plot
]
coordinates{
 (466.736842105263,-6)(388.947368421053,0) 
};
\addplot [
color=darkgray,
solid,
forget plot
]
coordinates{
 (466.736842105263,-8)(388.947368421053,0) 
};
\addplot [
color=darkgray,
solid,
forget plot
]
coordinates{
 (466.736842105263,-10)(388.947368421053,0) 
};
\addplot [
color=darkgray,
solid,
forget plot
]
coordinates{
 (466.736842105263,0)(388.947368421053,-2) 
};
\addplot [
color=darkgray,
solid,
forget plot
]
coordinates{
 (466.736842105263,-2)(388.947368421053,-2) 
};
\addplot [
color=darkgray,
solid,
forget plot
]
coordinates{
 (466.736842105263,-4)(388.947368421053,-2) 
};
\addplot [
color=darkgray,
solid,
forget plot
]
coordinates{
 (466.736842105263,-6)(388.947368421053,-2) 
};
\addplot [
color=darkgray,
solid,
forget plot
]
coordinates{
 (466.736842105263,-8)(388.947368421053,-2) 
};
\addplot [
color=darkgray,
solid,
forget plot
]
coordinates{
 (466.736842105263,-10)(388.947368421053,-2) 
};
\addplot [
color=darkgray,
solid,
forget plot
]
coordinates{
 (466.736842105263,-12)(388.947368421053,-2) 
};
\addplot [
color=darkgray,
solid,
forget plot
]
coordinates{
 (466.736842105263,0)(388.947368421053,-4) 
};
\addplot [
color=darkgray,
solid,
forget plot
]
coordinates{
 (466.736842105263,-2)(388.947368421053,-4) 
};
\addplot [
color=darkgray,
solid,
forget plot
]
coordinates{
 (466.736842105263,-4)(388.947368421053,-4) 
};
\addplot [
color=darkgray,
solid,
forget plot
]
coordinates{
 (466.736842105263,-6)(388.947368421053,-4) 
};
\addplot [
color=darkgray,
solid,
forget plot
]
coordinates{
 (466.736842105263,-8)(388.947368421053,-4) 
};
\addplot [
color=darkgray,
solid,
forget plot
]
coordinates{
 (466.736842105263,-10)(388.947368421053,-4) 
};
\addplot [
color=darkgray,
solid,
forget plot
]
coordinates{
 (466.736842105263,-12)(388.947368421053,-4) 
};
\addplot [
color=darkgray,
solid,
forget plot
]
coordinates{
 (466.736842105263,-14)(388.947368421053,-4) 
};
\addplot [
color=darkgray,
solid,
forget plot
]
coordinates{
 (466.736842105263,0)(388.947368421053,-6) 
};
\addplot [
color=darkgray,
solid,
forget plot
]
coordinates{
 (466.736842105263,-2)(388.947368421053,-6) 
};
\addplot [
color=darkgray,
solid,
forget plot
]
coordinates{
 (466.736842105263,-4)(388.947368421053,-6) 
};
\addplot [
color=darkgray,
solid,
forget plot
]
coordinates{
 (466.736842105263,-6)(388.947368421053,-6) 
};
\addplot [
color=darkgray,
solid,
forget plot
]
coordinates{
 (466.736842105263,-8)(388.947368421053,-6) 
};
\addplot [
color=darkgray,
solid,
forget plot
]
coordinates{
 (466.736842105263,-10)(388.947368421053,-6) 
};
\addplot [
color=darkgray,
solid,
forget plot
]
coordinates{
 (466.736842105263,-12)(388.947368421053,-6) 
};
\addplot [
color=darkgray,
solid,
forget plot
]
coordinates{
 (466.736842105263,-14)(388.947368421053,-6) 
};
\addplot [
color=darkgray,
solid,
forget plot
]
coordinates{
 (466.736842105263,-16)(388.947368421053,-6) 
};
\addplot [
color=darkgray,
solid,
forget plot
]
coordinates{
 (466.736842105263,0)(388.947368421053,-8) 
};
\addplot [
color=darkgray,
solid,
forget plot
]
coordinates{
 (466.736842105263,-2)(388.947368421053,-8) 
};
\addplot [
color=darkgray,
solid,
forget plot
]
coordinates{
 (466.736842105263,-4)(388.947368421053,-8) 
};
\addplot [
color=darkgray,
solid,
forget plot
]
coordinates{
 (466.736842105263,-6)(388.947368421053,-8) 
};
\addplot [
color=darkgray,
solid,
forget plot
]
coordinates{
 (466.736842105263,-8)(388.947368421053,-8) 
};
\addplot [
color=darkgray,
solid,
forget plot
]
coordinates{
 (466.736842105263,-10)(388.947368421053,-8) 
};
\addplot [
color=darkgray,
solid,
forget plot
]
coordinates{
 (466.736842105263,-12)(388.947368421053,-8) 
};
\addplot [
color=darkgray,
solid,
forget plot
]
coordinates{
 (466.736842105263,-14)(388.947368421053,-8) 
};
\addplot [
color=darkgray,
solid,
forget plot
]
coordinates{
 (466.736842105263,-16)(388.947368421053,-8) 
};
\addplot [
color=darkgray,
solid,
forget plot
]
coordinates{
 (466.736842105263,-18)(388.947368421053,-8) 
};
\addplot [
color=darkgray,
solid,
forget plot
]
coordinates{
 (466.736842105263,0)(388.947368421053,-10) 
};
\addplot [
color=darkgray,
solid,
forget plot
]
coordinates{
 (466.736842105263,-2)(388.947368421053,-10) 
};
\addplot [
color=darkgray,
solid,
forget plot
]
coordinates{
 (466.736842105263,-4)(388.947368421053,-10) 
};
\addplot [
color=darkgray,
solid,
forget plot
]
coordinates{
 (466.736842105263,-6)(388.947368421053,-10) 
};
\addplot [
color=darkgray,
solid,
forget plot
]
coordinates{
 (466.736842105263,-8)(388.947368421053,-10) 
};
\addplot [
color=darkgray,
solid,
forget plot
]
coordinates{
 (466.736842105263,-10)(388.947368421053,-10) 
};
\addplot [
color=darkgray,
solid,
forget plot
]
coordinates{
 (466.736842105263,-12)(388.947368421053,-10) 
};
\addplot [
color=darkgray,
solid,
forget plot
]
coordinates{
 (466.736842105263,-14)(388.947368421053,-10) 
};
\addplot [
color=darkgray,
solid,
forget plot
]
coordinates{
 (466.736842105263,-16)(388.947368421053,-10) 
};
\addplot [
color=darkgray,
solid,
forget plot
]
coordinates{
 (466.736842105263,-18)(388.947368421053,-10) 
};
\addplot [
color=darkgray,
solid,
forget plot
]
coordinates{
 (466.736842105263,-2)(388.947368421053,-12) 
};
\addplot [
color=darkgray,
solid,
forget plot
]
coordinates{
 (466.736842105263,-4)(388.947368421053,-12) 
};
\addplot [
color=darkgray,
solid,
forget plot
]
coordinates{
 (466.736842105263,-6)(388.947368421053,-12) 
};
\addplot [
color=darkgray,
solid,
forget plot
]
coordinates{
 (466.736842105263,-8)(388.947368421053,-12) 
};
\addplot [
color=darkgray,
solid,
forget plot
]
coordinates{
 (466.736842105263,-10)(388.947368421053,-12) 
};
\addplot [
color=darkgray,
solid,
forget plot
]
coordinates{
 (466.736842105263,-12)(388.947368421053,-12) 
};
\addplot [
color=darkgray,
solid,
forget plot
]
coordinates{
 (466.736842105263,-14)(388.947368421053,-12) 
};
\addplot [
color=darkgray,
solid,
forget plot
]
coordinates{
 (466.736842105263,-16)(388.947368421053,-12) 
};
\addplot [
color=darkgray,
solid,
forget plot
]
coordinates{
 (466.736842105263,-18)(388.947368421053,-12) 
};
\addplot [
color=darkgray,
solid,
forget plot
]
coordinates{
 (466.736842105263,-4)(388.947368421053,-14) 
};
\addplot [
color=darkgray,
solid,
forget plot
]
coordinates{
 (466.736842105263,-6)(388.947368421053,-14) 
};
\addplot [
color=darkgray,
solid,
forget plot
]
coordinates{
 (466.736842105263,-8)(388.947368421053,-14) 
};
\addplot [
color=darkgray,
solid,
forget plot
]
coordinates{
 (466.736842105263,-10)(388.947368421053,-14) 
};
\addplot [
color=darkgray,
solid,
forget plot
]
coordinates{
 (466.736842105263,-12)(388.947368421053,-14) 
};
\addplot [
color=darkgray,
solid,
forget plot
]
coordinates{
 (466.736842105263,-14)(388.947368421053,-14) 
};
\addplot [
color=darkgray,
solid,
forget plot
]
coordinates{
 (466.736842105263,-16)(388.947368421053,-14) 
};
\addplot [
color=darkgray,
solid,
forget plot
]
coordinates{
 (466.736842105263,-18)(388.947368421053,-14) 
};
\addplot [
color=darkgray,
solid,
forget plot
]
coordinates{
 (466.736842105263,-6)(388.947368421053,-16) 
};
\addplot [
color=darkgray,
solid,
forget plot
]
coordinates{
 (466.736842105263,-8)(388.947368421053,-16) 
};
\addplot [
color=darkgray,
solid,
forget plot
]
coordinates{
 (466.736842105263,-10)(388.947368421053,-16) 
};
\addplot [
color=darkgray,
solid,
forget plot
]
coordinates{
 (466.736842105263,-12)(388.947368421053,-16) 
};
\addplot [
color=darkgray,
solid,
forget plot
]
coordinates{
 (466.736842105263,-14)(388.947368421053,-16) 
};
\addplot [
color=darkgray,
solid,
forget plot
]
coordinates{
 (466.736842105263,-16)(388.947368421053,-16) 
};
\addplot [
color=darkgray,
solid,
forget plot
]
coordinates{
 (466.736842105263,-18)(388.947368421053,-16) 
};
\addplot [
color=darkgray,
solid,
forget plot
]
coordinates{
 (466.736842105263,-8)(388.947368421053,-18) 
};
\addplot [
color=darkgray,
solid,
forget plot
]
coordinates{
 (466.736842105263,-10)(388.947368421053,-18) 
};
\addplot [
color=darkgray,
solid,
forget plot
]
coordinates{
 (466.736842105263,-12)(388.947368421053,-18) 
};
\addplot [
color=darkgray,
solid,
forget plot
]
coordinates{
 (466.736842105263,-14)(388.947368421053,-18) 
};
\addplot [
color=darkgray,
solid,
forget plot
]
coordinates{
 (466.736842105263,-16)(388.947368421053,-18) 
};
\addplot [
color=darkgray,
solid,
forget plot
]
coordinates{
 (466.736842105263,-18)(388.947368421053,-18) 
};
\addplot [
color=darkgray,
solid,
forget plot
]
coordinates{
 (388.947368421053,0)(311.157894736842,0) 
};
\addplot [
color=darkgray,
solid,
forget plot
]
coordinates{
 (388.947368421053,-2)(311.157894736842,0) 
};
\addplot [
color=darkgray,
solid,
forget plot
]
coordinates{
 (388.947368421053,-4)(311.157894736842,0) 
};
\addplot [
color=darkgray,
solid,
forget plot
]
coordinates{
 (388.947368421053,-6)(311.157894736842,0) 
};
\addplot [
color=darkgray,
solid,
forget plot
]
coordinates{
 (388.947368421053,-8)(311.157894736842,0) 
};
\addplot [
color=darkgray,
solid,
forget plot
]
coordinates{
 (388.947368421053,-10)(311.157894736842,0) 
};
\addplot [
color=darkgray,
solid,
forget plot
]
coordinates{
 (388.947368421053,0)(311.157894736842,-2) 
};
\addplot [
color=darkgray,
solid,
forget plot
]
coordinates{
 (388.947368421053,-2)(311.157894736842,-2) 
};
\addplot [
color=darkgray,
solid,
forget plot
]
coordinates{
 (388.947368421053,-4)(311.157894736842,-2) 
};
\addplot [
color=darkgray,
solid,
forget plot
]
coordinates{
 (388.947368421053,-6)(311.157894736842,-2) 
};
\addplot [
color=darkgray,
solid,
forget plot
]
coordinates{
 (388.947368421053,-8)(311.157894736842,-2) 
};
\addplot [
color=darkgray,
solid,
forget plot
]
coordinates{
 (388.947368421053,-10)(311.157894736842,-2) 
};
\addplot [
color=darkgray,
solid,
forget plot
]
coordinates{
 (388.947368421053,-12)(311.157894736842,-2) 
};
\addplot [
color=darkgray,
solid,
forget plot
]
coordinates{
 (388.947368421053,0)(311.157894736842,-4) 
};
\addplot [
color=darkgray,
solid,
forget plot
]
coordinates{
 (388.947368421053,-2)(311.157894736842,-4) 
};
\addplot [
color=darkgray,
solid,
forget plot
]
coordinates{
 (388.947368421053,-4)(311.157894736842,-4) 
};
\addplot [
color=darkgray,
solid,
forget plot
]
coordinates{
 (388.947368421053,-6)(311.157894736842,-4) 
};
\addplot [
color=darkgray,
solid,
forget plot
]
coordinates{
 (388.947368421053,-8)(311.157894736842,-4) 
};
\addplot [
color=darkgray,
solid,
forget plot
]
coordinates{
 (388.947368421053,-10)(311.157894736842,-4) 
};
\addplot [
color=darkgray,
solid,
forget plot
]
coordinates{
 (388.947368421053,-12)(311.157894736842,-4) 
};
\addplot [
color=darkgray,
solid,
forget plot
]
coordinates{
 (388.947368421053,-14)(311.157894736842,-4) 
};
\addplot [
color=darkgray,
solid,
forget plot
]
coordinates{
 (388.947368421053,0)(311.157894736842,-6) 
};
\addplot [
color=darkgray,
solid,
forget plot
]
coordinates{
 (388.947368421053,-2)(311.157894736842,-6) 
};
\addplot [
color=darkgray,
solid,
forget plot
]
coordinates{
 (388.947368421053,-4)(311.157894736842,-6) 
};
\addplot [
color=darkgray,
solid,
forget plot
]
coordinates{
 (388.947368421053,-6)(311.157894736842,-6) 
};
\addplot [
color=darkgray,
solid,
forget plot
]
coordinates{
 (388.947368421053,-8)(311.157894736842,-6) 
};
\addplot [
color=darkgray,
solid,
forget plot
]
coordinates{
 (388.947368421053,-10)(311.157894736842,-6) 
};
\addplot [
color=darkgray,
solid,
forget plot
]
coordinates{
 (388.947368421053,-12)(311.157894736842,-6) 
};
\addplot [
color=darkgray,
solid,
forget plot
]
coordinates{
 (388.947368421053,-14)(311.157894736842,-6) 
};
\addplot [
color=darkgray,
solid,
forget plot
]
coordinates{
 (388.947368421053,-16)(311.157894736842,-6) 
};
\addplot [
color=darkgray,
solid,
forget plot
]
coordinates{
 (388.947368421053,0)(311.157894736842,-8) 
};
\addplot [
color=darkgray,
solid,
forget plot
]
coordinates{
 (388.947368421053,-2)(311.157894736842,-8) 
};
\addplot [
color=darkgray,
solid,
forget plot
]
coordinates{
 (388.947368421053,-4)(311.157894736842,-8) 
};
\addplot [
color=darkgray,
solid,
forget plot
]
coordinates{
 (388.947368421053,-6)(311.157894736842,-8) 
};
\addplot [
color=darkgray,
solid,
forget plot
]
coordinates{
 (388.947368421053,-8)(311.157894736842,-8) 
};
\addplot [
color=darkgray,
solid,
forget plot
]
coordinates{
 (388.947368421053,-10)(311.157894736842,-8) 
};
\addplot [
color=darkgray,
solid,
forget plot
]
coordinates{
 (388.947368421053,-12)(311.157894736842,-8) 
};
\addplot [
color=darkgray,
solid,
forget plot
]
coordinates{
 (388.947368421053,-14)(311.157894736842,-8) 
};
\addplot [
color=darkgray,
solid,
forget plot
]
coordinates{
 (388.947368421053,-16)(311.157894736842,-8) 
};
\addplot [
color=darkgray,
solid,
forget plot
]
coordinates{
 (388.947368421053,-18)(311.157894736842,-8) 
};
\addplot [
color=darkgray,
solid,
forget plot
]
coordinates{
 (388.947368421053,0)(311.157894736842,-10) 
};
\addplot [
color=darkgray,
solid,
forget plot
]
coordinates{
 (388.947368421053,-2)(311.157894736842,-10) 
};
\addplot [
color=darkgray,
solid,
forget plot
]
coordinates{
 (388.947368421053,-4)(311.157894736842,-10) 
};
\addplot [
color=darkgray,
solid,
forget plot
]
coordinates{
 (388.947368421053,-6)(311.157894736842,-10) 
};
\addplot [
color=darkgray,
solid,
forget plot
]
coordinates{
 (388.947368421053,-8)(311.157894736842,-10) 
};
\addplot [
color=darkgray,
solid,
forget plot
]
coordinates{
 (388.947368421053,-10)(311.157894736842,-10) 
};
\addplot [
color=darkgray,
solid,
forget plot
]
coordinates{
 (388.947368421053,-12)(311.157894736842,-10) 
};
\addplot [
color=darkgray,
solid,
forget plot
]
coordinates{
 (388.947368421053,-14)(311.157894736842,-10) 
};
\addplot [
color=darkgray,
solid,
forget plot
]
coordinates{
 (388.947368421053,-16)(311.157894736842,-10) 
};
\addplot [
color=darkgray,
solid,
forget plot
]
coordinates{
 (388.947368421053,-18)(311.157894736842,-10) 
};
\addplot [
color=darkgray,
solid,
forget plot
]
coordinates{
 (388.947368421053,-2)(311.157894736842,-12) 
};
\addplot [
color=darkgray,
solid,
forget plot
]
coordinates{
 (388.947368421053,-4)(311.157894736842,-12) 
};
\addplot [
color=darkgray,
solid,
forget plot
]
coordinates{
 (388.947368421053,-6)(311.157894736842,-12) 
};
\addplot [
color=darkgray,
solid,
forget plot
]
coordinates{
 (388.947368421053,-8)(311.157894736842,-12) 
};
\addplot [
color=darkgray,
solid,
forget plot
]
coordinates{
 (388.947368421053,-10)(311.157894736842,-12) 
};
\addplot [
color=darkgray,
solid,
forget plot
]
coordinates{
 (388.947368421053,-12)(311.157894736842,-12) 
};
\addplot [
color=darkgray,
solid,
forget plot
]
coordinates{
 (388.947368421053,-14)(311.157894736842,-12) 
};
\addplot [
color=darkgray,
solid,
forget plot
]
coordinates{
 (388.947368421053,-16)(311.157894736842,-12) 
};
\addplot [
color=darkgray,
solid,
forget plot
]
coordinates{
 (388.947368421053,-18)(311.157894736842,-12) 
};
\addplot [
color=darkgray,
solid,
forget plot
]
coordinates{
 (388.947368421053,-4)(311.157894736842,-14) 
};
\addplot [
color=darkgray,
solid,
forget plot
]
coordinates{
 (388.947368421053,-6)(311.157894736842,-14) 
};
\addplot [
color=darkgray,
solid,
forget plot
]
coordinates{
 (388.947368421053,-8)(311.157894736842,-14) 
};
\addplot [
color=darkgray,
solid,
forget plot
]
coordinates{
 (388.947368421053,-10)(311.157894736842,-14) 
};
\addplot [
color=darkgray,
solid,
forget plot
]
coordinates{
 (388.947368421053,-12)(311.157894736842,-14) 
};
\addplot [
color=darkgray,
solid,
forget plot
]
coordinates{
 (388.947368421053,-14)(311.157894736842,-14) 
};
\addplot [
color=darkgray,
solid,
forget plot
]
coordinates{
 (388.947368421053,-16)(311.157894736842,-14) 
};
\addplot [
color=darkgray,
solid,
forget plot
]
coordinates{
 (388.947368421053,-18)(311.157894736842,-14) 
};
\addplot [
color=darkgray,
solid,
forget plot
]
coordinates{
 (388.947368421053,-6)(311.157894736842,-16) 
};
\addplot [
color=darkgray,
solid,
forget plot
]
coordinates{
 (388.947368421053,-8)(311.157894736842,-16) 
};
\addplot [
color=darkgray,
solid,
forget plot
]
coordinates{
 (388.947368421053,-10)(311.157894736842,-16) 
};
\addplot [
color=darkgray,
solid,
forget plot
]
coordinates{
 (388.947368421053,-12)(311.157894736842,-16) 
};
\addplot [
color=darkgray,
solid,
forget plot
]
coordinates{
 (388.947368421053,-14)(311.157894736842,-16) 
};
\addplot [
color=darkgray,
solid,
forget plot
]
coordinates{
 (388.947368421053,-16)(311.157894736842,-16) 
};
\addplot [
color=darkgray,
solid,
forget plot
]
coordinates{
 (388.947368421053,-18)(311.157894736842,-16) 
};
\addplot [
color=darkgray,
solid,
forget plot
]
coordinates{
 (388.947368421053,-8)(311.157894736842,-18) 
};
\addplot [
color=darkgray,
solid,
forget plot
]
coordinates{
 (388.947368421053,-10)(311.157894736842,-18) 
};
\addplot [
color=darkgray,
solid,
forget plot
]
coordinates{
 (388.947368421053,-12)(311.157894736842,-18) 
};
\addplot [
color=darkgray,
solid,
forget plot
]
coordinates{
 (388.947368421053,-14)(311.157894736842,-18) 
};
\addplot [
color=darkgray,
solid,
forget plot
]
coordinates{
 (388.947368421053,-16)(311.157894736842,-18) 
};
\addplot [
color=darkgray,
solid,
forget plot
]
coordinates{
 (388.947368421053,-18)(311.157894736842,-18) 
};
\addplot [
color=darkgray,
solid,
forget plot
]
coordinates{
 (311.157894736842,0)(233.368421052632,0) 
};
\addplot [
color=darkgray,
solid,
forget plot
]
coordinates{
 (311.157894736842,-2)(233.368421052632,0) 
};
\addplot [
color=darkgray,
solid,
forget plot
]
coordinates{
 (311.157894736842,-4)(233.368421052632,0) 
};
\addplot [
color=darkgray,
solid,
forget plot
]
coordinates{
 (311.157894736842,-6)(233.368421052632,0) 
};
\addplot [
color=darkgray,
solid,
forget plot
]
coordinates{
 (311.157894736842,-8)(233.368421052632,0) 
};
\addplot [
color=darkgray,
solid,
forget plot
]
coordinates{
 (311.157894736842,-10)(233.368421052632,0) 
};
\addplot [
color=darkgray,
solid,
forget plot
]
coordinates{
 (311.157894736842,0)(233.368421052632,-2) 
};
\addplot [
color=darkgray,
solid,
forget plot
]
coordinates{
 (311.157894736842,-2)(233.368421052632,-2) 
};
\addplot [
color=darkgray,
solid,
forget plot
]
coordinates{
 (311.157894736842,-4)(233.368421052632,-2) 
};
\addplot [
color=darkgray,
solid,
forget plot
]
coordinates{
 (311.157894736842,-6)(233.368421052632,-2) 
};
\addplot [
color=darkgray,
solid,
forget plot
]
coordinates{
 (311.157894736842,-8)(233.368421052632,-2) 
};
\addplot [
color=darkgray,
solid,
forget plot
]
coordinates{
 (311.157894736842,-10)(233.368421052632,-2) 
};
\addplot [
color=darkgray,
solid,
forget plot
]
coordinates{
 (311.157894736842,-12)(233.368421052632,-2) 
};
\addplot [
color=darkgray,
solid,
forget plot
]
coordinates{
 (311.157894736842,0)(233.368421052632,-4) 
};
\addplot [
color=darkgray,
solid,
forget plot
]
coordinates{
 (311.157894736842,-2)(233.368421052632,-4) 
};
\addplot [
color=darkgray,
solid,
forget plot
]
coordinates{
 (311.157894736842,-4)(233.368421052632,-4) 
};
\addplot [
color=darkgray,
solid,
forget plot
]
coordinates{
 (311.157894736842,-6)(233.368421052632,-4) 
};
\addplot [
color=darkgray,
solid,
forget plot
]
coordinates{
 (311.157894736842,-8)(233.368421052632,-4) 
};
\addplot [
color=darkgray,
solid,
forget plot
]
coordinates{
 (311.157894736842,-10)(233.368421052632,-4) 
};
\addplot [
color=darkgray,
solid,
forget plot
]
coordinates{
 (311.157894736842,-12)(233.368421052632,-4) 
};
\addplot [
color=darkgray,
solid,
forget plot
]
coordinates{
 (311.157894736842,-14)(233.368421052632,-4) 
};
\addplot [
color=darkgray,
solid,
forget plot
]
coordinates{
 (311.157894736842,0)(233.368421052632,-6) 
};
\addplot [
color=darkgray,
solid,
forget plot
]
coordinates{
 (311.157894736842,-2)(233.368421052632,-6) 
};
\addplot [
color=darkgray,
solid,
forget plot
]
coordinates{
 (311.157894736842,-4)(233.368421052632,-6) 
};
\addplot [
color=darkgray,
solid,
forget plot
]
coordinates{
 (311.157894736842,-6)(233.368421052632,-6) 
};
\addplot [
color=darkgray,
solid,
forget plot
]
coordinates{
 (311.157894736842,-8)(233.368421052632,-6) 
};
\addplot [
color=darkgray,
solid,
forget plot
]
coordinates{
 (311.157894736842,-10)(233.368421052632,-6) 
};
\addplot [
color=darkgray,
solid,
forget plot
]
coordinates{
 (311.157894736842,-12)(233.368421052632,-6) 
};
\addplot [
color=darkgray,
solid,
forget plot
]
coordinates{
 (311.157894736842,-14)(233.368421052632,-6) 
};
\addplot [
color=darkgray,
solid,
forget plot
]
coordinates{
 (311.157894736842,-16)(233.368421052632,-6) 
};
\addplot [
color=darkgray,
solid,
forget plot
]
coordinates{
 (311.157894736842,0)(233.368421052632,-8) 
};
\addplot [
color=darkgray,
solid,
forget plot
]
coordinates{
 (311.157894736842,-2)(233.368421052632,-8) 
};
\addplot [
color=darkgray,
solid,
forget plot
]
coordinates{
 (311.157894736842,-4)(233.368421052632,-8) 
};
\addplot [
color=darkgray,
solid,
forget plot
]
coordinates{
 (311.157894736842,-6)(233.368421052632,-8) 
};
\addplot [
color=darkgray,
solid,
forget plot
]
coordinates{
 (311.157894736842,-8)(233.368421052632,-8) 
};
\addplot [
color=darkgray,
solid,
forget plot
]
coordinates{
 (311.157894736842,-10)(233.368421052632,-8) 
};
\addplot [
color=darkgray,
solid,
forget plot
]
coordinates{
 (311.157894736842,-12)(233.368421052632,-8) 
};
\addplot [
color=darkgray,
solid,
forget plot
]
coordinates{
 (311.157894736842,-14)(233.368421052632,-8) 
};
\addplot [
color=darkgray,
solid,
forget plot
]
coordinates{
 (311.157894736842,-16)(233.368421052632,-8) 
};
\addplot [
color=darkgray,
solid,
forget plot
]
coordinates{
 (311.157894736842,-18)(233.368421052632,-8) 
};
\addplot [
color=darkgray,
solid,
forget plot
]
coordinates{
 (311.157894736842,0)(233.368421052632,-10) 
};
\addplot [
color=darkgray,
solid,
forget plot
]
coordinates{
 (311.157894736842,-2)(233.368421052632,-10) 
};
\addplot [
color=darkgray,
solid,
forget plot
]
coordinates{
 (311.157894736842,-4)(233.368421052632,-10) 
};
\addplot [
color=darkgray,
solid,
forget plot
]
coordinates{
 (311.157894736842,-6)(233.368421052632,-10) 
};
\addplot [
color=darkgray,
solid,
forget plot
]
coordinates{
 (311.157894736842,-8)(233.368421052632,-10) 
};
\addplot [
color=darkgray,
solid,
forget plot
]
coordinates{
 (311.157894736842,-10)(233.368421052632,-10) 
};
\addplot [
color=darkgray,
solid,
forget plot
]
coordinates{
 (311.157894736842,-12)(233.368421052632,-10) 
};
\addplot [
color=darkgray,
solid,
forget plot
]
coordinates{
 (311.157894736842,-14)(233.368421052632,-10) 
};
\addplot [
color=darkgray,
solid,
forget plot
]
coordinates{
 (311.157894736842,-16)(233.368421052632,-10) 
};
\addplot [
color=darkgray,
solid,
forget plot
]
coordinates{
 (311.157894736842,-18)(233.368421052632,-10) 
};
\addplot [
color=darkgray,
solid,
forget plot
]
coordinates{
 (311.157894736842,-2)(233.368421052632,-12) 
};
\addplot [
color=darkgray,
solid,
forget plot
]
coordinates{
 (311.157894736842,-4)(233.368421052632,-12) 
};
\addplot [
color=darkgray,
solid,
forget plot
]
coordinates{
 (311.157894736842,-6)(233.368421052632,-12) 
};
\addplot [
color=darkgray,
solid,
forget plot
]
coordinates{
 (311.157894736842,-8)(233.368421052632,-12) 
};
\addplot [
color=darkgray,
solid,
forget plot
]
coordinates{
 (311.157894736842,-10)(233.368421052632,-12) 
};
\addplot [
color=darkgray,
solid,
forget plot
]
coordinates{
 (311.157894736842,-12)(233.368421052632,-12) 
};
\addplot [
color=darkgray,
solid,
forget plot
]
coordinates{
 (311.157894736842,-14)(233.368421052632,-12) 
};
\addplot [
color=darkgray,
solid,
forget plot
]
coordinates{
 (311.157894736842,-16)(233.368421052632,-12) 
};
\addplot [
color=darkgray,
solid,
forget plot
]
coordinates{
 (311.157894736842,-18)(233.368421052632,-12) 
};
\addplot [
color=darkgray,
solid,
forget plot
]
coordinates{
 (311.157894736842,-4)(233.368421052632,-14) 
};
\addplot [
color=darkgray,
solid,
forget plot
]
coordinates{
 (311.157894736842,-6)(233.368421052632,-14) 
};
\addplot [
color=darkgray,
solid,
forget plot
]
coordinates{
 (311.157894736842,-8)(233.368421052632,-14) 
};
\addplot [
color=darkgray,
solid,
forget plot
]
coordinates{
 (311.157894736842,-10)(233.368421052632,-14) 
};
\addplot [
color=darkgray,
solid,
forget plot
]
coordinates{
 (311.157894736842,-12)(233.368421052632,-14) 
};
\addplot [
color=darkgray,
solid,
forget plot
]
coordinates{
 (311.157894736842,-14)(233.368421052632,-14) 
};
\addplot [
color=darkgray,
solid,
forget plot
]
coordinates{
 (311.157894736842,-16)(233.368421052632,-14) 
};
\addplot [
color=darkgray,
solid,
forget plot
]
coordinates{
 (311.157894736842,-18)(233.368421052632,-14) 
};
\addplot [
color=darkgray,
solid,
forget plot
]
coordinates{
 (311.157894736842,-6)(233.368421052632,-16) 
};
\addplot [
color=darkgray,
solid,
forget plot
]
coordinates{
 (311.157894736842,-8)(233.368421052632,-16) 
};
\addplot [
color=darkgray,
solid,
forget plot
]
coordinates{
 (311.157894736842,-10)(233.368421052632,-16) 
};
\addplot [
color=darkgray,
solid,
forget plot
]
coordinates{
 (311.157894736842,-12)(233.368421052632,-16) 
};
\addplot [
color=darkgray,
solid,
forget plot
]
coordinates{
 (311.157894736842,-14)(233.368421052632,-16) 
};
\addplot [
color=darkgray,
solid,
forget plot
]
coordinates{
 (311.157894736842,-16)(233.368421052632,-16) 
};
\addplot [
color=darkgray,
solid,
forget plot
]
coordinates{
 (311.157894736842,-18)(233.368421052632,-16) 
};
\addplot [
color=darkgray,
solid,
forget plot
]
coordinates{
 (311.157894736842,-8)(233.368421052632,-18) 
};
\addplot [
color=darkgray,
solid,
forget plot
]
coordinates{
 (311.157894736842,-10)(233.368421052632,-18) 
};
\addplot [
color=darkgray,
solid,
forget plot
]
coordinates{
 (311.157894736842,-12)(233.368421052632,-18) 
};
\addplot [
color=darkgray,
solid,
forget plot
]
coordinates{
 (311.157894736842,-14)(233.368421052632,-18) 
};
\addplot [
color=darkgray,
solid,
forget plot
]
coordinates{
 (311.157894736842,-16)(233.368421052632,-18) 
};
\addplot [
color=darkgray,
solid,
forget plot
]
coordinates{
 (311.157894736842,-18)(233.368421052632,-18) 
};
\addplot [
color=darkgray,
solid,
forget plot
]
coordinates{
 (233.368421052632,0)(155.578947368421,0) 
};
\addplot [
color=darkgray,
solid,
forget plot
]
coordinates{
 (233.368421052632,-2)(155.578947368421,0) 
};
\addplot [
color=darkgray,
solid,
forget plot
]
coordinates{
 (233.368421052632,-4)(155.578947368421,0) 
};
\addplot [
color=darkgray,
solid,
forget plot
]
coordinates{
 (233.368421052632,-6)(155.578947368421,0) 
};
\addplot [
color=darkgray,
solid,
forget plot
]
coordinates{
 (233.368421052632,-8)(155.578947368421,0) 
};
\addplot [
color=darkgray,
solid,
forget plot
]
coordinates{
 (233.368421052632,-10)(155.578947368421,0) 
};
\addplot [
color=darkgray,
solid,
forget plot
]
coordinates{
 (233.368421052632,0)(155.578947368421,-2) 
};
\addplot [
color=darkgray,
solid,
forget plot
]
coordinates{
 (233.368421052632,-2)(155.578947368421,-2) 
};
\addplot [
color=darkgray,
solid,
forget plot
]
coordinates{
 (233.368421052632,-4)(155.578947368421,-2) 
};
\addplot [
color=darkgray,
solid,
forget plot
]
coordinates{
 (233.368421052632,-6)(155.578947368421,-2) 
};
\addplot [
color=darkgray,
solid,
forget plot
]
coordinates{
 (233.368421052632,-8)(155.578947368421,-2) 
};
\addplot [
color=darkgray,
solid,
forget plot
]
coordinates{
 (233.368421052632,-10)(155.578947368421,-2) 
};
\addplot [
color=darkgray,
solid,
forget plot
]
coordinates{
 (233.368421052632,-12)(155.578947368421,-2) 
};
\addplot [
color=darkgray,
solid,
forget plot
]
coordinates{
 (233.368421052632,0)(155.578947368421,-4) 
};
\addplot [
color=darkgray,
solid,
forget plot
]
coordinates{
 (233.368421052632,-2)(155.578947368421,-4) 
};
\addplot [
color=darkgray,
solid,
forget plot
]
coordinates{
 (233.368421052632,-4)(155.578947368421,-4) 
};
\addplot [
color=darkgray,
solid,
forget plot
]
coordinates{
 (233.368421052632,-6)(155.578947368421,-4) 
};
\addplot [
color=darkgray,
solid,
forget plot
]
coordinates{
 (233.368421052632,-8)(155.578947368421,-4) 
};
\addplot [
color=darkgray,
solid,
forget plot
]
coordinates{
 (233.368421052632,-10)(155.578947368421,-4) 
};
\addplot [
color=darkgray,
solid,
forget plot
]
coordinates{
 (233.368421052632,-12)(155.578947368421,-4) 
};
\addplot [
color=darkgray,
solid,
forget plot
]
coordinates{
 (233.368421052632,-14)(155.578947368421,-4) 
};
\addplot [
color=darkgray,
solid,
forget plot
]
coordinates{
 (233.368421052632,0)(155.578947368421,-6) 
};
\addplot [
color=darkgray,
solid,
forget plot
]
coordinates{
 (233.368421052632,-2)(155.578947368421,-6) 
};
\addplot [
color=darkgray,
solid,
forget plot
]
coordinates{
 (233.368421052632,-4)(155.578947368421,-6) 
};
\addplot [
color=darkgray,
solid,
forget plot
]
coordinates{
 (233.368421052632,-6)(155.578947368421,-6) 
};
\addplot [
color=darkgray,
solid,
forget plot
]
coordinates{
 (233.368421052632,-8)(155.578947368421,-6) 
};
\addplot [
color=darkgray,
solid,
forget plot
]
coordinates{
 (233.368421052632,-10)(155.578947368421,-6) 
};
\addplot [
color=darkgray,
solid,
forget plot
]
coordinates{
 (233.368421052632,-12)(155.578947368421,-6) 
};
\addplot [
color=darkgray,
solid,
forget plot
]
coordinates{
 (233.368421052632,-14)(155.578947368421,-6) 
};
\addplot [
color=darkgray,
solid,
forget plot
]
coordinates{
 (233.368421052632,-16)(155.578947368421,-6) 
};
\addplot [
color=darkgray,
solid,
forget plot
]
coordinates{
 (233.368421052632,0)(155.578947368421,-8) 
};
\addplot [
color=darkgray,
solid,
forget plot
]
coordinates{
 (233.368421052632,-2)(155.578947368421,-8) 
};
\addplot [
color=darkgray,
solid,
forget plot
]
coordinates{
 (233.368421052632,-4)(155.578947368421,-8) 
};
\addplot [
color=darkgray,
solid,
forget plot
]
coordinates{
 (233.368421052632,-6)(155.578947368421,-8) 
};
\addplot [
color=darkgray,
solid,
forget plot
]
coordinates{
 (233.368421052632,-8)(155.578947368421,-8) 
};
\addplot [
color=darkgray,
solid,
forget plot
]
coordinates{
 (233.368421052632,-10)(155.578947368421,-8) 
};
\addplot [
color=darkgray,
solid,
forget plot
]
coordinates{
 (233.368421052632,-12)(155.578947368421,-8) 
};
\addplot [
color=darkgray,
solid,
forget plot
]
coordinates{
 (233.368421052632,-14)(155.578947368421,-8) 
};
\addplot [
color=darkgray,
solid,
forget plot
]
coordinates{
 (233.368421052632,-16)(155.578947368421,-8) 
};
\addplot [
color=darkgray,
solid,
forget plot
]
coordinates{
 (233.368421052632,-18)(155.578947368421,-8) 
};
\addplot [
color=darkgray,
solid,
forget plot
]
coordinates{
 (233.368421052632,0)(155.578947368421,-10) 
};
\addplot [
color=darkgray,
solid,
forget plot
]
coordinates{
 (233.368421052632,-2)(155.578947368421,-10) 
};
\addplot [
color=darkgray,
solid,
forget plot
]
coordinates{
 (233.368421052632,-4)(155.578947368421,-10) 
};
\addplot [
color=darkgray,
solid,
forget plot
]
coordinates{
 (233.368421052632,-6)(155.578947368421,-10) 
};
\addplot [
color=darkgray,
solid,
forget plot
]
coordinates{
 (233.368421052632,-8)(155.578947368421,-10) 
};
\addplot [
color=darkgray,
solid,
forget plot
]
coordinates{
 (233.368421052632,-10)(155.578947368421,-10) 
};
\addplot [
color=darkgray,
solid,
forget plot
]
coordinates{
 (233.368421052632,-12)(155.578947368421,-10) 
};
\addplot [
color=darkgray,
solid,
forget plot
]
coordinates{
 (233.368421052632,-14)(155.578947368421,-10) 
};
\addplot [
color=darkgray,
solid,
forget plot
]
coordinates{
 (233.368421052632,-16)(155.578947368421,-10) 
};
\addplot [
color=darkgray,
solid,
forget plot
]
coordinates{
 (233.368421052632,-18)(155.578947368421,-10) 
};
\addplot [
color=darkgray,
solid,
forget plot
]
coordinates{
 (233.368421052632,-2)(155.578947368421,-12) 
};
\addplot [
color=darkgray,
solid,
forget plot
]
coordinates{
 (233.368421052632,-4)(155.578947368421,-12) 
};
\addplot [
color=darkgray,
solid,
forget plot
]
coordinates{
 (233.368421052632,-6)(155.578947368421,-12) 
};
\addplot [
color=darkgray,
solid,
forget plot
]
coordinates{
 (233.368421052632,-8)(155.578947368421,-12) 
};
\addplot [
color=darkgray,
solid,
forget plot
]
coordinates{
 (233.368421052632,-10)(155.578947368421,-12) 
};
\addplot [
color=darkgray,
solid,
forget plot
]
coordinates{
 (233.368421052632,-12)(155.578947368421,-12) 
};
\addplot [
color=darkgray,
solid,
forget plot
]
coordinates{
 (233.368421052632,-14)(155.578947368421,-12) 
};
\addplot [
color=darkgray,
solid,
forget plot
]
coordinates{
 (233.368421052632,-16)(155.578947368421,-12) 
};
\addplot [
color=darkgray,
solid,
forget plot
]
coordinates{
 (233.368421052632,-18)(155.578947368421,-12) 
};
\addplot [
color=darkgray,
solid,
forget plot
]
coordinates{
 (233.368421052632,-4)(155.578947368421,-14) 
};
\addplot [
color=darkgray,
solid,
forget plot
]
coordinates{
 (233.368421052632,-6)(155.578947368421,-14) 
};
\addplot [
color=darkgray,
solid,
forget plot
]
coordinates{
 (233.368421052632,-8)(155.578947368421,-14) 
};
\addplot [
color=darkgray,
solid,
forget plot
]
coordinates{
 (233.368421052632,-10)(155.578947368421,-14) 
};
\addplot [
color=darkgray,
solid,
forget plot
]
coordinates{
 (233.368421052632,-12)(155.578947368421,-14) 
};
\addplot [
color=darkgray,
solid,
forget plot
]
coordinates{
 (233.368421052632,-14)(155.578947368421,-14) 
};
\addplot [
color=darkgray,
solid,
forget plot
]
coordinates{
 (233.368421052632,-16)(155.578947368421,-14) 
};
\addplot [
color=darkgray,
solid,
forget plot
]
coordinates{
 (233.368421052632,-18)(155.578947368421,-14) 
};
\addplot [
color=darkgray,
solid,
forget plot
]
coordinates{
 (233.368421052632,-6)(155.578947368421,-16) 
};
\addplot [
color=darkgray,
solid,
forget plot
]
coordinates{
 (233.368421052632,-8)(155.578947368421,-16) 
};
\addplot [
color=darkgray,
solid,
forget plot
]
coordinates{
 (233.368421052632,-10)(155.578947368421,-16) 
};
\addplot [
color=darkgray,
solid,
forget plot
]
coordinates{
 (233.368421052632,-12)(155.578947368421,-16) 
};
\addplot [
color=darkgray,
solid,
forget plot
]
coordinates{
 (233.368421052632,-14)(155.578947368421,-16) 
};
\addplot [
color=darkgray,
solid,
forget plot
]
coordinates{
 (233.368421052632,-16)(155.578947368421,-16) 
};
\addplot [
color=darkgray,
solid,
forget plot
]
coordinates{
 (233.368421052632,-18)(155.578947368421,-16) 
};
\addplot [
color=darkgray,
solid,
forget plot
]
coordinates{
 (233.368421052632,-8)(155.578947368421,-18) 
};
\addplot [
color=darkgray,
solid,
forget plot
]
coordinates{
 (233.368421052632,-10)(155.578947368421,-18) 
};
\addplot [
color=darkgray,
solid,
forget plot
]
coordinates{
 (233.368421052632,-12)(155.578947368421,-18) 
};
\addplot [
color=darkgray,
solid,
forget plot
]
coordinates{
 (233.368421052632,-14)(155.578947368421,-18) 
};
\addplot [
color=darkgray,
solid,
forget plot
]
coordinates{
 (233.368421052632,-16)(155.578947368421,-18) 
};
\addplot [
color=darkgray,
solid,
forget plot
]
coordinates{
 (233.368421052632,-18)(155.578947368421,-18) 
};
\addplot [
color=darkgray,
solid,
forget plot
]
coordinates{
 (155.578947368421,0)(77.7894736842105,0) 
};
\addplot [
color=darkgray,
solid,
forget plot
]
coordinates{
 (155.578947368421,-2)(77.7894736842105,0) 
};
\addplot [
color=darkgray,
solid,
forget plot
]
coordinates{
 (155.578947368421,-4)(77.7894736842105,0) 
};
\addplot [
color=darkgray,
solid,
forget plot
]
coordinates{
 (155.578947368421,-6)(77.7894736842105,0) 
};
\addplot [
color=darkgray,
solid,
forget plot
]
coordinates{
 (155.578947368421,-8)(77.7894736842105,0) 
};
\addplot [
color=darkgray,
solid,
forget plot
]
coordinates{
 (155.578947368421,-10)(77.7894736842105,0) 
};
\addplot [
color=darkgray,
solid,
forget plot
]
coordinates{
 (155.578947368421,0)(77.7894736842105,-2) 
};
\addplot [
color=darkgray,
solid,
forget plot
]
coordinates{
 (155.578947368421,-2)(77.7894736842105,-2) 
};
\addplot [
color=darkgray,
solid,
forget plot
]
coordinates{
 (155.578947368421,-4)(77.7894736842105,-2) 
};
\addplot [
color=darkgray,
solid,
forget plot
]
coordinates{
 (155.578947368421,-6)(77.7894736842105,-2) 
};
\addplot [
color=darkgray,
solid,
forget plot
]
coordinates{
 (155.578947368421,-8)(77.7894736842105,-2) 
};
\addplot [
color=darkgray,
solid,
forget plot
]
coordinates{
 (155.578947368421,-10)(77.7894736842105,-2) 
};
\addplot [
color=darkgray,
solid,
forget plot
]
coordinates{
 (155.578947368421,-12)(77.7894736842105,-2) 
};
\addplot [
color=darkgray,
solid,
forget plot
]
coordinates{
 (155.578947368421,0)(77.7894736842105,-4) 
};
\addplot [
color=darkgray,
solid,
forget plot
]
coordinates{
 (155.578947368421,-2)(77.7894736842105,-4) 
};
\addplot [
color=darkgray,
solid,
forget plot
]
coordinates{
 (155.578947368421,-4)(77.7894736842105,-4) 
};
\addplot [
color=darkgray,
solid,
forget plot
]
coordinates{
 (155.578947368421,-6)(77.7894736842105,-4) 
};
\addplot [
color=darkgray,
solid,
forget plot
]
coordinates{
 (155.578947368421,-8)(77.7894736842105,-4) 
};
\addplot [
color=darkgray,
solid,
forget plot
]
coordinates{
 (155.578947368421,-10)(77.7894736842105,-4) 
};
\addplot [
color=darkgray,
solid,
forget plot
]
coordinates{
 (155.578947368421,-12)(77.7894736842105,-4) 
};
\addplot [
color=darkgray,
solid,
forget plot
]
coordinates{
 (155.578947368421,-14)(77.7894736842105,-4) 
};
\addplot [
color=darkgray,
solid,
forget plot
]
coordinates{
 (155.578947368421,0)(77.7894736842105,-6) 
};
\addplot [
color=darkgray,
solid,
forget plot
]
coordinates{
 (155.578947368421,-2)(77.7894736842105,-6) 
};
\addplot [
color=darkgray,
solid,
forget plot
]
coordinates{
 (155.578947368421,-4)(77.7894736842105,-6) 
};
\addplot [
color=darkgray,
solid,
forget plot
]
coordinates{
 (155.578947368421,-6)(77.7894736842105,-6) 
};
\addplot [
color=darkgray,
solid,
forget plot
]
coordinates{
 (155.578947368421,-8)(77.7894736842105,-6) 
};
\addplot [
color=darkgray,
solid,
forget plot
]
coordinates{
 (155.578947368421,-10)(77.7894736842105,-6) 
};
\addplot [
color=darkgray,
solid,
forget plot
]
coordinates{
 (155.578947368421,-12)(77.7894736842105,-6) 
};
\addplot [
color=darkgray,
solid,
forget plot
]
coordinates{
 (155.578947368421,-14)(77.7894736842105,-6) 
};
\addplot [
color=darkgray,
solid,
forget plot
]
coordinates{
 (155.578947368421,-16)(77.7894736842105,-6) 
};
\addplot [
color=darkgray,
solid,
forget plot
]
coordinates{
 (155.578947368421,0)(77.7894736842105,-8) 
};
\addplot [
color=darkgray,
solid,
forget plot
]
coordinates{
 (155.578947368421,-2)(77.7894736842105,-8) 
};
\addplot [
color=darkgray,
solid,
forget plot
]
coordinates{
 (155.578947368421,-4)(77.7894736842105,-8) 
};
\addplot [
color=darkgray,
solid,
forget plot
]
coordinates{
 (155.578947368421,-6)(77.7894736842105,-8) 
};
\addplot [
color=darkgray,
solid,
forget plot
]
coordinates{
 (155.578947368421,-8)(77.7894736842105,-8) 
};
\addplot [
color=darkgray,
solid,
forget plot
]
coordinates{
 (155.578947368421,-10)(77.7894736842105,-8) 
};
\addplot [
color=darkgray,
solid,
forget plot
]
coordinates{
 (155.578947368421,-12)(77.7894736842105,-8) 
};
\addplot [
color=darkgray,
solid,
forget plot
]
coordinates{
 (155.578947368421,-14)(77.7894736842105,-8) 
};
\addplot [
color=darkgray,
solid,
forget plot
]
coordinates{
 (155.578947368421,-16)(77.7894736842105,-8) 
};
\addplot [
color=darkgray,
solid,
forget plot
]
coordinates{
 (155.578947368421,-18)(77.7894736842105,-8) 
};
\addplot [
color=darkgray,
solid,
forget plot
]
coordinates{
 (155.578947368421,0)(77.7894736842105,-10) 
};
\addplot [
color=darkgray,
solid,
forget plot
]
coordinates{
 (155.578947368421,-2)(77.7894736842105,-10) 
};
\addplot [
color=darkgray,
solid,
forget plot
]
coordinates{
 (155.578947368421,-4)(77.7894736842105,-10) 
};
\addplot [
color=darkgray,
solid,
forget plot
]
coordinates{
 (155.578947368421,-6)(77.7894736842105,-10) 
};
\addplot [
color=darkgray,
solid,
forget plot
]
coordinates{
 (155.578947368421,-8)(77.7894736842105,-10) 
};
\addplot [
color=darkgray,
solid,
forget plot
]
coordinates{
 (155.578947368421,-10)(77.7894736842105,-10) 
};
\addplot [
color=darkgray,
solid,
forget plot
]
coordinates{
 (155.578947368421,-12)(77.7894736842105,-10) 
};
\addplot [
color=darkgray,
solid,
forget plot
]
coordinates{
 (155.578947368421,-14)(77.7894736842105,-10) 
};
\addplot [
color=darkgray,
solid,
forget plot
]
coordinates{
 (155.578947368421,-16)(77.7894736842105,-10) 
};
\addplot [
color=darkgray,
solid,
forget plot
]
coordinates{
 (155.578947368421,-18)(77.7894736842105,-10) 
};
\addplot [
color=darkgray,
solid,
forget plot
]
coordinates{
 (155.578947368421,-2)(77.7894736842105,-12) 
};
\addplot [
color=darkgray,
solid,
forget plot
]
coordinates{
 (155.578947368421,-4)(77.7894736842105,-12) 
};
\addplot [
color=darkgray,
solid,
forget plot
]
coordinates{
 (155.578947368421,-6)(77.7894736842105,-12) 
};
\addplot [
color=darkgray,
solid,
forget plot
]
coordinates{
 (155.578947368421,-8)(77.7894736842105,-12) 
};
\addplot [
color=darkgray,
solid,
forget plot
]
coordinates{
 (155.578947368421,-10)(77.7894736842105,-12) 
};
\addplot [
color=darkgray,
solid,
forget plot
]
coordinates{
 (155.578947368421,-12)(77.7894736842105,-12) 
};
\addplot [
color=darkgray,
solid,
forget plot
]
coordinates{
 (155.578947368421,-14)(77.7894736842105,-12) 
};
\addplot [
color=darkgray,
solid,
forget plot
]
coordinates{
 (155.578947368421,-16)(77.7894736842105,-12) 
};
\addplot [
color=darkgray,
solid,
forget plot
]
coordinates{
 (155.578947368421,-18)(77.7894736842105,-12) 
};
\addplot [
color=darkgray,
solid,
forget plot
]
coordinates{
 (155.578947368421,-4)(77.7894736842105,-14) 
};
\addplot [
color=darkgray,
solid,
forget plot
]
coordinates{
 (155.578947368421,-6)(77.7894736842105,-14) 
};
\addplot [
color=darkgray,
solid,
forget plot
]
coordinates{
 (155.578947368421,-8)(77.7894736842105,-14) 
};
\addplot [
color=darkgray,
solid,
forget plot
]
coordinates{
 (155.578947368421,-10)(77.7894736842105,-14) 
};
\addplot [
color=darkgray,
solid,
forget plot
]
coordinates{
 (155.578947368421,-12)(77.7894736842105,-14) 
};
\addplot [
color=darkgray,
solid,
forget plot
]
coordinates{
 (155.578947368421,-14)(77.7894736842105,-14) 
};
\addplot [
color=darkgray,
solid,
forget plot
]
coordinates{
 (155.578947368421,-16)(77.7894736842105,-14) 
};
\addplot [
color=darkgray,
solid,
forget plot
]
coordinates{
 (155.578947368421,-18)(77.7894736842105,-14) 
};
\addplot [
color=darkgray,
solid,
forget plot
]
coordinates{
 (155.578947368421,-6)(77.7894736842105,-16) 
};
\addplot [
color=darkgray,
solid,
forget plot
]
coordinates{
 (155.578947368421,-8)(77.7894736842105,-16) 
};
\addplot [
color=darkgray,
solid,
forget plot
]
coordinates{
 (155.578947368421,-10)(77.7894736842105,-16) 
};
\addplot [
color=darkgray,
solid,
forget plot
]
coordinates{
 (155.578947368421,-12)(77.7894736842105,-16) 
};
\addplot [
color=darkgray,
solid,
forget plot
]
coordinates{
 (155.578947368421,-14)(77.7894736842105,-16) 
};
\addplot [
color=darkgray,
solid,
forget plot
]
coordinates{
 (155.578947368421,-16)(77.7894736842105,-16) 
};
\addplot [
color=darkgray,
solid,
forget plot
]
coordinates{
 (155.578947368421,-18)(77.7894736842105,-16) 
};
\addplot [
color=darkgray,
solid,
forget plot
]
coordinates{
 (155.578947368421,-8)(77.7894736842105,-18) 
};
\addplot [
color=darkgray,
solid,
forget plot
]
coordinates{
 (155.578947368421,-10)(77.7894736842105,-18) 
};
\addplot [
color=darkgray,
solid,
forget plot
]
coordinates{
 (155.578947368421,-12)(77.7894736842105,-18) 
};
\addplot [
color=darkgray,
solid,
forget plot
]
coordinates{
 (155.578947368421,-14)(77.7894736842105,-18) 
};
\addplot [
color=darkgray,
solid,
forget plot
]
coordinates{
 (155.578947368421,-16)(77.7894736842105,-18) 
};
\addplot [
color=darkgray,
solid,
forget plot
]
coordinates{
 (155.578947368421,-18)(77.7894736842105,-18) 
};

\addplot [
color=red,
solid,
line width=1pt,
forget plot
]
coordinates{
 (77.7894736842105,0)(0,0) 
};
\addplot [
color=red,
solid,
line width=1pt,
forget plot
]
coordinates{
 (77.7894736842105,-2)(0,0) 
};
\addplot [
color=red,
solid,
line width=1pt,
forget plot
]
coordinates{
 (77.7894736842105,-4)(0,0) 
};
\addplot [
color=red,
solid,
line width=1pt,
forget plot
]
coordinates{
 (77.7894736842105,-6)(0,0) 
};
\addplot [
color=red,
solid,
line width=1pt,
forget plot
]
coordinates{
 (77.7894736842105,-8)(0,0) 
};
\addplot [
color=red,
solid,
line width=1pt,
forget plot
]
coordinates{
 (77.7894736842105,-10)(0,0) 
};

\addplot [
mark size=1.5pt,
only marks,
mark=*,
mark options={solid,fill=black,draw=black},
forget plot
]
coordinates{
 (0,0)(0,-2)(0,-4)(0,-6)(0,-8)(0,-10)(0,-12)(0,-14)(0,-16)(0,-18)(77.7894736842105,0)(77.7894736842105,-2)(77.7894736842105,-4)(77.7894736842105,-6)(77.7894736842105,-8)(77.7894736842105,-10)(77.7894736842105,-12)(77.7894736842105,-14)(77.7894736842105,-16)(77.7894736842105,-18)(155.578947368421,0)(155.578947368421,-2)(155.578947368421,-4)(155.578947368421,-6)(155.578947368421,-8)(155.578947368421,-10)(155.578947368421,-12)(155.578947368421,-14)(155.578947368421,-16)(155.578947368421,-18)(233.368421052632,0)(233.368421052632,-2)(233.368421052632,-4)(233.368421052632,-6)(233.368421052632,-8)(233.368421052632,-10)(233.368421052632,-12)(233.368421052632,-14)(233.368421052632,-16)(233.368421052632,-18)(311.157894736842,0)(311.157894736842,-2)(311.157894736842,-4)(311.157894736842,-6)(311.157894736842,-8)(311.157894736842,-10)(311.157894736842,-12)(311.157894736842,-14)(311.157894736842,-16)(311.157894736842,-18)(388.947368421053,0)(388.947368421053,-2)(388.947368421053,-4)(388.947368421053,-6)(388.947368421053,-8)(388.947368421053,-10)(388.947368421053,-12)(388.947368421053,-14)(388.947368421053,-16)(388.947368421053,-18)(466.736842105263,0)(466.736842105263,-2)(466.736842105263,-4)(466.736842105263,-6)(466.736842105263,-8)(466.736842105263,-10)(466.736842105263,-12)(466.736842105263,-14)(466.736842105263,-16)(466.736842105263,-18)(544.526315789474,0)(544.526315789474,-2)(544.526315789474,-4)(544.526315789474,-6)(544.526315789474,-8)(544.526315789474,-10)(544.526315789474,-12)(544.526315789474,-14)(544.526315789474,-16)(544.526315789474,-18)(622.315789473684,0)(622.315789473684,-2)(622.315789473684,-4)(622.315789473684,-6)(622.315789473684,-8)(622.315789473684,-10)(622.315789473684,-12)(622.315789473684,-14)(622.315789473684,-16)(622.315789473684,-18)(700.105263157895,0)(700.105263157895,-2)(700.105263157895,-4)(700.105263157895,-6)(700.105263157895,-8)(700.105263157895,-10)(700.105263157895,-12)(700.105263157895,-14)(700.105263157895,-16)(700.105263157895,-18)(777.894736842105,0)(777.894736842105,-2)(777.894736842105,-4)(777.894736842105,-6)(777.894736842105,-8)(777.894736842105,-10)(777.894736842105,-12)(777.894736842105,-14)(777.894736842105,-16)(777.894736842105,-18)(855.684210526316,0)(855.684210526316,-2)(855.684210526316,-4)(855.684210526316,-6)(855.684210526316,-8)(855.684210526316,-10)(855.684210526316,-12)(855.684210526316,-14)(855.684210526316,-16)(855.684210526316,-18)(933.473684210526,0)(933.473684210526,-2)(933.473684210526,-4)(933.473684210526,-6)(933.473684210526,-8)(933.473684210526,-10)(933.473684210526,-12)(933.473684210526,-14)(933.473684210526,-16)(933.473684210526,-18)(1011.26315789474,0)(1011.26315789474,-2)(1011.26315789474,-4)(1011.26315789474,-6)(1011.26315789474,-8)(1011.26315789474,-10)(1011.26315789474,-12)(1011.26315789474,-14)(1011.26315789474,-16)(1011.26315789474,-18)(1089.05263157895,0)(1089.05263157895,-2)(1089.05263157895,-4)(1089.05263157895,-6)(1089.05263157895,-8)(1089.05263157895,-10)(1089.05263157895,-12)(1089.05263157895,-14)(1089.05263157895,-16)(1089.05263157895,-18)(1166.84210526316,0)(1166.84210526316,-2)(1166.84210526316,-4)(1166.84210526316,-6)(1166.84210526316,-8)(1166.84210526316,-10)(1166.84210526316,-12)(1166.84210526316,-14)(1166.84210526316,-16)(1166.84210526316,-18)(1244.63157894737,0)(1244.63157894737,-2)(1244.63157894737,-4)(1244.63157894737,-6)(1244.63157894737,-8)(1244.63157894737,-10)(1244.63157894737,-12)(1244.63157894737,-14)(1244.63157894737,-16)(1244.63157894737,-18)(1322.42105263158,0)(1322.42105263158,-2)(1322.42105263158,-4)(1322.42105263158,-6)(1322.42105263158,-8)(1322.42105263158,-10)(1322.42105263158,-12)(1322.42105263158,-14)(1322.42105263158,-16)(1322.42105263158,-18)(1400.21052631579,0)(1400.21052631579,-2)(1400.21052631579,-4)(1400.21052631579,-6)(1400.21052631579,-8)(1400.21052631579,-10)(1400.21052631579,-12)(1400.21052631579,-14)(1400.21052631579,-16)(1400.21052631579,-18)(1478,0)(1478,-2)(1478,-4)(1478,-6)(1478,-8)(1478,-10)(1478,-12)(1478,-14)(1478,-16)(1478,-18)(1400.21052631579,0)(1400.21052631579,-2)(1400.21052631579,-4)(1400.21052631579,-6)(1400.21052631579,-8)(1400.21052631579,-10)(1400.21052631579,-12)(1400.21052631579,-14)(1400.21052631579,-16)(1400.21052631579,-18)(1322.42105263158,0)(1322.42105263158,-2)(1322.42105263158,-4)(1322.42105263158,-6)(1322.42105263158,-8)(1322.42105263158,-10)(1322.42105263158,-12)(1322.42105263158,-14)(1322.42105263158,-16)(1322.42105263158,-18)(1244.63157894737,0)(1244.63157894737,-2)(1244.63157894737,-4)(1244.63157894737,-6)(1244.63157894737,-8)(1244.63157894737,-10)(1244.63157894737,-12)(1244.63157894737,-14)(1244.63157894737,-16)(1244.63157894737,-18)(1166.84210526316,0)(1166.84210526316,-2)(1166.84210526316,-4)(1166.84210526316,-6)(1166.84210526316,-8)(1166.84210526316,-10)(1166.84210526316,-12)(1166.84210526316,-14)(1166.84210526316,-16)(1166.84210526316,-18)(1089.05263157895,0)(1089.05263157895,-2)(1089.05263157895,-4)(1089.05263157895,-6)(1089.05263157895,-8)(1089.05263157895,-10)(1089.05263157895,-12)(1089.05263157895,-14)(1089.05263157895,-16)(1089.05263157895,-18)(1011.26315789474,0)(1011.26315789474,-2)(1011.26315789474,-4)(1011.26315789474,-6)(1011.26315789474,-8)(1011.26315789474,-10)(1011.26315789474,-12)(1011.26315789474,-14)(1011.26315789474,-16)(1011.26315789474,-18)(933.473684210526,0)(933.473684210526,-2)(933.473684210526,-4)(933.473684210526,-6)(933.473684210526,-8)(933.473684210526,-10)(933.473684210526,-12)(933.473684210526,-14)(933.473684210526,-16)(933.473684210526,-18)(855.684210526316,0)(855.684210526316,-2)(855.684210526316,-4)(855.684210526316,-6)(855.684210526316,-8)(855.684210526316,-10)(855.684210526316,-12)(855.684210526316,-14)(855.684210526316,-16)(855.684210526316,-18)(777.894736842105,0)(777.894736842105,-2)(777.894736842105,-4)(777.894736842105,-6)(777.894736842105,-8)(777.894736842105,-10)(777.894736842105,-12)(777.894736842105,-14)(777.894736842105,-16)(777.894736842105,-18)(700.105263157895,0)(700.105263157895,-2)(700.105263157895,-4)(700.105263157895,-6)(700.105263157895,-8)(700.105263157895,-10)(700.105263157895,-12)(700.105263157895,-14)(700.105263157895,-16)(700.105263157895,-18)(622.315789473684,0)(622.315789473684,-2)(622.315789473684,-4)(622.315789473684,-6)(622.315789473684,-8)(622.315789473684,-10)(622.315789473684,-12)(622.315789473684,-14)(622.315789473684,-16)(622.315789473684,-18)(544.526315789474,0)(544.526315789474,-2)(544.526315789474,-4)(544.526315789474,-6)(544.526315789474,-8)(544.526315789474,-10)(544.526315789474,-12)(544.526315789474,-14)(544.526315789474,-16)(544.526315789474,-18)(466.736842105263,0)(466.736842105263,-2)(466.736842105263,-4)(466.736842105263,-6)(466.736842105263,-8)(466.736842105263,-10)(466.736842105263,-12)(466.736842105263,-14)(466.736842105263,-16)(466.736842105263,-18)(388.947368421053,0)(388.947368421053,-2)(388.947368421053,-4)(388.947368421053,-6)(388.947368421053,-8)(388.947368421053,-10)(388.947368421053,-12)(388.947368421053,-14)(388.947368421053,-16)(388.947368421053,-18)(311.157894736842,0)(311.157894736842,-2)(311.157894736842,-4)(311.157894736842,-6)(311.157894736842,-8)(311.157894736842,-10)(311.157894736842,-12)(311.157894736842,-14)(311.157894736842,-16)(311.157894736842,-18)(233.368421052632,0)(233.368421052632,-2)(233.368421052632,-4)(233.368421052632,-6)(233.368421052632,-8)(233.368421052632,-10)(233.368421052632,-12)(233.368421052632,-14)(233.368421052632,-16)(233.368421052632,-18)(155.578947368421,0)(155.578947368421,-2)(155.578947368421,-4)(155.578947368421,-6)(155.578947368421,-8)(155.578947368421,-10)(155.578947368421,-12)(155.578947368421,-14)(155.578947368421,-16)(155.578947368421,-18)(77.7894736842105,0)(77.7894736842105,-2)(77.7894736842105,-4)(77.7894736842105,-6)(77.7894736842105,-8)(77.7894736842105,-10)(77.7894736842105,-12)(77.7894736842105,-14)(77.7894736842105,-16)(77.7894736842105,-18) 
};
\end{axis}
\end{tikzpicture}%

\begin{figure}[tb]
  \centering
  \adjincludegraphics[width=\linewidth,clip=true,trim=\trimlen{} \trimlen{} \trimlen{} \trimlen{}]{figures/ch03/graph}
  \caption{An example $10\times 20$ path planning graph. Note that each node
           is only connected to the nodes of the next column that can be
           reached according to the physical constraints of the application
           (e.g. edges of top left node in red).}
  \label{fig:graph}
\end{figure}

\begin{algorithm}[tb]
  \caption{Graph-based path planning}
  \label{alg:ppgraph}
\small{
\begin{algorithmic}[1]
  \REQUIRE graph $\mathcal{G}$, GP prior ($\mu_0 = 0$, $k$, $\sigma_0$),\\
           \hspace{1.35em}measurements per edge $n_e$, lookahead $k$,\\
           \hspace{1.35em}threshold value $h$, accuracy parameter $\epsilon$
  \ENSURE predicted sets $\hat{H}$, $\hat{L}$
  \LET{$D$}{$\bigcup_{(i,j)\in\mathcal{E}}M_{ij}$}
  \STATE{}\COMMENT{Initialize sets and confidence regions as in \acl}
  \LET{$Q$}{( )}
  \LET{$s$}{$0$}
  \LET{$t$}{$1$}
  \WHILE{$U_{t-1} \neq \varnothing$}
    \STATE{}\COMMENT{Classify as in \acl}
    \LET{$P$}{$\argmax_{P\in\mathcal{P}_{k,s}} I^u(P;f\mid Q)$} \label{lin:maxmi}
    \STATE\COMMENT{Obtain measurements at points $M_{sP[1]}$}
    \LET{$s$}{$P[1]$}
    \LET{$Q$}{$(Q,s)$}
    \STATE{}\COMMENT{GP inference as in \acl}
    \LET{$t$}{$t+1$}
  \ENDWHILE
\end{algorithmic}
}
\end{algorithm}

We now come to the issue of computing an appropriate path through the graph.
Intuitively, we would like at each step to find a path of length $k$ that
starts from the current node and maximizes some measure of informativeness
with respect to level set estimation. There is no straightforward way to
extend the notion of ambiguity of \acl to a set of points, but we can instead
use the conditional information gain of yet unclassified points as the quantity
to be maximized.

More formally, let $\mathcal{G} = (\mathcal{V}, \mathcal{E})$ be the graph
and $\mathcal{P}_{k,i}$ the set of all paths of length
$k$ starting from node $i$, where we defined a path $P$ as a vector of nodes
and $P[n]$ is the $n$-th node in the path. Also, let $M_{ij}$ be the set of
$n_e$ points that are sampled when edge $(i, j)$ is traversed and $M_P$ be
the set of all points that are sampled when path $P$ is traversed.
Then, we define
the input space in this setting as the set of all points that may be sampled
along any edge of the graph, i.e.
\begin{align*}
D = \bigcup_{(i, j)\in \mathcal{E}} M_{ij}.
\end{align*}
We also define the conditional information gain about $f$ of a path $P$ given
the already traversed path $Q$ as the conditional information gain of all
points that are sampled along path $P$ given all points that have already
been sampled along $Q$
\begin{align*}
I(P;f\mid Q) = I(M_P; f\mid M_Q)
\end{align*}
Finally, we define the conditional information gain of yet unclassified points
as follows
\begin{align*}
I^u(P;f\mid Q) = I(M^u_P; f\mid M_Q),
\end{align*}
where $M^u_P = \{\*x\in M_P\mid \*x \notin U_t\}$ and $U_t$ denotes as in \acl
the subset of $D$ containing all yet unclassified points at iteration $t$.

\algoref{alg:ppgraph} shows a high-level overview of the path planning
procedure. Note that at each iteration we plan for $k$ steps ahead, but
only advance one step, i.e. traverse one edge of the graph, and then
replan after reclassifying using the newly obtained data.

The complexity of the algorithm is concentrated on the search of the path that
maximizes the conditional information gain (\lineref{lin:maxmi}).
In particular, the information gain is a submodular function and maximizing it
is an NP-hard problem~\cite{krause08}, therefore we have to resort to
approximate solutions in practice. The recursive-greedy~\cite{chekuri05}
algorithm provides near-optimal guarantees and has been used in practice
for offline graph-based path planning~\cite{singh09}. However, in our online
setting, where we have to reoptimize at each iteration in real-time, the
quasi-polynomial complexity of recursive-greedy is still prohibitively slow.
Instead, we use an even coarser
approximation by applying a Viterbi-style~\cite{forney73} dynamic programming
algorithm with a complexity of $\mathcal{O}(k n_v^2)$ per path planning
iteration, where $n_v$ is the vertical grid resolution of our graph.

%\setlength\figureheight{1.2in}\setlength\figurewidth{5.2in}
%% This file was created by matlab2tikz v0.2.3.
% Copyright (c) 2008--2012, Nico Schlömer <nico.schloemer@gmail.com>
% All rights reserved.
% 
% 
% 

\begin{tikzpicture}

\begin{axis}[%
tick label style={font=\tiny},
label style={font=\tiny},
xlabel shift={-10pt},
ylabel shift={-17pt},
legend style={font=\tiny},
view={0}{90},
width=\figurewidth,
height=\figureheight,
scale only axis,
xmin=0, xmax=1478,
xtick={0, 400, 1000, 1400},
xlabel={Length (m)},
ymin=-18, ymax=0,
ytick={0, -4, -14, -18},
ylabel={Depth (m)},
name=plot1,
axis lines*=box,
axis line style={draw=none},
tickwidth=0.0cm,
clip=false
]

\addplot [fill=darkgray,draw=black,forget plot] coordinates{ (0,0)(4.94314381270903,0)(9.88628762541806,0)(14.8294314381271,0)(19.7725752508361,0)(24.7157190635452,0)(29.6588628762542,0)(34.6020066889632,0)(39.5451505016722,0)(44.4882943143813,0)(49.4314381270903,0)(54.3745819397993,0)(59.3177257525084,0)(64.2608695652174,0)(69.2040133779264,0)(74.1471571906355,0)(79.0903010033445,0)(84.0334448160535,0)(88.9765886287625,0)(93.9197324414716,0)(98.8628762541806,0)(103.80602006689,0)(108.749163879599,0)(113.692307692308,0)(118.635451505017,0)(123.578595317726,0)(128.521739130435,0)(133.464882943144,0)(138.408026755853,0)(143.351170568562,0)(148.294314381271,0)(153.23745819398,0)(158.180602006689,0)(163.123745819398,0)(168.066889632107,0)(173.010033444816,0)(177.953177257525,0)(182.896321070234,0)(187.839464882943,0)(192.782608695652,0)(197.725752508361,0)(202.66889632107,0)(207.612040133779,0)(212.555183946488,0)(217.498327759197,0)(222.441471571906,0)(227.384615384615,0)(232.327759197324,0)(237.270903010033,0)(242.214046822742,0)(247.157190635452,0)(252.100334448161,0)(257.04347826087,0)(261.986622073579,0)(266.929765886288,0)(271.872909698997,0)(276.816053511706,0)(281.759197324415,0)(286.702341137124,0)(291.645484949833,0)(296.588628762542,0)(301.531772575251,0)(306.47491638796,0)(311.418060200669,0)(316.361204013378,0)(321.304347826087,0)(326.247491638796,0)(331.190635451505,0)(336.133779264214,0)(341.076923076923,0)(346.020066889632,0)(350.963210702341,0)(355.90635451505,0)(360.849498327759,0)(365.792642140468,0)(370.735785953177,0)(375.678929765886,0)(380.622073578595,0)(385.565217391304,0)(390.508361204013,0)(395.451505016722,0)(400.394648829431,0)(405.33779264214,0)(410.28093645485,0)(415.224080267559,0)(420.167224080268,0)(425.110367892977,0)(430.053511705686,0)(434.996655518395,0)(439.939799331104,0)(444.882943143813,0)(449.826086956522,0)(454.769230769231,0)(459.71237458194,0)(464.655518394649,0)(469.598662207358,0)(474.541806020067,0)(479.484949832776,0)(484.428093645485,0)(489.371237458194,0)(494.314381270903,0)(499.257525083612,0)(504.200668896321,0)(509.14381270903,0)(514.086956521739,0)(519.030100334448,0)(523.973244147157,0)(528.916387959866,0)(533.859531772575,0)(538.802675585284,0)(543.745819397993,0)(548.688963210702,0)(553.632107023411,0)(558.57525083612,0)(563.518394648829,0)(568.461538461538,0)(573.404682274248,0)(578.347826086957,0)(583.290969899666,0)(588.234113712375,0)(593.177257525084,0)(598.120401337793,0)(603.063545150502,0)(608.006688963211,0)(612.94983277592,0)(617.892976588629,0)(622.836120401338,0)(627.779264214047,0)(632.722408026756,0)(637.665551839465,0)(642.608695652174,0)(647.551839464883,0)(652.494983277592,0)(657.438127090301,0)(662.38127090301,0)(667.324414715719,0)(672.267558528428,0)(677.210702341137,0)(682.153846153846,0)(687.096989966555,0)(692.040133779264,0)(696.983277591973,0)(701.926421404682,0)(706.869565217391,0)(711.8127090301,0)(716.755852842809,0)(721.698996655518,0)(726.642140468227,0)(731.585284280936,0)(736.528428093646,0)(741.471571906354,0)(746.414715719064,0)(751.357859531773,0)(756.301003344482,0)(761.244147157191,0)(766.1872909699,0)(771.130434782609,0)(776.073578595318,0)(781.016722408027,0)(785.959866220736,0)(790.903010033445,0)(795.846153846154,0)(800.789297658863,0)(805.732441471572,0)(810.675585284281,0)(815.61872909699,0)(820.561872909699,0)(825.505016722408,0)(830.448160535117,0)(835.391304347826,0)(840.334448160535,0)(845.277591973244,0)(850.220735785953,0)(855.163879598662,0)(860.107023411371,0)(865.05016722408,0)(869.993311036789,0)(874.936454849498,0)(879.879598662207,0)(884.822742474916,0)(889.765886287625,0)(894.709030100334,0)(899.652173913044,0)(904.595317725752,0)(909.538461538462,0)(914.481605351171,0)(919.42474916388,0)(924.367892976589,0)(929.311036789298,0)(934.254180602007,0)(939.197324414716,0)(944.140468227425,0)(949.083612040134,0)(954.026755852843,0)(958.969899665552,0)(963.913043478261,0)(968.85618729097,0)(973.799331103679,0)(978.742474916388,0)(983.685618729097,0)(988.628762541806,0)(993.571906354515,0)(998.515050167224,0)(1003.45819397993,0)(1008.40133779264,0)(1013.34448160535,0)(1018.28762541806,0)(1023.23076923077,0)(1028.17391304348,0)(1033.11705685619,0)(1038.0602006689,0)(1043.00334448161,0)(1047.94648829431,0)(1052.88963210702,0)(1057.83277591973,0)(1062.77591973244,0)(1067.71906354515,0)(1072.66220735786,0)(1077.60535117057,0)(1082.54849498328,0)(1087.49163879599,0)(1092.4347826087,0)(1097.3779264214,0)(1102.32107023411,0)(1107.26421404682,0)(1112.20735785953,0)(1117.15050167224,0)(1122.09364548495,0)(1127.03678929766,0)(1131.97993311037,0)(1136.92307692308,0)(1141.86622073579,0)(1146.8093645485,0)(1151.7525083612,0)(1156.69565217391,0)(1161.63879598662,0)(1166.58193979933,0)(1171.52508361204,0)(1176.46822742475,0)(1181.41137123746,0)(1186.35451505017,0)(1191.29765886288,0)(1196.24080267559,0)(1201.18394648829,0)(1206.127090301,0)(1211.07023411371,0)(1216.01337792642,0)(1220.95652173913,0)(1225.89966555184,0)(1230.84280936455,0)(1235.78595317726,0)(1240.72909698997,0)(1245.67224080268,0)(1250.61538461538,0)(1255.55852842809,0)(1260.5016722408,0)(1265.44481605351,0)(1270.38795986622,0)(1275.33110367893,0)(1280.27424749164,0)(1285.21739130435,0)(1290.16053511706,0)(1295.10367892977,0)(1300.04682274247,0)(1304.98996655518,0)(1309.93311036789,0)(1314.8762541806,0)(1319.81939799331,0)(1324.76254180602,0)(1329.70568561873,0)(1334.64882943144,0)(1339.59197324415,0)(1344.53511705686,0)(1349.47826086957,0)(1354.42140468227,0)(1359.36454849498,0)(1364.30769230769,0)(1369.2508361204,0)(1374.19397993311,0)(1379.13712374582,0)(1384.08026755853,0)(1389.02341137124,0)(1393.96655518395,0)(1398.90969899666,0)(1403.85284280936,0)(1408.79598662207,0)(1413.73913043478,0)(1418.68227424749,0)(1423.6254180602,0)(1428.56856187291,0)(1433.51170568562,0)(1438.45484949833,0)(1443.39799331104,0)(1448.34113712375,0)(1453.28428093645,0)(1458.22742474916,0)(1463.17056856187,0)(1468.11371237458,0)(1473.05685618729,0)(1478,0)(1478,-0.0602006688963215)(1478,-0.120401337792643)(1478,-0.180602006688964)(1478,-0.240802675585286)(1478,-0.301003344481604)(1478,-0.361204013377925)(1478,-0.421404682274247)(1478,-0.481605351170568)(1478,-0.54180602006689)(1478,-0.602006688963211)(1478,-0.662207357859533)(1478,-0.722408026755854)(1478,-0.782608695652176)(1478,-0.842809364548494)(1478,-0.903010033444815)(1478,-0.963210702341136)(1478,-1.02341137123746)(1478,-1.08361204013378)(1478,-1.1438127090301)(1478,-1.20401337792642)(1478,-1.26421404682274)(1478,-1.32441471571906)(1478,-1.38461538461538)(1478,-1.4448160535117)(1478,-1.50501672240803)(1478,-1.56521739130435)(1478,-1.62541806020067)(1478,-1.68561872909699)(1478,-1.74581939799331)(1478,-1.80602006688963)(1478,-1.86622073578595)(1478,-1.92642140468227)(1478,-1.98662207357859)(1478,-2.04682274247492)(1478,-2.10702341137124)(1478,-2.16722408026756)(1478,-2.22742474916388)(1478,-2.2876254180602)(1478,-2.34782608695652)(1478,-2.40802675585284)(1478,-2.46822742474916)(1478,-2.52842809364548)(1478,-2.58862876254181)(1478,-2.64882943143813)(1478,-2.70903010033445)(1478,-2.76923076923077)(1478,-2.82943143812709)(1478,-2.88963210702341)(1478,-2.94983277591973)(1478,-3.01003344481605)(1478,-3.07023411371237)(1478,-3.1304347826087)(1478,-3.19063545150502)(1478,-3.25083612040134)(1478,-3.31103678929766)(1478,-3.37123745819398)(1478,-3.4314381270903)(1478,-3.49163879598662)(1478,-3.55183946488294)(1478,-3.61204013377926)(1478,-3.67224080267559)(1478,-3.73244147157191)(1478,-3.79264214046823)(1478,-3.85284280936455)(1478,-3.91304347826087)(1478,-3.97324414715719)(1478,-4.03344481605351)(1478,-4.09364548494983)(1478,-4.15384615384615)(1478,-4.21404682274247)(1478,-4.2742474916388)(1478,-4.33444816053512)(1478,-4.39464882943144)(1478,-4.45484949832776)(1478,-4.51505016722408)(1478,-4.5752508361204)(1478,-4.63545150501672)(1478,-4.69565217391304)(1478,-4.75585284280936)(1478,-4.81605351170569)(1478,-4.87625418060201)(1478,-4.93645484949833)(1478,-4.99665551839465)(1478,-5.05685618729097)(1478,-5.11705685618729)(1478,-5.17725752508361)(1478,-5.23745819397993)(1478,-5.29765886287625)(1478,-5.35785953177258)(1478,-5.4180602006689)(1478,-5.47826086956522)(1478,-5.53846153846154)(1478,-5.59866220735786)(1478,-5.65886287625418)(1478,-5.7190635451505)(1478,-5.77926421404682)(1478,-5.83946488294314)(1478,-5.89966555183947)(1478,-5.95986622073579)(1478,-6.02006688963211)(1478,-6.08026755852843)(1478,-6.14046822742475)(1478,-6.20066889632107)(1478,-6.26086956521739)(1478,-6.32107023411371)(1478,-6.38127090301003)(1478,-6.44147157190636)(1478,-6.50167224080267)(1478,-6.561872909699)(1478,-6.62207357859532)(1478,-6.68227424749164)(1478,-6.74247491638796)(1478,-6.80267558528428)(1478,-6.8628762541806)(1478,-6.92307692307692)(1478,-6.98327759197324)(1478,-7.04347826086956)(1478,-7.10367892976589)(1478,-7.16387959866221)(1478,-7.22408026755853)(1478,-7.28428093645485)(1478,-7.34448160535117)(1478,-7.40468227424749)(1478,-7.46488294314381)(1478,-7.52508361204013)(1478,-7.58528428093645)(1478,-7.64548494983278)(1478,-7.7056856187291)(1478,-7.76588628762542)(1478,-7.82608695652174)(1478,-7.88628762541806)(1478,-7.94648829431438)(1478,-8.0066889632107)(1478,-8.06688963210702)(1478,-8.12709030100334)(1478,-8.18729096989967)(1478,-8.24749163879599)(1478,-8.30769230769231)(1478,-8.36789297658863)(1478,-8.42809364548495)(1478,-8.48829431438127)(1478,-8.54849498327759)(1478,-8.60869565217391)(1478,-8.66889632107023)(1478,-8.72909698996656)(1478,-8.78929765886288)(1478,-8.8494983277592)(1478,-8.90969899665552)(1478,-8.96989966555184)(1478,-9.03010033444816)(1478,-9.09030100334448)(1478,-9.1505016722408)(1478,-9.21070234113712)(1478,-9.27090301003344)(1478,-9.33110367892977)(1478,-9.39130434782609)(1478,-9.45150501672241)(1478,-9.51170568561873)(1478,-9.57190635451505)(1478,-9.63210702341137)(1478,-9.69230769230769)(1478,-9.75250836120401)(1478,-9.81270903010033)(1478,-9.87290969899666)(1478,-9.93311036789298)(1478,-9.9933110367893)(1478,-10.0535117056856)(1478,-10.1137123745819)(1478,-10.1739130434783)(1478,-10.2341137123746)(1478,-10.2943143812709)(1478,-10.3545150501672)(1478,-10.4147157190635)(1478,-10.4749163879599)(1478,-10.5351170568562)(1478,-10.5953177257525)(1478,-10.6555183946488)(1478,-10.7157190635452)(1478,-10.7759197324415)(1478,-10.8361204013378)(1478,-10.8963210702341)(1478,-10.9565217391304)(1478,-11.0167224080268)(1478,-11.0769230769231)(1478,-11.1371237458194)(1478,-11.1973244147157)(1478,-11.257525083612)(1478,-11.3177257525084)(1478,-11.3779264214047)(1478,-11.438127090301)(1478,-11.4983277591973)(1478,-11.5585284280936)(1478,-11.61872909699)(1478,-11.6789297658863)(1478,-11.7391304347826)(1478,-11.7993311036789)(1478,-11.8595317725753)(1478,-11.9197324414716)(1478,-11.9799331103679)(1478,-12.0401337792642)(1478,-12.1003344481605)(1478,-12.1605351170569)(1478,-12.2207357859532)(1478,-12.2809364548495)(1478,-12.3411371237458)(1478,-12.4013377926421)(1478,-12.4615384615385)(1478,-12.5217391304348)(1478,-12.5819397993311)(1478,-12.6421404682274)(1478,-12.7023411371237)(1478,-12.7625418060201)(1478,-12.8227424749164)(1478,-12.8829431438127)(1478,-12.943143812709)(1478,-13.0033444816054)(1478,-13.0635451505017)(1478,-13.123745819398)(1478,-13.1839464882943)(1478,-13.2441471571906)(1478,-13.304347826087)(1478,-13.3645484949833)(1478,-13.4247491638796)(1478,-13.4849498327759)(1478,-13.5451505016722)(1478,-13.6053511705686)(1478,-13.6655518394649)(1478,-13.7257525083612)(1478,-13.7859531772575)(1478,-13.8461538461538)(1478,-13.9063545150502)(1478,-13.9665551839465)(1478,-14.0267558528428)(1478,-14.0869565217391)(1478,-14.1471571906355)(1478,-14.2073578595318)(1478,-14.2675585284281)(1478,-14.3277591973244)(1478,-14.3879598662207)(1478,-14.4481605351171)(1478,-14.5083612040134)(1478,-14.5685618729097)(1478,-14.628762541806)(1478,-14.6889632107023)(1478,-14.7491638795987)(1478,-14.809364548495)(1478,-14.8695652173913)(1478,-14.9297658862876)(1478,-14.9899665551839)(1478,-15.0501672240803)(1478,-15.1103678929766)(1478,-15.1705685618729)(1478,-15.2307692307692)(1478,-15.2909698996656)(1478,-15.3511705685619)(1478,-15.4113712374582)(1478,-15.4715719063545)(1478,-15.5317725752508)(1478,-15.5919732441472)(1478,-15.6521739130435)(1478,-15.7123745819398)(1478,-15.7725752508361)(1478,-15.8327759197324)(1478,-15.8929765886288)(1478,-15.9531772575251)(1478,-16.0133779264214)(1478,-16.0735785953177)(1478,-16.133779264214)(1478,-16.1939799331104)(1478,-16.2541806020067)(1478,-16.314381270903)(1478,-16.3745819397993)(1478,-16.4347826086957)(1478,-16.494983277592)(1478,-16.5551839464883)(1478,-16.6153846153846)(1478,-16.6755852842809)(1478,-16.7357859531773)(1478,-16.7959866220736)(1478,-16.8561872909699)(1478,-16.9163879598662)(1478,-16.9765886287625)(1478,-17.0367892976589)(1478,-17.0969899665552)(1478,-17.1571906354515)(1478,-17.2173913043478)(1478,-17.2775919732441)(1478,-17.3377926421405)(1478,-17.3979933110368)(1478,-17.4581939799331)(1478,-17.5183946488294)(1478,-17.5785953177258)(1478,-17.6387959866221)(1478,-17.6989966555184)(1478,-17.7591973244147)(1478,-17.819397993311)(1478,-17.8795986622074)(1478,-17.9397993311037)(1478,-18)(1473.05685618729,-18)(1468.11371237458,-18)(1463.17056856187,-18)(1458.22742474916,-18)(1453.28428093645,-18)(1448.34113712375,-18)(1443.39799331104,-18)(1438.45484949833,-18)(1433.51170568562,-18)(1428.56856187291,-18)(1423.6254180602,-18)(1418.68227424749,-18)(1413.73913043478,-18)(1408.79598662207,-18)(1403.85284280936,-18)(1398.90969899666,-18)(1393.96655518395,-18)(1389.02341137124,-18)(1384.08026755853,-18)(1379.13712374582,-18)(1374.19397993311,-18)(1369.2508361204,-18)(1364.30769230769,-18)(1359.36454849498,-18)(1354.42140468227,-18)(1349.47826086957,-18)(1344.53511705686,-18)(1339.59197324415,-18)(1334.64882943144,-18)(1329.70568561873,-18)(1324.76254180602,-18)(1319.81939799331,-18)(1314.8762541806,-18)(1309.93311036789,-18)(1304.98996655518,-18)(1300.04682274247,-18)(1295.10367892977,-18)(1290.16053511706,-18)(1285.21739130435,-18)(1280.27424749164,-18)(1275.33110367893,-18)(1270.38795986622,-18)(1265.44481605351,-18)(1260.5016722408,-18)(1255.55852842809,-18)(1250.61538461538,-18)(1245.67224080268,-18)(1240.72909698997,-18)(1235.78595317726,-18)(1230.84280936455,-18)(1225.89966555184,-18)(1220.95652173913,-18)(1216.01337792642,-18)(1211.07023411371,-18)(1206.127090301,-18)(1201.18394648829,-18)(1196.24080267559,-18)(1191.29765886288,-18)(1186.35451505017,-18)(1181.41137123746,-18)(1176.46822742475,-18)(1171.52508361204,-18)(1166.58193979933,-18)(1161.63879598662,-18)(1156.69565217391,-18)(1151.7525083612,-18)(1146.8093645485,-18)(1141.86622073579,-18)(1136.92307692308,-18)(1131.97993311037,-18)(1127.03678929766,-18)(1122.09364548495,-18)(1117.15050167224,-18)(1112.20735785953,-18)(1107.26421404682,-18)(1102.32107023411,-18)(1097.3779264214,-18)(1092.4347826087,-18)(1087.49163879599,-18)(1082.54849498328,-18)(1077.60535117057,-18)(1072.66220735786,-18)(1067.71906354515,-18)(1062.77591973244,-18)(1057.83277591973,-18)(1052.88963210702,-18)(1047.94648829431,-18)(1043.00334448161,-18)(1038.0602006689,-18)(1033.11705685619,-18)(1028.17391304348,-18)(1023.23076923077,-18)(1018.28762541806,-18)(1013.34448160535,-18)(1008.40133779264,-18)(1003.45819397993,-18)(998.515050167224,-18)(993.571906354515,-18)(988.628762541806,-18)(983.685618729097,-18)(978.742474916388,-18)(973.799331103679,-18)(968.85618729097,-18)(963.913043478261,-18)(958.969899665552,-18)(954.026755852843,-18)(949.083612040134,-18)(944.140468227425,-18)(939.197324414716,-18)(934.254180602007,-18)(929.311036789298,-18)(924.367892976589,-18)(919.42474916388,-18)(914.481605351171,-18)(909.538461538462,-18)(904.595317725752,-18)(899.652173913044,-18)(894.709030100334,-18)(889.765886287625,-18)(884.822742474916,-18)(879.879598662207,-18)(874.936454849498,-18)(869.993311036789,-18)(865.05016722408,-18)(860.107023411371,-18)(855.163879598662,-18)(850.220735785953,-18)(845.277591973244,-18)(840.334448160535,-18)(835.391304347826,-18)(830.448160535117,-18)(825.505016722408,-18)(820.561872909699,-18)(815.61872909699,-18)(810.675585284281,-18)(805.732441471572,-18)(800.789297658863,-18)(795.846153846154,-18)(790.903010033445,-18)(785.959866220736,-18)(781.016722408027,-18)(776.073578595318,-18)(771.130434782609,-18)(766.1872909699,-18)(761.244147157191,-18)(756.301003344482,-18)(751.357859531773,-18)(746.414715719064,-18)(741.471571906354,-18)(736.528428093646,-18)(731.585284280936,-18)(726.642140468227,-18)(721.698996655518,-18)(716.755852842809,-18)(711.8127090301,-18)(706.869565217391,-18)(701.926421404682,-18)(696.983277591973,-18)(692.040133779264,-18)(687.096989966555,-18)(682.153846153846,-18)(677.210702341137,-18)(672.267558528428,-18)(667.324414715719,-18)(662.38127090301,-18)(657.438127090301,-18)(652.494983277592,-18)(647.551839464883,-18)(642.608695652174,-18)(637.665551839465,-18)(632.722408026756,-18)(627.779264214047,-18)(622.836120401338,-18)(617.892976588629,-18)(612.94983277592,-18)(608.006688963211,-18)(603.063545150502,-18)(598.120401337793,-18)(593.177257525084,-18)(588.234113712375,-18)(583.290969899666,-18)(578.347826086957,-18)(573.404682274248,-18)(568.461538461538,-18)(563.518394648829,-18)(558.57525083612,-18)(553.632107023411,-18)(548.688963210702,-18)(543.745819397993,-18)(538.802675585284,-18)(533.859531772575,-18)(528.916387959866,-18)(523.973244147157,-18)(519.030100334448,-18)(514.086956521739,-18)(509.14381270903,-18)(504.200668896321,-18)(499.257525083612,-18)(494.314381270903,-18)(489.371237458194,-18)(484.428093645485,-18)(479.484949832776,-18)(474.541806020067,-18)(469.598662207358,-18)(464.655518394649,-18)(459.71237458194,-18)(454.769230769231,-18)(449.826086956522,-18)(444.882943143813,-18)(439.939799331104,-18)(434.996655518395,-18)(430.053511705686,-18)(425.110367892977,-18)(420.167224080268,-18)(415.224080267559,-18)(410.28093645485,-18)(405.33779264214,-18)(400.394648829431,-18)(395.451505016722,-18)(390.508361204013,-18)(385.565217391304,-18)(380.622073578595,-18)(375.678929765886,-18)(370.735785953177,-18)(365.792642140468,-18)(360.849498327759,-18)(355.90635451505,-18)(350.963210702341,-18)(346.020066889632,-18)(341.076923076923,-18)(336.133779264214,-18)(331.190635451505,-18)(326.247491638796,-18)(321.304347826087,-18)(316.361204013378,-18)(311.418060200669,-18)(306.47491638796,-18)(301.531772575251,-18)(296.588628762542,-18)(291.645484949833,-18)(286.702341137124,-18)(281.759197324415,-18)(276.816053511706,-18)(271.872909698997,-18)(266.929765886288,-18)(261.986622073579,-18)(257.04347826087,-18)(252.100334448161,-18)(247.157190635452,-18)(242.214046822742,-18)(237.270903010033,-18)(232.327759197324,-18)(227.384615384615,-18)(222.441471571906,-18)(217.498327759197,-18)(212.555183946488,-18)(207.612040133779,-18)(202.66889632107,-18)(197.725752508361,-18)(192.782608695652,-18)(187.839464882943,-18)(182.896321070234,-18)(177.953177257525,-18)(173.010033444816,-18)(168.066889632107,-18)(163.123745819398,-18)(158.180602006689,-18)(153.23745819398,-18)(148.294314381271,-18)(143.351170568562,-18)(138.408026755853,-18)(133.464882943144,-18)(128.521739130435,-18)(123.578595317726,-18)(118.635451505017,-18)(113.692307692308,-18)(108.749163879599,-18)(103.80602006689,-18)(98.8628762541806,-18)(93.9197324414716,-18)(88.9765886287625,-18)(84.0334448160535,-18)(79.0903010033445,-18)(74.1471571906355,-18)(69.2040133779264,-18)(64.2608695652174,-18)(59.3177257525084,-18)(54.3745819397993,-18)(49.4314381270903,-18)(44.4882943143813,-18)(39.5451505016722,-18)(34.6020066889632,-18)(29.6588628762542,-18)(24.7157190635452,-18)(19.7725752508361,-18)(14.8294314381271,-18)(9.88628762541806,-18)(4.94314381270903,-18)(0,-18.0000060194649)(-0.000494264954775508,-18)(0,-17.9397993311037)(0,-17.8795986622074)(0,-17.819397993311)(0,-17.7591973244147)(0,-17.6989966555184)(0,-17.6387959866221)(0,-17.5785953177258)(0,-17.5183946488294)(0,-17.4581939799331)(0,-17.3979933110368)(0,-17.3377926421405)(0,-17.2775919732441)(0,-17.2173913043478)(0,-17.1571906354515)(0,-17.0969899665552)(0,-17.0367892976589)(0,-16.9765886287625)(0,-16.9163879598662)(0,-16.8561872909699)(0,-16.7959866220736)(0,-16.7357859531773)(0,-16.6755852842809)(0,-16.6153846153846)(0,-16.5551839464883)(0,-16.494983277592)(0,-16.4347826086957)(0,-16.3745819397993)(0,-16.314381270903)(0,-16.2541806020067)(0,-16.1939799331104)(0,-16.133779264214)(0,-16.0735785953177)(0,-16.0133779264214)(0,-15.9531772575251)(0,-15.8929765886288)(0,-15.8327759197324)(0,-15.7725752508361)(0,-15.7123745819398)(0,-15.6521739130435)(0,-15.5919732441472)(0,-15.5317725752508)(0,-15.4715719063545)(0,-15.4113712374582)(0,-15.3511705685619)(0,-15.2909698996656)(0,-15.2307692307692)(0,-15.1705685618729)(0,-15.1103678929766)(0,-15.0501672240803)(0,-14.9899665551839)(0,-14.9297658862876)(0,-14.8695652173913)(0,-14.809364548495)(0,-14.7491638795987)(0,-14.6889632107023)(0,-14.628762541806)(0,-14.5685618729097)(0,-14.5083612040134)(0,-14.4481605351171)(0,-14.3879598662207)(0,-14.3277591973244)(0,-14.2675585284281)(0,-14.2073578595318)(0,-14.1471571906355)(0,-14.0869565217391)(0,-14.0267558528428)(0,-13.9665551839465)(0,-13.9063545150502)(0,-13.8461538461538)(0,-13.7859531772575)(0,-13.7257525083612)(0,-13.6655518394649)(0,-13.6053511705686)(0,-13.5451505016722)(0,-13.4849498327759)(0,-13.4247491638796)(0,-13.3645484949833)(0,-13.304347826087)(0,-13.2441471571906)(0,-13.1839464882943)(0,-13.123745819398)(0,-13.0635451505017)(0,-13.0033444816054)(0,-12.943143812709)(0,-12.8829431438127)(0,-12.8227424749164)(0,-12.7625418060201)(0,-12.7023411371237)(0,-12.6421404682274)(0,-12.5819397993311)(0,-12.5217391304348)(0,-12.4615384615385)(0,-12.4013377926421)(0,-12.3411371237458)(0,-12.2809364548495)(0,-12.2207357859532)(0,-12.1605351170569)(0,-12.1003344481605)(0,-12.0401337792642)(0,-11.9799331103679)(0,-11.9197324414716)(0,-11.8595317725753)(0,-11.7993311036789)(0,-11.7391304347826)(0,-11.6789297658863)(0,-11.61872909699)(0,-11.5585284280936)(0,-11.4983277591973)(0,-11.438127090301)(0,-11.3779264214047)(0,-11.3177257525084)(0,-11.257525083612)(0,-11.1973244147157)(0,-11.1371237458194)(0,-11.0769230769231)(0,-11.0167224080268)(0,-10.9565217391304)(0,-10.8963210702341)(0,-10.8361204013378)(0,-10.7759197324415)(0,-10.7157190635452)(0,-10.6555183946488)(0,-10.5953177257525)(0,-10.5351170568562)(0,-10.4749163879599)(0,-10.4147157190635)(0,-10.3545150501672)(0,-10.2943143812709)(0,-10.2341137123746)(0,-10.1739130434783)(0,-10.1137123745819)(0,-10.0535117056856)(0,-9.9933110367893)(0,-9.93311036789298)(0,-9.87290969899666)(0,-9.81270903010033)(0,-9.75250836120401)(0,-9.69230769230769)(0,-9.63210702341137)(0,-9.57190635451505)(0,-9.51170568561873)(0,-9.45150501672241)(0,-9.39130434782609)(0,-9.33110367892977)(0,-9.27090301003344)(0,-9.21070234113712)(0,-9.1505016722408)(0,-9.09030100334448)(0,-9.03010033444816)(0,-8.96989966555184)(0,-8.90969899665552)(0,-8.8494983277592)(0,-8.78929765886288)(0,-8.72909698996656)(0,-8.66889632107023)(0,-8.60869565217391)(0,-8.54849498327759)(0,-8.48829431438127)(0,-8.42809364548495)(0,-8.36789297658863)(0,-8.30769230769231)(0,-8.24749163879599)(0,-8.18729096989967)(0,-8.12709030100334)(0,-8.06688963210702)(0,-8.0066889632107)(0,-7.94648829431438)(0,-7.88628762541806)(0,-7.82608695652174)(0,-7.76588628762542)(0,-7.7056856187291)(0,-7.64548494983278)(0,-7.58528428093645)(0,-7.52508361204013)(0,-7.46488294314381)(0,-7.40468227424749)(0,-7.34448160535117)(0,-7.28428093645485)(0,-7.22408026755853)(0,-7.16387959866221)(0,-7.10367892976589)(0,-7.04347826086956)(0,-6.98327759197324)(0,-6.92307692307692)(0,-6.8628762541806)(0,-6.80267558528428)(0,-6.74247491638796)(0,-6.68227424749164)(0,-6.62207357859532)(0,-6.561872909699)(0,-6.50167224080267)(0,-6.44147157190636)(0,-6.38127090301003)(0,-6.32107023411371)(0,-6.26086956521739)(0,-6.20066889632107)(0,-6.14046822742475)(0,-6.08026755852843)(0,-6.02006688963211)(0,-5.95986622073579)(0,-5.89966555183947)(0,-5.83946488294314)(0,-5.77926421404682)(0,-5.7190635451505)(0,-5.65886287625418)(0,-5.59866220735786)(0,-5.53846153846154)(0,-5.47826086956522)(0,-5.4180602006689)(0,-5.35785953177258)(0,-5.29765886287625)(0,-5.23745819397993)(0,-5.17725752508361)(0,-5.11705685618729)(0,-5.05685618729097)(0,-4.99665551839465)(0,-4.93645484949833)(0,-4.87625418060201)(0,-4.81605351170569)(0,-4.75585284280936)(0,-4.69565217391304)(0,-4.63545150501672)(0,-4.5752508361204)(0,-4.51505016722408)(0,-4.45484949832776)(0,-4.39464882943144)(0,-4.33444816053512)(0,-4.2742474916388)(0,-4.21404682274247)(0,-4.15384615384615)(0,-4.09364548494983)(0,-4.03344481605351)(0,-3.97324414715719)(0,-3.91304347826087)(0,-3.85284280936455)(0,-3.79264214046823)(0,-3.73244147157191)(0,-3.67224080267559)(0,-3.61204013377926)(0,-3.55183946488294)(0,-3.49163879598662)(0,-3.4314381270903)(0,-3.37123745819398)(0,-3.31103678929766)(0,-3.25083612040134)(0,-3.19063545150502)(0,-3.1304347826087)(0,-3.07023411371237)(0,-3.01003344481605)(0,-2.94983277591973)(0,-2.88963210702341)(0,-2.82943143812709)(0,-2.76923076923077)(0,-2.70903010033445)(0,-2.64882943143813)(0,-2.58862876254181)(0,-2.52842809364548)(0,-2.46822742474916)(0,-2.40802675585284)(0,-2.34782608695652)(0,-2.2876254180602)(0,-2.22742474916388)(0,-2.16722408026756)(0,-2.10702341137124)(0,-2.04682274247492)(0,-1.98662207357859)(0,-1.92642140468227)(0,-1.86622073578595)(0,-1.80602006688963)(0,-1.74581939799331)(0,-1.68561872909699)(0,-1.62541806020067)(0,-1.56521739130435)(0,-1.50501672240803)(0,-1.4448160535117)(0,-1.38461538461538)(0,-1.32441471571906)(0,-1.26421404682274)(0,-1.20401337792642)(0,-1.1438127090301)(0,-1.08361204013378)(0,-1.02341137123746)(0,-0.963210702341136)(0,-0.903010033444815)(0,-0.842809364548494)(0,-0.782608695652176)(0,-0.722408026755854)(0,-0.662207357859533)(0,-0.602006688963211)(0,-0.54180602006689)(0,-0.481605351170568)(0,-0.421404682274247)(0,-0.361204013377925)(0,-0.301003344481604)(0,-0.240802675585286)(0,-0.180602006688964)(0,-0.120401337792643)(0,-0.0602006688963215)(0,0)};

\addplot [
color=red,
solid,
line width=1.0pt,
forget plot
]
coordinates{
 (0,0)(105.571428571429,0)(211.142857142857,-10.4210526315789)(316.714285714286,-18)(422.285714285714,-10.4210526315789)(527.857142857143,-3.78947368421053)(633.428571428571,0)(739,-9.47368421052632)(844.571428571429,-18)(950.142857142857,-8.52631578947369)(1055.71428571429,0) 
};

\addplot [
mark size=0.8pt,
only marks,
mark=*,
mark options={solid,fill=black,draw=black},
forget plot
]
coordinates{
 (0,0)(0,-0.947368421052632)(0,-1.89473684210526)(0,-2.84210526315789)(0,-3.78947368421053)(0,-4.73684210526316)(0,-5.68421052631579)(0,-6.63157894736842)(0,-7.57894736842105)(0,-8.52631578947369)(0,-9.47368421052632)(0,-10.4210526315789)(0,-11.3684210526316)(0,-12.3157894736842)(0,-13.2631578947368)(0,-14.2105263157895)(0,-15.1578947368421)(0,-16.1052631578947)(0,-17.0526315789474)(0,-18)(105.571428571429,0)(105.571428571429,-0.947368421052632)(105.571428571429,-1.89473684210526)(105.571428571429,-2.84210526315789)(105.571428571429,-3.78947368421053)(105.571428571429,-4.73684210526316)(105.571428571429,-5.68421052631579)(105.571428571429,-6.63157894736842)(105.571428571429,-7.57894736842105)(105.571428571429,-8.52631578947369)(105.571428571429,-9.47368421052632)(105.571428571429,-10.4210526315789)(105.571428571429,-11.3684210526316)(105.571428571429,-12.3157894736842)(105.571428571429,-13.2631578947368)(105.571428571429,-14.2105263157895)(105.571428571429,-15.1578947368421)(105.571428571429,-16.1052631578947)(105.571428571429,-17.0526315789474)(105.571428571429,-18)(211.142857142857,0)(211.142857142857,-0.947368421052632)(211.142857142857,-1.89473684210526)(211.142857142857,-2.84210526315789)(211.142857142857,-3.78947368421053)(211.142857142857,-4.73684210526316)(211.142857142857,-5.68421052631579)(211.142857142857,-6.63157894736842)(211.142857142857,-7.57894736842105)(211.142857142857,-8.52631578947369)(211.142857142857,-9.47368421052632)(211.142857142857,-10.4210526315789)(211.142857142857,-11.3684210526316)(211.142857142857,-12.3157894736842)(211.142857142857,-13.2631578947368)(211.142857142857,-14.2105263157895)(211.142857142857,-15.1578947368421)(211.142857142857,-16.1052631578947)(211.142857142857,-17.0526315789474)(211.142857142857,-18)(316.714285714286,0)(316.714285714286,-0.947368421052632)(316.714285714286,-1.89473684210526)(316.714285714286,-2.84210526315789)(316.714285714286,-3.78947368421053)(316.714285714286,-4.73684210526316)(316.714285714286,-5.68421052631579)(316.714285714286,-6.63157894736842)(316.714285714286,-7.57894736842105)(316.714285714286,-8.52631578947369)(316.714285714286,-9.47368421052632)(316.714285714286,-10.4210526315789)(316.714285714286,-11.3684210526316)(316.714285714286,-12.3157894736842)(316.714285714286,-13.2631578947368)(316.714285714286,-14.2105263157895)(316.714285714286,-15.1578947368421)(316.714285714286,-16.1052631578947)(316.714285714286,-17.0526315789474)(316.714285714286,-18)(422.285714285714,0)(422.285714285714,-0.947368421052632)(422.285714285714,-1.89473684210526)(422.285714285714,-2.84210526315789)(422.285714285714,-3.78947368421053)(422.285714285714,-4.73684210526316)(422.285714285714,-5.68421052631579)(422.285714285714,-6.63157894736842)(422.285714285714,-7.57894736842105)(422.285714285714,-8.52631578947369)(422.285714285714,-9.47368421052632)(422.285714285714,-10.4210526315789)(422.285714285714,-11.3684210526316)(422.285714285714,-12.3157894736842)(422.285714285714,-13.2631578947368)(422.285714285714,-14.2105263157895)(422.285714285714,-15.1578947368421)(422.285714285714,-16.1052631578947)(422.285714285714,-17.0526315789474)(422.285714285714,-18)(527.857142857143,0)(527.857142857143,-0.947368421052632)(527.857142857143,-1.89473684210526)(527.857142857143,-2.84210526315789)(527.857142857143,-3.78947368421053)(527.857142857143,-4.73684210526316)(527.857142857143,-5.68421052631579)(527.857142857143,-6.63157894736842)(527.857142857143,-7.57894736842105)(527.857142857143,-8.52631578947369)(527.857142857143,-9.47368421052632)(527.857142857143,-10.4210526315789)(527.857142857143,-11.3684210526316)(527.857142857143,-12.3157894736842)(527.857142857143,-13.2631578947368)(527.857142857143,-14.2105263157895)(527.857142857143,-15.1578947368421)(527.857142857143,-16.1052631578947)(527.857142857143,-17.0526315789474)(527.857142857143,-18)(633.428571428571,0)(633.428571428571,-0.947368421052632)(633.428571428571,-1.89473684210526)(633.428571428571,-2.84210526315789)(633.428571428571,-3.78947368421053)(633.428571428571,-4.73684210526316)(633.428571428571,-5.68421052631579)(633.428571428571,-6.63157894736842)(633.428571428571,-7.57894736842105)(633.428571428571,-8.52631578947369)(633.428571428571,-9.47368421052632)(633.428571428571,-10.4210526315789)(633.428571428571,-11.3684210526316)(633.428571428571,-12.3157894736842)(633.428571428571,-13.2631578947368)(633.428571428571,-14.2105263157895)(633.428571428571,-15.1578947368421)(633.428571428571,-16.1052631578947)(633.428571428571,-17.0526315789474)(633.428571428571,-18)(739,0)(739,-0.947368421052632)(739,-1.89473684210526)(739,-2.84210526315789)(739,-3.78947368421053)(739,-4.73684210526316)(739,-5.68421052631579)(739,-6.63157894736842)(739,-7.57894736842105)(739,-8.52631578947369)(739,-9.47368421052632)(739,-10.4210526315789)(739,-11.3684210526316)(739,-12.3157894736842)(739,-13.2631578947368)(739,-14.2105263157895)(739,-15.1578947368421)(739,-16.1052631578947)(739,-17.0526315789474)(739,-18)(844.571428571429,0)(844.571428571429,-0.947368421052632)(844.571428571429,-1.89473684210526)(844.571428571429,-2.84210526315789)(844.571428571429,-3.78947368421053)(844.571428571429,-4.73684210526316)(844.571428571429,-5.68421052631579)(844.571428571429,-6.63157894736842)(844.571428571429,-7.57894736842105)(844.571428571429,-8.52631578947369)(844.571428571429,-9.47368421052632)(844.571428571429,-10.4210526315789)(844.571428571429,-11.3684210526316)(844.571428571429,-12.3157894736842)(844.571428571429,-13.2631578947368)(844.571428571429,-14.2105263157895)(844.571428571429,-15.1578947368421)(844.571428571429,-16.1052631578947)(844.571428571429,-17.0526315789474)(844.571428571429,-18)(950.142857142857,0)(950.142857142857,-0.947368421052632)(950.142857142857,-1.89473684210526)(950.142857142857,-2.84210526315789)(950.142857142857,-3.78947368421053)(950.142857142857,-4.73684210526316)(950.142857142857,-5.68421052631579)(950.142857142857,-6.63157894736842)(950.142857142857,-7.57894736842105)(950.142857142857,-8.52631578947369)(950.142857142857,-9.47368421052632)(950.142857142857,-10.4210526315789)(950.142857142857,-11.3684210526316)(950.142857142857,-12.3157894736842)(950.142857142857,-13.2631578947368)(950.142857142857,-14.2105263157895)(950.142857142857,-15.1578947368421)(950.142857142857,-16.1052631578947)(950.142857142857,-17.0526315789474)(950.142857142857,-18)(1055.71428571429,0)(1055.71428571429,-0.947368421052632)(1055.71428571429,-1.89473684210526)(1055.71428571429,-2.84210526315789)(1055.71428571429,-3.78947368421053)(1055.71428571429,-4.73684210526316)(1055.71428571429,-5.68421052631579)(1055.71428571429,-6.63157894736842)(1055.71428571429,-7.57894736842105)(1055.71428571429,-8.52631578947369)(1055.71428571429,-9.47368421052632)(1055.71428571429,-10.4210526315789)(1055.71428571429,-11.3684210526316)(1055.71428571429,-12.3157894736842)(1055.71428571429,-13.2631578947368)(1055.71428571429,-14.2105263157895)(1055.71428571429,-15.1578947368421)(1055.71428571429,-16.1052631578947)(1055.71428571429,-17.0526315789474)(1055.71428571429,-18)(1161.28571428571,0)(1161.28571428571,-0.947368421052632)(1161.28571428571,-1.89473684210526)(1161.28571428571,-2.84210526315789)(1161.28571428571,-3.78947368421053)(1161.28571428571,-4.73684210526316)(1161.28571428571,-5.68421052631579)(1161.28571428571,-6.63157894736842)(1161.28571428571,-7.57894736842105)(1161.28571428571,-8.52631578947369)(1161.28571428571,-9.47368421052632)(1161.28571428571,-10.4210526315789)(1161.28571428571,-11.3684210526316)(1161.28571428571,-12.3157894736842)(1161.28571428571,-13.2631578947368)(1161.28571428571,-14.2105263157895)(1161.28571428571,-15.1578947368421)(1161.28571428571,-16.1052631578947)(1161.28571428571,-17.0526315789474)(1161.28571428571,-18)(1266.85714285714,0)(1266.85714285714,-0.947368421052632)(1266.85714285714,-1.89473684210526)(1266.85714285714,-2.84210526315789)(1266.85714285714,-3.78947368421053)(1266.85714285714,-4.73684210526316)(1266.85714285714,-5.68421052631579)(1266.85714285714,-6.63157894736842)(1266.85714285714,-7.57894736842105)(1266.85714285714,-8.52631578947369)(1266.85714285714,-9.47368421052632)(1266.85714285714,-10.4210526315789)(1266.85714285714,-11.3684210526316)(1266.85714285714,-12.3157894736842)(1266.85714285714,-13.2631578947368)(1266.85714285714,-14.2105263157895)(1266.85714285714,-15.1578947368421)(1266.85714285714,-16.1052631578947)(1266.85714285714,-17.0526315789474)(1266.85714285714,-18)(1372.42857142857,0)(1372.42857142857,-0.947368421052632)(1372.42857142857,-1.89473684210526)(1372.42857142857,-2.84210526315789)(1372.42857142857,-3.78947368421053)(1372.42857142857,-4.73684210526316)(1372.42857142857,-5.68421052631579)(1372.42857142857,-6.63157894736842)(1372.42857142857,-7.57894736842105)(1372.42857142857,-8.52631578947369)(1372.42857142857,-9.47368421052632)(1372.42857142857,-10.4210526315789)(1372.42857142857,-11.3684210526316)(1372.42857142857,-12.3157894736842)(1372.42857142857,-13.2631578947368)(1372.42857142857,-14.2105263157895)(1372.42857142857,-15.1578947368421)(1372.42857142857,-16.1052631578947)(1372.42857142857,-17.0526315789474)(1372.42857142857,-18)(1478,0)(1478,-0.947368421052632)(1478,-1.89473684210526)(1478,-2.84210526315789)(1478,-3.78947368421053)(1478,-4.73684210526316)(1478,-5.68421052631579)(1478,-6.63157894736842)(1478,-7.57894736842105)(1478,-8.52631578947369)(1478,-9.47368421052632)(1478,-10.4210526315789)(1478,-11.3684210526316)(1478,-12.3157894736842)(1478,-13.2631578947368)(1478,-14.2105263157895)(1478,-15.1578947368421)(1478,-16.1052631578947)(1478,-17.0526315789474)(1478,-18)(1372.42857142857,0)(1372.42857142857,-0.947368421052632)(1372.42857142857,-1.89473684210526)(1372.42857142857,-2.84210526315789)(1372.42857142857,-3.78947368421053)(1372.42857142857,-4.73684210526316)(1372.42857142857,-5.68421052631579)(1372.42857142857,-6.63157894736842)(1372.42857142857,-7.57894736842105)(1372.42857142857,-8.52631578947369)(1372.42857142857,-9.47368421052632)(1372.42857142857,-10.4210526315789)(1372.42857142857,-11.3684210526316)(1372.42857142857,-12.3157894736842)(1372.42857142857,-13.2631578947368)(1372.42857142857,-14.2105263157895)(1372.42857142857,-15.1578947368421)(1372.42857142857,-16.1052631578947)(1372.42857142857,-17.0526315789474)(1372.42857142857,-18)(1266.85714285714,0)(1266.85714285714,-0.947368421052632)(1266.85714285714,-1.89473684210526)(1266.85714285714,-2.84210526315789)(1266.85714285714,-3.78947368421053)(1266.85714285714,-4.73684210526316)(1266.85714285714,-5.68421052631579)(1266.85714285714,-6.63157894736842)(1266.85714285714,-7.57894736842105)(1266.85714285714,-8.52631578947369)(1266.85714285714,-9.47368421052632)(1266.85714285714,-10.4210526315789)(1266.85714285714,-11.3684210526316)(1266.85714285714,-12.3157894736842)(1266.85714285714,-13.2631578947368)(1266.85714285714,-14.2105263157895)(1266.85714285714,-15.1578947368421)(1266.85714285714,-16.1052631578947)(1266.85714285714,-17.0526315789474)(1266.85714285714,-18)(1161.28571428571,0)(1161.28571428571,-0.947368421052632)(1161.28571428571,-1.89473684210526)(1161.28571428571,-2.84210526315789)(1161.28571428571,-3.78947368421053)(1161.28571428571,-4.73684210526316)(1161.28571428571,-5.68421052631579)(1161.28571428571,-6.63157894736842)(1161.28571428571,-7.57894736842105)(1161.28571428571,-8.52631578947369)(1161.28571428571,-9.47368421052632)(1161.28571428571,-10.4210526315789)(1161.28571428571,-11.3684210526316)(1161.28571428571,-12.3157894736842)(1161.28571428571,-13.2631578947368)(1161.28571428571,-14.2105263157895)(1161.28571428571,-15.1578947368421)(1161.28571428571,-16.1052631578947)(1161.28571428571,-17.0526315789474)(1161.28571428571,-18)(1055.71428571429,0)(1055.71428571429,-0.947368421052632)(1055.71428571429,-1.89473684210526)(1055.71428571429,-2.84210526315789)(1055.71428571429,-3.78947368421053)(1055.71428571429,-4.73684210526316)(1055.71428571429,-5.68421052631579)(1055.71428571429,-6.63157894736842)(1055.71428571429,-7.57894736842105)(1055.71428571429,-8.52631578947369)(1055.71428571429,-9.47368421052632)(1055.71428571429,-10.4210526315789)(1055.71428571429,-11.3684210526316)(1055.71428571429,-12.3157894736842)(1055.71428571429,-13.2631578947368)(1055.71428571429,-14.2105263157895)(1055.71428571429,-15.1578947368421)(1055.71428571429,-16.1052631578947)(1055.71428571429,-17.0526315789474)(1055.71428571429,-18)(950.142857142857,0)(950.142857142857,-0.947368421052632)(950.142857142857,-1.89473684210526)(950.142857142857,-2.84210526315789)(950.142857142857,-3.78947368421053)(950.142857142857,-4.73684210526316)(950.142857142857,-5.68421052631579)(950.142857142857,-6.63157894736842)(950.142857142857,-7.57894736842105)(950.142857142857,-8.52631578947369)(950.142857142857,-9.47368421052632)(950.142857142857,-10.4210526315789)(950.142857142857,-11.3684210526316)(950.142857142857,-12.3157894736842)(950.142857142857,-13.2631578947368)(950.142857142857,-14.2105263157895)(950.142857142857,-15.1578947368421)(950.142857142857,-16.1052631578947)(950.142857142857,-17.0526315789474)(950.142857142857,-18)(844.571428571429,0)(844.571428571429,-0.947368421052632)(844.571428571429,-1.89473684210526)(844.571428571429,-2.84210526315789)(844.571428571429,-3.78947368421053)(844.571428571429,-4.73684210526316)(844.571428571429,-5.68421052631579)(844.571428571429,-6.63157894736842)(844.571428571429,-7.57894736842105)(844.571428571429,-8.52631578947369)(844.571428571429,-9.47368421052632)(844.571428571429,-10.4210526315789)(844.571428571429,-11.3684210526316)(844.571428571429,-12.3157894736842)(844.571428571429,-13.2631578947368)(844.571428571429,-14.2105263157895)(844.571428571429,-15.1578947368421)(844.571428571429,-16.1052631578947)(844.571428571429,-17.0526315789474)(844.571428571429,-18)(739,0)(739,-0.947368421052632)(739,-1.89473684210526)(739,-2.84210526315789)(739,-3.78947368421053)(739,-4.73684210526316)(739,-5.68421052631579)(739,-6.63157894736842)(739,-7.57894736842105)(739,-8.52631578947369)(739,-9.47368421052632)(739,-10.4210526315789)(739,-11.3684210526316)(739,-12.3157894736842)(739,-13.2631578947368)(739,-14.2105263157895)(739,-15.1578947368421)(739,-16.1052631578947)(739,-17.0526315789474)(739,-18)(633.428571428571,0)(633.428571428571,-0.947368421052632)(633.428571428571,-1.89473684210526)(633.428571428571,-2.84210526315789)(633.428571428571,-3.78947368421053)(633.428571428571,-4.73684210526316)(633.428571428571,-5.68421052631579)(633.428571428571,-6.63157894736842)(633.428571428571,-7.57894736842105)(633.428571428571,-8.52631578947369)(633.428571428571,-9.47368421052632)(633.428571428571,-10.4210526315789)(633.428571428571,-11.3684210526316)(633.428571428571,-12.3157894736842)(633.428571428571,-13.2631578947368)(633.428571428571,-14.2105263157895)(633.428571428571,-15.1578947368421)(633.428571428571,-16.1052631578947)(633.428571428571,-17.0526315789474)(633.428571428571,-18)(527.857142857143,0)(527.857142857143,-0.947368421052632)(527.857142857143,-1.89473684210526)(527.857142857143,-2.84210526315789)(527.857142857143,-3.78947368421053)(527.857142857143,-4.73684210526316)(527.857142857143,-5.68421052631579)(527.857142857143,-6.63157894736842)(527.857142857143,-7.57894736842105)(527.857142857143,-8.52631578947369)(527.857142857143,-9.47368421052632)(527.857142857143,-10.4210526315789)(527.857142857143,-11.3684210526316)(527.857142857143,-12.3157894736842)(527.857142857143,-13.2631578947368)(527.857142857143,-14.2105263157895)(527.857142857143,-15.1578947368421)(527.857142857143,-16.1052631578947)(527.857142857143,-17.0526315789474)(527.857142857143,-18)(422.285714285714,0)(422.285714285714,-0.947368421052632)(422.285714285714,-1.89473684210526)(422.285714285714,-2.84210526315789)(422.285714285714,-3.78947368421053)(422.285714285714,-4.73684210526316)(422.285714285714,-5.68421052631579)(422.285714285714,-6.63157894736842)(422.285714285714,-7.57894736842105)(422.285714285714,-8.52631578947369)(422.285714285714,-9.47368421052632)(422.285714285714,-10.4210526315789)(422.285714285714,-11.3684210526316)(422.285714285714,-12.3157894736842)(422.285714285714,-13.2631578947368)(422.285714285714,-14.2105263157895)(422.285714285714,-15.1578947368421)(422.285714285714,-16.1052631578947)(422.285714285714,-17.0526315789474)(422.285714285714,-18)(316.714285714286,0)(316.714285714286,-0.947368421052632)(316.714285714286,-1.89473684210526)(316.714285714286,-2.84210526315789)(316.714285714286,-3.78947368421053)(316.714285714286,-4.73684210526316)(316.714285714286,-5.68421052631579)(316.714285714286,-6.63157894736842)(316.714285714286,-7.57894736842105)(316.714285714286,-8.52631578947369)(316.714285714286,-9.47368421052632)(316.714285714286,-10.4210526315789)(316.714285714286,-11.3684210526316)(316.714285714286,-12.3157894736842)(316.714285714286,-13.2631578947368)(316.714285714286,-14.2105263157895)(316.714285714286,-15.1578947368421)(316.714285714286,-16.1052631578947)(316.714285714286,-17.0526315789474)(316.714285714286,-18)(211.142857142857,0)(211.142857142857,-0.947368421052632)(211.142857142857,-1.89473684210526)(211.142857142857,-2.84210526315789)(211.142857142857,-3.78947368421053)(211.142857142857,-4.73684210526316)(211.142857142857,-5.68421052631579)(211.142857142857,-6.63157894736842)(211.142857142857,-7.57894736842105)(211.142857142857,-8.52631578947369)(211.142857142857,-9.47368421052632)(211.142857142857,-10.4210526315789)(211.142857142857,-11.3684210526316)(211.142857142857,-12.3157894736842)(211.142857142857,-13.2631578947368)(211.142857142857,-14.2105263157895)(211.142857142857,-15.1578947368421)(211.142857142857,-16.1052631578947)(211.142857142857,-17.0526315789474)(211.142857142857,-18)(105.571428571429,0)(105.571428571429,-0.947368421052632)(105.571428571429,-1.89473684210526)(105.571428571429,-2.84210526315789)(105.571428571429,-3.78947368421053)(105.571428571429,-4.73684210526316)(105.571428571429,-5.68421052631579)(105.571428571429,-6.63157894736842)(105.571428571429,-7.57894736842105)(105.571428571429,-8.52631578947369)(105.571428571429,-9.47368421052632)(105.571428571429,-10.4210526315789)(105.571428571429,-11.3684210526316)(105.571428571429,-12.3157894736842)(105.571428571429,-13.2631578947368)(105.571428571429,-14.2105263157895)(105.571428571429,-15.1578947368421)(105.571428571429,-16.1052631578947)(105.571428571429,-17.0526315789474)(105.571428571429,-18) 
};

\addplot [
color=blue,
mark size=3.5pt,
only marks,
mark=o,
mark options={solid,draw=lime!80!black},
line width=2.7pt,
forget plot
]
coordinates{
 (0,0) 
};

\end{axis}
\end{tikzpicture}%

%% This file was created by matlab2tikz v0.2.3.
% Copyright (c) 2008--2012, Nico Schlömer <nico.schloemer@gmail.com>
% All rights reserved.
% 
% 
% 
\definecolor{locol}{rgb}{0.26, 0.45, 0.65}

\begin{tikzpicture}

\begin{axis}[%
tick label style={font=\tiny},
label style={font=\tiny},
xlabel shift={-10pt},
ylabel shift={-17pt},
legend style={font=\tiny},
view={0}{90},
width=\figurewidth,
height=\figureheight,
scale only axis,
xmin=0, xmax=1478,
xtick={0, 400, 1000, 1400},
xlabel={Length (m)},
ymin=-18, ymax=0,
ytick={0, -4, -14, -18},
ylabel={Depth (m)},
name=plot1,
axis lines*=box,
axis line style={draw=none},
tickwidth=0.0cm,
clip=false
]

\addplot [fill=locol,draw=black,forget plot] coordinates{ (0,0)(4.94314381270903,0)(9.88628762541806,0)(14.8294314381271,0)(19.7725752508361,0)(24.7157190635452,0)(29.6588628762542,0)(34.6020066889632,0)(39.5451505016722,0)(44.4882943143813,0)(49.4314381270903,0)(54.3745819397993,0)(59.3177257525084,0)(64.2608695652174,0)(69.2040133779264,0)(74.1471571906355,0)(79.0903010033445,0)(84.0334448160535,0)(88.9765886287625,0)(93.9197324414716,0)(98.8628762541806,0)(103.80602006689,0)(108.749163879599,0)(113.692307692308,0)(118.635451505017,0)(123.578595317726,0)(128.521739130435,0)(133.464882943144,0)(138.408026755853,0)(143.351170568562,0)(148.294314381271,0)(153.23745819398,0)(158.180602006689,0)(163.123745819398,0)(168.066889632107,0)(173.010033444816,0)(177.953177257525,0)(182.896321070234,0)(187.839464882943,0)(192.782608695652,0)(197.725752508361,0)(202.66889632107,0)(207.612040133779,0)(212.555183946488,0)(217.498327759197,0)(222.441471571906,0)(227.384615384615,0)(232.327759197324,0)(237.270903010033,0)(242.214046822742,0)(247.157190635452,0)(252.100334448161,0)(257.04347826087,0)(261.986622073579,0)(266.929765886288,0)(271.872909698997,0)(276.816053511706,0)(281.759197324415,0)(286.702341137124,0)(291.645484949833,0)(296.588628762542,0)(301.531772575251,0)(306.47491638796,0)(311.418060200669,0)(316.361204013378,0)(321.304347826087,0)(326.247491638796,0)(331.190635451505,0)(336.133779264214,0)(341.076923076923,0)(346.020066889632,0)(350.963210702341,0)(355.90635451505,0)(360.849498327759,0)(365.792642140468,0)(370.735785953177,3.00988295082389e-06)(375.678929765886,3.00988295082389e-06)(380.622073578595,3.00988295082389e-06)(385.565217391304,3.00988295082389e-06)(390.508361204013,3.00988295082389e-06)(395.451505016722,3.00988295082389e-06)(400.394648829431,3.00988295082389e-06)(405.33779264214,3.00988295082389e-06)(410.28093645485,3.00988295082389e-06)(415.224080267559,3.00988295082389e-06)(420.167224080268,3.00988295082389e-06)(425.110367892977,3.00988295082389e-06)(430.053511705686,3.00988295082389e-06)(434.996655518395,3.00988295082389e-06)(439.939799331104,3.00988295082389e-06)(444.882943143813,3.00988295082389e-06)(449.826086956522,3.00988295082389e-06)(454.769230769231,3.00988295082389e-06)(459.71237458194,3.00988295082389e-06)(464.655518394649,3.00988295082389e-06)(469.598662207358,3.00988295082389e-06)(474.541806020067,3.00988295082389e-06)(479.484949832776,3.00988295082389e-06)(484.428093645485,3.00988295082389e-06)(489.371237458194,3.00988295082389e-06)(494.314381270903,3.00988295082389e-06)(499.257525083612,3.00988295082389e-06)(504.200668896321,3.00988295082389e-06)(509.14381270903,3.00988295082389e-06)(514.086956521739,3.00988295082389e-06)(519.030100334448,3.00988295082389e-06)(523.973244147157,3.00988295082389e-06)(528.916387959866,3.00988295082389e-06)(533.859531772575,3.00988295082389e-06)(538.802675585284,3.00988295082389e-06)(543.745819397993,3.00988295082389e-06)(548.688963210702,3.00988295082389e-06)(553.632107023411,3.00988295082389e-06)(558.57525083612,3.00988295082389e-06)(563.518394648829,3.00988295082389e-06)(568.461538461538,3.00988295082389e-06)(573.404682274248,3.00988295082389e-06)(578.347826086957,3.00988295082389e-06)(583.290969899666,3.00988295082389e-06)(588.234113712375,3.00988295082389e-06)(593.177257525084,3.00988295082389e-06)(598.120401337793,3.00988295082389e-06)(603.063545150502,3.00988295082389e-06)(608.006688963211,3.00988295082389e-06)(612.94983277592,3.00988295082389e-06)(617.892976588629,3.00988295082389e-06)(622.836120401338,3.00988295082389e-06)(627.779264214047,3.00988295082389e-06)(632.722408026756,3.00988295082389e-06)(637.665551839465,3.00988295082389e-06)(642.608695652174,3.00988295082389e-06)(647.551839464883,3.00988295082389e-06)(652.494983277592,3.00988295082389e-06)(657.438127090301,3.00988295082389e-06)(662.38127090301,3.00988295082389e-06)(667.324414715719,3.00988295082389e-06)(672.267558528428,3.00988295082389e-06)(677.210702341137,3.00988295082389e-06)(682.153846153846,3.00988295082389e-06)(687.096989966555,3.00988295082389e-06)(692.040133779264,3.00988295082389e-06)(696.983277591973,3.00988295082389e-06)(701.926421404682,3.00988295082389e-06)(706.869565217391,3.00988295082389e-06)(711.8127090301,3.00988295082389e-06)(716.755852842809,3.00988295082389e-06)(721.698996655518,3.00988295082389e-06)(726.642140468227,3.00988295082389e-06)(731.585284280936,3.00988295082389e-06)(736.528428093646,3.00988295082389e-06)(741.471571906354,3.00988295082389e-06)(746.414715719064,3.00988295082389e-06)(751.357859531773,3.00988295082389e-06)(756.301003344482,3.00988295082389e-06)(761.244147157191,3.00988295082389e-06)(766.1872909699,3.00988295082389e-06)(771.130434782609,3.00988295082389e-06)(776.073578595318,3.00988295082389e-06)(781.016722408027,3.00988295082389e-06)(785.959866220736,3.00988295082389e-06)(790.903010033445,3.00988295082389e-06)(795.846153846154,3.00988295082389e-06)(800.789297658863,3.00988295082389e-06)(805.732441471572,3.00988295082389e-06)(810.675585284281,3.00988295082389e-06)(815.61872909699,3.00988295082389e-06)(820.561872909699,3.00988295082389e-06)(825.505016722408,3.00988295082389e-06)(830.448160535117,3.00988295082389e-06)(835.391304347826,3.00988295082389e-06)(840.334448160535,3.00988295082389e-06)(845.277591973244,3.00988295082389e-06)(850.220735785953,3.00988295082389e-06)(855.163879598662,3.00988295082389e-06)(860.107023411371,3.00988295082389e-06)(865.05016722408,3.00988295082389e-06)(869.993311036789,3.00988295082389e-06)(874.936454849498,3.00988295082389e-06)(879.879598662207,3.00988295082389e-06)(884.822742474916,3.00988295082389e-06)(889.765886287625,3.00988295082389e-06)(894.709030100334,3.00988295082389e-06)(899.652173913044,3.00988295082389e-06)(904.595317725752,3.00988295082389e-06)(909.538461538462,3.00988295082389e-06)(914.481605351171,3.00988295082389e-06)(919.42474916388,3.00988295082389e-06)(924.367892976589,3.00988295082389e-06)(929.311036789298,3.00988295082389e-06)(934.254180602007,3.00988295082389e-06)(939.197324414716,3.00988295082389e-06)(944.140468227425,3.00988295082389e-06)(949.083612040134,3.00988295082389e-06)(954.026755852843,3.00988295082389e-06)(958.969899665552,3.00988295082389e-06)(963.913043478261,3.00988295082389e-06)(968.85618729097,3.00988295082389e-06)(973.799331103679,3.00988295082389e-06)(978.742474916388,3.00988295082389e-06)(983.685618729097,3.00988295082389e-06)(988.628762541806,3.00988295082389e-06)(993.571906354515,3.00988295082389e-06)(998.515050167224,3.00988295082389e-06)(1003.45819397993,3.00988295082389e-06)(1008.40133779264,3.00988295082389e-06)(1013.34448160535,3.00988295082389e-06)(1018.28762541806,3.00988295082389e-06)(1023.23076923077,3.00988295082389e-06)(1028.17391304348,3.00988295082389e-06)(1033.11705685619,3.00988295082389e-06)(1038.0602006689,3.00988295082389e-06)(1043.00334448161,3.00988295082389e-06)(1047.94648829431,3.00988295082389e-06)(1052.88963210702,3.00988295082389e-06)(1057.83277591973,3.00988295082389e-06)(1062.77591973244,3.00988295082389e-06)(1067.71906354515,3.00988295082389e-06)(1072.66220735786,3.00988295082389e-06)(1077.60535117057,3.00988295082389e-06)(1082.54849498328,3.00988295082389e-06)(1087.49163879599,3.00988295082389e-06)(1092.4347826087,3.00988295082389e-06)(1097.3779264214,3.00988295082389e-06)(1102.32107023411,3.00988295082389e-06)(1107.26421404682,3.00988295082389e-06)(1112.20735785953,3.00988295082389e-06)(1117.15050167224,3.00988295082389e-06)(1122.09364548495,3.00988295082389e-06)(1127.03678929766,3.00988295082389e-06)(1131.97993311037,3.00988295082389e-06)(1136.92307692308,3.00988295082389e-06)(1141.86622073579,3.00988295082389e-06)(1146.8093645485,3.00988295082389e-06)(1151.7525083612,3.00988295082389e-06)(1156.69565217391,3.00988295082389e-06)(1161.63879598662,3.00988295082389e-06)(1166.58193979933,3.00988295082389e-06)(1171.52508361204,3.00988295082389e-06)(1176.46822742475,3.00988295082389e-06)(1181.41137123746,3.00988295082389e-06)(1186.35451505017,3.00988295082389e-06)(1191.29765886288,3.00988295082389e-06)(1196.24080267559,3.00988295082389e-06)(1201.18394648829,3.00988295082389e-06)(1206.127090301,3.00988295082389e-06)(1211.07023411371,3.00988295082389e-06)(1216.01337792642,3.00988295082389e-06)(1220.95652173913,3.00988295082389e-06)(1225.89966555184,3.00988295082389e-06)(1230.84280936455,3.00988295082389e-06)(1235.78595317726,3.00988295082389e-06)(1240.72909698997,3.00988295082389e-06)(1245.67224080268,3.00988295082389e-06)(1250.61538461538,3.00988295082389e-06)(1255.55852842809,3.00988295082389e-06)(1260.5016722408,3.00988295082389e-06)(1265.44481605351,3.00988295082389e-06)(1270.38795986622,3.00988295082389e-06)(1275.33110367893,3.00988295082389e-06)(1280.27424749164,3.00988295082389e-06)(1285.21739130435,3.00988295082389e-06)(1290.16053511706,3.00988295082389e-06)(1295.10367892977,3.00988295082389e-06)(1300.04682274247,3.00988295082389e-06)(1304.98996655518,3.00988295082389e-06)(1309.93311036789,3.00988295082389e-06)(1314.8762541806,3.00988295082389e-06)(1319.81939799331,3.00988295082389e-06)(1324.76254180602,3.00988295082389e-06)(1329.70568561873,3.00988295082389e-06)(1334.64882943144,3.00988295082389e-06)(1339.59197324415,3.00988295082389e-06)(1344.53511705686,3.00988295082389e-06)(1349.47826086957,3.00988295082389e-06)(1354.42140468227,3.00988295082389e-06)(1359.36454849498,3.00988295082389e-06)(1364.30769230769,3.00988295082389e-06)(1369.2508361204,3.00988295082389e-06)(1374.19397993311,3.00988295082389e-06)(1379.13712374582,3.00988295082389e-06)(1384.08026755853,3.00988295082389e-06)(1389.02341137124,3.00988295082389e-06)(1393.96655518395,3.00988295082389e-06)(1398.90969899666,3.00988295082389e-06)(1403.85284280936,3.00988295082389e-06)(1408.79598662207,3.00988295082389e-06)(1413.73913043478,3.00988295082389e-06)(1418.68227424749,3.00988295082389e-06)(1423.6254180602,3.00988295082389e-06)(1428.56856187291,3.00988295082389e-06)(1433.51170568562,3.00988295082389e-06)(1438.45484949833,3.00988295082389e-06)(1443.39799331104,3.00988295082389e-06)(1448.34113712375,3.00988295082389e-06)(1453.28428093645,3.00988295082389e-06)(1458.22742474916,3.00988295082389e-06)(1463.17056856187,3.00988295082389e-06)(1468.11371237458,3.00988295082389e-06)(1473.05685618729,3.00988295082389e-06)(1478,3.00988295082389e-06)(1478.00024714483,0)(1478.00024714483,-0.0602006688963215)(1478.00024714483,-0.120401337792643)(1478.00024714483,-0.180602006688964)(1478.00024714483,-0.240802675585286)(1478.00024714483,-0.301003344481604)(1478.00024714483,-0.361204013377925)(1478.00024714483,-0.421404682274247)(1478.00024714483,-0.481605351170568)(1478.00024714483,-0.54180602006689)(1478.00024714483,-0.602006688963211)(1478.00024714483,-0.662207357859533)(1478.00024714483,-0.722408026755854)(1478.00024714483,-0.782608695652176)(1478.00024714483,-0.842809364548494)(1478.00024714483,-0.903010033444815)(1478.00024714483,-0.963210702341136)(1478.00024714483,-1.02341137123746)(1478.00024714483,-1.08361204013378)(1478.00024714483,-1.1438127090301)(1478.00024714483,-1.20401337792642)(1478.00024714483,-1.26421404682274)(1478.00024714483,-1.32441471571906)(1478.00024714483,-1.38461538461538)(1478.00024714483,-1.4448160535117)(1478.00024714483,-1.50501672240803)(1478.00024714483,-1.56521739130435)(1478.00024714483,-1.62541806020067)(1478.00024714483,-1.68561872909699)(1478.00024714483,-1.74581939799331)(1478.00024714483,-1.80602006688963)(1478.00024714483,-1.86622073578595)(1478.00024714483,-1.92642140468227)(1478.00024714483,-1.98662207357859)(1478.00024714483,-2.04682274247492)(1478.00024714483,-2.10702341137124)(1478.00024714483,-2.16722408026756)(1478.00024714483,-2.22742474916388)(1478.00024714483,-2.2876254180602)(1478.00024714483,-2.34782608695652)(1478.00024714483,-2.40802675585284)(1478.00024714483,-2.46822742474916)(1478.00024714483,-2.52842809364548)(1478.00024714483,-2.58862876254181)(1478.00024714483,-2.64882943143813)(1478.00024714483,-2.70903010033445)(1478.00024714483,-2.76923076923077)(1478.00024714483,-2.82943143812709)(1478.00024714483,-2.88963210702341)(1478.00024714483,-2.94983277591973)(1478.00024714483,-3.01003344481605)(1478.00024714483,-3.07023411371237)(1478.00024714483,-3.1304347826087)(1478.00024714483,-3.19063545150502)(1478.00024714483,-3.25083612040134)(1478.00024714483,-3.31103678929766)(1478.00024714483,-3.37123745819398)(1478.00024714483,-3.4314381270903)(1478.00024714483,-3.49163879598662)(1478.00024714483,-3.55183946488294)(1478.00024714483,-3.61204013377926)(1478.00024714483,-3.67224080267559)(1478.00024714483,-3.73244147157191)(1478.00024714483,-3.79264214046823)(1478.00024714483,-3.85284280936455)(1478.00024714483,-3.91304347826087)(1478.00024714483,-3.97324414715719)(1478.00024714483,-4.03344481605351)(1478.00024714483,-4.09364548494983)(1478.00024714483,-4.15384615384615)(1478.00024714483,-4.21404682274247)(1478.00024714483,-4.2742474916388)(1478.00024714483,-4.33444816053512)(1478.00024714483,-4.39464882943144)(1478.00024714483,-4.45484949832776)(1478.00024714483,-4.51505016722408)(1478.00024714483,-4.5752508361204)(1478.00024714483,-4.63545150501672)(1478.00024714483,-4.69565217391304)(1478.00024714483,-4.75585284280936)(1478.00024714483,-4.81605351170569)(1478.00024714483,-4.87625418060201)(1478.00024714483,-4.93645484949833)(1478.00024714483,-4.99665551839465)(1478.00024714483,-5.05685618729097)(1478.00024714483,-5.11705685618729)(1478.00024714483,-5.17725752508361)(1478.00024714483,-5.23745819397993)(1478.00024714483,-5.29765886287625)(1478.00024714483,-5.35785953177258)(1478.00024714483,-5.4180602006689)(1478.00024714483,-5.47826086956522)(1478.00024714483,-5.53846153846154)(1478.00024714483,-5.59866220735786)(1478.00024714483,-5.65886287625418)(1478.00024714483,-5.7190635451505)(1478.00024714483,-5.77926421404682)(1478.00024714483,-5.83946488294314)(1478.00024714483,-5.89966555183947)(1478.00024714483,-5.95986622073579)(1478.00024714483,-6.02006688963211)(1478.00024714483,-6.08026755852843)(1478.00024714483,-6.14046822742475)(1478.00024714483,-6.20066889632107)(1478.00024714483,-6.26086956521739)(1478.00024714483,-6.32107023411371)(1478.00024714483,-6.38127090301003)(1478.00024714483,-6.44147157190636)(1478.00024714483,-6.50167224080267)(1478.00024714483,-6.561872909699)(1478.00024714483,-6.62207357859532)(1478.00024714483,-6.68227424749164)(1478.00024714483,-6.74247491638796)(1478.00024714483,-6.80267558528428)(1478.00024714483,-6.8628762541806)(1478.00024714483,-6.92307692307692)(1478.00024714483,-6.98327759197324)(1478.00024714483,-7.04347826086956)(1478.00024714483,-7.10367892976589)(1478.00024714483,-7.16387959866221)(1478.00024714483,-7.22408026755853)(1478.00024714483,-7.28428093645485)(1478.00024714483,-7.34448160535117)(1478.00024714483,-7.40468227424749)(1478.00024714483,-7.46488294314381)(1478.00024714483,-7.52508361204013)(1478.00024714483,-7.58528428093645)(1478.00024714483,-7.64548494983278)(1478.00024714483,-7.7056856187291)(1478.00024714483,-7.76588628762542)(1478.00024714483,-7.82608695652174)(1478.00024714483,-7.88628762541806)(1478.00024714483,-7.94648829431438)(1478.00024714483,-8.0066889632107)(1478.00024714483,-8.06688963210702)(1478.00024714483,-8.12709030100334)(1478.00024714483,-8.18729096989967)(1478.00024714483,-8.24749163879599)(1478.00024714483,-8.30769230769231)(1478.00024714483,-8.36789297658863)(1478.00024714483,-8.42809364548495)(1478.00024714483,-8.48829431438127)(1478.00024714483,-8.54849498327759)(1478.00024714483,-8.60869565217391)(1478.00024714483,-8.66889632107023)(1478.00024714483,-8.72909698996656)(1478.00024714483,-8.78929765886288)(1478.00024714483,-8.8494983277592)(1478.00024714483,-8.90969899665552)(1478.00024714483,-8.96989966555184)(1478.00024714483,-9.03010033444816)(1478.00024714483,-9.09030100334448)(1478.00024714483,-9.1505016722408)(1478.00024714483,-9.21070234113712)(1478.00024714483,-9.27090301003344)(1478.00024714483,-9.33110367892977)(1478.00024714483,-9.39130434782609)(1478.00024714483,-9.45150501672241)(1478.00024714483,-9.51170568561873)(1478.00024714483,-9.57190635451505)(1478.00024714483,-9.63210702341137)(1478.00024714483,-9.69230769230769)(1478.00024714483,-9.75250836120401)(1478.00024714483,-9.81270903010033)(1478.00024714483,-9.87290969899666)(1478.00024714483,-9.93311036789298)(1478.00024714483,-9.9933110367893)(1478.00024714483,-10.0535117056856)(1478.00024714483,-10.1137123745819)(1478.00024714483,-10.1739130434783)(1478.00024714483,-10.2341137123746)(1478.00024714483,-10.2943143812709)(1478.00024714483,-10.3545150501672)(1478.00024714483,-10.4147157190635)(1478.00024714483,-10.4749163879599)(1478.00024714483,-10.5351170568562)(1478.00024714483,-10.5953177257525)(1478.00024714483,-10.6555183946488)(1478.00024714483,-10.7157190635452)(1478.00024714483,-10.7759197324415)(1478.00024714483,-10.8361204013378)(1478.00024714483,-10.8963210702341)(1478.00024714483,-10.9565217391304)(1478.00024714483,-11.0167224080268)(1478.00024714483,-11.0769230769231)(1478.00024714483,-11.1371237458194)(1478.00024714483,-11.1973244147157)(1478.00024714483,-11.257525083612)(1478.00024714483,-11.3177257525084)(1478.00024714483,-11.3779264214047)(1478.00024714483,-11.438127090301)(1478.00024714483,-11.4983277591973)(1478.00024714483,-11.5585284280936)(1478.00024714483,-11.61872909699)(1478.00024714483,-11.6789297658863)(1478.00024714483,-11.7391304347826)(1478.00024714483,-11.7993311036789)(1478.00024714483,-11.8595317725753)(1478.00024714483,-11.9197324414716)(1478.00024714483,-11.9799331103679)(1478.00024714483,-12.0401337792642)(1478.00024714483,-12.1003344481605)(1478.00024714483,-12.1605351170569)(1478.00024714483,-12.2207357859532)(1478.00024714483,-12.2809364548495)(1478.00024714483,-12.3411371237458)(1478.00024714483,-12.4013377926421)(1478.00024714483,-12.4615384615385)(1478.00024714483,-12.5217391304348)(1478.00024714483,-12.5819397993311)(1478.00024714483,-12.6421404682274)(1478.00024714483,-12.7023411371237)(1478.00024714483,-12.7625418060201)(1478.00024714483,-12.8227424749164)(1478.00024714483,-12.8829431438127)(1478.00024714483,-12.943143812709)(1478.00024714483,-13.0033444816054)(1478.00024714483,-13.0635451505017)(1478.00024714483,-13.123745819398)(1478.00024714483,-13.1839464882943)(1478.00024714483,-13.2441471571906)(1478.00024714483,-13.304347826087)(1478.00024714483,-13.3645484949833)(1478.00024714483,-13.4247491638796)(1478.00024714483,-13.4849498327759)(1478.00024714483,-13.5451505016722)(1478.00024714483,-13.6053511705686)(1478.00024714483,-13.6655518394649)(1478.00024714483,-13.7257525083612)(1478.00024714483,-13.7859531772575)(1478.00024714483,-13.8461538461538)(1478.00024714483,-13.9063545150502)(1478.00024714483,-13.9665551839465)(1478.00024714483,-14.0267558528428)(1478.00024714483,-14.0869565217391)(1478.00024714483,-14.1471571906355)(1478.00024714483,-14.2073578595318)(1478.00024714483,-14.2675585284281)(1478.00024714483,-14.3277591973244)(1478.00024714483,-14.3879598662207)(1478.00024714483,-14.4481605351171)(1478.00024714483,-14.5083612040134)(1478.00024714483,-14.5685618729097)(1478.00024714483,-14.628762541806)(1478.00024714483,-14.6889632107023)(1478.00024714483,-14.7491638795987)(1478.00024714483,-14.809364548495)(1478.00024714483,-14.8695652173913)(1478.00024714483,-14.9297658862876)(1478.00024714483,-14.9899665551839)(1478.00024714483,-15.0501672240803)(1478.00024714483,-15.1103678929766)(1478.00024714483,-15.1705685618729)(1478.00024714483,-15.2307692307692)(1478.00024714483,-15.2909698996656)(1478.00024714483,-15.3511705685619)(1478.00024714483,-15.4113712374582)(1478.00024714483,-15.4715719063545)(1478.00024714483,-15.5317725752508)(1478.00024714483,-15.5919732441472)(1478.00024714483,-15.6521739130435)(1478.00024714483,-15.7123745819398)(1478.00024714483,-15.7725752508361)(1478.00024714483,-15.8327759197324)(1478.00024714483,-15.8929765886288)(1478.00024714483,-15.9531772575251)(1478.00024714483,-16.0133779264214)(1478.00024714483,-16.0735785953177)(1478.00024714483,-16.133779264214)(1478.00024714483,-16.1939799331104)(1478.00024714483,-16.2541806020067)(1478.00024714483,-16.314381270903)(1478.00024714483,-16.3745819397993)(1478.00024714483,-16.4347826086957)(1478.00024714483,-16.494983277592)(1478.00024714483,-16.5551839464883)(1478.00024714483,-16.6153846153846)(1478.00024714483,-16.6755852842809)(1478.00024714483,-16.7357859531773)(1478.00024714483,-16.7959866220736)(1478.00024714483,-16.8561872909699)(1478.00024714483,-16.9163879598662)(1478.00024714483,-16.9765886287625)(1478.00024714483,-17.0367892976589)(1478.00024714483,-17.0969899665552)(1478.00024714483,-17.1571906354515)(1478.00024714483,-17.2173913043478)(1478.00024714483,-17.2775919732441)(1478.00024714483,-17.3377926421405)(1478.00024714483,-17.3979933110368)(1478.00024714483,-17.4581939799331)(1478.00024714483,-17.5183946488294)(1478.00024714483,-17.5785953177258)(1478.00024714483,-17.6387959866221)(1478.00024714483,-17.6989966555184)(1478.00024714483,-17.7591973244147)(1478.00024714483,-17.819397993311)(1478.00024714483,-17.8795986622074)(1478.00024714483,-17.9397993311037)(1478.00024714483,-18)(1478,-18.000003009883)(1473.05685618729,-18.000003009883)(1468.11371237458,-18.000003009883)(1463.17056856187,-18.000003009883)(1458.22742474916,-18.000003009883)(1453.28428093645,-18.000003009883)(1448.34113712375,-18.000003009883)(1443.39799331104,-18.000003009883)(1438.45484949833,-18.000003009883)(1433.51170568562,-18.000003009883)(1428.56856187291,-18.000003009883)(1423.6254180602,-18.000003009883)(1418.68227424749,-18.000003009883)(1413.73913043478,-18.000003009883)(1408.79598662207,-18.000003009883)(1403.85284280936,-18.000003009883)(1398.90969899666,-18.000003009883)(1393.96655518395,-18.000003009883)(1389.02341137124,-18.000003009883)(1384.08026755853,-18.000003009883)(1379.13712374582,-18.000003009883)(1374.19397993311,-18.000003009883)(1369.2508361204,-18.000003009883)(1364.30769230769,-18.000003009883)(1359.36454849498,-18.000003009883)(1354.42140468227,-18.000003009883)(1349.47826086957,-18.000003009883)(1344.53511705686,-18.000003009883)(1339.59197324415,-18.000003009883)(1334.64882943144,-18.000003009883)(1329.70568561873,-18.000003009883)(1324.76254180602,-18.000003009883)(1319.81939799331,-18.000003009883)(1314.8762541806,-18.000003009883)(1309.93311036789,-18.000003009883)(1304.98996655518,-18.000003009883)(1300.04682274247,-18.000003009883)(1295.10367892977,-18.000003009883)(1290.16053511706,-18.000003009883)(1285.21739130435,-18.000003009883)(1280.27424749164,-18.000003009883)(1275.33110367893,-18.000003009883)(1270.38795986622,-18.000003009883)(1265.44481605351,-18.000003009883)(1260.5016722408,-18.000003009883)(1255.55852842809,-18.000003009883)(1250.61538461538,-18.000003009883)(1245.67224080268,-18.000003009883)(1240.72909698997,-18.000003009883)(1235.78595317726,-18.000003009883)(1230.84280936455,-18.000003009883)(1225.89966555184,-18.000003009883)(1220.95652173913,-18.000003009883)(1216.01337792642,-18.000003009883)(1211.07023411371,-18.000003009883)(1206.127090301,-18.000003009883)(1201.18394648829,-18.000003009883)(1196.24080267559,-18.000003009883)(1191.29765886288,-18.000003009883)(1186.35451505017,-18.000003009883)(1181.41137123746,-18.000003009883)(1176.46822742475,-18.000003009883)(1171.52508361204,-18.000003009883)(1166.58193979933,-18.000003009883)(1161.63879598662,-18.000003009883)(1156.69565217391,-18.000003009883)(1151.7525083612,-18.000003009883)(1146.8093645485,-18.000003009883)(1141.86622073579,-18.000003009883)(1136.92307692308,-18.000003009883)(1131.97993311037,-18.000003009883)(1127.03678929766,-18.000003009883)(1122.09364548495,-18.000003009883)(1117.15050167224,-18.000003009883)(1112.20735785953,-18.000003009883)(1107.26421404682,-18.000003009883)(1102.32107023411,-18.000003009883)(1097.3779264214,-18.000003009883)(1092.4347826087,-18.000003009883)(1087.49163879599,-18.000003009883)(1082.54849498328,-18.000003009883)(1077.60535117057,-18.000003009883)(1072.66220735786,-18.000003009883)(1067.71906354515,-18.000003009883)(1062.77591973244,-18.000003009883)(1057.83277591973,-18.000003009883)(1052.88963210702,-18.000003009883)(1047.94648829431,-18.000003009883)(1043.00334448161,-18.000003009883)(1038.0602006689,-18.000003009883)(1033.11705685619,-18.000003009883)(1028.17391304348,-18.000003009883)(1023.23076923077,-18.000003009883)(1018.28762541806,-18.000003009883)(1013.34448160535,-18.000003009883)(1008.40133779264,-18.000003009883)(1003.45819397993,-18.000003009883)(998.515050167224,-18.000003009883)(993.571906354515,-18.000003009883)(988.628762541806,-18.000003009883)(983.685618729097,-18.000003009883)(978.742474916388,-18.000003009883)(973.799331103679,-18.000003009883)(968.85618729097,-18.000003009883)(963.913043478261,-18.000003009883)(958.969899665552,-18.000003009883)(954.026755852843,-18.000003009883)(949.083612040134,-18.000003009883)(944.140468227425,-18.000003009883)(939.197324414716,-18.000003009883)(934.254180602007,-18.000003009883)(929.311036789298,-18.000003009883)(924.367892976589,-18.000003009883)(919.42474916388,-18.000003009883)(914.481605351171,-18.000003009883)(909.538461538462,-18.000003009883)(904.595317725752,-18.000003009883)(899.652173913044,-18.000003009883)(894.709030100334,-18.000003009883)(889.765886287625,-18.000003009883)(884.822742474916,-18.000003009883)(879.879598662207,-18.000003009883)(874.936454849498,-18.000003009883)(869.993311036789,-18.000003009883)(865.05016722408,-18.000003009883)(860.107023411371,-18.000003009883)(855.163879598662,-18.000003009883)(850.220735785953,-18.000003009883)(845.277591973244,-18.000003009883)(840.334448160535,-18.000003009883)(835.391304347826,-18.000003009883)(830.448160535117,-18.000003009883)(825.505016722408,-18.000003009883)(820.561872909699,-18.000003009883)(815.61872909699,-18.000003009883)(810.675585284281,-18.000003009883)(805.732441471572,-18.000003009883)(800.789297658863,-18.000003009883)(795.846153846154,-18.000003009883)(790.903010033445,-18.000003009883)(785.959866220736,-18.000003009883)(781.016722408027,-18.000003009883)(776.073578595318,-18.000003009883)(771.130434782609,-18.000003009883)(766.1872909699,-18.000003009883)(761.244147157191,-18.000003009883)(756.301003344482,-18.000003009883)(751.357859531773,-18.000003009883)(746.414715719064,-18.000003009883)(741.471571906354,-18.000003009883)(736.528428093646,-18.000003009883)(731.585284280936,-18.000003009883)(726.642140468227,-18.000003009883)(721.698996655518,-18.000003009883)(716.755852842809,-18.000003009883)(711.8127090301,-18.000003009883)(706.869565217391,-18.000003009883)(701.926421404682,-18.000003009883)(696.983277591973,-18.000003009883)(692.040133779264,-18.000003009883)(687.096989966555,-18.000003009883)(682.153846153846,-18.000003009883)(677.210702341137,-18.000003009883)(672.267558528428,-18.000003009883)(667.324414715719,-18.000003009883)(662.38127090301,-18.000003009883)(657.438127090301,-18.000003009883)(652.494983277592,-18.000003009883)(647.551839464883,-18.000003009883)(642.608695652174,-18.000003009883)(637.665551839465,-18.000003009883)(632.722408026756,-18.000003009883)(627.779264214047,-18.000003009883)(622.836120401338,-18.000003009883)(617.892976588629,-18.000003009883)(612.94983277592,-18.000003009883)(608.006688963211,-18.000003009883)(603.063545150502,-18.000003009883)(598.120401337793,-18.000003009883)(593.177257525084,-18.000003009883)(588.234113712375,-18.000003009883)(583.290969899666,-18.000003009883)(578.347826086957,-18.000003009883)(573.404682274248,-18.000003009883)(568.461538461538,-18.000003009883)(563.518394648829,-18)(558.57525083612,-18)(553.632107023411,-18)(548.688963210702,-18)(543.745819397993,-18)(538.802675585284,-18)(533.859531772575,-18)(528.916387959866,-18)(523.973244147157,-18)(519.030100334448,-18)(514.086956521739,-18)(509.14381270903,-18)(504.200668896321,-18)(499.257525083612,-18)(494.314381270903,-18)(489.371237458194,-18)(484.428093645485,-18)(479.484949832776,-18)(474.541806020067,-18)(469.598662207358,-18)(464.655518394649,-18)(459.71237458194,-18)(454.769230769231,-18)(449.826086956522,-18)(444.882943143813,-18)(439.939799331104,-18)(434.996655518395,-18)(430.053511705686,-18)(425.110367892977,-18)(420.167224080268,-18)(415.224080267559,-18)(410.28093645485,-18)(405.33779264214,-18)(400.394648829431,-18)(395.451505016722,-18)(390.508361204013,-18)(385.565217391304,-18)(380.622073578595,-18)(375.678929765886,-18)(370.735785953177,-18)(365.792642140468,-18)(360.849498327759,-18)(355.90635451505,-18)(350.963210702341,-18)(346.020066889632,-18)(341.076923076923,-18)(336.133779264214,-18)(331.190635451505,-18)(326.247491638796,-18)(321.304347826087,-18)(316.361204013378,-18)(311.418060200669,-18)(306.47491638796,-18)(301.531772575251,-18)(296.588628762542,-18)(291.645484949833,-18)(286.702341137124,-18)(281.759197324415,-18)(276.816053511706,-18)(271.872909698997,-18)(266.929765886288,-18)(261.986622073579,-18)(257.04347826087,-18)(252.100334448161,-18)(247.157190635452,-18)(242.214046822742,-18)(237.270903010033,-18)(232.327759197324,-18)(227.384615384615,-18)(222.441471571906,-18)(217.498327759197,-18)(212.555183946488,-18)(207.612040133779,-18)(202.66889632107,-18)(197.725752508361,-18)(192.782608695652,-18)(187.839464882943,-18)(182.896321070234,-18)(177.953177257525,-18)(173.010033444816,-18)(168.066889632107,-18)(163.123745819398,-18)(158.180602006689,-18)(153.23745819398,-18)(148.294314381271,-18)(143.351170568562,-18)(138.408026755853,-18)(133.464882943144,-18)(128.521739130435,-18)(123.578595317726,-18)(118.635451505017,-18)(113.692307692308,-18)(108.749163879599,-18)(103.80602006689,-18)(98.8628762541806,-18)(93.9197324414716,-18)(88.9765886287625,-18)(84.0334448160535,-18)(79.0903010033445,-18)(74.1471571906355,-18)(69.2040133779264,-18)(64.2608695652174,-18)(59.3177257525084,-18)(54.3745819397993,-18)(49.4314381270903,-18)(44.4882943143813,-18)(39.5451505016722,-18.000003009883)(34.6020066889632,-18.000003009883)(29.6588628762542,-18.000003009883)(24.7157190635452,-18.000003009883)(19.7725752508361,-18.000003009883)(14.8294314381271,-18.000003009883)(9.88628762541806,-18.000003009883)(4.94314381270903,-18.000003009883)(0,-18.0000060194649)(-0.000494264954775508,-18)(-0.000247144833393367,-17.9397993311037)(-0.000247144833393367,-17.8795986622074)(-0.000247144833393367,-17.819397993311)(-0.000247144833393367,-17.7591973244147)(-0.000247144833393367,-17.6989966555184)(-0.000247144833393367,-17.6387959866221)(-0.000247144833393367,-17.5785953177258)(-0.000247144833393367,-17.5183946488294)(-0.000247144833393367,-17.4581939799331)(-0.000247144833393367,-17.3979933110368)(-0.000247144833393367,-17.3377926421405)(-0.000247144833393367,-17.2775919732441)(-0.000247144833393367,-17.2173913043478)(-0.000247144833393367,-17.1571906354515)(-0.000247144833393367,-17.0969899665552)(-0.000247144833393367,-17.0367892976589)(-0.000247144833393367,-16.9765886287625)(-0.000247144833393367,-16.9163879598662)(-0.000247144833393367,-16.8561872909699)(-0.000247144833393367,-16.7959866220736)(-0.000247144833393367,-16.7357859531773)(-0.000247144833393367,-16.6755852842809)(-0.000247144833393367,-16.6153846153846)(-0.000247144833393367,-16.5551839464883)(-0.000247144833393367,-16.494983277592)(-0.000247144833393367,-16.4347826086957)(-0.000247144833393367,-16.3745819397993)(-0.000247144833393367,-16.314381270903)(-0.000247144833393367,-16.2541806020067)(-0.000247144833393367,-16.1939799331104)(-0.000247144833393367,-16.133779264214)(-0.000247144833393367,-16.0735785953177)(-0.000247144833393367,-16.0133779264214)(-0.000247144833393367,-15.9531772575251)(-0.000247144833393367,-15.8929765886288)(-0.000247144833393367,-15.8327759197324)(-0.000247144833393367,-15.7725752508361)(-0.000247144833393367,-15.7123745819398)(-0.000247144833393367,-15.6521739130435)(-0.000247144833393367,-15.5919732441472)(-0.000247144833393367,-15.5317725752508)(-0.000247144833393367,-15.4715719063545)(-0.000247144833393367,-15.4113712374582)(-0.000247144833393367,-15.3511705685619)(-0.000247144833393367,-15.2909698996656)(-0.000247144833393367,-15.2307692307692)(-0.000247144833393367,-15.1705685618729)(-0.000247144833393367,-15.1103678929766)(-0.000247144833393367,-15.0501672240803)(-0.000247144833393367,-14.9899665551839)(-0.000247144833393367,-14.9297658862876)(-0.000247144833393367,-14.8695652173913)(-0.000247144833393367,-14.809364548495)(-0.000247144833393367,-14.7491638795987)(-0.000247144833393367,-14.6889632107023)(-0.000247144833393367,-14.628762541806)(-0.000247144833393367,-14.5685618729097)(-0.000247144833393367,-14.5083612040134)(-0.000247144833393367,-14.4481605351171)(-0.000247144833393367,-14.3879598662207)(-0.000247144833393367,-14.3277591973244)(-0.000247144833393367,-14.2675585284281)(-0.000247144833393367,-14.2073578595318)(-0.000247144833393367,-14.1471571906355)(-0.000247144833393367,-14.0869565217391)(-0.000247144833393367,-14.0267558528428)(-0.000247144833393367,-13.9665551839465)(-0.000247144833393367,-13.9063545150502)(-0.000247144833393367,-13.8461538461538)(-0.000247144833393367,-13.7859531772575)(-0.000247144833393367,-13.7257525083612)(-0.000247144833393367,-13.6655518394649)(-0.000247144833393367,-13.6053511705686)(-0.000247144833393367,-13.5451505016722)(-0.000247144833393367,-13.4849498327759)(-0.000247144833393367,-13.4247491638796)(-0.000247144833393367,-13.3645484949833)(-0.000247144833393367,-13.304347826087)(-0.000247144833393367,-13.2441471571906)(-0.000247144833393367,-13.1839464882943)(-0.000247144833393367,-13.123745819398)(-0.000247144833393367,-13.0635451505017)(-0.000247144833393367,-13.0033444816054)(-0.000247144833393367,-12.943143812709)(-0.000247144833393367,-12.8829431438127)(-0.000247144833393367,-12.8227424749164)(-0.000247144833393367,-12.7625418060201)(-0.000247144833393367,-12.7023411371237)(-0.000247144833393367,-12.6421404682274)(-0.000247144833393367,-12.5819397993311)(-0.000247144833393367,-12.5217391304348)(-0.000247144833393367,-12.4615384615385)(-0.000247144833393367,-12.4013377926421)(-0.000247144833393367,-12.3411371237458)(-0.000247144833393367,-12.2809364548495)(-0.000247144833393367,-12.2207357859532)(-0.000247144833393367,-12.1605351170569)(-0.000247144833393367,-12.1003344481605)(-0.000247144833393367,-12.0401337792642)(-0.000247144833393367,-11.9799331103679)(-0.000247144833393367,-11.9197324414716)(-0.000247144833393367,-11.8595317725753)(-0.000247144833393367,-11.7993311036789)(-0.000247144833393367,-11.7391304347826)(-0.000247144833393367,-11.6789297658863)(-0.000247144833393367,-11.61872909699)(-0.000247144833393367,-11.5585284280936)(-0.000247144833393367,-11.4983277591973)(-0.000247144833393367,-11.438127090301)(-0.000247144833393367,-11.3779264214047)(-0.000247144833393367,-11.3177257525084)(-0.000247144833393367,-11.257525083612)(-0.000247144833393367,-11.1973244147157)(-0.000247144833393367,-11.1371237458194)(-0.000247144833393367,-11.0769230769231)(-0.000247144833393367,-11.0167224080268)(-0.000247144833393367,-10.9565217391304)(-0.000247144833393367,-10.8963210702341)(-0.000247144833393367,-10.8361204013378)(-0.000247144833393367,-10.7759197324415)(-0.000247144833393367,-10.7157190635452)(-0.000247144833393367,-10.6555183946488)(-0.000247144833393367,-10.5953177257525)(-0.000247144833393367,-10.5351170568562)(-0.000247144833393367,-10.4749163879599)(-0.000247144833393367,-10.4147157190635)(-0.000247144833393367,-10.3545150501672)(-0.000247144833393367,-10.2943143812709)(-0.000247144833393367,-10.2341137123746)(-0.000247144833393367,-10.1739130434783)(-0.000247144833393367,-10.1137123745819)(-0.000247144833393367,-10.0535117056856)(-0.000247144833393367,-9.9933110367893)(-0.000247144833393367,-9.93311036789298)(-0.000247144833393367,-9.87290969899666)(-0.000247144833393367,-9.81270903010033)(-0.000247144833393367,-9.75250836120401)(-0.000247144833393367,-9.69230769230769)(-0.000247144833393367,-9.63210702341137)(-0.000247144833393367,-9.57190635451505)(-0.000247144833393367,-9.51170568561873)(-0.000247144833393367,-9.45150501672241)(-0.000247144833393367,-9.39130434782609)(-0.000247144833393367,-9.33110367892977)(-0.000247144833393367,-9.27090301003344)(-0.000247144833393367,-9.21070234113712)(-0.000247144833393367,-9.1505016722408)(-0.000247144833393367,-9.09030100334448)(-0.000247144833393367,-9.03010033444816)(-0.000247144833393367,-8.96989966555184)(-0.000247144833393367,-8.90969899665552)(-0.000247144833393367,-8.8494983277592)(-0.000247144833393367,-8.78929765886288)(-0.000247144833393367,-8.72909698996656)(-0.000247144833393367,-8.66889632107023)(-0.000247144833393367,-8.60869565217391)(-0.000247144833393367,-8.54849498327759)(-0.000247144833393367,-8.48829431438127)(-0.000247144833393367,-8.42809364548495)(-0.000247144833393367,-8.36789297658863)(-0.000247144833393367,-8.30769230769231)(-0.000247144833393367,-8.24749163879599)(-0.000247144833393367,-8.18729096989967)(-0.000247144833393367,-8.12709030100334)(-0.000247144833393367,-8.06688963210702)(-0.000247144833393367,-8.0066889632107)(-0.000247144833393367,-7.94648829431438)(-0.000247144833393367,-7.88628762541806)(-0.000247144833393367,-7.82608695652174)(-0.000247144833393367,-7.76588628762542)(-0.000247144833393367,-7.7056856187291)(-0.000247144833393367,-7.64548494983278)(-0.000247144833393367,-7.58528428093645)(-0.000247144833393367,-7.52508361204013)(-0.000247144833393367,-7.46488294314381)(-0.000247144833393367,-7.40468227424749)(-0.000247144833393367,-7.34448160535117)(-0.000247144833393367,-7.28428093645485)(-0.000247144833393367,-7.22408026755853)(-0.000247144833393367,-7.16387959866221)(-0.000247144833393367,-7.10367892976589)(-0.000247144833393367,-7.04347826086956)(-0.000247144833393367,-6.98327759197324)(-0.000247144833393367,-6.92307692307692)(-0.000247144833393367,-6.8628762541806)(-0.000247144833393367,-6.80267558528428)(-0.000247144833393367,-6.74247491638796)(-0.000247144833393367,-6.68227424749164)(-0.000247144833393367,-6.62207357859532)(-0.000247144833393367,-6.561872909699)(-0.000247144833393367,-6.50167224080267)(-0.000247144833393367,-6.44147157190636)(-0.000247144833393367,-6.38127090301003)(-0.000247144833393367,-6.32107023411371)(-0.000247144833393367,-6.26086956521739)(-0.000247144833393367,-6.20066889632107)(-0.000247144833393367,-6.14046822742475)(-0.000247144833393367,-6.08026755852843)(-0.000247144833393367,-6.02006688963211)(-0.000247144833393367,-5.95986622073579)(-0.000247144833393367,-5.89966555183947)(-0.000247144833393367,-5.83946488294314)(-0.000247144833393367,-5.77926421404682)(-0.000247144833393367,-5.7190635451505)(-0.000247144833393367,-5.65886287625418)(-0.000247144833393367,-5.59866220735786)(-0.000247144833393367,-5.53846153846154)(-0.000247144833393367,-5.47826086956522)(-0.000247144833393367,-5.4180602006689)(-0.000247144833393367,-5.35785953177258)(-0.000247144833393367,-5.29765886287625)(-0.000247144833393367,-5.23745819397993)(-0.000247144833393367,-5.17725752508361)(-0.000247144833393367,-5.11705685618729)(-0.000247144833393367,-5.05685618729097)(-0.000247144833393367,-4.99665551839465)(-0.000247144833393367,-4.93645484949833)(-0.000247144833393367,-4.87625418060201)(-0.000247144833393367,-4.81605351170569)(-0.000247144833393367,-4.75585284280936)(-0.000247144833393367,-4.69565217391304)(-0.000247144833393367,-4.63545150501672)(-0.000247144833393367,-4.5752508361204)(-0.000247144833393367,-4.51505016722408)(0,-4.45484949832776)(0,-4.39464882943144)(0,-4.33444816053512)(0,-4.2742474916388)(0,-4.21404682274247)(0,-4.15384615384615)(0,-4.09364548494983)(0,-4.03344481605351)(0,-3.97324414715719)(0,-3.91304347826087)(0,-3.85284280936455)(0,-3.79264214046823)(0,-3.73244147157191)(0,-3.67224080267559)(0,-3.61204013377926)(0,-3.55183946488294)(0,-3.49163879598662)(0,-3.4314381270903)(0,-3.37123745819398)(0,-3.31103678929766)(0,-3.25083612040134)(0,-3.19063545150502)(0,-3.1304347826087)(0,-3.07023411371237)(0,-3.01003344481605)(0,-2.94983277591973)(0,-2.88963210702341)(0,-2.82943143812709)(0,-2.76923076923077)(0,-2.70903010033445)(0,-2.64882943143813)(0,-2.58862876254181)(0,-2.52842809364548)(0,-2.46822742474916)(0,-2.40802675585284)(0,-2.34782608695652)(0,-2.2876254180602)(0,-2.22742474916388)(0,-2.16722408026756)(0,-2.10702341137124)(0,-2.04682274247492)(0,-1.98662207357859)(0,-1.92642140468227)(0,-1.86622073578595)(0,-1.80602006688963)(0,-1.74581939799331)(0,-1.68561872909699)(0,-1.62541806020067)(0,-1.56521739130435)(0,-1.50501672240803)(0,-1.4448160535117)(0,-1.38461538461538)(0,-1.32441471571906)(0,-1.26421404682274)(0,-1.20401337792642)(0,-1.1438127090301)(0,-1.08361204013378)(0,-1.02341137123746)(0,-0.963210702341136)(0,-0.903010033444815)(0,-0.842809364548494)(0,-0.782608695652176)(0,-0.722408026755854)(0,-0.662207357859533)(0,-0.602006688963211)(0,-0.54180602006689)(0,-0.481605351170568)(0,-0.421404682274247)(0,-0.361204013377925)(0,-0.301003344481604)(0,-0.240802675585286)(0,-0.180602006688964)(0,-0.120401337792643)(0,-0.0602006688963215)(0,0)};

\addplot [fill=darkgray,draw=black,forget plot] coordinates{ (0,-4.51505016722408)(0,-4.5752508361204)(4.94314381270903,-4.63545150501672)(9.88628762541806,-4.69565217391304)(14.8294314381271,-4.75585284280936)(19.7725752508361,-4.81605351170569)(24.7157190635452,-4.87625418060201)(24.7157190635452,-4.93645484949833)(29.6588628762542,-4.99665551839465)(34.6020066889632,-5.05685618729097)(39.5451505016722,-5.11705685618729)(44.4882943143813,-5.17725752508361)(49.4314381270903,-5.23745819397993)(54.3745819397993,-5.29765886287625)(59.3177257525084,-5.35785953177258)(59.3177257525084,-5.4180602006689)(64.2608695652174,-5.47826086956522)(69.2040133779264,-5.53846153846154)(74.1471571906355,-5.59866220735786)(79.0903010033445,-5.65886287625418)(84.0334448160535,-5.7190635451505)(88.9765886287625,-5.77926421404682)(93.9197324414716,-5.83946488294314)(98.8628762541806,-5.89966555183947)(98.8628762541806,-5.95986622073579)(103.80602006689,-6.02006688963211)(108.749163879599,-6.08026755852843)(113.692307692308,-6.14046822742475)(118.635451505017,-6.20066889632107)(123.578595317726,-6.26086956521739)(123.578595317726,-6.32107023411371)(128.521739130435,-6.38127090301003)(133.464882943144,-6.44147157190636)(138.408026755853,-6.50167224080267)(138.408026755853,-6.561872909699)(143.351170568562,-6.62207357859532)(148.294314381271,-6.68227424749164)(148.294314381271,-6.74247491638796)(153.23745819398,-6.80267558528428)(158.180602006689,-6.8628762541806)(158.180602006689,-6.92307692307692)(163.123745819398,-6.98327759197324)(168.066889632107,-7.04347826086956)(173.010033444816,-7.10367892976589)(177.953177257525,-7.16387959866221)(182.896321070234,-7.16387959866221)(187.839464882943,-7.22408026755853)(192.782608695652,-7.22408026755853)(197.725752508361,-7.28428093645485)(202.66889632107,-7.34448160535117)(207.612040133779,-7.34448160535117)(212.555183946488,-7.34448160535117)(217.498327759197,-7.34448160535117)(222.441471571906,-7.34448160535117)(227.384615384615,-7.28428093645485)(232.327759197324,-7.28428093645485)(237.270903010033,-7.22408026755853)(242.214046822742,-7.16387959866221)(247.157190635452,-7.16387959866221)(252.100334448161,-7.10367892976589)(252.100334448161,-7.04347826086956)(257.04347826087,-6.98327759197324)(261.986622073579,-6.92307692307692)(261.986622073579,-6.8628762541806)(266.929765886288,-6.80267558528428)(266.929765886288,-6.74247491638796)(266.929765886288,-6.68227424749164)(271.872909698997,-6.62207357859532)(271.872909698997,-6.561872909699)(271.872909698997,-6.50167224080267)(271.872909698997,-6.44147157190636)(276.816053511706,-6.38127090301003)(276.816053511706,-6.32107023411371)(276.816053511706,-6.26086956521739)(276.816053511706,-6.20066889632107)(276.816053511706,-6.14046822742475)(276.816053511706,-6.08026755852843)(281.759197324415,-6.02006688963211)(281.759197324415,-5.95986622073579)(286.702341137124,-5.89966555183947)(286.702341137124,-5.83946488294314)(286.702341137124,-5.77926421404682)(291.645484949833,-5.7190635451505)(291.645484949833,-5.65886287625418)(291.645484949833,-5.59866220735786)(296.588628762542,-5.53846153846154)(296.588628762542,-5.47826086956522)(296.588628762542,-5.4180602006689)(301.531772575251,-5.35785953177258)(301.531772575251,-5.29765886287625)(306.47491638796,-5.23745819397993)(306.47491638796,-5.17725752508361)(306.47491638796,-5.11705685618729)(311.418060200669,-5.05685618729097)(311.418060200669,-4.99665551839465)(311.418060200669,-4.93645484949833)(316.361204013378,-4.87625418060201)(316.361204013378,-4.81605351170569)(321.304347826087,-4.75585284280936)(321.304347826087,-4.69565217391304)(321.304347826087,-4.63545150501672)(326.247491638796,-4.5752508361204)(326.247491638796,-4.51505016722408)(326.247491638796,-4.45484949832776)(331.190635451505,-4.39464882943144)(331.190635451505,-4.33444816053512)(331.190635451505,-4.2742474916388)(336.133779264214,-4.21404682274247)(336.133779264214,-4.15384615384615)(336.133779264214,-4.09364548494983)(341.076923076923,-4.03344481605351)(341.076923076923,-3.97324414715719)(341.076923076923,-3.91304347826087)(346.020066889632,-3.85284280936455)(346.020066889632,-3.79264214046823)(346.020066889632,-3.73244147157191)(346.020066889632,-3.67224080267559)(350.963210702341,-3.61204013377926)(350.963210702341,-3.55183946488294)(350.963210702341,-3.49163879598662)(350.963210702341,-3.4314381270903)(355.90635451505,-3.37123745819398)(355.90635451505,-3.31103678929766)(355.90635451505,-3.25083612040134)(355.90635451505,-3.19063545150502)(355.90635451505,-3.1304347826087)(360.849498327759,-3.07023411371237)(360.849498327759,-3.01003344481605)(360.849498327759,-2.94983277591973)(360.849498327759,-2.88963210702341)(360.849498327759,-2.82943143812709)(365.792642140468,-2.76923076923077)(365.792642140468,-2.70903010033445)(365.792642140468,-2.64882943143813)(365.792642140468,-2.58862876254181)(365.792642140468,-2.52842809364548)(370.735785953177,-2.46822742474916)(370.735785953177,-2.40802675585284)(370.735785953177,-2.34782608695652)(370.735785953177,-2.2876254180602)(370.735785953177,-2.22742474916388)(370.735785953177,-2.16722408026756)(370.735785953177,-2.10702341137124)(375.678929765886,-2.04682274247492)(375.678929765886,-1.98662207357859)(375.678929765886,-1.92642140468227)(375.678929765886,-1.86622073578595)(375.678929765886,-1.80602006688963)(375.678929765886,-1.74581939799331)(375.678929765886,-1.68561872909699)(375.678929765886,-1.62541806020067)(380.622073578595,-1.56521739130435)(380.622073578595,-1.50501672240803)(380.622073578595,-1.4448160535117)(380.622073578595,-1.38461538461538)(380.622073578595,-1.32441471571906)(380.622073578595,-1.26421404682274)(380.622073578595,-1.20401337792642)(380.622073578595,-1.1438127090301)(380.622073578595,-1.08361204013378)(380.622073578595,-1.02341137123746)(380.622073578595,-0.963210702341136)(380.622073578595,-0.903010033444815)(380.622073578595,-0.842809364548494)(380.622073578595,-0.782608695652176)(380.622073578595,-0.722408026755854)(380.622073578595,-0.662207357859533)(380.622073578595,-0.602006688963211)(380.622073578595,-0.54180602006689)(375.678929765886,-0.481605351170568)(375.678929765886,-0.421404682274247)(375.678929765886,-0.361204013377925)(375.678929765886,-0.301003344481604)(375.678929765886,-0.240802675585286)(375.678929765886,-0.180602006688964)(375.678929765886,-0.120401337792643)(370.735785953177,-0.0602006688963215)(370.735785953177,0)(375.678929765886,0)(380.622073578595,0)(385.565217391304,0)(390.508361204013,0)(395.451505016722,0)(400.394648829431,0)(405.33779264214,0)(410.28093645485,0)(415.224080267559,0)(420.167224080268,0)(425.110367892977,0)(430.053511705686,0)(434.996655518395,0)(439.939799331104,0)(444.882943143813,0)(449.826086956522,0)(454.769230769231,0)(459.71237458194,0)(464.655518394649,0)(469.598662207358,0)(474.541806020067,0)(479.484949832776,0)(484.428093645485,0)(489.371237458194,0)(494.314381270903,0)(499.257525083612,0)(504.200668896321,0)(509.14381270903,0)(514.086956521739,0)(519.030100334448,0)(523.973244147157,0)(528.916387959866,0)(533.859531772575,0)(538.802675585284,0)(543.745819397993,0)(548.688963210702,0)(553.632107023411,0)(558.57525083612,0)(563.518394648829,0)(568.461538461538,0)(573.404682274248,0)(578.347826086957,0)(583.290969899666,0)(588.234113712375,0)(593.177257525084,0)(598.120401337793,0)(603.063545150502,0)(608.006688963211,0)(612.94983277592,0)(617.892976588629,0)(622.836120401338,0)(627.779264214047,0)(632.722408026756,0)(637.665551839465,0)(642.608695652174,0)(647.551839464883,0)(652.494983277592,0)(657.438127090301,0)(662.38127090301,0)(667.324414715719,0)(672.267558528428,0)(677.210702341137,0)(682.153846153846,0)(687.096989966555,0)(692.040133779264,0)(696.983277591973,0)(701.926421404682,0)(706.869565217391,0)(711.8127090301,0)(716.755852842809,0)(721.698996655518,0)(726.642140468227,0)(731.585284280936,0)(736.528428093646,0)(741.471571906354,0)(746.414715719064,0)(751.357859531773,0)(756.301003344482,0)(761.244147157191,0)(766.1872909699,0)(771.130434782609,0)(776.073578595318,0)(781.016722408027,0)(785.959866220736,0)(790.903010033445,0)(795.846153846154,0)(800.789297658863,0)(805.732441471572,0)(810.675585284281,0)(815.61872909699,0)(820.561872909699,0)(825.505016722408,0)(830.448160535117,0)(835.391304347826,0)(840.334448160535,0)(845.277591973244,0)(850.220735785953,0)(855.163879598662,0)(860.107023411371,0)(865.05016722408,0)(869.993311036789,0)(874.936454849498,0)(879.879598662207,0)(884.822742474916,0)(889.765886287625,0)(894.709030100334,0)(899.652173913044,0)(904.595317725752,0)(909.538461538462,0)(914.481605351171,0)(919.42474916388,0)(924.367892976589,0)(929.311036789298,0)(934.254180602007,0)(939.197324414716,0)(944.140468227425,0)(949.083612040134,0)(954.026755852843,0)(958.969899665552,0)(963.913043478261,0)(968.85618729097,0)(973.799331103679,0)(978.742474916388,0)(983.685618729097,0)(988.628762541806,0)(993.571906354515,0)(998.515050167224,0)(1003.45819397993,0)(1008.40133779264,0)(1013.34448160535,0)(1018.28762541806,0)(1023.23076923077,0)(1028.17391304348,0)(1033.11705685619,0)(1038.0602006689,0)(1043.00334448161,0)(1047.94648829431,0)(1052.88963210702,0)(1057.83277591973,0)(1062.77591973244,0)(1067.71906354515,0)(1072.66220735786,0)(1077.60535117057,0)(1082.54849498328,0)(1087.49163879599,0)(1092.4347826087,0)(1097.3779264214,0)(1102.32107023411,0)(1107.26421404682,0)(1112.20735785953,0)(1117.15050167224,0)(1122.09364548495,0)(1127.03678929766,0)(1131.97993311037,0)(1136.92307692308,0)(1141.86622073579,0)(1146.8093645485,0)(1151.7525083612,0)(1156.69565217391,0)(1161.63879598662,0)(1166.58193979933,0)(1171.52508361204,0)(1176.46822742475,0)(1181.41137123746,0)(1186.35451505017,0)(1191.29765886288,0)(1196.24080267559,0)(1201.18394648829,0)(1206.127090301,0)(1211.07023411371,0)(1216.01337792642,0)(1220.95652173913,0)(1225.89966555184,0)(1230.84280936455,0)(1235.78595317726,0)(1240.72909698997,0)(1245.67224080268,0)(1250.61538461538,0)(1255.55852842809,0)(1260.5016722408,0)(1265.44481605351,0)(1270.38795986622,0)(1275.33110367893,0)(1280.27424749164,0)(1285.21739130435,0)(1290.16053511706,0)(1295.10367892977,0)(1300.04682274247,0)(1304.98996655518,0)(1309.93311036789,0)(1314.8762541806,0)(1319.81939799331,0)(1324.76254180602,0)(1329.70568561873,0)(1334.64882943144,0)(1339.59197324415,0)(1344.53511705686,0)(1349.47826086957,0)(1354.42140468227,0)(1359.36454849498,0)(1364.30769230769,0)(1369.2508361204,0)(1374.19397993311,0)(1379.13712374582,0)(1384.08026755853,0)(1389.02341137124,0)(1393.96655518395,0)(1398.90969899666,0)(1403.85284280936,0)(1408.79598662207,0)(1413.73913043478,0)(1418.68227424749,0)(1423.6254180602,0)(1428.56856187291,0)(1433.51170568562,0)(1438.45484949833,0)(1443.39799331104,0)(1448.34113712375,0)(1453.28428093645,0)(1458.22742474916,0)(1463.17056856187,0)(1468.11371237458,0)(1473.05685618729,0)(1478,0)(1478,-0.0602006688963215)(1478,-0.120401337792643)(1478,-0.180602006688964)(1478,-0.240802675585286)(1478,-0.301003344481604)(1478,-0.361204013377925)(1478,-0.421404682274247)(1478,-0.481605351170568)(1478,-0.54180602006689)(1478,-0.602006688963211)(1478,-0.662207357859533)(1478,-0.722408026755854)(1478,-0.782608695652176)(1478,-0.842809364548494)(1478,-0.903010033444815)(1478,-0.963210702341136)(1478,-1.02341137123746)(1478,-1.08361204013378)(1478,-1.1438127090301)(1478,-1.20401337792642)(1478,-1.26421404682274)(1478,-1.32441471571906)(1478,-1.38461538461538)(1478,-1.4448160535117)(1478,-1.50501672240803)(1478,-1.56521739130435)(1478,-1.62541806020067)(1478,-1.68561872909699)(1478,-1.74581939799331)(1478,-1.80602006688963)(1478,-1.86622073578595)(1478,-1.92642140468227)(1478,-1.98662207357859)(1478,-2.04682274247492)(1478,-2.10702341137124)(1478,-2.16722408026756)(1478,-2.22742474916388)(1478,-2.2876254180602)(1478,-2.34782608695652)(1478,-2.40802675585284)(1478,-2.46822742474916)(1478,-2.52842809364548)(1478,-2.58862876254181)(1478,-2.64882943143813)(1478,-2.70903010033445)(1478,-2.76923076923077)(1478,-2.82943143812709)(1478,-2.88963210702341)(1478,-2.94983277591973)(1478,-3.01003344481605)(1478,-3.07023411371237)(1478,-3.1304347826087)(1478,-3.19063545150502)(1478,-3.25083612040134)(1478,-3.31103678929766)(1478,-3.37123745819398)(1478,-3.4314381270903)(1478,-3.49163879598662)(1478,-3.55183946488294)(1478,-3.61204013377926)(1478,-3.67224080267559)(1478,-3.73244147157191)(1478,-3.79264214046823)(1478,-3.85284280936455)(1478,-3.91304347826087)(1478,-3.97324414715719)(1478,-4.03344481605351)(1478,-4.09364548494983)(1478,-4.15384615384615)(1478,-4.21404682274247)(1478,-4.2742474916388)(1478,-4.33444816053512)(1478,-4.39464882943144)(1478,-4.45484949832776)(1478,-4.51505016722408)(1478,-4.5752508361204)(1478,-4.63545150501672)(1478,-4.69565217391304)(1478,-4.75585284280936)(1478,-4.81605351170569)(1478,-4.87625418060201)(1478,-4.93645484949833)(1478,-4.99665551839465)(1478,-5.05685618729097)(1478,-5.11705685618729)(1478,-5.17725752508361)(1478,-5.23745819397993)(1478,-5.29765886287625)(1478,-5.35785953177258)(1478,-5.4180602006689)(1478,-5.47826086956522)(1478,-5.53846153846154)(1478,-5.59866220735786)(1478,-5.65886287625418)(1478,-5.7190635451505)(1478,-5.77926421404682)(1478,-5.83946488294314)(1478,-5.89966555183947)(1478,-5.95986622073579)(1478,-6.02006688963211)(1478,-6.08026755852843)(1478,-6.14046822742475)(1478,-6.20066889632107)(1478,-6.26086956521739)(1478,-6.32107023411371)(1478,-6.38127090301003)(1478,-6.44147157190636)(1478,-6.50167224080267)(1478,-6.561872909699)(1478,-6.62207357859532)(1478,-6.68227424749164)(1478,-6.74247491638796)(1478,-6.80267558528428)(1478,-6.8628762541806)(1478,-6.92307692307692)(1478,-6.98327759197324)(1478,-7.04347826086956)(1478,-7.10367892976589)(1478,-7.16387959866221)(1478,-7.22408026755853)(1478,-7.28428093645485)(1478,-7.34448160535117)(1478,-7.40468227424749)(1478,-7.46488294314381)(1478,-7.52508361204013)(1478,-7.58528428093645)(1478,-7.64548494983278)(1478,-7.7056856187291)(1478,-7.76588628762542)(1478,-7.82608695652174)(1478,-7.88628762541806)(1478,-7.94648829431438)(1478,-8.0066889632107)(1478,-8.06688963210702)(1478,-8.12709030100334)(1478,-8.18729096989967)(1478,-8.24749163879599)(1478,-8.30769230769231)(1478,-8.36789297658863)(1478,-8.42809364548495)(1478,-8.48829431438127)(1478,-8.54849498327759)(1478,-8.60869565217391)(1478,-8.66889632107023)(1478,-8.72909698996656)(1478,-8.78929765886288)(1478,-8.8494983277592)(1478,-8.90969899665552)(1478,-8.96989966555184)(1478,-9.03010033444816)(1478,-9.09030100334448)(1478,-9.1505016722408)(1478,-9.21070234113712)(1478,-9.27090301003344)(1478,-9.33110367892977)(1478,-9.39130434782609)(1478,-9.45150501672241)(1478,-9.51170568561873)(1478,-9.57190635451505)(1478,-9.63210702341137)(1478,-9.69230769230769)(1478,-9.75250836120401)(1478,-9.81270903010033)(1478,-9.87290969899666)(1478,-9.93311036789298)(1478,-9.9933110367893)(1478,-10.0535117056856)(1478,-10.1137123745819)(1478,-10.1739130434783)(1478,-10.2341137123746)(1478,-10.2943143812709)(1478,-10.3545150501672)(1478,-10.4147157190635)(1478,-10.4749163879599)(1478,-10.5351170568562)(1478,-10.5953177257525)(1478,-10.6555183946488)(1478,-10.7157190635452)(1478,-10.7759197324415)(1478,-10.8361204013378)(1478,-10.8963210702341)(1478,-10.9565217391304)(1478,-11.0167224080268)(1478,-11.0769230769231)(1478,-11.1371237458194)(1478,-11.1973244147157)(1478,-11.257525083612)(1478,-11.3177257525084)(1478,-11.3779264214047)(1478,-11.438127090301)(1478,-11.4983277591973)(1478,-11.5585284280936)(1478,-11.61872909699)(1478,-11.6789297658863)(1478,-11.7391304347826)(1478,-11.7993311036789)(1478,-11.8595317725753)(1478,-11.9197324414716)(1478,-11.9799331103679)(1478,-12.0401337792642)(1478,-12.1003344481605)(1478,-12.1605351170569)(1478,-12.2207357859532)(1478,-12.2809364548495)(1478,-12.3411371237458)(1478,-12.4013377926421)(1478,-12.4615384615385)(1478,-12.5217391304348)(1478,-12.5819397993311)(1478,-12.6421404682274)(1478,-12.7023411371237)(1478,-12.7625418060201)(1478,-12.8227424749164)(1478,-12.8829431438127)(1478,-12.943143812709)(1478,-13.0033444816054)(1478,-13.0635451505017)(1478,-13.123745819398)(1478,-13.1839464882943)(1478,-13.2441471571906)(1478,-13.304347826087)(1478,-13.3645484949833)(1478,-13.4247491638796)(1478,-13.4849498327759)(1478,-13.5451505016722)(1478,-13.6053511705686)(1478,-13.6655518394649)(1478,-13.7257525083612)(1478,-13.7859531772575)(1478,-13.8461538461538)(1478,-13.9063545150502)(1478,-13.9665551839465)(1478,-14.0267558528428)(1478,-14.0869565217391)(1478,-14.1471571906355)(1478,-14.2073578595318)(1478,-14.2675585284281)(1478,-14.3277591973244)(1478,-14.3879598662207)(1478,-14.4481605351171)(1478,-14.5083612040134)(1478,-14.5685618729097)(1478,-14.628762541806)(1478,-14.6889632107023)(1478,-14.7491638795987)(1478,-14.809364548495)(1478,-14.8695652173913)(1478,-14.9297658862876)(1478,-14.9899665551839)(1478,-15.0501672240803)(1478,-15.1103678929766)(1478,-15.1705685618729)(1478,-15.2307692307692)(1478,-15.2909698996656)(1478,-15.3511705685619)(1478,-15.4113712374582)(1478,-15.4715719063545)(1478,-15.5317725752508)(1478,-15.5919732441472)(1478,-15.6521739130435)(1478,-15.7123745819398)(1478,-15.7725752508361)(1478,-15.8327759197324)(1478,-15.8929765886288)(1478,-15.9531772575251)(1478,-16.0133779264214)(1478,-16.0735785953177)(1478,-16.133779264214)(1478,-16.1939799331104)(1478,-16.2541806020067)(1478,-16.314381270903)(1478,-16.3745819397993)(1478,-16.4347826086957)(1478,-16.494983277592)(1478,-16.5551839464883)(1478,-16.6153846153846)(1478,-16.6755852842809)(1478,-16.7357859531773)(1478,-16.7959866220736)(1478,-16.8561872909699)(1478,-16.9163879598662)(1478,-16.9765886287625)(1478,-17.0367892976589)(1478,-17.0969899665552)(1478,-17.1571906354515)(1478,-17.2173913043478)(1478,-17.2775919732441)(1478,-17.3377926421405)(1478,-17.3979933110368)(1478,-17.4581939799331)(1478,-17.5183946488294)(1478,-17.5785953177258)(1478,-17.6387959866221)(1478,-17.6989966555184)(1478,-17.7591973244147)(1478,-17.819397993311)(1478,-17.8795986622074)(1478,-17.9397993311037)(1478,-18)(1473.05685618729,-18)(1468.11371237458,-18)(1463.17056856187,-18)(1458.22742474916,-18)(1453.28428093645,-18)(1448.34113712375,-18)(1443.39799331104,-18)(1438.45484949833,-18)(1433.51170568562,-18)(1428.56856187291,-18)(1423.6254180602,-18)(1418.68227424749,-18)(1413.73913043478,-18)(1408.79598662207,-18)(1403.85284280936,-18)(1398.90969899666,-18)(1393.96655518395,-18)(1389.02341137124,-18)(1384.08026755853,-18)(1379.13712374582,-18)(1374.19397993311,-18)(1369.2508361204,-18)(1364.30769230769,-18)(1359.36454849498,-18)(1354.42140468227,-18)(1349.47826086957,-18)(1344.53511705686,-18)(1339.59197324415,-18)(1334.64882943144,-18)(1329.70568561873,-18)(1324.76254180602,-18)(1319.81939799331,-18)(1314.8762541806,-18)(1309.93311036789,-18)(1304.98996655518,-18)(1300.04682274247,-18)(1295.10367892977,-18)(1290.16053511706,-18)(1285.21739130435,-18)(1280.27424749164,-18)(1275.33110367893,-18)(1270.38795986622,-18)(1265.44481605351,-18)(1260.5016722408,-18)(1255.55852842809,-18)(1250.61538461538,-18)(1245.67224080268,-18)(1240.72909698997,-18)(1235.78595317726,-18)(1230.84280936455,-18)(1225.89966555184,-18)(1220.95652173913,-18)(1216.01337792642,-18)(1211.07023411371,-18)(1206.127090301,-18)(1201.18394648829,-18)(1196.24080267559,-18)(1191.29765886288,-18)(1186.35451505017,-18)(1181.41137123746,-18)(1176.46822742475,-18)(1171.52508361204,-18)(1166.58193979933,-18)(1161.63879598662,-18)(1156.69565217391,-18)(1151.7525083612,-18)(1146.8093645485,-18)(1141.86622073579,-18)(1136.92307692308,-18)(1131.97993311037,-18)(1127.03678929766,-18)(1122.09364548495,-18)(1117.15050167224,-18)(1112.20735785953,-18)(1107.26421404682,-18)(1102.32107023411,-18)(1097.3779264214,-18)(1092.4347826087,-18)(1087.49163879599,-18)(1082.54849498328,-18)(1077.60535117057,-18)(1072.66220735786,-18)(1067.71906354515,-18)(1062.77591973244,-18)(1057.83277591973,-18)(1052.88963210702,-18)(1047.94648829431,-18)(1043.00334448161,-18)(1038.0602006689,-18)(1033.11705685619,-18)(1028.17391304348,-18)(1023.23076923077,-18)(1018.28762541806,-18)(1013.34448160535,-18)(1008.40133779264,-18)(1003.45819397993,-18)(998.515050167224,-18)(993.571906354515,-18)(988.628762541806,-18)(983.685618729097,-18)(978.742474916388,-18)(973.799331103679,-18)(968.85618729097,-18)(963.913043478261,-18)(958.969899665552,-18)(954.026755852843,-18)(949.083612040134,-18)(944.140468227425,-18)(939.197324414716,-18)(934.254180602007,-18)(929.311036789298,-18)(924.367892976589,-18)(919.42474916388,-18)(914.481605351171,-18)(909.538461538462,-18)(904.595317725752,-18)(899.652173913044,-18)(894.709030100334,-18)(889.765886287625,-18)(884.822742474916,-18)(879.879598662207,-18)(874.936454849498,-18)(869.993311036789,-18)(865.05016722408,-18)(860.107023411371,-18)(855.163879598662,-18)(850.220735785953,-18)(845.277591973244,-18)(840.334448160535,-18)(835.391304347826,-18)(830.448160535117,-18)(825.505016722408,-18)(820.561872909699,-18)(815.61872909699,-18)(810.675585284281,-18)(805.732441471572,-18)(800.789297658863,-18)(795.846153846154,-18)(790.903010033445,-18)(785.959866220736,-18)(781.016722408027,-18)(776.073578595318,-18)(771.130434782609,-18)(766.1872909699,-18)(761.244147157191,-18)(756.301003344482,-18)(751.357859531773,-18)(746.414715719064,-18)(741.471571906354,-18)(736.528428093646,-18)(731.585284280936,-18)(726.642140468227,-18)(721.698996655518,-18)(716.755852842809,-18)(711.8127090301,-18)(706.869565217391,-18)(701.926421404682,-18)(696.983277591973,-18)(692.040133779264,-18)(687.096989966555,-18)(682.153846153846,-18)(677.210702341137,-18)(672.267558528428,-18)(667.324414715719,-18)(662.38127090301,-18)(657.438127090301,-18)(652.494983277592,-18)(647.551839464883,-18)(642.608695652174,-18)(637.665551839465,-18)(632.722408026756,-18)(627.779264214047,-18)(622.836120401338,-18)(617.892976588629,-18)(612.94983277592,-18)(608.006688963211,-18)(603.063545150502,-18)(598.120401337793,-18)(593.177257525084,-18)(588.234113712375,-18)(583.290969899666,-18)(578.347826086957,-18)(573.404682274248,-18)(568.461538461538,-18)(573.404682274248,-17.9397993311037)(573.404682274248,-17.8795986622074)(573.404682274248,-17.819397993311)(573.404682274248,-17.7591973244147)(573.404682274248,-17.6989966555184)(573.404682274248,-17.6387959866221)(573.404682274248,-17.5785953177258)(578.347826086957,-17.5183946488294)(578.347826086957,-17.4581939799331)(578.347826086957,-17.3979933110368)(578.347826086957,-17.3377926421405)(578.347826086957,-17.2775919732441)(578.347826086957,-17.2173913043478)(578.347826086957,-17.1571906354515)(583.290969899666,-17.0969899665552)(583.290969899666,-17.0367892976589)(583.290969899666,-16.9765886287625)(583.290969899666,-16.9163879598662)(583.290969899666,-16.8561872909699)(583.290969899666,-16.7959866220736)(583.290969899666,-16.7357859531773)(588.234113712375,-16.6755852842809)(588.234113712375,-16.6153846153846)(588.234113712375,-16.5551839464883)(588.234113712375,-16.494983277592)(588.234113712375,-16.4347826086957)(588.234113712375,-16.3745819397993)(588.234113712375,-16.314381270903)(588.234113712375,-16.2541806020067)(588.234113712375,-16.1939799331104)(593.177257525084,-16.133779264214)(593.177257525084,-16.0735785953177)(593.177257525084,-16.0133779264214)(593.177257525084,-15.9531772575251)(593.177257525084,-15.8929765886288)(593.177257525084,-15.8327759197324)(593.177257525084,-15.7725752508361)(593.177257525084,-15.7123745819398)(593.177257525084,-15.6521739130435)(593.177257525084,-15.5919732441472)(593.177257525084,-15.5317725752508)(593.177257525084,-15.4715719063545)(593.177257525084,-15.4113712374582)(593.177257525084,-15.3511705685619)(593.177257525084,-15.2909698996656)(593.177257525084,-15.2307692307692)(593.177257525084,-15.1705685618729)(593.177257525084,-15.1103678929766)(593.177257525084,-15.0501672240803)(593.177257525084,-14.9899665551839)(593.177257525084,-14.9297658862876)(593.177257525084,-14.8695652173913)(593.177257525084,-14.809364548495)(593.177257525084,-14.7491638795987)(593.177257525084,-14.6889632107023)(593.177257525084,-14.628762541806)(593.177257525084,-14.5685618729097)(593.177257525084,-14.5083612040134)(593.177257525084,-14.4481605351171)(593.177257525084,-14.3879598662207)(593.177257525084,-14.3277591973244)(593.177257525084,-14.2675585284281)(593.177257525084,-14.2073578595318)(593.177257525084,-14.1471571906355)(593.177257525084,-14.0869565217391)(588.234113712375,-14.0267558528428)(588.234113712375,-13.9665551839465)(588.234113712375,-13.9063545150502)(588.234113712375,-13.8461538461538)(588.234113712375,-13.7859531772575)(588.234113712375,-13.7257525083612)(588.234113712375,-13.6655518394649)(588.234113712375,-13.6053511705686)(583.290969899666,-13.5451505016722)(583.290969899666,-13.4849498327759)(583.290969899666,-13.4247491638796)(583.290969899666,-13.3645484949833)(583.290969899666,-13.304347826087)(583.290969899666,-13.2441471571906)(578.347826086957,-13.1839464882943)(578.347826086957,-13.123745819398)(578.347826086957,-13.0635451505017)(578.347826086957,-13.0033444816054)(573.404682274248,-12.943143812709)(573.404682274248,-12.8829431438127)(573.404682274248,-12.8227424749164)(573.404682274248,-12.7625418060201)(568.461538461538,-12.7023411371237)(568.461538461538,-12.6421404682274)(568.461538461538,-12.5819397993311)(563.518394648829,-12.5217391304348)(563.518394648829,-12.4615384615385)(563.518394648829,-12.4013377926421)(558.57525083612,-12.3411371237458)(558.57525083612,-12.2809364548495)(558.57525083612,-12.2207357859532)(553.632107023411,-12.1605351170569)(553.632107023411,-12.1003344481605)(553.632107023411,-12.0401337792642)(548.688963210702,-11.9799331103679)(548.688963210702,-11.9197324414716)(543.745819397993,-11.8595317725753)(543.745819397993,-11.7993311036789)(543.745819397993,-11.7391304347826)(538.802675585284,-11.6789297658863)(538.802675585284,-11.61872909699)(533.859531772575,-11.5585284280936)(533.859531772575,-11.4983277591973)(533.859531772575,-11.438127090301)(528.916387959866,-11.3779264214047)(528.916387959866,-11.3177257525084)(528.916387959866,-11.257525083612)(523.973244147157,-11.1973244147157)(523.973244147157,-11.1371237458194)(523.973244147157,-11.0769230769231)(519.030100334448,-11.0167224080268)(519.030100334448,-10.9565217391304)(519.030100334448,-10.8963210702341)(514.086956521739,-10.8361204013378)(514.086956521739,-10.7759197324415)(514.086956521739,-10.7157190635452)(509.14381270903,-10.6555183946488)(509.14381270903,-10.5953177257525)(509.14381270903,-10.5351170568562)(504.200668896321,-10.4749163879599)(504.200668896321,-10.4147157190635)(504.200668896321,-10.3545150501672)(499.257525083612,-10.2943143812709)(499.257525083612,-10.2341137123746)(499.257525083612,-10.1739130434783)(499.257525083612,-10.1137123745819)(494.314381270903,-10.0535117056856)(494.314381270903,-9.9933110367893)(494.314381270903,-9.93311036789298)(489.371237458194,-9.87290969899666)(489.371237458194,-9.81270903010033)(484.428093645485,-9.75250836120401)(484.428093645485,-9.69230769230769)(479.484949832776,-9.63210702341137)(479.484949832776,-9.57190635451505)(474.541806020067,-9.51170568561873)(469.598662207358,-9.45150501672241)(464.655518394649,-9.39130434782609)(459.71237458194,-9.33110367892977)(454.769230769231,-9.27090301003344)(449.826086956522,-9.21070234113712)(444.882943143813,-9.21070234113712)(439.939799331104,-9.21070234113712)(434.996655518395,-9.21070234113712)(430.053511705686,-9.1505016722408)(425.110367892977,-9.1505016722408)(420.167224080268,-9.21070234113712)(415.224080267559,-9.21070234113712)(410.28093645485,-9.21070234113712)(405.33779264214,-9.21070234113712)(400.394648829431,-9.27090301003344)(395.451505016722,-9.27090301003344)(390.508361204013,-9.33110367892977)(385.565217391304,-9.33110367892977)(380.622073578595,-9.39130434782609)(375.678929765886,-9.39130434782609)(370.735785953177,-9.45150501672241)(365.792642140468,-9.45150501672241)(360.849498327759,-9.51170568561873)(355.90635451505,-9.51170568561873)(350.963210702341,-9.51170568561873)(346.020066889632,-9.57190635451505)(341.076923076923,-9.57190635451505)(336.133779264214,-9.57190635451505)(331.190635451505,-9.57190635451505)(326.247491638796,-9.57190635451505)(321.304347826087,-9.57190635451505)(316.361204013378,-9.51170568561873)(311.418060200669,-9.51170568561873)(306.47491638796,-9.51170568561873)(301.531772575251,-9.51170568561873)(296.588628762542,-9.45150501672241)(291.645484949833,-9.45150501672241)(286.702341137124,-9.39130434782609)(281.759197324415,-9.39130434782609)(276.816053511706,-9.39130434782609)(271.872909698997,-9.33110367892977)(266.929765886288,-9.33110367892977)(261.986622073579,-9.27090301003344)(257.04347826087,-9.27090301003344)(252.100334448161,-9.21070234113712)(247.157190635452,-9.21070234113712)(242.214046822742,-9.1505016722408)(237.270903010033,-9.1505016722408)(232.327759197324,-9.1505016722408)(227.384615384615,-9.1505016722408)(222.441471571906,-9.1505016722408)(217.498327759197,-9.21070234113712)(212.555183946488,-9.21070234113712)(207.612040133779,-9.27090301003344)(202.66889632107,-9.27090301003344)(197.725752508361,-9.33110367892977)(192.782608695652,-9.39130434782609)(187.839464882943,-9.45150501672241)(182.896321070234,-9.45150501672241)(177.953177257525,-9.51170568561873)(173.010033444816,-9.57190635451505)(168.066889632107,-9.63210702341137)(163.123745819398,-9.69230769230769)(158.180602006689,-9.75250836120401)(153.23745819398,-9.81270903010033)(148.294314381271,-9.87290969899666)(143.351170568562,-9.93311036789298)(138.408026755853,-9.9933110367893)(133.464882943144,-10.0535117056856)(128.521739130435,-10.1137123745819)(123.578595317726,-10.1739130434783)(118.635451505017,-10.2341137123746)(118.635451505017,-10.2943143812709)(113.692307692308,-10.3545150501672)(108.749163879599,-10.4147157190635)(108.749163879599,-10.4749163879599)(103.80602006689,-10.5351170568562)(98.8628762541806,-10.5953177257525)(98.8628762541806,-10.6555183946488)(93.9197324414716,-10.7157190635452)(88.9765886287625,-10.7759197324415)(88.9765886287625,-10.8361204013378)(84.0334448160535,-10.8963210702341)(84.0334448160535,-10.9565217391304)(79.0903010033445,-11.0167224080268)(79.0903010033445,-11.0769230769231)(79.0903010033445,-11.1371237458194)(74.1471571906355,-11.1973244147157)(74.1471571906355,-11.257525083612)(69.2040133779264,-11.3177257525084)(69.2040133779264,-11.3779264214047)(69.2040133779264,-11.438127090301)(64.2608695652174,-11.4983277591973)(64.2608695652174,-11.5585284280936)(64.2608695652174,-11.61872909699)(64.2608695652174,-11.6789297658863)(59.3177257525084,-11.7391304347826)(59.3177257525084,-11.7993311036789)(59.3177257525084,-11.8595317725753)(59.3177257525084,-11.9197324414716)(59.3177257525084,-11.9799331103679)(59.3177257525084,-12.0401337792642)(54.3745819397993,-12.1003344481605)(54.3745819397993,-12.1605351170569)(54.3745819397993,-12.2207357859532)(54.3745819397993,-12.2809364548495)(54.3745819397993,-12.3411371237458)(54.3745819397993,-12.4013377926421)(54.3745819397993,-12.4615384615385)(54.3745819397993,-12.5217391304348)(54.3745819397993,-12.5819397993311)(54.3745819397993,-12.6421404682274)(54.3745819397993,-12.7023411371237)(54.3745819397993,-12.7625418060201)(54.3745819397993,-12.8227424749164)(54.3745819397993,-12.8829431438127)(54.3745819397993,-12.943143812709)(54.3745819397993,-13.0033444816054)(59.3177257525084,-13.0635451505017)(59.3177257525084,-13.123745819398)(59.3177257525084,-13.1839464882943)(59.3177257525084,-13.2441471571906)(59.3177257525084,-13.304347826087)(59.3177257525084,-13.3645484949833)(59.3177257525084,-13.4247491638796)(59.3177257525084,-13.4849498327759)(59.3177257525084,-13.5451505016722)(59.3177257525084,-13.6053511705686)(59.3177257525084,-13.6655518394649)(64.2608695652174,-13.7257525083612)(64.2608695652174,-13.7859531772575)(64.2608695652174,-13.8461538461538)(64.2608695652174,-13.9063545150502)(64.2608695652174,-13.9665551839465)(64.2608695652174,-14.0267558528428)(64.2608695652174,-14.0869565217391)(64.2608695652174,-14.1471571906355)(64.2608695652174,-14.2073578595318)(64.2608695652174,-14.2675585284281)(59.3177257525084,-14.3277591973244)(59.3177257525084,-14.3879598662207)(59.3177257525084,-14.4481605351171)(59.3177257525084,-14.5083612040134)(59.3177257525084,-14.5685618729097)(59.3177257525084,-14.628762541806)(59.3177257525084,-14.6889632107023)(54.3745819397993,-14.7491638795987)(54.3745819397993,-14.809364548495)(54.3745819397993,-14.8695652173913)(54.3745819397993,-14.9297658862876)(54.3745819397993,-14.9899665551839)(49.4314381270903,-15.0501672240803)(49.4314381270903,-15.1103678929766)(49.4314381270903,-15.1705685618729)(49.4314381270903,-15.2307692307692)(49.4314381270903,-15.2909698996656)(44.4882943143813,-15.3511705685619)(44.4882943143813,-15.4113712374582)(44.4882943143813,-15.4715719063545)(44.4882943143813,-15.5317725752508)(44.4882943143813,-15.5919732441472)(39.5451505016722,-15.6521739130435)(39.5451505016722,-15.7123745819398)(39.5451505016722,-15.7725752508361)(39.5451505016722,-15.8327759197324)(39.5451505016722,-15.8929765886288)(39.5451505016722,-15.9531772575251)(34.6020066889632,-16.0133779264214)(34.6020066889632,-16.0735785953177)(34.6020066889632,-16.133779264214)(34.6020066889632,-16.1939799331104)(34.6020066889632,-16.2541806020067)(34.6020066889632,-16.314381270903)(34.6020066889632,-16.3745819397993)(34.6020066889632,-16.4347826086957)(34.6020066889632,-16.494983277592)(34.6020066889632,-16.5551839464883)(34.6020066889632,-16.6153846153846)(34.6020066889632,-16.6755852842809)(34.6020066889632,-16.7357859531773)(34.6020066889632,-16.7959866220736)(34.6020066889632,-16.8561872909699)(34.6020066889632,-16.9163879598662)(34.6020066889632,-16.9765886287625)(34.6020066889632,-17.0367892976589)(34.6020066889632,-17.0969899665552)(34.6020066889632,-17.1571906354515)(34.6020066889632,-17.2173913043478)(34.6020066889632,-17.2775919732441)(34.6020066889632,-17.3377926421405)(34.6020066889632,-17.3979933110368)(34.6020066889632,-17.4581939799331)(34.6020066889632,-17.5183946488294)(34.6020066889632,-17.5785953177258)(34.6020066889632,-17.6387959866221)(39.5451505016722,-17.6989966555184)(39.5451505016722,-17.7591973244147)(39.5451505016722,-17.819397993311)(39.5451505016722,-17.8795986622074)(39.5451505016722,-17.9397993311037)(39.5451505016722,-18)(34.6020066889632,-18)(29.6588628762542,-18)(24.7157190635452,-18)(19.7725752508361,-18)(14.8294314381271,-18)(9.88628762541806,-18)(4.94314381270903,-18)(0,-18.0000030097325)(-0.000247132477387754,-18)(0,-17.9397993311037)(0,-17.8795986622074)(0,-17.819397993311)(0,-17.7591973244147)(0,-17.6989966555184)(0,-17.6387959866221)(0,-17.5785953177258)(0,-17.5183946488294)(0,-17.4581939799331)(0,-17.3979933110368)(0,-17.3377926421405)(0,-17.2775919732441)(0,-17.2173913043478)(0,-17.1571906354515)(0,-17.0969899665552)(0,-17.0367892976589)(0,-16.9765886287625)(0,-16.9163879598662)(0,-16.8561872909699)(0,-16.7959866220736)(0,-16.7357859531773)(0,-16.6755852842809)(0,-16.6153846153846)(0,-16.5551839464883)(0,-16.494983277592)(0,-16.4347826086957)(0,-16.3745819397993)(0,-16.314381270903)(0,-16.2541806020067)(0,-16.1939799331104)(0,-16.133779264214)(0,-16.0735785953177)(0,-16.0133779264214)(0,-15.9531772575251)(0,-15.8929765886288)(0,-15.8327759197324)(0,-15.7725752508361)(0,-15.7123745819398)(0,-15.6521739130435)(0,-15.5919732441472)(0,-15.5317725752508)(0,-15.4715719063545)(0,-15.4113712374582)(0,-15.3511705685619)(0,-15.2909698996656)(0,-15.2307692307692)(0,-15.1705685618729)(0,-15.1103678929766)(0,-15.0501672240803)(0,-14.9899665551839)(0,-14.9297658862876)(0,-14.8695652173913)(0,-14.809364548495)(0,-14.7491638795987)(0,-14.6889632107023)(0,-14.628762541806)(0,-14.5685618729097)(0,-14.5083612040134)(0,-14.4481605351171)(0,-14.3879598662207)(0,-14.3277591973244)(0,-14.2675585284281)(0,-14.2073578595318)(0,-14.1471571906355)(0,-14.0869565217391)(0,-14.0267558528428)(0,-13.9665551839465)(0,-13.9063545150502)(0,-13.8461538461538)(0,-13.7859531772575)(0,-13.7257525083612)(0,-13.6655518394649)(0,-13.6053511705686)(0,-13.5451505016722)(0,-13.4849498327759)(0,-13.4247491638796)(0,-13.3645484949833)(0,-13.304347826087)(0,-13.2441471571906)(0,-13.1839464882943)(0,-13.123745819398)(0,-13.0635451505017)(0,-13.0033444816054)(0,-12.943143812709)(0,-12.8829431438127)(0,-12.8227424749164)(0,-12.7625418060201)(0,-12.7023411371237)(0,-12.6421404682274)(0,-12.5819397993311)(0,-12.5217391304348)(0,-12.4615384615385)(0,-12.4013377926421)(0,-12.3411371237458)(0,-12.2809364548495)(0,-12.2207357859532)(0,-12.1605351170569)(0,-12.1003344481605)(0,-12.0401337792642)(0,-11.9799331103679)(0,-11.9197324414716)(0,-11.8595317725753)(0,-11.7993311036789)(0,-11.7391304347826)(0,-11.6789297658863)(0,-11.61872909699)(0,-11.5585284280936)(0,-11.4983277591973)(0,-11.438127090301)(0,-11.3779264214047)(0,-11.3177257525084)(0,-11.257525083612)(0,-11.1973244147157)(0,-11.1371237458194)(0,-11.0769230769231)(0,-11.0167224080268)(0,-10.9565217391304)(0,-10.8963210702341)(0,-10.8361204013378)(0,-10.7759197324415)(0,-10.7157190635452)(0,-10.6555183946488)(0,-10.5953177257525)(0,-10.5351170568562)(0,-10.4749163879599)(0,-10.4147157190635)(0,-10.3545150501672)(0,-10.2943143812709)(0,-10.2341137123746)(0,-10.1739130434783)(0,-10.1137123745819)(0,-10.0535117056856)(0,-9.9933110367893)(0,-9.93311036789298)(0,-9.87290969899666)(0,-9.81270903010033)(0,-9.75250836120401)(0,-9.69230769230769)(0,-9.63210702341137)(0,-9.57190635451505)(0,-9.51170568561873)(0,-9.45150501672241)(0,-9.39130434782609)(0,-9.33110367892977)(0,-9.27090301003344)(0,-9.21070234113712)(0,-9.1505016722408)(0,-9.09030100334448)(0,-9.03010033444816)(0,-8.96989966555184)(0,-8.90969899665552)(0,-8.8494983277592)(0,-8.78929765886288)(0,-8.72909698996656)(0,-8.66889632107023)(0,-8.60869565217391)(0,-8.54849498327759)(0,-8.48829431438127)(0,-8.42809364548495)(0,-8.36789297658863)(0,-8.30769230769231)(0,-8.24749163879599)(0,-8.18729096989967)(0,-8.12709030100334)(0,-8.06688963210702)(0,-8.0066889632107)(0,-7.94648829431438)(0,-7.88628762541806)(0,-7.82608695652174)(0,-7.76588628762542)(0,-7.7056856187291)(0,-7.64548494983278)(0,-7.58528428093645)(0,-7.52508361204013)(0,-7.46488294314381)(0,-7.40468227424749)(0,-7.34448160535117)(0,-7.28428093645485)(0,-7.22408026755853)(0,-7.16387959866221)(0,-7.10367892976589)(0,-7.04347826086956)(0,-6.98327759197324)(0,-6.92307692307692)(0,-6.8628762541806)(0,-6.80267558528428)(0,-6.74247491638796)(0,-6.68227424749164)(0,-6.62207357859532)(0,-6.561872909699)(0,-6.50167224080267)(0,-6.44147157190636)(0,-6.38127090301003)(0,-6.32107023411371)(0,-6.26086956521739)(0,-6.20066889632107)(0,-6.14046822742475)(0,-6.08026755852843)(0,-6.02006688963211)(0,-5.95986622073579)(0,-5.89966555183947)(0,-5.83946488294314)(0,-5.77926421404682)(0,-5.7190635451505)(0,-5.65886287625418)(0,-5.59866220735786)(0,-5.53846153846154)(0,-5.47826086956522)(0,-5.4180602006689)(0,-5.35785953177258)(0,-5.29765886287625)(0,-5.23745819397993)(0,-5.17725752508361)(0,-5.11705685618729)(0,-5.05685618729097)(0,-4.99665551839465)(0,-4.93645484949833)(0,-4.87625418060201)(0,-4.81605351170569)(0,-4.75585284280936)(0,-4.69565217391304)(0,-4.63545150501672)(0,-4.5752508361204)(0,-4.51505016722408)};

\addplot [
color=white,
draw=white,
only marks,
mark=x,
mark options={solid},
mark size=2.0pt,
line width=0.3pt,
forget plot
]
coordinates{
 (10.5571428571429,0)(21.1142857142857,0)(31.6714285714286,0)(42.2285714285714,0)(52.7857142857143,0)(63.3428571428571,0)(73.9,0)(84.4571428571429,0)(95.0142857142857,0)(105.571428571429,0)(116.128571428571,-0.663157894736842)(126.685714285714,-1.32631578947368)(137.242857142857,-1.98947368421053)(147.8,-2.65263157894737)(158.357142857143,-3.31578947368421)(168.914285714286,-3.97894736842105)(179.471428571429,-4.64210526315789)(190.028571428571,-5.30526315789474)(200.585714285714,-5.96842105263158)(211.142857142857,-6.63157894736842)(221.7,-7.76842105263158)(232.257142857143,-8.90526315789474)(242.814285714286,-10.0421052631579)(253.371428571429,-11.1789473684211)(263.928571428571,-12.3157894736842)(274.485714285714,-13.4526315789474)(285.042857142857,-14.5894736842105)(295.6,-15.7263157894737)(306.157142857143,-16.8631578947368)(316.714285714286,-18)(327.271428571429,-17.1473684210526)(337.828571428571,-16.2947368421053)(348.385714285714,-15.4421052631579)(358.942857142857,-14.5894736842105)(369.5,-13.7368421052632)(380.057142857143,-12.8842105263158)(390.614285714286,-12.0315789473684)(401.171428571429,-11.1789473684211)(411.728571428571,-10.3263157894737)(422.285714285714,-9.47368421052632) 
};

\addplot [
color=red,
solid,
line width=1.0pt,
forget plot
]
coordinates{
 (422.285714285714,-9.47368421052632)(527.857142857143,0)(633.428571428571,-8.52631578947369)(739,-18)(844.571428571429,-16.1052631578947)(950.142857142857,-7.57894736842105)(1055.71428571429,0)(1161.28571428571,-9.47368421052632)(1266.85714285714,-18)(1372.42857142857,-10.4210526315789)(1478,0) 
};

\addplot [
mark size=0.8pt,
only marks,
mark=*,
mark options={solid,fill=black,draw=black},
forget plot
]
coordinates{
 (0,0)(0,-0.947368421052632)(0,-1.89473684210526)(0,-2.84210526315789)(0,-3.78947368421053)(0,-4.73684210526316)(0,-5.68421052631579)(0,-6.63157894736842)(0,-7.57894736842105)(0,-8.52631578947369)(0,-9.47368421052632)(0,-10.4210526315789)(0,-11.3684210526316)(0,-12.3157894736842)(0,-13.2631578947368)(0,-14.2105263157895)(0,-15.1578947368421)(0,-16.1052631578947)(0,-17.0526315789474)(0,-18)(105.571428571429,0)(105.571428571429,-0.947368421052632)(105.571428571429,-1.89473684210526)(105.571428571429,-2.84210526315789)(105.571428571429,-3.78947368421053)(105.571428571429,-4.73684210526316)(105.571428571429,-5.68421052631579)(105.571428571429,-6.63157894736842)(105.571428571429,-7.57894736842105)(105.571428571429,-8.52631578947369)(105.571428571429,-9.47368421052632)(105.571428571429,-10.4210526315789)(105.571428571429,-11.3684210526316)(105.571428571429,-12.3157894736842)(105.571428571429,-13.2631578947368)(105.571428571429,-14.2105263157895)(105.571428571429,-15.1578947368421)(105.571428571429,-16.1052631578947)(105.571428571429,-17.0526315789474)(105.571428571429,-18)(211.142857142857,0)(211.142857142857,-0.947368421052632)(211.142857142857,-1.89473684210526)(211.142857142857,-2.84210526315789)(211.142857142857,-3.78947368421053)(211.142857142857,-4.73684210526316)(211.142857142857,-5.68421052631579)(211.142857142857,-6.63157894736842)(211.142857142857,-7.57894736842105)(211.142857142857,-8.52631578947369)(211.142857142857,-9.47368421052632)(211.142857142857,-10.4210526315789)(211.142857142857,-11.3684210526316)(211.142857142857,-12.3157894736842)(211.142857142857,-13.2631578947368)(211.142857142857,-14.2105263157895)(211.142857142857,-15.1578947368421)(211.142857142857,-16.1052631578947)(211.142857142857,-17.0526315789474)(211.142857142857,-18)(316.714285714286,0)(316.714285714286,-0.947368421052632)(316.714285714286,-1.89473684210526)(316.714285714286,-2.84210526315789)(316.714285714286,-3.78947368421053)(316.714285714286,-4.73684210526316)(316.714285714286,-5.68421052631579)(316.714285714286,-6.63157894736842)(316.714285714286,-7.57894736842105)(316.714285714286,-8.52631578947369)(316.714285714286,-9.47368421052632)(316.714285714286,-10.4210526315789)(316.714285714286,-11.3684210526316)(316.714285714286,-12.3157894736842)(316.714285714286,-13.2631578947368)(316.714285714286,-14.2105263157895)(316.714285714286,-15.1578947368421)(316.714285714286,-16.1052631578947)(316.714285714286,-17.0526315789474)(316.714285714286,-18)(422.285714285714,0)(422.285714285714,-0.947368421052632)(422.285714285714,-1.89473684210526)(422.285714285714,-2.84210526315789)(422.285714285714,-3.78947368421053)(422.285714285714,-4.73684210526316)(422.285714285714,-5.68421052631579)(422.285714285714,-6.63157894736842)(422.285714285714,-7.57894736842105)(422.285714285714,-8.52631578947369)(422.285714285714,-9.47368421052632)(422.285714285714,-10.4210526315789)(422.285714285714,-11.3684210526316)(422.285714285714,-12.3157894736842)(422.285714285714,-13.2631578947368)(422.285714285714,-14.2105263157895)(422.285714285714,-15.1578947368421)(422.285714285714,-16.1052631578947)(422.285714285714,-17.0526315789474)(422.285714285714,-18)(527.857142857143,0)(527.857142857143,-0.947368421052632)(527.857142857143,-1.89473684210526)(527.857142857143,-2.84210526315789)(527.857142857143,-3.78947368421053)(527.857142857143,-4.73684210526316)(527.857142857143,-5.68421052631579)(527.857142857143,-6.63157894736842)(527.857142857143,-7.57894736842105)(527.857142857143,-8.52631578947369)(527.857142857143,-9.47368421052632)(527.857142857143,-10.4210526315789)(527.857142857143,-11.3684210526316)(527.857142857143,-12.3157894736842)(527.857142857143,-13.2631578947368)(527.857142857143,-14.2105263157895)(527.857142857143,-15.1578947368421)(527.857142857143,-16.1052631578947)(527.857142857143,-17.0526315789474)(527.857142857143,-18)(633.428571428571,0)(633.428571428571,-0.947368421052632)(633.428571428571,-1.89473684210526)(633.428571428571,-2.84210526315789)(633.428571428571,-3.78947368421053)(633.428571428571,-4.73684210526316)(633.428571428571,-5.68421052631579)(633.428571428571,-6.63157894736842)(633.428571428571,-7.57894736842105)(633.428571428571,-8.52631578947369)(633.428571428571,-9.47368421052632)(633.428571428571,-10.4210526315789)(633.428571428571,-11.3684210526316)(633.428571428571,-12.3157894736842)(633.428571428571,-13.2631578947368)(633.428571428571,-14.2105263157895)(633.428571428571,-15.1578947368421)(633.428571428571,-16.1052631578947)(633.428571428571,-17.0526315789474)(633.428571428571,-18)(739,0)(739,-0.947368421052632)(739,-1.89473684210526)(739,-2.84210526315789)(739,-3.78947368421053)(739,-4.73684210526316)(739,-5.68421052631579)(739,-6.63157894736842)(739,-7.57894736842105)(739,-8.52631578947369)(739,-9.47368421052632)(739,-10.4210526315789)(739,-11.3684210526316)(739,-12.3157894736842)(739,-13.2631578947368)(739,-14.2105263157895)(739,-15.1578947368421)(739,-16.1052631578947)(739,-17.0526315789474)(739,-18)(844.571428571429,0)(844.571428571429,-0.947368421052632)(844.571428571429,-1.89473684210526)(844.571428571429,-2.84210526315789)(844.571428571429,-3.78947368421053)(844.571428571429,-4.73684210526316)(844.571428571429,-5.68421052631579)(844.571428571429,-6.63157894736842)(844.571428571429,-7.57894736842105)(844.571428571429,-8.52631578947369)(844.571428571429,-9.47368421052632)(844.571428571429,-10.4210526315789)(844.571428571429,-11.3684210526316)(844.571428571429,-12.3157894736842)(844.571428571429,-13.2631578947368)(844.571428571429,-14.2105263157895)(844.571428571429,-15.1578947368421)(844.571428571429,-16.1052631578947)(844.571428571429,-17.0526315789474)(844.571428571429,-18)(950.142857142857,0)(950.142857142857,-0.947368421052632)(950.142857142857,-1.89473684210526)(950.142857142857,-2.84210526315789)(950.142857142857,-3.78947368421053)(950.142857142857,-4.73684210526316)(950.142857142857,-5.68421052631579)(950.142857142857,-6.63157894736842)(950.142857142857,-7.57894736842105)(950.142857142857,-8.52631578947369)(950.142857142857,-9.47368421052632)(950.142857142857,-10.4210526315789)(950.142857142857,-11.3684210526316)(950.142857142857,-12.3157894736842)(950.142857142857,-13.2631578947368)(950.142857142857,-14.2105263157895)(950.142857142857,-15.1578947368421)(950.142857142857,-16.1052631578947)(950.142857142857,-17.0526315789474)(950.142857142857,-18)(1055.71428571429,0)(1055.71428571429,-0.947368421052632)(1055.71428571429,-1.89473684210526)(1055.71428571429,-2.84210526315789)(1055.71428571429,-3.78947368421053)(1055.71428571429,-4.73684210526316)(1055.71428571429,-5.68421052631579)(1055.71428571429,-6.63157894736842)(1055.71428571429,-7.57894736842105)(1055.71428571429,-8.52631578947369)(1055.71428571429,-9.47368421052632)(1055.71428571429,-10.4210526315789)(1055.71428571429,-11.3684210526316)(1055.71428571429,-12.3157894736842)(1055.71428571429,-13.2631578947368)(1055.71428571429,-14.2105263157895)(1055.71428571429,-15.1578947368421)(1055.71428571429,-16.1052631578947)(1055.71428571429,-17.0526315789474)(1055.71428571429,-18)(1161.28571428571,0)(1161.28571428571,-0.947368421052632)(1161.28571428571,-1.89473684210526)(1161.28571428571,-2.84210526315789)(1161.28571428571,-3.78947368421053)(1161.28571428571,-4.73684210526316)(1161.28571428571,-5.68421052631579)(1161.28571428571,-6.63157894736842)(1161.28571428571,-7.57894736842105)(1161.28571428571,-8.52631578947369)(1161.28571428571,-9.47368421052632)(1161.28571428571,-10.4210526315789)(1161.28571428571,-11.3684210526316)(1161.28571428571,-12.3157894736842)(1161.28571428571,-13.2631578947368)(1161.28571428571,-14.2105263157895)(1161.28571428571,-15.1578947368421)(1161.28571428571,-16.1052631578947)(1161.28571428571,-17.0526315789474)(1161.28571428571,-18)(1266.85714285714,0)(1266.85714285714,-0.947368421052632)(1266.85714285714,-1.89473684210526)(1266.85714285714,-2.84210526315789)(1266.85714285714,-3.78947368421053)(1266.85714285714,-4.73684210526316)(1266.85714285714,-5.68421052631579)(1266.85714285714,-6.63157894736842)(1266.85714285714,-7.57894736842105)(1266.85714285714,-8.52631578947369)(1266.85714285714,-9.47368421052632)(1266.85714285714,-10.4210526315789)(1266.85714285714,-11.3684210526316)(1266.85714285714,-12.3157894736842)(1266.85714285714,-13.2631578947368)(1266.85714285714,-14.2105263157895)(1266.85714285714,-15.1578947368421)(1266.85714285714,-16.1052631578947)(1266.85714285714,-17.0526315789474)(1266.85714285714,-18)(1372.42857142857,0)(1372.42857142857,-0.947368421052632)(1372.42857142857,-1.89473684210526)(1372.42857142857,-2.84210526315789)(1372.42857142857,-3.78947368421053)(1372.42857142857,-4.73684210526316)(1372.42857142857,-5.68421052631579)(1372.42857142857,-6.63157894736842)(1372.42857142857,-7.57894736842105)(1372.42857142857,-8.52631578947369)(1372.42857142857,-9.47368421052632)(1372.42857142857,-10.4210526315789)(1372.42857142857,-11.3684210526316)(1372.42857142857,-12.3157894736842)(1372.42857142857,-13.2631578947368)(1372.42857142857,-14.2105263157895)(1372.42857142857,-15.1578947368421)(1372.42857142857,-16.1052631578947)(1372.42857142857,-17.0526315789474)(1372.42857142857,-18)(1478,0)(1478,-0.947368421052632)(1478,-1.89473684210526)(1478,-2.84210526315789)(1478,-3.78947368421053)(1478,-4.73684210526316)(1478,-5.68421052631579)(1478,-6.63157894736842)(1478,-7.57894736842105)(1478,-8.52631578947369)(1478,-9.47368421052632)(1478,-10.4210526315789)(1478,-11.3684210526316)(1478,-12.3157894736842)(1478,-13.2631578947368)(1478,-14.2105263157895)(1478,-15.1578947368421)(1478,-16.1052631578947)(1478,-17.0526315789474)(1478,-18)(1372.42857142857,0)(1372.42857142857,-0.947368421052632)(1372.42857142857,-1.89473684210526)(1372.42857142857,-2.84210526315789)(1372.42857142857,-3.78947368421053)(1372.42857142857,-4.73684210526316)(1372.42857142857,-5.68421052631579)(1372.42857142857,-6.63157894736842)(1372.42857142857,-7.57894736842105)(1372.42857142857,-8.52631578947369)(1372.42857142857,-9.47368421052632)(1372.42857142857,-10.4210526315789)(1372.42857142857,-11.3684210526316)(1372.42857142857,-12.3157894736842)(1372.42857142857,-13.2631578947368)(1372.42857142857,-14.2105263157895)(1372.42857142857,-15.1578947368421)(1372.42857142857,-16.1052631578947)(1372.42857142857,-17.0526315789474)(1372.42857142857,-18)(1266.85714285714,0)(1266.85714285714,-0.947368421052632)(1266.85714285714,-1.89473684210526)(1266.85714285714,-2.84210526315789)(1266.85714285714,-3.78947368421053)(1266.85714285714,-4.73684210526316)(1266.85714285714,-5.68421052631579)(1266.85714285714,-6.63157894736842)(1266.85714285714,-7.57894736842105)(1266.85714285714,-8.52631578947369)(1266.85714285714,-9.47368421052632)(1266.85714285714,-10.4210526315789)(1266.85714285714,-11.3684210526316)(1266.85714285714,-12.3157894736842)(1266.85714285714,-13.2631578947368)(1266.85714285714,-14.2105263157895)(1266.85714285714,-15.1578947368421)(1266.85714285714,-16.1052631578947)(1266.85714285714,-17.0526315789474)(1266.85714285714,-18)(1161.28571428571,0)(1161.28571428571,-0.947368421052632)(1161.28571428571,-1.89473684210526)(1161.28571428571,-2.84210526315789)(1161.28571428571,-3.78947368421053)(1161.28571428571,-4.73684210526316)(1161.28571428571,-5.68421052631579)(1161.28571428571,-6.63157894736842)(1161.28571428571,-7.57894736842105)(1161.28571428571,-8.52631578947369)(1161.28571428571,-9.47368421052632)(1161.28571428571,-10.4210526315789)(1161.28571428571,-11.3684210526316)(1161.28571428571,-12.3157894736842)(1161.28571428571,-13.2631578947368)(1161.28571428571,-14.2105263157895)(1161.28571428571,-15.1578947368421)(1161.28571428571,-16.1052631578947)(1161.28571428571,-17.0526315789474)(1161.28571428571,-18)(1055.71428571429,0)(1055.71428571429,-0.947368421052632)(1055.71428571429,-1.89473684210526)(1055.71428571429,-2.84210526315789)(1055.71428571429,-3.78947368421053)(1055.71428571429,-4.73684210526316)(1055.71428571429,-5.68421052631579)(1055.71428571429,-6.63157894736842)(1055.71428571429,-7.57894736842105)(1055.71428571429,-8.52631578947369)(1055.71428571429,-9.47368421052632)(1055.71428571429,-10.4210526315789)(1055.71428571429,-11.3684210526316)(1055.71428571429,-12.3157894736842)(1055.71428571429,-13.2631578947368)(1055.71428571429,-14.2105263157895)(1055.71428571429,-15.1578947368421)(1055.71428571429,-16.1052631578947)(1055.71428571429,-17.0526315789474)(1055.71428571429,-18)(950.142857142857,0)(950.142857142857,-0.947368421052632)(950.142857142857,-1.89473684210526)(950.142857142857,-2.84210526315789)(950.142857142857,-3.78947368421053)(950.142857142857,-4.73684210526316)(950.142857142857,-5.68421052631579)(950.142857142857,-6.63157894736842)(950.142857142857,-7.57894736842105)(950.142857142857,-8.52631578947369)(950.142857142857,-9.47368421052632)(950.142857142857,-10.4210526315789)(950.142857142857,-11.3684210526316)(950.142857142857,-12.3157894736842)(950.142857142857,-13.2631578947368)(950.142857142857,-14.2105263157895)(950.142857142857,-15.1578947368421)(950.142857142857,-16.1052631578947)(950.142857142857,-17.0526315789474)(950.142857142857,-18)(844.571428571429,0)(844.571428571429,-0.947368421052632)(844.571428571429,-1.89473684210526)(844.571428571429,-2.84210526315789)(844.571428571429,-3.78947368421053)(844.571428571429,-4.73684210526316)(844.571428571429,-5.68421052631579)(844.571428571429,-6.63157894736842)(844.571428571429,-7.57894736842105)(844.571428571429,-8.52631578947369)(844.571428571429,-9.47368421052632)(844.571428571429,-10.4210526315789)(844.571428571429,-11.3684210526316)(844.571428571429,-12.3157894736842)(844.571428571429,-13.2631578947368)(844.571428571429,-14.2105263157895)(844.571428571429,-15.1578947368421)(844.571428571429,-16.1052631578947)(844.571428571429,-17.0526315789474)(844.571428571429,-18)(739,0)(739,-0.947368421052632)(739,-1.89473684210526)(739,-2.84210526315789)(739,-3.78947368421053)(739,-4.73684210526316)(739,-5.68421052631579)(739,-6.63157894736842)(739,-7.57894736842105)(739,-8.52631578947369)(739,-9.47368421052632)(739,-10.4210526315789)(739,-11.3684210526316)(739,-12.3157894736842)(739,-13.2631578947368)(739,-14.2105263157895)(739,-15.1578947368421)(739,-16.1052631578947)(739,-17.0526315789474)(739,-18)(633.428571428571,0)(633.428571428571,-0.947368421052632)(633.428571428571,-1.89473684210526)(633.428571428571,-2.84210526315789)(633.428571428571,-3.78947368421053)(633.428571428571,-4.73684210526316)(633.428571428571,-5.68421052631579)(633.428571428571,-6.63157894736842)(633.428571428571,-7.57894736842105)(633.428571428571,-8.52631578947369)(633.428571428571,-9.47368421052632)(633.428571428571,-10.4210526315789)(633.428571428571,-11.3684210526316)(633.428571428571,-12.3157894736842)(633.428571428571,-13.2631578947368)(633.428571428571,-14.2105263157895)(633.428571428571,-15.1578947368421)(633.428571428571,-16.1052631578947)(633.428571428571,-17.0526315789474)(633.428571428571,-18)(527.857142857143,0)(527.857142857143,-0.947368421052632)(527.857142857143,-1.89473684210526)(527.857142857143,-2.84210526315789)(527.857142857143,-3.78947368421053)(527.857142857143,-4.73684210526316)(527.857142857143,-5.68421052631579)(527.857142857143,-6.63157894736842)(527.857142857143,-7.57894736842105)(527.857142857143,-8.52631578947369)(527.857142857143,-9.47368421052632)(527.857142857143,-10.4210526315789)(527.857142857143,-11.3684210526316)(527.857142857143,-12.3157894736842)(527.857142857143,-13.2631578947368)(527.857142857143,-14.2105263157895)(527.857142857143,-15.1578947368421)(527.857142857143,-16.1052631578947)(527.857142857143,-17.0526315789474)(527.857142857143,-18)(422.285714285714,0)(422.285714285714,-0.947368421052632)(422.285714285714,-1.89473684210526)(422.285714285714,-2.84210526315789)(422.285714285714,-3.78947368421053)(422.285714285714,-4.73684210526316)(422.285714285714,-5.68421052631579)(422.285714285714,-6.63157894736842)(422.285714285714,-7.57894736842105)(422.285714285714,-8.52631578947369)(422.285714285714,-9.47368421052632)(422.285714285714,-10.4210526315789)(422.285714285714,-11.3684210526316)(422.285714285714,-12.3157894736842)(422.285714285714,-13.2631578947368)(422.285714285714,-14.2105263157895)(422.285714285714,-15.1578947368421)(422.285714285714,-16.1052631578947)(422.285714285714,-17.0526315789474)(422.285714285714,-18)(316.714285714286,0)(316.714285714286,-0.947368421052632)(316.714285714286,-1.89473684210526)(316.714285714286,-2.84210526315789)(316.714285714286,-3.78947368421053)(316.714285714286,-4.73684210526316)(316.714285714286,-5.68421052631579)(316.714285714286,-6.63157894736842)(316.714285714286,-7.57894736842105)(316.714285714286,-8.52631578947369)(316.714285714286,-9.47368421052632)(316.714285714286,-10.4210526315789)(316.714285714286,-11.3684210526316)(316.714285714286,-12.3157894736842)(316.714285714286,-13.2631578947368)(316.714285714286,-14.2105263157895)(316.714285714286,-15.1578947368421)(316.714285714286,-16.1052631578947)(316.714285714286,-17.0526315789474)(316.714285714286,-18)(211.142857142857,0)(211.142857142857,-0.947368421052632)(211.142857142857,-1.89473684210526)(211.142857142857,-2.84210526315789)(211.142857142857,-3.78947368421053)(211.142857142857,-4.73684210526316)(211.142857142857,-5.68421052631579)(211.142857142857,-6.63157894736842)(211.142857142857,-7.57894736842105)(211.142857142857,-8.52631578947369)(211.142857142857,-9.47368421052632)(211.142857142857,-10.4210526315789)(211.142857142857,-11.3684210526316)(211.142857142857,-12.3157894736842)(211.142857142857,-13.2631578947368)(211.142857142857,-14.2105263157895)(211.142857142857,-15.1578947368421)(211.142857142857,-16.1052631578947)(211.142857142857,-17.0526315789474)(211.142857142857,-18)(105.571428571429,0)(105.571428571429,-0.947368421052632)(105.571428571429,-1.89473684210526)(105.571428571429,-2.84210526315789)(105.571428571429,-3.78947368421053)(105.571428571429,-4.73684210526316)(105.571428571429,-5.68421052631579)(105.571428571429,-6.63157894736842)(105.571428571429,-7.57894736842105)(105.571428571429,-8.52631578947369)(105.571428571429,-9.47368421052632)(105.571428571429,-10.4210526315789)(105.571428571429,-11.3684210526316)(105.571428571429,-12.3157894736842)(105.571428571429,-13.2631578947368)(105.571428571429,-14.2105263157895)(105.571428571429,-15.1578947368421)(105.571428571429,-16.1052631578947)(105.571428571429,-17.0526315789474)(105.571428571429,-18) 
};

\addplot [
color=blue,
mark size=3.5pt,
only marks,
mark=o,
mark options={solid,draw=lime!80!black},
line width=2.7pt,
forget plot
]
coordinates{
 (422.285714285714,-9.47368421052632) 
};

\end{axis}
\end{tikzpicture}%

%% This file was created by matlab2tikz v0.2.3.
% Copyright (c) 2008--2012, Nico Schlömer <nico.schloemer@gmail.com>
% All rights reserved.
% 
% 
% 
\definecolor{locol}{rgb}{0.26, 0.45, 0.65}

\begin{tikzpicture}

\begin{axis}[%
tick label style={font=\tiny},
label style={font=\tiny},
xlabel shift={-10pt},
ylabel shift={-17pt},
legend style={font=\tiny},
view={0}{90},
width=\figurewidth,
height=\figureheight,
scale only axis,
xmin=0, xmax=1478,
xtick={0, 400, 1000, 1400},
xlabel={Length (m)},
ymin=-18, ymax=0,
ytick={0, -4, -14, -18},
ylabel={Depth (m)},
name=plot1,
axis lines*=box,
axis line style={draw=none},
tickwidth=0.0cm,
clip=false
]

\addplot [fill=locol,draw=black,forget plot] coordinates{ (0,0)(4.94314381270903,0)(9.88628762541806,0)(14.8294314381271,0)(19.7725752508361,0)(24.7157190635452,0)(29.6588628762542,0)(34.6020066889632,0)(39.5451505016722,0)(44.4882943143813,0)(49.4314381270903,0)(54.3745819397993,0)(59.3177257525084,0)(64.2608695652174,0)(69.2040133779264,0)(74.1471571906355,0)(79.0903010033445,0)(84.0334448160535,0)(88.9765886287625,0)(93.9197324414716,0)(98.8628762541806,0)(103.80602006689,0)(108.749163879599,0)(113.692307692308,0)(118.635451505017,0)(123.578595317726,0)(128.521739130435,0)(133.464882943144,0)(138.408026755853,0)(143.351170568562,0)(148.294314381271,0)(153.23745819398,0)(158.180602006689,0)(163.123745819398,0)(168.066889632107,0)(173.010033444816,0)(177.953177257525,0)(182.896321070234,0)(187.839464882943,0)(192.782608695652,0)(197.725752508361,0)(202.66889632107,0)(207.612040133779,0)(212.555183946488,0)(217.498327759197,0)(222.441471571906,0)(227.384615384615,0)(232.327759197324,0)(237.270903010033,0)(242.214046822742,0)(247.157190635452,0)(252.100334448161,0)(257.04347826087,0)(261.986622073579,0)(266.929765886288,0)(271.872909698997,0)(276.816053511706,0)(281.759197324415,0)(286.702341137124,0)(291.645484949833,0)(296.588628762542,0)(301.531772575251,0)(306.47491638796,0)(311.418060200669,0)(316.361204013378,0)(321.304347826087,0)(326.247491638796,0)(331.190635451505,0)(336.133779264214,0)(341.076923076923,0)(346.020066889632,0)(350.963210702341,0)(355.90635451505,0)(360.849498327759,0)(365.792642140468,0)(370.735785953177,0)(375.678929765886,0)(380.622073578595,0)(385.565217391304,0)(390.508361204013,0)(395.451505016722,0)(400.394648829431,0)(405.33779264214,0)(410.28093645485,0)(415.224080267559,0)(420.167224080268,0)(425.110367892977,0)(430.053511705686,0)(434.996655518395,0)(439.939799331104,0)(444.882943143813,0)(449.826086956522,0)(454.769230769231,0)(459.71237458194,0)(464.655518394649,0)(469.598662207358,0)(474.541806020067,0)(479.484949832776,0)(484.428093645485,0)(489.371237458194,0)(494.314381270903,0)(499.257525083612,0)(504.200668896321,0)(509.14381270903,0)(514.086956521739,0)(519.030100334448,0)(523.973244147157,0)(528.916387959866,0)(533.859531772575,0)(538.802675585284,0)(543.745819397993,0)(548.688963210702,0)(553.632107023411,0)(558.57525083612,0)(563.518394648829,0)(568.461538461538,0)(573.404682274248,0)(578.347826086957,0)(583.290969899666,0)(588.234113712375,0)(593.177257525084,0)(598.120401337793,0)(603.063545150502,0)(608.006688963211,0)(612.94983277592,0)(617.892976588629,0)(622.836120401338,0)(627.779264214047,0)(632.722408026756,0)(637.665551839465,0)(642.608695652174,0)(647.551839464883,0)(652.494983277592,0)(657.438127090301,0)(662.38127090301,0)(667.324414715719,0)(672.267558528428,0)(677.210702341137,0)(682.153846153846,0)(687.096989966555,0)(692.040133779264,0)(696.983277591973,0)(701.926421404682,0)(706.869565217391,0)(711.8127090301,0)(716.755852842809,0)(721.698996655518,0)(726.642140468227,0)(731.585284280936,0)(736.528428093646,0)(741.471571906354,0)(746.414715719064,0)(751.357859531773,0)(756.301003344482,0)(761.244147157191,0)(766.1872909699,0)(771.130434782609,0)(776.073578595318,0)(781.016722408027,0)(785.959866220736,0)(790.903010033445,0)(795.846153846154,0)(800.789297658863,0)(805.732441471572,0)(810.675585284281,0)(815.61872909699,0)(820.561872909699,0)(825.505016722408,0)(830.448160535117,0)(835.391304347826,0)(840.334448160535,0)(845.277591973244,0)(850.220735785953,0)(855.163879598662,0)(860.107023411371,0)(865.05016722408,0)(869.993311036789,0)(874.936454849498,0)(879.879598662207,0)(884.822742474916,0)(889.765886287625,0)(894.709030100334,0)(899.652173913044,0)(904.595317725752,0)(909.538461538462,0)(914.481605351171,0)(919.42474916388,0)(924.367892976589,0)(929.311036789298,0)(934.254180602007,0)(939.197324414716,0)(944.140468227425,0)(949.083612040134,0)(954.026755852843,0)(958.969899665552,0)(963.913043478261,0)(968.85618729097,0)(973.799331103679,0)(978.742474916388,0)(983.685618729097,0)(988.628762541806,0)(993.571906354515,0)(998.515050167224,0)(1003.45819397993,0)(1008.40133779264,0)(1013.34448160535,0)(1018.28762541806,0)(1023.23076923077,0)(1028.17391304348,0)(1033.11705685619,0)(1038.0602006689,0)(1043.00334448161,0)(1047.94648829431,0)(1052.88963210702,0)(1057.83277591973,0)(1062.77591973244,0)(1067.71906354515,0)(1072.66220735786,0)(1077.60535117057,0)(1082.54849498328,0)(1087.49163879599,0)(1092.4347826087,0)(1097.3779264214,0)(1102.32107023411,0)(1107.26421404682,0)(1112.20735785953,0)(1117.15050167224,0)(1122.09364548495,0)(1127.03678929766,0)(1131.97993311037,0)(1136.92307692308,0)(1141.86622073579,0)(1146.8093645485,0)(1151.7525083612,0)(1156.69565217391,0)(1161.63879598662,0)(1166.58193979933,0)(1171.52508361204,0)(1176.46822742475,0)(1181.41137123746,0)(1186.35451505017,0)(1191.29765886288,3.00988295082389e-06)(1196.24080267559,3.00988295082389e-06)(1201.18394648829,3.00988295082389e-06)(1206.127090301,3.00988295082389e-06)(1211.07023411371,3.00988295082389e-06)(1216.01337792642,3.00988295082389e-06)(1220.95652173913,3.00988295082389e-06)(1225.89966555184,3.00988295082389e-06)(1230.84280936455,3.00988295082389e-06)(1235.78595317726,3.00988295082389e-06)(1240.72909698997,3.00988295082389e-06)(1245.67224080268,3.00988295082389e-06)(1250.61538461538,3.00988295082389e-06)(1255.55852842809,3.00988295082389e-06)(1260.5016722408,3.00988295082389e-06)(1265.44481605351,3.00988295082389e-06)(1270.38795986622,3.00988295082389e-06)(1275.33110367893,3.00988295082389e-06)(1280.27424749164,3.00988295082389e-06)(1285.21739130435,3.00988295082389e-06)(1290.16053511706,3.00988295082389e-06)(1295.10367892977,3.00988295082389e-06)(1300.04682274247,3.00988295082389e-06)(1304.98996655518,3.00988295082389e-06)(1309.93311036789,3.00988295082389e-06)(1314.8762541806,3.00988295082389e-06)(1319.81939799331,3.00988295082389e-06)(1324.76254180602,3.00988295082389e-06)(1329.70568561873,3.00988295082389e-06)(1334.64882943144,3.00988295082389e-06)(1339.59197324415,3.00988295082389e-06)(1344.53511705686,3.00988295082389e-06)(1349.47826086957,3.00988295082389e-06)(1354.42140468227,3.00988295082389e-06)(1359.36454849498,3.00988295082389e-06)(1364.30769230769,3.00988295082389e-06)(1369.2508361204,3.00988295082389e-06)(1374.19397993311,3.00988295082389e-06)(1379.13712374582,3.00988295082389e-06)(1384.08026755853,3.00988295082389e-06)(1389.02341137124,3.00988295082389e-06)(1393.96655518395,3.00988295082389e-06)(1398.90969899666,3.00988295082389e-06)(1403.85284280936,3.00988295082389e-06)(1408.79598662207,3.00988295082389e-06)(1413.73913043478,3.00988295082389e-06)(1418.68227424749,3.00988295082389e-06)(1423.6254180602,3.00988295082389e-06)(1428.56856187291,3.00988295082389e-06)(1433.51170568562,3.00988295082389e-06)(1438.45484949833,3.00988295082389e-06)(1443.39799331104,3.00988295082389e-06)(1448.34113712375,3.00988295082389e-06)(1453.28428093645,3.00988295082389e-06)(1458.22742474916,3.00988295082389e-06)(1463.17056856187,3.00988295082389e-06)(1468.11371237458,3.00988295082389e-06)(1473.05685618729,3.00988295082389e-06)(1478,3.00988295082389e-06)(1478.00024714483,0)(1478.00024714483,-0.0602006688963215)(1478.00024714483,-0.120401337792643)(1478.00024714483,-0.180602006688964)(1478.00024714483,-0.240802675585286)(1478.00024714483,-0.301003344481604)(1478.00024714483,-0.361204013377925)(1478.00024714483,-0.421404682274247)(1478.00024714483,-0.481605351170568)(1478.00024714483,-0.54180602006689)(1478.00024714483,-0.602006688963211)(1478.00024714483,-0.662207357859533)(1478.00024714483,-0.722408026755854)(1478.00024714483,-0.782608695652176)(1478.00024714483,-0.842809364548494)(1478.00024714483,-0.903010033444815)(1478.00024714483,-0.963210702341136)(1478.00024714483,-1.02341137123746)(1478.00024714483,-1.08361204013378)(1478.00024714483,-1.1438127090301)(1478.00024714483,-1.20401337792642)(1478.00024714483,-1.26421404682274)(1478.00024714483,-1.32441471571906)(1478.00024714483,-1.38461538461538)(1478.00024714483,-1.4448160535117)(1478.00024714483,-1.50501672240803)(1478.00024714483,-1.56521739130435)(1478.00024714483,-1.62541806020067)(1478.00024714483,-1.68561872909699)(1478.00024714483,-1.74581939799331)(1478.00024714483,-1.80602006688963)(1478.00024714483,-1.86622073578595)(1478.00024714483,-1.92642140468227)(1478.00024714483,-1.98662207357859)(1478.00024714483,-2.04682274247492)(1478.00024714483,-2.10702341137124)(1478.00024714483,-2.16722408026756)(1478.00024714483,-2.22742474916388)(1478.00024714483,-2.2876254180602)(1478.00024714483,-2.34782608695652)(1478.00024714483,-2.40802675585284)(1478.00024714483,-2.46822742474916)(1478.00024714483,-2.52842809364548)(1478.00024714483,-2.58862876254181)(1478.00024714483,-2.64882943143813)(1478.00024714483,-2.70903010033445)(1478.00024714483,-2.76923076923077)(1478.00024714483,-2.82943143812709)(1478.00024714483,-2.88963210702341)(1478.00024714483,-2.94983277591973)(1478.00024714483,-3.01003344481605)(1478.00024714483,-3.07023411371237)(1478.00024714483,-3.1304347826087)(1478.00024714483,-3.19063545150502)(1478.00024714483,-3.25083612040134)(1478.00024714483,-3.31103678929766)(1478.00024714483,-3.37123745819398)(1478.00024714483,-3.4314381270903)(1478.00024714483,-3.49163879598662)(1478.00024714483,-3.55183946488294)(1478.00024714483,-3.61204013377926)(1478.00024714483,-3.67224080267559)(1478.00024714483,-3.73244147157191)(1478.00024714483,-3.79264214046823)(1478.00024714483,-3.85284280936455)(1478.00024714483,-3.91304347826087)(1478.00024714483,-3.97324414715719)(1478.00024714483,-4.03344481605351)(1478.00024714483,-4.09364548494983)(1478.00024714483,-4.15384615384615)(1478.00024714483,-4.21404682274247)(1478.00024714483,-4.2742474916388)(1478.00024714483,-4.33444816053512)(1478.00024714483,-4.39464882943144)(1478.00024714483,-4.45484949832776)(1478.00024714483,-4.51505016722408)(1478.00024714483,-4.5752508361204)(1478.00024714483,-4.63545150501672)(1478.00024714483,-4.69565217391304)(1478.00024714483,-4.75585284280936)(1478.00024714483,-4.81605351170569)(1478.00024714483,-4.87625418060201)(1478.00024714483,-4.93645484949833)(1478.00024714483,-4.99665551839465)(1478.00024714483,-5.05685618729097)(1478.00024714483,-5.11705685618729)(1478.00024714483,-5.17725752508361)(1478.00024714483,-5.23745819397993)(1478.00024714483,-5.29765886287625)(1478.00024714483,-5.35785953177258)(1478.00024714483,-5.4180602006689)(1478.00024714483,-5.47826086956522)(1478.00024714483,-5.53846153846154)(1478.00024714483,-5.59866220735786)(1478.00024714483,-5.65886287625418)(1478.00024714483,-5.7190635451505)(1478.00024714483,-5.77926421404682)(1478.00024714483,-5.83946488294314)(1478.00024714483,-5.89966555183947)(1478.00024714483,-5.95986622073579)(1478.00024714483,-6.02006688963211)(1478.00024714483,-6.08026755852843)(1478.00024714483,-6.14046822742475)(1478.00024714483,-6.20066889632107)(1478.00024714483,-6.26086956521739)(1478.00024714483,-6.32107023411371)(1478.00024714483,-6.38127090301003)(1478.00024714483,-6.44147157190636)(1478.00024714483,-6.50167224080267)(1478.00024714483,-6.561872909699)(1478.00024714483,-6.62207357859532)(1478.00024714483,-6.68227424749164)(1478.00024714483,-6.74247491638796)(1478.00024714483,-6.80267558528428)(1478.00024714483,-6.8628762541806)(1478.00024714483,-6.92307692307692)(1478.00024714483,-6.98327759197324)(1478.00024714483,-7.04347826086956)(1478.00024714483,-7.10367892976589)(1478.00024714483,-7.16387959866221)(1478.00024714483,-7.22408026755853)(1478.00024714483,-7.28428093645485)(1478.00024714483,-7.34448160535117)(1478.00024714483,-7.40468227424749)(1478.00024714483,-7.46488294314381)(1478.00024714483,-7.52508361204013)(1478.00024714483,-7.58528428093645)(1478.00024714483,-7.64548494983278)(1478.00024714483,-7.7056856187291)(1478.00024714483,-7.76588628762542)(1478.00024714483,-7.82608695652174)(1478.00024714483,-7.88628762541806)(1478.00024714483,-7.94648829431438)(1478.00024714483,-8.0066889632107)(1478.00024714483,-8.06688963210702)(1478.00024714483,-8.12709030100334)(1478.00024714483,-8.18729096989967)(1478.00024714483,-8.24749163879599)(1478.00024714483,-8.30769230769231)(1478.00024714483,-8.36789297658863)(1478.00024714483,-8.42809364548495)(1478.00024714483,-8.48829431438127)(1478.00024714483,-8.54849498327759)(1478.00024714483,-8.60869565217391)(1478.00024714483,-8.66889632107023)(1478.00024714483,-8.72909698996656)(1478.00024714483,-8.78929765886288)(1478.00024714483,-8.8494983277592)(1478.00024714483,-8.90969899665552)(1478.00024714483,-8.96989966555184)(1478.00024714483,-9.03010033444816)(1478.00024714483,-9.09030100334448)(1478.00024714483,-9.1505016722408)(1478.00024714483,-9.21070234113712)(1478.00024714483,-9.27090301003344)(1478.00024714483,-9.33110367892977)(1478.00024714483,-9.39130434782609)(1478.00024714483,-9.45150501672241)(1478.00024714483,-9.51170568561873)(1478.00024714483,-9.57190635451505)(1478.00024714483,-9.63210702341137)(1478.00024714483,-9.69230769230769)(1478.00024714483,-9.75250836120401)(1478.00024714483,-9.81270903010033)(1478.00024714483,-9.87290969899666)(1478.00024714483,-9.93311036789298)(1478.00024714483,-9.9933110367893)(1478.00024714483,-10.0535117056856)(1478.00024714483,-10.1137123745819)(1478.00024714483,-10.1739130434783)(1478.00024714483,-10.2341137123746)(1478.00024714483,-10.2943143812709)(1478.00024714483,-10.3545150501672)(1478.00024714483,-10.4147157190635)(1478.00024714483,-10.4749163879599)(1478.00024714483,-10.5351170568562)(1478.00024714483,-10.5953177257525)(1478.00024714483,-10.6555183946488)(1478.00024714483,-10.7157190635452)(1478.00024714483,-10.7759197324415)(1478.00024714483,-10.8361204013378)(1478.00024714483,-10.8963210702341)(1478.00024714483,-10.9565217391304)(1478.00024714483,-11.0167224080268)(1478.00024714483,-11.0769230769231)(1478.00024714483,-11.1371237458194)(1478.00024714483,-11.1973244147157)(1478.00024714483,-11.257525083612)(1478.00024714483,-11.3177257525084)(1478.00024714483,-11.3779264214047)(1478.00024714483,-11.438127090301)(1478.00024714483,-11.4983277591973)(1478.00024714483,-11.5585284280936)(1478.00024714483,-11.61872909699)(1478.00024714483,-11.6789297658863)(1478.00024714483,-11.7391304347826)(1478.00024714483,-11.7993311036789)(1478.00024714483,-11.8595317725753)(1478.00024714483,-11.9197324414716)(1478.00024714483,-11.9799331103679)(1478.00024714483,-12.0401337792642)(1478.00024714483,-12.1003344481605)(1478.00024714483,-12.1605351170569)(1478.00024714483,-12.2207357859532)(1478.00024714483,-12.2809364548495)(1478.00024714483,-12.3411371237458)(1478.00024714483,-12.4013377926421)(1478.00024714483,-12.4615384615385)(1478.00024714483,-12.5217391304348)(1478.00024714483,-12.5819397993311)(1478.00024714483,-12.6421404682274)(1478.00024714483,-12.7023411371237)(1478.00024714483,-12.7625418060201)(1478.00024714483,-12.8227424749164)(1478.00024714483,-12.8829431438127)(1478.00024714483,-12.943143812709)(1478.00024714483,-13.0033444816054)(1478.00024714483,-13.0635451505017)(1478.00024714483,-13.123745819398)(1478.00024714483,-13.1839464882943)(1478.00024714483,-13.2441471571906)(1478.00024714483,-13.304347826087)(1478.00024714483,-13.3645484949833)(1478.00024714483,-13.4247491638796)(1478.00024714483,-13.4849498327759)(1478.00024714483,-13.5451505016722)(1478.00024714483,-13.6053511705686)(1478.00024714483,-13.6655518394649)(1478.00024714483,-13.7257525083612)(1478.00024714483,-13.7859531772575)(1478.00024714483,-13.8461538461538)(1478.00024714483,-13.9063545150502)(1478.00024714483,-13.9665551839465)(1478.00024714483,-14.0267558528428)(1478.00024714483,-14.0869565217391)(1478.00024714483,-14.1471571906355)(1478.00024714483,-14.2073578595318)(1478.00024714483,-14.2675585284281)(1478.00024714483,-14.3277591973244)(1478.00024714483,-14.3879598662207)(1478.00024714483,-14.4481605351171)(1478.00024714483,-14.5083612040134)(1478.00024714483,-14.5685618729097)(1478.00024714483,-14.628762541806)(1478.00024714483,-14.6889632107023)(1478.00024714483,-14.7491638795987)(1478.00024714483,-14.809364548495)(1478.00024714483,-14.8695652173913)(1478.00024714483,-14.9297658862876)(1478.00024714483,-14.9899665551839)(1478.00024714483,-15.0501672240803)(1478.00024714483,-15.1103678929766)(1478.00024714483,-15.1705685618729)(1478.00024714483,-15.2307692307692)(1478.00024714483,-15.2909698996656)(1478.00024714483,-15.3511705685619)(1478.00024714483,-15.4113712374582)(1478.00024714483,-15.4715719063545)(1478.00024714483,-15.5317725752508)(1478.00024714483,-15.5919732441472)(1478.00024714483,-15.6521739130435)(1478.00024714483,-15.7123745819398)(1478.00024714483,-15.7725752508361)(1478.00024714483,-15.8327759197324)(1478.00024714483,-15.8929765886288)(1478.00024714483,-15.9531772575251)(1478.00024714483,-16.0133779264214)(1478.00024714483,-16.0735785953177)(1478.00024714483,-16.133779264214)(1478.00024714483,-16.1939799331104)(1478.00024714483,-16.2541806020067)(1478.00024714483,-16.314381270903)(1478.00024714483,-16.3745819397993)(1478.00024714483,-16.4347826086957)(1478.00024714483,-16.494983277592)(1478.00024714483,-16.5551839464883)(1478.00024714483,-16.6153846153846)(1478.00024714483,-16.6755852842809)(1478.00024714483,-16.7357859531773)(1478.00024714483,-16.7959866220736)(1478.00024714483,-16.8561872909699)(1478.00024714483,-16.9163879598662)(1478.00024714483,-16.9765886287625)(1478.00024714483,-17.0367892976589)(1478.00024714483,-17.0969899665552)(1478.00024714483,-17.1571906354515)(1478.00024714483,-17.2173913043478)(1478.00024714483,-17.2775919732441)(1478.00024714483,-17.3377926421405)(1478.00024714483,-17.3979933110368)(1478.00024714483,-17.4581939799331)(1478.00024714483,-17.5183946488294)(1478.00024714483,-17.5785953177258)(1478.00024714483,-17.6387959866221)(1478.00024714483,-17.6989966555184)(1478.00024714483,-17.7591973244147)(1478.00024714483,-17.819397993311)(1478.00024714483,-17.8795986622074)(1478.00024714483,-17.9397993311037)(1478.00024714483,-18)(1478,-18.000003009883)(1473.05685618729,-18.000003009883)(1468.11371237458,-18.000003009883)(1463.17056856187,-18.000003009883)(1458.22742474916,-18.000003009883)(1453.28428093645,-18.000003009883)(1448.34113712375,-18.000003009883)(1443.39799331104,-18.000003009883)(1438.45484949833,-18.000003009883)(1433.51170568562,-18.000003009883)(1428.56856187291,-18.000003009883)(1423.6254180602,-18.000003009883)(1418.68227424749,-18.000003009883)(1413.73913043478,-18.000003009883)(1408.79598662207,-18.000003009883)(1403.85284280936,-18.000003009883)(1398.90969899666,-18.000003009883)(1393.96655518395,-18.000003009883)(1389.02341137124,-18.000003009883)(1384.08026755853,-18.000003009883)(1379.13712374582,-18.000003009883)(1374.19397993311,-18.000003009883)(1369.2508361204,-18.000003009883)(1364.30769230769,-18.000003009883)(1359.36454849498,-18.000003009883)(1354.42140468227,-18.000003009883)(1349.47826086957,-18.000003009883)(1344.53511705686,-18.000003009883)(1339.59197324415,-18.000003009883)(1334.64882943144,-18.000003009883)(1329.70568561873,-18.000003009883)(1324.76254180602,-18.000003009883)(1319.81939799331,-18.000003009883)(1314.8762541806,-18.000003009883)(1309.93311036789,-18.000003009883)(1304.98996655518,-18.000003009883)(1300.04682274247,-18.000003009883)(1295.10367892977,-18.000003009883)(1290.16053511706,-18.000003009883)(1285.21739130435,-18.000003009883)(1280.27424749164,-18.000003009883)(1275.33110367893,-18.000003009883)(1270.38795986622,-18.000003009883)(1265.44481605351,-18.000003009883)(1260.5016722408,-18.000003009883)(1255.55852842809,-18.000003009883)(1250.61538461538,-18.000003009883)(1245.67224080268,-18.000003009883)(1240.72909698997,-18.000003009883)(1235.78595317726,-18.000003009883)(1230.84280936455,-18.000003009883)(1225.89966555184,-18.000003009883)(1220.95652173913,-18.000003009883)(1216.01337792642,-18.000003009883)(1211.07023411371,-18.000003009883)(1206.127090301,-18.000003009883)(1201.18394648829,-18.000003009883)(1196.24080267559,-18.000003009883)(1191.29765886288,-18.000003009883)(1186.35451505017,-18.000003009883)(1181.41137123746,-18.000003009883)(1176.46822742475,-18.000003009883)(1171.52508361204,-18.000003009883)(1166.58193979933,-18.000003009883)(1161.63879598662,-18.000003009883)(1156.69565217391,-18.000003009883)(1151.7525083612,-18.000003009883)(1146.8093645485,-18.000003009883)(1141.86622073579,-18.000003009883)(1136.92307692308,-18.000003009883)(1131.97993311037,-18.000003009883)(1127.03678929766,-18.000003009883)(1122.09364548495,-18.000003009883)(1117.15050167224,-18.000003009883)(1112.20735785953,-18.000003009883)(1107.26421404682,-18.000003009883)(1102.32107023411,-18.000003009883)(1097.3779264214,-18.000003009883)(1092.4347826087,-18.000003009883)(1087.49163879599,-18.000003009883)(1082.54849498328,-18.000003009883)(1077.60535117057,-18.000003009883)(1072.66220735786,-18.000003009883)(1067.71906354515,-18.000003009883)(1062.77591973244,-18.000003009883)(1057.83277591973,-18.000003009883)(1052.88963210702,-18.000003009883)(1047.94648829431,-18.000003009883)(1043.00334448161,-18.000003009883)(1038.0602006689,-18.000003009883)(1033.11705685619,-18.000003009883)(1028.17391304348,-18.000003009883)(1023.23076923077,-18.000003009883)(1018.28762541806,-18.000003009883)(1013.34448160535,-18.000003009883)(1008.40133779264,-18.000003009883)(1003.45819397993,-18.000003009883)(998.515050167224,-18.000003009883)(993.571906354515,-18.000003009883)(988.628762541806,-18.000003009883)(983.685618729097,-18.000003009883)(978.742474916388,-18.000003009883)(973.799331103679,-18.000003009883)(968.85618729097,-18.000003009883)(963.913043478261,-18)(958.969899665552,-18)(954.026755852843,-18)(949.083612040134,-18)(944.140468227425,-18)(939.197324414716,-18)(934.254180602007,-18)(929.311036789298,-18)(924.367892976589,-18)(919.42474916388,-18)(914.481605351171,-18)(909.538461538462,-18)(904.595317725752,-18)(899.652173913044,-18)(894.709030100334,-18)(889.765886287625,-18)(884.822742474916,-18)(879.879598662207,-18)(874.936454849498,-18)(869.993311036789,-18)(865.05016722408,-18)(860.107023411371,-18)(855.163879598662,-18)(850.220735785953,-18)(845.277591973244,-18)(840.334448160535,-18)(835.391304347826,-18)(830.448160535117,-18)(825.505016722408,-18)(820.561872909699,-18)(815.61872909699,-18)(810.675585284281,-18)(805.732441471572,-18)(800.789297658863,-18)(795.846153846154,-18)(790.903010033445,-18)(785.959866220736,-18)(781.016722408027,-18)(776.073578595318,-18)(771.130434782609,-18)(766.1872909699,-18)(761.244147157191,-18)(756.301003344482,-18)(751.357859531773,-18)(746.414715719064,-18)(741.471571906354,-18)(736.528428093646,-18)(731.585284280936,-18)(726.642140468227,-18)(721.698996655518,-18)(716.755852842809,-18)(711.8127090301,-18)(706.869565217391,-18)(701.926421404682,-18)(696.983277591973,-18)(692.040133779264,-18)(687.096989966555,-18)(682.153846153846,-18)(677.210702341137,-18)(672.267558528428,-18)(667.324414715719,-18)(662.38127090301,-18)(657.438127090301,-18)(652.494983277592,-18)(647.551839464883,-18)(642.608695652174,-18)(637.665551839465,-18)(632.722408026756,-18)(627.779264214047,-18)(622.836120401338,-18)(617.892976588629,-18)(612.94983277592,-18)(608.006688963211,-18)(603.063545150502,-18)(598.120401337793,-18)(593.177257525084,-18)(588.234113712375,-18)(583.290969899666,-18)(578.347826086957,-18)(573.404682274248,-18)(568.461538461538,-18)(563.518394648829,-18)(558.57525083612,-18)(553.632107023411,-18)(548.688963210702,-18)(543.745819397993,-18)(538.802675585284,-18)(533.859531772575,-18)(528.916387959866,-18)(523.973244147157,-18)(519.030100334448,-18)(514.086956521739,-18)(509.14381270903,-18)(504.200668896321,-18)(499.257525083612,-18)(494.314381270903,-18)(489.371237458194,-18)(484.428093645485,-18)(479.484949832776,-18)(474.541806020067,-18)(469.598662207358,-18)(464.655518394649,-18)(459.71237458194,-18)(454.769230769231,-18)(449.826086956522,-18)(444.882943143813,-18)(439.939799331104,-18)(434.996655518395,-18)(430.053511705686,-18)(425.110367892977,-18)(420.167224080268,-18)(415.224080267559,-18)(410.28093645485,-18)(405.33779264214,-18)(400.394648829431,-18)(395.451505016722,-18)(390.508361204013,-18)(385.565217391304,-18)(380.622073578595,-18)(375.678929765886,-18)(370.735785953177,-18)(365.792642140468,-18)(360.849498327759,-18)(355.90635451505,-18)(350.963210702341,-18)(346.020066889632,-18)(341.076923076923,-18)(336.133779264214,-18)(331.190635451505,-18)(326.247491638796,-18)(321.304347826087,-18)(316.361204013378,-18)(311.418060200669,-18)(306.47491638796,-18)(301.531772575251,-18)(296.588628762542,-18)(291.645484949833,-18)(286.702341137124,-18)(281.759197324415,-18)(276.816053511706,-18)(271.872909698997,-18)(266.929765886288,-18)(261.986622073579,-18)(257.04347826087,-18)(252.100334448161,-18)(247.157190635452,-18)(242.214046822742,-18)(237.270903010033,-18)(232.327759197324,-18)(227.384615384615,-18)(222.441471571906,-18)(217.498327759197,-18)(212.555183946488,-18)(207.612040133779,-18)(202.66889632107,-18)(197.725752508361,-18)(192.782608695652,-18)(187.839464882943,-18)(182.896321070234,-18)(177.953177257525,-18)(173.010033444816,-18)(168.066889632107,-18)(163.123745819398,-18)(158.180602006689,-18)(153.23745819398,-18)(148.294314381271,-18)(143.351170568562,-18)(138.408026755853,-18)(133.464882943144,-18)(128.521739130435,-18)(123.578595317726,-18)(118.635451505017,-18)(113.692307692308,-18)(108.749163879599,-18)(103.80602006689,-18)(98.8628762541806,-18)(93.9197324414716,-18)(88.9765886287625,-18)(84.0334448160535,-18)(79.0903010033445,-18)(74.1471571906355,-18)(69.2040133779264,-18)(64.2608695652174,-18)(59.3177257525084,-18)(54.3745819397993,-18)(49.4314381270903,-18)(44.4882943143813,-18)(39.5451505016722,-18.000003009883)(34.6020066889632,-18.000003009883)(29.6588628762542,-18.000003009883)(24.7157190635452,-18.000003009883)(19.7725752508361,-18.000003009883)(14.8294314381271,-18.000003009883)(9.88628762541806,-18.000003009883)(4.94314381270903,-18.000003009883)(0,-18.0000060194649)(-0.000494264954775508,-18)(-0.000247144833393367,-17.9397993311037)(-0.000247144833393367,-17.8795986622074)(-0.000247144833393367,-17.819397993311)(-0.000247144833393367,-17.7591973244147)(-0.000247144833393367,-17.6989966555184)(-0.000247144833393367,-17.6387959866221)(-0.000247144833393367,-17.5785953177258)(-0.000247144833393367,-17.5183946488294)(-0.000247144833393367,-17.4581939799331)(-0.000247144833393367,-17.3979933110368)(-0.000247144833393367,-17.3377926421405)(-0.000247144833393367,-17.2775919732441)(-0.000247144833393367,-17.2173913043478)(-0.000247144833393367,-17.1571906354515)(-0.000247144833393367,-17.0969899665552)(-0.000247144833393367,-17.0367892976589)(-0.000247144833393367,-16.9765886287625)(-0.000247144833393367,-16.9163879598662)(-0.000247144833393367,-16.8561872909699)(-0.000247144833393367,-16.7959866220736)(-0.000247144833393367,-16.7357859531773)(-0.000247144833393367,-16.6755852842809)(-0.000247144833393367,-16.6153846153846)(-0.000247144833393367,-16.5551839464883)(-0.000247144833393367,-16.494983277592)(-0.000247144833393367,-16.4347826086957)(-0.000247144833393367,-16.3745819397993)(-0.000247144833393367,-16.314381270903)(-0.000247144833393367,-16.2541806020067)(-0.000247144833393367,-16.1939799331104)(-0.000247144833393367,-16.133779264214)(-0.000247144833393367,-16.0735785953177)(-0.000247144833393367,-16.0133779264214)(-0.000247144833393367,-15.9531772575251)(-0.000247144833393367,-15.8929765886288)(-0.000247144833393367,-15.8327759197324)(-0.000247144833393367,-15.7725752508361)(-0.000247144833393367,-15.7123745819398)(-0.000247144833393367,-15.6521739130435)(-0.000247144833393367,-15.5919732441472)(-0.000247144833393367,-15.5317725752508)(-0.000247144833393367,-15.4715719063545)(-0.000247144833393367,-15.4113712374582)(-0.000247144833393367,-15.3511705685619)(-0.000247144833393367,-15.2909698996656)(-0.000247144833393367,-15.2307692307692)(-0.000247144833393367,-15.1705685618729)(-0.000247144833393367,-15.1103678929766)(-0.000247144833393367,-15.0501672240803)(-0.000247144833393367,-14.9899665551839)(-0.000247144833393367,-14.9297658862876)(-0.000247144833393367,-14.8695652173913)(-0.000247144833393367,-14.809364548495)(-0.000247144833393367,-14.7491638795987)(-0.000247144833393367,-14.6889632107023)(-0.000247144833393367,-14.628762541806)(-0.000247144833393367,-14.5685618729097)(-0.000247144833393367,-14.5083612040134)(-0.000247144833393367,-14.4481605351171)(-0.000247144833393367,-14.3879598662207)(-0.000247144833393367,-14.3277591973244)(-0.000247144833393367,-14.2675585284281)(-0.000247144833393367,-14.2073578595318)(-0.000247144833393367,-14.1471571906355)(-0.000247144833393367,-14.0869565217391)(-0.000247144833393367,-14.0267558528428)(-0.000247144833393367,-13.9665551839465)(-0.000247144833393367,-13.9063545150502)(-0.000247144833393367,-13.8461538461538)(-0.000247144833393367,-13.7859531772575)(-0.000247144833393367,-13.7257525083612)(-0.000247144833393367,-13.6655518394649)(-0.000247144833393367,-13.6053511705686)(-0.000247144833393367,-13.5451505016722)(-0.000247144833393367,-13.4849498327759)(-0.000247144833393367,-13.4247491638796)(-0.000247144833393367,-13.3645484949833)(-0.000247144833393367,-13.304347826087)(-0.000247144833393367,-13.2441471571906)(-0.000247144833393367,-13.1839464882943)(-0.000247144833393367,-13.123745819398)(-0.000247144833393367,-13.0635451505017)(-0.000247144833393367,-13.0033444816054)(-0.000247144833393367,-12.943143812709)(-0.000247144833393367,-12.8829431438127)(-0.000247144833393367,-12.8227424749164)(-0.000247144833393367,-12.7625418060201)(-0.000247144833393367,-12.7023411371237)(-0.000247144833393367,-12.6421404682274)(-0.000247144833393367,-12.5819397993311)(-0.000247144833393367,-12.5217391304348)(-0.000247144833393367,-12.4615384615385)(-0.000247144833393367,-12.4013377926421)(-0.000247144833393367,-12.3411371237458)(-0.000247144833393367,-12.2809364548495)(-0.000247144833393367,-12.2207357859532)(-0.000247144833393367,-12.1605351170569)(-0.000247144833393367,-12.1003344481605)(-0.000247144833393367,-12.0401337792642)(-0.000247144833393367,-11.9799331103679)(-0.000247144833393367,-11.9197324414716)(-0.000247144833393367,-11.8595317725753)(-0.000247144833393367,-11.7993311036789)(-0.000247144833393367,-11.7391304347826)(-0.000247144833393367,-11.6789297658863)(-0.000247144833393367,-11.61872909699)(-0.000247144833393367,-11.5585284280936)(-0.000247144833393367,-11.4983277591973)(-0.000247144833393367,-11.438127090301)(-0.000247144833393367,-11.3779264214047)(-0.000247144833393367,-11.3177257525084)(-0.000247144833393367,-11.257525083612)(-0.000247144833393367,-11.1973244147157)(-0.000247144833393367,-11.1371237458194)(-0.000247144833393367,-11.0769230769231)(-0.000247144833393367,-11.0167224080268)(-0.000247144833393367,-10.9565217391304)(-0.000247144833393367,-10.8963210702341)(-0.000247144833393367,-10.8361204013378)(-0.000247144833393367,-10.7759197324415)(-0.000247144833393367,-10.7157190635452)(-0.000247144833393367,-10.6555183946488)(-0.000247144833393367,-10.5953177257525)(-0.000247144833393367,-10.5351170568562)(-0.000247144833393367,-10.4749163879599)(-0.000247144833393367,-10.4147157190635)(-0.000247144833393367,-10.3545150501672)(-0.000247144833393367,-10.2943143812709)(-0.000247144833393367,-10.2341137123746)(-0.000247144833393367,-10.1739130434783)(-0.000247144833393367,-10.1137123745819)(-0.000247144833393367,-10.0535117056856)(-0.000247144833393367,-9.9933110367893)(-0.000247144833393367,-9.93311036789298)(-0.000247144833393367,-9.87290969899666)(-0.000247144833393367,-9.81270903010033)(-0.000247144833393367,-9.75250836120401)(-0.000247144833393367,-9.69230769230769)(-0.000247144833393367,-9.63210702341137)(-0.000247144833393367,-9.57190635451505)(-0.000247144833393367,-9.51170568561873)(-0.000247144833393367,-9.45150501672241)(-0.000247144833393367,-9.39130434782609)(-0.000247144833393367,-9.33110367892977)(-0.000247144833393367,-9.27090301003344)(-0.000247144833393367,-9.21070234113712)(-0.000247144833393367,-9.1505016722408)(-0.000247144833393367,-9.09030100334448)(-0.000247144833393367,-9.03010033444816)(-0.000247144833393367,-8.96989966555184)(-0.000247144833393367,-8.90969899665552)(-0.000247144833393367,-8.8494983277592)(-0.000247144833393367,-8.78929765886288)(-0.000247144833393367,-8.72909698996656)(-0.000247144833393367,-8.66889632107023)(-0.000247144833393367,-8.60869565217391)(-0.000247144833393367,-8.54849498327759)(-0.000247144833393367,-8.48829431438127)(-0.000247144833393367,-8.42809364548495)(-0.000247144833393367,-8.36789297658863)(-0.000247144833393367,-8.30769230769231)(-0.000247144833393367,-8.24749163879599)(-0.000247144833393367,-8.18729096989967)(-0.000247144833393367,-8.12709030100334)(-0.000247144833393367,-8.06688963210702)(-0.000247144833393367,-8.0066889632107)(-0.000247144833393367,-7.94648829431438)(-0.000247144833393367,-7.88628762541806)(-0.000247144833393367,-7.82608695652174)(-0.000247144833393367,-7.76588628762542)(-0.000247144833393367,-7.7056856187291)(-0.000247144833393367,-7.64548494983278)(-0.000247144833393367,-7.58528428093645)(-0.000247144833393367,-7.52508361204013)(-0.000247144833393367,-7.46488294314381)(-0.000247144833393367,-7.40468227424749)(-0.000247144833393367,-7.34448160535117)(-0.000247144833393367,-7.28428093645485)(-0.000247144833393367,-7.22408026755853)(-0.000247144833393367,-7.16387959866221)(-0.000247144833393367,-7.10367892976589)(-0.000247144833393367,-7.04347826086956)(-0.000247144833393367,-6.98327759197324)(-0.000247144833393367,-6.92307692307692)(-0.000247144833393367,-6.8628762541806)(-0.000247144833393367,-6.80267558528428)(-0.000247144833393367,-6.74247491638796)(-0.000247144833393367,-6.68227424749164)(-0.000247144833393367,-6.62207357859532)(-0.000247144833393367,-6.561872909699)(-0.000247144833393367,-6.50167224080267)(-0.000247144833393367,-6.44147157190636)(-0.000247144833393367,-6.38127090301003)(-0.000247144833393367,-6.32107023411371)(-0.000247144833393367,-6.26086956521739)(-0.000247144833393367,-6.20066889632107)(-0.000247144833393367,-6.14046822742475)(-0.000247144833393367,-6.08026755852843)(-0.000247144833393367,-6.02006688963211)(-0.000247144833393367,-5.95986622073579)(-0.000247144833393367,-5.89966555183947)(-0.000247144833393367,-5.83946488294314)(-0.000247144833393367,-5.77926421404682)(-0.000247144833393367,-5.7190635451505)(-0.000247144833393367,-5.65886287625418)(-0.000247144833393367,-5.59866220735786)(-0.000247144833393367,-5.53846153846154)(-0.000247144833393367,-5.47826086956522)(-0.000247144833393367,-5.4180602006689)(-0.000247144833393367,-5.35785953177258)(-0.000247144833393367,-5.29765886287625)(-0.000247144833393367,-5.23745819397993)(-0.000247144833393367,-5.17725752508361)(-0.000247144833393367,-5.11705685618729)(-0.000247144833393367,-5.05685618729097)(-0.000247144833393367,-4.99665551839465)(-0.000247144833393367,-4.93645484949833)(-0.000247144833393367,-4.87625418060201)(-0.000247144833393367,-4.81605351170569)(-0.000247144833393367,-4.75585284280936)(-0.000247144833393367,-4.69565217391304)(-0.000247144833393367,-4.63545150501672)(0,-4.5752508361204)(0,-4.51505016722408)(0,-4.45484949832776)(0,-4.39464882943144)(0,-4.33444816053512)(0,-4.2742474916388)(0,-4.21404682274247)(0,-4.15384615384615)(0,-4.09364548494983)(0,-4.03344481605351)(0,-3.97324414715719)(0,-3.91304347826087)(0,-3.85284280936455)(0,-3.79264214046823)(0,-3.73244147157191)(0,-3.67224080267559)(0,-3.61204013377926)(0,-3.55183946488294)(0,-3.49163879598662)(0,-3.4314381270903)(0,-3.37123745819398)(0,-3.31103678929766)(0,-3.25083612040134)(0,-3.19063545150502)(0,-3.1304347826087)(0,-3.07023411371237)(0,-3.01003344481605)(0,-2.94983277591973)(0,-2.88963210702341)(0,-2.82943143812709)(0,-2.76923076923077)(0,-2.70903010033445)(0,-2.64882943143813)(0,-2.58862876254181)(0,-2.52842809364548)(0,-2.46822742474916)(0,-2.40802675585284)(0,-2.34782608695652)(0,-2.2876254180602)(0,-2.22742474916388)(0,-2.16722408026756)(0,-2.10702341137124)(0,-2.04682274247492)(0,-1.98662207357859)(0,-1.92642140468227)(0,-1.86622073578595)(0,-1.80602006688963)(0,-1.74581939799331)(0,-1.68561872909699)(0,-1.62541806020067)(0,-1.56521739130435)(0,-1.50501672240803)(0,-1.4448160535117)(0,-1.38461538461538)(0,-1.32441471571906)(0,-1.26421404682274)(0,-1.20401337792642)(0,-1.1438127090301)(0,-1.08361204013378)(0,-1.02341137123746)(0,-0.963210702341136)(0,-0.903010033444815)(0,-0.842809364548494)(0,-0.782608695652176)(0,-0.722408026755854)(0,-0.662207357859533)(0,-0.602006688963211)(0,-0.54180602006689)(0,-0.481605351170568)(0,-0.421404682274247)(0,-0.361204013377925)(0,-0.301003344481604)(0,-0.240802675585286)(0,-0.180602006688964)(0,-0.120401337792643)(0,-0.0602006688963215)(0,0)};

\addplot [fill=darkgray,draw=black,forget plot] coordinates{ (0,-4.63545150501672)(4.94314381270903,-4.69565217391304)(9.88628762541806,-4.75585284280936)(9.88628762541806,-4.81605351170569)(14.8294314381271,-4.87625418060201)(19.7725752508361,-4.93645484949833)(24.7157190635452,-4.99665551839465)(24.7157190635452,-5.05685618729097)(29.6588628762542,-5.11705685618729)(34.6020066889632,-5.17725752508361)(39.5451505016722,-5.23745819397993)(44.4882943143813,-5.29765886287625)(44.4882943143813,-5.35785953177258)(49.4314381270903,-5.4180602006689)(54.3745819397993,-5.47826086956522)(59.3177257525084,-5.53846153846154)(64.2608695652174,-5.59866220735786)(64.2608695652174,-5.65886287625418)(69.2040133779264,-5.7190635451505)(74.1471571906355,-5.77926421404682)(79.0903010033445,-5.83946488294314)(84.0334448160535,-5.89966555183947)(88.9765886287625,-5.95986622073579)(88.9765886287625,-6.02006688963211)(93.9197324414716,-6.08026755852843)(98.8628762541806,-6.14046822742475)(103.80602006689,-6.20066889632107)(108.749163879599,-6.26086956521739)(108.749163879599,-6.32107023411371)(113.692307692308,-6.38127090301003)(118.635451505017,-6.44147157190636)(123.578595317726,-6.50167224080267)(128.521739130435,-6.561872909699)(128.521739130435,-6.62207357859532)(133.464882943144,-6.68227424749164)(138.408026755853,-6.74247491638796)(143.351170568562,-6.80267558528428)(148.294314381271,-6.8628762541806)(153.23745819398,-6.92307692307692)(158.180602006689,-6.98327759197324)(163.123745819398,-7.04347826086956)(168.066889632107,-7.10367892976589)(173.010033444816,-7.16387959866221)(177.953177257525,-7.16387959866221)(182.896321070234,-7.22408026755853)(187.839464882943,-7.28428093645485)(192.782608695652,-7.28428093645485)(197.725752508361,-7.34448160535117)(202.66889632107,-7.34448160535117)(207.612040133779,-7.34448160535117)(212.555183946488,-7.34448160535117)(217.498327759197,-7.34448160535117)(222.441471571906,-7.34448160535117)(227.384615384615,-7.28428093645485)(232.327759197324,-7.28428093645485)(237.270903010033,-7.22408026755853)(242.214046822742,-7.16387959866221)(247.157190635452,-7.16387959866221)(252.100334448161,-7.10367892976589)(252.100334448161,-7.04347826086956)(257.04347826087,-6.98327759197324)(261.986622073579,-6.92307692307692)(261.986622073579,-6.8628762541806)(266.929765886288,-6.80267558528428)(266.929765886288,-6.74247491638796)(266.929765886288,-6.68227424749164)(271.872909698997,-6.62207357859532)(276.816053511706,-6.561872909699)(276.816053511706,-6.50167224080267)(281.759197324415,-6.44147157190636)(286.702341137124,-6.44147157190636)(291.645484949833,-6.38127090301003)(296.588628762542,-6.32107023411371)(301.531772575251,-6.26086956521739)(306.47491638796,-6.26086956521739)(311.418060200669,-6.26086956521739)(316.361204013378,-6.20066889632107)(321.304347826087,-6.20066889632107)(326.247491638796,-6.20066889632107)(331.190635451505,-6.20066889632107)(336.133779264214,-6.20066889632107)(341.076923076923,-6.20066889632107)(346.020066889632,-6.20066889632107)(350.963210702341,-6.20066889632107)(355.90635451505,-6.26086956521739)(360.849498327759,-6.26086956521739)(365.792642140468,-6.26086956521739)(370.735785953177,-6.32107023411371)(375.678929765886,-6.32107023411371)(380.622073578595,-6.38127090301003)(385.565217391304,-6.38127090301003)(390.508361204013,-6.44147157190636)(395.451505016722,-6.44147157190636)(400.394648829431,-6.50167224080267)(405.33779264214,-6.50167224080267)(410.28093645485,-6.561872909699)(415.224080267559,-6.561872909699)(420.167224080268,-6.62207357859532)(425.110367892977,-6.62207357859532)(430.053511705686,-6.68227424749164)(434.996655518395,-6.68227424749164)(439.939799331104,-6.68227424749164)(444.882943143813,-6.74247491638796)(449.826086956522,-6.74247491638796)(454.769230769231,-6.74247491638796)(459.71237458194,-6.74247491638796)(464.655518394649,-6.74247491638796)(469.598662207358,-6.74247491638796)(474.541806020067,-6.74247491638796)(479.484949832776,-6.74247491638796)(484.428093645485,-6.68227424749164)(489.371237458194,-6.68227424749164)(494.314381270903,-6.68227424749164)(499.257525083612,-6.68227424749164)(504.200668896321,-6.68227424749164)(509.14381270903,-6.68227424749164)(514.086956521739,-6.68227424749164)(519.030100334448,-6.68227424749164)(523.973244147157,-6.68227424749164)(528.916387959866,-6.68227424749164)(533.859531772575,-6.68227424749164)(538.802675585284,-6.68227424749164)(543.745819397993,-6.68227424749164)(548.688963210702,-6.68227424749164)(553.632107023411,-6.74247491638796)(558.57525083612,-6.74247491638796)(563.518394648829,-6.74247491638796)(568.461538461538,-6.74247491638796)(573.404682274248,-6.80267558528428)(578.347826086957,-6.80267558528428)(583.290969899666,-6.80267558528428)(588.234113712375,-6.80267558528428)(593.177257525084,-6.80267558528428)(598.120401337793,-6.80267558528428)(603.063545150502,-6.80267558528428)(608.006688963211,-6.80267558528428)(612.94983277592,-6.80267558528428)(617.892976588629,-6.80267558528428)(622.836120401338,-6.80267558528428)(627.779264214047,-6.80267558528428)(632.722408026756,-6.74247491638796)(637.665551839465,-6.74247491638796)(642.608695652174,-6.68227424749164)(647.551839464883,-6.68227424749164)(652.494983277592,-6.62207357859532)(657.438127090301,-6.62207357859532)(662.38127090301,-6.561872909699)(667.324414715719,-6.561872909699)(672.267558528428,-6.561872909699)(677.210702341137,-6.50167224080267)(682.153846153846,-6.50167224080267)(687.096989966555,-6.50167224080267)(692.040133779264,-6.44147157190636)(696.983277591973,-6.44147157190636)(701.926421404682,-6.44147157190636)(706.869565217391,-6.44147157190636)(711.8127090301,-6.44147157190636)(716.755852842809,-6.44147157190636)(721.698996655518,-6.44147157190636)(726.642140468227,-6.44147157190636)(731.585284280936,-6.44147157190636)(736.528428093646,-6.50167224080267)(741.471571906354,-6.50167224080267)(746.414715719064,-6.50167224080267)(751.357859531773,-6.561872909699)(756.301003344482,-6.561872909699)(761.244147157191,-6.62207357859532)(766.1872909699,-6.62207357859532)(771.130434782609,-6.68227424749164)(776.073578595318,-6.68227424749164)(781.016722408027,-6.74247491638796)(785.959866220736,-6.80267558528428)(790.903010033445,-6.80267558528428)(795.846153846154,-6.8628762541806)(800.789297658863,-6.92307692307692)(805.732441471572,-6.92307692307692)(810.675585284281,-6.98327759197324)(815.61872909699,-7.04347826086956)(820.561872909699,-7.04347826086956)(825.505016722408,-7.10367892976589)(830.448160535117,-7.10367892976589)(835.391304347826,-7.16387959866221)(840.334448160535,-7.16387959866221)(845.277591973244,-7.16387959866221)(850.220735785953,-7.16387959866221)(855.163879598662,-7.16387959866221)(860.107023411371,-7.16387959866221)(865.05016722408,-7.16387959866221)(869.993311036789,-7.16387959866221)(874.936454849498,-7.16387959866221)(879.879598662207,-7.10367892976589)(884.822742474916,-7.10367892976589)(889.765886287625,-7.10367892976589)(894.709030100334,-7.10367892976589)(899.652173913044,-7.04347826086956)(904.595317725752,-7.04347826086956)(909.538461538462,-7.04347826086956)(914.481605351171,-6.98327759197324)(919.42474916388,-6.98327759197324)(924.367892976589,-6.98327759197324)(929.311036789298,-6.98327759197324)(934.254180602007,-6.98327759197324)(939.197324414716,-6.98327759197324)(944.140468227425,-6.98327759197324)(949.083612040134,-6.98327759197324)(954.026755852843,-6.98327759197324)(958.969899665552,-6.98327759197324)(963.913043478261,-6.98327759197324)(968.85618729097,-6.98327759197324)(973.799331103679,-6.98327759197324)(978.742474916388,-7.04347826086956)(983.685618729097,-7.04347826086956)(988.628762541806,-7.04347826086956)(993.571906354515,-7.04347826086956)(998.515050167224,-7.10367892976589)(1003.45819397993,-7.10367892976589)(1008.40133779264,-7.16387959866221)(1013.34448160535,-7.16387959866221)(1018.28762541806,-7.16387959866221)(1023.23076923077,-7.22408026755853)(1028.17391304348,-7.22408026755853)(1033.11705685619,-7.22408026755853)(1038.0602006689,-7.22408026755853)(1043.00334448161,-7.22408026755853)(1047.94648829431,-7.22408026755853)(1052.88963210702,-7.22408026755853)(1057.83277591973,-7.16387959866221)(1062.77591973244,-7.16387959866221)(1067.71906354515,-7.10367892976589)(1072.66220735786,-7.04347826086956)(1077.60535117057,-6.98327759197324)(1082.54849498328,-6.92307692307692)(1087.49163879599,-6.8628762541806)(1092.4347826087,-6.80267558528428)(1097.3779264214,-6.74247491638796)(1102.32107023411,-6.68227424749164)(1102.32107023411,-6.62207357859532)(1107.26421404682,-6.561872909699)(1112.20735785953,-6.50167224080267)(1112.20735785953,-6.44147157190636)(1117.15050167224,-6.38127090301003)(1117.15050167224,-6.32107023411371)(1122.09364548495,-6.26086956521739)(1122.09364548495,-6.20066889632107)(1127.03678929766,-6.14046822742475)(1127.03678929766,-6.08026755852843)(1131.97993311037,-6.02006688963211)(1131.97993311037,-5.95986622073579)(1136.92307692308,-5.89966555183947)(1136.92307692308,-5.83946488294314)(1141.86622073579,-5.77926421404682)(1141.86622073579,-5.7190635451505)(1141.86622073579,-5.65886287625418)(1146.8093645485,-5.59866220735786)(1146.8093645485,-5.53846153846154)(1151.7525083612,-5.47826086956522)(1151.7525083612,-5.4180602006689)(1151.7525083612,-5.35785953177258)(1156.69565217391,-5.29765886287625)(1156.69565217391,-5.23745819397993)(1156.69565217391,-5.17725752508361)(1161.63879598662,-5.11705685618729)(1161.63879598662,-5.05685618729097)(1161.63879598662,-4.99665551839465)(1166.58193979933,-4.93645484949833)(1166.58193979933,-4.87625418060201)(1166.58193979933,-4.81605351170569)(1171.52508361204,-4.75585284280936)(1171.52508361204,-4.69565217391304)(1171.52508361204,-4.63545150501672)(1176.46822742475,-4.5752508361204)(1176.46822742475,-4.51505016722408)(1176.46822742475,-4.45484949832776)(1181.41137123746,-4.39464882943144)(1181.41137123746,-4.33444816053512)(1181.41137123746,-4.2742474916388)(1181.41137123746,-4.21404682274247)(1186.35451505017,-4.15384615384615)(1186.35451505017,-4.09364548494983)(1186.35451505017,-4.03344481605351)(1186.35451505017,-3.97324414715719)(1191.29765886288,-3.91304347826087)(1191.29765886288,-3.85284280936455)(1191.29765886288,-3.79264214046823)(1191.29765886288,-3.73244147157191)(1191.29765886288,-3.67224080267559)(1196.24080267559,-3.61204013377926)(1196.24080267559,-3.55183946488294)(1196.24080267559,-3.49163879598662)(1196.24080267559,-3.4314381270903)(1196.24080267559,-3.37123745819398)(1201.18394648829,-3.31103678929766)(1201.18394648829,-3.25083612040134)(1201.18394648829,-3.19063545150502)(1201.18394648829,-3.1304347826087)(1201.18394648829,-3.07023411371237)(1201.18394648829,-3.01003344481605)(1206.127090301,-2.94983277591973)(1206.127090301,-2.88963210702341)(1206.127090301,-2.82943143812709)(1206.127090301,-2.76923076923077)(1206.127090301,-2.70903010033445)(1206.127090301,-2.64882943143813)(1211.07023411371,-2.58862876254181)(1211.07023411371,-2.52842809364548)(1211.07023411371,-2.46822742474916)(1211.07023411371,-2.40802675585284)(1211.07023411371,-2.34782608695652)(1211.07023411371,-2.2876254180602)(1211.07023411371,-2.22742474916388)(1211.07023411371,-2.16722408026756)(1211.07023411371,-2.10702341137124)(1216.01337792642,-2.04682274247492)(1216.01337792642,-1.98662207357859)(1216.01337792642,-1.92642140468227)(1216.01337792642,-1.86622073578595)(1216.01337792642,-1.80602006688963)(1216.01337792642,-1.74581939799331)(1216.01337792642,-1.68561872909699)(1216.01337792642,-1.62541806020067)(1216.01337792642,-1.56521739130435)(1216.01337792642,-1.50501672240803)(1216.01337792642,-1.4448160535117)(1216.01337792642,-1.38461538461538)(1216.01337792642,-1.32441471571906)(1216.01337792642,-1.26421404682274)(1216.01337792642,-1.20401337792642)(1216.01337792642,-1.1438127090301)(1211.07023411371,-1.08361204013378)(1211.07023411371,-1.02341137123746)(1211.07023411371,-0.963210702341136)(1211.07023411371,-0.903010033444815)(1211.07023411371,-0.842809364548494)(1211.07023411371,-0.782608695652176)(1211.07023411371,-0.722408026755854)(1206.127090301,-0.662207357859533)(1206.127090301,-0.602006688963211)(1206.127090301,-0.54180602006689)(1206.127090301,-0.481605351170568)(1201.18394648829,-0.421404682274247)(1201.18394648829,-0.361204013377925)(1201.18394648829,-0.301003344481604)(1196.24080267559,-0.240802675585286)(1196.24080267559,-0.180602006688964)(1196.24080267559,-0.120401337792643)(1191.29765886288,-0.0602006688963215)(1191.29765886288,0)(1196.24080267559,0)(1201.18394648829,0)(1206.127090301,0)(1211.07023411371,0)(1216.01337792642,0)(1220.95652173913,0)(1225.89966555184,0)(1230.84280936455,0)(1235.78595317726,0)(1240.72909698997,0)(1245.67224080268,0)(1250.61538461538,0)(1255.55852842809,0)(1260.5016722408,0)(1265.44481605351,0)(1270.38795986622,0)(1275.33110367893,0)(1280.27424749164,0)(1285.21739130435,0)(1290.16053511706,0)(1295.10367892977,0)(1300.04682274247,0)(1304.98996655518,0)(1309.93311036789,0)(1314.8762541806,0)(1319.81939799331,0)(1324.76254180602,0)(1329.70568561873,0)(1334.64882943144,0)(1339.59197324415,0)(1344.53511705686,0)(1349.47826086957,0)(1354.42140468227,0)(1359.36454849498,0)(1364.30769230769,0)(1369.2508361204,0)(1374.19397993311,0)(1379.13712374582,0)(1384.08026755853,0)(1389.02341137124,0)(1393.96655518395,0)(1398.90969899666,0)(1403.85284280936,0)(1408.79598662207,0)(1413.73913043478,0)(1418.68227424749,0)(1423.6254180602,0)(1428.56856187291,0)(1433.51170568562,0)(1438.45484949833,0)(1443.39799331104,0)(1448.34113712375,0)(1453.28428093645,0)(1458.22742474916,0)(1463.17056856187,0)(1468.11371237458,0)(1473.05685618729,0)(1478,0)(1478,-0.0602006688963215)(1478,-0.120401337792643)(1478,-0.180602006688964)(1478,-0.240802675585286)(1478,-0.301003344481604)(1478,-0.361204013377925)(1478,-0.421404682274247)(1478,-0.481605351170568)(1478,-0.54180602006689)(1478,-0.602006688963211)(1478,-0.662207357859533)(1478,-0.722408026755854)(1478,-0.782608695652176)(1478,-0.842809364548494)(1478,-0.903010033444815)(1478,-0.963210702341136)(1478,-1.02341137123746)(1478,-1.08361204013378)(1478,-1.1438127090301)(1478,-1.20401337792642)(1478,-1.26421404682274)(1478,-1.32441471571906)(1478,-1.38461538461538)(1478,-1.4448160535117)(1478,-1.50501672240803)(1478,-1.56521739130435)(1478,-1.62541806020067)(1478,-1.68561872909699)(1478,-1.74581939799331)(1478,-1.80602006688963)(1478,-1.86622073578595)(1478,-1.92642140468227)(1478,-1.98662207357859)(1478,-2.04682274247492)(1478,-2.10702341137124)(1478,-2.16722408026756)(1478,-2.22742474916388)(1478,-2.2876254180602)(1478,-2.34782608695652)(1478,-2.40802675585284)(1478,-2.46822742474916)(1478,-2.52842809364548)(1478,-2.58862876254181)(1478,-2.64882943143813)(1478,-2.70903010033445)(1478,-2.76923076923077)(1478,-2.82943143812709)(1478,-2.88963210702341)(1478,-2.94983277591973)(1478,-3.01003344481605)(1478,-3.07023411371237)(1478,-3.1304347826087)(1478,-3.19063545150502)(1478,-3.25083612040134)(1478,-3.31103678929766)(1478,-3.37123745819398)(1478,-3.4314381270903)(1478,-3.49163879598662)(1478,-3.55183946488294)(1478,-3.61204013377926)(1478,-3.67224080267559)(1478,-3.73244147157191)(1478,-3.79264214046823)(1478,-3.85284280936455)(1478,-3.91304347826087)(1478,-3.97324414715719)(1478,-4.03344481605351)(1478,-4.09364548494983)(1478,-4.15384615384615)(1478,-4.21404682274247)(1478,-4.2742474916388)(1478,-4.33444816053512)(1478,-4.39464882943144)(1478,-4.45484949832776)(1478,-4.51505016722408)(1478,-4.5752508361204)(1478,-4.63545150501672)(1478,-4.69565217391304)(1478,-4.75585284280936)(1478,-4.81605351170569)(1478,-4.87625418060201)(1478,-4.93645484949833)(1478,-4.99665551839465)(1478,-5.05685618729097)(1478,-5.11705685618729)(1478,-5.17725752508361)(1478,-5.23745819397993)(1478,-5.29765886287625)(1478,-5.35785953177258)(1478,-5.4180602006689)(1478,-5.47826086956522)(1478,-5.53846153846154)(1478,-5.59866220735786)(1478,-5.65886287625418)(1478,-5.7190635451505)(1478,-5.77926421404682)(1478,-5.83946488294314)(1478,-5.89966555183947)(1478,-5.95986622073579)(1478,-6.02006688963211)(1478,-6.08026755852843)(1478,-6.14046822742475)(1478,-6.20066889632107)(1478,-6.26086956521739)(1478,-6.32107023411371)(1478,-6.38127090301003)(1478,-6.44147157190636)(1478,-6.50167224080267)(1478,-6.561872909699)(1478,-6.62207357859532)(1478,-6.68227424749164)(1478,-6.74247491638796)(1478,-6.80267558528428)(1478,-6.8628762541806)(1478,-6.92307692307692)(1478,-6.98327759197324)(1478,-7.04347826086956)(1478,-7.10367892976589)(1478,-7.16387959866221)(1478,-7.22408026755853)(1478,-7.28428093645485)(1478,-7.34448160535117)(1478,-7.40468227424749)(1478,-7.46488294314381)(1478,-7.52508361204013)(1478,-7.58528428093645)(1478,-7.64548494983278)(1478,-7.7056856187291)(1478,-7.76588628762542)(1478,-7.82608695652174)(1478,-7.88628762541806)(1478,-7.94648829431438)(1478,-8.0066889632107)(1478,-8.06688963210702)(1478,-8.12709030100334)(1478,-8.18729096989967)(1478,-8.24749163879599)(1478,-8.30769230769231)(1478,-8.36789297658863)(1478,-8.42809364548495)(1478,-8.48829431438127)(1478,-8.54849498327759)(1478,-8.60869565217391)(1478,-8.66889632107023)(1478,-8.72909698996656)(1478,-8.78929765886288)(1478,-8.8494983277592)(1478,-8.90969899665552)(1478,-8.96989966555184)(1478,-9.03010033444816)(1478,-9.09030100334448)(1478,-9.1505016722408)(1478,-9.21070234113712)(1478,-9.27090301003344)(1478,-9.33110367892977)(1478,-9.39130434782609)(1478,-9.45150501672241)(1478,-9.51170568561873)(1478,-9.57190635451505)(1478,-9.63210702341137)(1478,-9.69230769230769)(1478,-9.75250836120401)(1478,-9.81270903010033)(1478,-9.87290969899666)(1478,-9.93311036789298)(1478,-9.9933110367893)(1478,-10.0535117056856)(1478,-10.1137123745819)(1478,-10.1739130434783)(1478,-10.2341137123746)(1478,-10.2943143812709)(1478,-10.3545150501672)(1478,-10.4147157190635)(1478,-10.4749163879599)(1478,-10.5351170568562)(1478,-10.5953177257525)(1478,-10.6555183946488)(1478,-10.7157190635452)(1478,-10.7759197324415)(1478,-10.8361204013378)(1478,-10.8963210702341)(1478,-10.9565217391304)(1478,-11.0167224080268)(1478,-11.0769230769231)(1478,-11.1371237458194)(1478,-11.1973244147157)(1478,-11.257525083612)(1478,-11.3177257525084)(1478,-11.3779264214047)(1478,-11.438127090301)(1478,-11.4983277591973)(1478,-11.5585284280936)(1478,-11.61872909699)(1478,-11.6789297658863)(1478,-11.7391304347826)(1478,-11.7993311036789)(1478,-11.8595317725753)(1478,-11.9197324414716)(1478,-11.9799331103679)(1478,-12.0401337792642)(1478,-12.1003344481605)(1478,-12.1605351170569)(1478,-12.2207357859532)(1478,-12.2809364548495)(1478,-12.3411371237458)(1478,-12.4013377926421)(1478,-12.4615384615385)(1478,-12.5217391304348)(1478,-12.5819397993311)(1478,-12.6421404682274)(1478,-12.7023411371237)(1478,-12.7625418060201)(1478,-12.8227424749164)(1478,-12.8829431438127)(1478,-12.943143812709)(1478,-13.0033444816054)(1478,-13.0635451505017)(1478,-13.123745819398)(1478,-13.1839464882943)(1478,-13.2441471571906)(1478,-13.304347826087)(1478,-13.3645484949833)(1478,-13.4247491638796)(1478,-13.4849498327759)(1478,-13.5451505016722)(1478,-13.6053511705686)(1478,-13.6655518394649)(1478,-13.7257525083612)(1478,-13.7859531772575)(1478,-13.8461538461538)(1478,-13.9063545150502)(1478,-13.9665551839465)(1478,-14.0267558528428)(1478,-14.0869565217391)(1478,-14.1471571906355)(1478,-14.2073578595318)(1478,-14.2675585284281)(1478,-14.3277591973244)(1478,-14.3879598662207)(1478,-14.4481605351171)(1478,-14.5083612040134)(1478,-14.5685618729097)(1478,-14.628762541806)(1478,-14.6889632107023)(1478,-14.7491638795987)(1478,-14.809364548495)(1478,-14.8695652173913)(1478,-14.9297658862876)(1478,-14.9899665551839)(1478,-15.0501672240803)(1478,-15.1103678929766)(1478,-15.1705685618729)(1478,-15.2307692307692)(1478,-15.2909698996656)(1478,-15.3511705685619)(1478,-15.4113712374582)(1478,-15.4715719063545)(1478,-15.5317725752508)(1478,-15.5919732441472)(1478,-15.6521739130435)(1478,-15.7123745819398)(1478,-15.7725752508361)(1478,-15.8327759197324)(1478,-15.8929765886288)(1478,-15.9531772575251)(1478,-16.0133779264214)(1478,-16.0735785953177)(1478,-16.133779264214)(1478,-16.1939799331104)(1478,-16.2541806020067)(1478,-16.314381270903)(1478,-16.3745819397993)(1478,-16.4347826086957)(1478,-16.494983277592)(1478,-16.5551839464883)(1478,-16.6153846153846)(1478,-16.6755852842809)(1478,-16.7357859531773)(1478,-16.7959866220736)(1478,-16.8561872909699)(1478,-16.9163879598662)(1478,-16.9765886287625)(1478,-17.0367892976589)(1478,-17.0969899665552)(1478,-17.1571906354515)(1478,-17.2173913043478)(1478,-17.2775919732441)(1478,-17.3377926421405)(1478,-17.3979933110368)(1478,-17.4581939799331)(1478,-17.5183946488294)(1478,-17.5785953177258)(1478,-17.6387959866221)(1478,-17.6989966555184)(1478,-17.7591973244147)(1478,-17.819397993311)(1478,-17.8795986622074)(1478,-17.9397993311037)(1478,-18)(1473.05685618729,-18)(1468.11371237458,-18)(1463.17056856187,-18)(1458.22742474916,-18)(1453.28428093645,-18)(1448.34113712375,-18)(1443.39799331104,-18)(1438.45484949833,-18)(1433.51170568562,-18)(1428.56856187291,-18)(1423.6254180602,-18)(1418.68227424749,-18)(1413.73913043478,-18)(1408.79598662207,-18)(1403.85284280936,-18)(1398.90969899666,-18)(1393.96655518395,-18)(1389.02341137124,-18)(1384.08026755853,-18)(1379.13712374582,-18)(1374.19397993311,-18)(1369.2508361204,-18)(1364.30769230769,-18)(1359.36454849498,-18)(1354.42140468227,-18)(1349.47826086957,-18)(1344.53511705686,-18)(1339.59197324415,-18)(1334.64882943144,-18)(1329.70568561873,-18)(1324.76254180602,-18)(1319.81939799331,-18)(1314.8762541806,-18)(1309.93311036789,-18)(1304.98996655518,-18)(1300.04682274247,-18)(1295.10367892977,-18)(1290.16053511706,-18)(1285.21739130435,-18)(1280.27424749164,-18)(1275.33110367893,-18)(1270.38795986622,-18)(1265.44481605351,-18)(1260.5016722408,-18)(1255.55852842809,-18)(1250.61538461538,-18)(1245.67224080268,-18)(1240.72909698997,-18)(1235.78595317726,-18)(1230.84280936455,-18)(1225.89966555184,-18)(1220.95652173913,-18)(1216.01337792642,-18)(1211.07023411371,-18)(1206.127090301,-18)(1201.18394648829,-18)(1196.24080267559,-18)(1191.29765886288,-18)(1186.35451505017,-18)(1181.41137123746,-18)(1176.46822742475,-18)(1171.52508361204,-18)(1166.58193979933,-18)(1161.63879598662,-18)(1156.69565217391,-18)(1151.7525083612,-18)(1146.8093645485,-18)(1141.86622073579,-18)(1136.92307692308,-18)(1131.97993311037,-18)(1127.03678929766,-18)(1122.09364548495,-18)(1117.15050167224,-18)(1112.20735785953,-18)(1107.26421404682,-18)(1102.32107023411,-18)(1097.3779264214,-18)(1092.4347826087,-18)(1087.49163879599,-18)(1082.54849498328,-18)(1077.60535117057,-18)(1072.66220735786,-18)(1067.71906354515,-18)(1062.77591973244,-18)(1057.83277591973,-18)(1052.88963210702,-18)(1047.94648829431,-18)(1043.00334448161,-18)(1038.0602006689,-18)(1033.11705685619,-18)(1028.17391304348,-18)(1023.23076923077,-18)(1018.28762541806,-18)(1013.34448160535,-18)(1008.40133779264,-18)(1003.45819397993,-18)(998.515050167224,-18)(993.571906354515,-18)(988.628762541806,-18)(983.685618729097,-18)(978.742474916388,-18)(973.799331103679,-18)(968.85618729097,-18)(968.85618729097,-17.9397993311037)(968.85618729097,-17.8795986622074)(968.85618729097,-17.819397993311)(968.85618729097,-17.7591973244147)(968.85618729097,-17.6989966555184)(968.85618729097,-17.6387959866221)(968.85618729097,-17.5785953177258)(968.85618729097,-17.5183946488294)(968.85618729097,-17.4581939799331)(963.913043478261,-17.3979933110368)(963.913043478261,-17.3377926421405)(963.913043478261,-17.2775919732441)(963.913043478261,-17.2173913043478)(963.913043478261,-17.1571906354515)(963.913043478261,-17.0969899665552)(963.913043478261,-17.0367892976589)(963.913043478261,-16.9765886287625)(958.969899665552,-16.9163879598662)(958.969899665552,-16.8561872909699)(958.969899665552,-16.7959866220736)(958.969899665552,-16.7357859531773)(958.969899665552,-16.6755852842809)(958.969899665552,-16.6153846153846)(958.969899665552,-16.5551839464883)(958.969899665552,-16.494983277592)(958.969899665552,-16.4347826086957)(958.969899665552,-16.3745819397993)(958.969899665552,-16.314381270903)(958.969899665552,-16.2541806020067)(958.969899665552,-16.1939799331104)(958.969899665552,-16.133779264214)(963.913043478261,-16.0735785953177)(963.913043478261,-16.0133779264214)(963.913043478261,-15.9531772575251)(963.913043478261,-15.8929765886288)(963.913043478261,-15.8327759197324)(963.913043478261,-15.7725752508361)(963.913043478261,-15.7123745819398)(963.913043478261,-15.6521739130435)(963.913043478261,-15.5919732441472)(963.913043478261,-15.5317725752508)(968.85618729097,-15.4715719063545)(968.85618729097,-15.4113712374582)(968.85618729097,-15.3511705685619)(968.85618729097,-15.2909698996656)(968.85618729097,-15.2307692307692)(968.85618729097,-15.1705685618729)(968.85618729097,-15.1103678929766)(968.85618729097,-15.0501672240803)(968.85618729097,-14.9899665551839)(968.85618729097,-14.9297658862876)(968.85618729097,-14.8695652173913)(973.799331103679,-14.809364548495)(973.799331103679,-14.7491638795987)(973.799331103679,-14.6889632107023)(973.799331103679,-14.628762541806)(973.799331103679,-14.5685618729097)(973.799331103679,-14.5083612040134)(973.799331103679,-14.4481605351171)(973.799331103679,-14.3879598662207)(973.799331103679,-14.3277591973244)(973.799331103679,-14.2675585284281)(973.799331103679,-14.2073578595318)(973.799331103679,-14.1471571906355)(973.799331103679,-14.0869565217391)(973.799331103679,-14.0267558528428)(973.799331103679,-13.9665551839465)(973.799331103679,-13.9063545150502)(973.799331103679,-13.8461538461538)(973.799331103679,-13.7859531772575)(973.799331103679,-13.7257525083612)(973.799331103679,-13.6655518394649)(973.799331103679,-13.6053511705686)(973.799331103679,-13.5451505016722)(973.799331103679,-13.4849498327759)(973.799331103679,-13.4247491638796)(973.799331103679,-13.3645484949833)(973.799331103679,-13.304347826087)(973.799331103679,-13.2441471571906)(973.799331103679,-13.1839464882943)(973.799331103679,-13.123745819398)(973.799331103679,-13.0635451505017)(973.799331103679,-13.0033444816054)(973.799331103679,-12.943143812709)(973.799331103679,-12.8829431438127)(973.799331103679,-12.8227424749164)(973.799331103679,-12.7625418060201)(968.85618729097,-12.7023411371237)(968.85618729097,-12.6421404682274)(968.85618729097,-12.5819397993311)(968.85618729097,-12.5217391304348)(968.85618729097,-12.4615384615385)(968.85618729097,-12.4013377926421)(963.913043478261,-12.3411371237458)(963.913043478261,-12.2809364548495)(963.913043478261,-12.2207357859532)(963.913043478261,-12.1605351170569)(963.913043478261,-12.1003344481605)(958.969899665552,-12.0401337792642)(958.969899665552,-11.9799331103679)(958.969899665552,-11.9197324414716)(954.026755852843,-11.8595317725753)(954.026755852843,-11.7993311036789)(954.026755852843,-11.7391304347826)(949.083612040134,-11.6789297658863)(949.083612040134,-11.61872909699)(949.083612040134,-11.5585284280936)(944.140468227425,-11.4983277591973)(944.140468227425,-11.438127090301)(939.197324414716,-11.3779264214047)(939.197324414716,-11.3177257525084)(934.254180602007,-11.257525083612)(934.254180602007,-11.1973244147157)(929.311036789298,-11.1371237458194)(929.311036789298,-11.0769230769231)(924.367892976589,-11.0167224080268)(924.367892976589,-10.9565217391304)(919.42474916388,-10.8963210702341)(914.481605351171,-10.8361204013378)(914.481605351171,-10.7759197324415)(909.538461538462,-10.7157190635452)(904.595317725752,-10.6555183946488)(904.595317725752,-10.5953177257525)(899.652173913044,-10.5351170568562)(894.709030100334,-10.4749163879599)(889.765886287625,-10.4147157190635)(884.822742474916,-10.3545150501672)(879.879598662207,-10.2943143812709)(874.936454849498,-10.2341137123746)(869.993311036789,-10.1739130434783)(865.05016722408,-10.1137123745819)(860.107023411371,-10.0535117056856)(855.163879598662,-9.9933110367893)(850.220735785953,-9.93311036789298)(845.277591973244,-9.87290969899666)(840.334448160535,-9.81270903010033)(835.391304347826,-9.81270903010033)(830.448160535117,-9.75250836120401)(825.505016722408,-9.75250836120401)(820.561872909699,-9.75250836120401)(815.61872909699,-9.75250836120401)(810.675585284281,-9.69230769230769)(805.732441471572,-9.69230769230769)(800.789297658863,-9.69230769230769)(795.846153846154,-9.69230769230769)(790.903010033445,-9.63210702341137)(785.959866220736,-9.63210702341137)(781.016722408027,-9.63210702341137)(776.073578595318,-9.63210702341137)(771.130434782609,-9.63210702341137)(766.1872909699,-9.63210702341137)(761.244147157191,-9.63210702341137)(756.301003344482,-9.57190635451505)(751.357859531773,-9.57190635451505)(746.414715719064,-9.57190635451505)(741.471571906354,-9.57190635451505)(736.528428093646,-9.51170568561873)(731.585284280936,-9.51170568561873)(726.642140468227,-9.51170568561873)(721.698996655518,-9.45150501672241)(716.755852842809,-9.39130434782609)(711.8127090301,-9.33110367892977)(706.869565217391,-9.27090301003344)(701.926421404682,-9.21070234113712)(696.983277591973,-9.1505016722408)(692.040133779264,-9.09030100334448)(687.096989966555,-9.03010033444816)(682.153846153846,-8.96989966555184)(677.210702341137,-8.96989966555184)(672.267558528428,-8.90969899665552)(667.324414715719,-8.8494983277592)(662.38127090301,-8.78929765886288)(657.438127090301,-8.72909698996656)(652.494983277592,-8.72909698996656)(647.551839464883,-8.66889632107023)(642.608695652174,-8.66889632107023)(637.665551839465,-8.60869565217391)(632.722408026756,-8.60869565217391)(627.779264214047,-8.60869565217391)(622.836120401338,-8.60869565217391)(617.892976588629,-8.66889632107023)(612.94983277592,-8.66889632107023)(608.006688963211,-8.66889632107023)(603.063545150502,-8.66889632107023)(598.120401337793,-8.72909698996656)(593.177257525084,-8.72909698996656)(588.234113712375,-8.78929765886288)(583.290969899666,-8.78929765886288)(578.347826086957,-8.8494983277592)(573.404682274248,-8.8494983277592)(568.461538461538,-8.90969899665552)(563.518394648829,-8.90969899665552)(558.57525083612,-8.96989966555184)(553.632107023411,-8.96989966555184)(548.688963210702,-9.03010033444816)(543.745819397993,-9.03010033444816)(538.802675585284,-9.03010033444816)(533.859531772575,-9.09030100334448)(528.916387959866,-9.09030100334448)(523.973244147157,-9.09030100334448)(519.030100334448,-9.09030100334448)(514.086956521739,-9.09030100334448)(509.14381270903,-9.09030100334448)(504.200668896321,-9.09030100334448)(499.257525083612,-9.09030100334448)(494.314381270903,-9.03010033444816)(489.371237458194,-9.03010033444816)(484.428093645485,-9.03010033444816)(479.484949832776,-8.96989966555184)(474.541806020067,-8.96989966555184)(469.598662207358,-8.90969899665552)(464.655518394649,-8.90969899665552)(459.71237458194,-8.8494983277592)(454.769230769231,-8.8494983277592)(449.826086956522,-8.8494983277592)(444.882943143813,-8.78929765886288)(439.939799331104,-8.78929765886288)(434.996655518395,-8.78929765886288)(430.053511705686,-8.8494983277592)(425.110367892977,-8.8494983277592)(420.167224080268,-8.8494983277592)(415.224080267559,-8.90969899665552)(410.28093645485,-8.90969899665552)(405.33779264214,-8.96989966555184)(400.394648829431,-9.03010033444816)(395.451505016722,-9.09030100334448)(390.508361204013,-9.09030100334448)(385.565217391304,-9.1505016722408)(380.622073578595,-9.21070234113712)(375.678929765886,-9.21070234113712)(370.735785953177,-9.27090301003344)(365.792642140468,-9.33110367892977)(360.849498327759,-9.33110367892977)(355.90635451505,-9.33110367892977)(350.963210702341,-9.39130434782609)(346.020066889632,-9.39130434782609)(341.076923076923,-9.39130434782609)(336.133779264214,-9.45150501672241)(331.190635451505,-9.45150501672241)(326.247491638796,-9.45150501672241)(321.304347826087,-9.45150501672241)(316.361204013378,-9.45150501672241)(311.418060200669,-9.45150501672241)(306.47491638796,-9.39130434782609)(301.531772575251,-9.39130434782609)(296.588628762542,-9.39130434782609)(291.645484949833,-9.39130434782609)(286.702341137124,-9.33110367892977)(281.759197324415,-9.33110367892977)(276.816053511706,-9.33110367892977)(271.872909698997,-9.27090301003344)(266.929765886288,-9.27090301003344)(261.986622073579,-9.21070234113712)(257.04347826087,-9.21070234113712)(252.100334448161,-9.21070234113712)(247.157190635452,-9.1505016722408)(242.214046822742,-9.1505016722408)(237.270903010033,-9.1505016722408)(232.327759197324,-9.1505016722408)(227.384615384615,-9.1505016722408)(222.441471571906,-9.1505016722408)(217.498327759197,-9.1505016722408)(212.555183946488,-9.21070234113712)(207.612040133779,-9.21070234113712)(202.66889632107,-9.27090301003344)(197.725752508361,-9.27090301003344)(192.782608695652,-9.33110367892977)(187.839464882943,-9.39130434782609)(182.896321070234,-9.45150501672241)(177.953177257525,-9.45150501672241)(173.010033444816,-9.51170568561873)(168.066889632107,-9.57190635451505)(163.123745819398,-9.63210702341137)(158.180602006689,-9.69230769230769)(153.23745819398,-9.75250836120401)(148.294314381271,-9.81270903010033)(143.351170568562,-9.87290969899666)(138.408026755853,-9.93311036789298)(133.464882943144,-9.9933110367893)(128.521739130435,-10.0535117056856)(123.578595317726,-10.1137123745819)(118.635451505017,-10.1739130434783)(118.635451505017,-10.2341137123746)(113.692307692308,-10.2943143812709)(108.749163879599,-10.3545150501672)(103.80602006689,-10.4147157190635)(103.80602006689,-10.4749163879599)(98.8628762541806,-10.5351170568562)(93.9197324414716,-10.5953177257525)(93.9197324414716,-10.6555183946488)(88.9765886287625,-10.7157190635452)(88.9765886287625,-10.7759197324415)(84.0334448160535,-10.8361204013378)(79.0903010033445,-10.8963210702341)(79.0903010033445,-10.9565217391304)(74.1471571906355,-11.0167224080268)(74.1471571906355,-11.0769230769231)(74.1471571906355,-11.1371237458194)(69.2040133779264,-11.1973244147157)(69.2040133779264,-11.257525083612)(64.2608695652174,-11.3177257525084)(64.2608695652174,-11.3779264214047)(64.2608695652174,-11.438127090301)(59.3177257525084,-11.4983277591973)(59.3177257525084,-11.5585284280936)(59.3177257525084,-11.61872909699)(59.3177257525084,-11.6789297658863)(54.3745819397993,-11.7391304347826)(54.3745819397993,-11.7993311036789)(54.3745819397993,-11.8595317725753)(54.3745819397993,-11.9197324414716)(54.3745819397993,-11.9799331103679)(49.4314381270903,-12.0401337792642)(49.4314381270903,-12.1003344481605)(49.4314381270903,-12.1605351170569)(49.4314381270903,-12.2207357859532)(49.4314381270903,-12.2809364548495)(49.4314381270903,-12.3411371237458)(49.4314381270903,-12.4013377926421)(49.4314381270903,-12.4615384615385)(49.4314381270903,-12.5217391304348)(49.4314381270903,-12.5819397993311)(49.4314381270903,-12.6421404682274)(49.4314381270903,-12.7023411371237)(49.4314381270903,-12.7625418060201)(49.4314381270903,-12.8227424749164)(49.4314381270903,-12.8829431438127)(49.4314381270903,-12.943143812709)(49.4314381270903,-13.0033444816054)(49.4314381270903,-13.0635451505017)(49.4314381270903,-13.123745819398)(54.3745819397993,-13.1839464882943)(54.3745819397993,-13.2441471571906)(54.3745819397993,-13.304347826087)(54.3745819397993,-13.3645484949833)(54.3745819397993,-13.4247491638796)(54.3745819397993,-13.4849498327759)(54.3745819397993,-13.5451505016722)(54.3745819397993,-13.6053511705686)(54.3745819397993,-13.6655518394649)(54.3745819397993,-13.7257525083612)(54.3745819397993,-13.7859531772575)(59.3177257525084,-13.8461538461538)(59.3177257525084,-13.9063545150502)(59.3177257525084,-13.9665551839465)(59.3177257525084,-14.0267558528428)(59.3177257525084,-14.0869565217391)(59.3177257525084,-14.1471571906355)(59.3177257525084,-14.2073578595318)(54.3745819397993,-14.2675585284281)(54.3745819397993,-14.3277591973244)(54.3745819397993,-14.3879598662207)(54.3745819397993,-14.4481605351171)(54.3745819397993,-14.5083612040134)(54.3745819397993,-14.5685618729097)(54.3745819397993,-14.628762541806)(54.3745819397993,-14.6889632107023)(54.3745819397993,-14.7491638795987)(49.4314381270903,-14.809364548495)(49.4314381270903,-14.8695652173913)(49.4314381270903,-14.9297658862876)(49.4314381270903,-14.9899665551839)(49.4314381270903,-15.0501672240803)(44.4882943143813,-15.1103678929766)(44.4882943143813,-15.1705685618729)(44.4882943143813,-15.2307692307692)(44.4882943143813,-15.2909698996656)(44.4882943143813,-15.3511705685619)(44.4882943143813,-15.4113712374582)(39.5451505016722,-15.4715719063545)(39.5451505016722,-15.5317725752508)(39.5451505016722,-15.5919732441472)(39.5451505016722,-15.6521739130435)(39.5451505016722,-15.7123745819398)(39.5451505016722,-15.7725752508361)(34.6020066889632,-15.8327759197324)(34.6020066889632,-15.8929765886288)(34.6020066889632,-15.9531772575251)(34.6020066889632,-16.0133779264214)(34.6020066889632,-16.0735785953177)(34.6020066889632,-16.133779264214)(34.6020066889632,-16.1939799331104)(34.6020066889632,-16.2541806020067)(34.6020066889632,-16.314381270903)(34.6020066889632,-16.3745819397993)(34.6020066889632,-16.4347826086957)(34.6020066889632,-16.494983277592)(34.6020066889632,-16.5551839464883)(34.6020066889632,-16.6153846153846)(34.6020066889632,-16.6755852842809)(34.6020066889632,-16.7357859531773)(34.6020066889632,-16.7959866220736)(34.6020066889632,-16.8561872909699)(34.6020066889632,-16.9163879598662)(34.6020066889632,-16.9765886287625)(34.6020066889632,-17.0367892976589)(34.6020066889632,-17.0969899665552)(34.6020066889632,-17.1571906354515)(34.6020066889632,-17.2173913043478)(34.6020066889632,-17.2775919732441)(34.6020066889632,-17.3377926421405)(34.6020066889632,-17.3979933110368)(34.6020066889632,-17.4581939799331)(34.6020066889632,-17.5183946488294)(34.6020066889632,-17.5785953177258)(34.6020066889632,-17.6387959866221)(39.5451505016722,-17.6989966555184)(39.5451505016722,-17.7591973244147)(39.5451505016722,-17.819397993311)(39.5451505016722,-17.8795986622074)(39.5451505016722,-17.9397993311037)(39.5451505016722,-18)(34.6020066889632,-18)(29.6588628762542,-18)(24.7157190635452,-18)(19.7725752508361,-18)(14.8294314381271,-18)(9.88628762541806,-18)(4.94314381270903,-18)(0,-18.0000030097325)(-0.000247132477387754,-18)(0,-17.9397993311037)(0,-17.8795986622074)(0,-17.819397993311)(0,-17.7591973244147)(0,-17.6989966555184)(0,-17.6387959866221)(0,-17.5785953177258)(0,-17.5183946488294)(0,-17.4581939799331)(0,-17.3979933110368)(0,-17.3377926421405)(0,-17.2775919732441)(0,-17.2173913043478)(0,-17.1571906354515)(0,-17.0969899665552)(0,-17.0367892976589)(0,-16.9765886287625)(0,-16.9163879598662)(0,-16.8561872909699)(0,-16.7959866220736)(0,-16.7357859531773)(0,-16.6755852842809)(0,-16.6153846153846)(0,-16.5551839464883)(0,-16.494983277592)(0,-16.4347826086957)(0,-16.3745819397993)(0,-16.314381270903)(0,-16.2541806020067)(0,-16.1939799331104)(0,-16.133779264214)(0,-16.0735785953177)(0,-16.0133779264214)(0,-15.9531772575251)(0,-15.8929765886288)(0,-15.8327759197324)(0,-15.7725752508361)(0,-15.7123745819398)(0,-15.6521739130435)(0,-15.5919732441472)(0,-15.5317725752508)(0,-15.4715719063545)(0,-15.4113712374582)(0,-15.3511705685619)(0,-15.2909698996656)(0,-15.2307692307692)(0,-15.1705685618729)(0,-15.1103678929766)(0,-15.0501672240803)(0,-14.9899665551839)(0,-14.9297658862876)(0,-14.8695652173913)(0,-14.809364548495)(0,-14.7491638795987)(0,-14.6889632107023)(0,-14.628762541806)(0,-14.5685618729097)(0,-14.5083612040134)(0,-14.4481605351171)(0,-14.3879598662207)(0,-14.3277591973244)(0,-14.2675585284281)(0,-14.2073578595318)(0,-14.1471571906355)(0,-14.0869565217391)(0,-14.0267558528428)(0,-13.9665551839465)(0,-13.9063545150502)(0,-13.8461538461538)(0,-13.7859531772575)(0,-13.7257525083612)(0,-13.6655518394649)(0,-13.6053511705686)(0,-13.5451505016722)(0,-13.4849498327759)(0,-13.4247491638796)(0,-13.3645484949833)(0,-13.304347826087)(0,-13.2441471571906)(0,-13.1839464882943)(0,-13.123745819398)(0,-13.0635451505017)(0,-13.0033444816054)(0,-12.943143812709)(0,-12.8829431438127)(0,-12.8227424749164)(0,-12.7625418060201)(0,-12.7023411371237)(0,-12.6421404682274)(0,-12.5819397993311)(0,-12.5217391304348)(0,-12.4615384615385)(0,-12.4013377926421)(0,-12.3411371237458)(0,-12.2809364548495)(0,-12.2207357859532)(0,-12.1605351170569)(0,-12.1003344481605)(0,-12.0401337792642)(0,-11.9799331103679)(0,-11.9197324414716)(0,-11.8595317725753)(0,-11.7993311036789)(0,-11.7391304347826)(0,-11.6789297658863)(0,-11.61872909699)(0,-11.5585284280936)(0,-11.4983277591973)(0,-11.438127090301)(0,-11.3779264214047)(0,-11.3177257525084)(0,-11.257525083612)(0,-11.1973244147157)(0,-11.1371237458194)(0,-11.0769230769231)(0,-11.0167224080268)(0,-10.9565217391304)(0,-10.8963210702341)(0,-10.8361204013378)(0,-10.7759197324415)(0,-10.7157190635452)(0,-10.6555183946488)(0,-10.5953177257525)(0,-10.5351170568562)(0,-10.4749163879599)(0,-10.4147157190635)(0,-10.3545150501672)(0,-10.2943143812709)(0,-10.2341137123746)(0,-10.1739130434783)(0,-10.1137123745819)(0,-10.0535117056856)(0,-9.9933110367893)(0,-9.93311036789298)(0,-9.87290969899666)(0,-9.81270903010033)(0,-9.75250836120401)(0,-9.69230769230769)(0,-9.63210702341137)(0,-9.57190635451505)(0,-9.51170568561873)(0,-9.45150501672241)(0,-9.39130434782609)(0,-9.33110367892977)(0,-9.27090301003344)(0,-9.21070234113712)(0,-9.1505016722408)(0,-9.09030100334448)(0,-9.03010033444816)(0,-8.96989966555184)(0,-8.90969899665552)(0,-8.8494983277592)(0,-8.78929765886288)(0,-8.72909698996656)(0,-8.66889632107023)(0,-8.60869565217391)(0,-8.54849498327759)(0,-8.48829431438127)(0,-8.42809364548495)(0,-8.36789297658863)(0,-8.30769230769231)(0,-8.24749163879599)(0,-8.18729096989967)(0,-8.12709030100334)(0,-8.06688963210702)(0,-8.0066889632107)(0,-7.94648829431438)(0,-7.88628762541806)(0,-7.82608695652174)(0,-7.76588628762542)(0,-7.7056856187291)(0,-7.64548494983278)(0,-7.58528428093645)(0,-7.52508361204013)(0,-7.46488294314381)(0,-7.40468227424749)(0,-7.34448160535117)(0,-7.28428093645485)(0,-7.22408026755853)(0,-7.16387959866221)(0,-7.10367892976589)(0,-7.04347826086956)(0,-6.98327759197324)(0,-6.92307692307692)(0,-6.8628762541806)(0,-6.80267558528428)(0,-6.74247491638796)(0,-6.68227424749164)(0,-6.62207357859532)(0,-6.561872909699)(0,-6.50167224080267)(0,-6.44147157190636)(0,-6.38127090301003)(0,-6.32107023411371)(0,-6.26086956521739)(0,-6.20066889632107)(0,-6.14046822742475)(0,-6.08026755852843)(0,-6.02006688963211)(0,-5.95986622073579)(0,-5.89966555183947)(0,-5.83946488294314)(0,-5.77926421404682)(0,-5.7190635451505)(0,-5.65886287625418)(0,-5.59866220735786)(0,-5.53846153846154)(0,-5.47826086956522)(0,-5.4180602006689)(0,-5.35785953177258)(0,-5.29765886287625)(0,-5.23745819397993)(0,-5.17725752508361)(0,-5.11705685618729)(0,-5.05685618729097)(0,-4.99665551839465)(0,-4.93645484949833)(0,-4.87625418060201)(0,-4.81605351170569)(0,-4.75585284280936)(0,-4.69565217391304)(0,-4.63545150501672)};

\addplot [fill=red!40!yellow,draw=black,forget plot] coordinates{ (1092.4347826087,-8.51839464882943)(1094.90635451505,-8.54849498327759)(1094.90635451505,-8.60869565217391)(1092.4347826087,-8.63879598662207)(1089.96321070234,-8.66889632107023)(1089.96321070234,-8.72909698996656)(1089.96321070234,-8.78929765886288)(1089.96321070234,-8.8494983277592)(1087.49163879599,-8.87959866220736)(1085.02006688963,-8.90969899665552)(1082.54849498328,-8.93979933110368)(1080.07692307692,-8.96989966555184)(1077.60535117057,-9)(1075.13377926421,-9.03010033444816)(1072.66220735786,-9.06020066889632)(1070.1906354515,-9.09030100334448)(1067.71906354515,-9.12040133779264)(1062.77591973244,-9.12040133779264)(1060.30434782609,-9.1505016722408)(1057.83277591973,-9.18060200668896)(1052.88963210702,-9.18060200668896)(1047.94648829431,-9.18060200668896)(1043.00334448161,-9.18060200668896)(1038.0602006689,-9.18060200668896)(1033.11705685619,-9.18060200668896)(1028.17391304348,-9.18060200668896)(1023.23076923077,-9.18060200668896)(1018.28762541806,-9.18060200668896)(1013.34448160535,-9.18060200668896)(1008.40133779264,-9.18060200668896)(1003.45819397993,-9.18060200668896)(1000.98662207358,-9.1505016722408)(998.515050167224,-9.12040133779264)(993.571906354515,-9.12040133779264)(988.628762541806,-9.12040133779264)(986.157190635452,-9.09030100334448)(983.685618729097,-9.06020066889632)(978.742474916388,-9.06020066889632)(973.799331103679,-9.06020066889632)(971.327759197324,-9.03010033444816)(968.85618729097,-9)(963.913043478261,-9)(961.441471571906,-8.96989966555184)(958.969899665552,-8.93979933110368)(954.026755852843,-8.93979933110368)(949.083612040134,-8.93979933110368)(946.612040133779,-8.90969899665552)(944.140468227425,-8.87959866220736)(939.197324414716,-8.87959866220736)(934.254180602007,-8.87959866220736)(929.311036789298,-8.87959866220736)(924.367892976589,-8.87959866220736)(919.42474916388,-8.87959866220736)(914.481605351171,-8.87959866220736)(912.010033444816,-8.90969899665552)(909.538461538462,-8.93979933110368)(904.595317725752,-8.93979933110368)(899.652173913044,-8.93979933110368)(897.180602006689,-8.96989966555184)(894.709030100334,-9)(889.765886287625,-9)(884.822742474916,-9)(882.351170568562,-9.03010033444816)(879.879598662207,-9.06020066889632)(874.936454849498,-9.06020066889632)(869.993311036789,-9.06020066889632)(865.05016722408,-9.06020066889632)(860.107023411371,-9.06020066889632)(855.163879598662,-9.06020066889632)(850.220735785953,-9.06020066889632)(845.277591973244,-9.06020066889632)(840.334448160535,-9.06020066889632)(835.391304347826,-9.06020066889632)(832.919732441472,-9.03010033444816)(830.448160535117,-9)(827.976588628763,-8.96989966555184)(825.505016722408,-8.93979933110368)(820.561872909699,-8.93979933110368)(818.090301003345,-8.90969899665552)(815.61872909699,-8.87959866220736)(813.147157190635,-8.8494983277592)(810.675585284281,-8.81939799331104)(808.204013377926,-8.78929765886288)(808.204013377926,-8.72909698996656)(805.732441471572,-8.69899665551839)(803.260869565217,-8.66889632107023)(803.260869565217,-8.60869565217391)(803.260869565217,-8.54849498327759)(800.789297658863,-8.51839464882943)(798.317725752508,-8.48829431438127)(798.317725752508,-8.42809364548495)(800.789297658863,-8.39799331103679)(803.260869565217,-8.36789297658863)(803.260869565217,-8.30769230769231)(803.260869565217,-8.24749163879599)(805.732441471572,-8.21739130434783)(808.204013377926,-8.18729096989967)(808.204013377926,-8.12709030100334)(810.675585284281,-8.09698996655519)(813.147157190635,-8.06688963210702)(815.61872909699,-8.03678929765886)(818.090301003345,-8.0066889632107)(820.561872909699,-7.97658862876254)(825.505016722408,-7.97658862876254)(827.976588628763,-7.94648829431438)(830.448160535117,-7.91638795986622)(835.391304347826,-7.91638795986622)(840.334448160535,-7.91638795986622)(845.277591973244,-7.91638795986622)(850.220735785953,-7.91638795986622)(855.163879598662,-7.91638795986622)(860.107023411371,-7.91638795986622)(862.578595317726,-7.94648829431438)(865.05016722408,-7.97658862876254)(869.993311036789,-7.97658862876254)(872.464882943144,-8.0066889632107)(874.936454849498,-8.03678929765886)(879.879598662207,-8.03678929765886)(882.351170568562,-8.06688963210702)(884.822742474916,-8.09698996655519)(889.765886287625,-8.09698996655519)(892.23745819398,-8.12709030100334)(894.709030100334,-8.1571906354515)(899.652173913044,-8.1571906354515)(902.123745819398,-8.18729096989967)(904.595317725752,-8.21739130434783)(907.066889632107,-8.24749163879599)(909.538461538462,-8.27759197324415)(914.481605351171,-8.27759197324415)(916.953177257525,-8.30769230769231)(919.42474916388,-8.33779264214047)(924.367892976589,-8.33779264214047)(929.311036789298,-8.33779264214047)(934.254180602007,-8.33779264214047)(939.197324414716,-8.33779264214047)(944.140468227425,-8.33779264214047)(949.083612040134,-8.33779264214047)(954.026755852843,-8.33779264214047)(956.498327759197,-8.30769230769231)(958.969899665552,-8.27759197324415)(963.913043478261,-8.27759197324415)(966.384615384615,-8.24749163879599)(968.85618729097,-8.21739130434783)(973.799331103679,-8.21739130434783)(976.270903010033,-8.18729096989967)(978.742474916388,-8.1571906354515)(983.685618729097,-8.1571906354515)(986.157190635452,-8.12709030100334)(988.628762541806,-8.09698996655519)(993.571906354515,-8.09698996655519)(996.04347826087,-8.06688963210702)(998.515050167224,-8.03678929765886)(1003.45819397993,-8.03678929765886)(1005.92976588629,-8.0066889632107)(1008.40133779264,-7.97658862876254)(1013.34448160535,-7.97658862876254)(1015.81605351171,-7.94648829431438)(1018.28762541806,-7.91638795986622)(1023.23076923077,-7.91638795986622)(1025.70234113712,-7.88628762541806)(1028.17391304348,-7.8561872909699)(1033.11705685619,-7.8561872909699)(1038.0602006689,-7.8561872909699)(1043.00334448161,-7.8561872909699)(1047.94648829431,-7.8561872909699)(1052.88963210702,-7.8561872909699)(1055.36120401338,-7.88628762541806)(1057.83277591973,-7.91638795986622)(1062.77591973244,-7.91638795986622)(1065.2474916388,-7.94648829431438)(1067.71906354515,-7.97658862876254)(1070.1906354515,-8.0066889632107)(1072.66220735786,-8.03678929765886)(1075.13377926421,-8.06688963210702)(1077.60535117057,-8.09698996655519)(1080.07692307692,-8.12709030100334)(1080.07692307692,-8.18729096989967)(1082.54849498328,-8.21739130434783)(1085.02006688963,-8.24749163879599)(1085.02006688963,-8.30769230769231)(1087.49163879599,-8.33779264214047)(1089.96321070234,-8.36789297658863)(1089.96321070234,-8.42809364548495)(1089.96321070234,-8.48829431438127)(1092.4347826087,-8.51839464882943)};

\addplot [
color=white,
draw=white,
only marks,
mark=x,
mark options={solid},
mark size=2.0pt,
line width=0.3pt,
forget plot
]
coordinates{
 (10.5571428571429,0)(21.1142857142857,0)(31.6714285714286,0)(42.2285714285714,0)(52.7857142857143,0)(63.3428571428571,0)(73.9,0)(84.4571428571429,0)(95.0142857142857,0)(105.571428571429,0)(116.128571428571,-0.663157894736842)(126.685714285714,-1.32631578947368)(137.242857142857,-1.98947368421053)(147.8,-2.65263157894737)(158.357142857143,-3.31578947368421)(168.914285714286,-3.97894736842105)(179.471428571429,-4.64210526315789)(190.028571428571,-5.30526315789474)(200.585714285714,-5.96842105263158)(211.142857142857,-6.63157894736842)(221.7,-7.76842105263158)(232.257142857143,-8.90526315789474)(242.814285714286,-10.0421052631579)(253.371428571429,-11.1789473684211)(263.928571428571,-12.3157894736842)(274.485714285714,-13.4526315789474)(285.042857142857,-14.5894736842105)(295.6,-15.7263157894737)(306.157142857143,-16.8631578947368)(316.714285714286,-18)(327.271428571429,-17.1473684210526)(337.828571428571,-16.2947368421053)(348.385714285714,-15.4421052631579)(358.942857142857,-14.5894736842105)(369.5,-13.7368421052632)(380.057142857143,-12.8842105263158)(390.614285714286,-12.0315789473684)(401.171428571429,-11.1789473684211)(411.728571428571,-10.3263157894737)(422.285714285714,-9.47368421052632)(432.842857142857,-8.52631578947368)(443.4,-7.57894736842105)(453.957142857143,-6.63157894736842)(464.514285714286,-5.68421052631579)(475.071428571429,-4.73684210526316)(485.628571428571,-3.78947368421053)(496.185714285714,-2.84210526315789)(506.742857142857,-1.89473684210526)(517.3,-0.947368421052632)(527.857142857143,0)(538.414285714286,-0.947368421052632)(548.971428571429,-1.89473684210526)(559.528571428571,-2.84210526315789)(570.085714285714,-3.78947368421053)(580.642857142857,-4.73684210526316)(591.2,-5.68421052631579)(601.757142857143,-6.63157894736842)(612.314285714286,-7.57894736842105)(622.871428571429,-8.52631578947368)(633.428571428571,-9.47368421052632)(643.985714285714,-10.3263157894737)(654.542857142857,-11.1789473684211)(665.1,-12.0315789473684)(675.657142857143,-12.8842105263158)(686.214285714286,-13.7368421052632)(696.771428571429,-14.5894736842105)(707.328571428571,-15.4421052631579)(717.885714285714,-16.2947368421053)(728.442857142857,-17.1473684210526)(739,-18)(749.557142857143,-16.9578947368421)(760.114285714286,-15.9157894736842)(770.671428571429,-14.8736842105263)(781.228571428571,-13.8315789473684)(791.785714285714,-12.7894736842105)(802.342857142857,-11.7473684210526)(812.9,-10.7052631578947)(823.457142857143,-9.66315789473684)(834.014285714286,-8.62105263157895)(844.571428571429,-7.57894736842105)(855.128571428571,-6.82105263157895)(865.685714285714,-6.06315789473684)(876.242857142857,-5.30526315789474)(886.8,-4.54736842105263)(897.357142857143,-3.78947368421053)(907.914285714286,-3.03157894736842)(918.471428571429,-2.27368421052632)(929.028571428571,-1.51578947368421)(939.585714285714,-0.757894736842106)(950.142857142857,0)(960.7,-0.852631578947369)(971.257142857143,-1.70526315789474)(981.814285714286,-2.55789473684211)(992.371428571429,-3.41052631578947)(1002.92857142857,-4.26315789473684)(1013.48571428571,-5.11578947368421)(1024.04285714286,-5.96842105263158)(1034.6,-6.82105263157895)(1045.15714285714,-7.67368421052632)(1055.71428571429,-8.52631578947369) 
};

\addplot [
color=red,
solid,
line width=1.0pt,
forget plot
]
coordinates{
 (1055.71428571429,-8.52631578947369)(1161.28571428571,-18)(1266.85714285714,-10.4210526315789)(1372.42857142857,0)(1478,-10.4210526315789)(1372.42857142857,-18)(1266.85714285714,-3.78947368421053)(1161.28571428571,-9.47368421052632)(1055.71428571429,-10.4210526315789)(950.142857142857,-10.4210526315789)(844.571428571429,-9.47368421052632) 
};

\addplot [
mark size=0.8pt,
only marks,
mark=*,
mark options={solid,fill=black,draw=black},
forget plot
]
coordinates{
 (0,0)(0,-0.947368421052632)(0,-1.89473684210526)(0,-2.84210526315789)(0,-3.78947368421053)(0,-4.73684210526316)(0,-5.68421052631579)(0,-6.63157894736842)(0,-7.57894736842105)(0,-8.52631578947369)(0,-9.47368421052632)(0,-10.4210526315789)(0,-11.3684210526316)(0,-12.3157894736842)(0,-13.2631578947368)(0,-14.2105263157895)(0,-15.1578947368421)(0,-16.1052631578947)(0,-17.0526315789474)(0,-18)(105.571428571429,0)(105.571428571429,-0.947368421052632)(105.571428571429,-1.89473684210526)(105.571428571429,-2.84210526315789)(105.571428571429,-3.78947368421053)(105.571428571429,-4.73684210526316)(105.571428571429,-5.68421052631579)(105.571428571429,-6.63157894736842)(105.571428571429,-7.57894736842105)(105.571428571429,-8.52631578947369)(105.571428571429,-9.47368421052632)(105.571428571429,-10.4210526315789)(105.571428571429,-11.3684210526316)(105.571428571429,-12.3157894736842)(105.571428571429,-13.2631578947368)(105.571428571429,-14.2105263157895)(105.571428571429,-15.1578947368421)(105.571428571429,-16.1052631578947)(105.571428571429,-17.0526315789474)(105.571428571429,-18)(211.142857142857,0)(211.142857142857,-0.947368421052632)(211.142857142857,-1.89473684210526)(211.142857142857,-2.84210526315789)(211.142857142857,-3.78947368421053)(211.142857142857,-4.73684210526316)(211.142857142857,-5.68421052631579)(211.142857142857,-6.63157894736842)(211.142857142857,-7.57894736842105)(211.142857142857,-8.52631578947369)(211.142857142857,-9.47368421052632)(211.142857142857,-10.4210526315789)(211.142857142857,-11.3684210526316)(211.142857142857,-12.3157894736842)(211.142857142857,-13.2631578947368)(211.142857142857,-14.2105263157895)(211.142857142857,-15.1578947368421)(211.142857142857,-16.1052631578947)(211.142857142857,-17.0526315789474)(211.142857142857,-18)(316.714285714286,0)(316.714285714286,-0.947368421052632)(316.714285714286,-1.89473684210526)(316.714285714286,-2.84210526315789)(316.714285714286,-3.78947368421053)(316.714285714286,-4.73684210526316)(316.714285714286,-5.68421052631579)(316.714285714286,-6.63157894736842)(316.714285714286,-7.57894736842105)(316.714285714286,-8.52631578947369)(316.714285714286,-9.47368421052632)(316.714285714286,-10.4210526315789)(316.714285714286,-11.3684210526316)(316.714285714286,-12.3157894736842)(316.714285714286,-13.2631578947368)(316.714285714286,-14.2105263157895)(316.714285714286,-15.1578947368421)(316.714285714286,-16.1052631578947)(316.714285714286,-17.0526315789474)(316.714285714286,-18)(422.285714285714,0)(422.285714285714,-0.947368421052632)(422.285714285714,-1.89473684210526)(422.285714285714,-2.84210526315789)(422.285714285714,-3.78947368421053)(422.285714285714,-4.73684210526316)(422.285714285714,-5.68421052631579)(422.285714285714,-6.63157894736842)(422.285714285714,-7.57894736842105)(422.285714285714,-8.52631578947369)(422.285714285714,-9.47368421052632)(422.285714285714,-10.4210526315789)(422.285714285714,-11.3684210526316)(422.285714285714,-12.3157894736842)(422.285714285714,-13.2631578947368)(422.285714285714,-14.2105263157895)(422.285714285714,-15.1578947368421)(422.285714285714,-16.1052631578947)(422.285714285714,-17.0526315789474)(422.285714285714,-18)(527.857142857143,0)(527.857142857143,-0.947368421052632)(527.857142857143,-1.89473684210526)(527.857142857143,-2.84210526315789)(527.857142857143,-3.78947368421053)(527.857142857143,-4.73684210526316)(527.857142857143,-5.68421052631579)(527.857142857143,-6.63157894736842)(527.857142857143,-7.57894736842105)(527.857142857143,-8.52631578947369)(527.857142857143,-9.47368421052632)(527.857142857143,-10.4210526315789)(527.857142857143,-11.3684210526316)(527.857142857143,-12.3157894736842)(527.857142857143,-13.2631578947368)(527.857142857143,-14.2105263157895)(527.857142857143,-15.1578947368421)(527.857142857143,-16.1052631578947)(527.857142857143,-17.0526315789474)(527.857142857143,-18)(633.428571428571,0)(633.428571428571,-0.947368421052632)(633.428571428571,-1.89473684210526)(633.428571428571,-2.84210526315789)(633.428571428571,-3.78947368421053)(633.428571428571,-4.73684210526316)(633.428571428571,-5.68421052631579)(633.428571428571,-6.63157894736842)(633.428571428571,-7.57894736842105)(633.428571428571,-8.52631578947369)(633.428571428571,-9.47368421052632)(633.428571428571,-10.4210526315789)(633.428571428571,-11.3684210526316)(633.428571428571,-12.3157894736842)(633.428571428571,-13.2631578947368)(633.428571428571,-14.2105263157895)(633.428571428571,-15.1578947368421)(633.428571428571,-16.1052631578947)(633.428571428571,-17.0526315789474)(633.428571428571,-18)(739,0)(739,-0.947368421052632)(739,-1.89473684210526)(739,-2.84210526315789)(739,-3.78947368421053)(739,-4.73684210526316)(739,-5.68421052631579)(739,-6.63157894736842)(739,-7.57894736842105)(739,-8.52631578947369)(739,-9.47368421052632)(739,-10.4210526315789)(739,-11.3684210526316)(739,-12.3157894736842)(739,-13.2631578947368)(739,-14.2105263157895)(739,-15.1578947368421)(739,-16.1052631578947)(739,-17.0526315789474)(739,-18)(844.571428571429,0)(844.571428571429,-0.947368421052632)(844.571428571429,-1.89473684210526)(844.571428571429,-2.84210526315789)(844.571428571429,-3.78947368421053)(844.571428571429,-4.73684210526316)(844.571428571429,-5.68421052631579)(844.571428571429,-6.63157894736842)(844.571428571429,-7.57894736842105)(844.571428571429,-8.52631578947369)(844.571428571429,-9.47368421052632)(844.571428571429,-10.4210526315789)(844.571428571429,-11.3684210526316)(844.571428571429,-12.3157894736842)(844.571428571429,-13.2631578947368)(844.571428571429,-14.2105263157895)(844.571428571429,-15.1578947368421)(844.571428571429,-16.1052631578947)(844.571428571429,-17.0526315789474)(844.571428571429,-18)(950.142857142857,0)(950.142857142857,-0.947368421052632)(950.142857142857,-1.89473684210526)(950.142857142857,-2.84210526315789)(950.142857142857,-3.78947368421053)(950.142857142857,-4.73684210526316)(950.142857142857,-5.68421052631579)(950.142857142857,-6.63157894736842)(950.142857142857,-7.57894736842105)(950.142857142857,-8.52631578947369)(950.142857142857,-9.47368421052632)(950.142857142857,-10.4210526315789)(950.142857142857,-11.3684210526316)(950.142857142857,-12.3157894736842)(950.142857142857,-13.2631578947368)(950.142857142857,-14.2105263157895)(950.142857142857,-15.1578947368421)(950.142857142857,-16.1052631578947)(950.142857142857,-17.0526315789474)(950.142857142857,-18)(1055.71428571429,0)(1055.71428571429,-0.947368421052632)(1055.71428571429,-1.89473684210526)(1055.71428571429,-2.84210526315789)(1055.71428571429,-3.78947368421053)(1055.71428571429,-4.73684210526316)(1055.71428571429,-5.68421052631579)(1055.71428571429,-6.63157894736842)(1055.71428571429,-7.57894736842105)(1055.71428571429,-8.52631578947369)(1055.71428571429,-9.47368421052632)(1055.71428571429,-10.4210526315789)(1055.71428571429,-11.3684210526316)(1055.71428571429,-12.3157894736842)(1055.71428571429,-13.2631578947368)(1055.71428571429,-14.2105263157895)(1055.71428571429,-15.1578947368421)(1055.71428571429,-16.1052631578947)(1055.71428571429,-17.0526315789474)(1055.71428571429,-18)(1161.28571428571,0)(1161.28571428571,-0.947368421052632)(1161.28571428571,-1.89473684210526)(1161.28571428571,-2.84210526315789)(1161.28571428571,-3.78947368421053)(1161.28571428571,-4.73684210526316)(1161.28571428571,-5.68421052631579)(1161.28571428571,-6.63157894736842)(1161.28571428571,-7.57894736842105)(1161.28571428571,-8.52631578947369)(1161.28571428571,-9.47368421052632)(1161.28571428571,-10.4210526315789)(1161.28571428571,-11.3684210526316)(1161.28571428571,-12.3157894736842)(1161.28571428571,-13.2631578947368)(1161.28571428571,-14.2105263157895)(1161.28571428571,-15.1578947368421)(1161.28571428571,-16.1052631578947)(1161.28571428571,-17.0526315789474)(1161.28571428571,-18)(1266.85714285714,0)(1266.85714285714,-0.947368421052632)(1266.85714285714,-1.89473684210526)(1266.85714285714,-2.84210526315789)(1266.85714285714,-3.78947368421053)(1266.85714285714,-4.73684210526316)(1266.85714285714,-5.68421052631579)(1266.85714285714,-6.63157894736842)(1266.85714285714,-7.57894736842105)(1266.85714285714,-8.52631578947369)(1266.85714285714,-9.47368421052632)(1266.85714285714,-10.4210526315789)(1266.85714285714,-11.3684210526316)(1266.85714285714,-12.3157894736842)(1266.85714285714,-13.2631578947368)(1266.85714285714,-14.2105263157895)(1266.85714285714,-15.1578947368421)(1266.85714285714,-16.1052631578947)(1266.85714285714,-17.0526315789474)(1266.85714285714,-18)(1372.42857142857,0)(1372.42857142857,-0.947368421052632)(1372.42857142857,-1.89473684210526)(1372.42857142857,-2.84210526315789)(1372.42857142857,-3.78947368421053)(1372.42857142857,-4.73684210526316)(1372.42857142857,-5.68421052631579)(1372.42857142857,-6.63157894736842)(1372.42857142857,-7.57894736842105)(1372.42857142857,-8.52631578947369)(1372.42857142857,-9.47368421052632)(1372.42857142857,-10.4210526315789)(1372.42857142857,-11.3684210526316)(1372.42857142857,-12.3157894736842)(1372.42857142857,-13.2631578947368)(1372.42857142857,-14.2105263157895)(1372.42857142857,-15.1578947368421)(1372.42857142857,-16.1052631578947)(1372.42857142857,-17.0526315789474)(1372.42857142857,-18)(1478,0)(1478,-0.947368421052632)(1478,-1.89473684210526)(1478,-2.84210526315789)(1478,-3.78947368421053)(1478,-4.73684210526316)(1478,-5.68421052631579)(1478,-6.63157894736842)(1478,-7.57894736842105)(1478,-8.52631578947369)(1478,-9.47368421052632)(1478,-10.4210526315789)(1478,-11.3684210526316)(1478,-12.3157894736842)(1478,-13.2631578947368)(1478,-14.2105263157895)(1478,-15.1578947368421)(1478,-16.1052631578947)(1478,-17.0526315789474)(1478,-18)(1372.42857142857,0)(1372.42857142857,-0.947368421052632)(1372.42857142857,-1.89473684210526)(1372.42857142857,-2.84210526315789)(1372.42857142857,-3.78947368421053)(1372.42857142857,-4.73684210526316)(1372.42857142857,-5.68421052631579)(1372.42857142857,-6.63157894736842)(1372.42857142857,-7.57894736842105)(1372.42857142857,-8.52631578947369)(1372.42857142857,-9.47368421052632)(1372.42857142857,-10.4210526315789)(1372.42857142857,-11.3684210526316)(1372.42857142857,-12.3157894736842)(1372.42857142857,-13.2631578947368)(1372.42857142857,-14.2105263157895)(1372.42857142857,-15.1578947368421)(1372.42857142857,-16.1052631578947)(1372.42857142857,-17.0526315789474)(1372.42857142857,-18)(1266.85714285714,0)(1266.85714285714,-0.947368421052632)(1266.85714285714,-1.89473684210526)(1266.85714285714,-2.84210526315789)(1266.85714285714,-3.78947368421053)(1266.85714285714,-4.73684210526316)(1266.85714285714,-5.68421052631579)(1266.85714285714,-6.63157894736842)(1266.85714285714,-7.57894736842105)(1266.85714285714,-8.52631578947369)(1266.85714285714,-9.47368421052632)(1266.85714285714,-10.4210526315789)(1266.85714285714,-11.3684210526316)(1266.85714285714,-12.3157894736842)(1266.85714285714,-13.2631578947368)(1266.85714285714,-14.2105263157895)(1266.85714285714,-15.1578947368421)(1266.85714285714,-16.1052631578947)(1266.85714285714,-17.0526315789474)(1266.85714285714,-18)(1161.28571428571,0)(1161.28571428571,-0.947368421052632)(1161.28571428571,-1.89473684210526)(1161.28571428571,-2.84210526315789)(1161.28571428571,-3.78947368421053)(1161.28571428571,-4.73684210526316)(1161.28571428571,-5.68421052631579)(1161.28571428571,-6.63157894736842)(1161.28571428571,-7.57894736842105)(1161.28571428571,-8.52631578947369)(1161.28571428571,-9.47368421052632)(1161.28571428571,-10.4210526315789)(1161.28571428571,-11.3684210526316)(1161.28571428571,-12.3157894736842)(1161.28571428571,-13.2631578947368)(1161.28571428571,-14.2105263157895)(1161.28571428571,-15.1578947368421)(1161.28571428571,-16.1052631578947)(1161.28571428571,-17.0526315789474)(1161.28571428571,-18)(1055.71428571429,0)(1055.71428571429,-0.947368421052632)(1055.71428571429,-1.89473684210526)(1055.71428571429,-2.84210526315789)(1055.71428571429,-3.78947368421053)(1055.71428571429,-4.73684210526316)(1055.71428571429,-5.68421052631579)(1055.71428571429,-6.63157894736842)(1055.71428571429,-7.57894736842105)(1055.71428571429,-8.52631578947369)(1055.71428571429,-9.47368421052632)(1055.71428571429,-10.4210526315789)(1055.71428571429,-11.3684210526316)(1055.71428571429,-12.3157894736842)(1055.71428571429,-13.2631578947368)(1055.71428571429,-14.2105263157895)(1055.71428571429,-15.1578947368421)(1055.71428571429,-16.1052631578947)(1055.71428571429,-17.0526315789474)(1055.71428571429,-18)(950.142857142857,0)(950.142857142857,-0.947368421052632)(950.142857142857,-1.89473684210526)(950.142857142857,-2.84210526315789)(950.142857142857,-3.78947368421053)(950.142857142857,-4.73684210526316)(950.142857142857,-5.68421052631579)(950.142857142857,-6.63157894736842)(950.142857142857,-7.57894736842105)(950.142857142857,-8.52631578947369)(950.142857142857,-9.47368421052632)(950.142857142857,-10.4210526315789)(950.142857142857,-11.3684210526316)(950.142857142857,-12.3157894736842)(950.142857142857,-13.2631578947368)(950.142857142857,-14.2105263157895)(950.142857142857,-15.1578947368421)(950.142857142857,-16.1052631578947)(950.142857142857,-17.0526315789474)(950.142857142857,-18)(844.571428571429,0)(844.571428571429,-0.947368421052632)(844.571428571429,-1.89473684210526)(844.571428571429,-2.84210526315789)(844.571428571429,-3.78947368421053)(844.571428571429,-4.73684210526316)(844.571428571429,-5.68421052631579)(844.571428571429,-6.63157894736842)(844.571428571429,-7.57894736842105)(844.571428571429,-8.52631578947369)(844.571428571429,-9.47368421052632)(844.571428571429,-10.4210526315789)(844.571428571429,-11.3684210526316)(844.571428571429,-12.3157894736842)(844.571428571429,-13.2631578947368)(844.571428571429,-14.2105263157895)(844.571428571429,-15.1578947368421)(844.571428571429,-16.1052631578947)(844.571428571429,-17.0526315789474)(844.571428571429,-18)(739,0)(739,-0.947368421052632)(739,-1.89473684210526)(739,-2.84210526315789)(739,-3.78947368421053)(739,-4.73684210526316)(739,-5.68421052631579)(739,-6.63157894736842)(739,-7.57894736842105)(739,-8.52631578947369)(739,-9.47368421052632)(739,-10.4210526315789)(739,-11.3684210526316)(739,-12.3157894736842)(739,-13.2631578947368)(739,-14.2105263157895)(739,-15.1578947368421)(739,-16.1052631578947)(739,-17.0526315789474)(739,-18)(633.428571428571,0)(633.428571428571,-0.947368421052632)(633.428571428571,-1.89473684210526)(633.428571428571,-2.84210526315789)(633.428571428571,-3.78947368421053)(633.428571428571,-4.73684210526316)(633.428571428571,-5.68421052631579)(633.428571428571,-6.63157894736842)(633.428571428571,-7.57894736842105)(633.428571428571,-8.52631578947369)(633.428571428571,-9.47368421052632)(633.428571428571,-10.4210526315789)(633.428571428571,-11.3684210526316)(633.428571428571,-12.3157894736842)(633.428571428571,-13.2631578947368)(633.428571428571,-14.2105263157895)(633.428571428571,-15.1578947368421)(633.428571428571,-16.1052631578947)(633.428571428571,-17.0526315789474)(633.428571428571,-18)(527.857142857143,0)(527.857142857143,-0.947368421052632)(527.857142857143,-1.89473684210526)(527.857142857143,-2.84210526315789)(527.857142857143,-3.78947368421053)(527.857142857143,-4.73684210526316)(527.857142857143,-5.68421052631579)(527.857142857143,-6.63157894736842)(527.857142857143,-7.57894736842105)(527.857142857143,-8.52631578947369)(527.857142857143,-9.47368421052632)(527.857142857143,-10.4210526315789)(527.857142857143,-11.3684210526316)(527.857142857143,-12.3157894736842)(527.857142857143,-13.2631578947368)(527.857142857143,-14.2105263157895)(527.857142857143,-15.1578947368421)(527.857142857143,-16.1052631578947)(527.857142857143,-17.0526315789474)(527.857142857143,-18)(422.285714285714,0)(422.285714285714,-0.947368421052632)(422.285714285714,-1.89473684210526)(422.285714285714,-2.84210526315789)(422.285714285714,-3.78947368421053)(422.285714285714,-4.73684210526316)(422.285714285714,-5.68421052631579)(422.285714285714,-6.63157894736842)(422.285714285714,-7.57894736842105)(422.285714285714,-8.52631578947369)(422.285714285714,-9.47368421052632)(422.285714285714,-10.4210526315789)(422.285714285714,-11.3684210526316)(422.285714285714,-12.3157894736842)(422.285714285714,-13.2631578947368)(422.285714285714,-14.2105263157895)(422.285714285714,-15.1578947368421)(422.285714285714,-16.1052631578947)(422.285714285714,-17.0526315789474)(422.285714285714,-18)(316.714285714286,0)(316.714285714286,-0.947368421052632)(316.714285714286,-1.89473684210526)(316.714285714286,-2.84210526315789)(316.714285714286,-3.78947368421053)(316.714285714286,-4.73684210526316)(316.714285714286,-5.68421052631579)(316.714285714286,-6.63157894736842)(316.714285714286,-7.57894736842105)(316.714285714286,-8.52631578947369)(316.714285714286,-9.47368421052632)(316.714285714286,-10.4210526315789)(316.714285714286,-11.3684210526316)(316.714285714286,-12.3157894736842)(316.714285714286,-13.2631578947368)(316.714285714286,-14.2105263157895)(316.714285714286,-15.1578947368421)(316.714285714286,-16.1052631578947)(316.714285714286,-17.0526315789474)(316.714285714286,-18)(211.142857142857,0)(211.142857142857,-0.947368421052632)(211.142857142857,-1.89473684210526)(211.142857142857,-2.84210526315789)(211.142857142857,-3.78947368421053)(211.142857142857,-4.73684210526316)(211.142857142857,-5.68421052631579)(211.142857142857,-6.63157894736842)(211.142857142857,-7.57894736842105)(211.142857142857,-8.52631578947369)(211.142857142857,-9.47368421052632)(211.142857142857,-10.4210526315789)(211.142857142857,-11.3684210526316)(211.142857142857,-12.3157894736842)(211.142857142857,-13.2631578947368)(211.142857142857,-14.2105263157895)(211.142857142857,-15.1578947368421)(211.142857142857,-16.1052631578947)(211.142857142857,-17.0526315789474)(211.142857142857,-18)(105.571428571429,0)(105.571428571429,-0.947368421052632)(105.571428571429,-1.89473684210526)(105.571428571429,-2.84210526315789)(105.571428571429,-3.78947368421053)(105.571428571429,-4.73684210526316)(105.571428571429,-5.68421052631579)(105.571428571429,-6.63157894736842)(105.571428571429,-7.57894736842105)(105.571428571429,-8.52631578947369)(105.571428571429,-9.47368421052632)(105.571428571429,-10.4210526315789)(105.571428571429,-11.3684210526316)(105.571428571429,-12.3157894736842)(105.571428571429,-13.2631578947368)(105.571428571429,-14.2105263157895)(105.571428571429,-15.1578947368421)(105.571428571429,-16.1052631578947)(105.571428571429,-17.0526315789474)(105.571428571429,-18) 
};

\addplot [
color=blue,
mark size=3.5pt,
only marks,
mark=o,
mark options={solid,draw=lime!80!black},
line width=2.7pt,
forget plot
]
coordinates{
 (1055.71428571429,-8.52631578947369)
};

\end{axis}
\end{tikzpicture}%

%% This file was created by matlab2tikz v0.2.3.
% Copyright (c) 2008--2012, Nico Schlömer <nico.schloemer@gmail.com>
% All rights reserved.
% 
% 
% 
\definecolor{locol}{rgb}{0.26, 0.45, 0.65}

\begin{tikzpicture}

\begin{axis}[%
tick label style={font=\tiny},
label style={font=\tiny},
xlabel shift={-10pt},
ylabel shift={-17pt},
legend style={font=\tiny},
view={0}{90},
width=\figurewidth,
height=\figureheight,
scale only axis,
xmin=0, xmax=1478,
xtick={0, 400, 1000, 1400},
xlabel={Length (m)},
ymin=-18, ymax=0,
ytick={0, -4, -14, -18},
ylabel={Depth (m)},
name=plot1,
axis lines*=box,
axis line style={draw=none},
tickwidth=0.0cm,
clip=false
]

\addplot [fill=locol,draw=black,forget plot] coordinates{ (0,0)(4.94314381270903,0)(9.88628762541806,0)(14.8294314381271,0)(19.7725752508361,0)(24.7157190635452,0)(29.6588628762542,0)(34.6020066889632,0)(39.5451505016722,0)(44.4882943143813,0)(49.4314381270903,0)(54.3745819397993,0)(59.3177257525084,0)(64.2608695652174,0)(69.2040133779264,0)(74.1471571906355,0)(79.0903010033445,0)(84.0334448160535,0)(88.9765886287625,0)(93.9197324414716,0)(98.8628762541806,0)(103.80602006689,0)(108.749163879599,0)(113.692307692308,0)(118.635451505017,0)(123.578595317726,0)(128.521739130435,0)(133.464882943144,0)(138.408026755853,0)(143.351170568562,0)(148.294314381271,0)(153.23745819398,0)(158.180602006689,0)(163.123745819398,0)(168.066889632107,0)(173.010033444816,0)(177.953177257525,0)(182.896321070234,0)(187.839464882943,0)(192.782608695652,0)(197.725752508361,0)(202.66889632107,0)(207.612040133779,0)(212.555183946488,0)(217.498327759197,0)(222.441471571906,0)(227.384615384615,0)(232.327759197324,0)(237.270903010033,0)(242.214046822742,0)(247.157190635452,0)(252.100334448161,0)(257.04347826087,0)(261.986622073579,0)(266.929765886288,0)(271.872909698997,0)(276.816053511706,0)(281.759197324415,0)(286.702341137124,0)(291.645484949833,0)(296.588628762542,0)(301.531772575251,0)(306.47491638796,0)(311.418060200669,0)(316.361204013378,0)(321.304347826087,0)(326.247491638796,0)(331.190635451505,0)(336.133779264214,0)(341.076923076923,0)(346.020066889632,0)(350.963210702341,0)(355.90635451505,0)(360.849498327759,0)(365.792642140468,0)(370.735785953177,0)(375.678929765886,0)(380.622073578595,0)(385.565217391304,0)(390.508361204013,0)(395.451505016722,0)(400.394648829431,0)(405.33779264214,0)(410.28093645485,0)(415.224080267559,0)(420.167224080268,0)(425.110367892977,0)(430.053511705686,0)(434.996655518395,0)(439.939799331104,0)(444.882943143813,0)(449.826086956522,0)(454.769230769231,0)(459.71237458194,0)(464.655518394649,0)(469.598662207358,0)(474.541806020067,0)(479.484949832776,0)(484.428093645485,0)(489.371237458194,0)(494.314381270903,0)(499.257525083612,0)(504.200668896321,0)(509.14381270903,0)(514.086956521739,0)(519.030100334448,0)(523.973244147157,0)(528.916387959866,0)(533.859531772575,0)(538.802675585284,0)(543.745819397993,0)(548.688963210702,0)(553.632107023411,0)(558.57525083612,0)(563.518394648829,0)(568.461538461538,0)(573.404682274248,0)(578.347826086957,0)(583.290969899666,0)(588.234113712375,0)(593.177257525084,0)(598.120401337793,0)(603.063545150502,0)(608.006688963211,0)(612.94983277592,0)(617.892976588629,0)(622.836120401338,0)(627.779264214047,0)(632.722408026756,0)(637.665551839465,0)(642.608695652174,0)(647.551839464883,0)(652.494983277592,0)(657.438127090301,0)(662.38127090301,0)(667.324414715719,0)(672.267558528428,0)(677.210702341137,0)(682.153846153846,0)(687.096989966555,0)(692.040133779264,0)(696.983277591973,0)(701.926421404682,0)(706.869565217391,0)(711.8127090301,0)(716.755852842809,0)(721.698996655518,0)(726.642140468227,0)(731.585284280936,0)(736.528428093646,0)(741.471571906354,0)(746.414715719064,0)(751.357859531773,0)(756.301003344482,0)(761.244147157191,0)(766.1872909699,0)(771.130434782609,0)(776.073578595318,0)(781.016722408027,0)(785.959866220736,0)(790.903010033445,0)(795.846153846154,0)(800.789297658863,0)(805.732441471572,0)(810.675585284281,0)(815.61872909699,0)(820.561872909699,0)(825.505016722408,0)(830.448160535117,0)(835.391304347826,0)(840.334448160535,0)(845.277591973244,0)(850.220735785953,0)(855.163879598662,0)(860.107023411371,0)(865.05016722408,0)(869.993311036789,0)(874.936454849498,0)(879.879598662207,0)(884.822742474916,0)(889.765886287625,0)(894.709030100334,0)(899.652173913044,0)(904.595317725752,0)(909.538461538462,0)(914.481605351171,0)(919.42474916388,0)(924.367892976589,0)(929.311036789298,0)(934.254180602007,0)(939.197324414716,0)(944.140468227425,0)(949.083612040134,0)(954.026755852843,0)(958.969899665552,0)(963.913043478261,0)(968.85618729097,0)(973.799331103679,0)(978.742474916388,0)(983.685618729097,0)(988.628762541806,0)(993.571906354515,0)(998.515050167224,0)(1003.45819397993,0)(1008.40133779264,0)(1013.34448160535,0)(1018.28762541806,0)(1023.23076923077,0)(1028.17391304348,0)(1033.11705685619,0)(1038.0602006689,0)(1043.00334448161,0)(1047.94648829431,0)(1052.88963210702,0)(1057.83277591973,0)(1062.77591973244,0)(1067.71906354515,0)(1072.66220735786,0)(1077.60535117057,0)(1082.54849498328,0)(1087.49163879599,0)(1092.4347826087,0)(1097.3779264214,0)(1102.32107023411,0)(1107.26421404682,0)(1112.20735785953,0)(1117.15050167224,0)(1122.09364548495,0)(1127.03678929766,0)(1131.97993311037,0)(1136.92307692308,0)(1141.86622073579,0)(1146.8093645485,0)(1151.7525083612,0)(1156.69565217391,0)(1161.63879598662,0)(1166.58193979933,0)(1171.52508361204,0)(1176.46822742475,0)(1181.41137123746,0)(1186.35451505017,0)(1191.29765886288,0)(1196.24080267559,0)(1201.18394648829,0)(1206.127090301,0)(1211.07023411371,0)(1216.01337792642,0)(1220.95652173913,0)(1225.89966555184,0)(1230.84280936455,0)(1235.78595317726,0)(1240.72909698997,0)(1245.67224080268,0)(1250.61538461538,0)(1255.55852842809,0)(1260.5016722408,0)(1265.44481605351,0)(1270.38795986622,0)(1275.33110367893,0)(1280.27424749164,0)(1285.21739130435,0)(1290.16053511706,0)(1295.10367892977,0)(1300.04682274247,0)(1304.98996655518,0)(1309.93311036789,0)(1314.8762541806,0)(1319.81939799331,0)(1324.76254180602,0)(1329.70568561873,0)(1334.64882943144,0)(1339.59197324415,0)(1344.53511705686,0)(1349.47826086957,0)(1354.42140468227,0)(1359.36454849498,0)(1364.30769230769,0)(1369.2508361204,0)(1374.19397993311,0)(1379.13712374582,0)(1384.08026755853,0)(1389.02341137124,0)(1393.96655518395,0)(1398.90969899666,0)(1403.85284280936,0)(1408.79598662207,0)(1413.73913043478,0)(1418.68227424749,0)(1423.6254180602,0)(1428.56856187291,0)(1433.51170568562,0)(1438.45484949833,0)(1443.39799331104,0)(1448.34113712375,0)(1453.28428093645,0)(1458.22742474916,0)(1463.17056856187,0)(1468.11371237458,0)(1473.05685618729,0)(1478,0)(1478,-0.0602006688963215)(1478,-0.120401337792643)(1478,-0.180602006688964)(1478,-0.240802675585286)(1478,-0.301003344481604)(1478,-0.361204013377925)(1478,-0.421404682274247)(1478,-0.481605351170568)(1478,-0.54180602006689)(1478,-0.602006688963211)(1478,-0.662207357859533)(1478,-0.722408026755854)(1478,-0.782608695652176)(1478,-0.842809364548494)(1478,-0.903010033444815)(1478,-0.963210702341136)(1478,-1.02341137123746)(1478,-1.08361204013378)(1478,-1.1438127090301)(1478,-1.20401337792642)(1478,-1.26421404682274)(1478,-1.32441471571906)(1478,-1.38461538461538)(1478,-1.4448160535117)(1478,-1.50501672240803)(1478,-1.56521739130435)(1478,-1.62541806020067)(1478,-1.68561872909699)(1478,-1.74581939799331)(1478,-1.80602006688963)(1478,-1.86622073578595)(1478,-1.92642140468227)(1478,-1.98662207357859)(1478,-2.04682274247492)(1478,-2.10702341137124)(1478,-2.16722408026756)(1478,-2.22742474916388)(1478,-2.2876254180602)(1478,-2.34782608695652)(1478,-2.40802675585284)(1478,-2.46822742474916)(1478,-2.52842809364548)(1478,-2.58862876254181)(1478,-2.64882943143813)(1478,-2.70903010033445)(1478,-2.76923076923077)(1478,-2.82943143812709)(1478,-2.88963210702341)(1478,-2.94983277591973)(1478,-3.01003344481605)(1478,-3.07023411371237)(1478,-3.1304347826087)(1478,-3.19063545150502)(1478,-3.25083612040134)(1478,-3.31103678929766)(1478,-3.37123745819398)(1478,-3.4314381270903)(1478,-3.49163879598662)(1478,-3.55183946488294)(1478,-3.61204013377926)(1478,-3.67224080267559)(1478,-3.73244147157191)(1478,-3.79264214046823)(1478,-3.85284280936455)(1478,-3.91304347826087)(1478,-3.97324414715719)(1478,-4.03344481605351)(1478,-4.09364548494983)(1478,-4.15384615384615)(1478,-4.21404682274247)(1478,-4.2742474916388)(1478,-4.33444816053512)(1478,-4.39464882943144)(1478,-4.45484949832776)(1478,-4.51505016722408)(1478,-4.5752508361204)(1478,-4.63545150501672)(1478,-4.69565217391304)(1478,-4.75585284280936)(1478,-4.81605351170569)(1478,-4.87625418060201)(1478,-4.93645484949833)(1478,-4.99665551839465)(1478,-5.05685618729097)(1478,-5.11705685618729)(1478,-5.17725752508361)(1478,-5.23745819397993)(1478,-5.29765886287625)(1478,-5.35785953177258)(1478,-5.4180602006689)(1478,-5.47826086956522)(1478,-5.53846153846154)(1478,-5.59866220735786)(1478,-5.65886287625418)(1478,-5.7190635451505)(1478,-5.77926421404682)(1478,-5.83946488294314)(1478,-5.89966555183947)(1478,-5.95986622073579)(1478,-6.02006688963211)(1478,-6.08026755852843)(1478,-6.14046822742475)(1478,-6.20066889632107)(1478,-6.26086956521739)(1478,-6.32107023411371)(1478,-6.38127090301003)(1478,-6.44147157190636)(1478,-6.50167224080267)(1478,-6.561872909699)(1478,-6.62207357859532)(1478,-6.68227424749164)(1478,-6.74247491638796)(1478,-6.80267558528428)(1478,-6.8628762541806)(1478,-6.92307692307692)(1478,-6.98327759197324)(1478,-7.04347826086956)(1478,-7.10367892976589)(1478,-7.16387959866221)(1478,-7.22408026755853)(1478,-7.28428093645485)(1478,-7.34448160535117)(1478,-7.40468227424749)(1478,-7.46488294314381)(1478,-7.52508361204013)(1478,-7.58528428093645)(1478,-7.64548494983278)(1478,-7.7056856187291)(1478,-7.76588628762542)(1478,-7.82608695652174)(1478,-7.88628762541806)(1478,-7.94648829431438)(1478,-8.0066889632107)(1478,-8.06688963210702)(1478,-8.12709030100334)(1478,-8.18729096989967)(1478,-8.24749163879599)(1478,-8.30769230769231)(1478,-8.36789297658863)(1478.00024714483,-8.42809364548495)(1478.00024714483,-8.48829431438127)(1478.00024714483,-8.54849498327759)(1478.00024714483,-8.60869565217391)(1478.00024714483,-8.66889632107023)(1478.00024714483,-8.72909698996656)(1478.00024714483,-8.78929765886288)(1478.00024714483,-8.8494983277592)(1478.00024714483,-8.90969899665552)(1478.00024714483,-8.96989966555184)(1478.00024714483,-9.03010033444816)(1478.00024714483,-9.09030100334448)(1478.00024714483,-9.1505016722408)(1478.00024714483,-9.21070234113712)(1478.00024714483,-9.27090301003344)(1478.00024714483,-9.33110367892977)(1478.00024714483,-9.39130434782609)(1478.00024714483,-9.45150501672241)(1478.00024714483,-9.51170568561873)(1478.00024714483,-9.57190635451505)(1478.00024714483,-9.63210702341137)(1478.00024714483,-9.69230769230769)(1478.00024714483,-9.75250836120401)(1478.00024714483,-9.81270903010033)(1478.00024714483,-9.87290969899666)(1478,-9.93311036789298)(1478,-9.9933110367893)(1478,-10.0535117056856)(1478,-10.1137123745819)(1478,-10.1739130434783)(1478,-10.2341137123746)(1478,-10.2943143812709)(1478,-10.3545150501672)(1478,-10.4147157190635)(1478,-10.4749163879599)(1478,-10.5351170568562)(1478,-10.5953177257525)(1478,-10.6555183946488)(1478,-10.7157190635452)(1478,-10.7759197324415)(1478,-10.8361204013378)(1478,-10.8963210702341)(1478,-10.9565217391304)(1478,-11.0167224080268)(1478,-11.0769230769231)(1478,-11.1371237458194)(1478,-11.1973244147157)(1478,-11.257525083612)(1478,-11.3177257525084)(1478,-11.3779264214047)(1478,-11.438127090301)(1478,-11.4983277591973)(1478,-11.5585284280936)(1478,-11.61872909699)(1478,-11.6789297658863)(1478,-11.7391304347826)(1478,-11.7993311036789)(1478,-11.8595317725753)(1478,-11.9197324414716)(1478,-11.9799331103679)(1478,-12.0401337792642)(1478,-12.1003344481605)(1478,-12.1605351170569)(1478,-12.2207357859532)(1478,-12.2809364548495)(1478,-12.3411371237458)(1478,-12.4013377926421)(1478,-12.4615384615385)(1478,-12.5217391304348)(1478,-12.5819397993311)(1478,-12.6421404682274)(1478,-12.7023411371237)(1478,-12.7625418060201)(1478,-12.8227424749164)(1478,-12.8829431438127)(1478,-12.943143812709)(1478,-13.0033444816054)(1478,-13.0635451505017)(1478,-13.123745819398)(1478,-13.1839464882943)(1478,-13.2441471571906)(1478,-13.304347826087)(1478,-13.3645484949833)(1478,-13.4247491638796)(1478,-13.4849498327759)(1478,-13.5451505016722)(1478,-13.6053511705686)(1478,-13.6655518394649)(1478,-13.7257525083612)(1478,-13.7859531772575)(1478,-13.8461538461538)(1478,-13.9063545150502)(1478,-13.9665551839465)(1478,-14.0267558528428)(1478,-14.0869565217391)(1478,-14.1471571906355)(1478,-14.2073578595318)(1478,-14.2675585284281)(1478,-14.3277591973244)(1478,-14.3879598662207)(1478,-14.4481605351171)(1478,-14.5083612040134)(1478,-14.5685618729097)(1478,-14.628762541806)(1478,-14.6889632107023)(1478,-14.7491638795987)(1478,-14.809364548495)(1478,-14.8695652173913)(1478,-14.9297658862876)(1478,-14.9899665551839)(1478,-15.0501672240803)(1478,-15.1103678929766)(1478,-15.1705685618729)(1478,-15.2307692307692)(1478,-15.2909698996656)(1478,-15.3511705685619)(1478,-15.4113712374582)(1478,-15.4715719063545)(1478,-15.5317725752508)(1478,-15.5919732441472)(1478,-15.6521739130435)(1478,-15.7123745819398)(1478,-15.7725752508361)(1478,-15.8327759197324)(1478,-15.8929765886288)(1478,-15.9531772575251)(1478,-16.0133779264214)(1478,-16.0735785953177)(1478,-16.133779264214)(1478,-16.1939799331104)(1478,-16.2541806020067)(1478,-16.314381270903)(1478,-16.3745819397993)(1478,-16.4347826086957)(1478,-16.494983277592)(1478,-16.5551839464883)(1478,-16.6153846153846)(1478,-16.6755852842809)(1478,-16.7357859531773)(1478,-16.7959866220736)(1478,-16.8561872909699)(1478,-16.9163879598662)(1478,-16.9765886287625)(1478,-17.0367892976589)(1478,-17.0969899665552)(1478,-17.1571906354515)(1478,-17.2173913043478)(1478,-17.2775919732441)(1478,-17.3377926421405)(1478,-17.3979933110368)(1478,-17.4581939799331)(1478,-17.5183946488294)(1478,-17.5785953177258)(1478,-17.6387959866221)(1478,-17.6989966555184)(1478,-17.7591973244147)(1478.00024714483,-17.819397993311)(1478.00024714483,-17.8795986622074)(1478.00024714483,-17.9397993311037)(1478.00024714483,-18)(1478,-18.000003009883)(1473.05685618729,-18.000003009883)(1468.11371237458,-18.000003009883)(1463.17056856187,-18.000003009883)(1458.22742474916,-18.000003009883)(1453.28428093645,-18)(1448.34113712375,-18)(1443.39799331104,-18)(1438.45484949833,-18)(1433.51170568562,-18)(1428.56856187291,-18)(1423.6254180602,-18)(1418.68227424749,-18)(1413.73913043478,-18)(1408.79598662207,-18)(1403.85284280936,-18)(1398.90969899666,-18)(1393.96655518395,-18)(1389.02341137124,-18)(1384.08026755853,-18)(1379.13712374582,-18)(1374.19397993311,-18)(1369.2508361204,-18)(1364.30769230769,-18)(1359.36454849498,-18)(1354.42140468227,-18)(1349.47826086957,-18)(1344.53511705686,-18)(1339.59197324415,-18)(1334.64882943144,-18)(1329.70568561873,-18)(1324.76254180602,-18)(1319.81939799331,-18)(1314.8762541806,-18)(1309.93311036789,-18)(1304.98996655518,-18)(1300.04682274247,-18)(1295.10367892977,-18)(1290.16053511706,-18)(1285.21739130435,-18)(1280.27424749164,-18)(1275.33110367893,-18)(1270.38795986622,-18)(1265.44481605351,-18)(1260.5016722408,-18)(1255.55852842809,-18)(1250.61538461538,-18)(1245.67224080268,-18)(1240.72909698997,-18)(1235.78595317726,-18)(1230.84280936455,-18)(1225.89966555184,-18)(1220.95652173913,-18)(1216.01337792642,-18)(1211.07023411371,-18)(1206.127090301,-18)(1201.18394648829,-18)(1196.24080267559,-18)(1191.29765886288,-18)(1186.35451505017,-18)(1181.41137123746,-18)(1176.46822742475,-18)(1171.52508361204,-18)(1166.58193979933,-18)(1161.63879598662,-18)(1156.69565217391,-18)(1151.7525083612,-18)(1146.8093645485,-18)(1141.86622073579,-18)(1136.92307692308,-18)(1131.97993311037,-18)(1127.03678929766,-18)(1122.09364548495,-18)(1117.15050167224,-18)(1112.20735785953,-18)(1107.26421404682,-18)(1102.32107023411,-18)(1097.3779264214,-18)(1092.4347826087,-18)(1087.49163879599,-18)(1082.54849498328,-18)(1077.60535117057,-18)(1072.66220735786,-18)(1067.71906354515,-18)(1062.77591973244,-18)(1057.83277591973,-18)(1052.88963210702,-18)(1047.94648829431,-18)(1043.00334448161,-18)(1038.0602006689,-18)(1033.11705685619,-18)(1028.17391304348,-18)(1023.23076923077,-18)(1018.28762541806,-18)(1013.34448160535,-18)(1008.40133779264,-18)(1003.45819397993,-18)(998.515050167224,-18)(993.571906354515,-18)(988.628762541806,-18)(983.685618729097,-18)(978.742474916388,-18)(973.799331103679,-18)(968.85618729097,-18)(963.913043478261,-18)(958.969899665552,-18)(954.026755852843,-18)(949.083612040134,-18)(944.140468227425,-18)(939.197324414716,-18)(934.254180602007,-18)(929.311036789298,-18)(924.367892976589,-18)(919.42474916388,-18)(914.481605351171,-18)(909.538461538462,-18)(904.595317725752,-18)(899.652173913044,-18)(894.709030100334,-18)(889.765886287625,-18)(884.822742474916,-18)(879.879598662207,-18)(874.936454849498,-18)(869.993311036789,-18)(865.05016722408,-18)(860.107023411371,-18)(855.163879598662,-18)(850.220735785953,-18)(845.277591973244,-18)(840.334448160535,-18)(835.391304347826,-18)(830.448160535117,-18)(825.505016722408,-18)(820.561872909699,-18)(815.61872909699,-18)(810.675585284281,-18)(805.732441471572,-18)(800.789297658863,-18)(795.846153846154,-18)(790.903010033445,-18)(785.959866220736,-18)(781.016722408027,-18)(776.073578595318,-18)(771.130434782609,-18)(766.1872909699,-18)(761.244147157191,-18)(756.301003344482,-18)(751.357859531773,-18)(746.414715719064,-18)(741.471571906354,-18)(736.528428093646,-18)(731.585284280936,-18)(726.642140468227,-18)(721.698996655518,-18)(716.755852842809,-18)(711.8127090301,-18)(706.869565217391,-18)(701.926421404682,-18)(696.983277591973,-18)(692.040133779264,-18)(687.096989966555,-18)(682.153846153846,-18)(677.210702341137,-18)(672.267558528428,-18)(667.324414715719,-18)(662.38127090301,-18)(657.438127090301,-18)(652.494983277592,-18)(647.551839464883,-18)(642.608695652174,-18)(637.665551839465,-18)(632.722408026756,-18)(627.779264214047,-18)(622.836120401338,-18)(617.892976588629,-18)(612.94983277592,-18)(608.006688963211,-18)(603.063545150502,-18)(598.120401337793,-18)(593.177257525084,-18)(588.234113712375,-18)(583.290969899666,-18)(578.347826086957,-18)(573.404682274248,-18)(568.461538461538,-18)(563.518394648829,-18)(558.57525083612,-18)(553.632107023411,-18)(548.688963210702,-18)(543.745819397993,-18)(538.802675585284,-18)(533.859531772575,-18)(528.916387959866,-18)(523.973244147157,-18)(519.030100334448,-18)(514.086956521739,-18)(509.14381270903,-18)(504.200668896321,-18)(499.257525083612,-18)(494.314381270903,-18)(489.371237458194,-18)(484.428093645485,-18)(479.484949832776,-18)(474.541806020067,-18)(469.598662207358,-18)(464.655518394649,-18)(459.71237458194,-18)(454.769230769231,-18)(449.826086956522,-18)(444.882943143813,-18)(439.939799331104,-18)(434.996655518395,-18)(430.053511705686,-18)(425.110367892977,-18)(420.167224080268,-18)(415.224080267559,-18)(410.28093645485,-18)(405.33779264214,-18)(400.394648829431,-18)(395.451505016722,-18)(390.508361204013,-18)(385.565217391304,-18)(380.622073578595,-18)(375.678929765886,-18)(370.735785953177,-18)(365.792642140468,-18)(360.849498327759,-18)(355.90635451505,-18)(350.963210702341,-18)(346.020066889632,-18)(341.076923076923,-18)(336.133779264214,-18)(331.190635451505,-18)(326.247491638796,-18)(321.304347826087,-18)(316.361204013378,-18)(311.418060200669,-18)(306.47491638796,-18)(301.531772575251,-18)(296.588628762542,-18)(291.645484949833,-18)(286.702341137124,-18)(281.759197324415,-18)(276.816053511706,-18)(271.872909698997,-18)(266.929765886288,-18)(261.986622073579,-18)(257.04347826087,-18)(252.100334448161,-18)(247.157190635452,-18)(242.214046822742,-18)(237.270903010033,-18)(232.327759197324,-18)(227.384615384615,-18)(222.441471571906,-18)(217.498327759197,-18)(212.555183946488,-18)(207.612040133779,-18)(202.66889632107,-18)(197.725752508361,-18)(192.782608695652,-18)(187.839464882943,-18)(182.896321070234,-18)(177.953177257525,-18)(173.010033444816,-18)(168.066889632107,-18)(163.123745819398,-18)(158.180602006689,-18)(153.23745819398,-18)(148.294314381271,-18)(143.351170568562,-18)(138.408026755853,-18)(133.464882943144,-18)(128.521739130435,-18)(123.578595317726,-18)(118.635451505017,-18)(113.692307692308,-18)(108.749163879599,-18)(103.80602006689,-18)(98.8628762541806,-18)(93.9197324414716,-18)(88.9765886287625,-18)(84.0334448160535,-18)(79.0903010033445,-18)(74.1471571906355,-18)(69.2040133779264,-18)(64.2608695652174,-18)(59.3177257525084,-18)(54.3745819397993,-18)(49.4314381270903,-18)(44.4882943143813,-18)(39.5451505016722,-18.000003009883)(34.6020066889632,-18.000003009883)(29.6588628762542,-18.000003009883)(24.7157190635452,-18.000003009883)(19.7725752508361,-18.000003009883)(14.8294314381271,-18.000003009883)(9.88628762541806,-18.000003009883)(4.94314381270903,-18.000003009883)(0,-18.0000060194649)(-0.000494264954775508,-18)(-0.000247144833393367,-17.9397993311037)(-0.000247144833393367,-17.8795986622074)(-0.000247144833393367,-17.819397993311)(-0.000247144833393367,-17.7591973244147)(-0.000247144833393367,-17.6989966555184)(-0.000247144833393367,-17.6387959866221)(-0.000247144833393367,-17.5785953177258)(-0.000247144833393367,-17.5183946488294)(-0.000247144833393367,-17.4581939799331)(-0.000247144833393367,-17.3979933110368)(-0.000247144833393367,-17.3377926421405)(-0.000247144833393367,-17.2775919732441)(-0.000247144833393367,-17.2173913043478)(-0.000247144833393367,-17.1571906354515)(-0.000247144833393367,-17.0969899665552)(-0.000247144833393367,-17.0367892976589)(-0.000247144833393367,-16.9765886287625)(-0.000247144833393367,-16.9163879598662)(-0.000247144833393367,-16.8561872909699)(-0.000247144833393367,-16.7959866220736)(-0.000247144833393367,-16.7357859531773)(-0.000247144833393367,-16.6755852842809)(-0.000247144833393367,-16.6153846153846)(-0.000247144833393367,-16.5551839464883)(-0.000247144833393367,-16.494983277592)(-0.000247144833393367,-16.4347826086957)(-0.000247144833393367,-16.3745819397993)(-0.000247144833393367,-16.314381270903)(-0.000247144833393367,-16.2541806020067)(-0.000247144833393367,-16.1939799331104)(-0.000247144833393367,-16.133779264214)(-0.000247144833393367,-16.0735785953177)(-0.000247144833393367,-16.0133779264214)(-0.000247144833393367,-15.9531772575251)(-0.000247144833393367,-15.8929765886288)(-0.000247144833393367,-15.8327759197324)(-0.000247144833393367,-15.7725752508361)(-0.000247144833393367,-15.7123745819398)(-0.000247144833393367,-15.6521739130435)(-0.000247144833393367,-15.5919732441472)(-0.000247144833393367,-15.5317725752508)(-0.000247144833393367,-15.4715719063545)(-0.000247144833393367,-15.4113712374582)(-0.000247144833393367,-15.3511705685619)(-0.000247144833393367,-15.2909698996656)(-0.000247144833393367,-15.2307692307692)(-0.000247144833393367,-15.1705685618729)(-0.000247144833393367,-15.1103678929766)(-0.000247144833393367,-15.0501672240803)(-0.000247144833393367,-14.9899665551839)(-0.000247144833393367,-14.9297658862876)(-0.000247144833393367,-14.8695652173913)(-0.000247144833393367,-14.809364548495)(-0.000247144833393367,-14.7491638795987)(-0.000247144833393367,-14.6889632107023)(-0.000247144833393367,-14.628762541806)(-0.000247144833393367,-14.5685618729097)(-0.000247144833393367,-14.5083612040134)(-0.000247144833393367,-14.4481605351171)(-0.000247144833393367,-14.3879598662207)(-0.000247144833393367,-14.3277591973244)(-0.000247144833393367,-14.2675585284281)(-0.000247144833393367,-14.2073578595318)(-0.000247144833393367,-14.1471571906355)(-0.000247144833393367,-14.0869565217391)(-0.000247144833393367,-14.0267558528428)(-0.000247144833393367,-13.9665551839465)(-0.000247144833393367,-13.9063545150502)(-0.000247144833393367,-13.8461538461538)(-0.000247144833393367,-13.7859531772575)(-0.000247144833393367,-13.7257525083612)(-0.000247144833393367,-13.6655518394649)(-0.000247144833393367,-13.6053511705686)(-0.000247144833393367,-13.5451505016722)(-0.000247144833393367,-13.4849498327759)(-0.000247144833393367,-13.4247491638796)(-0.000247144833393367,-13.3645484949833)(-0.000247144833393367,-13.304347826087)(-0.000247144833393367,-13.2441471571906)(-0.000247144833393367,-13.1839464882943)(-0.000247144833393367,-13.123745819398)(-0.000247144833393367,-13.0635451505017)(-0.000247144833393367,-13.0033444816054)(-0.000247144833393367,-12.943143812709)(-0.000247144833393367,-12.8829431438127)(-0.000247144833393367,-12.8227424749164)(-0.000247144833393367,-12.7625418060201)(-0.000247144833393367,-12.7023411371237)(-0.000247144833393367,-12.6421404682274)(-0.000247144833393367,-12.5819397993311)(-0.000247144833393367,-12.5217391304348)(-0.000247144833393367,-12.4615384615385)(-0.000247144833393367,-12.4013377926421)(-0.000247144833393367,-12.3411371237458)(-0.000247144833393367,-12.2809364548495)(-0.000247144833393367,-12.2207357859532)(-0.000247144833393367,-12.1605351170569)(-0.000247144833393367,-12.1003344481605)(-0.000247144833393367,-12.0401337792642)(-0.000247144833393367,-11.9799331103679)(-0.000247144833393367,-11.9197324414716)(-0.000247144833393367,-11.8595317725753)(-0.000247144833393367,-11.7993311036789)(-0.000247144833393367,-11.7391304347826)(-0.000247144833393367,-11.6789297658863)(-0.000247144833393367,-11.61872909699)(-0.000247144833393367,-11.5585284280936)(-0.000247144833393367,-11.4983277591973)(-0.000247144833393367,-11.438127090301)(-0.000247144833393367,-11.3779264214047)(-0.000247144833393367,-11.3177257525084)(-0.000247144833393367,-11.257525083612)(-0.000247144833393367,-11.1973244147157)(-0.000247144833393367,-11.1371237458194)(-0.000247144833393367,-11.0769230769231)(-0.000247144833393367,-11.0167224080268)(-0.000247144833393367,-10.9565217391304)(-0.000247144833393367,-10.8963210702341)(-0.000247144833393367,-10.8361204013378)(-0.000247144833393367,-10.7759197324415)(-0.000247144833393367,-10.7157190635452)(-0.000247144833393367,-10.6555183946488)(-0.000247144833393367,-10.5953177257525)(-0.000247144833393367,-10.5351170568562)(-0.000247144833393367,-10.4749163879599)(-0.000247144833393367,-10.4147157190635)(-0.000247144833393367,-10.3545150501672)(-0.000247144833393367,-10.2943143812709)(-0.000247144833393367,-10.2341137123746)(-0.000247144833393367,-10.1739130434783)(-0.000247144833393367,-10.1137123745819)(-0.000247144833393367,-10.0535117056856)(-0.000247144833393367,-9.9933110367893)(-0.000247144833393367,-9.93311036789298)(-0.000247144833393367,-9.87290969899666)(-0.000247144833393367,-9.81270903010033)(-0.000247144833393367,-9.75250836120401)(-0.000247144833393367,-9.69230769230769)(-0.000247144833393367,-9.63210702341137)(-0.000247144833393367,-9.57190635451505)(-0.000247144833393367,-9.51170568561873)(-0.000247144833393367,-9.45150501672241)(-0.000247144833393367,-9.39130434782609)(-0.000247144833393367,-9.33110367892977)(-0.000247144833393367,-9.27090301003344)(-0.000247144833393367,-9.21070234113712)(-0.000247144833393367,-9.1505016722408)(-0.000247144833393367,-9.09030100334448)(-0.000247144833393367,-9.03010033444816)(-0.000247144833393367,-8.96989966555184)(-0.000247144833393367,-8.90969899665552)(-0.000247144833393367,-8.8494983277592)(-0.000247144833393367,-8.78929765886288)(-0.000247144833393367,-8.72909698996656)(-0.000247144833393367,-8.66889632107023)(-0.000247144833393367,-8.60869565217391)(-0.000247144833393367,-8.54849498327759)(-0.000247144833393367,-8.48829431438127)(-0.000247144833393367,-8.42809364548495)(-0.000247144833393367,-8.36789297658863)(-0.000247144833393367,-8.30769230769231)(-0.000247144833393367,-8.24749163879599)(-0.000247144833393367,-8.18729096989967)(-0.000247144833393367,-8.12709030100334)(-0.000247144833393367,-8.06688963210702)(-0.000247144833393367,-8.0066889632107)(-0.000247144833393367,-7.94648829431438)(-0.000247144833393367,-7.88628762541806)(-0.000247144833393367,-7.82608695652174)(-0.000247144833393367,-7.76588628762542)(-0.000247144833393367,-7.7056856187291)(-0.000247144833393367,-7.64548494983278)(-0.000247144833393367,-7.58528428093645)(-0.000247144833393367,-7.52508361204013)(-0.000247144833393367,-7.46488294314381)(-0.000247144833393367,-7.40468227424749)(-0.000247144833393367,-7.34448160535117)(-0.000247144833393367,-7.28428093645485)(-0.000247144833393367,-7.22408026755853)(-0.000247144833393367,-7.16387959866221)(-0.000247144833393367,-7.10367892976589)(-0.000247144833393367,-7.04347826086956)(-0.000247144833393367,-6.98327759197324)(-0.000247144833393367,-6.92307692307692)(-0.000247144833393367,-6.8628762541806)(-0.000247144833393367,-6.80267558528428)(-0.000247144833393367,-6.74247491638796)(-0.000247144833393367,-6.68227424749164)(-0.000247144833393367,-6.62207357859532)(-0.000247144833393367,-6.561872909699)(-0.000247144833393367,-6.50167224080267)(-0.000247144833393367,-6.44147157190636)(-0.000247144833393367,-6.38127090301003)(-0.000247144833393367,-6.32107023411371)(-0.000247144833393367,-6.26086956521739)(-0.000247144833393367,-6.20066889632107)(-0.000247144833393367,-6.14046822742475)(-0.000247144833393367,-6.08026755852843)(-0.000247144833393367,-6.02006688963211)(-0.000247144833393367,-5.95986622073579)(-0.000247144833393367,-5.89966555183947)(-0.000247144833393367,-5.83946488294314)(-0.000247144833393367,-5.77926421404682)(-0.000247144833393367,-5.7190635451505)(-0.000247144833393367,-5.65886287625418)(-0.000247144833393367,-5.59866220735786)(-0.000247144833393367,-5.53846153846154)(-0.000247144833393367,-5.47826086956522)(-0.000247144833393367,-5.4180602006689)(-0.000247144833393367,-5.35785953177258)(-0.000247144833393367,-5.29765886287625)(-0.000247144833393367,-5.23745819397993)(-0.000247144833393367,-5.17725752508361)(-0.000247144833393367,-5.11705685618729)(-0.000247144833393367,-5.05685618729097)(-0.000247144833393367,-4.99665551839465)(-0.000247144833393367,-4.93645484949833)(-0.000247144833393367,-4.87625418060201)(-0.000247144833393367,-4.81605351170569)(-0.000247144833393367,-4.75585284280936)(-0.000247144833393367,-4.69565217391304)(-0.000247144833393367,-4.63545150501672)(0,-4.5752508361204)(0,-4.51505016722408)(0,-4.45484949832776)(0,-4.39464882943144)(0,-4.33444816053512)(0,-4.2742474916388)(0,-4.21404682274247)(0,-4.15384615384615)(0,-4.09364548494983)(0,-4.03344481605351)(0,-3.97324414715719)(0,-3.91304347826087)(0,-3.85284280936455)(0,-3.79264214046823)(0,-3.73244147157191)(0,-3.67224080267559)(0,-3.61204013377926)(0,-3.55183946488294)(0,-3.49163879598662)(0,-3.4314381270903)(0,-3.37123745819398)(0,-3.31103678929766)(0,-3.25083612040134)(0,-3.19063545150502)(0,-3.1304347826087)(0,-3.07023411371237)(0,-3.01003344481605)(0,-2.94983277591973)(0,-2.88963210702341)(0,-2.82943143812709)(0,-2.76923076923077)(0,-2.70903010033445)(0,-2.64882943143813)(0,-2.58862876254181)(0,-2.52842809364548)(0,-2.46822742474916)(0,-2.40802675585284)(0,-2.34782608695652)(0,-2.2876254180602)(0,-2.22742474916388)(0,-2.16722408026756)(0,-2.10702341137124)(0,-2.04682274247492)(0,-1.98662207357859)(0,-1.92642140468227)(0,-1.86622073578595)(0,-1.80602006688963)(0,-1.74581939799331)(0,-1.68561872909699)(0,-1.62541806020067)(0,-1.56521739130435)(0,-1.50501672240803)(0,-1.4448160535117)(0,-1.38461538461538)(0,-1.32441471571906)(0,-1.26421404682274)(0,-1.20401337792642)(0,-1.1438127090301)(0,-1.08361204013378)(0,-1.02341137123746)(0,-0.963210702341136)(0,-0.903010033444815)(0,-0.842809364548494)(0,-0.782608695652176)(0,-0.722408026755854)(0,-0.662207357859533)(0,-0.602006688963211)(0,-0.54180602006689)(0,-0.481605351170568)(0,-0.421404682274247)(0,-0.361204013377925)(0,-0.301003344481604)(0,-0.240802675585286)(0,-0.180602006688964)(0,-0.120401337792643)(0,-0.0602006688963215)(0,0)};

\addplot [fill=darkgray,draw=black,forget plot] coordinates{ (0,-4.63545150501672)(4.94314381270903,-4.69565217391304)(9.88628762541806,-4.75585284280936)(9.88628762541806,-4.81605351170569)(14.8294314381271,-4.87625418060201)(19.7725752508361,-4.93645484949833)(24.7157190635452,-4.99665551839465)(24.7157190635452,-5.05685618729097)(29.6588628762542,-5.11705685618729)(34.6020066889632,-5.17725752508361)(39.5451505016722,-5.23745819397993)(44.4882943143813,-5.29765886287625)(44.4882943143813,-5.35785953177258)(49.4314381270903,-5.4180602006689)(54.3745819397993,-5.47826086956522)(59.3177257525084,-5.53846153846154)(64.2608695652174,-5.59866220735786)(64.2608695652174,-5.65886287625418)(69.2040133779264,-5.7190635451505)(74.1471571906355,-5.77926421404682)(79.0903010033445,-5.83946488294314)(84.0334448160535,-5.89966555183947)(88.9765886287625,-5.95986622073579)(88.9765886287625,-6.02006688963211)(93.9197324414716,-6.08026755852843)(98.8628762541806,-6.14046822742475)(103.80602006689,-6.20066889632107)(108.749163879599,-6.26086956521739)(108.749163879599,-6.32107023411371)(113.692307692308,-6.38127090301003)(118.635451505017,-6.44147157190636)(123.578595317726,-6.50167224080267)(128.521739130435,-6.561872909699)(128.521739130435,-6.62207357859532)(133.464882943144,-6.68227424749164)(138.408026755853,-6.74247491638796)(143.351170568562,-6.80267558528428)(148.294314381271,-6.8628762541806)(153.23745819398,-6.92307692307692)(158.180602006689,-6.98327759197324)(163.123745819398,-7.04347826086956)(168.066889632107,-7.10367892976589)(173.010033444816,-7.16387959866221)(177.953177257525,-7.16387959866221)(182.896321070234,-7.22408026755853)(187.839464882943,-7.28428093645485)(192.782608695652,-7.28428093645485)(197.725752508361,-7.34448160535117)(202.66889632107,-7.34448160535117)(207.612040133779,-7.34448160535117)(212.555183946488,-7.34448160535117)(217.498327759197,-7.34448160535117)(222.441471571906,-7.34448160535117)(227.384615384615,-7.28428093645485)(232.327759197324,-7.28428093645485)(237.270903010033,-7.22408026755853)(242.214046822742,-7.16387959866221)(247.157190635452,-7.16387959866221)(252.100334448161,-7.10367892976589)(252.100334448161,-7.04347826086956)(257.04347826087,-6.98327759197324)(261.986622073579,-6.92307692307692)(261.986622073579,-6.8628762541806)(266.929765886288,-6.80267558528428)(266.929765886288,-6.74247491638796)(266.929765886288,-6.68227424749164)(271.872909698997,-6.62207357859532)(276.816053511706,-6.561872909699)(276.816053511706,-6.50167224080267)(281.759197324415,-6.44147157190636)(286.702341137124,-6.44147157190636)(291.645484949833,-6.38127090301003)(296.588628762542,-6.32107023411371)(301.531772575251,-6.26086956521739)(306.47491638796,-6.26086956521739)(311.418060200669,-6.26086956521739)(316.361204013378,-6.20066889632107)(321.304347826087,-6.20066889632107)(326.247491638796,-6.20066889632107)(331.190635451505,-6.20066889632107)(336.133779264214,-6.20066889632107)(341.076923076923,-6.20066889632107)(346.020066889632,-6.20066889632107)(350.963210702341,-6.20066889632107)(355.90635451505,-6.26086956521739)(360.849498327759,-6.26086956521739)(365.792642140468,-6.26086956521739)(370.735785953177,-6.32107023411371)(375.678929765886,-6.32107023411371)(380.622073578595,-6.38127090301003)(385.565217391304,-6.38127090301003)(390.508361204013,-6.44147157190636)(395.451505016722,-6.44147157190636)(400.394648829431,-6.50167224080267)(405.33779264214,-6.50167224080267)(410.28093645485,-6.561872909699)(415.224080267559,-6.561872909699)(420.167224080268,-6.62207357859532)(425.110367892977,-6.62207357859532)(430.053511705686,-6.68227424749164)(434.996655518395,-6.68227424749164)(439.939799331104,-6.68227424749164)(444.882943143813,-6.74247491638796)(449.826086956522,-6.74247491638796)(454.769230769231,-6.74247491638796)(459.71237458194,-6.74247491638796)(464.655518394649,-6.74247491638796)(469.598662207358,-6.74247491638796)(474.541806020067,-6.74247491638796)(479.484949832776,-6.74247491638796)(484.428093645485,-6.68227424749164)(489.371237458194,-6.68227424749164)(494.314381270903,-6.68227424749164)(499.257525083612,-6.68227424749164)(504.200668896321,-6.68227424749164)(509.14381270903,-6.68227424749164)(514.086956521739,-6.68227424749164)(519.030100334448,-6.68227424749164)(523.973244147157,-6.68227424749164)(528.916387959866,-6.68227424749164)(533.859531772575,-6.68227424749164)(538.802675585284,-6.68227424749164)(543.745819397993,-6.68227424749164)(548.688963210702,-6.68227424749164)(553.632107023411,-6.74247491638796)(558.57525083612,-6.74247491638796)(563.518394648829,-6.74247491638796)(568.461538461538,-6.74247491638796)(573.404682274248,-6.80267558528428)(578.347826086957,-6.80267558528428)(583.290969899666,-6.80267558528428)(588.234113712375,-6.80267558528428)(593.177257525084,-6.80267558528428)(598.120401337793,-6.80267558528428)(603.063545150502,-6.80267558528428)(608.006688963211,-6.80267558528428)(612.94983277592,-6.80267558528428)(617.892976588629,-6.80267558528428)(622.836120401338,-6.80267558528428)(627.779264214047,-6.80267558528428)(632.722408026756,-6.74247491638796)(637.665551839465,-6.74247491638796)(642.608695652174,-6.68227424749164)(647.551839464883,-6.68227424749164)(652.494983277592,-6.62207357859532)(657.438127090301,-6.62207357859532)(662.38127090301,-6.561872909699)(667.324414715719,-6.561872909699)(672.267558528428,-6.561872909699)(677.210702341137,-6.50167224080267)(682.153846153846,-6.50167224080267)(687.096989966555,-6.50167224080267)(692.040133779264,-6.44147157190636)(696.983277591973,-6.44147157190636)(701.926421404682,-6.44147157190636)(706.869565217391,-6.44147157190636)(711.8127090301,-6.44147157190636)(716.755852842809,-6.44147157190636)(721.698996655518,-6.44147157190636)(726.642140468227,-6.44147157190636)(731.585284280936,-6.44147157190636)(736.528428093646,-6.50167224080267)(741.471571906354,-6.50167224080267)(746.414715719064,-6.50167224080267)(751.357859531773,-6.561872909699)(756.301003344482,-6.561872909699)(761.244147157191,-6.62207357859532)(766.1872909699,-6.62207357859532)(771.130434782609,-6.68227424749164)(776.073578595318,-6.68227424749164)(781.016722408027,-6.74247491638796)(785.959866220736,-6.80267558528428)(790.903010033445,-6.80267558528428)(795.846153846154,-6.8628762541806)(800.789297658863,-6.92307692307692)(805.732441471572,-6.92307692307692)(810.675585284281,-6.98327759197324)(815.61872909699,-7.04347826086956)(820.561872909699,-7.04347826086956)(825.505016722408,-7.10367892976589)(830.448160535117,-7.10367892976589)(835.391304347826,-7.16387959866221)(840.334448160535,-7.16387959866221)(845.277591973244,-7.16387959866221)(850.220735785953,-7.16387959866221)(855.163879598662,-7.16387959866221)(860.107023411371,-7.16387959866221)(865.05016722408,-7.16387959866221)(869.993311036789,-7.16387959866221)(874.936454849498,-7.16387959866221)(879.879598662207,-7.16387959866221)(884.822742474916,-7.10367892976589)(889.765886287625,-7.10367892976589)(894.709030100334,-7.10367892976589)(899.652173913044,-7.10367892976589)(904.595317725752,-7.04347826086956)(909.538461538462,-7.04347826086956)(914.481605351171,-7.04347826086956)(919.42474916388,-7.04347826086956)(924.367892976589,-7.04347826086956)(929.311036789298,-7.04347826086956)(934.254180602007,-7.04347826086956)(939.197324414716,-6.98327759197324)(944.140468227425,-6.98327759197324)(949.083612040134,-6.98327759197324)(954.026755852843,-7.04347826086956)(958.969899665552,-7.04347826086956)(963.913043478261,-7.04347826086956)(968.85618729097,-7.04347826086956)(973.799331103679,-7.04347826086956)(978.742474916388,-7.04347826086956)(983.685618729097,-7.10367892976589)(988.628762541806,-7.10367892976589)(993.571906354515,-7.10367892976589)(998.515050167224,-7.16387959866221)(1003.45819397993,-7.16387959866221)(1008.40133779264,-7.16387959866221)(1013.34448160535,-7.22408026755853)(1018.28762541806,-7.22408026755853)(1023.23076923077,-7.22408026755853)(1028.17391304348,-7.22408026755853)(1033.11705685619,-7.22408026755853)(1038.0602006689,-7.22408026755853)(1043.00334448161,-7.22408026755853)(1047.94648829431,-7.22408026755853)(1052.88963210702,-7.22408026755853)(1057.83277591973,-7.16387959866221)(1062.77591973244,-7.16387959866221)(1067.71906354515,-7.10367892976589)(1072.66220735786,-7.10367892976589)(1077.60535117057,-7.04347826086956)(1082.54849498328,-7.04347826086956)(1087.49163879599,-6.98327759197324)(1092.4347826087,-6.92307692307692)(1097.3779264214,-6.92307692307692)(1102.32107023411,-6.8628762541806)(1107.26421404682,-6.80267558528428)(1112.20735785953,-6.80267558528428)(1117.15050167224,-6.74247491638796)(1122.09364548495,-6.74247491638796)(1127.03678929766,-6.74247491638796)(1131.97993311037,-6.68227424749164)(1136.92307692308,-6.68227424749164)(1141.86622073579,-6.68227424749164)(1146.8093645485,-6.62207357859532)(1151.7525083612,-6.62207357859532)(1156.69565217391,-6.62207357859532)(1161.63879598662,-6.62207357859532)(1166.58193979933,-6.62207357859532)(1171.52508361204,-6.62207357859532)(1176.46822742475,-6.68227424749164)(1181.41137123746,-6.68227424749164)(1186.35451505017,-6.68227424749164)(1191.29765886288,-6.74247491638796)(1196.24080267559,-6.74247491638796)(1201.18394648829,-6.74247491638796)(1206.127090301,-6.80267558528428)(1211.07023411371,-6.8628762541806)(1216.01337792642,-6.8628762541806)(1220.95652173913,-6.92307692307692)(1225.89966555184,-6.92307692307692)(1230.84280936455,-6.98327759197324)(1235.78595317726,-7.04347826086956)(1240.72909698997,-7.04347826086956)(1245.67224080268,-7.10367892976589)(1250.61538461538,-7.16387959866221)(1255.55852842809,-7.22408026755853)(1260.5016722408,-7.22408026755853)(1265.44481605351,-7.28428093645485)(1270.38795986622,-7.34448160535117)(1275.33110367893,-7.34448160535117)(1280.27424749164,-7.40468227424749)(1285.21739130435,-7.40468227424749)(1290.16053511706,-7.46488294314381)(1295.10367892977,-7.46488294314381)(1300.04682274247,-7.52508361204013)(1304.98996655518,-7.52508361204013)(1309.93311036789,-7.52508361204013)(1314.8762541806,-7.58528428093645)(1319.81939799331,-7.58528428093645)(1324.76254180602,-7.58528428093645)(1329.70568561873,-7.58528428093645)(1334.64882943144,-7.64548494983278)(1339.59197324415,-7.64548494983278)(1344.53511705686,-7.64548494983278)(1349.47826086957,-7.64548494983278)(1354.42140468227,-7.7056856187291)(1359.36454849498,-7.7056856187291)(1364.30769230769,-7.76588628762542)(1369.2508361204,-7.76588628762542)(1374.19397993311,-7.76588628762542)(1379.13712374582,-7.82608695652174)(1384.08026755853,-7.88628762541806)(1389.02341137124,-7.88628762541806)(1393.96655518395,-7.94648829431438)(1398.90969899666,-8.0066889632107)(1403.85284280936,-8.06688963210702)(1408.79598662207,-8.12709030100334)(1413.73913043478,-8.12709030100334)(1418.68227424749,-8.18729096989967)(1423.6254180602,-8.24749163879599)(1428.56856187291,-8.30769230769231)(1433.51170568562,-8.36789297658863)(1438.45484949833,-8.42809364548495)(1443.39799331104,-8.48829431438127)(1448.34113712375,-8.48829431438127)(1453.28428093645,-8.54849498327759)(1458.22742474916,-8.54849498327759)(1463.17056856187,-8.54849498327759)(1468.11371237458,-8.48829431438127)(1473.05685618729,-8.48829431438127)(1478,-8.42809364548495)(1478,-8.48829431438127)(1478,-8.54849498327759)(1478,-8.60869565217391)(1478,-8.66889632107023)(1478,-8.72909698996656)(1478,-8.78929765886288)(1478,-8.8494983277592)(1478,-8.90969899665552)(1478,-8.96989966555184)(1478,-9.03010033444816)(1478,-9.09030100334448)(1478,-9.1505016722408)(1478,-9.21070234113712)(1478,-9.27090301003344)(1478,-9.33110367892977)(1478,-9.39130434782609)(1478,-9.45150501672241)(1478,-9.51170568561873)(1478,-9.57190635451505)(1478,-9.63210702341137)(1478,-9.69230769230769)(1478,-9.75250836120401)(1478,-9.81270903010033)(1478,-9.87290969899666)(1473.05685618729,-9.87290969899666)(1468.11371237458,-9.81270903010033)(1463.17056856187,-9.81270903010033)(1458.22742474916,-9.81270903010033)(1453.28428093645,-9.87290969899666)(1448.34113712375,-9.87290969899666)(1443.39799331104,-9.93311036789298)(1438.45484949833,-9.9933110367893)(1433.51170568562,-9.9933110367893)(1428.56856187291,-10.0535117056856)(1423.6254180602,-10.1137123745819)(1418.68227424749,-10.1739130434783)(1413.73913043478,-10.2341137123746)(1408.79598662207,-10.2943143812709)(1403.85284280936,-10.3545150501672)(1398.90969899666,-10.4147157190635)(1393.96655518395,-10.4147157190635)(1389.02341137124,-10.4749163879599)(1384.08026755853,-10.4749163879599)(1379.13712374582,-10.5351170568562)(1374.19397993311,-10.5351170568562)(1369.2508361204,-10.5953177257525)(1364.30769230769,-10.5953177257525)(1359.36454849498,-10.5953177257525)(1354.42140468227,-10.5953177257525)(1349.47826086957,-10.5953177257525)(1344.53511705686,-10.5953177257525)(1339.59197324415,-10.5953177257525)(1334.64882943144,-10.5953177257525)(1329.70568561873,-10.5953177257525)(1324.76254180602,-10.5953177257525)(1319.81939799331,-10.5953177257525)(1314.8762541806,-10.5351170568562)(1309.93311036789,-10.5351170568562)(1304.98996655518,-10.5351170568562)(1300.04682274247,-10.4749163879599)(1295.10367892977,-10.4749163879599)(1290.16053511706,-10.4749163879599)(1285.21739130435,-10.4749163879599)(1280.27424749164,-10.4147157190635)(1275.33110367893,-10.4147157190635)(1270.38795986622,-10.4147157190635)(1265.44481605351,-10.4147157190635)(1260.5016722408,-10.4147157190635)(1255.55852842809,-10.4749163879599)(1250.61538461538,-10.4749163879599)(1245.67224080268,-10.4749163879599)(1240.72909698997,-10.5351170568562)(1235.78595317726,-10.5351170568562)(1230.84280936455,-10.5953177257525)(1225.89966555184,-10.5953177257525)(1220.95652173913,-10.5953177257525)(1216.01337792642,-10.6555183946488)(1211.07023411371,-10.6555183946488)(1206.127090301,-10.6555183946488)(1201.18394648829,-10.7157190635452)(1196.24080267559,-10.7157190635452)(1191.29765886288,-10.7157190635452)(1186.35451505017,-10.7157190635452)(1181.41137123746,-10.7759197324415)(1176.46822742475,-10.7759197324415)(1171.52508361204,-10.7759197324415)(1166.58193979933,-10.7759197324415)(1161.63879598662,-10.7759197324415)(1156.69565217391,-10.7759197324415)(1151.7525083612,-10.7759197324415)(1146.8093645485,-10.7759197324415)(1141.86622073579,-10.7759197324415)(1136.92307692308,-10.7157190635452)(1131.97993311037,-10.7157190635452)(1127.03678929766,-10.7157190635452)(1122.09364548495,-10.7157190635452)(1117.15050167224,-10.6555183946488)(1112.20735785953,-10.6555183946488)(1107.26421404682,-10.6555183946488)(1102.32107023411,-10.6555183946488)(1097.3779264214,-10.5953177257525)(1092.4347826087,-10.5953177257525)(1087.49163879599,-10.5953177257525)(1082.54849498328,-10.5953177257525)(1077.60535117057,-10.5953177257525)(1072.66220735786,-10.5953177257525)(1067.71906354515,-10.5953177257525)(1062.77591973244,-10.5953177257525)(1057.83277591973,-10.5953177257525)(1052.88963210702,-10.6555183946488)(1047.94648829431,-10.6555183946488)(1043.00334448161,-10.6555183946488)(1038.0602006689,-10.7157190635452)(1033.11705685619,-10.7157190635452)(1028.17391304348,-10.7759197324415)(1023.23076923077,-10.7759197324415)(1018.28762541806,-10.7759197324415)(1013.34448160535,-10.8361204013378)(1008.40133779264,-10.8361204013378)(1003.45819397993,-10.8361204013378)(998.515050167224,-10.8963210702341)(993.571906354515,-10.8963210702341)(988.628762541806,-10.8963210702341)(983.685618729097,-10.8963210702341)(978.742474916388,-10.8963210702341)(973.799331103679,-10.8963210702341)(968.85618729097,-10.8963210702341)(963.913043478261,-10.8963210702341)(958.969899665552,-10.8963210702341)(954.026755852843,-10.8963210702341)(949.083612040134,-10.8963210702341)(944.140468227425,-10.8361204013378)(939.197324414716,-10.8361204013378)(934.254180602007,-10.7759197324415)(929.311036789298,-10.7759197324415)(924.367892976589,-10.7157190635452)(919.42474916388,-10.7157190635452)(914.481605351171,-10.6555183946488)(909.538461538462,-10.5953177257525)(904.595317725752,-10.5953177257525)(899.652173913044,-10.5351170568562)(894.709030100334,-10.4749163879599)(889.765886287625,-10.4147157190635)(884.822742474916,-10.3545150501672)(879.879598662207,-10.2943143812709)(874.936454849498,-10.2341137123746)(869.993311036789,-10.1739130434783)(865.05016722408,-10.1137123745819)(860.107023411371,-10.0535117056856)(855.163879598662,-9.9933110367893)(850.220735785953,-9.93311036789298)(845.277591973244,-9.87290969899666)(840.334448160535,-9.81270903010033)(835.391304347826,-9.81270903010033)(830.448160535117,-9.75250836120401)(825.505016722408,-9.75250836120401)(820.561872909699,-9.75250836120401)(815.61872909699,-9.75250836120401)(810.675585284281,-9.69230769230769)(805.732441471572,-9.69230769230769)(800.789297658863,-9.69230769230769)(795.846153846154,-9.69230769230769)(790.903010033445,-9.63210702341137)(785.959866220736,-9.63210702341137)(781.016722408027,-9.63210702341137)(776.073578595318,-9.63210702341137)(771.130434782609,-9.63210702341137)(766.1872909699,-9.63210702341137)(761.244147157191,-9.63210702341137)(756.301003344482,-9.57190635451505)(751.357859531773,-9.57190635451505)(746.414715719064,-9.57190635451505)(741.471571906354,-9.57190635451505)(736.528428093646,-9.51170568561873)(731.585284280936,-9.51170568561873)(726.642140468227,-9.51170568561873)(721.698996655518,-9.45150501672241)(716.755852842809,-9.39130434782609)(711.8127090301,-9.33110367892977)(706.869565217391,-9.27090301003344)(701.926421404682,-9.21070234113712)(696.983277591973,-9.1505016722408)(692.040133779264,-9.09030100334448)(687.096989966555,-9.03010033444816)(682.153846153846,-8.96989966555184)(677.210702341137,-8.96989966555184)(672.267558528428,-8.90969899665552)(667.324414715719,-8.8494983277592)(662.38127090301,-8.78929765886288)(657.438127090301,-8.72909698996656)(652.494983277592,-8.72909698996656)(647.551839464883,-8.66889632107023)(642.608695652174,-8.66889632107023)(637.665551839465,-8.60869565217391)(632.722408026756,-8.60869565217391)(627.779264214047,-8.60869565217391)(622.836120401338,-8.60869565217391)(617.892976588629,-8.66889632107023)(612.94983277592,-8.66889632107023)(608.006688963211,-8.66889632107023)(603.063545150502,-8.66889632107023)(598.120401337793,-8.72909698996656)(593.177257525084,-8.72909698996656)(588.234113712375,-8.78929765886288)(583.290969899666,-8.78929765886288)(578.347826086957,-8.8494983277592)(573.404682274248,-8.8494983277592)(568.461538461538,-8.90969899665552)(563.518394648829,-8.90969899665552)(558.57525083612,-8.96989966555184)(553.632107023411,-8.96989966555184)(548.688963210702,-9.03010033444816)(543.745819397993,-9.03010033444816)(538.802675585284,-9.03010033444816)(533.859531772575,-9.09030100334448)(528.916387959866,-9.09030100334448)(523.973244147157,-9.09030100334448)(519.030100334448,-9.09030100334448)(514.086956521739,-9.09030100334448)(509.14381270903,-9.09030100334448)(504.200668896321,-9.09030100334448)(499.257525083612,-9.09030100334448)(494.314381270903,-9.03010033444816)(489.371237458194,-9.03010033444816)(484.428093645485,-9.03010033444816)(479.484949832776,-8.96989966555184)(474.541806020067,-8.96989966555184)(469.598662207358,-8.90969899665552)(464.655518394649,-8.90969899665552)(459.71237458194,-8.8494983277592)(454.769230769231,-8.8494983277592)(449.826086956522,-8.8494983277592)(444.882943143813,-8.78929765886288)(439.939799331104,-8.78929765886288)(434.996655518395,-8.78929765886288)(430.053511705686,-8.8494983277592)(425.110367892977,-8.8494983277592)(420.167224080268,-8.8494983277592)(415.224080267559,-8.90969899665552)(410.28093645485,-8.90969899665552)(405.33779264214,-8.96989966555184)(400.394648829431,-9.03010033444816)(395.451505016722,-9.09030100334448)(390.508361204013,-9.09030100334448)(385.565217391304,-9.1505016722408)(380.622073578595,-9.21070234113712)(375.678929765886,-9.21070234113712)(370.735785953177,-9.27090301003344)(365.792642140468,-9.33110367892977)(360.849498327759,-9.33110367892977)(355.90635451505,-9.33110367892977)(350.963210702341,-9.39130434782609)(346.020066889632,-9.39130434782609)(341.076923076923,-9.39130434782609)(336.133779264214,-9.45150501672241)(331.190635451505,-9.45150501672241)(326.247491638796,-9.45150501672241)(321.304347826087,-9.45150501672241)(316.361204013378,-9.45150501672241)(311.418060200669,-9.45150501672241)(306.47491638796,-9.39130434782609)(301.531772575251,-9.39130434782609)(296.588628762542,-9.39130434782609)(291.645484949833,-9.39130434782609)(286.702341137124,-9.33110367892977)(281.759197324415,-9.33110367892977)(276.816053511706,-9.33110367892977)(271.872909698997,-9.27090301003344)(266.929765886288,-9.27090301003344)(261.986622073579,-9.21070234113712)(257.04347826087,-9.21070234113712)(252.100334448161,-9.21070234113712)(247.157190635452,-9.1505016722408)(242.214046822742,-9.1505016722408)(237.270903010033,-9.1505016722408)(232.327759197324,-9.1505016722408)(227.384615384615,-9.1505016722408)(222.441471571906,-9.1505016722408)(217.498327759197,-9.1505016722408)(212.555183946488,-9.21070234113712)(207.612040133779,-9.21070234113712)(202.66889632107,-9.27090301003344)(197.725752508361,-9.27090301003344)(192.782608695652,-9.33110367892977)(187.839464882943,-9.39130434782609)(182.896321070234,-9.45150501672241)(177.953177257525,-9.45150501672241)(173.010033444816,-9.51170568561873)(168.066889632107,-9.57190635451505)(163.123745819398,-9.63210702341137)(158.180602006689,-9.69230769230769)(153.23745819398,-9.75250836120401)(148.294314381271,-9.81270903010033)(143.351170568562,-9.87290969899666)(138.408026755853,-9.93311036789298)(133.464882943144,-9.9933110367893)(128.521739130435,-10.0535117056856)(123.578595317726,-10.1137123745819)(118.635451505017,-10.1739130434783)(118.635451505017,-10.2341137123746)(113.692307692308,-10.2943143812709)(108.749163879599,-10.3545150501672)(103.80602006689,-10.4147157190635)(103.80602006689,-10.4749163879599)(98.8628762541806,-10.5351170568562)(93.9197324414716,-10.5953177257525)(93.9197324414716,-10.6555183946488)(88.9765886287625,-10.7157190635452)(88.9765886287625,-10.7759197324415)(84.0334448160535,-10.8361204013378)(79.0903010033445,-10.8963210702341)(79.0903010033445,-10.9565217391304)(74.1471571906355,-11.0167224080268)(74.1471571906355,-11.0769230769231)(74.1471571906355,-11.1371237458194)(69.2040133779264,-11.1973244147157)(69.2040133779264,-11.257525083612)(64.2608695652174,-11.3177257525084)(64.2608695652174,-11.3779264214047)(64.2608695652174,-11.438127090301)(59.3177257525084,-11.4983277591973)(59.3177257525084,-11.5585284280936)(59.3177257525084,-11.61872909699)(59.3177257525084,-11.6789297658863)(54.3745819397993,-11.7391304347826)(54.3745819397993,-11.7993311036789)(54.3745819397993,-11.8595317725753)(54.3745819397993,-11.9197324414716)(54.3745819397993,-11.9799331103679)(49.4314381270903,-12.0401337792642)(49.4314381270903,-12.1003344481605)(49.4314381270903,-12.1605351170569)(49.4314381270903,-12.2207357859532)(49.4314381270903,-12.2809364548495)(49.4314381270903,-12.3411371237458)(49.4314381270903,-12.4013377926421)(49.4314381270903,-12.4615384615385)(49.4314381270903,-12.5217391304348)(49.4314381270903,-12.5819397993311)(49.4314381270903,-12.6421404682274)(49.4314381270903,-12.7023411371237)(49.4314381270903,-12.7625418060201)(49.4314381270903,-12.8227424749164)(49.4314381270903,-12.8829431438127)(49.4314381270903,-12.943143812709)(49.4314381270903,-13.0033444816054)(49.4314381270903,-13.0635451505017)(49.4314381270903,-13.123745819398)(54.3745819397993,-13.1839464882943)(54.3745819397993,-13.2441471571906)(54.3745819397993,-13.304347826087)(54.3745819397993,-13.3645484949833)(54.3745819397993,-13.4247491638796)(54.3745819397993,-13.4849498327759)(54.3745819397993,-13.5451505016722)(54.3745819397993,-13.6053511705686)(54.3745819397993,-13.6655518394649)(54.3745819397993,-13.7257525083612)(54.3745819397993,-13.7859531772575)(59.3177257525084,-13.8461538461538)(59.3177257525084,-13.9063545150502)(59.3177257525084,-13.9665551839465)(59.3177257525084,-14.0267558528428)(59.3177257525084,-14.0869565217391)(59.3177257525084,-14.1471571906355)(59.3177257525084,-14.2073578595318)(54.3745819397993,-14.2675585284281)(54.3745819397993,-14.3277591973244)(54.3745819397993,-14.3879598662207)(54.3745819397993,-14.4481605351171)(54.3745819397993,-14.5083612040134)(54.3745819397993,-14.5685618729097)(54.3745819397993,-14.628762541806)(54.3745819397993,-14.6889632107023)(54.3745819397993,-14.7491638795987)(49.4314381270903,-14.809364548495)(49.4314381270903,-14.8695652173913)(49.4314381270903,-14.9297658862876)(49.4314381270903,-14.9899665551839)(49.4314381270903,-15.0501672240803)(44.4882943143813,-15.1103678929766)(44.4882943143813,-15.1705685618729)(44.4882943143813,-15.2307692307692)(44.4882943143813,-15.2909698996656)(44.4882943143813,-15.3511705685619)(39.5451505016722,-15.4113712374582)(39.5451505016722,-15.4715719063545)(39.5451505016722,-15.5317725752508)(39.5451505016722,-15.5919732441472)(39.5451505016722,-15.6521739130435)(39.5451505016722,-15.7123745819398)(39.5451505016722,-15.7725752508361)(34.6020066889632,-15.8327759197324)(34.6020066889632,-15.8929765886288)(34.6020066889632,-15.9531772575251)(34.6020066889632,-16.0133779264214)(34.6020066889632,-16.0735785953177)(34.6020066889632,-16.133779264214)(34.6020066889632,-16.1939799331104)(34.6020066889632,-16.2541806020067)(34.6020066889632,-16.314381270903)(34.6020066889632,-16.3745819397993)(34.6020066889632,-16.4347826086957)(34.6020066889632,-16.494983277592)(34.6020066889632,-16.5551839464883)(34.6020066889632,-16.6153846153846)(34.6020066889632,-16.6755852842809)(34.6020066889632,-16.7357859531773)(34.6020066889632,-16.7959866220736)(34.6020066889632,-16.8561872909699)(34.6020066889632,-16.9163879598662)(34.6020066889632,-16.9765886287625)(34.6020066889632,-17.0367892976589)(34.6020066889632,-17.0969899665552)(34.6020066889632,-17.1571906354515)(34.6020066889632,-17.2173913043478)(34.6020066889632,-17.2775919732441)(34.6020066889632,-17.3377926421405)(34.6020066889632,-17.3979933110368)(34.6020066889632,-17.4581939799331)(34.6020066889632,-17.5183946488294)(34.6020066889632,-17.5785953177258)(34.6020066889632,-17.6387959866221)(39.5451505016722,-17.6989966555184)(39.5451505016722,-17.7591973244147)(39.5451505016722,-17.819397993311)(39.5451505016722,-17.8795986622074)(39.5451505016722,-17.9397993311037)(39.5451505016722,-18)(34.6020066889632,-18)(29.6588628762542,-18)(24.7157190635452,-18)(19.7725752508361,-18)(14.8294314381271,-18)(9.88628762541806,-18)(4.94314381270903,-18)(0,-18.0000030097325)(-0.000247132477387754,-18)(0,-17.9397993311037)(0,-17.8795986622074)(0,-17.819397993311)(0,-17.7591973244147)(0,-17.6989966555184)(0,-17.6387959866221)(0,-17.5785953177258)(0,-17.5183946488294)(0,-17.4581939799331)(0,-17.3979933110368)(0,-17.3377926421405)(0,-17.2775919732441)(0,-17.2173913043478)(0,-17.1571906354515)(0,-17.0969899665552)(0,-17.0367892976589)(0,-16.9765886287625)(0,-16.9163879598662)(0,-16.8561872909699)(0,-16.7959866220736)(0,-16.7357859531773)(0,-16.6755852842809)(0,-16.6153846153846)(0,-16.5551839464883)(0,-16.494983277592)(0,-16.4347826086957)(0,-16.3745819397993)(0,-16.314381270903)(0,-16.2541806020067)(0,-16.1939799331104)(0,-16.133779264214)(0,-16.0735785953177)(0,-16.0133779264214)(0,-15.9531772575251)(0,-15.8929765886288)(0,-15.8327759197324)(0,-15.7725752508361)(0,-15.7123745819398)(0,-15.6521739130435)(0,-15.5919732441472)(0,-15.5317725752508)(0,-15.4715719063545)(0,-15.4113712374582)(0,-15.3511705685619)(0,-15.2909698996656)(0,-15.2307692307692)(0,-15.1705685618729)(0,-15.1103678929766)(0,-15.0501672240803)(0,-14.9899665551839)(0,-14.9297658862876)(0,-14.8695652173913)(0,-14.809364548495)(0,-14.7491638795987)(0,-14.6889632107023)(0,-14.628762541806)(0,-14.5685618729097)(0,-14.5083612040134)(0,-14.4481605351171)(0,-14.3879598662207)(0,-14.3277591973244)(0,-14.2675585284281)(0,-14.2073578595318)(0,-14.1471571906355)(0,-14.0869565217391)(0,-14.0267558528428)(0,-13.9665551839465)(0,-13.9063545150502)(0,-13.8461538461538)(0,-13.7859531772575)(0,-13.7257525083612)(0,-13.6655518394649)(0,-13.6053511705686)(0,-13.5451505016722)(0,-13.4849498327759)(0,-13.4247491638796)(0,-13.3645484949833)(0,-13.304347826087)(0,-13.2441471571906)(0,-13.1839464882943)(0,-13.123745819398)(0,-13.0635451505017)(0,-13.0033444816054)(0,-12.943143812709)(0,-12.8829431438127)(0,-12.8227424749164)(0,-12.7625418060201)(0,-12.7023411371237)(0,-12.6421404682274)(0,-12.5819397993311)(0,-12.5217391304348)(0,-12.4615384615385)(0,-12.4013377926421)(0,-12.3411371237458)(0,-12.2809364548495)(0,-12.2207357859532)(0,-12.1605351170569)(0,-12.1003344481605)(0,-12.0401337792642)(0,-11.9799331103679)(0,-11.9197324414716)(0,-11.8595317725753)(0,-11.7993311036789)(0,-11.7391304347826)(0,-11.6789297658863)(0,-11.61872909699)(0,-11.5585284280936)(0,-11.4983277591973)(0,-11.438127090301)(0,-11.3779264214047)(0,-11.3177257525084)(0,-11.257525083612)(0,-11.1973244147157)(0,-11.1371237458194)(0,-11.0769230769231)(0,-11.0167224080268)(0,-10.9565217391304)(0,-10.8963210702341)(0,-10.8361204013378)(0,-10.7759197324415)(0,-10.7157190635452)(0,-10.6555183946488)(0,-10.5953177257525)(0,-10.5351170568562)(0,-10.4749163879599)(0,-10.4147157190635)(0,-10.3545150501672)(0,-10.2943143812709)(0,-10.2341137123746)(0,-10.1739130434783)(0,-10.1137123745819)(0,-10.0535117056856)(0,-9.9933110367893)(0,-9.93311036789298)(0,-9.87290969899666)(0,-9.81270903010033)(0,-9.75250836120401)(0,-9.69230769230769)(0,-9.63210702341137)(0,-9.57190635451505)(0,-9.51170568561873)(0,-9.45150501672241)(0,-9.39130434782609)(0,-9.33110367892977)(0,-9.27090301003344)(0,-9.21070234113712)(0,-9.1505016722408)(0,-9.09030100334448)(0,-9.03010033444816)(0,-8.96989966555184)(0,-8.90969899665552)(0,-8.8494983277592)(0,-8.78929765886288)(0,-8.72909698996656)(0,-8.66889632107023)(0,-8.60869565217391)(0,-8.54849498327759)(0,-8.48829431438127)(0,-8.42809364548495)(0,-8.36789297658863)(0,-8.30769230769231)(0,-8.24749163879599)(0,-8.18729096989967)(0,-8.12709030100334)(0,-8.06688963210702)(0,-8.0066889632107)(0,-7.94648829431438)(0,-7.88628762541806)(0,-7.82608695652174)(0,-7.76588628762542)(0,-7.7056856187291)(0,-7.64548494983278)(0,-7.58528428093645)(0,-7.52508361204013)(0,-7.46488294314381)(0,-7.40468227424749)(0,-7.34448160535117)(0,-7.28428093645485)(0,-7.22408026755853)(0,-7.16387959866221)(0,-7.10367892976589)(0,-7.04347826086956)(0,-6.98327759197324)(0,-6.92307692307692)(0,-6.8628762541806)(0,-6.80267558528428)(0,-6.74247491638796)(0,-6.68227424749164)(0,-6.62207357859532)(0,-6.561872909699)(0,-6.50167224080267)(0,-6.44147157190636)(0,-6.38127090301003)(0,-6.32107023411371)(0,-6.26086956521739)(0,-6.20066889632107)(0,-6.14046822742475)(0,-6.08026755852843)(0,-6.02006688963211)(0,-5.95986622073579)(0,-5.89966555183947)(0,-5.83946488294314)(0,-5.77926421404682)(0,-5.7190635451505)(0,-5.65886287625418)(0,-5.59866220735786)(0,-5.53846153846154)(0,-5.47826086956522)(0,-5.4180602006689)(0,-5.35785953177258)(0,-5.29765886287625)(0,-5.23745819397993)(0,-5.17725752508361)(0,-5.11705685618729)(0,-5.05685618729097)(0,-4.99665551839465)(0,-4.93645484949833)(0,-4.87625418060201)(0,-4.81605351170569)(0,-4.75585284280936)(0,-4.69565217391304)(0,-4.63545150501672)};

\addplot [fill=red!40!yellow,draw=black,forget plot] coordinates{ (1324.76254180602,-8.75919732441472)(1327.23411371237,-8.78929765886288)(1327.23411371237,-8.8494983277592)(1327.23411371237,-8.90969899665552)(1327.23411371237,-8.96989966555184)(1324.76254180602,-9)(1322.29096989967,-9.03010033444816)(1322.29096989967,-9.09030100334448)(1322.29096989967,-9.1505016722408)(1319.81939799331,-9.18060200668896)(1317.34782608696,-9.21070234113712)(1317.34782608696,-9.27090301003344)(1314.8762541806,-9.30100334448161)(1312.40468227425,-9.33110367892977)(1309.93311036789,-9.36120401337793)(1307.46153846154,-9.39130434782609)(1304.98996655518,-9.42140468227425)(1302.51839464883,-9.45150501672241)(1300.04682274247,-9.48160535117057)(1297.57525083612,-9.51170568561873)(1295.10367892977,-9.54180602006689)(1292.63210702341,-9.57190635451505)(1290.16053511706,-9.60200668896321)(1285.21739130435,-9.60200668896321)(1280.27424749164,-9.60200668896321)(1277.80267558528,-9.63210702341137)(1275.33110367893,-9.66220735785953)(1270.38795986622,-9.66220735785953)(1265.44481605351,-9.66220735785953)(1260.5016722408,-9.66220735785953)(1255.55852842809,-9.66220735785953)(1250.61538461538,-9.66220735785953)(1245.67224080268,-9.66220735785953)(1243.20066889632,-9.63210702341137)(1240.72909698997,-9.60200668896321)(1235.78595317726,-9.60200668896321)(1230.84280936455,-9.60200668896321)(1228.37123745819,-9.57190635451505)(1225.89966555184,-9.54180602006689)(1220.95652173913,-9.54180602006689)(1216.01337792642,-9.54180602006689)(1213.54180602007,-9.51170568561873)(1211.07023411371,-9.48160535117057)(1206.127090301,-9.48160535117057)(1201.18394648829,-9.48160535117057)(1196.24080267559,-9.48160535117057)(1193.76923076923,-9.45150501672241)(1191.29765886288,-9.42140468227425)(1186.35451505017,-9.42140468227425)(1181.41137123746,-9.42140468227425)(1176.46822742475,-9.42140468227425)(1173.99665551839,-9.45150501672241)(1171.52508361204,-9.48160535117057)(1166.58193979933,-9.48160535117057)(1161.63879598662,-9.48160535117057)(1159.16722408027,-9.51170568561873)(1156.69565217391,-9.54180602006689)(1151.7525083612,-9.54180602006689)(1146.8093645485,-9.54180602006689)(1144.33779264214,-9.57190635451505)(1141.86622073579,-9.60200668896321)(1136.92307692308,-9.60200668896321)(1134.45150501672,-9.63210702341137)(1131.97993311037,-9.66220735785953)(1129.50836120401,-9.69230769230769)(1127.03678929766,-9.72240802675585)(1122.09364548495,-9.72240802675585)(1119.6220735786,-9.75250836120401)(1117.15050167224,-9.78260869565217)(1112.20735785953,-9.78260869565217)(1109.73578595318,-9.81270903010033)(1107.26421404682,-9.8428093645485)(1102.32107023411,-9.8428093645485)(1099.84949832776,-9.87290969899666)(1097.3779264214,-9.90301003344482)(1092.4347826087,-9.90301003344482)(1087.49163879599,-9.90301003344482)(1085.02006688963,-9.93311036789298)(1082.54849498328,-9.96321070234114)(1077.60535117057,-9.96321070234114)(1072.66220735786,-9.96321070234114)(1067.71906354515,-9.96321070234114)(1062.77591973244,-9.96321070234114)(1060.30434782609,-9.93311036789298)(1057.83277591973,-9.90301003344482)(1052.88963210702,-9.90301003344482)(1047.94648829431,-9.90301003344482)(1045.47491638796,-9.87290969899666)(1043.00334448161,-9.8428093645485)(1038.0602006689,-9.8428093645485)(1035.58862876254,-9.81270903010033)(1033.11705685619,-9.78260869565217)(1028.17391304348,-9.78260869565217)(1025.70234113712,-9.75250836120401)(1023.23076923077,-9.72240802675585)(1020.75919732441,-9.69230769230769)(1018.28762541806,-9.66220735785953)(1013.34448160535,-9.66220735785953)(1010.872909699,-9.63210702341137)(1008.40133779264,-9.60200668896321)(1005.92976588629,-9.57190635451505)(1003.45819397993,-9.54180602006689)(1000.98662207358,-9.51170568561873)(998.515050167224,-9.48160535117057)(993.571906354515,-9.48160535117057)(991.100334448161,-9.45150501672241)(988.628762541806,-9.42140468227425)(986.157190635452,-9.39130434782609)(983.685618729097,-9.36120401337793)(981.214046822742,-9.33110367892977)(978.742474916388,-9.30100334448161)(973.799331103679,-9.30100334448161)(971.327759197324,-9.27090301003344)(968.85618729097,-9.24080267558528)(966.384615384615,-9.21070234113712)(963.913043478261,-9.18060200668896)(958.969899665552,-9.18060200668896)(956.498327759197,-9.1505016722408)(954.026755852843,-9.12040133779264)(949.083612040134,-9.12040133779264)(946.612040133779,-9.09030100334448)(944.140468227425,-9.06020066889632)(939.197324414716,-9.06020066889632)(936.725752508361,-9.03010033444816)(934.254180602007,-9)(929.311036789298,-9)(924.367892976589,-9)(919.42474916388,-9)(914.481605351171,-9)(909.538461538462,-9)(904.595317725752,-9)(899.652173913044,-9)(894.709030100334,-9)(892.23745819398,-9.03010033444816)(889.765886287625,-9.06020066889632)(884.822742474916,-9.06020066889632)(879.879598662207,-9.06020066889632)(874.936454849498,-9.06020066889632)(869.993311036789,-9.06020066889632)(867.521739130435,-9.09030100334448)(865.05016722408,-9.12040133779264)(860.107023411371,-9.12040133779264)(855.163879598662,-9.12040133779264)(850.220735785953,-9.12040133779264)(847.749163879599,-9.09030100334448)(845.277591973244,-9.06020066889632)(840.334448160535,-9.06020066889632)(835.391304347826,-9.06020066889632)(832.919732441472,-9.03010033444816)(830.448160535117,-9)(827.976588628763,-8.96989966555184)(825.505016722408,-8.93979933110368)(820.561872909699,-8.93979933110368)(818.090301003345,-8.90969899665552)(815.61872909699,-8.87959866220736)(813.147157190635,-8.8494983277592)(810.675585284281,-8.81939799331104)(808.204013377926,-8.78929765886288)(808.204013377926,-8.72909698996656)(805.732441471572,-8.69899665551839)(803.260869565217,-8.66889632107023)(803.260869565217,-8.60869565217391)(803.260869565217,-8.54849498327759)(800.789297658863,-8.51839464882943)(798.317725752508,-8.48829431438127)(798.317725752508,-8.42809364548495)(800.789297658863,-8.39799331103679)(803.260869565217,-8.36789297658863)(803.260869565217,-8.30769230769231)(803.260869565217,-8.24749163879599)(805.732441471572,-8.21739130434783)(808.204013377926,-8.18729096989967)(808.204013377926,-8.12709030100334)(810.675585284281,-8.09698996655519)(813.147157190635,-8.06688963210702)(815.61872909699,-8.03678929765886)(818.090301003345,-8.0066889632107)(820.561872909699,-7.97658862876254)(825.505016722408,-7.97658862876254)(827.976588628763,-7.94648829431438)(830.448160535117,-7.91638795986622)(835.391304347826,-7.91638795986622)(840.334448160535,-7.91638795986622)(845.277591973244,-7.91638795986622)(850.220735785953,-7.91638795986622)(855.163879598662,-7.91638795986622)(860.107023411371,-7.91638795986622)(862.578595317726,-7.94648829431438)(865.05016722408,-7.97658862876254)(869.993311036789,-7.97658862876254)(874.936454849498,-7.97658862876254)(877.408026755853,-8.0066889632107)(879.879598662207,-8.03678929765886)(882.351170568562,-8.06688963210702)(884.822742474916,-8.09698996655519)(889.765886287625,-8.09698996655519)(892.23745819398,-8.12709030100334)(894.709030100334,-8.1571906354515)(899.652173913044,-8.1571906354515)(902.123745819398,-8.18729096989967)(904.595317725752,-8.21739130434783)(909.538461538462,-8.21739130434783)(912.010033444816,-8.24749163879599)(914.481605351171,-8.27759197324415)(919.42474916388,-8.27759197324415)(924.367892976589,-8.27759197324415)(926.839464882943,-8.30769230769231)(929.311036789298,-8.33779264214047)(934.254180602007,-8.33779264214047)(939.197324414716,-8.33779264214047)(944.140468227425,-8.33779264214047)(946.612040133779,-8.30769230769231)(949.083612040134,-8.27759197324415)(954.026755852843,-8.27759197324415)(958.969899665552,-8.27759197324415)(961.441471571906,-8.24749163879599)(963.913043478261,-8.21739130434783)(968.85618729097,-8.21739130434783)(971.327759197324,-8.18729096989967)(973.799331103679,-8.1571906354515)(978.742474916388,-8.1571906354515)(981.214046822742,-8.12709030100334)(983.685618729097,-8.09698996655519)(988.628762541806,-8.09698996655519)(991.100334448161,-8.06688963210702)(993.571906354515,-8.03678929765886)(998.515050167224,-8.03678929765886)(1000.98662207358,-8.0066889632107)(1003.45819397993,-7.97658862876254)(1008.40133779264,-7.97658862876254)(1010.872909699,-7.94648829431438)(1013.34448160535,-7.91638795986622)(1018.28762541806,-7.91638795986622)(1020.75919732441,-7.88628762541806)(1023.23076923077,-7.8561872909699)(1028.17391304348,-7.8561872909699)(1033.11705685619,-7.8561872909699)(1038.0602006689,-7.8561872909699)(1043.00334448161,-7.8561872909699)(1047.94648829431,-7.8561872909699)(1052.88963210702,-7.8561872909699)(1057.83277591973,-7.8561872909699)(1062.77591973244,-7.8561872909699)(1067.71906354515,-7.8561872909699)(1070.1906354515,-7.88628762541806)(1072.66220735786,-7.91638795986622)(1077.60535117057,-7.91638795986622)(1080.07692307692,-7.94648829431438)(1082.54849498328,-7.97658862876254)(1087.49163879599,-7.97658862876254)(1089.96321070234,-8.0066889632107)(1092.4347826087,-8.03678929765886)(1097.3779264214,-8.03678929765886)(1099.84949832776,-8.06688963210702)(1102.32107023411,-8.09698996655519)(1107.26421404682,-8.09698996655519)(1109.73578595318,-8.12709030100334)(1112.20735785953,-8.1571906354515)(1117.15050167224,-8.1571906354515)(1119.6220735786,-8.18729096989967)(1122.09364548495,-8.21739130434783)(1127.03678929766,-8.21739130434783)(1129.50836120401,-8.24749163879599)(1131.97993311037,-8.27759197324415)(1136.92307692308,-8.27759197324415)(1139.39464882943,-8.30769230769231)(1141.86622073579,-8.33779264214047)(1146.8093645485,-8.33779264214047)(1149.28093645485,-8.36789297658863)(1151.7525083612,-8.39799331103679)(1156.69565217391,-8.39799331103679)(1161.63879598662,-8.39799331103679)(1166.58193979933,-8.39799331103679)(1169.05351170569,-8.42809364548495)(1171.52508361204,-8.45819397993311)(1176.46822742475,-8.45819397993311)(1181.41137123746,-8.45819397993311)(1186.35451505017,-8.45819397993311)(1188.82608695652,-8.42809364548495)(1191.29765886288,-8.39799331103679)(1196.24080267559,-8.39799331103679)(1201.18394648829,-8.39799331103679)(1206.127090301,-8.39799331103679)(1208.59866220736,-8.36789297658863)(1211.07023411371,-8.33779264214047)(1216.01337792642,-8.33779264214047)(1220.95652173913,-8.33779264214047)(1223.42809364549,-8.30769230769231)(1225.89966555184,-8.27759197324415)(1230.84280936455,-8.27759197324415)(1235.78595317726,-8.27759197324415)(1238.25752508361,-8.24749163879599)(1240.72909698997,-8.21739130434783)(1245.67224080268,-8.21739130434783)(1250.61538461538,-8.21739130434783)(1253.08695652174,-8.18729096989967)(1255.55852842809,-8.1571906354515)(1260.5016722408,-8.1571906354515)(1265.44481605351,-8.1571906354515)(1270.38795986622,-8.1571906354515)(1275.33110367893,-8.1571906354515)(1280.27424749164,-8.1571906354515)(1285.21739130435,-8.1571906354515)(1287.6889632107,-8.18729096989967)(1290.16053511706,-8.21739130434783)(1295.10367892977,-8.21739130434783)(1297.57525083612,-8.24749163879599)(1300.04682274247,-8.27759197324415)(1302.51839464883,-8.30769230769231)(1304.98996655518,-8.33779264214047)(1307.46153846154,-8.36789297658863)(1309.93311036789,-8.39799331103679)(1312.40468227425,-8.42809364548495)(1314.8762541806,-8.45819397993311)(1317.34782608696,-8.48829431438127)(1317.34782608696,-8.54849498327759)(1319.81939799331,-8.57859531772575)(1322.29096989967,-8.60869565217391)(1322.29096989967,-8.66889632107023)(1322.29096989967,-8.72909698996656)(1324.76254180602,-8.75919732441472)};

\addplot [fill=darkgray,draw=black,forget plot] coordinates{ (1478,-17.819397993311)(1478,-17.8795986622074)(1478,-17.9397993311037)(1478,-18)(1473.05685618729,-18)(1468.11371237458,-18)(1463.17056856187,-18)(1458.22742474916,-18)(1463.17056856187,-18)(1468.11371237458,-17.9397993311037)(1473.05685618729,-17.8795986622074)(1478,-17.819397993311)};

\addplot [
color=white,
draw=white,
only marks,
mark=x,
mark options={solid},
mark size=2.0pt,
line width=0.3pt,
forget plot
]
coordinates{
 (10.5571428571429,0)(21.1142857142857,0)(31.6714285714286,0)(42.2285714285714,0)(52.7857142857143,0)(63.3428571428571,0)(73.9,0)(84.4571428571429,0)(95.0142857142857,0)(105.571428571429,0)(116.128571428571,-0.663157894736842)(126.685714285714,-1.32631578947368)(137.242857142857,-1.98947368421053)(147.8,-2.65263157894737)(158.357142857143,-3.31578947368421)(168.914285714286,-3.97894736842105)(179.471428571429,-4.64210526315789)(190.028571428571,-5.30526315789474)(200.585714285714,-5.96842105263158)(211.142857142857,-6.63157894736842)(221.7,-7.76842105263158)(232.257142857143,-8.90526315789474)(242.814285714286,-10.0421052631579)(253.371428571429,-11.1789473684211)(263.928571428571,-12.3157894736842)(274.485714285714,-13.4526315789474)(285.042857142857,-14.5894736842105)(295.6,-15.7263157894737)(306.157142857143,-16.8631578947368)(316.714285714286,-18)(327.271428571429,-17.1473684210526)(337.828571428571,-16.2947368421053)(348.385714285714,-15.4421052631579)(358.942857142857,-14.5894736842105)(369.5,-13.7368421052632)(380.057142857143,-12.8842105263158)(390.614285714286,-12.0315789473684)(401.171428571429,-11.1789473684211)(411.728571428571,-10.3263157894737)(422.285714285714,-9.47368421052632)(432.842857142857,-8.52631578947368)(443.4,-7.57894736842105)(453.957142857143,-6.63157894736842)(464.514285714286,-5.68421052631579)(475.071428571429,-4.73684210526316)(485.628571428571,-3.78947368421053)(496.185714285714,-2.84210526315789)(506.742857142857,-1.89473684210526)(517.3,-0.947368421052632)(527.857142857143,0)(538.414285714286,-0.947368421052632)(548.971428571429,-1.89473684210526)(559.528571428571,-2.84210526315789)(570.085714285714,-3.78947368421053)(580.642857142857,-4.73684210526316)(591.2,-5.68421052631579)(601.757142857143,-6.63157894736842)(612.314285714286,-7.57894736842105)(622.871428571429,-8.52631578947368)(633.428571428571,-9.47368421052632)(643.985714285714,-10.3263157894737)(654.542857142857,-11.1789473684211)(665.1,-12.0315789473684)(675.657142857143,-12.8842105263158)(686.214285714286,-13.7368421052632)(696.771428571429,-14.5894736842105)(707.328571428571,-15.4421052631579)(717.885714285714,-16.2947368421053)(728.442857142857,-17.1473684210526)(739,-18)(749.557142857143,-16.9578947368421)(760.114285714286,-15.9157894736842)(770.671428571429,-14.8736842105263)(781.228571428571,-13.8315789473684)(791.785714285714,-12.7894736842105)(802.342857142857,-11.7473684210526)(812.9,-10.7052631578947)(823.457142857143,-9.66315789473684)(834.014285714286,-8.62105263157895)(844.571428571429,-7.57894736842105)(855.128571428571,-6.82105263157895)(865.685714285714,-6.06315789473684)(876.242857142857,-5.30526315789474)(886.8,-4.54736842105263)(897.357142857143,-3.78947368421053)(907.914285714286,-3.03157894736842)(918.471428571429,-2.27368421052632)(929.028571428571,-1.51578947368421)(939.585714285714,-0.757894736842106)(950.142857142857,0)(960.7,-0.852631578947369)(971.257142857143,-1.70526315789474)(981.814285714286,-2.55789473684211)(992.371428571429,-3.41052631578947)(1002.92857142857,-4.26315789473684)(1013.48571428571,-5.11578947368421)(1024.04285714286,-5.96842105263158)(1034.6,-6.82105263157895)(1045.15714285714,-7.67368421052632)(1055.71428571429,-8.52631578947369)(1066.27142857143,-9.47368421052632)(1076.82857142857,-10.4210526315789)(1087.38571428571,-11.3684210526316)(1097.94285714286,-12.3157894736842)(1108.5,-13.2631578947368)(1119.05714285714,-14.2105263157895)(1129.61428571429,-15.1578947368421)(1140.17142857143,-16.1052631578947)(1150.72857142857,-17.0526315789474)(1161.28571428571,-18)(1171.84285714286,-17.2421052631579)(1182.4,-16.4842105263158)(1192.95714285714,-15.7263157894737)(1203.51428571429,-14.9684210526316)(1214.07142857143,-14.2105263157895)(1224.62857142857,-13.4526315789474)(1235.18571428571,-12.6947368421053)(1245.74285714286,-11.9368421052632)(1256.3,-11.1789473684211)(1266.85714285714,-10.4210526315789)(1277.41428571429,-9.37894736842105)(1287.97142857143,-8.33684210526316)(1298.52857142857,-7.29473684210526)(1309.08571428571,-6.25263157894737)(1319.64285714286,-5.21052631578947)(1330.2,-4.16842105263158)(1340.75714285714,-3.12631578947368)(1351.31428571429,-2.08421052631579)(1361.87142857143,-1.04210526315789)(1372.42857142857,0)(1382.98571428571,-1.42105263157895)(1393.54285714286,-2.84210526315789)(1404.1,-4.26315789473684)(1414.65714285714,-5.68421052631579)(1425.21428571429,-7.10526315789474)(1435.77142857143,-8.52631578947369)(1446.32857142857,-9.94736842105263)(1456.88571428571,-11.3684210526316)(1467.44285714286,-12.7894736842105)(1478,-14.2105263157895) 
};

\addplot [
color=red,
solid,
line width=1.0pt,
forget plot
]
coordinates{
 (1478,-14.2105263157895)(1372.42857142857,-9.47368421052632)(1266.85714285714,-6.63157894736842)(1161.28571428571,-10.4210526315789)(1055.71428571429,-10.4210526315789)(950.142857142857,-10.4210526315789)(844.571428571429,-6.63157894736842)(739,-9.47368421052632)(633.428571428571,-6.63157894736842)(527.857142857143,-8.52631578947369)(422.285714285714,-6.63157894736842) 
};

\addplot [
mark size=0.8pt,
only marks,
mark=*,
mark options={solid,fill=black,draw=black},
forget plot
]
coordinates{
 (0,0)(0,-0.947368421052632)(0,-1.89473684210526)(0,-2.84210526315789)(0,-3.78947368421053)(0,-4.73684210526316)(0,-5.68421052631579)(0,-6.63157894736842)(0,-7.57894736842105)(0,-8.52631578947369)(0,-9.47368421052632)(0,-10.4210526315789)(0,-11.3684210526316)(0,-12.3157894736842)(0,-13.2631578947368)(0,-14.2105263157895)(0,-15.1578947368421)(0,-16.1052631578947)(0,-17.0526315789474)(0,-18)(105.571428571429,0)(105.571428571429,-0.947368421052632)(105.571428571429,-1.89473684210526)(105.571428571429,-2.84210526315789)(105.571428571429,-3.78947368421053)(105.571428571429,-4.73684210526316)(105.571428571429,-5.68421052631579)(105.571428571429,-6.63157894736842)(105.571428571429,-7.57894736842105)(105.571428571429,-8.52631578947369)(105.571428571429,-9.47368421052632)(105.571428571429,-10.4210526315789)(105.571428571429,-11.3684210526316)(105.571428571429,-12.3157894736842)(105.571428571429,-13.2631578947368)(105.571428571429,-14.2105263157895)(105.571428571429,-15.1578947368421)(105.571428571429,-16.1052631578947)(105.571428571429,-17.0526315789474)(105.571428571429,-18)(211.142857142857,0)(211.142857142857,-0.947368421052632)(211.142857142857,-1.89473684210526)(211.142857142857,-2.84210526315789)(211.142857142857,-3.78947368421053)(211.142857142857,-4.73684210526316)(211.142857142857,-5.68421052631579)(211.142857142857,-6.63157894736842)(211.142857142857,-7.57894736842105)(211.142857142857,-8.52631578947369)(211.142857142857,-9.47368421052632)(211.142857142857,-10.4210526315789)(211.142857142857,-11.3684210526316)(211.142857142857,-12.3157894736842)(211.142857142857,-13.2631578947368)(211.142857142857,-14.2105263157895)(211.142857142857,-15.1578947368421)(211.142857142857,-16.1052631578947)(211.142857142857,-17.0526315789474)(211.142857142857,-18)(316.714285714286,0)(316.714285714286,-0.947368421052632)(316.714285714286,-1.89473684210526)(316.714285714286,-2.84210526315789)(316.714285714286,-3.78947368421053)(316.714285714286,-4.73684210526316)(316.714285714286,-5.68421052631579)(316.714285714286,-6.63157894736842)(316.714285714286,-7.57894736842105)(316.714285714286,-8.52631578947369)(316.714285714286,-9.47368421052632)(316.714285714286,-10.4210526315789)(316.714285714286,-11.3684210526316)(316.714285714286,-12.3157894736842)(316.714285714286,-13.2631578947368)(316.714285714286,-14.2105263157895)(316.714285714286,-15.1578947368421)(316.714285714286,-16.1052631578947)(316.714285714286,-17.0526315789474)(316.714285714286,-18)(422.285714285714,0)(422.285714285714,-0.947368421052632)(422.285714285714,-1.89473684210526)(422.285714285714,-2.84210526315789)(422.285714285714,-3.78947368421053)(422.285714285714,-4.73684210526316)(422.285714285714,-5.68421052631579)(422.285714285714,-6.63157894736842)(422.285714285714,-7.57894736842105)(422.285714285714,-8.52631578947369)(422.285714285714,-9.47368421052632)(422.285714285714,-10.4210526315789)(422.285714285714,-11.3684210526316)(422.285714285714,-12.3157894736842)(422.285714285714,-13.2631578947368)(422.285714285714,-14.2105263157895)(422.285714285714,-15.1578947368421)(422.285714285714,-16.1052631578947)(422.285714285714,-17.0526315789474)(422.285714285714,-18)(527.857142857143,0)(527.857142857143,-0.947368421052632)(527.857142857143,-1.89473684210526)(527.857142857143,-2.84210526315789)(527.857142857143,-3.78947368421053)(527.857142857143,-4.73684210526316)(527.857142857143,-5.68421052631579)(527.857142857143,-6.63157894736842)(527.857142857143,-7.57894736842105)(527.857142857143,-8.52631578947369)(527.857142857143,-9.47368421052632)(527.857142857143,-10.4210526315789)(527.857142857143,-11.3684210526316)(527.857142857143,-12.3157894736842)(527.857142857143,-13.2631578947368)(527.857142857143,-14.2105263157895)(527.857142857143,-15.1578947368421)(527.857142857143,-16.1052631578947)(527.857142857143,-17.0526315789474)(527.857142857143,-18)(633.428571428571,0)(633.428571428571,-0.947368421052632)(633.428571428571,-1.89473684210526)(633.428571428571,-2.84210526315789)(633.428571428571,-3.78947368421053)(633.428571428571,-4.73684210526316)(633.428571428571,-5.68421052631579)(633.428571428571,-6.63157894736842)(633.428571428571,-7.57894736842105)(633.428571428571,-8.52631578947369)(633.428571428571,-9.47368421052632)(633.428571428571,-10.4210526315789)(633.428571428571,-11.3684210526316)(633.428571428571,-12.3157894736842)(633.428571428571,-13.2631578947368)(633.428571428571,-14.2105263157895)(633.428571428571,-15.1578947368421)(633.428571428571,-16.1052631578947)(633.428571428571,-17.0526315789474)(633.428571428571,-18)(739,0)(739,-0.947368421052632)(739,-1.89473684210526)(739,-2.84210526315789)(739,-3.78947368421053)(739,-4.73684210526316)(739,-5.68421052631579)(739,-6.63157894736842)(739,-7.57894736842105)(739,-8.52631578947369)(739,-9.47368421052632)(739,-10.4210526315789)(739,-11.3684210526316)(739,-12.3157894736842)(739,-13.2631578947368)(739,-14.2105263157895)(739,-15.1578947368421)(739,-16.1052631578947)(739,-17.0526315789474)(739,-18)(844.571428571429,0)(844.571428571429,-0.947368421052632)(844.571428571429,-1.89473684210526)(844.571428571429,-2.84210526315789)(844.571428571429,-3.78947368421053)(844.571428571429,-4.73684210526316)(844.571428571429,-5.68421052631579)(844.571428571429,-6.63157894736842)(844.571428571429,-7.57894736842105)(844.571428571429,-8.52631578947369)(844.571428571429,-9.47368421052632)(844.571428571429,-10.4210526315789)(844.571428571429,-11.3684210526316)(844.571428571429,-12.3157894736842)(844.571428571429,-13.2631578947368)(844.571428571429,-14.2105263157895)(844.571428571429,-15.1578947368421)(844.571428571429,-16.1052631578947)(844.571428571429,-17.0526315789474)(844.571428571429,-18)(950.142857142857,0)(950.142857142857,-0.947368421052632)(950.142857142857,-1.89473684210526)(950.142857142857,-2.84210526315789)(950.142857142857,-3.78947368421053)(950.142857142857,-4.73684210526316)(950.142857142857,-5.68421052631579)(950.142857142857,-6.63157894736842)(950.142857142857,-7.57894736842105)(950.142857142857,-8.52631578947369)(950.142857142857,-9.47368421052632)(950.142857142857,-10.4210526315789)(950.142857142857,-11.3684210526316)(950.142857142857,-12.3157894736842)(950.142857142857,-13.2631578947368)(950.142857142857,-14.2105263157895)(950.142857142857,-15.1578947368421)(950.142857142857,-16.1052631578947)(950.142857142857,-17.0526315789474)(950.142857142857,-18)(1055.71428571429,0)(1055.71428571429,-0.947368421052632)(1055.71428571429,-1.89473684210526)(1055.71428571429,-2.84210526315789)(1055.71428571429,-3.78947368421053)(1055.71428571429,-4.73684210526316)(1055.71428571429,-5.68421052631579)(1055.71428571429,-6.63157894736842)(1055.71428571429,-7.57894736842105)(1055.71428571429,-8.52631578947369)(1055.71428571429,-9.47368421052632)(1055.71428571429,-10.4210526315789)(1055.71428571429,-11.3684210526316)(1055.71428571429,-12.3157894736842)(1055.71428571429,-13.2631578947368)(1055.71428571429,-14.2105263157895)(1055.71428571429,-15.1578947368421)(1055.71428571429,-16.1052631578947)(1055.71428571429,-17.0526315789474)(1055.71428571429,-18)(1161.28571428571,0)(1161.28571428571,-0.947368421052632)(1161.28571428571,-1.89473684210526)(1161.28571428571,-2.84210526315789)(1161.28571428571,-3.78947368421053)(1161.28571428571,-4.73684210526316)(1161.28571428571,-5.68421052631579)(1161.28571428571,-6.63157894736842)(1161.28571428571,-7.57894736842105)(1161.28571428571,-8.52631578947369)(1161.28571428571,-9.47368421052632)(1161.28571428571,-10.4210526315789)(1161.28571428571,-11.3684210526316)(1161.28571428571,-12.3157894736842)(1161.28571428571,-13.2631578947368)(1161.28571428571,-14.2105263157895)(1161.28571428571,-15.1578947368421)(1161.28571428571,-16.1052631578947)(1161.28571428571,-17.0526315789474)(1161.28571428571,-18)(1266.85714285714,0)(1266.85714285714,-0.947368421052632)(1266.85714285714,-1.89473684210526)(1266.85714285714,-2.84210526315789)(1266.85714285714,-3.78947368421053)(1266.85714285714,-4.73684210526316)(1266.85714285714,-5.68421052631579)(1266.85714285714,-6.63157894736842)(1266.85714285714,-7.57894736842105)(1266.85714285714,-8.52631578947369)(1266.85714285714,-9.47368421052632)(1266.85714285714,-10.4210526315789)(1266.85714285714,-11.3684210526316)(1266.85714285714,-12.3157894736842)(1266.85714285714,-13.2631578947368)(1266.85714285714,-14.2105263157895)(1266.85714285714,-15.1578947368421)(1266.85714285714,-16.1052631578947)(1266.85714285714,-17.0526315789474)(1266.85714285714,-18)(1372.42857142857,0)(1372.42857142857,-0.947368421052632)(1372.42857142857,-1.89473684210526)(1372.42857142857,-2.84210526315789)(1372.42857142857,-3.78947368421053)(1372.42857142857,-4.73684210526316)(1372.42857142857,-5.68421052631579)(1372.42857142857,-6.63157894736842)(1372.42857142857,-7.57894736842105)(1372.42857142857,-8.52631578947369)(1372.42857142857,-9.47368421052632)(1372.42857142857,-10.4210526315789)(1372.42857142857,-11.3684210526316)(1372.42857142857,-12.3157894736842)(1372.42857142857,-13.2631578947368)(1372.42857142857,-14.2105263157895)(1372.42857142857,-15.1578947368421)(1372.42857142857,-16.1052631578947)(1372.42857142857,-17.0526315789474)(1372.42857142857,-18)(1478,0)(1478,-0.947368421052632)(1478,-1.89473684210526)(1478,-2.84210526315789)(1478,-3.78947368421053)(1478,-4.73684210526316)(1478,-5.68421052631579)(1478,-6.63157894736842)(1478,-7.57894736842105)(1478,-8.52631578947369)(1478,-9.47368421052632)(1478,-10.4210526315789)(1478,-11.3684210526316)(1478,-12.3157894736842)(1478,-13.2631578947368)(1478,-14.2105263157895)(1478,-15.1578947368421)(1478,-16.1052631578947)(1478,-17.0526315789474)(1478,-18)(1372.42857142857,0)(1372.42857142857,-0.947368421052632)(1372.42857142857,-1.89473684210526)(1372.42857142857,-2.84210526315789)(1372.42857142857,-3.78947368421053)(1372.42857142857,-4.73684210526316)(1372.42857142857,-5.68421052631579)(1372.42857142857,-6.63157894736842)(1372.42857142857,-7.57894736842105)(1372.42857142857,-8.52631578947369)(1372.42857142857,-9.47368421052632)(1372.42857142857,-10.4210526315789)(1372.42857142857,-11.3684210526316)(1372.42857142857,-12.3157894736842)(1372.42857142857,-13.2631578947368)(1372.42857142857,-14.2105263157895)(1372.42857142857,-15.1578947368421)(1372.42857142857,-16.1052631578947)(1372.42857142857,-17.0526315789474)(1372.42857142857,-18)(1266.85714285714,0)(1266.85714285714,-0.947368421052632)(1266.85714285714,-1.89473684210526)(1266.85714285714,-2.84210526315789)(1266.85714285714,-3.78947368421053)(1266.85714285714,-4.73684210526316)(1266.85714285714,-5.68421052631579)(1266.85714285714,-6.63157894736842)(1266.85714285714,-7.57894736842105)(1266.85714285714,-8.52631578947369)(1266.85714285714,-9.47368421052632)(1266.85714285714,-10.4210526315789)(1266.85714285714,-11.3684210526316)(1266.85714285714,-12.3157894736842)(1266.85714285714,-13.2631578947368)(1266.85714285714,-14.2105263157895)(1266.85714285714,-15.1578947368421)(1266.85714285714,-16.1052631578947)(1266.85714285714,-17.0526315789474)(1266.85714285714,-18)(1161.28571428571,0)(1161.28571428571,-0.947368421052632)(1161.28571428571,-1.89473684210526)(1161.28571428571,-2.84210526315789)(1161.28571428571,-3.78947368421053)(1161.28571428571,-4.73684210526316)(1161.28571428571,-5.68421052631579)(1161.28571428571,-6.63157894736842)(1161.28571428571,-7.57894736842105)(1161.28571428571,-8.52631578947369)(1161.28571428571,-9.47368421052632)(1161.28571428571,-10.4210526315789)(1161.28571428571,-11.3684210526316)(1161.28571428571,-12.3157894736842)(1161.28571428571,-13.2631578947368)(1161.28571428571,-14.2105263157895)(1161.28571428571,-15.1578947368421)(1161.28571428571,-16.1052631578947)(1161.28571428571,-17.0526315789474)(1161.28571428571,-18)(1055.71428571429,0)(1055.71428571429,-0.947368421052632)(1055.71428571429,-1.89473684210526)(1055.71428571429,-2.84210526315789)(1055.71428571429,-3.78947368421053)(1055.71428571429,-4.73684210526316)(1055.71428571429,-5.68421052631579)(1055.71428571429,-6.63157894736842)(1055.71428571429,-7.57894736842105)(1055.71428571429,-8.52631578947369)(1055.71428571429,-9.47368421052632)(1055.71428571429,-10.4210526315789)(1055.71428571429,-11.3684210526316)(1055.71428571429,-12.3157894736842)(1055.71428571429,-13.2631578947368)(1055.71428571429,-14.2105263157895)(1055.71428571429,-15.1578947368421)(1055.71428571429,-16.1052631578947)(1055.71428571429,-17.0526315789474)(1055.71428571429,-18)(950.142857142857,0)(950.142857142857,-0.947368421052632)(950.142857142857,-1.89473684210526)(950.142857142857,-2.84210526315789)(950.142857142857,-3.78947368421053)(950.142857142857,-4.73684210526316)(950.142857142857,-5.68421052631579)(950.142857142857,-6.63157894736842)(950.142857142857,-7.57894736842105)(950.142857142857,-8.52631578947369)(950.142857142857,-9.47368421052632)(950.142857142857,-10.4210526315789)(950.142857142857,-11.3684210526316)(950.142857142857,-12.3157894736842)(950.142857142857,-13.2631578947368)(950.142857142857,-14.2105263157895)(950.142857142857,-15.1578947368421)(950.142857142857,-16.1052631578947)(950.142857142857,-17.0526315789474)(950.142857142857,-18)(844.571428571429,0)(844.571428571429,-0.947368421052632)(844.571428571429,-1.89473684210526)(844.571428571429,-2.84210526315789)(844.571428571429,-3.78947368421053)(844.571428571429,-4.73684210526316)(844.571428571429,-5.68421052631579)(844.571428571429,-6.63157894736842)(844.571428571429,-7.57894736842105)(844.571428571429,-8.52631578947369)(844.571428571429,-9.47368421052632)(844.571428571429,-10.4210526315789)(844.571428571429,-11.3684210526316)(844.571428571429,-12.3157894736842)(844.571428571429,-13.2631578947368)(844.571428571429,-14.2105263157895)(844.571428571429,-15.1578947368421)(844.571428571429,-16.1052631578947)(844.571428571429,-17.0526315789474)(844.571428571429,-18)(739,0)(739,-0.947368421052632)(739,-1.89473684210526)(739,-2.84210526315789)(739,-3.78947368421053)(739,-4.73684210526316)(739,-5.68421052631579)(739,-6.63157894736842)(739,-7.57894736842105)(739,-8.52631578947369)(739,-9.47368421052632)(739,-10.4210526315789)(739,-11.3684210526316)(739,-12.3157894736842)(739,-13.2631578947368)(739,-14.2105263157895)(739,-15.1578947368421)(739,-16.1052631578947)(739,-17.0526315789474)(739,-18)(633.428571428571,0)(633.428571428571,-0.947368421052632)(633.428571428571,-1.89473684210526)(633.428571428571,-2.84210526315789)(633.428571428571,-3.78947368421053)(633.428571428571,-4.73684210526316)(633.428571428571,-5.68421052631579)(633.428571428571,-6.63157894736842)(633.428571428571,-7.57894736842105)(633.428571428571,-8.52631578947369)(633.428571428571,-9.47368421052632)(633.428571428571,-10.4210526315789)(633.428571428571,-11.3684210526316)(633.428571428571,-12.3157894736842)(633.428571428571,-13.2631578947368)(633.428571428571,-14.2105263157895)(633.428571428571,-15.1578947368421)(633.428571428571,-16.1052631578947)(633.428571428571,-17.0526315789474)(633.428571428571,-18)(527.857142857143,0)(527.857142857143,-0.947368421052632)(527.857142857143,-1.89473684210526)(527.857142857143,-2.84210526315789)(527.857142857143,-3.78947368421053)(527.857142857143,-4.73684210526316)(527.857142857143,-5.68421052631579)(527.857142857143,-6.63157894736842)(527.857142857143,-7.57894736842105)(527.857142857143,-8.52631578947369)(527.857142857143,-9.47368421052632)(527.857142857143,-10.4210526315789)(527.857142857143,-11.3684210526316)(527.857142857143,-12.3157894736842)(527.857142857143,-13.2631578947368)(527.857142857143,-14.2105263157895)(527.857142857143,-15.1578947368421)(527.857142857143,-16.1052631578947)(527.857142857143,-17.0526315789474)(527.857142857143,-18)(422.285714285714,0)(422.285714285714,-0.947368421052632)(422.285714285714,-1.89473684210526)(422.285714285714,-2.84210526315789)(422.285714285714,-3.78947368421053)(422.285714285714,-4.73684210526316)(422.285714285714,-5.68421052631579)(422.285714285714,-6.63157894736842)(422.285714285714,-7.57894736842105)(422.285714285714,-8.52631578947369)(422.285714285714,-9.47368421052632)(422.285714285714,-10.4210526315789)(422.285714285714,-11.3684210526316)(422.285714285714,-12.3157894736842)(422.285714285714,-13.2631578947368)(422.285714285714,-14.2105263157895)(422.285714285714,-15.1578947368421)(422.285714285714,-16.1052631578947)(422.285714285714,-17.0526315789474)(422.285714285714,-18)(316.714285714286,0)(316.714285714286,-0.947368421052632)(316.714285714286,-1.89473684210526)(316.714285714286,-2.84210526315789)(316.714285714286,-3.78947368421053)(316.714285714286,-4.73684210526316)(316.714285714286,-5.68421052631579)(316.714285714286,-6.63157894736842)(316.714285714286,-7.57894736842105)(316.714285714286,-8.52631578947369)(316.714285714286,-9.47368421052632)(316.714285714286,-10.4210526315789)(316.714285714286,-11.3684210526316)(316.714285714286,-12.3157894736842)(316.714285714286,-13.2631578947368)(316.714285714286,-14.2105263157895)(316.714285714286,-15.1578947368421)(316.714285714286,-16.1052631578947)(316.714285714286,-17.0526315789474)(316.714285714286,-18)(211.142857142857,0)(211.142857142857,-0.947368421052632)(211.142857142857,-1.89473684210526)(211.142857142857,-2.84210526315789)(211.142857142857,-3.78947368421053)(211.142857142857,-4.73684210526316)(211.142857142857,-5.68421052631579)(211.142857142857,-6.63157894736842)(211.142857142857,-7.57894736842105)(211.142857142857,-8.52631578947369)(211.142857142857,-9.47368421052632)(211.142857142857,-10.4210526315789)(211.142857142857,-11.3684210526316)(211.142857142857,-12.3157894736842)(211.142857142857,-13.2631578947368)(211.142857142857,-14.2105263157895)(211.142857142857,-15.1578947368421)(211.142857142857,-16.1052631578947)(211.142857142857,-17.0526315789474)(211.142857142857,-18)(105.571428571429,0)(105.571428571429,-0.947368421052632)(105.571428571429,-1.89473684210526)(105.571428571429,-2.84210526315789)(105.571428571429,-3.78947368421053)(105.571428571429,-4.73684210526316)(105.571428571429,-5.68421052631579)(105.571428571429,-6.63157894736842)(105.571428571429,-7.57894736842105)(105.571428571429,-8.52631578947369)(105.571428571429,-9.47368421052632)(105.571428571429,-10.4210526315789)(105.571428571429,-11.3684210526316)(105.571428571429,-12.3157894736842)(105.571428571429,-13.2631578947368)(105.571428571429,-14.2105263157895)(105.571428571429,-15.1578947368421)(105.571428571429,-16.1052631578947)(105.571428571429,-17.0526315789474)(105.571428571429,-18) 
};

\addplot [
color=blue,
mark size=3.5pt,
only marks,
mark=o,
mark options={solid,draw=lime!80!black},
line width=2.7pt,
forget plot
]
coordinates{
 (1478,-14.2105263157895)
};

\end{axis}
\end{tikzpicture}%

%% This file was created by matlab2tikz v0.2.3.
% Copyright (c) 2008--2012, Nico Schlömer <nico.schloemer@gmail.com>
% All rights reserved.
% 
% 
% 
\definecolor{locol}{rgb}{0.26, 0.45, 0.65}

\begin{tikzpicture}

\begin{axis}[%
tick label style={font=\tiny},
label style={font=\tiny},
xlabel shift={-10pt},
ylabel shift={-17pt},
legend style={font=\tiny},
view={0}{90},
width=\figurewidth,
height=\figureheight,
scale only axis,
xmin=0, xmax=1478,
xtick={0, 400, 1000, 1400},
xlabel={Length (m)},
ymin=-18, ymax=0,
ytick={0, -4, -14, -18},
ylabel={Depth (m)},
name=plot1,
axis lines*=box,
axis line style={draw=none},
tickwidth=0.0cm,
clip=false
]

\addplot [fill=locol,draw=black,forget plot] coordinates{ (0,0)(4.94314381270903,0)(9.88628762541806,0)(14.8294314381271,0)(19.7725752508361,0)(24.7157190635452,0)(29.6588628762542,0)(34.6020066889632,0)(39.5451505016722,0)(44.4882943143813,0)(49.4314381270903,0)(54.3745819397993,0)(59.3177257525084,0)(64.2608695652174,0)(69.2040133779264,0)(74.1471571906355,0)(79.0903010033445,0)(84.0334448160535,0)(88.9765886287625,0)(93.9197324414716,0)(98.8628762541806,0)(103.80602006689,0)(108.749163879599,0)(113.692307692308,0)(118.635451505017,0)(123.578595317726,0)(128.521739130435,0)(133.464882943144,0)(138.408026755853,0)(143.351170568562,0)(148.294314381271,0)(153.23745819398,0)(158.180602006689,0)(163.123745819398,0)(168.066889632107,0)(173.010033444816,0)(177.953177257525,0)(182.896321070234,0)(187.839464882943,0)(192.782608695652,0)(197.725752508361,0)(202.66889632107,0)(207.612040133779,0)(212.555183946488,0)(217.498327759197,0)(222.441471571906,0)(227.384615384615,0)(232.327759197324,0)(237.270903010033,0)(242.214046822742,0)(247.157190635452,0)(252.100334448161,0)(257.04347826087,0)(261.986622073579,0)(266.929765886288,0)(271.872909698997,0)(276.816053511706,0)(281.759197324415,0)(286.702341137124,0)(291.645484949833,0)(296.588628762542,0)(301.531772575251,0)(306.47491638796,0)(311.418060200669,0)(316.361204013378,0)(321.304347826087,0)(326.247491638796,0)(331.190635451505,0)(336.133779264214,0)(341.076923076923,0)(346.020066889632,0)(350.963210702341,0)(355.90635451505,0)(360.849498327759,0)(365.792642140468,0)(370.735785953177,0)(375.678929765886,0)(380.622073578595,0)(385.565217391304,0)(390.508361204013,0)(395.451505016722,0)(400.394648829431,0)(405.33779264214,0)(410.28093645485,0)(415.224080267559,0)(420.167224080268,0)(425.110367892977,0)(430.053511705686,0)(434.996655518395,0)(439.939799331104,0)(444.882943143813,0)(449.826086956522,0)(454.769230769231,0)(459.71237458194,0)(464.655518394649,0)(469.598662207358,0)(474.541806020067,0)(479.484949832776,0)(484.428093645485,0)(489.371237458194,0)(494.314381270903,0)(499.257525083612,0)(504.200668896321,0)(509.14381270903,0)(514.086956521739,0)(519.030100334448,0)(523.973244147157,0)(528.916387959866,0)(533.859531772575,0)(538.802675585284,0)(543.745819397993,0)(548.688963210702,0)(553.632107023411,0)(558.57525083612,0)(563.518394648829,0)(568.461538461538,0)(573.404682274248,0)(578.347826086957,0)(583.290969899666,0)(588.234113712375,0)(593.177257525084,0)(598.120401337793,0)(603.063545150502,0)(608.006688963211,0)(612.94983277592,0)(617.892976588629,0)(622.836120401338,0)(627.779264214047,0)(632.722408026756,0)(637.665551839465,0)(642.608695652174,0)(647.551839464883,0)(652.494983277592,0)(657.438127090301,0)(662.38127090301,0)(667.324414715719,0)(672.267558528428,0)(677.210702341137,0)(682.153846153846,0)(687.096989966555,0)(692.040133779264,0)(696.983277591973,0)(701.926421404682,0)(706.869565217391,0)(711.8127090301,0)(716.755852842809,0)(721.698996655518,0)(726.642140468227,0)(731.585284280936,0)(736.528428093646,0)(741.471571906354,0)(746.414715719064,0)(751.357859531773,0)(756.301003344482,0)(761.244147157191,0)(766.1872909699,0)(771.130434782609,0)(776.073578595318,0)(781.016722408027,0)(785.959866220736,0)(790.903010033445,0)(795.846153846154,0)(800.789297658863,0)(805.732441471572,0)(810.675585284281,0)(815.61872909699,0)(820.561872909699,0)(825.505016722408,0)(830.448160535117,0)(835.391304347826,0)(840.334448160535,0)(845.277591973244,0)(850.220735785953,0)(855.163879598662,0)(860.107023411371,0)(865.05016722408,0)(869.993311036789,0)(874.936454849498,0)(879.879598662207,0)(884.822742474916,0)(889.765886287625,0)(894.709030100334,0)(899.652173913044,0)(904.595317725752,0)(909.538461538462,0)(914.481605351171,0)(919.42474916388,0)(924.367892976589,0)(929.311036789298,0)(934.254180602007,0)(939.197324414716,0)(944.140468227425,0)(949.083612040134,0)(954.026755852843,0)(958.969899665552,0)(963.913043478261,0)(968.85618729097,0)(973.799331103679,0)(978.742474916388,0)(983.685618729097,0)(988.628762541806,0)(993.571906354515,0)(998.515050167224,0)(1003.45819397993,0)(1008.40133779264,0)(1013.34448160535,0)(1018.28762541806,0)(1023.23076923077,0)(1028.17391304348,0)(1033.11705685619,0)(1038.0602006689,0)(1043.00334448161,0)(1047.94648829431,0)(1052.88963210702,0)(1057.83277591973,0)(1062.77591973244,0)(1067.71906354515,0)(1072.66220735786,0)(1077.60535117057,0)(1082.54849498328,0)(1087.49163879599,0)(1092.4347826087,0)(1097.3779264214,0)(1102.32107023411,0)(1107.26421404682,0)(1112.20735785953,0)(1117.15050167224,0)(1122.09364548495,0)(1127.03678929766,0)(1131.97993311037,0)(1136.92307692308,0)(1141.86622073579,0)(1146.8093645485,0)(1151.7525083612,0)(1156.69565217391,0)(1161.63879598662,0)(1166.58193979933,0)(1171.52508361204,0)(1176.46822742475,0)(1181.41137123746,0)(1186.35451505017,0)(1191.29765886288,0)(1196.24080267559,0)(1201.18394648829,0)(1206.127090301,0)(1211.07023411371,0)(1216.01337792642,0)(1220.95652173913,0)(1225.89966555184,0)(1230.84280936455,0)(1235.78595317726,0)(1240.72909698997,0)(1245.67224080268,0)(1250.61538461538,0)(1255.55852842809,0)(1260.5016722408,0)(1265.44481605351,0)(1270.38795986622,0)(1275.33110367893,0)(1280.27424749164,0)(1285.21739130435,0)(1290.16053511706,0)(1295.10367892977,0)(1300.04682274247,0)(1304.98996655518,0)(1309.93311036789,0)(1314.8762541806,0)(1319.81939799331,0)(1324.76254180602,0)(1329.70568561873,0)(1334.64882943144,0)(1339.59197324415,0)(1344.53511705686,0)(1349.47826086957,0)(1354.42140468227,0)(1359.36454849498,0)(1364.30769230769,0)(1369.2508361204,0)(1374.19397993311,0)(1379.13712374582,0)(1384.08026755853,0)(1389.02341137124,0)(1393.96655518395,0)(1398.90969899666,0)(1403.85284280936,0)(1408.79598662207,0)(1413.73913043478,0)(1418.68227424749,0)(1423.6254180602,0)(1428.56856187291,0)(1433.51170568562,0)(1438.45484949833,0)(1443.39799331104,0)(1448.34113712375,0)(1453.28428093645,0)(1458.22742474916,0)(1463.17056856187,0)(1468.11371237458,0)(1473.05685618729,0)(1478,0)(1478,-0.0602006688963215)(1478,-0.120401337792643)(1478,-0.180602006688964)(1478,-0.240802675585286)(1478,-0.301003344481604)(1478,-0.361204013377925)(1478,-0.421404682274247)(1478,-0.481605351170568)(1478,-0.54180602006689)(1478,-0.602006688963211)(1478,-0.662207357859533)(1478,-0.722408026755854)(1478,-0.782608695652176)(1478,-0.842809364548494)(1478,-0.903010033444815)(1478,-0.963210702341136)(1478,-1.02341137123746)(1478,-1.08361204013378)(1478,-1.1438127090301)(1478,-1.20401337792642)(1478,-1.26421404682274)(1478,-1.32441471571906)(1478,-1.38461538461538)(1478,-1.4448160535117)(1478,-1.50501672240803)(1478,-1.56521739130435)(1478,-1.62541806020067)(1478,-1.68561872909699)(1478,-1.74581939799331)(1478,-1.80602006688963)(1478,-1.86622073578595)(1478,-1.92642140468227)(1478,-1.98662207357859)(1478,-2.04682274247492)(1478,-2.10702341137124)(1478,-2.16722408026756)(1478,-2.22742474916388)(1478,-2.2876254180602)(1478,-2.34782608695652)(1478,-2.40802675585284)(1478,-2.46822742474916)(1478,-2.52842809364548)(1478,-2.58862876254181)(1478,-2.64882943143813)(1478,-2.70903010033445)(1478,-2.76923076923077)(1478,-2.82943143812709)(1478,-2.88963210702341)(1478,-2.94983277591973)(1478,-3.01003344481605)(1478,-3.07023411371237)(1478,-3.1304347826087)(1478,-3.19063545150502)(1478,-3.25083612040134)(1478,-3.31103678929766)(1478,-3.37123745819398)(1478,-3.4314381270903)(1478,-3.49163879598662)(1478,-3.55183946488294)(1478,-3.61204013377926)(1478,-3.67224080267559)(1478,-3.73244147157191)(1478,-3.79264214046823)(1478,-3.85284280936455)(1478,-3.91304347826087)(1478,-3.97324414715719)(1478,-4.03344481605351)(1478,-4.09364548494983)(1478,-4.15384615384615)(1478,-4.21404682274247)(1478,-4.2742474916388)(1478,-4.33444816053512)(1478,-4.39464882943144)(1478,-4.45484949832776)(1478,-4.51505016722408)(1478,-4.5752508361204)(1478,-4.63545150501672)(1478,-4.69565217391304)(1478,-4.75585284280936)(1478,-4.81605351170569)(1478,-4.87625418060201)(1478,-4.93645484949833)(1478,-4.99665551839465)(1478,-5.05685618729097)(1478,-5.11705685618729)(1478,-5.17725752508361)(1478,-5.23745819397993)(1478,-5.29765886287625)(1478,-5.35785953177258)(1478,-5.4180602006689)(1478,-5.47826086956522)(1478,-5.53846153846154)(1478,-5.59866220735786)(1478,-5.65886287625418)(1478,-5.7190635451505)(1478,-5.77926421404682)(1478,-5.83946488294314)(1478,-5.89966555183947)(1478,-5.95986622073579)(1478,-6.02006688963211)(1478,-6.08026755852843)(1478,-6.14046822742475)(1478,-6.20066889632107)(1478,-6.26086956521739)(1478,-6.32107023411371)(1478,-6.38127090301003)(1478,-6.44147157190636)(1478,-6.50167224080267)(1478,-6.561872909699)(1478,-6.62207357859532)(1478,-6.68227424749164)(1478,-6.74247491638796)(1478,-6.80267558528428)(1478,-6.8628762541806)(1478,-6.92307692307692)(1478,-6.98327759197324)(1478,-7.04347826086956)(1478,-7.10367892976589)(1478,-7.16387959866221)(1478,-7.22408026755853)(1478,-7.28428093645485)(1478,-7.34448160535117)(1478,-7.40468227424749)(1478,-7.46488294314381)(1478,-7.52508361204013)(1478,-7.58528428093645)(1478,-7.64548494983278)(1478,-7.7056856187291)(1478,-7.76588628762542)(1478,-7.82608695652174)(1478,-7.88628762541806)(1478,-7.94648829431438)(1478,-8.0066889632107)(1478,-8.06688963210702)(1478,-8.12709030100334)(1478,-8.18729096989967)(1478,-8.24749163879599)(1478,-8.30769230769231)(1478,-8.36789297658863)(1478,-8.42809364548495)(1478,-8.48829431438127)(1478,-8.54849498327759)(1478.00024714483,-8.60869565217391)(1478.00024714483,-8.66889632107023)(1478.00024714483,-8.72909698996656)(1478.00024714483,-8.78929765886288)(1478.00024714483,-8.8494983277592)(1478.00024714483,-8.90969899665552)(1478.00024714483,-8.96989966555184)(1478.00024714483,-9.03010033444816)(1478.00024714483,-9.09030100334448)(1478.00024714483,-9.1505016722408)(1478.00024714483,-9.21070234113712)(1478.00024714483,-9.27090301003344)(1478.00024714483,-9.33110367892977)(1478.00024714483,-9.39130434782609)(1478.00024714483,-9.45150501672241)(1478.00024714483,-9.51170568561873)(1478.00024714483,-9.57190635451505)(1478.00024714483,-9.63210702341137)(1478.00024714483,-9.69230769230769)(1478.00024714483,-9.75250836120401)(1478.00024714483,-9.81270903010033)(1478,-9.87290969899666)(1478,-9.93311036789298)(1478,-9.9933110367893)(1478,-10.0535117056856)(1478,-10.1137123745819)(1478,-10.1739130434783)(1478,-10.2341137123746)(1478,-10.2943143812709)(1478,-10.3545150501672)(1478,-10.4147157190635)(1478,-10.4749163879599)(1478,-10.5351170568562)(1478,-10.5953177257525)(1478,-10.6555183946488)(1478,-10.7157190635452)(1478,-10.7759197324415)(1478,-10.8361204013378)(1478,-10.8963210702341)(1478,-10.9565217391304)(1478,-11.0167224080268)(1478,-11.0769230769231)(1478,-11.1371237458194)(1478,-11.1973244147157)(1478,-11.257525083612)(1478,-11.3177257525084)(1478,-11.3779264214047)(1478,-11.438127090301)(1478,-11.4983277591973)(1478,-11.5585284280936)(1478,-11.61872909699)(1478,-11.6789297658863)(1478,-11.7391304347826)(1478,-11.7993311036789)(1478,-11.8595317725753)(1478,-11.9197324414716)(1478,-11.9799331103679)(1478,-12.0401337792642)(1478,-12.1003344481605)(1478,-12.1605351170569)(1478,-12.2207357859532)(1478,-12.2809364548495)(1478,-12.3411371237458)(1478,-12.4013377926421)(1478,-12.4615384615385)(1478,-12.5217391304348)(1478,-12.5819397993311)(1478,-12.6421404682274)(1478,-12.7023411371237)(1478,-12.7625418060201)(1478,-12.8227424749164)(1478,-12.8829431438127)(1478,-12.943143812709)(1478,-13.0033444816054)(1478,-13.0635451505017)(1478,-13.123745819398)(1478,-13.1839464882943)(1478,-13.2441471571906)(1478,-13.304347826087)(1478,-13.3645484949833)(1478,-13.4247491638796)(1478,-13.4849498327759)(1478,-13.5451505016722)(1478,-13.6053511705686)(1478,-13.6655518394649)(1478,-13.7257525083612)(1478,-13.7859531772575)(1478,-13.8461538461538)(1478,-13.9063545150502)(1478,-13.9665551839465)(1478,-14.0267558528428)(1478,-14.0869565217391)(1478,-14.1471571906355)(1478,-14.2073578595318)(1478,-14.2675585284281)(1478,-14.3277591973244)(1478,-14.3879598662207)(1478,-14.4481605351171)(1478,-14.5083612040134)(1478,-14.5685618729097)(1478,-14.628762541806)(1478,-14.6889632107023)(1478,-14.7491638795987)(1478,-14.809364548495)(1478,-14.8695652173913)(1478,-14.9297658862876)(1478,-14.9899665551839)(1478,-15.0501672240803)(1478,-15.1103678929766)(1478,-15.1705685618729)(1478,-15.2307692307692)(1478,-15.2909698996656)(1478,-15.3511705685619)(1478,-15.4113712374582)(1478,-15.4715719063545)(1478,-15.5317725752508)(1478,-15.5919732441472)(1478,-15.6521739130435)(1478,-15.7123745819398)(1478,-15.7725752508361)(1478,-15.8327759197324)(1478,-15.8929765886288)(1478,-15.9531772575251)(1478,-16.0133779264214)(1478,-16.0735785953177)(1478,-16.133779264214)(1478,-16.1939799331104)(1478,-16.2541806020067)(1478,-16.314381270903)(1478,-16.3745819397993)(1478,-16.4347826086957)(1478,-16.494983277592)(1478,-16.5551839464883)(1478,-16.6153846153846)(1478,-16.6755852842809)(1478,-16.7357859531773)(1478,-16.7959866220736)(1478,-16.8561872909699)(1478,-16.9163879598662)(1478,-16.9765886287625)(1478,-17.0367892976589)(1478,-17.0969899665552)(1478,-17.1571906354515)(1478,-17.2173913043478)(1478,-17.2775919732441)(1478,-17.3377926421405)(1478,-17.3979933110368)(1478,-17.4581939799331)(1478,-17.5183946488294)(1478,-17.5785953177258)(1478,-17.6387959866221)(1478,-17.6989966555184)(1478,-17.7591973244147)(1478.00024714483,-17.819397993311)(1478.00024714483,-17.8795986622074)(1478.00024714483,-17.9397993311037)(1478.00024714483,-18)(1478,-18.000003009883)(1473.05685618729,-18.000003009883)(1468.11371237458,-18.000003009883)(1463.17056856187,-18.000003009883)(1458.22742474916,-18.000003009883)(1453.28428093645,-18)(1448.34113712375,-18)(1443.39799331104,-18)(1438.45484949833,-18)(1433.51170568562,-18)(1428.56856187291,-18)(1423.6254180602,-18)(1418.68227424749,-18)(1413.73913043478,-18)(1408.79598662207,-18)(1403.85284280936,-18)(1398.90969899666,-18)(1393.96655518395,-18)(1389.02341137124,-18)(1384.08026755853,-18)(1379.13712374582,-18)(1374.19397993311,-18)(1369.2508361204,-18)(1364.30769230769,-18)(1359.36454849498,-18)(1354.42140468227,-18)(1349.47826086957,-18)(1344.53511705686,-18)(1339.59197324415,-18)(1334.64882943144,-18)(1329.70568561873,-18)(1324.76254180602,-18)(1319.81939799331,-18)(1314.8762541806,-18)(1309.93311036789,-18)(1304.98996655518,-18)(1300.04682274247,-18)(1295.10367892977,-18)(1290.16053511706,-18)(1285.21739130435,-18)(1280.27424749164,-18)(1275.33110367893,-18)(1270.38795986622,-18)(1265.44481605351,-18)(1260.5016722408,-18)(1255.55852842809,-18)(1250.61538461538,-18)(1245.67224080268,-18)(1240.72909698997,-18)(1235.78595317726,-18)(1230.84280936455,-18)(1225.89966555184,-18)(1220.95652173913,-18)(1216.01337792642,-18)(1211.07023411371,-18)(1206.127090301,-18)(1201.18394648829,-18)(1196.24080267559,-18)(1191.29765886288,-18)(1186.35451505017,-18)(1181.41137123746,-18)(1176.46822742475,-18)(1171.52508361204,-18)(1166.58193979933,-18)(1161.63879598662,-18)(1156.69565217391,-18)(1151.7525083612,-18)(1146.8093645485,-18)(1141.86622073579,-18)(1136.92307692308,-18)(1131.97993311037,-18)(1127.03678929766,-18)(1122.09364548495,-18)(1117.15050167224,-18)(1112.20735785953,-18)(1107.26421404682,-18)(1102.32107023411,-18)(1097.3779264214,-18)(1092.4347826087,-18)(1087.49163879599,-18)(1082.54849498328,-18)(1077.60535117057,-18)(1072.66220735786,-18)(1067.71906354515,-18)(1062.77591973244,-18)(1057.83277591973,-18)(1052.88963210702,-18)(1047.94648829431,-18)(1043.00334448161,-18)(1038.0602006689,-18)(1033.11705685619,-18)(1028.17391304348,-18)(1023.23076923077,-18)(1018.28762541806,-18)(1013.34448160535,-18)(1008.40133779264,-18)(1003.45819397993,-18)(998.515050167224,-18)(993.571906354515,-18)(988.628762541806,-18)(983.685618729097,-18)(978.742474916388,-18)(973.799331103679,-18)(968.85618729097,-18)(963.913043478261,-18)(958.969899665552,-18)(954.026755852843,-18)(949.083612040134,-18)(944.140468227425,-18)(939.197324414716,-18)(934.254180602007,-18)(929.311036789298,-18)(924.367892976589,-18)(919.42474916388,-18)(914.481605351171,-18)(909.538461538462,-18)(904.595317725752,-18)(899.652173913044,-18)(894.709030100334,-18)(889.765886287625,-18)(884.822742474916,-18)(879.879598662207,-18)(874.936454849498,-18)(869.993311036789,-18)(865.05016722408,-18)(860.107023411371,-18)(855.163879598662,-18)(850.220735785953,-18)(845.277591973244,-18)(840.334448160535,-18)(835.391304347826,-18)(830.448160535117,-18)(825.505016722408,-18)(820.561872909699,-18)(815.61872909699,-18)(810.675585284281,-18)(805.732441471572,-18)(800.789297658863,-18)(795.846153846154,-18)(790.903010033445,-18)(785.959866220736,-18)(781.016722408027,-18)(776.073578595318,-18)(771.130434782609,-18)(766.1872909699,-18)(761.244147157191,-18)(756.301003344482,-18)(751.357859531773,-18)(746.414715719064,-18)(741.471571906354,-18)(736.528428093646,-18)(731.585284280936,-18)(726.642140468227,-18)(721.698996655518,-18)(716.755852842809,-18)(711.8127090301,-18)(706.869565217391,-18)(701.926421404682,-18)(696.983277591973,-18)(692.040133779264,-18)(687.096989966555,-18)(682.153846153846,-18)(677.210702341137,-18)(672.267558528428,-18)(667.324414715719,-18)(662.38127090301,-18)(657.438127090301,-18)(652.494983277592,-18)(647.551839464883,-18)(642.608695652174,-18)(637.665551839465,-18)(632.722408026756,-18)(627.779264214047,-18)(622.836120401338,-18)(617.892976588629,-18)(612.94983277592,-18)(608.006688963211,-18)(603.063545150502,-18)(598.120401337793,-18)(593.177257525084,-18)(588.234113712375,-18)(583.290969899666,-18)(578.347826086957,-18)(573.404682274248,-18)(568.461538461538,-18)(563.518394648829,-18)(558.57525083612,-18)(553.632107023411,-18)(548.688963210702,-18)(543.745819397993,-18)(538.802675585284,-18)(533.859531772575,-18)(528.916387959866,-18)(523.973244147157,-18)(519.030100334448,-18)(514.086956521739,-18)(509.14381270903,-18)(504.200668896321,-18)(499.257525083612,-18)(494.314381270903,-18)(489.371237458194,-18)(484.428093645485,-18)(479.484949832776,-18)(474.541806020067,-18)(469.598662207358,-18)(464.655518394649,-18)(459.71237458194,-18)(454.769230769231,-18)(449.826086956522,-18)(444.882943143813,-18)(439.939799331104,-18)(434.996655518395,-18)(430.053511705686,-18)(425.110367892977,-18)(420.167224080268,-18)(415.224080267559,-18)(410.28093645485,-18)(405.33779264214,-18)(400.394648829431,-18)(395.451505016722,-18)(390.508361204013,-18)(385.565217391304,-18)(380.622073578595,-18)(375.678929765886,-18)(370.735785953177,-18)(365.792642140468,-18)(360.849498327759,-18)(355.90635451505,-18)(350.963210702341,-18)(346.020066889632,-18)(341.076923076923,-18)(336.133779264214,-18)(331.190635451505,-18)(326.247491638796,-18)(321.304347826087,-18)(316.361204013378,-18)(311.418060200669,-18)(306.47491638796,-18)(301.531772575251,-18)(296.588628762542,-18)(291.645484949833,-18)(286.702341137124,-18)(281.759197324415,-18)(276.816053511706,-18)(271.872909698997,-18)(266.929765886288,-18)(261.986622073579,-18)(257.04347826087,-18)(252.100334448161,-18)(247.157190635452,-18)(242.214046822742,-18)(237.270903010033,-18)(232.327759197324,-18)(227.384615384615,-18)(222.441471571906,-18)(217.498327759197,-18)(212.555183946488,-18)(207.612040133779,-18)(202.66889632107,-18)(197.725752508361,-18)(192.782608695652,-18)(187.839464882943,-18)(182.896321070234,-18)(177.953177257525,-18)(173.010033444816,-18)(168.066889632107,-18)(163.123745819398,-18)(158.180602006689,-18)(153.23745819398,-18)(148.294314381271,-18)(143.351170568562,-18)(138.408026755853,-18)(133.464882943144,-18)(128.521739130435,-18)(123.578595317726,-18)(118.635451505017,-18)(113.692307692308,-18)(108.749163879599,-18)(103.80602006689,-18)(98.8628762541806,-18)(93.9197324414716,-18)(88.9765886287625,-18)(84.0334448160535,-18)(79.0903010033445,-18)(74.1471571906355,-18)(69.2040133779264,-18)(64.2608695652174,-18)(59.3177257525084,-18)(54.3745819397993,-18)(49.4314381270903,-18)(44.4882943143813,-18)(39.5451505016722,-18)(34.6020066889632,-18)(29.6588628762542,-18)(24.7157190635452,-18)(19.7725752508361,-18)(14.8294314381271,-18)(9.88628762541806,-18)(4.94314381270903,-18)(0,-18.0000060194649)(-0.000494264954775508,-18)(0,-17.9397993311037)(0,-17.8795986622074)(0,-17.819397993311)(0,-17.7591973244147)(0,-17.6989966555184)(0,-17.6387959866221)(0,-17.5785953177258)(0,-17.5183946488294)(0,-17.4581939799331)(0,-17.3979933110368)(0,-17.3377926421405)(0,-17.2775919732441)(0,-17.2173913043478)(0,-17.1571906354515)(0,-17.0969899665552)(0,-17.0367892976589)(0,-16.9765886287625)(0,-16.9163879598662)(0,-16.8561872909699)(0,-16.7959866220736)(0,-16.7357859531773)(0,-16.6755852842809)(0,-16.6153846153846)(0,-16.5551839464883)(0,-16.494983277592)(0,-16.4347826086957)(0,-16.3745819397993)(0,-16.314381270903)(0,-16.2541806020067)(0,-16.1939799331104)(0,-16.133779264214)(0,-16.0735785953177)(0,-16.0133779264214)(0,-15.9531772575251)(0,-15.8929765886288)(0,-15.8327759197324)(0,-15.7725752508361)(0,-15.7123745819398)(0,-15.6521739130435)(0,-15.5919732441472)(0,-15.5317725752508)(0,-15.4715719063545)(0,-15.4113712374582)(0,-15.3511705685619)(0,-15.2909698996656)(0,-15.2307692307692)(0,-15.1705685618729)(0,-15.1103678929766)(0,-15.0501672240803)(0,-14.9899665551839)(0,-14.9297658862876)(0,-14.8695652173913)(0,-14.809364548495)(0,-14.7491638795987)(0,-14.6889632107023)(0,-14.628762541806)(0,-14.5685618729097)(0,-14.5083612040134)(0,-14.4481605351171)(0,-14.3879598662207)(0,-14.3277591973244)(0,-14.2675585284281)(0,-14.2073578595318)(0,-14.1471571906355)(0,-14.0869565217391)(0,-14.0267558528428)(0,-13.9665551839465)(0,-13.9063545150502)(0,-13.8461538461538)(0,-13.7859531772575)(0,-13.7257525083612)(0,-13.6655518394649)(0,-13.6053511705686)(0,-13.5451505016722)(0,-13.4849498327759)(0,-13.4247491638796)(0,-13.3645484949833)(0,-13.304347826087)(0,-13.2441471571906)(0,-13.1839464882943)(0,-13.123745819398)(0,-13.0635451505017)(0,-13.0033444816054)(0,-12.943143812709)(0,-12.8829431438127)(0,-12.8227424749164)(0,-12.7625418060201)(0,-12.7023411371237)(0,-12.6421404682274)(0,-12.5819397993311)(0,-12.5217391304348)(0,-12.4615384615385)(0,-12.4013377926421)(0,-12.3411371237458)(0,-12.2809364548495)(0,-12.2207357859532)(0,-12.1605351170569)(0,-12.1003344481605)(0,-12.0401337792642)(0,-11.9799331103679)(0,-11.9197324414716)(0,-11.8595317725753)(0,-11.7993311036789)(0,-11.7391304347826)(0,-11.6789297658863)(0,-11.61872909699)(0,-11.5585284280936)(0,-11.4983277591973)(0,-11.438127090301)(0,-11.3779264214047)(0,-11.3177257525084)(0,-11.257525083612)(0,-11.1973244147157)(0,-11.1371237458194)(0,-11.0769230769231)(0,-11.0167224080268)(0,-10.9565217391304)(0,-10.8963210702341)(0,-10.8361204013378)(0,-10.7759197324415)(0,-10.7157190635452)(0,-10.6555183946488)(0,-10.5953177257525)(0,-10.5351170568562)(0,-10.4749163879599)(0,-10.4147157190635)(0,-10.3545150501672)(0,-10.2943143812709)(0,-10.2341137123746)(0,-10.1739130434783)(0,-10.1137123745819)(0,-10.0535117056856)(0,-9.9933110367893)(0,-9.93311036789298)(0,-9.87290969899666)(0,-9.81270903010033)(0,-9.75250836120401)(0,-9.69230769230769)(0,-9.63210702341137)(0,-9.57190635451505)(0,-9.51170568561873)(0,-9.45150501672241)(0,-9.39130434782609)(0,-9.33110367892977)(0,-9.27090301003344)(0,-9.21070234113712)(0,-9.1505016722408)(-0.000247144833393367,-9.09030100334448)(-0.000247144833393367,-9.03010033444816)(-0.000247144833393367,-8.96989966555184)(-0.000247144833393367,-8.90969899665552)(-0.000247144833393367,-8.8494983277592)(-0.000247144833393367,-8.78929765886288)(-0.000247144833393367,-8.72909698996656)(-0.000247144833393367,-8.66889632107023)(-0.000247144833393367,-8.60869565217391)(-0.000247144833393367,-8.54849498327759)(-0.000247144833393367,-8.48829431438127)(-0.000247144833393367,-8.42809364548495)(-0.000247144833393367,-8.36789297658863)(-0.000247144833393367,-8.30769230769231)(-0.000247144833393367,-8.24749163879599)(-0.000247144833393367,-8.18729096989967)(-0.000247144833393367,-8.12709030100334)(-0.000247144833393367,-8.06688963210702)(-0.000247144833393367,-8.0066889632107)(-0.000247144833393367,-7.94648829431438)(-0.000247144833393367,-7.88628762541806)(-0.000247144833393367,-7.82608695652174)(-0.000247144833393367,-7.76588628762542)(-0.000247144833393367,-7.7056856187291)(-0.000247144833393367,-7.64548494983278)(-0.000247144833393367,-7.58528428093645)(-0.000247144833393367,-7.52508361204013)(-0.000247144833393367,-7.46488294314381)(-0.000247144833393367,-7.40468227424749)(-0.000247144833393367,-7.34448160535117)(-0.000247144833393367,-7.28428093645485)(-0.000247144833393367,-7.22408026755853)(-0.000247144833393367,-7.16387959866221)(-0.000247144833393367,-7.10367892976589)(-0.000247144833393367,-7.04347826086956)(-0.000247144833393367,-6.98327759197324)(-0.000247144833393367,-6.92307692307692)(-0.000247144833393367,-6.8628762541806)(-0.000247144833393367,-6.80267558528428)(-0.000247144833393367,-6.74247491638796)(-0.000247144833393367,-6.68227424749164)(-0.000247144833393367,-6.62207357859532)(-0.000247144833393367,-6.561872909699)(-0.000247144833393367,-6.50167224080267)(-0.000247144833393367,-6.44147157190636)(-0.000247144833393367,-6.38127090301003)(-0.000247144833393367,-6.32107023411371)(-0.000247144833393367,-6.26086956521739)(-0.000247144833393367,-6.20066889632107)(-0.000247144833393367,-6.14046822742475)(-0.000247144833393367,-6.08026755852843)(-0.000247144833393367,-6.02006688963211)(0,-5.95986622073579)(0,-5.89966555183947)(0,-5.83946488294314)(0,-5.77926421404682)(0,-5.7190635451505)(0,-5.65886287625418)(0,-5.59866220735786)(0,-5.53846153846154)(0,-5.47826086956522)(0,-5.4180602006689)(0,-5.35785953177258)(0,-5.29765886287625)(0,-5.23745819397993)(0,-5.17725752508361)(0,-5.11705685618729)(0,-5.05685618729097)(0,-4.99665551839465)(0,-4.93645484949833)(0,-4.87625418060201)(0,-4.81605351170569)(0,-4.75585284280936)(0,-4.69565217391304)(0,-4.63545150501672)(0,-4.5752508361204)(0,-4.51505016722408)(0,-4.45484949832776)(0,-4.39464882943144)(0,-4.33444816053512)(0,-4.2742474916388)(0,-4.21404682274247)(0,-4.15384615384615)(0,-4.09364548494983)(0,-4.03344481605351)(0,-3.97324414715719)(0,-3.91304347826087)(0,-3.85284280936455)(0,-3.79264214046823)(0,-3.73244147157191)(0,-3.67224080267559)(0,-3.61204013377926)(0,-3.55183946488294)(0,-3.49163879598662)(0,-3.4314381270903)(0,-3.37123745819398)(0,-3.31103678929766)(0,-3.25083612040134)(0,-3.19063545150502)(0,-3.1304347826087)(0,-3.07023411371237)(0,-3.01003344481605)(0,-2.94983277591973)(0,-2.88963210702341)(0,-2.82943143812709)(0,-2.76923076923077)(0,-2.70903010033445)(0,-2.64882943143813)(0,-2.58862876254181)(0,-2.52842809364548)(0,-2.46822742474916)(0,-2.40802675585284)(0,-2.34782608695652)(0,-2.2876254180602)(0,-2.22742474916388)(0,-2.16722408026756)(0,-2.10702341137124)(0,-2.04682274247492)(0,-1.98662207357859)(0,-1.92642140468227)(0,-1.86622073578595)(0,-1.80602006688963)(0,-1.74581939799331)(0,-1.68561872909699)(0,-1.62541806020067)(0,-1.56521739130435)(0,-1.50501672240803)(0,-1.4448160535117)(0,-1.38461538461538)(0,-1.32441471571906)(0,-1.26421404682274)(0,-1.20401337792642)(0,-1.1438127090301)(0,-1.08361204013378)(0,-1.02341137123746)(0,-0.963210702341136)(0,-0.903010033444815)(0,-0.842809364548494)(0,-0.782608695652176)(0,-0.722408026755854)(0,-0.662207357859533)(0,-0.602006688963211)(0,-0.54180602006689)(0,-0.481605351170568)(0,-0.421404682274247)(0,-0.361204013377925)(0,-0.301003344481604)(0,-0.240802675585286)(0,-0.180602006688964)(0,-0.120401337792643)(0,-0.0602006688963215)(0,0)};

\addplot [fill=darkgray,draw=black,forget plot] coordinates{ (1478,-8.60869565217391)(1478,-8.66889632107023)(1478,-8.72909698996656)(1478,-8.78929765886288)(1478,-8.8494983277592)(1478,-8.90969899665552)(1478,-8.96989966555184)(1478,-9.03010033444816)(1478,-9.09030100334448)(1478,-9.1505016722408)(1478,-9.21070234113712)(1478,-9.27090301003344)(1478,-9.33110367892977)(1478,-9.39130434782609)(1478,-9.45150501672241)(1478,-9.51170568561873)(1478,-9.57190635451505)(1478,-9.63210702341137)(1478,-9.69230769230769)(1478,-9.75250836120401)(1478,-9.81270903010033)(1473.05685618729,-9.87290969899666)(1468.11371237458,-9.81270903010033)(1463.17056856187,-9.81270903010033)(1458.22742474916,-9.81270903010033)(1453.28428093645,-9.87290969899666)(1448.34113712375,-9.87290969899666)(1443.39799331104,-9.93311036789298)(1438.45484949833,-9.9933110367893)(1433.51170568562,-9.9933110367893)(1428.56856187291,-10.0535117056856)(1423.6254180602,-10.1137123745819)(1418.68227424749,-10.1137123745819)(1413.73913043478,-10.1137123745819)(1408.79598662207,-10.1739130434783)(1403.85284280936,-10.1739130434783)(1398.90969899666,-10.2341137123746)(1393.96655518395,-10.2341137123746)(1389.02341137124,-10.2943143812709)(1384.08026755853,-10.2943143812709)(1379.13712374582,-10.3545150501672)(1374.19397993311,-10.3545150501672)(1369.2508361204,-10.4147157190635)(1364.30769230769,-10.4147157190635)(1359.36454849498,-10.4749163879599)(1354.42140468227,-10.4749163879599)(1349.47826086957,-10.4749163879599)(1344.53511705686,-10.4749163879599)(1339.59197324415,-10.5351170568562)(1334.64882943144,-10.5351170568562)(1329.70568561873,-10.5351170568562)(1324.76254180602,-10.5351170568562)(1319.81939799331,-10.5351170568562)(1314.8762541806,-10.5351170568562)(1309.93311036789,-10.5351170568562)(1304.98996655518,-10.5351170568562)(1300.04682274247,-10.4749163879599)(1295.10367892977,-10.4749163879599)(1290.16053511706,-10.4749163879599)(1285.21739130435,-10.4749163879599)(1280.27424749164,-10.4147157190635)(1275.33110367893,-10.4147157190635)(1270.38795986622,-10.4147157190635)(1265.44481605351,-10.4147157190635)(1260.5016722408,-10.4147157190635)(1255.55852842809,-10.4147157190635)(1250.61538461538,-10.4147157190635)(1245.67224080268,-10.4147157190635)(1240.72909698997,-10.4147157190635)(1235.78595317726,-10.4147157190635)(1230.84280936455,-10.4147157190635)(1225.89966555184,-10.4147157190635)(1220.95652173913,-10.3545150501672)(1216.01337792642,-10.3545150501672)(1211.07023411371,-10.3545150501672)(1206.127090301,-10.3545150501672)(1201.18394648829,-10.3545150501672)(1196.24080267559,-10.3545150501672)(1191.29765886288,-10.2943143812709)(1186.35451505017,-10.2943143812709)(1181.41137123746,-10.2943143812709)(1176.46822742475,-10.2943143812709)(1171.52508361204,-10.2943143812709)(1166.58193979933,-10.2341137123746)(1161.63879598662,-10.2341137123746)(1156.69565217391,-10.2341137123746)(1151.7525083612,-10.2341137123746)(1146.8093645485,-10.2341137123746)(1141.86622073579,-10.2341137123746)(1136.92307692308,-10.2341137123746)(1131.97993311037,-10.2341137123746)(1127.03678929766,-10.2341137123746)(1122.09364548495,-10.2341137123746)(1117.15050167224,-10.2341137123746)(1112.20735785953,-10.2341137123746)(1107.26421404682,-10.2341137123746)(1102.32107023411,-10.2341137123746)(1097.3779264214,-10.2341137123746)(1092.4347826087,-10.2341137123746)(1087.49163879599,-10.2341137123746)(1082.54849498328,-10.2341137123746)(1077.60535117057,-10.2341137123746)(1072.66220735786,-10.2341137123746)(1067.71906354515,-10.2341137123746)(1062.77591973244,-10.2341137123746)(1057.83277591973,-10.2341137123746)(1052.88963210702,-10.2341137123746)(1047.94648829431,-10.2341137123746)(1043.00334448161,-10.2341137123746)(1038.0602006689,-10.2341137123746)(1033.11705685619,-10.2341137123746)(1028.17391304348,-10.2341137123746)(1023.23076923077,-10.2341137123746)(1018.28762541806,-10.2341137123746)(1013.34448160535,-10.2341137123746)(1008.40133779264,-10.2341137123746)(1003.45819397993,-10.2341137123746)(998.515050167224,-10.2341137123746)(993.571906354515,-10.2341137123746)(988.628762541806,-10.2341137123746)(983.685618729097,-10.2341137123746)(978.742474916388,-10.2341137123746)(973.799331103679,-10.2341137123746)(968.85618729097,-10.2341137123746)(963.913043478261,-10.2341137123746)(958.969899665552,-10.2341137123746)(954.026755852843,-10.2341137123746)(949.083612040134,-10.2341137123746)(944.140468227425,-10.2341137123746)(939.197324414716,-10.1739130434783)(934.254180602007,-10.1739130434783)(929.311036789298,-10.1739130434783)(924.367892976589,-10.1739130434783)(919.42474916388,-10.1739130434783)(914.481605351171,-10.1739130434783)(909.538461538462,-10.1739130434783)(904.595317725752,-10.1739130434783)(899.652173913044,-10.1137123745819)(894.709030100334,-10.1137123745819)(889.765886287625,-10.1137123745819)(884.822742474916,-10.0535117056856)(879.879598662207,-10.0535117056856)(874.936454849498,-10.0535117056856)(869.993311036789,-9.9933110367893)(865.05016722408,-9.9933110367893)(860.107023411371,-9.93311036789298)(855.163879598662,-9.93311036789298)(850.220735785953,-9.87290969899666)(845.277591973244,-9.87290969899666)(840.334448160535,-9.81270903010033)(835.391304347826,-9.81270903010033)(830.448160535117,-9.75250836120401)(825.505016722408,-9.75250836120401)(820.561872909699,-9.75250836120401)(815.61872909699,-9.69230769230769)(810.675585284281,-9.63210702341137)(805.732441471572,-9.63210702341137)(800.789297658863,-9.57190635451505)(795.846153846154,-9.51170568561873)(790.903010033445,-9.45150501672241)(785.959866220736,-9.39130434782609)(781.016722408027,-9.33110367892977)(781.016722408027,-9.27090301003344)(776.073578595318,-9.21070234113712)(771.130434782609,-9.1505016722408)(771.130434782609,-9.09030100334448)(766.1872909699,-9.03010033444816)(766.1872909699,-8.96989966555184)(761.244147157191,-8.90969899665552)(761.244147157191,-8.8494983277592)(761.244147157191,-8.78929765886288)(756.301003344482,-8.72909698996656)(756.301003344482,-8.66889632107023)(756.301003344482,-8.60869565217391)(751.357859531773,-8.54849498327759)(751.357859531773,-8.48829431438127)(751.357859531773,-8.42809364548495)(751.357859531773,-8.36789297658863)(751.357859531773,-8.30769230769231)(756.301003344482,-8.24749163879599)(756.301003344482,-8.18729096989967)(761.244147157191,-8.12709030100334)(761.244147157191,-8.06688963210702)(766.1872909699,-8.0066889632107)(771.130434782609,-7.94648829431438)(776.073578595318,-7.94648829431438)(781.016722408027,-7.88628762541806)(785.959866220736,-7.88628762541806)(790.903010033445,-7.82608695652174)(795.846153846154,-7.82608695652174)(800.789297658863,-7.76588628762542)(805.732441471572,-7.76588628762542)(810.675585284281,-7.7056856187291)(815.61872909699,-7.7056856187291)(820.561872909699,-7.7056856187291)(825.505016722408,-7.64548494983278)(830.448160535117,-7.64548494983278)(835.391304347826,-7.58528428093645)(840.334448160535,-7.58528428093645)(845.277591973244,-7.58528428093645)(850.220735785953,-7.52508361204013)(855.163879598662,-7.52508361204013)(860.107023411371,-7.46488294314381)(865.05016722408,-7.46488294314381)(869.993311036789,-7.40468227424749)(874.936454849498,-7.40468227424749)(879.879598662207,-7.34448160535117)(884.822742474916,-7.34448160535117)(889.765886287625,-7.34448160535117)(894.709030100334,-7.28428093645485)(899.652173913044,-7.28428093645485)(904.595317725752,-7.28428093645485)(909.538461538462,-7.22408026755853)(914.481605351171,-7.22408026755853)(919.42474916388,-7.22408026755853)(924.367892976589,-7.16387959866221)(929.311036789298,-7.16387959866221)(934.254180602007,-7.16387959866221)(939.197324414716,-7.16387959866221)(944.140468227425,-7.16387959866221)(949.083612040134,-7.16387959866221)(954.026755852843,-7.16387959866221)(958.969899665552,-7.16387959866221)(963.913043478261,-7.16387959866221)(968.85618729097,-7.16387959866221)(973.799331103679,-7.16387959866221)(978.742474916388,-7.16387959866221)(983.685618729097,-7.16387959866221)(988.628762541806,-7.16387959866221)(993.571906354515,-7.22408026755853)(998.515050167224,-7.22408026755853)(1003.45819397993,-7.22408026755853)(1008.40133779264,-7.22408026755853)(1013.34448160535,-7.22408026755853)(1018.28762541806,-7.22408026755853)(1023.23076923077,-7.22408026755853)(1028.17391304348,-7.28428093645485)(1033.11705685619,-7.28428093645485)(1038.0602006689,-7.28428093645485)(1043.00334448161,-7.22408026755853)(1047.94648829431,-7.22408026755853)(1052.88963210702,-7.22408026755853)(1057.83277591973,-7.22408026755853)(1062.77591973244,-7.22408026755853)(1067.71906354515,-7.16387959866221)(1072.66220735786,-7.16387959866221)(1077.60535117057,-7.16387959866221)(1082.54849498328,-7.10367892976589)(1087.49163879599,-7.10367892976589)(1092.4347826087,-7.10367892976589)(1097.3779264214,-7.04347826086956)(1102.32107023411,-7.04347826086956)(1107.26421404682,-7.04347826086956)(1112.20735785953,-6.98327759197324)(1117.15050167224,-6.98327759197324)(1122.09364548495,-6.98327759197324)(1127.03678929766,-6.98327759197324)(1131.97993311037,-6.98327759197324)(1136.92307692308,-6.98327759197324)(1141.86622073579,-6.98327759197324)(1146.8093645485,-6.98327759197324)(1151.7525083612,-6.98327759197324)(1156.69565217391,-6.98327759197324)(1161.63879598662,-6.98327759197324)(1166.58193979933,-7.04347826086956)(1171.52508361204,-7.04347826086956)(1176.46822742475,-7.04347826086956)(1181.41137123746,-7.10367892976589)(1186.35451505017,-7.10367892976589)(1191.29765886288,-7.16387959866221)(1196.24080267559,-7.16387959866221)(1201.18394648829,-7.22408026755853)(1206.127090301,-7.22408026755853)(1211.07023411371,-7.28428093645485)(1216.01337792642,-7.28428093645485)(1220.95652173913,-7.34448160535117)(1225.89966555184,-7.34448160535117)(1230.84280936455,-7.40468227424749)(1235.78595317726,-7.40468227424749)(1240.72909698997,-7.46488294314381)(1245.67224080268,-7.46488294314381)(1250.61538461538,-7.52508361204013)(1255.55852842809,-7.52508361204013)(1260.5016722408,-7.58528428093645)(1265.44481605351,-7.58528428093645)(1270.38795986622,-7.58528428093645)(1275.33110367893,-7.64548494983278)(1280.27424749164,-7.64548494983278)(1285.21739130435,-7.64548494983278)(1290.16053511706,-7.7056856187291)(1295.10367892977,-7.7056856187291)(1300.04682274247,-7.7056856187291)(1304.98996655518,-7.76588628762542)(1309.93311036789,-7.76588628762542)(1314.8762541806,-7.76588628762542)(1319.81939799331,-7.82608695652174)(1324.76254180602,-7.82608695652174)(1329.70568561873,-7.82608695652174)(1334.64882943144,-7.82608695652174)(1339.59197324415,-7.88628762541806)(1344.53511705686,-7.88628762541806)(1349.47826086957,-7.88628762541806)(1354.42140468227,-7.94648829431438)(1359.36454849498,-7.94648829431438)(1364.30769230769,-7.94648829431438)(1369.2508361204,-8.0066889632107)(1374.19397993311,-8.0066889632107)(1379.13712374582,-8.06688963210702)(1384.08026755853,-8.06688963210702)(1389.02341137124,-8.12709030100334)(1393.96655518395,-8.12709030100334)(1398.90969899666,-8.18729096989967)(1403.85284280936,-8.18729096989967)(1408.79598662207,-8.24749163879599)(1413.73913043478,-8.30769230769231)(1418.68227424749,-8.36789297658863)(1423.6254180602,-8.36789297658863)(1428.56856187291,-8.42809364548495)(1433.51170568562,-8.42809364548495)(1438.45484949833,-8.48829431438127)(1443.39799331104,-8.48829431438127)(1448.34113712375,-8.54849498327759)(1453.28428093645,-8.54849498327759)(1458.22742474916,-8.54849498327759)(1463.17056856187,-8.60869565217391)(1468.11371237458,-8.60869565217391)(1473.05685618729,-8.60869565217391)(1478,-8.60869565217391)};

\addplot [fill=red!40!yellow,draw=black,forget plot] coordinates{ (1398.90969899666,-9.36120401337793)(1401.38127090301,-9.39130434782609)(1398.90969899666,-9.42140468227425)(1396.4381270903,-9.45150501672241)(1396.4381270903,-9.51170568561873)(1396.4381270903,-9.57190635451505)(1396.4381270903,-9.63210702341137)(1393.96655518395,-9.66220735785953)(1391.49498327759,-9.69230769230769)(1389.02341137124,-9.72240802675585)(1386.55183946488,-9.75250836120401)(1384.08026755853,-9.78260869565217)(1379.13712374582,-9.78260869565217)(1376.66555183947,-9.81270903010033)(1374.19397993311,-9.8428093645485)(1369.2508361204,-9.8428093645485)(1364.30769230769,-9.8428093645485)(1361.83612040134,-9.87290969899666)(1359.36454849498,-9.90301003344482)(1354.42140468227,-9.90301003344482)(1349.47826086957,-9.90301003344482)(1344.53511705686,-9.90301003344482)(1339.59197324415,-9.90301003344482)(1334.64882943144,-9.90301003344482)(1329.70568561873,-9.90301003344482)(1324.76254180602,-9.90301003344482)(1319.81939799331,-9.90301003344482)(1317.34782608696,-9.87290969899666)(1314.8762541806,-9.8428093645485)(1309.93311036789,-9.8428093645485)(1304.98996655518,-9.8428093645485)(1300.04682274247,-9.8428093645485)(1295.10367892977,-9.8428093645485)(1290.16053511706,-9.8428093645485)(1285.21739130435,-9.8428093645485)(1280.27424749164,-9.8428093645485)(1275.33110367893,-9.8428093645485)(1270.38795986622,-9.8428093645485)(1265.44481605351,-9.8428093645485)(1260.5016722408,-9.8428093645485)(1255.55852842809,-9.8428093645485)(1253.08695652174,-9.87290969899666)(1250.61538461538,-9.90301003344482)(1245.67224080268,-9.90301003344482)(1240.72909698997,-9.90301003344482)(1235.78595317726,-9.90301003344482)(1230.84280936455,-9.90301003344482)(1225.89966555184,-9.90301003344482)(1220.95652173913,-9.90301003344482)(1216.01337792642,-9.90301003344482)(1211.07023411371,-9.90301003344482)(1208.59866220736,-9.93311036789298)(1206.127090301,-9.96321070234114)(1201.18394648829,-9.96321070234114)(1196.24080267559,-9.96321070234114)(1191.29765886288,-9.96321070234114)(1186.35451505017,-9.96321070234114)(1181.41137123746,-9.96321070234114)(1176.46822742475,-9.96321070234114)(1171.52508361204,-9.96321070234114)(1166.58193979933,-9.96321070234114)(1161.63879598662,-9.96321070234114)(1156.69565217391,-9.96321070234114)(1151.7525083612,-9.96321070234114)(1146.8093645485,-9.96321070234114)(1141.86622073579,-9.96321070234114)(1136.92307692308,-9.96321070234114)(1131.97993311037,-9.96321070234114)(1127.03678929766,-9.96321070234114)(1122.09364548495,-9.96321070234114)(1117.15050167224,-9.96321070234114)(1112.20735785953,-9.96321070234114)(1107.26421404682,-9.96321070234114)(1102.32107023411,-9.96321070234114)(1097.3779264214,-9.96321070234114)(1092.4347826087,-9.96321070234114)(1087.49163879599,-9.96321070234114)(1082.54849498328,-9.96321070234114)(1077.60535117057,-9.96321070234114)(1072.66220735786,-9.96321070234114)(1067.71906354515,-9.96321070234114)(1062.77591973244,-9.96321070234114)(1057.83277591973,-9.96321070234114)(1052.88963210702,-9.96321070234114)(1050.41806020067,-9.9933110367893)(1047.94648829431,-10.0234113712375)(1043.00334448161,-10.0234113712375)(1038.0602006689,-10.0234113712375)(1035.58862876254,-9.9933110367893)(1033.11705685619,-9.96321070234114)(1028.17391304348,-9.96321070234114)(1023.23076923077,-9.96321070234114)(1018.28762541806,-9.96321070234114)(1013.34448160535,-9.96321070234114)(1008.40133779264,-9.96321070234114)(1003.45819397993,-9.96321070234114)(998.515050167224,-9.96321070234114)(993.571906354515,-9.96321070234114)(988.628762541806,-9.96321070234114)(983.685618729097,-9.96321070234114)(978.742474916388,-9.96321070234114)(973.799331103679,-9.96321070234114)(968.85618729097,-9.96321070234114)(963.913043478261,-9.96321070234114)(958.969899665552,-9.96321070234114)(954.026755852843,-9.96321070234114)(949.083612040134,-9.96321070234114)(946.612040133779,-9.93311036789298)(944.140468227425,-9.90301003344482)(939.197324414716,-9.90301003344482)(934.254180602007,-9.90301003344482)(929.311036789298,-9.90301003344482)(926.839464882943,-9.87290969899666)(924.367892976589,-9.8428093645485)(919.42474916388,-9.8428093645485)(916.953177257525,-9.81270903010033)(914.481605351171,-9.78260869565217)(909.538461538462,-9.78260869565217)(904.595317725752,-9.78260869565217)(902.123745819398,-9.75250836120401)(899.652173913044,-9.72240802675585)(894.709030100334,-9.72240802675585)(892.23745819398,-9.69230769230769)(889.765886287625,-9.66220735785953)(884.822742474916,-9.66220735785953)(882.351170568562,-9.63210702341137)(879.879598662207,-9.60200668896321)(877.408026755853,-9.57190635451505)(874.936454849498,-9.54180602006689)(869.993311036789,-9.54180602006689)(867.521739130435,-9.51170568561873)(865.05016722408,-9.48160535117057)(860.107023411371,-9.48160535117057)(857.635451505017,-9.45150501672241)(855.163879598662,-9.42140468227425)(852.692307692308,-9.39130434782609)(850.220735785953,-9.36120401337793)(845.277591973244,-9.36120401337793)(842.80602006689,-9.33110367892977)(840.334448160535,-9.30100334448161)(837.862876254181,-9.27090301003344)(835.391304347826,-9.24080267558528)(832.919732441472,-9.21070234113712)(830.448160535117,-9.18060200668896)(827.976588628763,-9.1505016722408)(825.505016722408,-9.12040133779264)(823.033444816054,-9.09030100334448)(820.561872909699,-9.06020066889632)(818.090301003345,-9.03010033444816)(815.61872909699,-9)(813.147157190635,-8.96989966555184)(810.675585284281,-8.93979933110368)(808.204013377926,-8.90969899665552)(808.204013377926,-8.8494983277592)(805.732441471572,-8.81939799331104)(803.260869565217,-8.78929765886288)(803.260869565217,-8.72909698996656)(803.260869565217,-8.66889632107023)(803.260869565217,-8.60869565217391)(803.260869565217,-8.54849498327759)(800.789297658863,-8.51839464882943)(798.317725752508,-8.48829431438127)(798.317725752508,-8.42809364548495)(800.789297658863,-8.39799331103679)(803.260869565217,-8.36789297658863)(803.260869565217,-8.30769230769231)(803.260869565217,-8.24749163879599)(805.732441471572,-8.21739130434783)(808.204013377926,-8.18729096989967)(808.204013377926,-8.12709030100334)(810.675585284281,-8.09698996655519)(813.147157190635,-8.06688963210702)(815.61872909699,-8.03678929765886)(818.090301003345,-8.0066889632107)(820.561872909699,-7.97658862876254)(825.505016722408,-7.97658862876254)(827.976588628763,-7.94648829431438)(830.448160535117,-7.91638795986622)(835.391304347826,-7.91638795986622)(840.334448160535,-7.91638795986622)(842.80602006689,-7.88628762541806)(845.277591973244,-7.8561872909699)(847.749163879599,-7.82608695652174)(850.220735785953,-7.79598662207358)(855.163879598662,-7.79598662207358)(860.107023411371,-7.79598662207358)(865.05016722408,-7.79598662207358)(869.993311036789,-7.79598662207358)(874.936454849498,-7.79598662207358)(879.879598662207,-7.79598662207358)(884.822742474916,-7.79598662207358)(889.765886287625,-7.79598662207358)(894.709030100334,-7.79598662207358)(899.652173913044,-7.79598662207358)(904.595317725752,-7.79598662207358)(909.538461538462,-7.79598662207358)(914.481605351171,-7.79598662207358)(919.42474916388,-7.79598662207358)(924.367892976589,-7.79598662207358)(926.839464882943,-7.82608695652174)(929.311036789298,-7.8561872909699)(934.254180602007,-7.8561872909699)(939.197324414716,-7.8561872909699)(944.140468227425,-7.8561872909699)(949.083612040134,-7.8561872909699)(954.026755852843,-7.8561872909699)(958.969899665552,-7.8561872909699)(963.913043478261,-7.8561872909699)(968.85618729097,-7.8561872909699)(973.799331103679,-7.8561872909699)(978.742474916388,-7.8561872909699)(983.685618729097,-7.8561872909699)(988.628762541806,-7.8561872909699)(993.571906354515,-7.8561872909699)(998.515050167224,-7.8561872909699)(1003.45819397993,-7.8561872909699)(1008.40133779264,-7.8561872909699)(1013.34448160535,-7.8561872909699)(1018.28762541806,-7.8561872909699)(1023.23076923077,-7.8561872909699)(1025.70234113712,-7.82608695652174)(1028.17391304348,-7.79598662207358)(1033.11705685619,-7.79598662207358)(1038.0602006689,-7.79598662207358)(1043.00334448161,-7.79598662207358)(1047.94648829431,-7.79598662207358)(1052.88963210702,-7.79598662207358)(1055.36120401338,-7.82608695652174)(1057.83277591973,-7.8561872909699)(1062.77591973244,-7.8561872909699)(1067.71906354515,-7.8561872909699)(1072.66220735786,-7.8561872909699)(1077.60535117057,-7.8561872909699)(1082.54849498328,-7.8561872909699)(1087.49163879599,-7.8561872909699)(1089.96321070234,-7.88628762541806)(1092.4347826087,-7.91638795986622)(1097.3779264214,-7.91638795986622)(1102.32107023411,-7.91638795986622)(1107.26421404682,-7.91638795986622)(1112.20735785953,-7.91638795986622)(1117.15050167224,-7.91638795986622)(1119.6220735786,-7.94648829431438)(1122.09364548495,-7.97658862876254)(1127.03678929766,-7.97658862876254)(1131.97993311037,-7.97658862876254)(1136.92307692308,-7.97658862876254)(1141.86622073579,-7.97658862876254)(1146.8093645485,-7.97658862876254)(1151.7525083612,-7.97658862876254)(1154.22408026756,-7.94648829431438)(1156.69565217391,-7.91638795986622)(1161.63879598662,-7.91638795986622)(1166.58193979933,-7.91638795986622)(1171.52508361204,-7.91638795986622)(1176.46822742475,-7.91638795986622)(1181.41137123746,-7.91638795986622)(1186.35451505017,-7.91638795986622)(1188.82608695652,-7.88628762541806)(1191.29765886288,-7.8561872909699)(1196.24080267559,-7.8561872909699)(1201.18394648829,-7.8561872909699)(1206.127090301,-7.8561872909699)(1211.07023411371,-7.8561872909699)(1216.01337792642,-7.8561872909699)(1220.95652173913,-7.8561872909699)(1225.89966555184,-7.8561872909699)(1230.84280936455,-7.8561872909699)(1235.78595317726,-7.8561872909699)(1240.72909698997,-7.8561872909699)(1245.67224080268,-7.8561872909699)(1250.61538461538,-7.8561872909699)(1255.55852842809,-7.8561872909699)(1258.03010033445,-7.88628762541806)(1260.5016722408,-7.91638795986622)(1265.44481605351,-7.91638795986622)(1270.38795986622,-7.91638795986622)(1272.85953177258,-7.94648829431438)(1275.33110367893,-7.97658862876254)(1280.27424749164,-7.97658862876254)(1285.21739130435,-7.97658862876254)(1287.6889632107,-8.0066889632107)(1290.16053511706,-8.03678929765886)(1295.10367892977,-8.03678929765886)(1297.57525083612,-8.06688963210702)(1300.04682274247,-8.09698996655519)(1304.98996655518,-8.09698996655519)(1307.46153846154,-8.12709030100334)(1309.93311036789,-8.1571906354515)(1314.8762541806,-8.1571906354515)(1317.34782608696,-8.18729096989967)(1319.81939799331,-8.21739130434783)(1324.76254180602,-8.21739130434783)(1327.23411371237,-8.24749163879599)(1329.70568561873,-8.27759197324415)(1332.17725752508,-8.30769230769231)(1334.64882943144,-8.33779264214047)(1339.59197324415,-8.33779264214047)(1342.0635451505,-8.36789297658863)(1344.53511705686,-8.39799331103679)(1347.00668896321,-8.42809364548495)(1349.47826086957,-8.45819397993311)(1351.94983277592,-8.48829431438127)(1354.42140468227,-8.51839464882943)(1356.89297658863,-8.54849498327759)(1359.36454849498,-8.57859531772575)(1361.83612040134,-8.60869565217391)(1364.30769230769,-8.63879598662207)(1366.77926421405,-8.66889632107023)(1369.2508361204,-8.69899665551839)(1371.72240802676,-8.72909698996656)(1374.19397993311,-8.75919732441472)(1376.66555183947,-8.78929765886288)(1376.66555183947,-8.8494983277592)(1379.13712374582,-8.87959866220736)(1381.60869565217,-8.90969899665552)(1384.08026755853,-8.93979933110368)(1386.55183946488,-8.96989966555184)(1389.02341137124,-9)(1391.49498327759,-9.03010033444816)(1391.49498327759,-9.09030100334448)(1393.96655518395,-9.12040133779264)(1396.4381270903,-9.1505016722408)(1396.4381270903,-9.21070234113712)(1396.4381270903,-9.27090301003344)(1396.4381270903,-9.33110367892977)(1398.90969899666,-9.36120401337793)};

\addplot [fill=darkgray,draw=black,forget plot] coordinates{ (0,-6.02006688963211)(0,-6.08026755852843)(4.94314381270903,-6.14046822742475)(4.94314381270903,-6.20066889632107)(9.88628762541806,-6.26086956521739)(9.88628762541806,-6.32107023411371)(14.8294314381271,-6.38127090301003)(19.7725752508361,-6.44147157190636)(19.7725752508361,-6.50167224080267)(24.7157190635452,-6.561872909699)(29.6588628762542,-6.62207357859532)(29.6588628762542,-6.68227424749164)(34.6020066889632,-6.74247491638796)(39.5451505016722,-6.80267558528428)(44.4882943143813,-6.8628762541806)(49.4314381270903,-6.92307692307692)(54.3745819397993,-6.98327759197324)(59.3177257525084,-6.98327759197324)(64.2608695652174,-7.04347826086956)(69.2040133779264,-7.10367892976589)(74.1471571906355,-7.10367892976589)(79.0903010033445,-7.16387959866221)(84.0334448160535,-7.16387959866221)(88.9765886287625,-7.22408026755853)(93.9197324414716,-7.22408026755853)(98.8628762541806,-7.28428093645485)(103.80602006689,-7.28428093645485)(108.749163879599,-7.28428093645485)(113.692307692308,-7.28428093645485)(118.635451505017,-7.28428093645485)(123.578595317726,-7.34448160535117)(128.521739130435,-7.34448160535117)(133.464882943144,-7.34448160535117)(138.408026755853,-7.28428093645485)(143.351170568562,-7.28428093645485)(148.294314381271,-7.28428093645485)(153.23745819398,-7.28428093645485)(158.180602006689,-7.28428093645485)(163.123745819398,-7.28428093645485)(168.066889632107,-7.22408026755853)(173.010033444816,-7.22408026755853)(177.953177257525,-7.22408026755853)(182.896321070234,-7.22408026755853)(187.839464882943,-7.28428093645485)(192.782608695652,-7.28428093645485)(197.725752508361,-7.34448160535117)(202.66889632107,-7.34448160535117)(207.612040133779,-7.34448160535117)(212.555183946488,-7.34448160535117)(217.498327759197,-7.34448160535117)(222.441471571906,-7.40468227424749)(227.384615384615,-7.40468227424749)(232.327759197324,-7.40468227424749)(237.270903010033,-7.46488294314381)(242.214046822742,-7.46488294314381)(247.157190635452,-7.46488294314381)(252.100334448161,-7.52508361204013)(257.04347826087,-7.52508361204013)(261.986622073579,-7.52508361204013)(266.929765886288,-7.52508361204013)(271.872909698997,-7.58528428093645)(276.816053511706,-7.58528428093645)(281.759197324415,-7.58528428093645)(286.702341137124,-7.64548494983278)(291.645484949833,-7.64548494983278)(296.588628762542,-7.64548494983278)(301.531772575251,-7.64548494983278)(306.47491638796,-7.7056856187291)(311.418060200669,-7.7056856187291)(316.361204013378,-7.7056856187291)(321.304347826087,-7.7056856187291)(326.247491638796,-7.7056856187291)(331.190635451505,-7.7056856187291)(336.133779264214,-7.7056856187291)(341.076923076923,-7.7056856187291)(346.020066889632,-7.64548494983278)(350.963210702341,-7.64548494983278)(355.90635451505,-7.58528428093645)(360.849498327759,-7.52508361204013)(365.792642140468,-7.46488294314381)(365.792642140468,-7.40468227424749)(370.735785953177,-7.34448160535117)(375.678929765886,-7.28428093645485)(380.622073578595,-7.22408026755853)(385.565217391304,-7.16387959866221)(390.508361204013,-7.10367892976589)(395.451505016722,-7.04347826086956)(400.394648829431,-7.04347826086956)(405.33779264214,-6.98327759197324)(410.28093645485,-6.92307692307692)(415.224080267559,-6.92307692307692)(420.167224080268,-6.92307692307692)(425.110367892977,-6.8628762541806)(430.053511705686,-6.8628762541806)(434.996655518395,-6.8628762541806)(439.939799331104,-6.80267558528428)(444.882943143813,-6.80267558528428)(449.826086956522,-6.80267558528428)(454.769230769231,-6.74247491638796)(459.71237458194,-6.74247491638796)(464.655518394649,-6.74247491638796)(469.598662207358,-6.74247491638796)(474.541806020067,-6.74247491638796)(479.484949832776,-6.74247491638796)(484.428093645485,-6.68227424749164)(489.371237458194,-6.68227424749164)(494.314381270903,-6.68227424749164)(499.257525083612,-6.68227424749164)(504.200668896321,-6.68227424749164)(509.14381270903,-6.68227424749164)(514.086956521739,-6.68227424749164)(519.030100334448,-6.68227424749164)(523.973244147157,-6.68227424749164)(528.916387959866,-6.68227424749164)(533.859531772575,-6.68227424749164)(538.802675585284,-6.68227424749164)(543.745819397993,-6.68227424749164)(548.688963210702,-6.68227424749164)(553.632107023411,-6.74247491638796)(558.57525083612,-6.74247491638796)(563.518394648829,-6.80267558528428)(568.461538461538,-6.80267558528428)(573.404682274248,-6.8628762541806)(578.347826086957,-6.92307692307692)(583.290969899666,-6.98327759197324)(588.234113712375,-7.04347826086956)(593.177257525084,-7.10367892976589)(598.120401337793,-7.16387959866221)(603.063545150502,-7.16387959866221)(608.006688963211,-7.22408026755853)(612.94983277592,-7.28428093645485)(617.892976588629,-7.34448160535117)(622.836120401338,-7.40468227424749)(627.779264214047,-7.46488294314381)(632.722408026756,-7.52508361204013)(637.665551839465,-7.58528428093645)(637.665551839465,-7.64548494983278)(642.608695652174,-7.7056856187291)(647.551839464883,-7.76588628762542)(647.551839464883,-7.82608695652174)(647.551839464883,-7.88628762541806)(652.494983277592,-7.94648829431438)(652.494983277592,-8.0066889632107)(647.551839464883,-8.06688963210702)(647.551839464883,-8.12709030100334)(647.551839464883,-8.18729096989967)(642.608695652174,-8.24749163879599)(642.608695652174,-8.30769230769231)(637.665551839465,-8.36789297658863)(632.722408026756,-8.42809364548495)(627.779264214047,-8.48829431438127)(622.836120401338,-8.54849498327759)(617.892976588629,-8.60869565217391)(612.94983277592,-8.60869565217391)(608.006688963211,-8.66889632107023)(603.063545150502,-8.66889632107023)(598.120401337793,-8.72909698996656)(593.177257525084,-8.72909698996656)(588.234113712375,-8.78929765886288)(583.290969899666,-8.78929765886288)(578.347826086957,-8.8494983277592)(573.404682274248,-8.8494983277592)(568.461538461538,-8.90969899665552)(563.518394648829,-8.90969899665552)(558.57525083612,-8.96989966555184)(553.632107023411,-8.96989966555184)(548.688963210702,-9.03010033444816)(543.745819397993,-9.03010033444816)(538.802675585284,-9.03010033444816)(533.859531772575,-9.09030100334448)(528.916387959866,-9.09030100334448)(523.973244147157,-9.09030100334448)(519.030100334448,-9.09030100334448)(514.086956521739,-9.09030100334448)(509.14381270903,-9.09030100334448)(504.200668896321,-9.09030100334448)(499.257525083612,-9.09030100334448)(494.314381270903,-9.03010033444816)(489.371237458194,-9.03010033444816)(484.428093645485,-9.03010033444816)(479.484949832776,-8.96989966555184)(474.541806020067,-8.96989966555184)(469.598662207358,-8.90969899665552)(464.655518394649,-8.90969899665552)(459.71237458194,-8.8494983277592)(454.769230769231,-8.8494983277592)(449.826086956522,-8.8494983277592)(444.882943143813,-8.78929765886288)(439.939799331104,-8.78929765886288)(434.996655518395,-8.78929765886288)(430.053511705686,-8.8494983277592)(425.110367892977,-8.78929765886288)(420.167224080268,-8.72909698996656)(415.224080267559,-8.66889632107023)(410.28093645485,-8.60869565217391)(405.33779264214,-8.54849498327759)(400.394648829431,-8.54849498327759)(395.451505016722,-8.48829431438127)(390.508361204013,-8.42809364548495)(385.565217391304,-8.36789297658863)(380.622073578595,-8.36789297658863)(375.678929765886,-8.30769230769231)(370.735785953177,-8.30769230769231)(365.792642140468,-8.24749163879599)(360.849498327759,-8.24749163879599)(355.90635451505,-8.30769230769231)(350.963210702341,-8.30769230769231)(346.020066889632,-8.36789297658863)(341.076923076923,-8.42809364548495)(336.133779264214,-8.48829431438127)(331.190635451505,-8.54849498327759)(326.247491638796,-8.60869565217391)(321.304347826087,-8.66889632107023)(316.361204013378,-8.72909698996656)(311.418060200669,-8.78929765886288)(306.47491638796,-8.8494983277592)(301.531772575251,-8.8494983277592)(296.588628762542,-8.90969899665552)(291.645484949833,-8.90969899665552)(286.702341137124,-8.96989966555184)(281.759197324415,-8.96989966555184)(276.816053511706,-8.96989966555184)(271.872909698997,-8.96989966555184)(266.929765886288,-8.96989966555184)(261.986622073579,-8.96989966555184)(257.04347826087,-8.96989966555184)(252.100334448161,-8.96989966555184)(247.157190635452,-8.96989966555184)(242.214046822742,-8.96989966555184)(237.270903010033,-8.96989966555184)(232.327759197324,-8.96989966555184)(227.384615384615,-8.96989966555184)(222.441471571906,-8.96989966555184)(217.498327759197,-8.96989966555184)(212.555183946488,-8.96989966555184)(207.612040133779,-8.96989966555184)(202.66889632107,-8.96989966555184)(197.725752508361,-8.96989966555184)(192.782608695652,-9.03010033444816)(187.839464882943,-9.03010033444816)(182.896321070234,-9.03010033444816)(177.953177257525,-9.03010033444816)(173.010033444816,-9.03010033444816)(168.066889632107,-9.03010033444816)(163.123745819398,-9.03010033444816)(158.180602006689,-9.03010033444816)(153.23745819398,-9.03010033444816)(148.294314381271,-8.96989966555184)(143.351170568562,-8.96989966555184)(138.408026755853,-8.96989966555184)(133.464882943144,-8.90969899665552)(128.521739130435,-8.90969899665552)(123.578595317726,-8.8494983277592)(118.635451505017,-8.8494983277592)(113.692307692308,-8.78929765886288)(108.749163879599,-8.78929765886288)(103.80602006689,-8.72909698996656)(98.8628762541806,-8.72909698996656)(93.9197324414716,-8.72909698996656)(88.9765886287625,-8.66889632107023)(84.0334448160535,-8.66889632107023)(79.0903010033445,-8.66889632107023)(74.1471571906355,-8.66889632107023)(69.2040133779264,-8.66889632107023)(64.2608695652174,-8.66889632107023)(59.3177257525084,-8.66889632107023)(54.3745819397993,-8.66889632107023)(49.4314381270903,-8.72909698996656)(44.4882943143813,-8.72909698996656)(39.5451505016722,-8.72909698996656)(34.6020066889632,-8.78929765886288)(29.6588628762542,-8.78929765886288)(24.7157190635452,-8.8494983277592)(19.7725752508361,-8.90969899665552)(14.8294314381271,-8.96989966555184)(9.88628762541806,-8.96989966555184)(4.94314381270903,-9.03010033444816)(0,-9.09030100334448)(0,-9.03010033444816)(0,-8.96989966555184)(0,-8.90969899665552)(0,-8.8494983277592)(0,-8.78929765886288)(0,-8.72909698996656)(0,-8.66889632107023)(0,-8.60869565217391)(0,-8.54849498327759)(0,-8.48829431438127)(0,-8.42809364548495)(0,-8.36789297658863)(0,-8.30769230769231)(0,-8.24749163879599)(0,-8.18729096989967)(0,-8.12709030100334)(0,-8.06688963210702)(0,-8.0066889632107)(0,-7.94648829431438)(0,-7.88628762541806)(0,-7.82608695652174)(0,-7.76588628762542)(0,-7.7056856187291)(0,-7.64548494983278)(0,-7.58528428093645)(0,-7.52508361204013)(0,-7.46488294314381)(0,-7.40468227424749)(0,-7.34448160535117)(0,-7.28428093645485)(0,-7.22408026755853)(0,-7.16387959866221)(0,-7.10367892976589)(0,-7.04347826086956)(0,-6.98327759197324)(0,-6.92307692307692)(0,-6.8628762541806)(0,-6.80267558528428)(0,-6.74247491638796)(0,-6.68227424749164)(0,-6.62207357859532)(0,-6.561872909699)(0,-6.50167224080267)(0,-6.44147157190636)(0,-6.38127090301003)(0,-6.32107023411371)(0,-6.26086956521739)(0,-6.20066889632107)(0,-6.14046822742475)(0,-6.08026755852843)(0,-6.02006688963211)};

\addplot [fill=red!40!yellow,draw=black,forget plot] coordinates{ (469.598662207358,-7.8561872909699)(472.070234113712,-7.82608695652174)(472.070234113712,-7.76588628762542)(474.541806020067,-7.73578595317726)(477.013377926421,-7.7056856187291)(479.484949832776,-7.67558528428094)(481.95652173913,-7.64548494983278)(484.428093645485,-7.61538461538462)(489.371237458194,-7.61538461538462)(491.842809364549,-7.58528428093645)(494.314381270903,-7.55518394648829)(499.257525083612,-7.55518394648829)(501.729096989967,-7.52508361204013)(504.200668896321,-7.49498327759197)(509.14381270903,-7.49498327759197)(514.086956521739,-7.49498327759197)(519.030100334448,-7.49498327759197)(521.501672240803,-7.46488294314381)(523.973244147157,-7.43478260869565)(528.916387959866,-7.43478260869565)(533.859531772575,-7.43478260869565)(538.802675585284,-7.43478260869565)(543.745819397993,-7.43478260869565)(548.688963210702,-7.43478260869565)(553.632107023411,-7.43478260869565)(558.57525083612,-7.43478260869565)(563.518394648829,-7.43478260869565)(565.989966555184,-7.46488294314381)(568.461538461538,-7.49498327759197)(573.404682274248,-7.49498327759197)(578.347826086957,-7.49498327759197)(580.819397993311,-7.52508361204013)(583.290969899666,-7.55518394648829)(585.76254180602,-7.58528428093645)(588.234113712375,-7.61538461538462)(590.705685618729,-7.64548494983278)(590.705685618729,-7.7056856187291)(593.177257525084,-7.73578595317726)(595.648829431438,-7.76588628762542)(595.648829431438,-7.82608695652174)(595.648829431438,-7.88628762541806)(595.648829431438,-7.94648829431438)(595.648829431438,-8.0066889632107)(593.177257525084,-8.03678929765886)(590.705685618729,-8.06688963210702)(590.705685618729,-8.12709030100334)(590.705685618729,-8.18729096989967)(588.234113712375,-8.21739130434783)(585.76254180602,-8.24749163879599)(585.76254180602,-8.30769230769231)(583.290969899666,-8.33779264214047)(580.819397993311,-8.36789297658863)(580.819397993311,-8.42809364548495)(578.347826086957,-8.45819397993311)(575.876254180602,-8.48829431438127)(573.404682274248,-8.51839464882943)(570.933110367893,-8.54849498327759)(568.461538461538,-8.57859531772575)(565.989966555184,-8.60869565217391)(563.518394648829,-8.63879598662207)(561.046822742475,-8.66889632107023)(558.57525083612,-8.69899665551839)(556.103678929766,-8.72909698996656)(553.632107023411,-8.75919732441472)(551.160535117057,-8.78929765886288)(548.688963210702,-8.81939799331104)(543.745819397993,-8.81939799331104)(541.274247491639,-8.8494983277592)(538.802675585284,-8.87959866220736)(533.859531772575,-8.87959866220736)(531.387959866221,-8.90969899665552)(528.916387959866,-8.93979933110368)(523.973244147157,-8.93979933110368)(519.030100334448,-8.93979933110368)(514.086956521739,-8.93979933110368)(509.14381270903,-8.93979933110368)(504.200668896321,-8.93979933110368)(499.257525083612,-8.93979933110368)(496.785953177258,-8.90969899665552)(494.314381270903,-8.87959866220736)(489.371237458194,-8.87959866220736)(486.899665551839,-8.8494983277592)(484.428093645485,-8.81939799331104)(481.95652173913,-8.78929765886288)(481.95652173913,-8.72909698996656)(479.484949832776,-8.69899665551839)(477.013377926421,-8.66889632107023)(477.013377926421,-8.60869565217391)(474.541806020067,-8.57859531772575)(472.070234113712,-8.54849498327759)(472.070234113712,-8.48829431438127)(472.070234113712,-8.42809364548495)(469.598662207358,-8.39799331103679)(467.127090301003,-8.36789297658863)(467.127090301003,-8.30769230769231)(467.127090301003,-8.24749163879599)(467.127090301003,-8.18729096989967)(467.127090301003,-8.12709030100334)(467.127090301003,-8.06688963210702)(467.127090301003,-8.0066889632107)(467.127090301003,-7.94648829431438)(467.127090301003,-7.88628762541806)(469.598662207358,-7.8561872909699)};

\addplot [fill=red!40!yellow,draw=black,forget plot] coordinates{ (192.782608695652,-8.21739130434783)(195.254180602007,-8.18729096989967)(197.725752508361,-8.1571906354515)(202.66889632107,-8.1571906354515)(207.612040133779,-8.1571906354515)(210.083612040134,-8.18729096989967)(212.555183946488,-8.21739130434783)(217.498327759197,-8.21739130434783)(222.441471571906,-8.21739130434783)(224.913043478261,-8.24749163879599)(227.384615384615,-8.27759197324415)(229.85618729097,-8.30769230769231)(229.85618729097,-8.36789297658863)(227.384615384615,-8.39799331103679)(224.913043478261,-8.42809364548495)(222.441471571906,-8.45819397993311)(217.498327759197,-8.45819397993311)(212.555183946488,-8.45819397993311)(207.612040133779,-8.45819397993311)(202.66889632107,-8.45819397993311)(200.197324414716,-8.42809364548495)(197.725752508361,-8.39799331103679)(192.782608695652,-8.39799331103679)(190.311036789298,-8.36789297658863)(190.311036789298,-8.30769230769231)(190.311036789298,-8.24749163879599)(192.782608695652,-8.21739130434783)};

\addplot [fill=darkgray,draw=black,forget plot] coordinates{ (1478,-17.819397993311)(1478,-17.8795986622074)(1478,-17.9397993311037)(1478,-18)(1473.05685618729,-18)(1468.11371237458,-18)(1463.17056856187,-18)(1458.22742474916,-18)(1463.17056856187,-18)(1468.11371237458,-17.9397993311037)(1473.05685618729,-17.8795986622074)(1478,-17.819397993311)};

\addplot [fill=darkgray,draw=black,forget plot] coordinates{ (0,-17.9698996655518)(2.47157190635452,-18)(0,-18.0000030097325)(-0.000247132477387754,-18)(0,-17.9698996655518)};

\addplot [
color=white,
draw=white,
only marks,
mark=x,
mark options={solid},
mark size=2.0pt,
line width=0.3pt,
forget plot
]
coordinates{
  (10.5571428571429,0)(21.1142857142857,0)(31.6714285714286,0)(42.2285714285714,0)(52.7857142857143,0)(63.3428571428571,0)(73.9,0)(84.4571428571429,0)(95.0142857142857,0)(105.571428571429,0)(116.128571428571,-0.663157894736842)(126.685714285714,-1.32631578947368)(137.242857142857,-1.98947368421053)(147.8,-2.65263157894737)(158.357142857143,-3.31578947368421)(168.914285714286,-3.97894736842105)(179.471428571429,-4.64210526315789)(190.028571428571,-5.30526315789474)(200.585714285714,-5.96842105263158)(211.142857142857,-6.63157894736842)(221.7,-7.76842105263158)(232.257142857143,-8.90526315789474)(242.814285714286,-10.0421052631579)(253.371428571429,-11.1789473684211)(263.928571428571,-12.3157894736842)(274.485714285714,-13.4526315789474)(285.042857142857,-14.5894736842105)(295.6,-15.7263157894737)(306.157142857143,-16.8631578947368)(316.714285714286,-18)(327.271428571429,-17.1473684210526)(337.828571428571,-16.2947368421053)(348.385714285714,-15.4421052631579)(358.942857142857,-14.5894736842105)(369.5,-13.7368421052632)(380.057142857143,-12.8842105263158)(390.614285714286,-12.0315789473684)(401.171428571429,-11.1789473684211)(411.728571428571,-10.3263157894737)(422.285714285714,-9.47368421052632)(432.842857142857,-8.52631578947368)(443.4,-7.57894736842105)(453.957142857143,-6.63157894736842)(464.514285714286,-5.68421052631579)(475.071428571429,-4.73684210526316)(485.628571428571,-3.78947368421053)(496.185714285714,-2.84210526315789)(506.742857142857,-1.89473684210526)(517.3,-0.947368421052632)(527.857142857143,0)(538.414285714286,-0.947368421052632)(548.971428571429,-1.89473684210526)(559.528571428571,-2.84210526315789)(570.085714285714,-3.78947368421053)(580.642857142857,-4.73684210526316)(591.2,-5.68421052631579)(601.757142857143,-6.63157894736842)(612.314285714286,-7.57894736842105)(622.871428571429,-8.52631578947368)(633.428571428571,-9.47368421052632)(643.985714285714,-10.3263157894737)(654.542857142857,-11.1789473684211)(665.1,-12.0315789473684)(675.657142857143,-12.8842105263158)(686.214285714286,-13.7368421052632)(696.771428571429,-14.5894736842105)(707.328571428571,-15.4421052631579)(717.885714285714,-16.2947368421053)(728.442857142857,-17.1473684210526)(739,-18)(749.557142857143,-16.9578947368421)(760.114285714286,-15.9157894736842)(770.671428571429,-14.8736842105263)(781.228571428571,-13.8315789473684)(791.785714285714,-12.7894736842105)(802.342857142857,-11.7473684210526)(812.9,-10.7052631578947)(823.457142857143,-9.66315789473684)(834.014285714286,-8.62105263157895)(844.571428571429,-7.57894736842105)(855.128571428571,-6.82105263157895)(865.685714285714,-6.06315789473684)(876.242857142857,-5.30526315789474)(886.8,-4.54736842105263)(897.357142857143,-3.78947368421053)(907.914285714286,-3.03157894736842)(918.471428571429,-2.27368421052632)(929.028571428571,-1.51578947368421)(939.585714285714,-0.757894736842106)(950.142857142857,0)(960.7,-0.852631578947369)(971.257142857143,-1.70526315789474)(981.814285714286,-2.55789473684211)(992.371428571429,-3.41052631578947)(1002.92857142857,-4.26315789473684)(1013.48571428571,-5.11578947368421)(1024.04285714286,-5.96842105263158)(1034.6,-6.82105263157895)(1045.15714285714,-7.67368421052632)(1055.71428571429,-8.52631578947369)(1066.27142857143,-9.47368421052632)(1076.82857142857,-10.4210526315789)(1087.38571428571,-11.3684210526316)(1097.94285714286,-12.3157894736842)(1108.5,-13.2631578947368)(1119.05714285714,-14.2105263157895)(1129.61428571429,-15.1578947368421)(1140.17142857143,-16.1052631578947)(1150.72857142857,-17.0526315789474)(1161.28571428571,-18)(1171.84285714286,-17.2421052631579)(1182.4,-16.4842105263158)(1192.95714285714,-15.7263157894737)(1203.51428571429,-14.9684210526316)(1214.07142857143,-14.2105263157895)(1224.62857142857,-13.4526315789474)(1235.18571428571,-12.6947368421053)(1245.74285714286,-11.9368421052632)(1256.3,-11.1789473684211)(1266.85714285714,-10.4210526315789)(1277.41428571429,-9.37894736842105)(1287.97142857143,-8.33684210526316)(1298.52857142857,-7.29473684210526)(1309.08571428571,-6.25263157894737)(1319.64285714286,-5.21052631578947)(1330.2,-4.16842105263158)(1340.75714285714,-3.12631578947368)(1351.31428571429,-2.08421052631579)(1361.87142857143,-1.04210526315789)(1372.42857142857,0)(1382.98571428571,-1.42105263157895)(1393.54285714286,-2.84210526315789)(1404.1,-4.26315789473684)(1414.65714285714,-5.68421052631579)(1425.21428571429,-7.10526315789474)(1435.77142857143,-8.52631578947369)(1446.32857142857,-9.94736842105263)(1456.88571428571,-11.3684210526316)(1467.44285714286,-12.7894736842105)(1478,-14.2105263157895)(1467.44285714286,-13.7368421052632)(1456.88571428571,-13.2631578947368)(1446.32857142857,-12.7894736842105)(1435.77142857143,-12.3157894736842)(1425.21428571429,-11.8421052631579)(1414.65714285714,-11.3684210526316)(1404.1,-10.8947368421053)(1393.54285714286,-10.4210526315789)(1382.98571428571,-9.94736842105263)(1372.42857142857,-9.47368421052632)(1361.87142857143,-9.18947368421053)(1351.31428571429,-8.90526315789474)(1340.75714285714,-8.62105263157895)(1330.2,-8.33684210526316)(1319.64285714286,-8.05263157894737)(1309.08571428571,-7.76842105263158)(1298.52857142857,-7.48421052631579)(1287.97142857143,-7.2)(1277.41428571429,-6.91578947368421)(1266.85714285714,-6.63157894736842)(1256.3,-7.01052631578947)(1245.74285714286,-7.38947368421053)(1235.18571428571,-7.76842105263158)(1224.62857142857,-8.14736842105263)(1214.07142857143,-8.52631578947369)(1203.51428571429,-8.90526315789474)(1192.95714285714,-9.28421052631579)(1182.4,-9.66315789473684)(1171.84285714286,-10.0421052631579)(1161.28571428571,-10.4210526315789)(1150.72857142857,-10.4210526315789)(1140.17142857143,-10.4210526315789)(1129.61428571429,-10.4210526315789)(1119.05714285714,-10.4210526315789)(1108.5,-10.4210526315789)(1097.94285714286,-10.4210526315789)(1087.38571428571,-10.4210526315789)(1076.82857142857,-10.4210526315789)(1066.27142857143,-10.4210526315789)(1055.71428571429,-10.4210526315789)(1045.15714285714,-10.4210526315789)(1034.6,-10.4210526315789)(1024.04285714286,-10.4210526315789)(1013.48571428571,-10.4210526315789)(1002.92857142857,-10.4210526315789)(992.371428571429,-10.4210526315789)(981.814285714286,-10.4210526315789)(971.257142857143,-10.4210526315789)(960.7,-10.4210526315789)(950.142857142857,-10.4210526315789)(939.585714285714,-10.0421052631579)(929.028571428571,-9.66315789473684)(918.471428571429,-9.28421052631579)(907.914285714286,-8.90526315789474)(897.357142857143,-8.52631578947368)(886.8,-8.14736842105263)(876.242857142857,-7.76842105263158)(865.685714285714,-7.38947368421053)(855.128571428571,-7.01052631578947)(844.571428571429,-6.63157894736842)(834.014285714286,-6.91578947368421)(823.457142857143,-7.2)(812.9,-7.48421052631579)(802.342857142857,-7.76842105263158)(791.785714285714,-8.05263157894737)(781.228571428571,-8.33684210526316)(770.671428571429,-8.62105263157895)(760.114285714286,-8.90526315789474)(749.557142857143,-9.18947368421053)(739,-9.47368421052632)(728.442857142857,-9.18947368421053)(717.885714285714,-8.90526315789474)(707.328571428571,-8.62105263157895)(696.771428571429,-8.33684210526316)(686.214285714286,-8.05263157894737)(675.657142857143,-7.76842105263158)(665.1,-7.48421052631579)(654.542857142857,-7.2)(643.985714285714,-6.91578947368421)(633.428571428571,-6.63157894736842)(622.871428571429,-6.82105263157895)(612.314285714286,-7.01052631578947)(601.757142857143,-7.2)(591.2,-7.38947368421053)(580.642857142857,-7.57894736842105)(570.085714285714,-7.76842105263158)(559.528571428571,-7.95789473684211)(548.971428571429,-8.14736842105263)(538.414285714286,-8.33684210526316)(527.857142857143,-8.52631578947369)(517.3,-8.71578947368421)(506.742857142857,-8.90526315789474)(496.185714285714,-9.09473684210526)(485.628571428571,-9.28421052631579)(475.071428571429,-9.47368421052632)(464.514285714286,-9.66315789473684)(453.957142857143,-9.85263157894737)(443.4,-10.0421052631579)(432.842857142857,-10.2315789473684)(422.285714285714,-10.4210526315789)(411.728571428571,-10.0421052631579)(401.171428571429,-9.66315789473684)(390.614285714286,-9.28421052631579)(380.057142857143,-8.90526315789474)(369.5,-8.52631578947368)(358.942857142857,-8.14736842105263)(348.385714285714,-7.76842105263158)(337.828571428571,-7.38947368421053)(327.271428571429,-7.01052631578947)(316.714285714286,-6.63157894736842)(306.157142857143,-6.91578947368421)(295.6,-7.2)(285.042857142857,-7.48421052631579)(274.485714285714,-7.76842105263158)(263.928571428571,-8.05263157894737)(253.371428571429,-8.33684210526316)(242.814285714286,-8.62105263157895)(232.257142857143,-8.90526315789474)(221.7,-9.18947368421053)(211.142857142857,-9.47368421052632)(200.585714285714,-9.18947368421053)(190.028571428571,-8.90526315789474)(179.471428571429,-8.62105263157895)(168.914285714286,-8.33684210526316)(158.357142857143,-8.05263157894737)(147.8,-7.76842105263158)(137.242857142857,-7.48421052631579)(126.685714285714,-7.2)(116.128571428571,-6.91578947368421)(105.571428571429,-6.63157894736842)(95.0142857142857,-7.76842105263158)(84.4571428571429,-8.90526315789474)(73.9,-10.0421052631579)(63.3428571428571,-11.1789473684211)(52.7857142857143,-12.3157894736842)(42.2285714285714,-13.4526315789474)(31.6714285714286,-14.5894736842105)(21.1142857142857,-15.7263157894737)(10.5571428571429,-16.8631578947368)(0,-18) 
};

\addplot [
mark size=0.8pt,
only marks,
mark=*,
mark options={solid,fill=black,draw=black},
forget plot
]
coordinates{
 (0,0)(0,-0.947368421052632)(0,-1.89473684210526)(0,-2.84210526315789)(0,-3.78947368421053)(0,-4.73684210526316)(0,-5.68421052631579)(0,-6.63157894736842)(0,-7.57894736842105)(0,-8.52631578947369)(0,-9.47368421052632)(0,-10.4210526315789)(0,-11.3684210526316)(0,-12.3157894736842)(0,-13.2631578947368)(0,-14.2105263157895)(0,-15.1578947368421)(0,-16.1052631578947)(0,-17.0526315789474)(0,-18)(105.571428571429,0)(105.571428571429,-0.947368421052632)(105.571428571429,-1.89473684210526)(105.571428571429,-2.84210526315789)(105.571428571429,-3.78947368421053)(105.571428571429,-4.73684210526316)(105.571428571429,-5.68421052631579)(105.571428571429,-6.63157894736842)(105.571428571429,-7.57894736842105)(105.571428571429,-8.52631578947369)(105.571428571429,-9.47368421052632)(105.571428571429,-10.4210526315789)(105.571428571429,-11.3684210526316)(105.571428571429,-12.3157894736842)(105.571428571429,-13.2631578947368)(105.571428571429,-14.2105263157895)(105.571428571429,-15.1578947368421)(105.571428571429,-16.1052631578947)(105.571428571429,-17.0526315789474)(105.571428571429,-18)(211.142857142857,0)(211.142857142857,-0.947368421052632)(211.142857142857,-1.89473684210526)(211.142857142857,-2.84210526315789)(211.142857142857,-3.78947368421053)(211.142857142857,-4.73684210526316)(211.142857142857,-5.68421052631579)(211.142857142857,-6.63157894736842)(211.142857142857,-7.57894736842105)(211.142857142857,-8.52631578947369)(211.142857142857,-9.47368421052632)(211.142857142857,-10.4210526315789)(211.142857142857,-11.3684210526316)(211.142857142857,-12.3157894736842)(211.142857142857,-13.2631578947368)(211.142857142857,-14.2105263157895)(211.142857142857,-15.1578947368421)(211.142857142857,-16.1052631578947)(211.142857142857,-17.0526315789474)(211.142857142857,-18)(316.714285714286,0)(316.714285714286,-0.947368421052632)(316.714285714286,-1.89473684210526)(316.714285714286,-2.84210526315789)(316.714285714286,-3.78947368421053)(316.714285714286,-4.73684210526316)(316.714285714286,-5.68421052631579)(316.714285714286,-6.63157894736842)(316.714285714286,-7.57894736842105)(316.714285714286,-8.52631578947369)(316.714285714286,-9.47368421052632)(316.714285714286,-10.4210526315789)(316.714285714286,-11.3684210526316)(316.714285714286,-12.3157894736842)(316.714285714286,-13.2631578947368)(316.714285714286,-14.2105263157895)(316.714285714286,-15.1578947368421)(316.714285714286,-16.1052631578947)(316.714285714286,-17.0526315789474)(316.714285714286,-18)(422.285714285714,0)(422.285714285714,-0.947368421052632)(422.285714285714,-1.89473684210526)(422.285714285714,-2.84210526315789)(422.285714285714,-3.78947368421053)(422.285714285714,-4.73684210526316)(422.285714285714,-5.68421052631579)(422.285714285714,-6.63157894736842)(422.285714285714,-7.57894736842105)(422.285714285714,-8.52631578947369)(422.285714285714,-9.47368421052632)(422.285714285714,-10.4210526315789)(422.285714285714,-11.3684210526316)(422.285714285714,-12.3157894736842)(422.285714285714,-13.2631578947368)(422.285714285714,-14.2105263157895)(422.285714285714,-15.1578947368421)(422.285714285714,-16.1052631578947)(422.285714285714,-17.0526315789474)(422.285714285714,-18)(527.857142857143,0)(527.857142857143,-0.947368421052632)(527.857142857143,-1.89473684210526)(527.857142857143,-2.84210526315789)(527.857142857143,-3.78947368421053)(527.857142857143,-4.73684210526316)(527.857142857143,-5.68421052631579)(527.857142857143,-6.63157894736842)(527.857142857143,-7.57894736842105)(527.857142857143,-8.52631578947369)(527.857142857143,-9.47368421052632)(527.857142857143,-10.4210526315789)(527.857142857143,-11.3684210526316)(527.857142857143,-12.3157894736842)(527.857142857143,-13.2631578947368)(527.857142857143,-14.2105263157895)(527.857142857143,-15.1578947368421)(527.857142857143,-16.1052631578947)(527.857142857143,-17.0526315789474)(527.857142857143,-18)(633.428571428571,0)(633.428571428571,-0.947368421052632)(633.428571428571,-1.89473684210526)(633.428571428571,-2.84210526315789)(633.428571428571,-3.78947368421053)(633.428571428571,-4.73684210526316)(633.428571428571,-5.68421052631579)(633.428571428571,-6.63157894736842)(633.428571428571,-7.57894736842105)(633.428571428571,-8.52631578947369)(633.428571428571,-9.47368421052632)(633.428571428571,-10.4210526315789)(633.428571428571,-11.3684210526316)(633.428571428571,-12.3157894736842)(633.428571428571,-13.2631578947368)(633.428571428571,-14.2105263157895)(633.428571428571,-15.1578947368421)(633.428571428571,-16.1052631578947)(633.428571428571,-17.0526315789474)(633.428571428571,-18)(739,0)(739,-0.947368421052632)(739,-1.89473684210526)(739,-2.84210526315789)(739,-3.78947368421053)(739,-4.73684210526316)(739,-5.68421052631579)(739,-6.63157894736842)(739,-7.57894736842105)(739,-8.52631578947369)(739,-9.47368421052632)(739,-10.4210526315789)(739,-11.3684210526316)(739,-12.3157894736842)(739,-13.2631578947368)(739,-14.2105263157895)(739,-15.1578947368421)(739,-16.1052631578947)(739,-17.0526315789474)(739,-18)(844.571428571429,0)(844.571428571429,-0.947368421052632)(844.571428571429,-1.89473684210526)(844.571428571429,-2.84210526315789)(844.571428571429,-3.78947368421053)(844.571428571429,-4.73684210526316)(844.571428571429,-5.68421052631579)(844.571428571429,-6.63157894736842)(844.571428571429,-7.57894736842105)(844.571428571429,-8.52631578947369)(844.571428571429,-9.47368421052632)(844.571428571429,-10.4210526315789)(844.571428571429,-11.3684210526316)(844.571428571429,-12.3157894736842)(844.571428571429,-13.2631578947368)(844.571428571429,-14.2105263157895)(844.571428571429,-15.1578947368421)(844.571428571429,-16.1052631578947)(844.571428571429,-17.0526315789474)(844.571428571429,-18)(950.142857142857,0)(950.142857142857,-0.947368421052632)(950.142857142857,-1.89473684210526)(950.142857142857,-2.84210526315789)(950.142857142857,-3.78947368421053)(950.142857142857,-4.73684210526316)(950.142857142857,-5.68421052631579)(950.142857142857,-6.63157894736842)(950.142857142857,-7.57894736842105)(950.142857142857,-8.52631578947369)(950.142857142857,-9.47368421052632)(950.142857142857,-10.4210526315789)(950.142857142857,-11.3684210526316)(950.142857142857,-12.3157894736842)(950.142857142857,-13.2631578947368)(950.142857142857,-14.2105263157895)(950.142857142857,-15.1578947368421)(950.142857142857,-16.1052631578947)(950.142857142857,-17.0526315789474)(950.142857142857,-18)(1055.71428571429,0)(1055.71428571429,-0.947368421052632)(1055.71428571429,-1.89473684210526)(1055.71428571429,-2.84210526315789)(1055.71428571429,-3.78947368421053)(1055.71428571429,-4.73684210526316)(1055.71428571429,-5.68421052631579)(1055.71428571429,-6.63157894736842)(1055.71428571429,-7.57894736842105)(1055.71428571429,-8.52631578947369)(1055.71428571429,-9.47368421052632)(1055.71428571429,-10.4210526315789)(1055.71428571429,-11.3684210526316)(1055.71428571429,-12.3157894736842)(1055.71428571429,-13.2631578947368)(1055.71428571429,-14.2105263157895)(1055.71428571429,-15.1578947368421)(1055.71428571429,-16.1052631578947)(1055.71428571429,-17.0526315789474)(1055.71428571429,-18)(1161.28571428571,0)(1161.28571428571,-0.947368421052632)(1161.28571428571,-1.89473684210526)(1161.28571428571,-2.84210526315789)(1161.28571428571,-3.78947368421053)(1161.28571428571,-4.73684210526316)(1161.28571428571,-5.68421052631579)(1161.28571428571,-6.63157894736842)(1161.28571428571,-7.57894736842105)(1161.28571428571,-8.52631578947369)(1161.28571428571,-9.47368421052632)(1161.28571428571,-10.4210526315789)(1161.28571428571,-11.3684210526316)(1161.28571428571,-12.3157894736842)(1161.28571428571,-13.2631578947368)(1161.28571428571,-14.2105263157895)(1161.28571428571,-15.1578947368421)(1161.28571428571,-16.1052631578947)(1161.28571428571,-17.0526315789474)(1161.28571428571,-18)(1266.85714285714,0)(1266.85714285714,-0.947368421052632)(1266.85714285714,-1.89473684210526)(1266.85714285714,-2.84210526315789)(1266.85714285714,-3.78947368421053)(1266.85714285714,-4.73684210526316)(1266.85714285714,-5.68421052631579)(1266.85714285714,-6.63157894736842)(1266.85714285714,-7.57894736842105)(1266.85714285714,-8.52631578947369)(1266.85714285714,-9.47368421052632)(1266.85714285714,-10.4210526315789)(1266.85714285714,-11.3684210526316)(1266.85714285714,-12.3157894736842)(1266.85714285714,-13.2631578947368)(1266.85714285714,-14.2105263157895)(1266.85714285714,-15.1578947368421)(1266.85714285714,-16.1052631578947)(1266.85714285714,-17.0526315789474)(1266.85714285714,-18)(1372.42857142857,0)(1372.42857142857,-0.947368421052632)(1372.42857142857,-1.89473684210526)(1372.42857142857,-2.84210526315789)(1372.42857142857,-3.78947368421053)(1372.42857142857,-4.73684210526316)(1372.42857142857,-5.68421052631579)(1372.42857142857,-6.63157894736842)(1372.42857142857,-7.57894736842105)(1372.42857142857,-8.52631578947369)(1372.42857142857,-9.47368421052632)(1372.42857142857,-10.4210526315789)(1372.42857142857,-11.3684210526316)(1372.42857142857,-12.3157894736842)(1372.42857142857,-13.2631578947368)(1372.42857142857,-14.2105263157895)(1372.42857142857,-15.1578947368421)(1372.42857142857,-16.1052631578947)(1372.42857142857,-17.0526315789474)(1372.42857142857,-18)(1478,0)(1478,-0.947368421052632)(1478,-1.89473684210526)(1478,-2.84210526315789)(1478,-3.78947368421053)(1478,-4.73684210526316)(1478,-5.68421052631579)(1478,-6.63157894736842)(1478,-7.57894736842105)(1478,-8.52631578947369)(1478,-9.47368421052632)(1478,-10.4210526315789)(1478,-11.3684210526316)(1478,-12.3157894736842)(1478,-13.2631578947368)(1478,-14.2105263157895)(1478,-15.1578947368421)(1478,-16.1052631578947)(1478,-17.0526315789474)(1478,-18)(1372.42857142857,0)(1372.42857142857,-0.947368421052632)(1372.42857142857,-1.89473684210526)(1372.42857142857,-2.84210526315789)(1372.42857142857,-3.78947368421053)(1372.42857142857,-4.73684210526316)(1372.42857142857,-5.68421052631579)(1372.42857142857,-6.63157894736842)(1372.42857142857,-7.57894736842105)(1372.42857142857,-8.52631578947369)(1372.42857142857,-9.47368421052632)(1372.42857142857,-10.4210526315789)(1372.42857142857,-11.3684210526316)(1372.42857142857,-12.3157894736842)(1372.42857142857,-13.2631578947368)(1372.42857142857,-14.2105263157895)(1372.42857142857,-15.1578947368421)(1372.42857142857,-16.1052631578947)(1372.42857142857,-17.0526315789474)(1372.42857142857,-18)(1266.85714285714,0)(1266.85714285714,-0.947368421052632)(1266.85714285714,-1.89473684210526)(1266.85714285714,-2.84210526315789)(1266.85714285714,-3.78947368421053)(1266.85714285714,-4.73684210526316)(1266.85714285714,-5.68421052631579)(1266.85714285714,-6.63157894736842)(1266.85714285714,-7.57894736842105)(1266.85714285714,-8.52631578947369)(1266.85714285714,-9.47368421052632)(1266.85714285714,-10.4210526315789)(1266.85714285714,-11.3684210526316)(1266.85714285714,-12.3157894736842)(1266.85714285714,-13.2631578947368)(1266.85714285714,-14.2105263157895)(1266.85714285714,-15.1578947368421)(1266.85714285714,-16.1052631578947)(1266.85714285714,-17.0526315789474)(1266.85714285714,-18)(1161.28571428571,0)(1161.28571428571,-0.947368421052632)(1161.28571428571,-1.89473684210526)(1161.28571428571,-2.84210526315789)(1161.28571428571,-3.78947368421053)(1161.28571428571,-4.73684210526316)(1161.28571428571,-5.68421052631579)(1161.28571428571,-6.63157894736842)(1161.28571428571,-7.57894736842105)(1161.28571428571,-8.52631578947369)(1161.28571428571,-9.47368421052632)(1161.28571428571,-10.4210526315789)(1161.28571428571,-11.3684210526316)(1161.28571428571,-12.3157894736842)(1161.28571428571,-13.2631578947368)(1161.28571428571,-14.2105263157895)(1161.28571428571,-15.1578947368421)(1161.28571428571,-16.1052631578947)(1161.28571428571,-17.0526315789474)(1161.28571428571,-18)(1055.71428571429,0)(1055.71428571429,-0.947368421052632)(1055.71428571429,-1.89473684210526)(1055.71428571429,-2.84210526315789)(1055.71428571429,-3.78947368421053)(1055.71428571429,-4.73684210526316)(1055.71428571429,-5.68421052631579)(1055.71428571429,-6.63157894736842)(1055.71428571429,-7.57894736842105)(1055.71428571429,-8.52631578947369)(1055.71428571429,-9.47368421052632)(1055.71428571429,-10.4210526315789)(1055.71428571429,-11.3684210526316)(1055.71428571429,-12.3157894736842)(1055.71428571429,-13.2631578947368)(1055.71428571429,-14.2105263157895)(1055.71428571429,-15.1578947368421)(1055.71428571429,-16.1052631578947)(1055.71428571429,-17.0526315789474)(1055.71428571429,-18)(950.142857142857,0)(950.142857142857,-0.947368421052632)(950.142857142857,-1.89473684210526)(950.142857142857,-2.84210526315789)(950.142857142857,-3.78947368421053)(950.142857142857,-4.73684210526316)(950.142857142857,-5.68421052631579)(950.142857142857,-6.63157894736842)(950.142857142857,-7.57894736842105)(950.142857142857,-8.52631578947369)(950.142857142857,-9.47368421052632)(950.142857142857,-10.4210526315789)(950.142857142857,-11.3684210526316)(950.142857142857,-12.3157894736842)(950.142857142857,-13.2631578947368)(950.142857142857,-14.2105263157895)(950.142857142857,-15.1578947368421)(950.142857142857,-16.1052631578947)(950.142857142857,-17.0526315789474)(950.142857142857,-18)(844.571428571429,0)(844.571428571429,-0.947368421052632)(844.571428571429,-1.89473684210526)(844.571428571429,-2.84210526315789)(844.571428571429,-3.78947368421053)(844.571428571429,-4.73684210526316)(844.571428571429,-5.68421052631579)(844.571428571429,-6.63157894736842)(844.571428571429,-7.57894736842105)(844.571428571429,-8.52631578947369)(844.571428571429,-9.47368421052632)(844.571428571429,-10.4210526315789)(844.571428571429,-11.3684210526316)(844.571428571429,-12.3157894736842)(844.571428571429,-13.2631578947368)(844.571428571429,-14.2105263157895)(844.571428571429,-15.1578947368421)(844.571428571429,-16.1052631578947)(844.571428571429,-17.0526315789474)(844.571428571429,-18)(739,0)(739,-0.947368421052632)(739,-1.89473684210526)(739,-2.84210526315789)(739,-3.78947368421053)(739,-4.73684210526316)(739,-5.68421052631579)(739,-6.63157894736842)(739,-7.57894736842105)(739,-8.52631578947369)(739,-9.47368421052632)(739,-10.4210526315789)(739,-11.3684210526316)(739,-12.3157894736842)(739,-13.2631578947368)(739,-14.2105263157895)(739,-15.1578947368421)(739,-16.1052631578947)(739,-17.0526315789474)(739,-18)(633.428571428571,0)(633.428571428571,-0.947368421052632)(633.428571428571,-1.89473684210526)(633.428571428571,-2.84210526315789)(633.428571428571,-3.78947368421053)(633.428571428571,-4.73684210526316)(633.428571428571,-5.68421052631579)(633.428571428571,-6.63157894736842)(633.428571428571,-7.57894736842105)(633.428571428571,-8.52631578947369)(633.428571428571,-9.47368421052632)(633.428571428571,-10.4210526315789)(633.428571428571,-11.3684210526316)(633.428571428571,-12.3157894736842)(633.428571428571,-13.2631578947368)(633.428571428571,-14.2105263157895)(633.428571428571,-15.1578947368421)(633.428571428571,-16.1052631578947)(633.428571428571,-17.0526315789474)(633.428571428571,-18)(527.857142857143,0)(527.857142857143,-0.947368421052632)(527.857142857143,-1.89473684210526)(527.857142857143,-2.84210526315789)(527.857142857143,-3.78947368421053)(527.857142857143,-4.73684210526316)(527.857142857143,-5.68421052631579)(527.857142857143,-6.63157894736842)(527.857142857143,-7.57894736842105)(527.857142857143,-8.52631578947369)(527.857142857143,-9.47368421052632)(527.857142857143,-10.4210526315789)(527.857142857143,-11.3684210526316)(527.857142857143,-12.3157894736842)(527.857142857143,-13.2631578947368)(527.857142857143,-14.2105263157895)(527.857142857143,-15.1578947368421)(527.857142857143,-16.1052631578947)(527.857142857143,-17.0526315789474)(527.857142857143,-18)(422.285714285714,0)(422.285714285714,-0.947368421052632)(422.285714285714,-1.89473684210526)(422.285714285714,-2.84210526315789)(422.285714285714,-3.78947368421053)(422.285714285714,-4.73684210526316)(422.285714285714,-5.68421052631579)(422.285714285714,-6.63157894736842)(422.285714285714,-7.57894736842105)(422.285714285714,-8.52631578947369)(422.285714285714,-9.47368421052632)(422.285714285714,-10.4210526315789)(422.285714285714,-11.3684210526316)(422.285714285714,-12.3157894736842)(422.285714285714,-13.2631578947368)(422.285714285714,-14.2105263157895)(422.285714285714,-15.1578947368421)(422.285714285714,-16.1052631578947)(422.285714285714,-17.0526315789474)(422.285714285714,-18)(316.714285714286,0)(316.714285714286,-0.947368421052632)(316.714285714286,-1.89473684210526)(316.714285714286,-2.84210526315789)(316.714285714286,-3.78947368421053)(316.714285714286,-4.73684210526316)(316.714285714286,-5.68421052631579)(316.714285714286,-6.63157894736842)(316.714285714286,-7.57894736842105)(316.714285714286,-8.52631578947369)(316.714285714286,-9.47368421052632)(316.714285714286,-10.4210526315789)(316.714285714286,-11.3684210526316)(316.714285714286,-12.3157894736842)(316.714285714286,-13.2631578947368)(316.714285714286,-14.2105263157895)(316.714285714286,-15.1578947368421)(316.714285714286,-16.1052631578947)(316.714285714286,-17.0526315789474)(316.714285714286,-18)(211.142857142857,0)(211.142857142857,-0.947368421052632)(211.142857142857,-1.89473684210526)(211.142857142857,-2.84210526315789)(211.142857142857,-3.78947368421053)(211.142857142857,-4.73684210526316)(211.142857142857,-5.68421052631579)(211.142857142857,-6.63157894736842)(211.142857142857,-7.57894736842105)(211.142857142857,-8.52631578947369)(211.142857142857,-9.47368421052632)(211.142857142857,-10.4210526315789)(211.142857142857,-11.3684210526316)(211.142857142857,-12.3157894736842)(211.142857142857,-13.2631578947368)(211.142857142857,-14.2105263157895)(211.142857142857,-15.1578947368421)(211.142857142857,-16.1052631578947)(211.142857142857,-17.0526315789474)(211.142857142857,-18)(105.571428571429,0)(105.571428571429,-0.947368421052632)(105.571428571429,-1.89473684210526)(105.571428571429,-2.84210526315789)(105.571428571429,-3.78947368421053)(105.571428571429,-4.73684210526316)(105.571428571429,-5.68421052631579)(105.571428571429,-6.63157894736842)(105.571428571429,-7.57894736842105)(105.571428571429,-8.52631578947369)(105.571428571429,-9.47368421052632)(105.571428571429,-10.4210526315789)(105.571428571429,-11.3684210526316)(105.571428571429,-12.3157894736842)(105.571428571429,-13.2631578947368)(105.571428571429,-14.2105263157895)(105.571428571429,-15.1578947368421)(105.571428571429,-16.1052631578947)(105.571428571429,-17.0526315789474)(105.571428571429,-18) 
};

\addplot [
color=blue,
mark size=3.5pt,
only marks,
mark=o,
mark options={solid,draw=lime!80!black},
line width=2.7pt,
forget plot
]
coordinates{
 (0,-18)
};

\end{axis}
\end{tikzpicture}%

\renewcommand\trimlen{0pt}
\begin{figure}[tbp]
  \begin{subfigure}[b]{0.99\linewidth}
    \centering
    \adjincludegraphics[width=\linewidth,clip=true,trim=\trimlen{} \trimlen{} \trimlen{} \trimlen{}]{figures/ch03/limno_bgape_ppgraph_0}
    \vspace{-18pt}
    \caption{$t = 0$}
    \label{fig:graph}
  \end{subfigure}

  \begin{subfigure}[b]{0.99\linewidth}
    \centering
    \vspace{8pt} % space of this row from above captions
    \adjincludegraphics[width=\linewidth,clip=true,trim=\trimlen{} \trimlen{} \trimlen{} \trimlen{}]{figures/ch03/limno_bgape_ppgraph_1}
    \vspace{-18pt}
    \caption{$t = 4$}
    \label{fig:graph}
  \end{subfigure}
  
  \begin{subfigure}[b]{0.99\linewidth}
    \centering
    \vspace{8pt} % space of this row from above captions
    \adjincludegraphics[width=\linewidth,clip=true,trim=\trimlen{} \trimlen{} \trimlen{} \trimlen{}]{figures/ch03/limno_bgape_ppgraph_2}
    \vspace{-18pt}
    \caption{$t = 10$}
    \label{fig:graph}
  \end{subfigure}
  
  \begin{subfigure}[b]{0.99\linewidth}
    \centering
    \vspace{8pt} % space of this row from above captions
    \adjincludegraphics[width=\linewidth,clip=true,trim=\trimlen{} \trimlen{} \trimlen{} \trimlen{}]{figures/ch03/limno_bgape_ppgraph_3}
    \vspace{-18pt}
    \caption{$t = 14$}
    \label{fig:graph}
  \end{subfigure}
  
  \begin{subfigure}[b]{0.99\linewidth}
    \centering
    \vspace{8pt} % space of this row from above captions
    \adjincludegraphics[width=\linewidth,clip=true,trim=\trimlen{} \trimlen{} \trimlen{} \trimlen{}]{figures/ch03/limno_bgape_ppgraph_4}
    \vspace{-18pt}
    \caption{$t = 28$}
    \label{fig:graph}
  \end{subfigure}
  
  \caption{
    Path planning with the graph-based algorithm
    on the the inferred algae concentration GP of \figref{fig:limno_bgape}
    using a $20\times 15$ graph, $\epsilon = 0.1$, $k = 10$, and $n_e=10$.
    Rather than running the algorithm until every point has been classified,
    we only run it for two traversals of the transect.
    Note how the predicted path (red line) gets adjusted as newer
    measurements become available.
  }
  \label{fig:limno_bgape_ppbatch}
\end{figure}
