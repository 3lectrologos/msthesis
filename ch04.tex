\chapter{Experiments} \label{sect:exp} 
In this section, we evaluate our proposed algorithms on two real-world datasets
and compare them to the state-of-the-art.
In more detail, the algorithms and their setup are as follows.
\begin{description}[labelindent=0pt,leftmargin=7pt,itemindent=-2pt,itemsep=0pt]
\item[\acl/\iacl:] Since the bound of \theoremref{thm:acl} is fairly conservative,
  in our experiments we used a constant value of $\beta_t^{1/2} = 3$, which is
  somewhat smaller than the values suggested by the theorem.
\item[\bacl/\ibacl:] We used $\eta_t^{1/2} = 4$ and a batch size of $B = 30$.
  Furthermore, as proposed by \citet{desautels12}, we initialized the
  algorithms with $T^{init} = 30$ sequentially chosen maximum variance samples.
\item[\str:] The state-of-the-art straddle heuristic,
  as proposed by \citet{bryan05}, with the selection rule
  ${\*x_t = \argmax_{\*x \in D} \left(1.96\sigma_{t-1}(\*x) - |\mu_{t-1}(\*x) - h|\right)}$.
\item[\istr:] For the implicit threshold setting, we have defined a
  variant of the straddle heuristic that uses at each step the
  implicitly defined threshold level with respect to the maximum of the
  inferred mean, i.e.
  ${h_t = \omega \max_{\*x\in D} \mu_{t-1}(\*x)}$.
\item[\rstr/\bstr:] We have defined two batch versions of the straddle heuristic:
  \rstr selects the $B = 30$ points with
  the largest straddle score, while \bstr follows a similar approach to \bacl
  by using the following selection rule
  \begin{align*}
  {\*x_t = \argmax_{\*x \in D}\left(1.96\sigma_{t-1}(\*x) - |\mu_{\fb[t]}(\*x) - h|\right)}.
  \end{align*}
\item[\var:]  The maximum variance rule
  $\*x_t = \argmax_{\*x \in D} \sigma_{t-1}(\*x)$.
\end{description}

We assess the classification accuracy for all algorithms using the $F_1$-score, i.e.
the harmonic mean of precision and recall, by considering points in the super-
and sublevel sets as positives and negatives respectively.
Finally, in all evaluations of \acl and its extensions, the accuracy parameter
$\epsilon$ is chosen to increase exponentially from $2\%$ up to $20\%$ of the
maximum value of each dataset.

\paragraph{Dataset 1: Network latency}
Our first dataset consists of round-trip time (RTT) measurements obtained
by ``pinging'' 1768 servers spread around the world.
The sample space consists of the longitude and latitude coordinates of each
server, as determined by a commercial geolocation
database\footnote{\url{http://www.maxmind.com}}.
Example applications for geographic RTT level set estimation, include
monitoring global networks and improving quality of service for applications
such as internet telephony or online games. Furthermore,
selecting samples in batches results in significant time savings, since
sending out and receiving multiple ICMP packets can be virtually performed in
parallel.

We used 200 randomly selected measurements to fit suitable hyperparameters for
an anisotropic Mat\'{e}rn-5~\mbox{\cite{rasmussen06}} kernel by maximum
likelihood  and the remaining 1568 for evaluation.
The threshold level we chose for the experiments was $h = 200$ ms.


\paragraph{Dataset 2: Environmental monitoring}
Our second dataset comes from the domain of environmental monitoring of inland
waters and consists of 2024 \emph{in situ} measurements of chlorophyll
%and {\it planktothrix rubescens}\footnote{a genus
%of the blue-green algae, which can produce toxins}
%\cite{hitz12}. cite this in the final version
concentration within a vertical transect plane, collected by an autonomous
surface vessel in a lake.
Monitoring chlorophyll concentration is useful in analyzing limnological
phenomena such as algal bloom. Since chlorophyll levels can vary throughout
the year, in addition to having a fixed threshold concentration, 
it can also be useful to be able to detect
relative ``hotspots'' of chlorophyll, i.e. regions of high concentration
with respect to the current maximum. Furthermore, selecting batches of points
can be used to plan sampling paths and reduce the required traveling
distances.

In our evaluation, we used $10,000$ points sampled in a $100 \times 100$ grid
from the GP posterior that was derived using the 2024 original measurements
(see~\figref{fig:limno_chl}).
Again, an anisotropic Mat\'{e}rn-5 kernel was used and suitable
hyperparameters were fitted by maximizing the
likelihood of a second available chlorophyll dataset from the same lake.
We used an explicit threshold level of $h = 1.3$ RFU and chose $\omega = 0.74$
for the implicit threshold case, so that the resulting implicit level is
identical to the explicit one, which enables us to compare the two settings on
equal ground.
The effect of batch sampling on the required traveling distance was evaluated
using an approximate Euclidean TSP solver to create paths connecting each
batch of samples selected by \bacl.

\setlength\figureheight{1.3in}\setlength\figurewidth{2.1in}
\renewcommand\trimlen{2pt}
\begin{figure*}[t]
  \begin{subfigure}[b]{0.49\textwidth}
    \centering
    %% This file was created by matlab2tikz v0.2.3.
% Copyright (c) 2008--2012, Nico Schlömer <nico.schloemer@gmail.com>
% All rights reserved.
%
%
%
\begin{tikzpicture}

\begin{axis}[%
tick label style={font=\tiny},
label style={font=\tiny},
label shift={-4pt},
xlabel shift={-6pt},
legend style={font=\tiny},
view={0}{90},
width=\figurewidth,
height=\figureheight,
scale only axis,
xmin=0, xmax=650,
xlabel={Samples},
ymin=0.58, ymax=1,
ylabel={$F_1$-score},
axis lines*=left,
legend cell align=left,
legend style={at={(1.03,0)},anchor=south east,fill=none,draw=none,align=left,row sep=-0.2em},
clip=false]


\addplot [
color=red,
densely dotted,
line width=0.7pt,
mark=x,
mark size=2pt,
mark options=solid
]
coordinates{
 (20,0.739299610894942)(56,0.79219512195122)(92,0.780487804878049)(128,0.813868613138686)(164,0.827149321266968)(199,0.835603996366939)(235,0.842105263157895)(271,0.848816029143898)(307,0.853747714808044)(343,0.874646559849199)(378,0.887823585810163)(414,0.906953966699314)(450,0.913385826771653)(486,0.917485265225933)(522,0.922624877571009)(557,0.939663699307616)(593,0.942687747035573)(629,0.944664031620553) 
};
\addlegendentry{\str};

\addplot [
color=orange,
densely dotted,
line width=0.7pt,
mark=+,
mark size=2pt,
mark options=solid
]
coordinates{
 (20,0.59672619047619)(56,0.72972972972973)(92,0.753246753246753)(128,0.771739130434783)(164,0.773980154355016)(199,0.795309168443497)(235,0.789699570815451)(271,0.797323135755258)(307,0.808712121212121)(343,0.823641563393708)(378,0.834586466165413)(414,0.84101599247413)(450,0.849340866290019)(486,0.853635505193579)(522,0.867783985102421)(557,0.874418604651163)(593,0.873949579831933)(629,0.879326473339569) 
};
\addlegendentry{\var};

\addplot [
color=blue,
solid,
line width=1.0pt,
mark=*,
mark size=1.0pt,
mark options=solid
]
coordinates{
 (650,0.95766129032258)(638,0.956609485368315)(625,0.953441295546559)(613,0.953629032258064)(565,0.932286555446516)(503,0.913043478260869)(488,0.917355371900826)(490,0.910331384015594)(436,0.888888888888889)(405,0.888888888888889)(395,0.874184529356943)(361,0.876876876876877)(307,0.851412944393801)(290,0.850828729281768)(268,0.846648301193756)(238,0.839708561020036)(213,0.829891838741396)(195,0.82804995196926)(172,0.818767249310027)(149,0.812098991750687)(132,0.810246679316888)(114,0.782112274024738)(103,0.792490118577075)(89,0.787172011661808) 
};
\addlegendentry{\acl};


\end{axis}
\end{tikzpicture}%

    \adjincludegraphics[width=\linewidth,clip=true,trim=\trimlen{} \trimlen{} \trimlen{} \trimlen{}]{figures/p_ping_seq}
    \caption{\textsf{[N]} Explicit threshold -- sequential}
	  \label{fig:ping-seq}
  \end{subfigure}
  \hfill
  \begin{subfigure}[b]{0.49\textwidth}
    \centering
    %% This file was created by matlab2tikz v0.2.3.
% Copyright (c) 2008--2012, Nico Schlömer <nico.schloemer@gmail.com>
% All rights reserved.
%
%
%
\begin{tikzpicture}

\begin{axis}[%
tick label style={font=\tiny},
label style={font=\tiny},
label shift={-4pt},
xlabel shift={-6pt},
legend style={font=\tiny},
view={0}{90},
width=\figurewidth,
height=\figureheight,
scale only axis,
xmin=0, xmax=230,
xlabel={Samples},
ymin=0.37, ymax=1,
ylabel={$F_1$-score},
axis lines*=left,
legend cell align=left,
legend style={at={(1.03,0)},anchor=south east,fill=none,draw=none,align=left,row sep=-0.2em},
clip=false]


\addplot [
color=red,
densely dotted,
line width=0.7pt,
mark=x,
mark size=2pt,
mark options=solid
]
coordinates{
 (20,0.666693706777946)(43,0.922211105964519)(66,0.949231006424866)(88,0.970407393451795)(111,0.978539126007941)(134,0.985378154601384)(156,0.98806244423496)(179,0.990745988176034)(202,0.992239730404345)(224,0.994329584437538)%(247,0.995225546720995)(269,0.996419873247655)(292,0.996718350300788)(315,0.997315914655458)(337,0.998807206170819)(360,0.999105424454402)(383,0.999403629493321)(405,0.999701821328293)(428,0.999701821328293)(450,1) 
};
\addlegendentry{\str};

\addplot [
color=orange,
densely dotted,
line width=0.7pt,
mark=+,
mark size=2pt,
mark options=solid
]
coordinates{
 (20,0.401360823274472)(43,0.818989322390159)(66,0.8295936348324)(88,0.874969089344893)(111,0.907484772654931)(134,0.929950280949145)(156,0.933416971745513)(179,0.936419455751627)(202,0.941265255893629)(224,0.945285821704858) 
};
\addlegendentry{\var};

\addplot [
color=blue,
solid,
line width=1.0pt,
mark=*,
mark size=1.0pt,
mark options=solid
]
coordinates{
 (216,0.990108337258596)(172,0.985370457763096)(144,0.982770828416332)(118,0.977304964539007)(97,0.971807628524046)(82,0.963104022851702)(69,0.956666666666667)(64,0.946517708580937)(53,0.921433657726086)(48,0.931974401516947)(45,0.902336448598131)(40,0.880417517585659)(35,0.870524656608872)(29,0.853575482406356)(22,0.770660638030646) 
};
\addlegendentry{\acl};

\end{axis}
\end{tikzpicture}%

    \adjincludegraphics[width=\linewidth,clip=true,trim=\trimlen{} \trimlen{} \trimlen{} \trimlen{}]{figures/p_chl_seq}
    \caption{\textsf{[E]} Explicit threshold -- sequential}
	\label{fig:chl-seq}
  \end{subfigure}

  \begin{subfigure}[b]{0.49\textwidth}
    \centering
    \vspace{12pt} % space of this row from above captions
    %% This file was created by matlab2tikz v0.2.3.
% Copyright (c) 2008--2012, Nico Schlömer <nico.schloemer@gmail.com>
% All rights reserved.
%
%
%

% defining custom colors
\definecolor{mycolor1}{rgb}{0,0.75,0.75}

\begin{tikzpicture}

\begin{axis}[%
tick label style={font=\tiny},
label style={font=\tiny},
label shift={-4pt},
xlabel shift={-6pt},
legend style={font=\tiny},
view={0}{90},
width=\figurewidth,
height=\figureheight,
scale only axis,
xmin=0, xmax=650,
xlabel={Samples},
ymin=0.6, ymax=1,
ylabel={$F_1$-score},
axis lines*=left,
legend cell align=left,
legend style={at={(1.03,0)},anchor=south east,fill=none,draw=none,align=left,row sep=-0.2em},
clip=false]


\addplot [
color=red,
densely dotted,
line width=0.7pt,
mark=x,
mark size=2pt,
mark options=solid
]
coordinates{
 (20,0.739299610894942)(56,0.79219512195122)(92,0.780487804878049)(128,0.813868613138686)(164,0.827149321266968)(199,0.835603996366939)(235,0.842105263157895)(271,0.848816029143898)(307,0.853747714808044)(343,0.874646559849199)(378,0.887823585810163)(414,0.906953966699314)(450,0.913385826771653)(486,0.917485265225933)(522,0.922624877571009)(557,0.939663699307616)(593,0.942687747035573)(629,0.944664031620553) 
};
\addlegendentry{\str};


\addplot [
color=orange,
densely dotted,
line width=0.7pt,
mark=+,
mark size=2pt,
mark options=solid
]
coordinates{
 (30,0.647657841140529)(60,0.629553264604811)(90,0.673381294964029)(120,0.721489526764934)(150,0.742485783915516)(180,0.747342600163532)(210,0.757701915070774)(240,0.757297748123436)(270,0.779334500875657)(300,0.784)(330,0.829268292682927)(360,0.836772983114446)(390,0.848837209302326)(420,0.856343283582089)(450,0.85820895522388)(480,0.863000931966449)(510,0.863805970149254)(540,0.875584658559401)(570,0.882242990654206)(600,0.898120672601385)(630,0.920920920920921) 
};
\addlegendentry{\rstr};


\addplot [
color=green!50!black,
densely dotted,
line width=0.7pt,
mark=triangle*,
mark size=1.5pt,
mark options=solid
]
coordinates{
 (30,0.649586776859504)(60,0.756805807622504)(90,0.778388278388278)(120,0.779929577464789)(150,0.792486583184257)(180,0.804701627486437)(210,0.815356489945155)(240,0.824723247232472)(270,0.836431226765799)(300,0.843082636954503)(330,0.856338028169014)(360,0.864966949952786)(390,0.885436893203883)(420,0.894190871369294)(450,0.909844559585492)(480,0.916235780765253)(510,0.92020725388601)(540,0.924508790072389)(570,0.934426229508197)(600,0.943089430894309)(630,0.950455005055612) 
};
\addlegendentry{\bstr};


\addplot [
color=blue,
solid,
line width=1.0pt,
mark=*,
mark size=1.0pt,
mark options=solid
]
coordinates{
 (630,0.942828485456369)(570,0.929718875502008)(518,0.910679611650485)(468,0.891732283464567)(390,0.88041237113402)(320,0.847866419294991)(270,0.838649155722326)(240,0.850333651096282)(210,0.805405405405405)(180,0.818867924528302)(150,0.810506566604127)(120,0.798715203426124)(90,0.777242044358727)(81,0.718367346938775)(60,0.717255717255717) 
};
\addlegendentry{\bacl};


\end{axis}
\end{tikzpicture}%

    \adjincludegraphics[width=\linewidth,clip=true,trim=\trimlen{} \trimlen{} \trimlen{} \trimlen{}]{figures/p_ping_batch}
    \caption{\textsf{[N]} Explicit threshold -- batch}
	  \label{fig:ping-batch}
  \end{subfigure}
  \hfill
  \begin{subfigure}[b]{0.49\textwidth}
    \centering
    %% This file was created by matlab2tikz v0.2.3.
% Copyright (c) 2008--2012, Nico Schlömer <nico.schloemer@gmail.com>
% All rights reserved.
%
%
%

% defining custom colors
\definecolor{mycolor1}{rgb}{0,0.75,0.75}

\begin{tikzpicture}

\begin{axis}[%
tick label style={font=\tiny},
label style={font=\tiny},
label shift={-4pt},
xlabel shift={-6pt},
legend style={font=\tiny},
view={0}{90},
width=\figurewidth,
height=\figureheight,
scale only axis,
xmin=0, xmax=370,
xlabel={Samples},
ymin=0.18, ymax=1,
ylabel={$F_1$-score},
axis lines*=left,
legend cell align=left,
legend style={at={(1.03,0)},anchor=south east,fill=none,draw=none,align=left,row sep=-0.2em},
clip=false]


\addplot [
color=red,
densely dotted,
line width=0.7pt,
mark=x,
mark size=2pt,
mark options=solid
]
coordinates{
 (20,0.666693706777946)(43,0.922211105964519)(66,0.949231006424866)(88,0.970407393451795)(111,0.978539126007941)(134,0.985378154601384)(156,0.98806244423496)(179,0.990745988176034)(202,0.992239730404345)(224,0.994329584437538)(247,0.995225546720995)(269,0.996419873247655)(292,0.996718350300788)(315,0.997315914655458)(337,0.998807206170819)(360,0.999105424454402)%(383,0.999403629493321)(405,0.999701821328293)(428,0.999701821328293)(450,1) 
};
\addlegendentry{\str};


\addplot [
color=orange,
densely dotted,
line width=0.7pt,
mark=+,
mark size=2pt,
mark options=solid
]
coordinates{
 (60,0.218811659186721)(90,0.576931804893578)(120,0.597122039036391)(150,0.596748097093111)(180,0.595625364747026)(210,0.651046329095089)(240,0.753254982695503)(270,0.761401927537089)(300,0.777030496340667)(330,0.796237965621637)(360,0.797282370780275)%(390,0.85877480968477)(420,0.908003049795615)(450,0.924429037896539) 
};
\addlegendentry{\rstr};

\addplot [
color=green!50!black,
densely dotted,
line width=0.7pt,
mark=triangle*,
mark size=1.5pt,
mark options=solid
]
coordinates{
 (30,0.616799995328497)(60,0.907166143276133)(90,0.947557759563669)(120,0.97195331988199)(150,0.978800667304354)(180,0.98388373845251)(210,0.98806244423496)(240,0.990448240614354)(270,0.992836033658206)(300,0.995225939862017)(330,0.997016947084897)(360,0.998210453142179)%(390,0.998807081051637)(420,0.99910537504784)(450,0.999403655792747) 
};
\addlegendentry{\bstr};

\addplot [
color=blue,
solid,
line width=1.0pt,
mark=*,
mark size=1.0pt,
mark options=solid
]
coordinates{
 (365,0.99835564951844)(300,0.995296331138288)(270,0.992226148409894)(210,0.988207547169811)(180,0.984194385468271)(150,0.979225684608121)(120,0.967466160056993)(90,0.941261434761675)(60,0.903389439704865) 
};
\addlegendentry{\bacl};


\end{axis}
\end{tikzpicture}%

    \adjincludegraphics[width=\linewidth,clip=true,trim=\trimlen{} \trimlen{} \trimlen{} \trimlen{}]{figures/p_chl_batch}
    \caption{\textsf{[E]} Explicit threshold -- batch}
	  \label{fig:chl-batch}
  \end{subfigure}

  \begin{subfigure}[b]{0.49\textwidth}
    \centering
    \vspace{12pt} % space of this row from above captions
    %% This file was created by matlab2tikz v0.2.3.
% Copyright (c) 2008--2012, Nico Schlömer <nico.schloemer@gmail.com>
% All rights reserved.
%
%
%
\begin{tikzpicture}

\begin{axis}[%
tick label style={font=\tiny},
label style={font=\tiny},
label shift={-4pt},
xlabel shift={-6pt},
legend style={font=\tiny},
view={0}{90},
width=\figurewidth,
height=\figureheight,
scale only axis,
xmin=0, xmax=520,
xlabel={Samples},
ymin=0.65, ymax=1,
ylabel={$F_1$-score},
axis lines*=left,
legend cell align=left,
legend style={at={(1.03,0)},anchor=south east,fill=none,draw=none,align=left,row sep=-0.2em},
clip=false]


\addplot [
color=red,
densely dotted,
line width=0.7pt,
mark=x,
mark size=2pt,
mark options=solid
]
coordinates{
 (20,0.666693706777946)(47,0.923035811259576)(73,0.962054989797891)(99,0.972256143357283)(126,0.983587588237688)(152,0.987762441723646)(178,0.990745988176034)(205,0.992239730404345)(231,0.994926850481122)(257,0.995524116524706)(284,0.996419873247655)(310,0.997315914655458)(336,0.998807206170819)(363,0.999105424454402)(389,0.999403629493321)(415,0.999701821328293)(442,1)(468,1)(494,1)(520,1) 
};
\addlegendentry{\str};


\addplot [
color=orange,
densely dotted,
line width=0.7pt,
mark=+,
mark size=2pt,
mark options=solid
]
coordinates{
 (20,0.689606292531263)(47,0.822847192505354)(73,0.876338293844459)(99,0.876111009342508)(126,0.878617505938897)(152,0.88136795479191)(178,0.881597922544582)(205,0.881828008317907)(231,0.881597922544582)(257,0.880449261705193)(284,0.880449261705193)(310,0.879990620322686)(336,0.878846066311138)(363,0.878389062114004)(389,0.878389062114004)(415,0.878160734724183)(442,0.877932523657309)(468,0.877248587275352)(494,0.877476450044646)(520,0.877476450044646) 
};
\addlegendentry{\istr};


\addplot [
color=blue,
solid,
line width=1.0pt,
mark=*,
mark size=1.0pt,
mark options=solid
]
coordinates{
 (514,0.986559773638293)(415,0.981996726677578)(340,0.978286247957039)(274,0.976624590930341)(225,0.975415593537813)(187,0.973036342321219)(149,0.967651195499297)(125,0.95947219604147)(106,0.959357277882798)(86,0.954141319606925)(70,0.938200906704844)(58,0.922318704529453)(47,0.865531704909943)(41,0.828895849647611)(34,0.815880322209436)(30,0.720432342548327) 
};
\addlegendentry{\iacl};


\addplot [
color=green!50!black,
solid,
line width=1.0pt,
mark=triangle*,
mark size=1.0pt,
mark options=solid
]
coordinates{
 (480,0.980602944613227)(394,0.978733348913297)(330,0.978444236176195)(300,0.975210477081384)(270,0.973264540337711)(210,0.972150713784227)(180,0.968917971488666)(150,0.965759849906191)(150,0.966034200046849)(120,0.959303693248647)(120,0.959303693248647)(120,0.959303693248647)(90,0.936110461034402)(90,0.936110461034402)(90,0.936110461034402)(90,0.936110461034402)(60,0.83785974868261)(60,0.83785974868261)(60,0.83785974868261)(60,0.83785974868261) 
};
\addlegendentry{\ibacl};


\end{axis}
\end{tikzpicture}%

    \adjincludegraphics[width=\linewidth,clip=true,trim=\trimlen{} \trimlen{} \trimlen{} \trimlen{}]{figures/p_chl_imp}
    \caption{\textsf{[E]} Implicit threshold}
	  \label{fig:chl-imp}
  \end{subfigure}
  \hfill
  \begin{subfigure}[b]{0.49\textwidth}
    \centering
    %% This file was created by matlab2tikz v0.2.3.
% Copyright (c) 2008--2012, Nico Schlömer <nico.schloemer@gmail.com>
% All rights reserved.
%
%
%
\begin{tikzpicture}

\begin{axis}[%
tick label style={font=\tiny},
label style={font=\tiny},
label shift={-4pt},
xlabel shift={-6pt},
legend style={font=\tiny},
view={0}{90},
width=\figurewidth,
height=\figureheight,
scale only axis,
xmin=0, xmax=35500,
xtick={0, 10000, 20000, 30000},
scaled ticks=false,
xlabel={Total travel length (m)},
ymin=0.82, ymax=1,
ylabel={$F_1$-score},
axis lines*=left,
legend cell align=left,
legend style={at={(1.03,0)},anchor=south east,fill=none,draw=none,align=left,row sep=-0.2em},
clip=false]


\addplot [
color=blue,
solid,
line width=0.7pt,
mark=*,
mark size=1.0pt
]
coordinates{
 (35441.7915161753,0.957391304347826)(33157.5621087165,0.95248522005693)(31560.1032547878,0.952360043907794)(30634.45271324,0.939486286460601)(27448.7430095821,0.928116069465816)(26881.429149883,0.925925925925926)(26879.2473317012,0.925478127060893)(23206.3907665825,0.914397724737899)(20697.6615741044,0.898086337338674)(17084.6155859159,0.885610531093963)(16890.4897306863,0.857716003005259)(13172.3209064758,0.831753113961703)
};
\addlegendentry{$B = 1$};


\addplot [
color=green!50!black,
solid,
line width=0.7pt,
mark=triangle*,
mark size=1.5pt
]
coordinates{
 (35067.7075979092,0.968180820715383)(30782.8938308443,0.967515364354697)(28809.8694473494,0.966301969365426)(24043.7029051417,0.964104822726272)(21462.6419802734,0.962897914379802)(18595.3589808005,0.959895379250218)(15083.8047448388,0.952673093777388)(15083.8047448388,0.952673093777388)(15083.8047448388,0.952673093777388)(12944.2507124109,0.928336620644313)(12944.2507124109,0.928336620644313)(11623.8315504192,0.925478127060893)(10294.5584013895,0.903901895206243)(7810.97862144332,0.899066251667408)(7810.97862144332,0.899066251667408)(7810.97862144332,0.899066251667408)(5206.59100527247,0.831753113961703)(5206.59100527247,0.831753113961703)(5206.59100527247,0.831753113961703)(5206.59100527247,0.831753113961703) 
};
\addlegendentry{$B = 5$};


\addplot [
color=orange,
solid,
line width=0.7pt,
mark=+,
mark size=2.0pt
]
coordinates{
 (35014.1448943239,0.98159509202454)(29684.3651063146,0.978313253012048)(26378.7116513409,0.976457645764576)(22519.3686084673,0.974494283201407)(18041.2714673298,0.97231985940246)(14854.3634784288,0.968476357267951)(11804.286348918,0.967090829311101)(10423.3003742316,0.963468309859155)(8748.49078414153,0.959250381346699)(7539.14777929507,0.959965187119234)(6169.43663053071,0.950841162333406)(6169.43663053071,0.950841162333406)(6169.43663053071,0.950841162333406)(4679.60338047477,0.926251097453907)(4679.60338047477,0.926251097453907)(3188.3146325471,0.901420959147424)(3188.3146325471,0.901420959147424)(3188.3146325471,0.901420959147424)(3188.3146325471,0.901420959147424) 
};
\addlegendentry{$B = 15$};

\addplot [
color=red,
solid,
line width=0.7pt,
mark=x,
mark size=2.0pt
]
coordinates{
 (35243.9756201591,0.987535534659961)(27025.0410478517,0.986990077177508)(22605.0964593641,0.983549023908752)(18077.2098956749,0.981132075471698)(15338.0661142578,0.977407326168019)(12260.0479449651,0.975362956445226)(9245.59060646341,0.972795085563844)(8798.21751940939,0.97082693573152)(6245.29196977966,0.968434896975011)(4631.60043447352,0.964535901926445)(4627.36401498911,0.963353083168751)(4627.36401498911,0.963353083168751)(3251.15220234956,0.938347179711412)(3251.15220234956,0.938347179711412)(3251.15220234956,0.938347179711412)(1739.56463320755,0.839450404667796)(1739.56463320755,0.839450404667796)(1739.56463320755,0.839450404667796) 
};
\addlegendentry{$B = 30$};

%\addplot [
%color=black,
%solid,
%line width=0.7pt,
%mark=x,
%mark size=1.0pt
%]
%coordinates{
% (7047.5913169571,1)(6679.35163135661,1)(6017.94788495517,1)(4372.29466183875,1)(4448.27730876844,1)(2970.77645629461,0.957575757575758)(2943.58265225566,0.948328267477203)(3185.16777820071,0.869565217391304)(3016.90620567418,0.849056603773585)(1479.28410738491,0.849056603773585) 
%};
%\addlegendentry{$B = 60$};


\end{axis}
\end{tikzpicture}%

    \adjincludegraphics[width=\linewidth,clip=true,trim=\trimlen{} \trimlen{} \trimlen{} \trimlen{}]{figures/p_chl_tsp}
    \caption{\textsf{[E]} Travel lengths for varying batch sizes}
	  \label{fig:chl-tsp}
  \end{subfigure}
  \caption{Results for the network latency \textsf{[N]} and environmental
           monitoring \textsf{[E]} datasets.
           (a), (b) \acl is competitive with \str, while both clearly
           outperform \var.
           (c), (d) \bacl and \bstr are only slightly worse than
           \str, while the naive \rstr is far inferior.
           (e) \iacl and \ibacl achieve high accuracy
           at a slower rate than the explicit threshold algorithms,
           while \istr fails to do so.
           (f) Using larger batch sizes for planning dramatically reduces
           the traveled path lengths.
           \vspace{-1em}}
  \label{fig:exp}
\end{figure*}

\paragraph{Results and discussion}
\figsref{fig:ping-seq} and~\ref{fig:chl-seq} compare the performance of the
strictly sequential algorithms on the two datasets. In both cases, \acl and
\str are comparable in performance, which is expected given the similarity
of their selection rules (see \sectref{sect:acl}). The slight superiority
of \str in \figref{fig:ping-seq} can be explained by the fact that \acl uses
a more conservative
value of $\beta_t$, which is required to avoid premature (mis)classification
phenomena in its monotonic classification scheme. On the other hand, note
that due to its aggressive sampling behavior, \str can in some cases suffer
from insufficient exploration.\footnotemark[2]
Although \var is commonly used for estimating functions over their entire
domain, it is clearly outperformed by both algorithms and, thus, deemed
unsuitable for the task of level set estimation.

In \figsref{fig:chl-batch} and~\ref{fig:ping-batch} we show the performance
of the batch algorithms on the two datasets. The \bacl and \bstr algorithms,
which use the always up-to-date variance estimates for selecting batches,
achieve similar performance. Furthermore, there is only a slight performance
penalty when compared to the strictly sequential \str, which can easily be
outweighed by the benefits of batch point selection (e.g. in the network
latency example, the batch algorithms would have about $B = 30$ times higher
throughput). As expected, the \rstr algorithm, performs significantly
worse than the other two batch algorithms, since it selects a lot of
redundant samples in areas of high straddle score
(cf. \sectref{sect:bacl}).

\figref{fig:chl-imp} shows the results of the implicit threshold experiments
on the environmental monitoring dataset. The difficulty of estimating the
function maximum at the same time as performing classification with respect to
the implicit threshold level is manifested in the notably larger sampling cost
of \iacl required to achieve high accuracy compared to the explicit threshold
experiments. As before, the batch version of
\iacl is only slightly worse that its sequential counterpart.
More importantly, the naive \istr algorithm completely
fails to achieve high accuracy, as it gets stuck with a wrong estimate of
the maximum and never recovers, since the straddle rule is not
sufficiently exploratory.

Finally, \figref{fig:chl-tsp} displays the dramatically lower required travel
length by using batches of samples for path planning: to achieve
an $F_1$-score of 0.95 using sequential sampling requires more than six times
larger traveling distance than planning ahead with $B = 30$
samples per batch.