\begin{abstract}
Many information gathering problems require %can be reduced to 
%In many applications, rather than estimating the value of some unknown
%function over its entire domain, it is of interest to accurately 
%the problem of 
determining the
set of points, for which an unknown function takes value above or below some given
threshold level. We formalize this task as a classification problem with
sequential measurements, where the unknown function is modeled as sample from a
Gaussian process (GP). We propose \acl, an algorithm that guides both sampling
and classification based on GP-derived confidence bounds,
and provide theoretical guarantees about its sample complexity.
%the number of
%samples required to achieve a certain classification accuracy. 
Furthermore, we extend \acl and its theory to two more natural settings:
% that naturally arise in practice: the setting 
(1) where the threshold level is implicitly
defined as a percentage of the (unknown) maximum of the target function and
(2) where samples are selected in batches. We evaluate the
effectiveness of our proposed methods on two problems of practical interest,
namely autonomous monitoring of algal populations in a lake environment and
geolocating network latency.
\end{abstract} 
\begin{ack}
I would like to thank\ldots
\end{ack}