\begin{abstract}
Many information gathering problems require determining the
set of points, for which an unknown function takes value above or below some
given threshold level.
As a concrete example, in the context of environmental monitoring of Lake
Zurich we would like to estimate the regions of the lake where the
concentration of chlorophyll or algae is greater than some critical value,
which would serve as an indicator of algal bloom phenomena.
A critical factor in such applications is the high cost in terms of time,
battery power, etc. that is associated with each measurement, therefore it
is of importance that we be
careful about selecting ``informative'' locations to sample, in order to
reduce the total sampling effort required.

We formalize the task of level set estimation as a classification problem with
sequential measurements, where the unknown function is modeled as a sample
from a Gaussian process (GP). We propose \acl, an active learning algorithm
that guides both sampling and classification based on GP-derived confidence
bounds, and provide theoretical guarantees about its sample complexity.
Furthermore, we extend \acl and its theory to two more natural settings:
(1) where the threshold level is implicitly
defined as a percentage of the (unknown) maximum of the target function and
(2) where samples are selected in batches. We evaluate the
effectiveness of our proposed methods on two problems of practical interest,
namely the aforementioned autonomous monitoring of algal populations in Lake
Zurich and geolocating network latency.
\end{abstract}

\cleardoublepage
\vspace{2in}
\chapter*{Acknowledgements}
I consider myself very lucky to have had such an excellent supervisor as
Prof. Andreas Krause, who has the gift of making complex things simple and
intuitive.

I would also like to thank Nathalie Casati and Gregory Hitz for our
collaboration on parts of this thesis.

Finally, thanks to the Limnological Station of University of Zurich and
the Autonomous Systems Lab of ETH for providing the environmental
monitoring datasets from Lake Zurich.

\vspace{2in}
\begingroup
\let\clearpage\relax
\chapter*{Contact}
The best way to contact me is to search my name on the Web and retrieve
my current e-mail address from my homepage. Of course, any questions or
comments are welcome.
\endgroup